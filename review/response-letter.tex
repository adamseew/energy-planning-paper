
\documentclass[10pt]{letter}
\usepackage[utf8]{inputenc}
\PassOptionsToPackage{hyphens}{url}
\usepackage{xcolor}
\definecolor{foo}{RGB}{0,51,102}
\usepackage[colorlinks=true,linkcolor=foo,urlcolor=foo]{hyperref}
\usepackage{amsmath}
\usepackage{amsfonts}
\DeclareMathOperator*{\argmax}{arg\,max}
\usepackage{cleveref}
\usepackage{tabto}
\usepackage{graphicx}
\usepackage{tikz}
\usepackage[left=1.1in,right=1.1in,top=.9in,bottom=.9in,%
            footskip=.25in]{geometry}
\usepackage{fancyhdr}
\fancypagestyle{plain}{%
  \fancyhf{}%
    \fancyfoot[C]{%\fontfamily{phv}\selectfont\small\thepage
    }%
    \renewcommand{\headrulewidth}{0pt} % line at the header
    \renewcommand{\footrulewidth}{0pt} % line at the footer (both not visible)
}
\renewcommand\familydefault{\sfdefault}
\renewcommand\sfdefault{phv}
\normalfont
\usepackage{endnotes}
\usepackage{float,graphicx}
\usepackage{hyperendnotes}
\let\footnote=\endnote
\makeatletter
\def\enoteheading{
  \mbox{}\par\vskip-2.3\baselineskip\noindent\rule{.5\textwidth}{0.4pt}\par\vskip\baselineskip}
\makeatother
\renewcommand\enoteformat{%
  \raggedright
  \leftskip=1.8em
  \vspace{-1.8em}
  \makebox[0pt][r]{\theenmark. \rule{0pt}{\dimexpr\ht\strutbox+\baselineskip}}%
}

\interfootnotelinepenalty=10000


\makeatletter
\newcommand\footnoteref[1]{\protected@xdef\@thefnmark{\ref{#1}}\@footnotemark}
\makeatother


\usepackage{advdate}
\newcommand{\yesterday}{{\AdvanceDate[-1]\today}}

%% pseudocode package
\usepackage{algorithm}
\usepackage[noend]{algorithmic}

% counter to maintain the line numbering
\newcommand{\setalglineno}[1]{%
  \setcounter{ALC@line}{\numexpr#1-1}}


% for adjustwidth environment
\usepackage[strict]{changepage}

% for formal definitions
\usepackage{framed}

% environment derived from framed.sty: see leftbar environment definition
\definecolor{formalshade}{rgb}{1,0.95,0.95}

\newenvironment{formal}{%
  \def\FrameCommand{%
    \hspace{1pt}%
    {\color{red}\vrule width 2pt}%
    {\color{formalshade}\vrule width 4pt}%
    \colorbox{formalshade}%
  }%
  \MakeFramed{\advance\hsize-\width\FrameRestore}%
  \noindent\hspace{-4.55pt}% disable indenting first paragraph
  \begin{adjustwidth}{}{7pt}%
  \vspace{2pt}\vspace{2pt}%
}
{%
  \vspace{2pt}\end{adjustwidth}\endMakeFramed%
}

\usepackage{fancyhdr}
\usepackage{lastpage}

\pagestyle{plain}
\fancyhf{}

\rfoot{Page \thepage \hspace{1pt} of \pageref{LastPage}}
\usepackage{mathtools}
\DeclarePairedDelimiter\norm{\lVert}{\rVert}%



\begin{document}
\pagestyle{plain}
\fontfamily{phv}\selectfont


%\vspace{4em}
%{\center{Adam Seewald \\Dept. \\University of Southern Denmark \\Campusvej 55 \\5230 Odense \\Denmark %\\\href{https://adamseewald.cc}{adamseewald.cc}, \href{mailto:adam@seewald.email}{adam@seewald.email}, \href{tel:+4527849313}{+45 2784 9313}\\
%}}

%\vspace{3em}

\begin{flushright}%Odense, Denmark, 
  \yesterday\end{flushright}

\vspace{2em}

%Dr. Hanna Kurniawati \\
%Reliable Robot Control (R2C) Lab, Cognitive Robotics Department\\
%Delft University of Technology\\
%Mekelweg 2 \\
%2628 CD Delft \\
%The Netherlands

{\centering Response letter to the editor and the referees}

\vspace{5em}

Dear Referees, Dr. Kurniawati:

\vspace{1em}

We greatly appreciate your time and effort in providing us with an invaluable assessment of our work, helping us to significantly improve both the quality and presentation of our energy-aware planning-scheduling for the aerial robotics domain. We have been able to address all your comments and questions in a revised version that we have attached once again for your kind consideration. 

The response letter is split into two sections, each for one review. The referees' comments are in black, and our response is highlighted in blue, similarly to the edits in the attached revised version. The quotes from the revised version are incorporated here in a box with a light red background and a thick red left border.

Thank you for your time and consideration. 

\vspace{1em}


\noindent Kind regards, 

\begin{flushright}
Authors of the manuscript 22-0795\,\,\,\,\,\,\,\,\,\,\,\,\,\,\,\\
%\includegraphics[scale=.35]{seewald_sig}
\end{flushright}

\vspace{5em}

\newpage 

{First review}

\vspace{3em}

{\hspace*{-4.5em}\textbf{[R1]}\vspace*{-1.9em}}

The paper addresses the problem of planning-scheduling for aerial robots with energy awareness for coverage path planning, in which a robot follows a path to cover a polygon. The objective is to maximize the configuration of parameters and satisfy constraints. The authors pose the optimization problem as an MPC and solve for the optimal parameters. They also present an energy model and propose a solution that adjusts the flight path and computational tasks online to deal with energy limitations. This approach extends the energy-aware planning-scheduling to UAVs.

Overall, the paper addresses a very interesting problem and the proposed solution is well-structured.
Related work is well reviewed in this letter. Below are some detailed comments and questions.

{\color{blue}

{\hspace*{-4.5em}{[R1]}\vspace*{-1.9em}}

We would like to thank the first referee for assessing our work with great detail, proposing edits, and posing important questions. These helped us to strengthen our contribution and clarify aspects we did not consider in the first version.

In this work, we choose to leverage planning-scheduling to aerial robotics, as we believe this is a relevant application and advantageous system for our devised approach. We are thus excited to see the referee sharing similar findings while bringing up numerous important points.}

\vspace{2em}

Major Comments:

{\hspace*{-4.5em}\textbf{[R1:1]}\vspace*{-1.9em}}

1. By definition, $t_s$ is the estimated time to complete the coverage, and $t_r$ is the remaining time to complete coverage at time instant $t$. Shouldn't we have $t_r=t_s-t$? However, in line 25 of
algorithm 2, $t_r=(t_s/\overline{t})(\overline{t}-t)$. Let's assume that $t_s=\underline{t}$. At time instant $t=\underline{t}$, by line 25, $t_r=(\underline{t}/\overline{t})(\overline{t}-\underline{t})=\underline{t}-\underline{t}^2/\overline{t}>0$. This does not make sense since the remaining time $t_r$ should equal to $0$. Could you clarify this?

{\color{blue}

{\hspace*{-4.5em}{[R1:1]}\vspace*{-1.9em}}

Thank you for underlying this issue. It is indeed an incorrect transcription of the algorithm, which we have corrected in the revised version. For the sake of clarity, we have further merged line 24 with 25.

\begin{formal}
  \begin{algorithmic}[1]
    \small
      \makeatletter
      \setcounter{ALC@line}{23}
      \makeatother
      \color{blue}\STATE $t_r\gets (\mathrm{diag}(\nu_i^\rho)\vspace*{.3ex}c_i^\rho(t)+\tau_i^\rho)[\overbrace{\begin{matrix}1&1&\cdots&1\end{matrix}}^{\rho}]-t$\vspace*{.3ex}\label{alg:traj1}
        
      \vspace*{.8ex}
    \end{algorithmic}
  \end{formal}  

}

{\hspace*{-4.5em}\textbf{[R1:2]}\vspace*{-1.9em}}

2. Line 24-27 of algorithm 2 only re-plans $c_i^\rho$. What about $c_i^\sigma$? Since the re-planning is conducted after solving the MPC, optimality can not be claimed for the re-planning. It's possible that there are multiple options to adjust the parameters during re-planning: 1. simultaneously adjusting both $c_i^\rho$ and $c_i^\sigma$; 2. only adjust $c_i^\rho$; 3. only adjust $c_i^\sigma$. It might be useful to consider all of these cases and discuss how they affect optimality.

{\color{blue} 

{\hspace*{-4.5em}{[R1:2]}\vspace*{-1.9em}}

Thank you for the comment as well as for proposing a way to address the issue. It is correct that in Lines 24--27 we re-plan solely $c_i^\rho$, i.e., the parameters related to the path. The remaining parameters $c_i^\sigma$ related to the computations are then re-planned utilizing the model predictive controller (MPC) in Line 16. The reason why we do not use MPC to plan both the parameters is due to the practical feasibility of re-planning in flight, %While an initial version of the algorithm optimized both via MPC, to assess the satisfaction of the battery constraint w.r.t. the path parameters, in the worst case, we would have to increase the MPC horizon up to the end of the flight.
and due to the different effects of the computation and path parameters on the energy consumption. While the computation parameters have an immediate effect, i.e., they affect the instantaneous energy consumption, the path parameters affect the flight time and thus the overall energy consumption. Concretely and by assumption, a change in the computations alters the schedule, hence the power drained by the computing hardware on the aerial robot, whereas a change in the path parameters alters the time when the aerial robot terminates the coverage.

We have thus decided to utilize a hybrid approach. We employ MPC for the schedule (computation parameters $c_i^\sigma$) and a greedy approach for the coverage (path parameters $c_i^\rho$). Similar combination of techniques has been investigated in the past planning-scheduling literature by Ondr\'{u}\v{s}ka et al.~(reference [{\color{green}19}] in the revised manuscript).
Nonetheless, as underlined in your observation, this was not clear from the text. We have updated the algorithm in Line 16.

\begin{formal}
\begin{algorithmic}[1]
  \small
    \makeatletter
    \setcounter{ALC@line}{15}
    \makeatother
    \color{blue}\STATE $\mathbf{q}(\mathcal{K}\setminus\{t+N\}),c_i^\sigma(\mathcal{K})\gets${ \vspace*{.3ex}solve NLP }$\argmax_{\mathbf{q}(k),c_i(k)}$\newline \vspace*{.7ex}\hspace*{1em}${l_f(\mathbf{q}(t+\hspace*{-.1ex}N),t+\hspace*{-.1ex}N)}+\hspace*{-.2ex}{\sum_{k\in\mathcal{K}}{l_d(\mathbf{q}(k),c_i(k),k)}}${ in Eq.~({\color{red}18})\newline \hspace*{1em}on }$\mathcal{K}=\{t,t+h,\dots,t+N\}$\vspace*{.3ex}
      
    \vspace*{.8ex}
  \end{algorithmic}
\end{formal}

  We have altered the introduction and the problem formulation in Section II to address the observation. We have further added an explanatory text in Section IV.B as well as in the conclusion. In the introduction we have stated explicitly that we utilize both the heuristics and optimal control.

  \begin{formal}
  {\color{black} [...] our approach uses optimal control {\color{blue} and heuristics} where both the paths and schedules variations are trajectories, varying between given bounds [...]} Hybrid approaches~[{\color{green}18}] are also available, where the techniques are mixed.
  \vspace*{1ex}
  \end{formal}

  We have altered the problem formulation, clearly stating that the objective is to find energy-aware trajectory of parameters rather than optimal.

  \begin{formal}
    {\color{black} [...] the \emph{re-planning-scheduling problem} is finding the {\color{blue}energy-aware} trajectory of parameters $c_i$ in time{\color{blue}, optimizing battery state of charge (SoC)}.}
    \vspace*{1ex}
  \end{formal}

  We have then altered Section IV.B.

  \begin{formal}
  \color{black}
  Past literature on planning-scheduling often relies on %optimal control and 
  optimization {\color{blue} as well as heuristics-}related approaches~[{\color{green}13}],~[{\color{green}14}],~[{\color{green}19}],~[{\color{green}23}]. We similarly derive an optimal control problem {\color{blue}and a greedy approach} returning the trajectory of parameters [...] {\color{blue} We utilize MPC to derive the trajectory of the computation parameters and the greedy approach with heuristics remaining coverage time for the path parameters.}

  An optimal control problem (OCP) that selects the highest configuration of {\color{blue} $c_i^\rho$} and respects the constraints [...]

  Line~{\color{red}16} in Algorithm~{\color{red}2} contains [...] a nonlinear program (NLP) that can be solved with available NLP solvers~[{\color{green}47}]. Its solution leads to both trajectories of {\color{blue} computation} parameters and states [...]
  
  Lines~{\color{red}17}--{\color{red}23} estimate the time [...] to [...] drain the battery, [...] The {\color{blue}path parameters and thus the} coverage is then re{\color{blue}-}planned accordingly on Lines~{\color{red}24}--{\color{red}26} using {\color{blue} the heuristics with the} %Lem.~\ref{lem:new} and 
  scaling factors from Eq.~({\color{red}11}) [...]

  \vspace*{1ex}
  \end{formal}

  Finally, we have altered the conclusion.

  \begin{formal}
  \color{black}
  Further directions include {\color{blue}the use of a purely optimization-based technique, e.g., MPC derives both the path and computation parameters trajectories and} the study of different energy models.
  \vspace*{1ex}
  \end{formal}
}

{\hspace*{-4.5em}\textbf{[R1:3]}\vspace*{-1.9em}}

3. It is not clear how the parameters of the initial path are obtained. Do the authors use random parameters from the constraint set, or use the highest configuration of parameters? Could you explain the reason for not using the energy-aware planning for the initial trajectory?

{\color{blue}

{\hspace*{-4.5em}{[R1:3]}\vspace*{-1.9em}}

Thank you for observing this omission. We have utilized the highest and the lowest configurations of parameters for the experiments denoted with roman numerals i, I and ii, II respectively, simulating planning-scheduling in the best- and worst-case scenarios. Nonetheless, we do agree it might be advantageous to utilize the algorithm in an ideal simulation first, to estimate the initial parameters.

We have explicitly stated how we obtain the initial configurations in Section V and outlined the possibility of utilizing the methodology to estimate the initial parameters.

\begin{formal}
  \color{black} [...] {\color{blue} The initial values of path and computation parameters are chosen to represent the highest and lowest configurations in the search space in {\color{red}I}--{\color{red}i} and {\color{red}II}--{\color{red}ii} respectively, modeling the behavior of the best- and worst-case scenario. Different search strategies are possible by, e.g., running an ideal instance of planning-scheduling prior to the flight.}
  \vspace*{1ex}
\end{formal}

}

{\hspace*{-4.5em}\textbf{[R1:4]}\vspace*{-1.9em}}


4. The simulation shows some promising results. However, the organization of Fig.~6 makes it difficult to read. For example, what is the top right plot in Fig.~6(b) showing? I understand that there are a number of figures to present and the space is limited, but I would suggest adding legends and axis titles to the plots if possible. You could also consider rewriting the caption of Fig.~6 to make it descriptive enough to be understood. Otherwise, readers have to repeatedly refer to the main text to understand what a specific plot is presenting.

{\color{blue} 


{\hspace*{-4.5em}{[R1:4]}\vspace*{-1.9em}}

Thank you for the concern regarding the readability of the figure. We have split the figure to address the concern into two Figures, 6 and 7. Subfigures that are related are then on a separate row, each for one case. I,II in Figure 6 are related to CPP at boundary values of path parameters and varying atmospheric conditions, and i,ii Figure 7 are related to planning-scheduling of I,II with varying battery conditions. We have rewritten the captions, adding additional commentary to ease the understanding.

\begin{formal}
  \footnotesize
  
% ! replace blue with black
\definecolor{cd9d9d9}{RGB}{235,235,235}
\definecolor{cffffff}{RGB}{255,255,255}
\definecolor{ca0a0a4}{RGB}{160,160,164}
\definecolor{ce10000}{RGB}{225,0,0}
\definecolor{cff0000}{RGB}{255,0,0}


\def \globalscale {0.940000}
\begin{tikzpicture}[y=0.80pt, x=0.80pt, yscale=-1*\globalscale, xscale=1*\globalscale, inner sep=0pt, outer sep=0pt]
\color{blue}
\begin{scope}[shift={(598.26327,165.72719)},draw=blue,even odd rule,line cap=rect,line join=bevel,line width=0.800pt]
  \path[draw=cd9d9d9,line cap=butt,line join=miter,line width=1.440pt,miter limit=4.00] (-302.5903,-136.2302) -- (-281.2030,-152.5408);



  \path[draw=cd9d9d9,line cap=butt,line join=miter,line width=1.440pt,miter limit=4.00] (-303.0158,-64.4976) -- (-281.5953,-47.1121);



  \path[fill=cd9d9d9,dash pattern=on 0.99pt off 0.99pt,even odd rule,line cap=round,line width=0.248pt,miter limit=4.00,rounded corners=0.0000cm] (-617.3423,-41.6805) rectangle (-187.4691,79.0929);



  \path[draw=cffffff,line cap=butt,line join=miter,line width=1.440pt,miter limit=4.00] (-302.9246,46.2884) -- (-281.1456,66.6309);



  \begin{scope}[draw=blue,line cap=rect,line join=bevel,line width=0.800pt]
  \end{scope}
  \begin{scope}[scale=1.006,draw=blue,line cap=rect,line join=bevel,line width=0.800pt]
  \end{scope}
  \begin{scope}[scale=1.006,draw=blue,line cap=rect,line join=bevel,line width=0.800pt]
  \end{scope}
  \begin{scope}[cm={{1.00588,0.0,0.0,1.00588,(39.2294,93.5471)}},draw=blue,line cap=rect,line join=bevel,line width=0.800pt]
  \end{scope}
  \begin{scope}[cm={{1.00588,0.0,0.0,1.00588,(39.2294,93.5471)}},draw=blue,line cap=rect,line join=bevel,line width=0.800pt]
  \end{scope}
  \begin{scope}[cm={{1.00588,0.0,0.0,1.00588,(39.2294,93.5471)}},draw=blue,line cap=rect,line join=bevel,line width=0.800pt]
  \end{scope}
  \begin{scope}[cm={{1.00588,0.0,0.0,1.00588,(39.2294,93.5471)}},draw=blue,line cap=rect,line join=bevel,line width=0.800pt]
  \end{scope}
  \begin{scope}[cm={{1.00588,0.0,0.0,1.00588,(39.2294,93.5471)}},draw=blue,line cap=rect,line join=bevel,line width=0.800pt]
  \end{scope}
  \begin{scope}[cm={{1.00588,0.0,0.0,1.00588,(-426.03325,-54.43023)}},draw=blue,line cap=rect,line join=bevel,line width=0.800pt]
    \path[fill=blue] (0.0000,0.0000) node[above right] (text34) {32};



  \end{scope}
  \begin{scope}[cm={{1.00588,0.0,0.0,1.00588,(39.2294,93.5471)}},draw=blue,line cap=rect,line join=bevel,line width=0.800pt]
  \end{scope}
  \begin{scope}[scale=1.006,draw=blue,line cap=rect,line join=bevel,line width=0.800pt]
  \end{scope}
  \begin{scope}[scale=1.006,draw=blue,line cap=rect,line join=bevel,line width=0.800pt]
  \end{scope}
  \begin{scope}[cm={{1.00588,0.0,0.0,1.00588,(39.2294,68.4)}},draw=blue,line cap=rect,line join=bevel,line width=0.800pt]
  \end{scope}
  \begin{scope}[cm={{1.00588,0.0,0.0,1.00588,(39.2294,68.4)}},draw=blue,line cap=rect,line join=bevel,line width=0.800pt]
  \end{scope}
  \begin{scope}[cm={{1.00588,0.0,0.0,1.00588,(39.2294,68.4)}},draw=blue,line cap=rect,line join=bevel,line width=0.800pt]
  \end{scope}
  \begin{scope}[cm={{1.00588,0.0,0.0,1.00588,(39.2294,68.4)}},draw=blue,line cap=rect,line join=bevel,line width=0.800pt]
  \end{scope}
  \begin{scope}[cm={{1.00588,0.0,0.0,1.00588,(39.2294,68.4)}},draw=blue,line cap=rect,line join=bevel,line width=0.800pt]
  \end{scope}
  \begin{scope}[cm={{1.00588,0.0,0.0,1.00588,(-425.97692,-85.57732)}},draw=blue,line cap=rect,line join=bevel,line width=0.800pt]
    \path[fill=blue] (0.0000,0.0000) node[above right] (text64) {36};



  \end{scope}
  \begin{scope}[cm={{1.00588,0.0,0.0,1.00588,(39.2294,68.4)}},draw=blue,line cap=rect,line join=bevel,line width=0.800pt]
  \end{scope}
  \begin{scope}[scale=1.006,draw=blue,line cap=rect,line join=bevel,line width=0.800pt]
  \end{scope}
  \begin{scope}[scale=1.006,draw=blue,line cap=rect,line join=bevel,line width=0.800pt]
  \end{scope}
  \begin{scope}[cm={{1.00588,0.0,0.0,1.00588,(39.2294,43.2529)}},draw=blue,line cap=rect,line join=bevel,line width=0.800pt]
  \end{scope}
  \begin{scope}[cm={{1.00588,0.0,0.0,1.00588,(39.2294,43.2529)}},draw=blue,line cap=rect,line join=bevel,line width=0.800pt]
  \end{scope}
  \begin{scope}[cm={{1.00588,0.0,0.0,1.00588,(39.2294,43.2529)}},draw=blue,line cap=rect,line join=bevel,line width=0.800pt]
  \end{scope}
  \begin{scope}[cm={{1.00588,0.0,0.0,1.00588,(39.2294,43.2529)}},draw=blue,line cap=rect,line join=bevel,line width=0.800pt]
  \end{scope}
  \begin{scope}[cm={{1.00588,0.0,0.0,1.00588,(39.2294,43.2529)}},draw=blue,line cap=rect,line join=bevel,line width=0.800pt]
  \end{scope}
  \begin{scope}[cm={{1.00588,0.0,0.0,1.00588,(-426.0413,-118.22444)}},draw=blue,line cap=rect,line join=bevel,line width=0.800pt]
    \path[fill=blue] (0.0000,0.0000) node[above right] (text94) {40};



  \end{scope}
  \begin{scope}[cm={{1.00588,0.0,0.0,1.00588,(39.2294,43.2529)}},draw=blue,line cap=rect,line join=bevel,line width=0.800pt]
  \end{scope}
  \begin{scope}[scale=1.006,draw=blue,line cap=rect,line join=bevel,line width=0.800pt]
  \end{scope}
  \begin{scope}[scale=1.006,draw=blue,line cap=rect,line join=bevel,line width=0.800pt]
  \end{scope}
  \begin{scope}[cm={{1.00588,0.0,0.0,1.00588,(53.3118,110.647)}},draw=blue,line cap=rect,line join=bevel,line width=0.800pt]
  \end{scope}
  \begin{scope}[cm={{1.00588,0.0,0.0,1.00588,(53.3118,110.647)}},draw=blue,line cap=rect,line join=bevel,line width=0.800pt]
  \end{scope}
  \begin{scope}[cm={{1.00588,0.0,0.0,1.00588,(53.3118,110.647)}},draw=blue,line cap=rect,line join=bevel,line width=0.800pt]
  \end{scope}
  \begin{scope}[cm={{1.00588,0.0,0.0,1.00588,(53.3118,110.647)}},draw=blue,line cap=rect,line join=bevel,line width=0.800pt]
  \end{scope}
  \begin{scope}[cm={{1.00588,0.0,0.0,1.00588,(53.3118,110.647)}},draw=blue,line cap=rect,line join=bevel,line width=0.800pt]
  \end{scope}
  \begin{scope}[cm={{1.00588,0.0,0.0,1.00588,(-170.37572,77.03586)}},draw=blue,line cap=rect,line join=bevel,line width=0.800pt]
    \path[fill=blue] (0.0000,0.0000) node[above right] (text124) {\scriptsize 0};



  \end{scope}
  \begin{scope}[cm={{1.00588,0.0,0.0,1.00588,(53.3118,110.647)}},draw=blue,line cap=rect,line join=bevel,line width=0.800pt]
  \end{scope}
  \begin{scope}[scale=1.006,draw=blue,line cap=rect,line join=bevel,line width=0.800pt]
  \end{scope}
  \begin{scope}[scale=1.006,draw=blue,line cap=rect,line join=bevel,line width=0.800pt]
  \end{scope}
  \begin{scope}[cm={{1.00588,0.0,0.0,1.00588,(79.4647,110.647)}},draw=blue,line cap=rect,line join=bevel,line width=0.800pt]
  \end{scope}
  \begin{scope}[cm={{1.00588,0.0,0.0,1.00588,(79.4647,110.647)}},draw=blue,line cap=rect,line join=bevel,line width=0.800pt]
  \end{scope}
  \begin{scope}[cm={{1.00588,0.0,0.0,1.00588,(79.4647,110.647)}},draw=blue,line cap=rect,line join=bevel,line width=0.800pt]
  \end{scope}
  \begin{scope}[cm={{1.00588,0.0,0.0,1.00588,(79.4647,110.647)}},draw=blue,line cap=rect,line join=bevel,line width=0.800pt]
  \end{scope}
  \begin{scope}[cm={{1.00588,0.0,0.0,1.00588,(79.4647,110.647)}},draw=blue,line cap=rect,line join=bevel,line width=0.800pt]
  \end{scope}
  \begin{scope}[cm={{1.00588,0.0,0.0,1.00588,(-144.22284,77.03586)}},draw=blue,line cap=rect,line join=bevel,line width=0.800pt]
    \path[fill=blue] (0.0000,0.0000) node[above right] (text156) {\scriptsize 1};



  \end{scope}
  \begin{scope}[cm={{1.00588,0.0,0.0,1.00588,(79.4647,110.647)}},draw=blue,line cap=rect,line join=bevel,line width=0.800pt]
  \end{scope}
  \begin{scope}[scale=1.006,draw=blue,line cap=rect,line join=bevel,line width=0.800pt]
  \end{scope}
  \begin{scope}[scale=1.006,draw=blue,line cap=rect,line join=bevel,line width=0.800pt]
  \end{scope}
  \begin{scope}[cm={{1.00588,0.0,0.0,1.00588,(105.618,110.647)}},draw=blue,line cap=rect,line join=bevel,line width=0.800pt]
  \end{scope}
  \begin{scope}[cm={{1.00588,0.0,0.0,1.00588,(105.618,110.647)}},draw=blue,line cap=rect,line join=bevel,line width=0.800pt]
  \end{scope}
  \begin{scope}[cm={{1.00588,0.0,0.0,1.00588,(105.618,110.647)}},draw=blue,line cap=rect,line join=bevel,line width=0.800pt]
  \end{scope}
  \begin{scope}[cm={{1.00588,0.0,0.0,1.00588,(105.618,110.647)}},draw=blue,line cap=rect,line join=bevel,line width=0.800pt]
  \end{scope}
  \begin{scope}[cm={{1.00588,0.0,0.0,1.00588,(105.618,110.647)}},draw=blue,line cap=rect,line join=bevel,line width=0.800pt]
  \end{scope}
  \begin{scope}[cm={{1.00588,0.0,0.0,1.00588,(-118.06953,77.03586)}},draw=blue,line cap=rect,line join=bevel,line width=0.800pt]
    \path[fill=blue] (0.0000,0.0000) node[above right] (text188) {\scriptsize 2};



  \end{scope}
  \begin{scope}[cm={{1.00588,0.0,0.0,1.00588,(105.618,110.647)}},draw=blue,line cap=rect,line join=bevel,line width=0.800pt]
  \end{scope}
  \begin{scope}[scale=1.006,draw=blue,line cap=rect,line join=bevel,line width=0.800pt]
  \end{scope}
  \begin{scope}[scale=1.006,draw=blue,line cap=rect,line join=bevel,line width=0.800pt]
  \end{scope}
  \begin{scope}[cm={{1.00588,0.0,0.0,1.00588,(132.274,110.647)}},draw=blue,line cap=rect,line join=bevel,line width=0.800pt]
  \end{scope}
  \begin{scope}[cm={{1.00588,0.0,0.0,1.00588,(132.274,110.647)}},draw=blue,line cap=rect,line join=bevel,line width=0.800pt]
  \end{scope}
  \begin{scope}[cm={{1.00588,0.0,0.0,1.00588,(132.274,110.647)}},draw=blue,line cap=rect,line join=bevel,line width=0.800pt]
  \end{scope}
  \begin{scope}[cm={{1.00588,0.0,0.0,1.00588,(132.274,110.647)}},draw=blue,line cap=rect,line join=bevel,line width=0.800pt]
  \end{scope}
  \begin{scope}[cm={{1.00588,0.0,0.0,1.00588,(132.274,110.647)}},draw=blue,line cap=rect,line join=bevel,line width=0.800pt]
  \end{scope}
  \begin{scope}[cm={{1.00588,0.0,0.0,1.00588,(-91.41353,-12.96416)}},draw=blue,line cap=rect,line join=bevel,line width=0.800pt]
    \path[fill=blue] (0.0000,89.4739) node[above right] (text218) {\scriptsize 3};



  \end{scope}
  \begin{scope}[cm={{1.00588,0.0,0.0,1.00588,(132.274,110.647)}},draw=blue,line cap=rect,line join=bevel,line width=0.800pt]
  \end{scope}
  \begin{scope}[scale=1.006,draw=blue,line cap=rect,line join=bevel,line width=0.800pt]
  \end{scope}
  \begin{scope}[scale=1.006,draw=blue,line cap=rect,line join=bevel,line width=0.800pt]
  \end{scope}
  \begin{scope}[scale=1.006,draw=blue,line cap=rect,line join=bevel,line width=0.800pt]
  \end{scope}
  \begin{scope}[scale=1.006,draw=blue,line cap=rect,line join=bevel,line width=0.800pt]
  \end{scope}
  \begin{scope}[scale=1.006,draw=blue,line cap=rect,line join=bevel,line width=0.800pt]
  \end{scope}
  \begin{scope}[scale=1.006,draw=blue,line cap=rect,line join=bevel,line width=0.800pt]
  \end{scope}
  \begin{scope}[cm={{1.00588,0.0,0.0,1.00588,(128.753,29.1706)}},draw=blue,line cap=rect,line join=bevel,line width=0.800pt]
  \end{scope}
  \begin{scope}[cm={{1.00588,0.0,0.0,1.00588,(128.753,29.1706)}},draw=blue,line cap=rect,line join=bevel,line width=0.800pt]
  \end{scope}
  \begin{scope}[cm={{1.00588,0.0,0.0,1.00588,(128.753,29.1706)}},draw=blue,line cap=rect,line join=bevel,line width=0.800pt]
  \end{scope}
  \begin{scope}[cm={{1.00588,0.0,0.0,1.00588,(128.753,29.1706)}},draw=blue,line cap=rect,line join=bevel,line width=0.800pt]
  \end{scope}
  \begin{scope}[cm={{1.00588,0.0,0.0,1.00588,(128.753,29.1706)}},draw=blue,line cap=rect,line join=bevel,line width=0.800pt]
  \end{scope}
  \begin{scope}[cm={{1.00588,0.0,0.0,1.00588,(128.753,29.1706)}},draw=blue,line cap=rect,line join=bevel,line width=0.800pt]
  \end{scope}
  \begin{scope}[cm={{0.0,-1.00588,1.00588,0.0,(29.1706,189.106)}},draw=blue,line cap=rect,line join=bevel,line width=0.800pt]
  \end{scope}
  \begin{scope}[cm={{0.0,-1.00588,1.00588,0.0,(29.1706,189.106)}},draw=blue,line cap=rect,line join=bevel,line width=0.800pt]
  \end{scope}
  \begin{scope}[cm={{0.0,-1.00588,1.00588,0.0,(29.1706,189.106)}},draw=blue,line cap=rect,line join=bevel,line width=0.800pt]
  \end{scope}
  \begin{scope}[cm={{0.0,-1.00588,1.00588,0.0,(29.1706,189.106)}},draw=blue,line cap=rect,line join=bevel,line width=0.800pt]
  \end{scope}
  \begin{scope}[cm={{0.0,-1.00588,1.00588,0.0,(29.1706,189.106)}},draw=blue,line cap=rect,line join=bevel,line width=0.800pt]
  \end{scope}
  \begin{scope}[cm={{0.0,-1.00588,1.00588,0.0,(29.1706,189.106)}},draw=blue,line cap=rect,line join=bevel,line width=0.800pt]
  \end{scope}
  \begin{scope}[cm={{1.00588,0.0,0.0,1.00588,(62.3647,28.1647)}},draw=blue,line cap=rect,line join=bevel,line width=0.800pt]
  \end{scope}
  \begin{scope}[cm={{1.00588,0.0,0.0,1.00588,(62.3647,28.1647)}},draw=blue,line cap=rect,line join=bevel,line width=0.800pt]
  \end{scope}
  \begin{scope}[cm={{1.00588,0.0,0.0,1.00588,(62.3647,28.1647)}},draw=blue,line cap=rect,line join=bevel,line width=0.800pt]
  \end{scope}
  \begin{scope}[cm={{1.00588,0.0,0.0,1.00588,(62.3647,28.1647)}},draw=blue,line cap=rect,line join=bevel,line width=0.800pt]
  \end{scope}
  \begin{scope}[cm={{1.00588,0.0,0.0,1.00588,(62.3647,28.1647)}},draw=blue,line cap=rect,line join=bevel,line width=0.800pt]
  \end{scope}
  \begin{scope}[cm={{1.00588,0.0,0.0,1.00588,(62.3647,28.1647)}},draw=blue,line cap=rect,line join=bevel,line width=0.800pt]
  \end{scope}
  \begin{scope}[scale=1.006,draw=blue,line cap=rect,line join=bevel,line width=0.800pt]
  \end{scope}
  \begin{scope}[scale=1.006,draw=blue,line cap=rect,line join=bevel,line width=0.800pt]
  \end{scope}
  \begin{scope}[scale=1.006,draw=blue,line cap=rect,line join=bevel,line width=0.800pt]
  \end{scope}
  \begin{scope}[scale=1.006,draw=blue,line cap=rect,line join=bevel,line width=0.800pt]
  \end{scope}
  \begin{scope}[scale=1.006,draw=blue,line cap=rect,line join=bevel,line width=0.800pt]
  \end{scope}
  \begin{scope}[scale=1.006,draw=blue,line cap=rect,line join=bevel,line width=0.800pt]
  \end{scope}
  \begin{scope}[cm={{1.00588,0.0,0.0,1.00588,(60.3529,36.2118)}},draw=blue,line cap=rect,line join=bevel,line width=0.800pt]
  \end{scope}
  \begin{scope}[cm={{1.00588,0.0,0.0,1.00588,(60.3529,36.2118)}},draw=blue,line cap=rect,line join=bevel,line width=0.800pt]
  \end{scope}
  \begin{scope}[cm={{1.00588,0.0,0.0,1.00588,(60.3529,36.2118)}},draw=blue,line cap=rect,line join=bevel,line width=0.800pt]
  \end{scope}
  \begin{scope}[cm={{1.00588,0.0,0.0,1.00588,(60.3529,36.2118)}},draw=blue,line cap=rect,line join=bevel,line width=0.800pt]
  \end{scope}
  \begin{scope}[cm={{1.00588,0.0,0.0,1.00588,(60.3529,36.2118)}},draw=blue,line cap=rect,line join=bevel,line width=0.800pt]
  \end{scope}
  \begin{scope}[cm={{1.00588,0.0,0.0,1.00588,(60.3529,36.2118)}},draw=blue,line cap=rect,line join=bevel,line width=0.800pt]
  \end{scope}
  \begin{scope}[scale=1.006,draw=blue,line cap=rect,line join=bevel,line width=0.800pt]
  \end{scope}
  \begin{scope}[scale=1.006,draw=blue,line cap=rect,line join=bevel,line width=0.800pt]
  \end{scope}
  \begin{scope}[scale=1.006,draw=blue,line cap=rect,line join=bevel,line width=0.800pt]
  \end{scope}
  \begin{scope}[scale=1.006,draw=blue,line cap=rect,line join=bevel,line width=0.800pt]
  \end{scope}
  \begin{scope}[scale=1.006,draw=blue,line cap=rect,line join=bevel,line width=0.800pt]
  \end{scope}
  \begin{scope}[scale=1.006,draw=blue,line cap=rect,line join=bevel,line width=0.800pt]
  \end{scope}
  \begin{scope}[scale=1.006,draw=blue,line cap=rect,line join=bevel,line width=0.800pt]
  \end{scope}
  \begin{scope}[scale=1.006,draw=blue,line cap=round,line join=round,line width=0.480pt]
    \path[cm={{1.25275,0.0,0.0,1.25275,(-479.34266,-167.80381)}},draw] (56.5000,13.5000) -- (56.5000,95.5000) -- (142.5000,95.5000) -- (142.5000,13.5000) -- (56.5000,13.5000);



    \begin{scope}[cm={{0.99623,0.0,0.0,1.3704,(-430.13329,-152.91751)}},draw=ca0a0a4,dash pattern=on 1.02pt off 1.02pt,line cap=round,line join=round,line width=0.255pt,miter limit=4.00]
      \path[shift={(110.39113,-5.49717)},draw,dash pattern=on 1.02pt off 1.02pt,line width=0.255pt,miter limit=4.00] (70.5000,164.5000) -- (70.5000,88.5000);



    \end{scope}
    \begin{scope}[cm={{0.99623,0.0,0.0,1.3704,(-320.15845,-159.02501)}},draw=ca0a0a4,dash pattern=on 1.02pt off 1.02pt,line cap=round,line join=round,line width=0.255pt,miter limit=4.00]
      \path[draw,dash pattern=on 1.02pt off 1.02pt,line width=0.255pt,miter limit=4.00] (98.5000,164.5000) -- (98.5000,88.5000);



    \end{scope}
    \begin{scope}[cm={{0.74279,0.0,0.0,1.28515,(-186.22138,-161.30028)}},draw=ca0a0a4,dash pattern=on 1.22pt off 1.22pt,line cap=round,line join=round,line width=0.305pt,miter limit=4.00]
      \path[draw,dash pattern=on 1.22pt off 1.22pt,line width=0.305pt,miter limit=4.00] (25.5000,26.5000) -- (108.5000,26.5000);



      \path[draw,dash pattern=on 1.22pt off 1.22pt,line width=0.305pt,miter limit=4.00] (137.5000,26.5000) -- (142.5000,26.5000);



    \end{scope}
    \begin{scope}[cm={{0.74279,0.0,0.0,1.28515,(-186.22138,-161.30028)}},draw=blue,line cap=round,line join=round,line width=0.480pt]
      \path[cm={{1.54975,0.0,0.0,1.0,(-13.85377,0.0)}},draw] (25.5000,26.5000) -- (28.5000,26.5000);



      \path[cm={{1.54975,0.0,0.0,1.0,(-78.53317,0.0)}},draw] (142.5000,26.5000) -- (139.5000,26.5000);



    \end{scope}
    \begin{scope}[cm={{0.95389,0.0,0.0,0.95389,(-180.91222,-124.57711)}},draw=blue,fill=ce10000,line cap=rect,line join=bevel,line width=0.800pt]
      \path[fill=ce10000] (0.0000,0.0000) node[above right] (text36) {\scriptsize 30};



    \end{scope}
    \begin{scope}[cm={{0.74279,0.0,0.0,1.28515,(-186.22138,-161.30028)}},draw=ca0a0a4,dash pattern=on 1.22pt off 1.22pt,line cap=round,line join=round,line width=0.305pt,miter limit=4.00]
      \path[draw,dash pattern=on 1.22pt off 1.22pt,line width=0.305pt,miter limit=4.00] (25.5000,11.5000) -- (142.5000,11.5000);



    \end{scope}
    \begin{scope}[cm={{0.74279,0.0,0.0,1.28515,(-186.22138,-161.30028)}},draw=blue,line cap=round,line join=round,line width=0.480pt]
      \path[cm={{1.54975,0.0,0.0,1.0,(-13.85377,0.0)}},draw] (25.5000,11.5000) -- (28.5000,11.5000);



      \path[cm={{1.54975,0.0,0.0,1.0,(-78.53317,0.0)}},draw] (142.5000,11.5000) -- (139.5000,11.5000);



    \end{scope}
    \begin{scope}[cm={{0.95389,0.0,0.0,0.95389,(-180.91222,-144.27915)}},draw=blue,fill=ce10000,line cap=rect,line join=bevel,line width=0.800pt]
      \path[fill=ce10000] (0.0000,0.0000) node[above right] (text66) {\scriptsize 40};



    \end{scope}
    \begin{scope}[cm={{0.74279,0.0,0.0,1.28515,(-186.22138,-161.30028)}},draw=ca0a0a4,dash pattern=on 0.40pt off 0.80pt,line cap=round,line join=round,line width=0.400pt]
      \path[draw] (25.5000,32.5000) -- (25.5000,8.5000);



    \end{scope}
    \begin{scope}[cm={{0.74279,0.0,0.0,1.28515,(-186.22138,-161.30028)}},draw=blue,line cap=round,line join=round,line width=0.480pt]
      \path[draw] (25.5000,32.5000) -- (25.5000,28.5000);



      \path[draw] (25.5000,8.5000) -- (25.5000,11.5000);



    \end{scope}
    \begin{scope}[cm={{0.74279,0.0,0.0,1.28515,(-186.22138,-161.30028)}},draw=ca0a0a4,dash pattern=on 1.22pt off 1.22pt,line cap=round,line join=round,line width=0.305pt,miter limit=4.00]
      \path[draw,dash pattern=on 1.22pt off 1.22pt,line width=0.305pt,miter limit=4.00] (60.5000,32.5000) -- (60.5000,8.5000);



    \end{scope}
    \begin{scope}[cm={{0.74279,0.0,0.0,1.28515,(-186.22138,-161.30028)}},draw=blue,line cap=round,line join=round,line width=0.480pt]
      \path[cm={{1.0,0.0,0.0,0.70101,(0.0,9.59194)}},draw] (60.5000,32.5000) -- (60.5000,28.5000);



      \path[cm={{1.0,0.0,0.0,0.89573,(0.0,0.855)}},draw] (60.5000,8.5000) -- (60.5000,11.5000);



    \end{scope}
    \begin{scope}[cm={{0.74279,0.0,0.0,1.28515,(-186.22138,-161.30028)}},draw=ca0a0a4,dash pattern=on 1.22pt off 1.22pt,line cap=round,line join=round,line width=0.305pt,miter limit=4.00]
      \path[draw,dash pattern=on 1.22pt off 1.22pt,line width=0.305pt,miter limit=4.00] (95.5000,32.5000) -- (95.5000,8.5000);



    \end{scope}
    \begin{scope}[cm={{0.74279,0.0,0.0,1.28515,(-186.22138,-161.30028)}},draw=blue,line cap=round,line join=round,line width=0.480pt]
      \path[cm={{1.0,0.0,0.0,0.70101,(0.0,9.59194)}},draw] (95.5000,32.5000) -- (95.5000,28.5000);



      \path[cm={{1.0,0.0,0.0,0.89573,(0.0,0.855)}},draw] (95.5000,8.5000) -- (95.5000,11.5000);



    \end{scope}
    \begin{scope}[cm={{0.74279,0.0,0.0,1.28515,(-186.22138,-161.30028)}},draw=ca0a0a4,dash pattern=on 1.22pt off 1.22pt,line cap=round,line join=round,line width=0.305pt,miter limit=4.00]
      \path[draw,dash pattern=on 1.22pt off 1.22pt,line width=0.305pt,miter limit=4.00] (130.5000,20.5000) -- (130.5000,8.5000);



    \end{scope}
    \begin{scope}[cm={{0.74279,0.0,0.0,1.28515,(-186.22138,-161.30028)}},draw=blue,line cap=round,line join=round,line width=0.480pt]
      \path[cm={{1.0,0.0,0.0,0.70101,(0.0,9.59194)}},draw] (130.5000,32.5000) -- (130.5000,28.5000);



      \path[cm={{1.0,0.0,0.0,0.89573,(0.0,0.855)}},draw] (130.5000,8.5000) -- (130.5000,11.5000);



    \end{scope}
    \begin{scope}[cm={{0.74279,0.0,0.0,1.28515,(-186.22138,-161.30028)}},draw=blue,line cap=round,line join=round,line width=0.480pt]
      \path[draw] (25.5000,8.5000) -- (25.5000,32.5000) -- (142.5000,32.5000) -- (142.5000,8.5000) -- (25.5000,8.5000);



    \end{scope}
    \begin{scope}[cm={{0.95389,0.0,0.0,0.95389,(-104.99275,-127.89091)}},draw=blue,line cap=rect,line join=bevel,line width=0.800pt]
      \path[fill=blue] (0.0000,0.0000) node[above right] (text194) {\scriptsize $\alpha_0$};



    \end{scope}
    \begin{scope}[cm={{0.74279,0.0,0.0,1.28515,(-184.07401,-166.61499)}},draw=blue,line cap=round,line join=round,line width=0.480pt]
      \path[draw,even odd rule] (123.5000,28.5000) -- (132.5000,28.5000);



    \end{scope}
    \begin{scope}[cm={{0.74279,0.0,0.0,1.28515,(-186.22138,-161.30028)}},draw=blue,line cap=round,line join=round,line width=0.480pt]
      \path[draw] (25.8000,32.0000) -- (26.2000,14.3000) -- (26.7000,17.2000) -- (27.1000,18.1000) -- (27.5000,14.6000) -- (27.9000,12.6000) -- (28.3000,14.9000) -- (28.8000,17.3000) -- (29.2000,10.3000) -- (29.6000,14.5000) -- (30.0000,19.1000) -- (30.5000,15.7000) -- (30.9000,12.7000) -- (31.3000,15.0000) -- (31.7000,18.7000) -- (32.2000,19.3000) -- (32.6000,16.6000) -- (33.0000,13.7000) -- (33.4000,12.8000) -- (33.8000,14.3000) -- (34.3000,16.9000) -- (34.7000,18.9000) -- (35.1000,19.4000) -- (35.5000,18.6000) -- (36.0000,17.2000) -- (36.4000,15.9000) -- (36.8000,15.5000) -- (37.2000,15.9000) -- (37.6000,17.1000) -- (38.1000,18.6000) -- (38.5000,20.0000) -- (38.9000,21.0000) -- (39.3000,21.3000) -- (39.8000,21.0000) -- (40.2000,20.3000) -- (40.6000,19.3000) -- (41.0000,18.3000) -- (41.5000,17.4000) -- (41.9000,16.9000) -- (42.3000,16.8000) -- (42.7000,17.1000) -- (43.1000,17.6000) -- (43.6000,18.3000) -- (44.0000,19.0000) -- (44.4000,19.7000) -- (44.8000,20.1000) -- (45.3000,20.4000) -- (45.7000,20.5000) -- (46.1000,20.4000) -- (46.5000,20.1000) -- (47.0000,19.8000) -- (47.4000,19.3000) -- (47.8000,18.5000) -- (48.2000,17.8000) -- (48.6000,17.2000) -- (49.1000,16.7000) -- (49.5000,16.5000) -- (49.9000,16.4000) -- (50.3000,16.4000) -- (50.8000,16.5000) -- (51.2000,16.7000) -- (51.6000,17.0000) -- (52.0000,17.2000) -- (52.4000,17.4000) -- (52.9000,17.5000) -- (53.3000,17.5000) -- (53.7000,17.3000) -- (54.1000,17.0000) -- (54.6000,16.6000) -- (55.0000,16.1000) -- (55.4000,15.6000) -- (55.8000,15.0000) -- (56.3000,14.4000) -- (56.7000,13.9000) -- (57.1000,13.5000) -- (57.5000,13.2000) -- (57.9000,13.2000) -- (58.4000,13.3000) -- (58.8000,13.5000) -- (59.2000,13.9000) -- (59.6000,14.3000) -- (60.1000,14.7000) -- (60.5000,15.0000) -- (60.9000,15.4000) -- (61.3000,15.7000) -- (61.8000,16.0000) -- (62.2000,16.2000) -- (62.6000,16.3000) -- (63.0000,16.2000) -- (63.4000,16.2000) -- (63.9000,16.1000) -- (64.3000,16.0000) -- (64.7000,15.9000) -- (65.1000,15.9000) -- (65.6000,15.9000) -- (66.0000,16.0000) -- (66.4000,16.2000) -- (66.8000,16.4000) -- (67.3000,16.7000) -- (67.7000,16.9000) -- (68.1000,17.2000) -- (68.5000,17.3000) -- (68.9000,17.4000) -- (69.4000,17.5000) -- (69.8000,17.7000) -- (70.2000,17.7000) -- (70.6000,17.7000) -- (71.1000,17.6000) -- (71.5000,17.5000) -- (71.9000,17.3000) -- (72.3000,17.1000) -- (72.7000,17.0000) -- (73.2000,16.9000) -- (73.6000,16.8000) -- (74.0000,16.9000) -- (74.4000,17.0000) -- (74.9000,16.9000) -- (75.3000,16.8000) -- (75.7000,16.8000) -- (76.1000,16.8000) -- (76.6000,16.9000) -- (77.0000,17.1000) -- (77.4000,17.2000) -- (77.8000,17.3000) -- (78.2000,17.4000) -- (78.7000,17.4000) -- (79.1000,17.4000) -- (79.5000,17.4000) -- (79.9000,17.4000) -- (80.4000,17.3000) -- (80.8000,17.2000) -- (81.2000,17.1000) -- (81.6000,17.1000) -- (82.1000,17.0000) -- (82.5000,17.0000) -- (82.9000,16.9000) -- (83.3000,16.7000) -- (83.7000,16.6000) -- (84.2000,16.4000) -- (84.6000,16.4000) -- (85.0000,16.3000) -- (85.4000,16.3000) -- (85.9000,16.3000) -- (86.3000,16.3000) -- (86.7000,16.2000) -- (87.1000,16.2000) -- (87.6000,16.1000) -- (88.0000,16.1000) -- (88.4000,16.2000) -- (88.8000,16.3000) -- (89.2000,16.3000) -- (89.7000,16.3000) -- (90.1000,16.3000) -- (90.5000,16.3000) -- (90.9000,16.3000) -- (91.4000,16.3000) -- (91.8000,16.3000) -- (92.2000,16.3000) -- (92.6000,16.3000) -- (93.0000,16.2000) -- (93.5000,16.2000) -- (93.9000,16.2000) -- (94.3000,16.2000) -- (94.7000,16.2000) -- (95.2000,16.1000) -- (95.6000,16.1000) -- (96.0000,16.1000) -- (96.4000,16.2000) -- (96.9000,16.4000) -- (97.3000,16.5000) -- (97.7000,16.5000) -- (98.1000,16.6000) -- (98.5000,16.6000) -- (99.0000,16.6000) -- (99.4000,16.6000) -- (99.8000,16.5000) -- (100.2000,16.5000) -- (100.7000,16.6000) -- (101.1000,16.7000) -- (101.5000,16.6000) -- (101.9000,16.5000) -- (102.3000,16.3000) -- (102.8000,16.3000) -- (103.2000,16.3000) -- (103.6000,16.3000) -- (104.0000,16.3000) -- (104.5000,16.3000) -- (104.9000,16.3000) -- (105.3000,16.3000) -- (105.7000,16.2000) -- (106.2000,16.2000) -- (106.6000,16.2000) -- (107.0000,16.2000) -- (107.4000,16.2000) -- (107.8000,16.2000) -- (108.3000,16.3000) -- (108.7000,16.3000) -- (109.1000,16.4000) -- (109.5000,16.5000) -- (110.0000,16.5000) -- (110.4000,16.5000) -- (110.8000,16.5000) -- (111.2000,16.5000) -- (111.7000,16.5000) -- (112.1000,16.6000) -- (112.5000,16.6000) -- (112.9000,16.6000) -- (113.3000,16.5000) -- (113.8000,16.5000) -- (114.2000,16.4000) -- (114.6000,16.3000) -- (115.0000,16.3000) -- (115.5000,16.4000) -- (115.9000,16.5000) -- (116.3000,16.5000) -- (116.7000,16.5000) -- (117.2000,16.5000) -- (117.6000,16.5000) -- (118.0000,16.5000) -- (118.4000,16.5000) -- (118.8000,16.6000) -- (119.3000,16.6000) -- (119.7000,16.6000) -- (120.1000,16.5000) -- (120.5000,16.5000) -- (121.0000,16.5000) -- (121.4000,16.5000) -- (121.8000,16.5000) -- (122.2000,16.3000) -- (122.6000,16.3000) -- (123.1000,16.3000) -- (123.5000,16.4000) -- (123.9000,16.4000) -- (124.3000,16.4000) -- (124.8000,16.4000) -- (125.2000,16.4000) -- (125.6000,16.3000) -- (126.0000,16.3000) -- (126.5000,16.2000) -- (126.9000,16.2000) -- (127.3000,16.3000) -- (127.7000,16.5000) -- (128.1000,16.5000) -- (128.6000,16.4000) -- (129.0000,16.3000) -- (129.4000,16.3000) -- (129.8000,16.4000) -- (130.3000,16.5000) -- (130.7000,16.5000) -- (131.1000,16.6000) -- (131.5000,16.6000) -- (131.9000,16.6000) -- (132.4000,16.6000) -- (132.8000,16.5000) -- (133.2000,16.5000) -- (133.6000,16.4000) -- (134.1000,16.4000) -- (134.5000,16.3000) -- (134.9000,16.3000) -- (135.3000,16.3000) -- (135.8000,16.4000) -- (136.2000,16.4000) -- (136.6000,16.4000) -- (137.0000,16.4000) -- (137.4000,16.4000) -- (137.9000,16.4000) -- (138.3000,16.4000) -- (138.7000,16.5000) -- (139.1000,16.5000) -- (139.6000,16.5000) -- (140.0000,16.5000) -- (140.4000,16.5000) -- (140.8000,16.4000) -- (141.3000,16.4000) -- (141.7000,16.3000) -- (142.1000,16.4000) -- (142.3000,16.4000);



    \end{scope}
    \begin{scope}[cm={{0.74279,0.0,0.0,1.28515,(-186.22138,-161.30028)}},draw=blue,line cap=round,line join=round,line width=0.480pt]
      \path[draw] (25.5000,8.5000) -- (25.5000,32.5000) -- (142.5000,32.5000) -- (142.5000,8.5000) -- (25.5000,8.5000);



    \end{scope}
    \begin{scope}[cm={{0.74279,0.0,0.0,1.28515,(-186.22138,-161.30028)}},draw=ca0a0a4,dash pattern=on 1.22pt off 1.22pt,line cap=round,line join=round,line width=0.305pt,miter limit=4.00]
      \path[draw,dash pattern=on 1.22pt off 1.22pt,line width=0.305pt,miter limit=4.00] (25.5000,46.5000) -- (108.5000,46.5000);



      \path[draw,dash pattern=on 1.22pt off 1.22pt,line width=0.305pt,miter limit=4.00] (137.5000,46.5000) -- (142.5000,46.5000);



    \end{scope}
    \begin{scope}[cm={{0.74279,0.0,0.0,1.28515,(-186.22138,-161.30028)}},draw=blue,line cap=round,line join=round,line width=0.480pt]
      \path[cm={{1.54975,0.0,0.0,1.0,(-13.85377,0.0)}},draw] (25.5000,46.5000) -- (28.5000,46.5000);



      \path[cm={{1.54975,0.0,0.0,1.0,(-78.53317,0.0)}},draw] (142.5000,46.5000) -- (139.5000,46.5000);



    \end{scope}
    \begin{scope}[cm={{0.95389,0.0,0.0,0.95389,(-183.45337,-99.34397)}},draw=blue,fill=ce10000,line cap=rect,line join=bevel,line width=0.800pt]
      \path[fill=ce10000] (0.0000,0.0000) node[above right] (text250) {\scriptsize -10};



    \end{scope}
    \begin{scope}[cm={{0.74279,0.0,0.0,1.28515,(-186.22138,-161.30028)}},draw=ca0a0a4,dash pattern=on 1.22pt off 1.22pt,line cap=round,line join=round,line width=0.305pt,miter limit=4.00]
      \path[draw,dash pattern=on 1.22pt off 1.22pt,line width=0.305pt,miter limit=4.00] (25.5000,34.5000) -- (142.5000,34.5000);



    \end{scope}
    \begin{scope}[cm={{0.74279,0.0,0.0,1.28515,(-186.22138,-161.30028)}},draw=blue,line cap=round,line join=round,line width=0.480pt]
      \path[cm={{1.54975,0.0,0.0,1.0,(-13.85377,0.0)}},draw] (25.5000,34.5000) -- (28.5000,34.5000);



      \path[cm={{1.54975,0.0,0.0,1.0,(-78.53317,0.0)}},draw] (142.5000,34.5000) -- (139.5000,34.5000);



    \end{scope}
    \begin{scope}[cm={{0.95389,0.0,0.0,0.95389,(-180.91222,-114.12005)}},draw=blue,fill=ce10000,line cap=rect,line join=bevel,line width=0.800pt]
      \path[fill=ce10000] (0.0000,0.0000) node[above right] (text280) {\scriptsize 10};



    \end{scope}
    \begin{scope}[cm={{0.74279,0.0,0.0,1.28515,(-186.22138,-161.30028)}},draw=ca0a0a4,dash pattern=on 0.40pt off 0.80pt,line cap=round,line join=round,line width=0.400pt]
      \path[draw] (25.5000,56.5000) -- (25.5000,32.5000);



    \end{scope}
    \begin{scope}[cm={{0.74279,0.0,0.0,1.28515,(-186.22138,-161.30028)}},draw=blue,line cap=round,line join=round,line width=0.480pt]
      \path[draw] (25.5000,56.5000) -- (25.5000,51.5000);



      \path[draw] (25.5000,32.5000) -- (25.5000,36.5000);



    \end{scope}
    \begin{scope}[cm={{0.74279,0.0,0.0,1.28515,(-186.22138,-161.30028)}},draw=ca0a0a4,dash pattern=on 1.22pt off 1.22pt,line cap=round,line join=round,line width=0.305pt,miter limit=4.00]
      \path[draw,dash pattern=on 1.22pt off 1.22pt,line width=0.305pt,miter limit=4.00] (60.5000,56.5000) -- (60.5000,32.5000);



    \end{scope}
    \begin{scope}[cm={{0.74279,0.0,0.0,1.28515,(-186.22138,-161.30028)}},draw=blue,line cap=round,line join=round,line width=0.480pt]
      \path[cm={{1.0,0.0,0.0,0.57583,(0.0,24.03834)}},draw] (60.5000,56.5000) -- (60.5000,51.5000);



      \path[cm={{1.0,0.0,0.0,0.70101,(0.0,9.62756)}},draw] (60.5000,32.5000) -- (60.5000,36.5000);



    \end{scope}
    \begin{scope}[cm={{0.74279,0.0,0.0,1.28515,(-186.22138,-161.30028)}},draw=ca0a0a4,dash pattern=on 1.22pt off 1.22pt,line cap=round,line join=round,line width=0.305pt,miter limit=4.00]
      \path[draw,dash pattern=on 1.22pt off 1.22pt,line width=0.305pt,miter limit=4.00] (95.5000,56.5000) -- (95.5000,32.5000);



    \end{scope}
    \begin{scope}[cm={{0.74279,0.0,0.0,1.28515,(-186.22138,-161.30028)}},draw=blue,line cap=round,line join=round,line width=0.480pt]
      \path[cm={{1.0,0.0,0.0,0.57583,(0.0,24.03834)}},draw] (95.5000,56.5000) -- (95.5000,51.5000);



      \path[cm={{1.0,0.0,0.0,0.70101,(0.0,9.62756)}},draw] (95.5000,32.5000) -- (95.5000,36.5000);



    \end{scope}
    \begin{scope}[cm={{0.74279,0.0,0.0,1.28515,(-186.22138,-161.30028)}},draw=ca0a0a4,dash pattern=on 1.22pt off 1.22pt,line cap=round,line join=round,line width=0.305pt,miter limit=4.00]
      \path[draw,dash pattern=on 1.22pt off 1.22pt,line width=0.305pt,miter limit=4.00] (130.5000,44.5000) -- (130.5000,32.5000);



    \end{scope}
    \begin{scope}[cm={{0.74279,0.0,0.0,1.28515,(-186.22138,-161.30028)}},draw=blue,line cap=round,line join=round,line width=0.480pt]
      \path[cm={{1.0,0.0,0.0,0.57583,(0.0,24.03834)}},draw] (130.5000,56.5000) -- (130.5000,51.5000);



      \path[cm={{1.0,0.0,0.0,0.70101,(0.0,9.62756)}},draw] (130.5000,32.5000) -- (130.5000,36.5000);



    \end{scope}
    \begin{scope}[cm={{0.74279,0.0,0.0,1.28515,(-186.22138,-161.30028)}},draw=blue,line cap=round,line join=round,line width=0.480pt]
      \path[draw] (25.5000,32.5000) -- (25.5000,56.5000) -- (142.5000,56.5000) -- (142.5000,32.5000) -- (25.5000,32.5000);



    \end{scope}
    \begin{scope}[cm={{0.95389,0.0,0.0,0.95389,(-104.99275,-97.16859)}},draw=blue,line cap=rect,line join=bevel,line width=0.800pt]
      \path[fill=blue] (0.0000,0.0000) node[above right] (text408) {\scriptsize $\alpha_1$};



    \end{scope}
    \begin{scope}[cm={{0.74279,0.0,0.0,1.28515,(-184.07401,-166.61499)}},draw=blue,line cap=round,line join=round,line width=0.480pt]
      \path[draw,even odd rule] (123.5000,52.5000) -- (132.5000,52.5000);



    \end{scope}
    \begin{scope}[cm={{0.74279,0.0,0.0,1.28515,(-186.22138,-161.30028)}},draw=blue,line cap=round,line join=round,line width=0.480pt]
      \path[draw] (25.8000,36.4000) -- (25.8000,36.4000) -- (26.2000,41.9000) -- (26.7000,39.4000) -- (27.1000,40.5000) -- (27.5000,40.1000) -- (27.9000,39.9000) -- (28.3000,40.6000) -- (28.8000,43.1000) -- (29.2000,39.4000) -- (29.6000,39.6000) -- (30.0000,41.8000) -- (30.5000,40.8000) -- (30.9000,39.0000) -- (31.3000,39.5000) -- (31.7000,41.2000) -- (32.2000,42.0000) -- (32.6000,41.2000) -- (33.0000,39.7000) -- (33.4000,38.9000) -- (33.8000,39.1000) -- (34.3000,40.2000) -- (34.7000,41.3000) -- (35.1000,41.9000) -- (35.5000,41.9000) -- (36.0000,41.5000) -- (36.4000,40.9000) -- (36.8000,40.4000) -- (37.2000,40.4000) -- (37.6000,40.7000) -- (38.1000,41.2000) -- (38.5000,41.8000) -- (38.9000,42.2000) -- (39.3000,42.3000) -- (39.8000,42.2000) -- (40.2000,41.8000) -- (40.6000,41.3000) -- (41.0000,40.6000) -- (41.5000,39.9000) -- (41.9000,39.3000) -- (42.3000,38.9000) -- (42.7000,38.7000) -- (43.1000,38.7000) -- (43.6000,38.8000) -- (44.0000,39.0000) -- (44.4000,39.2000) -- (44.8000,39.4000) -- (45.3000,39.6000) -- (45.7000,39.8000) -- (46.1000,39.8000) -- (46.5000,39.8000) -- (47.0000,39.8000) -- (47.4000,39.7000) -- (47.8000,39.5000) -- (48.2000,39.3000) -- (48.6000,39.1000) -- (49.1000,39.0000) -- (49.5000,38.9000) -- (49.9000,38.9000) -- (50.3000,39.0000) -- (50.8000,39.2000) -- (51.2000,39.4000) -- (51.6000,39.7000) -- (52.0000,39.9000) -- (52.4000,40.2000) -- (52.9000,40.5000) -- (53.3000,40.7000) -- (53.7000,40.8000) -- (54.1000,40.9000) -- (54.6000,40.9000) -- (55.0000,40.9000) -- (55.4000,40.8000) -- (55.8000,40.6000) -- (56.3000,40.5000) -- (56.7000,40.3000) -- (57.1000,40.1000) -- (57.5000,40.0000) -- (57.9000,40.0000) -- (58.4000,40.0000) -- (58.8000,40.1000) -- (59.2000,40.3000) -- (59.6000,40.5000) -- (60.1000,40.7000) -- (60.5000,40.9000) -- (60.9000,41.1000) -- (61.3000,41.4000) -- (61.8000,41.6000) -- (62.2000,41.7000) -- (62.6000,41.8000) -- (63.0000,41.9000) -- (63.4000,41.9000) -- (63.9000,41.8000) -- (64.3000,41.8000) -- (64.7000,41.7000) -- (65.1000,41.6000) -- (65.6000,41.6000) -- (66.0000,41.5000) -- (66.4000,41.4000) -- (66.8000,41.4000) -- (67.3000,41.4000) -- (67.7000,41.3000) -- (68.1000,41.3000) -- (68.5000,41.2000) -- (68.9000,41.1000) -- (69.4000,41.0000) -- (69.8000,40.9000) -- (70.2000,40.8000) -- (70.6000,40.7000) -- (71.1000,40.5000) -- (71.5000,40.3000) -- (71.9000,40.1000) -- (72.3000,39.9000) -- (72.7000,39.7000) -- (73.2000,39.5000) -- (73.6000,39.4000) -- (74.0000,39.3000) -- (74.4000,39.2000) -- (74.9000,39.0000) -- (75.3000,38.9000) -- (75.7000,38.9000) -- (76.1000,38.8000) -- (76.6000,38.8000) -- (77.0000,38.9000) -- (77.4000,38.9000) -- (77.8000,39.0000) -- (78.2000,39.0000) -- (78.7000,39.1000) -- (79.1000,39.2000) -- (79.5000,39.3000) -- (79.9000,39.3000) -- (80.4000,39.4000) -- (80.8000,39.5000) -- (81.2000,39.6000) -- (81.6000,39.7000) -- (82.1000,39.8000) -- (82.5000,39.9000) -- (82.9000,40.0000) -- (83.3000,40.1000) -- (83.7000,40.1000) -- (84.2000,40.2000) -- (84.6000,40.3000) -- (85.0000,40.4000) -- (85.4000,40.5000) -- (85.9000,40.7000) -- (86.3000,40.8000) -- (86.7000,40.9000) -- (87.1000,41.0000) -- (87.6000,41.1000) -- (88.0000,41.1000) -- (88.4000,41.3000) -- (88.8000,41.4000) -- (89.2000,41.5000) -- (89.7000,41.6000) -- (90.1000,41.6000) -- (90.5000,41.7000) -- (90.9000,41.7000) -- (91.4000,41.7000) -- (91.8000,41.7000) -- (92.2000,41.7000) -- (92.6000,41.7000) -- (93.0000,41.6000) -- (93.5000,41.6000) -- (93.9000,41.5000) -- (94.3000,41.5000) -- (94.7000,41.4000) -- (95.2000,41.3000) -- (95.6000,41.2000) -- (96.0000,41.1000) -- (96.4000,41.0000) -- (96.9000,41.0000) -- (97.3000,40.9000) -- (97.7000,40.8000) -- (98.1000,40.7000) -- (98.5000,40.6000) -- (99.0000,40.5000) -- (99.4000,40.4000) -- (99.8000,40.3000) -- (100.2000,40.1000) -- (100.7000,40.1000) -- (101.1000,40.0000) -- (101.5000,39.9000) -- (101.9000,39.7000) -- (102.3000,39.6000) -- (102.8000,39.5000) -- (103.2000,39.4000) -- (103.6000,39.3000) -- (104.0000,39.2000) -- (104.5000,39.2000) -- (104.9000,39.1000) -- (105.3000,39.1000) -- (105.7000,39.1000) -- (106.2000,39.1000) -- (106.6000,39.1000) -- (107.0000,39.1000) -- (107.4000,39.2000) -- (107.8000,39.2000) -- (108.3000,39.3000) -- (108.7000,39.4000) -- (109.1000,39.5000) -- (109.5000,39.7000) -- (110.0000,39.8000) -- (110.4000,39.9000) -- (110.8000,40.0000) -- (111.2000,40.1000) -- (111.7000,40.3000) -- (112.1000,40.4000) -- (112.5000,40.6000) -- (112.9000,40.7000) -- (113.3000,40.8000) -- (113.8000,40.9000) -- (114.2000,41.0000) -- (114.6000,41.1000) -- (115.0000,41.2000) -- (115.5000,41.3000) -- (115.9000,41.4000) -- (116.3000,41.5000) -- (116.7000,41.6000) -- (117.2000,41.6000) -- (117.6000,41.7000) -- (118.0000,41.7000) -- (118.4000,41.8000) -- (118.8000,41.8000) -- (119.3000,41.8000) -- (119.7000,41.8000) -- (120.1000,41.7000) -- (120.5000,41.7000) -- (121.0000,41.6000) -- (121.4000,41.6000) -- (121.8000,41.5000) -- (122.2000,41.3000) -- (122.6000,41.2000) -- (123.1000,41.1000) -- (123.5000,41.0000) -- (123.9000,40.9000) -- (124.3000,40.8000) -- (124.8000,40.7000) -- (125.2000,40.5000) -- (125.6000,40.4000) -- (126.0000,40.2000) -- (126.5000,40.1000) -- (126.9000,39.9000) -- (127.3000,39.8000) -- (127.7000,39.8000) -- (128.1000,39.7000) -- (128.6000,39.5000) -- (129.0000,39.4000) -- (129.4000,39.4000) -- (129.8000,39.3000) -- (130.3000,39.3000) -- (130.7000,39.2000) -- (131.1000,39.2000) -- (131.5000,39.2000) -- (131.9000,39.2000) -- (132.4000,39.2000) -- (132.8000,39.2000) -- (133.2000,39.2000) -- (133.6000,39.2000) -- (134.1000,39.3000) -- (134.5000,39.3000) -- (134.9000,39.4000) -- (135.3000,39.5000) -- (135.8000,39.6000) -- (136.2000,39.7000) -- (136.6000,39.8000) -- (137.0000,39.9000) -- (137.4000,40.0000) -- (137.9000,40.1000) -- (138.3000,40.3000) -- (138.7000,40.4000) -- (139.1000,40.6000) -- (139.6000,40.7000) -- (140.0000,40.8000) -- (140.4000,40.9000) -- (140.8000,41.0000) -- (141.3000,41.1000) -- (141.7000,41.2000) -- (142.1000,41.3000) -- (142.3000,41.4000);



    \end{scope}
    \begin{scope}[cm={{0.74279,0.0,0.0,1.28515,(-186.22138,-161.30028)}},draw=blue,line cap=round,line join=round,line width=0.480pt]
      \path[draw] (25.5000,32.5000) -- (25.5000,56.5000) -- (142.5000,56.5000) -- (142.5000,32.5000) -- (25.5000,32.5000);



    \end{scope}
    \begin{scope}[cm={{0.74279,0.0,0.0,1.28515,(-186.22138,-161.30028)}},draw=ca0a0a4,dash pattern=on 1.22pt off 1.22pt,line cap=round,line join=round,line width=0.305pt,miter limit=4.00]
      \path[draw,dash pattern=on 1.22pt off 1.22pt,line width=0.305pt,miter limit=4.00] (25.5000,70.5000) -- (108.5000,70.5000);



      \path[draw,dash pattern=on 1.22pt off 1.22pt,line width=0.305pt,miter limit=4.00] (137.5000,70.5000) -- (142.5000,70.5000);



    \end{scope}
    \begin{scope}[cm={{0.74279,0.0,0.0,1.28515,(-186.22138,-161.30028)}},draw=blue,line cap=round,line join=round,line width=0.480pt]
      \path[cm={{1.54975,0.0,0.0,1.0,(-13.85377,0.0)}},draw] (25.5000,70.5000) -- (28.5000,70.5000);



      \path[cm={{1.54975,0.0,0.0,1.0,(-78.53317,0.15196)}},draw] (142.5000,70.5000) -- (139.5000,70.5000);



    \end{scope}
    \begin{scope}[cm={{0.95389,0.0,0.0,0.95389,(-183.45337,-67.69359)}},draw=blue,fill=ce10000,line cap=rect,line join=bevel,line width=0.800pt]
      \path[fill=ce10000] (0.0000,0.0000) node[above right] (text464) {\scriptsize -10};



    \end{scope}
    \begin{scope}[cm={{0.74279,0.0,0.0,1.28515,(-186.22138,-161.30028)}},draw=ca0a0a4,dash pattern=on 1.22pt off 1.22pt,line cap=round,line join=round,line width=0.305pt,miter limit=4.00]
      \path[draw,dash pattern=on 1.22pt off 1.22pt,line width=0.305pt,miter limit=4.00] (25.5000,58.5000) -- (142.5000,58.5000);



    \end{scope}
    \begin{scope}[cm={{0.74279,0.0,0.0,1.28515,(-186.22138,-161.30028)}},draw=blue,line cap=round,line join=round,line width=0.480pt]
      \path[cm={{1.54975,0.0,0.0,1.0,(-13.85377,0.0)}},draw] (25.5000,58.5000) -- (28.5000,58.5000);



      \path[cm={{1.54975,0.0,0.0,1.0,(-78.53317,0.0)}},draw] (142.5000,58.5000) -- (139.5000,58.5000);



    \end{scope}
    \begin{scope}[cm={{0.95389,0.0,0.0,0.95389,(-180.91222,-84.15133)}},draw=blue,fill=ce10000,line cap=rect,line join=bevel,line width=0.800pt]
      \path[fill=ce10000] (0.0000,0.0000) node[above right] (text494) {\scriptsize 10};



    \end{scope}
    \begin{scope}[cm={{0.74279,0.0,0.0,1.28515,(-186.22138,-161.30028)}},draw=ca0a0a4,dash pattern=on 0.40pt off 0.80pt,line cap=round,line join=round,line width=0.400pt]
      \path[draw] (25.5000,80.5000) -- (25.5000,56.5000);



    \end{scope}
    \begin{scope}[cm={{0.74279,0.0,0.0,1.28515,(-186.22138,-161.30028)}},draw=blue,line cap=round,line join=round,line width=0.480pt]
      \path[draw] (25.5000,80.5000) -- (25.5000,75.5000);



      \path[draw] (25.5000,56.5000) -- (25.5000,60.5000);



    \end{scope}
    \begin{scope}[cm={{0.74279,0.0,0.0,1.28515,(-186.22138,-161.30028)}},draw=ca0a0a4,dash pattern=on 1.22pt off 1.22pt,line cap=round,line join=round,line width=0.305pt,miter limit=4.00]
      \path[draw,dash pattern=on 1.22pt off 1.22pt,line width=0.305pt,miter limit=4.00] (60.5000,80.5000) -- (60.5000,56.5000);



    \end{scope}
    \begin{scope}[cm={{0.74279,0.0,0.0,1.28515,(-186.22138,-161.30028)}},draw=blue,line cap=round,line join=round,line width=0.480pt]
      \path[cm={{1.0,0.0,0.0,0.57583,(0.0,34.21848)}},draw] (60.5000,80.5000) -- (60.5000,75.5000);



      \path[cm={{1.0,0.0,0.0,0.70101,(0.0,16.80337)}},draw] (60.5000,56.5000) -- (60.5000,60.5000);



    \end{scope}
    \begin{scope}[cm={{0.74279,0.0,0.0,1.28515,(-186.22138,-161.30028)}},draw=ca0a0a4,dash pattern=on 1.22pt off 1.22pt,line cap=round,line join=round,line width=0.305pt,miter limit=4.00]
      \path[draw,dash pattern=on 1.22pt off 1.22pt,line width=0.305pt,miter limit=4.00] (95.5000,80.5000) -- (95.5000,56.5000);



    \end{scope}
    \begin{scope}[cm={{0.74279,0.0,0.0,1.28515,(-186.22138,-161.30028)}},draw=blue,line cap=round,line join=round,line width=0.480pt]
      \path[cm={{1.0,0.0,0.0,0.57583,(0.0,34.21848)}},draw] (95.5000,80.5000) -- (95.5000,75.5000);



      \path[cm={{1.0,0.0,0.0,0.70101,(0.0,16.80337)}},draw] (95.5000,56.5000) -- (95.5000,60.5000);



    \end{scope}
    \begin{scope}[cm={{0.74279,0.0,0.0,1.28515,(-186.22138,-161.30028)}},draw=ca0a0a4,dash pattern=on 1.22pt off 1.22pt,line cap=round,line join=round,line width=0.305pt,miter limit=4.00]
      \path[draw,dash pattern=on 1.22pt off 1.22pt,line width=0.305pt,miter limit=4.00] (130.5000,68.5000) -- (130.5000,56.5000);



    \end{scope}
    \begin{scope}[cm={{0.74279,0.0,0.0,1.28515,(-186.22138,-161.30028)}},draw=blue,line cap=round,line join=round,line width=0.480pt]
      \path[cm={{1.0,0.0,0.0,0.57583,(0.0,34.21848)}},draw] (130.5000,80.5000) -- (130.5000,75.5000);



      \path[cm={{1.0,0.0,0.0,0.70101,(0.0,16.80337)}},draw] (130.5000,56.5000) -- (130.5000,60.5000);



    \end{scope}
    \begin{scope}[cm={{0.74279,0.0,0.0,1.28515,(-186.22138,-161.30028)}},draw=blue,line cap=round,line join=round,line width=0.480pt]
      \path[draw] (25.5000,56.5000) -- (25.5000,80.5000) -- (142.5000,80.5000) -- (142.5000,56.5000) -- (25.5000,56.5000);



    \end{scope}
    \begin{scope}[cm={{0.95389,0.0,0.0,0.95389,(-104.73329,-65.32779)}},draw=blue,line cap=rect,line join=bevel,line width=0.800pt]
      \path[fill=blue] (0.0000,0.0000) node[above right] (text622) {\scriptsize $\beta_1$};



    \end{scope}
    \begin{scope}[cm={{0.74279,0.0,0.0,1.28515,(-184.07401,-166.61499)}},draw=blue,line cap=round,line join=round,line width=0.480pt]
      \path[draw,even odd rule] (123.5000,76.5000) -- (132.5000,76.5000);



    \end{scope}
    \begin{scope}[cm={{0.74279,0.0,0.0,1.28515,(-186.22138,-161.30028)}},draw=blue,line cap=round,line join=round,line width=0.480pt]
      \path[draw] (25.8000,67.9000) -- (25.8000,67.9000) -- (26.2000,64.4000) -- (26.7000,63.4000) -- (27.1000,65.9000) -- (27.5000,64.8000) -- (27.9000,64.0000) -- (28.3000,65.9000) -- (28.8000,65.0000) -- (29.2000,61.7000) -- (29.6000,64.2000) -- (30.0000,65.6000) -- (30.5000,63.7000) -- (30.9000,62.9000) -- (31.3000,64.5000) -- (31.7000,65.9000) -- (32.2000,65.6000) -- (32.6000,64.1000) -- (33.0000,62.9000) -- (33.4000,63.0000) -- (33.8000,64.1000) -- (34.3000,65.3000) -- (34.7000,66.0000) -- (35.1000,65.8000) -- (35.5000,65.1000) -- (36.0000,64.3000) -- (36.4000,63.6000) -- (36.8000,63.4000) -- (37.2000,63.6000) -- (37.6000,64.0000) -- (38.1000,64.5000) -- (38.5000,64.8000) -- (38.9000,64.8000) -- (39.3000,64.6000) -- (39.8000,64.1000) -- (40.2000,63.5000) -- (40.6000,63.0000) -- (41.0000,62.5000) -- (41.5000,62.2000) -- (41.9000,62.1000) -- (42.3000,62.3000) -- (42.7000,62.7000) -- (43.1000,63.2000) -- (43.6000,63.7000) -- (44.0000,64.1000) -- (44.4000,64.6000) -- (44.8000,64.9000) -- (45.3000,65.1000) -- (45.7000,65.3000) -- (46.1000,65.3000) -- (46.5000,65.3000) -- (47.0000,65.2000) -- (47.4000,65.1000) -- (47.8000,64.9000) -- (48.2000,64.8000) -- (48.6000,64.7000) -- (49.1000,64.8000) -- (49.5000,64.8000) -- (49.9000,65.0000) -- (50.3000,65.2000) -- (50.8000,65.4000) -- (51.2000,65.6000) -- (51.6000,65.8000) -- (52.0000,66.0000) -- (52.4000,66.1000) -- (52.9000,66.2000) -- (53.3000,66.1000) -- (53.7000,66.0000) -- (54.1000,65.9000) -- (54.6000,65.7000) -- (55.0000,65.4000) -- (55.4000,65.2000) -- (55.8000,64.9000) -- (56.3000,64.7000) -- (56.7000,64.5000) -- (57.1000,64.4000) -- (57.5000,64.3000) -- (57.9000,64.4000) -- (58.4000,64.5000) -- (58.8000,64.6000) -- (59.2000,64.8000) -- (59.6000,64.9000) -- (60.1000,65.0000) -- (60.5000,65.1000) -- (60.9000,65.2000) -- (61.3000,65.2000) -- (61.8000,65.2000) -- (62.2000,65.1000) -- (62.6000,65.0000) -- (63.0000,64.8000) -- (63.4000,64.6000) -- (63.9000,64.3000) -- (64.3000,64.1000) -- (64.7000,63.9000) -- (65.1000,63.8000) -- (65.6000,63.7000) -- (66.0000,63.6000) -- (66.4000,63.5000) -- (66.8000,63.4000) -- (67.3000,63.4000) -- (67.7000,63.4000) -- (68.1000,63.4000) -- (68.5000,63.3000) -- (68.9000,63.3000) -- (69.4000,63.3000) -- (69.8000,63.3000) -- (70.2000,63.2000) -- (70.6000,63.2000) -- (71.1000,63.1000) -- (71.5000,63.1000) -- (71.9000,63.0000) -- (72.3000,63.0000) -- (72.7000,63.0000) -- (73.2000,63.1000) -- (73.6000,63.2000) -- (74.0000,63.3000) -- (74.4000,63.4000) -- (74.9000,63.5000) -- (75.3000,63.6000) -- (75.7000,63.8000) -- (76.1000,64.0000) -- (76.6000,64.2000) -- (77.0000,64.4000) -- (77.4000,64.6000) -- (77.8000,64.8000) -- (78.2000,64.9000) -- (78.7000,65.1000) -- (79.1000,65.2000) -- (79.5000,65.3000) -- (79.9000,65.4000) -- (80.4000,65.5000) -- (80.8000,65.6000) -- (81.2000,65.6000) -- (81.6000,65.7000) -- (82.1000,65.7000) -- (82.5000,65.8000) -- (82.9000,65.8000) -- (83.3000,65.8000) -- (83.7000,65.8000) -- (84.2000,65.7000) -- (84.6000,65.7000) -- (85.0000,65.7000) -- (85.4000,65.7000) -- (85.9000,65.7000) -- (86.3000,65.6000) -- (86.7000,65.6000) -- (87.1000,65.5000) -- (87.6000,65.5000) -- (88.0000,65.4000) -- (88.4000,65.3000) -- (88.8000,65.3000) -- (89.2000,65.2000) -- (89.7000,65.1000) -- (90.1000,65.0000) -- (90.5000,64.9000) -- (90.9000,64.7000) -- (91.4000,64.6000) -- (91.8000,64.5000) -- (92.2000,64.4000) -- (92.6000,64.2000) -- (93.0000,64.1000) -- (93.5000,64.0000) -- (93.9000,63.8000) -- (94.3000,63.7000) -- (94.7000,63.6000) -- (95.2000,63.5000) -- (95.6000,63.4000) -- (96.0000,63.4000) -- (96.4000,63.4000) -- (96.9000,63.4000) -- (97.3000,63.4000) -- (97.7000,63.3000) -- (98.1000,63.3000) -- (98.5000,63.3000) -- (99.0000,63.3000) -- (99.4000,63.3000) -- (99.8000,63.3000) -- (100.2000,63.3000) -- (100.7000,63.3000) -- (101.1000,63.4000) -- (101.5000,63.4000) -- (101.9000,63.4000) -- (102.3000,63.5000) -- (102.8000,63.6000) -- (103.2000,63.7000) -- (103.6000,63.8000) -- (104.0000,63.9000) -- (104.5000,64.0000) -- (104.9000,64.1000) -- (105.3000,64.3000) -- (105.7000,64.4000) -- (106.2000,64.5000) -- (106.6000,64.6000) -- (107.0000,64.8000) -- (107.4000,64.9000) -- (107.8000,65.0000) -- (108.3000,65.2000) -- (108.7000,65.3000) -- (109.1000,65.4000) -- (109.5000,65.5000) -- (110.0000,65.6000) -- (110.4000,65.7000) -- (110.8000,65.7000) -- (111.2000,65.7000) -- (111.7000,65.8000) -- (112.1000,65.8000) -- (112.5000,65.8000) -- (112.9000,65.8000) -- (113.3000,65.7000) -- (113.8000,65.7000) -- (114.2000,65.6000) -- (114.6000,65.5000) -- (115.0000,65.4000) -- (115.5000,65.4000) -- (115.9000,65.3000) -- (116.3000,65.2000) -- (116.7000,65.1000) -- (117.2000,65.0000) -- (117.6000,64.9000) -- (118.0000,64.7000) -- (118.4000,64.6000) -- (118.8000,64.5000) -- (119.3000,64.4000) -- (119.7000,64.2000) -- (120.1000,64.1000) -- (120.5000,64.0000) -- (121.0000,63.8000) -- (121.4000,63.7000) -- (121.8000,63.6000) -- (122.2000,63.4000) -- (122.6000,63.3000) -- (123.1000,63.3000) -- (123.5000,63.2000) -- (123.9000,63.2000) -- (124.3000,63.2000) -- (124.8000,63.1000) -- (125.2000,63.1000) -- (125.6000,63.1000) -- (126.0000,63.1000) -- (126.5000,63.1000) -- (126.9000,63.1000) -- (127.3000,63.2000) -- (127.7000,63.4000) -- (128.1000,63.4000) -- (128.6000,63.5000) -- (129.0000,63.5000) -- (129.4000,63.6000) -- (129.8000,63.8000) -- (130.3000,63.9000) -- (130.7000,64.1000) -- (131.1000,64.2000) -- (131.5000,64.3000) -- (131.9000,64.4000) -- (132.4000,64.6000) -- (132.8000,64.7000) -- (133.2000,64.8000) -- (133.6000,64.9000) -- (134.1000,65.0000) -- (134.5000,65.1000) -- (134.9000,65.2000) -- (135.3000,65.3000) -- (135.8000,65.4000) -- (136.2000,65.5000) -- (136.6000,65.6000) -- (137.0000,65.6000) -- (137.4000,65.6000) -- (137.9000,65.7000) -- (138.3000,65.7000) -- (138.7000,65.8000) -- (139.1000,65.8000) -- (139.6000,65.8000) -- (140.0000,65.7000) -- (140.4000,65.7000) -- (140.8000,65.6000) -- (141.3000,65.5000) -- (141.7000,65.4000) -- (142.1000,65.4000) -- (142.3000,65.3000);



    \end{scope}
    \begin{scope}[cm={{0.74279,0.0,0.0,1.28515,(-186.22138,-161.30028)}},draw=blue,line cap=round,line join=round,line width=0.480pt]
      \path[draw] (25.5000,56.5000) -- (25.5000,80.5000) -- (142.5000,80.5000) -- (142.5000,56.5000) -- (25.5000,56.5000);



    \end{scope}
    \begin{scope}[cm={{0.74279,0.0,0.0,1.28515,(-186.22138,-161.30028)}},draw=ca0a0a4,dash pattern=on 1.22pt off 1.22pt,line cap=round,line join=round,line width=0.305pt,miter limit=4.00]
      \path[draw,dash pattern=on 1.22pt off 1.22pt,line width=0.305pt,miter limit=4.00] (25.5000,94.5000) -- (108.5000,94.5000);



      \path[draw,dash pattern=on 1.22pt off 1.22pt,line width=0.305pt,miter limit=4.00] (137.5000,94.5000) -- (142.5000,94.5000);



    \end{scope}
    \begin{scope}[cm={{0.74279,0.0,0.0,1.28515,(-186.22138,-161.30028)}},draw=blue,line cap=round,line join=round,line width=0.480pt]
      \path[cm={{1.54975,0.0,0.0,1.0,(-13.85377,0.0)}},draw] (25.5000,94.5000) -- (28.5000,94.5000);



      \path[cm={{1.54975,0.0,0.0,1.0,(-78.53317,0.0)}},draw] (142.5000,94.5000) -- (139.5000,94.5000);



    \end{scope}
    \begin{scope}[cm={{0.95389,0.0,0.0,0.95389,(-183.45337,-37.34406)}},draw=blue,fill=ce10000,line cap=rect,line join=bevel,line width=0.800pt]
      \path[fill=ce10000] (0.0000,0.0000) node[above right] (text678) {\scriptsize -10};



    \end{scope}
    \begin{scope}[cm={{0.74279,0.0,0.0,1.28515,(-186.22138,-161.30028)}},draw=ca0a0a4,dash pattern=on 1.22pt off 1.22pt,line cap=round,line join=round,line width=0.305pt,miter limit=4.00]
      \path[draw,dash pattern=on 1.22pt off 1.22pt,line width=0.305pt,miter limit=4.00] (25.5000,82.5000) -- (142.5000,82.5000);



    \end{scope}
    \begin{scope}[cm={{0.74279,0.0,0.0,1.28515,(-186.22138,-161.30028)}},draw=blue,line cap=round,line join=round,line width=0.480pt]
      \path[cm={{1.54975,0.0,0.0,1.0,(-13.85377,0.0)}},draw] (25.5000,82.5000) -- (28.5000,82.5000);



      \path[cm={{1.54975,0.0,0.0,1.0,(-78.53317,0.0)}},draw] (142.5000,82.5000) -- (139.5000,82.5000);



    \end{scope}
    \begin{scope}[cm={{0.95389,0.0,0.0,0.95389,(-180.91222,-52.31054)}},draw=blue,fill=ce10000,line cap=rect,line join=bevel,line width=0.800pt]
      \path[fill=ce10000] (0.0000,0.0000) node[above right] (text708) {\scriptsize 10};



    \end{scope}
    \begin{scope}[cm={{0.74279,0.0,0.0,1.28515,(-186.22138,-161.30028)}},draw=ca0a0a4,dash pattern=on 0.40pt off 0.80pt,line cap=round,line join=round,line width=0.400pt]
      \path[draw] (25.5000,104.5000) -- (25.5000,80.5000);



    \end{scope}
    \begin{scope}[cm={{0.74279,0.0,0.0,1.28515,(-186.22138,-161.30028)}},draw=blue,line cap=round,line join=round,line width=0.480pt]
      \path[draw] (25.5000,104.5000) -- (25.5000,99.5000);



      \path[draw] (25.5000,80.5000) -- (25.5000,84.5000);



    \end{scope}
    \begin{scope}[cm={{0.74279,0.0,0.0,1.28515,(-186.22138,-161.30028)}},draw=ca0a0a4,dash pattern=on 1.22pt off 1.22pt,line cap=round,line join=round,line width=0.305pt,miter limit=4.00]
      \path[draw,dash pattern=on 1.22pt off 1.22pt,line width=0.305pt,miter limit=4.00] (60.5000,104.5000) -- (60.5000,80.5000);



    \end{scope}
    \begin{scope}[cm={{0.74279,0.0,0.0,1.28515,(-186.22138,-161.30028)}},draw=blue,line cap=round,line join=round,line width=0.480pt]
      \path[cm={{1.0,0.0,0.0,0.57583,(0.0,44.39861)}},draw] (60.5000,104.5000) -- (60.5000,99.5000);



      \path[cm={{1.0,0.0,0.0,0.70101,(0.0,23.97919)}},draw] (60.5000,80.5000) -- (60.5000,84.5000);



    \end{scope}
    \begin{scope}[cm={{0.74279,0.0,0.0,1.28515,(-186.22138,-161.30028)}},draw=ca0a0a4,dash pattern=on 1.22pt off 1.22pt,line cap=round,line join=round,line width=0.305pt,miter limit=4.00]
      \path[draw,dash pattern=on 1.22pt off 1.22pt,line width=0.305pt,miter limit=4.00] (95.5000,104.5000) -- (95.5000,80.5000);



    \end{scope}
    \begin{scope}[cm={{0.74279,0.0,0.0,1.28515,(-186.22138,-161.30028)}},draw=blue,line cap=round,line join=round,line width=0.480pt]
      \path[cm={{1.0,0.0,0.0,0.57583,(0.0,44.39861)}},draw] (95.5000,104.5000) -- (95.5000,99.5000);



      \path[cm={{1.0,0.0,0.0,0.70101,(0.0,23.97919)}},draw] (95.5000,80.5000) -- (95.5000,84.5000);



    \end{scope}
    \begin{scope}[cm={{0.74279,0.0,0.0,1.28515,(-186.22138,-161.30028)}},draw=ca0a0a4,dash pattern=on 1.22pt off 1.22pt,line cap=round,line join=round,line width=0.305pt,miter limit=4.00]
      \path[draw,dash pattern=on 1.22pt off 1.22pt,line width=0.305pt,miter limit=4.00] (130.5000,92.5000) -- (130.5000,80.5000);



    \end{scope}
    \begin{scope}[cm={{0.74279,0.0,0.0,1.28515,(-186.22138,-161.30028)}},draw=blue,line cap=round,line join=round,line width=0.480pt]
      \path[cm={{1.0,0.0,0.0,0.57583,(0.0,44.39861)}},draw] (130.5000,104.5000) -- (130.5000,99.5000);



      \path[cm={{1.0,0.0,0.0,0.70101,(0.0,23.97919)}},draw] (130.5000,80.5000) -- (130.5000,84.5000);



    \end{scope}
    \begin{scope}[cm={{0.74279,0.0,0.0,1.28515,(-186.22138,-161.30028)}},draw=blue,line cap=round,line join=round,line width=0.480pt]
      \path[draw] (25.5000,80.5000) -- (25.5000,104.5000) -- (142.5000,104.5000) -- (142.5000,80.5000) -- (25.5000,80.5000);



    \end{scope}
    \begin{scope}[cm={{0.95389,0.0,0.0,0.95389,(-104.99275,-34.97829)}},draw=blue,line cap=rect,line join=bevel,line width=0.800pt]
      \path[fill=blue] (0.0000,0.0000) node[above right] (text836) {\scriptsize $\alpha_2$};



    \end{scope}
    \begin{scope}[cm={{0.74279,0.0,0.0,1.28515,(-184.07401,-166.61499)}},draw=blue,line cap=round,line join=round,line width=0.480pt]
      \path[draw,even odd rule] (123.5000,100.5000) -- (132.5000,100.5000);



    \end{scope}
    \begin{scope}[cm={{0.74279,0.0,0.0,1.28515,(-186.22138,-161.30028)}},draw=blue,line cap=round,line join=round,line width=0.480pt]
      \path[draw] (25.8000,86.8000) -- (25.8000,86.8000) -- (26.2000,88.5000) -- (26.7000,88.7000) -- (27.1000,88.0000) -- (27.5000,88.0000) -- (27.9000,90.0000) -- (28.3000,85.4000) -- (28.8000,92.2000) -- (29.2000,86.5000) -- (29.6000,86.2000) -- (30.0000,92.2000) -- (30.5000,91.0000) -- (30.9000,85.7000) -- (31.3000,84.6000) -- (31.7000,88.3000) -- (32.2000,91.6000) -- (32.6000,91.4000) -- (33.0000,88.6000) -- (33.4000,86.0000) -- (33.8000,85.5000) -- (34.3000,87.1000) -- (34.7000,89.4000) -- (35.1000,91.2000) -- (35.5000,91.5000) -- (36.0000,90.5000) -- (36.4000,88.8000) -- (36.8000,87.2000) -- (37.2000,86.2000) -- (37.6000,86.2000) -- (38.1000,86.9000) -- (38.5000,88.0000) -- (38.9000,89.2000) -- (39.3000,90.2000) -- (39.8000,90.6000) -- (40.2000,90.5000) -- (40.6000,89.9000) -- (41.0000,89.0000) -- (41.5000,88.0000) -- (41.9000,87.1000) -- (42.3000,86.4000) -- (42.7000,86.1000) -- (43.1000,86.2000) -- (43.6000,86.6000) -- (44.0000,87.3000) -- (44.4000,88.1000) -- (44.8000,88.9000) -- (45.3000,89.6000) -- (45.7000,90.1000) -- (46.1000,90.5000) -- (46.5000,90.6000) -- (47.0000,90.6000) -- (47.4000,90.3000) -- (47.8000,89.7000) -- (48.2000,89.0000) -- (48.6000,88.3000) -- (49.1000,87.6000) -- (49.5000,86.9000) -- (49.9000,86.4000) -- (50.3000,86.0000) -- (50.8000,85.9000) -- (51.2000,86.0000) -- (51.6000,86.2000) -- (52.0000,86.6000) -- (52.4000,87.2000) -- (52.9000,87.8000) -- (53.3000,88.3000) -- (53.7000,88.9000) -- (54.1000,89.3000) -- (54.6000,89.6000) -- (55.0000,89.8000) -- (55.4000,89.8000) -- (55.8000,89.6000) -- (56.3000,89.4000) -- (56.7000,89.0000) -- (57.1000,88.6000) -- (57.5000,88.2000) -- (57.9000,87.9000) -- (58.4000,87.7000) -- (58.8000,87.6000) -- (59.2000,87.6000) -- (59.6000,87.7000) -- (60.1000,87.9000) -- (60.5000,88.2000) -- (60.9000,88.5000) -- (61.3000,88.8000) -- (61.8000,89.1000) -- (62.2000,89.4000) -- (62.6000,89.5000) -- (63.0000,89.6000) -- (63.4000,89.6000) -- (63.9000,89.4000) -- (64.3000,89.2000) -- (64.7000,89.0000) -- (65.1000,88.7000) -- (65.6000,88.5000) -- (66.0000,88.2000) -- (66.4000,88.1000) -- (66.8000,87.9000) -- (67.3000,87.9000) -- (67.7000,87.9000) -- (68.1000,87.9000) -- (68.5000,88.0000) -- (68.9000,88.1000) -- (69.4000,88.2000) -- (69.8000,88.4000) -- (70.2000,88.5000) -- (70.6000,88.6000) -- (71.1000,88.7000) -- (71.5000,88.7000) -- (71.9000,88.7000) -- (72.3000,88.6000) -- (72.7000,88.6000) -- (73.2000,88.5000) -- (73.6000,88.5000) -- (74.0000,88.5000) -- (74.4000,88.5000) -- (74.9000,88.4000) -- (75.3000,88.4000) -- (75.7000,88.3000) -- (76.1000,88.3000) -- (76.6000,88.4000) -- (77.0000,88.4000) -- (77.4000,88.4000) -- (77.8000,88.5000) -- (78.2000,88.5000) -- (78.7000,88.6000) -- (79.1000,88.6000) -- (79.5000,88.6000) -- (79.9000,88.6000) -- (80.4000,88.6000) -- (80.8000,88.5000) -- (81.2000,88.5000) -- (81.6000,88.4000) -- (82.1000,88.3000) -- (82.5000,88.3000) -- (82.9000,88.2000) -- (83.3000,88.1000) -- (83.7000,88.0000) -- (84.2000,88.0000) -- (84.6000,87.9000) -- (85.0000,87.9000) -- (85.4000,87.9000) -- (85.9000,87.9000) -- (86.3000,88.0000) -- (86.7000,88.0000) -- (87.1000,88.1000) -- (87.6000,88.2000) -- (88.0000,88.3000) -- (88.4000,88.4000) -- (88.8000,88.5000) -- (89.2000,88.7000) -- (89.7000,88.8000) -- (90.1000,88.9000) -- (90.5000,89.0000) -- (90.9000,89.0000) -- (91.4000,89.1000) -- (91.8000,89.1000) -- (92.2000,89.0000) -- (92.6000,89.0000) -- (93.0000,88.9000) -- (93.5000,88.8000) -- (93.9000,88.7000) -- (94.3000,88.6000) -- (94.7000,88.5000) -- (95.2000,88.3000) -- (95.6000,88.2000) -- (96.0000,88.1000) -- (96.4000,88.0000) -- (96.9000,88.0000) -- (97.3000,88.0000) -- (97.7000,88.0000) -- (98.1000,88.0000) -- (98.5000,88.0000) -- (99.0000,88.1000) -- (99.4000,88.1000) -- (99.8000,88.2000) -- (100.2000,88.2000) -- (100.7000,88.3000) -- (101.1000,88.5000) -- (101.5000,88.5000) -- (101.9000,88.5000) -- (102.3000,88.5000) -- (102.8000,88.6000) -- (103.2000,88.6000) -- (103.6000,88.6000) -- (104.0000,88.7000) -- (104.5000,88.7000) -- (104.9000,88.7000) -- (105.3000,88.7000) -- (105.7000,88.7000) -- (106.2000,88.6000) -- (106.6000,88.6000) -- (107.0000,88.6000) -- (107.4000,88.5000) -- (107.8000,88.5000) -- (108.3000,88.4000) -- (108.7000,88.4000) -- (109.1000,88.4000) -- (109.5000,88.4000) -- (110.0000,88.4000) -- (110.4000,88.4000) -- (110.8000,88.3000) -- (111.2000,88.3000) -- (111.7000,88.3000) -- (112.1000,88.3000) -- (112.5000,88.3000) -- (112.9000,88.3000) -- (113.3000,88.3000) -- (113.8000,88.3000) -- (114.2000,88.3000) -- (114.6000,88.3000) -- (115.0000,88.3000) -- (115.5000,88.3000) -- (115.9000,88.4000) -- (116.3000,88.4000) -- (116.7000,88.4000) -- (117.2000,88.5000) -- (117.6000,88.5000) -- (118.0000,88.6000) -- (118.4000,88.6000) -- (118.8000,88.6000) -- (119.3000,88.7000) -- (119.7000,88.7000) -- (120.1000,88.7000) -- (120.5000,88.6000) -- (121.0000,88.6000) -- (121.4000,88.6000) -- (121.8000,88.5000) -- (122.2000,88.4000) -- (122.6000,88.4000) -- (123.1000,88.3000) -- (123.5000,88.3000) -- (123.9000,88.3000) -- (124.3000,88.3000) -- (124.8000,88.2000) -- (125.2000,88.2000) -- (125.6000,88.2000) -- (126.0000,88.1000) -- (126.5000,88.1000) -- (126.9000,88.2000) -- (127.3000,88.2000) -- (127.7000,88.3000) -- (128.1000,88.4000) -- (128.6000,88.4000) -- (129.0000,88.4000) -- (129.4000,88.5000) -- (129.8000,88.6000) -- (130.3000,88.6000) -- (130.7000,88.7000) -- (131.1000,88.8000) -- (131.5000,88.8000) -- (131.9000,88.8000) -- (132.4000,88.8000) -- (132.8000,88.8000) -- (133.2000,88.7000) -- (133.6000,88.7000) -- (134.1000,88.6000) -- (134.5000,88.5000) -- (134.9000,88.4000) -- (135.3000,88.4000) -- (135.8000,88.3000) -- (136.2000,88.3000) -- (136.6000,88.2000) -- (137.0000,88.2000) -- (137.4000,88.1000) -- (137.9000,88.1000) -- (138.3000,88.1000) -- (138.7000,88.1000) -- (139.1000,88.2000) -- (139.6000,88.2000) -- (140.0000,88.3000) -- (140.4000,88.3000) -- (140.8000,88.3000) -- (141.3000,88.4000) -- (141.7000,88.4000) -- (142.1000,88.5000) -- (142.3000,88.5000);



    \end{scope}
    \begin{scope}[cm={{0.74279,0.0,0.0,1.28515,(-186.22138,-161.30028)}},draw=blue,line cap=round,line join=round,line width=0.480pt]
      \path[draw] (25.5000,80.5000) -- (25.5000,104.5000) -- (142.5000,104.5000) -- (142.5000,80.5000) -- (25.5000,80.5000);



    \end{scope}
    \begin{scope}[cm={{0.74279,0.0,0.0,1.28515,(-186.22138,-161.30028)}},draw=ca0a0a4,dash pattern=on 1.22pt off 1.22pt,line cap=round,line join=round,line width=0.305pt,miter limit=4.00]
      \path[draw,dash pattern=on 1.22pt off 1.22pt,line width=0.305pt,miter limit=4.00] (25.5000,118.5000) -- (108.5000,118.5000);



      \path[draw,dash pattern=on 1.22pt off 1.22pt,line width=0.305pt,miter limit=4.00] (137.5000,118.5000) -- (142.5000,118.5000);



    \end{scope}
    \begin{scope}[cm={{0.74279,0.0,0.0,1.28515,(-186.22138,-161.30028)}},draw=blue,line cap=round,line join=round,line width=0.480pt]
      \path[cm={{1.54975,0.0,0.0,1.0,(-13.85377,0.0)}},draw] (25.5000,118.5000) -- (28.5000,118.5000);



      \path[cm={{1.54975,0.0,0.0,1.0,(-78.53317,0.0)}},draw] (142.5000,118.5000) -- (139.5000,118.5000);



    \end{scope}
    \begin{scope}[cm={{0.95389,0.0,0.0,0.95389,(-183.45337,-6.99453)}},draw=blue,fill=ce10000,line cap=rect,line join=bevel,line width=0.800pt]
      \path[fill=ce10000] (0.0000,0.0000) node[above right] (text892) {\scriptsize -10};



    \end{scope}
    \begin{scope}[cm={{0.74279,0.0,0.0,1.28515,(-186.22138,-161.30028)}},draw=ca0a0a4,dash pattern=on 1.22pt off 1.22pt,line cap=round,line join=round,line width=0.305pt,miter limit=4.00]
      \path[draw,dash pattern=on 1.22pt off 1.22pt,line width=0.305pt,miter limit=4.00] (25.5000,106.5000) -- (142.5000,106.5000);



    \end{scope}
    \begin{scope}[cm={{0.74279,0.0,0.0,1.28515,(-186.22138,-161.30028)}},draw=blue,line cap=round,line join=round,line width=0.480pt]
      \path[cm={{1.54975,0.0,0.0,1.0,(-13.85377,0.0)}},draw] (25.5000,106.5000) -- (28.5000,106.5000);



      \path[cm={{1.54975,0.0,0.0,1.0,(-78.53317,0.0)}},draw] (142.5000,106.5000) -- (139.5000,106.5000);



    \end{scope}
    \begin{scope}[cm={{0.95389,0.0,0.0,0.95389,(-180.91222,-21.96101)}},draw=blue,fill=ce10000,line cap=rect,line join=bevel,line width=0.800pt]
      \path[fill=ce10000] (0.0000,0.0000) node[above right] (text922) {\scriptsize 10};



    \end{scope}
    \begin{scope}[cm={{0.74279,0.0,0.0,1.28515,(-186.22138,-161.30028)}},draw=ca0a0a4,dash pattern=on 0.40pt off 0.80pt,line cap=round,line join=round,line width=0.400pt]
      \path[draw] (25.5000,128.5000) -- (25.5000,104.5000);



    \end{scope}
    \begin{scope}[cm={{0.74279,0.0,0.0,1.28515,(-186.22138,-161.30028)}},draw=blue,line cap=round,line join=round,line width=0.480pt]
      \path[draw] (25.5000,128.5000) -- (25.5000,123.5000);



      \path[draw] (25.5000,104.5000) -- (25.5000,108.5000);



    \end{scope}
    \begin{scope}[cm={{0.74279,0.0,0.0,1.28515,(-186.22138,-161.30028)}},draw=ca0a0a4,dash pattern=on 1.22pt off 1.22pt,line cap=round,line join=round,line width=0.305pt,miter limit=4.00]
      \path[draw,dash pattern=on 1.22pt off 1.22pt,line width=0.305pt,miter limit=4.00] (60.5000,128.5000) -- (60.5000,104.5000);



    \end{scope}
    \begin{scope}[cm={{0.74279,0.0,0.0,1.28515,(-186.22138,-161.30028)}},draw=blue,line cap=round,line join=round,line width=0.480pt]
      \path[cm={{1.0,0.0,0.0,0.57583,(0.0,54.57875)}},draw] (60.5000,128.5000) -- (60.5000,123.5000);



      \path[cm={{1.0,0.0,0.0,0.70101,(0.0,31.15501)}},draw] (60.5000,104.5000) -- (60.5000,108.5000);



    \end{scope}
    \begin{scope}[cm={{0.74279,0.0,0.0,1.28515,(-186.22138,-161.30028)}},draw=ca0a0a4,dash pattern=on 1.22pt off 1.22pt,line cap=round,line join=round,line width=0.305pt,miter limit=4.00]
      \path[draw,dash pattern=on 1.22pt off 1.22pt,line width=0.305pt,miter limit=4.00] (95.5000,128.5000) -- (95.5000,104.5000);



    \end{scope}
    \begin{scope}[cm={{0.74279,0.0,0.0,1.28515,(-147.4494,-161.30028)}},draw=blue,line cap=round,line join=round,line width=0.480pt]
      \path[cm={{1.0,0.0,0.0,0.57583,(-52.1975,54.57875)}},draw] (95.5000,128.5000) -- (95.5000,123.5000);



      \path[cm={{1.0,0.0,0.0,0.70101,(-52.1975,31.15501)}},draw] (95.5000,104.5000) -- (95.5000,108.5000);



    \end{scope}
    \begin{scope}[cm={{0.74279,0.0,0.0,1.28515,(-186.22138,-161.30028)}},draw=ca0a0a4,dash pattern=on 1.22pt off 1.22pt,line cap=round,line join=round,line width=0.305pt,miter limit=4.00]
      \path[draw,dash pattern=on 1.22pt off 1.22pt,line width=0.305pt,miter limit=4.00] (130.5000,116.5000) -- (130.5000,104.5000);



    \end{scope}
    \begin{scope}[cm={{0.74279,0.0,0.0,1.28515,(-186.22138,-161.30028)}},draw=blue,line cap=round,line join=round,line width=0.480pt]
      \path[cm={{1.0,0.0,0.0,0.57583,(0.0,54.57875)}},draw] (130.5000,128.5000) -- (130.5000,123.5000);



      \path[cm={{1.0,0.0,0.0,0.70101,(0.0,31.15501)}},draw] (130.5000,104.5000) -- (130.5000,108.5000);



    \end{scope}
    \begin{scope}[cm={{0.74279,0.0,0.0,1.28515,(-186.22138,-161.30028)}},draw=blue,line cap=round,line join=round,line width=0.480pt]
      \path[draw] (25.5000,104.5000) -- (25.5000,128.5000) -- (142.5000,128.5000) -- (142.5000,104.5000) -- (25.5000,104.5000);



    \end{scope}
    \begin{scope}[cm={{0.95389,0.0,0.0,0.95389,(-104.73329,-4.06956)}},draw=blue,line cap=rect,line join=bevel,line width=0.800pt]
      \path[fill=blue] (0.0000,0.0000) node[above right] (text1050) {\scriptsize $\beta_2$};



    \end{scope}
    \begin{scope}[cm={{0.74279,0.0,0.0,1.28515,(-184.07401,-166.61499)}},draw=blue,line cap=round,line join=round,line width=0.480pt]
      \path[draw,even odd rule] (123.5000,124.5000) -- (132.5000,124.5000);



    \end{scope}
    \begin{scope}[cm={{0.74279,0.0,0.0,1.28515,(-186.22138,-161.30028)}},draw=blue,line cap=round,line join=round,line width=0.480pt]
      \path[draw] (25.8000,114.2000) -- (25.8000,114.2000) -- (26.2000,112.5000) -- (26.7000,112.2000) -- (27.1000,112.0000) -- (27.5000,115.1000) -- (27.9000,109.2000) -- (28.3000,114.1000) -- (28.8000,113.9000) -- (29.2000,108.7000) -- (29.6000,114.6000) -- (30.0000,115.3000) -- (30.5000,109.5000) -- (30.9000,108.5000) -- (31.3000,113.0000) -- (31.7000,116.4000) -- (32.2000,115.2000) -- (32.6000,111.5000) -- (33.0000,109.2000) -- (33.4000,109.7000) -- (33.8000,112.1000) -- (34.3000,114.6000) -- (34.7000,115.4000) -- (35.1000,114.5000) -- (35.5000,112.6000) -- (36.0000,110.6000) -- (36.4000,109.5000) -- (36.8000,109.5000) -- (37.2000,110.5000) -- (37.6000,112.0000) -- (38.1000,113.4000) -- (38.5000,114.4000) -- (38.9000,114.7000) -- (39.3000,114.3000) -- (39.8000,113.4000) -- (40.2000,112.3000) -- (40.6000,111.3000) -- (41.0000,110.6000) -- (41.5000,110.2000) -- (41.9000,110.3000) -- (42.3000,110.8000) -- (42.7000,111.6000) -- (43.1000,112.5000) -- (43.6000,113.4000) -- (44.0000,114.1000) -- (44.4000,114.5000) -- (44.8000,114.7000) -- (45.3000,114.6000) -- (45.7000,114.3000) -- (46.1000,113.7000) -- (46.5000,113.0000) -- (47.0000,112.4000) -- (47.4000,111.7000) -- (47.8000,111.0000) -- (48.2000,110.5000) -- (48.6000,110.2000) -- (49.1000,110.2000) -- (49.5000,110.4000) -- (49.9000,110.8000) -- (50.3000,111.4000) -- (50.8000,112.0000) -- (51.2000,112.7000) -- (51.6000,113.4000) -- (52.0000,114.0000) -- (52.4000,114.5000) -- (52.9000,114.8000) -- (53.3000,114.8000) -- (53.7000,114.7000) -- (54.1000,114.5000) -- (54.6000,114.0000) -- (55.0000,113.5000) -- (55.4000,113.0000) -- (55.8000,112.5000) -- (56.3000,112.0000) -- (56.7000,111.6000) -- (57.1000,111.4000) -- (57.5000,111.3000) -- (57.9000,111.3000) -- (58.4000,111.5000) -- (58.8000,111.8000) -- (59.2000,112.2000) -- (59.6000,112.5000) -- (60.1000,112.8000) -- (60.5000,113.0000) -- (60.9000,113.2000) -- (61.3000,113.2000) -- (61.8000,113.2000) -- (62.2000,113.1000) -- (62.6000,112.8000) -- (63.0000,112.6000) -- (63.4000,112.2000) -- (63.9000,111.9000) -- (64.3000,111.7000) -- (64.7000,111.5000) -- (65.1000,111.4000) -- (65.6000,111.3000) -- (66.0000,111.4000) -- (66.4000,111.5000) -- (66.8000,111.7000) -- (67.3000,112.0000) -- (67.7000,112.2000) -- (68.1000,112.4000) -- (68.5000,112.6000) -- (68.9000,112.7000) -- (69.4000,112.8000) -- (69.8000,112.9000) -- (70.2000,112.9000) -- (70.6000,112.9000) -- (71.1000,112.8000) -- (71.5000,112.7000) -- (71.9000,112.6000) -- (72.3000,112.5000) -- (72.7000,112.4000) -- (73.2000,112.3000) -- (73.6000,112.3000) -- (74.0000,112.3000) -- (74.4000,112.3000) -- (74.9000,112.3000) -- (75.3000,112.3000) -- (75.7000,112.3000) -- (76.1000,112.3000) -- (76.6000,112.4000) -- (77.0000,112.5000) -- (77.4000,112.5000) -- (77.8000,112.5000) -- (78.2000,112.5000) -- (78.7000,112.5000) -- (79.1000,112.5000) -- (79.5000,112.5000) -- (79.9000,112.4000) -- (80.4000,112.4000) -- (80.8000,112.3000) -- (81.2000,112.3000) -- (81.6000,112.3000) -- (82.1000,112.2000) -- (82.5000,112.3000) -- (82.9000,112.3000) -- (83.3000,112.3000) -- (83.7000,112.3000) -- (84.2000,112.3000) -- (84.6000,112.4000) -- (85.0000,112.5000) -- (85.4000,112.6000) -- (85.9000,112.7000) -- (86.3000,112.8000) -- (86.7000,112.8000) -- (87.1000,112.9000) -- (87.6000,112.9000) -- (88.0000,113.0000) -- (88.4000,113.0000) -- (88.8000,113.0000) -- (89.2000,113.0000) -- (89.7000,113.0000) -- (90.1000,112.9000) -- (90.5000,112.8000) -- (90.9000,112.7000) -- (91.4000,112.6000) -- (91.8000,112.4000) -- (92.2000,112.3000) -- (92.6000,112.2000) -- (93.0000,112.0000) -- (93.5000,111.9000) -- (93.9000,111.9000) -- (94.3000,111.8000) -- (94.7000,111.8000) -- (95.2000,111.8000) -- (95.6000,111.8000) -- (96.0000,111.9000) -- (96.4000,112.0000) -- (96.9000,112.2000) -- (97.3000,112.3000) -- (97.7000,112.4000) -- (98.1000,112.5000) -- (98.5000,112.6000) -- (99.0000,112.6000) -- (99.4000,112.7000) -- (99.8000,112.7000) -- (100.2000,112.8000) -- (100.7000,112.8000) -- (101.1000,112.9000) -- (101.5000,112.9000) -- (101.9000,112.8000) -- (102.3000,112.7000) -- (102.8000,112.7000) -- (103.2000,112.6000) -- (103.6000,112.6000) -- (104.0000,112.6000) -- (104.5000,112.5000) -- (104.9000,112.5000) -- (105.3000,112.4000) -- (105.7000,112.3000) -- (106.2000,112.3000) -- (106.6000,112.3000) -- (107.0000,112.2000) -- (107.4000,112.2000) -- (107.8000,112.2000) -- (108.3000,112.2000) -- (108.7000,112.2000) -- (109.1000,112.3000) -- (109.5000,112.3000) -- (110.0000,112.3000) -- (110.4000,112.3000) -- (110.8000,112.3000) -- (111.2000,112.3000) -- (111.7000,112.4000) -- (112.1000,112.4000) -- (112.5000,112.4000) -- (112.9000,112.5000) -- (113.3000,112.5000) -- (113.8000,112.5000) -- (114.2000,112.5000) -- (114.6000,112.5000) -- (115.0000,112.5000) -- (115.5000,112.6000) -- (115.9000,112.6000) -- (116.3000,112.6000) -- (116.7000,112.6000) -- (117.2000,112.6000) -- (117.6000,112.6000) -- (118.0000,112.6000) -- (118.4000,112.6000) -- (118.8000,112.6000) -- (119.3000,112.5000) -- (119.7000,112.5000) -- (120.1000,112.4000) -- (120.5000,112.4000) -- (121.0000,112.3000) -- (121.4000,112.3000) -- (121.8000,112.2000) -- (122.2000,112.2000) -- (122.6000,112.2000) -- (123.1000,112.2000) -- (123.5000,112.2000) -- (123.9000,112.3000) -- (124.3000,112.3000) -- (124.8000,112.3000) -- (125.2000,112.4000) -- (125.6000,112.4000) -- (126.0000,112.4000) -- (126.5000,112.5000) -- (126.9000,112.5000) -- (127.3000,112.6000) -- (127.7000,112.7000) -- (128.1000,112.7000) -- (128.6000,112.7000) -- (129.0000,112.7000) -- (129.4000,112.7000) -- (129.8000,112.7000) -- (130.3000,112.7000) -- (130.7000,112.7000) -- (131.1000,112.6000) -- (131.5000,112.6000) -- (131.9000,112.5000) -- (132.4000,112.4000) -- (132.8000,112.3000) -- (133.2000,112.2000) -- (133.6000,112.2000) -- (134.1000,112.1000) -- (134.5000,112.1000) -- (134.9000,112.1000) -- (135.3000,112.1000) -- (135.8000,112.1000) -- (136.2000,112.2000) -- (136.6000,112.2000) -- (137.0000,112.2000) -- (137.4000,112.3000) -- (137.9000,112.4000) -- (138.3000,112.4000) -- (138.7000,112.5000) -- (139.1000,112.6000) -- (139.6000,112.7000) -- (140.0000,112.7000) -- (140.4000,112.7000) -- (140.8000,112.7000) -- (141.3000,112.7000) -- (141.7000,112.7000) -- (142.1000,112.7000) -- (142.3000,112.7000);



    \end{scope}
    \begin{scope}[cm={{0.74279,0.0,0.0,1.28515,(-186.22138,-161.30028)}},draw=blue,line cap=round,line join=round,line width=0.480pt]
      \path[draw] (25.5000,104.5000) -- (25.5000,128.5000) -- (142.5000,128.5000) -- (142.5000,104.5000) -- (25.5000,104.5000);



    \end{scope}
    \begin{scope}[cm={{0.74279,0.0,0.0,1.28515,(-186.22138,-161.30028)}},draw=ca0a0a4,dash pattern=on 1.22pt off 1.22pt,line cap=round,line join=round,line width=0.305pt,miter limit=4.00]
      \path[draw,dash pattern=on 1.22pt off 1.22pt,line width=0.305pt,miter limit=4.00] (25.5000,144.5000) -- (108.5000,144.5000);



      \path[draw,dash pattern=on 1.22pt off 1.22pt,line width=0.305pt,miter limit=4.00] (137.5000,144.5000) -- (142.5000,144.5000);



    \end{scope}
    \begin{scope}[cm={{0.74279,0.0,0.0,1.28515,(-186.22138,-161.30028)}},draw=blue,line cap=round,line join=round,line width=0.480pt]
      \path[cm={{1.54975,0.0,0.0,1.0,(-13.85377,0.0)}},draw] (25.5000,144.5000) -- (28.5000,144.5000);



      \path[cm={{1.54975,0.0,0.0,1.0,(-78.53317,0.0)}},draw] (142.5000,144.5000) -- (139.5000,144.5000);



    \end{scope}
    \begin{scope}[cm={{0.95389,0.0,0.0,0.95389,(-183.45337,27.70791)}},draw=blue,fill=ce10000,line cap=rect,line join=bevel,line width=0.800pt]
      \path[fill=ce10000] (0.0000,0.0000) node[above right] (text1106) {\scriptsize -20};



    \end{scope}
    \begin{scope}[cm={{0.74279,0.0,0.0,1.28515,(-186.22138,-161.30028)}},draw=ca0a0a4,dash pattern=on 1.22pt off 1.22pt,line cap=round,line join=round,line width=0.305pt,miter limit=4.00]
      \path[draw,dash pattern=on 1.22pt off 1.22pt,line width=0.305pt,miter limit=4.00] (25.5000,129.5000) -- (142.5000,129.5000);



    \end{scope}
    \begin{scope}[cm={{0.74279,0.0,0.0,1.28515,(-186.22138,-161.30028)}},draw=blue,line cap=round,line join=round,line width=0.480pt]
      \path[cm={{1.54975,0.0,0.0,1.0,(-13.85377,0.0)}},draw] (25.5000,129.5000) -- (28.5000,129.5000);



      \path[cm={{1.54975,0.0,0.0,1.0,(-78.53317,0.0)}},draw] (142.5000,129.5000) -- (139.5000,129.5000);



    \end{scope}
    \begin{scope}[cm={{0.95389,0.0,0.0,0.95389,(-180.91222,7.43463)}},draw=blue,fill=ce10000,line cap=rect,line join=bevel,line width=0.800pt]
      \path[fill=ce10000] (0.0000,0.0000) node[above right] (text1136) {\scriptsize 20};



    \end{scope}
    \begin{scope}[cm={{0.74279,0.0,0.0,1.28515,(-186.22138,-161.30028)}},draw=ca0a0a4,dash pattern=on 0.40pt off 0.80pt,line cap=round,line join=round,line width=0.400pt]
      \path[draw] (25.5000,152.5000) -- (25.5000,128.5000);



    \end{scope}
    \begin{scope}[cm={{0.74279,0.0,0.0,1.28515,(-186.22138,-161.30028)}},draw=blue,line cap=round,line join=round,line width=0.480pt]
      \path[draw] (25.5000,152.5000) -- (25.5000,147.5000);



      \path[draw] (25.5000,128.5000) -- (25.5000,132.5000);



    \end{scope}
    \begin{scope}[cm={{0.74279,0.0,0.0,1.28515,(-186.22138,-161.30028)}},draw=ca0a0a4,dash pattern=on 1.22pt off 1.22pt,line cap=round,line join=round,line width=0.305pt,miter limit=4.00]
      \path[draw,dash pattern=on 1.22pt off 1.22pt,line width=0.305pt,miter limit=4.00] (60.5000,152.5000) -- (60.5000,128.5000);



    \end{scope}
    \begin{scope}[cm={{0.74279,0.0,0.0,1.28515,(-186.22138,-161.30028)}},draw=blue,line cap=round,line join=round,line width=0.480pt]
      \path[cm={{1.0,0.0,0.0,0.57583,(0.0,64.75889)}},draw] (60.5000,152.5000) -- (60.5000,147.5000);



      \path[cm={{1.0,0.0,0.0,0.70101,(0.0,38.33083)}},draw] (60.5000,128.5000) -- (60.5000,132.5000);



    \end{scope}
    \begin{scope}[cm={{0.74279,0.0,0.0,1.28515,(-186.22138,-161.30028)}},draw=ca0a0a4,dash pattern=on 1.22pt off 1.22pt,line cap=round,line join=round,line width=0.305pt,miter limit=4.00]
      \path[draw,dash pattern=on 1.22pt off 1.22pt,line width=0.305pt,miter limit=4.00] (95.5000,152.5000) -- (95.5000,128.5000);



    \end{scope}
    \begin{scope}[cm={{0.74279,0.0,0.0,1.28515,(-186.22138,-161.30028)}},draw=blue,line cap=round,line join=round,line width=0.480pt]
      \path[cm={{1.0,0.0,0.0,0.57583,(0.0,64.75889)}},draw] (95.5000,152.5000) -- (95.5000,147.5000);



      \path[cm={{1.0,0.0,0.0,0.70101,(0.0,38.33083)}},draw] (95.5000,128.5000) -- (95.5000,132.5000);



    \end{scope}
    \begin{scope}[cm={{0.74279,0.0,0.0,1.28515,(-186.22138,-161.30028)}},draw=ca0a0a4,dash pattern=on 1.22pt off 1.22pt,line cap=round,line join=round,line width=0.305pt,miter limit=4.00]
      \path[draw,dash pattern=on 1.22pt off 1.22pt,line width=0.305pt,miter limit=4.00] (130.5000,140.5000) -- (130.5000,128.5000);



    \end{scope}
    \begin{scope}[cm={{0.99623,0.0,0.0,1.3704,(-320.15815,-158.86873)}},draw=ca0a0a4,dash pattern=on 1.02pt off 1.02pt,even odd rule,line cap=round,line join=round,line width=0.255pt,miter limit=4.00]
      \path[draw,dash pattern=on 1.02pt off 1.02pt,even odd rule,line cap=round,line width=0.255pt,miter limit=4.00] (41.5000,101.5000) -- (127.5000,101.5000);



    \end{scope}
    \begin{scope}[cm={{0.95389,0.0,0.0,0.95389,(-228.06584,308.36388)}},draw=blue,line cap=rect,line join=bevel,line width=0.800pt]
      \begin{scope}[cm={{1.04485,0.0,0.0,1.39271,(-96.58638,-490.878)}},draw=ca0a0a4,dash pattern=on 1.56pt off 1.56pt,line cap=round,line join=round,line width=0.259pt,miter limit=4.00]
        \path[draw,dash pattern=on 1.56pt off 1.56pt,line width=0.259pt,miter limit=4.00] (70.5000,84.5000) -- (70.5000,30.5000);



        \path[draw,dash pattern=on 1.56pt off 1.56pt,line width=0.259pt,miter limit=4.00] (70.5000,14.5000) -- (70.5000,8.5000);



      \end{scope}
      \begin{scope}[cm={{1.04485,0.0,0.0,1.39271,(-96.58638,-490.7838)}},draw=ca0a0a4,dash pattern=on 1.56pt off 1.56pt,line cap=round,line join=round,line width=0.259pt,miter limit=4.00]
        \path[draw,dash pattern=on 1.56pt off 1.56pt,line width=0.259pt,miter limit=4.00] (98.5000,84.5000) -- (98.5000,30.5000);



        \path[draw,dash pattern=on 1.56pt off 1.56pt,line width=0.259pt,miter limit=4.00] (98.5000,14.5000) -- (98.5000,8.5000);



      \end{scope}
      \path[fill=ce10000] (46.7690,-262.2919) node[above right] (text1320) {\scriptsize -20};



      \begin{scope}[cm={{1.04439,0.0,0.0,1.41697,(-96.54445,-493.13085)}},draw=blue,line cap=round,line join=round,line width=0.480pt]
        \path[draw=cd9d9d9,line cap=rect,line join=miter,line width=2.127pt,miter limit=4.00] (41.5000,8.5000) -- (41.5000,84.5000) -- (127.5000,84.5000) -- (127.5000,8.5000) -- (41.5000,8.5000);



      \end{scope}
      \begin{scope}[cm={{1.04439,0.0,0.0,1.41697,(-96.54445,-493.13085)}},draw=ca0a0a4,dash pattern=on 1.03pt off 1.03pt,line cap=round,line join=round,line width=0.257pt,miter limit=4.00]
        \path[draw,dash pattern=on 1.03pt off 1.03pt,line width=0.257pt,miter limit=4.00] (41.5000,74.5000) -- (127.5000,74.5000);



      \end{scope}
      \begin{scope}[cm={{1.04439,0.0,0.0,1.41697,(-96.54445,-493.13085)}},draw=blue,line cap=round,line join=round,line width=0.480pt]
        \path[cm={{1.15551,0.0,0.0,1.0,(-6.40682,0.0)}},draw] (41.5000,74.5000) -- (44.5000,74.5000);



        \path[cm={{1.15551,0.0,0.0,1.0,(-19.98953,0.0)}},draw] (127.5000,74.5000) -- (124.5000,74.5000);



      \end{scope}
      \begin{scope}[cm={{1.05023,0.0,0.0,1.05023,(-193.26478,-275.39592)}},draw=blue,line cap=rect,line join=bevel,line width=0.800pt]
        \path[fill=blue] (119.0842,-104.1987) node[above right] (text34-5) {\scriptsize 32};



      \end{scope}
      \begin{scope}[cm={{1.04439,0.0,0.0,1.41697,(-96.54445,-493.13085)}},draw=ca0a0a4,dash pattern=on 1.03pt off 1.03pt,line cap=round,line join=round,line width=0.257pt,miter limit=4.00]
        \path[draw,dash pattern=on 1.03pt off 1.03pt,line width=0.257pt,miter limit=4.00] (41.5000,48.5000) -- (127.5000,48.5000);



      \end{scope}
      \begin{scope}[cm={{1.04439,0.0,0.0,1.41697,(-96.54445,-493.13085)}},draw=blue,line cap=round,line join=round,line width=0.480pt]
        \path[cm={{1.15551,0.0,0.0,1.0,(-6.40682,0.0)}},draw] (41.5000,48.5000) -- (44.5000,48.5000);



        \path[cm={{1.15551,0.0,0.0,1.0,(-19.98953,0.0)}},draw] (127.5000,48.5000) -- (124.5000,48.5000);



      \end{scope}
      \begin{scope}[cm={{1.05023,0.0,0.0,1.05023,(-68.14069,-421.00079)}},draw=blue,line cap=rect,line join=bevel,line width=0.800pt]
        \path[fill=blue] (0.0000,0.0000) node[above right] (text64-7) {\scriptsize 36};



      \end{scope}
      \begin{scope}[cm={{1.04485,0.0,0.0,1.39271,(-88.8098,-492.60283)}},draw=ca0a0a4,dash pattern=on 1.56pt off 1.56pt,line cap=round,line join=round,line width=0.259pt,miter limit=4.00]
        \path[shift={(-7.44276,0)},draw,dash pattern=on 1.56pt off 1.56pt,line width=0.259pt,miter limit=4.00] (41.5000,22.5000) -- (46.5000,22.5000);



        \path[shift={(-7.44276,0)},draw,dash pattern=on 1.56pt off 1.56pt,line width=0.259pt,miter limit=4.00] (103.5000,22.5000) -- (127.5000,22.5000);



      \end{scope}
      \begin{scope}[cm={{1.04439,0.0,0.0,1.41697,(-96.54445,-493.13085)}},draw=blue,line cap=round,line join=round,line width=0.480pt]
        \path[cm={{1.15551,0.0,0.0,1.0,(-6.40682,0.0)}},draw] (41.5000,22.5000) -- (44.5000,22.5000);



        \path[cm={{1.15551,0.0,0.0,1.0,(-19.98953,0.0)}},draw] (127.5000,22.5000) -- (124.5000,22.5000);



      \end{scope}
      \begin{scope}[cm={{1.05023,0.0,0.0,1.05023,(-68.2079,-459.24995)}},draw=blue,line cap=rect,line join=bevel,line width=0.800pt]
        \path[fill=blue] (0.0000,0.0000) node[above right] (text96) {\scriptsize 40};



      \end{scope}
      \begin{scope}[cm={{1.04439,0.0,0.0,1.41697,(-96.54445,-493.13085)}},draw=ca0a0a4,dash pattern=on 0.40pt off 0.80pt,line cap=round,line join=round,line width=0.400pt]
        \path[draw] (41.5000,84.5000) -- (41.5000,8.5000);



      \end{scope}
      \begin{scope}[cm={{1.04439,0.0,0.0,1.41697,(-96.54445,-493.13085)}},draw=blue,line cap=round,line join=round,line width=0.480pt]
        \path[draw] (41.5000,84.5000) -- (41.5000,80.5000);



        \path[draw] (41.5000,8.5000) -- (41.5000,11.5000);



      \end{scope}
      \begin{scope}[cm={{1.04439,0.0,0.0,1.41697,(-96.54445,-493.13085)}},draw=blue,line cap=round,line join=round,line width=0.480pt]
        \path[cm={{1.0,0.0,0.0,0.66652,(0.0,27.84811)}},draw] (70.5000,84.5000) -- (70.5000,80.5000);



        \path[cm={{1.0,0.0,0.0,0.85167,(0.00012,1.21632)}},draw] (70.5000,8.5000) -- (70.5000,11.5000);



      \end{scope}
      \begin{scope}[cm={{1.04439,0.0,0.0,1.41697,(-96.54445,-493.13085)}},draw=blue,line cap=round,line join=round,line width=0.480pt]
        \path[cm={{1.0,0.0,0.0,0.66652,(0.0,27.84811)}},draw] (98.5000,84.5000) -- (98.5000,80.5000);



        \path[cm={{1.0,0.0,0.0,0.85167,(0.14844,1.21632)}},draw] (98.5000,8.5000) -- (98.5000,11.5000);



      \end{scope}
      \begin{scope}[cm={{1.04439,0.0,0.0,1.41697,(-96.54445,-493.13085)}},draw=ca0a0a4,dash pattern=on 0.40pt off 0.80pt,line cap=round,line join=round,line width=0.400pt]
        \path[draw] (127.5000,84.5000) -- (127.5000,8.5000);



      \end{scope}
      \begin{scope}[cm={{1.04439,0.0,0.0,1.41697,(-96.54445,-493.13085)}},draw=blue,line cap=round,line join=round,line width=0.480pt]
        \path[draw] (127.5000,84.5000) -- (127.5000,80.5000);



        \path[draw] (127.5000,8.5000) -- (127.5000,11.5000);



      \end{scope}
      \begin{scope}[cm={{1.05016,0.0,0.0,1.02141,(-35.68793,-461.26773)}},draw=blue,line cap=rect,line join=bevel,line width=0.800pt]
        \path[fill=blue] (0.0000,0.0000) node[above right] (text276) {\scriptsize $\Upsilon(t)$};



      \end{scope}
      \begin{scope}[cm={{0.76164,0.0,0.0,1.39271,(-67.04184,-490.7838)}},draw=blue,line cap=round,line join=round,line width=0.480pt]
        \path[draw,even odd rule] (71.5000,18.5000) -- (98.5000,18.5000);



      \end{scope}
      \begin{scope}[cm={{1.04439,0.0,0.0,1.41697,(-96.54445,-493.13085)}},draw=blue,line cap=round,line join=round,line width=0.480pt]
        \path[draw] (41.6000,17.0000) -- (41.6000,17.0000) -- (41.7000,17.8000) -- (41.9000,18.6000) -- (42.0000,19.4000) -- (42.2000,20.2000) -- (42.3000,20.9000) -- (42.5000,21.6000) -- (42.6000,22.4000) -- (42.7000,23.1000) -- (42.9000,23.8000) -- (43.0000,24.5000) -- (43.2000,25.2000) -- (43.3000,25.8000) -- (43.5000,26.5000) -- (43.6000,27.1000) -- (43.7000,27.7000) -- (43.9000,28.4000) -- (44.0000,29.0000) -- (44.2000,29.6000) -- (44.3000,30.1000) -- (44.5000,30.7000) -- (44.6000,31.3000) -- (44.7000,31.8000) -- (44.9000,32.4000) -- (45.0000,32.9000) -- (45.2000,33.4000) -- (45.3000,34.0000) -- (45.5000,34.5000) -- (45.6000,35.0000) -- (45.7000,35.5000) -- (45.9000,35.9000) -- (46.0000,36.4000) -- (46.2000,36.9000) -- (46.3000,37.3000) -- (46.5000,37.8000) -- (46.6000,38.2000) -- (46.7000,38.7000) -- (46.9000,39.1000) -- (47.0000,39.5000) -- (47.2000,39.9000) -- (47.3000,40.3000) -- (47.5000,40.7000) -- (47.6000,41.1000) -- (47.7000,41.5000) -- (47.9000,41.9000) -- (48.0000,42.2000) -- (48.2000,42.6000) -- (48.3000,43.0000) -- (48.5000,43.3000) -- (48.6000,43.6000) -- (48.7000,44.0000) -- (48.9000,44.3000) -- (49.0000,44.6000) -- (49.2000,45.0000) -- (49.3000,45.3000) -- (49.5000,45.6000) -- (49.6000,45.9000) -- (49.7000,46.2000) -- (49.9000,46.5000) -- (50.0000,46.8000) -- (50.2000,47.0000) -- (50.3000,47.3000) -- (50.5000,47.6000) -- (50.6000,47.8000) -- (50.7000,48.1000) -- (50.9000,48.4000) -- (51.0000,48.6000) -- (51.2000,48.8000) -- (51.3000,49.1000) -- (51.5000,49.3000) -- (51.6000,49.5000) -- (51.7000,49.8000) -- (51.9000,50.0000) -- (52.0000,50.2000) -- (52.2000,50.4000) -- (52.3000,50.6000) -- (52.5000,50.8000) -- (52.6000,51.0000) -- (52.7000,51.2000) -- (52.9000,51.4000) -- (53.0000,51.6000) -- (53.2000,51.7000) -- (53.3000,51.9000) -- (53.5000,52.1000) -- (53.6000,52.3000) -- (53.7000,52.4000) -- (53.9000,52.6000) -- (54.0000,52.7000) -- (54.2000,52.9000) -- (54.3000,53.0000) -- (54.5000,53.1000) -- (54.6000,53.3000) -- (54.7000,53.4000) -- (54.9000,53.5000) -- (55.0000,53.7000) -- (55.2000,53.8000) -- (55.3000,53.9000) -- (55.5000,54.0000) -- (55.6000,54.1000) -- (55.7000,54.2000) -- (55.9000,54.3000) -- (56.0000,54.4000) -- (56.2000,54.5000) -- (56.3000,54.6000) -- (56.5000,54.6000) -- (56.6000,54.7000) -- (56.7000,54.8000) -- (56.9000,54.9000) -- (57.0000,54.9000) -- (57.2000,55.0000) -- (57.3000,55.0000) -- (57.5000,55.1000) -- (57.6000,55.1000) -- (57.7000,55.2000) -- (57.9000,55.2000) -- (58.0000,55.2000) -- (58.2000,55.3000) -- (58.3000,55.3000) -- (58.5000,55.3000) -- (58.6000,55.3000) -- (58.7000,55.3000) -- (58.9000,55.3000) -- (59.0000,55.3000) -- (59.2000,55.3000) -- (59.3000,55.3000) -- (59.5000,55.3000) -- (59.6000,55.2000) -- (59.7000,55.2000) -- (59.9000,55.2000) -- (60.0000,55.1000) -- (60.2000,55.1000) -- (60.3000,55.0000) -- (60.5000,54.9000) -- (60.6000,54.9000) -- (60.7000,54.8000) -- (60.9000,54.7000) -- (61.0000,54.6000) -- (61.2000,54.5000) -- (61.3000,54.4000) -- (61.5000,54.2000) -- (61.6000,54.1000) -- (61.7000,54.0000) -- (61.9000,53.8000) -- (62.0000,53.7000) -- (62.2000,53.5000) -- (62.3000,53.3000) -- (62.5000,53.1000) -- (62.6000,53.0000) -- (62.7000,52.8000) -- (62.9000,52.6000) -- (63.0000,52.4000) -- (63.2000,52.2000) -- (63.3000,52.0000) -- (63.5000,51.7000) -- (63.6000,51.5000) -- (63.7000,51.3000) -- (63.9000,51.1000) -- (64.0000,50.8000) -- (64.2000,50.6000) -- (64.3000,50.4000) -- (64.5000,50.1000) -- (64.6000,49.9000) -- (64.7000,49.7000) -- (64.9000,49.4000) -- (65.0000,49.2000) -- (65.2000,49.0000) -- (65.3000,48.7000) -- (65.5000,48.5000) -- (65.6000,48.2000) -- (65.7000,48.0000) -- (65.9000,47.8000) -- (66.0000,47.5000) -- (66.2000,47.3000) -- (66.3000,47.1000) -- (66.5000,46.8000) -- (66.6000,46.6000) -- (66.7000,46.4000) -- (66.9000,46.1000) -- (67.0000,45.9000) -- (67.2000,45.7000) -- (67.3000,45.4000) -- (67.5000,45.2000) -- (67.6000,45.0000) -- (67.7000,44.8000) -- (67.9000,44.5000) -- (68.0000,44.3000) -- (68.2000,44.1000) -- (68.3000,43.9000) -- (68.5000,43.7000) -- (68.6000,43.5000) -- (68.7000,43.2000) -- (68.9000,43.0000) -- (69.0000,42.8000) -- (69.2000,42.6000) -- (69.3000,42.4000) -- (69.5000,42.2000) -- (69.6000,42.0000) -- (69.7000,41.8000) -- (69.9000,41.6000) -- (70.0000,41.4000) -- (70.2000,41.2000) -- (70.3000,41.0000) -- (70.5000,40.9000) -- (70.6000,40.7000) -- (70.7000,40.5000) -- (70.9000,40.3000) -- (71.0000,40.1000) -- (71.2000,40.0000) -- (71.3000,39.8000) -- (71.5000,39.6000) -- (71.6000,39.4000) -- (71.7000,39.3000) -- (71.9000,39.1000) -- (72.0000,38.9000) -- (72.2000,38.8000) -- (72.3000,38.6000) -- (72.5000,38.5000) -- (72.6000,38.3000) -- (72.7000,38.1000) -- (72.9000,38.0000) -- (73.0000,37.9000) -- (73.2000,37.7000) -- (73.3000,37.6000) -- (73.5000,37.4000) -- (73.6000,37.3000) -- (73.7000,37.1000) -- (73.9000,37.0000) -- (74.0000,36.9000) -- (74.2000,36.8000) -- (74.3000,36.6000) -- (74.5000,36.5000) -- (74.6000,36.4000) -- (74.7000,36.2000) -- (74.9000,36.1000) -- (75.0000,36.0000) -- (75.2000,35.9000) -- (75.3000,35.8000) -- (75.5000,35.7000) -- (75.6000,35.6000) -- (75.7000,35.5000) -- (75.9000,35.4000) -- (76.0000,35.3000) -- (76.2000,35.2000) -- (76.3000,35.1000) -- (76.5000,35.0000) -- (76.6000,34.9000) -- (76.7000,34.8000) -- (76.9000,34.7000) -- (77.0000,34.6000) -- (77.2000,34.5000) -- (77.3000,34.4000) -- (77.5000,34.4000) -- (77.6000,34.3000) -- (77.7000,34.2000) -- (77.9000,34.1000) -- (78.0000,34.1000) -- (78.2000,34.0000) -- (78.3000,33.9000) -- (78.5000,33.9000) -- (78.6000,33.8000) -- (78.7000,33.7000) -- (78.9000,33.7000) -- (79.0000,33.6000) -- (79.2000,33.6000) -- (79.3000,33.5000) -- (79.5000,33.5000) -- (79.6000,33.4000) -- (79.7000,33.4000) -- (79.9000,33.3000) -- (80.0000,33.3000) -- (80.2000,33.2000) -- (80.3000,33.2000) -- (80.5000,33.1000) -- (80.6000,33.1000) -- (80.7000,33.1000) -- (80.9000,33.0000) -- (81.0000,33.0000) -- (81.2000,33.0000) -- (81.3000,32.9000) -- (81.5000,32.9000) -- (81.6000,32.9000) -- (81.7000,32.9000) -- (81.9000,32.8000) -- (82.0000,32.8000) -- (82.2000,32.8000) -- (82.3000,32.8000) -- (82.5000,32.8000) -- (82.6000,32.7000) -- (82.7000,32.7000) -- (82.9000,32.7000) -- (83.0000,32.7000) -- (83.2000,32.7000) -- (83.3000,32.7000) -- (83.5000,32.7000) -- (83.6000,32.7000) -- (83.7000,32.7000) -- (83.9000,32.6000) -- (84.0000,32.6000) -- (84.2000,32.6000) -- (84.3000,32.6000) -- (84.5000,32.6000) -- (84.6000,32.6000) -- (84.7000,32.6000) -- (84.9000,32.7000) -- (85.0000,32.7000) -- (85.2000,32.7000) -- (85.3000,32.7000) -- (85.4000,32.7000) -- (85.6000,32.7000) -- (85.7000,32.7000) -- (85.9000,32.7000) -- (86.0000,32.7000) -- (86.2000,32.7000) -- (86.3000,32.8000) -- (86.4000,32.8000) -- (86.6000,32.8000) -- (86.7000,32.8000) -- (86.9000,32.8000) -- (87.0000,32.8000) -- (87.2000,32.9000) -- (87.3000,32.9000) -- (87.4000,32.9000) -- (87.6000,32.9000) -- (87.7000,33.0000) -- (87.9000,33.0000) -- (88.0000,33.0000) -- (88.2000,33.0000) -- (88.3000,33.1000) -- (88.4000,33.1000) -- (88.6000,33.1000) -- (88.7000,33.1000) -- (88.9000,33.2000) -- (89.0000,33.2000) -- (89.2000,33.2000) -- (89.3000,33.3000) -- (89.4000,33.3000) -- (89.6000,33.3000) -- (89.7000,33.4000) -- (89.9000,33.4000) -- (90.0000,33.4000) -- (90.2000,33.5000) -- (90.3000,33.5000) -- (90.4000,33.6000) -- (90.6000,33.6000) -- (90.7000,33.6000) -- (90.9000,33.7000) -- (91.0000,33.7000) -- (91.2000,33.8000) -- (91.3000,33.8000) -- (91.4000,33.8000) -- (91.6000,33.9000) -- (91.7000,33.9000) -- (91.9000,34.0000) -- (92.0000,34.0000) -- (92.2000,34.1000) -- (92.3000,34.1000) -- (92.4000,34.1000) -- (92.6000,34.2000) -- (92.7000,34.2000) -- (92.9000,34.3000) -- (93.0000,34.3000) -- (93.2000,34.4000) -- (93.3000,34.4000) -- (93.4000,34.5000) -- (93.6000,34.5000) -- (93.7000,34.6000) -- (93.9000,34.6000) -- (94.0000,34.7000) -- (94.2000,34.7000) -- (94.3000,34.8000) -- (94.4000,34.8000) -- (94.6000,34.9000) -- (94.7000,34.9000) -- (94.9000,35.0000) -- (95.0000,35.0000) -- (95.2000,35.1000) -- (95.3000,35.1000) -- (95.4000,35.2000) -- (95.6000,35.2000) -- (95.7000,35.3000) -- (95.9000,35.3000) -- (96.0000,35.4000) -- (96.2000,35.4000) -- (96.3000,35.5000) -- (96.4000,35.5000) -- (96.6000,35.6000) -- (96.7000,35.6000) -- (96.9000,35.7000) -- (97.0000,35.7000) -- (97.2000,35.8000) -- (97.3000,35.9000) -- (97.4000,35.9000) -- (97.6000,36.0000) -- (97.7000,36.0000) -- (97.9000,36.1000) -- (98.0000,36.1000) -- (98.2000,36.2000) -- (98.3000,36.2000) -- (98.4000,36.3000) -- (98.6000,36.4000) -- (98.7000,36.4000) -- (98.9000,36.5000) -- (99.0000,36.6000) -- (99.2000,36.6000) -- (99.3000,36.7000) -- (99.4000,36.8000) -- (99.6000,36.8000) -- (99.7000,36.9000) -- (99.9000,37.0000) -- (100.0000,37.0000) -- (100.2000,37.1000) -- (100.3000,37.2000) -- (100.4000,37.2000) -- (100.6000,37.3000) -- (100.7000,37.4000) -- (100.9000,37.4000) -- (101.0000,37.5000) -- (101.2000,37.6000) -- (101.3000,37.6000) -- (101.4000,37.7000) -- (101.6000,37.8000) -- (101.7000,37.9000) -- (101.9000,37.9000) -- (102.0000,38.0000) -- (102.2000,38.1000) -- (102.3000,38.2000) -- (102.4000,38.2000) -- (102.6000,38.3000) -- (102.7000,38.4000) -- (102.9000,38.4000) -- (103.0000,38.5000) -- (103.2000,38.6000) -- (103.3000,38.7000) -- (103.4000,38.7000) -- (103.6000,38.8000) -- (103.7000,38.9000) -- (103.9000,38.9000) -- (104.0000,39.0000) -- (104.2000,39.1000) -- (104.3000,39.2000) -- (104.4000,39.2000) -- (104.6000,39.3000) -- (104.7000,39.4000) -- (104.9000,39.4000) -- (105.0000,39.5000) -- (105.2000,39.6000) -- (105.3000,39.7000) -- (105.4000,39.7000) -- (105.6000,39.8000) -- (105.7000,39.9000) -- (105.9000,39.9000) -- (106.0000,40.0000) -- (106.2000,40.1000) -- (106.3000,40.1000) -- (106.4000,40.2000) -- (106.6000,40.3000) -- (106.7000,40.3000) -- (106.9000,40.4000) -- (107.0000,40.5000) -- (107.2000,40.5000) -- (107.3000,40.6000) -- (107.4000,40.6000) -- (107.6000,40.7000) -- (107.7000,40.8000) -- (107.9000,40.8000) -- (108.0000,40.9000) -- (108.2000,41.0000) -- (108.3000,41.0000) -- (108.4000,41.1000) -- (108.6000,41.1000) -- (108.7000,41.2000) -- (108.9000,41.3000) -- (109.0000,41.3000) -- (109.2000,41.4000) -- (109.3000,41.4000) -- (109.4000,41.5000) -- (109.6000,41.5000) -- (109.7000,41.6000) -- (109.9000,41.6000) -- (110.0000,41.7000) -- (110.2000,41.7000) -- (110.3000,41.8000) -- (110.4000,41.9000) -- (110.6000,41.9000) -- (110.7000,42.0000) -- (110.9000,42.0000) -- (111.0000,42.1000) -- (111.2000,42.1000) -- (111.3000,42.1000) -- (111.4000,42.2000) -- (111.6000,42.2000) -- (111.7000,42.3000) -- (111.9000,42.3000) -- (112.0000,42.4000) -- (112.2000,42.4000) -- (112.3000,42.5000) -- (112.4000,42.5000) -- (112.6000,42.6000) -- (112.7000,42.6000) -- (112.9000,42.6000) -- (113.0000,42.7000) -- (113.2000,42.7000) -- (113.3000,42.8000) -- (113.4000,42.8000) -- (113.6000,42.8000) -- (113.7000,42.9000) -- (113.9000,42.9000) -- (114.0000,43.0000) -- (114.2000,43.0000) -- (114.3000,43.0000) -- (114.4000,43.1000) -- (114.6000,43.1000) -- (114.7000,43.1000) -- (114.9000,43.2000) -- (115.0000,43.2000) -- (115.2000,43.2000) -- (115.3000,43.3000) -- (115.4000,43.3000) -- (115.6000,43.3000) -- (115.7000,43.4000) -- (115.9000,43.4000) -- (116.0000,43.4000) -- (116.2000,43.4000) -- (116.3000,43.5000) -- (116.4000,43.5000) -- (116.6000,43.5000) -- (116.7000,43.6000) -- (116.9000,43.6000) -- (117.0000,43.6000) -- (117.2000,43.6000) -- (117.3000,43.7000) -- (117.4000,43.7000) -- (117.6000,43.7000) -- (117.7000,43.7000) -- (117.9000,43.7000) -- (118.0000,43.8000) -- (118.2000,43.8000) -- (118.3000,43.8000) -- (118.4000,43.8000) -- (118.6000,43.9000) -- (118.7000,43.9000) -- (118.9000,43.9000) -- (119.0000,43.9000) -- (119.2000,43.9000) -- (119.3000,44.0000) -- (119.4000,44.0000) -- (119.6000,44.0000) -- (119.7000,44.0000) -- (119.9000,44.0000) -- (120.0000,44.0000) -- (120.2000,44.0000) -- (120.3000,44.1000) -- (120.4000,44.1000) -- (120.6000,44.1000) -- (120.7000,44.1000) -- (120.9000,44.1000) -- (121.0000,44.1000) -- (121.2000,44.1000) -- (121.3000,44.2000) -- (121.4000,44.2000) -- (121.6000,44.2000) -- (121.7000,44.2000) -- (121.9000,44.2000) -- (122.0000,44.2000) -- (122.2000,44.2000) -- (122.3000,44.2000) -- (122.4000,44.2000) -- (122.6000,44.2000) -- (122.7000,44.3000) -- (122.9000,44.3000) -- (123.0000,44.3000) -- (123.2000,44.3000) -- (123.3000,44.3000) -- (123.4000,44.3000) -- (123.6000,44.3000) -- (123.7000,44.3000) -- (123.9000,44.3000) -- (124.0000,44.3000) -- (124.2000,44.3000) -- (124.3000,44.3000) -- (124.4000,44.3000) -- (124.6000,44.3000) -- (124.7000,44.3000) -- (124.9000,44.3000) -- (125.0000,44.3000) -- (125.2000,44.3000) -- (125.3000,44.3000) -- (125.4000,44.4000) -- (125.6000,44.4000) -- (125.7000,44.4000) -- (125.9000,44.4000) -- (126.0000,44.4000) -- (126.2000,44.4000) -- (126.3000,44.4000) -- (126.4000,44.4000) -- (126.6000,44.4000) -- (126.7000,44.4000) -- (126.9000,44.4000) -- (127.0000,44.4000) -- (127.2000,44.4000) -- (127.3000,44.4000);



      \end{scope}
      \begin{scope}[cm={{1.05016,0.0,0.0,1.02141,(-33.29971,-448.94892)}},draw=blue,line cap=rect,line join=bevel,line width=0.800pt]
        \path[fill=blue] (0.0000,0.0000) node[above right] (text312) {\scriptsize $\hat{y}(t)$};



      \end{scope}
      \begin{scope}[cm={{1.04485,0.0,0.0,1.39271,(-87.20649,-490.7838)}},draw=cff0000,line cap=round,line join=round,line width=0.480pt]
        \path[draw,even odd rule,line width=0.410pt] (71.4187,26.5000) -- (91.1002,26.5000);



      \end{scope}
      \begin{scope}[cm={{1.04439,0.0,0.0,1.41697,(-96.54445,-493.13085)}},draw=cff0000,line cap=round,line join=round,line width=0.480pt]
        \path[draw] (41.6000,21.8000) -- (41.6000,21.8000) -- (41.7000,17.5000) -- (41.9000,18.2000) -- (42.0000,19.1000) -- (42.2000,20.0000) -- (42.3000,20.9000) -- (42.5000,21.7000) -- (42.6000,22.5000) -- (42.7000,23.3000) -- (42.9000,24.1000) -- (43.0000,24.8000) -- (43.2000,25.5000) -- (43.3000,26.2000) -- (43.5000,26.9000) -- (43.6000,27.6000) -- (43.7000,28.2000) -- (43.9000,28.8000) -- (44.0000,29.4000) -- (44.2000,30.0000) -- (44.3000,30.5000) -- (44.5000,31.1000) -- (44.6000,31.6000) -- (44.7000,32.1000) -- (44.9000,32.7000) -- (45.0000,33.2000) -- (45.2000,33.7000) -- (45.3000,34.2000) -- (45.5000,34.6000) -- (45.6000,35.1000) -- (45.7000,35.6000) -- (45.9000,36.1000) -- (46.0000,36.5000) -- (46.2000,37.0000) -- (46.3000,37.5000) -- (46.5000,37.9000) -- (46.6000,38.4000) -- (46.7000,38.8000) -- (46.9000,39.3000) -- (47.0000,39.7000) -- (47.2000,40.1000) -- (47.3000,40.6000) -- (47.5000,41.0000) -- (47.6000,41.4000) -- (47.7000,41.8000) -- (47.9000,42.2000) -- (48.0000,42.5000) -- (48.2000,42.9000) -- (48.3000,43.3000) -- (48.5000,43.6000) -- (48.6000,43.9000) -- (48.7000,44.3000) -- (48.9000,44.6000) -- (49.0000,44.9000) -- (49.2000,45.2000) -- (49.3000,45.5000) -- (49.5000,45.8000) -- (49.6000,46.0000) -- (49.7000,46.3000) -- (49.9000,46.6000) -- (50.0000,46.8000) -- (50.2000,47.1000) -- (50.3000,47.4000) -- (50.5000,47.6000) -- (50.6000,47.9000) -- (50.7000,48.1000) -- (50.9000,48.4000) -- (51.0000,48.7000) -- (51.2000,48.9000) -- (51.3000,49.2000) -- (51.5000,49.4000) -- (51.6000,49.7000) -- (51.7000,50.0000) -- (51.9000,50.2000) -- (52.0000,50.5000) -- (52.2000,50.8000) -- (52.3000,51.0000) -- (52.5000,51.3000) -- (52.6000,51.5000) -- (52.7000,51.7000) -- (52.9000,51.9000) -- (53.0000,52.1000) -- (53.2000,52.3000) -- (53.3000,52.5000) -- (53.5000,52.7000) -- (53.6000,52.8000) -- (53.7000,52.9000) -- (53.9000,53.1000) -- (54.0000,53.2000) -- (54.2000,53.3000) -- (54.3000,53.3000) -- (54.5000,53.4000) -- (54.6000,53.5000) -- (54.7000,53.5000) -- (54.9000,53.6000) -- (55.0000,53.6000) -- (55.2000,53.6000) -- (55.3000,53.7000) -- (55.5000,53.7000) -- (55.6000,53.7000) -- (55.7000,53.8000) -- (55.9000,53.8000) -- (56.0000,53.8000) -- (56.2000,53.9000) -- (56.3000,54.0000) -- (56.5000,54.0000) -- (56.6000,54.1000) -- (56.7000,54.2000) -- (56.9000,54.3000) -- (57.0000,54.4000) -- (57.2000,54.5000) -- (57.3000,54.6000) -- (57.5000,54.7000) -- (57.6000,54.8000) -- (57.7000,54.9000) -- (57.9000,55.0000) -- (58.0000,55.1000) -- (58.2000,55.2000) -- (58.3000,55.3000) -- (58.5000,55.3000) -- (58.6000,55.4000) -- (58.7000,55.4000) -- (58.9000,55.5000) -- (59.0000,55.5000) -- (59.2000,55.5000) -- (59.3000,55.5000) -- (59.5000,55.5000) -- (59.6000,55.5000) -- (59.7000,55.5000) -- (59.9000,55.4000) -- (60.0000,55.3000) -- (60.2000,55.3000) -- (60.3000,55.2000) -- (60.5000,55.1000) -- (60.6000,55.0000) -- (60.7000,54.9000) -- (60.9000,54.8000) -- (61.0000,54.6000) -- (61.2000,54.5000) -- (61.3000,54.4000) -- (61.5000,54.2000) -- (61.6000,54.0000) -- (61.7000,53.9000) -- (61.9000,53.7000) -- (62.0000,53.5000) -- (62.2000,53.3000) -- (62.3000,53.1000) -- (62.5000,52.9000) -- (62.6000,52.7000) -- (62.7000,52.5000) -- (62.9000,52.3000) -- (63.0000,52.1000) -- (63.2000,51.9000) -- (63.3000,51.7000) -- (63.5000,51.5000) -- (63.6000,51.3000) -- (63.7000,51.1000) -- (63.9000,50.9000) -- (64.0000,50.7000) -- (64.2000,50.5000) -- (64.3000,50.3000) -- (64.5000,50.0000) -- (64.6000,49.8000) -- (64.7000,49.6000) -- (64.9000,49.4000) -- (65.0000,49.2000) -- (65.2000,49.0000) -- (65.3000,48.8000) -- (65.5000,48.6000) -- (65.6000,48.3000) -- (65.7000,48.1000) -- (65.9000,47.9000) -- (66.0000,47.7000) -- (66.2000,47.5000) -- (66.3000,47.3000) -- (66.5000,47.0000) -- (66.6000,46.8000) -- (66.7000,46.6000) -- (66.9000,46.4000) -- (67.0000,46.2000) -- (67.2000,45.9000) -- (67.3000,45.7000) -- (67.5000,45.5000) -- (67.6000,45.3000) -- (67.7000,45.0000) -- (67.9000,44.8000) -- (68.0000,44.6000) -- (68.2000,44.3000) -- (68.3000,44.1000) -- (68.5000,43.9000) -- (68.6000,43.6000) -- (68.7000,43.4000) -- (68.9000,43.2000) -- (69.0000,43.0000) -- (69.2000,42.7000) -- (69.3000,42.5000) -- (69.5000,42.3000) -- (69.6000,42.1000) -- (69.7000,41.8000) -- (69.9000,41.6000) -- (70.0000,41.4000) -- (70.2000,41.2000) -- (70.3000,41.0000) -- (70.5000,40.8000) -- (70.6000,40.6000) -- (70.7000,40.4000) -- (70.9000,40.2000) -- (71.0000,40.0000) -- (71.2000,39.8000) -- (71.3000,39.6000) -- (71.5000,39.4000) -- (71.6000,39.2000) -- (71.7000,39.0000) -- (71.9000,38.9000) -- (72.0000,38.7000) -- (72.2000,38.5000) -- (72.3000,38.3000) -- (72.5000,38.2000) -- (72.6000,38.0000) -- (72.7000,37.9000) -- (72.9000,37.7000) -- (73.0000,37.6000) -- (73.2000,37.4000) -- (73.3000,37.3000) -- (73.5000,37.2000) -- (73.6000,37.0000) -- (73.7000,36.9000) -- (73.9000,36.8000) -- (74.0000,36.7000) -- (74.2000,36.5000) -- (74.3000,36.4000) -- (74.5000,36.3000) -- (74.6000,36.2000) -- (74.7000,36.1000) -- (74.9000,36.0000) -- (75.0000,35.9000) -- (75.2000,35.8000) -- (75.3000,35.7000) -- (75.5000,35.6000) -- (75.6000,35.5000) -- (75.7000,35.4000) -- (75.9000,35.3000) -- (76.0000,35.3000) -- (76.2000,35.2000) -- (76.3000,35.1000) -- (76.5000,35.0000) -- (76.6000,34.9000) -- (76.7000,34.9000) -- (76.9000,34.8000) -- (77.0000,34.7000) -- (77.2000,34.6000) -- (77.3000,34.6000) -- (77.5000,34.5000) -- (77.6000,34.4000) -- (77.7000,34.4000) -- (77.9000,34.3000) -- (78.0000,34.3000) -- (78.2000,34.2000) -- (78.3000,34.1000) -- (78.5000,34.1000) -- (78.6000,34.0000) -- (78.7000,34.0000) -- (78.9000,33.9000) -- (79.0000,33.9000) -- (79.2000,33.8000) -- (79.3000,33.7000) -- (79.5000,33.7000) -- (79.6000,33.6000) -- (79.7000,33.6000) -- (79.9000,33.5000) -- (80.0000,33.5000) -- (80.2000,33.5000) -- (80.3000,33.4000) -- (80.5000,33.4000) -- (80.6000,33.3000) -- (80.7000,33.3000) -- (80.9000,33.2000) -- (81.0000,33.2000) -- (81.2000,33.2000) -- (81.3000,33.1000) -- (81.5000,33.1000) -- (81.6000,33.0000) -- (81.7000,33.0000) -- (81.9000,33.0000) -- (82.0000,32.9000) -- (82.2000,32.9000) -- (82.3000,32.9000) -- (82.5000,32.9000) -- (82.6000,32.8000) -- (82.7000,32.8000) -- (82.9000,32.8000) -- (83.0000,32.8000) -- (83.2000,32.7000) -- (83.3000,32.7000) -- (83.5000,32.7000) -- (83.6000,32.7000) -- (83.7000,32.7000) -- (83.9000,32.7000) -- (84.0000,32.6000) -- (84.2000,32.6000) -- (84.3000,32.6000) -- (84.5000,32.6000) -- (84.6000,32.6000) -- (84.7000,32.6000) -- (84.9000,32.6000) -- (85.0000,32.6000) -- (85.2000,32.6000) -- (85.3000,32.6000) -- (85.4000,32.6000) -- (85.6000,32.6000) -- (85.7000,32.6000) -- (85.9000,32.6000) -- (86.0000,32.6000) -- (86.2000,32.6000) -- (86.3000,32.6000) -- (86.4000,32.7000) -- (86.6000,32.7000) -- (86.7000,32.7000) -- (86.9000,32.7000) -- (87.0000,32.7000) -- (87.2000,32.7000) -- (87.3000,32.8000) -- (87.4000,32.8000) -- (87.6000,32.8000) -- (87.7000,32.8000) -- (87.9000,32.8000) -- (88.0000,32.9000) -- (88.2000,32.9000) -- (88.3000,32.9000) -- (88.4000,33.0000) -- (88.6000,33.0000) -- (88.7000,33.0000) -- (88.9000,33.1000) -- (89.0000,33.1000) -- (89.2000,33.1000) -- (89.3000,33.2000) -- (89.4000,33.2000) -- (89.6000,33.2000) -- (89.7000,33.3000) -- (89.9000,33.3000) -- (90.0000,33.4000) -- (90.2000,33.4000) -- (90.3000,33.5000) -- (90.4000,33.5000) -- (90.6000,33.5000) -- (90.7000,33.6000) -- (90.9000,33.6000) -- (91.0000,33.7000) -- (91.2000,33.7000) -- (91.3000,33.8000) -- (91.4000,33.8000) -- (91.6000,33.9000) -- (91.7000,33.9000) -- (91.9000,34.0000) -- (92.0000,34.0000) -- (92.2000,34.1000) -- (92.3000,34.1000) -- (92.4000,34.2000) -- (92.6000,34.2000) -- (92.7000,34.3000) -- (92.9000,34.3000) -- (93.0000,34.4000) -- (93.2000,34.4000) -- (93.3000,34.5000) -- (93.4000,34.5000) -- (93.6000,34.6000) -- (93.7000,34.6000) -- (93.9000,34.7000) -- (94.0000,34.7000) -- (94.2000,34.8000) -- (94.3000,34.8000) -- (94.4000,34.9000) -- (94.6000,34.9000) -- (94.7000,35.0000) -- (94.9000,35.1000) -- (95.0000,35.1000) -- (95.2000,35.2000) -- (95.3000,35.2000) -- (95.4000,35.3000) -- (95.6000,35.3000) -- (95.7000,35.4000) -- (95.9000,35.4000) -- (96.0000,35.5000) -- (96.2000,35.5000) -- (96.3000,35.6000) -- (96.4000,35.6000) -- (96.6000,35.7000) -- (96.7000,35.7000) -- (96.9000,35.8000) -- (97.0000,35.9000) -- (97.2000,35.9000) -- (97.3000,36.0000) -- (97.4000,36.0000) -- (97.6000,36.1000) -- (97.7000,36.1000) -- (97.9000,36.2000) -- (98.0000,36.2000) -- (98.2000,36.3000) -- (98.3000,36.3000) -- (98.4000,36.4000) -- (98.6000,36.5000) -- (98.7000,36.5000) -- (98.9000,36.6000) -- (99.0000,36.6000) -- (99.2000,36.7000) -- (99.3000,36.8000) -- (99.4000,36.8000) -- (99.6000,36.9000) -- (99.7000,37.0000) -- (99.9000,37.0000) -- (100.0000,37.1000) -- (100.2000,37.1000) -- (100.3000,37.2000) -- (100.4000,37.3000) -- (100.6000,37.3000) -- (100.7000,37.4000) -- (100.9000,37.5000) -- (101.0000,37.5000) -- (101.2000,37.6000) -- (101.3000,37.7000) -- (101.4000,37.7000) -- (101.6000,37.8000) -- (101.7000,37.9000) -- (101.9000,37.9000) -- (102.0000,38.0000) -- (102.2000,38.1000) -- (102.3000,38.1000) -- (102.4000,38.2000) -- (102.6000,38.3000) -- (102.7000,38.4000) -- (102.9000,38.4000) -- (103.0000,38.5000) -- (103.2000,38.6000) -- (103.3000,38.6000) -- (103.4000,38.7000) -- (103.6000,38.8000) -- (103.7000,38.8000) -- (103.9000,38.9000) -- (104.0000,39.0000) -- (104.2000,39.0000) -- (104.3000,39.1000) -- (104.4000,39.2000) -- (104.6000,39.2000) -- (104.7000,39.3000) -- (104.9000,39.4000) -- (105.0000,39.4000) -- (105.2000,39.5000) -- (105.3000,39.6000) -- (105.4000,39.6000) -- (105.6000,39.7000) -- (105.7000,39.8000) -- (105.9000,39.8000) -- (106.0000,39.9000) -- (106.2000,40.0000) -- (106.3000,40.0000) -- (106.4000,40.1000) -- (106.6000,40.2000) -- (106.7000,40.2000) -- (106.9000,40.3000) -- (107.0000,40.4000) -- (107.2000,40.4000) -- (107.3000,40.5000) -- (107.4000,40.5000) -- (107.6000,40.6000) -- (107.7000,40.7000) -- (107.9000,40.7000) -- (108.0000,40.8000) -- (108.2000,40.9000) -- (108.3000,40.9000) -- (108.4000,41.0000) -- (108.6000,41.0000) -- (108.7000,41.1000) -- (108.9000,41.2000) -- (109.0000,41.2000) -- (109.2000,41.3000) -- (109.3000,41.3000) -- (109.4000,41.4000) -- (109.6000,41.4000) -- (109.7000,41.5000) -- (109.9000,41.5000) -- (110.0000,41.6000) -- (110.2000,41.7000) -- (110.3000,41.7000) -- (110.4000,41.8000) -- (110.6000,41.8000) -- (110.7000,41.9000) -- (110.9000,41.9000) -- (111.0000,42.0000) -- (111.2000,42.0000) -- (111.3000,42.1000) -- (111.4000,42.1000) -- (111.6000,42.2000) -- (111.7000,42.2000) -- (111.9000,42.3000) -- (112.0000,42.3000) -- (112.2000,42.4000) -- (112.3000,42.4000) -- (112.4000,42.5000) -- (112.6000,42.5000) -- (112.7000,42.6000) -- (112.9000,42.6000) -- (113.0000,42.6000) -- (113.2000,42.7000) -- (113.3000,42.7000) -- (113.4000,42.8000) -- (113.6000,42.8000) -- (113.7000,42.9000) -- (113.9000,42.9000) -- (114.0000,42.9000) -- (114.2000,43.0000) -- (114.3000,43.0000) -- (114.4000,43.1000) -- (114.6000,43.1000) -- (114.7000,43.1000) -- (114.9000,43.2000) -- (115.0000,43.2000) -- (115.2000,43.2000) -- (115.3000,43.3000) -- (115.4000,43.3000) -- (115.6000,43.3000) -- (115.7000,43.4000) -- (115.9000,43.4000) -- (116.0000,43.4000) -- (116.2000,43.5000) -- (116.3000,43.5000) -- (116.4000,43.5000) -- (116.6000,43.6000) -- (116.7000,43.6000) -- (116.9000,43.6000) -- (117.0000,43.6000) -- (117.2000,43.7000) -- (117.3000,43.7000) -- (117.4000,43.7000) -- (117.6000,43.7000) -- (117.7000,43.8000) -- (117.9000,43.8000) -- (118.0000,43.8000) -- (118.2000,43.8000) -- (118.3000,43.9000) -- (118.4000,43.9000) -- (118.6000,43.9000) -- (118.7000,43.9000) -- (118.9000,44.0000) -- (119.0000,44.0000) -- (119.2000,44.0000) -- (119.3000,44.0000) -- (119.4000,44.0000) -- (119.6000,44.1000) -- (119.7000,44.1000) -- (119.9000,44.1000) -- (120.0000,44.1000) -- (120.2000,44.1000) -- (120.3000,44.1000) -- (120.4000,44.1000) -- (120.6000,44.2000) -- (120.7000,44.2000) -- (120.9000,44.2000) -- (121.0000,44.2000) -- (121.2000,44.2000) -- (121.3000,44.2000) -- (121.4000,44.2000) -- (121.6000,44.3000) -- (121.7000,44.3000) -- (121.9000,44.3000) -- (122.0000,44.3000) -- (122.2000,44.3000) -- (122.3000,44.3000) -- (122.4000,44.3000) -- (122.6000,44.3000) -- (122.7000,44.3000) -- (122.9000,44.3000) -- (123.0000,44.3000) -- (123.2000,44.4000) -- (123.3000,44.4000) -- (123.4000,44.4000) -- (123.6000,44.4000) -- (123.7000,44.4000) -- (123.9000,44.4000) -- (124.0000,44.4000) -- (124.2000,44.4000) -- (124.3000,44.4000) -- (124.4000,44.4000) -- (124.6000,44.4000) -- (124.7000,44.4000) -- (124.9000,44.4000) -- (125.0000,44.4000) -- (125.2000,44.4000) -- (125.3000,44.4000) -- (125.4000,44.4000) -- (125.6000,44.4000) -- (125.7000,44.4000) -- (125.9000,44.4000) -- (126.0000,44.4000) -- (126.2000,44.4000) -- (126.3000,44.4000) -- (126.4000,44.4000) -- (126.6000,44.4000) -- (126.7000,44.4000) -- (126.9000,44.4000) -- (127.0000,44.4000) -- (127.2000,44.4000) -- (127.3000,44.4000);



      \end{scope}
      \begin{scope}[cm={{1.04439,0.0,0.0,1.41697,(-96.54445,-495.32852)}},draw=blue,line cap=round,line join=round,line width=0.480pt]
        \path[draw=blue] (41.5000,10.0509) -- (41.5000,86.0509) -- (127.5000,86.0509) -- (127.5000,10.0509) -- (41.5000,10.0509);



      \end{scope}
      \begin{scope}[cm={{1.27491,0.0,0.0,1.41542,(-124.58597,-493.0531)}},fill=cffffff]
        \path[fill,rounded corners=0.0000cm] (81.0000,50.0000) rectangle (121.0000,78.0000);



      \end{scope}
      \begin{scope}[cm={{1.27491,0.0,0.0,1.41542,(-124.58597,-493.0531)}},draw=ca0a0a4,dash pattern=on 0.93pt off 0.93pt,line cap=round,line join=round,line width=0.233pt,miter limit=4.00]
        \path[draw,dash pattern=on 0.93pt off 0.93pt,line width=0.233pt,miter limit=4.00] (81.5000,55.5000) -- (121.5000,55.5000);



      \end{scope}
      \begin{scope}[cm={{1.27491,0.0,0.0,1.41542,(-124.58597,-493.0531)}},draw=blue,line cap=round,line join=round,line width=0.480pt]
        \path[draw] (81.5000,55.5000) -- (82.6777,55.5000);



        \path[draw] (121.5000,55.5000) -- (120.1220,55.5000);



      \end{scope}
      \begin{scope}[cm={{1.27491,0.0,0.0,1.41542,(-124.58597,-493.0531)}},draw=ca0a0a4,dash pattern=on 0.93pt off 0.93pt,line cap=round,line join=round,line width=0.233pt,miter limit=4.00]
        \path[draw,dash pattern=on 0.93pt off 0.93pt,line width=0.233pt,miter limit=4.00] (97.5000,78.5000) -- (97.5000,50.5000);



      \end{scope}
      \begin{scope}[cm={{1.27491,0.0,0.0,1.41542,(-124.58597,-493.0531)}},draw=blue,line cap=round,line join=round,line width=0.480pt]
        \path[draw] (97.5000,50.5000) -- (97.5000,50.5000) -- (97.5000,51.6564);



      \end{scope}
      \begin{scope}[cm={{1.27491,0.0,0.0,1.41542,(-124.58597,-493.0531)}},draw=blue,line cap=round,line join=round,line width=0.480pt]
        \path[draw] (81.5000,50.5000) -- (81.5000,78.5000) -- (121.5000,78.5000) -- (121.5000,50.5000) -- (81.5000,50.5000);



      \end{scope}
      \begin{scope}[cm={{1.27491,0.0,0.0,1.41542,(-124.58597,-493.0531)}},draw=blue,line cap=round,line join=round,line width=0.480pt]
        \path[draw] (81.2000,77.5000) -- (81.2000,77.5000) -- (81.2000,77.5000) -- (81.2000,77.5000) -- (81.2000,77.5000) -- (81.2000,77.5000) -- (81.2000,77.5000) -- (81.2000,77.5000) -- (81.2000,77.5000) -- (81.2000,77.5000) -- (81.2000,77.5000) -- (81.2000,77.5000) -- (81.2000,77.5000) -- (81.2000,77.5000) -- (81.2000,77.5000) -- (81.2000,77.5000) -- (81.2000,77.4000) -- (81.2000,77.4000) -- (81.2000,77.4000) -- (81.2000,77.4000) -- (81.2000,77.4000) -- (81.2000,77.4000) -- (81.2000,77.4000) -- (81.2000,77.4000) -- (81.2000,77.4000) -- (81.2000,77.4000) -- (81.3000,77.4000) -- (81.3000,77.4000) -- (81.3000,77.4000) -- (81.3000,77.4000) -- (81.3000,77.4000) -- (81.3000,77.4000) -- (81.3000,77.4000) -- (81.3000,77.4000) -- (81.3000,77.4000) -- (81.3000,77.4000) -- (81.3000,77.4000) -- (81.3000,77.3000) -- (81.3000,77.3000) -- (81.3000,77.3000) -- (81.3000,77.3000) -- (81.3000,77.3000) -- (81.3000,77.3000) -- (81.3000,77.3000) -- (81.3000,77.3000) -- (81.3000,77.3000) -- (81.3000,77.3000) -- (81.3000,77.3000) -- (81.3000,77.3000) -- (81.3000,77.3000) -- (81.3000,77.3000) -- (81.3000,77.3000) -- (81.3000,77.3000) -- (81.3000,77.3000) -- (81.3000,77.3000) -- (81.3000,77.3000) -- (81.3000,77.3000) -- (81.3000,77.2000) -- (81.3000,77.2000) -- (81.3000,77.2000) -- (81.3000,77.2000) -- (81.3000,77.2000) -- (81.3000,77.2000) -- (81.3000,77.2000) -- (81.3000,77.2000) -- (81.3000,77.2000) -- (81.3000,77.2000) -- (81.3000,77.2000) -- (81.3000,77.2000) -- (81.3000,77.2000) -- (81.3000,77.2000) -- (81.3000,77.2000) -- (81.3000,77.2000) -- (81.3000,77.2000) -- (81.3000,77.2000) -- (81.3000,77.2000) -- (81.4000,77.2000) -- (81.4000,77.2000) -- (81.4000,77.1000) -- (81.4000,77.1000) -- (81.4000,77.1000) -- (81.4000,77.1000) -- (81.4000,77.1000) -- (81.4000,77.1000) -- (81.4000,77.1000) -- (81.4000,77.1000) -- (81.4000,77.1000) -- (81.4000,77.1000) -- (81.4000,77.1000) -- (81.4000,77.1000) -- (81.4000,77.1000) -- (81.4000,77.1000) -- (81.4000,77.1000) -- (81.4000,77.1000) -- (81.4000,77.1000) -- (81.4000,77.1000) -- (81.4000,77.1000) -- (81.4000,77.1000) -- (81.4000,77.1000) -- (81.4000,77.0000) -- (81.4000,77.0000) -- (81.4000,77.0000) -- (81.4000,77.0000) -- (81.4000,77.0000) -- (81.4000,77.0000) -- (81.4000,77.0000) -- (81.4000,77.0000) -- (81.4000,77.0000) -- (81.4000,77.0000) -- (81.4000,77.0000) -- (81.4000,77.0000) -- (81.4000,77.0000) -- (81.4000,77.0000) -- (81.4000,77.0000) -- (81.4000,77.0000) -- (81.4000,77.0000) -- (81.4000,77.0000) -- (81.4000,77.0000) -- (81.4000,77.0000) -- (81.4000,77.0000) -- (81.4000,76.9000) -- (81.4000,76.9000) -- (81.4000,76.9000) -- (81.4000,76.9000) -- (81.4000,76.9000) -- (81.5000,76.9000) -- (81.5000,76.9000) -- (81.5000,76.9000) -- (81.5000,76.9000) -- (81.5000,76.9000) -- (81.5000,76.9000) -- (81.5000,76.9000) -- (81.5000,76.9000) -- (81.5000,76.9000) -- (81.5000,76.9000) -- (81.5000,76.9000) -- (81.5000,76.9000) -- (81.5000,76.9000) -- (81.5000,76.9000) -- (81.5000,76.9000) -- (81.5000,76.9000) -- (81.5000,76.8000) -- (81.5000,76.8000) -- (81.5000,76.8000) -- (81.5000,76.8000) -- (81.5000,76.8000) -- (81.5000,76.8000) -- (81.5000,76.8000) -- (81.5000,76.8000) -- (81.5000,76.8000) -- (81.5000,76.8000) -- (81.5000,76.8000) -- (81.5000,76.8000) -- (81.5000,76.8000) -- (81.5000,76.8000) -- (81.5000,76.8000) -- (81.5000,76.8000) -- (81.5000,76.8000) -- (81.5000,76.8000) -- (81.5000,76.8000) -- (81.5000,76.8000) -- (81.5000,76.7000) -- (81.5000,76.7000) -- (81.5000,76.7000) -- (81.5000,76.7000) -- (81.5000,76.7000) -- (81.5000,76.7000) -- (81.5000,76.7000) -- (81.5000,76.7000) -- (81.5000,76.7000) -- (81.5000,76.7000) -- (81.5000,76.7000) -- (81.5000,76.7000) -- (81.5000,76.7000) -- (81.5000,76.7000) -- (81.6000,76.7000) -- (81.6000,76.7000) -- (81.6000,76.7000) -- (81.6000,76.7000) -- (81.6000,76.7000) -- (81.6000,76.7000) -- (81.6000,76.7000) -- (81.6000,76.6000) -- (81.6000,76.6000) -- (81.6000,76.6000) -- (81.6000,76.6000) -- (81.6000,76.6000) -- (81.6000,76.6000) -- (81.6000,76.6000) -- (81.6000,76.6000) -- (81.6000,76.6000) -- (81.6000,76.6000) -- (81.6000,76.6000) -- (81.6000,76.6000) -- (81.6000,76.6000) -- (81.6000,76.6000) -- (81.6000,76.6000) -- (81.6000,76.6000) -- (81.6000,76.6000) -- (81.6000,76.6000) -- (81.6000,76.6000) -- (81.6000,76.6000) -- (81.6000,76.6000) -- (81.6000,76.5000) -- (81.6000,76.5000) -- (81.6000,76.5000) -- (81.6000,76.5000) -- (81.6000,76.5000) -- (81.6000,76.5000) -- (81.6000,76.5000) -- (81.6000,76.5000) -- (81.6000,76.5000) -- (81.6000,76.5000) -- (81.6000,76.5000) -- (81.6000,76.5000) -- (81.6000,76.5000) -- (81.6000,76.5000) -- (81.6000,76.5000) -- (81.6000,76.5000) -- (81.6000,76.5000) -- (81.6000,76.5000) -- (81.6000,76.5000) -- (81.6000,76.5000) -- (81.6000,76.5000) -- (81.7000,76.4000) -- (81.7000,76.4000) -- (81.7000,76.4000) -- (81.7000,76.4000) -- (81.7000,76.4000) -- (81.7000,76.4000) -- (81.7000,76.4000) -- (81.7000,76.4000) -- (81.7000,76.4000) -- (81.7000,76.4000) -- (81.7000,76.4000) -- (81.7000,76.4000) -- (81.7000,76.4000) -- (81.7000,76.4000) -- (81.7000,76.4000) -- (81.7000,76.4000) -- (81.7000,76.4000) -- (81.7000,76.4000) -- (81.7000,76.4000) -- (81.7000,76.4000) -- (81.7000,76.4000) -- (81.7000,76.3000) -- (81.7000,76.3000) -- (81.7000,76.3000) -- (81.7000,76.3000) -- (81.7000,76.3000) -- (81.7000,76.3000) -- (81.7000,76.3000) -- (81.7000,76.3000) -- (81.7000,76.3000) -- (81.7000,76.3000) -- (81.7000,76.3000) -- (81.7000,76.3000) -- (81.7000,76.3000) -- (81.7000,76.3000) -- (81.7000,76.3000) -- (81.7000,76.3000) -- (81.7000,76.3000) -- (81.7000,76.3000) -- (81.7000,76.3000) -- (81.7000,76.3000) -- (81.7000,76.2000) -- (81.7000,76.2000) -- (81.7000,76.2000) -- (81.7000,76.2000) -- (81.7000,76.2000) -- (81.7000,76.2000) -- (81.7000,76.2000) -- (81.7000,76.2000) -- (81.7000,76.2000) -- (81.8000,76.2000) -- (81.8000,76.2000) -- (81.8000,76.2000) -- (81.8000,76.2000) -- (81.8000,76.2000) -- (81.8000,76.2000) -- (81.8000,76.2000) -- (81.8000,76.2000) -- (81.8000,76.2000) -- (81.8000,76.2000) -- (81.8000,76.2000) -- (81.8000,76.2000) -- (81.8000,76.1000) -- (81.8000,76.1000) -- (81.8000,76.1000) -- (81.8000,76.1000) -- (81.8000,76.1000) -- (81.8000,76.1000) -- (81.8000,76.1000) -- (81.8000,76.1000) -- (81.8000,76.1000) -- (81.8000,76.1000) -- (81.8000,76.1000) -- (81.8000,76.1000) -- (81.8000,76.1000) -- (81.8000,76.1000) -- (81.8000,76.1000) -- (81.8000,76.1000) -- (81.8000,76.1000) -- (81.8000,76.1000) -- (81.8000,76.1000) -- (81.8000,76.1000) -- (81.8000,76.1000) -- (81.8000,76.0000) -- (81.8000,76.0000) -- (81.8000,76.0000) -- (81.8000,76.0000) -- (81.8000,76.0000) -- (81.8000,76.0000) -- (81.8000,76.0000) -- (81.8000,76.0000) -- (81.8000,76.0000) -- (81.8000,76.0000) -- (81.8000,76.0000) -- (81.8000,76.0000) -- (81.8000,76.0000) -- (81.8000,76.0000) -- (81.8000,76.0000) -- (81.8000,76.0000) -- (81.9000,76.0000) -- (81.9000,76.0000) -- (81.9000,76.0000) -- (81.9000,76.0000) -- (81.9000,76.0000) -- (81.9000,75.9000) -- (81.9000,75.9000) -- (81.9000,75.9000) -- (81.9000,75.9000) -- (81.9000,75.9000) -- (81.9000,75.9000) -- (81.9000,75.9000) -- (81.9000,75.9000) -- (81.9000,75.9000) -- (81.9000,75.9000) -- (81.9000,75.9000) -- (81.9000,75.9000) -- (81.9000,75.9000) -- (81.9000,75.9000) -- (81.9000,75.9000) -- (81.9000,75.9000) -- (81.9000,75.9000) -- (81.9000,75.9000) -- (81.9000,75.9000) -- (81.9000,75.9000) -- (81.9000,75.8000) -- (81.9000,75.8000) -- (81.9000,75.8000) -- (81.9000,75.8000) -- (81.9000,75.8000) -- (81.9000,75.8000) -- (81.9000,75.8000) -- (81.9000,75.8000) -- (81.9000,75.8000) -- (81.9000,75.8000) -- (81.9000,75.8000) -- (81.9000,75.8000) -- (81.9000,75.8000) -- (81.9000,75.8000) -- (81.9000,75.8000) -- (81.9000,75.8000) -- (81.9000,75.8000) -- (81.9000,75.8000) -- (81.9000,75.8000) -- (81.9000,75.8000) -- (81.9000,75.8000) -- (81.9000,75.7000) -- (81.9000,75.7000) -- (81.9000,75.7000) -- (81.9000,75.7000) -- (82.0000,75.7000) -- (82.0000,75.7000) -- (82.0000,75.7000) -- (82.0000,75.7000) -- (82.0000,75.7000) -- (82.0000,75.7000) -- (82.0000,75.7000) -- (82.0000,75.7000) -- (82.0000,75.7000) -- (82.0000,75.7000) -- (82.0000,75.7000) -- (82.0000,75.7000) -- (82.0000,75.7000) -- (82.0000,75.7000) -- (82.0000,75.7000) -- (82.0000,75.7000) -- (82.0000,75.7000) -- (82.0000,75.6000) -- (82.0000,75.6000) -- (82.0000,75.6000) -- (82.0000,75.6000) -- (82.0000,75.6000) -- (82.0000,75.6000) -- (82.0000,75.6000) -- (82.0000,75.6000) -- (82.0000,75.6000) -- (82.0000,75.6000) -- (82.0000,75.6000) -- (82.0000,75.6000) -- (82.0000,75.6000) -- (82.0000,75.6000) -- (82.0000,75.6000) -- (82.0000,75.6000) -- (82.0000,75.6000) -- (82.0000,75.6000) -- (82.0000,75.6000) -- (82.0000,75.6000) -- (82.0000,75.6000) -- (82.0000,75.5000) -- (82.0000,75.5000) -- (82.0000,75.5000) -- (82.0000,75.5000) -- (82.0000,75.5000) -- (82.0000,75.5000) -- (82.0000,75.5000) -- (82.0000,75.5000) -- (82.0000,75.5000) -- (82.0000,75.5000) -- (82.0000,75.5000) -- (82.1000,75.5000) -- (82.1000,75.5000) -- (82.1000,75.5000) -- (82.1000,75.5000) -- (82.1000,75.5000) -- (82.1000,75.5000) -- (82.1000,75.5000) -- (82.1000,75.5000) -- (82.1000,75.5000) -- (82.1000,75.5000) -- (82.1000,75.4000) -- (82.1000,75.4000) -- (82.1000,75.4000) -- (82.1000,75.4000) -- (82.1000,75.4000) -- (82.1000,75.4000) -- (82.1000,75.4000) -- (82.1000,75.4000) -- (82.1000,75.4000) -- (82.1000,75.4000) -- (82.1000,75.4000) -- (82.1000,75.4000) -- (82.1000,75.4000) -- (82.1000,75.4000) -- (82.1000,75.4000) -- (82.1000,75.4000) -- (82.1000,75.4000) -- (82.1000,75.4000) -- (82.1000,75.4000) -- (82.1000,75.4000) -- (82.1000,75.3000) -- (82.1000,75.3000) -- (82.1000,75.3000) -- (82.1000,75.3000) -- (82.1000,75.3000) -- (82.1000,75.3000) -- (82.1000,75.3000) -- (82.1000,75.3000) -- (82.1000,75.3000) -- (82.1000,75.3000) -- (82.1000,75.3000) -- (82.1000,75.3000) -- (82.1000,75.3000) -- (82.1000,75.3000) -- (82.1000,75.3000) -- (82.1000,75.3000) -- (82.1000,75.3000) -- (82.1000,75.3000) -- (82.1000,75.3000) -- (82.1000,75.3000) -- (82.2000,75.3000) -- (82.2000,75.2000) -- (82.2000,75.2000) -- (82.2000,75.2000) -- (82.2000,75.2000) -- (82.2000,75.2000) -- (82.2000,75.2000) -- (82.2000,75.2000) -- (82.2000,75.2000) -- (82.2000,75.2000) -- (82.2000,75.2000) -- (82.2000,75.2000) -- (82.2000,75.2000) -- (82.2000,75.2000) -- (82.2000,75.2000) -- (82.2000,75.2000) -- (82.2000,75.2000) -- (82.2000,75.2000) -- (82.2000,75.2000) -- (82.2000,75.2000) -- (82.2000,75.2000) -- (82.2000,75.2000) -- (82.2000,75.1000) -- (82.2000,75.1000) -- (82.2000,75.1000) -- (82.2000,75.1000) -- (82.2000,75.1000) -- (82.2000,75.1000) -- (82.2000,75.1000) -- (82.2000,75.1000) -- (82.2000,75.1000) -- (82.2000,75.1000) -- (82.2000,75.1000) -- (82.2000,75.1000) -- (82.2000,75.1000) -- (82.2000,75.1000) -- (82.2000,75.1000) -- (82.2000,75.1000) -- (82.2000,75.1000) -- (82.2000,75.1000) -- (82.2000,75.1000) -- (82.2000,75.1000) -- (82.2000,75.1000) -- (82.2000,75.0000) -- (82.2000,75.0000) -- (82.2000,75.0000) -- (82.2000,75.0000) -- (82.2000,75.0000) -- (82.2000,75.0000) -- (82.3000,75.0000) -- (82.3000,75.0000) -- (82.3000,75.0000) -- (82.3000,75.0000) -- (82.3000,75.0000) -- (82.3000,75.0000) -- (82.3000,75.0000) -- (82.3000,75.0000) -- (82.3000,75.0000) -- (82.3000,75.0000) -- (82.3000,75.0000) -- (82.3000,75.0000) -- (82.3000,75.0000) -- (82.3000,75.0000) -- (82.3000,75.0000) -- (82.3000,74.9000) -- (82.3000,74.9000) -- (82.3000,74.9000) -- (82.3000,74.9000) -- (82.3000,74.9000) -- (82.3000,74.9000) -- (82.3000,74.9000) -- (82.3000,74.9000) -- (82.3000,74.9000) -- (82.3000,74.9000) -- (82.3000,74.9000) -- (82.3000,74.9000) -- (82.3000,74.9000) -- (82.3000,74.9000) -- (82.3000,74.9000) -- (82.3000,74.9000) -- (82.3000,74.9000) -- (82.3000,74.9000) -- (82.3000,74.9000) -- (82.3000,74.9000) -- (82.3000,74.8000) -- (82.3000,74.8000) -- (82.3000,74.8000) -- (82.3000,74.8000) -- (82.3000,74.8000) -- (82.3000,74.8000) -- (82.3000,74.8000) -- (82.3000,74.8000) -- (82.3000,74.8000) -- (82.3000,74.8000) -- (82.3000,74.8000) -- (82.3000,74.8000) -- (82.3000,74.8000) -- (82.3000,74.8000) -- (82.3000,74.8000) -- (82.4000,74.8000) -- (82.4000,74.8000) -- (82.4000,74.8000) -- (82.4000,74.8000) -- (82.4000,74.8000) -- (82.4000,74.8000) -- (82.4000,74.7000) -- (82.4000,74.7000) -- (82.4000,74.7000) -- (82.4000,74.7000) -- (82.4000,74.7000) -- (82.4000,74.7000) -- (82.4000,74.7000) -- (82.4000,74.7000) -- (82.4000,74.7000) -- (82.4000,74.7000) -- (82.4000,74.7000) -- (82.4000,74.7000) -- (82.4000,74.7000) -- (82.4000,74.7000) -- (82.4000,74.7000) -- (82.4000,74.7000) -- (82.4000,74.7000) -- (82.4000,74.7000) -- (82.4000,74.7000) -- (82.4000,74.7000) -- (82.4000,74.7000) -- (82.4000,74.6000) -- (82.4000,74.6000) -- (82.4000,74.6000) -- (82.4000,74.6000) -- (82.4000,74.6000) -- (82.4000,74.6000) -- (82.4000,74.6000) -- (82.4000,74.6000) -- (82.4000,74.6000) -- (82.4000,74.6000) -- (82.4000,74.6000) -- (82.4000,74.6000) -- (82.4000,74.6000) -- (82.4000,74.6000) -- (82.4000,74.6000) -- (82.4000,74.6000) -- (82.4000,74.6000) -- (82.4000,74.6000) -- (82.4000,74.6000) -- (82.4000,74.6000) -- (82.4000,74.6000) -- (82.4000,74.5000) -- (82.5000,74.5000) -- (82.5000,74.5000) -- (82.5000,74.5000) -- (82.5000,74.5000) -- (82.5000,74.5000) -- (82.5000,74.5000) -- (82.5000,74.5000) -- (82.5000,74.5000) -- (82.5000,74.5000) -- (82.5000,74.5000) -- (82.5000,74.5000) -- (82.5000,74.5000) -- (82.5000,74.5000) -- (82.5000,74.5000) -- (82.5000,74.5000) -- (82.5000,74.5000) -- (82.5000,74.5000) -- (82.5000,74.5000) -- (82.5000,74.5000) -- (82.5000,74.4000) -- (82.5000,74.4000) -- (82.5000,74.4000) -- (82.5000,74.4000) -- (82.5000,74.4000) -- (82.5000,74.4000) -- (82.5000,74.4000) -- (82.5000,74.4000) -- (82.5000,74.4000) -- (82.5000,74.4000) -- (82.5000,74.4000) -- (82.5000,74.4000) -- (82.5000,74.4000) -- (82.5000,74.4000) -- (82.5000,74.4000) -- (82.5000,74.4000) -- (82.5000,74.4000) -- (82.5000,74.4000) -- (82.5000,74.4000) -- (82.5000,74.4000) -- (82.5000,74.4000) -- (82.5000,74.3000) -- (82.5000,74.3000) -- (82.5000,74.3000) -- (82.5000,74.3000) -- (82.5000,74.3000) -- (82.5000,74.3000) -- (82.5000,74.3000) -- (82.5000,74.3000) -- (82.5000,74.3000) -- (82.5000,74.3000) -- (82.6000,74.3000) -- (82.6000,74.3000) -- (82.6000,74.3000) -- (82.6000,74.3000) -- (82.6000,74.3000) -- (82.6000,74.3000) -- (82.6000,74.3000) -- (82.6000,74.3000) -- (82.6000,74.3000) -- (82.6000,74.3000) -- (82.6000,74.3000) -- (82.6000,74.2000) -- (82.6000,74.2000) -- (82.6000,74.2000) -- (82.6000,74.2000) -- (82.6000,74.2000) -- (82.6000,74.2000) -- (82.6000,74.2000) -- (82.6000,74.2000) -- (82.6000,74.2000) -- (82.6000,74.2000) -- (82.6000,74.2000) -- (82.6000,74.2000) -- (82.6000,74.2000) -- (82.6000,74.2000) -- (82.6000,74.2000) -- (82.6000,74.2000) -- (82.6000,74.2000) -- (82.6000,74.2000) -- (82.6000,74.2000) -- (82.6000,74.2000) -- (82.6000,74.2000) -- (82.6000,74.1000) -- (82.6000,74.1000) -- (82.6000,74.1000) -- (82.6000,74.1000) -- (82.6000,74.1000) -- (82.6000,74.1000) -- (82.6000,74.1000) -- (82.6000,74.1000) -- (82.6000,74.1000) -- (82.6000,74.1000) -- (82.6000,74.1000) -- (82.6000,74.1000) -- (82.6000,74.1000) -- (82.6000,74.1000) -- (82.6000,74.1000) -- (82.6000,74.1000) -- (82.6000,74.1000) -- (82.7000,74.1000) -- (82.7000,74.1000) -- (82.7000,74.1000) -- (82.7000,74.1000) -- (82.7000,74.0000) -- (82.7000,74.0000) -- (82.7000,74.0000) -- (82.7000,74.0000) -- (82.7000,74.0000) -- (82.7000,74.0000) -- (82.7000,74.0000) -- (82.7000,74.0000) -- (82.7000,74.0000) -- (82.7000,74.0000) -- (82.7000,74.0000) -- (82.7000,74.0000) -- (82.7000,74.0000) -- (82.7000,74.0000) -- (82.7000,74.0000) -- (82.7000,74.0000) -- (82.7000,74.0000) -- (82.7000,74.0000) -- (82.7000,74.0000) -- (82.7000,74.0000) -- (82.7000,73.9000) -- (82.7000,73.9000) -- (82.7000,73.9000) -- (82.7000,73.9000) -- (82.7000,73.9000) -- (82.7000,73.9000) -- (82.7000,73.9000) -- (82.7000,73.9000) -- (82.7000,73.9000) -- (82.7000,73.9000) -- (82.7000,73.9000) -- (82.7000,73.9000) -- (82.7000,73.9000) -- (82.7000,73.9000) -- (82.7000,73.9000) -- (82.7000,73.9000) -- (82.7000,73.9000) -- (82.7000,73.9000) -- (82.7000,73.9000) -- (82.7000,73.9000) -- (82.7000,73.9000) -- (82.7000,73.8000) -- (82.7000,73.8000) -- (82.7000,73.8000) -- (82.7000,73.8000) -- (82.7000,73.8000) -- (82.8000,73.8000) -- (82.8000,73.8000) -- (82.8000,73.8000) -- (82.8000,73.8000) -- (82.8000,73.8000) -- (82.8000,73.8000) -- (82.8000,73.8000) -- (82.8000,73.8000) -- (82.8000,73.8000) -- (82.8000,73.8000) -- (82.8000,73.8000) -- (82.8000,73.8000) -- (82.8000,73.8000) -- (82.8000,73.8000) -- (82.8000,73.8000) -- (82.8000,73.8000) -- (82.8000,73.7000) -- (82.8000,73.7000) -- (82.8000,73.7000) -- (82.8000,73.7000) -- (82.8000,73.7000) -- (82.8000,73.7000) -- (82.8000,73.7000) -- (82.8000,73.7000) -- (82.8000,73.7000) -- (82.8000,73.7000) -- (82.8000,73.7000) -- (82.8000,73.7000) -- (82.8000,73.7000) -- (82.8000,73.7000) -- (82.8000,73.7000) -- (82.8000,73.7000) -- (82.8000,73.7000) -- (82.8000,73.7000) -- (82.8000,73.7000) -- (82.8000,73.7000) -- (82.8000,73.7000) -- (82.8000,73.6000) -- (82.8000,73.6000) -- (82.8000,73.6000) -- (82.8000,73.6000) -- (82.8000,73.6000) -- (82.8000,73.6000) -- (82.8000,73.6000) -- (82.8000,73.6000) -- (82.8000,73.6000) -- (82.8000,73.6000) -- (82.8000,73.6000) -- (82.8000,73.6000) -- (82.9000,73.6000) -- (82.9000,73.6000) -- (82.9000,73.6000) -- (82.9000,73.6000) -- (82.9000,73.6000) -- (82.9000,73.6000) -- (82.9000,73.6000) -- (82.9000,73.6000) -- (82.9000,73.6000) -- (82.9000,73.5000) -- (82.9000,73.5000) -- (82.9000,73.5000) -- (82.9000,73.5000) -- (82.9000,73.5000) -- (82.9000,73.5000) -- (82.9000,73.5000) -- (82.9000,73.5000) -- (82.9000,73.5000) -- (82.9000,73.5000) -- (82.9000,73.5000) -- (82.9000,73.5000) -- (82.9000,73.5000) -- (82.9000,73.5000) -- (82.9000,73.5000) -- (82.9000,73.5000) -- (82.9000,73.5000) -- (82.9000,73.5000) -- (82.9000,73.5000) -- (82.9000,73.5000) -- (82.9000,73.4000) -- (82.9000,73.4000) -- (82.9000,73.4000) -- (82.9000,73.4000) -- (82.9000,73.4000) -- (82.9000,73.4000) -- (82.9000,73.4000) -- (82.9000,73.4000) -- (82.9000,73.4000) -- (82.9000,73.4000) -- (82.9000,73.4000) -- (82.9000,73.4000) -- (82.9000,73.4000) -- (82.9000,73.4000) -- (82.9000,73.4000) -- (82.9000,73.4000) -- (82.9000,73.4000) -- (82.9000,73.4000) -- (82.9000,73.4000) -- (82.9000,73.4000) -- (82.9000,73.4000) -- (83.0000,73.3000) -- (83.0000,73.3000) -- (83.0000,73.3000) -- (83.0000,73.3000) -- (83.0000,73.3000) -- (83.0000,73.3000) -- (83.0000,73.3000) -- (83.0000,73.3000) -- (83.0000,73.3000) -- (83.0000,73.3000) -- (83.0000,73.3000) -- (83.0000,73.3000) -- (83.0000,73.3000) -- (83.0000,73.3000) -- (83.0000,73.3000) -- (83.0000,73.3000) -- (83.0000,73.3000) -- (83.0000,73.3000) -- (83.0000,73.3000) -- (83.0000,73.3000) -- (83.0000,73.3000) -- (83.0000,73.2000) -- (83.0000,73.2000) -- (83.0000,73.2000) -- (83.0000,73.2000) -- (83.0000,73.2000) -- (83.0000,73.2000) -- (83.0000,73.2000) -- (83.0000,73.2000) -- (83.0000,73.2000) -- (83.0000,73.2000) -- (83.0000,73.2000) -- (83.0000,73.2000) -- (83.0000,73.2000) -- (83.0000,73.2000) -- (83.0000,73.2000) -- (83.0000,73.2000) -- (83.0000,73.2000) -- (83.0000,73.2000) -- (83.0000,73.2000) -- (83.0000,73.2000) -- (83.0000,73.2000) -- (83.0000,73.1000) -- (83.0000,73.1000) -- (83.0000,73.1000) -- (83.0000,73.1000) -- (83.0000,73.1000) -- (83.0000,73.1000) -- (83.0000,73.1000) -- (83.1000,73.1000) -- (83.1000,73.1000) -- (83.1000,73.1000) -- (83.1000,73.1000) -- (83.1000,73.1000) -- (83.1000,73.1000) -- (83.1000,73.1000) -- (83.1000,73.1000) -- (83.1000,73.1000) -- (83.1000,73.1000) -- (83.1000,73.1000) -- (83.1000,73.1000) -- (83.1000,73.1000) -- (83.1000,73.1000) -- (83.1000,73.0000) -- (83.1000,73.0000) -- (83.1000,73.0000) -- (83.1000,73.0000) -- (83.1000,73.0000) -- (83.1000,73.0000) -- (83.1000,73.0000) -- (83.1000,73.0000) -- (83.1000,73.0000) -- (83.1000,73.0000) -- (83.1000,73.0000) -- (83.1000,73.0000) -- (83.1000,73.0000) -- (83.1000,73.0000) -- (83.1000,73.0000) -- (83.1000,73.0000) -- (83.1000,73.0000) -- (83.1000,73.0000) -- (83.1000,73.0000) -- (83.1000,73.0000) -- (83.1000,72.9000) -- (83.1000,72.9000) -- (83.1000,72.9000) -- (83.1000,72.9000) -- (83.1000,72.9000) -- (83.1000,72.9000) -- (83.1000,72.9000) -- (83.1000,72.9000) -- (83.1000,72.9000) -- (83.1000,72.9000) -- (83.1000,72.9000) -- (83.1000,72.9000) -- (83.1000,72.9000) -- (83.1000,72.9000) -- (83.1000,72.9000) -- (83.1000,72.9000) -- (83.2000,72.9000) -- (83.2000,72.9000) -- (83.2000,72.9000) -- (83.2000,72.9000) -- (83.2000,72.9000) -- (83.2000,72.8000) -- (83.2000,72.8000) -- (83.2000,72.8000) -- (83.2000,72.8000) -- (83.2000,72.8000) -- (83.2000,72.8000) -- (83.2000,72.8000) -- (83.2000,72.8000) -- (83.2000,72.8000) -- (83.2000,72.8000) -- (83.2000,72.8000) -- (83.2000,72.8000) -- (83.2000,72.8000) -- (83.2000,72.8000) -- (83.2000,72.8000) -- (83.2000,72.8000) -- (83.2000,72.8000) -- (83.2000,72.8000) -- (83.2000,72.8000) -- (83.2000,72.8000) -- (83.2000,72.8000) -- (83.2000,72.7000) -- (83.2000,72.7000) -- (83.2000,72.7000) -- (83.2000,72.7000) -- (83.2000,72.7000) -- (83.2000,72.7000) -- (83.2000,72.7000) -- (83.2000,72.7000) -- (83.2000,72.7000) -- (83.2000,72.7000) -- (83.2000,72.7000) -- (83.2000,72.7000) -- (83.2000,72.7000) -- (83.2000,72.7000) -- (83.2000,72.7000) -- (83.2000,72.7000) -- (83.2000,72.7000) -- (83.2000,72.7000) -- (83.2000,72.7000) -- (83.2000,72.7000) -- (83.2000,72.7000) -- (83.2000,72.6000) -- (83.2000,72.6000) -- (83.3000,72.6000) -- (83.3000,72.6000) -- (83.3000,72.6000) -- (83.3000,72.6000) -- (83.3000,72.6000) -- (83.3000,72.6000) -- (83.3000,72.6000) -- (83.3000,72.6000) -- (83.3000,72.6000) -- (83.3000,72.6000) -- (83.3000,72.6000) -- (83.3000,72.6000) -- (83.3000,72.6000) -- (83.3000,72.6000) -- (83.3000,72.6000) -- (83.3000,72.6000) -- (83.3000,72.6000) -- (83.3000,72.6000) -- (83.3000,72.5000) -- (83.3000,72.5000) -- (83.3000,72.5000) -- (83.3000,72.5000) -- (83.3000,72.5000) -- (83.3000,72.5000) -- (83.3000,72.5000) -- (83.3000,72.5000) -- (83.3000,72.5000) -- (83.3000,72.5000) -- (83.3000,72.5000) -- (83.3000,72.5000) -- (83.3000,72.5000) -- (83.3000,72.5000) -- (83.3000,72.5000) -- (83.3000,72.5000) -- (83.3000,72.5000) -- (83.3000,72.5000) -- (83.3000,72.5000) -- (83.3000,72.5000) -- (83.3000,72.5000) -- (83.3000,72.4000) -- (83.3000,72.4000) -- (83.3000,72.4000) -- (83.3000,72.4000) -- (83.3000,72.4000) -- (83.3000,72.4000) -- (83.3000,72.4000) -- (83.3000,72.4000) -- (83.3000,72.4000) -- (83.3000,72.4000) -- (83.3000,72.4000) -- (83.4000,72.4000) -- (83.4000,72.4000) -- (83.4000,72.4000) -- (83.4000,72.4000) -- (83.4000,72.4000) -- (83.4000,72.4000) -- (83.4000,72.4000) -- (83.4000,72.4000) -- (83.4000,72.4000) -- (83.4000,72.4000) -- (83.4000,72.3000) -- (83.4000,72.3000) -- (83.4000,72.3000) -- (83.4000,72.3000) -- (83.4000,72.3000) -- (83.4000,72.3000) -- (83.4000,72.3000) -- (83.4000,72.3000) -- (83.4000,72.3000) -- (83.4000,72.3000) -- (83.4000,72.3000) -- (83.4000,72.3000) -- (83.4000,72.3000) -- (83.4000,72.3000) -- (83.4000,72.3000) -- (83.4000,72.3000) -- (83.4000,72.3000) -- (83.4000,72.3000) -- (83.4000,72.3000) -- (83.4000,72.3000) -- (83.4000,72.3000) -- (83.4000,72.2000) -- (83.4000,72.2000) -- (83.4000,72.2000) -- (83.4000,72.2000) -- (83.4000,72.2000) -- (83.4000,72.2000) -- (83.4000,72.2000) -- (83.4000,72.2000) -- (83.4000,72.2000) -- (83.4000,72.2000) -- (83.4000,72.2000) -- (83.4000,72.2000) -- (83.4000,72.2000) -- (83.4000,72.2000) -- (83.4000,72.2000) -- (83.4000,72.2000) -- (83.4000,72.2000) -- (83.4000,72.2000) -- (83.5000,72.2000) -- (83.5000,72.2000) -- (83.5000,72.2000) -- (83.5000,72.1000) -- (83.5000,72.1000) -- (83.5000,72.1000) -- (83.5000,72.1000) -- (83.5000,72.1000) -- (83.5000,72.1000) -- (83.5000,72.1000) -- (83.5000,72.1000) -- (83.5000,72.1000) -- (83.5000,72.1000) -- (83.5000,72.1000) -- (83.5000,72.1000) -- (83.5000,72.1000) -- (83.5000,72.1000) -- (83.5000,72.1000) -- (83.5000,72.1000) -- (83.5000,72.1000) -- (83.5000,72.1000) -- (83.5000,72.1000) -- (83.5000,72.1000) -- (83.5000,72.0000) -- (83.5000,72.0000) -- (83.5000,72.0000) -- (83.5000,72.0000) -- (83.5000,72.0000) -- (83.5000,72.0000) -- (83.5000,72.0000) -- (83.5000,72.0000) -- (83.5000,72.0000) -- (83.5000,72.0000) -- (83.5000,72.0000) -- (83.5000,72.0000) -- (83.5000,72.0000) -- (83.5000,72.0000) -- (83.5000,72.0000) -- (83.5000,72.0000) -- (83.5000,72.0000) -- (83.5000,72.0000) -- (83.5000,72.0000) -- (83.5000,72.0000) -- (83.5000,72.0000) -- (83.5000,71.9000) -- (83.5000,71.9000) -- (83.5000,71.9000) -- (83.5000,71.9000) -- (83.5000,71.9000) -- (83.5000,71.9000) -- (83.6000,71.9000) -- (83.6000,71.9000) -- (83.6000,71.9000) -- (83.6000,71.9000) -- (83.6000,71.9000) -- (83.6000,71.9000) -- (83.6000,71.9000) -- (83.6000,71.9000) -- (83.6000,71.9000) -- (83.6000,71.9000) -- (83.6000,71.9000) -- (83.6000,71.9000) -- (83.6000,71.9000) -- (83.6000,71.9000) -- (83.6000,71.9000) -- (83.6000,71.8000) -- (83.6000,71.8000) -- (83.6000,71.8000) -- (83.6000,71.8000) -- (83.6000,71.8000) -- (83.6000,71.8000) -- (83.6000,71.8000) -- (83.6000,71.8000) -- (83.6000,71.8000) -- (83.6000,71.8000) -- (83.6000,71.8000) -- (83.6000,71.8000) -- (83.6000,71.8000) -- (83.6000,71.8000) -- (83.6000,71.8000) -- (83.6000,71.8000) -- (83.6000,71.8000) -- (83.6000,71.8000) -- (83.6000,71.8000) -- (83.6000,71.8000) -- (83.6000,71.8000) -- (83.6000,71.7000) -- (83.6000,71.7000) -- (83.6000,71.7000) -- (83.6000,71.7000) -- (83.6000,71.7000) -- (83.6000,71.7000) -- (83.6000,71.7000) -- (83.6000,71.7000) -- (83.6000,71.7000) -- (83.6000,71.7000) -- (83.6000,71.7000) -- (83.6000,71.7000) -- (83.6000,71.7000) -- (83.7000,71.7000) -- (83.7000,71.7000) -- (83.7000,71.7000) -- (83.7000,71.7000) -- (83.7000,71.7000) -- (83.7000,71.7000) -- (83.7000,71.7000) -- (83.7000,71.7000) -- (83.7000,71.6000) -- (83.7000,71.6000) -- (83.7000,71.6000) -- (83.7000,71.6000) -- (83.7000,71.6000) -- (83.7000,71.6000) -- (83.7000,71.6000) -- (83.7000,71.6000) -- (83.7000,71.6000) -- (83.7000,71.6000) -- (83.7000,71.6000) -- (83.7000,71.6000) -- (83.7000,71.6000) -- (83.7000,71.6000) -- (83.7000,71.6000) -- (83.7000,71.6000) -- (83.7000,71.6000) -- (83.7000,71.6000) -- (83.7000,71.6000) -- (83.7000,71.6000) -- (83.7000,71.5000) -- (83.7000,71.5000) -- (83.7000,71.5000) -- (83.7000,71.5000) -- (83.7000,71.5000) -- (83.7000,71.5000) -- (83.7000,71.5000) -- (83.7000,71.5000) -- (83.7000,71.5000) -- (83.7000,71.5000) -- (83.7000,71.5000) -- (83.7000,71.5000) -- (83.7000,71.5000) -- (83.7000,71.5000) -- (83.7000,71.5000) -- (83.7000,71.5000) -- (83.7000,71.5000) -- (83.7000,71.5000) -- (83.7000,71.5000) -- (83.7000,71.5000) -- (83.7000,71.5000) -- (83.7000,71.4000) -- (83.8000,71.4000) -- (83.8000,71.4000) -- (83.8000,71.4000) -- (83.8000,71.4000) -- (83.8000,71.4000) -- (83.8000,71.4000) -- (83.8000,71.4000) -- (83.8000,71.4000) -- (83.8000,71.4000) -- (83.8000,71.4000) -- (83.8000,71.4000) -- (83.8000,71.4000) -- (83.8000,71.4000) -- (83.8000,71.4000) -- (83.8000,71.4000) -- (83.8000,71.4000) -- (83.8000,71.4000) -- (83.8000,71.4000) -- (83.8000,71.4000) -- (83.8000,71.4000) -- (83.8000,71.3000) -- (83.8000,71.3000) -- (83.8000,71.3000) -- (83.8000,71.3000) -- (83.8000,71.3000) -- (83.8000,71.3000) -- (83.8000,71.3000) -- (83.8000,71.3000) -- (83.8000,71.3000) -- (83.8000,71.3000) -- (83.8000,71.3000) -- (83.8000,71.3000) -- (83.8000,71.3000) -- (83.8000,71.3000) -- (83.8000,71.3000) -- (83.8000,71.3000) -- (83.8000,71.3000) -- (83.8000,71.3000) -- (83.8000,71.3000) -- (83.8000,71.3000) -- (83.8000,71.3000) -- (83.8000,71.2000) -- (83.8000,71.2000) -- (83.8000,71.2000) -- (83.8000,71.2000) -- (83.8000,71.2000) -- (83.8000,71.2000) -- (83.8000,71.2000) -- (83.8000,71.2000) -- (83.9000,71.2000) -- (83.9000,71.2000) -- (83.9000,71.2000) -- (83.9000,71.2000) -- (83.9000,71.2000) -- (83.9000,71.2000) -- (83.9000,71.2000) -- (83.9000,71.2000) -- (83.9000,71.2000) -- (83.9000,71.2000) -- (83.9000,71.2000) -- (83.9000,71.2000) -- (83.9000,71.2000) -- (83.9000,71.1000) -- (83.9000,71.1000) -- (83.9000,71.1000) -- (83.9000,71.1000) -- (83.9000,71.1000) -- (83.9000,71.1000) -- (83.9000,71.1000) -- (83.9000,71.1000) -- (83.9000,71.1000) -- (83.9000,71.1000) -- (83.9000,71.1000) -- (83.9000,71.1000) -- (83.9000,71.1000) -- (83.9000,71.1000) -- (83.9000,71.1000) -- (83.9000,71.1000) -- (83.9000,71.1000) -- (83.9000,71.1000) -- (83.9000,71.1000) -- (83.9000,71.1000) -- (83.9000,71.0000) -- (83.9000,71.0000) -- (83.9000,71.0000) -- (83.9000,71.0000) -- (83.9000,71.0000) -- (83.9000,71.0000) -- (83.9000,71.0000) -- (83.9000,71.0000) -- (83.9000,71.0000) -- (83.9000,71.0000) -- (83.9000,71.0000) -- (83.9000,71.0000) -- (83.9000,71.0000) -- (83.9000,71.0000) -- (83.9000,71.0000) -- (83.9000,71.0000) -- (83.9000,71.0000) -- (84.0000,71.0000) -- (84.0000,71.0000) -- (84.0000,71.0000) -- (84.0000,71.0000) -- (84.0000,70.9000) -- (84.0000,70.9000) -- (84.0000,70.9000) -- (84.0000,70.9000) -- (84.0000,70.9000) -- (84.0000,70.9000) -- (84.0000,70.9000) -- (84.0000,70.9000) -- (84.0000,70.9000) -- (84.0000,70.9000) -- (84.0000,70.9000) -- (84.0000,70.9000) -- (84.0000,70.9000) -- (84.0000,70.9000) -- (84.0000,70.9000) -- (84.0000,70.9000) -- (84.0000,70.9000) -- (84.0000,70.9000) -- (84.0000,70.9000) -- (84.0000,70.9000) -- (84.0000,70.9000) -- (84.0000,70.8000) -- (84.0000,70.8000) -- (84.0000,70.8000) -- (84.0000,70.8000) -- (84.0000,70.8000) -- (84.0000,70.8000) -- (84.0000,70.8000) -- (84.0000,70.8000) -- (84.0000,70.8000) -- (84.0000,70.8000) -- (84.0000,70.8000) -- (84.0000,70.8000) -- (84.0000,70.8000) -- (84.0000,70.8000) -- (84.0000,70.8000) -- (84.0000,70.8000) -- (84.0000,70.8000) -- (84.0000,70.8000) -- (84.0000,70.8000) -- (84.0000,70.8000) -- (84.0000,70.8000) -- (84.0000,70.7000) -- (84.0000,70.7000) -- (84.0000,70.7000) -- (84.1000,70.7000) -- (84.1000,70.7000) -- (84.1000,70.7000) -- (84.1000,70.7000) -- (84.1000,70.7000) -- (84.1000,70.7000) -- (84.1000,70.7000) -- (84.1000,70.7000) -- (84.1000,70.7000) -- (84.1000,70.7000) -- (84.1000,70.7000) -- (84.1000,70.7000) -- (84.1000,70.7000) -- (84.1000,70.7000) -- (84.1000,70.7000) -- (84.1000,70.7000) -- (84.1000,70.7000) -- (84.1000,70.6000) -- (84.1000,70.6000) -- (84.1000,70.6000) -- (84.1000,70.6000) -- (84.1000,70.6000) -- (84.1000,70.6000) -- (84.1000,70.6000) -- (84.1000,70.6000) -- (84.1000,70.6000) -- (84.1000,70.6000) -- (84.1000,70.6000) -- (84.1000,70.6000) -- (84.1000,70.6000) -- (84.1000,70.6000) -- (84.1000,70.6000) -- (84.1000,70.6000) -- (84.1000,70.6000) -- (84.1000,70.6000) -- (84.1000,70.6000) -- (84.1000,70.6000) -- (84.1000,70.6000) -- (84.1000,70.5000) -- (84.1000,70.5000) -- (84.1000,70.5000) -- (84.1000,70.5000) -- (84.1000,70.5000) -- (84.1000,70.5000) -- (84.1000,70.5000) -- (84.1000,70.5000) -- (84.1000,70.5000) -- (84.1000,70.5000) -- (84.1000,70.5000) -- (84.1000,70.5000) -- (84.2000,70.5000) -- (84.2000,70.5000) -- (84.2000,70.5000) -- (84.2000,70.5000) -- (84.2000,70.5000) -- (84.2000,70.5000) -- (84.2000,70.5000) -- (84.2000,70.5000) -- (84.2000,70.5000) -- (84.2000,70.4000) -- (84.2000,70.4000) -- (84.2000,70.4000) -- (84.2000,70.4000) -- (84.2000,70.4000) -- (84.2000,70.4000) -- (84.2000,70.4000) -- (84.2000,70.4000) -- (84.2000,70.4000) -- (84.2000,70.4000) -- (84.2000,70.4000) -- (84.2000,70.4000) -- (84.2000,70.4000) -- (84.2000,70.4000) -- (84.2000,70.4000) -- (84.2000,70.4000) -- (84.2000,70.4000) -- (84.2000,70.4000) -- (84.2000,70.4000) -- (84.2000,70.4000) -- (84.2000,70.4000) -- (84.2000,70.3000) -- (84.2000,70.3000) -- (84.2000,70.3000) -- (84.2000,70.3000) -- (84.2000,70.3000) -- (84.2000,70.3000) -- (84.2000,70.3000) -- (84.2000,70.3000) -- (84.2000,70.3000) -- (84.2000,70.3000) -- (84.2000,70.3000) -- (84.2000,70.3000) -- (84.2000,70.3000) -- (84.2000,70.3000) -- (84.2000,70.3000) -- (84.2000,70.3000) -- (84.2000,70.3000) -- (84.2000,70.3000) -- (84.2000,70.3000) -- (84.3000,70.3000) -- (84.3000,70.3000) -- (84.3000,70.2000) -- (84.3000,70.2000) -- (84.3000,70.2000) -- (84.3000,70.2000) -- (84.3000,70.2000) -- (84.3000,70.2000) -- (84.3000,70.2000) -- (84.3000,70.2000) -- (84.3000,70.2000) -- (84.3000,70.2000) -- (84.3000,70.2000) -- (84.3000,70.2000) -- (84.3000,70.2000) -- (84.3000,70.2000) -- (84.3000,70.2000) -- (84.3000,70.2000) -- (84.3000,70.2000) -- (84.3000,70.2000) -- (84.3000,70.2000) -- (84.3000,70.2000) -- (84.3000,70.1000) -- (84.3000,70.1000) -- (84.3000,70.1000) -- (84.3000,70.1000) -- (84.3000,70.1000) -- (84.3000,70.1000) -- (84.3000,70.1000) -- (84.3000,70.1000) -- (84.3000,70.1000) -- (84.3000,70.1000) -- (84.3000,70.1000) -- (84.3000,70.1000) -- (84.3000,70.1000) -- (84.3000,70.1000) -- (84.3000,70.1000) -- (84.3000,70.1000) -- (84.3000,70.1000) -- (84.3000,70.1000) -- (84.3000,70.1000) -- (84.3000,70.1000) -- (84.3000,70.1000) -- (84.3000,70.0000) -- (84.3000,70.0000) -- (84.3000,70.0000) -- (84.3000,70.0000) -- (84.3000,70.0000) -- (84.3000,70.0000) -- (84.3000,70.0000) -- (84.4000,70.0000) -- (84.4000,70.0000) -- (84.4000,70.0000) -- (84.4000,70.0000) -- (84.4000,70.0000) -- (84.4000,70.0000) -- (84.4000,70.0000) -- (84.4000,70.0000) -- (84.4000,70.0000) -- (84.4000,70.0000) -- (84.4000,70.0000) -- (84.4000,70.0000) -- (84.4000,70.0000) -- (84.4000,70.0000) -- (84.4000,69.9000) -- (84.4000,69.9000) -- (84.4000,69.9000) -- (84.4000,69.9000) -- (84.4000,69.9000) -- (84.4000,69.9000) -- (84.4000,69.9000) -- (84.4000,69.9000) -- (84.4000,69.9000) -- (84.4000,69.9000) -- (84.4000,69.9000) -- (84.4000,69.9000) -- (84.4000,69.9000) -- (84.4000,69.9000) -- (84.4000,69.9000) -- (84.4000,69.9000) -- (84.4000,69.9000) -- (84.4000,69.9000) -- (84.4000,69.9000) -- (84.4000,69.9000) -- (84.4000,69.9000) -- (84.4000,69.8000) -- (84.4000,69.8000) -- (84.4000,69.8000) -- (84.4000,69.8000) -- (84.4000,69.8000) -- (84.4000,69.8000) -- (84.4000,69.8000) -- (84.4000,69.8000) -- (84.4000,69.8000) -- (84.4000,69.8000) -- (84.4000,69.8000) -- (84.4000,69.8000) -- (84.4000,69.8000) -- (84.4000,69.8000) -- (84.5000,69.8000) -- (84.5000,69.8000) -- (84.5000,69.8000) -- (84.5000,69.8000) -- (84.5000,69.8000) -- (84.5000,69.8000) -- (84.5000,69.8000) -- (84.5000,69.7000) -- (84.5000,69.7000) -- (84.5000,69.7000) -- (84.5000,69.7000) -- (84.5000,69.7000) -- (84.5000,69.7000) -- (84.5000,69.7000) -- (84.5000,69.7000) -- (84.5000,69.7000) -- (84.5000,69.7000) -- (84.5000,69.7000) -- (84.5000,69.7000) -- (84.5000,69.7000) -- (84.5000,69.7000) -- (84.5000,69.7000) -- (84.5000,69.7000) -- (84.5000,69.7000) -- (84.5000,69.7000) -- (84.5000,69.7000) -- (84.5000,69.7000) -- (84.5000,69.6000) -- (84.5000,69.6000) -- (84.5000,69.6000) -- (84.5000,69.6000) -- (84.5000,69.6000) -- (84.5000,69.6000) -- (84.5000,69.6000) -- (84.5000,69.6000) -- (84.5000,69.6000) -- (84.5000,69.6000) -- (84.5000,69.6000) -- (84.5000,69.6000) -- (84.5000,69.6000) -- (84.5000,69.6000) -- (84.5000,69.6000) -- (84.5000,69.6000) -- (84.5000,69.6000) -- (84.5000,69.6000) -- (84.5000,69.6000) -- (84.5000,69.6000) -- (84.5000,69.6000) -- (84.5000,69.5000) -- (84.5000,69.5000) -- (84.6000,69.5000) -- (84.6000,69.5000) -- (84.6000,69.5000) -- (84.6000,69.5000) -- (84.6000,69.5000) -- (84.6000,69.5000) -- (84.6000,69.5000) -- (84.6000,69.5000) -- (84.6000,69.5000) -- (84.6000,69.5000) -- (84.6000,69.5000) -- (84.6000,69.5000) -- (84.6000,69.5000) -- (84.6000,69.5000) -- (84.6000,69.5000) -- (84.6000,69.5000) -- (84.6000,69.5000) -- (84.6000,69.5000) -- (84.6000,69.5000) -- (84.6000,69.4000) -- (84.6000,69.4000) -- (84.6000,69.4000) -- (84.6000,69.4000) -- (84.6000,69.4000) -- (84.6000,69.4000) -- (84.6000,69.4000) -- (84.6000,69.4000) -- (84.6000,69.4000) -- (84.6000,69.4000) -- (84.6000,69.4000) -- (84.6000,69.4000) -- (84.6000,69.4000) -- (84.6000,69.4000) -- (84.6000,69.4000) -- (84.6000,69.4000) -- (84.6000,69.4000) -- (84.6000,69.4000) -- (84.6000,69.4000) -- (84.6000,69.4000) -- (84.6000,69.4000) -- (84.6000,69.3000) -- (84.6000,69.3000) -- (84.6000,69.3000) -- (84.6000,69.3000) -- (84.6000,69.3000) -- (84.6000,69.3000) -- (84.6000,69.3000) -- (84.6000,69.3000) -- (84.6000,69.3000) -- (84.7000,69.3000) -- (84.7000,69.3000) -- (84.7000,69.3000) -- (84.7000,69.3000) -- (84.7000,69.3000) -- (84.7000,69.3000) -- (84.7000,69.3000) -- (84.7000,69.3000) -- (84.7000,69.3000) -- (84.7000,69.3000) -- (84.7000,69.3000) -- (84.7000,69.2000) -- (84.7000,69.2000) -- (84.7000,69.2000) -- (84.7000,69.2000) -- (84.7000,69.2000) -- (84.7000,69.2000) -- (84.7000,69.2000) -- (84.7000,69.2000) -- (84.7000,69.2000) -- (84.7000,69.2000) -- (84.7000,69.2000) -- (84.7000,69.2000) -- (84.7000,69.2000) -- (84.7000,69.2000) -- (84.7000,69.2000) -- (84.7000,69.2000) -- (84.7000,69.2000) -- (84.7000,69.2000) -- (84.7000,69.2000) -- (84.7000,69.2000) -- (84.7000,69.2000) -- (84.7000,69.1000) -- (84.7000,69.1000) -- (84.7000,69.1000) -- (84.7000,69.1000) -- (84.7000,69.1000) -- (84.7000,69.1000) -- (84.7000,69.1000) -- (84.7000,69.1000) -- (84.7000,69.1000) -- (84.7000,69.1000) -- (84.7000,69.1000) -- (84.7000,69.1000) -- (84.7000,69.1000) -- (84.7000,69.1000) -- (84.7000,69.1000) -- (84.7000,69.1000) -- (84.7000,69.1000) -- (84.7000,69.1000) -- (84.8000,69.1000) -- (84.8000,69.1000) -- (84.8000,69.1000) -- (84.8000,69.0000) -- (84.8000,69.0000) -- (84.8000,69.0000) -- (84.8000,69.0000) -- (84.8000,69.0000) -- (84.8000,69.0000) -- (84.8000,69.0000) -- (84.8000,69.0000) -- (84.8000,69.0000) -- (84.8000,69.0000) -- (84.8000,69.0000) -- (84.8000,69.0000) -- (84.8000,69.0000) -- (84.8000,69.0000) -- (84.8000,69.0000) -- (84.8000,69.0000) -- (84.8000,69.0000) -- (84.8000,69.0000) -- (84.8000,69.0000) -- (84.8000,69.0000) -- (84.8000,69.0000) -- (84.8000,68.9000) -- (84.8000,68.9000) -- (84.8000,68.9000) -- (84.8000,68.9000) -- (84.8000,68.9000) -- (84.8000,68.9000) -- (84.8000,68.9000) -- (84.8000,68.9000) -- (84.8000,68.9000) -- (84.8000,68.9000) -- (84.8000,68.9000) -- (84.8000,68.9000) -- (84.8000,68.9000) -- (84.8000,68.9000) -- (84.8000,68.9000) -- (84.8000,68.9000) -- (84.8000,68.9000) -- (84.8000,68.9000) -- (84.8000,68.9000) -- (84.8000,68.9000) -- (84.8000,68.9000) -- (84.8000,68.8000) -- (84.8000,68.8000) -- (84.8000,68.8000) -- (84.8000,68.8000) -- (84.9000,68.8000) -- (84.9000,68.8000) -- (84.9000,68.8000) -- (84.9000,68.8000) -- (84.9000,68.8000) -- (84.9000,68.8000) -- (84.9000,68.8000) -- (84.9000,68.8000) -- (84.9000,68.8000) -- (84.9000,68.8000) -- (84.9000,68.8000) -- (84.9000,68.8000) -- (84.9000,68.8000) -- (84.9000,68.8000) -- (84.9000,68.8000) -- (84.9000,68.8000) -- (84.9000,68.7000) -- (84.9000,68.7000) -- (84.9000,68.7000) -- (84.9000,68.7000) -- (84.9000,68.7000) -- (84.9000,68.7000) -- (84.9000,68.7000) -- (84.9000,68.7000) -- (84.9000,68.7000) -- (84.9000,68.7000) -- (84.9000,68.7000) -- (84.9000,68.7000) -- (84.9000,68.7000) -- (84.9000,68.7000) -- (84.9000,68.7000) -- (84.9000,68.7000) -- (84.9000,68.7000) -- (84.9000,68.7000) -- (84.9000,68.7000) -- (84.9000,68.7000) -- (84.9000,68.7000) -- (84.9000,68.6000) -- (84.9000,68.6000) -- (84.9000,68.6000) -- (84.9000,68.6000) -- (84.9000,68.6000) -- (84.9000,68.6000) -- (84.9000,68.6000) -- (84.9000,68.6000) -- (84.9000,68.6000) -- (84.9000,68.6000) -- (84.9000,68.6000) -- (84.9000,68.6000) -- (84.9000,68.6000) -- (85.0000,68.6000) -- (85.0000,68.6000) -- (85.0000,68.6000) -- (85.0000,68.6000) -- (85.0000,68.6000) -- (85.0000,68.6000) -- (85.0000,68.6000) -- (85.0000,68.6000) -- (85.0000,68.5000) -- (85.0000,68.5000) -- (85.0000,68.5000) -- (85.0000,68.5000) -- (85.0000,68.5000) -- (85.0000,68.5000) -- (85.0000,68.5000) -- (85.0000,68.5000) -- (85.0000,68.5000) -- (85.0000,68.5000) -- (85.0000,68.5000) -- (85.0000,68.5000) -- (85.0000,68.5000) -- (85.0000,68.5000) -- (85.0000,68.5000) -- (85.0000,68.5000) -- (85.0000,68.5000) -- (85.0000,68.5000) -- (85.0000,68.5000) -- (85.0000,68.5000) -- (85.0000,68.5000) -- (85.0000,68.4000) -- (85.0000,68.4000) -- (85.0000,68.4000) -- (85.0000,68.4000) -- (85.0000,68.4000) -- (85.0000,68.4000) -- (85.0000,68.4000) -- (85.0000,68.4000) -- (85.0000,68.4000) -- (85.0000,68.4000) -- (85.0000,68.4000) -- (85.0000,68.4000) -- (85.0000,68.4000) -- (85.0000,68.4000) -- (85.0000,68.4000) -- (85.0000,68.4000) -- (85.0000,68.4000) -- (85.0000,68.4000) -- (85.0000,68.4000) -- (85.0000,68.4000) -- (85.1000,68.4000) -- (85.1000,68.3000) -- (85.1000,68.3000) -- (85.1000,68.3000) -- (85.1000,68.3000) -- (85.1000,68.3000) -- (85.1000,68.3000) -- (85.1000,68.3000) -- (85.1000,68.3000) -- (85.1000,68.3000) -- (85.1000,68.3000) -- (85.1000,68.3000) -- (85.1000,68.3000) -- (85.1000,68.3000) -- (85.1000,68.3000) -- (85.1000,68.3000) -- (85.1000,68.3000) -- (85.1000,68.3000) -- (85.1000,68.3000) -- (85.1000,68.3000) -- (85.1000,68.3000) -- (85.1000,68.2000) -- (85.1000,68.2000) -- (85.1000,68.2000) -- (85.1000,68.2000) -- (85.1000,68.2000) -- (85.1000,68.2000) -- (85.1000,68.2000) -- (85.1000,68.2000) -- (85.1000,68.2000) -- (85.1000,68.2000) -- (85.1000,68.2000) -- (85.1000,68.2000) -- (85.1000,68.2000) -- (85.1000,68.2000) -- (85.1000,68.2000) -- (85.1000,68.2000) -- (85.1000,68.2000) -- (85.1000,68.2000) -- (85.1000,68.2000) -- (85.1000,68.2000) -- (85.1000,68.2000) -- (85.1000,68.1000) -- (85.1000,68.1000) -- (85.1000,68.1000) -- (85.1000,68.1000) -- (85.1000,68.1000) -- (85.1000,68.1000) -- (85.1000,68.1000) -- (85.1000,68.1000) -- (85.2000,68.1000) -- (85.2000,68.1000) -- (85.2000,68.1000) -- (85.2000,68.1000) -- (85.2000,68.1000) -- (85.2000,68.1000) -- (85.2000,68.1000) -- (85.2000,68.1000) -- (85.2000,68.1000) -- (85.2000,68.1000) -- (85.2000,68.1000) -- (85.2000,68.1000) -- (85.2000,68.1000) -- (85.2000,68.0000) -- (85.2000,68.0000) -- (85.2000,68.0000) -- (85.2000,68.0000) -- (85.2000,68.0000) -- (85.2000,68.0000) -- (85.2000,68.0000) -- (85.2000,68.0000) -- (85.2000,68.0000) -- (85.2000,68.0000) -- (85.2000,68.0000) -- (85.2000,68.0000) -- (85.2000,68.0000) -- (85.2000,68.0000) -- (85.2000,68.0000) -- (85.2000,68.0000) -- (85.2000,68.0000) -- (85.2000,68.0000) -- (85.2000,68.0000) -- (85.2000,68.0000) -- (85.2000,68.0000) -- (85.2000,67.9000) -- (85.2000,67.9000) -- (85.2000,67.9000) -- (85.2000,67.9000) -- (85.2000,67.9000) -- (85.2000,67.9000) -- (85.2000,67.9000) -- (85.2000,67.9000) -- (85.2000,67.9000) -- (85.2000,67.9000) -- (85.2000,67.9000) -- (85.2000,67.9000) -- (85.2000,67.9000) -- (85.2000,67.9000) -- (85.2000,67.9000) -- (85.3000,67.9000) -- (85.3000,67.9000) -- (85.3000,67.9000) -- (85.3000,67.9000) -- (85.3000,67.9000) -- (85.3000,67.9000) -- (85.3000,67.8000) -- (85.3000,67.8000) -- (85.3000,67.8000) -- (85.3000,67.8000) -- (85.3000,67.8000) -- (85.3000,67.8000) -- (85.3000,67.8000) -- (85.3000,67.8000) -- (85.3000,67.8000) -- (85.3000,67.8000) -- (85.3000,67.8000) -- (85.3000,67.8000) -- (85.3000,67.8000) -- (85.3000,67.8000) -- (85.3000,67.8000) -- (85.3000,67.8000) -- (85.3000,67.8000) -- (85.3000,67.8000) -- (85.3000,67.8000) -- (85.3000,67.8000) -- (85.3000,67.7000) -- (85.3000,67.7000) -- (85.3000,67.7000) -- (85.3000,67.7000) -- (85.3000,67.7000) -- (85.3000,67.7000) -- (85.3000,67.7000) -- (85.3000,67.7000) -- (85.3000,67.7000) -- (85.3000,67.7000) -- (85.3000,67.7000) -- (85.3000,67.7000) -- (85.3000,67.7000) -- (85.3000,67.7000) -- (85.3000,67.7000) -- (85.3000,67.7000) -- (85.3000,67.7000) -- (85.3000,67.7000) -- (85.3000,67.7000) -- (85.3000,67.7000) -- (85.3000,67.7000) -- (85.3000,67.6000) -- (85.3000,67.6000) -- (85.3000,67.6000) -- (85.4000,67.6000) -- (85.4000,67.6000) -- (85.4000,67.6000) -- (85.4000,67.6000) -- (85.4000,67.6000) -- (85.4000,67.6000) -- (85.4000,67.6000) -- (85.4000,67.6000) -- (85.4000,67.6000) -- (85.4000,67.6000) -- (85.4000,67.6000) -- (85.4000,67.6000) -- (85.4000,67.6000) -- (85.4000,67.6000) -- (85.4000,67.6000) -- (85.4000,67.6000) -- (85.4000,67.6000) -- (85.4000,67.6000) -- (85.4000,67.5000) -- (85.4000,67.5000) -- (85.4000,67.5000) -- (85.4000,67.5000) -- (85.4000,67.5000) -- (85.4000,67.5000) -- (85.4000,67.5000) -- (85.4000,67.5000) -- (85.4000,67.5000) -- (85.4000,67.5000) -- (85.4000,67.5000) -- (85.4000,67.5000) -- (85.4000,67.5000) -- (85.4000,67.5000) -- (85.4000,67.5000) -- (85.4000,67.5000) -- (85.4000,67.5000) -- (85.4000,67.5000) -- (85.4000,67.5000) -- (85.4000,67.5000) -- (85.4000,67.5000) -- (85.4000,67.4000) -- (85.4000,67.4000) -- (85.4000,67.4000) -- (85.4000,67.4000) -- (85.4000,67.4000) -- (85.4000,67.4000) -- (85.4000,67.4000) -- (85.4000,67.4000) -- (85.4000,67.4000) -- (85.4000,67.4000) -- (85.5000,67.4000) -- (85.5000,67.4000) -- (85.5000,67.4000) -- (85.5000,67.4000) -- (85.5000,67.4000) -- (85.5000,67.4000) -- (85.5000,67.4000) -- (85.5000,67.4000) -- (85.5000,67.4000) -- (85.5000,67.4000) -- (85.5000,67.3000) -- (85.5000,67.3000) -- (85.5000,67.3000) -- (85.5000,67.3000) -- (85.5000,67.3000) -- (85.5000,67.3000) -- (85.5000,67.3000) -- (85.5000,67.3000) -- (85.5000,67.3000) -- (85.5000,67.3000) -- (85.5000,67.3000) -- (85.5000,67.3000) -- (85.5000,67.3000) -- (85.5000,67.3000) -- (85.5000,67.3000) -- (85.5000,67.3000) -- (85.5000,67.3000) -- (85.5000,67.3000) -- (85.5000,67.3000) -- (85.5000,67.3000) -- (85.5000,67.3000) -- (85.5000,67.2000) -- (85.5000,67.2000) -- (85.5000,67.2000) -- (85.5000,67.2000) -- (85.5000,67.2000) -- (85.5000,67.2000) -- (85.5000,67.2000) -- (85.5000,67.2000) -- (85.5000,67.2000) -- (85.5000,67.2000) -- (85.5000,67.2000) -- (85.5000,67.2000) -- (85.5000,67.2000) -- (85.5000,67.2000) -- (85.5000,67.2000) -- (85.5000,67.2000) -- (85.5000,67.2000) -- (85.5000,67.2000) -- (85.5000,67.2000) -- (85.6000,67.2000) -- (85.6000,67.2000) -- (85.6000,67.1000) -- (85.6000,67.1000) -- (85.6000,67.1000) -- (85.6000,67.1000) -- (85.6000,67.1000) -- (85.6000,67.1000) -- (85.6000,67.1000) -- (85.6000,67.1000) -- (85.6000,67.1000) -- (85.6000,67.1000) -- (85.6000,67.1000) -- (85.6000,67.1000) -- (85.6000,67.1000) -- (85.6000,67.1000) -- (85.6000,67.1000) -- (85.6000,67.1000) -- (85.6000,67.1000) -- (85.6000,67.1000) -- (85.6000,67.1000) -- (85.6000,67.1000) -- (85.6000,67.1000) -- (85.6000,67.0000) -- (85.6000,67.0000) -- (85.6000,67.0000) -- (85.6000,67.0000) -- (85.6000,67.0000) -- (85.6000,67.0000) -- (85.6000,67.0000) -- (85.6000,67.0000) -- (85.6000,67.0000) -- (85.6000,67.0000) -- (85.6000,67.0000) -- (85.6000,67.0000) -- (85.6000,67.0000) -- (85.6000,67.0000) -- (85.6000,67.0000) -- (85.6000,67.0000) -- (85.6000,67.0000) -- (85.6000,67.0000) -- (85.6000,67.0000) -- (85.6000,67.0000) -- (85.6000,67.0000) -- (85.6000,66.9000) -- (85.6000,66.9000) -- (85.6000,66.9000) -- (85.6000,66.9000) -- (85.6000,66.9000) -- (85.7000,66.9000) -- (85.7000,66.9000) -- (85.7000,66.9000) -- (85.7000,66.9000) -- (85.7000,66.9000) -- (85.7000,66.9000) -- (85.7000,66.9000) -- (85.7000,66.9000) -- (85.7000,66.9000) -- (85.7000,66.9000) -- (85.7000,66.9000) -- (85.7000,66.9000) -- (85.7000,66.9000) -- (85.7000,66.9000) -- (85.7000,66.9000) -- (85.7000,66.8000) -- (85.7000,66.8000) -- (85.7000,66.8000) -- (85.7000,66.8000) -- (85.7000,66.8000) -- (85.7000,66.8000) -- (85.7000,66.8000) -- (85.7000,66.8000) -- (85.7000,66.8000) -- (85.7000,66.8000) -- (85.7000,66.8000) -- (85.7000,66.8000) -- (85.7000,66.8000) -- (85.7000,66.8000) -- (85.7000,66.8000) -- (85.7000,66.8000) -- (85.7000,66.8000) -- (85.7000,66.8000) -- (85.7000,66.8000) -- (85.7000,66.8000) -- (85.7000,66.8000) -- (85.7000,66.7000) -- (85.7000,66.7000) -- (85.7000,66.7000) -- (85.7000,66.7000) -- (85.7000,66.7000) -- (85.7000,66.7000) -- (85.7000,66.7000) -- (85.7000,66.7000) -- (85.7000,66.7000) -- (85.7000,66.7000) -- (85.7000,66.7000) -- (85.7000,66.7000) -- (85.7000,66.7000) -- (85.7000,66.7000) -- (85.8000,66.7000) -- (85.8000,66.7000) -- (85.8000,66.7000) -- (85.8000,66.7000) -- (85.8000,66.7000) -- (85.8000,66.7000) -- (85.8000,66.7000) -- (85.8000,66.6000) -- (85.8000,66.6000) -- (85.8000,66.6000) -- (85.8000,66.6000) -- (85.8000,66.6000) -- (85.8000,66.6000) -- (85.8000,66.6000) -- (85.8000,66.6000) -- (85.8000,66.6000) -- (85.8000,66.6000) -- (85.8000,66.6000) -- (85.8000,66.6000) -- (85.8000,66.6000) -- (85.8000,66.6000) -- (85.8000,66.6000) -- (85.8000,66.6000) -- (85.8000,66.6000) -- (85.8000,66.6000) -- (85.8000,66.6000) -- (85.8000,66.6000) -- (85.8000,66.6000) -- (85.8000,66.5000) -- (85.8000,66.5000) -- (85.8000,66.5000) -- (85.8000,66.5000) -- (85.8000,66.5000) -- (85.8000,66.5000) -- (85.8000,66.5000) -- (85.8000,66.5000) -- (85.8000,66.5000) -- (85.8000,66.5000) -- (85.8000,66.5000) -- (85.8000,66.5000) -- (85.8000,66.5000) -- (85.8000,66.5000) -- (85.8000,66.5000) -- (85.8000,66.5000) -- (85.8000,66.5000) -- (85.8000,66.5000) -- (85.8000,66.5000) -- (85.8000,66.5000) -- (85.8000,66.5000) -- (85.9000,66.4000) -- (85.9000,66.4000) -- (85.9000,66.4000) -- (85.9000,66.4000) -- (85.9000,66.4000) -- (85.9000,66.4000) -- (85.9000,66.4000) -- (85.9000,66.4000) -- (85.9000,66.4000) -- (85.9000,66.4000) -- (85.9000,66.4000) -- (85.9000,66.4000) -- (85.9000,66.4000) -- (85.9000,66.4000) -- (85.9000,66.4000) -- (85.9000,66.4000) -- (85.9000,66.4000) -- (85.9000,66.4000) -- (85.9000,66.4000) -- (85.9000,66.4000) -- (85.9000,66.3000) -- (85.9000,66.3000) -- (85.9000,66.3000) -- (85.9000,66.3000) -- (85.9000,66.3000) -- (85.9000,66.3000) -- (85.9000,66.3000) -- (85.9000,66.3000) -- (85.9000,66.3000) -- (85.9000,66.3000) -- (85.9000,66.3000) -- (85.9000,66.3000) -- (85.9000,66.3000) -- (85.9000,66.3000) -- (85.9000,66.3000) -- (85.9000,66.3000) -- (85.9000,66.3000) -- (85.9000,66.3000) -- (85.9000,66.3000) -- (85.9000,66.3000) -- (85.9000,66.3000) -- (85.9000,66.2000) -- (85.9000,66.2000) -- (85.9000,66.2000) -- (85.9000,66.2000) -- (85.9000,66.2000) -- (85.9000,66.2000) -- (85.9000,66.2000) -- (85.9000,66.2000) -- (85.9000,66.2000) -- (86.0000,66.2000) -- (86.0000,66.2000) -- (86.0000,66.2000) -- (86.0000,66.2000) -- (86.0000,66.2000) -- (86.0000,66.2000) -- (86.0000,66.2000) -- (86.0000,66.2000) -- (86.0000,66.2000) -- (86.0000,66.2000) -- (86.0000,66.2000) -- (86.0000,66.2000) -- (86.0000,66.1000) -- (86.0000,66.1000) -- (86.0000,66.1000) -- (86.0000,66.1000) -- (86.0000,66.1000) -- (86.0000,66.1000) -- (86.0000,66.1000) -- (86.0000,66.1000) -- (86.0000,66.1000) -- (86.0000,66.1000) -- (86.0000,66.1000) -- (86.0000,66.1000) -- (86.0000,66.1000) -- (86.0000,66.1000) -- (86.0000,66.1000) -- (86.0000,66.1000) -- (86.0000,66.1000) -- (86.0000,66.1000) -- (86.0000,66.1000) -- (86.0000,66.1000) -- (86.0000,66.1000) -- (86.0000,66.0000) -- (86.0000,66.0000) -- (86.0000,66.0000) -- (86.0000,66.0000) -- (86.0000,66.0000) -- (86.0000,66.0000) -- (86.0000,66.0000) -- (86.0000,66.0000) -- (86.0000,66.0000) -- (86.0000,66.0000) -- (86.0000,66.0000) -- (86.0000,66.0000) -- (86.0000,66.0000) -- (86.0000,66.0000) -- (86.0000,66.0000) -- (86.0000,66.0000) -- (86.1000,66.0000) -- (86.1000,66.0000) -- (86.1000,66.0000) -- (86.1000,66.0000) -- (86.1000,66.0000) -- (86.1000,65.9000) -- (86.1000,65.9000) -- (86.1000,65.9000) -- (86.1000,65.9000) -- (86.1000,65.9000) -- (86.1000,65.9000) -- (86.1000,65.9000) -- (86.1000,65.9000) -- (86.1000,65.9000) -- (86.1000,65.9000) -- (86.1000,65.9000) -- (86.1000,65.9000) -- (86.1000,65.9000) -- (86.1000,65.9000) -- (86.1000,65.9000) -- (86.1000,65.9000) -- (86.1000,65.9000) -- (86.1000,65.9000) -- (86.1000,65.9000) -- (86.1000,65.9000) -- (86.1000,65.8000) -- (86.1000,65.8000) -- (86.1000,65.8000) -- (86.1000,65.8000) -- (86.1000,65.8000) -- (86.1000,65.8000) -- (86.1000,65.8000) -- (86.1000,65.8000) -- (86.1000,65.8000) -- (86.1000,65.8000) -- (86.1000,65.8000) -- (86.1000,65.8000) -- (86.1000,65.8000) -- (86.1000,65.8000) -- (86.1000,65.8000) -- (86.1000,65.8000) -- (86.1000,65.8000) -- (86.1000,65.8000) -- (86.1000,65.8000) -- (86.1000,65.8000) -- (86.1000,65.8000) -- (86.1000,65.7000) -- (86.1000,65.7000) -- (86.1000,65.7000) -- (86.1000,65.7000) -- (86.2000,65.7000) -- (86.2000,65.7000) -- (86.2000,65.7000) -- (86.2000,65.7000) -- (86.2000,65.7000) -- (86.2000,65.7000) -- (86.2000,65.7000) -- (86.2000,65.7000) -- (86.2000,65.7000) -- (86.2000,65.7000) -- (86.2000,65.7000) -- (86.2000,65.7000) -- (86.2000,65.7000) -- (86.2000,65.7000) -- (86.2000,65.7000) -- (86.2000,65.7000) -- (86.2000,65.7000) -- (86.2000,65.6000) -- (86.2000,65.6000) -- (86.2000,65.6000) -- (86.2000,65.6000) -- (86.2000,65.6000) -- (86.2000,65.6000) -- (86.2000,65.6000) -- (86.2000,65.6000) -- (86.2000,65.6000) -- (86.2000,65.6000) -- (86.2000,65.6000) -- (86.2000,65.6000) -- (86.2000,65.6000) -- (86.2000,65.6000) -- (86.2000,65.6000) -- (86.2000,65.6000) -- (86.2000,65.6000) -- (86.2000,65.6000) -- (86.2000,65.6000) -- (86.2000,65.6000) -- (86.2000,65.6000) -- (86.2000,65.5000) -- (86.2000,65.5000) -- (86.2000,65.5000) -- (86.2000,65.5000) -- (86.2000,65.5000) -- (86.2000,65.5000) -- (86.2000,65.5000) -- (86.2000,65.5000) -- (86.2000,65.5000) -- (86.2000,65.5000) -- (86.2000,65.5000) -- (86.2000,65.5000) -- (86.3000,65.5000) -- (86.3000,65.5000) -- (86.3000,65.5000) -- (86.3000,65.5000) -- (86.3000,65.5000) -- (86.3000,65.5000) -- (86.3000,65.5000) -- (86.3000,65.5000) -- (86.3000,65.4000) -- (86.3000,65.4000) -- (86.3000,65.4000) -- (86.3000,65.4000) -- (86.3000,65.4000) -- (86.3000,65.4000) -- (86.3000,65.4000) -- (86.3000,65.4000) -- (86.3000,65.4000) -- (86.3000,65.4000) -- (86.3000,65.4000) -- (86.3000,65.4000) -- (86.3000,65.4000) -- (86.3000,65.4000) -- (86.3000,65.4000) -- (86.3000,65.4000) -- (86.3000,65.4000) -- (86.3000,65.4000) -- (86.3000,65.4000) -- (86.3000,65.4000) -- (86.3000,65.4000) -- (86.3000,65.3000) -- (86.3000,65.3000) -- (86.3000,65.3000) -- (86.3000,65.3000) -- (86.3000,65.3000) -- (86.3000,65.3000) -- (86.3000,65.3000) -- (86.3000,65.3000) -- (86.3000,65.3000) -- (86.3000,65.3000) -- (86.3000,65.3000) -- (86.3000,65.3000) -- (86.3000,65.3000) -- (86.3000,65.3000) -- (86.3000,65.3000) -- (86.3000,65.3000) -- (86.3000,65.3000) -- (86.3000,65.3000) -- (86.3000,65.3000) -- (86.3000,65.3000) -- (86.4000,65.3000) -- (86.4000,65.2000) -- (86.4000,65.2000) -- (86.4000,65.2000) -- (86.4000,65.2000) -- (86.4000,65.2000) -- (86.4000,65.2000) -- (86.4000,65.2000) -- (86.4000,65.2000) -- (86.4000,65.2000) -- (86.4000,65.2000) -- (86.4000,65.2000) -- (86.4000,65.2000) -- (86.4000,65.2000) -- (86.4000,65.2000) -- (86.4000,65.2000) -- (86.4000,65.2000) -- (86.4000,65.2000) -- (86.4000,65.2000) -- (86.4000,65.2000) -- (86.4000,65.2000) -- (86.4000,65.2000) -- (86.4000,65.1000) -- (86.4000,65.1000) -- (86.4000,65.1000) -- (86.4000,65.1000) -- (86.4000,65.1000) -- (86.4000,65.1000) -- (86.4000,65.1000) -- (86.4000,65.1000) -- (86.4000,65.1000) -- (86.4000,65.1000) -- (86.4000,65.1000) -- (86.4000,65.1000) -- (86.4000,65.1000) -- (86.4000,65.1000) -- (86.4000,65.1000) -- (86.4000,65.1000) -- (86.4000,65.1000) -- (86.4000,65.1000) -- (86.4000,65.1000) -- (86.4000,65.1000) -- (86.4000,65.1000) -- (86.4000,65.0000) -- (86.4000,65.0000) -- (86.4000,65.0000) -- (86.4000,65.0000) -- (86.4000,65.0000) -- (86.4000,65.0000) -- (86.4000,65.0000) -- (86.5000,65.0000) -- (86.5000,65.0000) -- (86.5000,65.0000) -- (86.5000,65.0000) -- (86.5000,65.0000) -- (86.5000,65.0000) -- (86.5000,65.0000) -- (86.5000,65.0000) -- (86.5000,65.0000) -- (86.5000,65.0000) -- (86.5000,65.0000) -- (86.5000,65.0000) -- (86.5000,65.0000) -- (86.5000,64.9000) -- (86.5000,64.9000) -- (86.5000,64.9000) -- (86.5000,64.9000) -- (86.5000,64.9000) -- (86.5000,64.9000) -- (86.5000,64.9000) -- (86.5000,64.9000) -- (86.5000,64.9000) -- (86.5000,64.9000) -- (86.5000,64.9000) -- (86.5000,64.9000) -- (86.5000,64.9000) -- (86.5000,64.9000) -- (86.5000,64.9000) -- (86.5000,64.9000) -- (86.5000,64.9000) -- (86.5000,64.9000) -- (86.5000,64.9000) -- (86.5000,64.9000) -- (86.5000,64.9000) -- (86.5000,64.8000) -- (86.5000,64.8000) -- (86.5000,64.8000) -- (86.5000,64.8000) -- (86.5000,64.8000) -- (86.5000,64.8000) -- (86.5000,64.8000) -- (86.5000,64.8000) -- (86.5000,64.8000) -- (86.5000,64.8000) -- (86.5000,64.8000) -- (86.5000,64.8000) -- (86.5000,64.8000) -- (86.5000,64.8000) -- (86.5000,64.8000) -- (86.6000,64.8000) -- (86.6000,64.8000) -- (86.6000,64.8000) -- (86.6000,64.8000) -- (86.6000,64.8000) -- (86.6000,64.8000) -- (86.6000,64.7000) -- (86.6000,64.7000) -- (86.6000,64.7000) -- (86.6000,64.7000) -- (86.6000,64.7000) -- (86.6000,64.7000) -- (86.6000,64.7000) -- (86.6000,64.7000) -- (86.6000,64.7000) -- (86.6000,64.7000) -- (86.6000,64.7000) -- (86.6000,64.7000) -- (86.6000,64.7000) -- (86.6000,64.7000) -- (86.6000,64.7000) -- (86.6000,64.7000) -- (86.6000,64.7000) -- (86.6000,64.7000) -- (86.6000,64.7000) -- (86.6000,64.7000) -- (86.6000,64.7000) -- (86.6000,64.6000) -- (86.6000,64.6000) -- (86.6000,64.6000) -- (86.6000,64.6000) -- (86.6000,64.6000) -- (86.6000,64.6000) -- (86.6000,64.6000) -- (86.6000,64.6000) -- (86.6000,64.6000) -- (86.6000,64.6000) -- (86.6000,64.6000) -- (86.6000,64.6000) -- (86.6000,64.6000) -- (86.6000,64.6000) -- (86.6000,64.6000) -- (86.6000,64.6000) -- (86.6000,64.6000) -- (86.6000,64.6000) -- (86.6000,64.6000) -- (86.6000,64.6000) -- (86.6000,64.6000) -- (86.6000,64.5000) -- (86.6000,64.5000) -- (86.7000,64.5000) -- (86.7000,64.5000) -- (86.7000,64.5000) -- (86.7000,64.5000) -- (86.7000,64.5000) -- (86.7000,64.5000) -- (86.7000,64.5000) -- (86.7000,64.5000) -- (86.7000,64.5000) -- (86.7000,64.5000) -- (86.7000,64.5000) -- (86.7000,64.5000) -- (86.7000,64.5000) -- (86.7000,64.5000) -- (86.7000,64.5000) -- (86.7000,64.5000) -- (86.7000,64.5000) -- (86.7000,64.5000) -- (86.7000,64.4000) -- (86.7000,64.4000) -- (86.7000,64.4000) -- (86.7000,64.4000) -- (86.7000,64.4000) -- (86.7000,64.4000) -- (86.7000,64.4000) -- (86.7000,64.4000) -- (86.7000,64.4000) -- (86.7000,64.4000) -- (86.7000,64.4000) -- (86.7000,64.4000) -- (86.7000,64.4000) -- (86.7000,64.4000) -- (86.7000,64.4000) -- (86.7000,64.4000) -- (86.7000,64.4000) -- (86.7000,64.4000) -- (86.7000,64.4000) -- (86.7000,64.4000) -- (86.7000,64.4000) -- (86.7000,64.3000) -- (86.7000,64.3000) -- (86.7000,64.3000) -- (86.7000,64.3000) -- (86.7000,64.3000) -- (86.7000,64.3000) -- (86.7000,64.3000) -- (86.7000,64.3000) -- (86.7000,64.3000) -- (86.7000,64.3000) -- (86.8000,64.3000) -- (86.8000,64.3000) -- (86.8000,64.3000) -- (86.8000,64.3000) -- (86.8000,64.3000) -- (86.8000,64.3000) -- (86.8000,64.3000) -- (86.8000,64.3000) -- (86.8000,64.3000) -- (86.8000,64.3000) -- (86.8000,64.3000) -- (86.8000,64.2000) -- (86.8000,64.2000) -- (86.8000,64.2000) -- (86.8000,64.2000) -- (86.8000,64.2000) -- (86.8000,64.2000) -- (86.8000,64.2000) -- (86.8000,64.2000) -- (86.8000,64.2000) -- (86.8000,64.2000) -- (86.8000,64.2000) -- (86.8000,64.2000) -- (86.8000,64.2000) -- (86.8000,64.2000) -- (86.8000,64.2000) -- (86.8000,64.2000) -- (86.8000,64.2000) -- (86.8000,64.2000) -- (86.8000,64.2000) -- (86.8000,64.2000) -- (86.8000,64.2000) -- (86.8000,64.1000) -- (86.8000,64.1000) -- (86.8000,64.1000) -- (86.8000,64.1000) -- (86.8000,64.1000) -- (86.8000,64.1000) -- (86.8000,64.1000) -- (86.8000,64.1000) -- (86.8000,64.1000) -- (86.8000,64.1000) -- (86.8000,64.1000) -- (86.8000,64.1000) -- (86.8000,64.1000) -- (86.8000,64.1000) -- (86.8000,64.1000) -- (86.8000,64.1000) -- (86.8000,64.1000) -- (86.8000,64.1000) -- (86.9000,64.1000) -- (86.9000,64.1000) -- (86.9000,64.0000) -- (86.9000,64.0000) -- (86.9000,64.0000) -- (86.9000,64.0000) -- (86.9000,64.0000) -- (86.9000,64.0000) -- (86.9000,64.0000) -- (86.9000,64.0000) -- (86.9000,64.0000) -- (86.9000,64.0000) -- (86.9000,64.0000) -- (86.9000,64.0000) -- (86.9000,64.0000) -- (86.9000,64.0000) -- (86.9000,64.0000) -- (86.9000,64.0000) -- (86.9000,64.0000) -- (86.9000,64.0000) -- (86.9000,64.0000) -- (86.9000,64.0000) -- (86.9000,64.0000) -- (86.9000,63.9000) -- (86.9000,63.9000) -- (86.9000,63.9000) -- (86.9000,63.9000) -- (86.9000,63.9000) -- (86.9000,63.9000) -- (86.9000,63.9000) -- (86.9000,63.9000) -- (86.9000,63.9000) -- (86.9000,63.9000) -- (86.9000,63.9000) -- (86.9000,63.9000) -- (86.9000,63.9000) -- (86.9000,63.9000) -- (86.9000,63.9000) -- (86.9000,63.9000) -- (86.9000,63.9000) -- (86.9000,63.9000) -- (86.9000,63.9000) -- (86.9000,63.9000) -- (86.9000,63.9000) -- (86.9000,63.8000) -- (86.9000,63.8000) -- (86.9000,63.8000) -- (86.9000,63.8000) -- (86.9000,63.8000) -- (87.0000,63.8000) -- (87.0000,63.8000) -- (87.0000,63.8000) -- (87.0000,63.8000) -- (87.0000,63.8000) -- (87.0000,63.8000) -- (87.0000,63.8000) -- (87.0000,63.8000) -- (87.0000,63.8000) -- (87.0000,63.8000) -- (87.0000,63.8000) -- (87.0000,63.8000) -- (87.0000,63.8000) -- (87.0000,63.8000) -- (87.0000,63.8000) -- (87.0000,63.8000) -- (87.0000,63.7000) -- (87.0000,63.7000) -- (87.0000,63.7000) -- (87.0000,63.7000) -- (87.0000,63.7000) -- (87.0000,63.7000) -- (87.0000,63.7000) -- (87.0000,63.7000) -- (87.0000,63.7000) -- (87.0000,63.7000) -- (87.0000,63.7000) -- (87.0000,63.7000) -- (87.0000,63.7000) -- (87.0000,63.7000) -- (87.0000,63.7000) -- (87.0000,63.7000) -- (87.0000,63.7000) -- (87.0000,63.7000) -- (87.0000,63.7000) -- (87.0000,63.7000) -- (87.0000,63.7000) -- (87.0000,63.6000) -- (87.0000,63.6000) -- (87.0000,63.6000) -- (87.0000,63.6000) -- (87.0000,63.6000) -- (87.0000,63.6000) -- (87.0000,63.6000) -- (87.0000,63.6000) -- (87.0000,63.6000) -- (87.0000,63.6000) -- (87.0000,63.6000) -- (87.0000,63.6000) -- (87.0000,63.6000) -- (87.1000,63.6000) -- (87.1000,63.6000) -- (87.1000,63.6000) -- (87.1000,63.6000) -- (87.1000,63.6000) -- (87.1000,63.6000) -- (87.1000,63.6000) -- (87.1000,63.5000) -- (87.1000,63.5000) -- (87.1000,63.5000) -- (87.1000,63.5000) -- (87.1000,63.5000) -- (87.1000,63.5000) -- (87.1000,63.5000) -- (87.1000,63.5000) -- (87.1000,63.5000) -- (87.1000,63.5000) -- (87.1000,63.5000) -- (87.1000,63.5000) -- (87.1000,63.5000) -- (87.1000,63.5000) -- (87.1000,63.5000) -- (87.1000,63.5000) -- (87.1000,63.5000) -- (87.1000,63.5000) -- (87.1000,63.5000) -- (87.1000,63.5000) -- (87.1000,63.5000) -- (87.1000,63.4000) -- (87.1000,63.4000) -- (87.1000,63.4000) -- (87.1000,63.4000) -- (87.1000,63.4000) -- (87.1000,63.4000) -- (87.1000,63.4000) -- (87.1000,63.4000) -- (87.1000,63.4000) -- (87.1000,63.4000) -- (87.1000,63.4000) -- (87.1000,63.4000) -- (87.1000,63.4000) -- (87.1000,63.4000) -- (87.1000,63.4000) -- (87.1000,63.4000) -- (87.1000,63.4000) -- (87.1000,63.4000) -- (87.1000,63.4000) -- (87.1000,63.4000) -- (87.1000,63.4000) -- (87.2000,63.3000) -- (87.2000,63.3000) -- (87.2000,63.3000) -- (87.2000,63.3000) -- (87.2000,63.3000) -- (87.2000,63.3000) -- (87.2000,63.3000) -- (87.2000,63.3000) -- (87.2000,63.3000) -- (87.2000,63.3000) -- (87.2000,63.3000) -- (87.2000,63.3000) -- (87.2000,63.3000) -- (87.2000,63.3000) -- (87.2000,63.3000) -- (87.2000,63.3000) -- (87.2000,63.3000) -- (87.2000,63.3000) -- (87.2000,63.3000) -- (87.2000,63.3000) -- (87.2000,63.3000) -- (87.2000,63.2000) -- (87.2000,63.2000) -- (87.2000,63.2000) -- (87.2000,63.2000) -- (87.2000,63.2000) -- (87.2000,63.2000) -- (87.2000,63.2000) -- (87.2000,63.2000) -- (87.2000,63.2000) -- (87.2000,63.2000) -- (87.2000,63.2000) -- (87.2000,63.2000) -- (87.2000,63.2000) -- (87.2000,63.2000) -- (87.2000,63.2000) -- (87.2000,63.2000) -- (87.2000,63.2000) -- (87.2000,63.2000) -- (87.2000,63.2000) -- (87.2000,63.2000) -- (87.2000,63.2000) -- (87.2000,63.1000) -- (87.2000,63.1000) -- (87.2000,63.1000) -- (87.2000,63.1000) -- (87.2000,63.1000) -- (87.2000,63.1000) -- (87.2000,63.1000) -- (87.2000,63.1000) -- (87.3000,63.1000) -- (87.3000,63.1000) -- (87.3000,63.1000) -- (87.3000,63.1000) -- (87.3000,63.1000) -- (87.3000,63.1000) -- (87.3000,63.1000) -- (87.3000,63.1000) -- (87.3000,63.1000) -- (87.3000,63.1000) -- (87.3000,63.1000) -- (87.3000,63.1000) -- (87.3000,63.0000) -- (87.3000,63.0000) -- (87.3000,63.0000) -- (87.3000,63.0000) -- (87.3000,63.0000) -- (87.3000,63.0000) -- (87.3000,63.0000) -- (87.3000,63.0000) -- (87.3000,63.0000) -- (87.3000,63.0000) -- (87.3000,63.0000) -- (87.3000,63.0000) -- (87.3000,63.0000) -- (87.3000,63.0000) -- (87.3000,63.0000) -- (87.3000,63.0000) -- (87.3000,63.0000) -- (87.3000,63.0000) -- (87.3000,63.0000) -- (87.3000,63.0000) -- (87.3000,63.0000) -- (87.3000,62.9000) -- (87.3000,62.9000) -- (87.3000,62.9000) -- (87.3000,62.9000) -- (87.3000,62.9000) -- (87.3000,62.9000) -- (87.3000,62.9000) -- (87.3000,62.9000) -- (87.3000,62.9000) -- (87.3000,62.9000) -- (87.3000,62.9000) -- (87.3000,62.9000) -- (87.3000,62.9000) -- (87.3000,62.9000) -- (87.3000,62.9000) -- (87.3000,62.9000) -- (87.4000,62.9000) -- (87.4000,62.9000) -- (87.4000,62.9000) -- (87.4000,62.9000) -- (87.4000,62.9000) -- (87.4000,62.8000) -- (87.4000,62.8000) -- (87.4000,62.8000) -- (87.4000,62.8000) -- (87.4000,62.8000) -- (87.4000,62.8000) -- (87.4000,62.8000) -- (87.4000,62.8000) -- (87.4000,62.8000) -- (87.4000,62.8000) -- (87.4000,62.8000) -- (87.4000,62.8000) -- (87.4000,62.8000) -- (87.4000,62.8000) -- (87.4000,62.8000) -- (87.4000,62.8000) -- (87.4000,62.8000) -- (87.4000,62.8000) -- (87.4000,62.8000) -- (87.4000,62.8000) -- (87.4000,62.8000) -- (87.4000,62.7000) -- (87.4000,62.7000) -- (87.4000,62.7000) -- (87.4000,62.7000) -- (87.4000,62.7000) -- (87.4000,62.7000) -- (87.4000,62.7000) -- (87.4000,62.7000) -- (87.4000,62.7000) -- (87.4000,62.7000) -- (87.4000,62.7000) -- (87.4000,62.7000) -- (87.4000,62.7000) -- (87.4000,62.7000) -- (87.4000,62.7000) -- (87.4000,62.7000) -- (87.4000,62.7000) -- (87.4000,62.7000) -- (87.4000,62.7000) -- (87.4000,62.7000) -- (87.4000,62.7000) -- (87.4000,62.6000) -- (87.4000,62.6000) -- (87.4000,62.6000) -- (87.5000,62.6000) -- (87.5000,62.6000) -- (87.5000,62.6000) -- (87.5000,62.6000) -- (87.5000,62.6000) -- (87.5000,62.6000) -- (87.5000,62.6000) -- (87.5000,62.6000) -- (87.5000,62.6000) -- (87.5000,62.6000) -- (87.5000,62.6000) -- (87.5000,62.6000) -- (87.5000,62.6000) -- (87.5000,62.6000) -- (87.5000,62.6000) -- (87.5000,62.6000) -- (87.5000,62.6000) -- (87.5000,62.5000) -- (87.5000,62.5000) -- (87.5000,62.5000) -- (87.5000,62.5000) -- (87.5000,62.5000) -- (87.5000,62.5000) -- (87.5000,62.5000) -- (87.5000,62.5000) -- (87.5000,62.5000) -- (87.5000,62.5000) -- (87.5000,62.5000) -- (87.5000,62.5000) -- (87.5000,62.5000) -- (87.5000,62.5000) -- (87.5000,62.5000) -- (87.5000,62.5000) -- (87.5000,62.5000) -- (87.5000,62.5000) -- (87.5000,62.5000) -- (87.5000,62.5000) -- (87.5000,62.5000) -- (87.5000,62.4000) -- (87.5000,62.4000) -- (87.5000,62.4000) -- (87.5000,62.4000) -- (87.5000,62.4000) -- (87.5000,62.4000) -- (87.5000,62.4000) -- (87.5000,62.4000) -- (87.5000,62.4000) -- (87.5000,62.4000) -- (87.5000,62.4000) -- (87.6000,62.4000) -- (87.6000,62.4000) -- (87.6000,62.4000) -- (87.6000,62.4000) -- (87.6000,62.4000) -- (87.6000,62.4000) -- (87.6000,62.4000) -- (87.6000,62.4000) -- (87.6000,62.4000) -- (87.6000,62.4000) -- (87.6000,62.3000) -- (87.6000,62.3000) -- (87.6000,62.3000) -- (87.6000,62.3000) -- (87.6000,62.3000) -- (87.6000,62.3000) -- (87.6000,62.3000) -- (87.6000,62.3000) -- (87.6000,62.3000) -- (87.6000,62.3000) -- (87.6000,62.3000) -- (87.6000,62.3000) -- (87.6000,62.3000) -- (87.6000,62.3000) -- (87.6000,62.3000) -- (87.6000,62.3000) -- (87.6000,62.3000) -- (87.6000,62.3000) -- (87.6000,62.3000) -- (87.6000,62.3000) -- (87.6000,62.3000) -- (87.6000,62.2000) -- (87.6000,62.2000) -- (87.6000,62.2000) -- (87.6000,62.2000) -- (87.6000,62.2000) -- (87.6000,62.2000) -- (87.6000,62.2000) -- (87.6000,62.2000) -- (87.6000,62.2000) -- (87.6000,62.2000) -- (87.6000,62.2000) -- (87.6000,62.2000) -- (87.6000,62.2000) -- (87.6000,62.2000) -- (87.6000,62.2000) -- (87.6000,62.2000) -- (87.6000,62.2000) -- (87.6000,62.2000) -- (87.6000,62.2000) -- (87.7000,62.2000) -- (87.7000,62.1000) -- (87.7000,62.1000) -- (87.7000,62.1000) -- (87.7000,62.1000) -- (87.7000,62.1000) -- (87.7000,62.1000) -- (87.7000,62.1000) -- (87.7000,62.1000) -- (87.7000,62.1000) -- (87.7000,62.1000) -- (87.7000,62.1000) -- (87.7000,62.1000) -- (87.7000,62.1000) -- (87.7000,62.1000) -- (87.7000,62.1000) -- (87.7000,62.1000) -- (87.7000,62.1000) -- (87.7000,62.1000) -- (87.7000,62.1000) -- (87.7000,62.1000) -- (87.7000,62.1000) -- (87.7000,62.0000) -- (87.7000,62.0000) -- (87.7000,62.0000) -- (87.7000,62.0000) -- (87.7000,62.0000) -- (87.7000,62.0000) -- (87.7000,62.0000) -- (87.7000,62.0000) -- (87.7000,62.0000) -- (87.7000,62.0000) -- (87.7000,62.0000) -- (87.7000,62.0000) -- (87.7000,62.0000) -- (87.7000,62.0000) -- (87.7000,62.0000) -- (87.7000,62.0000) -- (87.7000,62.0000) -- (87.7000,62.0000) -- (87.7000,62.0000) -- (87.7000,62.0000) -- (87.7000,62.0000) -- (87.7000,61.9000) -- (87.7000,61.9000) -- (87.7000,61.9000) -- (87.7000,61.9000) -- (87.7000,61.9000) -- (87.7000,61.9000) -- (87.8000,61.9000) -- (87.8000,61.9000) -- (87.8000,61.9000) -- (87.8000,61.9000) -- (87.8000,61.9000) -- (87.8000,61.9000) -- (87.8000,61.9000) -- (87.8000,61.9000) -- (87.8000,61.9000) -- (87.8000,61.9000) -- (87.8000,61.9000) -- (87.8000,61.9000) -- (87.8000,61.9000) -- (87.8000,61.9000) -- (87.8000,61.9000) -- (87.8000,61.8000) -- (87.8000,61.8000) -- (87.8000,61.8000) -- (87.8000,61.8000) -- (87.8000,61.8000) -- (87.8000,61.8000) -- (87.8000,61.8000) -- (87.8000,61.8000) -- (87.8000,61.8000) -- (87.8000,61.8000) -- (87.8000,61.8000) -- (87.8000,61.8000) -- (87.8000,61.8000) -- (87.8000,61.8000) -- (87.8000,61.8000) -- (87.8000,61.8000) -- (87.8000,61.8000) -- (87.8000,61.8000) -- (87.8000,61.8000) -- (87.8000,61.8000) -- (87.8000,61.8000) -- (87.8000,61.7000) -- (87.8000,61.7000) -- (87.8000,61.7000) -- (87.8000,61.7000) -- (87.8000,61.7000) -- (87.8000,61.7000) -- (87.8000,61.7000) -- (87.8000,61.7000) -- (87.8000,61.7000) -- (87.8000,61.7000) -- (87.8000,61.7000) -- (87.8000,61.7000) -- (87.8000,61.7000) -- (87.8000,61.7000) -- (87.9000,61.7000) -- (87.9000,61.7000) -- (87.9000,61.7000) -- (87.9000,61.7000) -- (87.9000,61.7000) -- (87.9000,61.7000) -- (87.9000,61.6000) -- (87.9000,61.6000) -- (87.9000,61.6000) -- (87.9000,61.6000) -- (87.9000,61.6000) -- (87.9000,61.6000) -- (87.9000,61.6000) -- (87.9000,61.6000) -- (87.9000,61.6000) -- (87.9000,61.6000) -- (87.9000,61.6000) -- (87.9000,61.6000) -- (87.9000,61.6000) -- (87.9000,61.6000) -- (87.9000,61.6000) -- (87.9000,61.6000) -- (87.9000,61.6000) -- (87.9000,61.6000) -- (87.9000,61.6000) -- (87.9000,61.6000) -- (87.9000,61.6000) -- (87.9000,61.5000) -- (87.9000,61.5000) -- (87.9000,61.5000) -- (87.9000,61.5000) -- (87.9000,61.5000) -- (87.9000,61.5000) -- (87.9000,61.5000) -- (87.9000,61.5000) -- (87.9000,61.5000) -- (87.9000,61.5000) -- (87.9000,61.5000) -- (87.9000,61.5000) -- (87.9000,61.5000) -- (87.9000,61.5000) -- (87.9000,61.5000) -- (87.9000,61.5000) -- (87.9000,61.5000) -- (87.9000,61.5000) -- (87.9000,61.5000) -- (87.9000,61.5000) -- (87.9000,61.5000) -- (87.9000,61.4000) -- (88.0000,61.4000) -- (88.0000,61.4000) -- (88.0000,61.4000) -- (88.0000,61.4000) -- (88.0000,61.4000) -- (88.0000,61.4000) -- (88.0000,61.4000) -- (88.0000,61.4000) -- (88.0000,61.4000) -- (88.0000,61.4000) -- (88.0000,61.4000) -- (88.0000,61.4000) -- (88.0000,61.4000) -- (88.0000,61.4000) -- (88.0000,61.4000) -- (88.0000,61.4000) -- (88.0000,61.4000) -- (88.0000,61.4000) -- (88.0000,61.4000) -- (88.0000,61.4000) -- (88.0000,61.3000) -- (88.0000,61.3000) -- (88.0000,61.3000) -- (88.0000,61.3000) -- (88.0000,61.3000) -- (88.0000,61.3000) -- (88.0000,61.3000) -- (88.0000,61.3000) -- (88.0000,61.3000) -- (88.0000,61.3000) -- (88.0000,61.3000) -- (88.0000,61.3000) -- (88.0000,61.3000) -- (88.0000,61.3000) -- (88.0000,61.3000) -- (88.0000,61.3000) -- (88.0000,61.3000) -- (88.0000,61.3000) -- (88.0000,61.3000) -- (88.0000,61.3000) -- (88.0000,61.3000) -- (88.0000,61.2000) -- (88.0000,61.2000) -- (88.0000,61.2000) -- (88.0000,61.2000) -- (88.0000,61.2000) -- (88.0000,61.2000) -- (88.0000,61.2000) -- (88.0000,61.2000) -- (88.0000,61.2000) -- (88.1000,61.2000) -- (88.1000,61.2000) -- (88.1000,61.2000) -- (88.1000,61.2000) -- (88.1000,61.2000) -- (88.1000,61.2000) -- (88.1000,61.2000) -- (88.1000,61.2000) -- (88.1000,61.2000) -- (88.1000,61.2000) -- (88.1000,61.2000) -- (88.1000,61.1000) -- (88.1000,61.1000) -- (88.1000,61.1000) -- (88.1000,61.1000) -- (88.1000,61.1000) -- (88.1000,61.1000) -- (88.1000,61.1000) -- (88.1000,61.1000) -- (88.1000,61.1000) -- (88.1000,61.1000) -- (88.1000,61.1000) -- (88.1000,61.1000) -- (88.1000,61.1000) -- (88.1000,61.1000) -- (88.1000,61.1000) -- (88.1000,61.1000) -- (88.1000,61.1000) -- (88.1000,61.1000) -- (88.1000,61.1000) -- (88.1000,61.1000) -- (88.1000,61.1000) -- (88.1000,61.0000) -- (88.1000,61.0000) -- (88.1000,61.0000) -- (88.1000,61.0000) -- (88.1000,61.0000) -- (88.1000,61.0000) -- (88.1000,61.0000) -- (88.1000,61.0000) -- (88.1000,61.0000) -- (88.1000,61.0000) -- (88.1000,61.0000) -- (88.1000,61.0000) -- (88.1000,61.0000) -- (88.1000,61.0000) -- (88.1000,61.0000) -- (88.1000,61.0000) -- (88.1000,61.0000) -- (88.2000,61.0000) -- (88.2000,61.0000) -- (88.2000,61.0000) -- (88.2000,61.0000) -- (88.2000,60.9000) -- (88.2000,60.9000) -- (88.2000,60.9000) -- (88.2000,60.9000) -- (88.2000,60.9000) -- (88.2000,60.9000) -- (88.2000,60.9000) -- (88.2000,60.9000) -- (88.2000,60.9000) -- (88.2000,60.9000) -- (88.2000,60.9000) -- (88.2000,60.9000) -- (88.2000,60.9000) -- (88.2000,60.9000) -- (88.2000,60.9000) -- (88.2000,60.9000) -- (88.2000,60.9000) -- (88.2000,60.9000) -- (88.2000,60.9000) -- (88.2000,60.9000) -- (88.2000,60.9000) -- (88.2000,60.8000) -- (88.2000,60.8000) -- (88.2000,60.8000) -- (88.2000,60.8000) -- (88.2000,60.8000) -- (88.2000,60.8000) -- (88.2000,60.8000) -- (88.2000,60.8000) -- (88.2000,60.8000) -- (88.2000,60.8000) -- (88.2000,60.8000) -- (88.2000,60.8000) -- (88.2000,60.8000) -- (88.2000,60.8000) -- (88.2000,60.8000) -- (88.2000,60.8000) -- (88.2000,60.8000) -- (88.2000,60.8000) -- (88.2000,60.8000) -- (88.2000,60.8000) -- (88.2000,60.8000) -- (88.2000,60.7000) -- (88.2000,60.7000) -- (88.2000,60.7000) -- (88.2000,60.7000) -- (88.3000,60.7000) -- (88.3000,60.7000) -- (88.3000,60.7000) -- (88.3000,60.7000) -- (88.3000,60.7000) -- (88.3000,60.7000) -- (88.3000,60.7000) -- (88.3000,60.7000) -- (88.3000,60.7000) -- (88.3000,60.7000) -- (88.3000,60.7000) -- (88.3000,60.7000) -- (88.3000,60.7000) -- (88.3000,60.7000) -- (88.3000,60.7000) -- (88.3000,60.7000) -- (88.3000,60.6000) -- (88.3000,60.6000) -- (88.3000,60.6000) -- (88.3000,60.6000) -- (88.3000,60.6000) -- (88.3000,60.6000) -- (88.3000,60.6000) -- (88.3000,60.6000) -- (88.3000,60.6000) -- (88.3000,60.6000) -- (88.3000,60.6000) -- (88.3000,60.6000) -- (88.3000,60.6000) -- (88.3000,60.6000) -- (88.3000,60.6000) -- (88.3000,60.6000) -- (88.3000,60.6000) -- (88.3000,60.6000) -- (88.3000,60.6000) -- (88.3000,60.6000) -- (88.3000,60.6000) -- (95.7000,60.5000) -- (95.7000,60.5000) -- (95.7000,60.5000) -- (95.7000,60.5000) -- (95.7000,60.5000) -- (95.7000,60.5000) -- (95.7000,60.5000) -- (95.7000,60.5000) -- (95.7000,60.5000) -- (95.7000,60.5000) -- (95.7000,60.5000) -- (95.7000,60.5000) -- (95.7000,60.5000) -- (95.7000,60.5000) -- (95.7000,60.5000) -- (95.7000,60.5000) -- (95.7000,60.5000) -- (95.7000,60.5000) -- (95.7000,60.5000) -- (95.7000,60.5000) -- (95.7000,60.5000) -- (95.7000,60.4000) -- (95.7000,60.4000) -- (95.7000,60.4000) -- (95.7000,60.4000) -- (95.7000,60.4000) -- (95.7000,60.4000) -- (95.7000,60.4000) -- (95.7000,60.4000) -- (95.7000,60.4000) -- (95.7000,60.4000) -- (95.7000,60.4000) -- (95.7000,60.4000) -- (95.7000,60.4000) -- (95.7000,60.4000) -- (95.7000,60.4000) -- (95.7000,60.4000) -- (95.7000,60.4000) -- (95.7000,60.4000) -- (95.7000,60.4000) -- (95.7000,60.4000) -- (95.7000,60.4000) -- (95.7000,60.3000) -- (95.7000,60.3000) -- (95.7000,60.3000) -- (95.7000,60.3000) -- (95.7000,60.3000) -- (95.7000,60.3000) -- (95.8000,60.3000) -- (95.8000,60.3000) -- (95.8000,60.3000) -- (95.8000,60.3000) -- (95.8000,60.3000) -- (95.8000,60.3000) -- (95.8000,60.3000) -- (95.8000,60.3000) -- (95.8000,60.3000) -- (95.8000,60.3000) -- (95.8000,60.3000) -- (95.8000,60.3000) -- (95.8000,60.3000) -- (95.8000,60.3000) -- (95.8000,60.2000) -- (95.8000,60.2000) -- (95.8000,60.2000) -- (95.8000,60.2000) -- (95.8000,60.2000) -- (95.8000,60.2000) -- (95.8000,60.2000) -- (95.8000,60.2000) -- (95.8000,60.2000) -- (95.8000,60.2000) -- (95.8000,60.2000) -- (95.8000,60.2000) -- (95.8000,60.2000) -- (95.8000,60.2000) -- (95.8000,60.2000) -- (95.8000,60.2000) -- (95.8000,60.2000) -- (95.8000,60.2000) -- (95.8000,60.2000) -- (95.8000,60.2000) -- (95.8000,60.2000) -- (95.8000,60.1000) -- (95.8000,60.1000) -- (95.8000,60.1000) -- (95.8000,60.1000) -- (95.8000,60.1000) -- (95.8000,60.1000) -- (95.8000,60.1000) -- (95.8000,60.1000) -- (95.8000,60.1000) -- (95.8000,60.1000) -- (95.8000,60.1000) -- (95.8000,60.1000) -- (95.8000,60.1000) -- (95.8000,60.1000) -- (95.8000,60.1000) -- (95.9000,60.1000) -- (95.9000,60.1000) -- (95.9000,60.1000) -- (95.9000,60.1000) -- (95.9000,60.1000) -- (95.9000,60.1000) -- (95.9000,60.0000) -- (95.9000,60.0000) -- (95.9000,60.0000) -- (95.9000,60.0000) -- (95.9000,60.0000) -- (95.9000,60.0000) -- (95.9000,60.0000) -- (95.9000,60.0000) -- (95.9000,60.0000) -- (95.9000,60.0000) -- (95.9000,60.0000) -- (95.9000,60.0000) -- (95.9000,60.0000) -- (95.9000,60.0000) -- (95.9000,60.0000) -- (95.9000,60.0000) -- (95.9000,60.0000) -- (95.9000,60.0000) -- (95.9000,60.0000) -- (95.9000,60.0000) -- (95.9000,60.0000) -- (95.9000,59.9000) -- (95.9000,59.9000) -- (95.9000,59.9000) -- (95.9000,59.9000) -- (95.9000,59.9000) -- (95.9000,59.9000) -- (95.9000,59.9000) -- (95.9000,59.9000) -- (95.9000,59.9000) -- (95.9000,59.9000) -- (95.9000,59.9000) -- (95.9000,59.9000) -- (95.9000,59.9000) -- (95.9000,59.9000) -- (95.9000,59.9000) -- (95.9000,59.9000) -- (95.9000,59.9000) -- (95.9000,59.9000) -- (95.9000,59.9000) -- (95.9000,59.9000) -- (95.9000,59.9000) -- (95.9000,59.8000) -- (96.0000,59.8000) -- (96.0000,59.8000) -- (96.0000,59.8000) -- (96.0000,59.8000) -- (96.0000,59.8000) -- (96.0000,59.8000) -- (96.0000,59.8000) -- (96.0000,59.8000) -- (96.0000,59.8000) -- (96.0000,59.8000) -- (96.0000,59.8000) -- (96.0000,59.8000) -- (96.0000,59.8000) -- (96.0000,59.8000) -- (96.0000,59.8000) -- (96.0000,59.8000) -- (96.0000,59.8000) -- (96.0000,59.8000) -- (96.0000,59.8000) -- (96.0000,59.7000) -- (96.0000,59.7000) -- (96.0000,59.7000) -- (96.0000,59.7000) -- (96.0000,59.7000) -- (96.0000,59.7000) -- (96.0000,59.7000) -- (96.0000,59.7000) -- (96.0000,59.7000) -- (96.0000,59.7000) -- (96.0000,59.7000) -- (96.0000,59.7000) -- (96.0000,59.7000) -- (96.0000,59.7000) -- (96.0000,59.7000) -- (96.0000,59.7000) -- (96.0000,59.7000) -- (96.0000,59.7000) -- (96.0000,59.7000) -- (96.0000,59.7000) -- (96.0000,59.7000) -- (96.0000,59.6000) -- (96.0000,59.6000) -- (96.0000,59.6000) -- (96.0000,59.6000) -- (96.0000,59.6000) -- (96.0000,59.6000) -- (96.0000,59.6000) -- (96.0000,59.6000) -- (96.0000,59.6000) -- (96.0000,59.6000) -- (96.1000,59.6000) -- (96.1000,59.6000) -- (96.1000,59.6000) -- (96.1000,59.6000) -- (96.1000,59.6000) -- (96.1000,59.6000) -- (96.1000,59.6000) -- (96.1000,59.6000) -- (96.1000,59.6000) -- (96.1000,59.6000) -- (96.1000,59.6000) -- (96.1000,59.5000) -- (96.1000,59.5000) -- (96.1000,59.5000) -- (96.1000,59.5000) -- (96.1000,59.5000) -- (96.1000,59.5000) -- (96.1000,59.5000) -- (96.1000,59.5000) -- (96.1000,59.5000) -- (96.1000,59.5000) -- (96.1000,59.5000) -- (96.1000,59.5000) -- (96.1000,59.5000) -- (96.1000,59.5000) -- (96.1000,59.5000) -- (96.1000,59.5000) -- (96.1000,59.5000) -- (96.1000,59.5000) -- (96.1000,59.5000) -- (96.1000,59.5000) -- (96.1000,59.5000) -- (96.1000,59.4000) -- (96.1000,59.4000) -- (96.1000,59.4000) -- (96.1000,59.4000) -- (96.1000,59.4000) -- (96.1000,59.4000) -- (96.1000,59.4000) -- (96.1000,59.4000) -- (96.1000,59.4000) -- (96.1000,59.4000) -- (96.1000,59.4000) -- (96.1000,59.4000) -- (96.1000,59.4000) -- (96.1000,59.4000) -- (96.1000,59.4000) -- (96.1000,59.4000) -- (96.1000,59.4000) -- (96.2000,59.4000) -- (96.2000,59.4000) -- (96.2000,59.4000) -- (96.2000,59.4000) -- (96.2000,59.3000) -- (96.2000,59.3000) -- (96.2000,59.3000) -- (96.2000,59.3000) -- (96.2000,59.3000) -- (96.2000,59.3000) -- (96.2000,59.3000) -- (96.2000,59.3000) -- (96.2000,59.3000) -- (96.2000,59.3000) -- (96.2000,59.3000) -- (96.2000,59.3000) -- (96.2000,59.3000) -- (96.2000,59.3000) -- (96.2000,59.3000) -- (96.2000,59.3000) -- (96.2000,59.3000) -- (96.2000,59.3000) -- (96.2000,59.3000) -- (96.2000,59.3000) -- (96.2000,59.2000) -- (96.2000,59.2000) -- (96.2000,59.2000) -- (96.2000,59.2000) -- (96.2000,59.2000) -- (96.2000,59.2000) -- (96.2000,59.2000) -- (96.2000,59.2000) -- (96.2000,59.2000) -- (96.2000,59.2000) -- (96.2000,59.2000) -- (96.2000,59.2000) -- (96.2000,59.2000) -- (96.2000,59.2000) -- (96.2000,59.2000) -- (96.2000,59.2000) -- (96.2000,59.2000) -- (96.2000,59.2000) -- (96.2000,59.2000) -- (96.2000,59.2000) -- (96.2000,59.2000) -- (96.2000,59.1000) -- (96.2000,59.1000) -- (96.2000,59.1000) -- (96.2000,59.1000) -- (96.2000,59.1000) -- (96.3000,59.1000) -- (96.3000,59.1000) -- (96.3000,59.1000) -- (96.3000,59.1000) -- (96.3000,59.1000) -- (96.3000,59.1000) -- (96.3000,59.1000) -- (96.3000,59.1000) -- (96.3000,59.1000) -- (96.3000,59.1000) -- (96.3000,59.1000) -- (96.3000,59.1000) -- (96.3000,59.1000) -- (96.3000,59.1000) -- (96.3000,59.1000) -- (96.3000,59.1000) -- (96.3000,59.0000) -- (96.3000,59.0000) -- (96.3000,59.0000) -- (96.3000,59.0000) -- (96.3000,59.0000) -- (96.3000,59.0000) -- (96.3000,59.0000) -- (96.3000,59.0000) -- (96.3000,59.0000) -- (96.3000,59.0000) -- (96.3000,59.0000) -- (96.3000,59.0000) -- (96.3000,59.0000) -- (96.3000,59.0000) -- (96.3000,59.0000) -- (96.3000,59.0000) -- (96.3000,59.0000) -- (96.3000,59.0000) -- (96.3000,59.0000) -- (96.3000,59.0000) -- (96.3000,59.0000) -- (96.3000,58.9000) -- (96.3000,58.9000) -- (96.3000,58.9000) -- (96.3000,58.9000) -- (96.3000,58.9000) -- (96.3000,58.9000) -- (96.3000,58.9000) -- (96.3000,58.9000) -- (96.3000,58.9000) -- (96.3000,58.9000) -- (96.3000,58.9000) -- (96.3000,58.9000) -- (96.3000,58.9000) -- (96.4000,58.9000) -- (96.4000,58.9000) -- (96.4000,58.9000) -- (96.4000,58.9000) -- (96.4000,58.9000) -- (96.4000,58.9000) -- (96.4000,58.9000) -- (96.4000,58.9000) -- (96.4000,58.8000) -- (96.4000,58.8000) -- (96.4000,58.8000) -- (96.4000,58.8000) -- (96.4000,58.8000) -- (96.4000,58.8000) -- (96.4000,58.8000) -- (96.4000,58.8000) -- (96.4000,58.8000) -- (96.4000,58.8000) -- (96.4000,58.8000) -- (96.4000,58.8000) -- (96.4000,58.8000) -- (96.4000,58.8000) -- (96.4000,58.8000) -- (96.4000,58.8000) -- (96.4000,58.8000) -- (96.4000,58.8000) -- (96.4000,58.8000) -- (96.4000,58.8000) -- (96.4000,58.7000) -- (96.4000,58.7000) -- (96.4000,58.7000) -- (96.4000,58.7000) -- (96.4000,58.7000) -- (96.4000,58.7000) -- (96.4000,58.7000) -- (96.4000,58.7000) -- (96.4000,58.7000) -- (96.4000,58.7000) -- (96.4000,58.7000) -- (96.4000,58.7000) -- (96.4000,58.7000) -- (96.4000,58.7000) -- (96.4000,58.7000) -- (96.4000,58.7000) -- (96.4000,58.7000) -- (96.4000,58.7000) -- (96.4000,58.7000) -- (96.4000,58.7000) -- (96.4000,58.7000) -- (96.5000,58.6000) -- (96.5000,58.6000) -- (96.5000,58.6000) -- (96.5000,58.6000) -- (96.5000,58.6000) -- (96.5000,58.6000) -- (96.5000,58.6000) -- (96.5000,58.6000) -- (96.5000,58.6000) -- (96.5000,58.6000) -- (96.5000,58.6000) -- (96.5000,58.6000) -- (96.5000,58.6000) -- (96.5000,58.6000) -- (96.5000,58.6000) -- (96.5000,58.6000) -- (96.5000,58.6000) -- (96.5000,58.6000) -- (96.5000,58.6000) -- (96.5000,58.6000) -- (96.5000,58.6000) -- (96.5000,58.5000) -- (96.5000,58.5000) -- (96.5000,58.5000) -- (96.5000,58.5000) -- (96.5000,58.5000) -- (96.5000,58.5000) -- (96.5000,58.5000) -- (96.5000,58.5000) -- (96.5000,58.5000) -- (96.5000,58.5000) -- (96.5000,58.5000) -- (96.5000,58.5000) -- (96.5000,58.5000) -- (96.5000,58.5000) -- (96.5000,58.5000) -- (96.5000,58.5000) -- (96.5000,58.5000) -- (96.5000,58.5000) -- (96.5000,58.5000) -- (96.5000,58.5000) -- (96.5000,58.5000) -- (96.5000,58.4000) -- (96.5000,58.4000) -- (96.5000,58.4000) -- (96.5000,58.4000) -- (96.5000,58.4000) -- (96.5000,58.4000) -- (96.5000,58.4000) -- (96.5000,58.4000) -- (96.6000,58.4000) -- (96.6000,58.4000) -- (96.6000,58.4000) -- (96.6000,58.4000) -- (96.6000,58.4000) -- (96.6000,58.4000) -- (96.6000,58.4000) -- (96.6000,58.4000) -- (96.6000,58.4000) -- (96.6000,58.4000) -- (96.6000,58.4000) -- (96.6000,58.4000) -- (96.6000,58.3000) -- (96.6000,58.3000) -- (96.6000,58.3000) -- (96.6000,58.3000) -- (96.6000,58.3000) -- (96.6000,58.3000) -- (96.6000,58.3000) -- (96.6000,58.3000) -- (96.6000,58.3000) -- (96.6000,58.3000) -- (96.6000,58.3000) -- (96.6000,58.3000) -- (96.6000,58.3000) -- (96.6000,58.3000) -- (96.6000,58.3000) -- (96.6000,58.3000) -- (96.6000,58.3000) -- (96.6000,58.3000) -- (96.6000,58.3000) -- (96.6000,58.3000) -- (96.6000,58.3000) -- (96.6000,58.2000) -- (96.6000,58.2000) -- (96.6000,58.2000) -- (96.6000,58.2000) -- (96.6000,58.2000) -- (96.6000,58.2000) -- (96.6000,58.2000) -- (96.6000,58.2000) -- (96.6000,58.2000) -- (96.6000,58.2000) -- (96.6000,58.2000) -- (96.6000,58.2000) -- (96.6000,58.2000) -- (96.6000,58.2000) -- (96.6000,58.2000) -- (96.6000,58.2000) -- (96.7000,58.2000) -- (96.7000,58.2000) -- (96.7000,58.2000) -- (96.7000,58.2000) -- (96.7000,58.2000) -- (96.7000,58.1000) -- (96.7000,58.1000) -- (96.7000,58.1000) -- (96.7000,58.1000) -- (96.7000,58.1000) -- (96.7000,58.1000) -- (96.7000,58.1000) -- (96.7000,58.1000) -- (96.7000,58.1000) -- (96.7000,58.1000) -- (96.7000,58.1000) -- (96.7000,58.1000) -- (96.7000,58.1000) -- (96.7000,58.1000) -- (96.7000,58.1000) -- (96.7000,58.1000) -- (96.7000,58.1000) -- (96.7000,58.1000) -- (96.7000,58.1000) -- (96.7000,58.1000) -- (96.7000,58.1000) -- (96.7000,58.0000) -- (96.7000,58.0000) -- (96.7000,58.0000) -- (96.7000,58.0000) -- (96.7000,58.0000) -- (96.7000,58.0000) -- (96.7000,58.0000) -- (96.7000,58.0000) -- (96.7000,58.0000) -- (96.7000,58.0000) -- (96.7000,58.0000) -- (96.7000,58.0000) -- (96.7000,58.0000) -- (96.7000,58.0000) -- (96.7000,58.0000) -- (96.7000,58.0000) -- (96.7000,58.0000) -- (96.7000,58.0000) -- (96.7000,58.0000) -- (96.7000,58.0000) -- (96.7000,58.0000) -- (96.7000,57.9000) -- (96.7000,57.9000) -- (96.7000,57.9000) -- (96.8000,57.9000) -- (96.8000,57.9000) -- (96.8000,57.9000) -- (96.8000,57.9000) -- (96.8000,57.9000) -- (96.8000,57.9000) -- (96.8000,57.9000) -- (96.8000,57.9000) -- (96.8000,57.9000) -- (96.8000,57.9000) -- (96.8000,57.9000) -- (96.8000,57.9000) -- (96.8000,57.9000) -- (96.8000,57.9000) -- (96.8000,57.9000) -- (96.8000,57.9000) -- (96.8000,57.9000) -- (96.8000,57.8000) -- (96.8000,57.8000) -- (96.8000,57.8000) -- (96.8000,57.8000) -- (96.8000,57.8000) -- (96.8000,57.8000) -- (96.8000,57.8000) -- (96.8000,57.8000) -- (96.8000,57.8000) -- (96.8000,57.8000) -- (96.8000,57.8000) -- (96.8000,57.8000) -- (96.8000,57.8000) -- (96.8000,57.8000) -- (96.8000,57.8000) -- (96.8000,57.8000) -- (96.8000,57.8000) -- (96.8000,57.8000) -- (96.8000,57.8000) -- (96.8000,57.8000) -- (96.8000,57.8000) -- (96.8000,57.7000) -- (96.8000,57.7000) -- (96.8000,57.7000) -- (96.8000,57.7000) -- (96.8000,57.7000) -- (96.8000,57.7000) -- (96.8000,57.7000) -- (96.8000,57.7000) -- (96.8000,57.7000) -- (96.8000,57.7000) -- (96.8000,57.7000) -- (96.9000,57.7000) -- (96.9000,57.7000) -- (96.9000,57.7000) -- (96.9000,57.7000) -- (96.9000,57.7000) -- (96.9000,57.7000) -- (96.9000,57.7000) -- (96.9000,57.7000) -- (96.9000,57.7000) -- (96.9000,57.7000) -- (96.9000,57.6000) -- (96.9000,57.6000) -- (96.9000,57.6000) -- (96.9000,57.6000) -- (96.9000,57.6000) -- (96.9000,57.6000) -- (96.9000,57.6000) -- (96.9000,57.6000) -- (96.9000,57.6000) -- (96.9000,57.6000) -- (96.9000,57.6000) -- (96.9000,57.6000) -- (96.9000,57.6000) -- (96.9000,57.6000) -- (96.9000,57.6000) -- (96.9000,57.6000) -- (96.9000,57.6000) -- (96.9000,57.6000) -- (96.9000,57.6000) -- (96.9000,57.6000) -- (96.9000,57.6000) -- (96.9000,57.5000) -- (96.9000,57.5000) -- (96.9000,57.5000) -- (96.9000,57.5000) -- (96.9000,57.5000) -- (96.9000,57.5000) -- (96.9000,57.5000) -- (96.9000,57.5000) -- (96.9000,57.5000) -- (96.9000,57.5000) -- (96.9000,57.5000) -- (96.9000,57.5000) -- (96.9000,57.5000) -- (96.9000,57.5000) -- (96.9000,57.5000) -- (96.9000,57.5000) -- (96.9000,57.5000) -- (96.9000,57.5000) -- (96.9000,57.5000) -- (97.0000,57.5000) -- (97.0000,57.5000) -- (97.0000,57.4000) -- (97.0000,57.4000) -- (97.0000,57.4000) -- (97.0000,57.4000) -- (97.0000,57.4000) -- (97.0000,57.4000) -- (97.0000,57.4000) -- (97.0000,57.4000) -- (97.0000,57.4000) -- (97.0000,57.4000) -- (97.0000,57.4000) -- (97.0000,57.4000) -- (97.0000,57.4000) -- (97.0000,57.4000) -- (97.0000,57.4000) -- (97.0000,57.4000) -- (97.0000,57.4000) -- (97.0000,57.4000) -- (97.0000,57.4000) -- (97.0000,57.4000) -- (97.0000,57.3000) -- (97.0000,57.3000) -- (97.0000,57.3000) -- (97.0000,57.3000) -- (97.0000,57.3000) -- (97.0000,57.3000) -- (97.0000,57.3000) -- (97.0000,57.3000) -- (97.0000,57.3000) -- (97.0000,57.3000) -- (97.0000,57.3000) -- (97.0000,57.3000) -- (97.0000,57.3000) -- (97.0000,57.3000) -- (97.0000,57.3000) -- (97.0000,57.3000) -- (97.0000,57.3000) -- (97.0000,57.3000) -- (97.0000,57.3000) -- (97.0000,57.3000) -- (97.0000,57.3000) -- (97.0000,57.2000) -- (97.0000,57.2000) -- (97.0000,57.2000) -- (97.0000,57.2000) -- (97.0000,57.2000) -- (97.0000,57.2000) -- (97.1000,57.2000) -- (97.1000,57.2000) -- (97.1000,57.2000) -- (97.1000,57.2000) -- (97.1000,57.2000) -- (97.1000,57.2000) -- (97.1000,57.2000) -- (97.1000,57.2000) -- (97.1000,57.2000) -- (97.1000,57.2000) -- (97.1000,57.2000) -- (97.1000,57.2000) -- (97.1000,57.2000) -- (97.1000,57.2000) -- (97.1000,57.2000) -- (97.1000,57.1000) -- (97.1000,57.1000) -- (97.1000,57.1000) -- (97.1000,57.1000) -- (97.1000,57.1000) -- (97.1000,57.1000) -- (97.1000,57.1000) -- (97.1000,57.1000) -- (97.1000,57.1000) -- (97.1000,57.1000) -- (97.1000,57.1000) -- (97.1000,57.1000) -- (97.1000,57.1000) -- (97.1000,57.1000) -- (97.1000,57.1000) -- (97.1000,57.1000) -- (97.1000,57.1000) -- (97.1000,57.1000) -- (97.1000,57.1000) -- (97.1000,57.1000) -- (97.1000,57.1000) -- (97.1000,57.0000) -- (97.1000,57.0000) -- (97.1000,57.0000) -- (97.1000,57.0000) -- (97.1000,57.0000) -- (97.1000,57.0000) -- (97.1000,57.0000) -- (97.1000,57.0000) -- (97.1000,57.0000) -- (97.1000,57.0000) -- (97.1000,57.0000) -- (97.1000,57.0000) -- (97.1000,57.0000) -- (97.1000,57.0000) -- (97.2000,57.0000) -- (97.2000,57.0000) -- (97.2000,57.0000) -- (97.2000,57.0000) -- (97.2000,57.0000) -- (97.2000,57.0000) -- (97.2000,56.9000) -- (97.2000,56.9000) -- (97.2000,56.9000) -- (97.2000,56.9000) -- (97.2000,56.9000) -- (97.2000,56.9000) -- (97.2000,56.9000) -- (97.2000,56.9000) -- (97.2000,56.9000) -- (97.2000,56.9000) -- (97.2000,56.9000) -- (97.2000,56.9000) -- (97.2000,56.9000) -- (97.2000,56.9000) -- (97.2000,56.9000) -- (97.2000,56.9000) -- (97.2000,56.9000) -- (97.2000,56.9000) -- (97.2000,56.9000) -- (97.2000,56.9000) -- (97.2000,56.9000) -- (97.2000,56.8000) -- (97.2000,56.8000) -- (97.2000,56.8000) -- (97.2000,56.8000) -- (97.2000,56.8000) -- (97.2000,56.8000) -- (97.2000,56.8000) -- (97.2000,56.8000) -- (97.2000,56.8000) -- (97.2000,56.8000) -- (97.2000,56.8000) -- (97.2000,56.8000) -- (97.2000,56.8000) -- (97.2000,56.8000) -- (97.2000,56.8000) -- (97.2000,56.8000) -- (97.2000,56.8000) -- (97.2000,56.8000) -- (97.2000,56.8000) -- (97.2000,56.8000) -- (97.2000,56.8000) -- (97.2000,56.7000) -- (97.3000,56.7000) -- (97.3000,56.7000) -- (97.3000,56.7000) -- (97.3000,56.7000) -- (97.3000,56.7000) -- (97.3000,56.7000) -- (97.3000,56.7000) -- (97.3000,56.7000) -- (97.3000,56.7000) -- (97.3000,56.7000) -- (97.3000,56.7000) -- (97.3000,56.7000) -- (97.3000,56.7000) -- (97.3000,56.7000) -- (97.3000,56.7000) -- (97.3000,56.7000) -- (97.3000,56.7000) -- (97.3000,56.7000) -- (97.3000,56.7000) -- (97.3000,56.7000) -- (97.3000,56.6000) -- (97.3000,56.6000) -- (97.3000,56.6000) -- (97.3000,56.6000) -- (97.3000,56.6000) -- (97.3000,56.6000) -- (97.3000,56.6000) -- (97.3000,56.6000) -- (97.3000,56.6000) -- (97.3000,56.6000) -- (97.3000,56.6000) -- (97.3000,56.6000) -- (97.3000,56.6000) -- (97.3000,56.6000) -- (97.3000,56.6000) -- (97.3000,56.6000) -- (97.3000,56.6000) -- (97.3000,56.6000) -- (97.3000,56.6000) -- (97.3000,56.6000) -- (97.3000,56.6000) -- (97.3000,56.5000) -- (97.3000,56.5000) -- (97.3000,56.5000) -- (97.3000,56.5000) -- (97.3000,56.5000) -- (97.3000,56.5000) -- (97.3000,56.5000) -- (97.3000,56.5000) -- (97.3000,56.5000) -- (97.4000,56.5000) -- (97.4000,56.5000) -- (97.4000,56.5000) -- (97.4000,56.5000) -- (97.4000,56.5000) -- (97.4000,56.5000) -- (97.4000,56.5000) -- (97.4000,56.5000) -- (97.4000,56.5000) -- (97.4000,56.5000) -- (97.4000,56.5000) -- (97.4000,56.4000) -- (97.4000,56.4000) -- (97.4000,56.4000) -- (97.4000,56.4000) -- (97.4000,56.4000) -- (97.4000,56.4000) -- (97.4000,56.4000) -- (97.4000,56.4000) -- (97.4000,56.4000) -- (97.4000,56.4000) -- (97.4000,56.4000) -- (97.4000,56.4000) -- (97.4000,56.4000) -- (97.4000,56.4000) -- (97.4000,56.4000) -- (97.4000,56.4000) -- (97.4000,56.4000) -- (97.4000,56.4000) -- (97.4000,56.4000) -- (97.4000,56.4000) -- (97.4000,56.4000) -- (97.4000,56.3000) -- (97.4000,56.3000) -- (97.4000,56.3000) -- (97.4000,56.3000) -- (97.4000,56.3000) -- (97.4000,56.3000) -- (97.4000,56.3000) -- (97.4000,56.3000) -- (97.4000,56.3000) -- (97.4000,56.3000) -- (97.4000,56.3000) -- (97.4000,56.3000) -- (97.4000,56.3000) -- (97.4000,56.3000) -- (97.4000,56.3000) -- (97.4000,56.3000) -- (97.4000,56.3000) -- (97.5000,56.3000) -- (97.5000,56.3000) -- (97.5000,56.3000) -- (97.5000,56.3000) -- (97.5000,56.2000) -- (97.5000,56.2000) -- (97.5000,56.2000) -- (97.5000,56.2000) -- (97.5000,56.2000) -- (97.5000,56.2000) -- (97.5000,56.2000) -- (97.5000,56.2000) -- (97.5000,56.2000) -- (97.5000,56.2000) -- (97.5000,56.2000) -- (97.5000,56.2000) -- (97.5000,56.2000) -- (97.5000,56.2000) -- (97.5000,56.2000) -- (97.5000,56.2000) -- (97.5000,56.2000) -- (97.5000,56.2000) -- (97.5000,56.2000) -- (97.5000,56.2000) -- (97.5000,56.2000) -- (97.5000,56.1000) -- (97.5000,56.1000) -- (97.5000,56.1000) -- (97.5000,56.1000) -- (97.5000,56.1000) -- (97.5000,56.1000) -- (97.5000,56.1000) -- (97.5000,56.1000) -- (97.5000,56.1000) -- (97.5000,56.1000) -- (97.5000,56.1000) -- (97.5000,56.1000) -- (97.5000,56.1000) -- (97.5000,56.1000) -- (97.5000,56.1000) -- (97.5000,56.1000) -- (97.5000,56.1000) -- (97.5000,56.1000) -- (97.5000,56.1000) -- (97.5000,56.1000) -- (97.5000,56.1000) -- (97.5000,56.0000) -- (97.5000,56.0000) -- (97.5000,56.0000) -- (97.5000,56.0000) -- (97.6000,56.0000) -- (97.6000,56.0000) -- (97.6000,56.0000) -- (97.6000,56.0000) -- (97.6000,56.0000) -- (97.6000,56.0000) -- (97.6000,56.0000) -- (97.6000,56.0000) -- (97.6000,56.0000) -- (97.6000,56.0000) -- (97.6000,56.0000) -- (97.6000,56.0000) -- (97.6000,56.0000) -- (97.6000,56.0000) -- (97.6000,56.0000) -- (97.6000,56.0000) -- (97.6000,55.9000) -- (97.6000,55.9000) -- (97.6000,55.9000) -- (97.6000,55.9000) -- (97.6000,55.9000) -- (97.6000,55.9000) -- (97.6000,55.9000) -- (97.6000,55.9000) -- (97.6000,55.9000) -- (97.6000,55.9000) -- (97.6000,55.9000) -- (97.6000,55.9000) -- (97.6000,55.9000) -- (97.6000,55.9000) -- (97.6000,55.9000) -- (97.6000,55.9000) -- (97.6000,55.9000) -- (97.6000,55.9000) -- (97.6000,55.9000) -- (97.6000,55.9000) -- (97.6000,55.9000) -- (97.6000,55.8000) -- (97.6000,55.8000) -- (97.6000,55.8000) -- (97.6000,55.8000) -- (97.6000,55.8000) -- (97.6000,55.8000) -- (97.6000,55.8000) -- (97.6000,55.8000) -- (97.6000,55.8000) -- (97.6000,55.8000) -- (97.6000,55.8000) -- (97.6000,55.8000) -- (97.7000,55.8000) -- (97.7000,55.8000) -- (97.7000,55.8000) -- (97.7000,55.8000) -- (97.7000,55.8000) -- (97.7000,55.8000) -- (97.7000,55.8000) -- (97.7000,55.8000) -- (97.7000,55.8000) -- (97.7000,55.7000) -- (97.7000,55.7000) -- (97.7000,55.7000) -- (97.7000,55.7000) -- (97.7000,55.7000) -- (97.7000,55.7000) -- (97.7000,55.7000) -- (121.4000,55.9000);



      \end{scope}
      \begin{scope}[cm={{1.27491,0.0,0.0,1.41542,(-124.58597,-493.0531)}},draw=blue,line cap=round,line join=round,line width=0.480pt]
        \path[draw] (81.5000,50.5000) -- (81.5000,78.5000) -- (121.5000,78.5000) -- (121.5000,50.5000) -- (81.5000,50.5000);



      \end{scope}
      \begin{scope}[cm={{1.05023,0.0,0.0,1.05023,(-51.68473,-410.86372)}},draw=blue,line cap=rect,line join=bevel,line width=0.800pt]
        \path[fill=blue] (0.0000,0.0000) node[above right] (text64-7-2) {\scriptsize $T$\hspace{.5ex}=\hspace{.5ex}46};



      \end{scope}
      \begin{scope}[cm={{1.05023,0.0,0.0,1.05023,(-3.79554,-425.35157)}},draw=blue,line cap=rect,line join=bevel,line width=0.800pt]
        \path[fill=blue] (0.0000,0.0000) node[above right] (text64-7-8) {\scriptsize 83};



      \end{scope}
    \end{scope}
    \begin{scope}[cm={{0.74279,0.0,0.0,1.28515,(-186.22138,-161.30028)}},draw=blue,line cap=round,line join=round,line width=0.480pt]
      \path[cm={{1.0,0.0,0.0,0.57583,(0.0,64.75889)}},draw] (130.5000,152.5000) -- (130.5000,147.5000);



      \path[cm={{1.0,0.0,0.0,0.70101,(-0.0,38.33083)}},draw] (130.5000,128.5000) -- (130.5000,132.5000);



    \end{scope}
    \begin{scope}[cm={{0.74279,0.0,0.0,1.28515,(-186.22138,-161.30028)}},draw=blue,line cap=round,line join=round,line width=0.480pt]
      \path[draw] (25.5000,128.5000) -- (25.5000,152.5000) -- (142.5000,152.5000) -- (142.5000,128.5000) -- (25.5000,128.5000);



    \end{scope}
    \begin{scope}[cm={{0.95389,0.0,0.0,0.95389,(-108.68182,28.37633)}},draw=blue,line cap=rect,line join=bevel,line width=0.800pt]
      \path[fill=blue] (3.8674,-1.6114) node[above right] (text1264) {\scriptsize $\alpha_3$};



    \end{scope}
    \begin{scope}[cm={{0.74279,0.0,0.0,1.28515,(-184.07401,-166.61499)}},draw=blue,line cap=round,line join=round,line width=0.480pt]
      \path[draw,even odd rule] (123.5000,148.5000) -- (132.5000,148.5000);



    \end{scope}
    \begin{scope}[cm={{0.74279,0.0,0.0,1.28515,(-186.22138,-161.30028)}},draw=blue,line cap=round,line join=round,line width=0.480pt]
      \path[draw] (25.8000,136.6000) -- (25.8000,136.6000) -- (26.2000,137.3000) -- (26.7000,137.7000) -- (27.1000,136.6000) -- (27.5000,137.5000) -- (27.9000,137.0000) -- (28.3000,139.3000) -- (28.8000,132.8000) -- (29.2000,140.7000) -- (29.6000,139.7000) -- (30.0000,133.1000) -- (30.5000,135.4000) -- (30.9000,140.8000) -- (31.3000,140.7000) -- (31.7000,136.2000) -- (32.2000,133.3000) -- (32.6000,134.6000) -- (33.0000,138.2000) -- (33.4000,140.7000) -- (33.8000,140.6000) -- (34.3000,138.4000) -- (34.7000,135.8000) -- (35.1000,134.3000) -- (35.5000,134.4000) -- (36.0000,136.0000) -- (36.4000,138.0000) -- (36.8000,139.6000) -- (37.2000,140.3000) -- (37.6000,140.0000) -- (38.1000,138.8000) -- (38.5000,137.3000) -- (38.9000,135.9000) -- (39.3000,135.0000) -- (39.8000,134.7000) -- (40.2000,135.0000) -- (40.6000,135.8000) -- (41.0000,136.8000) -- (41.5000,137.9000) -- (41.9000,138.8000) -- (42.3000,139.4000) -- (42.7000,139.6000) -- (43.1000,139.4000) -- (43.6000,139.0000) -- (44.0000,138.3000) -- (44.4000,137.4000) -- (44.8000,136.6000) -- (45.3000,136.0000) -- (45.7000,135.5000) -- (46.1000,135.2000) -- (46.5000,135.2000) -- (47.0000,135.5000) -- (47.4000,135.9000) -- (47.8000,136.4000) -- (48.2000,137.0000) -- (48.6000,137.6000) -- (49.1000,138.1000) -- (49.5000,138.5000) -- (49.9000,138.8000) -- (50.3000,138.9000) -- (50.8000,138.8000) -- (51.2000,138.5000) -- (51.6000,138.1000) -- (52.0000,137.6000) -- (52.4000,137.1000) -- (52.9000,136.5000) -- (53.3000,136.1000) -- (53.7000,135.7000) -- (54.1000,135.6000) -- (54.6000,135.6000) -- (55.0000,135.8000) -- (55.4000,136.1000) -- (55.8000,136.6000) -- (56.3000,137.1000) -- (56.7000,137.6000) -- (57.1000,138.0000) -- (57.5000,138.3000) -- (57.9000,138.6000) -- (58.4000,138.7000) -- (58.8000,138.6000) -- (59.2000,138.4000) -- (59.6000,138.1000) -- (60.1000,137.8000) -- (60.5000,137.3000) -- (60.9000,136.9000) -- (61.3000,136.6000) -- (61.8000,136.4000) -- (62.2000,136.3000) -- (62.6000,136.4000) -- (63.0000,136.5000) -- (63.4000,136.7000) -- (63.9000,137.0000) -- (64.3000,137.4000) -- (64.7000,137.7000) -- (65.1000,138.0000) -- (65.6000,138.2000) -- (66.0000,138.3000) -- (66.4000,138.2000) -- (66.8000,138.1000) -- (67.3000,137.9000) -- (67.7000,137.7000) -- (68.1000,137.4000) -- (68.5000,137.1000) -- (68.9000,136.8000) -- (69.4000,136.6000) -- (69.8000,136.5000) -- (70.2000,136.5000) -- (70.6000,136.6000) -- (71.1000,136.7000) -- (71.5000,136.9000) -- (71.9000,137.0000) -- (72.3000,137.2000) -- (72.7000,137.4000) -- (73.2000,137.6000) -- (73.6000,137.8000) -- (74.0000,137.9000) -- (74.4000,138.0000) -- (74.9000,138.0000) -- (75.3000,137.9000) -- (75.7000,137.8000) -- (76.1000,137.7000) -- (76.6000,137.5000) -- (77.0000,137.3000) -- (77.4000,137.2000) -- (77.8000,137.1000) -- (78.2000,136.9000) -- (78.7000,136.8000) -- (79.1000,136.8000) -- (79.5000,136.8000) -- (79.9000,136.8000) -- (80.4000,136.8000) -- (80.8000,136.9000) -- (81.2000,137.0000) -- (81.6000,137.1000) -- (82.1000,137.3000) -- (82.5000,137.4000) -- (82.9000,137.5000) -- (83.3000,137.6000) -- (83.7000,137.6000) -- (84.2000,137.6000) -- (84.6000,137.6000) -- (85.0000,137.6000) -- (85.4000,137.5000) -- (85.9000,137.4000) -- (86.3000,137.3000) -- (86.7000,137.2000) -- (87.1000,137.1000) -- (87.6000,137.0000) -- (88.0000,136.9000) -- (88.4000,136.9000) -- (88.8000,137.0000) -- (89.2000,137.0000) -- (89.7000,137.1000) -- (90.1000,137.2000) -- (90.5000,137.3000) -- (90.9000,137.4000) -- (91.4000,137.5000) -- (91.8000,137.5000) -- (92.2000,137.6000) -- (92.6000,137.6000) -- (93.0000,137.6000) -- (93.5000,137.5000) -- (93.9000,137.5000) -- (94.3000,137.4000) -- (94.7000,137.3000) -- (95.2000,137.2000) -- (95.6000,137.1000) -- (96.0000,137.1000) -- (96.4000,137.0000) -- (96.9000,137.1000) -- (97.3000,137.1000) -- (97.7000,137.1000) -- (98.1000,137.2000) -- (98.5000,137.3000) -- (99.0000,137.3000) -- (99.4000,137.4000) -- (99.8000,137.4000) -- (100.2000,137.5000) -- (100.7000,137.5000) -- (101.1000,137.6000) -- (101.5000,137.5000) -- (101.9000,137.4000) -- (102.3000,137.3000) -- (102.8000,137.3000) -- (103.2000,137.2000) -- (103.6000,137.1000) -- (104.0000,137.1000) -- (104.5000,137.0000) -- (104.9000,137.0000) -- (105.3000,137.0000) -- (105.7000,137.1000) -- (106.2000,137.1000) -- (106.6000,137.2000) -- (107.0000,137.3000) -- (107.4000,137.3000) -- (107.8000,137.4000) -- (108.3000,137.5000) -- (108.7000,137.6000) -- (109.1000,137.6000) -- (109.5000,137.7000) -- (110.0000,137.6000) -- (110.4000,137.6000) -- (110.8000,137.5000) -- (111.2000,137.4000) -- (111.7000,137.3000) -- (112.1000,137.2000) -- (112.5000,137.1000) -- (112.9000,137.0000) -- (113.3000,137.0000) -- (113.8000,136.9000) -- (114.2000,136.9000) -- (114.6000,136.9000) -- (115.0000,136.9000) -- (115.5000,137.0000) -- (115.9000,137.2000) -- (116.3000,137.3000) -- (116.7000,137.4000) -- (117.2000,137.5000) -- (117.6000,137.6000) -- (118.0000,137.7000) -- (118.4000,137.7000) -- (118.8000,137.7000) -- (119.3000,137.7000) -- (119.7000,137.6000) -- (120.1000,137.5000) -- (120.5000,137.4000) -- (121.0000,137.2000) -- (121.4000,137.1000) -- (121.8000,137.0000) -- (122.2000,136.8000) -- (122.6000,136.8000) -- (123.1000,136.8000) -- (123.5000,136.8000) -- (123.9000,136.9000) -- (124.3000,137.0000) -- (124.8000,137.2000) -- (125.2000,137.3000) -- (125.6000,137.4000) -- (126.0000,137.6000) -- (126.5000,137.7000) -- (126.9000,137.7000) -- (127.3000,137.8000) -- (127.7000,137.8000) -- (128.1000,137.8000) -- (128.6000,137.7000) -- (129.0000,137.5000) -- (129.4000,137.4000) -- (129.8000,137.3000) -- (130.3000,137.1000) -- (130.7000,137.0000) -- (131.1000,136.9000) -- (131.5000,136.9000) -- (131.9000,136.8000) -- (132.4000,136.8000) -- (132.8000,136.9000) -- (133.2000,137.0000) -- (133.6000,137.1000) -- (134.1000,137.2000) -- (134.5000,137.3000) -- (134.9000,137.4000) -- (135.3000,137.6000) -- (135.8000,137.7000) -- (136.2000,137.8000) -- (136.6000,137.8000) -- (137.0000,137.8000) -- (137.4000,137.7000) -- (137.9000,137.6000) -- (138.3000,137.5000) -- (138.7000,137.4000) -- (139.1000,137.3000) -- (139.6000,137.1000) -- (140.0000,137.0000) -- (140.4000,136.9000) -- (140.8000,136.8000) -- (141.3000,136.8000) -- (141.7000,136.8000) -- (142.1000,136.9000) -- (142.3000,136.9000);



    \end{scope}
    \begin{scope}[cm={{0.7438,0.0,0.0,0.77563,(-64.025,-157.04212)}},draw=blue,line cap=round,line join=round,line width=0.480pt]
    \end{scope}
    \begin{scope}[cm={{0.74279,0.0,0.0,1.28515,(-186.22138,-161.30028)}},draw=ca0a0a4,dash pattern=on 1.22pt off 1.22pt,line cap=round,line join=round,line width=0.305pt,miter limit=4.00]
      \path[draw,dash pattern=on 1.22pt off 1.22pt,line width=0.305pt,miter limit=4.00] (25.5000,168.5000) -- (108.5000,168.5000);



      \path[draw,dash pattern=on 1.22pt off 1.22pt,line width=0.305pt,miter limit=4.00] (137.5000,168.5000) -- (142.5000,168.5000);



    \end{scope}
    \begin{scope}[cm={{0.74279,0.0,0.0,1.28515,(-186.22138,-161.30028)}},draw=blue,line cap=round,line join=round,line width=0.480pt]
      \path[cm={{1.54975,0.0,0.0,1.0,(-13.85377,0.0)}},draw] (25.5000,168.5000) -- (28.5000,168.5000);



      \path[cm={{1.54975,0.0,0.0,1.0,(-78.53317,0.0)}},draw] (142.5000,168.5000) -- (139.5000,168.5000);



    \end{scope}
    \begin{scope}[cm={{0.74279,0.0,0.0,1.28515,(-186.22138,-161.30028)}},draw=ca0a0a4,dash pattern=on 1.22pt off 1.22pt,line cap=round,line join=round,line width=0.305pt,miter limit=4.00]
      \path[draw,dash pattern=on 1.22pt off 1.22pt,line width=0.305pt,miter limit=4.00] (25.5000,153.5000) -- (142.5000,153.5000);



    \end{scope}
    \begin{scope}[cm={{0.74279,0.0,0.0,1.28515,(-186.22138,-161.30028)}},draw=blue,line cap=round,line join=round,line width=0.480pt]
      \path[cm={{1.54975,0.0,0.0,1.0,(-13.85377,0.0)}},draw] (25.5000,153.5000) -- (28.5000,153.5000);



      \path[cm={{1.54975,0.0,0.0,1.0,(-78.53317,0.0)}},draw] (142.5000,153.5000) -- (139.5000,153.5000);



    \end{scope}
    \begin{scope}[cm={{0.95389,0.0,0.0,0.95389,(-180.91222,39.27539)}},draw=blue,fill=ce10000,line cap=rect,line join=bevel,line width=0.800pt]
      \path[fill=ce10000] (0.0000,0.0000) node[above right] (text1350) {\scriptsize 20};



    \end{scope}
    \begin{scope}[cm={{0.74279,0.0,0.0,1.28515,(-186.22138,-161.30028)}},draw=ca0a0a4,dash pattern=on 0.40pt off 0.80pt,line cap=round,line join=round,line width=0.400pt]
      \path[draw] (25.5000,176.5000) -- (25.5000,152.5000);



    \end{scope}
    \begin{scope}[cm={{0.74279,0.0,0.0,1.28515,(-186.22138,-161.30028)}},draw=blue,line cap=round,line join=round,line width=0.480pt]
      \path[draw] (25.5000,176.5000) -- (25.5000,171.5000);



      \path[draw] (25.5000,152.5000) -- (25.5000,156.5000);



    \end{scope}
    \begin{scope}[cm={{0.74279,0.0,0.0,1.28515,(-186.22138,-161.30028)}},draw=ca0a0a4,dash pattern=on 1.22pt off 1.22pt,line cap=round,line join=round,line width=0.305pt,miter limit=4.00]
      \path[draw,dash pattern=on 1.22pt off 1.22pt,line width=0.305pt,miter limit=4.00] (60.5000,176.5000) -- (60.5000,152.5000);



    \end{scope}
    \begin{scope}[cm={{0.74279,0.0,0.0,1.28515,(-186.22138,-161.30028)}},draw=blue,line cap=round,line join=round,line width=0.480pt]
      \path[cm={{1.0,0.0,0.0,0.57583,(0.0,74.93902)}},draw] (60.5000,176.5000) -- (60.5000,171.5000);



      \path[cm={{1.0,0.0,0.0,0.70101,(0.0,45.50665)}},draw] (60.5000,152.5000) -- (60.5000,156.5000);



    \end{scope}
    \begin{scope}[cm={{0.74279,0.0,0.0,1.28515,(-186.22138,-161.30028)}},draw=ca0a0a4,dash pattern=on 1.22pt off 1.22pt,line cap=round,line join=round,line width=0.305pt,miter limit=4.00]
      \path[draw,dash pattern=on 1.22pt off 1.22pt,line width=0.305pt,miter limit=4.00] (95.5000,176.5000) -- (95.5000,152.5000);



    \end{scope}
    \begin{scope}[cm={{0.74279,0.0,0.0,1.28515,(-186.22138,-161.30028)}},draw=blue,line cap=round,line join=round,line width=0.480pt]
      \path[cm={{1.0,0.0,0.0,0.57583,(0.0,74.93902)}},draw] (95.5000,176.5000) -- (95.5000,171.5000);



      \path[cm={{1.0,0.0,0.0,0.70101,(0.0,45.50665)}},draw] (95.5000,152.5000) -- (95.5000,156.5000);



    \end{scope}
    \begin{scope}[cm={{0.74279,0.0,0.0,1.28515,(-186.22138,-161.30028)}},draw=ca0a0a4,dash pattern=on 1.22pt off 1.22pt,line cap=round,line join=round,line width=0.305pt,miter limit=4.00]
      \path[draw,dash pattern=on 1.22pt off 1.22pt,line width=0.305pt,miter limit=4.00] (130.5000,164.5000) -- (130.5000,152.5000);



    \end{scope}
    \begin{scope}[cm={{0.74279,0.0,0.0,1.28515,(-186.22138,-161.30028)}},draw=blue,line cap=round,line join=round,line width=0.480pt]
      \path[cm={{1.0,0.0,0.0,0.57583,(0.0,74.93902)}},draw] (130.5000,176.5000) -- (130.5000,171.5000);



      \path[cm={{1.0,0.0,0.0,0.70101,(0.0,45.50665)}},draw] (130.5000,152.5000) -- (130.5000,156.5000);



    \end{scope}
    \begin{scope}[cm={{0.74279,0.0,0.0,1.28515,(-186.22138,-161.30028)}},draw=blue,line cap=round,line join=round,line width=0.480pt]
      \path[draw] (25.5000,152.5000) -- (25.5000,176.5000) -- (142.5000,176.5000) -- (142.5000,152.5000) -- (25.5000,152.5000);



    \end{scope}
    \begin{scope}[cm={{0.95389,0.0,0.0,0.95389,(-104.73329,57.56154)}},draw=blue,line cap=rect,line join=bevel,line width=0.800pt]
      \path[fill=blue] (0.0000,0.0000) node[above right] (text1494) {\scriptsize $\beta_3$};



    \end{scope}
    \begin{scope}[cm={{0.74279,0.0,0.0,1.28515,(-184.07401,-166.61499)}},draw=blue,line cap=round,line join=round,line width=0.480pt]
      \path[draw,even odd rule] (123.5000,172.5000) -- (132.5000,172.5000);



    \end{scope}
    \begin{scope}[cm={{0.74279,0.0,0.0,1.28515,(-186.22138,-161.30028)}},draw=blue,line cap=round,line join=round,line width=0.480pt]
      \path[draw] (25.8000,162.1000) -- (25.8000,162.1000) -- (26.2000,161.2000) -- (26.7000,161.1000) -- (27.1000,162.2000) -- (27.5000,159.4000) -- (27.9000,164.7000) -- (28.3000,157.2000) -- (28.8000,163.3000) -- (29.2000,164.4000) -- (29.6000,156.6000) -- (30.0000,159.9000) -- (30.5000,165.6000) -- (30.9000,163.5000) -- (31.3000,158.2000) -- (31.7000,157.3000) -- (32.2000,161.0000) -- (32.6000,164.7000) -- (33.0000,165.0000) -- (33.4000,162.4000) -- (33.8000,159.2000) -- (34.3000,157.7000) -- (34.7000,158.4000) -- (35.1000,160.6000) -- (35.5000,162.9000) -- (36.0000,164.3000) -- (36.4000,164.2000) -- (36.8000,163.0000) -- (37.2000,161.2000) -- (37.6000,159.5000) -- (38.1000,158.5000) -- (38.5000,158.3000) -- (38.9000,159.0000) -- (39.3000,160.2000) -- (39.8000,161.6000) -- (40.2000,162.8000) -- (40.6000,163.6000) -- (41.0000,163.9000) -- (41.5000,163.6000) -- (41.9000,163.0000) -- (42.3000,162.1000) -- (42.7000,161.1000) -- (43.1000,160.2000) -- (43.6000,159.5000) -- (44.0000,159.1000) -- (44.4000,159.0000) -- (44.8000,159.2000) -- (45.3000,159.7000) -- (45.7000,160.4000) -- (46.1000,161.1000) -- (46.5000,161.8000) -- (47.0000,162.5000) -- (47.4000,163.0000) -- (47.8000,163.2000) -- (48.2000,163.3000) -- (48.6000,163.1000) -- (49.1000,162.8000) -- (49.5000,162.4000) -- (49.9000,161.8000) -- (50.3000,161.2000) -- (50.8000,160.7000) -- (51.2000,160.2000) -- (51.6000,159.9000) -- (52.0000,159.7000) -- (52.4000,159.7000) -- (52.9000,159.9000) -- (53.3000,160.3000) -- (53.7000,160.8000) -- (54.1000,161.3000) -- (54.6000,161.9000) -- (55.0000,162.4000) -- (55.4000,162.7000) -- (55.8000,163.0000) -- (56.3000,163.0000) -- (56.7000,162.8000) -- (57.1000,162.6000) -- (57.5000,162.2000) -- (57.9000,161.7000) -- (58.4000,161.2000) -- (58.8000,160.8000) -- (59.2000,160.4000) -- (59.6000,160.1000) -- (60.1000,160.0000) -- (60.5000,160.0000) -- (60.9000,160.1000) -- (61.3000,160.4000) -- (61.8000,160.7000) -- (62.2000,161.1000) -- (62.6000,161.5000) -- (63.0000,161.8000) -- (63.4000,162.1000) -- (63.9000,162.2000) -- (64.3000,162.2000) -- (64.7000,162.1000) -- (65.1000,161.9000) -- (65.6000,161.6000) -- (66.0000,161.3000) -- (66.4000,161.0000) -- (66.8000,160.7000) -- (67.3000,160.5000) -- (67.7000,160.4000) -- (68.1000,160.3000) -- (68.5000,160.4000) -- (68.9000,160.6000) -- (69.4000,160.8000) -- (69.8000,161.1000) -- (70.2000,161.4000) -- (70.6000,161.6000) -- (71.1000,161.8000) -- (71.5000,162.0000) -- (71.9000,162.0000) -- (72.3000,162.0000) -- (72.7000,162.0000) -- (73.2000,161.9000) -- (73.6000,161.8000) -- (74.0000,161.6000) -- (74.4000,161.5000) -- (74.9000,161.2000) -- (75.3000,161.0000) -- (75.7000,160.8000) -- (76.1000,160.7000) -- (76.6000,160.7000) -- (77.0000,160.7000) -- (77.4000,160.7000) -- (77.8000,160.8000) -- (78.2000,160.9000) -- (78.7000,161.0000) -- (79.1000,161.2000) -- (79.5000,161.3000) -- (79.9000,161.5000) -- (80.4000,161.6000) -- (80.8000,161.7000) -- (81.2000,161.8000) -- (81.6000,161.8000) -- (82.1000,161.8000) -- (82.5000,161.8000) -- (82.9000,161.7000) -- (83.3000,161.6000) -- (83.7000,161.5000) -- (84.2000,161.3000) -- (84.6000,161.2000) -- (85.0000,161.1000) -- (85.4000,161.0000) -- (85.9000,161.0000) -- (86.3000,161.0000) -- (86.7000,161.0000) -- (87.1000,161.0000) -- (87.6000,161.1000) -- (88.0000,161.1000) -- (88.4000,161.3000) -- (88.8000,161.4000) -- (89.2000,161.5000) -- (89.7000,161.6000) -- (90.1000,161.6000) -- (90.5000,161.6000) -- (90.9000,161.6000) -- (91.4000,161.6000) -- (91.8000,161.5000) -- (92.2000,161.4000) -- (92.6000,161.3000) -- (93.0000,161.2000) -- (93.5000,161.1000) -- (93.9000,161.1000) -- (94.3000,161.0000) -- (94.7000,161.0000) -- (95.2000,161.0000) -- (95.6000,161.0000) -- (96.0000,161.1000) -- (96.4000,161.2000) -- (96.9000,161.3000) -- (97.3000,161.4000) -- (97.7000,161.5000) -- (98.1000,161.5000) -- (98.5000,161.5000) -- (99.0000,161.5000) -- (99.4000,161.5000) -- (99.8000,161.4000) -- (100.2000,161.4000) -- (100.7000,161.3000) -- (101.1000,161.3000) -- (101.5000,161.2000) -- (101.9000,161.1000) -- (102.3000,161.1000) -- (102.8000,161.0000) -- (103.2000,161.1000) -- (103.6000,161.1000) -- (104.0000,161.2000) -- (104.5000,161.2000) -- (104.9000,161.3000) -- (105.3000,161.4000) -- (105.7000,161.4000) -- (106.2000,161.5000) -- (106.6000,161.5000) -- (107.0000,161.6000) -- (107.4000,161.6000) -- (107.8000,161.5000) -- (108.3000,161.5000) -- (108.7000,161.4000) -- (109.1000,161.4000) -- (109.5000,161.3000) -- (110.0000,161.2000) -- (110.4000,161.1000) -- (110.8000,161.0000) -- (111.2000,161.0000) -- (111.7000,160.9000) -- (112.1000,160.9000) -- (112.5000,161.0000) -- (112.9000,161.0000) -- (113.3000,161.1000) -- (113.8000,161.2000) -- (114.2000,161.3000) -- (114.6000,161.4000) -- (115.0000,161.5000) -- (115.5000,161.6000) -- (115.9000,161.7000) -- (116.3000,161.7000) -- (116.7000,161.7000) -- (117.2000,161.6000) -- (117.6000,161.6000) -- (118.0000,161.5000) -- (118.4000,161.3000) -- (118.8000,161.2000) -- (119.3000,161.1000) -- (119.7000,161.0000) -- (120.1000,160.9000) -- (120.5000,160.8000) -- (121.0000,160.8000) -- (121.4000,160.9000) -- (121.8000,160.9000) -- (122.2000,161.0000) -- (122.6000,161.2000) -- (123.1000,161.3000) -- (123.5000,161.5000) -- (123.9000,161.6000) -- (124.3000,161.7000) -- (124.8000,161.8000) -- (125.2000,161.8000) -- (125.6000,161.8000) -- (126.0000,161.7000) -- (126.5000,161.6000) -- (126.9000,161.5000) -- (127.3000,161.4000) -- (127.7000,161.3000) -- (128.1000,161.1000) -- (128.6000,160.9000) -- (129.0000,160.8000) -- (129.4000,160.8000) -- (129.8000,160.8000) -- (130.3000,160.8000) -- (130.7000,160.9000) -- (131.1000,161.0000) -- (131.5000,161.1000) -- (131.9000,161.3000) -- (132.4000,161.4000) -- (132.8000,161.5000) -- (133.2000,161.6000) -- (133.6000,161.7000) -- (134.1000,161.7000) -- (134.5000,161.8000) -- (134.9000,161.7000) -- (135.3000,161.7000) -- (135.8000,161.6000) -- (136.2000,161.5000) -- (136.6000,161.3000) -- (137.0000,161.2000) -- (137.4000,161.0000) -- (137.9000,160.9000) -- (138.3000,160.9000) -- (138.7000,160.8000) -- (139.1000,160.8000) -- (139.6000,160.9000) -- (140.0000,160.9000) -- (140.4000,161.0000) -- (140.8000,161.1000) -- (141.3000,161.3000) -- (141.7000,161.4000) -- (142.1000,161.6000) -- (142.3000,161.6000);



    \end{scope}
    \path[draw=blue,line cap=butt,line join=miter,line width=0.747pt] (-429.4395,-113.5288) -- cycle;



    \begin{scope}[cm={{0.99667,0.0,0.0,1.34693,(-320.19845,-156.47287)}},draw=ca0a0a4,dash pattern=on 1.54pt off 1.54pt,line cap=round,line join=round,line width=0.257pt,miter limit=4.00]
      \path[draw,dash pattern=on 1.54pt off 1.54pt,line width=0.257pt,miter limit=4.00] (41.5000,153.5000) -- (127.5000,153.5000);



    \end{scope}
    \begin{scope}[cm={{0.99667,0.0,0.0,1.34693,(-320.19845,-156.47287)}},draw=cd9d9d9,line cap=rect,line join=miter,line width=2.113pt,miter limit=10.00]
      \path[draw=cffffff,line cap=rect,line join=miter,line width=2.113pt,miter limit=10.00] (41.5000,88.5000) -- (41.5000,164.5000) -- (127.5000,164.5000) -- (127.5000,88.5000) -- (41.5000,88.5000);



    \end{scope}
    \begin{scope}[cm={{0.99667,0.0,0.0,1.34693,(-320.19845,-156.47287)}},draw=blue,line cap=round,line join=round,line width=0.480pt]
      \path[cm={{1.155,0.0,0.0,1.0,(-6.38582,0.0)}},draw] (41.5000,153.5000) -- (44.5000,153.5000);



      \path[cm={{1.155,0.0,0.0,1.0,(-20.06101,0.0)}},draw] (127.5000,153.5000) -- (124.5000,153.5000);



    \end{scope}
    \begin{scope}[cm={{1.0018,0.0,0.0,1.0018,(-293.12869,52.89918)}},draw=blue,line cap=rect,line join=bevel,line width=0.800pt]
      \path[fill=blue] (0.0000,0.0000) node[above right] (text366) {\scriptsize 30};



    \end{scope}
    \begin{scope}[cm={{0.99667,0.0,0.0,1.34693,(-320.19845,-156.47287)}},draw=ca0a0a4,dash pattern=on 1.54pt off 1.54pt,line cap=round,line join=round,line width=0.257pt,miter limit=4.00]
      \path[draw,dash pattern=on 1.54pt off 1.54pt,line width=0.257pt,miter limit=4.00] (41.5000,127.5000) -- (127.5000,127.5000);



    \end{scope}
    \begin{scope}[cm={{0.99667,0.0,0.0,1.34693,(-320.19845,-156.47287)}},draw=blue,line cap=round,line join=round,line width=0.480pt]
      \path[cm={{1.155,0.0,0.0,1.0,(-6.38582,0.0)}},draw] (41.5000,127.5000) -- (44.5000,127.5000);



      \path[cm={{1.155,0.0,0.0,1.0,(-20.06101,0.0)}},draw] (127.5000,127.5000) -- (124.5000,127.5000);



    \end{scope}
    \begin{scope}[cm={{1.0018,0.0,0.0,1.0018,(-292.76805,19.17029)}},draw=blue,line cap=rect,line join=bevel,line width=0.800pt]
      \path[fill=blue] (0.0000,0.0000) node[above right] (text396) {\scriptsize 33};



    \end{scope}
    \begin{scope}[cm={{0.99667,0.0,0.0,1.34693,(-320.19845,-156.47287)}},draw=blue,line cap=round,line join=round,line width=0.480pt]
      \path[cm={{1.155,0.0,0.0,1.0,(-6.38582,0.0)}},draw] (41.5000,101.5000) -- (44.5000,101.5000);



      \path[cm={{1.155,0.0,0.0,1.0,(-20.06101,0.0)}},draw] (127.5000,101.5000) -- (124.5000,101.5000);



    \end{scope}
    \begin{scope}[cm={{1.0018,0.0,0.0,1.0018,(-293.06458,-16.50733)}},draw=blue,line cap=rect,line join=bevel,line width=0.800pt]
      \path[fill=blue] (0.0000,0.0000) node[above right] (text426) {\scriptsize 36};



    \end{scope}
    \begin{scope}[cm={{0.99667,0.0,0.0,1.34693,(-320.19845,-156.47287)}},draw=ca0a0a4,dash pattern=on 0.40pt off 0.80pt,line cap=round,line join=round,line width=0.400pt]
      \path[draw] (41.5000,164.5000) -- (41.5000,88.5000);



    \end{scope}
    \begin{scope}[cm={{0.99667,0.0,0.0,1.34693,(-320.19845,-156.47287)}},draw=blue,line cap=round,line join=round,line width=0.480pt]
      \path[draw] (41.5000,164.5000) -- (41.5000,159.5000);



      \path[draw] (41.5000,88.5000) -- (41.5000,92.5000);



    \end{scope}
    \begin{scope}[cm={{1.00174,0.0,0.0,1.01515,(-281.55885,76.66745)}},draw=blue,line cap=rect,line join=bevel,line width=0.800pt]
      \path[fill=blue] (0.0000,0.0000) node[above right] (text456) {\scriptsize 0};



    \end{scope}
    \begin{scope}[cm={{0.99667,0.0,0.0,1.34693,(-320.19845,-156.47287)}},draw=blue,line cap=round,line join=round,line width=0.480pt]
      \path[cm={{1.0,0.0,0.0,0.54942,(0.0,73.94736)}},draw] (70.5000,164.5000) -- (70.5000,159.5000);



      \path[cm={{1.0,0.0,0.0,0.66885,(0.0,29.20714)}},draw] (70.5000,88.5000) -- (70.5000,92.5000);



    \end{scope}
    \begin{scope}[cm={{1.00174,0.0,0.0,1.01515,(-253.51015,76.66745)}},draw=blue,line cap=rect,line join=bevel,line width=0.800pt]
      \path[fill=blue] (0.0000,0.0000) node[above right] (text486) {\scriptsize 2};



    \end{scope}
    \begin{scope}[cm={{0.99667,0.0,0.0,1.34693,(-320.19845,-156.47287)}},draw=blue,line cap=round,line join=round,line width=0.480pt]
      \path[cm={{1.0,0.0,0.0,0.54942,(0.0,73.94736)}},draw] (98.5000,164.5000) -- (98.5000,159.5000);



      \path[cm={{1.0,0.0,0.0,0.66885,(0.0,29.20714)}},draw] (98.5000,88.5000) -- (98.5000,92.5000);



    \end{scope}
    \begin{scope}[cm={{1.00174,0.0,0.0,1.01515,(-223.95891,76.66745)}},draw=blue,line cap=rect,line join=bevel,line width=0.800pt]
      \path[fill=blue] (0.0000,0.0000) node[above right] (text516) {\scriptsize 4};



    \end{scope}
    \begin{scope}[cm={{0.99667,0.0,0.0,1.34693,(-320.19845,-156.47287)}},draw=ca0a0a4,dash pattern=on 0.40pt off 0.80pt,line cap=round,line join=round,line width=0.400pt]
      \path[draw] (127.5000,164.5000) -- (127.5000,88.5000);



    \end{scope}
    \begin{scope}[cm={{0.99667,0.0,0.0,1.34693,(-320.19845,-156.47287)}},draw=blue,line cap=round,line join=round,line width=0.480pt]
      \path[draw] (127.5000,164.5000) -- (127.5000,159.5000);



      \path[draw] (127.5000,88.5000) -- (127.5000,92.5000);



    \end{scope}
    \begin{scope}[cm={{1.00174,0.0,0.0,1.01515,(-196.411,76.73242)}},draw=blue,line cap=rect,line join=bevel,line width=0.800pt]
      \path[fill=blue] (0.0000,0.0000) node[above right] (text546) {\scriptsize 6};



    \end{scope}
    \begin{scope}[cm={{0.99667,0.0,0.0,1.34693,(-320.19845,-156.47287)}},draw=blue,line cap=round,line join=round,line width=0.480pt]
      \path[draw] (41.6000,103.3000) -- (41.6000,103.3000) -- (41.7000,103.6000) -- (41.9000,104.0000) -- (42.0000,104.3000) -- (42.2000,104.6000) -- (42.3000,104.9000) -- (42.5000,105.1000) -- (42.6000,105.4000) -- (42.7000,105.7000) -- (42.9000,106.0000) -- (43.0000,106.3000) -- (43.2000,106.5000) -- (43.3000,106.8000) -- (43.5000,107.1000) -- (43.6000,107.3000) -- (43.7000,107.6000) -- (43.9000,107.9000) -- (44.0000,108.1000) -- (44.2000,108.4000) -- (44.3000,108.6000) -- (44.5000,108.8000) -- (44.6000,109.1000) -- (44.7000,109.3000) -- (44.9000,109.5000) -- (45.0000,109.8000) -- (45.2000,110.0000) -- (45.3000,110.2000) -- (45.5000,110.4000) -- (45.6000,110.7000) -- (45.7000,110.9000) -- (45.9000,111.1000) -- (46.0000,111.3000) -- (46.2000,111.5000) -- (46.3000,111.7000) -- (46.5000,111.9000) -- (46.6000,112.1000) -- (46.7000,112.3000) -- (46.9000,112.4000) -- (47.0000,112.6000) -- (47.2000,112.8000) -- (47.3000,113.0000) -- (47.5000,113.2000) -- (47.6000,113.3000) -- (47.7000,113.5000) -- (47.9000,113.7000) -- (48.0000,113.9000) -- (48.2000,114.0000) -- (48.3000,114.2000) -- (48.5000,114.3000) -- (48.6000,114.5000) -- (48.7000,114.7000) -- (48.9000,114.8000) -- (49.0000,115.0000) -- (49.2000,115.1000) -- (49.3000,115.2000) -- (49.5000,115.4000) -- (49.6000,115.5000) -- (49.7000,115.7000) -- (49.9000,115.8000) -- (50.0000,115.9000) -- (50.2000,116.1000) -- (50.3000,116.2000) -- (50.5000,116.3000) -- (50.6000,116.4000) -- (50.7000,116.6000) -- (50.9000,116.7000) -- (51.0000,116.8000) -- (51.2000,116.9000) -- (51.3000,117.0000) -- (51.5000,117.1000) -- (51.6000,117.2000) -- (51.7000,117.4000) -- (51.9000,117.5000) -- (52.0000,117.6000) -- (52.2000,117.7000) -- (52.3000,117.8000) -- (52.5000,117.9000) -- (52.6000,118.0000) -- (52.7000,118.0000) -- (52.9000,118.1000) -- (53.0000,118.2000) -- (53.2000,118.3000) -- (53.3000,118.4000) -- (53.5000,118.5000) -- (53.6000,118.6000) -- (53.7000,118.7000) -- (53.9000,118.7000) -- (54.0000,118.8000) -- (54.2000,118.9000) -- (54.3000,119.0000) -- (54.5000,119.0000) -- (54.6000,119.1000) -- (54.7000,119.2000) -- (54.9000,119.2000) -- (55.0000,119.3000) -- (55.2000,119.4000) -- (55.3000,119.4000) -- (55.5000,119.5000) -- (55.6000,119.6000) -- (55.7000,119.6000) -- (55.9000,119.7000) -- (56.0000,119.7000) -- (56.2000,119.8000) -- (56.3000,119.8000) -- (56.5000,119.9000) -- (56.6000,119.9000) -- (56.7000,120.0000) -- (56.9000,120.0000) -- (57.0000,120.1000) -- (57.2000,120.1000) -- (57.3000,120.2000) -- (57.5000,120.2000) -- (57.6000,120.2000) -- (57.7000,120.3000) -- (57.9000,120.3000) -- (58.0000,120.4000) -- (58.2000,120.4000) -- (58.3000,120.4000) -- (58.5000,120.5000) -- (58.6000,120.5000) -- (58.7000,120.5000) -- (58.9000,120.6000) -- (59.0000,120.6000) -- (59.2000,120.6000) -- (59.3000,120.6000) -- (59.5000,120.7000) -- (59.6000,120.7000) -- (59.7000,120.7000) -- (59.9000,120.7000) -- (60.0000,120.8000) -- (60.2000,120.8000) -- (60.3000,120.8000) -- (60.5000,120.8000) -- (60.6000,120.8000) -- (60.7000,120.9000) -- (60.9000,120.9000) -- (61.0000,120.9000) -- (61.2000,120.9000) -- (61.3000,120.9000) -- (61.5000,120.9000) -- (61.6000,120.9000) -- (61.7000,121.0000) -- (61.9000,121.0000) -- (62.0000,121.0000) -- (62.2000,121.0000) -- (62.3000,121.0000) -- (62.5000,121.0000) -- (62.6000,121.0000) -- (62.7000,121.0000) -- (62.9000,121.0000) -- (63.0000,121.0000) -- (63.2000,121.0000) -- (63.3000,121.0000) -- (63.5000,121.0000) -- (63.6000,121.0000) -- (63.7000,121.0000) -- (63.9000,121.0000) -- (64.0000,121.0000) -- (64.2000,121.0000) -- (64.3000,121.0000) -- (64.5000,121.0000) -- (64.6000,121.0000) -- (64.7000,121.0000) -- (64.9000,121.0000) -- (65.0000,121.0000) -- (65.2000,121.0000) -- (65.3000,121.0000) -- (65.5000,121.0000) -- (65.6000,121.0000) -- (65.7000,121.0000) -- (65.9000,121.0000) -- (66.0000,121.0000) -- (66.2000,121.0000) -- (66.3000,121.0000) -- (66.5000,120.9000) -- (66.6000,120.9000) -- (66.7000,120.9000) -- (66.9000,120.9000) -- (67.0000,120.9000) -- (67.2000,120.9000) -- (67.3000,120.9000) -- (67.5000,120.9000) -- (67.6000,120.9000) -- (67.7000,120.9000) -- (67.9000,120.9000) -- (68.0000,120.9000) -- (68.2000,120.8000) -- (68.3000,120.8000) -- (68.5000,120.8000) -- (68.6000,120.8000) -- (68.7000,120.8000) -- (68.9000,120.8000) -- (69.0000,120.8000) -- (69.2000,120.8000) -- (69.3000,120.8000) -- (69.5000,120.8000) -- (69.6000,120.8000) -- (69.7000,120.8000) -- (69.9000,120.8000) -- (70.0000,120.8000) -- (70.2000,120.8000) -- (70.3000,120.8000) -- (70.5000,120.8000) -- (70.6000,120.8000) -- (70.7000,120.8000) -- (70.9000,120.8000) -- (71.0000,120.8000) -- (71.2000,120.8000) -- (71.3000,120.8000) -- (71.5000,120.9000) -- (71.6000,120.9000) -- (71.7000,120.9000) -- (71.9000,120.9000) -- (72.0000,120.9000) -- (72.2000,120.9000) -- (72.3000,121.0000) -- (72.5000,121.0000) -- (72.6000,121.0000) -- (72.7000,121.0000) -- (72.9000,121.1000) -- (73.0000,121.1000) -- (73.2000,121.1000) -- (73.3000,121.2000) -- (73.5000,121.2000) -- (73.6000,121.2000) -- (73.7000,121.2000) -- (73.9000,121.3000) -- (74.0000,121.3000) -- (74.2000,121.3000) -- (74.3000,121.3000) -- (74.5000,121.4000) -- (74.6000,121.4000) -- (74.7000,121.4000) -- (74.9000,121.5000) -- (75.0000,121.5000) -- (75.2000,121.5000) -- (75.3000,121.5000) -- (75.5000,121.6000) -- (75.6000,121.6000) -- (75.7000,121.6000) -- (75.9000,121.6000) -- (76.0000,121.7000) -- (76.2000,121.7000) -- (76.3000,121.7000) -- (76.5000,121.7000) -- (76.6000,121.7000) -- (76.7000,121.7000) -- (76.9000,121.8000) -- (77.0000,121.8000) -- (77.2000,121.8000) -- (77.3000,121.8000) -- (77.5000,121.8000) -- (77.6000,121.8000) -- (77.7000,121.9000) -- (77.9000,121.9000) -- (78.0000,121.9000) -- (78.2000,121.9000) -- (78.3000,121.9000) -- (78.5000,121.9000) -- (78.6000,121.9000) -- (78.7000,121.9000) -- (78.9000,121.9000) -- (79.0000,121.9000) -- (79.2000,121.9000) -- (79.3000,121.9000) -- (79.5000,121.9000) -- (79.6000,121.9000) -- (79.7000,122.0000) -- (79.9000,122.0000) -- (80.0000,122.0000) -- (80.2000,122.0000) -- (80.3000,122.0000) -- (80.5000,121.9000) -- (80.6000,121.9000) -- (80.7000,121.9000) -- (80.9000,121.9000) -- (81.0000,121.9000) -- (81.2000,121.9000) -- (81.3000,121.9000) -- (81.5000,121.9000) -- (81.6000,121.9000) -- (81.7000,121.9000) -- (81.9000,121.9000) -- (82.0000,121.9000) -- (82.2000,121.9000) -- (82.3000,121.9000) -- (82.5000,121.9000) -- (82.6000,121.9000) -- (82.7000,121.8000) -- (82.9000,121.8000) -- (83.0000,121.8000) -- (83.2000,121.8000) -- (83.3000,121.8000) -- (83.5000,121.8000) -- (83.6000,121.8000) -- (83.7000,121.8000) -- (83.9000,121.7000) -- (84.0000,121.7000) -- (84.2000,121.7000) -- (84.3000,121.7000) -- (84.5000,121.7000) -- (84.6000,121.7000) -- (84.7000,121.6000) -- (84.9000,121.6000) -- (85.0000,121.6000) -- (85.2000,121.6000) -- (85.3000,121.6000) -- (85.5000,121.5000) -- (85.6000,121.5000) -- (85.7000,121.5000) -- (85.9000,121.5000) -- (86.0000,121.5000) -- (86.2000,121.5000) -- (86.3000,121.4000) -- (86.5000,121.4000) -- (86.6000,121.4000) -- (86.7000,121.4000) -- (86.9000,121.3000) -- (87.0000,121.3000) -- (87.2000,121.3000) -- (87.3000,121.3000) -- (87.5000,121.3000) -- (87.6000,121.2000) -- (87.7000,121.2000) -- (87.9000,121.2000) -- (88.0000,121.2000) -- (88.2000,121.1000) -- (88.3000,121.1000) -- (88.5000,121.1000) -- (88.6000,121.1000) -- (88.7000,121.0000) -- (88.9000,121.0000) -- (89.0000,121.0000) -- (89.2000,121.0000) -- (89.3000,120.9000) -- (89.5000,120.9000) -- (89.6000,120.9000) -- (89.7000,120.9000) -- (89.9000,120.8000) -- (90.0000,120.8000) -- (90.2000,120.8000) -- (90.3000,120.8000) -- (90.5000,120.7000) -- (90.6000,120.7000) -- (90.7000,120.7000) -- (90.9000,120.6000) -- (91.0000,120.6000) -- (91.2000,120.6000) -- (91.3000,120.6000) -- (91.5000,120.5000) -- (91.6000,120.5000) -- (91.7000,120.5000) -- (91.9000,120.5000) -- (92.0000,120.4000) -- (92.2000,120.4000) -- (92.3000,120.4000) -- (92.5000,120.4000) -- (92.6000,120.3000) -- (92.7000,120.3000) -- (92.9000,120.3000) -- (93.0000,120.2000) -- (93.2000,120.2000) -- (93.3000,120.2000) -- (93.4000,120.2000) -- (93.6000,120.1000) -- (93.7000,120.1000) -- (93.9000,120.1000) -- (94.0000,120.1000) -- (94.2000,120.0000) -- (94.3000,120.0000) -- (94.4000,120.0000) -- (94.6000,120.0000) -- (94.7000,119.9000) -- (94.9000,119.9000) -- (95.0000,119.9000) -- (95.2000,119.8000) -- (95.3000,119.8000) -- (95.4000,119.8000) -- (95.6000,119.8000) -- (95.7000,119.7000) -- (95.9000,119.7000) -- (96.0000,119.7000) -- (96.2000,119.7000) -- (96.3000,119.6000) -- (96.4000,119.6000) -- (96.6000,119.6000) -- (96.7000,119.6000) -- (96.9000,119.5000) -- (97.0000,119.5000) -- (97.2000,119.5000) -- (97.3000,119.5000) -- (97.4000,119.4000) -- (97.6000,119.4000) -- (97.7000,119.4000) -- (97.9000,119.3000) -- (98.0000,119.3000) -- (98.2000,119.3000) -- (98.3000,119.3000) -- (98.4000,119.2000) -- (98.6000,119.2000) -- (98.7000,119.2000) -- (98.9000,119.1000) -- (99.0000,119.1000) -- (99.2000,119.1000) -- (99.3000,119.1000) -- (99.4000,119.0000) -- (99.6000,119.0000) -- (99.7000,119.0000) -- (99.9000,119.0000) -- (100.0000,118.9000) -- (100.2000,118.9000) -- (100.3000,118.9000) -- (100.4000,118.9000) -- (100.6000,118.8000) -- (100.7000,118.8000) -- (100.9000,118.8000) -- (101.0000,118.8000) -- (101.2000,118.7000) -- (101.3000,118.7000) -- (101.4000,118.7000) -- (101.6000,118.7000) -- (101.7000,118.6000) -- (101.9000,118.6000) -- (102.0000,118.6000) -- (102.2000,118.6000) -- (102.3000,118.5000) -- (102.4000,118.5000) -- (102.6000,118.5000) -- (102.7000,118.5000) -- (102.9000,118.4000) -- (103.0000,118.4000) -- (103.2000,118.4000) -- (103.3000,118.4000) -- (103.4000,118.4000) -- (103.6000,118.3000) -- (103.7000,118.3000) -- (103.9000,118.3000) -- (104.0000,118.3000) -- (104.2000,118.3000) -- (104.3000,118.2000) -- (104.4000,118.2000) -- (104.6000,118.2000) -- (104.7000,118.2000) -- (104.9000,118.2000) -- (105.0000,118.1000) -- (105.2000,118.1000) -- (105.3000,118.1000) -- (105.4000,118.1000) -- (105.6000,118.1000) -- (105.7000,118.0000) -- (105.9000,118.0000) -- (106.0000,118.0000) -- (106.2000,118.0000) -- (106.3000,118.0000) -- (106.4000,118.0000) -- (106.6000,117.9000) -- (106.7000,117.9000) -- (106.9000,117.9000) -- (107.0000,117.9000) -- (107.2000,117.9000) -- (107.3000,117.9000) -- (107.4000,117.9000) -- (107.6000,117.8000) -- (107.7000,117.8000) -- (107.9000,117.8000) -- (108.0000,117.8000) -- (108.2000,117.8000) -- (108.3000,117.8000) -- (108.4000,117.8000) -- (108.6000,117.8000) -- (108.7000,117.7000) -- (108.9000,117.7000) -- (109.0000,117.7000) -- (109.2000,117.7000) -- (109.3000,117.7000) -- (109.4000,117.7000) -- (109.6000,117.7000) -- (109.7000,117.7000) -- (109.9000,117.7000) -- (110.0000,117.6000) -- (110.2000,117.6000) -- (110.3000,117.6000) -- (110.4000,117.6000) -- (110.6000,117.6000) -- (110.7000,117.6000) -- (110.9000,117.6000) -- (111.0000,117.6000) -- (111.2000,117.6000) -- (111.3000,117.6000) -- (111.4000,117.6000) -- (111.6000,117.6000) -- (111.7000,117.5000) -- (111.9000,117.5000) -- (112.0000,117.5000) -- (112.2000,117.5000) -- (112.3000,117.5000) -- (112.4000,117.5000) -- (112.6000,117.5000) -- (112.7000,117.5000) -- (112.9000,117.5000) -- (113.0000,117.5000) -- (113.2000,117.5000) -- (113.3000,117.5000) -- (113.4000,117.5000) -- (113.6000,117.5000) -- (113.7000,117.5000) -- (113.9000,117.5000) -- (114.0000,117.5000) -- (114.2000,117.5000) -- (114.3000,117.4000) -- (114.4000,117.4000) -- (114.6000,117.4000) -- (114.7000,117.4000) -- (114.9000,117.4000) -- (115.0000,117.4000) -- (115.2000,117.4000) -- (115.3000,117.4000) -- (115.4000,117.4000) -- (115.6000,117.4000) -- (115.7000,117.4000) -- (115.9000,117.4000) -- (116.0000,117.4000) -- (116.2000,117.4000) -- (116.3000,117.4000) -- (116.4000,117.4000) -- (116.6000,117.4000) -- (116.7000,117.4000) -- (116.9000,117.4000) -- (117.0000,117.4000) -- (117.2000,117.4000) -- (117.3000,117.4000) -- (117.4000,117.4000) -- (117.6000,117.4000) -- (117.7000,117.4000) -- (117.9000,117.4000) -- (118.0000,117.4000) -- (118.2000,117.4000) -- (118.3000,117.4000) -- (118.4000,117.4000) -- (118.6000,117.4000) -- (118.7000,117.4000) -- (118.9000,117.4000) -- (119.0000,117.4000) -- (119.2000,117.4000) -- (119.3000,117.4000) -- (119.4000,117.4000) -- (119.6000,117.4000) -- (119.7000,117.4000) -- (119.9000,117.4000) -- (120.0000,117.4000) -- (120.2000,117.4000) -- (120.3000,117.4000) -- (120.4000,117.4000) -- (120.6000,117.4000) -- (120.7000,117.4000) -- (120.9000,117.4000) -- (121.0000,117.4000) -- (121.2000,117.4000) -- (121.3000,117.4000) -- (121.4000,117.4000) -- (121.6000,117.4000) -- (121.7000,117.4000) -- (121.9000,117.4000) -- (122.0000,117.4000) -- (122.2000,117.4000) -- (122.3000,117.4000) -- (122.4000,117.4000) -- (122.6000,117.4000) -- (122.7000,117.4000) -- (122.9000,117.4000) -- (123.0000,117.4000) -- (123.2000,117.4000) -- (123.3000,117.4000) -- (123.4000,117.4000) -- (123.6000,117.4000) -- (123.7000,117.4000) -- (123.9000,117.4000) -- (124.0000,117.4000) -- (124.2000,117.4000) -- (124.3000,117.4000) -- (124.4000,117.5000) -- (124.6000,117.5000) -- (124.7000,117.5000) -- (124.9000,117.5000) -- (125.0000,117.5000) -- (125.2000,117.5000) -- (125.3000,117.5000) -- (125.4000,117.5000) -- (125.6000,117.5000) -- (125.7000,117.5000) -- (125.9000,117.5000) -- (126.0000,117.5000) -- (126.2000,117.5000) -- (126.3000,117.5000) -- (126.4000,117.5000) -- (126.6000,117.5000) -- (126.7000,117.5000) -- (126.9000,117.5000) -- (127.0000,117.5000) -- (127.2000,117.5000) -- (127.3000,117.5000);



    \end{scope}
    \begin{scope}[cm={{0.99667,0.0,0.0,1.34693,(-320.19845,-156.47287)}},draw=cff0000,line cap=round,line join=bevel,line width=0.480pt,miter limit=4.00]
      \path[draw,line cap=round,line join=round,line width=0.480pt,miter limit=4.00] (41.6000,109.8000) -- (41.6000,109.8000) -- (41.7000,103.2000) -- (41.9000,103.5000) -- (42.0000,103.9000) -- (42.2000,104.4000) -- (42.3000,104.8000) -- (42.5000,105.2000) -- (42.6000,105.6000) -- (42.7000,106.0000) -- (42.9000,106.4000) -- (43.0000,106.7000) -- (43.2000,107.1000) -- (43.3000,107.4000) -- (43.5000,107.7000) -- (43.6000,107.9000) -- (43.7000,108.2000) -- (43.9000,108.5000) -- (44.0000,108.7000) -- (44.2000,108.9000) -- (44.3000,109.1000) -- (44.5000,109.4000) -- (44.6000,109.6000) -- (44.7000,109.7000) -- (44.9000,109.9000) -- (45.0000,110.1000) -- (45.2000,110.3000) -- (45.3000,110.5000) -- (45.5000,110.7000) -- (45.6000,110.9000) -- (45.7000,111.1000) -- (45.9000,111.3000) -- (46.0000,111.5000) -- (46.2000,111.7000) -- (46.3000,111.9000) -- (46.5000,112.1000) -- (46.6000,112.3000) -- (46.7000,112.5000) -- (46.9000,112.7000) -- (47.0000,112.9000) -- (47.2000,113.1000) -- (47.3000,113.3000) -- (47.5000,113.5000) -- (47.6000,113.7000) -- (47.7000,113.9000) -- (47.9000,114.1000) -- (48.0000,114.3000) -- (48.2000,114.5000) -- (48.3000,114.6000) -- (48.5000,114.8000) -- (48.6000,114.9000) -- (48.7000,115.1000) -- (48.9000,115.2000) -- (49.0000,115.3000) -- (49.2000,115.4000) -- (49.3000,115.6000) -- (49.5000,115.7000) -- (49.6000,115.8000) -- (49.7000,115.9000) -- (49.9000,116.0000) -- (50.0000,116.1000) -- (50.2000,116.2000) -- (50.3000,116.3000) -- (50.5000,116.4000) -- (50.6000,116.5000) -- (50.7000,116.6000) -- (50.9000,116.8000) -- (51.0000,116.9000) -- (51.2000,117.0000) -- (51.3000,117.2000) -- (51.5000,117.3000) -- (51.6000,117.5000) -- (51.7000,117.7000) -- (51.9000,117.8000) -- (52.0000,118.0000) -- (52.2000,118.2000) -- (52.3000,118.3000) -- (52.5000,118.5000) -- (52.6000,118.6000) -- (52.7000,118.8000) -- (52.9000,118.9000) -- (53.0000,119.0000) -- (53.2000,119.1000) -- (53.3000,119.2000) -- (53.5000,119.3000) -- (53.6000,119.4000) -- (53.7000,119.4000) -- (53.9000,119.4000) -- (54.0000,119.5000) -- (54.2000,119.5000) -- (54.3000,119.5000) -- (54.5000,119.4000) -- (54.6000,119.4000) -- (54.7000,119.4000) -- (54.9000,119.3000) -- (55.0000,119.3000) -- (55.2000,119.2000) -- (55.3000,119.2000) -- (55.5000,119.1000) -- (55.6000,119.1000) -- (55.7000,119.1000) -- (55.9000,119.0000) -- (56.0000,119.0000) -- (56.2000,119.0000) -- (56.3000,119.1000) -- (56.5000,119.1000) -- (56.6000,119.2000) -- (56.7000,119.2000) -- (56.9000,119.3000) -- (57.0000,119.4000) -- (57.2000,119.5000) -- (57.3000,119.6000) -- (57.5000,119.7000) -- (57.6000,119.8000) -- (57.7000,119.9000) -- (57.9000,120.1000) -- (58.0000,120.2000) -- (58.2000,120.3000) -- (58.3000,120.4000) -- (58.5000,120.5000) -- (58.6000,120.6000) -- (58.7000,120.7000) -- (58.9000,120.8000) -- (59.0000,120.8000) -- (59.2000,120.9000) -- (59.3000,121.0000) -- (59.5000,121.0000) -- (59.6000,121.0000) -- (59.7000,121.1000) -- (59.9000,121.1000) -- (60.0000,121.1000) -- (60.2000,121.1000) -- (60.3000,121.1000) -- (60.5000,121.1000) -- (60.6000,121.1000) -- (60.7000,121.0000) -- (60.9000,121.0000) -- (61.0000,121.0000) -- (61.2000,121.0000) -- (61.3000,120.9000) -- (61.5000,120.9000) -- (61.6000,120.9000) -- (61.7000,120.8000) -- (61.9000,120.8000) -- (62.0000,120.8000) -- (62.2000,120.8000) -- (62.3000,120.7000) -- (62.5000,120.7000) -- (62.6000,120.7000) -- (62.7000,120.7000) -- (62.9000,120.7000) -- (63.0000,120.7000) -- (63.2000,120.7000) -- (63.3000,120.7000) -- (63.5000,120.7000) -- (63.6000,120.8000) -- (63.7000,120.8000) -- (63.9000,120.8000) -- (64.0000,120.8000) -- (64.2000,120.8000) -- (64.3000,120.9000) -- (64.5000,120.9000) -- (64.6000,120.9000) -- (64.7000,121.0000) -- (64.9000,121.0000) -- (65.0000,121.0000) -- (65.2000,121.0000) -- (65.3000,121.1000) -- (65.5000,121.1000) -- (65.6000,121.1000) -- (65.7000,121.1000) -- (65.9000,121.2000) -- (66.0000,121.2000) -- (66.2000,121.2000) -- (66.3000,121.2000) -- (66.5000,121.2000) -- (66.6000,121.2000) -- (66.7000,121.2000) -- (66.9000,121.2000) -- (67.0000,121.2000) -- (67.2000,121.2000) -- (67.3000,121.2000) -- (67.5000,121.2000) -- (67.6000,121.2000) -- (67.7000,121.2000) -- (67.9000,121.2000) -- (68.0000,121.2000) -- (68.2000,121.1000) -- (68.3000,121.1000) -- (68.5000,121.1000) -- (68.6000,121.1000) -- (68.7000,121.0000) -- (68.9000,121.0000) -- (69.0000,121.0000) -- (69.2000,120.9000) -- (69.3000,120.9000) -- (69.5000,120.9000) -- (69.6000,120.8000) -- (69.7000,120.8000) -- (69.9000,120.8000) -- (70.0000,120.7000) -- (70.2000,120.7000) -- (70.3000,120.7000) -- (70.5000,120.7000) -- (70.6000,120.6000) -- (70.7000,120.6000) -- (70.9000,120.6000) -- (71.0000,120.6000) -- (71.2000,120.6000) -- (71.3000,120.6000) -- (71.5000,120.6000) -- (71.6000,120.6000) -- (71.7000,120.6000) -- (71.9000,120.6000) -- (72.0000,120.6000) -- (72.2000,120.6000) -- (72.3000,120.6000) -- (72.5000,120.6000) -- (72.6000,120.6000) -- (72.7000,120.7000) -- (72.9000,120.7000) -- (73.0000,120.7000) -- (73.2000,120.8000) -- (73.3000,120.8000) -- (73.5000,120.8000) -- (73.6000,120.9000) -- (73.7000,120.9000) -- (73.9000,120.9000) -- (74.0000,121.0000) -- (74.2000,121.0000) -- (74.3000,121.1000) -- (74.5000,121.1000) -- (74.6000,121.2000) -- (74.7000,121.2000) -- (74.9000,121.3000) -- (75.0000,121.3000) -- (75.2000,121.3000) -- (75.3000,121.4000) -- (75.5000,121.4000) -- (75.6000,121.5000) -- (75.7000,121.5000) -- (75.9000,121.6000) -- (76.0000,121.6000) -- (76.2000,121.7000) -- (76.3000,121.7000) -- (76.5000,121.7000) -- (76.6000,121.8000) -- (76.7000,121.8000) -- (76.9000,121.9000) -- (77.0000,121.9000) -- (77.2000,121.9000) -- (77.3000,122.0000) -- (77.5000,122.0000) -- (77.6000,122.0000) -- (77.7000,122.1000) -- (77.9000,122.1000) -- (78.0000,122.1000) -- (78.2000,122.1000) -- (78.3000,122.1000) -- (78.5000,122.2000) -- (78.6000,122.2000) -- (78.7000,122.2000) -- (78.9000,122.2000) -- (79.0000,122.2000) -- (79.2000,122.2000) -- (79.3000,122.2000) -- (79.5000,122.2000) -- (79.6000,122.2000) -- (79.7000,122.2000) -- (79.9000,122.2000) -- (80.0000,122.2000) -- (80.2000,122.2000) -- (80.3000,122.2000) -- (80.5000,122.2000) -- (80.6000,122.2000) -- (80.7000,122.2000) -- (80.9000,122.2000) -- (81.0000,122.2000) -- (81.2000,122.2000) -- (81.3000,122.2000) -- (81.5000,122.1000) -- (81.6000,122.1000) -- (81.7000,122.1000) -- (81.9000,122.1000) -- (82.0000,122.1000) -- (82.2000,122.0000) -- (82.3000,122.0000) -- (82.5000,122.0000) -- (82.6000,122.0000) -- (82.7000,121.9000) -- (82.9000,121.9000) -- (83.0000,121.9000) -- (83.2000,121.9000) -- (83.3000,121.8000) -- (83.5000,121.8000) -- (83.6000,121.8000) -- (83.7000,121.8000) -- (83.9000,121.7000) -- (84.0000,121.7000) -- (84.2000,121.7000) -- (84.3000,121.7000) -- (84.5000,121.6000) -- (84.6000,121.6000) -- (84.7000,121.6000) -- (84.9000,121.5000) -- (85.0000,121.5000) -- (85.2000,121.5000) -- (85.3000,121.5000) -- (85.5000,121.4000) -- (85.6000,121.4000) -- (85.7000,121.4000) -- (85.9000,121.3000) -- (86.0000,121.3000) -- (86.2000,121.3000) -- (86.3000,121.3000) -- (86.5000,121.2000) -- (86.6000,121.2000) -- (86.7000,121.2000) -- (86.9000,121.2000) -- (87.0000,121.1000) -- (87.2000,121.1000) -- (87.3000,121.1000) -- (87.5000,121.1000) -- (87.6000,121.1000) -- (87.7000,121.0000) -- (87.9000,121.0000) -- (88.0000,121.0000) -- (88.2000,121.0000) -- (88.3000,120.9000) -- (88.5000,120.9000) -- (88.6000,120.9000) -- (88.7000,120.9000) -- (88.9000,120.9000) -- (89.0000,120.8000) -- (89.2000,120.8000) -- (89.3000,120.8000) -- (89.5000,120.8000) -- (89.6000,120.8000) -- (89.7000,120.7000) -- (89.9000,120.7000) -- (90.0000,120.7000) -- (90.2000,120.7000) -- (90.3000,120.7000) -- (90.5000,120.6000) -- (90.6000,120.6000) -- (90.7000,120.6000) -- (90.9000,120.6000) -- (91.0000,120.6000) -- (91.2000,120.5000) -- (91.3000,120.5000) -- (91.5000,120.5000) -- (91.6000,120.5000) -- (91.7000,120.5000) -- (91.9000,120.4000) -- (92.0000,120.4000) -- (92.2000,120.4000) -- (92.3000,120.4000) -- (92.5000,120.4000) -- (92.6000,120.4000) -- (92.7000,120.3000) -- (92.9000,120.3000) -- (93.0000,120.3000) -- (93.2000,120.3000) -- (93.3000,120.3000) -- (93.4000,120.2000) -- (93.6000,120.2000) -- (93.7000,120.2000) -- (93.9000,120.2000) -- (94.0000,120.2000) -- (94.2000,120.1000) -- (94.3000,120.1000) -- (94.4000,120.1000) -- (94.6000,120.1000) -- (94.7000,120.0000) -- (94.9000,120.0000) -- (95.0000,120.0000) -- (95.2000,120.0000) -- (95.3000,120.0000) -- (95.4000,119.9000) -- (95.6000,119.9000) -- (95.7000,119.9000) -- (95.9000,119.9000) -- (96.0000,119.8000) -- (96.2000,119.8000) -- (96.3000,119.8000) -- (96.4000,119.8000) -- (96.6000,119.7000) -- (96.7000,119.7000) -- (96.9000,119.7000) -- (97.0000,119.7000) -- (97.2000,119.6000) -- (97.3000,119.6000) -- (97.4000,119.6000) -- (97.6000,119.5000) -- (97.7000,119.5000) -- (97.9000,119.5000) -- (98.0000,119.5000) -- (98.2000,119.4000) -- (98.3000,119.4000) -- (98.4000,119.4000) -- (98.6000,119.3000) -- (98.7000,119.3000) -- (98.9000,119.3000) -- (99.0000,119.2000) -- (99.2000,119.2000) -- (99.3000,119.2000) -- (99.4000,119.1000) -- (99.6000,119.1000) -- (99.7000,119.1000) -- (99.9000,119.1000) -- (100.0000,119.0000) -- (100.2000,119.0000) -- (100.3000,119.0000) -- (100.4000,118.9000) -- (100.6000,118.9000) -- (100.7000,118.9000) -- (100.9000,118.8000) -- (101.0000,118.8000) -- (101.2000,118.8000) -- (101.3000,118.7000) -- (101.4000,118.7000) -- (101.6000,118.7000) -- (101.7000,118.6000) -- (101.9000,118.6000) -- (102.0000,118.6000) -- (102.2000,118.6000) -- (102.3000,118.5000) -- (102.4000,118.5000) -- (102.6000,118.5000) -- (102.7000,118.4000) -- (102.9000,118.4000) -- (103.0000,118.4000) -- (103.2000,118.4000) -- (103.3000,118.3000) -- (103.4000,118.3000) -- (103.6000,118.3000) -- (103.7000,118.3000) -- (103.9000,118.2000) -- (104.0000,118.2000) -- (104.2000,118.2000) -- (104.3000,118.2000) -- (104.4000,118.1000) -- (104.6000,118.1000) -- (104.7000,118.1000) -- (104.9000,118.1000) -- (105.0000,118.0000) -- (105.2000,118.0000) -- (105.3000,118.0000) -- (105.4000,118.0000) -- (105.6000,118.0000) -- (105.7000,117.9000) -- (105.9000,117.9000) -- (106.0000,117.9000) -- (106.2000,117.9000) -- (106.3000,117.9000) -- (106.4000,117.8000) -- (106.6000,117.8000) -- (106.7000,117.8000) -- (106.9000,117.8000) -- (107.0000,117.8000) -- (107.2000,117.8000) -- (107.3000,117.7000) -- (107.4000,117.7000) -- (107.6000,117.7000) -- (107.7000,117.7000) -- (107.9000,117.7000) -- (108.0000,117.7000) -- (108.2000,117.7000) -- (108.3000,117.6000) -- (108.4000,117.6000) -- (108.6000,117.6000) -- (108.7000,117.6000) -- (108.9000,117.6000) -- (109.0000,117.6000) -- (109.2000,117.6000) -- (109.3000,117.6000) -- (109.4000,117.6000) -- (109.6000,117.6000) -- (109.7000,117.5000) -- (109.9000,117.5000) -- (110.0000,117.5000) -- (110.2000,117.5000) -- (110.3000,117.5000) -- (110.4000,117.5000) -- (110.6000,117.5000) -- (110.7000,117.5000) -- (110.9000,117.5000) -- (111.0000,117.5000) -- (111.2000,117.5000) -- (111.3000,117.5000) -- (111.4000,117.5000) -- (111.6000,117.5000) -- (111.7000,117.5000) -- (111.9000,117.5000) -- (112.0000,117.5000) -- (112.2000,117.5000) -- (112.3000,117.4000) -- (112.4000,117.4000) -- (112.6000,117.4000) -- (112.7000,117.4000) -- (112.9000,117.4000) -- (113.0000,117.4000) -- (113.2000,117.4000) -- (113.3000,117.4000) -- (113.4000,117.4000) -- (113.6000,117.4000) -- (113.7000,117.4000) -- (113.9000,117.4000) -- (114.0000,117.4000) -- (114.2000,117.4000) -- (114.3000,117.4000) -- (114.4000,117.4000) -- (114.6000,117.4000) -- (114.7000,117.4000) -- (114.9000,117.4000) -- (115.0000,117.4000) -- (115.2000,117.4000) -- (115.3000,117.4000) -- (115.4000,117.4000) -- (115.6000,117.4000) -- (115.7000,117.4000) -- (115.9000,117.4000) -- (116.0000,117.4000) -- (116.2000,117.4000) -- (116.3000,117.4000) -- (116.4000,117.4000) -- (116.6000,117.4000) -- (116.7000,117.4000) -- (116.9000,117.5000) -- (117.0000,117.5000) -- (117.2000,117.5000) -- (117.3000,117.5000) -- (117.4000,117.5000) -- (117.6000,117.5000) -- (117.7000,117.5000) -- (117.9000,117.5000) -- (118.0000,117.5000) -- (118.2000,117.5000) -- (118.3000,117.5000) -- (118.4000,117.5000) -- (118.6000,117.5000) -- (118.7000,117.5000) -- (118.9000,117.5000) -- (119.0000,117.5000) -- (119.2000,117.5000) -- (119.3000,117.5000) -- (119.4000,117.5000) -- (119.6000,117.5000) -- (119.7000,117.5000) -- (119.9000,117.5000) -- (120.0000,117.5000) -- (120.2000,117.5000) -- (120.3000,117.5000) -- (120.4000,117.5000) -- (120.6000,117.5000) -- (120.7000,117.5000) -- (120.9000,117.5000) -- (121.0000,117.5000) -- (121.2000,117.5000) -- (121.3000,117.5000) -- (121.4000,117.5000) -- (121.6000,117.5000) -- (121.7000,117.5000) -- (121.9000,117.5000) -- (122.0000,117.5000) -- (122.2000,117.5000) -- (122.3000,117.5000) -- (122.4000,117.5000) -- (122.6000,117.5000) -- (122.7000,117.5000) -- (122.9000,117.5000) -- (123.0000,117.5000) -- (123.2000,117.5000) -- (123.3000,117.5000) -- (123.4000,117.5000) -- (123.6000,117.5000) -- (123.7000,117.5000) -- (123.9000,117.6000) -- (124.0000,117.6000) -- (124.2000,117.6000) -- (124.3000,117.6000) -- (124.4000,117.6000) -- (124.6000,117.6000) -- (124.7000,117.6000) -- (124.9000,117.6000) -- (125.0000,117.6000) -- (125.2000,117.6000) -- (125.3000,117.6000) -- (125.4000,117.6000) -- (125.6000,117.6000) -- (125.7000,117.6000) -- (125.9000,117.6000) -- (126.0000,117.6000) -- (126.2000,117.6000) -- (126.3000,117.6000) -- (126.4000,117.6000) -- (126.6000,117.6000) -- (126.7000,117.6000) -- (126.9000,117.6000) -- (127.0000,117.6000) -- (127.2000,117.6000) -- (127.3000,117.6000);



    \end{scope}
    \begin{scope}[cm={{1.21653,0.0,0.0,1.34548,(-346.94368,-156.28627)}},fill=cffffff]
      \path[fill=cd9d9d9,rounded corners=0.0000cm] (81.0000,129.0000) rectangle (121.0000,157.0000);



    \end{scope}
    \begin{scope}[cm={{1.21653,0.0,0.0,1.34548,(-346.94368,-156.28627)}},draw=ca0a0a4,dash pattern=on 1.40pt off 1.40pt,line cap=round,line join=round,line width=0.233pt,miter limit=4.00]
      \path[draw,dash pattern=on 1.40pt off 1.40pt,line width=0.233pt,miter limit=4.00] (81.5000,135.5000) -- (121.5000,135.5000);



    \end{scope}
    \begin{scope}[cm={{1.21653,0.0,0.0,1.34548,(-346.94368,-156.28627)}},draw=blue,line cap=round,line join=round,line width=0.480pt]
      \path[draw] (81.5000,135.5000) -- (82.6460,135.5000);



      \path[draw] (121.5000,135.5000) -- (120.1220,135.5000);



    \end{scope}
    \begin{scope}[cm={{1.00174,0.0,0.0,1.01515,(-276.4622,28.89387)}},draw=blue,line cap=rect,line join=bevel,line width=0.800pt]
      \path[fill=blue] (0.0000,0.0000) node[above right] (text766) {\scriptsize $T$\hspace{.5ex}=\hspace{.5ex}47};



    \end{scope}
    \begin{scope}[cm={{1.21653,0.0,0.0,1.34548,(-346.94368,-156.28627)}},draw=ca0a0a4,dash pattern=on 1.40pt off 1.40pt,line cap=round,line join=round,line width=0.233pt,miter limit=4.00]
      \path[draw,dash pattern=on 1.40pt off 1.40pt,line width=0.233pt,miter limit=4.00] (97.5000,157.5000) -- (97.5000,129.5000);



    \end{scope}
    \begin{scope}[cm={{1.21653,0.0,0.0,1.34548,(-346.94368,-156.28627)}},draw=blue,line cap=round,line join=round,line width=0.480pt]
      \path[draw] (97.5000,129.5000) -- (97.5000,129.5000) -- (97.5000,130.6564);



    \end{scope}
    \begin{scope}[cm={{1.00174,0.0,0.0,1.01515,(-232.9405,15.55865)}},draw=blue,line cap=rect,line join=bevel,line width=0.800pt]
      \path[fill=blue] (0.0000,0.0000) node[above right] (text794) {\scriptsize 84};



    \end{scope}
    \begin{scope}[cm={{1.21653,0.0,0.0,1.34548,(-346.94368,-156.28627)}},draw=blue,line cap=round,line join=round,line width=0.480pt]
      \path[draw] (81.5000,129.5000) -- (81.5000,157.5000) -- (121.5000,157.5000) -- (121.5000,129.5000) -- (81.5000,129.5000);



    \end{scope}
    \begin{scope}[cm={{1.21653,0.0,0.0,1.34548,(-346.94368,-156.28627)}},draw=blue,line cap=round,line join=round,line width=0.480pt]
      \path[draw] (81.2000,156.5000) -- (81.2000,156.5000) -- (81.2000,156.5000) -- (81.2000,156.5000) -- (81.2000,156.5000) -- (81.2000,156.5000) -- (81.2000,156.5000) -- (81.2000,156.5000) -- (81.2000,156.5000) -- (81.2000,156.5000) -- (81.2000,156.5000) -- (81.2000,156.5000) -- (81.2000,156.5000) -- (81.2000,156.5000) -- (81.2000,156.5000) -- (81.2000,156.5000) -- (81.2000,156.5000) -- (81.2000,156.5000) -- (81.2000,156.5000) -- (81.2000,156.5000) -- (81.2000,156.4000) -- (81.2000,156.4000) -- (81.2000,156.4000) -- (81.2000,156.4000) -- (81.2000,156.4000) -- (81.2000,156.4000) -- (81.3000,156.4000) -- (81.3000,156.4000) -- (81.3000,156.4000) -- (81.3000,156.4000) -- (81.3000,156.4000) -- (81.3000,156.4000) -- (81.3000,156.4000) -- (81.3000,156.4000) -- (81.3000,156.4000) -- (81.3000,156.4000) -- (81.3000,156.4000) -- (81.3000,156.4000) -- (81.3000,156.4000) -- (81.3000,156.4000) -- (81.3000,156.4000) -- (81.3000,156.3000) -- (81.3000,156.3000) -- (81.3000,156.3000) -- (81.3000,156.3000) -- (81.3000,156.3000) -- (81.3000,156.3000) -- (81.3000,156.3000) -- (81.3000,156.3000) -- (81.3000,156.3000) -- (81.3000,156.3000) -- (81.3000,156.3000) -- (81.3000,156.3000) -- (81.3000,156.3000) -- (81.3000,156.3000) -- (81.3000,156.3000) -- (81.3000,156.3000) -- (81.3000,156.3000) -- (81.3000,156.3000) -- (81.3000,156.3000) -- (81.3000,156.3000) -- (81.3000,156.3000) -- (81.3000,156.3000) -- (81.3000,156.2000) -- (81.3000,156.2000) -- (81.3000,156.2000) -- (81.3000,156.2000) -- (81.3000,156.2000) -- (81.3000,156.2000) -- (81.3000,156.2000) -- (81.3000,156.2000) -- (81.3000,156.2000) -- (81.3000,156.2000) -- (81.3000,156.2000) -- (81.3000,156.2000) -- (81.3000,156.2000) -- (81.4000,156.2000) -- (81.4000,156.2000) -- (81.4000,156.2000) -- (81.4000,156.2000) -- (81.4000,156.2000) -- (81.4000,156.2000) -- (81.4000,156.2000) -- (81.4000,156.2000) -- (81.4000,156.1000) -- (81.4000,156.1000) -- (81.4000,156.1000) -- (81.4000,156.1000) -- (81.4000,156.1000) -- (81.4000,156.1000) -- (81.4000,156.1000) -- (81.4000,156.1000) -- (81.4000,156.1000) -- (81.4000,156.1000) -- (81.4000,156.1000) -- (81.4000,156.1000) -- (81.4000,156.1000) -- (81.4000,156.1000) -- (81.4000,156.1000) -- (81.4000,156.1000) -- (81.4000,156.1000) -- (81.4000,156.1000) -- (81.4000,156.1000) -- (81.4000,156.1000) -- (81.4000,156.1000) -- (81.4000,156.1000) -- (81.4000,156.0000) -- (81.4000,156.0000) -- (81.4000,156.0000) -- (81.4000,156.0000) -- (81.4000,156.0000) -- (81.4000,156.0000) -- (81.4000,156.0000) -- (81.4000,156.0000) -- (81.4000,156.0000) -- (81.4000,156.0000) -- (81.4000,156.0000) -- (81.4000,156.0000) -- (81.4000,156.0000) -- (81.4000,156.0000) -- (81.4000,156.0000) -- (81.4000,156.0000) -- (81.4000,156.0000) -- (81.4000,156.0000) -- (81.4000,156.0000) -- (81.5000,156.0000) -- (81.5000,156.0000) -- (81.5000,155.9000) -- (81.5000,155.9000) -- (81.5000,155.9000) -- (81.5000,155.9000) -- (81.5000,155.9000) -- (81.5000,155.9000) -- (81.5000,155.9000) -- (81.5000,155.9000) -- (81.5000,155.9000) -- (81.5000,155.9000) -- (81.5000,155.9000) -- (81.5000,155.9000) -- (81.5000,155.9000) -- (81.5000,155.9000) -- (81.5000,155.9000) -- (81.5000,155.9000) -- (81.5000,155.9000) -- (81.5000,155.9000) -- (81.5000,155.9000) -- (81.5000,155.9000) -- (81.5000,155.9000) -- (81.5000,155.9000) -- (81.5000,155.8000) -- (81.5000,155.8000) -- (81.5000,155.8000) -- (81.5000,155.8000) -- (81.5000,155.8000) -- (81.5000,155.8000) -- (81.5000,155.8000) -- (81.5000,155.8000) -- (81.5000,155.8000) -- (81.5000,155.8000) -- (81.5000,155.8000) -- (81.5000,155.8000) -- (81.5000,155.8000) -- (81.5000,155.8000) -- (81.5000,155.8000) -- (81.5000,155.8000) -- (81.5000,155.8000) -- (81.5000,155.8000) -- (81.5000,155.8000) -- (81.5000,155.8000) -- (81.5000,155.8000) -- (81.5000,155.7000) -- (81.5000,155.7000) -- (81.5000,155.7000) -- (81.5000,155.7000) -- (81.5000,155.7000) -- (81.6000,155.7000) -- (81.6000,155.7000) -- (81.6000,155.7000) -- (81.6000,155.7000) -- (81.6000,155.7000) -- (81.6000,155.7000) -- (81.6000,155.7000) -- (81.6000,155.7000) -- (81.6000,155.7000) -- (81.6000,155.7000) -- (81.6000,155.7000) -- (81.6000,155.7000) -- (81.6000,155.7000) -- (81.6000,155.7000) -- (81.6000,155.7000) -- (81.6000,155.7000) -- (81.6000,155.7000) -- (81.6000,155.6000) -- (81.6000,155.6000) -- (81.6000,155.6000) -- (81.6000,155.6000) -- (81.6000,155.6000) -- (81.6000,155.6000) -- (81.6000,155.6000) -- (81.6000,155.6000) -- (81.6000,155.6000) -- (81.6000,155.6000) -- (81.6000,155.6000) -- (81.6000,155.6000) -- (81.6000,155.6000) -- (81.6000,155.6000) -- (81.6000,155.6000) -- (81.6000,155.6000) -- (81.6000,155.6000) -- (81.6000,155.6000) -- (81.6000,155.6000) -- (81.6000,155.6000) -- (81.6000,155.6000) -- (81.6000,155.5000) -- (81.6000,155.5000) -- (81.6000,155.5000) -- (81.6000,155.5000) -- (81.6000,155.5000) -- (81.6000,155.5000) -- (81.6000,155.5000) -- (81.6000,155.5000) -- (81.6000,155.5000) -- (81.6000,155.5000) -- (81.6000,155.5000) -- (81.7000,155.5000) -- (81.7000,155.5000) -- (81.7000,155.5000) -- (81.7000,155.5000) -- (81.7000,155.5000) -- (81.7000,155.5000) -- (81.7000,155.5000) -- (81.7000,155.5000) -- (81.7000,155.5000) -- (81.7000,155.5000) -- (81.7000,155.5000) -- (81.7000,155.4000) -- (81.7000,155.4000) -- (81.7000,155.4000) -- (81.7000,155.4000) -- (81.7000,155.4000) -- (81.7000,155.4000) -- (81.7000,155.4000) -- (81.7000,155.4000) -- (81.7000,155.4000) -- (81.7000,155.4000) -- (81.7000,155.4000) -- (81.7000,155.4000) -- (81.7000,155.4000) -- (81.7000,155.4000) -- (81.7000,155.4000) -- (81.7000,155.4000) -- (81.7000,155.4000) -- (81.7000,155.4000) -- (81.7000,155.4000) -- (81.7000,155.4000) -- (81.7000,155.4000) -- (81.7000,155.3000) -- (81.7000,155.3000) -- (81.7000,155.3000) -- (81.7000,155.3000) -- (81.7000,155.3000) -- (81.7000,155.3000) -- (81.7000,155.3000) -- (81.7000,155.3000) -- (81.7000,155.3000) -- (81.7000,155.3000) -- (81.7000,155.3000) -- (81.7000,155.3000) -- (81.7000,155.3000) -- (81.7000,155.3000) -- (81.7000,155.3000) -- (81.7000,155.3000) -- (81.7000,155.3000) -- (81.7000,155.3000) -- (81.8000,155.3000) -- (81.8000,155.3000) -- (81.8000,155.3000) -- (81.8000,155.3000) -- (81.8000,155.2000) -- (81.8000,155.2000) -- (81.8000,155.2000) -- (81.8000,155.2000) -- (81.8000,155.2000) -- (81.8000,155.2000) -- (81.8000,155.2000) -- (81.8000,155.2000) -- (81.8000,155.2000) -- (81.8000,155.2000) -- (81.8000,155.2000) -- (81.8000,155.2000) -- (81.8000,155.2000) -- (81.8000,155.2000) -- (81.8000,155.2000) -- (81.8000,155.2000) -- (81.8000,155.2000) -- (81.8000,155.2000) -- (81.8000,155.2000) -- (81.8000,155.2000) -- (81.8000,155.2000) -- (81.8000,155.1000) -- (81.8000,155.1000) -- (81.8000,155.1000) -- (81.8000,155.1000) -- (81.8000,155.1000) -- (81.8000,155.1000) -- (81.8000,155.1000) -- (81.8000,155.1000) -- (81.8000,155.1000) -- (81.8000,155.1000) -- (81.8000,155.1000) -- (81.8000,155.1000) -- (81.8000,155.1000) -- (81.8000,155.1000) -- (81.8000,155.1000) -- (81.8000,155.1000) -- (81.8000,155.1000) -- (81.8000,155.1000) -- (81.8000,155.1000) -- (81.8000,155.1000) -- (81.8000,155.1000) -- (81.8000,155.1000) -- (81.8000,155.0000) -- (81.8000,155.0000) -- (81.9000,155.0000) -- (81.9000,155.0000) -- (81.9000,155.0000) -- (81.9000,155.0000) -- (81.9000,155.0000) -- (81.9000,155.0000) -- (81.9000,155.0000) -- (81.9000,155.0000) -- (81.9000,155.0000) -- (81.9000,155.0000) -- (81.9000,155.0000) -- (81.9000,155.0000) -- (81.9000,155.0000) -- (81.9000,155.0000) -- (81.9000,155.0000) -- (81.9000,155.0000) -- (81.9000,155.0000) -- (81.9000,155.0000) -- (81.9000,155.0000) -- (81.9000,154.9000) -- (81.9000,154.9000) -- (81.9000,154.9000) -- (81.9000,154.9000) -- (81.9000,154.9000) -- (81.9000,154.9000) -- (81.9000,154.9000) -- (81.9000,154.9000) -- (81.9000,154.9000) -- (81.9000,154.9000) -- (81.9000,154.9000) -- (81.9000,154.9000) -- (81.9000,154.9000) -- (81.9000,154.9000) -- (81.9000,154.9000) -- (81.9000,154.9000) -- (81.9000,154.9000) -- (81.9000,154.9000) -- (81.9000,154.9000) -- (81.9000,154.9000) -- (81.9000,154.9000) -- (81.9000,154.9000) -- (81.9000,154.8000) -- (81.9000,154.8000) -- (81.9000,154.8000) -- (81.9000,154.8000) -- (81.9000,154.8000) -- (81.9000,154.8000) -- (81.9000,154.8000) -- (81.9000,154.8000) -- (81.9000,154.8000) -- (82.0000,154.8000) -- (82.0000,154.8000) -- (82.0000,154.8000) -- (82.0000,154.8000) -- (82.0000,154.8000) -- (82.0000,154.8000) -- (82.0000,154.8000) -- (82.0000,154.8000) -- (82.0000,154.8000) -- (82.0000,154.8000) -- (82.0000,154.8000) -- (82.0000,154.8000) -- (82.0000,154.7000) -- (82.0000,154.7000) -- (82.0000,154.7000) -- (82.0000,154.7000) -- (82.0000,154.7000) -- (82.0000,154.7000) -- (82.0000,154.7000) -- (82.0000,154.7000) -- (82.0000,154.7000) -- (82.0000,154.7000) -- (82.0000,154.7000) -- (82.0000,154.7000) -- (82.0000,154.7000) -- (82.0000,154.7000) -- (82.0000,154.7000) -- (82.0000,154.7000) -- (82.0000,154.7000) -- (82.0000,154.7000) -- (82.0000,154.7000) -- (82.0000,154.7000) -- (82.0000,154.7000) -- (82.0000,154.7000) -- (82.0000,154.6000) -- (82.0000,154.6000) -- (82.0000,154.6000) -- (82.0000,154.6000) -- (82.0000,154.6000) -- (82.0000,154.6000) -- (82.0000,154.6000) -- (82.0000,154.6000) -- (82.0000,154.6000) -- (82.0000,154.6000) -- (82.0000,154.6000) -- (82.0000,154.6000) -- (82.0000,154.6000) -- (82.0000,154.6000) -- (82.0000,154.6000) -- (82.1000,154.6000) -- (82.1000,154.6000) -- (82.1000,154.6000) -- (82.1000,154.6000) -- (82.1000,154.6000) -- (82.1000,154.6000) -- (82.1000,154.5000) -- (82.1000,154.5000) -- (82.1000,154.5000) -- (82.1000,154.5000) -- (82.1000,154.5000) -- (82.1000,154.5000) -- (82.1000,154.5000) -- (82.1000,154.5000) -- (82.1000,154.5000) -- (82.1000,154.5000) -- (82.1000,154.5000) -- (82.1000,154.5000) -- (82.1000,154.5000) -- (82.1000,154.5000) -- (82.1000,154.5000) -- (82.1000,154.5000) -- (82.1000,154.5000) -- (82.1000,154.5000) -- (82.1000,154.5000) -- (82.1000,154.5000) -- (82.1000,154.5000) -- (82.1000,154.5000) -- (82.1000,154.4000) -- (82.1000,154.4000) -- (82.1000,154.4000) -- (82.1000,154.4000) -- (82.1000,154.4000) -- (82.1000,154.4000) -- (82.1000,154.4000) -- (82.1000,154.4000) -- (82.1000,154.4000) -- (82.1000,154.4000) -- (82.1000,154.4000) -- (82.1000,154.4000) -- (82.1000,154.4000) -- (82.1000,154.4000) -- (82.1000,154.4000) -- (82.1000,154.4000) -- (82.1000,154.4000) -- (82.1000,154.4000) -- (82.1000,154.4000) -- (82.1000,154.4000) -- (82.1000,154.4000) -- (82.1000,154.3000) -- (82.2000,154.3000) -- (82.2000,154.3000) -- (82.2000,154.3000) -- (82.2000,154.3000) -- (82.2000,154.3000) -- (82.2000,154.3000) -- (82.2000,154.3000) -- (82.2000,154.3000) -- (82.2000,154.3000) -- (82.2000,154.3000) -- (82.2000,154.3000) -- (82.2000,154.3000) -- (82.2000,154.3000) -- (82.2000,154.3000) -- (82.2000,154.3000) -- (82.2000,154.3000) -- (82.2000,154.3000) -- (82.2000,154.3000) -- (82.2000,154.3000) -- (82.2000,154.3000) -- (82.2000,154.3000) -- (82.2000,154.2000) -- (82.2000,154.2000) -- (82.2000,154.2000) -- (82.2000,154.2000) -- (82.2000,154.2000) -- (82.2000,154.2000) -- (82.2000,154.2000) -- (82.2000,154.2000) -- (82.2000,154.2000) -- (82.2000,154.2000) -- (82.2000,154.2000) -- (82.2000,154.2000) -- (82.2000,154.2000) -- (82.2000,154.2000) -- (82.2000,154.2000) -- (82.2000,154.2000) -- (82.2000,154.2000) -- (82.2000,154.2000) -- (82.2000,154.2000) -- (82.2000,154.2000) -- (82.2000,154.2000) -- (82.2000,154.1000) -- (82.2000,154.1000) -- (82.2000,154.1000) -- (82.2000,154.1000) -- (82.2000,154.1000) -- (82.2000,154.1000) -- (82.2000,154.1000) -- (82.3000,154.1000) -- (82.3000,154.1000) -- (82.3000,154.1000) -- (82.3000,154.1000) -- (82.3000,154.1000) -- (82.3000,154.1000) -- (82.3000,154.1000) -- (82.3000,154.1000) -- (82.3000,154.1000) -- (82.3000,154.1000) -- (82.3000,154.1000) -- (82.3000,154.1000) -- (82.3000,154.1000) -- (82.3000,154.1000) -- (82.3000,154.1000) -- (82.3000,154.0000) -- (82.3000,154.0000) -- (82.3000,154.0000) -- (82.3000,154.0000) -- (82.3000,154.0000) -- (82.3000,154.0000) -- (82.3000,154.0000) -- (82.3000,154.0000) -- (82.3000,154.0000) -- (82.3000,154.0000) -- (82.3000,154.0000) -- (82.3000,154.0000) -- (82.3000,154.0000) -- (82.3000,154.0000) -- (82.3000,154.0000) -- (82.3000,154.0000) -- (82.3000,154.0000) -- (82.3000,154.0000) -- (82.3000,154.0000) -- (82.3000,154.0000) -- (82.3000,154.0000) -- (82.3000,153.9000) -- (82.3000,153.9000) -- (82.3000,153.9000) -- (82.3000,153.9000) -- (82.3000,153.9000) -- (82.3000,153.9000) -- (82.3000,153.9000) -- (82.3000,153.9000) -- (82.3000,153.9000) -- (82.3000,153.9000) -- (82.3000,153.9000) -- (82.3000,153.9000) -- (82.3000,153.9000) -- (82.3000,153.9000) -- (82.4000,153.9000) -- (82.4000,153.9000) -- (82.4000,153.9000) -- (82.4000,153.9000) -- (82.4000,153.9000) -- (82.4000,153.9000) -- (82.4000,153.9000) -- (82.4000,153.9000) -- (82.4000,153.8000) -- (82.4000,153.8000) -- (82.4000,153.8000) -- (82.4000,153.8000) -- (82.4000,153.8000) -- (82.4000,153.8000) -- (82.4000,153.8000) -- (82.4000,153.8000) -- (82.4000,153.8000) -- (82.4000,153.8000) -- (82.4000,153.8000) -- (82.4000,153.8000) -- (82.4000,153.8000) -- (82.4000,153.8000) -- (82.4000,153.8000) -- (82.4000,153.8000) -- (82.4000,153.8000) -- (82.4000,153.8000) -- (82.4000,153.8000) -- (82.4000,153.8000) -- (82.4000,153.8000) -- (82.4000,153.7000) -- (82.4000,153.7000) -- (82.4000,153.7000) -- (82.4000,153.7000) -- (82.4000,153.7000) -- (82.4000,153.7000) -- (82.4000,153.7000) -- (82.4000,153.7000) -- (82.4000,153.7000) -- (82.4000,153.7000) -- (82.4000,153.7000) -- (82.4000,153.7000) -- (82.4000,153.7000) -- (82.4000,153.7000) -- (82.4000,153.7000) -- (82.4000,153.7000) -- (82.4000,153.7000) -- (82.4000,153.7000) -- (82.4000,153.7000) -- (82.4000,153.7000) -- (82.5000,153.7000) -- (82.5000,153.7000) -- (82.5000,153.6000) -- (82.5000,153.6000) -- (82.5000,153.6000) -- (82.5000,153.6000) -- (82.5000,153.6000) -- (82.5000,153.6000) -- (82.5000,153.6000) -- (82.5000,153.6000) -- (82.5000,153.6000) -- (82.5000,153.6000) -- (82.5000,153.6000) -- (82.5000,153.6000) -- (82.5000,153.6000) -- (82.5000,153.6000) -- (82.5000,153.6000) -- (82.5000,153.6000) -- (82.5000,153.6000) -- (82.5000,153.6000) -- (82.5000,153.6000) -- (82.5000,153.6000) -- (82.5000,153.6000) -- (82.5000,153.5000) -- (82.5000,153.5000) -- (82.5000,153.5000) -- (82.5000,153.5000) -- (82.5000,153.5000) -- (82.5000,153.5000) -- (82.5000,153.5000) -- (82.5000,153.5000) -- (82.5000,153.5000) -- (82.5000,153.5000) -- (82.5000,153.5000) -- (82.5000,153.5000) -- (82.5000,153.5000) -- (82.5000,153.5000) -- (82.5000,153.5000) -- (82.5000,153.5000) -- (82.5000,153.5000) -- (82.5000,153.5000) -- (82.5000,153.5000) -- (82.5000,153.5000) -- (82.5000,153.5000) -- (82.5000,153.5000) -- (82.5000,153.4000) -- (82.5000,153.4000) -- (82.5000,153.4000) -- (82.5000,153.4000) -- (82.5000,153.4000) -- (82.6000,153.4000) -- (82.6000,153.4000) -- (82.6000,153.4000) -- (82.6000,153.4000) -- (82.6000,153.4000) -- (82.6000,153.4000) -- (82.6000,153.4000) -- (82.6000,153.4000) -- (82.6000,153.4000) -- (82.6000,153.4000) -- (82.6000,153.4000) -- (82.6000,153.4000) -- (82.6000,153.4000) -- (82.6000,153.4000) -- (82.6000,153.4000) -- (82.6000,153.4000) -- (82.6000,153.3000) -- (82.6000,153.3000) -- (82.6000,153.3000) -- (82.6000,153.3000) -- (82.6000,153.3000) -- (82.6000,153.3000) -- (82.6000,153.3000) -- (82.6000,153.3000) -- (82.6000,153.3000) -- (82.6000,153.3000) -- (82.6000,153.3000) -- (82.6000,153.3000) -- (82.6000,153.3000) -- (82.6000,153.3000) -- (82.6000,153.3000) -- (82.6000,153.3000) -- (82.6000,153.3000) -- (82.6000,153.3000) -- (82.6000,153.3000) -- (82.6000,153.3000) -- (82.6000,153.3000) -- (82.6000,153.3000) -- (82.6000,153.2000) -- (82.6000,153.2000) -- (82.6000,153.2000) -- (82.6000,153.2000) -- (82.6000,153.2000) -- (82.6000,153.2000) -- (82.6000,153.2000) -- (82.6000,153.2000) -- (82.6000,153.2000) -- (82.6000,153.2000) -- (82.6000,153.2000) -- (82.7000,153.2000) -- (82.7000,153.2000) -- (82.7000,153.2000) -- (82.7000,153.2000) -- (82.7000,153.2000) -- (82.7000,153.2000) -- (82.7000,153.2000) -- (82.7000,153.2000) -- (82.7000,153.2000) -- (82.7000,153.2000) -- (82.7000,153.1000) -- (82.7000,153.1000) -- (82.7000,153.1000) -- (82.7000,153.1000) -- (82.7000,153.1000) -- (82.7000,153.1000) -- (82.7000,153.1000) -- (82.7000,153.1000) -- (82.7000,153.1000) -- (82.7000,153.1000) -- (82.7000,153.1000) -- (82.7000,153.1000) -- (82.7000,153.1000) -- (82.7000,153.1000) -- (82.7000,153.1000) -- (82.7000,153.1000) -- (82.7000,153.1000) -- (82.7000,153.1000) -- (82.7000,153.1000) -- (82.7000,153.1000) -- (82.7000,153.1000) -- (82.7000,153.1000) -- (82.7000,153.0000) -- (82.7000,153.0000) -- (82.7000,153.0000) -- (82.7000,153.0000) -- (82.7000,153.0000) -- (82.7000,153.0000) -- (82.7000,153.0000) -- (82.7000,153.0000) -- (82.7000,153.0000) -- (82.7000,153.0000) -- (82.7000,153.0000) -- (82.7000,153.0000) -- (82.7000,153.0000) -- (82.7000,153.0000) -- (82.7000,153.0000) -- (82.7000,153.0000) -- (82.7000,153.0000) -- (82.7000,153.0000) -- (82.8000,153.0000) -- (82.8000,153.0000) -- (82.8000,153.0000) -- (82.8000,152.9000) -- (82.8000,152.9000) -- (82.8000,152.9000) -- (82.8000,152.9000) -- (82.8000,152.9000) -- (82.8000,152.9000) -- (82.8000,152.9000) -- (82.8000,152.9000) -- (82.8000,152.9000) -- (82.8000,152.9000) -- (82.8000,152.9000) -- (82.8000,152.9000) -- (82.8000,152.9000) -- (82.8000,152.9000) -- (82.8000,152.9000) -- (82.8000,152.9000) -- (82.8000,152.9000) -- (82.8000,152.9000) -- (82.8000,152.9000) -- (82.8000,152.9000) -- (82.8000,152.9000) -- (82.8000,152.9000) -- (82.8000,152.8000) -- (82.8000,152.8000) -- (82.8000,152.8000) -- (82.8000,152.8000) -- (82.8000,152.8000) -- (82.8000,152.8000) -- (82.8000,152.8000) -- (82.8000,152.8000) -- (82.8000,152.8000) -- (82.8000,152.8000) -- (82.8000,152.8000) -- (82.8000,152.8000) -- (82.8000,152.8000) -- (82.8000,152.8000) -- (82.8000,152.8000) -- (82.8000,152.8000) -- (82.8000,152.8000) -- (82.8000,152.8000) -- (82.8000,152.8000) -- (82.8000,152.8000) -- (82.8000,152.8000) -- (82.8000,152.7000) -- (82.8000,152.7000) -- (82.8000,152.7000) -- (82.9000,152.7000) -- (82.9000,152.7000) -- (82.9000,152.7000) -- (82.9000,152.7000) -- (82.9000,152.7000) -- (82.9000,152.7000) -- (82.9000,152.7000) -- (82.9000,152.7000) -- (82.9000,152.7000) -- (82.9000,152.7000) -- (82.9000,152.7000) -- (82.9000,152.7000) -- (82.9000,152.7000) -- (82.9000,152.7000) -- (82.9000,152.7000) -- (82.9000,152.7000) -- (82.9000,152.7000) -- (82.9000,152.7000) -- (82.9000,152.7000) -- (82.9000,152.6000) -- (82.9000,152.6000) -- (82.9000,152.6000) -- (82.9000,152.6000) -- (82.9000,152.6000) -- (82.9000,152.6000) -- (82.9000,152.6000) -- (82.9000,152.6000) -- (82.9000,152.6000) -- (82.9000,152.6000) -- (82.9000,152.6000) -- (82.9000,152.6000) -- (82.9000,152.6000) -- (82.9000,152.6000) -- (82.9000,152.6000) -- (82.9000,152.6000) -- (82.9000,152.6000) -- (82.9000,152.6000) -- (82.9000,152.6000) -- (82.9000,152.6000) -- (82.9000,152.6000) -- (82.9000,152.5000) -- (82.9000,152.5000) -- (82.9000,152.5000) -- (82.9000,152.5000) -- (82.9000,152.5000) -- (82.9000,152.5000) -- (82.9000,152.5000) -- (82.9000,152.5000) -- (82.9000,152.5000) -- (82.9000,152.5000) -- (83.0000,152.5000) -- (83.0000,152.5000) -- (83.0000,152.5000) -- (83.0000,152.5000) -- (83.0000,152.5000) -- (83.0000,152.5000) -- (83.0000,152.5000) -- (83.0000,152.5000) -- (83.0000,152.5000) -- (83.0000,152.5000) -- (83.0000,152.5000) -- (83.0000,152.5000) -- (83.0000,152.4000) -- (83.0000,152.4000) -- (83.0000,152.4000) -- (83.0000,152.4000) -- (83.0000,152.4000) -- (83.0000,152.4000) -- (83.0000,152.4000) -- (83.0000,152.4000) -- (83.0000,152.4000) -- (83.0000,152.4000) -- (83.0000,152.4000) -- (83.0000,152.4000) -- (83.0000,152.4000) -- (83.0000,152.4000) -- (83.0000,152.4000) -- (83.0000,152.4000) -- (83.0000,152.4000) -- (83.0000,152.4000) -- (83.0000,152.4000) -- (83.0000,152.4000) -- (83.0000,152.4000) -- (83.0000,152.3000) -- (83.0000,152.3000) -- (83.0000,152.3000) -- (83.0000,152.3000) -- (83.0000,152.3000) -- (83.0000,152.3000) -- (83.0000,152.3000) -- (83.0000,152.3000) -- (83.0000,152.3000) -- (83.0000,152.3000) -- (83.0000,152.3000) -- (83.0000,152.3000) -- (83.0000,152.3000) -- (83.0000,152.3000) -- (83.0000,152.3000) -- (83.0000,152.3000) -- (83.1000,152.3000) -- (83.1000,152.3000) -- (83.1000,152.3000) -- (83.1000,152.3000) -- (83.1000,152.3000) -- (83.1000,152.3000) -- (83.1000,152.2000) -- (83.1000,152.2000) -- (83.1000,152.2000) -- (83.1000,152.2000) -- (83.1000,152.2000) -- (83.1000,152.2000) -- (83.1000,152.2000) -- (83.1000,152.2000) -- (83.1000,152.2000) -- (83.1000,152.2000) -- (83.1000,152.2000) -- (83.1000,152.2000) -- (83.1000,152.2000) -- (83.1000,152.2000) -- (83.1000,152.2000) -- (83.1000,152.2000) -- (83.1000,152.2000) -- (83.1000,152.2000) -- (83.1000,152.2000) -- (83.1000,152.2000) -- (83.1000,152.2000) -- (83.1000,152.2000) -- (83.1000,152.1000) -- (83.1000,152.1000) -- (83.1000,152.1000) -- (83.1000,152.1000) -- (83.1000,152.1000) -- (83.1000,152.1000) -- (83.1000,152.1000) -- (83.1000,152.1000) -- (83.1000,152.1000) -- (83.1000,152.1000) -- (83.1000,152.1000) -- (83.1000,152.1000) -- (83.1000,152.1000) -- (83.1000,152.1000) -- (83.1000,152.1000) -- (83.1000,152.1000) -- (83.1000,152.1000) -- (83.1000,152.1000) -- (83.1000,152.1000) -- (83.1000,152.1000) -- (83.1000,152.1000) -- (83.1000,152.0000) -- (83.2000,152.0000) -- (83.2000,152.0000) -- (83.2000,152.0000) -- (83.2000,152.0000) -- (83.2000,152.0000) -- (83.2000,152.0000) -- (83.2000,152.0000) -- (83.2000,152.0000) -- (83.2000,152.0000) -- (83.2000,152.0000) -- (83.2000,152.0000) -- (83.2000,152.0000) -- (83.2000,152.0000) -- (83.2000,152.0000) -- (83.2000,152.0000) -- (83.2000,152.0000) -- (83.2000,152.0000) -- (83.2000,152.0000) -- (83.2000,152.0000) -- (83.2000,152.0000) -- (83.2000,152.0000) -- (83.2000,151.9000) -- (83.2000,151.9000) -- (83.2000,151.9000) -- (83.2000,151.9000) -- (83.2000,151.9000) -- (83.2000,151.9000) -- (83.2000,151.9000) -- (83.2000,151.9000) -- (83.2000,151.9000) -- (83.2000,151.9000) -- (83.2000,151.9000) -- (83.2000,151.9000) -- (83.2000,151.9000) -- (83.2000,151.9000) -- (83.2000,151.9000) -- (83.2000,151.9000) -- (83.2000,151.9000) -- (83.2000,151.9000) -- (83.2000,151.9000) -- (83.2000,151.9000) -- (83.2000,151.9000) -- (83.2000,151.8000) -- (83.2000,151.8000) -- (83.2000,151.8000) -- (83.2000,151.8000) -- (83.2000,151.8000) -- (83.2000,151.8000) -- (83.2000,151.8000) -- (83.3000,151.8000) -- (83.3000,151.8000) -- (83.3000,151.8000) -- (83.3000,151.8000) -- (83.3000,151.8000) -- (83.3000,151.8000) -- (83.3000,151.8000) -- (83.3000,151.8000) -- (83.3000,151.8000) -- (83.3000,151.8000) -- (83.3000,151.8000) -- (83.3000,151.8000) -- (83.3000,151.8000) -- (83.3000,151.8000) -- (83.3000,151.8000) -- (83.3000,151.7000) -- (83.3000,151.7000) -- (83.3000,151.7000) -- (83.3000,151.7000) -- (83.3000,151.7000) -- (83.3000,151.7000) -- (83.3000,151.7000) -- (83.3000,151.7000) -- (83.3000,151.7000) -- (83.3000,151.7000) -- (83.3000,151.7000) -- (83.3000,151.7000) -- (83.3000,151.7000) -- (83.3000,151.7000) -- (83.3000,151.7000) -- (83.3000,151.7000) -- (83.3000,151.7000) -- (83.3000,151.7000) -- (83.3000,151.7000) -- (83.3000,151.7000) -- (83.3000,151.7000) -- (83.3000,151.6000) -- (83.3000,151.6000) -- (83.3000,151.6000) -- (83.3000,151.6000) -- (83.3000,151.6000) -- (83.3000,151.6000) -- (83.3000,151.6000) -- (83.3000,151.6000) -- (83.3000,151.6000) -- (83.3000,151.6000) -- (83.3000,151.6000) -- (83.3000,151.6000) -- (83.3000,151.6000) -- (83.3000,151.6000) -- (83.4000,151.6000) -- (83.4000,151.6000) -- (83.4000,151.6000) -- (83.4000,151.6000) -- (83.4000,151.6000) -- (83.4000,151.6000) -- (83.4000,151.6000) -- (83.4000,151.6000) -- (83.4000,151.5000) -- (83.4000,151.5000) -- (83.4000,151.5000) -- (83.4000,151.5000) -- (83.4000,151.5000) -- (83.4000,151.5000) -- (83.4000,151.5000) -- (83.4000,151.5000) -- (83.4000,151.5000) -- (83.4000,151.5000) -- (83.4000,151.5000) -- (83.4000,151.5000) -- (83.4000,151.5000) -- (83.4000,151.5000) -- (83.4000,151.5000) -- (83.4000,151.5000) -- (83.4000,151.5000) -- (83.4000,151.5000) -- (83.4000,151.5000) -- (83.4000,151.5000) -- (83.4000,151.5000) -- (83.4000,151.4000) -- (83.4000,151.4000) -- (83.4000,151.4000) -- (83.4000,151.4000) -- (83.4000,151.4000) -- (83.4000,151.4000) -- (83.4000,151.4000) -- (83.4000,151.4000) -- (83.4000,151.4000) -- (83.4000,151.4000) -- (83.4000,151.4000) -- (83.4000,151.4000) -- (83.4000,151.4000) -- (83.4000,151.4000) -- (83.4000,151.4000) -- (83.4000,151.4000) -- (83.4000,151.4000) -- (83.4000,151.4000) -- (83.4000,151.4000) -- (83.4000,151.4000) -- (83.5000,151.4000) -- (83.5000,151.4000) -- (83.5000,151.3000) -- (83.5000,151.3000) -- (83.5000,151.3000) -- (83.5000,151.3000) -- (83.5000,151.3000) -- (83.5000,151.3000) -- (83.5000,151.3000) -- (83.5000,151.3000) -- (83.5000,151.3000) -- (83.5000,151.3000) -- (83.5000,151.3000) -- (83.5000,151.3000) -- (83.5000,151.3000) -- (83.5000,151.3000) -- (83.5000,151.3000) -- (83.5000,151.3000) -- (83.5000,151.3000) -- (83.5000,151.3000) -- (83.5000,151.3000) -- (83.5000,151.3000) -- (83.5000,151.3000) -- (83.5000,151.2000) -- (83.5000,151.2000) -- (83.5000,151.2000) -- (83.5000,151.2000) -- (83.5000,151.2000) -- (83.5000,151.2000) -- (83.5000,151.2000) -- (83.5000,151.2000) -- (83.5000,151.2000) -- (83.5000,151.2000) -- (83.5000,151.2000) -- (83.5000,151.2000) -- (83.5000,151.2000) -- (83.5000,151.2000) -- (83.5000,151.2000) -- (83.5000,151.2000) -- (83.5000,151.2000) -- (83.5000,151.2000) -- (83.5000,151.2000) -- (83.5000,151.2000) -- (83.5000,151.2000) -- (83.5000,151.2000) -- (83.5000,151.1000) -- (83.5000,151.1000) -- (83.5000,151.1000) -- (83.5000,151.1000) -- (83.5000,151.1000) -- (83.6000,151.1000) -- (83.6000,151.1000) -- (83.6000,151.1000) -- (83.6000,151.1000) -- (83.6000,151.1000) -- (83.6000,151.1000) -- (83.6000,151.1000) -- (83.6000,151.1000) -- (83.6000,151.1000) -- (83.6000,151.1000) -- (83.6000,151.1000) -- (83.6000,151.1000) -- (83.6000,151.1000) -- (83.6000,151.1000) -- (83.6000,151.1000) -- (83.6000,151.1000) -- (83.6000,151.0000) -- (83.6000,151.0000) -- (83.6000,151.0000) -- (83.6000,151.0000) -- (83.6000,151.0000) -- (83.6000,151.0000) -- (83.6000,151.0000) -- (83.6000,151.0000) -- (83.6000,151.0000) -- (83.6000,151.0000) -- (83.6000,151.0000) -- (83.6000,151.0000) -- (83.6000,151.0000) -- (83.6000,151.0000) -- (83.6000,151.0000) -- (83.6000,151.0000) -- (83.6000,151.0000) -- (83.6000,151.0000) -- (83.6000,151.0000) -- (83.6000,151.0000) -- (83.6000,151.0000) -- (83.6000,151.0000) -- (83.6000,150.9000) -- (83.6000,150.9000) -- (83.6000,150.9000) -- (83.6000,150.9000) -- (83.6000,150.9000) -- (83.6000,150.9000) -- (83.6000,150.9000) -- (83.6000,150.9000) -- (83.6000,150.9000) -- (83.6000,150.9000) -- (83.6000,150.9000) -- (83.7000,150.9000) -- (83.7000,150.9000) -- (83.7000,150.9000) -- (83.7000,150.9000) -- (83.7000,150.9000) -- (83.7000,150.9000) -- (83.7000,150.9000) -- (83.7000,150.9000) -- (83.7000,150.9000) -- (83.7000,150.9000) -- (83.7000,150.8000) -- (83.7000,150.8000) -- (83.7000,150.8000) -- (83.7000,150.8000) -- (83.7000,150.8000) -- (83.7000,150.8000) -- (83.7000,150.8000) -- (83.7000,150.8000) -- (83.7000,150.8000) -- (83.7000,150.8000) -- (83.7000,150.8000) -- (83.7000,150.8000) -- (83.7000,150.8000) -- (83.7000,150.8000) -- (83.7000,150.8000) -- (83.7000,150.8000) -- (83.7000,150.8000) -- (83.7000,150.8000) -- (83.7000,150.8000) -- (83.7000,150.8000) -- (83.7000,150.8000) -- (83.7000,150.8000) -- (83.7000,150.7000) -- (83.7000,150.7000) -- (83.7000,150.7000) -- (83.7000,150.7000) -- (83.7000,150.7000) -- (83.7000,150.7000) -- (83.7000,150.7000) -- (83.7000,150.7000) -- (83.7000,150.7000) -- (83.7000,150.7000) -- (83.7000,150.7000) -- (83.7000,150.7000) -- (83.7000,150.7000) -- (83.7000,150.7000) -- (83.7000,150.7000) -- (83.7000,150.7000) -- (83.7000,150.7000) -- (83.7000,150.7000) -- (83.8000,150.7000) -- (83.8000,150.7000) -- (83.8000,150.7000) -- (83.8000,150.6000) -- (83.8000,150.6000) -- (83.8000,150.6000) -- (83.8000,150.6000) -- (83.8000,150.6000) -- (83.8000,150.6000) -- (83.8000,150.6000) -- (83.8000,150.6000) -- (83.8000,150.6000) -- (83.8000,150.6000) -- (83.8000,150.6000) -- (83.8000,150.6000) -- (83.8000,150.6000) -- (83.8000,150.6000) -- (83.8000,150.6000) -- (83.8000,150.6000) -- (83.8000,150.6000) -- (83.8000,150.6000) -- (83.8000,150.6000) -- (83.8000,150.6000) -- (83.8000,150.6000) -- (83.8000,150.6000) -- (83.8000,150.5000) -- (83.8000,150.5000) -- (83.8000,150.5000) -- (83.8000,150.5000) -- (83.8000,150.5000) -- (83.8000,150.5000) -- (83.8000,150.5000) -- (83.8000,150.5000) -- (83.8000,150.5000) -- (83.8000,150.5000) -- (83.8000,150.5000) -- (83.8000,150.5000) -- (83.8000,150.5000) -- (83.8000,150.5000) -- (83.8000,150.5000) -- (83.8000,150.5000) -- (83.8000,150.5000) -- (83.8000,150.5000) -- (83.8000,150.5000) -- (83.8000,150.5000) -- (83.8000,150.5000) -- (83.8000,150.4000) -- (83.8000,150.4000) -- (83.8000,150.4000) -- (83.9000,150.4000) -- (83.9000,150.4000) -- (83.9000,150.4000) -- (83.9000,150.4000) -- (83.9000,150.4000) -- (83.9000,150.4000) -- (83.9000,150.4000) -- (83.9000,150.4000) -- (83.9000,150.4000) -- (83.9000,150.4000) -- (83.9000,150.4000) -- (83.9000,150.4000) -- (83.9000,150.4000) -- (83.9000,150.4000) -- (83.9000,150.4000) -- (83.9000,150.4000) -- (83.9000,150.4000) -- (83.9000,150.4000) -- (83.9000,150.4000) -- (83.9000,150.3000) -- (83.9000,150.3000) -- (83.9000,150.3000) -- (83.9000,150.3000) -- (83.9000,150.3000) -- (83.9000,150.3000) -- (83.9000,150.3000) -- (83.9000,150.3000) -- (83.9000,150.3000) -- (83.9000,150.3000) -- (83.9000,150.3000) -- (83.9000,150.3000) -- (83.9000,150.3000) -- (83.9000,150.3000) -- (83.9000,150.3000) -- (83.9000,150.3000) -- (83.9000,150.3000) -- (83.9000,150.3000) -- (83.9000,150.3000) -- (83.9000,150.3000) -- (83.9000,150.3000) -- (83.9000,150.2000) -- (83.9000,150.2000) -- (83.9000,150.2000) -- (83.9000,150.2000) -- (83.9000,150.2000) -- (83.9000,150.2000) -- (83.9000,150.2000) -- (83.9000,150.2000) -- (83.9000,150.2000) -- (83.9000,150.2000) -- (84.0000,150.2000) -- (84.0000,150.2000) -- (84.0000,150.2000) -- (84.0000,150.2000) -- (84.0000,150.2000) -- (84.0000,150.2000) -- (84.0000,150.2000) -- (84.0000,150.2000) -- (84.0000,150.2000) -- (84.0000,150.2000) -- (84.0000,150.2000) -- (84.0000,150.2000) -- (84.0000,150.1000) -- (84.0000,150.1000) -- (84.0000,150.1000) -- (84.0000,150.1000) -- (84.0000,150.1000) -- (84.0000,150.1000) -- (84.0000,150.1000) -- (84.0000,150.1000) -- (84.0000,150.1000) -- (84.0000,150.1000) -- (84.0000,150.1000) -- (84.0000,150.1000) -- (84.0000,150.1000) -- (84.0000,150.1000) -- (84.0000,150.1000) -- (84.0000,150.1000) -- (84.0000,150.1000) -- (84.0000,150.1000) -- (84.0000,150.1000) -- (84.0000,150.1000) -- (84.0000,150.1000) -- (84.0000,150.0000) -- (84.0000,150.0000) -- (84.0000,150.0000) -- (84.0000,150.0000) -- (84.0000,150.0000) -- (84.0000,150.0000) -- (84.0000,150.0000) -- (84.0000,150.0000) -- (84.0000,150.0000) -- (84.0000,150.0000) -- (84.0000,150.0000) -- (84.0000,150.0000) -- (84.0000,150.0000) -- (84.0000,150.0000) -- (84.0000,150.0000) -- (84.0000,150.0000) -- (84.1000,150.0000) -- (84.1000,150.0000) -- (84.1000,150.0000) -- (84.1000,150.0000) -- (84.1000,150.0000) -- (84.1000,150.0000) -- (84.1000,149.9000) -- (84.1000,149.9000) -- (84.1000,149.9000) -- (84.1000,149.9000) -- (84.1000,149.9000) -- (84.1000,149.9000) -- (84.1000,149.9000) -- (84.1000,149.9000) -- (84.1000,149.9000) -- (84.1000,149.9000) -- (84.1000,149.9000) -- (84.1000,149.9000) -- (84.1000,149.9000) -- (84.1000,149.9000) -- (84.1000,149.9000) -- (84.1000,149.9000) -- (84.1000,149.9000) -- (84.1000,149.9000) -- (84.1000,149.9000) -- (84.1000,149.9000) -- (84.1000,149.9000) -- (84.1000,149.8000) -- (84.1000,149.8000) -- (84.1000,149.8000) -- (84.1000,149.8000) -- (84.1000,149.8000) -- (84.1000,149.8000) -- (84.1000,149.8000) -- (84.1000,149.8000) -- (84.1000,149.8000) -- (84.1000,149.8000) -- (84.1000,149.8000) -- (84.1000,149.8000) -- (84.1000,149.8000) -- (84.1000,149.8000) -- (84.1000,149.8000) -- (84.1000,149.8000) -- (84.1000,149.8000) -- (84.1000,149.8000) -- (84.1000,149.8000) -- (84.1000,149.8000) -- (84.1000,149.8000) -- (84.1000,149.8000) -- (84.1000,149.7000) -- (84.2000,149.7000) -- (84.2000,149.7000) -- (84.2000,149.7000) -- (84.2000,149.7000) -- (84.2000,149.7000) -- (84.2000,149.7000) -- (84.2000,149.7000) -- (84.2000,149.7000) -- (84.2000,149.7000) -- (84.2000,149.7000) -- (84.2000,149.7000) -- (84.2000,149.7000) -- (84.2000,149.7000) -- (84.2000,149.7000) -- (84.2000,149.7000) -- (84.2000,149.7000) -- (84.2000,149.7000) -- (84.2000,149.7000) -- (84.2000,149.7000) -- (84.2000,149.7000) -- (84.2000,149.6000) -- (84.2000,149.6000) -- (84.2000,149.6000) -- (84.2000,149.6000) -- (84.2000,149.6000) -- (84.2000,149.6000) -- (84.2000,149.6000) -- (84.2000,149.6000) -- (84.2000,149.6000) -- (84.2000,149.6000) -- (84.2000,149.6000) -- (84.2000,149.6000) -- (84.2000,149.6000) -- (84.2000,149.6000) -- (84.2000,149.6000) -- (84.2000,149.6000) -- (84.2000,149.6000) -- (84.2000,149.6000) -- (84.2000,149.6000) -- (84.2000,149.6000) -- (84.2000,149.6000) -- (84.2000,149.6000) -- (84.2000,149.5000) -- (84.2000,149.5000) -- (84.2000,149.5000) -- (84.2000,149.5000) -- (84.2000,149.5000) -- (84.2000,149.5000) -- (84.2000,149.5000) -- (84.3000,149.5000) -- (84.3000,149.5000) -- (84.3000,149.5000) -- (84.3000,149.5000) -- (84.3000,149.5000) -- (84.3000,149.5000) -- (84.3000,149.5000) -- (84.3000,149.5000) -- (84.3000,149.5000) -- (84.3000,149.5000) -- (84.3000,149.5000) -- (84.3000,149.5000) -- (84.3000,149.5000) -- (84.3000,149.5000) -- (84.3000,149.4000) -- (84.3000,149.4000) -- (84.3000,149.4000) -- (84.3000,149.4000) -- (84.3000,149.4000) -- (84.3000,149.4000) -- (84.3000,149.4000) -- (84.3000,149.4000) -- (84.3000,149.4000) -- (84.3000,149.4000) -- (84.3000,149.4000) -- (84.3000,149.4000) -- (84.3000,149.4000) -- (84.3000,149.4000) -- (84.3000,149.4000) -- (84.3000,149.4000) -- (84.3000,149.4000) -- (84.3000,149.4000) -- (84.3000,149.4000) -- (84.3000,149.4000) -- (84.3000,149.4000) -- (84.3000,149.4000) -- (84.3000,149.3000) -- (84.3000,149.3000) -- (84.3000,149.3000) -- (84.3000,149.3000) -- (84.3000,149.3000) -- (84.3000,149.3000) -- (84.3000,149.3000) -- (84.3000,149.3000) -- (84.3000,149.3000) -- (84.3000,149.3000) -- (84.3000,149.3000) -- (84.3000,149.3000) -- (84.3000,149.3000) -- (84.3000,149.3000) -- (84.4000,149.3000) -- (84.4000,149.3000) -- (84.4000,149.3000) -- (84.4000,149.3000) -- (84.4000,149.3000) -- (84.4000,149.3000) -- (84.4000,149.3000) -- (84.4000,149.2000) -- (84.4000,149.2000) -- (84.4000,149.2000) -- (84.4000,149.2000) -- (84.4000,149.2000) -- (84.4000,149.2000) -- (84.4000,149.2000) -- (84.4000,149.2000) -- (84.4000,149.2000) -- (84.4000,149.2000) -- (84.4000,149.2000) -- (84.4000,149.2000) -- (84.4000,149.2000) -- (84.4000,149.2000) -- (84.4000,149.2000) -- (84.4000,149.2000) -- (84.4000,149.2000) -- (84.4000,149.2000) -- (84.4000,149.2000) -- (84.4000,149.2000) -- (84.4000,149.2000) -- (84.4000,149.2000) -- (84.4000,149.1000) -- (84.4000,149.1000) -- (84.4000,149.1000) -- (84.4000,149.1000) -- (84.4000,149.1000) -- (84.4000,149.1000) -- (84.4000,149.1000) -- (84.4000,149.1000) -- (84.4000,149.1000) -- (84.4000,149.1000) -- (84.4000,149.1000) -- (84.4000,149.1000) -- (84.4000,149.1000) -- (84.4000,149.1000) -- (84.4000,149.1000) -- (84.4000,149.1000) -- (84.4000,149.1000) -- (84.4000,149.1000) -- (84.4000,149.1000) -- (84.4000,149.1000) -- (84.5000,149.1000) -- (84.5000,149.0000) -- (84.5000,149.0000) -- (84.5000,149.0000) -- (84.5000,149.0000) -- (84.5000,149.0000) -- (84.5000,149.0000) -- (84.5000,149.0000) -- (84.5000,149.0000) -- (84.5000,149.0000) -- (84.5000,149.0000) -- (84.5000,149.0000) -- (84.5000,149.0000) -- (84.5000,149.0000) -- (84.5000,149.0000) -- (84.5000,149.0000) -- (84.5000,149.0000) -- (84.5000,149.0000) -- (84.5000,149.0000) -- (84.5000,149.0000) -- (84.5000,149.0000) -- (84.5000,149.0000) -- (84.5000,149.0000) -- (84.5000,148.9000) -- (84.5000,148.9000) -- (84.5000,148.9000) -- (84.5000,148.9000) -- (84.5000,148.9000) -- (84.5000,148.9000) -- (84.5000,148.9000) -- (84.5000,148.9000) -- (84.5000,148.9000) -- (84.5000,148.9000) -- (84.5000,148.9000) -- (84.5000,148.9000) -- (84.5000,148.9000) -- (84.5000,148.9000) -- (84.5000,148.9000) -- (84.5000,148.9000) -- (84.5000,148.9000) -- (84.5000,148.9000) -- (84.5000,148.9000) -- (84.5000,148.9000) -- (84.5000,148.9000) -- (84.5000,148.8000) -- (84.5000,148.8000) -- (84.5000,148.8000) -- (84.5000,148.8000) -- (84.5000,148.8000) -- (84.5000,148.8000) -- (84.6000,148.8000) -- (84.6000,148.8000) -- (84.6000,148.8000) -- (84.6000,148.8000) -- (84.6000,148.8000) -- (84.6000,148.8000) -- (84.6000,148.8000) -- (84.6000,148.8000) -- (84.6000,148.8000) -- (84.6000,148.8000) -- (84.6000,148.8000) -- (84.6000,148.8000) -- (84.6000,148.8000) -- (84.6000,148.8000) -- (84.6000,148.8000) -- (84.6000,148.8000) -- (84.6000,148.7000) -- (84.6000,148.7000) -- (84.6000,148.7000) -- (84.6000,148.7000) -- (84.6000,148.7000) -- (84.6000,148.7000) -- (84.6000,148.7000) -- (84.6000,148.7000) -- (84.6000,148.7000) -- (84.6000,148.7000) -- (84.6000,148.7000) -- (84.6000,148.7000) -- (84.6000,148.7000) -- (84.6000,148.7000) -- (84.6000,148.7000) -- (84.6000,148.7000) -- (84.6000,148.7000) -- (84.6000,148.7000) -- (84.6000,148.7000) -- (84.6000,148.7000) -- (84.6000,148.7000) -- (84.6000,148.6000) -- (84.6000,148.6000) -- (84.6000,148.6000) -- (84.6000,148.6000) -- (84.6000,148.6000) -- (84.6000,148.6000) -- (84.6000,148.6000) -- (84.6000,148.6000) -- (84.6000,148.6000) -- (84.6000,148.6000) -- (84.6000,148.6000) -- (84.6000,148.6000) -- (84.7000,148.6000) -- (84.7000,148.6000) -- (84.7000,148.6000) -- (84.7000,148.6000) -- (84.7000,148.6000) -- (84.7000,148.6000) -- (84.7000,148.6000) -- (84.7000,148.6000) -- (84.7000,148.6000) -- (84.7000,148.6000) -- (84.7000,148.5000) -- (84.7000,148.5000) -- (84.7000,148.5000) -- (84.7000,148.5000) -- (84.7000,148.5000) -- (84.7000,148.5000) -- (84.7000,148.5000) -- (84.7000,148.5000) -- (84.7000,148.5000) -- (84.7000,148.5000) -- (84.7000,148.5000) -- (84.7000,148.5000) -- (84.7000,148.5000) -- (84.7000,148.5000) -- (84.7000,148.5000) -- (84.7000,148.5000) -- (84.7000,148.5000) -- (84.7000,148.5000) -- (84.7000,148.5000) -- (84.7000,148.5000) -- (84.7000,148.5000) -- (84.7000,148.4000) -- (84.7000,148.4000) -- (84.7000,148.4000) -- (84.7000,148.4000) -- (84.7000,148.4000) -- (84.7000,148.4000) -- (84.7000,148.4000) -- (84.7000,148.4000) -- (84.7000,148.4000) -- (84.7000,148.4000) -- (84.7000,148.4000) -- (84.7000,148.4000) -- (84.7000,148.4000) -- (84.7000,148.4000) -- (84.7000,148.4000) -- (84.7000,148.4000) -- (84.7000,148.4000) -- (84.7000,148.4000) -- (84.7000,148.4000) -- (84.8000,148.4000) -- (84.8000,148.4000) -- (84.8000,148.4000) -- (84.8000,148.3000) -- (84.8000,148.3000) -- (84.8000,148.3000) -- (84.8000,148.3000) -- (84.8000,148.3000) -- (84.8000,148.3000) -- (84.8000,148.3000) -- (84.8000,148.3000) -- (84.8000,148.3000) -- (84.8000,148.3000) -- (84.8000,148.3000) -- (84.8000,148.3000) -- (84.8000,148.3000) -- (84.8000,148.3000) -- (84.8000,148.3000) -- (84.8000,148.3000) -- (84.8000,148.3000) -- (84.8000,148.3000) -- (84.8000,148.3000) -- (84.8000,148.3000) -- (84.8000,148.3000) -- (84.8000,148.2000) -- (84.8000,148.2000) -- (84.8000,148.2000) -- (84.8000,148.2000) -- (84.8000,148.2000) -- (84.8000,148.2000) -- (84.8000,148.2000) -- (84.8000,148.2000) -- (84.8000,148.2000) -- (84.8000,148.2000) -- (84.8000,148.2000) -- (84.8000,148.2000) -- (84.8000,148.2000) -- (84.8000,148.2000) -- (84.8000,148.2000) -- (84.8000,148.2000) -- (84.8000,148.2000) -- (84.8000,148.2000) -- (84.8000,148.2000) -- (84.8000,148.2000) -- (84.8000,148.2000) -- (84.8000,148.2000) -- (84.8000,148.1000) -- (84.8000,148.1000) -- (84.8000,148.1000) -- (84.9000,148.1000) -- (84.9000,148.1000) -- (84.9000,148.1000) -- (84.9000,148.1000) -- (84.9000,148.1000) -- (84.9000,148.1000) -- (84.9000,148.1000) -- (84.9000,148.1000) -- (84.9000,148.1000) -- (84.9000,148.1000) -- (84.9000,148.1000) -- (84.9000,148.1000) -- (84.9000,148.1000) -- (84.9000,148.1000) -- (84.9000,148.1000) -- (84.9000,148.1000) -- (84.9000,148.1000) -- (84.9000,148.1000) -- (84.9000,148.0000) -- (84.9000,148.0000) -- (84.9000,148.0000) -- (84.9000,148.0000) -- (84.9000,148.0000) -- (84.9000,148.0000) -- (84.9000,148.0000) -- (84.9000,148.0000) -- (84.9000,148.0000) -- (84.9000,148.0000) -- (84.9000,148.0000) -- (84.9000,148.0000) -- (84.9000,148.0000) -- (84.9000,148.0000) -- (84.9000,148.0000) -- (84.9000,148.0000) -- (84.9000,148.0000) -- (84.9000,148.0000) -- (84.9000,148.0000) -- (84.9000,148.0000) -- (84.9000,148.0000) -- (84.9000,148.0000) -- (84.9000,147.9000) -- (84.9000,147.9000) -- (84.9000,147.9000) -- (84.9000,147.9000) -- (84.9000,147.9000) -- (84.9000,147.9000) -- (84.9000,147.9000) -- (84.9000,147.9000) -- (84.9000,147.9000) -- (84.9000,147.9000) -- (85.0000,147.9000) -- (85.0000,147.9000) -- (85.0000,147.9000) -- (85.0000,147.9000) -- (85.0000,147.9000) -- (85.0000,147.9000) -- (85.0000,147.9000) -- (85.0000,147.9000) -- (85.0000,147.9000) -- (85.0000,147.9000) -- (85.0000,147.9000) -- (85.0000,147.8000) -- (85.0000,147.8000) -- (85.0000,147.8000) -- (85.0000,147.8000) -- (85.0000,147.8000) -- (85.0000,147.8000) -- (85.0000,147.8000) -- (85.0000,147.8000) -- (85.0000,147.8000) -- (85.0000,147.8000) -- (85.0000,147.8000) -- (85.0000,147.8000) -- (85.0000,147.8000) -- (85.0000,147.8000) -- (85.0000,147.8000) -- (85.0000,147.8000) -- (85.0000,147.8000) -- (85.0000,147.8000) -- (85.0000,147.8000) -- (85.0000,147.8000) -- (85.0000,147.8000) -- (85.0000,147.8000) -- (85.0000,147.7000) -- (85.0000,147.7000) -- (85.0000,147.7000) -- (85.0000,147.7000) -- (85.0000,147.7000) -- (85.0000,147.7000) -- (85.0000,147.7000) -- (85.0000,147.7000) -- (85.0000,147.7000) -- (85.0000,147.7000) -- (85.0000,147.7000) -- (85.0000,147.7000) -- (85.0000,147.7000) -- (85.0000,147.7000) -- (85.0000,147.7000) -- (85.0000,147.7000) -- (85.1000,147.7000) -- (85.1000,147.7000) -- (85.1000,147.7000) -- (85.1000,147.7000) -- (85.1000,147.7000) -- (85.1000,147.6000) -- (85.1000,147.6000) -- (85.1000,147.6000) -- (85.1000,147.6000) -- (85.1000,147.6000) -- (85.1000,147.6000) -- (85.1000,147.6000) -- (85.1000,147.6000) -- (85.1000,147.6000) -- (85.1000,147.6000) -- (85.1000,147.6000) -- (85.1000,147.6000) -- (85.1000,147.6000) -- (85.1000,147.6000) -- (85.1000,147.6000) -- (85.1000,147.6000) -- (85.1000,147.6000) -- (85.1000,147.6000) -- (85.1000,147.6000) -- (85.1000,147.6000) -- (85.1000,147.6000) -- (85.1000,147.6000) -- (85.1000,147.5000) -- (85.1000,147.5000) -- (85.1000,147.5000) -- (85.1000,147.5000) -- (85.1000,147.5000) -- (85.1000,147.5000) -- (85.1000,147.5000) -- (85.1000,147.5000) -- (85.1000,147.5000) -- (85.1000,147.5000) -- (85.1000,147.5000) -- (85.1000,147.5000) -- (85.1000,147.5000) -- (85.1000,147.5000) -- (85.1000,147.5000) -- (85.1000,147.5000) -- (85.1000,147.5000) -- (85.1000,147.5000) -- (85.1000,147.5000) -- (85.1000,147.5000) -- (85.1000,147.5000) -- (85.1000,147.4000) -- (85.1000,147.4000) -- (85.2000,147.4000) -- (85.2000,147.4000) -- (85.2000,147.4000) -- (85.2000,147.4000) -- (85.2000,147.4000) -- (85.2000,147.4000) -- (85.2000,147.4000) -- (85.2000,147.4000) -- (85.2000,147.4000) -- (85.2000,147.4000) -- (85.2000,147.4000) -- (85.2000,147.4000) -- (85.2000,147.4000) -- (85.2000,147.4000) -- (85.2000,147.4000) -- (85.2000,147.4000) -- (85.2000,147.4000) -- (85.2000,147.4000) -- (85.2000,147.4000) -- (85.2000,147.4000) -- (85.2000,147.3000) -- (85.2000,147.3000) -- (85.2000,147.3000) -- (85.2000,147.3000) -- (85.2000,147.3000) -- (85.2000,147.3000) -- (85.2000,147.3000) -- (85.2000,147.3000) -- (85.2000,147.3000) -- (85.2000,147.3000) -- (85.2000,147.3000) -- (85.2000,147.3000) -- (85.2000,147.3000) -- (85.2000,147.3000) -- (85.2000,147.3000) -- (85.2000,147.3000) -- (85.2000,147.3000) -- (85.2000,147.3000) -- (85.2000,147.3000) -- (85.2000,147.3000) -- (85.2000,147.3000) -- (85.2000,147.2000) -- (85.2000,147.2000) -- (85.2000,147.2000) -- (85.2000,147.2000) -- (85.2000,147.2000) -- (85.2000,147.2000) -- (85.2000,147.2000) -- (85.2000,147.2000) -- (85.3000,147.2000) -- (85.3000,147.2000) -- (85.3000,147.2000) -- (85.3000,147.2000) -- (85.3000,147.2000) -- (85.3000,147.2000) -- (85.3000,147.2000) -- (85.3000,147.2000) -- (85.3000,147.2000) -- (85.3000,147.2000) -- (85.3000,147.2000) -- (85.3000,147.2000) -- (85.3000,147.2000) -- (85.3000,147.2000) -- (85.3000,147.1000) -- (85.3000,147.1000) -- (85.3000,147.1000) -- (85.3000,147.1000) -- (85.3000,147.1000) -- (85.3000,147.1000) -- (85.3000,147.1000) -- (85.3000,147.1000) -- (85.3000,147.1000) -- (85.3000,147.1000) -- (85.3000,147.1000) -- (85.3000,147.1000) -- (85.3000,147.1000) -- (85.3000,147.1000) -- (85.3000,147.1000) -- (85.3000,147.1000) -- (85.3000,147.1000) -- (85.3000,147.1000) -- (85.3000,147.1000) -- (85.3000,147.1000) -- (85.3000,147.1000) -- (85.3000,147.0000) -- (85.3000,147.0000) -- (85.3000,147.0000) -- (85.3000,147.0000) -- (85.3000,147.0000) -- (85.3000,147.0000) -- (85.3000,147.0000) -- (85.3000,147.0000) -- (85.3000,147.0000) -- (85.3000,147.0000) -- (85.3000,147.0000) -- (85.3000,147.0000) -- (85.3000,147.0000) -- (85.3000,147.0000) -- (85.3000,147.0000) -- (85.4000,147.0000) -- (85.4000,147.0000) -- (85.4000,147.0000) -- (85.4000,147.0000) -- (85.4000,147.0000) -- (85.4000,147.0000) -- (85.4000,147.0000) -- (85.4000,146.9000) -- (85.4000,146.9000) -- (85.4000,146.9000) -- (85.4000,146.9000) -- (85.4000,146.9000) -- (85.4000,146.9000) -- (85.4000,146.9000) -- (85.4000,146.9000) -- (85.4000,146.9000) -- (85.4000,146.9000) -- (85.4000,146.9000) -- (85.4000,146.9000) -- (85.4000,146.9000) -- (85.4000,146.9000) -- (85.4000,146.9000) -- (85.4000,146.9000) -- (85.4000,146.9000) -- (85.4000,146.9000) -- (85.4000,146.9000) -- (85.4000,146.9000) -- (85.4000,146.9000) -- (85.4000,146.8000) -- (85.4000,146.8000) -- (85.4000,146.8000) -- (85.4000,146.8000) -- (85.4000,146.8000) -- (85.4000,146.8000) -- (85.4000,146.8000) -- (85.4000,146.8000) -- (85.4000,146.8000) -- (85.4000,146.8000) -- (85.4000,146.8000) -- (85.4000,146.8000) -- (85.4000,146.8000) -- (85.4000,146.8000) -- (85.4000,146.8000) -- (85.4000,146.8000) -- (85.4000,146.8000) -- (85.4000,146.8000) -- (85.4000,146.8000) -- (85.4000,146.8000) -- (85.4000,146.8000) -- (85.5000,146.8000) -- (85.5000,146.7000) -- (85.5000,146.7000) -- (85.5000,146.7000) -- (85.5000,146.7000) -- (85.5000,146.7000) -- (85.5000,146.7000) -- (85.5000,146.7000) -- (85.5000,146.7000) -- (85.5000,146.7000) -- (85.5000,146.7000) -- (85.5000,146.7000) -- (85.5000,146.7000) -- (85.5000,146.7000) -- (85.5000,146.7000) -- (85.5000,146.7000) -- (85.5000,146.7000) -- (85.5000,146.7000) -- (85.5000,146.7000) -- (85.5000,146.7000) -- (85.5000,146.7000) -- (85.5000,146.7000) -- (85.5000,146.6000) -- (85.5000,146.6000) -- (85.5000,146.6000) -- (85.5000,146.6000) -- (85.5000,146.6000) -- (85.5000,146.6000) -- (85.5000,146.6000) -- (85.5000,146.6000) -- (85.5000,146.6000) -- (85.5000,146.6000) -- (85.5000,146.6000) -- (85.5000,146.6000) -- (85.5000,146.6000) -- (85.5000,146.6000) -- (85.5000,146.6000) -- (85.5000,146.6000) -- (85.5000,146.6000) -- (85.5000,146.6000) -- (85.5000,146.6000) -- (85.5000,146.6000) -- (85.5000,146.6000) -- (85.5000,146.6000) -- (85.5000,146.5000) -- (85.5000,146.5000) -- (85.5000,146.5000) -- (85.5000,146.5000) -- (85.5000,146.5000) -- (85.5000,146.5000) -- (85.6000,146.5000) -- (85.6000,146.5000) -- (85.6000,146.5000) -- (85.6000,146.5000) -- (85.6000,146.5000) -- (85.6000,146.5000) -- (85.6000,146.5000) -- (85.6000,146.5000) -- (85.6000,146.5000) -- (85.6000,146.5000) -- (85.6000,146.5000) -- (85.6000,146.5000) -- (85.6000,146.5000) -- (85.6000,146.5000) -- (85.6000,146.5000) -- (85.6000,146.4000) -- (85.6000,146.4000) -- (85.6000,146.4000) -- (85.6000,146.4000) -- (85.6000,146.4000) -- (85.6000,146.4000) -- (85.6000,146.4000) -- (85.6000,146.4000) -- (85.6000,146.4000) -- (85.6000,146.4000) -- (85.6000,146.4000) -- (85.6000,146.4000) -- (85.6000,146.4000) -- (85.6000,146.4000) -- (85.6000,146.4000) -- (85.6000,146.4000) -- (85.6000,146.4000) -- (85.6000,146.4000) -- (85.6000,146.4000) -- (85.6000,146.4000) -- (85.6000,146.4000) -- (85.6000,146.4000) -- (85.6000,146.3000) -- (85.6000,146.3000) -- (85.6000,146.3000) -- (85.6000,146.3000) -- (85.6000,146.3000) -- (85.6000,146.3000) -- (85.6000,146.3000) -- (85.6000,146.3000) -- (85.6000,146.3000) -- (85.6000,146.3000) -- (85.6000,146.3000) -- (85.6000,146.3000) -- (85.7000,146.3000) -- (85.7000,146.3000) -- (85.7000,146.3000) -- (85.7000,146.3000) -- (85.7000,146.3000) -- (85.7000,146.3000) -- (85.7000,146.3000) -- (85.7000,146.3000) -- (85.7000,146.3000) -- (85.7000,146.2000) -- (85.7000,146.2000) -- (85.7000,146.2000) -- (85.7000,146.2000) -- (85.7000,146.2000) -- (85.7000,146.2000) -- (85.7000,146.2000) -- (85.7000,146.2000) -- (85.7000,146.2000) -- (85.7000,146.2000) -- (85.7000,146.2000) -- (85.7000,146.2000) -- (85.7000,146.2000) -- (85.7000,146.2000) -- (85.7000,146.2000) -- (85.7000,146.2000) -- (85.7000,146.2000) -- (85.7000,146.2000) -- (85.7000,146.2000) -- (85.7000,146.2000) -- (85.7000,146.2000) -- (85.7000,146.2000) -- (85.7000,146.1000) -- (85.7000,146.1000) -- (85.7000,146.1000) -- (85.7000,146.1000) -- (85.7000,146.1000) -- (85.7000,146.1000) -- (85.7000,146.1000) -- (85.7000,146.1000) -- (85.7000,146.1000) -- (85.7000,146.1000) -- (85.7000,146.1000) -- (85.7000,146.1000) -- (85.7000,146.1000) -- (85.7000,146.1000) -- (85.7000,146.1000) -- (85.7000,146.1000) -- (85.7000,146.1000) -- (85.7000,146.1000) -- (85.7000,146.1000) -- (85.8000,146.1000) -- (85.8000,146.1000) -- (85.8000,146.0000) -- (85.8000,146.0000) -- (85.8000,146.0000) -- (85.8000,146.0000) -- (85.8000,146.0000) -- (85.8000,146.0000) -- (85.8000,146.0000) -- (85.8000,146.0000) -- (85.8000,146.0000) -- (85.8000,146.0000) -- (85.8000,146.0000) -- (85.8000,146.0000) -- (85.8000,146.0000) -- (85.8000,146.0000) -- (85.8000,146.0000) -- (85.8000,146.0000) -- (85.8000,146.0000) -- (85.8000,146.0000) -- (85.8000,146.0000) -- (85.8000,146.0000) -- (85.8000,146.0000) -- (85.8000,146.0000) -- (85.8000,145.9000) -- (85.8000,145.9000) -- (85.8000,145.9000) -- (85.8000,145.9000) -- (85.8000,145.9000) -- (85.8000,145.9000) -- (85.8000,145.9000) -- (85.8000,145.9000) -- (85.8000,145.9000) -- (85.8000,145.9000) -- (85.8000,145.9000) -- (85.8000,145.9000) -- (85.8000,145.9000) -- (85.8000,145.9000) -- (85.8000,145.9000) -- (85.8000,145.9000) -- (85.8000,145.9000) -- (85.8000,145.9000) -- (85.8000,145.9000) -- (85.8000,145.9000) -- (85.8000,145.9000) -- (85.8000,145.8000) -- (85.8000,145.8000) -- (85.8000,145.8000) -- (85.8000,145.8000) -- (85.9000,145.8000) -- (85.9000,145.8000) -- (85.9000,145.8000) -- (85.9000,145.8000) -- (85.9000,145.8000) -- (85.9000,145.8000) -- (85.9000,145.8000) -- (85.9000,145.8000) -- (85.9000,145.8000) -- (85.9000,145.8000) -- (85.9000,145.8000) -- (85.9000,145.8000) -- (85.9000,145.8000) -- (85.9000,145.8000) -- (85.9000,145.8000) -- (85.9000,145.8000) -- (85.9000,145.8000) -- (85.9000,145.8000) -- (85.9000,145.7000) -- (85.9000,145.7000) -- (85.9000,145.7000) -- (85.9000,145.7000) -- (85.9000,145.7000) -- (85.9000,145.7000) -- (85.9000,145.7000) -- (85.9000,145.7000) -- (85.9000,145.7000) -- (85.9000,145.7000) -- (85.9000,145.7000) -- (85.9000,145.7000) -- (85.9000,145.7000) -- (85.9000,145.7000) -- (85.9000,145.7000) -- (85.9000,145.7000) -- (85.9000,145.7000) -- (85.9000,145.7000) -- (85.9000,145.7000) -- (85.9000,145.7000) -- (85.9000,145.7000) -- (85.9000,145.6000) -- (85.9000,145.6000) -- (85.9000,145.6000) -- (85.9000,145.6000) -- (85.9000,145.6000) -- (85.9000,145.6000) -- (85.9000,145.6000) -- (85.9000,145.6000) -- (85.9000,145.6000) -- (85.9000,145.6000) -- (85.9000,145.6000) -- (86.0000,145.6000) -- (86.0000,145.6000) -- (86.0000,145.6000) -- (86.0000,145.6000) -- (86.0000,145.6000) -- (86.0000,145.6000) -- (86.0000,145.6000) -- (86.0000,145.6000) -- (86.0000,145.6000) -- (86.0000,145.6000) -- (86.0000,145.6000) -- (86.0000,145.5000) -- (86.0000,145.5000) -- (86.0000,145.5000) -- (86.0000,145.5000) -- (86.0000,145.5000) -- (86.0000,145.5000) -- (86.0000,145.5000) -- (86.0000,145.5000) -- (86.0000,145.5000) -- (86.0000,145.5000) -- (86.0000,145.5000) -- (86.0000,145.5000) -- (86.0000,145.5000) -- (86.0000,145.5000) -- (86.0000,145.5000) -- (86.0000,145.5000) -- (86.0000,145.5000) -- (86.0000,145.5000) -- (86.0000,145.5000) -- (86.0000,145.5000) -- (86.0000,145.5000) -- (86.0000,145.4000) -- (86.0000,145.4000) -- (86.0000,145.4000) -- (86.0000,145.4000) -- (86.0000,145.4000) -- (86.0000,145.4000) -- (86.0000,145.4000) -- (86.0000,145.4000) -- (86.0000,145.4000) -- (86.0000,145.4000) -- (86.0000,145.4000) -- (86.0000,145.4000) -- (86.0000,145.4000) -- (86.0000,145.4000) -- (86.0000,145.4000) -- (86.0000,145.4000) -- (86.0000,145.4000) -- (86.1000,145.4000) -- (86.1000,145.4000) -- (86.1000,145.4000) -- (86.1000,145.4000) -- (86.1000,145.4000) -- (86.1000,145.3000) -- (86.1000,145.3000) -- (86.1000,145.3000) -- (86.1000,145.3000) -- (86.1000,145.3000) -- (86.1000,145.3000) -- (86.1000,145.3000) -- (86.1000,145.3000) -- (86.1000,145.3000) -- (86.1000,145.3000) -- (86.1000,145.3000) -- (86.1000,145.3000) -- (86.1000,145.3000) -- (86.1000,145.3000) -- (86.1000,145.3000) -- (86.1000,145.3000) -- (86.1000,145.3000) -- (86.1000,145.3000) -- (86.1000,145.3000) -- (86.1000,145.3000) -- (86.1000,145.3000) -- (86.1000,145.2000) -- (86.1000,145.2000) -- (86.1000,145.2000) -- (86.1000,145.2000) -- (86.1000,145.2000) -- (86.1000,145.2000) -- (86.1000,145.2000) -- (86.1000,145.2000) -- (86.1000,145.2000) -- (86.1000,145.2000) -- (86.1000,145.2000) -- (86.1000,145.2000) -- (86.1000,145.2000) -- (86.1000,145.2000) -- (86.1000,145.2000) -- (86.1000,145.2000) -- (86.1000,145.2000) -- (86.1000,145.2000) -- (86.1000,145.2000) -- (86.1000,145.2000) -- (86.1000,145.2000) -- (86.1000,145.2000) -- (86.1000,145.1000) -- (86.1000,145.1000) -- (86.2000,145.1000) -- (86.2000,145.1000) -- (86.2000,145.1000) -- (86.2000,145.1000) -- (86.2000,145.1000) -- (86.2000,145.1000) -- (86.2000,145.1000) -- (86.2000,145.1000) -- (86.2000,145.1000) -- (86.2000,145.1000) -- (86.2000,145.1000) -- (86.2000,145.1000) -- (86.2000,145.1000) -- (86.2000,145.1000) -- (86.2000,145.1000) -- (86.2000,145.1000) -- (86.2000,145.1000) -- (86.2000,145.1000) -- (86.2000,145.1000) -- (86.2000,145.0000) -- (86.2000,145.0000) -- (86.2000,145.0000) -- (86.2000,145.0000) -- (86.2000,145.0000) -- (86.2000,145.0000) -- (86.2000,145.0000) -- (86.2000,145.0000) -- (86.2000,145.0000) -- (86.2000,145.0000) -- (86.2000,145.0000) -- (86.2000,145.0000) -- (86.2000,145.0000) -- (86.2000,145.0000) -- (86.2000,145.0000) -- (86.2000,145.0000) -- (86.2000,145.0000) -- (86.2000,145.0000) -- (86.2000,145.0000) -- (86.2000,145.0000) -- (86.2000,145.0000) -- (86.2000,145.0000) -- (86.2000,144.9000) -- (86.2000,144.9000) -- (86.2000,144.9000) -- (86.2000,144.9000) -- (86.2000,144.9000) -- (86.2000,144.9000) -- (86.2000,144.9000) -- (86.2000,144.9000) -- (86.2000,144.9000) -- (86.3000,144.9000) -- (86.3000,144.9000) -- (86.3000,144.9000) -- (86.3000,144.9000) -- (86.3000,144.9000) -- (86.3000,144.9000) -- (86.3000,144.9000) -- (86.3000,144.9000) -- (86.3000,144.9000) -- (86.3000,144.9000) -- (86.3000,144.9000) -- (86.3000,144.9000) -- (86.3000,144.8000) -- (86.3000,144.8000) -- (86.3000,144.8000) -- (86.3000,144.8000) -- (86.3000,144.8000) -- (86.3000,144.8000) -- (86.3000,144.8000) -- (86.3000,144.8000) -- (86.3000,144.8000) -- (86.3000,144.8000) -- (86.3000,144.8000) -- (86.3000,144.8000) -- (86.3000,144.8000) -- (86.3000,144.8000) -- (86.3000,144.8000) -- (86.3000,144.8000) -- (86.3000,144.8000) -- (86.3000,144.8000) -- (86.3000,144.8000) -- (86.3000,144.8000) -- (86.3000,144.8000) -- (86.3000,144.8000) -- (86.3000,144.7000) -- (86.3000,144.7000) -- (86.3000,144.7000) -- (86.3000,144.7000) -- (86.3000,144.7000) -- (86.3000,144.7000) -- (86.3000,144.7000) -- (86.3000,144.7000) -- (86.3000,144.7000) -- (86.3000,144.7000) -- (86.3000,144.7000) -- (86.3000,144.7000) -- (86.3000,144.7000) -- (86.3000,144.7000) -- (86.3000,144.7000) -- (86.4000,144.7000) -- (86.4000,144.7000) -- (86.4000,144.7000) -- (86.4000,144.7000) -- (86.4000,144.7000) -- (86.4000,144.7000) -- (86.4000,144.6000) -- (86.4000,144.6000) -- (86.4000,144.6000) -- (86.4000,144.6000) -- (86.4000,144.6000) -- (86.4000,144.6000) -- (86.4000,144.6000) -- (86.4000,144.6000) -- (86.4000,144.6000) -- (86.4000,144.6000) -- (86.4000,144.6000) -- (86.4000,144.6000) -- (86.4000,144.6000) -- (86.4000,144.6000) -- (86.4000,144.6000) -- (86.4000,144.6000) -- (86.4000,144.6000) -- (86.4000,144.6000) -- (86.4000,144.6000) -- (86.4000,144.6000) -- (86.4000,144.6000) -- (86.4000,144.6000) -- (86.4000,144.5000) -- (86.4000,144.5000) -- (86.4000,144.5000) -- (86.4000,144.5000) -- (86.4000,144.5000) -- (86.4000,144.5000) -- (86.4000,144.5000) -- (86.4000,144.5000) -- (86.4000,144.5000) -- (86.4000,144.5000) -- (86.4000,144.5000) -- (86.4000,144.5000) -- (86.4000,144.5000) -- (86.4000,144.5000) -- (86.4000,144.5000) -- (86.4000,144.5000) -- (86.4000,144.5000) -- (86.4000,144.5000) -- (86.4000,144.5000) -- (86.4000,144.5000) -- (86.4000,144.5000) -- (86.4000,144.4000) -- (86.5000,144.4000) -- (86.5000,144.4000) -- (86.5000,144.4000) -- (86.5000,144.4000) -- (86.5000,144.4000) -- (86.5000,144.4000) -- (86.5000,144.4000) -- (86.5000,144.4000) -- (86.5000,144.4000) -- (86.5000,144.4000) -- (86.5000,144.4000) -- (86.5000,144.4000) -- (86.5000,144.4000) -- (86.5000,144.4000) -- (86.5000,144.4000) -- (86.5000,144.4000) -- (86.5000,144.4000) -- (86.5000,144.4000) -- (86.5000,144.4000) -- (86.5000,144.4000) -- (86.5000,144.4000) -- (86.5000,144.3000) -- (86.5000,144.3000) -- (86.5000,144.3000) -- (86.5000,144.3000) -- (86.5000,144.3000) -- (86.5000,144.3000) -- (86.5000,144.3000) -- (86.5000,144.3000) -- (86.5000,144.3000) -- (86.5000,144.3000) -- (86.5000,144.3000) -- (86.5000,144.3000) -- (86.5000,144.3000) -- (86.5000,144.3000) -- (86.5000,144.3000) -- (86.5000,144.3000) -- (86.5000,144.3000) -- (86.5000,144.3000) -- (86.5000,144.3000) -- (86.5000,144.3000) -- (86.5000,144.3000) -- (86.5000,144.2000) -- (86.5000,144.2000) -- (86.5000,144.2000) -- (86.5000,144.2000) -- (86.5000,144.2000) -- (86.5000,144.2000) -- (86.5000,144.2000) -- (86.6000,144.2000) -- (86.6000,144.2000) -- (86.6000,144.2000) -- (86.6000,144.2000) -- (86.6000,144.2000) -- (86.6000,144.2000) -- (86.6000,144.2000) -- (86.6000,144.2000) -- (86.6000,144.2000) -- (86.6000,144.2000) -- (86.6000,144.2000) -- (86.6000,144.2000) -- (86.6000,144.2000) -- (86.6000,144.2000) -- (86.6000,144.2000) -- (86.6000,144.1000) -- (86.6000,144.1000) -- (86.6000,144.1000) -- (86.6000,144.1000) -- (86.6000,144.1000) -- (86.6000,144.1000) -- (86.6000,144.1000) -- (86.6000,144.1000) -- (86.6000,144.1000) -- (86.6000,144.1000) -- (86.6000,144.1000) -- (86.6000,144.1000) -- (86.6000,144.1000) -- (86.6000,144.1000) -- (86.6000,144.1000) -- (86.6000,144.1000) -- (86.6000,144.1000) -- (86.6000,144.1000) -- (86.6000,144.1000) -- (86.6000,144.1000) -- (86.6000,144.1000) -- (86.6000,144.0000) -- (86.6000,144.0000) -- (86.6000,144.0000) -- (86.6000,144.0000) -- (86.6000,144.0000) -- (86.6000,144.0000) -- (86.6000,144.0000) -- (86.6000,144.0000) -- (86.6000,144.0000) -- (86.6000,144.0000) -- (86.6000,144.0000) -- (86.6000,144.0000) -- (86.6000,144.0000) -- (86.6000,144.0000) -- (86.7000,144.0000) -- (86.7000,144.0000) -- (86.7000,144.0000) -- (86.7000,144.0000) -- (86.7000,144.0000) -- (86.7000,144.0000) -- (86.7000,144.0000) -- (86.7000,144.0000) -- (86.7000,143.9000) -- (86.7000,143.9000) -- (86.7000,143.9000) -- (86.7000,143.9000) -- (86.7000,143.9000) -- (86.7000,143.9000) -- (86.7000,143.9000) -- (86.7000,143.9000) -- (86.7000,143.9000) -- (86.7000,143.9000) -- (86.7000,143.9000) -- (86.7000,143.9000) -- (86.7000,143.9000) -- (86.7000,143.9000) -- (86.7000,143.9000) -- (86.7000,143.9000) -- (86.7000,143.9000) -- (86.7000,143.9000) -- (86.7000,143.9000) -- (86.7000,143.9000) -- (86.7000,143.9000) -- (86.7000,143.8000) -- (86.7000,143.8000) -- (86.7000,143.8000) -- (86.7000,143.8000) -- (86.7000,143.8000) -- (86.7000,143.8000) -- (86.7000,143.8000) -- (86.7000,143.8000) -- (86.7000,143.8000) -- (86.7000,143.8000) -- (86.7000,143.8000) -- (86.7000,143.8000) -- (86.7000,143.8000) -- (86.7000,143.8000) -- (86.7000,143.8000) -- (86.7000,143.8000) -- (86.7000,143.8000) -- (86.7000,143.8000) -- (86.7000,143.8000) -- (86.7000,143.8000) -- (86.8000,143.8000) -- (86.8000,143.8000) -- (86.8000,143.7000) -- (86.8000,143.7000) -- (86.8000,143.7000) -- (86.8000,143.7000) -- (86.8000,143.7000) -- (86.8000,143.7000) -- (86.8000,143.7000) -- (86.8000,143.7000) -- (86.8000,143.7000) -- (86.8000,143.7000) -- (86.8000,143.7000) -- (86.8000,143.7000) -- (86.8000,143.7000) -- (86.8000,143.7000) -- (86.8000,143.7000) -- (86.8000,143.7000) -- (86.8000,143.7000) -- (86.8000,143.7000) -- (86.8000,143.7000) -- (86.8000,143.7000) -- (86.8000,143.7000) -- (86.8000,143.6000) -- (86.8000,143.6000) -- (86.8000,143.6000) -- (86.8000,143.6000) -- (86.8000,143.6000) -- (86.8000,143.6000) -- (86.8000,143.6000) -- (86.8000,143.6000) -- (86.8000,143.6000) -- (86.8000,143.6000) -- (86.8000,143.6000) -- (86.8000,143.6000) -- (86.8000,143.6000) -- (86.8000,143.6000) -- (86.8000,143.6000) -- (86.8000,143.6000) -- (86.8000,143.6000) -- (86.8000,143.6000) -- (86.8000,143.6000) -- (86.8000,143.6000) -- (86.8000,143.6000) -- (86.8000,143.6000) -- (86.8000,143.5000) -- (86.8000,143.5000) -- (86.8000,143.5000) -- (86.8000,143.5000) -- (86.8000,143.5000) -- (86.9000,143.5000) -- (86.9000,143.5000) -- (86.9000,143.5000) -- (86.9000,143.5000) -- (86.9000,143.5000) -- (86.9000,143.5000) -- (86.9000,143.5000) -- (86.9000,143.5000) -- (86.9000,143.5000) -- (86.9000,143.5000) -- (86.9000,143.5000) -- (86.9000,143.5000) -- (86.9000,143.5000) -- (86.9000,143.5000) -- (86.9000,143.5000) -- (86.9000,143.5000) -- (86.9000,143.4000) -- (86.9000,143.4000) -- (86.9000,143.4000) -- (86.9000,143.4000) -- (86.9000,143.4000) -- (86.9000,143.4000) -- (86.9000,143.4000) -- (86.9000,143.4000) -- (86.9000,143.4000) -- (86.9000,143.4000) -- (86.9000,143.4000) -- (86.9000,143.4000) -- (86.9000,143.4000) -- (86.9000,143.4000) -- (86.9000,143.4000) -- (86.9000,143.4000) -- (86.9000,143.4000) -- (86.9000,143.4000) -- (86.9000,143.4000) -- (86.9000,143.4000) -- (86.9000,143.4000) -- (86.9000,143.4000) -- (86.9000,143.3000) -- (86.9000,143.3000) -- (86.9000,143.3000) -- (86.9000,143.3000) -- (86.9000,143.3000) -- (86.9000,143.3000) -- (86.9000,143.3000) -- (86.9000,143.3000) -- (86.9000,143.3000) -- (86.9000,143.3000) -- (86.9000,143.3000) -- (87.0000,143.3000) -- (87.0000,143.3000) -- (87.0000,143.3000) -- (87.0000,143.3000) -- (87.0000,143.3000) -- (87.0000,143.3000) -- (87.0000,143.3000) -- (87.0000,143.3000) -- (87.0000,143.3000) -- (87.0000,143.3000) -- (87.0000,143.2000) -- (87.0000,143.2000) -- (87.0000,143.2000) -- (87.0000,143.2000) -- (87.0000,143.2000) -- (87.0000,143.2000) -- (87.0000,143.2000) -- (87.0000,143.2000) -- (87.0000,143.2000) -- (87.0000,143.2000) -- (87.0000,143.2000) -- (87.0000,143.2000) -- (87.0000,143.2000) -- (87.0000,143.2000) -- (87.0000,143.2000) -- (87.0000,143.2000) -- (87.0000,143.2000) -- (87.0000,143.2000) -- (87.0000,143.2000) -- (87.0000,143.2000) -- (87.0000,143.2000) -- (87.0000,143.2000) -- (87.0000,143.1000) -- (87.0000,143.1000) -- (87.0000,143.1000) -- (87.0000,143.1000) -- (87.0000,143.1000) -- (87.0000,143.1000) -- (87.0000,143.1000) -- (87.0000,143.1000) -- (87.0000,143.1000) -- (87.0000,143.1000) -- (87.0000,143.1000) -- (87.0000,143.1000) -- (87.0000,143.1000) -- (87.0000,143.1000) -- (87.0000,143.1000) -- (87.0000,143.1000) -- (87.0000,143.1000) -- (87.0000,143.1000) -- (87.1000,143.1000) -- (87.1000,143.1000) -- (87.1000,143.1000) -- (87.1000,143.0000) -- (87.1000,143.0000) -- (87.1000,143.0000) -- (87.1000,143.0000) -- (87.1000,143.0000) -- (87.1000,143.0000) -- (87.1000,143.0000) -- (87.1000,143.0000) -- (87.1000,143.0000) -- (87.1000,143.0000) -- (87.1000,143.0000) -- (87.1000,143.0000) -- (87.1000,143.0000) -- (87.1000,143.0000) -- (87.1000,143.0000) -- (87.1000,143.0000) -- (87.1000,143.0000) -- (87.1000,143.0000) -- (87.1000,143.0000) -- (87.1000,143.0000) -- (87.1000,143.0000) -- (87.1000,143.0000) -- (87.1000,142.9000) -- (87.1000,142.9000) -- (87.1000,142.9000) -- (87.1000,142.9000) -- (87.1000,142.9000) -- (87.1000,142.9000) -- (87.1000,142.9000) -- (87.1000,142.9000) -- (87.1000,142.9000) -- (87.1000,142.9000) -- (87.1000,142.9000) -- (87.1000,142.9000) -- (87.1000,142.9000) -- (87.1000,142.9000) -- (87.1000,142.9000) -- (87.1000,142.9000) -- (87.1000,142.9000) -- (87.1000,142.9000) -- (87.1000,142.9000) -- (87.1000,142.9000) -- (87.1000,142.9000) -- (87.1000,142.9000) -- (87.1000,142.8000) -- (87.1000,142.8000) -- (87.2000,142.8000) -- (87.2000,142.8000) -- (87.2000,142.8000) -- (87.2000,142.8000) -- (87.2000,142.8000) -- (87.2000,142.8000) -- (87.2000,142.8000) -- (87.2000,142.8000) -- (87.2000,142.8000) -- (87.2000,142.8000) -- (87.2000,142.8000) -- (87.2000,142.8000) -- (87.2000,142.8000) -- (87.2000,142.8000) -- (87.2000,142.8000) -- (87.2000,142.8000) -- (87.2000,142.8000) -- (87.2000,142.8000) -- (87.2000,142.8000) -- (87.2000,142.7000) -- (87.2000,142.7000) -- (87.2000,142.7000) -- (87.2000,142.7000) -- (87.2000,142.7000) -- (87.2000,142.7000) -- (87.2000,142.7000) -- (87.2000,142.7000) -- (87.2000,142.7000) -- (87.2000,142.7000) -- (87.2000,142.7000) -- (87.2000,142.7000) -- (87.2000,142.7000) -- (87.2000,142.7000) -- (87.2000,142.7000) -- (87.2000,142.7000) -- (87.2000,142.7000) -- (87.2000,142.7000) -- (87.2000,142.7000) -- (87.2000,142.7000) -- (87.2000,142.7000) -- (87.2000,142.7000) -- (87.2000,142.6000) -- (87.2000,142.6000) -- (87.2000,142.6000) -- (87.2000,142.6000) -- (87.2000,142.6000) -- (87.2000,142.6000) -- (87.2000,142.6000) -- (87.2000,142.6000) -- (87.2000,142.6000) -- (87.3000,142.6000) -- (87.3000,142.6000) -- (87.3000,142.6000) -- (87.3000,142.6000) -- (87.3000,142.6000) -- (87.3000,142.6000) -- (87.3000,142.6000) -- (87.3000,142.6000) -- (87.3000,142.6000) -- (87.3000,142.6000) -- (87.3000,142.6000) -- (87.3000,142.6000) -- (87.3000,142.5000) -- (87.3000,142.5000) -- (87.3000,142.5000) -- (87.3000,142.5000) -- (87.3000,142.5000) -- (87.3000,142.5000) -- (87.3000,142.5000) -- (87.3000,142.5000) -- (87.3000,142.5000) -- (87.3000,142.5000) -- (87.3000,142.5000) -- (87.3000,142.5000) -- (87.3000,142.5000) -- (87.3000,142.5000) -- (87.3000,142.5000) -- (87.3000,142.5000) -- (87.3000,142.5000) -- (87.3000,142.5000) -- (87.3000,142.5000) -- (87.3000,142.5000) -- (87.3000,142.5000) -- (87.3000,142.5000) -- (87.3000,142.4000) -- (87.3000,142.4000) -- (87.3000,142.4000) -- (87.3000,142.4000) -- (87.3000,142.4000) -- (87.3000,142.4000) -- (87.3000,142.4000) -- (87.3000,142.4000) -- (87.3000,142.4000) -- (87.3000,142.4000) -- (87.3000,142.4000) -- (87.3000,142.4000) -- (87.3000,142.4000) -- (87.3000,142.4000) -- (87.3000,142.4000) -- (87.4000,142.4000) -- (87.4000,142.4000) -- (87.4000,142.4000) -- (87.4000,142.4000) -- (87.4000,142.4000) -- (87.4000,142.4000) -- (87.4000,142.3000) -- (87.4000,142.3000) -- (87.4000,142.3000) -- (87.4000,142.3000) -- (87.4000,142.3000) -- (87.4000,142.3000) -- (87.4000,142.3000) -- (87.4000,142.3000) -- (87.4000,142.3000) -- (87.4000,142.3000) -- (87.4000,142.3000) -- (87.4000,142.3000) -- (87.4000,142.3000) -- (87.4000,142.3000) -- (87.4000,142.3000) -- (87.4000,142.3000) -- (87.4000,142.3000) -- (87.4000,142.3000) -- (87.4000,142.3000) -- (87.4000,142.3000) -- (87.4000,142.3000) -- (87.4000,142.3000) -- (87.4000,142.2000) -- (87.4000,142.2000) -- (87.4000,142.2000) -- (87.4000,142.2000) -- (87.4000,142.2000) -- (87.4000,142.2000) -- (87.4000,142.2000) -- (87.4000,142.2000) -- (87.4000,142.2000) -- (87.4000,142.2000) -- (87.4000,142.2000) -- (87.4000,142.2000) -- (87.4000,142.2000) -- (87.4000,142.2000) -- (87.4000,142.2000) -- (87.4000,142.2000) -- (87.4000,142.2000) -- (87.4000,142.2000) -- (87.4000,142.2000) -- (87.4000,142.2000) -- (87.4000,142.2000) -- (87.4000,142.1000) -- (87.5000,142.1000) -- (87.5000,142.1000) -- (87.5000,142.1000) -- (87.5000,142.1000) -- (87.5000,142.1000) -- (87.5000,142.1000) -- (87.5000,142.1000) -- (87.5000,142.1000) -- (87.5000,142.1000) -- (87.5000,142.1000) -- (87.5000,142.1000) -- (87.5000,142.1000) -- (87.5000,142.1000) -- (87.5000,142.1000) -- (87.5000,142.1000) -- (87.5000,142.1000) -- (87.5000,142.1000) -- (87.5000,142.1000) -- (87.5000,142.1000) -- (87.5000,142.1000) -- (87.5000,142.1000) -- (87.5000,142.0000) -- (87.5000,142.0000) -- (87.5000,142.0000) -- (87.5000,142.0000) -- (87.5000,142.0000) -- (87.5000,142.0000) -- (87.5000,142.0000) -- (87.5000,142.0000) -- (87.5000,142.0000) -- (87.5000,142.0000) -- (87.5000,142.0000) -- (87.5000,142.0000) -- (87.5000,142.0000) -- (87.5000,142.0000) -- (87.5000,142.0000) -- (87.5000,142.0000) -- (87.5000,142.0000) -- (87.5000,142.0000) -- (87.5000,142.0000) -- (87.5000,142.0000) -- (87.5000,142.0000) -- (87.5000,141.9000) -- (87.5000,141.9000) -- (87.5000,141.9000) -- (87.5000,141.9000) -- (87.5000,141.9000) -- (87.5000,141.9000) -- (87.5000,141.9000) -- (87.6000,141.9000) -- (87.6000,141.9000) -- (87.6000,141.9000) -- (87.6000,141.9000) -- (87.6000,141.9000) -- (87.6000,141.9000) -- (87.6000,141.9000) -- (87.6000,141.9000) -- (87.6000,141.9000) -- (87.6000,141.9000) -- (87.6000,141.9000) -- (87.6000,141.9000) -- (87.6000,141.9000) -- (87.6000,141.9000) -- (87.6000,141.9000) -- (87.6000,141.8000) -- (87.6000,141.8000) -- (87.6000,141.8000) -- (87.6000,141.8000) -- (87.6000,141.8000) -- (87.6000,141.8000) -- (87.6000,141.8000) -- (87.6000,141.8000) -- (87.6000,141.8000) -- (87.6000,141.8000) -- (87.6000,141.8000) -- (87.6000,141.8000) -- (87.6000,141.8000) -- (87.6000,141.8000) -- (87.6000,141.8000) -- (87.6000,141.8000) -- (87.6000,141.8000) -- (87.6000,141.8000) -- (87.6000,141.8000) -- (87.6000,141.8000) -- (87.6000,141.8000) -- (87.6000,141.7000) -- (87.6000,141.7000) -- (87.6000,141.7000) -- (87.6000,141.7000) -- (87.6000,141.7000) -- (87.6000,141.7000) -- (87.6000,141.7000) -- (87.6000,141.7000) -- (87.6000,141.7000) -- (87.6000,141.7000) -- (87.6000,141.7000) -- (87.6000,141.7000) -- (87.6000,141.7000) -- (87.6000,141.7000) -- (87.7000,141.7000) -- (87.7000,141.7000) -- (87.7000,141.7000) -- (87.7000,141.7000) -- (87.7000,141.7000) -- (87.7000,141.7000) -- (87.7000,141.7000) -- (87.7000,141.7000) -- (87.7000,141.6000) -- (87.7000,141.6000) -- (87.7000,141.6000) -- (87.7000,141.6000) -- (87.7000,141.6000) -- (87.7000,141.6000) -- (87.7000,141.6000) -- (87.7000,141.6000) -- (87.7000,141.6000) -- (87.7000,141.6000) -- (87.7000,141.6000) -- (87.7000,141.6000) -- (87.7000,141.6000) -- (87.7000,141.6000) -- (87.7000,141.6000) -- (87.7000,141.6000) -- (87.7000,141.6000) -- (87.7000,141.6000) -- (87.7000,141.6000) -- (87.7000,141.6000) -- (87.7000,141.6000) -- (87.7000,141.5000) -- (87.7000,141.5000) -- (87.7000,141.5000) -- (87.7000,141.5000) -- (87.7000,141.5000) -- (87.7000,141.5000) -- (87.7000,141.5000) -- (87.7000,141.5000) -- (87.7000,141.5000) -- (87.7000,141.5000) -- (87.7000,141.5000) -- (87.7000,141.5000) -- (87.7000,141.5000) -- (87.7000,141.5000) -- (87.7000,141.5000) -- (87.7000,141.5000) -- (87.7000,141.5000) -- (87.7000,141.5000) -- (87.7000,141.5000) -- (87.7000,141.5000) -- (87.8000,141.5000) -- (87.8000,141.5000) -- (87.8000,141.4000) -- (87.8000,141.4000) -- (87.8000,141.4000) -- (87.8000,141.4000) -- (87.8000,141.4000) -- (87.8000,141.4000) -- (87.8000,141.4000) -- (87.8000,141.4000) -- (87.8000,141.4000) -- (87.8000,141.4000) -- (87.8000,141.4000) -- (87.8000,141.4000) -- (87.8000,141.4000) -- (87.8000,141.4000) -- (87.8000,141.4000) -- (87.8000,141.4000) -- (87.8000,141.4000) -- (87.8000,141.4000) -- (87.8000,141.4000) -- (87.8000,141.4000) -- (87.8000,141.4000) -- (87.8000,141.3000) -- (87.8000,141.3000) -- (87.8000,141.3000) -- (87.8000,141.3000) -- (87.8000,141.3000) -- (87.8000,141.3000) -- (87.8000,141.3000) -- (87.8000,141.3000) -- (87.8000,141.3000) -- (87.8000,141.3000) -- (87.8000,141.3000) -- (87.8000,141.3000) -- (87.8000,141.3000) -- (87.8000,141.3000) -- (87.8000,141.3000) -- (87.8000,141.3000) -- (87.8000,141.3000) -- (87.8000,141.3000) -- (87.8000,141.3000) -- (87.8000,141.3000) -- (87.8000,141.3000) -- (87.8000,141.3000) -- (87.8000,141.2000) -- (87.8000,141.2000) -- (87.8000,141.2000) -- (87.8000,141.2000) -- (87.8000,141.2000) -- (87.9000,141.2000) -- (87.9000,141.2000) -- (87.9000,141.2000) -- (87.9000,141.2000) -- (87.9000,141.2000) -- (87.9000,141.2000) -- (87.9000,141.2000) -- (87.9000,141.2000) -- (87.9000,141.2000) -- (87.9000,141.2000) -- (87.9000,141.2000) -- (87.9000,141.2000) -- (87.9000,141.2000) -- (87.9000,141.2000) -- (87.9000,141.2000) -- (87.9000,141.2000) -- (87.9000,141.1000) -- (87.9000,141.1000) -- (87.9000,141.1000) -- (87.9000,141.1000) -- (87.9000,141.1000) -- (87.9000,141.1000) -- (87.9000,141.1000) -- (87.9000,141.1000) -- (87.9000,141.1000) -- (87.9000,141.1000) -- (87.9000,141.1000) -- (87.9000,141.1000) -- (87.9000,141.1000) -- (87.9000,141.1000) -- (87.9000,141.1000) -- (87.9000,141.1000) -- (87.9000,141.1000) -- (87.9000,141.1000) -- (87.9000,141.1000) -- (87.9000,141.1000) -- (87.9000,141.1000) -- (87.9000,141.1000) -- (87.9000,141.0000) -- (87.9000,141.0000) -- (87.9000,141.0000) -- (87.9000,141.0000) -- (87.9000,141.0000) -- (87.9000,141.0000) -- (87.9000,141.0000) -- (87.9000,141.0000) -- (87.9000,141.0000) -- (87.9000,141.0000) -- (87.9000,141.0000) -- (88.0000,141.0000) -- (88.0000,141.0000) -- (88.0000,141.0000) -- (88.0000,141.0000) -- (88.0000,141.0000) -- (88.0000,141.0000) -- (88.0000,141.0000) -- (88.0000,141.0000) -- (88.0000,141.0000) -- (88.0000,141.0000) -- (88.0000,140.9000) -- (88.0000,140.9000) -- (88.0000,140.9000) -- (88.0000,140.9000) -- (88.0000,140.9000) -- (88.0000,140.9000) -- (88.0000,140.9000) -- (88.0000,140.9000) -- (88.0000,140.9000) -- (88.0000,140.9000) -- (88.0000,140.9000) -- (88.0000,140.9000) -- (88.0000,140.9000) -- (88.0000,140.9000) -- (88.0000,140.9000) -- (88.0000,140.9000) -- (88.0000,140.9000) -- (88.0000,140.9000) -- (88.0000,140.9000) -- (88.0000,140.9000) -- (88.0000,140.9000) -- (88.0000,140.9000) -- (88.0000,140.8000) -- (88.0000,140.8000) -- (88.0000,140.8000) -- (88.0000,140.8000) -- (88.0000,140.8000) -- (88.0000,140.8000) -- (88.0000,140.8000) -- (88.0000,140.8000) -- (88.0000,140.8000) -- (88.0000,140.8000) -- (88.0000,140.8000) -- (88.0000,140.8000) -- (88.0000,140.8000) -- (88.0000,140.8000) -- (88.0000,140.8000) -- (88.0000,140.8000) -- (88.0000,140.8000) -- (88.0000,140.8000) -- (88.1000,140.8000) -- (88.1000,140.8000) -- (88.1000,140.8000) -- (88.1000,140.7000) -- (88.1000,140.7000) -- (88.1000,140.7000) -- (88.1000,140.7000) -- (88.1000,140.7000) -- (88.1000,140.7000) -- (88.1000,140.7000) -- (88.1000,140.7000) -- (88.1000,140.7000) -- (88.1000,140.7000) -- (88.1000,140.7000) -- (88.1000,140.7000) -- (88.1000,140.7000) -- (88.1000,140.7000) -- (88.1000,140.7000) -- (88.1000,140.7000) -- (88.1000,140.7000) -- (88.1000,140.7000) -- (88.1000,140.7000) -- (88.1000,140.7000) -- (88.1000,140.7000) -- (88.1000,140.7000) -- (88.1000,140.6000) -- (88.1000,140.6000) -- (88.1000,140.6000) -- (88.1000,140.6000) -- (88.1000,140.6000) -- (88.1000,140.6000) -- (88.1000,140.6000) -- (88.1000,140.6000) -- (88.1000,140.6000) -- (88.1000,140.6000) -- (88.1000,140.6000) -- (88.1000,140.6000) -- (88.1000,140.6000) -- (88.1000,140.6000) -- (88.1000,140.6000) -- (88.1000,140.6000) -- (88.1000,140.6000) -- (88.1000,140.6000) -- (88.1000,140.6000) -- (88.1000,140.6000) -- (88.1000,140.6000) -- (88.1000,140.5000) -- (88.1000,140.5000) -- (88.1000,140.5000) -- (88.2000,140.5000) -- (88.2000,140.5000) -- (88.2000,140.5000) -- (88.2000,140.5000) -- (88.2000,140.5000) -- (88.2000,140.5000) -- (88.2000,140.5000) -- (88.2000,140.5000) -- (88.2000,140.5000) -- (88.2000,140.5000) -- (88.2000,140.5000) -- (88.2000,140.5000) -- (88.2000,140.5000) -- (88.2000,140.5000) -- (88.2000,140.5000) -- (88.2000,140.5000) -- (88.2000,140.5000) -- (88.2000,140.5000) -- (88.2000,140.5000) -- (88.2000,140.4000) -- (88.2000,140.4000) -- (88.2000,140.4000) -- (88.2000,140.4000) -- (88.2000,140.4000) -- (88.2000,140.4000) -- (88.2000,140.4000) -- (88.2000,140.4000) -- (88.2000,140.4000) -- (88.2000,140.4000) -- (88.2000,140.4000) -- (88.2000,140.4000) -- (88.2000,140.4000) -- (88.2000,140.4000) -- (88.2000,140.4000) -- (88.2000,140.4000) -- (88.2000,140.4000) -- (88.2000,140.4000) -- (88.2000,140.4000) -- (88.2000,140.4000) -- (88.2000,140.4000) -- (88.2000,140.3000) -- (88.2000,140.3000) -- (88.2000,140.3000) -- (88.2000,140.3000) -- (88.2000,140.3000) -- (88.2000,140.3000) -- (88.2000,140.3000) -- (88.2000,140.3000) -- (88.2000,140.3000) -- (88.2000,140.3000) -- (88.3000,140.3000) -- (88.3000,140.3000) -- (88.3000,140.3000) -- (88.3000,140.3000) -- (88.3000,140.3000) -- (88.3000,140.3000) -- (88.3000,140.3000) -- (88.3000,140.3000) -- (88.3000,140.3000) -- (88.3000,140.3000) -- (88.3000,140.3000) -- (88.3000,140.3000) -- (88.3000,140.2000) -- (88.3000,140.2000) -- (88.3000,140.2000) -- (88.3000,140.2000) -- (88.3000,140.2000) -- (88.3000,140.2000) -- (88.3000,140.2000) -- (88.3000,140.2000) -- (88.3000,140.2000) -- (88.3000,140.2000) -- (88.3000,140.2000) -- (88.3000,140.2000) -- (88.3000,140.2000) -- (88.3000,140.2000) -- (88.3000,140.2000) -- (88.3000,140.2000) -- (88.3000,140.2000) -- (88.3000,140.2000) -- (88.3000,140.2000) -- (88.3000,140.2000) -- (88.3000,140.2000) -- (88.3000,140.1000) -- (88.3000,140.1000) -- (88.3000,140.1000) -- (88.3000,140.1000) -- (88.3000,140.1000) -- (88.3000,140.1000) -- (88.3000,140.1000) -- (88.3000,140.1000) -- (88.3000,140.1000) -- (88.3000,140.1000) -- (88.3000,140.1000) -- (88.3000,140.1000) -- (88.3000,140.1000) -- (88.3000,140.1000) -- (88.3000,140.1000) -- (88.3000,140.1000) -- (88.4000,140.1000) -- (88.4000,140.1000) -- (88.4000,140.1000) -- (88.4000,140.1000) -- (88.4000,140.1000) -- (88.4000,140.1000) -- (88.4000,140.0000) -- (88.4000,140.0000) -- (88.4000,140.0000) -- (88.4000,140.0000) -- (88.4000,140.0000) -- (88.4000,140.0000) -- (88.4000,140.0000) -- (88.4000,140.0000) -- (88.4000,140.0000) -- (88.4000,140.0000) -- (88.4000,140.0000) -- (88.4000,140.0000) -- (88.4000,140.0000) -- (88.4000,140.0000) -- (88.4000,140.0000) -- (88.4000,140.0000) -- (88.4000,140.0000) -- (88.4000,140.0000) -- (88.4000,140.0000) -- (88.4000,140.0000) -- (88.4000,140.0000) -- (88.4000,139.9000) -- (88.4000,139.9000) -- (88.4000,139.9000) -- (88.4000,139.9000) -- (88.4000,139.9000) -- (88.4000,139.9000) -- (88.4000,139.9000) -- (88.4000,139.9000) -- (88.4000,139.9000) -- (88.4000,139.9000) -- (88.4000,139.9000) -- (88.4000,139.9000) -- (88.4000,139.9000) -- (88.4000,139.9000) -- (88.4000,139.9000) -- (88.4000,139.9000) -- (88.4000,139.9000) -- (88.4000,139.9000) -- (88.4000,139.9000) -- (88.4000,139.9000) -- (88.4000,139.9000) -- (88.4000,139.9000) -- (88.4000,139.8000) -- (88.5000,139.8000) -- (88.5000,139.8000) -- (88.5000,139.8000) -- (88.5000,139.8000) -- (88.5000,139.8000) -- (88.5000,139.8000) -- (88.5000,139.8000) -- (88.5000,139.8000) -- (88.5000,139.8000) -- (88.5000,139.8000) -- (88.5000,139.8000) -- (88.5000,139.8000) -- (88.5000,139.8000) -- (88.5000,139.8000) -- (88.5000,139.8000) -- (88.5000,139.8000) -- (88.5000,139.8000) -- (96.0000,139.8000) -- (96.0000,139.8000) -- (96.0000,139.8000) -- (96.0000,139.7000) -- (96.0000,139.7000) -- (96.0000,139.7000) -- (96.0000,139.7000) -- (96.0000,139.7000) -- (96.0000,139.7000) -- (96.0000,139.7000) -- (96.0000,139.7000) -- (96.0000,139.7000) -- (96.0000,139.7000) -- (96.0000,139.7000) -- (96.0000,139.7000) -- (96.0000,139.7000) -- (96.0000,139.7000) -- (96.0000,139.7000) -- (96.0000,139.7000) -- (96.0000,139.7000) -- (96.0000,139.7000) -- (96.0000,139.7000) -- (96.0000,139.7000) -- (96.0000,139.7000) -- (96.0000,139.7000) -- (96.0000,139.6000) -- (96.0000,139.6000) -- (96.0000,139.6000) -- (96.0000,139.6000) -- (96.0000,139.6000) -- (96.0000,139.6000) -- (96.0000,139.6000) -- (96.0000,139.6000) -- (96.0000,139.6000) -- (96.0000,139.6000) -- (96.0000,139.6000) -- (96.0000,139.6000) -- (96.0000,139.6000) -- (96.0000,139.6000) -- (96.1000,139.6000) -- (96.1000,139.6000) -- (96.1000,139.6000) -- (96.1000,139.6000) -- (96.1000,139.6000) -- (96.1000,139.6000) -- (96.1000,139.6000) -- (96.1000,139.5000) -- (96.1000,139.5000) -- (96.1000,139.5000) -- (96.1000,139.5000) -- (96.1000,139.5000) -- (96.1000,139.5000) -- (96.1000,139.5000) -- (96.1000,139.5000) -- (96.1000,139.5000) -- (96.1000,139.5000) -- (96.1000,139.5000) -- (96.1000,139.5000) -- (96.1000,139.5000) -- (96.1000,139.5000) -- (96.1000,139.5000) -- (96.1000,139.5000) -- (96.1000,139.5000) -- (96.1000,139.5000) -- (96.1000,139.5000) -- (96.1000,139.5000) -- (96.1000,139.5000) -- (96.1000,139.5000) -- (96.1000,139.4000) -- (96.1000,139.4000) -- (96.1000,139.4000) -- (96.1000,139.4000) -- (96.1000,139.4000) -- (96.1000,139.4000) -- (96.1000,139.4000) -- (96.1000,139.4000) -- (96.1000,139.4000) -- (96.1000,139.4000) -- (96.1000,139.4000) -- (96.1000,139.4000) -- (96.1000,139.4000) -- (96.1000,139.4000) -- (96.1000,139.4000) -- (96.1000,139.4000) -- (96.1000,139.4000) -- (96.1000,139.4000) -- (96.1000,139.4000) -- (96.1000,139.4000) -- (96.2000,139.4000) -- (96.2000,139.3000) -- (96.2000,139.3000) -- (96.2000,139.3000) -- (96.2000,139.3000) -- (96.2000,139.3000) -- (96.2000,139.3000) -- (96.2000,139.3000) -- (96.2000,139.3000) -- (96.2000,139.3000) -- (96.2000,139.3000) -- (96.2000,139.3000) -- (96.2000,139.3000) -- (96.2000,139.3000) -- (96.2000,139.3000) -- (96.2000,139.3000) -- (96.2000,139.3000) -- (96.2000,139.3000) -- (96.2000,139.3000) -- (96.2000,139.3000) -- (96.2000,139.3000) -- (96.2000,139.3000) -- (96.2000,139.3000) -- (96.2000,139.2000) -- (96.2000,139.2000) -- (96.2000,139.2000) -- (96.2000,139.2000) -- (96.2000,139.2000) -- (96.2000,139.2000) -- (96.2000,139.2000) -- (96.2000,139.2000) -- (96.2000,139.2000) -- (96.2000,139.2000) -- (96.2000,139.2000) -- (96.2000,139.2000) -- (96.2000,139.2000) -- (96.2000,139.2000) -- (96.2000,139.2000) -- (96.2000,139.2000) -- (96.2000,139.2000) -- (96.2000,139.2000) -- (96.2000,139.2000) -- (96.2000,139.2000) -- (96.2000,139.2000) -- (96.2000,139.1000) -- (96.2000,139.1000) -- (96.2000,139.1000) -- (96.2000,139.1000) -- (96.2000,139.1000) -- (96.2000,139.1000) -- (96.3000,139.1000) -- (96.3000,139.1000) -- (96.3000,139.1000) -- (96.3000,139.1000) -- (96.3000,139.1000) -- (96.3000,139.1000) -- (96.3000,139.1000) -- (96.3000,139.1000) -- (96.3000,139.1000) -- (96.3000,139.1000) -- (96.3000,139.1000) -- (96.3000,139.1000) -- (96.3000,139.1000) -- (96.3000,139.1000) -- (96.3000,139.1000) -- (96.3000,139.1000) -- (96.3000,139.0000) -- (96.3000,139.0000) -- (96.3000,139.0000) -- (96.3000,139.0000) -- (96.3000,139.0000) -- (96.3000,139.0000) -- (96.3000,139.0000) -- (96.3000,139.0000) -- (96.3000,139.0000) -- (96.3000,139.0000) -- (96.3000,139.0000) -- (96.3000,139.0000) -- (96.3000,139.0000) -- (96.3000,139.0000) -- (96.3000,139.0000) -- (96.3000,139.0000) -- (96.3000,139.0000) -- (96.3000,139.0000) -- (96.3000,139.0000) -- (96.3000,139.0000) -- (96.3000,139.0000) -- (96.3000,138.9000) -- (96.3000,138.9000) -- (96.3000,138.9000) -- (96.3000,138.9000) -- (96.3000,138.9000) -- (96.3000,138.9000) -- (96.3000,138.9000) -- (96.3000,138.9000) -- (96.3000,138.9000) -- (96.3000,138.9000) -- (96.3000,138.9000) -- (96.3000,138.9000) -- (96.3000,138.9000) -- (96.4000,138.9000) -- (96.4000,138.9000) -- (96.4000,138.9000) -- (96.4000,138.9000) -- (96.4000,138.9000) -- (96.4000,138.9000) -- (96.4000,138.9000) -- (96.4000,138.9000) -- (96.4000,138.9000) -- (96.4000,138.8000) -- (96.4000,138.8000) -- (96.4000,138.8000) -- (96.4000,138.8000) -- (96.4000,138.8000) -- (96.4000,138.8000) -- (96.4000,138.8000) -- (96.4000,138.8000) -- (96.4000,138.8000) -- (96.4000,138.8000) -- (96.4000,138.8000) -- (96.4000,138.8000) -- (96.4000,138.8000) -- (96.4000,138.8000) -- (96.4000,138.8000) -- (96.4000,138.8000) -- (96.4000,138.8000) -- (96.4000,138.8000) -- (96.4000,138.8000) -- (96.4000,138.8000) -- (96.4000,138.8000) -- (96.4000,138.7000) -- (96.4000,138.7000) -- (96.4000,138.7000) -- (96.4000,138.7000) -- (96.4000,138.7000) -- (96.4000,138.7000) -- (96.4000,138.7000) -- (96.4000,138.7000) -- (96.4000,138.7000) -- (96.4000,138.7000) -- (96.4000,138.7000) -- (96.4000,138.7000) -- (96.4000,138.7000) -- (96.4000,138.7000) -- (96.4000,138.7000) -- (96.4000,138.7000) -- (96.4000,138.7000) -- (96.4000,138.7000) -- (96.4000,138.7000) -- (96.5000,138.7000) -- (96.5000,138.7000) -- (96.5000,138.7000) -- (96.5000,138.6000) -- (96.5000,138.6000) -- (96.5000,138.6000) -- (96.5000,138.6000) -- (96.5000,138.6000) -- (96.5000,138.6000) -- (96.5000,138.6000) -- (96.5000,138.6000) -- (96.5000,138.6000) -- (96.5000,138.6000) -- (96.5000,138.6000) -- (96.5000,138.6000) -- (96.5000,138.6000) -- (96.5000,138.6000) -- (96.5000,138.6000) -- (96.5000,138.6000) -- (96.5000,138.6000) -- (96.5000,138.6000) -- (96.5000,138.6000) -- (96.5000,138.6000) -- (96.5000,138.6000) -- (96.5000,138.5000) -- (96.5000,138.5000) -- (96.5000,138.5000) -- (96.5000,138.5000) -- (96.5000,138.5000) -- (96.5000,138.5000) -- (96.5000,138.5000) -- (96.5000,138.5000) -- (96.5000,138.5000) -- (96.5000,138.5000) -- (96.5000,138.5000) -- (96.5000,138.5000) -- (96.5000,138.5000) -- (96.5000,138.5000) -- (96.5000,138.5000) -- (96.5000,138.5000) -- (96.5000,138.5000) -- (96.5000,138.5000) -- (96.5000,138.5000) -- (96.5000,138.5000) -- (96.5000,138.5000) -- (96.5000,138.5000) -- (96.5000,138.4000) -- (96.5000,138.4000) -- (96.5000,138.4000) -- (96.5000,138.4000) -- (96.6000,138.4000) -- (96.6000,138.4000) -- (96.6000,138.4000) -- (96.6000,138.4000) -- (96.6000,138.4000) -- (96.6000,138.4000) -- (96.6000,138.4000) -- (96.6000,138.4000) -- (96.6000,138.4000) -- (96.6000,138.4000) -- (96.6000,138.4000) -- (96.6000,138.4000) -- (96.6000,138.4000) -- (96.6000,138.4000) -- (96.6000,138.4000) -- (96.6000,138.4000) -- (96.6000,138.4000) -- (96.6000,138.3000) -- (96.6000,138.3000) -- (96.6000,138.3000) -- (96.6000,138.3000) -- (96.6000,138.3000) -- (96.6000,138.3000) -- (96.6000,138.3000) -- (96.6000,138.3000) -- (96.6000,138.3000) -- (96.6000,138.3000) -- (96.6000,138.3000) -- (96.6000,138.3000) -- (96.6000,138.3000) -- (96.6000,138.3000) -- (96.6000,138.3000) -- (96.6000,138.3000) -- (96.6000,138.3000) -- (96.6000,138.3000) -- (96.6000,138.3000) -- (96.6000,138.3000) -- (96.6000,138.3000) -- (96.6000,138.3000) -- (96.6000,138.2000) -- (96.6000,138.2000) -- (96.6000,138.2000) -- (96.6000,138.2000) -- (96.6000,138.2000) -- (96.6000,138.2000) -- (96.6000,138.2000) -- (96.6000,138.2000) -- (96.6000,138.2000) -- (96.6000,138.2000) -- (96.7000,138.2000) -- (96.7000,138.2000) -- (96.7000,138.2000) -- (96.7000,138.2000) -- (96.7000,138.2000) -- (96.7000,138.2000) -- (96.7000,138.2000) -- (96.7000,138.2000) -- (96.7000,138.2000) -- (96.7000,138.2000) -- (96.7000,138.2000) -- (96.7000,138.1000) -- (96.7000,138.1000) -- (96.7000,138.1000) -- (96.7000,138.1000) -- (96.7000,138.1000) -- (96.7000,138.1000) -- (96.7000,138.1000) -- (96.7000,138.1000) -- (96.7000,138.1000) -- (96.7000,138.1000) -- (96.7000,138.1000) -- (96.7000,138.1000) -- (96.7000,138.1000) -- (96.7000,138.1000) -- (96.7000,138.1000) -- (96.7000,138.1000) -- (96.7000,138.1000) -- (96.7000,138.1000) -- (96.7000,138.1000) -- (96.7000,138.1000) -- (96.7000,138.1000) -- (96.7000,138.1000) -- (96.7000,138.0000) -- (96.7000,138.0000) -- (96.7000,138.0000) -- (96.7000,138.0000) -- (96.7000,138.0000) -- (96.7000,138.0000) -- (96.7000,138.0000) -- (96.7000,138.0000) -- (96.7000,138.0000) -- (96.7000,138.0000) -- (96.7000,138.0000) -- (96.7000,138.0000) -- (96.7000,138.0000) -- (96.7000,138.0000) -- (96.7000,138.0000) -- (96.7000,138.0000) -- (96.7000,138.0000) -- (96.8000,138.0000) -- (96.8000,138.0000) -- (96.8000,138.0000) -- (96.8000,138.0000) -- (96.8000,137.9000) -- (96.8000,137.9000) -- (96.8000,137.9000) -- (96.8000,137.9000) -- (96.8000,137.9000) -- (96.8000,137.9000) -- (96.8000,137.9000) -- (96.8000,137.9000) -- (96.8000,137.9000) -- (96.8000,137.9000) -- (96.8000,137.9000) -- (96.8000,137.9000) -- (96.8000,137.9000) -- (96.8000,137.9000) -- (96.8000,137.9000) -- (96.8000,137.9000) -- (96.8000,137.9000) -- (96.8000,137.9000) -- (96.8000,137.9000) -- (96.8000,137.9000) -- (96.8000,137.9000) -- (96.8000,137.9000) -- (96.8000,137.8000) -- (96.8000,137.8000) -- (96.8000,137.8000) -- (96.8000,137.8000) -- (96.8000,137.8000) -- (96.8000,137.8000) -- (96.8000,137.8000) -- (96.8000,137.8000) -- (96.8000,137.8000) -- (96.8000,137.8000) -- (96.8000,137.8000) -- (96.8000,137.8000) -- (96.8000,137.8000) -- (96.8000,137.8000) -- (96.8000,137.8000) -- (96.8000,137.8000) -- (96.8000,137.8000) -- (96.8000,137.8000) -- (96.8000,137.8000) -- (96.8000,137.8000) -- (96.8000,137.8000) -- (96.8000,137.7000) -- (96.8000,137.7000) -- (96.9000,137.7000) -- (96.9000,137.7000) -- (96.9000,137.7000) -- (96.9000,137.7000) -- (96.9000,137.7000) -- (96.9000,137.7000) -- (96.9000,137.7000) -- (96.9000,137.7000) -- (96.9000,137.7000) -- (96.9000,137.7000) -- (96.9000,137.7000) -- (96.9000,137.7000) -- (96.9000,137.7000) -- (96.9000,137.7000) -- (96.9000,137.7000) -- (96.9000,137.7000) -- (96.9000,137.7000) -- (96.9000,137.7000) -- (96.9000,137.7000) -- (96.9000,137.7000) -- (96.9000,137.6000) -- (96.9000,137.6000) -- (96.9000,137.6000) -- (96.9000,137.6000) -- (96.9000,137.6000) -- (96.9000,137.6000) -- (96.9000,137.6000) -- (96.9000,137.6000) -- (96.9000,137.6000) -- (96.9000,137.6000) -- (96.9000,137.6000) -- (96.9000,137.6000) -- (96.9000,137.6000) -- (96.9000,137.6000) -- (96.9000,137.6000) -- (96.9000,137.6000) -- (96.9000,137.6000) -- (96.9000,137.6000) -- (96.9000,137.6000) -- (96.9000,137.6000) -- (96.9000,137.6000) -- (96.9000,137.5000) -- (96.9000,137.5000) -- (96.9000,137.5000) -- (96.9000,137.5000) -- (96.9000,137.5000) -- (96.9000,137.5000) -- (96.9000,137.5000) -- (96.9000,137.5000) -- (96.9000,137.5000) -- (97.0000,137.5000) -- (97.0000,137.5000) -- (97.0000,137.5000) -- (97.0000,137.5000) -- (97.0000,137.5000) -- (97.0000,137.5000) -- (97.0000,137.5000) -- (97.0000,137.5000) -- (97.0000,137.5000) -- (97.0000,137.5000) -- (97.0000,137.5000) -- (97.0000,137.5000) -- (97.0000,137.5000) -- (97.0000,137.4000) -- (97.0000,137.4000) -- (97.0000,137.4000) -- (97.0000,137.4000) -- (97.0000,137.4000) -- (97.0000,137.4000) -- (97.0000,137.4000) -- (97.0000,137.4000) -- (97.0000,137.4000) -- (97.0000,137.4000) -- (97.0000,137.4000) -- (97.0000,137.4000) -- (97.0000,137.4000) -- (97.0000,137.4000) -- (97.0000,137.4000) -- (97.0000,137.4000) -- (97.0000,137.4000) -- (97.0000,137.4000) -- (97.0000,137.4000) -- (97.0000,137.4000) -- (97.0000,137.4000) -- (97.0000,137.3000) -- (97.0000,137.3000) -- (97.0000,137.3000) -- (97.0000,137.3000) -- (97.0000,137.3000) -- (97.0000,137.3000) -- (97.0000,137.3000) -- (97.0000,137.3000) -- (97.0000,137.3000) -- (97.0000,137.3000) -- (97.0000,137.3000) -- (97.0000,137.3000) -- (97.0000,137.3000) -- (97.0000,137.3000) -- (97.0000,137.3000) -- (97.1000,137.3000) -- (97.1000,137.3000) -- (97.1000,137.3000) -- (97.1000,137.3000) -- (97.1000,137.3000) -- (97.1000,137.3000) -- (97.1000,137.3000) -- (97.1000,137.2000) -- (97.1000,137.2000) -- (97.1000,137.2000) -- (97.1000,137.2000) -- (97.1000,137.2000) -- (97.1000,137.2000) -- (97.1000,137.2000) -- (97.1000,137.2000) -- (97.1000,137.2000) -- (97.1000,137.2000) -- (97.1000,137.2000) -- (97.1000,137.2000) -- (97.1000,137.2000) -- (97.1000,137.2000) -- (97.1000,137.2000) -- (97.1000,137.2000) -- (97.1000,137.2000) -- (97.1000,137.2000) -- (97.1000,137.2000) -- (97.1000,137.2000) -- (97.1000,137.2000) -- (97.1000,137.1000) -- (97.1000,137.1000) -- (97.1000,137.1000) -- (97.1000,137.1000) -- (97.1000,137.1000) -- (97.1000,137.1000) -- (97.1000,137.1000) -- (97.1000,137.1000) -- (97.1000,137.1000) -- (97.1000,137.1000) -- (97.1000,137.1000) -- (97.1000,137.1000) -- (97.1000,137.1000) -- (97.1000,137.1000) -- (97.1000,137.1000) -- (97.1000,137.1000) -- (97.1000,137.1000) -- (97.1000,137.1000) -- (97.1000,137.1000) -- (97.1000,137.1000) -- (97.1000,137.1000) -- (97.1000,137.1000) -- (97.2000,137.0000) -- (97.2000,137.0000) -- (97.2000,137.0000) -- (97.2000,137.0000) -- (97.2000,137.0000) -- (97.2000,137.0000) -- (97.2000,137.0000) -- (97.2000,137.0000) -- (97.2000,137.0000) -- (97.2000,137.0000) -- (97.2000,137.0000) -- (97.2000,137.0000) -- (97.2000,137.0000) -- (97.2000,137.0000) -- (97.2000,137.0000) -- (97.2000,137.0000) -- (97.2000,137.0000) -- (97.2000,137.0000) -- (97.2000,137.0000) -- (97.2000,137.0000) -- (97.2000,137.0000) -- (97.2000,136.9000) -- (97.2000,136.9000) -- (97.2000,136.9000) -- (97.2000,136.9000) -- (97.2000,136.9000) -- (97.2000,136.9000) -- (97.2000,136.9000) -- (97.2000,136.9000) -- (97.2000,136.9000) -- (97.2000,136.9000) -- (97.2000,136.9000) -- (97.2000,136.9000) -- (97.2000,136.9000) -- (97.2000,136.9000) -- (97.2000,136.9000) -- (97.2000,136.9000) -- (97.2000,136.9000) -- (97.2000,136.9000) -- (97.2000,136.9000) -- (97.2000,136.9000) -- (97.2000,136.9000) -- (97.2000,136.9000) -- (97.2000,136.8000) -- (97.2000,136.8000) -- (97.2000,136.8000) -- (97.2000,136.8000) -- (97.2000,136.8000) -- (97.2000,136.8000) -- (97.3000,136.8000) -- (97.3000,136.8000) -- (97.3000,136.8000) -- (97.3000,136.8000) -- (97.3000,136.8000) -- (97.3000,136.8000) -- (97.3000,136.8000) -- (97.3000,136.8000) -- (97.3000,136.8000) -- (97.3000,136.8000) -- (97.3000,136.8000) -- (97.3000,136.8000) -- (97.3000,136.8000) -- (97.3000,136.8000) -- (97.3000,136.8000) -- (97.3000,136.7000) -- (97.3000,136.7000) -- (97.3000,136.7000) -- (97.3000,136.7000) -- (97.3000,136.7000) -- (97.3000,136.7000) -- (97.3000,136.7000) -- (97.3000,136.7000) -- (97.3000,136.7000) -- (97.3000,136.7000) -- (97.3000,136.7000) -- (97.3000,136.7000) -- (97.3000,136.7000) -- (97.3000,136.7000) -- (97.3000,136.7000) -- (97.3000,136.7000) -- (97.3000,136.7000) -- (97.3000,136.7000) -- (97.3000,136.7000) -- (97.3000,136.7000) -- (97.3000,136.7000) -- (97.3000,136.7000) -- (97.3000,136.6000) -- (97.3000,136.6000) -- (97.3000,136.6000) -- (97.3000,136.6000) -- (97.3000,136.6000) -- (97.3000,136.6000) -- (97.3000,136.6000) -- (97.3000,136.6000) -- (97.3000,136.6000) -- (97.3000,136.6000) -- (97.3000,136.6000) -- (97.3000,136.6000) -- (97.3000,136.6000) -- (97.4000,136.6000) -- (97.4000,136.6000) -- (97.4000,136.6000) -- (97.4000,136.6000) -- (97.4000,136.6000) -- (97.4000,136.6000) -- (97.4000,136.6000) -- (97.4000,136.6000) -- (97.4000,136.5000) -- (97.4000,136.5000) -- (97.4000,136.5000) -- (97.4000,136.5000) -- (97.4000,136.5000) -- (97.4000,136.5000) -- (97.4000,136.5000) -- (97.4000,136.5000) -- (97.4000,136.5000) -- (97.4000,136.5000) -- (97.4000,136.5000) -- (97.4000,136.5000) -- (97.4000,136.5000) -- (97.4000,136.5000) -- (97.4000,136.5000) -- (97.4000,136.5000) -- (97.4000,136.5000) -- (97.4000,136.5000) -- (97.4000,136.5000) -- (97.4000,136.5000) -- (97.4000,136.5000) -- (97.4000,136.5000) -- (97.4000,136.4000) -- (97.4000,136.4000) -- (97.4000,136.4000) -- (97.4000,136.4000) -- (97.4000,136.4000) -- (97.4000,136.4000) -- (97.4000,136.4000) -- (97.4000,136.4000) -- (97.4000,136.4000) -- (97.4000,136.4000) -- (97.4000,136.4000) -- (97.4000,136.4000) -- (97.4000,136.4000) -- (97.4000,136.4000) -- (97.4000,136.4000) -- (97.4000,136.4000) -- (97.4000,136.4000) -- (97.4000,136.4000) -- (97.4000,136.4000) -- (97.5000,136.4000) -- (97.5000,136.4000) -- (97.5000,136.3000) -- (97.5000,136.3000) -- (97.5000,136.3000) -- (97.5000,136.3000) -- (97.5000,136.3000) -- (97.5000,136.3000) -- (97.5000,136.3000) -- (97.5000,136.3000) -- (97.5000,136.3000) -- (97.5000,136.3000) -- (97.5000,136.3000) -- (97.5000,136.3000) -- (97.5000,136.3000) -- (97.5000,136.3000) -- (97.5000,136.3000) -- (97.5000,136.3000) -- (97.5000,136.3000) -- (97.5000,136.3000) -- (97.5000,136.3000) -- (97.5000,136.3000) -- (97.5000,136.3000) -- (97.5000,136.3000) -- (97.5000,136.2000) -- (97.5000,136.2000) -- (97.5000,136.2000) -- (97.5000,136.2000) -- (97.5000,136.2000) -- (97.5000,136.2000) -- (97.5000,136.2000) -- (97.5000,136.2000) -- (97.5000,136.2000) -- (97.5000,136.2000) -- (97.5000,136.2000) -- (97.5000,136.2000) -- (97.5000,136.2000) -- (97.5000,136.2000) -- (97.5000,136.2000) -- (97.5000,136.2000) -- (97.5000,136.2000) -- (97.5000,136.2000) -- (97.5000,136.2000) -- (97.5000,136.2000) -- (97.5000,136.2000) -- (97.5000,136.1000) -- (97.5000,136.1000) -- (97.5000,136.1000) -- (97.5000,136.1000) -- (97.5000,136.1000) -- (97.6000,136.1000) -- (97.6000,136.1000) -- (97.6000,136.1000) -- (97.6000,136.1000) -- (97.6000,136.1000) -- (97.6000,136.1000) -- (97.6000,136.1000) -- (97.6000,136.1000) -- (97.6000,136.1000) -- (97.6000,136.1000) -- (97.6000,136.1000) -- (97.6000,136.1000) -- (97.6000,136.1000) -- (97.6000,136.1000) -- (97.6000,136.1000) -- (97.6000,136.1000) -- (97.6000,136.1000) -- (97.6000,136.0000) -- (97.6000,136.0000) -- (97.6000,136.0000) -- (97.6000,136.0000) -- (97.6000,136.0000) -- (97.6000,136.0000) -- (97.6000,136.0000) -- (97.6000,136.0000) -- (97.6000,136.0000) -- (97.6000,136.0000) -- (97.6000,136.0000) -- (97.6000,136.0000) -- (97.6000,136.0000) -- (97.6000,136.0000) -- (97.6000,136.0000) -- (97.6000,136.0000) -- (97.6000,136.0000) -- (97.6000,136.0000) -- (97.6000,136.0000) -- (97.6000,136.0000) -- (97.6000,136.0000) -- (97.6000,135.9000) -- (97.6000,135.9000) -- (97.6000,135.9000) -- (97.6000,135.9000) -- (97.6000,135.9000) -- (97.6000,135.9000) -- (97.6000,135.9000) -- (97.6000,135.9000) -- (97.6000,135.9000) -- (97.6000,135.9000) -- (97.6000,135.9000) -- (97.7000,135.9000) -- (97.7000,135.9000) -- (97.7000,135.9000) -- (97.7000,135.9000) -- (97.7000,135.9000) -- (97.7000,135.9000) -- (97.7000,135.9000) -- (97.7000,135.9000) -- (97.7000,135.9000) -- (97.7000,135.9000) -- (97.7000,135.9000) -- (97.7000,135.8000) -- (97.7000,135.8000) -- (97.7000,135.8000) -- (97.7000,135.8000) -- (97.7000,135.8000) -- (97.7000,135.8000) -- (97.7000,135.8000) -- (97.7000,135.8000) -- (97.7000,135.8000) -- (97.7000,135.8000) -- (97.7000,135.8000) -- (97.7000,135.8000) -- (97.7000,135.8000) -- (97.7000,135.8000) -- (97.7000,135.8000) -- (97.7000,135.8000) -- (97.7000,135.8000) -- (97.7000,135.8000) -- (97.7000,135.8000) -- (97.7000,135.8000) -- (97.7000,135.8000) -- (97.7000,135.7000) -- (97.7000,135.7000) -- (97.7000,135.7000) -- (97.7000,135.7000) -- (97.7000,135.7000) -- (97.7000,135.7000) -- (97.7000,135.7000) -- (97.7000,135.7000) -- (97.7000,135.7000) -- (97.7000,135.7000) -- (97.7000,135.7000) -- (97.7000,135.7000) -- (97.7000,135.7000) -- (97.7000,135.7000) -- (97.7000,135.7000) -- (97.7000,135.7000) -- (97.7000,135.7000) -- (97.7000,135.7000) -- (97.8000,135.7000) -- (97.8000,135.7000) -- (97.8000,135.7000) -- (97.8000,135.7000) -- (97.8000,135.6000) -- (97.8000,135.6000) -- (97.8000,135.6000) -- (97.8000,135.6000) -- (97.8000,135.6000) -- (97.8000,135.6000) -- (97.8000,135.6000) -- (97.8000,135.6000) -- (97.8000,135.6000) -- (97.8000,135.6000) -- (97.8000,135.6000) -- (97.8000,135.6000) -- (97.8000,135.6000) -- (97.8000,135.6000) -- (97.8000,135.6000) -- (97.8000,135.6000) -- (97.8000,135.6000) -- (97.8000,135.6000) -- (97.8000,135.6000) -- (97.8000,135.6000) -- (97.8000,135.6000) -- (97.8000,135.5000) -- (97.8000,135.5000) -- (97.8000,135.5000) -- (97.8000,135.5000) -- (97.8000,135.5000) -- (97.8000,135.5000) -- (97.8000,135.5000) -- (97.8000,135.5000) -- (97.8000,135.5000) -- (97.8000,135.5000) -- (97.8000,135.5000) -- (97.8000,135.5000) -- (97.8000,135.5000) -- (97.8000,135.5000) -- (97.8000,135.5000) -- (97.8000,135.5000) -- (97.8000,135.5000) -- (97.8000,135.5000) -- (97.8000,135.5000) -- (97.8000,135.5000) -- (97.8000,135.5000) -- (97.8000,135.5000) -- (97.8000,135.4000) -- (97.8000,135.4000) -- (97.9000,135.4000) -- (97.9000,135.4000) -- (97.9000,135.4000) -- (97.9000,135.4000) -- (97.9000,135.4000) -- (97.9000,135.4000) -- (97.9000,135.4000) -- (97.9000,135.4000) -- (97.9000,135.4000) -- (97.9000,135.4000) -- (97.9000,135.4000) -- (97.9000,135.4000) -- (97.9000,135.4000) -- (97.9000,135.4000) -- (97.9000,135.4000) -- (97.9000,135.4000) -- (97.9000,135.4000) -- (97.9000,135.4000) -- (97.9000,135.4000) -- (97.9000,135.3000) -- (97.9000,135.3000) -- (97.9000,135.3000) -- (97.9000,135.3000) -- (97.9000,135.3000) -- (97.9000,135.3000) -- (97.9000,135.3000) -- (97.9000,135.3000) -- (97.9000,135.3000) -- (97.9000,135.3000) -- (97.9000,135.3000) -- (97.9000,135.3000) -- (97.9000,135.3000) -- (107.3000,135.3000) -- (107.3000,135.3000) -- (107.3000,135.3000) -- (107.4000,135.3000) -- (121.2657,135.3000);



    \end{scope}
    \begin{scope}[cm={{1.21653,0.0,0.0,1.34548,(-346.94368,-156.28627)}},draw=blue,line cap=round,line join=round,line width=0.480pt]
      \path[draw] (81.5000,129.5000) -- (81.5000,157.5000) -- (121.5000,157.5000) -- (121.5000,129.5000) -- (81.5000,129.5000);



    \end{scope}
    \path[cm={{0.99667,0.0,0.0,1.34693,(-320.19845,-156.47287)}},draw=blue] (41.5000,88.5000) -- (41.5000,164.5000) -- (127.5000,164.5000) -- (127.5000,88.5000) -- (41.5000,88.5000);



    \path[draw=cffffff,line cap=butt,line join=miter,line width=1.312pt,miter limit=4.00] (-300.8820,-23.4840) -- (-279.6507,-38.3822);



  \end{scope}
  \begin{scope}[cm={{1.26012,0.0,0.0,1.26012,(-619.61892,-54.78689)}},draw=ca0a0a4,dash pattern=on 0.95pt off 0.95pt,line cap=round,line join=round,line width=0.238pt,miter limit=4.00]
    \path[draw,dash pattern=on 0.95pt off 0.95pt,line width=0.238pt,miter limit=4.00] (165.5000,88.5000) -- (251.5000,88.5000);



  \end{scope}
  \begin{scope}[cm={{1.26012,0.0,0.0,1.26012,(-619.61892,-54.78689)}},draw=blue,line cap=round,line join=round,line width=0.480pt]
    \path[cm={{0.9189,0.0,0.0,1.0,(13.39849,0.0)}},draw] (165.5000,88.5000) -- (168.5000,88.5000);



    \path[cm={{0.9189,0.0,0.0,1.0,(20.28216,0.0)}},draw] (251.5000,88.5000) -- (248.5000,88.5000);



  \end{scope}
  \begin{scope}[scale=1.006,draw=blue,line cap=rect,line join=bevel,line width=0.800pt]
  \end{scope}
  \begin{scope}[cm={{1.00588,0.0,0.0,1.00588,(148.871,93.5471)}},draw=blue,line cap=rect,line join=bevel,line width=0.800pt]
  \end{scope}
  \begin{scope}[cm={{1.00588,0.0,0.0,1.00588,(148.871,93.5471)}},draw=blue,line cap=rect,line join=bevel,line width=0.800pt]
  \end{scope}
  \begin{scope}[cm={{1.00588,0.0,0.0,1.00588,(148.871,93.5471)}},draw=blue,line cap=rect,line join=bevel,line width=0.800pt]
  \end{scope}
  \begin{scope}[cm={{1.00588,0.0,0.0,1.00588,(148.871,93.5471)}},draw=blue,line cap=rect,line join=bevel,line width=0.800pt]
  \end{scope}
  \begin{scope}[cm={{1.00588,0.0,0.0,1.00588,(148.871,93.5471)}},draw=blue,line cap=rect,line join=bevel,line width=0.800pt]
  \end{scope}
  \begin{scope}[cm={{1.00588,0.0,0.0,1.00588,(-425.82403,58.61615)}},draw=blue,line cap=rect,line join=bevel,line width=0.800pt]
    \path[fill=blue] (0.0000,0.0000) node[above right] (text380) {27};



  \end{scope}
  \begin{scope}[cm={{1.00588,0.0,0.0,1.00588,(148.871,93.5471)}},draw=blue,line cap=rect,line join=bevel,line width=0.800pt]
  \end{scope}
  \begin{scope}[scale=1.006,draw=blue,line cap=rect,line join=bevel,line width=0.800pt]
  \end{scope}
  \begin{scope}[cm={{1.26012,0.0,0.0,1.26012,(-619.61892,-54.78689)}},draw=ca0a0a4,dash pattern=on 0.95pt off 0.95pt,line cap=round,line join=round,line width=0.238pt,miter limit=4.00]
    \path[draw,dash pattern=on 0.95pt off 0.95pt,line width=0.238pt,miter limit=4.00] (165.5000,63.5000) -- (251.5000,63.5000);



  \end{scope}
  \begin{scope}[cm={{1.26012,0.0,0.0,1.26012,(-619.61892,-54.78689)}},draw=blue,line cap=round,line join=round,line width=0.480pt]
    \path[cm={{0.9189,0.0,0.0,1.0,(13.39849,0.0)}},draw] (165.5000,63.5000) -- (168.5000,63.5000);



    \path[cm={{0.9189,0.0,0.0,1.0,(20.28216,0.0)}},draw] (251.5000,63.5000) -- (248.5000,63.5000);



  \end{scope}
  \begin{scope}[scale=1.006,draw=blue,line cap=rect,line join=bevel,line width=0.800pt]
  \end{scope}
  \begin{scope}[cm={{1.00588,0.0,0.0,1.00588,(149.876,68.4)}},draw=blue,line cap=rect,line join=bevel,line width=0.800pt]
  \end{scope}
  \begin{scope}[cm={{1.00588,0.0,0.0,1.00588,(149.876,68.4)}},draw=blue,line cap=rect,line join=bevel,line width=0.800pt]
  \end{scope}
  \begin{scope}[cm={{1.00588,0.0,0.0,1.00588,(149.876,68.4)}},draw=blue,line cap=rect,line join=bevel,line width=0.800pt]
  \end{scope}
  \begin{scope}[cm={{1.00588,0.0,0.0,1.00588,(149.876,68.4)}},draw=blue,line cap=rect,line join=bevel,line width=0.800pt]
  \end{scope}
  \begin{scope}[cm={{1.00588,0.0,0.0,1.00588,(149.876,68.4)}},draw=blue,line cap=rect,line join=bevel,line width=0.800pt]
  \end{scope}
  \begin{scope}[cm={{1.00588,0.0,0.0,1.00588,(-425.38144,27.46905)}},draw=blue,line cap=rect,line join=bevel,line width=0.800pt]
    \path[fill=blue] (0.0000,0.0000) node[above right] (text410) {31};



  \end{scope}
  \begin{scope}[cm={{1.00588,0.0,0.0,1.00588,(149.876,68.4)}},draw=blue,line cap=rect,line join=bevel,line width=0.800pt]
  \end{scope}
  \begin{scope}[scale=1.006,draw=blue,line cap=rect,line join=bevel,line width=0.800pt]
  \end{scope}
  \begin{scope}[cm={{1.26012,0.0,0.0,1.26012,(-482.17027,-168.78691)}},draw=ca0a0a4,dash pattern=on 0.95pt off 0.95pt,line cap=round,line join=round,line width=0.238pt,miter limit=4.00]
    \path[draw,dash pattern=on 0.95pt off 0.95pt,line width=0.238pt,miter limit=4.00] (108.5000,95.5000) -- (108.5000,36.5000);



    \path[draw,dash pattern=on 0.95pt off 0.95pt,line width=0.238pt,miter limit=4.00] (108.5000,20.5000) -- (108.5000,13.5000);



  \end{scope}
  \begin{scope}[cm={{1.26012,0.0,0.0,1.26012,(-619.61892,-54.78689)}},draw=ca0a0a4,dash pattern=on 0.95pt off 0.95pt,line cap=round,line join=round,line width=0.238pt,miter limit=4.00]
    \path[draw,dash pattern=on 0.95pt off 0.95pt,line width=0.238pt,miter limit=4.00] (165.5000,38.5000) -- (251.5000,38.5000);



  \end{scope}
  \begin{scope}[cm={{1.26012,0.0,0.0,1.26012,(-619.61892,-54.78689)}},draw=blue,line cap=round,line join=round,line width=0.480pt]
    \path[cm={{0.9189,0.0,0.0,1.0,(13.39849,0.0)}},draw] (165.5000,38.5000) -- (168.5000,38.5000);



    \path[cm={{0.9189,0.0,0.0,1.0,(20.28216,0.0)}},draw] (251.5000,38.5000) -- (248.5000,38.5000);



  \end{scope}
  \begin{scope}[scale=1.006,draw=blue,line cap=rect,line join=bevel,line width=0.800pt]
  \end{scope}
  \begin{scope}[cm={{1.00588,0.0,0.0,1.00588,(149.876,43.2529)}},draw=blue,line cap=rect,line join=bevel,line width=0.800pt]
  \end{scope}
  \begin{scope}[cm={{1.00588,0.0,0.0,1.00588,(149.876,43.2529)}},draw=blue,line cap=rect,line join=bevel,line width=0.800pt]
  \end{scope}
  \begin{scope}[cm={{1.00588,0.0,0.0,1.00588,(149.876,43.2529)}},draw=blue,line cap=rect,line join=bevel,line width=0.800pt]
  \end{scope}
  \begin{scope}[cm={{1.00588,0.0,0.0,1.00588,(149.876,43.2529)}},draw=blue,line cap=rect,line join=bevel,line width=0.800pt]
  \end{scope}
  \begin{scope}[cm={{1.00588,0.0,0.0,1.00588,(149.876,43.2529)}},draw=blue,line cap=rect,line join=bevel,line width=0.800pt]
  \end{scope}
  \begin{scope}[cm={{1.00588,0.0,0.0,1.00588,(-425.73551,-5.17805)}},draw=blue,line cap=rect,line join=bevel,line width=0.800pt]
    \path[fill=blue] (0.0000,0.0000) node[above right] (text440) {35};



  \end{scope}
  \begin{scope}[cm={{1.00588,0.0,0.0,1.00588,(149.876,43.2529)}},draw=blue,line cap=rect,line join=bevel,line width=0.800pt]
  \end{scope}
  \begin{scope}[scale=1.006,draw=blue,line cap=rect,line join=bevel,line width=0.800pt]
  \end{scope}
  \begin{scope}[cm={{1.26012,0.0,0.0,1.26012,(-619.61892,-54.78689)}},draw=ca0a0a4,dash pattern=on 0.40pt off 0.80pt,line cap=round,line join=round,line width=0.400pt]
    \path[draw] (165.5000,95.5000) -- (165.5000,13.5000);



  \end{scope}
  \begin{scope}[cm={{1.26012,0.0,0.0,1.26012,(-619.61892,-54.78689)}},draw=blue,line cap=round,line join=round,line width=0.480pt]
    \path[draw] (165.5000,95.5000) -- (165.5000,92.5000);



    \path[draw] (165.5000,13.5000) -- (165.5000,16.5000);



  \end{scope}
  \begin{scope}[scale=1.006,draw=blue,line cap=rect,line join=bevel,line width=0.800pt]
  \end{scope}
  \begin{scope}[cm={{1.00588,0.0,0.0,1.00588,(162.953,110.647)}},draw=blue,line cap=rect,line join=bevel,line width=0.800pt]
  \end{scope}
  \begin{scope}[cm={{1.00588,0.0,0.0,1.00588,(162.953,110.647)}},draw=blue,line cap=rect,line join=bevel,line width=0.800pt]
  \end{scope}
  \begin{scope}[cm={{1.00588,0.0,0.0,1.00588,(162.953,110.647)}},draw=blue,line cap=rect,line join=bevel,line width=0.800pt]
  \end{scope}
  \begin{scope}[cm={{1.00588,0.0,0.0,1.00588,(162.953,110.647)}},draw=blue,line cap=rect,line join=bevel,line width=0.800pt]
  \end{scope}
  \begin{scope}[cm={{1.00588,0.0,0.0,1.00588,(162.953,110.647)}},draw=blue,line cap=rect,line join=bevel,line width=0.800pt]
  \end{scope}
  \begin{scope}[cm={{1.00588,0.0,0.0,1.00588,(-413.68137,77.03586)}},draw=blue,line cap=rect,line join=bevel,line width=0.800pt]
    \path[fill=blue] (0.0000,0.0000) node[above right] (text470) {0};



  \end{scope}
  \begin{scope}[cm={{1.00588,0.0,0.0,1.00588,(162.953,110.647)}},draw=blue,line cap=rect,line join=bevel,line width=0.800pt]
  \end{scope}
  \begin{scope}[scale=1.006,draw=blue,line cap=rect,line join=bevel,line width=0.800pt]
  \end{scope}
  \begin{scope}[cm={{1.26012,0.0,0.0,1.26012,(-619.6494,-55.17358)}},draw=ca0a0a4,dash pattern=on 1.03pt off 1.03pt,line cap=round,line join=round,line width=0.257pt,miter limit=4.00]
    \path[draw,dash pattern=on 1.03pt off 1.03pt,line width=0.257pt,miter limit=4.00] (191.5000,95.5000) -- (191.5000,13.5000);



  \end{scope}
  \begin{scope}[cm={{1.26012,0.0,0.0,1.26012,(-482.17027,-168.78691)}},draw=ca0a0a4,dash pattern=on 0.95pt off 0.95pt,line cap=round,line join=round,line width=0.238pt,miter limit=4.00]
    \path[draw,dash pattern=on 0.95pt off 0.95pt,line width=0.238pt,miter limit=4.00] (82.5000,95.5000) -- (82.5000,36.5000);



    \path[draw,dash pattern=on 0.95pt off 0.95pt,line width=0.238pt,miter limit=4.00] (82.5000,20.5000) -- (82.5000,13.5000);



  \end{scope}
  \begin{scope}[cm={{1.26012,0.0,0.0,1.26012,(-619.61892,-54.78689)}},draw=blue,line cap=round,line join=round,line width=0.480pt]
    \path[cm={{1.0,0.0,0.0,0.9189,(0.0,7.78184)}},draw] (191.5000,95.5000) -- (191.5000,92.5000);



    \path[cm={{1.0,0.0,0.0,0.9189,(0.0,1.07058)}},draw] (191.5000,13.5000) -- (191.5000,16.5000);



  \end{scope}
  \begin{scope}[scale=1.006,draw=blue,line cap=rect,line join=bevel,line width=0.800pt]
  \end{scope}
  \begin{scope}[cm={{1.00588,0.0,0.0,1.00588,(189.106,110.647)}},draw=blue,line cap=rect,line join=bevel,line width=0.800pt]
  \end{scope}
  \begin{scope}[cm={{1.00588,0.0,0.0,1.00588,(189.106,110.647)}},draw=blue,line cap=rect,line join=bevel,line width=0.800pt]
  \end{scope}
  \begin{scope}[cm={{1.00588,0.0,0.0,1.00588,(189.106,110.647)}},draw=blue,line cap=rect,line join=bevel,line width=0.800pt]
  \end{scope}
  \begin{scope}[cm={{1.00588,0.0,0.0,1.00588,(189.106,110.647)}},draw=blue,line cap=rect,line join=bevel,line width=0.800pt]
  \end{scope}
  \begin{scope}[cm={{1.00588,0.0,0.0,1.00588,(189.106,110.647)}},draw=blue,line cap=rect,line join=bevel,line width=0.800pt]
  \end{scope}
  \begin{scope}[cm={{1.00588,0.0,0.0,1.00588,(-380.02837,77.03586)}},draw=blue,line cap=rect,line join=bevel,line width=0.800pt]
    \path[fill=blue] (0.0000,0.0000) node[above right] (text500) {1};



  \end{scope}
  \begin{scope}[cm={{1.00588,0.0,0.0,1.00588,(189.106,110.647)}},draw=blue,line cap=rect,line join=bevel,line width=0.800pt]
  \end{scope}
  \begin{scope}[scale=1.006,draw=blue,line cap=rect,line join=bevel,line width=0.800pt]
  \end{scope}
  \begin{scope}[cm={{1.26012,0.0,0.0,1.26012,(-619.6494,-55.17358)}},draw=ca0a0a4,dash pattern=on 1.03pt off 1.03pt,line cap=round,line join=round,line width=0.257pt,miter limit=4.00]
    \path[draw,dash pattern=on 1.03pt off 1.03pt,line width=0.257pt,miter limit=4.00] (217.5000,95.5000) -- (217.5000,13.5000);



  \end{scope}
  \begin{scope}[cm={{1.26012,0.0,0.0,1.26012,(-619.61892,-54.78689)}},draw=blue,line cap=round,line join=round,line width=0.480pt]
    \path[cm={{1.0,0.0,0.0,0.9189,(0.0,7.78184)}},draw] (217.5000,95.5000) -- (217.5000,92.5000);



    \path[cm={{1.0,0.0,0.0,0.9189,(0.0,1.07058)}},draw] (217.5000,13.5000) -- (217.5000,16.5000);



  \end{scope}
  \begin{scope}[scale=1.006,draw=blue,line cap=rect,line join=bevel,line width=0.800pt]
  \end{scope}
  \begin{scope}[cm={{1.00588,0.0,0.0,1.00588,(215.259,110.647)}},draw=blue,line cap=rect,line join=bevel,line width=0.800pt]
  \end{scope}
  \begin{scope}[cm={{1.00588,0.0,0.0,1.00588,(215.259,110.647)}},draw=blue,line cap=rect,line join=bevel,line width=0.800pt]
  \end{scope}
  \begin{scope}[cm={{1.00588,0.0,0.0,1.00588,(215.259,110.647)}},draw=blue,line cap=rect,line join=bevel,line width=0.800pt]
  \end{scope}
  \begin{scope}[cm={{1.00588,0.0,0.0,1.00588,(215.259,110.647)}},draw=blue,line cap=rect,line join=bevel,line width=0.800pt]
  \end{scope}
  \begin{scope}[cm={{1.00588,0.0,0.0,1.00588,(215.259,110.647)}},draw=blue,line cap=rect,line join=bevel,line width=0.800pt]
  \end{scope}
  \begin{scope}[cm={{1.00588,0.0,0.0,1.00588,(-347.87533,77.03586)}},draw=blue,line cap=rect,line join=bevel,line width=0.800pt]
    \path[fill=blue] (0.0000,0.0000) node[above right] (text530) {2};



  \end{scope}
  \begin{scope}[cm={{1.00588,0.0,0.0,1.00588,(215.259,110.647)}},draw=blue,line cap=rect,line join=bevel,line width=0.800pt]
  \end{scope}
  \begin{scope}[scale=1.006,draw=blue,line cap=rect,line join=bevel,line width=0.800pt]
  \end{scope}
  \begin{scope}[cm={{1.26012,0.0,0.0,1.26012,(-619.61892,-55.16996)}},draw=ca0a0a4,dash pattern=on 0.95pt off 0.95pt,line cap=round,line join=round,line width=0.238pt,miter limit=4.00]
    \path[draw,dash pattern=on 0.95pt off 0.95pt,line width=0.238pt,miter limit=4.00] (243.5000,95.5000) -- (243.5000,13.5000);



  \end{scope}
  \begin{scope}[cm={{1.26012,0.0,0.0,1.26012,(-619.61892,-54.78689)}},draw=blue,line cap=round,line join=round,line width=0.480pt]
    \path[cm={{1.0,0.0,0.0,0.9189,(0.0,7.78184)}},draw] (243.5000,95.5000) -- (243.5000,92.5000);



    \path[cm={{1.0,0.0,0.0,0.9189,(0.0,1.07058)}},draw] (243.5000,13.5000) -- (243.5000,16.5000);



  \end{scope}
  \begin{scope}[scale=1.006,draw=blue,line cap=rect,line join=bevel,line width=0.800pt]
  \end{scope}
  \begin{scope}[cm={{1.00588,0.0,0.0,1.00588,(241.915,110.647)}},draw=blue,line cap=rect,line join=bevel,line width=0.800pt]
  \end{scope}
  \begin{scope}[cm={{1.00588,0.0,0.0,1.00588,(241.915,110.647)}},draw=blue,line cap=rect,line join=bevel,line width=0.800pt]
  \end{scope}
  \begin{scope}[cm={{1.00588,0.0,0.0,1.00588,(241.915,110.647)}},draw=blue,line cap=rect,line join=bevel,line width=0.800pt]
  \end{scope}
  \begin{scope}[cm={{1.00588,0.0,0.0,1.00588,(241.915,110.647)}},draw=blue,line cap=rect,line join=bevel,line width=0.800pt]
  \end{scope}
  \begin{scope}[cm={{1.00588,0.0,0.0,1.00588,(241.915,110.647)}},draw=blue,line cap=rect,line join=bevel,line width=0.800pt]
  \end{scope}
  \begin{scope}[cm={{1.00588,0.0,0.0,1.00588,(-315.21933,77.03586)}},draw=blue,line cap=rect,line join=bevel,line width=0.800pt]
    \path[fill=blue] (0.0000,0.0000) node[above right] (text560) {3};



  \end{scope}
  \begin{scope}[cm={{1.00588,0.0,0.0,1.00588,(241.915,110.647)}},draw=blue,line cap=rect,line join=bevel,line width=0.800pt]
  \end{scope}
  \begin{scope}[scale=1.006,draw=blue,line cap=rect,line join=bevel,line width=0.800pt]
  \end{scope}
  \begin{scope}[cm={{1.26012,0.0,0.0,1.26012,(-619.61892,-54.78689)}},draw=blue,line cap=round,line join=round,line width=0.480pt]
    \path[draw] (165.5000,13.5000) -- (165.5000,95.5000) -- (251.5000,95.5000) -- (251.5000,13.5000) -- (165.5000,13.5000);



  \end{scope}
  \begin{scope}[scale=1.006,draw=blue,line cap=rect,line join=bevel,line width=0.800pt]
  \end{scope}
  \begin{scope}[scale=1.006,draw=blue,line cap=rect,line join=bevel,line width=0.800pt]
  \end{scope}
  \begin{scope}[scale=1.006,draw=blue,line cap=rect,line join=bevel,line width=0.800pt]
  \end{scope}
  \begin{scope}[scale=1.006,draw=blue,line cap=rect,line join=bevel,line width=0.800pt]
  \end{scope}
  \begin{scope}[scale=1.006,draw=blue,line cap=rect,line join=bevel,line width=0.800pt]
  \end{scope}
  \begin{scope}[cm={{1.00588,0.0,0.0,1.00588,(235.376,29.1706)}},draw=blue,line cap=rect,line join=bevel,line width=0.800pt]
  \end{scope}
  \begin{scope}[cm={{1.00588,0.0,0.0,1.00588,(235.376,29.1706)}},draw=blue,line cap=rect,line join=bevel,line width=0.800pt]
  \end{scope}
  \begin{scope}[cm={{1.00588,0.0,0.0,1.00588,(235.376,29.1706)}},draw=blue,line cap=rect,line join=bevel,line width=0.800pt]
  \end{scope}
  \begin{scope}[cm={{1.00588,0.0,0.0,1.00588,(235.376,29.1706)}},draw=blue,line cap=rect,line join=bevel,line width=0.800pt]
  \end{scope}
  \begin{scope}[cm={{1.00588,0.0,0.0,1.00588,(235.376,29.1706)}},draw=blue,line cap=rect,line join=bevel,line width=0.800pt]
  \end{scope}
  \begin{scope}[cm={{1.00588,0.0,0.0,1.00588,(235.376,29.1706)}},draw=blue,line cap=rect,line join=bevel,line width=0.800pt]
  \end{scope}
  \begin{scope}[scale=1.006,draw=blue,line cap=rect,line join=bevel,line width=0.800pt]
  \end{scope}
  \begin{scope}[scale=1.006,draw=blue,line cap=rect,line join=bevel,line width=0.800pt]
  \end{scope}
  \begin{scope}[scale=1.006,draw=blue,line cap=rect,line join=bevel,line width=0.800pt]
  \end{scope}
  \begin{scope}[cm={{1.26012,0.0,0.0,1.26012,(-619.61892,-54.78689)}},draw=blue,line cap=round,line join=round,line width=0.480pt]
    \path[draw] (165.3000,36.9000) -- (165.3000,36.9000) -- (165.4000,41.9000) -- (165.5000,44.7000) -- (165.7000,46.2000) -- (165.8000,46.7000) -- (166.0000,46.7000) -- (166.1000,46.7000) -- (166.2000,47.2000) -- (166.4000,47.4000) -- (166.5000,47.2000) -- (166.6000,46.7000) -- (166.8000,46.1000) -- (166.9000,45.5000) -- (167.1000,44.9000) -- (167.2000,44.5000) -- (167.3000,44.2000) -- (167.5000,44.2000) -- (167.6000,44.2000) -- (167.8000,44.2000) -- (167.9000,44.3000) -- (168.0000,44.4000) -- (168.2000,44.4000) -- (168.3000,44.5000) -- (168.4000,44.5000) -- (168.6000,44.5000) -- (168.7000,44.5000) -- (168.9000,44.5000) -- (169.0000,44.5000) -- (169.1000,44.5000) -- (169.3000,44.5000) -- (169.4000,44.5000) -- (169.5000,44.5000) -- (169.7000,44.5000) -- (169.8000,44.5000) -- (170.0000,44.5000) -- (170.1000,44.5000) -- (170.2000,44.5000) -- (170.4000,44.5000) -- (170.5000,44.5000) -- (170.7000,44.5000) -- (170.8000,44.5000) -- (170.9000,44.5000) -- (171.1000,44.5000) -- (171.2000,44.5000) -- (171.3000,44.5000) -- (171.5000,44.5000) -- (171.6000,44.5000) -- (171.8000,44.6000) -- (171.9000,44.7000) -- (172.0000,44.9000) -- (172.2000,45.2000) -- (172.3000,45.5000) -- (172.5000,45.8000) -- (172.6000,46.3000) -- (172.7000,46.7000) -- (172.9000,47.3000) -- (173.0000,47.8000) -- (173.1000,48.5000) -- (173.3000,49.1000) -- (173.4000,49.9000) -- (173.6000,50.7000) -- (173.7000,51.5000) -- (173.8000,52.4000) -- (174.0000,53.4000) -- (174.1000,54.4000) -- (174.2000,55.4000) -- (174.4000,56.5000) -- (174.5000,57.7000) -- (174.7000,58.9000) -- (174.8000,60.2000) -- (174.9000,61.5000) -- (175.1000,62.8000) -- (175.2000,64.2000) -- (175.4000,65.6000) -- (175.5000,67.1000) -- (175.6000,68.5000) -- (175.8000,70.0000) -- (175.9000,71.5000) -- (176.0000,73.0000) -- (176.2000,74.5000) -- (176.3000,76.0000) -- (176.5000,77.4000) -- (176.6000,78.8000) -- (176.7000,80.2000) -- (176.9000,81.6000) -- (177.0000,83.1000) -- (177.1000,84.5000) -- (177.3000,85.7000) -- (177.4000,86.5000) -- (177.6000,87.0000) -- (177.7000,87.2000) -- (177.8000,87.2000) -- (178.0000,87.1000) -- (178.1000,86.9000) -- (178.3000,86.7000) -- (178.4000,86.6000) -- (178.5000,86.5000) -- (178.7000,86.4000) -- (178.8000,86.3000) -- (178.9000,86.3000) -- (179.1000,86.3000) -- (179.2000,86.3000) -- (179.4000,86.3000) -- (179.5000,86.3000) -- (179.6000,86.3000) -- (179.8000,86.3000) -- (179.9000,86.3000) -- (180.0000,86.3000) -- (180.2000,86.3000) -- (180.3000,86.3000) -- (180.5000,86.3000) -- (180.6000,86.3000) -- (180.7000,86.3000) -- (180.9000,86.3000) -- (181.0000,86.3000) -- (181.2000,86.3000) -- (181.3000,86.3000) -- (181.4000,86.2000) -- (181.6000,86.0000) -- (181.7000,85.8000) -- (181.8000,85.9000) -- (182.0000,86.1000) -- (182.1000,86.2000) -- (182.3000,86.0000) -- (182.4000,85.7000) -- (182.5000,85.0000) -- (182.7000,84.0000) -- (182.8000,82.8000) -- (183.0000,81.4000) -- (183.1000,79.9000) -- (183.2000,78.2000) -- (183.4000,76.4000) -- (183.5000,74.6000) -- (183.6000,72.8000) -- (183.8000,70.9000) -- (183.9000,69.1000) -- (184.1000,67.3000) -- (184.2000,65.5000) -- (184.3000,63.8000) -- (184.5000,62.1000) -- (184.6000,60.5000) -- (184.7000,59.0000) -- (184.9000,57.6000) -- (185.0000,56.2000) -- (185.2000,54.9000) -- (185.3000,53.7000) -- (185.4000,52.5000) -- (185.6000,51.5000) -- (185.7000,50.5000) -- (185.9000,49.6000) -- (186.0000,48.7000) -- (186.1000,48.0000) -- (186.3000,47.3000) -- (186.4000,46.7000) -- (186.5000,46.1000) -- (186.7000,45.6000) -- (186.8000,45.2000) -- (187.0000,44.9000) -- (187.1000,44.6000) -- (187.2000,44.3000) -- (187.4000,44.1000) -- (187.5000,44.1000) -- (187.6000,44.1000) -- (187.8000,44.2000) -- (187.9000,44.3000) -- (188.1000,44.4000) -- (188.2000,44.4000) -- (188.3000,44.5000) -- (188.5000,44.5000) -- (188.6000,44.5000) -- (188.8000,44.5000) -- (188.9000,44.5000) -- (189.0000,44.5000) -- (189.2000,44.5000) -- (189.3000,44.5000) -- (189.4000,44.5000) -- (189.6000,44.5000) -- (189.7000,44.5000) -- (189.9000,44.5000) -- (190.0000,44.5000) -- (190.1000,44.5000) -- (190.3000,44.5000) -- (190.4000,44.5000) -- (190.6000,44.5000) -- (190.7000,44.5000) -- (190.8000,44.5000) -- (191.0000,44.5000) -- (191.1000,44.5000) -- (191.2000,44.5000) -- (191.4000,44.5000) -- (191.5000,44.5000) -- (191.7000,44.5000) -- (191.8000,44.5000) -- (191.9000,44.6000) -- (192.1000,44.7000) -- (192.2000,44.9000) -- (192.3000,45.1000) -- (192.5000,45.4000) -- (192.6000,45.8000) -- (192.8000,46.2000) -- (192.9000,46.6000) -- (193.0000,47.2000) -- (193.2000,47.7000) -- (193.3000,48.3000) -- (193.5000,49.0000) -- (193.6000,49.7000) -- (193.7000,50.5000) -- (193.9000,51.3000) -- (194.0000,52.2000) -- (194.1000,53.1000) -- (194.3000,54.1000) -- (194.4000,55.1000) -- (194.6000,56.2000) -- (194.7000,57.3000) -- (194.8000,58.5000) -- (195.0000,59.8000) -- (195.1000,61.0000) -- (195.2000,62.4000) -- (195.4000,63.7000) -- (195.5000,65.1000) -- (195.7000,66.5000) -- (195.8000,68.0000) -- (195.9000,69.5000) -- (196.1000,71.0000) -- (196.2000,72.4000) -- (196.4000,73.9000) -- (196.5000,75.4000) -- (196.6000,76.8000) -- (196.8000,78.2000) -- (196.9000,79.6000) -- (197.0000,81.0000) -- (197.2000,82.5000) -- (197.3000,84.0000) -- (197.5000,85.3000) -- (197.6000,86.3000) -- (197.7000,86.9000) -- (197.9000,87.2000) -- (198.0000,87.3000) -- (198.1000,87.2000) -- (198.3000,87.0000) -- (198.4000,86.8000) -- (198.6000,86.6000) -- (198.7000,86.5000) -- (198.8000,86.4000) -- (199.0000,86.3000) -- (199.1000,86.3000) -- (199.3000,86.3000) -- (199.4000,86.3000) -- (199.5000,86.3000) -- (199.7000,86.3000) -- (199.8000,86.3000) -- (199.9000,86.3000) -- (200.1000,86.3000) -- (200.2000,86.3000) -- (200.4000,86.3000) -- (200.5000,86.3000) -- (200.6000,86.3000) -- (200.8000,86.3000) -- (200.9000,86.3000) -- (201.1000,86.3000) -- (201.2000,86.3000) -- (201.3000,86.3000) -- (201.5000,86.3000) -- (201.6000,86.3000) -- (201.7000,86.1000) -- (201.9000,85.9000) -- (202.0000,85.9000) -- (202.2000,86.0000) -- (202.3000,86.1000) -- (202.4000,86.1000) -- (202.6000,85.9000) -- (202.7000,85.3000) -- (202.8000,84.5000) -- (203.0000,83.5000) -- (203.1000,82.2000) -- (203.3000,80.7000) -- (203.4000,79.1000) -- (203.5000,77.3000) -- (203.7000,75.5000) -- (203.8000,73.7000) -- (204.0000,71.9000) -- (204.1000,70.0000) -- (204.2000,68.2000) -- (204.4000,66.4000) -- (204.5000,64.7000) -- (204.6000,63.0000) -- (204.8000,61.4000) -- (204.9000,59.8000) -- (205.1000,58.3000) -- (205.2000,56.9000) -- (205.3000,55.5000) -- (205.5000,54.3000) -- (205.6000,53.1000) -- (205.7000,52.0000) -- (205.9000,51.0000) -- (206.0000,50.0000) -- (206.2000,49.1000) -- (206.3000,48.4000) -- (206.4000,47.6000) -- (206.6000,47.0000) -- (206.7000,46.4000) -- (206.9000,45.9000) -- (207.0000,45.4000) -- (207.1000,45.1000) -- (207.3000,44.8000) -- (207.4000,44.5000) -- (207.5000,44.2000) -- (207.7000,44.1000) -- (207.8000,44.1000) -- (208.0000,44.1000) -- (208.1000,44.2000) -- (208.2000,44.3000) -- (208.4000,44.4000) -- (208.5000,44.4000) -- (208.6000,44.5000) -- (208.8000,44.5000) -- (208.9000,44.5000) -- (209.1000,44.5000) -- (209.2000,44.5000) -- (209.3000,44.5000) -- (209.5000,44.5000) -- (209.6000,44.5000) -- (209.8000,44.5000) -- (209.9000,44.5000) -- (210.0000,44.5000) -- (210.2000,44.5000) -- (210.3000,44.5000) -- (210.4000,44.5000) -- (210.6000,44.5000) -- (210.7000,44.5000) -- (210.9000,44.5000) -- (211.0000,44.5000) -- (211.1000,44.5000) -- (211.3000,44.5000) -- (211.4000,44.5000) -- (211.5000,44.5000) -- (211.7000,44.5000) -- (211.8000,44.5000) -- (212.0000,44.5000) -- (212.1000,44.6000) -- (212.2000,44.7000) -- (212.4000,44.8000) -- (212.5000,45.0000) -- (212.7000,45.3000) -- (212.8000,45.6000) -- (212.9000,45.9000) -- (213.1000,46.4000) -- (213.2000,46.9000) -- (213.3000,47.4000) -- (213.5000,48.0000) -- (213.6000,48.6000) -- (213.8000,49.3000) -- (213.9000,50.0000) -- (214.0000,50.8000) -- (214.2000,51.7000) -- (214.3000,52.6000) -- (214.5000,53.5000) -- (214.6000,54.5000) -- (214.7000,55.6000) -- (214.9000,56.7000) -- (215.0000,57.8000) -- (215.1000,59.0000) -- (215.3000,60.3000) -- (215.4000,61.6000) -- (215.6000,62.9000) -- (215.7000,64.3000) -- (215.8000,65.7000) -- (216.0000,67.1000) -- (216.1000,68.6000) -- (216.2000,70.1000) -- (216.4000,71.6000) -- (216.5000,73.0000) -- (216.7000,74.5000) -- (216.8000,76.0000) -- (216.9000,77.4000) -- (217.1000,78.8000) -- (217.2000,80.1000) -- (217.4000,81.6000) -- (217.5000,83.1000) -- (217.6000,84.6000) -- (217.8000,85.8000) -- (217.9000,86.6000) -- (218.0000,87.0000) -- (218.2000,87.2000) -- (218.3000,87.2000) -- (218.5000,87.1000) -- (218.6000,86.9000) -- (218.7000,86.7000) -- (218.9000,86.6000) -- (219.0000,86.5000) -- (219.1000,86.4000) -- (219.3000,86.3000) -- (219.4000,86.3000) -- (219.6000,86.3000) -- (219.7000,86.3000) -- (219.8000,86.3000) -- (220.0000,86.3000) -- (220.1000,86.3000) -- (220.3000,86.3000) -- (220.4000,86.3000) -- (220.5000,86.3000) -- (220.7000,86.3000) -- (220.8000,86.3000) -- (220.9000,86.3000) -- (221.1000,86.3000) -- (221.2000,86.3000) -- (221.4000,86.3000) -- (221.5000,86.3000) -- (221.6000,86.3000) -- (221.8000,86.3000) -- (221.9000,86.2000) -- (222.0000,86.0000) -- (222.2000,85.9000) -- (222.3000,85.9000) -- (222.5000,86.1000) -- (222.6000,86.2000) -- (222.7000,86.1000) -- (222.9000,85.7000) -- (223.0000,85.1000) -- (223.2000,84.1000) -- (223.3000,83.0000) -- (223.4000,81.6000) -- (223.6000,80.1000) -- (223.7000,78.4000) -- (223.8000,76.6000) -- (224.0000,74.8000) -- (224.1000,73.0000) -- (224.3000,71.2000) -- (224.4000,69.3000) -- (224.5000,67.5000) -- (224.7000,65.7000) -- (224.8000,64.0000) -- (225.0000,62.3000) -- (225.1000,60.7000) -- (225.2000,59.2000) -- (225.4000,57.7000) -- (225.5000,56.3000) -- (225.6000,55.0000) -- (225.8000,53.8000) -- (225.9000,52.7000) -- (226.1000,51.6000) -- (226.2000,50.6000) -- (226.3000,49.7000) -- (226.5000,48.8000) -- (226.6000,48.1000) -- (226.7000,47.4000) -- (226.9000,46.7000) -- (227.0000,46.2000) -- (227.2000,45.7000) -- (227.3000,45.3000) -- (227.4000,44.9000) -- (227.6000,44.7000) -- (227.7000,44.4000) -- (227.9000,44.2000) -- (228.0000,44.1000) -- (228.1000,44.1000) -- (228.3000,44.2000) -- (228.4000,44.3000) -- (228.5000,44.3000) -- (228.7000,44.4000) -- (228.8000,44.5000) -- (229.0000,44.5000) -- (229.1000,44.5000) -- (229.2000,44.5000) -- (229.4000,44.5000) -- (229.5000,44.5000) -- (229.6000,44.5000) -- (229.8000,44.5000) -- (229.9000,44.5000) -- (230.1000,44.5000) -- (230.2000,44.5000) -- (230.3000,44.5000) -- (230.5000,44.5000) -- (230.6000,44.5000) -- (230.8000,44.5000) -- (230.9000,44.5000) -- (231.0000,44.5000) -- (231.2000,44.5000) -- (231.3000,44.5000) -- (231.4000,44.5000) -- (231.6000,44.5000) -- (231.7000,44.5000) -- (231.9000,44.5000) -- (232.0000,44.5000) -- (232.1000,44.5000) -- (232.3000,44.5000) -- (232.4000,44.6000) -- (232.6000,44.7000) -- (232.7000,44.9000) -- (232.8000,45.1000) -- (233.0000,45.4000) -- (233.1000,45.7000) -- (233.2000,46.1000) -- (233.4000,46.6000) -- (233.5000,47.1000) -- (233.7000,47.6000) -- (233.8000,48.2000) -- (233.9000,48.9000) -- (234.1000,49.6000) -- (234.2000,50.3000) -- (234.3000,51.1000) -- (234.5000,52.0000) -- (234.6000,52.9000) -- (234.8000,53.9000) -- (234.9000,54.9000) -- (235.0000,56.0000) -- (235.2000,57.1000) -- (235.3000,58.3000) -- (235.5000,59.5000) -- (235.6000,60.8000) -- (235.7000,62.1000) -- (235.9000,63.4000) -- (236.0000,64.8000) -- (236.1000,66.3000) -- (236.3000,67.7000) -- (236.4000,69.2000) -- (236.6000,70.6000) -- (236.7000,72.1000) -- (236.8000,73.6000) -- (237.0000,75.1000) -- (237.1000,76.5000) -- (237.2000,77.9000) -- (237.4000,79.3000) -- (237.5000,80.7000) -- (237.7000,82.2000) -- (237.8000,83.7000) -- (237.9000,85.1000) -- (238.1000,86.1000) -- (238.2000,86.8000) -- (238.4000,87.2000) -- (238.5000,87.3000) -- (238.6000,87.2000) -- (238.8000,87.0000) -- (238.9000,86.9000) -- (239.0000,86.7000) -- (239.2000,86.5000) -- (239.3000,86.4000) -- (239.5000,86.3000) -- (239.6000,86.3000) -- (239.7000,86.3000) -- (239.9000,86.3000) -- (240.0000,86.3000) -- (240.2000,86.3000) -- (240.3000,86.3000) -- (240.4000,86.3000) -- (240.6000,86.3000) -- (240.7000,86.3000) -- (240.8000,86.3000) -- (241.0000,86.3000) -- (241.1000,86.3000) -- (241.3000,86.3000) -- (241.4000,86.3000) -- (241.5000,86.3000) -- (241.7000,86.3000) -- (241.8000,86.3000) -- (241.9000,86.3000) -- (242.1000,86.3000) -- (242.2000,86.2000) -- (242.4000,86.0000) -- (242.5000,85.9000) -- (242.6000,86.0000) -- (242.8000,86.1000) -- (242.9000,86.1000) -- (243.1000,85.9000) -- (243.2000,85.5000) -- (243.3000,84.7000) -- (243.5000,83.7000) -- (243.6000,82.5000) -- (243.7000,81.0000) -- (243.9000,79.4000) -- (244.0000,77.7000) -- (244.2000,75.9000) -- (244.3000,74.1000) -- (244.4000,72.3000) -- (244.6000,70.4000) -- (244.7000,68.6000) -- (244.8000,66.8000) -- (245.0000,65.1000) -- (245.1000,63.4000) -- (245.3000,61.7000) -- (245.4000,60.1000) -- (245.5000,58.6000) -- (245.7000,57.2000) -- (245.8000,55.8000) -- (246.0000,54.6000) -- (246.1000,53.3000) -- (246.2000,52.2000) -- (246.4000,51.2000) -- (246.5000,50.2000) -- (246.6000,49.3000) -- (246.8000,48.5000) -- (246.9000,47.8000) -- (247.1000,47.1000) -- (247.2000,46.5000) -- (247.3000,46.0000) -- (247.5000,45.5000) -- (247.6000,45.1000) -- (247.7000,44.8000) -- (247.9000,44.5000) -- (248.0000,44.3000) -- (248.2000,44.1000) -- (248.3000,44.1000) -- (248.4000,44.1000) -- (248.6000,44.2000) -- (248.7000,44.3000) -- (248.9000,44.4000) -- (249.0000,44.4000) -- (249.1000,44.5000) -- (249.3000,44.5000) -- (249.4000,44.5000) -- (249.5000,44.5000) -- (249.7000,44.5000) -- (249.8000,44.5000) -- (250.0000,44.5000) -- (250.1000,44.5000) -- (250.2000,44.5000) -- (250.4000,44.5000) -- (250.5000,44.5000) -- (250.7000,44.5000) -- (250.8000,44.5000) -- (250.9000,44.5000) -- (251.1000,44.5000) -- (251.2000,44.5000) -- (251.3000,44.5000) -- (251.5000,44.5000) -- (251.6000,44.5000) -- (251.8000,44.5000) -- (251.9000,44.5000) -- (251.9000,44.5000);



  \end{scope}
  \begin{scope}[scale=1.006,draw=blue,line cap=rect,line join=bevel,line width=0.800pt]
  \end{scope}
  \begin{scope}[scale=1.006,draw=blue,line cap=rect,line join=bevel,line width=0.800pt]
  \end{scope}
  \begin{scope}[scale=1.006,draw=blue,line cap=rect,line join=bevel,line width=0.800pt]
  \end{scope}
  \begin{scope}[scale=1.006,draw=blue,line cap=rect,line join=bevel,line width=0.800pt]
  \end{scope}
  \begin{scope}[cm={{1.26012,0.0,0.0,1.26012,(-619.61892,-54.78689)}},draw=cff0000,line cap=round,line join=round,line width=0.480pt]
    \path[draw] (165.2000,51.8000) -- (165.2000,51.8000) -- (165.3000,51.0000) -- (165.4000,50.3000) -- (165.5000,49.6000) -- (165.5000,48.9000) -- (165.6000,48.3000) -- (165.7000,47.8000) -- (165.8000,47.2000) -- (165.9000,46.8000) -- (166.0000,46.4000) -- (166.1000,46.0000) -- (166.2000,45.7000) -- (166.2000,45.4000) -- (166.3000,45.2000) -- (166.4000,45.0000) -- (166.5000,44.8000) -- (166.6000,44.7000) -- (166.7000,44.7000) -- (166.8000,44.7000) -- (166.8000,44.7000) -- (166.9000,44.7000) -- (167.0000,44.8000) -- (167.1000,44.9000) -- (167.2000,45.1000) -- (167.3000,45.3000) -- (167.4000,45.4000) -- (167.5000,45.6000) -- (167.5000,45.9000) -- (167.6000,46.1000) -- (167.7000,46.4000) -- (167.8000,46.6000) -- (167.9000,46.9000) -- (168.0000,47.1000) -- (168.1000,47.4000) -- (168.2000,47.6000) -- (168.2000,47.8000) -- (168.3000,48.0000) -- (168.4000,48.2000) -- (168.5000,48.4000) -- (168.6000,48.6000) -- (168.7000,48.8000) -- (168.8000,48.9000) -- (168.8000,49.0000) -- (168.9000,49.1000) -- (169.0000,49.1000) -- (169.1000,49.1000) -- (169.2000,49.1000) -- (169.3000,49.1000) -- (169.4000,49.0000) -- (169.5000,48.9000) -- (169.5000,48.8000) -- (169.6000,48.6000) -- (169.7000,48.5000) -- (169.8000,48.3000) -- (169.9000,48.0000) -- (170.0000,47.8000) -- (170.1000,47.5000) -- (170.1000,47.2000) -- (170.2000,46.9000) -- (170.3000,46.6000) -- (170.4000,46.2000) -- (170.5000,45.9000) -- (170.6000,45.5000) -- (170.7000,45.1000) -- (170.8000,44.8000) -- (170.8000,44.4000) -- (170.9000,44.0000) -- (171.0000,43.7000) -- (171.1000,43.4000) -- (171.2000,43.0000) -- (171.3000,42.7000) -- (171.4000,42.5000) -- (171.4000,42.2000) -- (171.5000,42.0000) -- (171.6000,41.8000) -- (171.7000,41.6000) -- (171.8000,41.5000) -- (171.9000,41.4000) -- (172.0000,41.3000) -- (172.1000,41.3000) -- (172.1000,41.3000) -- (172.2000,41.4000) -- (172.3000,41.6000) -- (172.4000,41.7000) -- (172.5000,42.0000) -- (172.6000,42.3000) -- (172.7000,42.6000) -- (172.8000,43.0000) -- (172.8000,43.4000) -- (172.9000,43.9000) -- (173.0000,44.5000) -- (173.1000,45.1000) -- (173.2000,45.7000) -- (173.3000,46.4000) -- (173.4000,47.1000) -- (173.4000,47.9000) -- (173.5000,48.8000) -- (173.6000,49.6000) -- (173.7000,50.5000) -- (173.8000,51.5000) -- (173.9000,52.5000) -- (174.0000,53.5000) -- (174.1000,54.5000) -- (174.1000,55.6000) -- (174.2000,56.6000) -- (174.3000,57.7000) -- (174.4000,58.9000) -- (174.5000,60.0000) -- (174.6000,61.1000) -- (174.7000,62.2000) -- (174.7000,63.4000) -- (174.8000,64.5000) -- (174.9000,65.6000) -- (175.0000,66.8000) -- (175.1000,67.8000) -- (175.2000,68.9000) -- (175.3000,70.0000) -- (175.4000,71.0000) -- (175.4000,72.1000) -- (175.5000,73.0000) -- (175.6000,74.0000) -- (175.7000,74.9000) -- (175.8000,75.8000) -- (175.9000,76.7000) -- (176.0000,77.5000) -- (176.0000,78.3000) -- (176.1000,79.0000) -- (176.2000,79.7000) -- (176.3000,80.3000) -- (176.4000,80.9000) -- (176.5000,81.5000) -- (176.6000,82.0000) -- (176.7000,82.5000) -- (176.7000,83.0000) -- (176.8000,83.4000) -- (176.9000,83.7000) -- (177.0000,84.0000) -- (177.1000,84.3000) -- (177.2000,84.6000) -- (177.3000,84.8000) -- (177.4000,85.0000) -- (177.4000,85.1000) -- (177.5000,85.3000) -- (177.6000,85.3000) -- (177.7000,85.4000) -- (177.8000,85.5000) -- (177.9000,85.5000) -- (178.0000,85.5000) -- (178.0000,85.5000) -- (178.1000,85.5000) -- (178.2000,85.5000) -- (178.3000,85.4000) -- (178.4000,85.4000) -- (178.5000,85.4000) -- (178.6000,85.3000) -- (178.7000,85.3000) -- (178.7000,85.2000) -- (178.8000,85.2000) -- (178.9000,85.2000) -- (179.0000,85.1000) -- (179.1000,85.1000) -- (179.2000,85.1000) -- (179.3000,85.1000) -- (179.3000,85.1000) -- (179.4000,85.1000) -- (179.5000,85.1000) -- (179.6000,85.2000) -- (179.7000,85.2000) -- (179.8000,85.3000) -- (179.9000,85.4000) -- (180.0000,85.4000) -- (180.0000,85.5000) -- (180.1000,85.6000) -- (180.2000,85.7000) -- (180.3000,85.8000) -- (180.4000,85.9000) -- (180.5000,86.0000) -- (180.6000,86.1000) -- (180.6000,86.2000) -- (180.7000,86.3000) -- (180.8000,86.4000) -- (180.9000,86.4000) -- (181.0000,86.5000) -- (181.1000,86.5000) -- (181.2000,86.6000) -- (181.3000,86.6000) -- (181.3000,86.6000) -- (181.4000,86.5000) -- (181.5000,86.4000) -- (181.6000,86.4000) -- (181.7000,86.2000) -- (181.8000,86.1000) -- (181.9000,85.9000) -- (182.0000,85.6000) -- (182.0000,85.4000) -- (182.1000,85.1000) -- (182.2000,84.7000) -- (182.3000,84.3000) -- (182.4000,83.9000) -- (182.5000,83.5000) -- (182.6000,82.9000) -- (182.6000,82.4000) -- (182.7000,81.8000) -- (182.8000,81.2000) -- (182.9000,80.5000) -- (183.0000,79.8000) -- (183.1000,79.0000) -- (183.2000,78.2000) -- (183.3000,77.4000) -- (183.3000,76.5000) -- (183.4000,75.6000) -- (183.5000,74.7000) -- (183.6000,73.7000) -- (183.7000,72.7000) -- (183.8000,71.7000) -- (183.9000,70.7000) -- (183.9000,69.6000) -- (184.0000,68.6000) -- (184.1000,67.5000) -- (184.2000,66.4000) -- (184.3000,65.3000) -- (184.4000,64.2000) -- (184.5000,63.1000) -- (184.6000,62.1000) -- (184.6000,61.0000) -- (184.7000,59.9000) -- (184.8000,58.9000) -- (184.9000,57.9000) -- (185.0000,56.9000) -- (185.1000,55.9000) -- (185.2000,54.9000) -- (185.2000,54.0000) -- (185.3000,53.1000) -- (185.4000,52.3000) -- (185.5000,51.5000) -- (185.6000,50.7000) -- (185.7000,50.0000) -- (185.8000,49.3000) -- (185.9000,48.7000) -- (185.9000,48.1000) -- (186.0000,47.5000) -- (186.1000,47.0000) -- (186.2000,46.6000) -- (186.3000,46.2000) -- (186.4000,45.8000) -- (186.5000,45.5000) -- (186.5000,45.3000) -- (186.6000,45.0000) -- (186.7000,44.9000) -- (186.8000,44.8000) -- (186.9000,44.7000) -- (187.0000,44.6000) -- (187.1000,44.6000) -- (187.2000,44.7000) -- (187.2000,44.7000) -- (187.3000,44.8000) -- (187.4000,45.0000) -- (187.5000,45.1000) -- (187.6000,45.3000) -- (187.7000,45.5000) -- (187.8000,45.7000) -- (187.9000,46.0000) -- (187.9000,46.2000) -- (188.0000,46.4000) -- (188.1000,46.7000) -- (188.2000,47.0000) -- (188.3000,47.2000) -- (188.4000,47.4000) -- (188.5000,47.7000) -- (188.5000,47.9000) -- (188.6000,48.1000) -- (188.7000,48.3000) -- (188.8000,48.5000) -- (188.9000,48.7000) -- (189.0000,48.8000) -- (189.1000,48.9000) -- (189.2000,49.0000) -- (189.2000,49.1000) -- (189.3000,49.1000) -- (189.4000,49.1000) -- (189.5000,49.1000) -- (189.6000,49.1000) -- (189.7000,49.0000) -- (189.8000,48.9000) -- (189.8000,48.8000) -- (189.9000,48.6000) -- (190.0000,48.4000) -- (190.1000,48.2000) -- (190.2000,47.9000) -- (190.3000,47.7000) -- (190.4000,47.4000) -- (190.5000,47.1000) -- (190.5000,46.8000) -- (190.6000,46.4000) -- (190.7000,46.1000) -- (190.8000,45.7000) -- (190.9000,45.4000) -- (191.0000,45.0000) -- (191.1000,44.6000) -- (191.1000,44.3000) -- (191.2000,43.9000) -- (191.3000,43.6000) -- (191.4000,43.2000) -- (191.5000,42.9000) -- (191.6000,42.6000) -- (191.7000,42.3000) -- (191.8000,42.1000) -- (191.8000,41.9000) -- (191.9000,41.7000) -- (192.0000,41.5000) -- (192.1000,41.4000) -- (192.2000,41.3000) -- (192.3000,41.3000) -- (192.4000,41.3000) -- (192.5000,41.3000) -- (192.5000,41.4000) -- (192.6000,41.6000) -- (192.7000,41.8000) -- (192.8000,42.0000) -- (192.9000,42.4000) -- (193.0000,42.7000) -- (193.1000,43.1000) -- (193.1000,43.6000) -- (193.2000,44.1000) -- (193.3000,44.7000) -- (193.4000,45.3000) -- (193.5000,45.9000) -- (193.6000,46.7000) -- (193.7000,47.4000) -- (193.8000,48.2000) -- (193.8000,49.1000) -- (193.9000,50.0000) -- (194.0000,50.9000) -- (194.1000,51.8000) -- (194.2000,52.8000) -- (194.3000,53.9000) -- (194.4000,54.9000) -- (194.4000,56.0000) -- (194.5000,57.1000) -- (194.6000,58.2000) -- (194.7000,59.3000) -- (194.8000,60.4000) -- (194.9000,61.6000) -- (195.0000,62.7000) -- (195.1000,63.8000) -- (195.1000,65.0000) -- (195.2000,66.1000) -- (195.3000,67.2000) -- (195.4000,68.3000) -- (195.5000,69.4000) -- (195.6000,70.4000) -- (195.7000,71.5000) -- (195.7000,72.5000) -- (195.8000,73.4000) -- (195.9000,74.4000) -- (196.0000,75.3000) -- (196.1000,76.2000) -- (196.2000,77.0000) -- (196.3000,77.8000) -- (196.4000,78.6000) -- (196.4000,79.3000) -- (196.5000,80.0000) -- (196.6000,80.6000) -- (196.7000,81.2000) -- (196.8000,81.7000) -- (196.9000,82.3000) -- (197.0000,82.7000) -- (197.1000,83.2000) -- (197.1000,83.5000) -- (197.2000,83.9000) -- (197.3000,84.2000) -- (197.4000,84.5000) -- (197.5000,84.7000) -- (197.6000,84.9000) -- (197.7000,85.1000) -- (197.7000,85.2000) -- (197.8000,85.3000) -- (197.9000,85.4000) -- (198.0000,85.5000) -- (198.1000,85.5000) -- (198.2000,85.5000) -- (198.3000,85.5000) -- (198.4000,85.5000) -- (198.4000,85.5000) -- (198.5000,85.5000) -- (198.6000,85.4000) -- (198.7000,85.4000) -- (198.8000,85.3000) -- (198.9000,85.3000) -- (199.0000,85.2000) -- (199.0000,85.2000) -- (199.1000,85.2000) -- (199.2000,85.1000) -- (199.3000,85.1000) -- (199.4000,85.1000) -- (199.5000,85.1000) -- (199.6000,85.1000) -- (199.7000,85.1000) -- (199.7000,85.1000) -- (199.8000,85.2000) -- (199.9000,85.2000) -- (200.0000,85.2000) -- (200.1000,85.3000) -- (200.2000,85.4000) -- (200.3000,85.5000) -- (200.3000,85.5000) -- (200.4000,85.6000) -- (200.5000,85.7000) -- (200.6000,85.8000) -- (200.7000,85.9000) -- (200.8000,86.0000) -- (200.9000,86.1000) -- (201.0000,86.2000) -- (201.0000,86.3000) -- (201.1000,86.4000) -- (201.2000,86.5000) -- (201.3000,86.5000) -- (201.4000,86.6000) -- (201.5000,86.6000) -- (201.6000,86.6000) -- (201.7000,86.6000) -- (201.7000,86.5000) -- (201.8000,86.4000) -- (201.9000,86.3000) -- (202.0000,86.2000) -- (202.1000,86.0000) -- (202.2000,85.8000) -- (202.3000,85.6000) -- (202.3000,85.3000) -- (202.4000,85.0000) -- (202.5000,84.6000) -- (202.6000,84.2000) -- (202.7000,83.8000) -- (202.8000,83.3000) -- (202.9000,82.8000) -- (203.0000,82.2000) -- (203.0000,81.6000) -- (203.1000,80.9000) -- (203.2000,80.2000) -- (203.3000,79.5000) -- (203.4000,78.7000) -- (203.5000,77.9000) -- (203.6000,77.0000) -- (203.6000,76.2000) -- (203.7000,75.2000) -- (203.8000,74.3000) -- (203.9000,73.3000) -- (204.0000,72.3000) -- (204.1000,71.3000) -- (204.2000,70.3000) -- (204.3000,69.2000) -- (204.3000,68.1000) -- (204.4000,67.1000) -- (204.5000,66.0000) -- (204.6000,64.9000) -- (204.7000,63.8000) -- (204.8000,62.7000) -- (204.9000,61.6000) -- (204.9000,60.6000) -- (205.0000,59.5000) -- (205.1000,58.5000) -- (205.2000,57.5000) -- (205.3000,56.5000) -- (205.4000,55.5000) -- (205.5000,54.6000) -- (205.6000,53.7000) -- (205.6000,52.8000) -- (205.7000,51.9000) -- (205.8000,51.1000) -- (205.9000,50.4000) -- (206.0000,49.7000) -- (206.1000,49.0000) -- (206.2000,48.4000) -- (206.3000,47.8000) -- (206.3000,47.3000) -- (206.4000,46.8000) -- (206.5000,46.4000) -- (206.6000,46.0000) -- (206.7000,45.7000) -- (206.8000,45.4000) -- (206.9000,45.1000) -- (206.9000,45.0000) -- (207.0000,44.8000) -- (207.1000,44.7000) -- (207.2000,44.6000) -- (207.3000,44.6000) -- (207.4000,44.6000) -- (207.5000,44.7000) -- (207.6000,44.8000) -- (207.6000,44.9000) -- (207.7000,45.0000) -- (207.8000,45.2000) -- (207.9000,45.4000) -- (208.0000,45.6000) -- (208.1000,45.8000) -- (208.2000,46.0000) -- (208.2000,46.3000) -- (208.3000,46.5000) -- (208.4000,46.8000) -- (208.5000,47.1000) -- (208.6000,47.3000) -- (208.7000,47.5000) -- (208.8000,47.8000) -- (208.9000,48.0000) -- (208.9000,48.2000) -- (209.0000,48.4000) -- (209.1000,48.6000) -- (209.2000,48.8000) -- (209.3000,48.9000) -- (209.4000,49.0000) -- (209.5000,49.1000) -- (209.5000,49.1000) -- (209.6000,49.2000) -- (209.7000,49.2000) -- (209.8000,49.1000) -- (209.9000,49.1000) -- (210.0000,49.0000) -- (210.1000,48.9000) -- (210.2000,48.7000) -- (210.2000,48.5000) -- (210.3000,48.3000) -- (210.4000,48.1000) -- (210.5000,47.9000) -- (210.6000,47.6000) -- (210.7000,47.3000) -- (210.8000,47.0000) -- (210.9000,46.6000) -- (210.9000,46.3000) -- (211.0000,45.9000) -- (211.1000,45.6000) -- (211.2000,45.2000) -- (211.3000,44.8000) -- (211.4000,44.5000) -- (211.5000,44.1000) -- (211.5000,43.8000) -- (211.6000,43.4000) -- (211.7000,43.1000) -- (211.8000,42.8000) -- (211.9000,42.5000) -- (212.0000,42.2000) -- (212.1000,42.0000) -- (212.2000,41.8000) -- (212.2000,41.6000) -- (212.3000,41.4000) -- (212.4000,41.3000) -- (212.5000,41.3000) -- (212.6000,41.3000) -- (212.7000,41.3000) -- (212.8000,41.4000) -- (212.8000,41.5000) -- (212.9000,41.6000) -- (213.0000,41.9000) -- (213.1000,42.1000) -- (213.2000,42.5000) -- (213.3000,42.8000) -- (213.4000,43.3000) -- (213.5000,43.7000) -- (213.5000,44.3000) -- (213.6000,44.9000) -- (213.7000,45.5000) -- (213.8000,46.2000) -- (213.9000,46.9000) -- (214.0000,47.7000) -- (214.1000,48.5000) -- (214.1000,49.4000) -- (214.2000,50.3000) -- (214.3000,51.2000) -- (214.4000,52.2000) -- (214.5000,53.2000) -- (214.6000,54.2000) -- (214.7000,55.3000) -- (214.8000,56.4000) -- (214.8000,57.5000) -- (214.9000,58.6000) -- (215.0000,59.7000) -- (215.1000,60.9000) -- (215.2000,62.0000) -- (215.3000,63.1000) -- (215.4000,64.3000) -- (215.5000,65.4000) -- (215.5000,66.5000) -- (215.6000,67.6000) -- (215.7000,68.7000) -- (215.8000,69.8000) -- (215.9000,70.8000) -- (216.0000,71.9000) -- (216.1000,72.9000) -- (216.1000,73.8000) -- (216.2000,74.8000) -- (216.3000,75.7000) -- (216.4000,76.5000) -- (216.5000,77.4000) -- (216.6000,78.1000) -- (216.7000,78.9000) -- (216.8000,79.6000) -- (216.8000,80.2000) -- (216.9000,80.9000) -- (217.0000,81.4000) -- (217.1000,82.0000) -- (217.2000,82.5000) -- (217.3000,82.9000) -- (217.4000,83.3000) -- (217.4000,83.7000) -- (217.5000,84.0000) -- (217.6000,84.3000) -- (217.7000,84.6000) -- (217.8000,84.8000) -- (217.9000,85.0000) -- (218.0000,85.1000) -- (218.1000,85.3000) -- (218.1000,85.4000) -- (218.2000,85.4000) -- (218.3000,85.5000) -- (218.4000,85.5000) -- (218.5000,85.5000) -- (218.6000,85.5000) -- (218.7000,85.5000) -- (218.7000,85.5000) -- (218.8000,85.5000) -- (218.9000,85.4000) -- (219.0000,85.4000) -- (219.1000,85.3000) -- (219.2000,85.3000) -- (219.3000,85.2000) -- (219.4000,85.2000) -- (219.4000,85.2000) -- (219.5000,85.1000) -- (219.6000,85.1000) -- (219.7000,85.1000) -- (219.8000,85.1000) -- (219.9000,85.1000) -- (220.0000,85.1000) -- (220.1000,85.1000) -- (220.1000,85.2000) -- (220.2000,85.2000) -- (220.3000,85.3000) -- (220.4000,85.3000) -- (220.5000,85.4000) -- (220.6000,85.5000) -- (220.7000,85.6000) -- (220.7000,85.7000) -- (220.8000,85.8000) -- (220.9000,85.9000) -- (221.0000,86.0000) -- (221.1000,86.1000) -- (221.2000,86.2000) -- (221.3000,86.3000) -- (221.4000,86.4000) -- (221.4000,86.4000) -- (221.5000,86.5000) -- (221.6000,86.5000) -- (221.7000,86.6000) -- (221.8000,86.6000) -- (221.9000,86.6000) -- (222.0000,86.6000) -- (222.0000,86.5000) -- (222.1000,86.4000) -- (222.2000,86.3000) -- (222.3000,86.1000) -- (222.4000,86.0000) -- (222.5000,85.7000) -- (222.6000,85.5000) -- (222.7000,85.2000) -- (222.7000,84.9000) -- (222.8000,84.5000) -- (222.9000,84.1000) -- (223.0000,83.6000) -- (223.1000,83.1000) -- (223.2000,82.6000) -- (223.3000,82.0000) -- (223.3000,81.3000) -- (223.4000,80.7000) -- (223.5000,80.0000) -- (223.6000,79.2000) -- (223.7000,78.4000) -- (223.8000,77.6000) -- (223.9000,76.7000) -- (224.0000,75.8000) -- (224.0000,74.9000) -- (224.1000,73.9000) -- (224.2000,73.0000) -- (224.3000,71.9000) -- (224.4000,70.9000) -- (224.5000,69.9000) -- (224.6000,68.8000) -- (224.7000,67.7000) -- (224.7000,66.6000) -- (224.8000,65.6000) -- (224.9000,64.5000) -- (225.0000,63.4000) -- (225.1000,62.3000) -- (225.2000,61.2000) -- (225.3000,60.1000) -- (225.3000,59.1000) -- (225.4000,58.1000) -- (225.5000,57.1000) -- (225.6000,56.1000) -- (225.7000,55.1000) -- (225.8000,54.2000) -- (225.9000,53.3000) -- (226.0000,52.4000) -- (226.0000,51.6000) -- (226.1000,50.8000) -- (226.2000,50.1000) -- (226.3000,49.4000) -- (226.4000,48.7000) -- (226.5000,48.1000) -- (226.6000,47.6000) -- (226.6000,47.1000) -- (226.7000,46.6000) -- (226.8000,46.2000) -- (226.9000,45.8000) -- (227.0000,45.5000) -- (227.1000,45.3000) -- (227.2000,45.0000) -- (227.3000,44.9000) -- (227.3000,44.7000) -- (227.4000,44.6000) -- (227.5000,44.6000) -- (227.6000,44.6000) -- (227.7000,44.6000) -- (227.8000,44.7000) -- (227.9000,44.8000) -- (227.9000,44.9000) -- (228.0000,45.1000) -- (228.1000,45.2000) -- (228.2000,45.4000) -- (228.3000,45.7000) -- (228.4000,45.9000) -- (228.5000,46.1000) -- (228.6000,46.4000) -- (228.6000,46.6000) -- (228.7000,46.9000) -- (228.8000,47.2000) -- (228.9000,47.4000) -- (229.0000,47.6000) -- (229.1000,47.9000) -- (229.2000,48.1000) -- (229.2000,48.3000) -- (229.3000,48.5000) -- (229.4000,48.7000) -- (229.5000,48.8000) -- (229.6000,48.9000) -- (229.7000,49.0000) -- (229.8000,49.1000) -- (229.9000,49.2000) -- (229.9000,49.2000) -- (230.0000,49.2000) -- (230.1000,49.1000) -- (230.2000,49.1000) -- (230.3000,49.0000) -- (230.4000,48.8000) -- (230.5000,48.7000) -- (230.6000,48.5000) -- (230.6000,48.3000) -- (230.7000,48.0000) -- (230.8000,47.8000) -- (230.9000,47.5000) -- (231.0000,47.2000) -- (231.1000,46.9000) -- (231.2000,46.5000) -- (231.2000,46.2000) -- (231.3000,45.8000) -- (231.4000,45.4000) -- (231.5000,45.1000) -- (231.6000,44.7000) -- (231.7000,44.3000) -- (231.8000,44.0000) -- (231.9000,43.6000) -- (231.9000,43.3000) -- (232.0000,43.0000) -- (232.1000,42.6000) -- (232.2000,42.4000) -- (232.3000,42.1000) -- (232.4000,41.9000) -- (232.5000,41.7000) -- (232.5000,41.5000) -- (232.6000,41.4000) -- (232.7000,41.3000) -- (232.8000,41.2000) -- (232.9000,41.2000) -- (233.0000,41.3000) -- (233.1000,41.4000) -- (233.2000,41.5000) -- (233.2000,41.7000) -- (233.3000,41.9000) -- (233.4000,42.2000) -- (233.5000,42.6000) -- (233.6000,43.0000) -- (233.7000,43.4000) -- (233.8000,43.9000) -- (233.8000,44.5000) -- (233.9000,45.1000) -- (234.0000,45.7000) -- (234.1000,46.4000) -- (234.2000,47.2000) -- (234.3000,48.0000) -- (234.4000,48.8000) -- (234.5000,49.7000) -- (234.5000,50.6000) -- (234.6000,51.6000) -- (234.7000,52.6000) -- (234.8000,53.6000) -- (234.9000,54.6000) -- (235.0000,55.7000) -- (235.1000,56.8000) -- (235.1000,57.9000) -- (235.2000,59.0000) -- (235.3000,60.2000) -- (235.4000,61.3000) -- (235.5000,62.4000) -- (235.6000,63.6000) -- (235.7000,64.7000) -- (235.8000,65.8000) -- (235.8000,67.0000) -- (235.9000,68.1000) -- (236.0000,69.2000) -- (236.1000,70.2000) -- (236.2000,71.3000) -- (236.3000,72.3000) -- (236.4000,73.3000) -- (236.5000,74.2000) -- (236.5000,75.1000) -- (236.6000,76.0000) -- (236.7000,76.9000) -- (236.8000,77.7000) -- (236.9000,78.5000) -- (237.0000,79.2000) -- (237.1000,79.9000) -- (237.1000,80.5000) -- (237.2000,81.1000) -- (237.3000,81.7000) -- (237.4000,82.2000) -- (237.5000,82.7000) -- (237.6000,83.1000) -- (237.7000,83.5000) -- (237.8000,83.9000) -- (237.8000,84.2000) -- (237.9000,84.4000) -- (238.0000,84.7000) -- (238.1000,84.9000) -- (238.2000,85.1000) -- (238.3000,85.2000) -- (238.4000,85.3000) -- (238.4000,85.4000) -- (238.5000,85.5000) -- (238.6000,85.5000) -- (238.7000,85.5000) -- (238.8000,85.5000) -- (238.9000,85.5000) -- (239.0000,85.5000) -- (239.1000,85.5000) -- (239.1000,85.4000) -- (239.2000,85.4000) -- (239.3000,85.4000) -- (239.4000,85.3000) -- (239.5000,85.3000) -- (239.6000,85.2000) -- (239.7000,85.2000) -- (239.7000,85.1000) -- (239.8000,85.1000) -- (239.9000,85.1000) -- (240.0000,85.1000) -- (240.1000,85.1000) -- (240.2000,85.1000) -- (240.3000,85.1000) -- (240.4000,85.1000) -- (240.4000,85.2000) -- (240.5000,85.2000) -- (240.6000,85.3000) -- (240.7000,85.4000) -- (240.8000,85.4000) -- (240.9000,85.5000) -- (241.0000,85.6000) -- (241.0000,85.7000) -- (241.1000,85.8000) -- (241.2000,85.9000) -- (241.3000,86.0000) -- (241.4000,86.1000) -- (241.5000,86.2000) -- (241.6000,86.3000) -- (241.7000,86.4000) -- (241.7000,86.5000) -- (241.8000,86.5000) -- (241.9000,86.6000) -- (242.0000,86.6000) -- (242.1000,86.6000) -- (242.2000,86.6000) -- (242.3000,86.5000) -- (242.4000,86.5000) -- (242.4000,86.4000) -- (242.5000,86.3000) -- (242.6000,86.1000) -- (242.7000,85.9000) -- (242.8000,85.7000) -- (242.9000,85.4000) -- (243.0000,85.1000) -- (243.0000,84.7000) -- (243.1000,84.3000) -- (243.2000,83.9000) -- (243.3000,83.4000) -- (243.4000,82.9000) -- (243.5000,82.4000) -- (243.6000,81.8000) -- (243.7000,81.1000) -- (243.7000,80.4000) -- (243.8000,79.7000) -- (243.9000,78.9000) -- (244.0000,78.1000) -- (244.1000,77.3000) -- (244.2000,76.4000) -- (244.3000,75.5000) -- (244.3000,74.5000) -- (244.4000,73.6000) -- (244.5000,72.6000) -- (244.6000,71.6000) -- (244.7000,70.5000) -- (244.8000,69.5000) -- (244.9000,68.4000) -- (245.0000,67.3000) -- (245.0000,66.2000) -- (245.1000,65.1000) -- (245.2000,64.0000) -- (245.3000,62.9000) -- (245.4000,61.9000) -- (245.5000,60.8000) -- (245.6000,59.7000) -- (245.7000,58.7000) -- (245.7000,57.7000) -- (245.8000,56.7000) -- (245.9000,55.7000) -- (246.0000,54.7000) -- (246.1000,53.8000) -- (246.2000,52.9000) -- (246.3000,52.1000) -- (246.3000,51.3000) -- (246.4000,50.5000) -- (246.5000,49.8000) -- (246.6000,49.1000) -- (246.7000,48.5000) -- (246.8000,47.9000) -- (246.9000,47.3000) -- (247.0000,46.9000) -- (247.0000,46.4000) -- (247.1000,46.0000) -- (247.2000,45.7000) -- (247.3000,45.4000) -- (247.4000,45.1000) -- (247.5000,44.9000) -- (247.6000,44.8000) -- (247.6000,44.7000) -- (247.7000,44.6000) -- (247.8000,44.6000) -- (247.9000,44.6000) -- (248.0000,44.6000) -- (248.1000,44.7000) -- (248.2000,44.8000) -- (248.3000,45.0000) -- (248.3000,45.1000) -- (248.4000,45.3000) -- (248.5000,45.5000) -- (248.6000,45.7000) -- (248.7000,46.0000) -- (248.8000,46.2000) -- (248.9000,46.5000) -- (248.9000,46.7000) -- (249.0000,47.0000) -- (249.1000,47.3000) -- (249.2000,47.5000) -- (249.3000,47.7000) -- (249.4000,48.0000) -- (249.5000,48.2000) -- (249.6000,48.4000) -- (249.6000,48.6000) -- (249.7000,48.8000) -- (249.8000,48.9000) -- (249.9000,49.0000) -- (250.0000,49.1000) -- (250.1000,49.2000) -- (250.2000,49.2000) -- (250.3000,49.2000) -- (250.3000,49.2000) -- (250.4000,49.1000) -- (250.5000,49.0000) -- (250.6000,48.9000) -- (250.7000,48.8000) -- (250.8000,48.6000) -- (250.9000,48.4000) -- (250.9000,48.2000) -- (251.0000,47.9000) -- (251.1000,47.7000) -- (251.2000,47.4000) -- (251.3000,47.1000) -- (251.4000,46.7000) -- (251.5000,46.4000) -- (251.6000,46.0000) -- (251.6000,45.7000) -- (251.7000,45.3000) -- (251.8000,44.9000) -- (251.9000,44.5000);



  \end{scope}
  \begin{scope}[scale=1.006,draw=blue,line cap=rect,line join=bevel,line width=0.800pt]
  \end{scope}
  \begin{scope}[scale=1.006,draw=blue,line cap=rect,line join=bevel,line width=0.800pt]
  \end{scope}
  \begin{scope}[cm={{1.26012,0.0,0.0,1.26012,(-619.61892,-54.78689)}},draw=blue,line cap=round,line join=round,line width=0.480pt]
    \path[draw] (165.5000,13.5000) -- (165.5000,95.5000) -- (251.5000,95.5000) -- (251.5000,13.5000) -- (165.5000,13.5000);



  \end{scope}
  \begin{scope}[scale=1.006,draw=blue,line cap=rect,line join=bevel,line width=0.800pt]
  \end{scope}
  \begin{scope}[cm={{1.00588,0.0,0.0,1.00588,(39.2294,199.165)}},draw=blue,line cap=rect,line join=bevel,line width=0.800pt]
  \end{scope}
  \begin{scope}[cm={{1.00588,0.0,0.0,1.00588,(39.2294,199.165)}},draw=blue,line cap=rect,line join=bevel,line width=0.800pt]
  \end{scope}
  \begin{scope}[cm={{1.00588,0.0,0.0,1.00588,(39.2294,199.165)}},draw=blue,line cap=rect,line join=bevel,line width=0.800pt]
  \end{scope}
  \begin{scope}[cm={{1.00588,0.0,0.0,1.00588,(39.2294,199.165)}},draw=blue,line cap=rect,line join=bevel,line width=0.800pt]
  \end{scope}
  \begin{scope}[cm={{1.00588,0.0,0.0,1.00588,(39.2294,199.165)}},draw=blue,line cap=rect,line join=bevel,line width=0.800pt]
  \end{scope}
  \begin{scope}[cm={{1.00588,0.0,0.0,1.00588,(39.2294,199.165)}},draw=blue,line cap=rect,line join=bevel,line width=0.800pt]
  \end{scope}
  \begin{scope}[scale=1.006,draw=blue,line cap=rect,line join=bevel,line width=0.800pt]
  \end{scope}
  \begin{scope}[scale=1.006,draw=blue,line cap=rect,line join=bevel,line width=0.800pt]
  \end{scope}
  \begin{scope}[cm={{1.00588,0.0,0.0,1.00588,(40.2353,177.035)}},draw=blue,line cap=rect,line join=bevel,line width=0.800pt]
  \end{scope}
  \begin{scope}[cm={{1.00588,0.0,0.0,1.00588,(40.2353,177.035)}},draw=blue,line cap=rect,line join=bevel,line width=0.800pt]
  \end{scope}
  \begin{scope}[cm={{1.00588,0.0,0.0,1.00588,(40.2353,177.035)}},draw=blue,line cap=rect,line join=bevel,line width=0.800pt]
  \end{scope}
  \begin{scope}[cm={{1.00588,0.0,0.0,1.00588,(40.2353,177.035)}},draw=blue,line cap=rect,line join=bevel,line width=0.800pt]
  \end{scope}
  \begin{scope}[cm={{1.00588,0.0,0.0,1.00588,(40.2353,177.035)}},draw=blue,line cap=rect,line join=bevel,line width=0.800pt]
  \end{scope}
  \begin{scope}[cm={{1.00588,0.0,0.0,1.00588,(40.2353,177.035)}},draw=blue,line cap=rect,line join=bevel,line width=0.800pt]
  \end{scope}
  \begin{scope}[scale=1.006,draw=blue,line cap=rect,line join=bevel,line width=0.800pt]
  \end{scope}
  \begin{scope}[scale=1.006,draw=blue,line cap=rect,line join=bevel,line width=0.800pt]
  \end{scope}
  \begin{scope}[cm={{1.00588,0.0,0.0,1.00588,(40.2353,153.9)}},draw=blue,line cap=rect,line join=bevel,line width=0.800pt]
  \end{scope}
  \begin{scope}[cm={{1.00588,0.0,0.0,1.00588,(40.2353,153.9)}},draw=blue,line cap=rect,line join=bevel,line width=0.800pt]
  \end{scope}
  \begin{scope}[cm={{1.00588,0.0,0.0,1.00588,(40.2353,153.9)}},draw=blue,line cap=rect,line join=bevel,line width=0.800pt]
  \end{scope}
  \begin{scope}[cm={{1.00588,0.0,0.0,1.00588,(40.2353,153.9)}},draw=blue,line cap=rect,line join=bevel,line width=0.800pt]
  \end{scope}
  \begin{scope}[cm={{1.00588,0.0,0.0,1.00588,(40.2353,153.9)}},draw=blue,line cap=rect,line join=bevel,line width=0.800pt]
  \end{scope}
  \begin{scope}[cm={{1.00588,0.0,0.0,1.00588,(40.2353,153.9)}},draw=blue,line cap=rect,line join=bevel,line width=0.800pt]
  \end{scope}
  \begin{scope}[scale=1.006,draw=blue,line cap=rect,line join=bevel,line width=0.800pt]
  \end{scope}
  \begin{scope}[scale=1.006,draw=blue,line cap=rect,line join=bevel,line width=0.800pt]
  \end{scope}
  \begin{scope}[cm={{1.00588,0.0,0.0,1.00588,(39.2294,131.771)}},draw=blue,line cap=rect,line join=bevel,line width=0.800pt]
  \end{scope}
  \begin{scope}[cm={{1.00588,0.0,0.0,1.00588,(39.2294,131.771)}},draw=blue,line cap=rect,line join=bevel,line width=0.800pt]
  \end{scope}
  \begin{scope}[cm={{1.00588,0.0,0.0,1.00588,(39.2294,131.771)}},draw=blue,line cap=rect,line join=bevel,line width=0.800pt]
  \end{scope}
  \begin{scope}[cm={{1.00588,0.0,0.0,1.00588,(39.2294,131.771)}},draw=blue,line cap=rect,line join=bevel,line width=0.800pt]
  \end{scope}
  \begin{scope}[cm={{1.00588,0.0,0.0,1.00588,(39.2294,131.771)}},draw=blue,line cap=rect,line join=bevel,line width=0.800pt]
  \end{scope}
  \begin{scope}[cm={{1.00588,0.0,0.0,1.00588,(39.2294,131.771)}},draw=blue,line cap=rect,line join=bevel,line width=0.800pt]
  \end{scope}
  \begin{scope}[scale=1.006,draw=blue,line cap=rect,line join=bevel,line width=0.800pt]
  \end{scope}
  \begin{scope}[scale=1.006,draw=blue,line cap=rect,line join=bevel,line width=0.800pt]
  \end{scope}
  \begin{scope}[cm={{1.00588,0.0,0.0,1.00588,(53.3118,217.271)}},draw=blue,line cap=rect,line join=bevel,line width=0.800pt]
  \end{scope}
  \begin{scope}[cm={{1.00588,0.0,0.0,1.00588,(53.3118,217.271)}},draw=blue,line cap=rect,line join=bevel,line width=0.800pt]
  \end{scope}
  \begin{scope}[cm={{1.00588,0.0,0.0,1.00588,(53.3118,217.271)}},draw=blue,line cap=rect,line join=bevel,line width=0.800pt]
  \end{scope}
  \begin{scope}[cm={{1.00588,0.0,0.0,1.00588,(53.3118,217.271)}},draw=blue,line cap=rect,line join=bevel,line width=0.800pt]
  \end{scope}
  \begin{scope}[cm={{1.00588,0.0,0.0,1.00588,(53.3118,217.271)}},draw=blue,line cap=rect,line join=bevel,line width=0.800pt]
  \end{scope}
  \begin{scope}[cm={{1.00588,0.0,0.0,1.00588,(53.3118,217.271)}},draw=blue,line cap=rect,line join=bevel,line width=0.800pt]
  \end{scope}
  \begin{scope}[scale=1.006,draw=blue,line cap=rect,line join=bevel,line width=0.800pt]
  \end{scope}
  \begin{scope}[scale=1.006,draw=blue,line cap=rect,line join=bevel,line width=0.800pt]
  \end{scope}
  \begin{scope}[cm={{1.00588,0.0,0.0,1.00588,(82.4824,217.271)}},draw=blue,line cap=rect,line join=bevel,line width=0.800pt]
  \end{scope}
  \begin{scope}[cm={{1.00588,0.0,0.0,1.00588,(82.4824,217.271)}},draw=blue,line cap=rect,line join=bevel,line width=0.800pt]
  \end{scope}
  \begin{scope}[cm={{1.00588,0.0,0.0,1.00588,(82.4824,217.271)}},draw=blue,line cap=rect,line join=bevel,line width=0.800pt]
  \end{scope}
  \begin{scope}[cm={{1.00588,0.0,0.0,1.00588,(82.4824,217.271)}},draw=blue,line cap=rect,line join=bevel,line width=0.800pt]
  \end{scope}
  \begin{scope}[cm={{1.00588,0.0,0.0,1.00588,(82.4824,217.271)}},draw=blue,line cap=rect,line join=bevel,line width=0.800pt]
  \end{scope}
  \begin{scope}[cm={{1.00588,0.0,0.0,1.00588,(82.4824,217.271)}},draw=blue,line cap=rect,line join=bevel,line width=0.800pt]
  \end{scope}
  \begin{scope}[scale=1.006,draw=blue,line cap=rect,line join=bevel,line width=0.800pt]
  \end{scope}
  \begin{scope}[scale=1.006,draw=blue,line cap=rect,line join=bevel,line width=0.800pt]
  \end{scope}
  \begin{scope}[cm={{1.00588,0.0,0.0,1.00588,(111.653,217.271)}},draw=blue,line cap=rect,line join=bevel,line width=0.800pt]
  \end{scope}
  \begin{scope}[cm={{1.00588,0.0,0.0,1.00588,(111.653,217.271)}},draw=blue,line cap=rect,line join=bevel,line width=0.800pt]
  \end{scope}
  \begin{scope}[cm={{1.00588,0.0,0.0,1.00588,(111.653,217.271)}},draw=blue,line cap=rect,line join=bevel,line width=0.800pt]
  \end{scope}
  \begin{scope}[cm={{1.00588,0.0,0.0,1.00588,(111.653,217.271)}},draw=blue,line cap=rect,line join=bevel,line width=0.800pt]
  \end{scope}
  \begin{scope}[cm={{1.00588,0.0,0.0,1.00588,(111.653,217.271)}},draw=blue,line cap=rect,line join=bevel,line width=0.800pt]
  \end{scope}
  \begin{scope}[cm={{1.00588,0.0,0.0,1.00588,(111.653,217.271)}},draw=blue,line cap=rect,line join=bevel,line width=0.800pt]
  \end{scope}
  \begin{scope}[scale=1.006,draw=blue,line cap=rect,line join=bevel,line width=0.800pt]
  \end{scope}
  \begin{scope}[scale=1.006,draw=blue,line cap=rect,line join=bevel,line width=0.800pt]
  \end{scope}
  \begin{scope}[cm={{1.00588,0.0,0.0,1.00588,(142.332,217.271)}},draw=blue,line cap=rect,line join=bevel,line width=0.800pt]
  \end{scope}
  \begin{scope}[cm={{1.00588,0.0,0.0,1.00588,(142.332,217.271)}},draw=blue,line cap=rect,line join=bevel,line width=0.800pt]
  \end{scope}
  \begin{scope}[cm={{1.00588,0.0,0.0,1.00588,(142.332,217.271)}},draw=blue,line cap=rect,line join=bevel,line width=0.800pt]
  \end{scope}
  \begin{scope}[cm={{1.00588,0.0,0.0,1.00588,(142.332,217.271)}},draw=blue,line cap=rect,line join=bevel,line width=0.800pt]
  \end{scope}
  \begin{scope}[cm={{1.00588,0.0,0.0,1.00588,(142.332,217.271)}},draw=blue,line cap=rect,line join=bevel,line width=0.800pt]
  \end{scope}
  \begin{scope}[cm={{1.00588,0.0,0.0,1.00588,(142.332,217.271)}},draw=blue,line cap=rect,line join=bevel,line width=0.800pt]
  \end{scope}
  \begin{scope}[scale=1.006,draw=blue,line cap=rect,line join=bevel,line width=0.800pt]
  \end{scope}
  \begin{scope}[scale=1.006,draw=blue,line cap=rect,line join=bevel,line width=0.800pt]
  \end{scope}
  \begin{scope}[cm={{1.00588,0.0,0.0,1.00588,(171.503,217.271)}},draw=blue,line cap=rect,line join=bevel,line width=0.800pt]
  \end{scope}
  \begin{scope}[cm={{1.00588,0.0,0.0,1.00588,(171.503,217.271)}},draw=blue,line cap=rect,line join=bevel,line width=0.800pt]
  \end{scope}
  \begin{scope}[cm={{1.00588,0.0,0.0,1.00588,(171.503,217.271)}},draw=blue,line cap=rect,line join=bevel,line width=0.800pt]
  \end{scope}
  \begin{scope}[cm={{1.00588,0.0,0.0,1.00588,(171.503,217.271)}},draw=blue,line cap=rect,line join=bevel,line width=0.800pt]
  \end{scope}
  \begin{scope}[cm={{1.00588,0.0,0.0,1.00588,(171.503,217.271)}},draw=blue,line cap=rect,line join=bevel,line width=0.800pt]
  \end{scope}
  \begin{scope}[cm={{1.00588,0.0,0.0,1.00588,(171.503,217.271)}},draw=blue,line cap=rect,line join=bevel,line width=0.800pt]
  \end{scope}
  \begin{scope}[scale=1.006,draw=blue,line cap=rect,line join=bevel,line width=0.800pt]
  \end{scope}
  \begin{scope}[scale=1.006,draw=blue,line cap=rect,line join=bevel,line width=0.800pt]
  \end{scope}
  \begin{scope}[cm={{1.00588,0.0,0.0,1.00588,(200.674,217.271)}},draw=blue,line cap=rect,line join=bevel,line width=0.800pt]
  \end{scope}
  \begin{scope}[cm={{1.00588,0.0,0.0,1.00588,(200.674,217.271)}},draw=blue,line cap=rect,line join=bevel,line width=0.800pt]
  \end{scope}
  \begin{scope}[cm={{1.00588,0.0,0.0,1.00588,(200.674,217.271)}},draw=blue,line cap=rect,line join=bevel,line width=0.800pt]
  \end{scope}
  \begin{scope}[cm={{1.00588,0.0,0.0,1.00588,(200.674,217.271)}},draw=blue,line cap=rect,line join=bevel,line width=0.800pt]
  \end{scope}
  \begin{scope}[cm={{1.00588,0.0,0.0,1.00588,(200.674,217.271)}},draw=blue,line cap=rect,line join=bevel,line width=0.800pt]
  \end{scope}
  \begin{scope}[cm={{1.00588,0.0,0.0,1.00588,(200.674,217.271)}},draw=blue,line cap=rect,line join=bevel,line width=0.800pt]
  \end{scope}
  \begin{scope}[scale=1.006,draw=blue,line cap=rect,line join=bevel,line width=0.800pt]
  \end{scope}
  \begin{scope}[scale=1.006,draw=blue,line cap=rect,line join=bevel,line width=0.800pt]
  \end{scope}
  \begin{scope}[cm={{1.00588,0.0,0.0,1.00588,(229.341,217.271)}},draw=blue,line cap=rect,line join=bevel,line width=0.800pt]
  \end{scope}
  \begin{scope}[cm={{1.00588,0.0,0.0,1.00588,(229.341,217.271)}},draw=blue,line cap=rect,line join=bevel,line width=0.800pt]
  \end{scope}
  \begin{scope}[cm={{1.00588,0.0,0.0,1.00588,(229.341,217.271)}},draw=blue,line cap=rect,line join=bevel,line width=0.800pt]
  \end{scope}
  \begin{scope}[cm={{1.00588,0.0,0.0,1.00588,(229.341,217.271)}},draw=blue,line cap=rect,line join=bevel,line width=0.800pt]
  \end{scope}
  \begin{scope}[cm={{1.00588,0.0,0.0,1.00588,(229.341,217.271)}},draw=blue,line cap=rect,line join=bevel,line width=0.800pt]
  \end{scope}
  \begin{scope}[cm={{1.00588,0.0,0.0,1.00588,(229.341,217.271)}},draw=blue,line cap=rect,line join=bevel,line width=0.800pt]
  \end{scope}
  \begin{scope}[scale=1.006,draw=blue,line cap=rect,line join=bevel,line width=0.800pt]
  \end{scope}
  \begin{scope}[scale=1.006,draw=blue,line cap=rect,line join=bevel,line width=0.800pt]
  \end{scope}
  \begin{scope}[scale=1.006,draw=blue,line cap=rect,line join=bevel,line width=0.800pt]
  \end{scope}
  \begin{scope}[scale=1.006,draw=blue,line cap=rect,line join=bevel,line width=0.800pt]
  \end{scope}
  \begin{scope}[scale=1.006,draw=blue,line cap=rect,line join=bevel,line width=0.800pt]
  \end{scope}
  \begin{scope}[scale=1.006,draw=blue,line cap=rect,line join=bevel,line width=0.800pt]
  \end{scope}
  \begin{scope}[cm={{1.00588,0.0,0.0,1.00588,(236.382,132.776)}},draw=blue,line cap=rect,line join=bevel,line width=0.800pt]
  \end{scope}
  \begin{scope}[cm={{1.00588,0.0,0.0,1.00588,(236.382,132.776)}},draw=blue,line cap=rect,line join=bevel,line width=0.800pt]
  \end{scope}
  \begin{scope}[cm={{1.00588,0.0,0.0,1.00588,(236.382,132.776)}},draw=blue,line cap=rect,line join=bevel,line width=0.800pt]
  \end{scope}
  \begin{scope}[cm={{1.00588,0.0,0.0,1.00588,(236.382,132.776)}},draw=blue,line cap=rect,line join=bevel,line width=0.800pt]
  \end{scope}
  \begin{scope}[cm={{1.00588,0.0,0.0,1.00588,(236.382,132.776)}},draw=blue,line cap=rect,line join=bevel,line width=0.800pt]
  \end{scope}
  \begin{scope}[cm={{1.00588,0.0,0.0,1.00588,(236.382,132.776)}},draw=blue,line cap=rect,line join=bevel,line width=0.800pt]
  \end{scope}
  \begin{scope}[scale=1.006,draw=blue,line cap=rect,line join=bevel,line width=0.800pt]
  \end{scope}
  \begin{scope}[scale=1.006,draw=blue,line cap=rect,line join=bevel,line width=0.800pt]
  \end{scope}
  \begin{scope}[scale=1.006,draw=blue,line cap=rect,line join=bevel,line width=0.800pt]
  \end{scope}
  \begin{scope}[scale=1.006,draw=blue,line cap=rect,line join=bevel,line width=0.800pt]
  \end{scope}
  \begin{scope}[cm={{1.00588,0.0,0.0,1.00588,(197.153,129.759)}},draw=blue,line cap=rect,line join=bevel,line width=0.800pt]
  \end{scope}
  \begin{scope}[cm={{1.00588,0.0,0.0,1.00588,(197.153,129.759)}},draw=blue,line cap=rect,line join=bevel,line width=0.800pt]
  \end{scope}
  \begin{scope}[cm={{1.00588,0.0,0.0,1.00588,(197.153,129.759)}},draw=blue,line cap=rect,line join=bevel,line width=0.800pt]
  \end{scope}
  \begin{scope}[cm={{1.00588,0.0,0.0,1.00588,(197.153,129.759)}},draw=blue,line cap=rect,line join=bevel,line width=0.800pt]
  \end{scope}
  \begin{scope}[cm={{1.00588,0.0,0.0,1.00588,(197.153,129.759)}},draw=blue,line cap=rect,line join=bevel,line width=0.800pt]
  \end{scope}
  \begin{scope}[cm={{1.00588,0.0,0.0,1.00588,(197.153,129.759)}},draw=blue,line cap=rect,line join=bevel,line width=0.800pt]
  \end{scope}
  \begin{scope}[scale=1.006,draw=blue,line cap=rect,line join=bevel,line width=0.800pt]
  \end{scope}
  \begin{scope}[scale=1.006,draw=blue,line cap=rect,line join=bevel,line width=0.800pt]
  \end{scope}
  \begin{scope}[scale=1.006,draw=blue,line cap=rect,line join=bevel,line width=0.800pt]
  \end{scope}
  \begin{scope}[scale=1.006,draw=blue,line cap=rect,line join=bevel,line width=0.800pt]
  \end{scope}
  \begin{scope}[scale=1.006,draw=blue,line cap=rect,line join=bevel,line width=0.800pt]
  \end{scope}
  \begin{scope}[scale=1.006,draw=blue,line cap=rect,line join=bevel,line width=0.800pt]
  \end{scope}
  \begin{scope}[scale=1.006,draw=blue,line cap=rect,line join=bevel,line width=0.800pt]
  \end{scope}
  \begin{scope}[scale=1.006,draw=blue,line cap=rect,line join=bevel,line width=0.800pt]
  \end{scope}
  \begin{scope}[cm={{1.00588,0.0,0.0,1.00588,(39.2294,305.788)}},draw=blue,line cap=rect,line join=bevel,line width=0.800pt]
  \end{scope}
  \begin{scope}[cm={{1.00588,0.0,0.0,1.00588,(39.2294,305.788)}},draw=blue,line cap=rect,line join=bevel,line width=0.800pt]
  \end{scope}
  \begin{scope}[cm={{1.00588,0.0,0.0,1.00588,(39.2294,305.788)}},draw=blue,line cap=rect,line join=bevel,line width=0.800pt]
  \end{scope}
  \begin{scope}[cm={{1.00588,0.0,0.0,1.00588,(39.2294,305.788)}},draw=blue,line cap=rect,line join=bevel,line width=0.800pt]
  \end{scope}
  \begin{scope}[cm={{1.00588,0.0,0.0,1.00588,(39.2294,305.788)}},draw=blue,line cap=rect,line join=bevel,line width=0.800pt]
  \end{scope}
  \begin{scope}[cm={{1.00588,0.0,0.0,1.00588,(39.2294,305.788)}},draw=blue,line cap=rect,line join=bevel,line width=0.800pt]
  \end{scope}
  \begin{scope}[scale=1.006,draw=blue,line cap=rect,line join=bevel,line width=0.800pt]
  \end{scope}
  \begin{scope}[scale=1.006,draw=blue,line cap=rect,line join=bevel,line width=0.800pt]
  \end{scope}
  \begin{scope}[cm={{1.00588,0.0,0.0,1.00588,(40.2353,282.653)}},draw=blue,line cap=rect,line join=bevel,line width=0.800pt]
  \end{scope}
  \begin{scope}[cm={{1.00588,0.0,0.0,1.00588,(40.2353,282.653)}},draw=blue,line cap=rect,line join=bevel,line width=0.800pt]
  \end{scope}
  \begin{scope}[cm={{1.00588,0.0,0.0,1.00588,(40.2353,282.653)}},draw=blue,line cap=rect,line join=bevel,line width=0.800pt]
  \end{scope}
  \begin{scope}[cm={{1.00588,0.0,0.0,1.00588,(40.2353,282.653)}},draw=blue,line cap=rect,line join=bevel,line width=0.800pt]
  \end{scope}
  \begin{scope}[cm={{1.00588,0.0,0.0,1.00588,(40.2353,282.653)}},draw=blue,line cap=rect,line join=bevel,line width=0.800pt]
  \end{scope}
  \begin{scope}[cm={{1.00588,0.0,0.0,1.00588,(40.2353,282.653)}},draw=blue,line cap=rect,line join=bevel,line width=0.800pt]
  \end{scope}
  \begin{scope}[scale=1.006,draw=blue,line cap=rect,line join=bevel,line width=0.800pt]
  \end{scope}
  \begin{scope}[scale=1.006,draw=blue,line cap=rect,line join=bevel,line width=0.800pt]
  \end{scope}
  \begin{scope}[cm={{1.00588,0.0,0.0,1.00588,(40.2353,260.524)}},draw=blue,line cap=rect,line join=bevel,line width=0.800pt]
  \end{scope}
  \begin{scope}[cm={{1.00588,0.0,0.0,1.00588,(40.2353,260.524)}},draw=blue,line cap=rect,line join=bevel,line width=0.800pt]
  \end{scope}
  \begin{scope}[cm={{1.00588,0.0,0.0,1.00588,(40.2353,260.524)}},draw=blue,line cap=rect,line join=bevel,line width=0.800pt]
  \end{scope}
  \begin{scope}[cm={{1.00588,0.0,0.0,1.00588,(40.2353,260.524)}},draw=blue,line cap=rect,line join=bevel,line width=0.800pt]
  \end{scope}
  \begin{scope}[cm={{1.00588,0.0,0.0,1.00588,(40.2353,260.524)}},draw=blue,line cap=rect,line join=bevel,line width=0.800pt]
  \end{scope}
  \begin{scope}[cm={{1.00588,0.0,0.0,1.00588,(40.2353,260.524)}},draw=blue,line cap=rect,line join=bevel,line width=0.800pt]
  \end{scope}
  \begin{scope}[scale=1.006,draw=blue,line cap=rect,line join=bevel,line width=0.800pt]
  \end{scope}
  \begin{scope}[scale=1.006,draw=blue,line cap=rect,line join=bevel,line width=0.800pt]
  \end{scope}
  \begin{scope}[cm={{1.00588,0.0,0.0,1.00588,(39.2294,237.388)}},draw=blue,line cap=rect,line join=bevel,line width=0.800pt]
  \end{scope}
  \begin{scope}[cm={{1.00588,0.0,0.0,1.00588,(39.2294,237.388)}},draw=blue,line cap=rect,line join=bevel,line width=0.800pt]
  \end{scope}
  \begin{scope}[cm={{1.00588,0.0,0.0,1.00588,(39.2294,237.388)}},draw=blue,line cap=rect,line join=bevel,line width=0.800pt]
  \end{scope}
  \begin{scope}[cm={{1.00588,0.0,0.0,1.00588,(39.2294,237.388)}},draw=blue,line cap=rect,line join=bevel,line width=0.800pt]
  \end{scope}
  \begin{scope}[cm={{1.00588,0.0,0.0,1.00588,(39.2294,237.388)}},draw=blue,line cap=rect,line join=bevel,line width=0.800pt]
  \end{scope}
  \begin{scope}[cm={{1.00588,0.0,0.0,1.00588,(39.2294,237.388)}},draw=blue,line cap=rect,line join=bevel,line width=0.800pt]
  \end{scope}
  \begin{scope}[scale=1.006,draw=blue,line cap=rect,line join=bevel,line width=0.800pt]
  \end{scope}
  \begin{scope}[scale=1.006,draw=blue,line cap=rect,line join=bevel,line width=0.800pt]
  \end{scope}
  \begin{scope}[cm={{1.00588,0.0,0.0,1.00588,(53.3118,322.888)}},draw=blue,line cap=rect,line join=bevel,line width=0.800pt]
  \end{scope}
  \begin{scope}[cm={{1.00588,0.0,0.0,1.00588,(53.3118,322.888)}},draw=blue,line cap=rect,line join=bevel,line width=0.800pt]
  \end{scope}
  \begin{scope}[cm={{1.00588,0.0,0.0,1.00588,(53.3118,322.888)}},draw=blue,line cap=rect,line join=bevel,line width=0.800pt]
  \end{scope}
  \begin{scope}[cm={{1.00588,0.0,0.0,1.00588,(53.3118,322.888)}},draw=blue,line cap=rect,line join=bevel,line width=0.800pt]
  \end{scope}
  \begin{scope}[cm={{1.00588,0.0,0.0,1.00588,(53.3118,322.888)}},draw=blue,line cap=rect,line join=bevel,line width=0.800pt]
  \end{scope}
  \begin{scope}[cm={{1.00588,0.0,0.0,1.00588,(53.3118,322.888)}},draw=blue,line cap=rect,line join=bevel,line width=0.800pt]
  \end{scope}
  \begin{scope}[scale=1.006,draw=blue,line cap=rect,line join=bevel,line width=0.800pt]
  \end{scope}
  \begin{scope}[scale=1.006,draw=blue,line cap=rect,line join=bevel,line width=0.800pt]
  \end{scope}
  \begin{scope}[cm={{1.00588,0.0,0.0,1.00588,(89.5235,322.888)}},draw=blue,line cap=rect,line join=bevel,line width=0.800pt]
  \end{scope}
  \begin{scope}[cm={{1.00588,0.0,0.0,1.00588,(89.5235,322.888)}},draw=blue,line cap=rect,line join=bevel,line width=0.800pt]
  \end{scope}
  \begin{scope}[cm={{1.00588,0.0,0.0,1.00588,(89.5235,322.888)}},draw=blue,line cap=rect,line join=bevel,line width=0.800pt]
  \end{scope}
  \begin{scope}[cm={{1.00588,0.0,0.0,1.00588,(89.5235,322.888)}},draw=blue,line cap=rect,line join=bevel,line width=0.800pt]
  \end{scope}
  \begin{scope}[cm={{1.00588,0.0,0.0,1.00588,(89.5235,322.888)}},draw=blue,line cap=rect,line join=bevel,line width=0.800pt]
  \end{scope}
  \begin{scope}[cm={{1.00588,0.0,0.0,1.00588,(89.5235,322.888)}},draw=blue,line cap=rect,line join=bevel,line width=0.800pt]
  \end{scope}
  \begin{scope}[scale=1.006,draw=blue,line cap=rect,line join=bevel,line width=0.800pt]
  \end{scope}
  \begin{scope}[scale=1.006,draw=blue,line cap=rect,line join=bevel,line width=0.800pt]
  \end{scope}
  \begin{scope}[cm={{1.00588,0.0,0.0,1.00588,(125.232,322.888)}},draw=blue,line cap=rect,line join=bevel,line width=0.800pt]
  \end{scope}
  \begin{scope}[cm={{1.00588,0.0,0.0,1.00588,(125.232,322.888)}},draw=blue,line cap=rect,line join=bevel,line width=0.800pt]
  \end{scope}
  \begin{scope}[cm={{1.00588,0.0,0.0,1.00588,(125.232,322.888)}},draw=blue,line cap=rect,line join=bevel,line width=0.800pt]
  \end{scope}
  \begin{scope}[cm={{1.00588,0.0,0.0,1.00588,(125.232,322.888)}},draw=blue,line cap=rect,line join=bevel,line width=0.800pt]
  \end{scope}
  \begin{scope}[cm={{1.00588,0.0,0.0,1.00588,(125.232,322.888)}},draw=blue,line cap=rect,line join=bevel,line width=0.800pt]
  \end{scope}
  \begin{scope}[cm={{1.00588,0.0,0.0,1.00588,(125.232,322.888)}},draw=blue,line cap=rect,line join=bevel,line width=0.800pt]
  \end{scope}
  \begin{scope}[scale=1.006,draw=blue,line cap=rect,line join=bevel,line width=0.800pt]
  \end{scope}
  \begin{scope}[scale=1.006,draw=blue,line cap=rect,line join=bevel,line width=0.800pt]
  \end{scope}
  \begin{scope}[cm={{1.00588,0.0,0.0,1.00588,(160.941,322.888)}},draw=blue,line cap=rect,line join=bevel,line width=0.800pt]
  \end{scope}
  \begin{scope}[cm={{1.00588,0.0,0.0,1.00588,(160.941,322.888)}},draw=blue,line cap=rect,line join=bevel,line width=0.800pt]
  \end{scope}
  \begin{scope}[cm={{1.00588,0.0,0.0,1.00588,(160.941,322.888)}},draw=blue,line cap=rect,line join=bevel,line width=0.800pt]
  \end{scope}
  \begin{scope}[cm={{1.00588,0.0,0.0,1.00588,(160.941,322.888)}},draw=blue,line cap=rect,line join=bevel,line width=0.800pt]
  \end{scope}
  \begin{scope}[cm={{1.00588,0.0,0.0,1.00588,(160.941,322.888)}},draw=blue,line cap=rect,line join=bevel,line width=0.800pt]
  \end{scope}
  \begin{scope}[cm={{1.00588,0.0,0.0,1.00588,(160.941,322.888)}},draw=blue,line cap=rect,line join=bevel,line width=0.800pt]
  \end{scope}
  \begin{scope}[scale=1.006,draw=blue,line cap=rect,line join=bevel,line width=0.800pt]
  \end{scope}
  \begin{scope}[scale=1.006,draw=blue,line cap=rect,line join=bevel,line width=0.800pt]
  \end{scope}
  \begin{scope}[cm={{1.00588,0.0,0.0,1.00588,(196.147,322.888)}},draw=blue,line cap=rect,line join=bevel,line width=0.800pt]
  \end{scope}
  \begin{scope}[cm={{1.00588,0.0,0.0,1.00588,(196.147,322.888)}},draw=blue,line cap=rect,line join=bevel,line width=0.800pt]
  \end{scope}
  \begin{scope}[cm={{1.00588,0.0,0.0,1.00588,(196.147,322.888)}},draw=blue,line cap=rect,line join=bevel,line width=0.800pt]
  \end{scope}
  \begin{scope}[cm={{1.00588,0.0,0.0,1.00588,(196.147,322.888)}},draw=blue,line cap=rect,line join=bevel,line width=0.800pt]
  \end{scope}
  \begin{scope}[cm={{1.00588,0.0,0.0,1.00588,(196.147,322.888)}},draw=blue,line cap=rect,line join=bevel,line width=0.800pt]
  \end{scope}
  \begin{scope}[cm={{1.00588,0.0,0.0,1.00588,(196.147,322.888)}},draw=blue,line cap=rect,line join=bevel,line width=0.800pt]
  \end{scope}
  \begin{scope}[scale=1.006,draw=blue,line cap=rect,line join=bevel,line width=0.800pt]
  \end{scope}
  \begin{scope}[scale=1.006,draw=blue,line cap=rect,line join=bevel,line width=0.800pt]
  \end{scope}
  \begin{scope}[cm={{1.00588,0.0,0.0,1.00588,(228.838,322.888)}},draw=blue,line cap=rect,line join=bevel,line width=0.800pt]
  \end{scope}
  \begin{scope}[cm={{1.00588,0.0,0.0,1.00588,(228.838,322.888)}},draw=blue,line cap=rect,line join=bevel,line width=0.800pt]
  \end{scope}
  \begin{scope}[cm={{1.00588,0.0,0.0,1.00588,(228.838,322.888)}},draw=blue,line cap=rect,line join=bevel,line width=0.800pt]
  \end{scope}
  \begin{scope}[cm={{1.00588,0.0,0.0,1.00588,(228.838,322.888)}},draw=blue,line cap=rect,line join=bevel,line width=0.800pt]
  \end{scope}
  \begin{scope}[cm={{1.00588,0.0,0.0,1.00588,(228.838,322.888)}},draw=blue,line cap=rect,line join=bevel,line width=0.800pt]
  \end{scope}
  \begin{scope}[cm={{1.00588,0.0,0.0,1.00588,(228.838,322.888)}},draw=blue,line cap=rect,line join=bevel,line width=0.800pt]
  \end{scope}
  \begin{scope}[scale=1.006,draw=blue,line cap=rect,line join=bevel,line width=0.800pt]
  \end{scope}
  \begin{scope}[scale=1.006,draw=blue,line cap=rect,line join=bevel,line width=0.800pt]
  \end{scope}
  \begin{scope}[scale=1.006,draw=blue,line cap=rect,line join=bevel,line width=0.800pt]
  \end{scope}
  \begin{scope}[scale=1.006,draw=blue,line cap=rect,line join=bevel,line width=0.800pt]
  \end{scope}
  \begin{scope}[scale=1.006,draw=blue,line cap=rect,line join=bevel,line width=0.800pt]
  \end{scope}
  \begin{scope}[scale=1.006,draw=blue,line cap=rect,line join=bevel,line width=0.800pt]
  \end{scope}
  \begin{scope}[cm={{1.00588,0.0,0.0,1.00588,(232.359,238.394)}},draw=blue,line cap=rect,line join=bevel,line width=0.800pt]
  \end{scope}
  \begin{scope}[cm={{1.00588,0.0,0.0,1.00588,(232.359,238.394)}},draw=blue,line cap=rect,line join=bevel,line width=0.800pt]
  \end{scope}
  \begin{scope}[cm={{1.00588,0.0,0.0,1.00588,(232.359,238.394)}},draw=blue,line cap=rect,line join=bevel,line width=0.800pt]
  \end{scope}
  \begin{scope}[cm={{1.00588,0.0,0.0,1.00588,(232.359,238.394)}},draw=blue,line cap=rect,line join=bevel,line width=0.800pt]
  \end{scope}
  \begin{scope}[cm={{1.00588,0.0,0.0,1.00588,(232.359,238.394)}},draw=blue,line cap=rect,line join=bevel,line width=0.800pt]
  \end{scope}
  \begin{scope}[cm={{1.00588,0.0,0.0,1.00588,(232.359,238.394)}},draw=blue,line cap=rect,line join=bevel,line width=0.800pt]
  \end{scope}
  \begin{scope}[cm={{1.00588,0.0,0.0,1.00588,(130.262,337.976)}},draw=blue,line cap=rect,line join=bevel,line width=0.800pt]
  \end{scope}
  \begin{scope}[cm={{1.00588,0.0,0.0,1.00588,(130.262,337.976)}},draw=blue,line cap=rect,line join=bevel,line width=0.800pt]
  \end{scope}
  \begin{scope}[cm={{1.00588,0.0,0.0,1.00588,(130.262,337.976)}},draw=blue,line cap=rect,line join=bevel,line width=0.800pt]
  \end{scope}
  \begin{scope}[cm={{1.00588,0.0,0.0,1.00588,(130.262,337.976)}},draw=blue,line cap=rect,line join=bevel,line width=0.800pt]
  \end{scope}
  \begin{scope}[cm={{1.00588,0.0,0.0,1.00588,(130.262,337.976)}},draw=blue,line cap=rect,line join=bevel,line width=0.800pt]
  \end{scope}
  \begin{scope}[cm={{1.00588,0.0,0.0,1.00588,(130.262,337.976)}},draw=blue,line cap=rect,line join=bevel,line width=0.800pt]
  \end{scope}
  \begin{scope}[scale=1.006,draw=blue,line cap=rect,line join=bevel,line width=0.800pt]
  \end{scope}
  \begin{scope}[scale=1.006,draw=blue,line cap=rect,line join=bevel,line width=0.800pt]
  \end{scope}
  \begin{scope}[scale=1.006,draw=blue,line cap=rect,line join=bevel,line width=0.800pt]
  \end{scope}
  \begin{scope}[scale=1.006,draw=blue,line cap=rect,line join=bevel,line width=0.800pt]
  \end{scope}
  \begin{scope}[scale=1.006,draw=blue,line cap=rect,line join=bevel,line width=0.800pt]
  \end{scope}
  \begin{scope}[scale=1.006,draw=blue,line cap=rect,line join=bevel,line width=0.800pt]
  \end{scope}
  \begin{scope}[scale=1.006,draw=blue,line cap=rect,line join=bevel,line width=0.800pt]
  \end{scope}
  \begin{scope}[scale=1.006,draw=blue,line cap=rect,line join=bevel,line width=0.800pt]
  \end{scope}
  \begin{scope}[draw=blue,line cap=rect,line join=bevel,line width=0.800pt]
  \end{scope}
  \begin{scope}[cm={{1.26012,0.0,0.0,1.26012,(-482.16108,-168.79053)}},draw=ca0a0a4,dash pattern=on 1.03pt off 1.03pt,line cap=round,line join=round,line width=0.257pt,miter limit=4.00]
    \path[draw,dash pattern=on 1.03pt off 1.03pt,line width=0.257pt,miter limit=4.00] (56.5000,88.5000) -- (142.5000,88.5000);



  \end{scope}
  \begin{scope}[cm={{1.26012,0.0,0.0,1.26012,(-482.16108,-168.79053)}},draw=blue,line cap=round,line join=round,line width=0.480pt]
    \path[cm={{0.9189,0.0,0.0,1.0,(4.55807,0.0)}},draw] (56.5000,88.5000) -- (59.5000,88.5000);



    \path[cm={{0.9189,0.0,0.0,1.0,(11.5047,0.0)}},draw] (142.5000,88.5000) -- (139.5000,88.5000);



  \end{scope}
  \begin{scope}[cm={{1.26012,0.0,0.0,1.26012,(-482.16108,-168.79053)}},draw=ca0a0a4,dash pattern=on 1.03pt off 1.03pt,line cap=round,line join=round,line width=0.257pt,miter limit=4.00]
    \path[draw,dash pattern=on 1.03pt off 1.03pt,line width=0.257pt,miter limit=4.00] (56.5000,63.5000) -- (142.5000,63.5000);



  \end{scope}
  \begin{scope}[cm={{1.26012,0.0,0.0,1.26012,(-482.16108,-168.79053)}},draw=blue,line cap=round,line join=round,line width=0.480pt]
    \path[cm={{0.9189,0.0,0.0,1.0,(4.55807,-3e-05)}},draw] (56.5000,63.5000) -- (59.5000,63.5000);



    \path[cm={{0.9189,0.0,0.0,1.0,(11.5047,0.0)}},draw] (142.5000,63.5000) -- (139.5000,63.5000);



  \end{scope}
  \begin{scope}[cm={{1.26012,0.0,0.0,1.26012,(-482.16108,-168.79053)}},draw=ca0a0a4,dash pattern=on 1.03pt off 1.03pt,line cap=round,line join=round,line width=0.257pt,miter limit=4.00]
    \path[draw,dash pattern=on 1.03pt off 1.03pt,line width=0.257pt,miter limit=4.00] (56.5000,38.5000) -- (142.5000,38.5000);



  \end{scope}
  \begin{scope}[cm={{1.26012,0.0,0.0,1.26012,(-482.16108,-168.79053)}},draw=blue,line cap=round,line join=round,line width=0.480pt]
    \path[cm={{0.9189,0.0,0.0,1.0,(4.55807,-3e-05)}},draw] (56.5000,38.5000) -- (59.5000,38.5000);



    \path[cm={{0.9189,0.0,0.0,1.0,(11.5047,0.0)}},draw] (142.5000,38.5000) -- (139.5000,38.5000);



  \end{scope}
  \begin{scope}[cm={{1.26012,0.0,0.0,1.26012,(-482.16108,-168.79053)}},draw=ca0a0a4,dash pattern=on 0.40pt off 0.80pt,line cap=round,line join=round,line width=0.400pt]
    \path[draw] (56.5000,95.5000) -- (56.5000,13.5000);



  \end{scope}
  \begin{scope}[cm={{1.26012,0.0,0.0,1.26012,(-482.16108,-168.79053)}},draw=blue,line cap=round,line join=round,line width=0.480pt]
    \path[draw] (56.5000,95.5000) -- (56.5000,92.5000);



    \path[draw] (56.5000,13.5000) -- (56.5000,16.5000);



  \end{scope}
  \begin{scope}[cm={{1.26012,0.0,0.0,1.26012,(-482.16108,-168.79053)}},draw=blue,line cap=round,line join=round,line width=0.480pt]
    \path[cm={{1.0,0.0,0.0,0.9189,(0.0,7.47783)}},draw] (82.5000,95.5000) -- (82.5000,92.5000);



    \path[cm={{1.0,0.0,0.0,0.9189,(0.0,1.07058)}},draw] (82.5000,13.5000) -- (82.5000,16.5000);



  \end{scope}
  \begin{scope}[cm={{1.26012,0.0,0.0,1.26012,(-482.16108,-168.79053)}},draw=blue,line cap=round,line join=round,line width=0.480pt]
    \path[cm={{1.0,0.0,0.0,0.9189,(0.0,7.47783)}},draw] (108.5000,95.5000) -- (108.5000,92.5000);



    \path[cm={{1.0,0.0,0.0,0.9189,(0.0,1.07058)}},draw] (108.5000,13.5000) -- (108.5000,16.5000);



  \end{scope}
  \begin{scope}[cm={{1.26012,0.0,0.0,1.26012,(-482.16108,-168.79053)}},draw=ca0a0a4,dash pattern=on 1.03pt off 1.03pt,line cap=round,line join=round,line width=0.257pt,miter limit=4.00]
    \path[draw,dash pattern=on 1.03pt off 1.03pt,line width=0.257pt,miter limit=4.00] (134.5000,95.5000) -- (134.5000,13.5000);



  \end{scope}
  \begin{scope}[cm={{1.26012,0.0,0.0,1.26012,(-482.16108,-168.79053)}},draw=blue,line cap=round,line join=round,line width=0.480pt]
    \path[cm={{1.0,0.0,0.0,0.9189,(0.0,7.47783)}},draw] (134.5000,95.5000) -- (134.5000,92.5000);



    \path[cm={{1.0,0.0,0.0,0.9189,(0.0,1.07058)}},draw] (134.5000,13.5000) -- (134.5000,16.5000);



  \end{scope}
  \begin{scope}[cm={{1.00588,0.0,0.0,1.00588,(-400.2762,-132.7966)}},draw=blue,line cap=rect,line join=bevel,line width=0.800pt]
    \path[fill=blue] (0.0000,-1.4912) node[above right] (text290) {\scriptsize $\Upsilon(t)$};



  \end{scope}
  \begin{scope}[cm={{1.26012,0.0,0.0,1.26012,(-487.73989,-169.45128)}},draw=blue,line cap=round,line join=round,line width=0.480pt]
    \path[draw,even odd rule] (86.5000,24.5000) -- (112.5000,24.5000);



  \end{scope}
  \begin{scope}[cm={{1.26012,0.0,0.0,1.26012,(-482.16108,-168.79053)}},draw=blue,line cap=round,line join=round,line width=0.480pt]
    \path[draw] (56.1000,52.3000) -- (56.1000,52.3000) -- (56.5000,69.8000) -- (56.8000,61.5000) -- (57.1000,50.5000) -- (57.4000,49.2000) -- (57.7000,52.5000) -- (58.0000,57.0000) -- (58.4000,59.6000) -- (58.7000,60.0000) -- (59.0000,59.6000) -- (59.3000,59.2000) -- (59.6000,59.1000) -- (59.9000,59.0000) -- (60.2000,59.0000) -- (60.6000,59.1000) -- (60.9000,59.1000) -- (61.2000,59.1000) -- (61.5000,59.1000) -- (61.8000,59.0000) -- (62.1000,59.5000) -- (62.4000,62.4000) -- (62.8000,66.1000) -- (63.1000,69.9000) -- (63.4000,73.4000) -- (63.7000,76.7000) -- (64.0000,79.8000) -- (64.3000,82.6000) -- (64.6000,84.9000) -- (65.0000,86.4000) -- (65.3000,87.1000) -- (65.6000,86.7000) -- (65.9000,85.4000) -- (66.2000,83.2000) -- (66.5000,80.3000) -- (66.8000,77.1000) -- (67.2000,73.7000) -- (67.5000,70.1000) -- (67.8000,65.0000) -- (68.1000,60.3000) -- (68.4000,58.1000) -- (68.7000,58.0000) -- (69.0000,58.5000) -- (69.4000,58.9000) -- (69.7000,59.1000) -- (70.0000,59.1000) -- (70.3000,59.1000) -- (70.6000,59.1000) -- (70.9000,59.1000) -- (71.2000,59.1000) -- (71.5000,59.0000) -- (71.9000,61.2000) -- (72.2000,62.2000) -- (72.5000,59.0000) -- (72.8000,55.3000) -- (73.1000,52.3000) -- (73.4000,50.0000) -- (73.7000,48.2000) -- (74.1000,46.8000) -- (74.4000,45.7000) -- (74.7000,44.9000) -- (75.0000,44.4000) -- (75.3000,44.2000) -- (75.6000,44.3000) -- (75.9000,44.7000) -- (76.3000,45.4000) -- (76.6000,46.4000) -- (76.9000,47.7000) -- (77.2000,49.3000) -- (77.5000,51.3000) -- (77.8000,53.6000) -- (78.1000,56.7000) -- (78.5000,59.2000) -- (78.8000,59.9000) -- (79.1000,59.6000) -- (79.4000,59.3000) -- (79.7000,59.1000) -- (80.0000,59.0000) -- (80.3000,59.0000) -- (80.7000,59.1000) -- (81.0000,59.1000) -- (81.3000,59.1000) -- (81.6000,59.1000) -- (81.9000,59.0000) -- (82.2000,59.1000) -- (82.5000,61.3000) -- (82.9000,65.2000) -- (83.2000,69.1000) -- (83.5000,72.7000) -- (83.8000,76.1000) -- (84.1000,79.3000) -- (84.4000,82.1000) -- (84.7000,84.6000) -- (85.1000,86.3000) -- (85.4000,87.1000) -- (85.7000,86.8000) -- (86.0000,85.5000) -- (86.3000,83.3000) -- (86.6000,80.4000) -- (86.9000,77.1000) -- (87.3000,73.6000) -- (87.6000,69.9000) -- (87.9000,65.0000) -- (88.2000,60.5000) -- (88.5000,58.3000) -- (88.8000,58.1000) -- (89.1000,58.5000) -- (89.5000,58.9000) -- (89.8000,59.1000) -- (90.1000,59.1000) -- (90.4000,59.1000) -- (90.7000,59.1000) -- (91.0000,59.1000) -- (91.3000,59.1000) -- (91.7000,59.0000) -- (92.0000,61.4000) -- (92.3000,62.0000) -- (92.6000,58.6000) -- (92.9000,55.0000) -- (93.2000,52.0000) -- (93.5000,49.7000) -- (93.9000,47.9000) -- (94.2000,46.6000) -- (94.5000,45.5000) -- (94.8000,44.8000) -- (95.1000,44.3000) -- (95.4000,44.2000) -- (95.7000,44.4000) -- (96.1000,44.8000) -- (96.4000,45.6000) -- (96.7000,46.7000) -- (97.0000,48.2000) -- (97.3000,50.0000) -- (97.6000,52.1000) -- (97.9000,54.7000) -- (98.3000,57.7000) -- (98.6000,59.4000) -- (98.9000,59.7000) -- (99.2000,59.5000) -- (99.5000,59.2000) -- (99.8000,59.1000) -- (100.1000,59.0000) -- (100.5000,59.1000) -- (100.8000,59.1000) -- (101.1000,59.1000) -- (101.4000,59.1000) -- (101.7000,59.1000) -- (102.0000,59.0000) -- (102.3000,59.5000) -- (102.7000,62.6000) -- (103.0000,66.8000) -- (103.3000,70.7000) -- (103.6000,74.2000) -- (103.9000,77.5000) -- (104.2000,80.5000) -- (104.5000,83.2000) -- (104.9000,85.3000) -- (105.2000,86.7000) -- (105.5000,87.1000) -- (105.8000,86.3000) -- (106.1000,84.5000) -- (106.4000,81.8000) -- (106.7000,78.6000) -- (107.0000,75.1000) -- (107.4000,71.6000) -- (107.7000,67.2000) -- (108.0000,62.3000) -- (108.3000,59.0000) -- (108.6000,58.0000) -- (108.9000,58.3000) -- (109.3000,58.7000) -- (109.6000,59.0000) -- (109.9000,59.1000) -- (110.2000,59.1000) -- (110.5000,59.1000) -- (110.8000,59.1000) -- (111.1000,59.1000) -- (111.4000,59.0000) -- (111.8000,60.0000) -- (112.1000,62.4000) -- (112.4000,60.2000) -- (112.7000,56.4000) -- (113.0000,53.1000) -- (113.3000,50.5000) -- (113.6000,48.6000) -- (114.0000,47.1000) -- (114.3000,45.9000) -- (114.6000,45.0000) -- (114.9000,44.5000) -- (115.2000,44.2000) -- (115.5000,44.3000) -- (115.8000,44.6000) -- (116.2000,45.3000) -- (116.5000,46.3000) -- (116.8000,47.6000) -- (117.1000,49.4000) -- (117.4000,51.4000) -- (117.7000,53.9000) -- (118.0000,56.8000) -- (118.4000,59.0000) -- (118.7000,59.7000) -- (119.0000,59.6000) -- (119.3000,59.3000) -- (119.6000,59.1000) -- (119.9000,59.1000) -- (120.2000,59.1000) -- (120.6000,59.1000) -- (120.9000,59.1000) -- (121.2000,59.1000) -- (121.5000,59.1000) -- (121.8000,59.0000) -- (122.1000,59.2000) -- (122.4000,61.3000) -- (122.8000,65.4000) -- (123.1000,69.6000) -- (123.4000,73.3000) -- (123.7000,76.6000) -- (124.0000,79.6000) -- (124.3000,82.4000) -- (124.6000,84.7000) -- (125.0000,86.4000) -- (125.3000,87.1000) -- (125.6000,86.7000) -- (125.9000,85.1000) -- (126.2000,82.5000) -- (126.5000,79.4000) -- (126.8000,76.0000) -- (127.2000,72.5000) -- (127.5000,68.4000) -- (127.8000,63.4000) -- (128.1000,59.6000) -- (128.4000,58.1000) -- (128.7000,58.2000) -- (129.0000,58.6000) -- (129.4000,59.0000) -- (129.7000,59.1000) -- (130.0000,59.1000) -- (130.3000,59.1000) -- (130.6000,59.1000) -- (130.9000,59.1000) -- (131.2000,59.0000) -- (131.6000,59.4000) -- (131.9000,62.2000) -- (132.2000,61.0000) -- (132.5000,57.3000) -- (132.8000,53.8000) -- (133.1000,51.1000) -- (133.4000,49.0000) -- (133.8000,47.4000) -- (134.1000,46.1000) -- (134.4000,45.2000) -- (134.7000,44.6000) -- (135.0000,44.2000) -- (135.3000,44.2000) -- (135.6000,44.5000) -- (136.0000,45.1000) -- (136.3000,46.0000) -- (136.6000,47.3000) -- (136.9000,49.0000) -- (137.2000,51.0000) -- (137.5000,53.3000) -- (137.8000,56.2000) -- (138.2000,58.6000) -- (138.5000,59.6000) -- (138.8000,59.6000) -- (139.1000,59.3000) -- (139.4000,59.1000) -- (139.7000,59.1000) -- (140.0000,59.1000) -- (140.4000,59.1000) -- (140.7000,59.1000) -- (141.0000,59.1000) -- (141.3000,59.1000) -- (141.6000,59.1000) -- (141.9000,59.0000) -- (142.2000,60.6000) -- (142.6000,64.5000) -- (142.7000,66.5000);



  \end{scope}
  \begin{scope}[cm={{1.00588,0.0,0.0,1.00588,(-399.05137,-123.74949)}},draw=blue,line cap=rect,line join=bevel,line width=0.800pt]
    \path[fill=blue] (0.0000,0.0000) node[above right] (text326) {\scriptsize $y(t)$};



  \end{scope}
  \begin{scope}[cm={{1.26012,0.0,0.0,1.26012,(-205.87067,-45.42881)}},draw=cff0000,line cap=round,line join=round,line width=0.480pt]
    \path[draw,even odd rule] (-137.1848,-65.9213) -- (-111.1848,-65.9213);



  \end{scope}
  \begin{scope}[cm={{1.26012,0.0,0.0,1.26012,(-482.16108,-168.79053)}},draw=cff0000,line cap=round,line join=round,line width=0.480pt]
    \path[draw] (56.0000,44.3000) -- (56.0000,44.3000) -- (56.1000,44.4000) -- (56.2000,44.6000) -- (56.3000,44.8000) -- (56.3000,45.1000) -- (56.4000,45.4000) -- (56.5000,45.7000) -- (56.6000,46.1000) -- (56.7000,46.5000) -- (56.8000,47.0000) -- (56.9000,47.4000) -- (57.0000,47.9000) -- (57.0000,48.4000) -- (57.1000,48.9000) -- (57.2000,49.5000) -- (57.3000,50.0000) -- (57.4000,50.6000) -- (57.5000,51.2000) -- (57.6000,51.8000) -- (57.6000,52.3000) -- (57.7000,52.9000) -- (57.8000,53.5000) -- (57.9000,54.0000) -- (58.0000,54.6000) -- (58.1000,55.1000) -- (58.2000,55.6000) -- (58.3000,56.1000) -- (58.3000,56.6000) -- (58.4000,57.1000) -- (58.5000,57.5000) -- (58.6000,57.9000) -- (58.7000,58.2000) -- (58.8000,58.6000) -- (58.9000,58.9000) -- (59.0000,59.1000) -- (59.0000,59.4000) -- (59.1000,59.6000) -- (59.2000,59.8000) -- (59.3000,59.9000) -- (59.4000,60.0000) -- (59.5000,60.1000) -- (59.6000,60.1000) -- (59.6000,60.1000) -- (59.7000,60.1000) -- (59.8000,60.1000) -- (59.9000,60.0000) -- (60.0000,60.0000) -- (60.1000,59.9000) -- (60.2000,59.7000) -- (60.3000,59.6000) -- (60.3000,59.5000) -- (60.4000,59.3000) -- (60.5000,59.2000) -- (60.6000,59.0000) -- (60.7000,58.9000) -- (60.8000,58.8000) -- (60.9000,58.6000) -- (60.9000,58.5000) -- (61.0000,58.4000) -- (61.1000,58.3000) -- (61.2000,58.2000) -- (61.3000,58.2000) -- (61.4000,58.2000) -- (61.5000,58.2000) -- (61.6000,58.3000) -- (61.6000,58.3000) -- (61.7000,58.5000) -- (61.8000,58.6000) -- (61.9000,58.8000) -- (62.0000,59.1000) -- (62.1000,59.3000) -- (62.2000,59.7000) -- (62.2000,60.0000) -- (62.3000,60.4000) -- (62.4000,60.9000) -- (62.5000,61.4000) -- (62.6000,61.9000) -- (62.7000,62.5000) -- (62.8000,63.1000) -- (62.9000,63.8000) -- (62.9000,64.4000) -- (63.0000,65.2000) -- (63.1000,65.9000) -- (63.2000,66.7000) -- (63.3000,67.5000) -- (63.4000,68.3000) -- (63.5000,69.2000) -- (63.6000,70.0000) -- (63.6000,70.9000) -- (63.7000,71.8000) -- (63.8000,72.7000) -- (63.9000,73.6000) -- (64.0000,74.4000) -- (64.1000,75.3000) -- (64.2000,76.2000) -- (64.2000,77.0000) -- (64.3000,77.8000) -- (64.4000,78.6000) -- (64.5000,79.4000) -- (64.6000,80.1000) -- (64.7000,80.8000) -- (64.8000,81.5000) -- (64.9000,82.1000) -- (64.9000,82.7000) -- (65.0000,83.2000) -- (65.1000,83.7000) -- (65.2000,84.1000) -- (65.3000,84.4000) -- (65.4000,84.7000) -- (65.5000,85.0000) -- (65.5000,85.2000) -- (65.6000,85.3000) -- (65.7000,85.3000) -- (65.8000,85.3000) -- (65.9000,85.3000) -- (66.0000,85.1000) -- (66.1000,84.9000) -- (66.2000,84.7000) -- (66.2000,84.3000) -- (66.3000,84.0000) -- (66.4000,83.5000) -- (66.5000,83.1000) -- (66.6000,82.5000) -- (66.7000,81.9000) -- (66.8000,81.3000) -- (66.8000,80.6000) -- (66.9000,79.9000) -- (67.0000,79.2000) -- (67.1000,78.4000) -- (67.2000,77.6000) -- (67.3000,76.7000) -- (67.4000,75.9000) -- (67.5000,75.0000) -- (67.5000,74.1000) -- (67.6000,73.2000) -- (67.7000,72.3000) -- (67.8000,71.4000) -- (67.9000,70.5000) -- (68.0000,69.6000) -- (68.1000,68.7000) -- (68.2000,67.9000) -- (68.2000,67.0000) -- (68.3000,66.2000) -- (68.4000,65.4000) -- (68.5000,64.7000) -- (68.6000,63.9000) -- (68.7000,63.2000) -- (68.8000,62.5000) -- (68.8000,61.9000) -- (68.9000,61.3000) -- (69.0000,60.8000) -- (69.1000,60.2000) -- (69.2000,59.8000) -- (69.3000,59.4000) -- (69.4000,59.0000) -- (69.5000,58.6000) -- (69.5000,58.3000) -- (69.6000,58.1000) -- (69.7000,57.8000) -- (69.8000,57.7000) -- (69.9000,57.5000) -- (70.0000,57.4000) -- (70.1000,57.3000) -- (70.1000,57.3000) -- (70.2000,57.3000) -- (70.3000,57.3000) -- (70.4000,57.3000) -- (70.5000,57.4000) -- (70.6000,57.5000) -- (70.7000,57.6000) -- (70.8000,57.7000) -- (70.8000,57.8000) -- (70.9000,57.9000) -- (71.0000,58.0000) -- (71.1000,58.2000) -- (71.2000,58.3000) -- (71.3000,58.4000) -- (71.4000,58.5000) -- (71.4000,58.6000) -- (71.5000,58.7000) -- (71.6000,58.7000) -- (71.7000,58.8000) -- (71.8000,58.8000) -- (71.9000,58.8000) -- (72.0000,58.8000) -- (72.1000,58.7000) -- (72.1000,58.6000) -- (72.2000,58.5000) -- (72.3000,58.3000) -- (72.4000,58.2000) -- (72.5000,57.9000) -- (72.6000,57.7000) -- (72.7000,57.4000) -- (72.8000,57.1000) -- (72.8000,56.8000) -- (72.9000,56.4000) -- (73.0000,56.0000) -- (73.1000,55.6000) -- (73.2000,55.2000) -- (73.3000,54.7000) -- (73.4000,54.3000) -- (73.4000,53.8000) -- (73.5000,53.3000) -- (73.6000,52.7000) -- (73.7000,52.2000) -- (73.8000,51.7000) -- (73.9000,51.1000) -- (74.0000,50.6000) -- (74.1000,50.0000) -- (74.1000,49.5000) -- (74.2000,49.0000) -- (74.3000,48.5000) -- (74.4000,48.0000) -- (74.5000,47.5000) -- (74.6000,47.1000) -- (74.7000,46.6000) -- (74.7000,46.2000) -- (74.8000,45.8000) -- (74.9000,45.5000) -- (75.0000,45.2000) -- (75.1000,44.9000) -- (75.2000,44.7000) -- (75.3000,44.5000) -- (75.4000,44.3000) -- (75.4000,44.2000) -- (75.5000,44.1000) -- (75.6000,44.1000) -- (75.7000,44.1000) -- (75.8000,44.2000) -- (75.9000,44.3000) -- (76.0000,44.4000) -- (76.0000,44.6000) -- (76.1000,44.8000) -- (76.2000,45.1000) -- (76.3000,45.4000) -- (76.4000,45.8000) -- (76.5000,46.1000) -- (76.6000,46.5000) -- (76.7000,47.0000) -- (76.7000,47.4000) -- (76.8000,47.9000) -- (76.9000,48.5000) -- (77.0000,49.0000) -- (77.1000,49.5000) -- (77.2000,50.1000) -- (77.3000,50.7000) -- (77.3000,51.2000) -- (77.4000,51.8000) -- (77.5000,52.4000) -- (77.6000,53.0000) -- (77.7000,53.5000) -- (77.8000,54.1000) -- (77.9000,54.6000) -- (78.0000,55.2000) -- (78.0000,55.7000) -- (78.1000,56.2000) -- (78.2000,56.7000) -- (78.3000,57.1000) -- (78.4000,57.5000) -- (78.5000,57.9000) -- (78.6000,58.3000) -- (78.7000,58.6000) -- (78.7000,58.9000) -- (78.8000,59.2000) -- (78.9000,59.4000) -- (79.0000,59.6000) -- (79.1000,59.8000) -- (79.2000,59.9000) -- (79.3000,60.0000) -- (79.3000,60.1000) -- (79.4000,60.1000) -- (79.5000,60.2000) -- (79.6000,60.1000) -- (79.7000,60.1000) -- (79.8000,60.0000) -- (79.9000,60.0000) -- (80.0000,59.9000) -- (80.0000,59.7000) -- (80.1000,59.6000) -- (80.2000,59.5000) -- (80.3000,59.3000) -- (80.4000,59.2000) -- (80.5000,59.0000) -- (80.6000,58.9000) -- (80.6000,58.7000) -- (80.7000,58.6000) -- (80.8000,58.5000) -- (80.9000,58.4000) -- (81.0000,58.3000) -- (81.1000,58.2000) -- (81.2000,58.2000) -- (81.3000,58.2000) -- (81.3000,58.2000) -- (81.4000,58.2000) -- (81.5000,58.3000) -- (81.6000,58.5000) -- (81.7000,58.6000) -- (81.8000,58.8000) -- (81.9000,59.1000) -- (81.9000,59.3000) -- (82.0000,59.7000) -- (82.1000,60.0000) -- (82.2000,60.4000) -- (82.3000,60.9000) -- (82.4000,61.4000) -- (82.5000,61.9000) -- (82.6000,62.5000) -- (82.6000,63.1000) -- (82.7000,63.8000) -- (82.8000,64.5000) -- (82.9000,65.2000) -- (83.0000,66.0000) -- (83.1000,66.8000) -- (83.2000,67.6000) -- (83.3000,68.4000) -- (83.3000,69.3000) -- (83.4000,70.1000) -- (83.5000,71.0000) -- (83.6000,71.9000) -- (83.7000,72.8000) -- (83.8000,73.7000) -- (83.9000,74.5000) -- (83.9000,75.4000) -- (84.0000,76.3000) -- (84.1000,77.1000) -- (84.2000,77.9000) -- (84.3000,78.7000) -- (84.4000,79.5000) -- (84.5000,80.2000) -- (84.6000,80.9000) -- (84.6000,81.6000) -- (84.7000,82.2000) -- (84.8000,82.8000) -- (84.9000,83.3000) -- (85.0000,83.7000) -- (85.1000,84.2000) -- (85.2000,84.5000) -- (85.2000,84.8000) -- (85.3000,85.0000) -- (85.4000,85.2000) -- (85.5000,85.3000) -- (85.6000,85.4000) -- (85.7000,85.4000) -- (85.8000,85.3000) -- (85.9000,85.1000) -- (85.9000,84.9000) -- (86.0000,84.7000) -- (86.1000,84.4000) -- (86.2000,84.0000) -- (86.3000,83.5000) -- (86.4000,83.0000) -- (86.5000,82.5000) -- (86.5000,81.9000) -- (86.6000,81.3000) -- (86.7000,80.6000) -- (86.8000,79.9000) -- (86.9000,79.1000) -- (87.0000,78.3000) -- (87.1000,77.5000) -- (87.2000,76.7000) -- (87.2000,75.8000) -- (87.3000,74.9000) -- (87.4000,74.0000) -- (87.5000,73.1000) -- (87.6000,72.2000) -- (87.7000,71.3000) -- (87.8000,70.4000) -- (87.9000,69.5000) -- (87.9000,68.7000) -- (88.0000,67.8000) -- (88.1000,67.0000) -- (88.2000,66.1000) -- (88.3000,65.3000) -- (88.4000,64.6000) -- (88.5000,63.8000) -- (88.5000,63.1000) -- (88.6000,62.5000) -- (88.7000,61.8000) -- (88.8000,61.2000) -- (88.9000,60.7000) -- (89.0000,60.2000) -- (89.1000,59.7000) -- (89.2000,59.3000) -- (89.2000,58.9000) -- (89.3000,58.6000) -- (89.4000,58.3000) -- (89.5000,58.0000) -- (89.6000,57.8000) -- (89.7000,57.6000) -- (89.8000,57.5000) -- (89.8000,57.4000) -- (89.9000,57.3000) -- (90.0000,57.3000) -- (90.1000,57.3000) -- (90.2000,57.3000) -- (90.3000,57.3000) -- (90.4000,57.4000) -- (90.5000,57.5000) -- (90.5000,57.6000) -- (90.6000,57.7000) -- (90.7000,57.8000) -- (90.8000,57.9000) -- (90.9000,58.1000) -- (91.0000,58.2000) -- (91.1000,58.3000) -- (91.1000,58.4000) -- (91.2000,58.5000) -- (91.3000,58.6000) -- (91.4000,58.7000) -- (91.5000,58.8000) -- (91.6000,58.8000) -- (91.7000,58.8000) -- (91.8000,58.8000) -- (91.8000,58.8000) -- (91.9000,58.7000) -- (92.0000,58.6000) -- (92.1000,58.5000) -- (92.2000,58.3000) -- (92.3000,58.2000) -- (92.4000,57.9000) -- (92.5000,57.7000) -- (92.5000,57.4000) -- (92.6000,57.1000) -- (92.7000,56.8000) -- (92.8000,56.4000) -- (92.9000,56.0000) -- (93.0000,55.6000) -- (93.1000,55.2000) -- (93.1000,54.7000) -- (93.2000,54.2000) -- (93.3000,53.7000) -- (93.4000,53.2000) -- (93.5000,52.7000) -- (93.6000,52.2000) -- (93.7000,51.6000) -- (93.8000,51.1000) -- (93.8000,50.5000) -- (93.9000,50.0000) -- (94.0000,49.5000) -- (94.1000,48.9000) -- (94.2000,48.4000) -- (94.3000,47.9000) -- (94.4000,47.5000) -- (94.4000,47.0000) -- (94.5000,46.6000) -- (94.6000,46.2000) -- (94.7000,45.8000) -- (94.8000,45.4000) -- (94.9000,45.1000) -- (95.0000,44.9000) -- (95.1000,44.6000) -- (95.1000,44.4000) -- (95.2000,44.3000) -- (95.3000,44.2000) -- (95.4000,44.1000) -- (95.5000,44.1000) -- (95.6000,44.1000) -- (95.7000,44.2000) -- (95.7000,44.3000) -- (95.8000,44.4000) -- (95.9000,44.6000) -- (96.0000,44.8000) -- (96.1000,45.1000) -- (96.2000,45.4000) -- (96.3000,45.8000) -- (96.4000,46.2000) -- (96.4000,46.6000) -- (96.5000,47.0000) -- (96.6000,47.5000) -- (96.7000,48.0000) -- (96.8000,48.5000) -- (96.9000,49.0000) -- (97.0000,49.6000) -- (97.1000,50.1000) -- (97.1000,50.7000) -- (97.2000,51.3000) -- (97.3000,51.9000) -- (97.4000,52.4000) -- (97.5000,53.0000) -- (97.6000,53.6000) -- (97.7000,54.2000) -- (97.7000,54.7000) -- (97.8000,55.2000) -- (97.9000,55.8000) -- (98.0000,56.2000) -- (98.1000,56.7000) -- (98.2000,57.2000) -- (98.3000,57.6000) -- (98.4000,58.0000) -- (98.4000,58.3000) -- (98.5000,58.7000) -- (98.6000,59.0000) -- (98.7000,59.2000) -- (98.8000,59.5000) -- (98.9000,59.7000) -- (99.0000,59.8000) -- (99.0000,60.0000) -- (99.1000,60.1000) -- (99.2000,60.1000) -- (99.3000,60.2000) -- (99.4000,60.2000) -- (99.5000,60.2000) -- (99.6000,60.1000) -- (99.7000,60.0000) -- (99.7000,60.0000) -- (99.8000,59.9000) -- (99.9000,59.7000) -- (100.0000,59.6000) -- (100.1000,59.5000) -- (100.2000,59.3000) -- (100.3000,59.2000) -- (100.3000,59.0000) -- (100.4000,58.9000) -- (100.5000,58.7000) -- (100.6000,58.6000) -- (100.7000,58.5000) -- (100.8000,58.4000) -- (100.9000,58.3000) -- (101.0000,58.2000) -- (101.0000,58.2000) -- (101.1000,58.2000) -- (101.2000,58.2000) -- (101.3000,58.2000) -- (101.4000,58.3000) -- (101.5000,58.4000) -- (101.6000,58.6000) -- (101.7000,58.8000) -- (101.7000,59.1000) -- (101.8000,59.3000) -- (101.9000,59.7000) -- (102.0000,60.0000) -- (102.1000,60.5000) -- (102.2000,60.9000) -- (102.3000,61.4000) -- (102.3000,62.0000) -- (102.4000,62.6000) -- (102.5000,63.2000) -- (102.6000,63.8000) -- (102.7000,64.5000) -- (102.8000,65.3000) -- (102.9000,66.0000) -- (103.0000,66.8000) -- (103.0000,67.6000) -- (103.1000,68.5000) -- (103.2000,69.3000) -- (103.3000,70.2000) -- (103.4000,71.1000) -- (103.5000,72.0000) -- (103.6000,72.8000) -- (103.6000,73.7000) -- (103.7000,74.6000) -- (103.8000,75.5000) -- (103.9000,76.4000) -- (104.0000,77.2000) -- (104.1000,78.0000) -- (104.2000,78.8000) -- (104.3000,79.6000) -- (104.3000,80.3000) -- (104.4000,81.0000) -- (104.5000,81.7000) -- (104.6000,82.3000) -- (104.7000,82.8000) -- (104.8000,83.4000) -- (104.9000,83.8000) -- (104.9000,84.2000) -- (105.0000,84.6000) -- (105.1000,84.9000) -- (105.2000,85.1000) -- (105.3000,85.3000) -- (105.4000,85.4000) -- (105.5000,85.4000) -- (105.6000,85.4000) -- (105.6000,85.3000) -- (105.7000,85.2000) -- (105.8000,85.0000) -- (105.9000,84.7000) -- (106.0000,84.4000) -- (106.1000,84.0000) -- (106.2000,83.5000) -- (106.3000,83.0000) -- (106.3000,82.5000) -- (106.4000,81.9000) -- (106.5000,81.2000) -- (106.6000,80.5000) -- (106.7000,79.8000) -- (106.8000,79.1000) -- (106.9000,78.3000) -- (106.9000,77.4000) -- (107.0000,76.6000) -- (107.1000,75.7000) -- (107.2000,74.8000) -- (107.3000,73.9000) -- (107.4000,73.0000) -- (107.5000,72.1000) -- (107.6000,71.2000) -- (107.6000,70.3000) -- (107.7000,69.4000) -- (107.8000,68.6000) -- (107.9000,67.7000) -- (108.0000,66.9000) -- (108.1000,66.0000) -- (108.2000,65.2000) -- (108.2000,64.5000) -- (108.3000,63.7000) -- (108.4000,63.0000) -- (108.5000,62.4000) -- (108.6000,61.8000) -- (108.7000,61.2000) -- (108.8000,60.6000) -- (108.9000,60.1000) -- (108.9000,59.6000) -- (109.0000,59.2000) -- (109.1000,58.9000) -- (109.2000,58.5000) -- (109.3000,58.2000) -- (109.4000,58.0000) -- (109.5000,57.8000) -- (109.5000,57.6000) -- (109.6000,57.5000) -- (109.7000,57.4000) -- (109.8000,57.3000) -- (109.9000,57.3000) -- (110.0000,57.3000) -- (110.1000,57.3000) -- (110.2000,57.3000) -- (110.2000,57.4000) -- (110.3000,57.5000) -- (110.4000,57.6000) -- (110.5000,57.7000) -- (110.6000,57.8000) -- (110.7000,57.9000) -- (110.8000,58.1000) -- (110.9000,58.2000) -- (110.9000,58.3000) -- (111.0000,58.4000) -- (111.1000,58.5000) -- (111.2000,58.6000) -- (111.3000,58.7000) -- (111.4000,58.8000) -- (111.5000,58.8000) -- (111.5000,58.8000) -- (111.6000,58.8000) -- (111.7000,58.8000) -- (111.8000,58.7000) -- (111.9000,58.6000) -- (112.0000,58.5000) -- (112.1000,58.3000) -- (112.2000,58.2000) -- (112.2000,57.9000) -- (112.3000,57.7000) -- (112.4000,57.4000) -- (112.5000,57.1000) -- (112.6000,56.8000) -- (112.7000,56.4000) -- (112.8000,56.0000) -- (112.8000,55.6000) -- (112.9000,55.1000) -- (113.0000,54.7000) -- (113.1000,54.2000) -- (113.2000,53.7000) -- (113.3000,53.2000) -- (113.4000,52.7000) -- (113.5000,52.1000) -- (113.5000,51.6000) -- (113.6000,51.0000) -- (113.7000,50.5000) -- (113.8000,49.9000) -- (113.9000,49.4000) -- (114.0000,48.9000) -- (114.1000,48.4000) -- (114.1000,47.9000) -- (114.2000,47.4000) -- (114.3000,46.9000) -- (114.4000,46.5000) -- (114.5000,46.1000) -- (114.6000,45.7000) -- (114.7000,45.4000) -- (114.8000,45.1000) -- (114.8000,44.8000) -- (114.9000,44.6000) -- (115.0000,44.4000) -- (115.1000,44.3000) -- (115.2000,44.1000) -- (115.3000,44.1000) -- (115.4000,44.1000) -- (115.5000,44.1000) -- (115.5000,44.1000) -- (115.6000,44.3000) -- (115.7000,44.4000) -- (115.8000,44.6000) -- (115.9000,44.8000) -- (116.0000,45.1000) -- (116.1000,45.4000) -- (116.1000,45.8000) -- (116.2000,46.2000) -- (116.3000,46.6000) -- (116.4000,47.0000) -- (116.5000,47.5000) -- (116.6000,48.0000) -- (116.7000,48.5000) -- (116.8000,49.1000) -- (116.8000,49.6000) -- (116.9000,50.2000) -- (117.0000,50.8000) -- (117.1000,51.3000) -- (117.2000,51.9000) -- (117.3000,52.5000) -- (117.4000,53.1000) -- (117.4000,53.6000) -- (117.5000,54.2000) -- (117.6000,54.8000) -- (117.7000,55.3000) -- (117.8000,55.8000) -- (117.9000,56.3000) -- (118.0000,56.8000) -- (118.1000,57.2000) -- (118.1000,57.6000) -- (118.2000,58.0000) -- (118.3000,58.4000) -- (118.4000,58.7000) -- (118.5000,59.0000) -- (118.6000,59.3000) -- (118.7000,59.5000) -- (118.7000,59.7000) -- (118.8000,59.9000) -- (118.9000,60.0000) -- (119.0000,60.1000) -- (119.1000,60.1000) -- (119.2000,60.2000) -- (119.3000,60.2000) -- (119.4000,60.2000) -- (119.4000,60.1000) -- (119.5000,60.1000) -- (119.6000,60.0000) -- (119.7000,59.9000) -- (119.8000,59.7000) -- (119.9000,59.6000) -- (120.0000,59.5000) -- (120.0000,59.3000) -- (120.1000,59.1000) -- (120.2000,59.0000) -- (120.3000,58.8000) -- (120.4000,58.7000) -- (120.5000,58.6000) -- (120.6000,58.4000) -- (120.7000,58.3000) -- (120.7000,58.2000) -- (120.8000,58.2000) -- (120.9000,58.2000) -- (121.0000,58.1000) -- (121.1000,58.2000) -- (121.2000,58.2000) -- (121.3000,58.3000) -- (121.4000,58.4000) -- (121.4000,58.6000) -- (121.5000,58.8000) -- (121.6000,59.1000) -- (121.7000,59.3000) -- (121.8000,59.7000) -- (121.9000,60.1000) -- (122.0000,60.5000) -- (122.0000,60.9000) -- (122.1000,61.5000) -- (122.2000,62.0000) -- (122.3000,62.6000) -- (122.4000,63.2000) -- (122.5000,63.9000) -- (122.6000,64.6000) -- (122.7000,65.3000) -- (122.7000,66.1000) -- (122.8000,66.9000) -- (122.9000,67.7000) -- (123.0000,68.5000) -- (123.1000,69.4000) -- (123.2000,70.3000) -- (123.3000,71.1000) -- (123.3000,72.0000) -- (123.4000,72.9000) -- (123.5000,73.8000) -- (123.6000,74.7000) -- (123.7000,75.6000) -- (123.8000,76.4000) -- (123.9000,77.3000) -- (124.0000,78.1000) -- (124.0000,78.9000) -- (124.1000,79.7000) -- (124.2000,80.4000) -- (124.3000,81.1000) -- (124.4000,81.7000) -- (124.5000,82.4000) -- (124.6000,82.9000) -- (124.6000,83.4000) -- (124.7000,83.9000) -- (124.8000,84.3000) -- (124.9000,84.6000) -- (125.0000,84.9000) -- (125.1000,85.2000) -- (125.2000,85.3000) -- (125.3000,85.4000) -- (125.3000,85.5000) -- (125.4000,85.4000) -- (125.5000,85.3000) -- (125.6000,85.2000) -- (125.7000,85.0000) -- (125.8000,84.7000) -- (125.9000,84.4000) -- (125.9000,84.0000) -- (126.0000,83.5000) -- (126.1000,83.0000) -- (126.2000,82.4000) -- (126.3000,81.8000) -- (126.4000,81.2000) -- (126.5000,80.5000) -- (126.6000,79.8000) -- (126.6000,79.0000) -- (126.7000,78.2000) -- (126.8000,77.4000) -- (126.9000,76.5000) -- (127.0000,75.7000) -- (127.1000,74.8000) -- (127.2000,73.9000) -- (127.3000,73.0000) -- (127.3000,72.1000) -- (127.4000,71.2000) -- (127.5000,70.3000) -- (127.6000,69.4000) -- (127.7000,68.5000) -- (127.8000,67.6000) -- (127.9000,66.8000) -- (127.9000,66.0000) -- (128.0000,65.2000) -- (128.1000,64.4000) -- (128.2000,63.7000) -- (128.3000,63.0000) -- (128.4000,62.3000) -- (128.5000,61.7000) -- (128.6000,61.1000) -- (128.6000,60.5000) -- (128.7000,60.0000) -- (128.8000,59.6000) -- (128.9000,59.2000) -- (129.0000,58.8000) -- (129.1000,58.5000) -- (129.2000,58.2000) -- (129.2000,57.9000) -- (129.3000,57.7000) -- (129.4000,57.6000) -- (129.5000,57.4000) -- (129.6000,57.3000) -- (129.7000,57.3000) -- (129.8000,57.2000) -- (129.9000,57.3000) -- (129.9000,57.3000) -- (130.0000,57.3000) -- (130.1000,57.4000) -- (130.2000,57.5000) -- (130.3000,57.6000) -- (130.4000,57.7000) -- (130.5000,57.8000) -- (130.5000,58.0000) -- (130.6000,58.1000) -- (130.7000,58.2000) -- (130.8000,58.3000) -- (130.9000,58.5000) -- (131.0000,58.6000) -- (131.1000,58.7000) -- (131.2000,58.7000) -- (131.2000,58.8000) -- (131.3000,58.8000) -- (131.4000,58.8000) -- (131.5000,58.8000) -- (131.6000,58.8000) -- (131.7000,58.7000) -- (131.8000,58.6000) -- (131.8000,58.5000) -- (131.9000,58.3000) -- (132.0000,58.2000) -- (132.1000,57.9000) -- (132.2000,57.7000) -- (132.3000,57.4000) -- (132.4000,57.1000) -- (132.5000,56.8000) -- (132.5000,56.4000) -- (132.6000,56.0000) -- (132.7000,55.6000) -- (132.8000,55.1000) -- (132.9000,54.6000) -- (133.0000,54.2000) -- (133.1000,53.7000) -- (133.2000,53.1000) -- (133.2000,52.6000) -- (133.3000,52.1000) -- (133.4000,51.5000) -- (133.5000,51.0000) -- (133.6000,50.4000) -- (133.7000,49.9000) -- (133.8000,49.4000) -- (133.8000,48.8000) -- (133.9000,48.3000) -- (134.0000,47.8000) -- (134.1000,47.3000) -- (134.2000,46.9000) -- (134.3000,46.5000) -- (134.4000,46.1000) -- (134.5000,45.7000) -- (134.5000,45.3000) -- (134.6000,45.0000) -- (134.7000,44.8000) -- (134.8000,44.6000) -- (134.9000,44.4000) -- (135.0000,44.2000) -- (135.1000,44.1000) -- (135.1000,44.1000) -- (135.2000,44.0000) -- (135.3000,44.1000) -- (135.4000,44.1000) -- (135.5000,44.2000) -- (135.6000,44.4000) -- (135.7000,44.6000) -- (135.8000,44.8000) -- (135.8000,45.1000) -- (135.9000,45.4000) -- (136.0000,45.8000) -- (136.1000,46.2000) -- (136.2000,46.6000) -- (136.3000,47.1000) -- (136.4000,47.5000) -- (136.5000,48.0000) -- (136.5000,48.6000) -- (136.6000,49.1000) -- (136.7000,49.7000) -- (136.8000,50.2000) -- (136.9000,50.8000) -- (137.0000,51.4000) -- (137.1000,52.0000) -- (137.1000,52.6000) -- (137.2000,53.1000) -- (137.3000,53.7000) -- (137.4000,54.3000) -- (137.5000,54.8000) -- (137.6000,55.4000) -- (137.7000,55.9000) -- (137.8000,56.4000) -- (137.8000,56.8000) -- (137.9000,57.3000) -- (138.0000,57.7000) -- (138.1000,58.1000) -- (138.2000,58.4000) -- (138.3000,58.8000) -- (138.4000,59.1000) -- (138.4000,59.3000) -- (138.5000,59.5000) -- (138.6000,59.7000) -- (138.7000,59.9000) -- (138.8000,60.0000) -- (138.9000,60.1000) -- (139.0000,60.2000) -- (139.1000,60.2000) -- (139.1000,60.2000) -- (139.2000,60.2000) -- (139.3000,60.1000) -- (139.4000,60.1000) -- (139.5000,60.0000) -- (139.6000,59.9000) -- (139.7000,59.7000) -- (139.7000,59.6000) -- (139.8000,59.4000) -- (139.9000,59.3000) -- (140.0000,59.1000) -- (140.1000,59.0000) -- (140.2000,58.8000) -- (140.3000,58.7000) -- (140.4000,58.5000) -- (140.4000,58.4000) -- (140.5000,58.3000) -- (140.6000,58.2000) -- (140.7000,58.2000) -- (140.8000,58.1000) -- (140.9000,58.1000) -- (141.0000,58.1000) -- (141.1000,58.2000) -- (141.1000,58.3000) -- (141.2000,58.4000) -- (141.3000,58.6000) -- (141.4000,58.8000) -- (141.5000,59.1000) -- (141.6000,59.4000) -- (141.7000,59.7000) -- (141.7000,60.1000) -- (141.8000,60.5000) -- (141.9000,61.0000) -- (142.0000,61.5000) -- (142.1000,62.0000) -- (142.2000,62.6000) -- (142.3000,63.3000) -- (142.4000,63.9000) -- (142.4000,64.6000) -- (142.5000,65.4000) -- (142.6000,66.1000) -- (142.7000,67.0000);



  \end{scope}
  \begin{scope}[cm={{1.26012,0.0,0.0,1.26012,(-482.16108,-168.79053)}},draw=blue,line cap=round,line join=round,line width=0.480pt]
    \path[draw] (56.5000,13.5000) -- (56.5000,95.5000) -- (142.5000,95.5000) -- (142.5000,13.5000) -- (56.5000,13.5000);



  \end{scope}
  \begin{scope}[cm={{1.15801,0.0,0.0,1.15801,(-615.96866,-167.12877)}},draw=ca0a0a4,dash pattern=on 0.40pt off 0.80pt,line cap=round,line join=round,line width=0.400pt]
    \path[draw] (44.5000,102.5000) -- (142.5000,102.5000);



  \end{scope}
  \begin{scope}[cm={{1.15801,0.0,0.0,1.15801,(-615.96866,-167.12877)}},draw=blue,line cap=round,line join=round,line width=0.480pt]
    \path[draw] (44.5000,102.5000) -- (48.5000,102.5000);



    \path[draw] (142.5000,102.5000) -- (139.5000,102.5000);



  \end{scope}
  \begin{scope}[cm={{1.00588,0.0,0.0,1.00588,(-585.77904,-46.49663)}},draw=blue,line cap=rect,line join=bevel,line width=0.800pt]
    \path[fill=blue] (0.0000,0.0000) node[above right] (text34-9) {-100};



  \end{scope}
  \begin{scope}[cm={{1.15801,0.0,0.0,1.15801,(-615.96866,-167.12877)}},draw=ca0a0a4,dash pattern=on 1.55pt off 1.55pt,line cap=round,line join=round,line width=0.259pt,miter limit=4.00]
    \path[draw,dash pattern=on 1.55pt off 1.55pt,line width=0.259pt,miter limit=4.00] (44.5000,79.5000) -- (142.5000,79.5000);



  \end{scope}
  \begin{scope}[cm={{1.15801,0.0,0.0,1.15801,(-615.96866,-167.12877)}},draw=blue,line cap=round,line join=round,line width=0.480pt]
    \path[draw] (44.5000,79.5000) -- (48.5000,79.5000);



    \path[draw] (142.5000,79.5000) -- (139.5000,79.5000);



  \end{scope}
  \begin{scope}[cm={{1.00588,0.0,0.0,1.00588,(-575.04441,-71.62653)}},draw=blue,line cap=rect,line join=bevel,line width=0.800pt]
    \path[fill=blue] (0.0000,0.0000) node[above right] (text64-6) {0};



  \end{scope}
  \begin{scope}[cm={{1.15801,0.0,0.0,1.15801,(-615.96866,-167.12877)}},draw=ca0a0a4,dash pattern=on 1.55pt off 1.55pt,line cap=round,line join=round,line width=0.259pt,miter limit=4.00]
    \path[draw,dash pattern=on 1.55pt off 1.55pt,line width=0.259pt,miter limit=4.00] (44.5000,57.5000) -- (142.5000,57.5000);



  \end{scope}
  \begin{scope}[cm={{1.15801,0.0,0.0,1.15801,(-615.96866,-167.12877)}},draw=blue,line cap=round,line join=round,line width=0.480pt]
    \path[draw] (44.5000,57.5000) -- (48.5000,57.5000);



    \path[draw] (142.5000,57.5000) -- (139.5000,57.5000);



  \end{scope}
  \begin{scope}[cm={{1.00588,0.0,0.0,1.00588,(-583.09145,-98.25592)}},draw=blue,line cap=rect,line join=bevel,line width=0.800pt]
    \path[fill=blue] (0.0000,0.0000) node[above right] (text94-7) {100};



  \end{scope}
  \begin{scope}[cm={{1.15801,0.0,0.0,1.15801,(-615.96866,-167.12877)}},draw=ca0a0a4,dash pattern=on 1.55pt off 1.55pt,line cap=round,line join=round,line width=0.259pt,miter limit=4.00]
    \path[draw,dash pattern=on 1.55pt off 1.55pt,line width=0.259pt,miter limit=4.00] (44.5000,35.5000) -- (142.5000,35.5000);



  \end{scope}
  \begin{scope}[cm={{1.15801,0.0,0.0,1.15801,(-615.96866,-167.12877)}},draw=blue,line cap=round,line join=round,line width=0.480pt]
    \path[draw] (44.5000,35.5000) -- (48.5000,35.5000);



    \path[draw] (142.5000,35.5000) -- (139.5000,35.5000);



  \end{scope}
  \begin{scope}[cm={{1.00588,0.0,0.0,1.00588,(-583.09145,-123.38532)}},draw=blue,line cap=rect,line join=bevel,line width=0.800pt]
    \path[fill=blue] (0.0000,0.0000) node[above right] (text124-9) {200};



  \end{scope}
  \begin{scope}[cm={{1.15801,0.0,0.0,1.15801,(-615.96866,-167.12877)}},draw=ca0a0a4,dash pattern=on 0.40pt off 0.80pt,line cap=round,line join=round,line width=0.400pt]
    \path[draw] (44.5000,13.5000) -- (142.5000,13.5000);



  \end{scope}
  \begin{scope}[cm={{1.15801,0.0,0.0,1.15801,(-615.96866,-167.12877)}},draw=blue,line cap=round,line join=round,line width=0.480pt]
    \path[draw] (44.5000,13.5000) -- (48.5000,13.5000);



    \path[draw] (142.5000,13.5000) -- (139.5000,13.5000);



  \end{scope}
  \begin{scope}[cm={{1.00588,0.0,0.0,1.00588,(-583.09145,-148.51472)}},draw=blue,line cap=rect,line join=bevel,line width=0.800pt]
    \path[fill=blue] (0.0000,0.0000) node[above right] (text154) {300};



  \end{scope}
  \begin{scope}[cm={{1.15801,0.0,0.0,1.15801,(-615.96866,-167.12877)}},draw=ca0a0a4,dash pattern=on 0.40pt off 0.80pt,line cap=round,line join=round,line width=0.400pt]
    \path[draw] (44.5000,102.5000) -- (44.5000,13.5000);



  \end{scope}
  \begin{scope}[cm={{1.15801,0.0,0.0,1.15801,(-615.96866,-167.12877)}},draw=blue,line cap=round,line join=round,line width=0.480pt]
    \path[draw] (44.5000,102.5000) -- (44.5000,99.5000);



    \path[draw] (44.5000,13.5000) -- (44.5000,16.5000);



  \end{scope}
  \begin{scope}[cm={{1.15801,0.0,0.0,1.15801,(-615.96866,-167.12877)}},draw=ca0a0a4,dash pattern=on 1.55pt off 1.55pt,line cap=round,line join=round,line width=0.259pt,miter limit=4.00]
    \path[draw,dash pattern=on 1.55pt off 1.55pt,line width=0.259pt,miter limit=4.00] (69.5000,102.5000) -- (69.5000,27.5000);



    \path[draw,dash pattern=on 1.55pt off 1.55pt,line width=0.259pt,miter limit=4.00] (69.5000,19.5000) -- (69.5000,13.5000);



  \end{scope}
  \begin{scope}[cm={{1.15801,0.0,0.0,1.15801,(-615.96866,-167.12877)}},draw=blue,line cap=round,line join=round,line width=0.480pt]
    \path[draw] (69.5000,102.5000) -- (69.5000,99.5000);



    \path[draw] (69.5000,13.5000) -- (69.5000,16.5000);



  \end{scope}
  \begin{scope}[cm={{1.15801,0.0,0.0,1.15801,(-615.96866,-167.12877)}},draw=ca0a0a4,dash pattern=on 1.55pt off 1.55pt,line cap=round,line join=round,line width=0.259pt,miter limit=4.00]
    \path[draw,dash pattern=on 1.55pt off 1.55pt,line width=0.259pt,miter limit=4.00] (93.5000,102.5000) -- (93.5000,27.5000);



    \path[draw,dash pattern=on 1.55pt off 1.55pt,line width=0.259pt,miter limit=4.00] (93.5000,19.5000) -- (93.5000,13.5000);



  \end{scope}
  \begin{scope}[cm={{1.15801,0.0,0.0,1.15801,(-615.96866,-167.12877)}},draw=blue,line cap=round,line join=round,line width=0.480pt]
    \path[draw] (93.5000,102.5000) -- (93.5000,99.5000);



    \path[draw] (93.5000,13.5000) -- (93.5000,16.5000);



  \end{scope}
  \begin{scope}[cm={{1.15801,0.0,0.0,1.15801,(-615.96866,-167.12877)}},draw=ca0a0a4,dash pattern=on 1.55pt off 1.55pt,line cap=round,line join=round,line width=0.259pt,miter limit=4.00]
    \path[draw,dash pattern=on 1.55pt off 1.55pt,line width=0.259pt,miter limit=4.00] (118.5000,102.5000) -- (118.5000,13.5000);



  \end{scope}
  \begin{scope}[cm={{1.15801,0.0,0.0,1.15801,(-615.96866,-167.12877)}},draw=blue,line cap=round,line join=round,line width=0.480pt]
    \path[draw] (118.5000,102.5000) -- (118.5000,99.5000);



    \path[draw] (118.5000,13.5000) -- (118.5000,16.5000);



  \end{scope}
  \begin{scope}[cm={{1.15801,0.0,0.0,1.15801,(-615.96866,-167.12877)}},draw=ca0a0a4,dash pattern=on 0.40pt off 0.80pt,line cap=round,line join=round,line width=0.400pt]
    \path[draw] (142.5000,102.5000) -- (142.5000,13.5000);



  \end{scope}
  \begin{scope}[cm={{1.15801,0.0,0.0,1.15801,(-615.96866,-167.12877)}},draw=blue,line cap=round,line join=round,line width=0.480pt]
    \path[draw] (142.5000,102.5000) -- (142.5000,99.5000);



    \path[draw] (142.5000,13.5000) -- (142.5000,16.5000);



  \end{scope}
  \begin{scope}[cm={{1.15801,0.0,0.0,1.15801,(-615.96866,-167.12877)}},draw=blue,line cap=round,line join=round,line width=0.480pt]
    \path[draw] (44.5000,13.5000) -- (44.5000,102.5000) -- (142.5000,102.5000) -- (142.5000,13.5000) -- (44.5000,13.5000);



  \end{scope}
  \begin{scope}[cm={{0.84029,0.0,0.0,0.84029,(-557.89843,-136.56808)}},draw=blue,line cap=rect,line join=bevel,line width=0.800pt]
    \path[fill=blue] (0.0000,0.0000) node[above right] (text360) {\scriptsize $\mathbf{p}(t)$};



  \end{scope}
  \begin{scope}[cm={{1.15801,0.0,0.0,1.15801,(-623.4871,-168.38184)}},draw=blue,line cap=round,line join=round,line width=0.480pt]
    \path[draw,even odd rule] (74.5000,23.5000) -- (101.5000,23.5000);



  \end{scope}
  \begin{scope}[cm={{1.15801,0.0,0.0,1.15801,(-615.96866,-167.12877)}},draw=blue,line cap=round,line join=round,line width=0.480pt]
    \path[draw] (57.5000,32.8000) -- (57.5000,32.8000) -- (58.3000,35.4000) -- (57.8000,37.2000) -- (56.3000,38.9000) -- (54.9000,41.1000) -- (53.9000,43.5000) -- (53.5000,46.1000) -- (53.4000,48.6000) -- (53.4000,51.1000) -- (53.4000,53.6000) -- (53.4000,56.1000) -- (53.4000,58.6000) -- (53.4000,61.1000) -- (53.4000,63.6000) -- (53.4000,66.1000) -- (53.4000,68.7000) -- (53.4000,71.2000) -- (53.4000,73.7000) -- (53.4000,76.2000) -- (53.4000,78.7000) -- (53.9000,81.3000) -- (54.7000,83.7000) -- (56.0000,86.0000) -- (57.6000,88.1000) -- (59.5000,89.9000) -- (61.7000,91.4000) -- (64.1000,92.6000) -- (66.6000,93.4000) -- (69.2000,93.9000) -- (71.8000,94.1000) -- (74.4000,93.9000) -- (77.0000,93.4000) -- (79.5000,92.6000) -- (81.8000,91.4000) -- (84.0000,90.0000) -- (85.9000,88.2000) -- (87.5000,86.1000) -- (88.6000,83.8000) -- (89.1000,81.3000) -- (89.2000,78.8000) -- (89.3000,76.2000) -- (89.3000,73.7000) -- (89.3000,71.1000) -- (89.3000,68.6000) -- (89.3000,66.1000) -- (89.3000,63.6000) -- (89.3000,61.0000) -- (89.3000,58.5000) -- (89.3000,56.0000) -- (89.3000,53.5000) -- (89.6000,51.0000) -- (90.0000,48.5000) -- (89.9000,46.0000) -- (89.4000,43.6000) -- (88.5000,41.3000) -- (87.2000,39.1000) -- (85.5000,37.2000) -- (83.5000,35.5000) -- (81.3000,34.1000) -- (78.9000,33.1000) -- (76.3000,32.5000) -- (73.6000,32.2000) -- (71.0000,32.3000) -- (68.3000,32.8000) -- (65.8000,33.7000) -- (63.5000,34.9000) -- (61.4000,36.5000) -- (59.6000,38.3000) -- (58.2000,40.4000) -- (57.1000,42.7000) -- (56.5000,45.2000) -- (56.3000,47.7000) -- (56.3000,50.1000) -- (56.3000,52.6000) -- (56.3000,55.1000) -- (56.3000,57.6000) -- (56.3000,60.2000) -- (56.3000,62.7000) -- (56.3000,65.2000) -- (56.3000,67.7000) -- (56.3000,70.2000) -- (56.3000,72.7000) -- (56.3000,75.3000) -- (56.3000,77.8000) -- (56.6000,80.3000) -- (57.3000,82.8000) -- (58.5000,85.1000) -- (60.1000,87.2000) -- (61.9000,89.1000) -- (64.1000,90.7000) -- (66.4000,91.9000) -- (68.9000,92.8000) -- (71.5000,93.4000) -- (74.1000,93.6000) -- (76.7000,93.5000) -- (79.3000,93.0000) -- (81.8000,92.2000) -- (84.1000,91.0000) -- (86.3000,89.5000) -- (88.2000,87.7000) -- (89.8000,85.7000) -- (90.9000,83.4000) -- (91.4000,80.9000) -- (91.6000,78.3000) -- (91.6000,75.8000) -- (91.6000,73.2000) -- (91.6000,70.7000) -- (91.6000,68.2000) -- (91.6000,65.6000) -- (91.6000,63.1000) -- (91.6000,60.6000) -- (91.6000,58.1000) -- (91.6000,55.6000) -- (91.6000,53.0000) -- (91.9000,50.5000) -- (92.3000,48.0000) -- (92.2000,45.6000) -- (91.7000,43.2000) -- (90.7000,40.9000) -- (89.4000,38.7000) -- (87.7000,36.8000) -- (85.6000,35.2000) -- (83.3000,33.9000) -- (80.9000,32.9000) -- (78.3000,32.4000) -- (75.6000,32.2000) -- (72.9000,32.4000) -- (70.3000,33.0000) -- (67.9000,33.9000) -- (65.6000,35.2000) -- (63.6000,36.9000) -- (61.9000,38.8000) -- (60.6000,41.0000) -- (59.7000,43.4000) -- (59.3000,45.9000) -- (59.2000,48.3000) -- (59.1000,50.8000) -- (59.1000,53.3000) -- (59.1000,55.8000) -- (59.1000,58.3000) -- (59.1000,60.9000) -- (59.1000,63.4000) -- (59.1000,65.9000) -- (59.1000,68.4000) -- (59.1000,70.9000) -- (59.1000,73.4000) -- (59.1000,76.0000) -- (59.2000,78.5000) -- (59.6000,81.0000) -- (60.6000,83.5000) -- (61.9000,85.7000) -- (63.6000,87.7000) -- (65.6000,89.4000) -- (67.8000,90.9000) -- (70.2000,92.0000) -- (72.7000,92.8000) -- (75.3000,93.2000) -- (78.0000,93.2000) -- (80.6000,92.9000) -- (83.1000,92.3000) -- (85.5000,91.3000) -- (87.8000,89.9000) -- (89.8000,88.3000) -- (91.6000,86.3000) -- (92.9000,84.1000) -- (93.6000,81.7000) -- (93.9000,79.2000) -- (94.0000,76.6000) -- (94.0000,74.0000) -- (94.1000,71.5000) -- (94.1000,69.0000) -- (94.1000,66.4000) -- (94.1000,63.9000) -- (94.1000,61.4000) -- (94.1000,58.9000) -- (94.1000,56.4000) -- (94.1000,53.9000) -- (94.2000,51.3000) -- (94.6000,48.8000) -- (94.7000,46.4000) -- (94.4000,43.9000) -- (93.6000,41.6000) -- (92.4000,39.3000) -- (90.8000,37.3000) -- (88.9000,35.6000) -- (86.7000,34.2000) -- (84.3000,33.1000) -- (81.7000,32.4000) -- (79.1000,32.0000) -- (76.4000,32.1000) -- (73.8000,32.5000) -- (71.3000,33.4000) -- (68.9000,34.6000) -- (66.8000,36.1000) -- (65.0000,37.9000) -- (63.6000,40.1000) -- (62.6000,42.4000) -- (62.0000,44.8000) -- (61.8000,47.3000) -- (61.8000,49.8000) -- (61.7000,52.3000) -- (61.7000,54.8000) -- (61.7000,57.3000) -- (61.7000,59.8000) -- (61.7000,62.4000) -- (61.7000,64.9000) -- (61.7000,67.4000) -- (61.7000,69.9000) -- (61.7000,72.4000) -- (61.7000,75.0000) -- (61.7000,77.5000) -- (62.0000,80.0000) -- (62.8000,82.5000) -- (64.0000,84.8000) -- (65.6000,86.9000) -- (67.5000,88.7000) -- (69.7000,90.2000) -- (72.1000,91.4000) -- (74.6000,92.3000) -- (77.1000,92.8000) -- (79.8000,93.0000) -- (82.4000,92.8000) -- (84.9000,92.2000) -- (87.4000,91.3000) -- (89.7000,90.0000) -- (91.8000,88.5000) -- (93.6000,86.6000) -- (95.1000,84.4000) -- (95.9000,82.1000) -- (96.3000,79.6000) -- (96.4000,77.0000) -- (96.5000,74.4000) -- (96.5000,71.9000) -- (96.5000,69.3000) -- (96.5000,66.8000) -- (96.5000,64.3000) -- (96.5000,61.8000) -- (96.5000,59.3000) -- (96.5000,56.7000) -- (96.5000,54.2000) -- (96.5000,51.7000) -- (97.0000,49.2000) -- (97.2000,46.7000) -- (96.9000,44.3000) -- (96.3000,41.9000) -- (95.1000,39.6000) -- (93.6000,37.6000) -- (91.8000,35.8000) -- (89.7000,34.3000) -- (87.3000,33.1000) -- (84.8000,32.3000) -- (82.1000,31.9000) -- (79.5000,31.8000) -- (76.8000,32.2000) -- (74.3000,32.9000) -- (71.9000,34.1000) -- (69.7000,35.5000) -- (67.9000,37.3000) -- (66.4000,39.4000) -- (65.2000,41.7000) -- (64.6000,44.1000) -- (64.3000,46.6000) -- (64.3000,49.1000) -- (64.2000,51.6000) -- (64.2000,54.1000) -- (64.2000,56.6000) -- (64.2000,59.1000) -- (64.2000,61.6000) -- (64.2000,64.1000) -- (64.2000,66.6000) -- (64.2000,69.2000) -- (64.2000,71.7000) -- (64.2000,74.2000) -- (64.2000,76.7000) -- (64.4000,79.3000) -- (65.1000,81.8000) -- (66.2000,84.1000) -- (67.8000,86.2000) -- (69.6000,88.1000) -- (71.7000,89.7000) -- (74.0000,91.0000) -- (76.5000,91.9000) -- (79.1000,92.5000) -- (81.7000,92.7000) -- (84.3000,92.6000) -- (86.9000,92.1000) -- (89.4000,91.3000) -- (91.7000,90.1000) -- (93.9000,88.6000) -- (95.7000,86.7000) -- (97.3000,84.6000) -- (98.2000,82.3000) -- (98.7000,79.8000) -- (98.9000,77.3000) -- (98.9000,74.7000) -- (98.9000,72.1000) -- (99.0000,69.6000) -- (99.0000,67.1000) -- (99.0000,64.6000) -- (99.0000,62.0000) -- (99.0000,59.5000) -- (99.0000,57.0000) -- (99.0000,54.5000) -- (99.0000,52.0000) -- (99.3000,49.4000) -- (99.6000,47.0000) -- (99.5000,44.5000) -- (98.9000,42.1000) -- (97.8000,39.8000) -- (96.4000,37.7000) -- (94.6000,35.9000) -- (92.5000,34.3000) -- (90.2000,33.1000) -- (87.7000,32.2000) -- (85.1000,31.7000) -- (82.4000,31.6000) -- (79.7000,31.9000) -- (77.2000,32.6000) -- (74.8000,33.6000) -- (72.6000,35.1000) -- (70.6000,36.8000) -- (69.1000,38.8000) -- (67.9000,41.1000) -- (67.2000,43.5000) -- (66.9000,46.0000) -- (66.8000,48.5000) -- (66.8000,51.0000) -- (66.7000,53.5000) -- (66.7000,56.0000) -- (66.7000,58.5000) -- (66.7000,61.0000) -- (66.7000,63.5000) -- (66.7000,66.0000) -- (66.7000,68.6000) -- (66.7000,71.1000) -- (66.7000,73.6000) -- (66.7000,76.1000) -- (66.9000,78.7000) -- (67.5000,81.2000) -- (68.6000,83.5000) -- (70.0000,85.7000) -- (71.8000,87.6000) -- (73.9000,89.2000) -- (76.2000,90.6000) -- (78.7000,91.6000) -- (81.2000,92.2000) -- (83.8000,92.5000) -- (86.5000,92.4000) -- (89.0000,92.0000) -- (91.5000,91.2000) -- (93.9000,90.0000) -- (96.1000,88.5000) -- (98.0000,86.7000) -- (99.5000,84.7000) -- (100.6000,82.4000) -- (101.1000,79.9000) -- (101.3000,77.3000) -- (101.4000,74.8000) -- (101.4000,72.2000) -- (101.4000,69.7000) -- (101.4000,67.2000) -- (101.4000,64.6000) -- (101.4000,62.1000) -- (101.4000,59.6000) -- (101.4000,57.1000) -- (101.4000,54.6000) -- (101.4000,52.0000) -- (101.7000,49.5000) -- (102.1000,47.0000) -- (102.0000,44.6000) -- (101.4000,42.2000) -- (100.4000,39.9000) -- (99.0000,37.7000) -- (97.3000,35.9000) -- (95.2000,34.2000) -- (92.9000,33.0000) -- (90.4000,32.1000) -- (87.8000,31.5000) -- (85.2000,31.4000) -- (82.5000,31.6000) -- (79.9000,32.3000) -- (77.5000,33.3000) -- (75.3000,34.6000) -- (73.3000,36.3000) -- (71.7000,38.3000) -- (70.5000,40.6000) -- (69.7000,43.0000) -- (69.3000,45.5000) -- (69.2000,47.9000) -- (69.2000,50.4000) -- (69.2000,52.9000) -- (69.2000,55.4000) -- (69.2000,57.9000) -- (69.2000,60.5000) -- (69.2000,63.0000) -- (69.2000,65.5000) -- (69.2000,68.0000) -- (69.2000,70.5000) -- (69.2000,73.1000) -- (69.2000,75.6000) -- (69.3000,78.1000) -- (69.8000,80.6000) -- (70.8000,83.0000) -- (72.3000,85.2000) -- (74.0000,87.2000) -- (76.1000,88.8000) -- (78.3000,90.2000) -- (80.8000,91.2000) -- (83.3000,91.9000) -- (85.9000,92.3000) -- (88.6000,92.2000) -- (91.2000,91.8000) -- (93.7000,91.1000) -- (96.0000,90.0000) -- (98.2000,88.5000) -- (100.2000,86.8000) -- (101.8000,84.7000) -- (102.9000,82.5000) -- (103.5000,80.0000) -- (103.7000,77.4000) -- (103.8000,74.9000) -- (103.8000,72.3000) -- (103.8000,69.8000) -- (103.8000,67.3000) -- (103.8000,64.7000) -- (103.9000,62.2000) -- (103.8000,59.7000) -- (103.8000,57.2000) -- (103.8000,54.7000) -- (103.8000,52.1000) -- (104.1000,49.6000) -- (104.5000,47.1000) -- (104.4000,44.7000) -- (103.9000,42.3000) -- (103.0000,39.9000) -- (101.7000,37.8000) -- (100.0000,35.9000) -- (98.0000,34.2000) -- (95.7000,32.9000) -- (93.2000,31.9000) -- (90.6000,31.4000) -- (88.0000,31.2000) -- (85.3000,31.3000) -- (82.7000,31.9000) -- (80.2000,32.9000) -- (78.0000,34.2000) -- (76.0000,35.9000) -- (74.3000,37.8000) -- (73.0000,40.0000) -- (72.2000,42.4000) -- (71.8000,44.9000) -- (71.7000,47.4000) -- (71.6000,49.9000) -- (71.6000,52.4000) -- (71.6000,54.9000) -- (71.6000,57.4000) -- (71.6000,59.9000) -- (71.6000,62.4000) -- (71.6000,65.0000) -- (71.6000,67.5000) -- (71.6000,70.0000) -- (71.6000,72.5000) -- (71.6000,75.0000) -- (71.7000,77.6000) -- (72.2000,80.1000) -- (73.1000,82.5000) -- (74.5000,84.7000) -- (76.2000,86.7000) -- (78.2000,88.4000) -- (80.5000,89.8000) -- (82.9000,90.9000) -- (85.4000,91.6000) -- (88.0000,92.0000) -- (90.7000,92.0000) -- (93.3000,91.7000) -- (95.8000,91.0000) -- (98.2000,89.9000) -- (100.4000,88.5000) -- (102.4000,86.8000) -- (104.1000,84.8000) -- (105.3000,82.6000) -- (105.9000,80.1000) -- (106.2000,77.6000) -- (106.2000,75.0000) -- (106.3000,72.4000) -- (106.3000,69.9000) -- (106.3000,67.4000) -- (106.3000,64.9000) -- (106.3000,62.3000) -- (106.3000,59.8000) -- (106.3000,57.3000) -- (106.3000,54.8000) -- (106.3000,52.3000) -- (106.5000,49.7000) -- (106.9000,47.2000) -- (106.9000,44.8000) -- (106.5000,42.4000) -- (105.6000,40.0000) -- (104.3000,37.8000) -- (102.7000,35.9000) -- (100.7000,34.2000) -- (98.4000,32.8000) -- (96.0000,31.8000) -- (93.4000,31.2000) -- (90.8000,30.9000) -- (88.1000,31.1000) -- (85.5000,31.6000) -- (83.0000,32.5000) -- (80.7000,33.8000) -- (78.7000,35.4000) -- (77.0000,37.4000) -- (75.6000,39.5000) -- (74.8000,41.9000) -- (74.3000,44.4000) -- (74.2000,46.9000) -- (74.1000,49.3000) -- (74.1000,51.8000) -- (74.1000,54.3000) -- (74.1000,56.9000) -- (74.1000,59.4000) -- (74.1000,61.9000) -- (74.1000,64.4000) -- (74.1000,66.9000) -- (74.1000,69.4000) -- (74.1000,72.0000) -- (74.1000,74.5000) -- (74.1000,77.0000) -- (74.5000,79.6000) -- (75.4000,82.0000) -- (76.8000,84.2000) -- (78.5000,86.2000) -- (80.4000,88.0000) -- (82.6000,89.4000) -- (85.0000,90.6000) -- (87.6000,91.3000) -- (90.2000,91.8000) -- (92.8000,91.8000) -- (95.4000,91.5000) -- (97.9000,90.9000) -- (100.3000,89.8000) -- (102.6000,88.5000) -- (104.6000,86.8000) -- (106.3000,84.8000) -- (107.6000,82.6000) -- (108.3000,80.2000) -- (108.6000,77.7000) -- (108.7000,75.1000) -- (108.7000,72.5000) -- (108.7000,70.0000) -- (108.7000,67.5000) -- (108.7000,64.9000) -- (108.7000,62.4000) -- (108.7000,59.9000) -- (108.7000,57.4000) -- (108.7000,54.9000) -- (108.7000,52.3000) -- (108.9000,49.8000) -- (109.3000,47.3000) -- (109.4000,44.9000) -- (109.0000,42.4000) -- (108.2000,40.1000) -- (106.9000,37.9000) -- (105.3000,35.9000) -- (103.4000,34.2000) -- (101.2000,32.8000) -- (98.8000,31.7000) -- (96.2000,31.0000) -- (93.5000,30.7000) -- (90.9000,30.8000) -- (88.2000,31.3000) -- (85.7000,32.2000) -- (83.4000,33.4000) -- (81.4000,35.0000) -- (79.6000,36.9000) -- (78.2000,39.0000) -- (77.3000,41.4000) -- (76.8000,43.8000) -- (76.6000,46.3000) -- (76.5000,48.8000) -- (76.5000,51.3000) -- (76.5000,53.8000) -- (76.5000,56.3000) -- (76.5000,58.8000) -- (76.5000,61.4000) -- (76.5000,63.9000) -- (76.5000,66.4000) -- (76.5000,68.9000) -- (76.5000,71.4000) -- (76.5000,73.9000) -- (76.5000,76.5000) -- (76.9000,79.0000) -- (77.7000,81.5000) -- (79.0000,83.7000) -- (80.7000,85.8000) -- (82.6000,87.6000) -- (84.8000,89.0000) -- (87.2000,90.2000) -- (89.7000,91.0000) -- (92.3000,91.5000) -- (94.9000,91.6000) -- (97.5000,91.4000) -- (100.1000,90.7000) -- (102.5000,89.8000) -- (104.8000,88.4000) -- (106.8000,86.8000) -- (108.6000,84.9000) -- (109.9000,82.7000) -- (110.7000,80.3000) -- (111.0000,77.8000) -- (111.1000,75.2000) -- (111.2000,72.6000) -- (111.2000,70.1000) -- (111.2000,67.6000) -- (111.2000,65.0000) -- (111.2000,62.5000) -- (111.2000,60.0000) -- (111.2000,57.5000) -- (111.2000,55.0000) -- (111.2000,52.4000) -- (111.3000,49.9000) -- (111.7000,47.4000) -- (111.9000,45.0000) -- (111.5000,42.5000) -- (110.8000,40.2000) -- (109.6000,37.9000) -- (108.0000,35.9000) -- (106.1000,34.1000) -- (103.9000,32.7000) -- (101.5000,31.6000) -- (99.0000,30.9000) -- (96.3000,30.5000) -- (93.6000,30.6000) -- (91.0000,31.0000) -- (88.5000,31.8000) -- (86.2000,33.0000) -- (84.1000,34.6000) -- (82.3000,36.4000) -- (80.8000,38.5000) -- (79.8000,40.8000) -- (79.3000,43.3000) -- (79.1000,45.8000) -- (79.0000,48.3000) -- (79.0000,50.8000) -- (79.0000,53.3000) -- (79.0000,55.8000) -- (79.0000,58.3000) -- (79.0000,60.8000) -- (79.0000,63.3000) -- (79.0000,65.9000) -- (79.0000,68.4000) -- (79.0000,70.9000) -- (79.0000,73.4000) -- (79.0000,75.9000) -- (79.3000,78.5000) -- (80.1000,81.0000) -- (81.3000,83.2000) -- (82.9000,85.3000) -- (84.8000,87.1000) -- (87.0000,88.6000) -- (89.3000,89.9000) -- (91.8000,90.7000) -- (94.4000,91.3000) -- (97.0000,91.4000) -- (99.6000,91.2000) -- (102.2000,90.6000) -- (104.6000,89.7000) -- (106.9000,88.4000) -- (109.0000,86.8000) -- (110.8000,84.9000) -- (112.2000,82.8000) -- (113.1000,80.4000) -- (113.4000,77.9000) -- (113.6000,75.3000) -- (113.6000,72.7000) -- (113.6000,70.2000) -- (113.6000,67.6000) -- (113.6000,65.1000) -- (113.6000,62.6000) -- (113.6000,60.1000) -- (113.6000,57.6000) -- (113.6000,55.1000) -- (113.6000,52.5000) -- (113.7000,50.0000) -- (114.1000,47.5000) -- (114.3000,45.0000) -- (114.0000,42.6000) -- (113.3000,40.2000) -- (112.2000,38.0000) -- (110.7000,35.9000) -- (108.8000,34.1000) -- (106.6000,32.6000) -- (104.3000,31.5000) -- (101.7000,30.7000) -- (99.1000,30.3000) -- (96.4000,30.3000) -- (93.8000,30.7000) -- (91.2000,31.5000) -- (88.9000,32.6000) -- (86.7000,34.1000) -- (84.9000,36.0000) -- (83.4000,38.0000) -- (82.4000,40.3000) -- (81.8000,42.8000) -- (81.6000,45.3000) -- (81.5000,47.8000) -- (81.4000,50.3000) -- (81.4000,52.8000) -- (81.4000,55.3000) -- (81.4000,57.8000) -- (81.4000,60.3000) -- (81.4000,62.8000) -- (81.4000,65.4000) -- (81.4000,67.9000) -- (81.4000,70.4000) -- (81.4000,72.9000) -- (81.4000,75.4000) -- (81.7000,78.0000) -- (82.4000,80.5000) -- (83.6000,82.8000) -- (85.2000,84.9000) -- (87.0000,86.7000) -- (89.2000,88.3000) -- (91.5000,89.5000) -- (94.0000,90.4000) -- (96.6000,91.0000) -- (99.2000,91.2000) -- (101.8000,91.0000) -- (104.4000,90.5000) -- (106.8000,89.6000) -- (109.2000,88.3000) -- (111.3000,86.8000) -- (113.1000,84.9000) -- (114.6000,82.8000) -- (115.5000,80.4000) -- (115.9000,77.9000) -- (116.0000,75.3000) -- (116.1000,72.8000) -- (116.1000,70.2000) -- (116.1000,67.7000) -- (116.1000,65.2000) -- (116.1000,62.7000) -- (116.1000,60.1000) -- (116.1000,57.6000) -- (116.1000,55.1000) -- (116.1000,52.6000) -- (116.1000,50.1000) -- (116.5000,47.5000) -- (116.8000,45.1000) -- (116.5000,42.6000) -- (115.9000,40.2000) -- (114.8000,38.0000) -- (113.3000,35.9000) -- (111.5000,34.1000) -- (109.3000,32.6000) -- (107.0000,31.4000) -- (104.5000,30.6000) -- (101.8000,30.1000) -- (99.2000,30.1000) -- (96.5000,30.4000) -- (94.0000,31.2000) -- (91.6000,32.3000) -- (89.4000,33.7000) -- (87.5000,35.5000) -- (86.0000,37.6000) -- (84.9000,39.9000) -- (84.3000,42.3000) -- (84.0000,44.8000) -- (83.9000,47.3000) -- (83.9000,49.8000) -- (83.9000,52.3000) -- (83.9000,54.8000) -- (83.9000,57.3000) -- (83.9000,59.8000) -- (83.9000,62.3000) -- (83.9000,64.8000) -- (83.9000,67.4000) -- (83.9000,69.9000) -- (83.9000,72.4000) -- (83.9000,74.9000) -- (84.0000,77.5000) -- (84.7000,80.0000) -- (85.9000,82.3000) -- (87.4000,84.4000) -- (89.2000,86.3000) -- (91.3000,87.9000) -- (93.7000,89.2000) -- (96.1000,90.1000) -- (98.7000,90.7000) -- (101.3000,91.0000) -- (103.9000,90.8000) -- (106.5000,90.3000) -- (109.0000,89.5000) -- (111.3000,88.3000) -- (113.5000,86.8000) -- (115.3000,84.9000) -- (116.9000,82.8000) -- (117.8000,80.5000) -- (118.3000,78.0000) -- (118.4000,75.4000) -- (118.5000,72.9000) -- (118.5000,70.3000) -- (118.5000,67.8000) -- (118.5000,65.3000) -- (118.5000,62.7000) -- (118.5000,60.2000) -- (118.5000,57.7000) -- (118.5000,55.2000) -- (118.5000,52.7000) -- (118.5000,50.2000) -- (118.9000,47.6000) -- (119.2000,45.1000) -- (119.0000,42.7000) -- (118.4000,40.3000) -- (117.4000,38.0000) -- (115.9000,35.9000) -- (114.1000,34.1000) -- (112.1000,32.5000) -- (109.7000,31.3000) -- (107.2000,30.4000) -- (104.6000,29.9000) -- (101.9000,29.9000) -- (99.3000,30.2000) -- (96.7000,30.8000) -- (94.3000,31.9000) -- (92.1000,33.3000) -- (90.2000,35.1000) -- (88.6000,37.1000) -- (87.5000,39.4000) -- (86.8000,41.8000) -- (86.5000,44.3000) -- (86.4000,46.8000) -- (86.3000,49.3000) -- (86.3000,51.8000) -- (86.3000,54.3000) -- (86.3000,56.8000) -- (86.3000,59.3000) -- (86.3000,61.8000) -- (86.3000,64.3000) -- (86.3000,66.8000) -- (86.3000,69.4000) -- (86.3000,71.9000) -- (86.3000,74.4000) -- (86.5000,76.9000) -- (87.1000,79.5000) -- (88.2000,81.8000) -- (89.7000,84.0000) -- (91.5000,85.9000) -- (93.6000,87.5000) -- (95.8000,88.8000) -- (98.3000,89.8000) -- (100.9000,90.5000) -- (103.5000,90.7000) -- (106.1000,90.6000) -- (108.7000,90.2000) -- (111.2000,89.4000) -- (113.5000,88.2000) -- (115.7000,86.7000) -- (117.6000,84.9000) -- (119.2000,82.9000) -- (120.2000,80.5000) -- (120.7000,78.1000) -- (120.9000,75.5000) -- (120.9000,72.9000) -- (121.0000,70.4000) -- (121.0000,67.8000) -- (121.0000,65.3000) -- (121.0000,62.8000) -- (121.0000,60.3000) -- (121.0000,57.8000) -- (121.0000,55.2000) -- (121.0000,52.7000) -- (121.0000,50.2000) -- (121.3000,47.7000) -- (121.6000,45.2000) -- (121.5000,42.8000) -- (121.0000,40.4000) -- (120.0000,38.0000) -- (118.6000,35.9000) -- (116.8000,34.0000) -- (114.8000,32.4000) -- (112.4000,31.2000) -- (110.0000,30.3000) -- (107.4000,29.7000) -- (104.7000,29.6000) -- (102.0000,29.9000) -- (99.4000,30.5000) -- (97.0000,31.5000) -- (94.8000,32.9000) -- (92.8000,34.6000) -- (91.2000,36.6000) -- (90.0000,38.8000) -- (89.2000,41.2000) -- (88.9000,43.7000) -- (88.8000,46.2000) -- (88.7000,48.7000) -- (88.7000,51.2000) -- (88.7000,53.7000) -- (88.7000,56.2000) -- (88.7000,58.7000) -- (88.7000,61.2000) -- (88.7000,63.8000) -- (88.7000,66.3000) -- (88.7000,68.8000) -- (88.7000,71.3000) -- (88.7000,73.8000) -- (88.8000,76.4000) -- (89.4000,78.9000) -- (90.4000,81.3000) -- (91.8000,83.5000) -- (93.6000,85.4000) -- (95.6000,87.1000) -- (97.9000,88.4000) -- (100.3000,89.5000) -- (102.9000,90.2000) -- (105.5000,90.5000) -- (108.1000,90.5000) -- (110.7000,90.1000) -- (113.2000,89.3000) -- (115.6000,88.2000) -- (117.8000,86.8000) -- (119.8000,85.0000) -- (121.4000,83.0000) -- (122.5000,80.7000) -- (123.1000,78.3000) -- (123.3000,75.7000) -- (123.4000,73.1000) -- (123.4000,70.6000) -- (123.4000,68.0000) -- (123.4000,65.5000) -- (123.4000,63.0000) -- (123.4000,60.5000) -- (123.4000,58.0000) -- (123.4000,55.4000) -- (123.4000,52.9000) -- (123.4000,50.4000) -- (123.7000,47.9000) -- (124.1000,45.4000) -- (124.0000,42.9000) -- (123.5000,40.5000) -- (122.6000,38.2000) -- (121.3000,36.0000) -- (119.6000,34.1000) -- (117.6000,32.5000) -- (115.3000,31.1000) -- (112.8000,30.2000) -- (110.2000,29.6000) -- (107.6000,29.4000) -- (104.9000,29.6000) -- (102.3000,30.1000) -- (99.9000,31.1000) -- (97.6000,32.4000) -- (95.6000,34.1000) -- (93.9000,36.0000) -- (92.6000,38.2000) -- (91.8000,40.6000) -- (91.4000,43.1000) -- (91.2000,45.6000) -- (91.2000,48.0000) -- (91.2000,50.5000) -- (91.2000,53.0000) -- (91.2000,55.6000) -- (91.2000,58.1000) -- (91.2000,60.6000) -- (91.2000,63.1000) -- (91.2000,65.6000) -- (91.2000,68.1000) -- (91.2000,70.7000) -- (91.2000,73.2000) -- (91.2000,75.7000) -- (91.7000,78.3000) -- (92.6000,80.7000) -- (94.0000,82.9000) -- (95.7000,84.9000) -- (97.7000,86.6000) -- (99.9000,88.0000) -- (102.4000,89.1000) -- (104.9000,89.9000) -- (107.5000,90.3000) -- (110.1000,90.3000) -- (112.7000,90.0000) -- (115.3000,89.3000) -- (117.7000,88.2000) -- (119.9000,86.8000) -- (121.9000,85.1000) -- (123.6000,83.1000) -- (124.8000,80.9000) -- (125.5000,78.5000) -- (125.7000,75.9000) -- (125.8000,73.4000) -- (125.9000,70.8000) -- (125.9000,68.2000) -- (125.9000,65.7000) -- (125.9000,63.2000) -- (125.9000,60.7000) -- (125.9000,58.2000) -- (125.9000,55.7000) -- (125.9000,53.1000) -- (125.9000,50.6000) -- (126.0000,48.1000) -- (126.5000,45.5000);



  \end{scope}
  \begin{scope}[cm={{1.15801,0.0,0.0,1.15801,(-615.96866,-167.12877)}},draw=blue,line cap=round,line join=round,line width=0.480pt]
    \path[draw] (44.5000,13.5000) -- (44.5000,102.5000) -- (142.5000,102.5000) -- (142.5000,13.5000) -- (44.5000,13.5000);



  \end{scope}
  \begin{scope}[cm={{1.15801,0.0,0.0,1.15801,(-736.40737,-53.12875)}},draw=ca0a0a4,dash pattern=on 0.40pt off 0.80pt,line cap=round,line join=round,line width=0.400pt]
    \path[draw] (148.5000,102.5000) -- (246.5000,102.5000);



  \end{scope}
  \begin{scope}[cm={{1.15801,0.0,0.0,1.15801,(-736.40737,-53.12875)}},draw=blue,line cap=round,line join=round,line width=0.480pt]
    \path[draw] (148.5000,102.5000) -- (151.5000,102.5000);



    \path[draw] (246.5000,102.5000) -- (243.5000,102.5000);



  \end{scope}
  \begin{scope}[cm={{1.15801,0.0,0.0,1.15801,(-736.40737,-53.12875)}},draw=ca0a0a4,dash pattern=on 1.55pt off 1.55pt,line cap=round,line join=round,line width=0.259pt,miter limit=4.00]
    \path[draw,dash pattern=on 1.55pt off 1.55pt,line width=0.259pt,miter limit=4.00] (148.5000,79.5000) -- (246.5000,79.5000);



  \end{scope}
  \begin{scope}[cm={{1.15801,0.0,0.0,1.15801,(-736.40737,-53.12875)}},draw=blue,line cap=round,line join=round,line width=0.480pt]
    \path[draw] (148.5000,79.5000) -- (151.5000,79.5000);



    \path[draw] (246.5000,79.5000) -- (243.5000,79.5000);



  \end{scope}
  \begin{scope}[cm={{1.15801,0.0,0.0,1.15801,(-736.40737,-53.12875)}},draw=ca0a0a4,dash pattern=on 1.55pt off 1.55pt,line cap=round,line join=round,line width=0.259pt,miter limit=4.00]
    \path[draw,dash pattern=on 1.55pt off 1.55pt,line width=0.259pt,miter limit=4.00] (148.5000,57.5000) -- (246.5000,57.5000);



  \end{scope}
  \begin{scope}[cm={{1.15801,0.0,0.0,1.15801,(-736.40737,-53.12875)}},draw=blue,line cap=round,line join=round,line width=0.480pt]
    \path[draw] (148.5000,57.5000) -- (151.5000,57.5000);



    \path[draw] (246.5000,57.5000) -- (243.5000,57.5000);



  \end{scope}
  \begin{scope}[cm={{1.15801,0.0,0.0,1.15801,(-736.40737,-53.12875)}},draw=ca0a0a4,dash pattern=on 1.55pt off 1.55pt,line cap=round,line join=round,line width=0.259pt,miter limit=4.00]
    \path[draw,dash pattern=on 1.55pt off 1.55pt,line width=0.259pt,miter limit=4.00] (148.5000,35.5000) -- (246.5000,35.5000);



  \end{scope}
  \begin{scope}[cm={{1.15801,0.0,0.0,1.15801,(-736.40737,-53.12875)}},draw=blue,line cap=round,line join=round,line width=0.480pt]
    \path[draw] (148.5000,35.5000) -- (151.5000,35.5000);



    \path[draw] (246.5000,35.5000) -- (243.5000,35.5000);



  \end{scope}
  \begin{scope}[cm={{1.15801,0.0,0.0,1.15801,(-736.40737,-53.12875)}},draw=ca0a0a4,dash pattern=on 0.40pt off 0.80pt,line cap=round,line join=round,line width=0.400pt]
    \path[draw] (148.5000,13.5000) -- (246.5000,13.5000);



  \end{scope}
  \begin{scope}[cm={{1.15801,0.0,0.0,1.15801,(-673.8747,-53.12875)}},draw=ca0a0a4,dash pattern=on 1.55pt off 1.55pt,line cap=round,line join=round,line width=0.259pt,miter limit=4.00]
    \path[draw,dash pattern=on 1.55pt off 1.55pt,line width=0.259pt,miter limit=4.00] (118.5000,102.5000) -- (118.5000,13.5000);



  \end{scope}
  \begin{scope}[cm={{1.15801,0.0,0.0,1.15801,(-736.40737,-53.12875)}},draw=blue,line cap=round,line join=round,line width=0.480pt]
    \path[draw] (148.5000,13.5000) -- (151.5000,13.5000);



    \path[draw] (246.5000,13.5000) -- (243.5000,13.5000);



  \end{scope}
  \begin{scope}[cm={{1.15801,0.0,0.0,1.15801,(-736.40737,-53.12875)}},draw=ca0a0a4,dash pattern=on 0.40pt off 0.80pt,line cap=round,line join=round,line width=0.400pt]
    \path[draw] (148.5000,102.5000) -- (148.5000,13.5000);



  \end{scope}
  \begin{scope}[cm={{1.15801,0.0,0.0,1.15801,(-736.40737,-53.12875)}},draw=blue,line cap=round,line join=round,line width=0.480pt]
    \path[draw] (148.5000,102.5000) -- (148.5000,99.5000);



    \path[draw] (148.5000,13.5000) -- (148.5000,16.5000);



  \end{scope}
  \begin{scope}[cm={{1.15801,0.0,0.0,1.15801,(-736.40737,-53.12875)}},draw=blue,line cap=round,line join=round,line width=0.480pt]
    \path[draw] (172.5000,102.5000) -- (172.5000,99.5000);



    \path[draw] (172.5000,13.5000) -- (172.5000,16.5000);



  \end{scope}
  \begin{scope}[cm={{1.15801,0.0,0.0,1.15801,(-644.92439,-53.12875)}},draw=ca0a0a4,dash pattern=on 1.55pt off 1.55pt,line cap=round,line join=round,line width=0.259pt,miter limit=4.00]
    \path[draw,dash pattern=on 1.55pt off 1.55pt,line width=0.259pt,miter limit=4.00] (118.5000,102.5000) -- (118.5000,13.5000);



  \end{scope}
  \begin{scope}[cm={{1.15801,0.0,0.0,1.15801,(-736.40737,-53.12875)}},draw=blue,line cap=round,line join=round,line width=0.480pt]
    \path[draw] (197.5000,102.5000) -- (197.5000,99.5000);



    \path[draw] (197.5000,13.5000) -- (197.5000,16.5000);



  \end{scope}
  \begin{scope}[cm={{1.15801,0.0,0.0,1.15801,(-617.13209,-53.12875)}},draw=ca0a0a4,dash pattern=on 1.55pt off 1.55pt,line cap=round,line join=round,line width=0.259pt,miter limit=4.00]
    \path[draw,dash pattern=on 1.55pt off 1.55pt,line width=0.259pt,miter limit=4.00] (118.5000,102.5000) -- (118.5000,13.5000);



  \end{scope}
  \begin{scope}[cm={{1.15801,0.0,0.0,1.15801,(-736.40737,-53.12875)}},draw=blue,line cap=round,line join=round,line width=0.480pt]
    \path[draw] (221.5000,102.5000) -- (221.5000,99.5000);



    \path[draw] (221.5000,13.5000) -- (221.5000,16.5000);



  \end{scope}
  \begin{scope}[cm={{1.15801,0.0,0.0,1.15801,(-736.40737,-53.12875)}},draw=ca0a0a4,dash pattern=on 0.40pt off 0.80pt,line cap=round,line join=round,line width=0.400pt]
    \path[draw] (246.5000,102.5000) -- (246.5000,27.5000);



    \path[draw] (246.5000,19.5000) -- (246.5000,13.5000);



  \end{scope}
  \begin{scope}[cm={{1.15801,0.0,0.0,1.15801,(-736.40737,-53.12875)}},draw=blue,line cap=round,line join=round,line width=0.480pt]
    \path[draw] (246.5000,102.5000) -- (246.5000,99.5000);



    \path[draw] (246.5000,13.5000) -- (246.5000,16.5000);



  \end{scope}
  \begin{scope}[cm={{1.15801,0.0,0.0,1.15801,(-736.40737,-53.12875)}},draw=blue,line cap=round,line join=round,line width=0.480pt]
    \path[draw] (148.5000,13.5000) -- (148.5000,102.5000) -- (246.5000,102.5000) -- (246.5000,13.5000) -- (148.5000,13.5000);



  \end{scope}
  \begin{scope}[cm={{0.84029,0.0,0.0,0.84029,(-689.6406,-75.21977)}},fill=cd9d9d9]
    \path[rounded corners=0.0000cm] (222.0000,18.0000) rectangle (238.0000,34.0000);



  \end{scope}
  \begin{scope}[cm={{1.15801,0.0,0.0,1.15801,(-736.40737,-53.12875)}},draw=blue,line cap=round,line join=round,line width=0.480pt]
    \path[draw] (160.6000,31.9000) -- (160.6000,31.9000) -- (160.7000,33.1000) -- (160.7000,34.2000) -- (160.8000,35.2000) -- (160.9000,36.2000) -- (160.9000,37.2000) -- (160.6000,38.2000) -- (159.9000,39.1000) -- (159.3000,40.0000) -- (158.7000,40.9000) -- (158.2000,41.8000) -- (157.7000,42.8000) -- (157.4000,43.8000) -- (157.2000,44.8000) -- (157.1000,45.8000) -- (157.0000,46.9000) -- (157.0000,48.0000) -- (157.0000,49.0000) -- (157.0000,50.1000) -- (157.0000,51.1000) -- (157.0000,52.2000) -- (157.0000,53.3000) -- (157.0000,54.3000) -- (157.0000,55.4000) -- (157.0000,56.4000) -- (157.0000,57.5000) -- (157.0000,58.5000) -- (157.0000,59.6000) -- (157.0000,60.6000) -- (157.0000,61.7000) -- (157.0000,62.8000) -- (157.0000,63.8000) -- (157.0000,64.9000) -- (157.0000,65.9000) -- (157.0000,67.0000) -- (157.0000,68.0000) -- (157.0000,69.1000) -- (157.0000,70.1000) -- (157.0000,71.2000) -- (157.0000,72.2000) -- (157.0000,73.3000) -- (157.0000,74.4000) -- (157.0000,75.4000) -- (157.0000,76.5000) -- (157.0000,77.5000) -- (157.0000,78.6000) -- (157.1000,79.6000) -- (157.3000,80.7000) -- (157.6000,81.7000) -- (157.9000,82.7000) -- (158.3000,83.7000) -- (158.8000,84.7000) -- (159.4000,85.6000) -- (160.0000,86.5000) -- (160.7000,87.4000) -- (161.4000,88.3000) -- (162.2000,89.0000) -- (163.1000,89.8000) -- (164.0000,90.5000) -- (165.0000,91.1000) -- (166.0000,91.7000) -- (167.1000,92.2000) -- (168.2000,92.7000) -- (169.3000,93.1000) -- (170.5000,93.4000) -- (171.7000,93.6000) -- (172.9000,93.8000) -- (174.1000,93.9000) -- (175.3000,94.0000) -- (176.6000,93.9000) -- (177.8000,93.8000) -- (179.0000,93.6000) -- (180.2000,93.4000) -- (181.4000,93.0000) -- (182.5000,92.6000) -- (183.7000,92.1000) -- (184.7000,91.6000) -- (185.8000,91.0000) -- (186.7000,90.3000) -- (187.7000,89.6000) -- (188.5000,88.8000) -- (189.4000,88.0000) -- (190.1000,87.2000) -- (190.8000,86.3000) -- (191.4000,85.3000) -- (191.9000,84.4000) -- (192.3000,83.3000) -- (192.6000,82.3000) -- (192.8000,81.3000) -- (193.0000,80.3000) -- (193.1000,79.2000) -- (193.2000,78.2000) -- (193.3000,77.2000) -- (193.3000,76.1000) -- (193.4000,75.1000) -- (193.4000,74.0000) -- (193.4000,73.0000) -- (193.4000,71.9000) -- (193.4000,70.9000) -- (193.4000,69.8000) -- (193.4000,68.8000) -- (193.5000,67.7000) -- (193.5000,66.6000) -- (193.5000,65.6000) -- (193.5000,64.5000) -- (193.5000,63.5000) -- (193.5000,62.4000) -- (193.5000,61.4000) -- (193.5000,60.3000) -- (193.5000,59.3000) -- (193.5000,58.2000) -- (193.5000,57.2000) -- (193.5000,56.1000) -- (193.5000,55.0000) -- (193.5000,54.0000) -- (193.5000,52.9000) -- (193.5000,51.9000) -- (193.5000,50.8000) -- (193.7000,49.8000) -- (193.9000,48.7000) -- (194.0000,47.7000) -- (194.0000,46.6000) -- (193.9000,45.6000) -- (193.8000,44.5000) -- (193.6000,43.5000) -- (193.2000,42.5000) -- (192.8000,41.5000) -- (192.4000,40.5000) -- (191.8000,39.6000) -- (191.2000,38.7000) -- (190.4000,37.8000) -- (189.6000,37.0000) -- (188.8000,36.3000) -- (187.8000,35.6000) -- (186.8000,34.9000) -- (185.8000,34.4000) -- (184.7000,33.9000) -- (183.5000,33.5000) -- (182.3000,33.2000) -- (181.1000,33.0000) -- (179.9000,32.8000) -- (178.7000,32.8000) -- (177.4000,32.8000) -- (176.2000,32.9000) -- (175.0000,33.1000) -- (173.8000,33.4000) -- (172.7000,33.8000) -- (171.6000,34.2000) -- (170.5000,34.8000) -- (169.5000,35.3000) -- (168.5000,36.0000) -- (167.6000,36.7000) -- (166.8000,37.5000) -- (166.0000,38.3000) -- (165.4000,39.2000) -- (164.7000,40.1000) -- (164.2000,41.1000) -- (163.8000,42.1000) -- (163.4000,43.1000) -- (163.2000,44.1000) -- (163.1000,45.2000) -- (163.0000,46.2000) -- (163.0000,47.3000) -- (163.0000,48.4000) -- (163.0000,49.4000) -- (163.0000,50.5000) -- (163.0000,51.5000) -- (163.0000,52.6000) -- (163.0000,53.6000) -- (163.0000,54.7000) -- (162.9000,55.7000) -- (162.9000,56.8000) -- (162.9000,57.9000) -- (162.9000,58.9000) -- (162.9000,60.0000) -- (162.9000,61.0000) -- (162.9000,62.1000) -- (162.9000,63.1000) -- (162.9000,64.2000) -- (162.9000,65.2000) -- (162.9000,66.3000) -- (162.9000,67.4000) -- (162.9000,68.4000) -- (162.9000,69.5000) -- (162.9000,70.5000) -- (162.9000,71.6000) -- (162.9000,72.6000) -- (162.9000,73.7000) -- (162.9000,74.7000) -- (162.9000,75.8000) -- (162.9000,76.9000) -- (163.0000,77.9000) -- (163.1000,79.0000) -- (163.3000,80.0000) -- (163.5000,81.0000) -- (163.9000,82.1000) -- (164.3000,83.1000) -- (164.7000,84.0000) -- (165.3000,85.0000) -- (165.9000,85.9000) -- (166.6000,86.8000) -- (167.3000,87.6000) -- (168.1000,88.4000) -- (168.9000,89.2000) -- (169.9000,89.9000) -- (170.8000,90.5000) -- (171.8000,91.1000) -- (172.9000,91.7000) -- (174.0000,92.1000) -- (175.1000,92.5000) -- (176.3000,92.9000) -- (177.4000,93.2000) -- (178.6000,93.4000) -- (179.9000,93.5000) -- (181.1000,93.5000) -- (182.3000,93.5000) -- (183.6000,93.4000) -- (184.8000,93.3000) -- (186.0000,93.0000) -- (187.2000,92.7000) -- (188.3000,92.3000) -- (189.5000,91.9000) -- (190.6000,91.4000) -- (191.6000,90.8000) -- (192.6000,90.1000) -- (193.6000,89.4000) -- (194.5000,88.7000) -- (195.3000,87.9000) -- (196.1000,87.1000) -- (196.8000,86.2000) -- (197.4000,85.2000) -- (197.9000,84.3000) -- (198.4000,83.3000) -- (198.7000,82.3000) -- (198.9000,81.2000) -- (199.1000,80.2000) -- (199.3000,79.2000) -- (199.4000,78.1000) -- (199.4000,77.1000) -- (199.5000,76.1000) -- (199.5000,75.0000) -- (199.6000,74.0000) -- (199.6000,72.9000) -- (199.6000,71.9000) -- (199.6000,70.8000) -- (199.6000,69.8000) -- (199.6000,68.7000) -- (199.6000,67.7000) -- (199.6000,66.6000) -- (199.6000,65.5000) -- (199.7000,64.5000) -- (199.7000,63.4000) -- (199.7000,62.4000) -- (199.7000,61.3000) -- (199.7000,60.3000) -- (199.7000,59.2000) -- (199.7000,58.2000) -- (199.7000,57.1000) -- (199.7000,56.0000) -- (199.7000,55.0000) -- (199.7000,53.9000) -- (199.7000,52.9000) -- (199.7000,51.8000) -- (199.7000,50.8000) -- (199.8000,49.7000) -- (200.0000,48.7000) -- (200.1000,47.6000) -- (200.2000,46.6000) -- (200.1000,45.5000) -- (200.0000,44.5000) -- (199.8000,43.4000) -- (199.6000,42.4000) -- (199.2000,41.4000) -- (198.8000,40.4000) -- (198.3000,39.5000) -- (197.6000,38.5000) -- (197.0000,37.7000) -- (196.2000,36.8000) -- (195.4000,36.1000) -- (194.5000,35.3000) -- (193.5000,34.7000) -- (192.5000,34.1000) -- (191.4000,33.5000) -- (190.2000,33.1000) -- (189.1000,32.7000) -- (187.9000,32.5000) -- (186.7000,32.3000) -- (185.4000,32.2000) -- (184.2000,32.2000) -- (183.0000,32.2000) -- (181.8000,32.4000) -- (180.6000,32.7000) -- (179.4000,33.0000) -- (178.3000,33.4000) -- (177.2000,33.9000) -- (176.1000,34.4000) -- (175.2000,35.1000) -- (174.2000,35.8000) -- (173.4000,36.5000) -- (172.6000,37.3000) -- (171.8000,38.2000) -- (171.2000,39.0000) -- (170.6000,40.0000) -- (170.1000,41.0000) -- (169.7000,42.0000) -- (169.4000,43.0000) -- (169.2000,44.0000) -- (169.2000,45.1000) -- (169.1000,46.1000) -- (169.1000,47.2000) -- (169.1000,48.3000) -- (169.1000,49.3000) -- (169.1000,50.4000) -- (169.1000,51.4000) -- (169.1000,52.5000) -- (169.1000,53.6000) -- (169.1000,54.6000) -- (169.1000,55.7000) -- (169.1000,56.7000) -- (169.1000,57.8000) -- (169.1000,58.8000) -- (169.1000,59.9000) -- (169.1000,60.9000) -- (169.1000,62.0000) -- (169.1000,63.1000) -- (169.1000,64.1000) -- (169.1000,65.2000) -- (169.1000,66.2000) -- (169.1000,67.3000) -- (169.1000,68.3000) -- (169.1000,69.4000) -- (169.1000,70.4000) -- (169.1000,71.5000) -- (169.1000,72.6000) -- (169.1000,73.6000) -- (169.1000,74.7000) -- (169.1000,75.7000) -- (169.1000,76.8000) -- (169.1000,77.8000) -- (169.3000,78.9000) -- (169.5000,79.9000) -- (169.8000,80.9000) -- (170.1000,82.0000) -- (170.6000,82.9000) -- (171.1000,83.9000) -- (171.7000,84.8000) -- (172.3000,85.7000) -- (173.0000,86.6000) -- (173.8000,87.4000) -- (174.6000,88.2000) -- (175.5000,88.9000) -- (176.4000,89.6000) -- (177.4000,90.2000) -- (178.4000,90.8000) -- (179.5000,91.3000) -- (180.6000,91.8000) -- (181.7000,92.2000) -- (182.9000,92.5000) -- (184.1000,92.7000) -- (185.3000,92.9000) -- (186.5000,93.0000) -- (187.8000,93.0000) -- (189.0000,93.0000) -- (190.2000,92.8000) -- (191.4000,92.6000) -- (192.6000,92.4000) -- (193.8000,92.0000) -- (195.0000,91.6000) -- (196.1000,91.1000) -- (197.2000,90.6000) -- (198.2000,90.0000) -- (199.2000,89.3000) -- (200.1000,88.6000) -- (201.0000,87.8000) -- (201.8000,87.0000) -- (202.5000,86.2000) -- (203.2000,85.3000) -- (203.8000,84.3000) -- (204.3000,83.3000) -- (204.7000,82.3000) -- (205.0000,81.3000) -- (205.2000,80.3000) -- (205.3000,79.2000) -- (205.5000,78.2000) -- (205.6000,77.2000) -- (205.6000,76.1000) -- (205.7000,75.1000) -- (205.7000,74.0000) -- (205.7000,73.0000) -- (205.8000,71.9000) -- (205.8000,70.9000) -- (205.8000,69.8000) -- (205.8000,68.8000) -- (205.8000,67.7000) -- (205.8000,66.7000) -- (205.8000,65.6000) -- (205.8000,64.6000) -- (205.8000,63.5000) -- (205.8000,62.5000) -- (205.8000,61.4000) -- (205.8000,60.3000) -- (205.8000,59.3000) -- (205.8000,58.2000) -- (205.8000,57.2000) -- (205.8000,56.1000) -- (205.8000,55.1000) -- (205.8000,54.0000) -- (205.8000,53.0000) -- (205.8000,51.9000) -- (205.8000,50.8000) -- (205.9000,49.8000) -- (206.0000,48.7000) -- (206.2000,47.7000) -- (206.3000,46.6000) -- (206.3000,45.6000) -- (206.3000,44.5000) -- (206.1000,43.5000) -- (205.9000,42.5000) -- (205.6000,41.4000) -- (205.2000,40.4000) -- (204.7000,39.5000) -- (204.2000,38.5000) -- (203.5000,37.6000) -- (202.8000,36.8000) -- (202.0000,36.0000) -- (201.2000,35.2000) -- (200.2000,34.5000) -- (199.2000,33.9000) -- (198.2000,33.3000) -- (197.1000,32.8000) -- (195.9000,32.4000) -- (194.8000,32.1000) -- (193.5000,31.8000) -- (192.3000,31.7000) -- (191.1000,31.6000) -- (189.9000,31.6000) -- (188.6000,31.7000) -- (187.4000,31.9000) -- (186.3000,32.2000) -- (185.1000,32.6000) -- (184.0000,33.0000) -- (182.9000,33.5000) -- (181.9000,34.1000) -- (180.9000,34.8000) -- (180.0000,35.5000) -- (179.2000,36.2000) -- (178.4000,37.1000) -- (177.7000,37.9000) -- (177.1000,38.9000) -- (176.6000,39.8000) -- (176.1000,40.8000) -- (175.7000,41.8000) -- (175.5000,42.8000) -- (175.3000,43.9000) -- (175.3000,44.9000) -- (175.3000,46.0000) -- (175.2000,47.1000) -- (175.2000,48.1000) -- (175.2000,49.2000) -- (175.2000,50.2000) -- (175.2000,51.3000) -- (175.2000,52.4000) -- (175.2000,53.4000) -- (175.2000,54.5000) -- (175.2000,55.5000) -- (175.2000,56.6000) -- (175.2000,57.6000) -- (175.2000,58.7000) -- (175.2000,59.7000) -- (175.2000,60.8000) -- (175.2000,61.9000) -- (175.2000,62.9000) -- (175.2000,64.0000) -- (175.2000,65.0000) -- (175.2000,66.1000) -- (175.2000,67.1000) -- (175.2000,68.2000) -- (175.2000,69.2000) -- (175.2000,70.3000) -- (175.2000,71.4000) -- (175.2000,72.4000) -- (175.2000,73.5000) -- (175.2000,74.5000) -- (175.2000,75.6000) -- (175.2000,76.6000) -- (175.3000,77.7000) -- (175.5000,78.7000) -- (175.7000,79.8000) -- (176.1000,80.8000) -- (176.5000,81.8000) -- (176.9000,82.8000) -- (177.4000,83.7000) -- (178.0000,84.6000) -- (178.7000,85.5000) -- (179.4000,86.4000) -- (180.2000,87.2000) -- (181.1000,87.9000) -- (182.0000,88.7000) -- (182.9000,89.3000) -- (183.9000,89.9000) -- (185.0000,90.5000) -- (186.0000,91.0000) -- (187.2000,91.4000) -- (188.3000,91.7000) -- (189.5000,92.0000) -- (190.7000,92.2000) -- (191.9000,92.4000) -- (193.1000,92.4000) -- (194.4000,92.4000) -- (195.6000,92.4000) -- (196.8000,92.2000) -- (198.0000,92.0000) -- (199.2000,91.7000) -- (200.4000,91.3000) -- (201.5000,90.9000) -- (202.6000,90.4000) -- (203.7000,89.8000) -- (204.7000,89.2000) -- (205.7000,88.5000) -- (206.6000,87.8000) -- (207.4000,87.0000) -- (208.2000,86.1000) -- (208.9000,85.3000) -- (209.6000,84.3000) -- (210.1000,83.4000) -- (210.6000,82.4000) -- (210.9000,81.4000) -- (211.2000,80.3000) -- (211.4000,79.3000) -- (211.5000,78.3000) -- (211.6000,77.3000) -- (211.7000,76.2000) -- (211.8000,75.2000) -- (211.8000,74.1000) -- (211.9000,73.1000) -- (211.9000,72.0000) -- (211.9000,71.0000) -- (211.9000,69.9000) -- (211.9000,68.9000) -- (211.9000,67.8000) -- (211.9000,66.8000) -- (211.9000,65.7000) -- (211.9000,64.7000) -- (211.9000,63.6000) -- (211.9000,62.5000) -- (211.9000,61.5000) -- (211.9000,60.4000) -- (211.9000,59.4000) -- (211.9000,58.3000) -- (211.9000,57.3000) -- (211.9000,56.2000) -- (211.9000,55.2000) -- (211.9000,54.1000) -- (211.9000,53.0000) -- (211.9000,52.0000) -- (211.9000,50.9000) -- (212.0000,49.9000) -- (212.1000,48.8000) -- (212.2000,47.8000) -- (212.4000,46.7000) -- (212.5000,45.7000) -- (212.4000,44.6000) -- (212.4000,43.6000) -- (212.2000,42.5000) -- (211.9000,41.5000) -- (211.6000,40.5000) -- (211.2000,39.5000) -- (210.7000,38.6000) -- (210.1000,37.6000) -- (209.4000,36.7000) -- (208.7000,35.9000) -- (207.8000,35.1000) -- (206.9000,34.4000) -- (206.0000,33.7000) -- (205.0000,33.1000) -- (203.9000,32.5000) -- (202.8000,32.1000) -- (201.6000,31.7000) -- (200.4000,31.4000) -- (199.2000,31.2000) -- (198.0000,31.1000) -- (196.8000,31.1000) -- (195.5000,31.1000) -- (194.3000,31.3000) -- (193.1000,31.5000) -- (191.9000,31.8000) -- (190.8000,32.2000) -- (189.7000,32.7000) -- (188.7000,33.2000) -- (187.7000,33.8000) -- (186.7000,34.5000) -- (185.8000,35.2000) -- (185.0000,36.0000) -- (184.3000,36.9000) -- (183.6000,37.7000) -- (183.0000,38.7000) -- (182.5000,39.6000) -- (182.1000,40.6000) -- (181.7000,41.6000) -- (181.6000,42.7000) -- (181.5000,43.7000) -- (181.4000,44.8000) -- (181.4000,45.9000) -- (181.4000,46.9000) -- (181.4000,48.0000) -- (181.4000,49.0000) -- (181.4000,50.1000) -- (181.4000,51.2000) -- (181.4000,52.2000) -- (181.4000,53.3000) -- (181.4000,54.3000) -- (181.4000,55.4000) -- (181.4000,56.4000) -- (181.4000,57.5000) -- (181.4000,58.5000) -- (181.4000,59.6000) -- (181.4000,60.7000) -- (181.4000,61.7000) -- (181.4000,62.8000) -- (181.4000,63.8000) -- (181.4000,64.9000) -- (181.4000,65.9000) -- (181.4000,67.0000) -- (181.4000,68.0000) -- (181.4000,69.1000) -- (181.4000,70.2000) -- (181.4000,71.2000) -- (181.4000,72.3000) -- (181.4000,73.3000) -- (181.4000,74.4000) -- (181.4000,75.4000) -- (181.4000,76.5000) -- (181.5000,77.5000) -- (181.7000,78.6000) -- (182.0000,79.6000) -- (182.3000,80.6000) -- (182.8000,81.6000) -- (183.3000,82.6000) -- (183.8000,83.5000) -- (184.4000,84.4000) -- (185.1000,85.3000) -- (185.9000,86.1000) -- (186.7000,86.9000) -- (187.6000,87.7000) -- (188.5000,88.4000) -- (189.4000,89.0000) -- (190.5000,89.6000) -- (191.5000,90.1000) -- (192.6000,90.6000) -- (193.8000,91.0000) -- (194.9000,91.3000) -- (196.1000,91.6000) -- (197.3000,91.7000) -- (198.5000,91.9000) -- (199.8000,91.9000) -- (201.0000,91.9000) -- (202.2000,91.8000) -- (203.5000,91.6000) -- (204.7000,91.3000) -- (205.8000,91.0000) -- (207.0000,90.6000) -- (208.1000,90.1000) -- (209.2000,89.6000) -- (210.2000,89.0000) -- (211.2000,88.4000) -- (212.2000,87.7000) -- (213.1000,86.9000) -- (213.9000,86.1000) -- (214.6000,85.2000) -- (215.3000,84.3000) -- (215.9000,83.4000) -- (216.5000,82.4000) -- (216.9000,81.4000) -- (217.2000,80.4000) -- (217.4000,79.4000) -- (217.6000,78.4000) -- (217.7000,77.3000) -- (217.8000,76.3000) -- (217.9000,75.3000) -- (217.9000,74.2000) -- (218.0000,73.2000) -- (218.0000,72.1000) -- (218.0000,71.1000) -- (218.1000,70.0000) -- (218.1000,69.0000) -- (218.1000,67.9000) -- (218.1000,66.9000) -- (218.1000,65.8000) -- (218.1000,64.7000) -- (218.1000,63.7000) -- (218.1000,62.6000) -- (218.1000,61.6000) -- (218.1000,60.5000) -- (218.1000,59.5000) -- (218.1000,58.4000) -- (218.1000,57.4000) -- (218.1000,56.3000) -- (218.1000,55.2000) -- (218.1000,54.2000) -- (218.1000,53.1000) -- (218.1000,52.1000) -- (218.1000,51.0000) -- (218.1000,50.0000) -- (218.1000,48.9000) -- (218.3000,47.9000) -- (218.4000,46.8000) -- (218.6000,45.8000) -- (218.6000,44.7000) -- (218.6000,43.7000) -- (218.4000,42.6000) -- (218.2000,41.6000) -- (217.9000,40.6000) -- (217.6000,39.6000) -- (217.1000,38.6000) -- (216.6000,37.6000) -- (216.0000,36.7000) -- (215.3000,35.9000) -- (214.5000,35.0000) -- (213.6000,34.3000) -- (212.7000,33.5000) -- (211.7000,32.9000) -- (210.7000,32.3000) -- (209.6000,31.8000) -- (208.5000,31.4000) -- (207.3000,31.0000) -- (206.1000,30.8000) -- (204.9000,30.6000) -- (203.6000,30.5000) -- (202.4000,30.5000) -- (201.2000,30.6000) -- (200.0000,30.8000) -- (198.8000,31.1000) -- (197.6000,31.4000) -- (196.5000,31.8000) -- (195.4000,32.3000) -- (194.4000,32.9000) -- (193.4000,33.5000) -- (192.5000,34.2000) -- (191.6000,35.0000) -- (190.9000,35.8000) -- (190.1000,36.6000) -- (189.5000,37.5000) -- (188.9000,38.5000) -- (188.5000,39.5000) -- (188.1000,40.5000) -- (187.8000,41.5000) -- (187.7000,42.6000) -- (187.6000,43.6000) -- (187.5000,44.7000) -- (187.5000,45.7000) -- (187.5000,46.8000) -- (187.5000,47.9000) -- (187.5000,48.9000) -- (187.5000,50.0000) -- (187.5000,51.0000) -- (187.5000,52.1000) -- (187.5000,53.1000) -- (187.5000,54.2000) -- (187.5000,55.2000) -- (187.5000,56.3000) -- (187.5000,57.4000) -- (187.5000,58.4000) -- (187.5000,59.5000) -- (187.5000,60.5000) -- (187.5000,61.6000) -- (187.5000,62.6000) -- (187.5000,63.7000) -- (187.5000,64.7000) -- (187.5000,65.8000) -- (187.5000,66.9000) -- (187.5000,67.9000) -- (187.5000,69.0000) -- (187.5000,70.0000) -- (187.5000,71.1000) -- (187.5000,72.1000) -- (187.5000,73.2000) -- (187.5000,74.2000) -- (187.5000,75.3000) -- (187.6000,76.4000) -- (187.7000,77.4000) -- (188.0000,78.4000) -- (188.3000,79.5000) -- (188.7000,80.5000) -- (189.1000,81.5000) -- (189.6000,82.4000) -- (190.2000,83.3000) -- (190.8000,84.2000) -- (191.6000,85.1000) -- (192.3000,85.9000) -- (193.2000,86.7000) -- (194.1000,87.4000) -- (195.0000,88.1000) -- (196.0000,88.7000) -- (197.0000,89.2000) -- (198.1000,89.7000) -- (199.2000,90.2000) -- (200.4000,90.6000) -- (201.5000,90.9000) -- (202.7000,91.1000) -- (204.0000,91.2000) -- (205.2000,91.3000) -- (206.4000,91.3000) -- (207.6000,91.3000) -- (208.9000,91.1000) -- (210.1000,90.9000) -- (211.3000,90.6000) -- (212.5000,90.3000) -- (213.6000,89.9000) -- (214.7000,89.4000) -- (215.8000,88.8000) -- (216.8000,88.2000) -- (217.8000,87.5000) -- (218.7000,86.8000) -- (219.5000,86.0000) -- (220.3000,85.2000) -- (221.1000,84.3000) -- (221.7000,83.4000) -- (222.3000,82.5000) -- (222.8000,81.5000) -- (223.1000,80.5000) -- (223.4000,79.4000) -- (223.6000,78.4000) -- (223.8000,77.4000) -- (223.9000,76.4000) -- (224.0000,75.3000) -- (224.1000,74.3000) -- (224.1000,73.2000) -- (224.1000,72.2000) -- (224.2000,71.1000) -- (224.2000,70.1000) -- (224.2000,69.0000) -- (224.2000,68.0000) -- (224.2000,66.9000) -- (224.2000,65.9000) -- (224.2000,64.8000) -- (224.2000,63.8000) -- (224.2000,62.7000) -- (224.2000,61.7000) -- (224.2000,60.6000) -- (224.2000,59.5000) -- (224.2000,58.5000) -- (224.2000,57.4000) -- (224.2000,56.4000) -- (224.2000,55.3000) -- (224.2000,54.3000) -- (224.2000,53.2000) -- (224.2000,52.2000) -- (224.2000,51.1000) -- (224.2000,50.0000) -- (224.2000,49.0000) -- (224.3000,47.9000) -- (224.5000,46.9000) -- (224.6000,45.8000) -- (224.7000,44.8000) -- (224.7000,43.7000) -- (224.7000,42.7000) -- (224.5000,41.6000) -- (224.3000,40.6000) -- (223.9000,39.6000) -- (223.5000,38.6000) -- (223.1000,37.6000) -- (222.5000,36.7000) -- (221.8000,35.8000) -- (221.1000,35.0000) -- (220.3000,34.2000) -- (219.4000,33.4000) -- (218.5000,32.7000) -- (217.5000,32.1000) -- (216.4000,31.5000) -- (215.3000,31.1000) -- (214.1000,30.7000) -- (213.0000,30.4000) -- (211.7000,30.1000) -- (210.5000,30.0000) -- (209.3000,29.9000) -- (208.1000,30.0000) -- (206.8000,30.1000) -- (205.6000,30.3000) -- (204.5000,30.6000) -- (203.3000,31.0000) -- (202.2000,31.5000) -- (201.2000,32.0000) -- (200.1000,32.6000) -- (199.2000,33.2000) -- (198.3000,34.0000) -- (197.5000,34.7000) -- (196.7000,35.6000) -- (196.0000,36.4000) -- (195.4000,37.4000) -- (194.9000,38.3000) -- (194.4000,39.3000) -- (194.1000,40.3000) -- (193.9000,41.4000) -- (193.8000,42.4000) -- (193.7000,43.5000) -- (193.7000,44.5000) -- (193.7000,45.6000) -- (193.7000,46.7000) -- (193.6000,47.7000) -- (193.6000,48.8000) -- (193.6000,49.8000) -- (193.6000,50.9000) -- (193.6000,51.9000) -- (193.6000,53.0000) -- (193.6000,54.1000) -- (193.6000,55.1000) -- (193.6000,56.2000) -- (193.6000,57.2000) -- (193.6000,58.3000) -- (193.6000,59.3000) -- (193.6000,60.4000) -- (193.6000,61.4000) -- (193.6000,62.5000) -- (193.6000,63.6000) -- (193.6000,64.6000) -- (193.6000,65.7000) -- (193.6000,66.7000) -- (193.6000,67.8000) -- (193.6000,68.8000) -- (193.6000,69.9000) -- (193.6000,70.9000) -- (193.6000,72.0000) -- (193.6000,73.1000) -- (193.6000,74.1000) -- (193.7000,75.2000) -- (193.8000,76.2000) -- (194.0000,77.3000) -- (194.2000,78.3000) -- (194.6000,79.3000) -- (195.0000,80.3000) -- (195.4000,81.3000) -- (196.0000,82.2000) -- (196.6000,83.1000) -- (197.3000,84.0000) -- (198.0000,84.9000) -- (198.8000,85.7000) -- (199.7000,86.4000) -- (200.6000,87.1000) -- (201.5000,87.8000) -- (202.5000,88.4000) -- (203.6000,88.9000) -- (204.7000,89.4000) -- (205.8000,89.8000) -- (207.0000,90.1000) -- (208.2000,90.4000) -- (209.4000,90.6000) -- (210.6000,90.7000) -- (211.8000,90.8000) -- (213.0000,90.8000) -- (214.3000,90.7000) -- (215.5000,90.5000) -- (216.7000,90.3000) -- (217.9000,90.0000) -- (219.0000,89.6000) -- (220.2000,89.1000) -- (221.3000,88.6000) -- (222.3000,88.0000) -- (223.3000,87.4000) -- (224.3000,86.7000) -- (225.2000,85.9000) -- (226.0000,85.2000) -- (226.8000,84.3000) -- (227.5000,83.4000) -- (228.1000,82.5000) -- (228.6000,81.5000) -- (229.1000,80.5000) -- (229.4000,79.5000) -- (229.7000,78.5000) -- (229.9000,77.5000) -- (230.0000,76.4000) -- (230.1000,75.4000) -- (230.2000,74.4000) -- (230.2000,73.3000) -- (230.3000,72.3000) -- (230.3000,71.2000) -- (230.3000,70.2000) -- (230.3000,69.1000) -- (230.3000,68.1000) -- (230.4000,67.0000) -- (230.4000,66.0000) -- (230.4000,64.9000) -- (230.4000,63.9000) -- (230.4000,62.8000) -- (230.4000,61.7000) -- (230.4000,60.7000) -- (230.4000,59.6000) -- (230.4000,58.6000) -- (230.4000,57.5000) -- (230.4000,56.5000) -- (230.4000,55.4000) -- (230.4000,54.4000) -- (230.4000,53.3000) -- (230.4000,52.2000) -- (230.4000,51.2000) -- (230.4000,50.1000) -- (230.4000,49.1000) -- (230.4000,48.0000) -- (230.5000,47.0000) -- (230.7000,45.9000) -- (230.8000,44.8000);



  \end{scope}
  \begin{scope}[cm={{1.15801,0.0,0.0,1.15801,(-736.40737,-53.12875)}},draw=blue,line cap=round,line join=round,line width=0.480pt]
    \path[draw] (148.5000,13.5000) -- (148.5000,102.5000) -- (246.5000,102.5000) -- (246.5000,13.5000) -- (148.5000,13.5000);



  \end{scope}
  \begin{scope}[cm={{0.84173,0.0,0.0,0.84173,(-601.60573,125.64086)}},fill=cffffff]
  \end{scope}
  \begin{scope}[cm={{1.00588,0.0,0.0,1.00588,(-513.7994,343.95446)}},draw=blue,line cap=rect,line join=bevel,line width=0.800pt]
    \begin{scope}[rotate around={-90.0:(-229.13106,-146.12949)},draw=blue,line cap=rect,line join=bevel,line width=0.800pt]
      \path[fill=blue] (0.0000,0.0000) node[above right] (text344-9) {\rotatebox{90}{y (m)}};



    \end{scope}
    \path[fill=blue] (-6.4779,-252.9639) node[above right] (text344-6-3) {x (m)};



    \path[fill=blue] (-76.6396,-237.2505) node[above right] (text344-6-3-1) {(a) Trajectories without re-planning};



    \path[fill=blue] (313 .5506,-237.2505) node[above right] (text344-6-3-1-1-9) {(c) State $(\alpha_0\dots\in\mathbf{q})$ evol. for {\hyperref[fig:trajs-I-static]{\color{red}I}}};



    \path[fill=blue] (100.2713,-237.2505) node[above right] (text344-6-3-1-6) {(b) Energies, detail of first instants, periods $T$};



  \end{scope}
  \begin{scope}[cm={{1.00588,0.0,0.0,1.00588,(-571.04217,77.13242)}},draw=blue,line cap=rect,line join=bevel,line width=0.800pt]
    \path[fill=blue] (0.0000,0.0000) node[above right] (text1768-1) {-150};



  \end{scope}
  \begin{scope}[cm={{1.00588,0.0,0.0,1.00588,(-543.83037,77.13242)}},draw=blue,line cap=rect,line join=bevel,line width=0.800pt]
    \path[fill=blue] (0.0000,0.0000) node[above right] (text1798-7) {-50};



  \end{scope}
  \begin{scope}[cm={{1.00588,0.0,0.0,1.00588,(-513.62717,77.13242)}},draw=blue,line cap=rect,line join=bevel,line width=0.800pt]
    \path[fill=blue] (0.0000,0.0000) node[above right] (text1828-9) {50};



  \end{scope}
  \begin{scope}[cm={{1.00588,0.0,0.0,1.00588,(-488.44517,77.13242)}},draw=blue,line cap=rect,line join=bevel,line width=0.800pt]
    \path[fill=blue] (0.0000,0.0000) node[above right] (text1858-6) {150};



  \end{scope}
  \begin{scope}[cm={{1.00588,0.0,0.0,1.00588,(-462.82756,77.13242)}},draw=blue,line cap=rect,line join=bevel,line width=0.800pt]
    \path[fill=blue] (0.0000,0.0000) node[above right] (text1858-6-6) {250};



  \end{scope}
  \begin{scope}[cm={{1.00588,0.0,0.0,1.00588,(-585.50071,67.88167)}},draw=blue,line cap=rect,line join=bevel,line width=0.800pt]
    \path[fill=blue] (0.0000,0.0000) node[above right] (text34-9-6) {-100};



  \end{scope}
  \begin{scope}[cm={{1.00588,0.0,0.0,1.00588,(-574.76608,42.75178)}},draw=blue,line cap=rect,line join=bevel,line width=0.800pt]
    \path[fill=blue] (0.0000,0.0000) node[above right] (text64-6-8) {0};



  \end{scope}
  \begin{scope}[cm={{1.00588,0.0,0.0,1.00588,(-582.81312,16.12239)}},draw=blue,line cap=rect,line join=bevel,line width=0.800pt]
    \path[fill=blue] (0.0000,0.0000) node[above right] (text94-7-9) {100};



  \end{scope}
  \begin{scope}[cm={{1.00588,0.0,0.0,1.00588,(-582.81312,-9.00701)}},draw=blue,line cap=rect,line join=bevel,line width=0.800pt]
    \path[fill=blue] (0.0000,0.0000) node[above right] (text124-9-0) {200};



  \end{scope}
  \begin{scope}[cm={{1.00588,0.0,0.0,1.00588,(-582.81312,-34.13641)}},draw=blue,line cap=rect,line join=bevel,line width=0.800pt]
    \path[fill=blue] (0.0000,0.0000) node[above right] (text154-3) {300};



  \end{scope}
  \begin{scope}[cm={{1.00769,0.0,0.0,1.00769,(-255.48702,89.55166)}},draw=blue,line cap=rect,line join=bevel,line width=0.800pt]
    \path[fill=blue] (0.0000,0.0000) node[above right] (text602-2) {\scriptsize Time (sec)};



  \end{scope}
  \begin{scope}[cm={{1.00588,0.0,0.0,1.00588,(-140.21331,89.54177)}},draw=blue,line cap=rect,line join=bevel,line width=0.800pt]
    \path[fill=blue] (0.0000,0.0000) node[above right] (text1412-73) {\scriptsize Time (min)};



  \end{scope}
  \begin{scope}[cm={{1.00588,0.0,0.0,1.00588,(-376.86267,89.54177)}},draw=blue,line cap=rect,line join=bevel,line width=0.800pt]
    \path[fill=blue] (0.0000,0.0000) node[above right] (text1412-73-1) {Time (min)};



  \end{scope}
  \begin{scope}[cm={{1.00588,0.0,0.0,1.00588,(-605.35515,-93.86128)}},draw=blue,line cap=rect,line join=bevel,line width=0.800pt]
    \path[fill=blue] (0.0000,0.0000) node[above right] (text154-9) {\Large{\label{fig:ener:static-I}\label{fig:trajs-I-static}I}};



  \end{scope}
  \begin{scope}[cm={{1.00588,0.0,0.0,1.00588,(-607.8469,20.77248)}},draw=blue,line cap=rect,line join=bevel,line width=0.800pt]
    \path[fill=blue] (2.4954,0.0000) node[above right] (text154-9-0) {\Large{\label{fig:ener:static-II}\label{fig:trajs-II-static}II}};



  \end{scope}
  \begin{scope}[cm={{0.0,-1.00588,1.00588,0.0,(154.28002,10.40301)}},draw=blue,line cap=rect,line join=bevel,line width=0.800pt]
    \path[fill=blue] (35.7896,-587.5454) node[above right] (text274-1) {\rotatebox{90}{Power (W)}};



  \end{scope}
  \begin{scope}[cm={{1.00588,0.0,0.0,1.00588,(-526.48136,104.32102)}},draw=blue,line cap=rect,line join=bevel,line width=0.800pt]
  \end{scope}
\end{scope}

\end{tikzpicture}


  {\color{blue}Fig.~6:~\color{blue}CPP with Zamboni-like motion using two boundary configurations. In \hyperref[fig:stat]{a} are the trajectories of the coverage--the highest in {\color{red}I} and the lowest in {\color{red}II}. In {\color{red}b} are the energy and the period evolutions for both {\color{red}I} and {\color{red}II} with different atmospheric conditions. In {\color{red}c} are the states that compose the energy model for {\color{red}I}.}

  
\definecolor{ca0a0a4}{RGB}{160,160,164}
\definecolor{cd9d9d9}{RGB}{235,235,235}
\definecolor{c00ff00}{RGB}{0,255,0}
\definecolor{cffffff}{RGB}{255,255,255}


\def \globalscale {1.000000}
\begin{tikzpicture}[y=0.80pt, x=0.80pt, yscale=-1.03*\globalscale, xscale=1.18*\globalscale, inner sep=0pt, outer sep=0pt]
\color{blue}
\begin{scope}[shift={(598.44106,-141.22792)},draw=blue,even odd rule,line cap=rect,line join=bevel,line width=0.800pt]
  \begin{scope}[cm={{0.84173,0.0,0.0,0.84173,(-603.85556,69.88496)}},draw=ca0a0a4,dash pattern=on 2.14pt off 2.14pt,line cap=round,line join=round,line width=0.356pt,miter limit=4.00]
    \path[draw,dash pattern=on 2.14pt off 2.14pt,line width=0.356pt,miter limit=4.00] (44.5000,181.5000) -- (179.5000,181.5000);



  \end{scope}
  \begin{scope}[cm={{0.84173,0.0,0.0,0.84173,(-603.85556,69.88496)}},draw=ca0a0a4,dash pattern=on 2.14pt off 2.14pt,line cap=round,line join=round,line width=0.356pt,miter limit=4.00]
    \path[draw,dash pattern=on 2.14pt off 2.14pt,line width=0.356pt,miter limit=4.00] (78.5000,210.5000) -- (78.5000,108.5000);



  \end{scope}
  \begin{scope}[cm={{0.84173,0.0,0.0,0.84173,(-603.85556,69.88496)}},draw=blue,line cap=round,line join=round,line width=0.480pt]
    \path[draw] (61.9000,118.7000) -- (61.9000,118.7000) -- (62.6000,120.9000) -- (63.3000,122.6000) -- (63.5000,124.1000) -- (62.7000,125.3000) -- (61.5000,126.5000) -- (60.3000,127.9000) -- (59.2000,129.5000) -- (58.3000,131.2000) -- (57.5000,133.0000) -- (57.0000,134.8000) -- (56.8000,136.7000) -- (56.7000,138.6000) -- (56.6000,140.4000) -- (56.6000,142.2000) -- (56.6000,144.1000) -- (56.6000,145.9000) -- (56.6000,147.7000) -- (56.6000,149.5000) -- (56.6000,151.4000) -- (56.6000,153.2000) -- (56.6000,155.1000) -- (56.6000,156.9000) -- (56.6000,158.8000) -- (56.6000,160.6000) -- (56.6000,162.5000) -- (56.6000,164.3000) -- (56.6000,166.2000) -- (56.6000,168.0000) -- (56.6000,169.9000) -- (56.6000,171.7000) -- (56.6000,173.6000) -- (56.6000,175.4000) -- (56.6000,177.2000) -- (56.6000,179.1000) -- (56.7000,181.0000) -- (57.1000,182.8000) -- (57.6000,184.7000) -- (58.3000,186.4000) -- (59.2000,188.2000) -- (60.2000,189.8000) -- (61.4000,191.3000) -- (62.7000,192.8000) -- (64.2000,194.1000) -- (65.7000,195.3000) -- (67.4000,196.4000) -- (69.1000,197.4000) -- (71.0000,198.2000) -- (72.9000,198.9000) -- (74.8000,199.5000) -- (76.8000,199.9000) -- (78.8000,200.2000) -- (80.8000,200.4000) -- (82.8000,200.4000) -- (84.8000,200.2000) -- (86.7000,200.0000) -- (88.7000,199.6000) -- (90.6000,199.0000) -- (92.5000,198.4000) -- (94.3000,197.5000) -- (96.1000,196.6000) -- (97.8000,195.5000) -- (99.3000,194.3000) -- (100.8000,193.0000) -- (102.1000,191.6000) -- (103.3000,190.0000) -- (104.2000,188.4000) -- (104.9000,186.7000) -- (105.3000,184.9000) -- (105.6000,183.0000) -- (105.7000,181.1000) -- (105.8000,179.2000) -- (105.8000,177.4000) -- (105.8000,175.5000) -- (105.8000,173.6000) -- (105.8000,171.7000) -- (105.8000,169.9000) -- (105.9000,168.0000) -- (105.9000,166.2000) -- (105.9000,164.3000) -- (105.9000,162.5000) -- (105.9000,160.6000) -- (105.9000,158.8000) -- (105.9000,157.0000) -- (105.9000,155.1000) -- (105.9000,153.3000) -- (105.9000,151.4000) -- (105.9000,149.6000) -- (105.9000,147.7000) -- (105.8000,145.9000) -- (106.1000,144.0000) -- (106.5000,142.1000) -- (106.8000,140.3000) -- (106.8000,138.6000) -- (106.7000,136.8000) -- (106.4000,135.0000) -- (105.9000,133.2000) -- (105.2000,131.5000) -- (104.3000,129.8000) -- (103.3000,128.3000) -- (102.1000,126.8000) -- (100.8000,125.4000) -- (99.3000,124.1000) -- (97.7000,122.9000) -- (96.0000,121.9000) -- (94.3000,121.0000) -- (92.4000,120.3000) -- (90.5000,119.7000) -- (88.5000,119.2000) -- (86.5000,118.9000) -- (84.5000,118.8000) -- (82.4000,118.8000) -- (80.4000,119.0000) -- (78.4000,119.3000) -- (76.4000,119.8000) -- (74.5000,120.4000) -- (72.6000,121.2000) -- (70.9000,122.1000) -- (69.2000,123.1000) -- (67.6000,124.3000) -- (66.2000,125.6000) -- (64.9000,127.0000) -- (63.8000,128.5000) -- (62.8000,130.1000) -- (62.0000,131.8000) -- (61.3000,133.6000) -- (60.9000,135.4000) -- (60.7000,137.2000) -- (60.6000,139.1000) -- (60.6000,140.9000) -- (60.6000,142.7000) -- (60.6000,144.5000) -- (60.6000,146.4000) -- (60.6000,148.2000) -- (60.6000,150.0000) -- (60.6000,151.9000) -- (60.6000,153.7000) -- (60.5000,155.6000) -- (60.5000,157.4000) -- (60.5000,159.3000) -- (60.5000,161.1000) -- (60.5000,163.0000) -- (60.5000,164.8000) -- (60.5000,166.7000) -- (60.5000,168.5000) -- (60.6000,170.4000) -- (60.6000,172.2000) -- (60.6000,174.1000) -- (60.6000,175.9000) -- (60.5000,177.7000) -- (60.6000,179.6000) -- (60.8000,181.5000) -- (61.2000,183.3000) -- (61.9000,185.1000) -- (62.7000,186.9000) -- (63.6000,188.6000) -- (64.7000,190.1000) -- (66.0000,191.6000) -- (67.4000,193.0000) -- (68.9000,194.3000) -- (70.6000,195.4000) -- (72.3000,196.4000) -- (74.1000,197.3000) -- (76.0000,198.1000) -- (77.9000,198.7000) -- (79.8000,199.2000) -- (81.8000,199.5000) -- (83.8000,199.7000) -- (85.8000,199.7000) -- (87.8000,199.6000) -- (89.8000,199.4000) -- (91.8000,199.0000) -- (93.7000,198.5000) -- (95.6000,197.8000) -- (97.4000,197.0000) -- (99.2000,196.1000) -- (100.8000,195.0000) -- (102.4000,193.8000) -- (103.9000,192.5000) -- (105.2000,191.1000) -- (106.4000,189.5000) -- (107.3000,187.9000) -- (108.0000,186.2000) -- (108.5000,184.4000) -- (108.7000,182.5000) -- (108.9000,180.6000) -- (109.0000,178.7000) -- (109.0000,176.9000) -- (109.0000,175.0000) -- (109.0000,173.1000) -- (109.0000,171.2000) -- (109.1000,169.4000) -- (109.1000,167.5000) -- (109.1000,165.7000) -- (109.1000,163.8000) -- (109.1000,162.0000) -- (109.1000,160.1000) -- (109.1000,158.3000) -- (109.1000,156.5000) -- (109.1000,154.6000) -- (109.1000,152.8000) -- (109.1000,150.9000) -- (109.1000,149.1000) -- (109.1000,147.2000) -- (109.1000,145.4000) -- (109.3000,143.5000) -- (109.7000,141.6000) -- (110.0000,139.8000) -- (110.0000,138.1000) -- (109.8000,136.3000) -- (109.5000,134.5000) -- (109.0000,132.7000) -- (108.3000,131.0000) -- (107.4000,129.4000) -- (106.3000,127.8000) -- (105.1000,126.3000) -- (103.7000,125.0000) -- (102.3000,123.7000) -- (100.6000,122.6000) -- (98.9000,121.6000) -- (97.1000,120.7000) -- (95.2000,120.0000) -- (93.3000,119.5000) -- (91.3000,119.1000) -- (89.3000,118.8000) -- (87.3000,118.7000) -- (85.2000,118.8000) -- (83.2000,119.0000) -- (81.2000,119.4000) -- (79.3000,119.9000) -- (77.4000,120.6000) -- (75.6000,121.5000) -- (73.9000,122.6000) -- (72.4000,123.8000) -- (70.9000,125.1000) -- (69.6000,126.5000) -- (68.5000,128.0000) -- (67.5000,129.6000) -- (66.7000,131.3000) -- (66.0000,133.0000) -- (65.7000,134.9000) -- (65.5000,136.7000) -- (65.4000,138.6000) -- (65.4000,140.4000) -- (65.4000,142.2000) -- (65.4000,144.0000) -- (65.3000,145.8000) -- (65.3000,147.7000) -- (65.3000,149.5000) -- (65.3000,151.4000) -- (65.3000,153.2000) -- (65.3000,155.1000) -- (65.3000,156.9000) -- (65.3000,158.8000) -- (65.3000,160.6000) -- (65.3000,162.4000) -- (65.3000,164.3000) -- (65.3000,166.1000) -- (65.3000,168.0000) -- (65.3000,169.8000) -- (65.3000,171.7000) -- (65.3000,173.5000) -- (65.3000,175.4000) -- (65.3000,177.2000) -- (65.4000,179.1000) -- (65.6000,181.0000) -- (66.0000,182.8000) -- (66.6000,184.6000) -- (67.4000,186.4000) -- (68.3000,188.1000) -- (69.4000,189.7000) -- (70.6000,191.2000) -- (71.9000,192.6000) -- (73.4000,193.9000) -- (75.0000,195.1000) -- (76.7000,196.2000) -- (78.5000,197.1000) -- (80.3000,198.0000) -- (82.2000,198.7000) -- (84.1000,199.2000) -- (86.1000,199.6000) -- (88.1000,199.9000) -- (90.1000,200.1000) -- (92.1000,200.2000) -- (94.1000,200.1000) -- (96.1000,199.8000) -- (98.0000,199.5000) -- (100.0000,199.0000) -- (101.9000,198.4000) -- (103.8000,197.7000) -- (105.6000,196.9000) -- (107.3000,195.9000) -- (109.0000,194.8000) -- (110.6000,193.6000) -- (112.0000,192.3000) -- (113.4000,190.9000) -- (114.6000,189.4000) -- (115.4000,187.7000) -- (115.9000,185.9000) -- (116.1000,184.1000) -- (116.2000,182.2000) -- (116.2000,180.3000) -- (116.3000,178.4000) -- (116.3000,176.5000) -- (116.3000,174.6000) -- (116.3000,172.8000) -- (116.3000,170.9000) -- (116.3000,169.1000) -- (116.3000,167.2000) -- (116.3000,165.4000) -- (116.3000,163.5000) -- (116.3000,161.7000) -- (116.3000,159.8000) -- (116.3000,158.0000) -- (116.3000,156.1000) -- (116.3000,154.3000) -- (116.3000,152.4000) -- (116.3000,150.6000) -- (116.3000,148.8000) -- (116.3000,146.9000) -- (116.3000,145.1000) -- (116.3000,143.2000) -- (116.5000,141.3000) -- (116.7000,139.5000) -- (116.7000,137.7000) -- (116.6000,135.9000) -- (116.3000,134.1000) -- (115.8000,132.3000) -- (115.1000,130.6000) -- (114.2000,129.0000) -- (113.2000,127.4000) -- (112.0000,125.9000) -- (110.6000,124.6000) -- (109.1000,123.3000) -- (107.5000,122.2000) -- (105.8000,121.2000) -- (104.0000,120.3000) -- (102.1000,119.6000) -- (100.2000,119.0000) -- (98.2000,118.6000) -- (96.2000,118.4000) -- (94.1000,118.3000) -- (92.1000,118.4000) -- (90.1000,118.6000) -- (88.1000,119.0000) -- (86.1000,119.5000) -- (84.2000,120.2000) -- (82.4000,121.1000) -- (80.7000,122.0000) -- (79.1000,123.2000) -- (77.6000,124.4000) -- (76.2000,125.8000) -- (75.0000,127.3000) -- (74.0000,128.8000) -- (73.1000,130.5000) -- (72.4000,132.2000) -- (72.0000,134.0000) -- (71.8000,135.9000) -- (71.7000,137.7000) -- (71.6000,139.5000) -- (71.6000,141.4000) -- (71.6000,143.2000) -- (71.6000,145.0000) -- (71.6000,146.8000) -- (71.6000,148.7000) -- (71.6000,150.5000) -- (71.6000,152.4000) -- (71.6000,154.2000) -- (71.6000,156.1000) -- (71.6000,157.9000) -- (71.6000,159.8000) -- (71.6000,161.6000) -- (71.6000,163.5000) -- (71.6000,165.3000) -- (71.6000,167.2000) -- (71.6000,169.0000) -- (71.6000,170.8000) -- (71.6000,172.7000) -- (71.6000,174.5000) -- (71.6000,176.4000) -- (71.6000,178.2000) -- (71.8000,180.1000) -- (72.2000,182.0000) -- (72.7000,183.8000) -- (73.5000,185.6000) -- (74.4000,187.3000) -- (75.5000,188.9000) -- (76.7000,190.4000) -- (78.0000,191.8000) -- (79.5000,193.1000) -- (81.1000,194.3000) -- (82.8000,195.4000) -- (84.6000,196.3000) -- (86.4000,197.1000) -- (88.3000,197.8000) -- (90.3000,198.3000) -- (92.2000,198.7000) -- (94.2000,199.0000) -- (96.2000,199.1000) -- (98.2000,199.1000) -- (100.2000,198.9000) -- (102.2000,198.6000) -- (104.2000,198.2000) -- (106.1000,197.6000) -- (107.9000,196.9000) -- (109.7000,196.0000) -- (111.5000,195.1000) -- (113.1000,194.0000) -- (114.7000,192.7000) -- (116.1000,191.4000) -- (117.4000,189.9000) -- (118.5000,188.4000) -- (119.4000,186.7000) -- (120.0000,185.0000) -- (120.4000,183.2000) -- (120.6000,181.3000) -- (120.7000,179.4000) -- (120.8000,177.5000) -- (120.8000,175.6000) -- (120.8000,173.8000) -- (120.8000,171.9000) -- (120.8000,170.0000) -- (120.9000,168.2000) -- (120.9000,166.3000) -- (120.9000,164.5000) -- (120.9000,162.6000) -- (120.9000,160.8000) -- (120.9000,158.9000) -- (120.9000,157.1000) -- (120.9000,155.2000) -- (120.9000,153.4000) -- (120.9000,151.5000) -- (120.9000,149.7000) -- (120.9000,147.8000) -- (120.9000,146.0000) -- (120.9000,144.1000) -- (121.2000,142.3000) -- (121.7000,140.4000) -- (121.9000,138.6000) -- (121.9000,136.9000) -- (121.8000,135.1000) -- (121.4000,133.3000) -- (120.9000,131.6000) -- (120.1000,129.9000) -- (119.2000,128.2000) -- (118.1000,126.7000) -- (116.9000,125.2000) -- (115.5000,123.9000) -- (114.0000,122.7000) -- (112.3000,121.6000) -- (110.6000,120.6000) -- (108.7000,119.8000) -- (106.8000,119.2000) -- (104.9000,118.7000) -- (102.8000,118.4000) -- (100.8000,118.2000) -- (98.8000,118.2000) -- (96.7000,118.4000) -- (94.7000,118.7000) -- (92.8000,119.2000) -- (90.9000,119.8000) -- (89.0000,120.6000) -- (87.3000,121.6000) -- (85.6000,122.6000) -- (84.1000,123.9000) -- (82.7000,125.2000) -- (81.5000,126.7000) -- (80.4000,128.2000) -- (79.5000,129.9000) -- (78.7000,131.6000) -- (78.3000,133.4000) -- (78.0000,135.2000) -- (77.9000,137.1000) -- (77.9000,138.9000) -- (77.8000,140.7000) -- (77.8000,142.5000) -- (77.8000,144.4000) -- (77.8000,146.2000) -- (77.8000,148.0000) -- (77.8000,149.9000) -- (77.8000,151.7000) -- (77.8000,153.6000) -- (77.8000,155.4000) -- (77.8000,157.3000) -- (77.8000,159.1000) -- (77.8000,161.0000) -- (77.8000,162.8000) -- (77.8000,164.6000) -- (77.8000,166.5000) -- (77.8000,168.3000) -- (77.8000,170.2000) -- (77.8000,172.0000) -- (77.8000,173.9000) -- (77.8000,175.7000) -- (77.8000,177.6000) -- (78.0000,179.5000) -- (78.4000,181.3000) -- (78.9000,183.1000) -- (79.7000,184.9000) -- (80.6000,186.6000) -- (81.7000,188.2000) -- (82.9000,189.7000) -- (84.2000,191.1000) -- (85.7000,192.5000) -- (87.3000,193.6000) -- (89.0000,194.7000) -- (90.8000,195.6000) -- (92.6000,196.4000) -- (94.5000,197.1000) -- (96.5000,197.6000) -- (98.5000,198.0000) -- (100.5000,198.2000) -- (102.5000,198.3000) -- (104.5000,198.3000) -- (106.5000,198.1000) -- (108.4000,197.7000) -- (110.4000,197.3000) -- (112.3000,196.7000) -- (114.1000,195.9000) -- (115.9000,195.0000) -- (117.6000,194.0000) -- (119.2000,192.9000) -- (120.8000,191.6000) -- (122.1000,190.2000) -- (123.4000,188.7000) -- (124.5000,187.1000) -- (125.3000,185.5000) -- (125.8000,183.7000) -- (126.1000,181.8000) -- (126.3000,180.0000) -- (126.4000,178.1000) -- (126.5000,176.2000) -- (126.5000,174.3000) -- (126.5000,172.4000) -- (126.5000,170.6000) -- (126.5000,168.7000) -- (126.5000,166.9000) -- (126.5000,165.0000) -- (126.5000,163.2000) -- (126.6000,161.3000) -- (126.6000,159.5000) -- (126.6000,157.6000) -- (126.6000,155.8000) -- (126.6000,153.9000) -- (126.6000,152.1000) -- (126.5000,150.2000) -- (126.5000,148.4000) -- (126.5000,146.5000) -- (126.5000,144.7000) -- (126.6000,142.8000) -- (127.1000,141.0000) -- (127.5000,139.2000) -- (127.6000,137.4000) -- (127.6000,135.6000) -- (127.3000,133.8000) -- (126.9000,132.0000) -- (126.3000,130.3000) -- (125.5000,128.6000) -- (124.5000,127.0000) -- (123.3000,125.5000) -- (122.0000,124.1000) -- (120.5000,122.8000) -- (118.9000,121.7000) -- (117.2000,120.7000) -- (115.4000,119.8000) -- (113.5000,119.1000) -- (111.6000,118.5000) -- (109.6000,118.1000) -- (107.6000,117.9000) -- (105.6000,117.8000) -- (103.5000,117.9000) -- (101.5000,118.2000) -- (99.5000,118.6000) -- (97.6000,119.2000) -- (95.7000,119.9000) -- (93.9000,120.8000) -- (92.2000,121.8000) -- (90.7000,123.0000) -- (89.2000,124.3000) -- (87.9000,125.7000) -- (86.8000,127.2000) -- (85.8000,128.8000) -- (85.0000,130.5000) -- (84.5000,132.3000) -- (84.1000,134.1000) -- (84.0000,136.0000) -- (83.9000,137.8000) -- (83.9000,139.6000) -- (83.9000,141.4000) -- (83.9000,143.3000) -- (83.8000,145.1000) -- (83.8000,146.9000) -- (83.8000,148.8000) -- (83.8000,150.6000) -- (83.8000,152.5000) -- (83.8000,154.3000) -- (83.8000,156.2000) -- (83.8000,158.0000) -- (83.8000,159.9000) -- (83.8000,161.7000) -- (83.8000,163.5000) -- (83.8000,165.4000) -- (83.8000,167.2000) -- (83.8000,169.1000) -- (83.8000,170.9000) -- (83.8000,172.8000) -- (83.8000,174.6000) -- (83.8000,176.5000) -- (83.9000,178.3000) -- (84.2000,180.2000) -- (84.8000,182.1000) -- (85.5000,183.8000) -- (86.3000,185.6000) -- (87.3000,187.2000) -- (88.5000,188.7000) -- (89.9000,190.2000) -- (91.3000,191.5000) -- (92.9000,192.7000) -- (94.5000,193.8000) -- (96.3000,194.8000) -- (98.1000,195.6000) -- (100.0000,196.3000) -- (102.0000,196.9000) -- (103.9000,197.3000) -- (105.9000,197.6000) -- (107.9000,197.7000) -- (109.9000,197.7000) -- (111.9000,197.5000) -- (113.9000,197.2000) -- (115.8000,196.8000) -- (117.7000,196.2000) -- (119.6000,195.5000) -- (121.4000,194.6000) -- (123.1000,193.6000) -- (124.8000,192.5000) -- (126.3000,191.3000) -- (127.7000,189.9000) -- (129.0000,188.4000) -- (130.1000,186.9000) -- (130.9000,185.2000) -- (131.5000,183.4000) -- (131.9000,181.6000) -- (132.1000,179.7000) -- (132.2000,177.8000) -- (132.3000,176.0000) -- (132.3000,174.1000) -- (132.3000,172.2000) -- (132.3000,170.3000) -- (132.3000,168.5000) -- (132.3000,166.6000) -- (132.3000,164.8000) -- (132.3000,162.9000) -- (132.3000,161.1000) -- (132.3000,159.2000) -- (132.3000,157.4000) -- (132.3000,155.5000) -- (132.3000,153.7000) -- (132.3000,151.8000) -- (132.3000,150.0000) -- (132.3000,148.1000) -- (132.3000,146.3000) -- (132.3000,144.4000) -- (132.4000,142.6000) -- (132.9000,140.7000) -- (133.2000,138.9000) -- (133.4000,137.1000) -- (133.4000,135.3000) -- (133.2000,133.5000) -- (132.8000,131.8000) -- (132.2000,130.0000) -- (131.3000,128.4000) -- (130.3000,126.8000) -- (129.1000,125.3000) -- (127.8000,123.9000) -- (126.3000,122.7000) -- (124.7000,121.6000) -- (122.9000,120.6000) -- (121.1000,119.8000) -- (119.2000,119.2000) -- (117.2000,118.7000) -- (115.2000,118.4000) -- (113.1000,118.3000) -- (111.1000,118.3000) -- (109.1000,118.6000) -- (107.1000,119.0000) -- (105.2000,119.6000) -- (103.3000,120.3000) -- (101.5000,121.2000) -- (99.9000,122.3000) -- (98.3000,123.5000) -- (96.9000,124.8000) -- (95.7000,126.3000) -- (94.6000,127.9000) -- (93.8000,129.5000) -- (93.1000,131.3000) -- (92.7000,133.1000) -- (92.6000,134.9000) -- (92.5000,136.8000) -- (92.4000,138.6000) -- (92.4000,140.4000) -- (92.4000,142.2000) -- (92.4000,144.1000) -- (92.4000,145.9000) -- (92.4000,147.7000) -- (92.4000,149.6000) -- (92.4000,151.4000) -- (92.4000,153.3000) -- (92.4000,155.1000) -- (92.4000,157.0000) -- (92.4000,158.8000) -- (92.4000,160.7000) -- (92.4000,162.5000) -- (92.4000,164.4000) -- (92.4000,166.2000) -- (92.4000,168.1000) -- (92.4000,169.9000) -- (92.4000,171.8000) -- (92.4000,173.6000) -- (92.4000,175.4000) -- (92.4000,177.3000) -- (92.7000,179.2000) -- (93.2000,181.0000) -- (93.8000,182.8000) -- (94.7000,184.5000) -- (95.7000,186.2000) -- (96.9000,187.8000) -- (98.2000,189.2000) -- (99.6000,190.6000) -- (101.1000,191.8000) -- (102.8000,192.9000) -- (104.5000,193.9000) -- (106.4000,194.8000) -- (108.2000,195.5000) -- (110.2000,196.0000) -- (112.1000,196.5000) -- (114.1000,196.8000) -- (116.1000,196.9000) -- (118.1000,196.9000) -- (120.1000,196.8000) -- (122.1000,196.5000) -- (124.1000,196.1000) -- (126.0000,195.5000) -- (127.8000,194.8000) -- (129.7000,194.0000) -- (131.4000,193.0000) -- (133.0000,191.9000) -- (134.6000,190.6000) -- (136.0000,189.3000) -- (137.3000,187.8000) -- (138.4000,186.3000) -- (139.3000,184.6000) -- (139.9000,182.8000) -- (140.3000,181.0000) -- (140.5000,179.2000) -- (140.6000,177.3000) -- (140.7000,175.4000) -- (140.7000,173.5000) -- (140.7000,171.6000) -- (140.7000,169.8000) -- (140.8000,167.9000) -- (140.8000,166.0000) -- (140.8000,164.2000) -- (140.8000,162.3000) -- (140.8000,160.5000) -- (140.8000,158.6000) -- (140.8000,156.8000) -- (140.8000,154.9000) -- (140.8000,153.1000) -- (140.8000,151.3000) -- (140.8000,149.4000) -- (140.8000,147.6000) -- (140.8000,145.7000) -- (140.8000,143.9000) -- (140.8000,142.0000) -- (141.3000,140.1000) -- (141.7000,138.3000) -- (142.0000,136.5000) -- (142.0000,134.8000) -- (141.8000,133.0000) -- (141.4000,131.2000) -- (140.8000,129.5000) -- (140.0000,127.8000) -- (139.0000,126.2000) -- (137.9000,124.7000) -- (136.5000,123.3000) -- (135.1000,122.0000) -- (133.5000,120.9000) -- (131.7000,119.9000) -- (129.9000,119.1000) -- (128.0000,118.4000) -- (126.1000,117.9000) -- (124.0000,117.6000) -- (122.0000,117.4000) -- (120.0000,117.4000) -- (117.9000,117.6000) -- (115.9000,118.0000) -- (114.0000,118.5000) -- (112.1000,119.2000) -- (110.3000,120.1000) -- (108.6000,121.1000) -- (107.0000,122.2000) -- (105.6000,123.5000) -- (104.3000,124.9000) -- (103.2000,126.5000) -- (102.2000,128.1000) -- (101.5000,129.8000) -- (101.0000,131.6000) -- (100.8000,133.5000) -- (100.6000,135.3000) -- (100.6000,137.1000) -- (100.6000,138.9000) -- (100.6000,140.8000) -- (100.5000,142.6000) -- (100.5000,144.4000) -- (100.5000,146.3000) -- (100.5000,148.1000) -- (100.5000,149.9000) -- (100.5000,151.8000) -- (100.5000,153.6000) -- (100.5000,155.5000) -- (100.5000,157.3000) -- (100.5000,159.2000) -- (100.5000,161.0000) -- (100.5000,162.9000) -- (100.5000,164.7000) -- (100.5000,166.6000) -- (100.5000,168.4000) -- (100.5000,170.3000) -- (100.5000,172.1000) -- (100.5000,174.0000) -- (100.5000,175.8000) -- (100.7000,177.7000) -- (101.1000,179.6000) -- (101.7000,181.4000) -- (102.5000,183.1000) -- (103.4000,184.8000) -- (104.5000,186.4000) -- (105.8000,187.9000) -- (107.1000,189.3000) -- (108.6000,190.6000) -- (110.2000,191.8000) -- (111.9000,192.8000) -- (113.7000,193.7000) -- (115.6000,194.5000) -- (117.5000,195.2000) -- (119.4000,195.7000) -- (121.4000,196.0000) -- (123.4000,196.2000) -- (125.4000,196.3000) -- (127.4000,196.2000) -- (129.4000,196.0000) -- (131.4000,195.7000) -- (133.3000,195.2000) -- (135.2000,194.6000) -- (137.1000,193.8000) -- (138.8000,192.9000) -- (140.5000,191.9000) -- (142.1000,190.7000) -- (143.6000,189.4000) -- (145.0000,188.0000) -- (146.2000,186.5000) -- (147.3000,184.9000) -- (148.0000,183.2000) -- (148.5000,181.4000) -- (148.8000,179.6000) -- (149.0000,177.7000) -- (149.1000,175.8000) -- (149.1000,173.9000) -- (149.2000,172.1000) -- (149.2000,170.2000) -- (149.2000,168.3000) -- (149.2000,166.5000) -- (149.2000,164.6000) -- (149.2000,162.8000) -- (149.2000,160.9000) -- (149.2000,159.1000) -- (149.2000,157.2000) -- (149.2000,155.4000) -- (149.2000,153.5000) -- (149.2000,151.7000) -- (149.2000,149.8000) -- (149.2000,148.0000) -- (149.2000,146.1000) -- (149.2000,144.3000) -- (149.2000,142.4000) -- (149.4000,140.6000) -- (149.9000,138.7000) -- (150.3000,136.9000) -- (150.4000,135.1000) -- (150.4000,133.4000) -- (150.1000,131.6000) -- (149.7000,129.8000) -- (149.0000,128.1000) -- (148.1000,126.5000) -- (147.1000,124.9000) -- (145.9000,123.4000) -- (144.5000,122.1000) -- (142.9000,120.8000) -- (141.3000,119.8000) -- (139.5000,118.8000) -- (137.7000,118.0000) -- (135.8000,117.4000) -- (133.8000,117.0000) -- (131.8000,116.7000) -- (129.7000,116.6000) -- (127.7000,116.7000) -- (125.7000,117.0000) -- (123.7000,117.4000) -- (121.7000,118.0000) -- (119.9000,118.7000) -- (118.1000,119.6000) -- (116.4000,120.7000) -- (114.9000,121.9000) -- (113.5000,123.2000) -- (112.3000,124.7000) -- (111.2000,126.3000) -- (110.3000,127.9000) -- (109.7000,129.7000) -- (109.3000,131.5000) -- (109.1000,133.3000) -- (109.0000,135.2000) -- (109.0000,137.0000) -- (109.0000,138.8000) -- (108.9000,140.6000) -- (108.9000,142.5000) -- (108.9000,144.3000) -- (108.9000,146.1000) -- (108.9000,148.0000) -- (108.9000,149.8000) -- (108.9000,151.7000) -- (108.9000,153.5000) -- (108.9000,155.4000) -- (108.9000,157.2000) -- (108.9000,159.1000) -- (108.9000,160.9000) -- (108.9000,162.8000) -- (108.9000,164.6000) -- (108.9000,166.5000) -- (108.9000,168.3000) -- (108.9000,170.1000) -- (108.9000,172.0000) -- (108.9000,173.8000) -- (109.0000,175.7000) -- (109.2000,177.6000) -- (109.7000,179.4000) -- (110.4000,181.2000) -- (111.2000,183.0000) -- (112.2000,184.6000) -- (113.3000,186.2000) -- (114.6000,187.7000) -- (116.0000,189.0000) -- (117.5000,190.3000) -- (119.2000,191.4000) -- (120.9000,192.4000) -- (122.7000,193.3000) -- (124.6000,194.0000) -- (126.5000,194.6000) -- (128.5000,195.1000) -- (130.5000,195.4000) -- (132.5000,195.6000) -- (134.5000,195.6000) -- (136.5000,195.5000) -- (138.5000,195.2000) -- (140.4000,194.8000) -- (142.3000,194.3000) -- (144.2000,193.6000) -- (146.0000,192.8000) -- (147.8000,191.9000) -- (149.5000,190.8000) -- (151.0000,189.6000) -- (152.5000,188.3000) -- (153.8000,186.9000) -- (155.0000,185.3000) -- (156.0000,183.7000) -- (156.6000,182.0000) -- (157.1000,180.2000) -- (157.3000,178.3000) -- (157.5000,176.4000) -- (157.6000,174.5000) -- (157.6000,172.7000) -- (157.6000,170.8000) -- (157.6000,168.9000) -- (157.6000,167.0000) -- (157.6000,165.2000) -- (157.6000,163.3000) -- (157.7000,161.5000) -- (157.7000,159.6000) -- (157.7000,157.8000) -- (157.7000,155.9000) -- (157.7000,154.1000) -- (157.7000,152.2000) -- (157.7000,150.4000) -- (157.7000,148.6000) -- (157.7000,146.7000) -- (157.7000,144.9000) -- (157.6000,143.0000) -- (157.6000,141.2000) -- (158.0000,139.3000) -- (158.5000,137.4000) -- (158.8000,135.6000);



  \end{scope}
  \path[fill=cd9d9d9,dash pattern=on 1.12pt off 1.12pt,even odd rule,line cap=round,line width=0.281pt,miter limit=4.00,rounded corners=0.0000cm] (-617.3423,257.0234) rectangle (-74.5817,379.8000);



  \begin{scope}[cm={{1.04177,0.0,0.0,1.04177,(-342.63646,42.1635)}},draw=ca0a0a4,dash pattern=on 1.73pt off 1.73pt,line cap=round,line join=round,line width=0.288pt,miter limit=4.00]
    \path[draw,dash pattern=on 1.73pt off 1.73pt,line width=0.288pt,miter limit=4.00] (56.5000,232.5000) -- (251.5000,232.5000);



  \end{scope}
  \begin{scope}[draw=blue,line cap=rect,line join=bevel,line width=0.800pt]
  \end{scope}
  \begin{scope}[scale=1.006,draw=blue,line cap=rect,line join=bevel,line width=0.800pt]
  \end{scope}
  \begin{scope}[scale=1.006,draw=blue,line cap=rect,line join=bevel,line width=0.800pt]
  \end{scope}
  \begin{scope}[cm={{1.00588,0.0,0.0,1.00588,(39.2294,93.5471)}},draw=blue,line cap=rect,line join=bevel,line width=0.800pt]
  \end{scope}
  \begin{scope}[cm={{1.00588,0.0,0.0,1.00588,(39.2294,93.5471)}},draw=blue,line cap=rect,line join=bevel,line width=0.800pt]
  \end{scope}
  \begin{scope}[cm={{1.00588,0.0,0.0,1.00588,(39.2294,93.5471)}},draw=blue,line cap=rect,line join=bevel,line width=0.800pt]
  \end{scope}
  \begin{scope}[cm={{1.00588,0.0,0.0,1.00588,(39.2294,93.5471)}},draw=blue,line cap=rect,line join=bevel,line width=0.800pt]
  \end{scope}
  \begin{scope}[cm={{1.00588,0.0,0.0,1.00588,(39.2294,93.5471)}},draw=blue,line cap=rect,line join=bevel,line width=0.800pt]
  \end{scope}
  \begin{scope}[cm={{1.00588,0.0,0.0,1.00588,(39.2294,93.5471)}},draw=blue,line cap=rect,line join=bevel,line width=0.800pt]
  \end{scope}
  \begin{scope}[scale=1.006,draw=blue,line cap=rect,line join=bevel,line width=0.800pt]
  \end{scope}
  \begin{scope}[scale=1.006,draw=blue,line cap=rect,line join=bevel,line width=0.800pt]
  \end{scope}
  \begin{scope}[cm={{1.00588,0.0,0.0,1.00588,(39.2294,68.4)}},draw=blue,line cap=rect,line join=bevel,line width=0.800pt]
  \end{scope}
  \begin{scope}[cm={{1.00588,0.0,0.0,1.00588,(39.2294,68.4)}},draw=blue,line cap=rect,line join=bevel,line width=0.800pt]
  \end{scope}
  \begin{scope}[cm={{1.00588,0.0,0.0,1.00588,(39.2294,68.4)}},draw=blue,line cap=rect,line join=bevel,line width=0.800pt]
  \end{scope}
  \begin{scope}[cm={{1.00588,0.0,0.0,1.00588,(39.2294,68.4)}},draw=blue,line cap=rect,line join=bevel,line width=0.800pt]
  \end{scope}
  \begin{scope}[cm={{1.00588,0.0,0.0,1.00588,(39.2294,68.4)}},draw=blue,line cap=rect,line join=bevel,line width=0.800pt]
  \end{scope}
  \begin{scope}[cm={{1.00588,0.0,0.0,1.00588,(39.2294,68.4)}},draw=blue,line cap=rect,line join=bevel,line width=0.800pt]
  \end{scope}
  \begin{scope}[scale=1.006,draw=blue,line cap=rect,line join=bevel,line width=0.800pt]
  \end{scope}
  \begin{scope}[scale=1.006,draw=blue,line cap=rect,line join=bevel,line width=0.800pt]
  \end{scope}
  \begin{scope}[cm={{1.00588,0.0,0.0,1.00588,(39.2294,43.2529)}},draw=blue,line cap=rect,line join=bevel,line width=0.800pt]
  \end{scope}
  \begin{scope}[cm={{1.00588,0.0,0.0,1.00588,(39.2294,43.2529)}},draw=blue,line cap=rect,line join=bevel,line width=0.800pt]
  \end{scope}
  \begin{scope}[cm={{1.00588,0.0,0.0,1.00588,(39.2294,43.2529)}},draw=blue,line cap=rect,line join=bevel,line width=0.800pt]
  \end{scope}
  \begin{scope}[cm={{1.00588,0.0,0.0,1.00588,(39.2294,43.2529)}},draw=blue,line cap=rect,line join=bevel,line width=0.800pt]
  \end{scope}
  \begin{scope}[cm={{1.00588,0.0,0.0,1.00588,(39.2294,43.2529)}},draw=blue,line cap=rect,line join=bevel,line width=0.800pt]
  \end{scope}
  \begin{scope}[cm={{1.00588,0.0,0.0,1.00588,(39.2294,43.2529)}},draw=blue,line cap=rect,line join=bevel,line width=0.800pt]
  \end{scope}
  \begin{scope}[scale=1.006,draw=blue,line cap=rect,line join=bevel,line width=0.800pt]
  \end{scope}
  \begin{scope}[scale=1.006,draw=blue,line cap=rect,line join=bevel,line width=0.800pt]
  \end{scope}
  \begin{scope}[cm={{1.00588,0.0,0.0,1.00588,(53.3118,110.647)}},draw=blue,line cap=rect,line join=bevel,line width=0.800pt]
  \end{scope}
  \begin{scope}[cm={{1.00588,0.0,0.0,1.00588,(53.3118,110.647)}},draw=blue,line cap=rect,line join=bevel,line width=0.800pt]
  \end{scope}
  \begin{scope}[cm={{1.00588,0.0,0.0,1.00588,(53.3118,110.647)}},draw=blue,line cap=rect,line join=bevel,line width=0.800pt]
  \end{scope}
  \begin{scope}[cm={{1.00588,0.0,0.0,1.00588,(53.3118,110.647)}},draw=blue,line cap=rect,line join=bevel,line width=0.800pt]
  \end{scope}
  \begin{scope}[cm={{1.00588,0.0,0.0,1.00588,(53.3118,110.647)}},draw=blue,line cap=rect,line join=bevel,line width=0.800pt]
  \end{scope}
  \begin{scope}[cm={{1.00588,0.0,0.0,1.00588,(53.3118,110.647)}},draw=blue,line cap=rect,line join=bevel,line width=0.800pt]
  \end{scope}
  \begin{scope}[scale=1.006,draw=blue,line cap=rect,line join=bevel,line width=0.800pt]
  \end{scope}
  \begin{scope}[scale=1.006,draw=blue,line cap=rect,line join=bevel,line width=0.800pt]
  \end{scope}
  \begin{scope}[cm={{1.00588,0.0,0.0,1.00588,(79.4647,110.647)}},draw=blue,line cap=rect,line join=bevel,line width=0.800pt]
  \end{scope}
  \begin{scope}[cm={{1.00588,0.0,0.0,1.00588,(79.4647,110.647)}},draw=blue,line cap=rect,line join=bevel,line width=0.800pt]
  \end{scope}
  \begin{scope}[cm={{1.00588,0.0,0.0,1.00588,(79.4647,110.647)}},draw=blue,line cap=rect,line join=bevel,line width=0.800pt]
  \end{scope}
  \begin{scope}[cm={{1.00588,0.0,0.0,1.00588,(79.4647,110.647)}},draw=blue,line cap=rect,line join=bevel,line width=0.800pt]
  \end{scope}
  \begin{scope}[cm={{1.00588,0.0,0.0,1.00588,(79.4647,110.647)}},draw=blue,line cap=rect,line join=bevel,line width=0.800pt]
  \end{scope}
  \begin{scope}[cm={{1.00588,0.0,0.0,1.00588,(79.4647,110.647)}},draw=blue,line cap=rect,line join=bevel,line width=0.800pt]
  \end{scope}
  \begin{scope}[scale=1.006,draw=blue,line cap=rect,line join=bevel,line width=0.800pt]
  \end{scope}
  \begin{scope}[scale=1.006,draw=blue,line cap=rect,line join=bevel,line width=0.800pt]
  \end{scope}
  \begin{scope}[cm={{1.00588,0.0,0.0,1.00588,(105.618,110.647)}},draw=blue,line cap=rect,line join=bevel,line width=0.800pt]
  \end{scope}
  \begin{scope}[cm={{1.00588,0.0,0.0,1.00588,(105.618,110.647)}},draw=blue,line cap=rect,line join=bevel,line width=0.800pt]
  \end{scope}
  \begin{scope}[cm={{1.00588,0.0,0.0,1.00588,(105.618,110.647)}},draw=blue,line cap=rect,line join=bevel,line width=0.800pt]
  \end{scope}
  \begin{scope}[cm={{1.00588,0.0,0.0,1.00588,(105.618,110.647)}},draw=blue,line cap=rect,line join=bevel,line width=0.800pt]
  \end{scope}
  \begin{scope}[cm={{1.00588,0.0,0.0,1.00588,(105.618,110.647)}},draw=blue,line cap=rect,line join=bevel,line width=0.800pt]
  \end{scope}
  \begin{scope}[cm={{1.00588,0.0,0.0,1.00588,(105.618,110.647)}},draw=blue,line cap=rect,line join=bevel,line width=0.800pt]
  \end{scope}
  \begin{scope}[scale=1.006,draw=blue,line cap=rect,line join=bevel,line width=0.800pt]
  \end{scope}
  \begin{scope}[scale=1.006,draw=blue,line cap=rect,line join=bevel,line width=0.800pt]
  \end{scope}
  \begin{scope}[cm={{1.00588,0.0,0.0,1.00588,(132.274,110.647)}},draw=blue,line cap=rect,line join=bevel,line width=0.800pt]
  \end{scope}
  \begin{scope}[cm={{1.00588,0.0,0.0,1.00588,(132.274,110.647)}},draw=blue,line cap=rect,line join=bevel,line width=0.800pt]
  \end{scope}
  \begin{scope}[cm={{1.00588,0.0,0.0,1.00588,(132.274,110.647)}},draw=blue,line cap=rect,line join=bevel,line width=0.800pt]
  \end{scope}
  \begin{scope}[cm={{1.00588,0.0,0.0,1.00588,(132.274,110.647)}},draw=blue,line cap=rect,line join=bevel,line width=0.800pt]
  \end{scope}
  \begin{scope}[cm={{1.00588,0.0,0.0,1.00588,(132.274,110.647)}},draw=blue,line cap=rect,line join=bevel,line width=0.800pt]
  \end{scope}
  \begin{scope}[cm={{1.00588,0.0,0.0,1.00588,(132.274,110.647)}},draw=blue,line cap=rect,line join=bevel,line width=0.800pt]
  \end{scope}
  \begin{scope}[scale=1.006,draw=blue,line cap=rect,line join=bevel,line width=0.800pt]
  \end{scope}
  \begin{scope}[scale=1.006,draw=blue,line cap=rect,line join=bevel,line width=0.800pt]
  \end{scope}
  \begin{scope}[scale=1.006,draw=blue,line cap=rect,line join=bevel,line width=0.800pt]
  \end{scope}
  \begin{scope}[scale=1.006,draw=blue,line cap=rect,line join=bevel,line width=0.800pt]
  \end{scope}
  \begin{scope}[scale=1.006,draw=blue,line cap=rect,line join=bevel,line width=0.800pt]
  \end{scope}
  \begin{scope}[scale=1.006,draw=blue,line cap=rect,line join=bevel,line width=0.800pt]
  \end{scope}
  \begin{scope}[cm={{1.00588,0.0,0.0,1.00588,(128.753,29.1706)}},draw=blue,line cap=rect,line join=bevel,line width=0.800pt]
  \end{scope}
  \begin{scope}[cm={{1.00588,0.0,0.0,1.00588,(128.753,29.1706)}},draw=blue,line cap=rect,line join=bevel,line width=0.800pt]
  \end{scope}
  \begin{scope}[cm={{1.00588,0.0,0.0,1.00588,(128.753,29.1706)}},draw=blue,line cap=rect,line join=bevel,line width=0.800pt]
  \end{scope}
  \begin{scope}[cm={{1.00588,0.0,0.0,1.00588,(128.753,29.1706)}},draw=blue,line cap=rect,line join=bevel,line width=0.800pt]
  \end{scope}
  \begin{scope}[cm={{1.00588,0.0,0.0,1.00588,(128.753,29.1706)}},draw=blue,line cap=rect,line join=bevel,line width=0.800pt]
  \end{scope}
  \begin{scope}[cm={{1.00588,0.0,0.0,1.00588,(128.753,29.1706)}},draw=blue,line cap=rect,line join=bevel,line width=0.800pt]
  \end{scope}
  \begin{scope}[cm={{0.0,-1.00588,1.00588,0.0,(29.1706,189.106)}},draw=blue,line cap=rect,line join=bevel,line width=0.800pt]
  \end{scope}
  \begin{scope}[cm={{0.0,-1.00588,1.00588,0.0,(29.1706,189.106)}},draw=blue,line cap=rect,line join=bevel,line width=0.800pt]
  \end{scope}
  \begin{scope}[cm={{0.0,-1.00588,1.00588,0.0,(29.1706,189.106)}},draw=blue,line cap=rect,line join=bevel,line width=0.800pt]
  \end{scope}
  \begin{scope}[cm={{0.0,-1.00588,1.00588,0.0,(29.1706,189.106)}},draw=blue,line cap=rect,line join=bevel,line width=0.800pt]
  \end{scope}
  \begin{scope}[cm={{0.0,-1.00588,1.00588,0.0,(29.1706,189.106)}},draw=blue,line cap=rect,line join=bevel,line width=0.800pt]
  \end{scope}
  \begin{scope}[cm={{0.0,-1.00588,1.00588,0.0,(281.91864,312.23446)}},draw=blue,line cap=rect,line join=bevel,line width=0.800pt]
    \path[fill=blue] (32.7896,-587.5454) node[above right] (text274) {\rotatebox{90}{Power (W)}};



  \end{scope}
  \begin{scope}[cm={{0.0,-1.00588,1.00588,0.0,(29.1706,189.106)}},draw=blue,line cap=rect,line join=bevel,line width=0.800pt]
  \end{scope}
  \begin{scope}[cm={{1.00588,0.0,0.0,1.00588,(62.3647,28.1647)}},draw=blue,line cap=rect,line join=bevel,line width=0.800pt]
  \end{scope}
  \begin{scope}[cm={{1.00588,0.0,0.0,1.00588,(62.3647,28.1647)}},draw=blue,line cap=rect,line join=bevel,line width=0.800pt]
  \end{scope}
  \begin{scope}[cm={{1.00588,0.0,0.0,1.00588,(62.3647,28.1647)}},draw=blue,line cap=rect,line join=bevel,line width=0.800pt]
  \end{scope}
  \begin{scope}[cm={{1.00588,0.0,0.0,1.00588,(62.3647,28.1647)}},draw=blue,line cap=rect,line join=bevel,line width=0.800pt]
  \end{scope}
  \begin{scope}[cm={{1.00588,0.0,0.0,1.00588,(62.3647,28.1647)}},draw=blue,line cap=rect,line join=bevel,line width=0.800pt]
  \end{scope}
  \begin{scope}[cm={{1.00588,0.0,0.0,1.00588,(62.3647,28.1647)}},draw=blue,line cap=rect,line join=bevel,line width=0.800pt]
  \end{scope}
  \begin{scope}[scale=1.006,draw=blue,line cap=rect,line join=bevel,line width=0.800pt]
  \end{scope}
  \begin{scope}[scale=1.006,draw=blue,line cap=rect,line join=bevel,line width=0.800pt]
  \end{scope}
  \begin{scope}[scale=1.006,draw=blue,line cap=rect,line join=bevel,line width=0.800pt]
  \end{scope}
  \begin{scope}[scale=1.006,draw=blue,line cap=rect,line join=bevel,line width=0.800pt]
  \end{scope}
  \begin{scope}[scale=1.006,draw=blue,line cap=rect,line join=bevel,line width=0.800pt]
  \end{scope}
  \begin{scope}[scale=1.006,draw=blue,line cap=rect,line join=bevel,line width=0.800pt]
  \end{scope}
  \begin{scope}[cm={{1.00588,0.0,0.0,1.00588,(60.3529,36.2118)}},draw=blue,line cap=rect,line join=bevel,line width=0.800pt]
  \end{scope}
  \begin{scope}[cm={{1.00588,0.0,0.0,1.00588,(60.3529,36.2118)}},draw=blue,line cap=rect,line join=bevel,line width=0.800pt]
  \end{scope}
  \begin{scope}[cm={{1.00588,0.0,0.0,1.00588,(60.3529,36.2118)}},draw=blue,line cap=rect,line join=bevel,line width=0.800pt]
  \end{scope}
  \begin{scope}[cm={{1.00588,0.0,0.0,1.00588,(60.3529,36.2118)}},draw=blue,line cap=rect,line join=bevel,line width=0.800pt]
  \end{scope}
  \begin{scope}[cm={{1.00588,0.0,0.0,1.00588,(60.3529,36.2118)}},draw=blue,line cap=rect,line join=bevel,line width=0.800pt]
  \end{scope}
  \begin{scope}[cm={{1.00588,0.0,0.0,1.00588,(60.3529,36.2118)}},draw=blue,line cap=rect,line join=bevel,line width=0.800pt]
  \end{scope}
  \begin{scope}[scale=1.006,draw=blue,line cap=rect,line join=bevel,line width=0.800pt]
  \end{scope}
  \begin{scope}[scale=1.006,draw=blue,line cap=rect,line join=bevel,line width=0.800pt]
  \end{scope}
  \begin{scope}[scale=1.006,draw=blue,line cap=rect,line join=bevel,line width=0.800pt]
  \end{scope}
  \begin{scope}[scale=1.006,draw=blue,line cap=rect,line join=bevel,line width=0.800pt]
  \end{scope}
  \begin{scope}[scale=1.006,draw=blue,line cap=rect,line join=bevel,line width=0.800pt]
  \end{scope}
  \begin{scope}[scale=1.006,draw=blue,line cap=rect,line join=bevel,line width=0.800pt]
  \end{scope}
  \begin{scope}[scale=1.006,draw=blue,line cap=rect,line join=bevel,line width=0.800pt]
  \end{scope}
  \begin{scope}[scale=1.006,draw=blue,line cap=rect,line join=bevel,line width=0.800pt]
  \end{scope}
  \begin{scope}[cm={{1.00588,0.0,0.0,1.00588,(148.871,93.5471)}},draw=blue,line cap=rect,line join=bevel,line width=0.800pt]
  \end{scope}
  \begin{scope}[cm={{1.00588,0.0,0.0,1.00588,(148.871,93.5471)}},draw=blue,line cap=rect,line join=bevel,line width=0.800pt]
  \end{scope}
  \begin{scope}[cm={{1.00588,0.0,0.0,1.00588,(148.871,93.5471)}},draw=blue,line cap=rect,line join=bevel,line width=0.800pt]
  \end{scope}
  \begin{scope}[cm={{1.00588,0.0,0.0,1.00588,(148.871,93.5471)}},draw=blue,line cap=rect,line join=bevel,line width=0.800pt]
  \end{scope}
  \begin{scope}[cm={{1.00588,0.0,0.0,1.00588,(148.871,93.5471)}},draw=blue,line cap=rect,line join=bevel,line width=0.800pt]
  \end{scope}
  \begin{scope}[cm={{1.00588,0.0,0.0,1.00588,(148.871,93.5471)}},draw=blue,line cap=rect,line join=bevel,line width=0.800pt]
  \end{scope}
  \begin{scope}[scale=1.006,draw=blue,line cap=rect,line join=bevel,line width=0.800pt]
  \end{scope}
  \begin{scope}[scale=1.006,draw=blue,line cap=rect,line join=bevel,line width=0.800pt]
  \end{scope}
  \begin{scope}[cm={{1.00588,0.0,0.0,1.00588,(149.876,68.4)}},draw=blue,line cap=rect,line join=bevel,line width=0.800pt]
  \end{scope}
  \begin{scope}[cm={{1.00588,0.0,0.0,1.00588,(149.876,68.4)}},draw=blue,line cap=rect,line join=bevel,line width=0.800pt]
  \end{scope}
  \begin{scope}[cm={{1.00588,0.0,0.0,1.00588,(149.876,68.4)}},draw=blue,line cap=rect,line join=bevel,line width=0.800pt]
  \end{scope}
  \begin{scope}[cm={{1.00588,0.0,0.0,1.00588,(149.876,68.4)}},draw=blue,line cap=rect,line join=bevel,line width=0.800pt]
  \end{scope}
  \begin{scope}[cm={{1.00588,0.0,0.0,1.00588,(149.876,68.4)}},draw=blue,line cap=rect,line join=bevel,line width=0.800pt]
  \end{scope}
  \begin{scope}[cm={{1.00588,0.0,0.0,1.00588,(149.876,68.4)}},draw=blue,line cap=rect,line join=bevel,line width=0.800pt]
  \end{scope}
  \begin{scope}[scale=1.006,draw=blue,line cap=rect,line join=bevel,line width=0.800pt]
  \end{scope}
  \begin{scope}[scale=1.006,draw=blue,line cap=rect,line join=bevel,line width=0.800pt]
  \end{scope}
  \begin{scope}[cm={{1.00588,0.0,0.0,1.00588,(149.876,43.2529)}},draw=blue,line cap=rect,line join=bevel,line width=0.800pt]
  \end{scope}
  \begin{scope}[cm={{1.00588,0.0,0.0,1.00588,(149.876,43.2529)}},draw=blue,line cap=rect,line join=bevel,line width=0.800pt]
  \end{scope}
  \begin{scope}[cm={{1.00588,0.0,0.0,1.00588,(149.876,43.2529)}},draw=blue,line cap=rect,line join=bevel,line width=0.800pt]
  \end{scope}
  \begin{scope}[cm={{1.00588,0.0,0.0,1.00588,(149.876,43.2529)}},draw=blue,line cap=rect,line join=bevel,line width=0.800pt]
  \end{scope}
  \begin{scope}[cm={{1.00588,0.0,0.0,1.00588,(149.876,43.2529)}},draw=blue,line cap=rect,line join=bevel,line width=0.800pt]
  \end{scope}
  \begin{scope}[cm={{1.00588,0.0,0.0,1.00588,(149.876,43.2529)}},draw=blue,line cap=rect,line join=bevel,line width=0.800pt]
  \end{scope}
  \begin{scope}[scale=1.006,draw=blue,line cap=rect,line join=bevel,line width=0.800pt]
  \end{scope}
  \begin{scope}[scale=1.006,draw=blue,line cap=rect,line join=bevel,line width=0.800pt]
  \end{scope}
  \begin{scope}[cm={{1.00588,0.0,0.0,1.00588,(162.953,110.647)}},draw=blue,line cap=rect,line join=bevel,line width=0.800pt]
  \end{scope}
  \begin{scope}[cm={{1.00588,0.0,0.0,1.00588,(162.953,110.647)}},draw=blue,line cap=rect,line join=bevel,line width=0.800pt]
  \end{scope}
  \begin{scope}[cm={{1.00588,0.0,0.0,1.00588,(162.953,110.647)}},draw=blue,line cap=rect,line join=bevel,line width=0.800pt]
  \end{scope}
  \begin{scope}[cm={{1.00588,0.0,0.0,1.00588,(162.953,110.647)}},draw=blue,line cap=rect,line join=bevel,line width=0.800pt]
  \end{scope}
  \begin{scope}[cm={{1.00588,0.0,0.0,1.00588,(162.953,110.647)}},draw=blue,line cap=rect,line join=bevel,line width=0.800pt]
  \end{scope}
  \begin{scope}[cm={{1.00588,0.0,0.0,1.00588,(162.953,110.647)}},draw=blue,line cap=rect,line join=bevel,line width=0.800pt]
  \end{scope}
  \begin{scope}[scale=1.006,draw=blue,line cap=rect,line join=bevel,line width=0.800pt]
  \end{scope}
  \begin{scope}[scale=1.006,draw=blue,line cap=rect,line join=bevel,line width=0.800pt]
  \end{scope}
  \begin{scope}[cm={{1.00588,0.0,0.0,1.00588,(189.106,110.647)}},draw=blue,line cap=rect,line join=bevel,line width=0.800pt]
  \end{scope}
  \begin{scope}[cm={{1.00588,0.0,0.0,1.00588,(189.106,110.647)}},draw=blue,line cap=rect,line join=bevel,line width=0.800pt]
  \end{scope}
  \begin{scope}[cm={{1.00588,0.0,0.0,1.00588,(189.106,110.647)}},draw=blue,line cap=rect,line join=bevel,line width=0.800pt]
  \end{scope}
  \begin{scope}[cm={{1.00588,0.0,0.0,1.00588,(189.106,110.647)}},draw=blue,line cap=rect,line join=bevel,line width=0.800pt]
  \end{scope}
  \begin{scope}[cm={{1.00588,0.0,0.0,1.00588,(189.106,110.647)}},draw=blue,line cap=rect,line join=bevel,line width=0.800pt]
  \end{scope}
  \begin{scope}[cm={{1.00588,0.0,0.0,1.00588,(189.106,110.647)}},draw=blue,line cap=rect,line join=bevel,line width=0.800pt]
  \end{scope}
  \begin{scope}[scale=1.006,draw=blue,line cap=rect,line join=bevel,line width=0.800pt]
  \end{scope}
  \begin{scope}[scale=1.006,draw=blue,line cap=rect,line join=bevel,line width=0.800pt]
  \end{scope}
  \begin{scope}[cm={{1.00588,0.0,0.0,1.00588,(215.259,110.647)}},draw=blue,line cap=rect,line join=bevel,line width=0.800pt]
  \end{scope}
  \begin{scope}[cm={{1.00588,0.0,0.0,1.00588,(215.259,110.647)}},draw=blue,line cap=rect,line join=bevel,line width=0.800pt]
  \end{scope}
  \begin{scope}[cm={{1.00588,0.0,0.0,1.00588,(215.259,110.647)}},draw=blue,line cap=rect,line join=bevel,line width=0.800pt]
  \end{scope}
  \begin{scope}[cm={{1.00588,0.0,0.0,1.00588,(215.259,110.647)}},draw=blue,line cap=rect,line join=bevel,line width=0.800pt]
  \end{scope}
  \begin{scope}[cm={{1.00588,0.0,0.0,1.00588,(215.259,110.647)}},draw=blue,line cap=rect,line join=bevel,line width=0.800pt]
  \end{scope}
  \begin{scope}[cm={{1.00588,0.0,0.0,1.00588,(215.259,110.647)}},draw=blue,line cap=rect,line join=bevel,line width=0.800pt]
  \end{scope}
  \begin{scope}[scale=1.006,draw=blue,line cap=rect,line join=bevel,line width=0.800pt]
  \end{scope}
  \begin{scope}[scale=1.006,draw=blue,line cap=rect,line join=bevel,line width=0.800pt]
  \end{scope}
  \begin{scope}[cm={{1.00588,0.0,0.0,1.00588,(241.915,110.647)}},draw=blue,line cap=rect,line join=bevel,line width=0.800pt]
  \end{scope}
  \begin{scope}[cm={{1.00588,0.0,0.0,1.00588,(241.915,110.647)}},draw=blue,line cap=rect,line join=bevel,line width=0.800pt]
  \end{scope}
  \begin{scope}[cm={{1.00588,0.0,0.0,1.00588,(241.915,110.647)}},draw=blue,line cap=rect,line join=bevel,line width=0.800pt]
  \end{scope}
  \begin{scope}[cm={{1.00588,0.0,0.0,1.00588,(241.915,110.647)}},draw=blue,line cap=rect,line join=bevel,line width=0.800pt]
  \end{scope}
  \begin{scope}[cm={{1.00588,0.0,0.0,1.00588,(241.915,110.647)}},draw=blue,line cap=rect,line join=bevel,line width=0.800pt]
  \end{scope}
  \begin{scope}[cm={{1.00588,0.0,0.0,1.00588,(241.915,110.647)}},draw=blue,line cap=rect,line join=bevel,line width=0.800pt]
  \end{scope}
  \begin{scope}[scale=1.006,draw=blue,line cap=rect,line join=bevel,line width=0.800pt]
  \end{scope}
  \begin{scope}[scale=1.006,draw=blue,line cap=rect,line join=bevel,line width=0.800pt]
  \end{scope}
  \begin{scope}[scale=1.006,draw=blue,line cap=rect,line join=bevel,line width=0.800pt]
  \end{scope}
  \begin{scope}[scale=1.006,draw=blue,line cap=rect,line join=bevel,line width=0.800pt]
  \end{scope}
  \begin{scope}[scale=1.006,draw=blue,line cap=rect,line join=bevel,line width=0.800pt]
  \end{scope}
  \begin{scope}[scale=1.006,draw=blue,line cap=rect,line join=bevel,line width=0.800pt]
  \end{scope}
  \begin{scope}[cm={{1.00588,0.0,0.0,1.00588,(235.376,29.1706)}},draw=blue,line cap=rect,line join=bevel,line width=0.800pt]
  \end{scope}
  \begin{scope}[cm={{1.00588,0.0,0.0,1.00588,(235.376,29.1706)}},draw=blue,line cap=rect,line join=bevel,line width=0.800pt]
  \end{scope}
  \begin{scope}[cm={{1.00588,0.0,0.0,1.00588,(235.376,29.1706)}},draw=blue,line cap=rect,line join=bevel,line width=0.800pt]
  \end{scope}
  \begin{scope}[cm={{1.00588,0.0,0.0,1.00588,(235.376,29.1706)}},draw=blue,line cap=rect,line join=bevel,line width=0.800pt]
  \end{scope}
  \begin{scope}[cm={{1.00588,0.0,0.0,1.00588,(235.376,29.1706)}},draw=blue,line cap=rect,line join=bevel,line width=0.800pt]
  \end{scope}
  \begin{scope}[cm={{1.00588,0.0,0.0,1.00588,(235.376,29.1706)}},draw=blue,line cap=rect,line join=bevel,line width=0.800pt]
  \end{scope}
  \begin{scope}[scale=1.006,draw=blue,line cap=rect,line join=bevel,line width=0.800pt]
  \end{scope}
  \begin{scope}[scale=1.006,draw=blue,line cap=rect,line join=bevel,line width=0.800pt]
  \end{scope}
  \begin{scope}[scale=1.006,draw=blue,line cap=rect,line join=bevel,line width=0.800pt]
  \end{scope}
  \begin{scope}[scale=1.006,draw=blue,line cap=rect,line join=bevel,line width=0.800pt]
  \end{scope}
  \begin{scope}[scale=1.006,draw=blue,line cap=rect,line join=bevel,line width=0.800pt]
  \end{scope}
  \begin{scope}[scale=1.006,draw=blue,line cap=rect,line join=bevel,line width=0.800pt]
  \end{scope}
  \begin{scope}[scale=1.006,draw=blue,line cap=rect,line join=bevel,line width=0.800pt]
  \end{scope}
  \begin{scope}[scale=1.006,draw=blue,line cap=rect,line join=bevel,line width=0.800pt]
  \end{scope}
  \begin{scope}[scale=1.006,draw=blue,line cap=rect,line join=bevel,line width=0.800pt]
  \end{scope}
  \begin{scope}[cm={{1.04177,0.0,0.0,1.04177,(-342.44493,31.82443)}},draw=ca0a0a4,dash pattern=on 1.73pt off 1.73pt,line cap=round,line join=round,line width=0.288pt,miter limit=4.00]
    \path[draw,dash pattern=on 1.73pt off 1.73pt,line width=0.288pt,miter limit=4.00] (56.5000,194.5000) -- (251.5000,194.5000);



  \end{scope}
  \begin{scope}[cm={{1.04177,0.0,0.0,1.04177,(-342.44493,31.82443)}},draw=blue,line cap=round,line join=round,line width=0.480pt]
    \path[cm={{1.1115,0.0,0.0,1.0,(-6.26603,0.0)}},draw] (56.5000,194.5000) -- (59.5000,194.5000);



    \path[cm={{1.1115,0.0,0.0,1.0,(-28.19312,0.0)}},draw] (251.5000,194.5000) -- (248.5000,194.5000);



  \end{scope}
  \begin{scope}[scale=1.006,draw=blue,line cap=rect,line join=bevel,line width=0.800pt]
  \end{scope}
  \begin{scope}[cm={{1.00588,0.0,0.0,1.00588,(39.2294,199.165)}},draw=blue,line cap=rect,line join=bevel,line width=0.800pt]
  \end{scope}
  \begin{scope}[cm={{1.00588,0.0,0.0,1.00588,(39.2294,199.165)}},draw=blue,line cap=rect,line join=bevel,line width=0.800pt]
  \end{scope}
  \begin{scope}[cm={{1.00588,0.0,0.0,1.00588,(39.2294,199.165)}},draw=blue,line cap=rect,line join=bevel,line width=0.800pt]
  \end{scope}
  \begin{scope}[cm={{1.00588,0.0,0.0,1.00588,(39.2294,199.165)}},draw=blue,line cap=rect,line join=bevel,line width=0.800pt]
  \end{scope}
  \begin{scope}[cm={{1.00588,0.0,0.0,1.00588,(39.2294,199.165)}},draw=blue,line cap=rect,line join=bevel,line width=0.800pt]
  \end{scope}
  \begin{scope}[cm={{1.00588,0.0,0.0,1.00588,(-298.32402,235.165)}},draw=blue,line cap=rect,line join=bevel,line width=0.800pt]
    \path[fill=blue] (0.0000,0.0000) node[above right] (text658) {27};



  \end{scope}
  \begin{scope}[cm={{1.00588,0.0,0.0,1.00588,(39.2294,199.165)}},draw=blue,line cap=rect,line join=bevel,line width=0.800pt]
  \end{scope}
  \begin{scope}[scale=1.006,draw=blue,line cap=rect,line join=bevel,line width=0.800pt]
  \end{scope}
  \begin{scope}[cm={{1.04177,0.0,0.0,1.04177,(-342.44493,31.82443)}},draw=ca0a0a4,dash pattern=on 1.73pt off 1.73pt,line cap=round,line join=round,line width=0.288pt,miter limit=4.00]
    \path[draw,dash pattern=on 1.73pt off 1.73pt,line width=0.288pt,miter limit=4.00] (56.5000,171.5000) -- (251.5000,171.5000);



  \end{scope}
  \begin{scope}[cm={{1.04177,0.0,0.0,1.04177,(-342.44493,31.82443)}},draw=blue,line cap=round,line join=round,line width=0.480pt]
    \path[cm={{1.1115,0.0,0.0,1.0,(-6.26603,0.0)}},draw] (56.5000,171.5000) -- (59.5000,171.5000);



    \path[cm={{1.1115,0.0,0.0,1.0,(-28.19312,0.0)}},draw] (251.5000,171.5000) -- (248.5000,171.5000);



  \end{scope}
  \begin{scope}[scale=1.006,draw=blue,line cap=rect,line join=bevel,line width=0.800pt]
  \end{scope}
  \begin{scope}[cm={{1.00588,0.0,0.0,1.00588,(40.2353,177.035)}},draw=blue,line cap=rect,line join=bevel,line width=0.800pt]
  \end{scope}
  \begin{scope}[cm={{1.00588,0.0,0.0,1.00588,(40.2353,177.035)}},draw=blue,line cap=rect,line join=bevel,line width=0.800pt]
  \end{scope}
  \begin{scope}[cm={{1.00588,0.0,0.0,1.00588,(40.2353,177.035)}},draw=blue,line cap=rect,line join=bevel,line width=0.800pt]
  \end{scope}
  \begin{scope}[cm={{1.00588,0.0,0.0,1.00588,(40.2353,177.035)}},draw=blue,line cap=rect,line join=bevel,line width=0.800pt]
  \end{scope}
  \begin{scope}[cm={{1.00588,0.0,0.0,1.00588,(40.2353,177.035)}},draw=blue,line cap=rect,line join=bevel,line width=0.800pt]
  \end{scope}
  \begin{scope}[cm={{1.00588,0.0,0.0,1.00588,(-298.41069,213.035)}},draw=blue,line cap=rect,line join=bevel,line width=0.800pt]
    \path[fill=blue] (0.5262,0.0000) node[above right] (text688) {31};



  \end{scope}
  \begin{scope}[cm={{1.00588,0.0,0.0,1.00588,(40.2353,177.035)}},draw=blue,line cap=rect,line join=bevel,line width=0.800pt]
  \end{scope}
  \begin{scope}[scale=1.006,draw=blue,line cap=rect,line join=bevel,line width=0.800pt]
  \end{scope}
  \begin{scope}[cm={{1.04177,0.0,0.0,1.04177,(-342.44493,31.82443)}},draw=ca0a0a4,dash pattern=on 1.73pt off 1.73pt,line cap=round,line join=round,line width=0.288pt,miter limit=4.00]
    \path[draw,dash pattern=on 1.73pt off 1.73pt,line width=0.288pt,miter limit=4.00] (56.5000,149.5000) -- (251.5000,149.5000);



  \end{scope}
  \begin{scope}[cm={{1.04177,0.0,0.0,1.04177,(-342.44493,31.82443)}},draw=blue,line cap=round,line join=round,line width=0.480pt]
    \path[cm={{1.1115,0.0,0.0,1.0,(-6.26603,0.0)}},draw] (56.5000,149.5000) -- (59.5000,149.5000);



    \path[cm={{1.1115,0.0,0.0,1.0,(-28.19312,0.0)}},draw] (251.5000,149.5000) -- (248.5000,149.5000);



  \end{scope}
  \begin{scope}[scale=1.006,draw=blue,line cap=rect,line join=bevel,line width=0.800pt]
  \end{scope}
  \begin{scope}[cm={{1.00588,0.0,0.0,1.00588,(40.2353,153.9)}},draw=blue,line cap=rect,line join=bevel,line width=0.800pt]
  \end{scope}
  \begin{scope}[cm={{1.00588,0.0,0.0,1.00588,(40.2353,153.9)}},draw=blue,line cap=rect,line join=bevel,line width=0.800pt]
  \end{scope}
  \begin{scope}[cm={{1.00588,0.0,0.0,1.00588,(40.2353,153.9)}},draw=blue,line cap=rect,line join=bevel,line width=0.800pt]
  \end{scope}
  \begin{scope}[cm={{1.00588,0.0,0.0,1.00588,(40.2353,153.9)}},draw=blue,line cap=rect,line join=bevel,line width=0.800pt]
  \end{scope}
  \begin{scope}[cm={{1.00588,0.0,0.0,1.00588,(40.2353,153.9)}},draw=blue,line cap=rect,line join=bevel,line width=0.800pt]
  \end{scope}
  \begin{scope}[cm={{1.00588,0.0,0.0,1.00588,(-298.23551,191.4)}},draw=blue,line cap=rect,line join=bevel,line width=0.800pt]
    \path[fill=blue] (0.0000,0.0000) node[above right] (text718) {35};



  \end{scope}
  \begin{scope}[cm={{1.00588,0.0,0.0,1.00588,(40.2353,153.9)}},draw=blue,line cap=rect,line join=bevel,line width=0.800pt]
  \end{scope}
  \begin{scope}[scale=1.006,draw=blue,line cap=rect,line join=bevel,line width=0.800pt]
  \end{scope}
  \begin{scope}[cm={{1.04164,0.0,0.0,1.04164,(-342.30377,31.82628)}},draw=ca0a0a4,dash pattern=on 1.73pt off 1.73pt,line cap=round,line join=round,line width=0.288pt,miter limit=4.00]
    \path[draw,dash pattern=on 1.73pt off 1.73pt,line width=0.288pt,miter limit=4.00] (56.5000,126.5000) -- (195.5000,126.5000);



    \path[draw,dash pattern=on 1.73pt off 1.73pt,line width=0.288pt,miter limit=4.00] (246.5000,126.5000) -- (251.5000,126.5000);



  \end{scope}
  \begin{scope}[cm={{1.04177,0.0,0.0,1.04177,(-342.44493,31.82443)}},draw=blue,line cap=round,line join=round,line width=0.480pt]
    \path[cm={{1.1115,0.0,0.0,1.0,(-6.26603,0.0)}},draw] (56.5000,126.5000) -- (59.5000,126.5000);



    \path[cm={{1.1115,0.0,0.0,1.0,(-28.19312,0.0)}},draw] (251.5000,126.5000) -- (248.5000,126.5000);



  \end{scope}
  \begin{scope}[scale=1.006,draw=blue,line cap=rect,line join=bevel,line width=0.800pt]
  \end{scope}
  \begin{scope}[cm={{1.00588,0.0,0.0,1.00588,(39.2294,131.771)}},draw=blue,line cap=rect,line join=bevel,line width=0.800pt]
  \end{scope}
  \begin{scope}[cm={{1.00588,0.0,0.0,1.00588,(39.2294,131.771)}},draw=blue,line cap=rect,line join=bevel,line width=0.800pt]
  \end{scope}
  \begin{scope}[cm={{1.00588,0.0,0.0,1.00588,(39.2294,131.771)}},draw=blue,line cap=rect,line join=bevel,line width=0.800pt]
  \end{scope}
  \begin{scope}[cm={{1.00588,0.0,0.0,1.00588,(39.2294,131.771)}},draw=blue,line cap=rect,line join=bevel,line width=0.800pt]
  \end{scope}
  \begin{scope}[cm={{1.00588,0.0,0.0,1.00588,(39.2294,131.771)}},draw=blue,line cap=rect,line join=bevel,line width=0.800pt]
  \end{scope}
  \begin{scope}[cm={{1.00588,0.0,0.0,1.00588,(-298.4045,166.271)}},draw=blue,line cap=rect,line join=bevel,line width=0.800pt]
    \path[fill=blue] (0.0000,0.0000) node[above right] (text750) {39};



  \end{scope}
  \begin{scope}[cm={{1.00588,0.0,0.0,1.00588,(39.2294,131.771)}},draw=blue,line cap=rect,line join=bevel,line width=0.800pt]
  \end{scope}
  \begin{scope}[scale=1.006,draw=blue,line cap=rect,line join=bevel,line width=0.800pt]
  \end{scope}
  \begin{scope}[cm={{1.04177,0.0,0.0,1.04177,(-342.44493,31.82443)}},draw=ca0a0a4,dash pattern=on 0.40pt off 0.80pt,line cap=round,line join=round,line width=0.400pt]
    \path[draw] (56.5000,200.5000) -- (56.5000,115.5000);



  \end{scope}
  \begin{scope}[cm={{1.04177,0.0,0.0,1.04177,(-342.44493,31.82443)}},draw=blue,line cap=round,line join=round,line width=0.480pt]
    \path[draw] (56.5000,200.5000) -- (56.5000,198.5000);



    \path[draw] (56.5000,115.5000) -- (56.5000,118.5000);



  \end{scope}
  \begin{scope}[scale=1.006,draw=blue,line cap=rect,line join=bevel,line width=0.800pt]
  \end{scope}
  \begin{scope}[cm={{1.00588,0.0,0.0,1.00588,(53.3118,217.271)}},draw=blue,line cap=rect,line join=bevel,line width=0.800pt]
  \end{scope}
  \begin{scope}[cm={{1.00588,0.0,0.0,1.00588,(53.3118,217.271)}},draw=blue,line cap=rect,line join=bevel,line width=0.800pt]
  \end{scope}
  \begin{scope}[cm={{1.00588,0.0,0.0,1.00588,(53.3118,217.271)}},draw=blue,line cap=rect,line join=bevel,line width=0.800pt]
  \end{scope}
  \begin{scope}[cm={{1.00588,0.0,0.0,1.00588,(53.3118,217.271)}},draw=blue,line cap=rect,line join=bevel,line width=0.800pt]
  \end{scope}
  \begin{scope}[cm={{1.00588,0.0,0.0,1.00588,(53.3118,217.271)}},draw=blue,line cap=rect,line join=bevel,line width=0.800pt]
  \end{scope}
  \begin{scope}[cm={{1.00588,0.0,0.0,1.00588,(-282.68826,252.63148)}},draw=blue,line cap=rect,line join=bevel,line width=0.800pt]
    \path[fill=blue] (0.0000,0.0000) node[above right] (text780) {0};



  \end{scope}
  \begin{scope}[cm={{1.00588,0.0,0.0,1.00588,(53.3118,217.271)}},draw=blue,line cap=rect,line join=bevel,line width=0.800pt]
  \end{scope}
  \begin{scope}[scale=1.006,draw=blue,line cap=rect,line join=bevel,line width=0.800pt]
  \end{scope}
  \begin{scope}[cm={{1.04177,0.0,0.0,1.04177,(-342.44493,31.82443)}},draw=ca0a0a4,dash pattern=on 1.73pt off 1.73pt,line cap=round,line join=round,line width=0.288pt,miter limit=4.00]
    \path[draw,dash pattern=on 1.73pt off 1.73pt,line width=0.288pt,miter limit=4.00] (85.5000,200.5000) -- (85.5000,115.5000);



  \end{scope}
  \begin{scope}[cm={{1.04177,0.0,0.0,1.04177,(-342.44493,31.82443)}},draw=blue,line cap=round,line join=round,line width=0.480pt]
    \path[cm={{1.0,0.0,0.0,1.53899,(0.0,-108.26833)}},draw] (85.5000,200.5000) -- (85.5000,198.5000);



    \path[cm={{1.0,0.0,0.0,1.1115,(0.0,-12.84424)}},draw] (85.5000,115.5000) -- (85.5000,118.5000);



  \end{scope}
  \begin{scope}[scale=1.006,draw=blue,line cap=rect,line join=bevel,line width=0.800pt]
  \end{scope}
  \begin{scope}[cm={{1.00588,0.0,0.0,1.00588,(82.4824,217.271)}},draw=blue,line cap=rect,line join=bevel,line width=0.800pt]
  \end{scope}
  \begin{scope}[cm={{1.00588,0.0,0.0,1.00588,(82.4824,217.271)}},draw=blue,line cap=rect,line join=bevel,line width=0.800pt]
  \end{scope}
  \begin{scope}[cm={{1.00588,0.0,0.0,1.00588,(82.4824,217.271)}},draw=blue,line cap=rect,line join=bevel,line width=0.800pt]
  \end{scope}
  \begin{scope}[cm={{1.00588,0.0,0.0,1.00588,(82.4824,217.271)}},draw=blue,line cap=rect,line join=bevel,line width=0.800pt]
  \end{scope}
  \begin{scope}[cm={{1.00588,0.0,0.0,1.00588,(82.4824,217.271)}},draw=blue,line cap=rect,line join=bevel,line width=0.800pt]
  \end{scope}
  \begin{scope}[cm={{1.00588,0.0,0.0,1.00588,(-253.51767,252.74414)}},draw=blue,line cap=rect,line join=bevel,line width=0.800pt]
    \path[fill=blue] (0.0000,0.0000) node[above right] (text810) {1};



  \end{scope}
  \begin{scope}[cm={{1.00588,0.0,0.0,1.00588,(82.4824,217.271)}},draw=blue,line cap=rect,line join=bevel,line width=0.800pt]
  \end{scope}
  \begin{scope}[scale=1.006,draw=blue,line cap=rect,line join=bevel,line width=0.800pt]
  \end{scope}
  \begin{scope}[cm={{1.04177,0.0,0.0,1.04177,(-342.44493,31.82443)}},draw=ca0a0a4,dash pattern=on 1.73pt off 1.73pt,line cap=round,line join=round,line width=0.288pt,miter limit=4.00]
    \path[draw,dash pattern=on 1.73pt off 1.73pt,line width=0.288pt,miter limit=4.00] (114.5000,200.5000) -- (114.5000,115.5000);



  \end{scope}
  \begin{scope}[cm={{1.04177,0.0,0.0,1.04177,(-342.44493,31.82443)}},draw=blue,line cap=round,line join=round,line width=0.480pt]
    \path[cm={{1.0,0.0,0.0,1.53899,(0.0,-108.26833)}},draw] (114.5000,200.5000) -- (114.5000,198.5000);



    \path[cm={{1.0,0.0,0.0,1.1115,(0.0,-12.84424)}},draw] (114.5000,115.5000) -- (114.5000,118.5000);



  \end{scope}
  \begin{scope}[scale=1.006,draw=blue,line cap=rect,line join=bevel,line width=0.800pt]
  \end{scope}
  \begin{scope}[cm={{1.00588,0.0,0.0,1.00588,(111.653,217.271)}},draw=blue,line cap=rect,line join=bevel,line width=0.800pt]
  \end{scope}
  \begin{scope}[cm={{1.00588,0.0,0.0,1.00588,(111.653,217.271)}},draw=blue,line cap=rect,line join=bevel,line width=0.800pt]
  \end{scope}
  \begin{scope}[cm={{1.00588,0.0,0.0,1.00588,(111.653,217.271)}},draw=blue,line cap=rect,line join=bevel,line width=0.800pt]
  \end{scope}
  \begin{scope}[cm={{1.00588,0.0,0.0,1.00588,(111.653,217.271)}},draw=blue,line cap=rect,line join=bevel,line width=0.800pt]
  \end{scope}
  \begin{scope}[cm={{1.00588,0.0,0.0,1.00588,(111.653,217.271)}},draw=blue,line cap=rect,line join=bevel,line width=0.800pt]
  \end{scope}
  \begin{scope}[cm={{1.00588,0.0,0.0,1.00588,(-224.34706,252.74414)}},draw=blue,line cap=rect,line join=bevel,line width=0.800pt]
    \path[fill=blue] (0.0000,0.0000) node[above right] (text840) {2};



  \end{scope}
  \begin{scope}[cm={{1.00588,0.0,0.0,1.00588,(111.653,217.271)}},draw=blue,line cap=rect,line join=bevel,line width=0.800pt]
  \end{scope}
  \begin{scope}[scale=1.006,draw=blue,line cap=rect,line join=bevel,line width=0.800pt]
  \end{scope}
  \begin{scope}[cm={{1.04177,0.0,0.0,1.04177,(-342.44493,31.82443)}},draw=ca0a0a4,dash pattern=on 1.73pt off 1.73pt,line cap=round,line join=round,line width=0.288pt,miter limit=4.00]
    \path[draw,dash pattern=on 1.73pt off 1.73pt,line width=0.288pt,miter limit=4.00] (143.5000,200.5000) -- (143.5000,115.5000);



  \end{scope}
  \begin{scope}[cm={{1.04177,0.0,0.0,1.04177,(-342.44493,31.82443)}},draw=blue,line cap=round,line join=round,line width=0.480pt]
    \path[cm={{1.0,0.0,0.0,1.53899,(0.0,-108.26833)}},draw] (143.5000,200.5000) -- (143.5000,198.5000);



    \path[cm={{1.0,0.0,0.0,1.1115,(0.0,-12.84424)}},draw] (143.5000,115.5000) -- (143.5000,118.5000);



  \end{scope}
  \begin{scope}[scale=1.006,draw=blue,line cap=rect,line join=bevel,line width=0.800pt]
  \end{scope}
  \begin{scope}[cm={{1.00588,0.0,0.0,1.00588,(142.332,217.271)}},draw=blue,line cap=rect,line join=bevel,line width=0.800pt]
  \end{scope}
  \begin{scope}[cm={{1.00588,0.0,0.0,1.00588,(142.332,217.271)}},draw=blue,line cap=rect,line join=bevel,line width=0.800pt]
  \end{scope}
  \begin{scope}[cm={{1.00588,0.0,0.0,1.00588,(142.332,217.271)}},draw=blue,line cap=rect,line join=bevel,line width=0.800pt]
  \end{scope}
  \begin{scope}[cm={{1.00588,0.0,0.0,1.00588,(142.332,217.271)}},draw=blue,line cap=rect,line join=bevel,line width=0.800pt]
  \end{scope}
  \begin{scope}[cm={{1.00588,0.0,0.0,1.00588,(142.332,217.271)}},draw=blue,line cap=rect,line join=bevel,line width=0.800pt]
  \end{scope}
  \begin{scope}[cm={{1.00588,0.0,0.0,1.00588,(-193.66806,252.63148)}},draw=blue,line cap=rect,line join=bevel,line width=0.800pt]
    \path[fill=blue] (0.0000,0.0000) node[above right] (text870) {3};



  \end{scope}
  \begin{scope}[cm={{1.00588,0.0,0.0,1.00588,(142.332,217.271)}},draw=blue,line cap=rect,line join=bevel,line width=0.800pt]
  \end{scope}
  \begin{scope}[scale=1.006,draw=blue,line cap=rect,line join=bevel,line width=0.800pt]
  \end{scope}
  \begin{scope}[cm={{1.04177,0.0,0.0,1.04177,(-342.44493,31.82443)}},draw=ca0a0a4,dash pattern=on 1.73pt off 1.73pt,line cap=round,line join=round,line width=0.288pt,miter limit=4.00]
    \path[draw,dash pattern=on 1.73pt off 1.73pt,line width=0.288pt,miter limit=4.00] (172.5000,200.5000) -- (172.5000,115.5000);



  \end{scope}
  \begin{scope}[cm={{1.04177,0.0,0.0,1.04177,(-342.44493,31.82443)}},draw=blue,line cap=round,line join=round,line width=0.480pt]
    \path[cm={{1.0,0.0,0.0,1.53899,(0.0,-108.26833)}},draw] (172.5000,200.5000) -- (172.5000,198.5000);



    \path[cm={{1.0,0.0,0.0,1.1115,(0.0,-12.84424)}},draw] (172.5000,115.5000) -- (172.5000,118.5000);



  \end{scope}
  \begin{scope}[scale=1.006,draw=blue,line cap=rect,line join=bevel,line width=0.800pt]
  \end{scope}
  \begin{scope}[cm={{1.00588,0.0,0.0,1.00588,(171.503,217.271)}},draw=blue,line cap=rect,line join=bevel,line width=0.800pt]
  \end{scope}
  \begin{scope}[cm={{1.00588,0.0,0.0,1.00588,(171.503,217.271)}},draw=blue,line cap=rect,line join=bevel,line width=0.800pt]
  \end{scope}
  \begin{scope}[cm={{1.00588,0.0,0.0,1.00588,(171.503,217.271)}},draw=blue,line cap=rect,line join=bevel,line width=0.800pt]
  \end{scope}
  \begin{scope}[cm={{1.00588,0.0,0.0,1.00588,(171.503,217.271)}},draw=blue,line cap=rect,line join=bevel,line width=0.800pt]
  \end{scope}
  \begin{scope}[cm={{1.00588,0.0,0.0,1.00588,(171.503,217.271)}},draw=blue,line cap=rect,line join=bevel,line width=0.800pt]
  \end{scope}
  \begin{scope}[cm={{1.00588,0.0,0.0,1.00588,(-164.49705,252.74414)}},draw=blue,line cap=rect,line join=bevel,line width=0.800pt]
    \path[fill=blue] (0.0000,0.0000) node[above right] (text900) {4};



  \end{scope}
  \begin{scope}[cm={{1.00588,0.0,0.0,1.00588,(171.503,217.271)}},draw=blue,line cap=rect,line join=bevel,line width=0.800pt]
  \end{scope}
  \begin{scope}[scale=1.006,draw=blue,line cap=rect,line join=bevel,line width=0.800pt]
  \end{scope}
  \begin{scope}[cm={{1.04164,0.0,0.0,1.04164,(-342.42871,32.01781)}},draw=ca0a0a4,dash pattern=on 1.73pt off 1.73pt,line cap=round,line join=round,line width=0.288pt,miter limit=4.00]
    \path[draw,dash pattern=on 1.73pt off 1.73pt,line width=0.288pt,miter limit=4.00] (201.5000,200.5000) -- (201.5000,129.5000);



    \path[draw,dash pattern=on 1.73pt off 1.73pt,line width=0.288pt,miter limit=4.00] (201.5000,121.5000) -- (201.5000,115.5000);



  \end{scope}
  \begin{scope}[cm={{1.04177,0.0,0.0,1.04177,(-342.51116,41.97197)}},draw=ca0a0a4,dash pattern=on 1.73pt off 1.73pt,line cap=round,line join=round,line width=0.288pt,miter limit=4.00]
    \path[draw,dash pattern=on 1.73pt off 1.73pt,line width=0.288pt,miter limit=4.00] (198.5000,306.5000) -- (198.5000,221.5000);



  \end{scope}
  \begin{scope}[cm={{1.04177,0.0,0.0,1.04177,(-342.44493,31.82445)}},draw=blue,line cap=round,line join=round,line width=0.480pt]
    \path[cm={{1.0,0.0,0.0,1.53899,(0.0,-108.26833)}},draw] (201.5000,200.5000) -- (201.5000,198.5000);



    \path[cm={{1.0,0.0,0.0,1.1115,(0.0,-12.84424)}},draw] (201.5000,115.5000) -- (201.5000,118.5000);



  \end{scope}
  \begin{scope}[scale=1.006,draw=blue,line cap=rect,line join=bevel,line width=0.800pt]
  \end{scope}
  \begin{scope}[cm={{1.00588,0.0,0.0,1.00588,(200.674,217.271)}},draw=blue,line cap=rect,line join=bevel,line width=0.800pt]
  \end{scope}
  \begin{scope}[cm={{1.00588,0.0,0.0,1.00588,(200.674,217.271)}},draw=blue,line cap=rect,line join=bevel,line width=0.800pt]
  \end{scope}
  \begin{scope}[cm={{1.00588,0.0,0.0,1.00588,(200.674,217.271)}},draw=blue,line cap=rect,line join=bevel,line width=0.800pt]
  \end{scope}
  \begin{scope}[cm={{1.00588,0.0,0.0,1.00588,(200.674,217.271)}},draw=blue,line cap=rect,line join=bevel,line width=0.800pt]
  \end{scope}
  \begin{scope}[cm={{1.00588,0.0,0.0,1.00588,(200.674,217.271)}},draw=blue,line cap=rect,line join=bevel,line width=0.800pt]
  \end{scope}
  \begin{scope}[cm={{1.00588,0.0,0.0,1.00588,(-136.82605,252.63148)}},draw=blue,line cap=rect,line join=bevel,line width=0.800pt]
    \path[fill=blue] (1.4912,0.0000) node[above right] (text932) {5};



  \end{scope}
  \begin{scope}[cm={{1.00588,0.0,0.0,1.00588,(200.674,217.271)}},draw=blue,line cap=rect,line join=bevel,line width=0.800pt]
  \end{scope}
  \begin{scope}[scale=1.006,draw=blue,line cap=rect,line join=bevel,line width=0.800pt]
  \end{scope}
  \begin{scope}[cm={{1.04164,0.0,0.0,1.04164,(-342.48662,32.01785)}},draw=ca0a0a4,dash pattern=on 1.73pt off 1.73pt,line cap=round,line join=round,line width=0.288pt,miter limit=4.00]
    \path[draw,dash pattern=on 1.73pt off 1.73pt,line width=0.288pt,miter limit=4.00] (231.5000,200.5000) -- (231.5000,129.5000);



    \path[draw,dash pattern=on 1.73pt off 1.73pt,line width=0.288pt,miter limit=4.00] (231.5000,121.5000) -- (231.5000,115.5000);



  \end{scope}
  \begin{scope}[cm={{1.04177,0.0,0.0,1.04177,(-342.45387,41.97197)}},draw=ca0a0a4,dash pattern=on 1.73pt off 1.73pt,line cap=round,line join=round,line width=0.288pt,miter limit=4.00]
    \path[draw,dash pattern=on 1.73pt off 1.73pt,line width=0.288pt,miter limit=4.00] (233.5000,306.5000) -- (233.5000,221.5000);



  \end{scope}
  \begin{scope}[cm={{1.04177,0.0,0.0,1.04177,(-342.44493,31.82443)}},draw=blue,line cap=round,line join=round,line width=0.480pt]
    \path[cm={{1.0,0.0,0.0,1.53899,(0.0,-108.26833)}},draw] (231.5000,200.5000) -- (231.5000,198.5000);



    \path[cm={{1.0,0.0,0.0,1.1115,(0.0,-12.84424)}},draw] (231.5000,115.5000) -- (231.5000,118.5000);



  \end{scope}
  \begin{scope}[scale=1.006,draw=blue,line cap=rect,line join=bevel,line width=0.800pt]
  \end{scope}
  \begin{scope}[cm={{1.00588,0.0,0.0,1.00588,(229.341,217.271)}},draw=blue,line cap=rect,line join=bevel,line width=0.800pt]
  \end{scope}
  \begin{scope}[cm={{1.00588,0.0,0.0,1.00588,(229.341,217.271)}},draw=blue,line cap=rect,line join=bevel,line width=0.800pt]
  \end{scope}
  \begin{scope}[cm={{1.00588,0.0,0.0,1.00588,(229.341,217.271)}},draw=blue,line cap=rect,line join=bevel,line width=0.800pt]
  \end{scope}
  \begin{scope}[cm={{1.00588,0.0,0.0,1.00588,(229.341,217.271)}},draw=blue,line cap=rect,line join=bevel,line width=0.800pt]
  \end{scope}
  \begin{scope}[cm={{1.00588,0.0,0.0,1.00588,(229.341,217.271)}},draw=blue,line cap=rect,line join=bevel,line width=0.800pt]
  \end{scope}
  \begin{scope}[cm={{1.00588,0.0,0.0,1.00588,(-106.65906,250.271)}},draw=blue,line cap=rect,line join=bevel,line width=0.800pt]
    \path[fill=blue] (0.0000,2.3467) node[above right] (text964) {6};



  \end{scope}
  \begin{scope}[cm={{1.00588,0.0,0.0,1.00588,(229.341,217.271)}},draw=blue,line cap=rect,line join=bevel,line width=0.800pt]
  \end{scope}
  \begin{scope}[scale=1.006,draw=blue,line cap=rect,line join=bevel,line width=0.800pt]
  \end{scope}
  \begin{scope}[cm={{1.04177,0.0,0.0,1.04177,(-342.44493,31.82443)}},draw=blue,line cap=round,line join=round,line width=0.480pt]
    \path[draw] (56.5000,115.5000) -- (56.5000,200.5000) -- (251.5000,200.5000) -- (251.5000,115.5000) -- (56.5000,115.5000);



  \end{scope}
  \begin{scope}[scale=1.006,draw=blue,line cap=rect,line join=bevel,line width=0.800pt]
  \end{scope}
  \begin{scope}[scale=1.006,draw=blue,line cap=rect,line join=bevel,line width=0.800pt]
  \end{scope}
  \begin{scope}[cm={{1.04164,0.0,0.0,1.04164,(-345.64402,156.32537)}},draw=c00ff00,line cap=round,line join=round,line width=0.480pt]
    \path[draw,even odd rule] (215.5000,6.2015) -- (241.5000,6.2015);



  \end{scope}
  \begin{scope}[scale=1.006,draw=blue,line cap=rect,line join=bevel,line width=0.800pt]
  \end{scope}
  \begin{scope}[scale=1.006,draw=blue,line cap=rect,line join=bevel,line width=0.800pt]
  \end{scope}
  \begin{scope}[scale=1.006,draw=blue,line cap=rect,line join=bevel,line width=0.800pt]
  \end{scope}
  \begin{scope}[cm={{1.00588,0.0,0.0,1.00588,(236.382,132.776)}},draw=blue,line cap=rect,line join=bevel,line width=0.800pt]
  \end{scope}
  \begin{scope}[cm={{1.00588,0.0,0.0,1.00588,(236.382,132.776)}},draw=blue,line cap=rect,line join=bevel,line width=0.800pt]
  \end{scope}
  \begin{scope}[cm={{1.00588,0.0,0.0,1.00588,(236.382,132.776)}},draw=blue,line cap=rect,line join=bevel,line width=0.800pt]
  \end{scope}
  \begin{scope}[cm={{1.00588,0.0,0.0,1.00588,(236.382,132.776)}},draw=blue,line cap=rect,line join=bevel,line width=0.800pt]
  \end{scope}
  \begin{scope}[cm={{1.00588,0.0,0.0,1.00588,(236.382,132.776)}},draw=blue,line cap=rect,line join=bevel,line width=0.800pt]
  \end{scope}
  \begin{scope}[cm={{1.00588,0.0,0.0,1.00588,(236.382,132.776)}},draw=blue,line cap=rect,line join=bevel,line width=0.800pt]
  \end{scope}
  \begin{scope}[scale=1.006,draw=blue,line cap=rect,line join=bevel,line width=0.800pt]
  \end{scope}
  \begin{scope}[scale=1.006,draw=blue,line cap=rect,line join=bevel,line width=0.800pt]
  \end{scope}
  \begin{scope}[scale=1.006,draw=blue,line cap=rect,line join=bevel,line width=0.800pt]
  \end{scope}
  \begin{scope}[cm={{1.04177,0.0,0.0,1.04177,(-342.44493,31.82443)}},draw=blue,line cap=round,line join=round,line width=0.480pt]
    \path[draw] (56.1000,127.5000) -- (56.1000,127.5000) -- (56.3000,142.6000) -- (56.5000,149.0000) -- (56.7000,148.4000) -- (56.9000,141.3000) -- (57.1000,134.7000) -- (57.3000,131.3000) -- (57.5000,130.4000) -- (57.7000,131.2000) -- (57.9000,132.8000) -- (58.1000,135.0000) -- (58.3000,137.3000) -- (58.4000,139.1000) -- (58.6000,140.1000) -- (58.8000,140.4000) -- (59.0000,140.4000) -- (59.2000,140.1000) -- (59.4000,139.9000) -- (59.6000,139.7000) -- (59.8000,139.6000) -- (60.0000,139.5000) -- (60.2000,139.5000) -- (60.4000,139.5000) -- (60.6000,139.5000) -- (60.8000,139.5000) -- (61.0000,139.5000) -- (61.2000,139.5000) -- (61.4000,139.5000) -- (61.6000,139.5000) -- (61.8000,139.5000) -- (62.0000,139.5000) -- (62.2000,139.5000) -- (62.4000,139.5000) -- (62.6000,139.5000) -- (62.8000,139.6000) -- (63.0000,140.3000) -- (63.1000,141.9000) -- (63.3000,143.7000) -- (63.5000,145.6000) -- (63.7000,147.5000) -- (63.9000,149.3000) -- (64.1000,151.1000) -- (64.3000,152.8000) -- (64.5000,154.5000) -- (64.7000,156.1000) -- (64.9000,157.6000) -- (65.1000,159.1000) -- (65.3000,160.5000) -- (65.5000,161.7000) -- (65.7000,162.7000) -- (65.9000,163.6000) -- (66.1000,164.2000) -- (66.3000,164.6000) -- (66.5000,164.7000) -- (66.7000,164.5000) -- (66.9000,164.0000) -- (67.1000,163.2000) -- (67.3000,162.2000) -- (67.5000,161.0000) -- (67.7000,159.7000) -- (67.9000,158.2000) -- (68.0000,156.6000) -- (68.2000,154.9000) -- (68.4000,153.2000) -- (68.6000,151.5000) -- (68.8000,149.7000) -- (69.0000,147.3000) -- (69.2000,144.6000) -- (69.4000,142.0000) -- (69.6000,140.1000) -- (69.8000,139.0000) -- (70.0000,138.5000) -- (70.2000,138.5000) -- (70.4000,138.7000) -- (70.6000,139.0000) -- (70.8000,139.2000) -- (71.0000,139.4000) -- (71.2000,139.5000) -- (71.4000,139.5000) -- (71.6000,139.6000) -- (71.8000,139.5000) -- (72.0000,139.5000) -- (72.2000,139.5000) -- (72.4000,139.5000) -- (72.6000,139.5000) -- (72.7000,139.5000) -- (72.9000,139.5000) -- (73.1000,139.5000) -- (73.3000,139.5000) -- (73.5000,139.4000) -- (73.7000,140.8000) -- (73.9000,142.7000) -- (74.1000,142.4000) -- (74.3000,141.0000) -- (74.5000,139.2000) -- (74.7000,137.3000) -- (74.9000,135.6000) -- (75.1000,134.1000) -- (75.3000,132.8000) -- (75.5000,131.6000) -- (75.7000,130.7000) -- (75.9000,129.8000) -- (76.1000,129.1000) -- (76.3000,128.4000) -- (76.5000,127.9000) -- (76.7000,127.4000) -- (76.9000,127.0000) -- (77.1000,126.7000) -- (77.3000,126.4000) -- (77.5000,126.3000) -- (77.6000,126.2000) -- (77.8000,126.2000) -- (78.0000,126.3000) -- (78.2000,126.5000) -- (78.4000,126.7000) -- (78.6000,127.1000) -- (78.8000,127.5000) -- (79.0000,128.0000) -- (79.2000,128.6000) -- (79.4000,129.3000) -- (79.6000,130.0000) -- (79.8000,130.9000) -- (80.0000,131.9000) -- (80.2000,132.9000) -- (80.4000,134.1000) -- (80.6000,135.3000) -- (80.8000,137.0000) -- (81.0000,138.6000) -- (81.2000,139.6000) -- (81.4000,140.1000) -- (81.6000,140.2000) -- (81.8000,140.1000) -- (82.0000,140.0000) -- (82.2000,139.8000) -- (82.4000,139.6000) -- (82.5000,139.6000) -- (82.7000,139.5000) -- (82.9000,139.5000) -- (83.1000,139.5000) -- (83.3000,139.5000) -- (83.5000,139.5000) -- (83.7000,139.5000) -- (83.9000,139.5000) -- (84.1000,139.5000) -- (84.3000,139.5000) -- (84.5000,139.5000) -- (84.7000,139.5000) -- (84.9000,139.5000) -- (85.1000,139.5000) -- (85.3000,139.5000) -- (85.5000,139.7000) -- (85.7000,140.9000) -- (85.9000,142.7000) -- (86.1000,144.6000) -- (86.3000,146.6000) -- (86.5000,148.5000) -- (86.7000,150.3000) -- (86.9000,152.1000) -- (87.1000,153.8000) -- (87.2000,155.5000) -- (87.4000,157.0000) -- (87.6000,158.6000) -- (87.8000,160.0000) -- (88.0000,161.3000) -- (88.2000,162.4000) -- (88.4000,163.3000) -- (88.6000,164.1000) -- (88.8000,164.5000) -- (89.0000,164.7000) -- (89.2000,164.6000) -- (89.4000,164.1000) -- (89.6000,163.4000) -- (89.8000,162.4000) -- (90.0000,161.2000) -- (90.2000,159.8000) -- (90.4000,158.3000) -- (90.6000,156.6000) -- (90.8000,154.9000) -- (91.0000,153.2000) -- (91.2000,151.5000) -- (91.4000,149.6000) -- (91.6000,147.3000) -- (91.8000,144.8000) -- (92.0000,142.3000) -- (92.1000,140.4000) -- (92.3000,139.2000) -- (92.5000,138.7000) -- (92.7000,138.6000) -- (92.9000,138.7000) -- (93.1000,139.0000) -- (93.3000,139.2000) -- (93.5000,139.4000) -- (93.7000,139.5000) -- (93.9000,139.5000) -- (94.1000,139.5000) -- (94.3000,139.5000) -- (94.5000,139.5000) -- (94.7000,139.5000) -- (94.9000,139.5000) -- (95.1000,139.5000) -- (95.3000,139.5000) -- (95.5000,139.5000) -- (95.7000,139.5000) -- (95.9000,139.5000) -- (96.1000,139.5000) -- (96.3000,141.0000) -- (96.5000,142.6000) -- (96.7000,142.3000) -- (96.8000,140.8000) -- (97.0000,139.0000) -- (97.2000,137.1000) -- (97.4000,135.4000) -- (97.6000,133.9000) -- (97.8000,132.6000) -- (98.0000,131.4000) -- (98.2000,130.5000) -- (98.4000,129.6000) -- (98.6000,128.9000) -- (98.8000,128.3000) -- (99.0000,127.7000) -- (99.2000,127.3000) -- (99.4000,126.9000) -- (99.6000,126.6000) -- (99.8000,126.4000) -- (100.0000,126.2000) -- (100.2000,126.2000) -- (100.4000,126.2000) -- (100.6000,126.3000) -- (100.8000,126.6000) -- (101.0000,126.8000) -- (101.2000,127.3000) -- (101.4000,127.9000) -- (101.6000,128.6000) -- (101.7000,129.3000) -- (101.9000,130.1000) -- (102.1000,130.9000) -- (102.3000,131.9000) -- (102.5000,132.9000) -- (102.7000,134.0000) -- (102.9000,135.3000) -- (103.1000,137.1000) -- (103.3000,138.7000) -- (103.5000,139.8000) -- (103.7000,140.2000) -- (103.9000,140.3000) -- (104.1000,140.2000) -- (104.3000,140.0000) -- (104.5000,139.8000) -- (104.7000,139.6000) -- (104.9000,139.5000) -- (105.1000,139.5000) -- (105.3000,139.5000) -- (105.5000,139.5000) -- (105.7000,139.5000) -- (105.9000,139.5000) -- (106.1000,139.5000) -- (106.3000,139.5000) -- (106.4000,139.5000) -- (106.6000,139.5000) -- (106.8000,139.5000) -- (107.0000,139.5000) -- (107.2000,139.5000) -- (107.4000,139.5000) -- (107.6000,139.5000) -- (107.8000,139.7000) -- (108.0000,140.8000) -- (108.2000,142.5000) -- (108.4000,144.4000) -- (108.6000,146.3000) -- (108.8000,148.1000) -- (109.0000,149.8000) -- (109.2000,151.5000) -- (109.4000,153.2000) -- (109.6000,154.8000) -- (109.8000,156.4000) -- (110.0000,157.9000) -- (110.2000,159.3000) -- (110.4000,160.6000) -- (110.6000,161.7000) -- (110.8000,162.7000) -- (111.0000,163.5000) -- (111.1000,164.1000) -- (111.3000,164.5000) -- (111.5000,164.6000) -- (111.7000,164.5000) -- (111.9000,164.1000) -- (112.1000,163.5000) -- (112.3000,162.7000) -- (112.5000,161.7000) -- (112.7000,160.6000) -- (112.9000,159.3000) -- (113.1000,157.9000) -- (113.3000,156.4000) -- (113.5000,154.9000) -- (113.7000,153.3000) -- (113.9000,151.7000) -- (114.1000,149.9000) -- (114.3000,146.7000) -- (114.5000,143.0000) -- (114.7000,140.3000) -- (114.9000,144.4000) -- (115.1000,145.2000) -- (115.3000,145.1000) -- (115.5000,145.4000) -- (115.7000,145.8000) -- (115.9000,146.2000) -- (116.0000,146.4000) -- (116.2000,146.5000) -- (116.4000,146.6000) -- (116.6000,146.6000) -- (116.8000,146.6000) -- (117.0000,146.6000) -- (117.2000,146.6000) -- (117.4000,146.6000) -- (117.6000,146.5000) -- (117.8000,146.5000) -- (118.0000,146.5000) -- (118.2000,146.5000) -- (118.4000,146.5000) -- (118.6000,146.5000) -- (118.8000,146.5000) -- (119.0000,146.7000) -- (119.2000,147.4000) -- (119.4000,147.9000) -- (119.6000,147.5000) -- (119.8000,146.2000) -- (120.0000,144.3000) -- (120.2000,146.9000) -- (120.4000,148.4000) -- (120.6000,147.9000) -- (120.8000,152.5000) -- (120.9000,151.7000) -- (121.1000,150.8000) -- (121.3000,150.0000) -- (121.5000,149.4000) -- (121.7000,148.8000) -- (121.9000,148.1000) -- (122.1000,152.5000) -- (122.3000,154.9000) -- (122.5000,154.5000) -- (122.7000,154.3000) -- (122.9000,154.3000) -- (123.1000,154.3000) -- (123.3000,154.5000) -- (123.5000,154.7000) -- (123.7000,155.0000) -- (123.9000,155.4000) -- (124.1000,155.8000) -- (124.3000,156.6000) -- (124.5000,155.2000) -- (124.7000,150.6000) -- (124.9000,151.6000) -- (125.1000,152.5000) -- (125.3000,153.6000) -- (125.5000,154.7000) -- (125.7000,156.0000) -- (125.8000,157.7000) -- (126.0000,153.9000) -- (126.2000,153.5000) -- (126.4000,154.2000) -- (126.6000,154.4000) -- (126.8000,154.4000) -- (127.0000,159.8000) -- (127.2000,161.1000) -- (127.4000,160.7000) -- (127.6000,160.6000) -- (127.8000,160.6000) -- (128.0000,160.6000) -- (128.2000,160.6000) -- (128.4000,160.6000) -- (128.6000,160.6000) -- (128.8000,160.6000) -- (129.0000,160.6000) -- (129.2000,160.6000) -- (129.4000,160.6000) -- (129.6000,160.6000) -- (129.8000,160.6000) -- (130.0000,160.6000) -- (130.2000,160.6000) -- (130.4000,160.6000) -- (130.5000,160.7000) -- (130.7000,161.7000) -- (130.9000,163.4000) -- (131.1000,165.5000) -- (131.3000,163.4000) -- (131.5000,161.8000) -- (131.7000,163.0000) -- (131.9000,158.9000) -- (132.1000,160.1000) -- (132.3000,161.9000) -- (132.5000,163.6000) -- (132.7000,164.1000) -- (132.9000,159.7000) -- (133.1000,160.7000) -- (133.3000,161.9000) -- (133.5000,162.9000) -- (133.7000,163.7000) -- (133.9000,164.3000) -- (134.1000,164.6000) -- (134.3000,164.6000) -- (134.5000,164.2000) -- (134.7000,165.2000) -- (134.9000,170.0000) -- (135.1000,169.0000) -- (135.2000,167.8000) -- (135.4000,166.2000) -- (135.6000,166.4000) -- (135.8000,170.3000) -- (136.0000,168.4000) -- (136.2000,171.9000) -- (136.4000,172.4000) -- (136.6000,170.3000) -- (136.8000,167.8000) -- (137.0000,165.0000) -- (137.2000,167.9000) -- (137.4000,168.1000) -- (137.6000,166.9000) -- (137.8000,166.6000) -- (138.0000,166.7000) -- (138.2000,166.9000) -- (138.4000,167.2000) -- (138.6000,167.4000) -- (138.8000,167.5000) -- (139.0000,167.6000) -- (139.2000,167.7000) -- (139.4000,167.7000) -- (139.6000,167.7000) -- (139.8000,167.7000) -- (140.0000,167.6000) -- (140.1000,167.6000) -- (140.3000,167.6000) -- (140.5000,167.6000) -- (140.7000,167.6000) -- (140.9000,167.6000) -- (141.1000,167.6000) -- (141.3000,167.7000) -- (141.5000,169.6000) -- (141.7000,171.2000) -- (141.9000,170.6000) -- (142.1000,169.0000) -- (142.3000,167.1000) -- (142.5000,165.1000) -- (142.7000,163.3000) -- (142.9000,161.8000) -- (143.1000,160.5000) -- (143.3000,159.3000) -- (143.5000,158.3000) -- (143.7000,157.5000) -- (143.9000,156.8000) -- (144.1000,156.1000) -- (144.3000,155.6000) -- (144.5000,155.2000) -- (144.7000,154.8000) -- (144.9000,154.6000) -- (145.0000,154.4000) -- (145.2000,154.3000) -- (145.4000,154.3000) -- (145.6000,154.4000) -- (145.8000,154.6000) -- (146.0000,154.8000) -- (146.2000,155.2000) -- (146.4000,155.7000) -- (146.6000,156.2000) -- (146.8000,156.9000) -- (147.0000,157.6000) -- (147.2000,158.4000) -- (147.4000,159.4000) -- (147.6000,160.4000) -- (147.8000,161.6000) -- (148.0000,162.8000) -- (148.2000,164.5000) -- (148.4000,166.2000) -- (148.6000,167.5000) -- (148.8000,168.2000) -- (149.0000,168.4000) -- (149.2000,168.4000) -- (149.4000,168.2000) -- (149.6000,168.0000) -- (149.8000,167.8000) -- (149.9000,167.7000) -- (150.1000,167.6000) -- (150.3000,167.6000) -- (150.5000,167.6000) -- (150.7000,167.6000) -- (150.9000,167.6000) -- (151.1000,167.6000) -- (151.3000,167.6000) -- (151.5000,167.6000) -- (151.7000,167.6000) -- (151.9000,167.6000) -- (152.1000,167.6000) -- (152.3000,167.6000) -- (152.5000,167.6000) -- (152.7000,167.6000) -- (152.9000,167.7000) -- (153.1000,168.6000) -- (153.3000,170.3000) -- (153.5000,172.2000) -- (153.7000,174.1000) -- (153.9000,176.0000) -- (154.1000,177.9000) -- (154.3000,179.6000) -- (154.5000,181.4000) -- (154.6000,183.0000) -- (154.8000,184.6000) -- (155.0000,186.2000) -- (155.2000,187.6000) -- (155.4000,189.0000) -- (155.6000,190.2000) -- (155.8000,191.2000) -- (156.0000,192.0000) -- (156.2000,192.5000) -- (156.4000,192.8000) -- (156.6000,192.8000) -- (156.8000,192.4000) -- (157.0000,191.8000) -- (157.2000,190.9000) -- (157.4000,189.8000) -- (157.6000,188.5000) -- (157.8000,187.0000) -- (158.0000,185.5000) -- (158.2000,183.8000) -- (158.4000,182.1000) -- (158.6000,180.3000) -- (158.8000,178.6000) -- (159.0000,176.5000) -- (159.2000,174.0000) -- (159.3000,171.4000) -- (159.5000,169.2000) -- (159.7000,167.7000) -- (159.9000,166.9000) -- (160.1000,166.7000) -- (160.3000,166.8000) -- (160.5000,167.0000) -- (160.7000,167.2000) -- (160.9000,167.4000) -- (161.1000,167.6000) -- (161.3000,167.6000) -- (161.5000,167.7000) -- (161.7000,167.7000) -- (161.9000,167.7000) -- (162.1000,167.6000) -- (162.3000,167.6000) -- (162.5000,167.6000) -- (162.7000,167.6000) -- (162.9000,167.6000) -- (163.1000,167.6000) -- (163.3000,167.6000) -- (163.5000,167.6000) -- (163.7000,168.2000) -- (163.9000,170.6000) -- (164.1000,171.2000) -- (164.2000,170.0000) -- (164.4000,168.2000) -- (164.6000,166.2000) -- (164.8000,164.3000) -- (165.0000,162.6000) -- (165.2000,161.2000) -- (165.4000,159.9000) -- (165.6000,158.8000) -- (165.8000,157.9000) -- (166.0000,157.1000) -- (166.2000,156.5000) -- (166.4000,155.9000) -- (166.6000,155.4000) -- (166.8000,155.0000) -- (167.0000,154.7000) -- (167.2000,154.5000) -- (167.4000,154.3000) -- (167.6000,154.3000) -- (167.8000,154.3000) -- (168.0000,154.5000) -- (168.2000,154.7000) -- (168.4000,155.0000) -- (168.6000,155.5000) -- (168.8000,156.0000) -- (169.0000,156.6000) -- (169.1000,157.3000) -- (169.3000,158.1000) -- (169.5000,159.0000) -- (169.7000,160.0000) -- (169.9000,161.2000) -- (170.1000,162.4000) -- (170.3000,163.9000) -- (170.5000,165.7000) -- (170.7000,167.1000) -- (170.9000,168.0000) -- (171.1000,168.4000) -- (171.3000,168.4000) -- (171.5000,168.2000) -- (171.7000,168.0000) -- (171.9000,167.8000) -- (172.1000,167.7000) -- (172.3000,167.7000) -- (172.5000,167.6000) -- (172.7000,167.6000) -- (172.9000,167.6000) -- (173.1000,167.6000) -- (173.3000,167.6000) -- (173.5000,167.6000) -- (173.7000,167.6000) -- (173.9000,167.6000) -- (174.0000,167.6000) -- (174.2000,167.6000) -- (174.4000,167.6000) -- (174.6000,167.6000) -- (174.8000,167.6000) -- (175.0000,167.6000) -- (175.2000,168.2000) -- (175.4000,169.6000) -- (175.6000,171.5000) -- (175.8000,173.5000) -- (176.0000,175.5000) -- (176.2000,177.4000) -- (176.4000,179.2000) -- (176.6000,180.9000) -- (176.8000,182.6000) -- (177.0000,184.2000) -- (177.2000,185.8000) -- (177.4000,187.2000) -- (177.6000,188.6000) -- (177.8000,189.8000) -- (178.0000,190.9000) -- (178.2000,191.8000) -- (178.4000,192.4000) -- (178.6000,192.7000) -- (178.7000,192.8000) -- (178.9000,192.5000) -- (179.1000,192.0000) -- (179.3000,191.2000) -- (179.5000,190.1000) -- (179.7000,188.8000) -- (179.9000,187.3000) -- (180.1000,185.7000) -- (180.3000,184.1000) -- (180.5000,182.4000) -- (180.7000,180.6000) -- (180.9000,178.9000) -- (181.1000,176.9000) -- (181.3000,174.5000) -- (181.5000,171.9000) -- (181.7000,169.6000) -- (181.9000,168.0000) -- (182.1000,167.0000) -- (182.3000,166.7000) -- (182.5000,166.8000) -- (182.7000,167.0000) -- (182.9000,167.2000) -- (183.1000,167.4000) -- (183.3000,167.5000) -- (183.4000,167.6000) -- (183.6000,167.7000) -- (183.8000,167.7000) -- (184.0000,167.7000) -- (184.2000,167.7000) -- (184.4000,167.6000) -- (184.6000,167.6000) -- (184.8000,167.6000) -- (185.0000,167.6000) -- (185.2000,167.6000) -- (185.4000,167.6000) -- (185.6000,167.6000) -- (185.8000,168.0000) -- (186.0000,170.4000) -- (186.2000,171.2000) -- (186.4000,170.3000) -- (186.6000,168.5000) -- (186.8000,166.5000) -- (187.0000,164.6000) -- (187.2000,162.8000) -- (187.4000,161.2000) -- (187.6000,159.9000) -- (187.8000,158.7000) -- (188.0000,157.7000) -- (188.2000,156.9000) -- (188.3000,156.2000) -- (188.5000,155.6000) -- (188.7000,155.1000) -- (188.9000,154.7000) -- (189.1000,154.5000) -- (189.3000,154.3000) -- (189.5000,154.2000) -- (189.7000,154.3000) -- (189.9000,154.4000) -- (190.1000,154.7000) -- (190.3000,155.0000) -- (190.5000,155.5000) -- (190.7000,156.0000) -- (190.9000,156.7000) -- (191.1000,157.5000) -- (191.3000,158.4000) -- (191.5000,159.4000) -- (191.7000,160.6000) -- (191.9000,161.8000) -- (192.1000,163.3000) -- (192.3000,165.2000) -- (192.5000,166.8000) -- (192.7000,167.8000) -- (192.9000,168.3000) -- (193.1000,168.4000) -- (193.2000,168.3000) -- (193.4000,168.1000) -- (193.6000,167.9000) -- (193.8000,167.8000) -- (194.0000,167.7000) -- (194.2000,167.6000) -- (194.4000,167.6000) -- (194.6000,167.6000) -- (194.8000,167.6000) -- (195.0000,167.6000) -- (195.2000,167.6000) -- (195.4000,167.6000) -- (195.6000,167.6000) -- (195.8000,167.6000) -- (196.0000,167.6000) -- (196.2000,167.6000) -- (196.4000,167.6000) -- (196.6000,167.6000) -- (196.8000,167.6000) -- (197.0000,168.0000) -- (197.2000,169.3000) -- (197.4000,171.2000) -- (197.6000,173.2000) -- (197.8000,175.2000) -- (197.9000,177.1000) -- (198.1000,178.9000) -- (198.3000,180.7000) -- (198.5000,182.3000) -- (198.7000,184.0000) -- (198.9000,185.5000) -- (199.1000,187.0000) -- (199.3000,188.4000) -- (199.5000,189.7000) -- (199.7000,190.8000) -- (199.9000,191.7000) -- (200.1000,192.3000) -- (200.3000,192.7000) -- (200.5000,192.8000) -- (200.7000,192.6000) -- (200.9000,192.1000) -- (201.1000,191.3000) -- (201.3000,190.2000) -- (201.5000,188.9000) -- (201.7000,187.5000) -- (201.9000,185.9000) -- (202.1000,184.3000) -- (202.3000,182.5000) -- (202.5000,180.8000) -- (202.7000,179.0000) -- (202.8000,177.1000) -- (203.0000,174.7000) -- (203.2000,172.1000) -- (203.4000,169.8000) -- (203.6000,168.1000) -- (203.8000,167.1000) -- (204.0000,166.7000) -- (204.2000,166.7000) -- (204.4000,166.9000) -- (204.6000,167.2000) -- (204.8000,167.4000) -- (205.0000,167.5000) -- (205.2000,167.6000) -- (205.4000,167.7000) -- (205.6000,167.7000) -- (205.8000,167.7000) -- (206.0000,167.7000) -- (206.2000,167.6000) -- (206.4000,167.6000) -- (206.6000,167.6000) -- (206.8000,167.6000) -- (207.0000,167.6000) -- (207.2000,167.6000) -- (207.4000,167.6000) -- (207.5000,167.8000) -- (207.7000,170.3000) -- (207.9000,171.6000) -- (208.1000,170.8000) -- (208.3000,169.0000) -- (208.5000,166.9000) -- (208.7000,164.8000) -- (208.9000,162.9000) -- (209.1000,161.3000) -- (209.3000,160.0000) -- (209.5000,158.8000) -- (209.7000,157.8000) -- (209.9000,157.0000) -- (210.1000,156.3000) -- (210.3000,155.7000) -- (210.5000,155.2000) -- (210.7000,154.8000) -- (210.9000,154.5000) -- (211.1000,154.3000) -- (211.3000,154.3000) -- (211.5000,154.3000) -- (211.7000,154.4000) -- (211.9000,154.6000) -- (212.1000,154.9000) -- (212.3000,155.3000) -- (212.4000,155.9000) -- (212.6000,156.5000) -- (212.8000,157.2000) -- (213.0000,158.1000) -- (213.2000,159.0000) -- (213.4000,160.1000) -- (213.6000,161.3000) -- (213.8000,162.7000) -- (214.0000,164.4000) -- (214.2000,166.3000) -- (214.4000,167.6000) -- (214.6000,168.3000) -- (214.8000,168.5000) -- (215.0000,168.4000) -- (215.2000,168.2000) -- (215.4000,168.0000) -- (215.6000,167.8000) -- (215.8000,167.7000) -- (216.0000,167.6000) -- (216.2000,167.6000) -- (216.4000,167.6000) -- (216.6000,167.6000) -- (216.8000,167.6000) -- (217.0000,167.6000) -- (217.1000,167.6000) -- (217.3000,167.6000) -- (217.5000,167.6000) -- (217.7000,167.6000) -- (217.9000,167.6000) -- (218.1000,167.6000) -- (218.3000,167.6000) -- (218.5000,167.6000) -- (218.7000,167.7000) -- (218.9000,168.7000) -- (219.1000,170.4000) -- (219.3000,172.4000) -- (219.5000,174.3000) -- (219.7000,176.2000) -- (219.9000,178.1000) -- (220.1000,179.8000) -- (220.3000,181.6000) -- (220.5000,183.2000) -- (220.7000,184.8000) -- (220.9000,186.4000) -- (221.1000,187.8000) -- (221.3000,189.1000) -- (221.5000,190.3000) -- (221.7000,191.3000) -- (221.9000,192.0000) -- (222.0000,192.5000) -- (222.2000,192.8000) -- (222.4000,192.7000) -- (222.6000,192.4000) -- (222.8000,191.7000) -- (223.0000,190.8000) -- (223.2000,189.7000) -- (223.4000,188.3000) -- (223.6000,186.8000) -- (223.8000,185.2000) -- (224.0000,183.6000) -- (224.2000,181.8000) -- (224.4000,180.1000) -- (224.6000,178.3000) -- (224.8000,176.2000) -- (225.0000,173.7000) -- (225.2000,171.1000) -- (225.4000,169.0000) -- (225.6000,167.6000) -- (225.8000,166.9000) -- (226.0000,166.7000) -- (226.2000,166.8000) -- (226.4000,167.0000) -- (226.6000,167.3000) -- (226.8000,167.4000) -- (226.9000,167.6000) -- (227.1000,167.6000) -- (227.3000,167.7000) -- (227.5000,167.7000) -- (227.7000,167.7000) -- (227.9000,167.6000) -- (228.1000,167.6000) -- (228.3000,167.6000) -- (228.5000,167.6000) -- (228.7000,167.6000) -- (228.9000,167.6000) -- (229.1000,167.6000) -- (229.3000,167.5000) -- (229.5000,168.5000) -- (229.7000,171.2000) -- (229.9000,171.5000) -- (230.1000,170.2000) -- (230.3000,168.2000) -- (230.5000,166.1000) -- (230.7000,164.1000) -- (230.9000,162.3000) -- (231.1000,160.8000) -- (231.3000,159.5000) -- (231.4000,158.4000) -- (231.6000,157.5000) -- (231.8000,156.7000) -- (232.0000,156.1000) -- (232.2000,155.5000) -- (232.4000,155.1000) -- (232.6000,154.7000) -- (232.8000,154.5000) -- (233.0000,154.3000) -- (233.2000,154.2000) -- (233.4000,154.3000) -- (233.6000,154.4000) -- (233.8000,154.7000) -- (234.0000,155.0000) -- (234.2000,155.5000) -- (234.4000,156.1000) -- (234.6000,156.7000) -- (234.8000,157.5000) -- (235.0000,158.4000) -- (235.2000,159.4000) -- (235.4000,160.5000) -- (235.6000,161.7000) -- (235.8000,163.2000) -- (236.0000,165.1000) -- (236.2000,166.8000) -- (236.3000,167.9000) -- (236.5000,168.4000) -- (236.7000,168.5000) -- (236.9000,168.3000) -- (237.1000,168.1000) -- (237.3000,167.9000) -- (237.5000,167.8000) -- (237.7000,167.7000) -- (237.9000,167.6000) -- (238.1000,167.6000) -- (238.3000,167.6000) -- (238.5000,167.6000) -- (238.7000,167.6000) -- (238.9000,167.6000) -- (239.1000,167.6000) -- (239.3000,167.6000) -- (239.5000,167.6000) -- (239.7000,167.6000) -- (239.9000,167.6000) -- (240.1000,167.6000) -- (240.3000,167.6000) -- (240.5000,167.6000) -- (240.7000,168.0000) -- (240.9000,169.3000) -- (241.1000,171.1000) -- (241.2000,173.0000) -- (241.4000,174.9000) -- (241.6000,176.8000) -- (241.8000,178.6000) -- (242.0000,180.4000) -- (242.2000,182.1000) -- (242.4000,183.7000) -- (242.6000,185.3000) -- (242.8000,186.8000) -- (243.0000,188.2000) -- (243.2000,189.5000) -- (243.4000,190.6000) -- (243.6000,191.5000) -- (243.8000,192.2000) -- (244.0000,192.7000) -- (244.2000,192.8000) -- (244.4000,192.6000) -- (244.6000,192.2000) -- (244.8000,191.5000) -- (245.0000,190.5000) -- (245.2000,189.3000) -- (245.4000,187.9000) -- (245.6000,186.4000) -- (245.8000,184.7000) -- (246.0000,183.0000) -- (246.1000,181.3000) -- (246.3000,179.6000) -- (246.5000,177.8000) -- (246.7000,175.5000) -- (246.9000,172.9000) -- (247.1000,170.4000) -- (247.3000,168.5000) -- (247.5000,167.3000) -- (247.7000,166.8000) -- (247.9000,166.7000) -- (248.1000,166.9000) -- (248.3000,167.1000) -- (248.5000,167.3000) -- (248.7000,167.5000) -- (248.9000,167.6000) -- (249.1000,167.6000) -- (249.3000,167.7000) -- (249.5000,167.7000) -- (249.7000,167.7000) -- (249.9000,167.6000) -- (250.1000,167.6000) -- (250.3000,167.6000) -- (250.5000,167.6000) -- (250.7000,167.6000) -- (250.9000,167.6000) -- (251.0000,167.6000) -- (251.2000,167.6000) -- (251.4000,169.4000) -- (251.6000,171.5000) -- (251.8000,171.2000);



  \end{scope}
  \begin{scope}[scale=1.006,draw=blue,line cap=rect,line join=bevel,line width=0.800pt]
  \end{scope}
  \begin{scope}[cm={{1.00588,0.0,0.0,1.00588,(197.153,129.759)}},draw=blue,line cap=rect,line join=bevel,line width=0.800pt]
  \end{scope}
  \begin{scope}[cm={{1.00588,0.0,0.0,1.00588,(197.153,129.759)}},draw=blue,line cap=rect,line join=bevel,line width=0.800pt]
  \end{scope}
  \begin{scope}[cm={{1.00588,0.0,0.0,1.00588,(197.153,129.759)}},draw=blue,line cap=rect,line join=bevel,line width=0.800pt]
  \end{scope}
  \begin{scope}[cm={{1.00588,0.0,0.0,1.00588,(197.153,129.759)}},draw=blue,line cap=rect,line join=bevel,line width=0.800pt]
  \end{scope}
  \begin{scope}[cm={{1.00588,0.0,0.0,1.00588,(197.153,129.759)}},draw=blue,line cap=rect,line join=bevel,line width=0.800pt]
  \end{scope}
  \begin{scope}[cm={{1.04164,0.0,0.0,1.04164,(-142.48887,167.27602)}},draw=blue,line cap=rect,line join=bevel,line width=0.800pt]
    \path[fill=blue] (0.0000,0.0000) node[above right] (text1032) {\scriptsize $b_0(t)$};



  \end{scope}
  \begin{scope}[cm={{1.00588,0.0,0.0,1.00588,(197.153,129.759)}},draw=blue,line cap=rect,line join=bevel,line width=0.800pt]
  \end{scope}
  \begin{scope}[scale=1.006,draw=blue,line cap=rect,line join=bevel,line width=0.800pt]
  \end{scope}
  \begin{scope}[scale=1.006,draw=blue,line cap=rect,line join=bevel,line width=0.800pt]
  \end{scope}
  \begin{scope}[scale=1.006,draw=blue,line cap=rect,line join=bevel,line width=0.800pt]
  \end{scope}
  \begin{scope}[scale=1.006,draw=blue,line cap=rect,line join=bevel,line width=0.800pt]
  \end{scope}
  \begin{scope}[scale=1.006,draw=blue,line cap=rect,line join=bevel,line width=0.800pt]
  \end{scope}
  \begin{scope}[cm={{1.04177,0.0,0.0,1.04177,(-342.44493,31.82443)}},draw=c00ff00,line cap=round,line join=round,line width=0.480pt]
    \path[draw] (101.0000,115.7000) -- (101.2000,122.5000) -- (101.7000,123.1000) -- (102.2000,123.7000) -- (102.7000,124.3000) -- (103.2000,124.9000) -- (103.6000,125.5000) -- (104.1000,126.1000) -- (104.6000,126.7000) -- (105.1000,127.3000) -- (105.6000,127.9000) -- (106.1000,128.4000) -- (106.6000,129.0000) -- (107.0000,129.6000) -- (107.5000,130.2000) -- (108.0000,130.8000) -- (108.5000,131.4000) -- (109.0000,132.0000) -- (109.5000,132.6000) -- (110.0000,133.2000) -- (110.4000,133.8000) -- (110.9000,134.3000) -- (111.4000,134.9000) -- (111.9000,135.5000) -- (112.4000,136.1000) -- (112.9000,136.7000) -- (113.4000,137.3000) -- (113.9000,137.9000) -- (114.3000,138.5000) -- (114.8000,139.1000) -- (115.3000,139.7000) -- (115.8000,140.2000) -- (116.3000,140.8000) -- (116.8000,141.4000) -- (117.3000,142.0000) -- (117.7000,142.6000) -- (118.2000,143.2000) -- (118.7000,143.8000) -- (119.2000,144.4000) -- (119.7000,145.0000) -- (120.2000,145.6000) -- (120.7000,146.1000) -- (121.1000,146.7000) -- (121.6000,147.3000) -- (122.1000,147.9000) -- (122.6000,148.5000) -- (123.1000,149.1000) -- (123.6000,149.7000) -- (124.1000,150.3000) -- (124.5000,150.9000) -- (125.0000,151.5000) -- (125.5000,152.0000) -- (126.0000,152.6000) -- (126.5000,153.2000) -- (127.0000,153.8000) -- (127.5000,154.4000) -- (128.0000,155.0000) -- (128.4000,155.6000) -- (128.9000,156.2000) -- (129.4000,156.8000) -- (129.9000,157.4000) -- (130.4000,157.9000) -- (130.9000,158.5000) -- (131.4000,159.1000) -- (131.8000,159.7000) -- (132.3000,160.3000) -- (132.8000,160.9000) -- (133.3000,161.5000) -- (133.8000,162.1000) -- (134.3000,162.7000) -- (134.8000,163.3000) -- (135.2000,163.9000) -- (135.7000,164.4000) -- (136.2000,165.0000) -- (136.7000,165.6000) -- (137.2000,166.2000) -- (137.7000,166.8000) -- (138.2000,167.4000) -- (138.6000,168.0000) -- (139.1000,168.6000) -- (139.6000,169.2000) -- (140.1000,169.7000) -- (140.6000,170.3000) -- (141.1000,170.9000) -- (141.6000,171.5000) -- (142.1000,172.1000) -- (142.5000,172.7000) -- (143.0000,173.3000) -- (143.5000,173.9000) -- (144.0000,174.5000) -- (144.5000,175.1000) -- (145.0000,175.7000) -- (145.5000,176.2000) -- (145.9000,176.8000) -- (146.4000,177.4000) -- (146.9000,178.0000) -- (147.4000,178.6000) -- (147.9000,179.2000) -- (148.4000,179.8000) -- (148.9000,180.4000) -- (149.3000,181.0000) -- (149.8000,181.6000) -- (150.3000,182.1000) -- (150.8000,182.7000) -- (151.3000,183.3000) -- (151.8000,183.9000) -- (152.3000,184.5000) -- (152.7000,185.1000) -- (153.2000,185.7000) -- (153.7000,186.3000) -- (154.2000,186.9000) -- (154.7000,187.5000) -- (155.2000,188.0000) -- (155.7000,188.6000) -- (156.2000,189.2000) -- (156.6000,189.8000) -- (157.1000,190.4000) -- (157.6000,191.0000) -- (158.1000,191.6000) -- (158.6000,192.2000) -- (159.1000,192.8000) -- (159.6000,193.4000) -- (160.0000,193.9000) -- (160.5000,194.5000) -- (161.0000,195.1000) -- (161.5000,195.7000) -- (162.0000,196.3000) -- (162.5000,196.9000) -- (163.0000,197.5000) -- (163.4000,198.1000) -- (163.9000,198.7000) -- (164.4000,199.3000) -- (164.9000,199.8000) -- (165.4000,200.4000) -- (165.5000,200.6000);



  \end{scope}
  \begin{scope}[scale=1.006,draw=blue,line cap=rect,line join=bevel,line width=0.800pt]
  \end{scope}
  \begin{scope}[scale=1.006,draw=blue,line cap=rect,line join=bevel,line width=0.800pt]
  \end{scope}
  \begin{scope}[cm={{1.04177,0.0,0.0,1.04177,(-342.44493,30.31402)}},draw=blue,line cap=round,line join=round,line width=0.480pt]
    \path[shift={(0,1.44988)},draw] (56.5000,115.5000) -- (56.5000,200.5000) -- (251.5000,200.5000) -- (251.5000,115.5000) -- (56.5000,115.5000);



  \end{scope}
  \begin{scope}[cm={{1.04164,0.0,0.0,1.04164,(-342.4953,42.14671)}},draw=ca0a0a4,dash pattern=on 1.73pt off 1.73pt,line cap=round,line join=round,line width=0.288pt,miter limit=4.00]
    \path[draw,dash pattern=on 1.73pt off 1.73pt,line width=0.288pt,miter limit=4.00] (56.5000,299.5000) -- (251.5000,299.5000);



  \end{scope}
  \begin{scope}[cm={{1.04164,0.0,0.0,1.04164,(-342.4953,42.14671)}},draw=blue,line cap=round,line join=round,line width=0.480pt]
    \path[cm={{1.11163,0.0,0.0,1.0,(-6.27372,0.0)}},draw] (56.5000,299.5000) -- (59.5000,299.5000);



    \path[cm={{1.11163,0.0,0.0,1.0,(-28.25871,0.0)}},draw] (251.5000,299.5000) -- (248.5000,299.5000);



  \end{scope}
  \begin{scope}[scale=1.006,draw=blue,line cap=rect,line join=bevel,line width=0.800pt]
  \end{scope}
  \begin{scope}[cm={{1.00588,0.0,0.0,1.00588,(39.2294,305.788)}},draw=blue,line cap=rect,line join=bevel,line width=0.800pt]
  \end{scope}
  \begin{scope}[cm={{1.00588,0.0,0.0,1.00588,(39.2294,305.788)}},draw=blue,line cap=rect,line join=bevel,line width=0.800pt]
  \end{scope}
  \begin{scope}[cm={{1.00588,0.0,0.0,1.00588,(39.2294,305.788)}},draw=blue,line cap=rect,line join=bevel,line width=0.800pt]
  \end{scope}
  \begin{scope}[cm={{1.00588,0.0,0.0,1.00588,(39.2294,305.788)}},draw=blue,line cap=rect,line join=bevel,line width=0.800pt]
  \end{scope}
  \begin{scope}[cm={{1.00588,0.0,0.0,1.00588,(39.2294,305.788)}},draw=blue,line cap=rect,line join=bevel,line width=0.800pt]
  \end{scope}
  \begin{scope}[cm={{1.00588,0.0,0.0,1.00588,(-298.32402,357.32974)}},draw=blue,line cap=rect,line join=bevel,line width=0.800pt]
    \path[fill=blue] (0.0000,0.0000) node[above right] (text1086) {27};



  \end{scope}
  \begin{scope}[cm={{1.00588,0.0,0.0,1.00588,(39.2294,305.788)}},draw=blue,line cap=rect,line join=bevel,line width=0.800pt]
  \end{scope}
  \begin{scope}[scale=1.006,draw=blue,line cap=rect,line join=bevel,line width=0.800pt]
  \end{scope}
  \begin{scope}[cm={{1.04164,0.0,0.0,1.04164,(-342.4953,42.14671)}},draw=ca0a0a4,dash pattern=on 1.73pt off 1.73pt,line cap=round,line join=round,line width=0.288pt,miter limit=4.00]
    \path[draw,dash pattern=on 1.73pt off 1.73pt,line width=0.288pt,miter limit=4.00] (56.5000,277.5000) -- (251.5000,277.5000);



  \end{scope}
  \begin{scope}[cm={{1.04164,0.0,0.0,1.04164,(-342.4953,42.14671)}},draw=blue,line cap=round,line join=round,line width=0.480pt]
    \path[cm={{1.11163,0.0,0.0,1.0,(-6.27372,0.0)}},draw] (56.5000,277.5000) -- (59.5000,277.5000);



    \path[cm={{1.11163,0.0,0.0,1.0,(-28.25871,0.0)}},draw] (251.5000,277.5000) -- (248.5000,277.5000);



  \end{scope}
  \begin{scope}[scale=1.006,draw=blue,line cap=rect,line join=bevel,line width=0.800pt]
  \end{scope}
  \begin{scope}[cm={{1.00588,0.0,0.0,1.00588,(40.2353,282.653)}},draw=blue,line cap=rect,line join=bevel,line width=0.800pt]
  \end{scope}
  \begin{scope}[cm={{1.00588,0.0,0.0,1.00588,(40.2353,282.653)}},draw=blue,line cap=rect,line join=bevel,line width=0.800pt]
  \end{scope}
  \begin{scope}[cm={{1.00588,0.0,0.0,1.00588,(40.2353,282.653)}},draw=blue,line cap=rect,line join=bevel,line width=0.800pt]
  \end{scope}
  \begin{scope}[cm={{1.00588,0.0,0.0,1.00588,(40.2353,282.653)}},draw=blue,line cap=rect,line join=bevel,line width=0.800pt]
  \end{scope}
  \begin{scope}[cm={{1.00588,0.0,0.0,1.00588,(40.2353,282.653)}},draw=blue,line cap=rect,line join=bevel,line width=0.800pt]
  \end{scope}
  \begin{scope}[cm={{1.00588,0.0,0.0,1.00588,(-298.23551,334.19474)}},draw=blue,line cap=rect,line join=bevel,line width=0.800pt]
    \path[fill=blue] (0.3520,0.0000) node[above right] (text1116) {31};



  \end{scope}
  \begin{scope}[cm={{1.00588,0.0,0.0,1.00588,(40.2353,282.653)}},draw=blue,line cap=rect,line join=bevel,line width=0.800pt]
  \end{scope}
  \begin{scope}[scale=1.006,draw=blue,line cap=rect,line join=bevel,line width=0.800pt]
  \end{scope}
  \begin{scope}[cm={{1.04164,0.0,0.0,1.04164,(-342.4953,42.14671)}},draw=ca0a0a4,dash pattern=on 1.73pt off 1.73pt,line cap=round,line join=round,line width=0.288pt,miter limit=4.00]
    \path[draw,dash pattern=on 1.73pt off 1.73pt,line width=0.288pt,miter limit=4.00] (56.5000,254.5000) -- (251.5000,254.5000);



  \end{scope}
  \begin{scope}[cm={{1.04164,0.0,0.0,1.04164,(-342.4953,42.14671)}},draw=blue,line cap=round,line join=round,line width=0.480pt]
    \path[cm={{1.11163,0.0,0.0,1.0,(-6.27372,0.0)}},draw] (56.5000,254.5000) -- (59.5000,254.5000);



    \path[cm={{1.11163,0.0,0.0,1.0,(-28.25871,0.0)}},draw] (251.5000,254.5000) -- (248.5000,254.5000);



  \end{scope}
  \begin{scope}[scale=1.006,draw=blue,line cap=rect,line join=bevel,line width=0.800pt]
  \end{scope}
  \begin{scope}[cm={{1.00588,0.0,0.0,1.00588,(40.2353,260.524)}},draw=blue,line cap=rect,line join=bevel,line width=0.800pt]
  \end{scope}
  \begin{scope}[cm={{1.00588,0.0,0.0,1.00588,(40.2353,260.524)}},draw=blue,line cap=rect,line join=bevel,line width=0.800pt]
  \end{scope}
  \begin{scope}[cm={{1.00588,0.0,0.0,1.00588,(40.2353,260.524)}},draw=blue,line cap=rect,line join=bevel,line width=0.800pt]
  \end{scope}
  \begin{scope}[cm={{1.00588,0.0,0.0,1.00588,(40.2353,260.524)}},draw=blue,line cap=rect,line join=bevel,line width=0.800pt]
  \end{scope}
  \begin{scope}[cm={{1.00588,0.0,0.0,1.00588,(40.2353,260.524)}},draw=blue,line cap=rect,line join=bevel,line width=0.800pt]
  \end{scope}
  \begin{scope}[cm={{1.00588,0.0,0.0,1.00588,(-298.23551,309.06574)}},draw=blue,line cap=rect,line join=bevel,line width=0.800pt]
    \path[fill=blue] (0.0000,0.0000) node[above right] (text1146) {35};



  \end{scope}
  \begin{scope}[cm={{1.00588,0.0,0.0,1.00588,(40.2353,260.524)}},draw=blue,line cap=rect,line join=bevel,line width=0.800pt]
  \end{scope}
  \begin{scope}[scale=1.006,draw=blue,line cap=rect,line join=bevel,line width=0.800pt]
  \end{scope}
  \begin{scope}[cm={{1.04164,0.0,0.0,1.04164,(-342.4953,42.14671)}},draw=blue,line cap=round,line join=round,line width=0.480pt]
    \path[cm={{1.11163,0.0,0.0,1.0,(-6.27372,0.0)}},draw] (56.5000,232.5000) -- (59.5000,232.5000);



    \path[cm={{1.11163,0.0,0.0,1.0,(-28.25871,0.0)}},draw] (251.5000,232.5000) -- (248.5000,232.5000);



  \end{scope}
  \begin{scope}[scale=1.006,draw=blue,line cap=rect,line join=bevel,line width=0.800pt]
  \end{scope}
  \begin{scope}[cm={{1.00588,0.0,0.0,1.00588,(39.2294,237.388)}},draw=blue,line cap=rect,line join=bevel,line width=0.800pt]
  \end{scope}
  \begin{scope}[cm={{1.00588,0.0,0.0,1.00588,(39.2294,237.388)}},draw=blue,line cap=rect,line join=bevel,line width=0.800pt]
  \end{scope}
  \begin{scope}[cm={{1.00588,0.0,0.0,1.00588,(39.2294,237.388)}},draw=blue,line cap=rect,line join=bevel,line width=0.800pt]
  \end{scope}
  \begin{scope}[cm={{1.00588,0.0,0.0,1.00588,(39.2294,237.388)}},draw=blue,line cap=rect,line join=bevel,line width=0.800pt]
  \end{scope}
  \begin{scope}[cm={{1.00588,0.0,0.0,1.00588,(39.2294,237.388)}},draw=blue,line cap=rect,line join=bevel,line width=0.800pt]
  \end{scope}
  \begin{scope}[cm={{1.00588,0.0,0.0,1.00588,(-298.4045,287.42974)}},draw=blue,line cap=rect,line join=bevel,line width=0.800pt]
    \path[fill=blue] (0.0000,0.0000) node[above right] (text1176) {39};



  \end{scope}
  \begin{scope}[cm={{1.00588,0.0,0.0,1.00588,(39.2294,237.388)}},draw=blue,line cap=rect,line join=bevel,line width=0.800pt]
  \end{scope}
  \begin{scope}[scale=1.006,draw=blue,line cap=rect,line join=bevel,line width=0.800pt]
  \end{scope}
  \begin{scope}[cm={{1.04164,0.0,0.0,1.04164,(-342.4953,42.14671)}},draw=ca0a0a4,dash pattern=on 0.40pt off 0.80pt,line cap=round,line join=round,line width=0.400pt]
    \path[draw] (56.5000,306.5000) -- (56.5000,221.5000);



  \end{scope}
  \begin{scope}[cm={{1.04164,0.0,0.0,1.04164,(-342.4953,42.14671)}},draw=blue,line cap=round,line join=round,line width=0.480pt]
    \path[draw] (56.5000,306.5000) -- (56.5000,303.5000);



    \path[draw] (56.5000,221.5000) -- (56.5000,223.5000);



  \end{scope}
  \begin{scope}[scale=1.006,draw=blue,line cap=rect,line join=bevel,line width=0.800pt]
  \end{scope}
  \begin{scope}[cm={{1.00588,0.0,0.0,1.00588,(53.3118,322.888)}},draw=blue,line cap=rect,line join=bevel,line width=0.800pt]
  \end{scope}
  \begin{scope}[cm={{1.00588,0.0,0.0,1.00588,(53.3118,322.888)}},draw=blue,line cap=rect,line join=bevel,line width=0.800pt]
  \end{scope}
  \begin{scope}[cm={{1.00588,0.0,0.0,1.00588,(53.3118,322.888)}},draw=blue,line cap=rect,line join=bevel,line width=0.800pt]
  \end{scope}
  \begin{scope}[cm={{1.00588,0.0,0.0,1.00588,(53.3118,322.888)}},draw=blue,line cap=rect,line join=bevel,line width=0.800pt]
  \end{scope}
  \begin{scope}[cm={{1.00588,0.0,0.0,1.00588,(53.3118,322.888)}},draw=blue,line cap=rect,line join=bevel,line width=0.800pt]
  \end{scope}
  \begin{scope}[cm={{1.00588,0.0,0.0,1.00588,(-285.68826,376.888)}},draw=blue,line cap=rect,line join=bevel,line width=0.800pt]
    \path[fill=blue] (0.0000,0.0000) node[above right] (text1206) {0};



  \end{scope}
  \begin{scope}[cm={{1.00588,0.0,0.0,1.00588,(53.3118,322.888)}},draw=blue,line cap=rect,line join=bevel,line width=0.800pt]
  \end{scope}
  \begin{scope}[scale=1.006,draw=blue,line cap=rect,line join=bevel,line width=0.800pt]
  \end{scope}
  \begin{scope}[cm={{1.04164,0.0,0.0,1.04164,(-342.4953,42.14671)}},draw=ca0a0a4,dash pattern=on 1.73pt off 1.73pt,line cap=round,line join=round,line width=0.288pt,miter limit=4.00]
    \path[draw,dash pattern=on 1.73pt off 1.73pt,line width=0.288pt,miter limit=4.00] (91.5000,306.5000) -- (91.5000,221.5000);



  \end{scope}
  \begin{scope}[cm={{1.04164,0.0,0.0,1.04164,(-342.4953,42.14671)}},draw=blue,line cap=round,line join=round,line width=0.480pt]
    \path[cm={{1.0,0.0,0.0,1.11163,(0.0,-34.30653)}},draw] (91.5000,306.5000) -- (91.5000,303.5000);



    \path[cm={{1.0,0.0,0.0,1.53918,(0.0,-119.26725)}},draw] (91.5000,221.5000) -- (91.5000,223.5000);



  \end{scope}
  \begin{scope}[scale=1.006,draw=blue,line cap=rect,line join=bevel,line width=0.800pt]
  \end{scope}
  \begin{scope}[cm={{1.00588,0.0,0.0,1.00588,(89.5235,322.888)}},draw=blue,line cap=rect,line join=bevel,line width=0.800pt]
  \end{scope}
  \begin{scope}[cm={{1.00588,0.0,0.0,1.00588,(89.5235,322.888)}},draw=blue,line cap=rect,line join=bevel,line width=0.800pt]
  \end{scope}
  \begin{scope}[cm={{1.00588,0.0,0.0,1.00588,(89.5235,322.888)}},draw=blue,line cap=rect,line join=bevel,line width=0.800pt]
  \end{scope}
  \begin{scope}[cm={{1.00588,0.0,0.0,1.00588,(89.5235,322.888)}},draw=blue,line cap=rect,line join=bevel,line width=0.800pt]
  \end{scope}
  \begin{scope}[cm={{1.00588,0.0,0.0,1.00588,(89.5235,322.888)}},draw=blue,line cap=rect,line join=bevel,line width=0.800pt]
  \end{scope}
  \begin{scope}[cm={{1.00588,0.0,0.0,1.00588,(-247.97657,377.00066)}},draw=blue,line cap=rect,line join=bevel,line width=0.800pt]
    \path[fill=blue] (0.0000,0.0000) node[above right] (text1236) {2};



  \end{scope}
  \begin{scope}[cm={{1.00588,0.0,0.0,1.00588,(89.5235,322.888)}},draw=blue,line cap=rect,line join=bevel,line width=0.800pt]
  \end{scope}
  \begin{scope}[scale=1.006,draw=blue,line cap=rect,line join=bevel,line width=0.800pt]
  \end{scope}
  \begin{scope}[cm={{1.04164,0.0,0.0,1.04164,(-342.4953,42.14671)}},draw=ca0a0a4,dash pattern=on 1.73pt off 1.73pt,line cap=round,line join=round,line width=0.288pt,miter limit=4.00]
    \path[draw,dash pattern=on 1.73pt off 1.73pt,line width=0.288pt,miter limit=4.00] (127.5000,306.5000) -- (127.5000,221.5000);



  \end{scope}
  \begin{scope}[cm={{1.04164,0.0,0.0,1.04164,(-342.4953,42.14671)}},draw=blue,line cap=round,line join=round,line width=0.480pt]
    \path[cm={{1.0,0.0,0.0,1.11163,(0.0,-34.30653)}},draw] (127.5000,306.5000) -- (127.5000,303.5000);



    \path[cm={{1.0,0.0,0.0,1.53918,(0.0,-119.26725)}},draw] (127.5000,221.5000) -- (127.5000,223.5000);



  \end{scope}
  \begin{scope}[scale=1.006,draw=blue,line cap=rect,line join=bevel,line width=0.800pt]
  \end{scope}
  \begin{scope}[cm={{1.00588,0.0,0.0,1.00588,(125.232,322.888)}},draw=blue,line cap=rect,line join=bevel,line width=0.800pt]
  \end{scope}
  \begin{scope}[cm={{1.00588,0.0,0.0,1.00588,(125.232,322.888)}},draw=blue,line cap=rect,line join=bevel,line width=0.800pt]
  \end{scope}
  \begin{scope}[cm={{1.00588,0.0,0.0,1.00588,(125.232,322.888)}},draw=blue,line cap=rect,line join=bevel,line width=0.800pt]
  \end{scope}
  \begin{scope}[cm={{1.00588,0.0,0.0,1.00588,(125.232,322.888)}},draw=blue,line cap=rect,line join=bevel,line width=0.800pt]
  \end{scope}
  \begin{scope}[cm={{1.00588,0.0,0.0,1.00588,(125.232,322.888)}},draw=blue,line cap=rect,line join=bevel,line width=0.800pt]
  \end{scope}
  \begin{scope}[cm={{1.00588,0.0,0.0,1.00588,(-212.26806,376.888)}},draw=blue,line cap=rect,line join=bevel,line width=0.800pt]
    \path[fill=blue] (0.0000,0.1120) node[above right] (text1266) {4};



  \end{scope}
  \begin{scope}[cm={{1.00588,0.0,0.0,1.00588,(125.232,322.888)}},draw=blue,line cap=rect,line join=bevel,line width=0.800pt]
  \end{scope}
  \begin{scope}[scale=1.006,draw=blue,line cap=rect,line join=bevel,line width=0.800pt]
  \end{scope}
  \begin{scope}[cm={{1.04164,0.0,0.0,1.04164,(-342.4953,42.14671)}},draw=ca0a0a4,dash pattern=on 1.73pt off 1.73pt,line cap=round,line join=round,line width=0.288pt,miter limit=4.00]
    \path[draw,dash pattern=on 1.73pt off 1.73pt,line width=0.288pt,miter limit=4.00] (162.5000,306.5000) -- (162.5000,221.5000);



  \end{scope}
  \begin{scope}[cm={{1.04164,0.0,0.0,1.04164,(-342.4953,42.14671)}},draw=blue,line cap=round,line join=round,line width=0.480pt]
    \path[cm={{1.0,0.0,0.0,1.11163,(0.0,-34.30653)}},draw] (162.5000,306.5000) -- (162.5000,303.5000);



    \path[cm={{1.0,0.0,0.0,1.53918,(0.0,-119.26725)}},draw] (162.5000,221.5000) -- (162.5000,223.5000);



  \end{scope}
  \begin{scope}[scale=1.006,draw=blue,line cap=rect,line join=bevel,line width=0.800pt]
  \end{scope}
  \begin{scope}[cm={{1.00588,0.0,0.0,1.00588,(160.941,322.888)}},draw=blue,line cap=rect,line join=bevel,line width=0.800pt]
  \end{scope}
  \begin{scope}[cm={{1.00588,0.0,0.0,1.00588,(160.941,322.888)}},draw=blue,line cap=rect,line join=bevel,line width=0.800pt]
  \end{scope}
  \begin{scope}[cm={{1.00588,0.0,0.0,1.00588,(160.941,322.888)}},draw=blue,line cap=rect,line join=bevel,line width=0.800pt]
  \end{scope}
  \begin{scope}[cm={{1.00588,0.0,0.0,1.00588,(160.941,322.888)}},draw=blue,line cap=rect,line join=bevel,line width=0.800pt]
  \end{scope}
  \begin{scope}[cm={{1.00588,0.0,0.0,1.00588,(160.941,322.888)}},draw=blue,line cap=rect,line join=bevel,line width=0.800pt]
  \end{scope}
  \begin{scope}[cm={{1.00588,0.0,0.0,1.00588,(-173.55905,376.888)}},draw=blue,line cap=rect,line join=bevel,line width=0.800pt]
    \path[fill=blue] (0.0000,0.0000) node[above right] (text1296) {6};



  \end{scope}
  \begin{scope}[cm={{1.00588,0.0,0.0,1.00588,(160.941,322.888)}},draw=blue,line cap=rect,line join=bevel,line width=0.800pt]
  \end{scope}
  \begin{scope}[scale=1.006,draw=blue,line cap=rect,line join=bevel,line width=0.800pt]
  \end{scope}
  \begin{scope}[cm={{1.04164,0.0,0.0,1.04164,(-342.4953,42.14671)}},draw=blue,line cap=round,line join=round,line width=0.480pt]
    \path[cm={{1.0,0.0,0.0,1.11163,(0.0,-34.30653)}},draw] (198.5000,306.5000) -- (198.5000,303.5000);



    \path[cm={{1.0,0.0,0.0,1.53918,(0.0,-119.26725)}},draw] (198.5000,221.5000) -- (198.5000,223.5000);



  \end{scope}
  \begin{scope}[scale=1.006,draw=blue,line cap=rect,line join=bevel,line width=0.800pt]
  \end{scope}
  \begin{scope}[cm={{1.00588,0.0,0.0,1.00588,(196.147,322.888)}},draw=blue,line cap=rect,line join=bevel,line width=0.800pt]
  \end{scope}
  \begin{scope}[cm={{1.00588,0.0,0.0,1.00588,(196.147,322.888)}},draw=blue,line cap=rect,line join=bevel,line width=0.800pt]
  \end{scope}
  \begin{scope}[cm={{1.00588,0.0,0.0,1.00588,(196.147,322.888)}},draw=blue,line cap=rect,line join=bevel,line width=0.800pt]
  \end{scope}
  \begin{scope}[cm={{1.00588,0.0,0.0,1.00588,(196.147,322.888)}},draw=blue,line cap=rect,line join=bevel,line width=0.800pt]
  \end{scope}
  \begin{scope}[cm={{1.00588,0.0,0.0,1.00588,(196.147,322.888)}},draw=blue,line cap=rect,line join=bevel,line width=0.800pt]
  \end{scope}
  \begin{scope}[cm={{1.00588,0.0,0.0,1.00588,(-136.85305,376.888)}},draw=blue,line cap=rect,line join=bevel,line width=0.800pt]
    \path[fill=blue] (0.0000,0.0000) node[above right] (text1326) {8};



  \end{scope}
  \begin{scope}[cm={{1.00588,0.0,0.0,1.00588,(196.147,322.888)}},draw=blue,line cap=rect,line join=bevel,line width=0.800pt]
  \end{scope}
  \begin{scope}[scale=1.006,draw=blue,line cap=rect,line join=bevel,line width=0.800pt]
  \end{scope}
  \begin{scope}[cm={{1.04164,0.0,0.0,1.04164,(-342.4953,42.14671)}},draw=blue,line cap=round,line join=round,line width=0.480pt]
    \path[cm={{1.0,0.0,0.0,1.11163,(0.0,-34.30653)}},draw] (233.5000,306.5000) -- (233.5000,303.5000);



    \path[cm={{1.0,0.0,0.0,1.53918,(0.0,-119.26725)}},draw] (233.5000,221.5000) -- (233.5000,223.5000);



  \end{scope}
  \begin{scope}[scale=1.006,draw=blue,line cap=rect,line join=bevel,line width=0.800pt]
  \end{scope}
  \begin{scope}[cm={{1.00588,0.0,0.0,1.00588,(228.838,322.888)}},draw=blue,line cap=rect,line join=bevel,line width=0.800pt]
  \end{scope}
  \begin{scope}[cm={{1.00588,0.0,0.0,1.00588,(228.838,322.888)}},draw=blue,line cap=rect,line join=bevel,line width=0.800pt]
  \end{scope}
  \begin{scope}[cm={{1.00588,0.0,0.0,1.00588,(228.838,322.888)}},draw=blue,line cap=rect,line join=bevel,line width=0.800pt]
  \end{scope}
  \begin{scope}[cm={{1.00588,0.0,0.0,1.00588,(228.838,322.888)}},draw=blue,line cap=rect,line join=bevel,line width=0.800pt]
  \end{scope}
  \begin{scope}[cm={{1.00588,0.0,0.0,1.00588,(228.838,322.888)}},draw=blue,line cap=rect,line join=bevel,line width=0.800pt]
  \end{scope}
  \begin{scope}[cm={{1.00588,0.0,0.0,1.00588,(-102.66205,376.888)}},draw=blue,line cap=rect,line join=bevel,line width=0.800pt]
    \path[fill=blue] (0.0000,0.0000) node[above right] (text1356) {10};



  \end{scope}
  \begin{scope}[cm={{1.00588,0.0,0.0,1.00588,(228.838,322.888)}},draw=blue,line cap=rect,line join=bevel,line width=0.800pt]
  \end{scope}
  \begin{scope}[scale=1.006,draw=blue,line cap=rect,line join=bevel,line width=0.800pt]
  \end{scope}
  \begin{scope}[cm={{1.04164,0.0,0.0,1.04164,(-342.4953,42.14671)}},draw=blue,line cap=round,line join=round,line width=0.480pt]
    \path[draw] (56.5000,221.5000) -- (56.5000,306.5000) -- (251.5000,306.5000) -- (251.5000,221.5000) -- (56.5000,221.5000);



  \end{scope}
  \begin{scope}[scale=1.006,draw=blue,line cap=rect,line join=bevel,line width=0.800pt]
  \end{scope}
  \begin{scope}[scale=1.006,draw=blue,line cap=rect,line join=bevel,line width=0.800pt]
  \end{scope}
  \begin{scope}[scale=1.006,draw=blue,line cap=rect,line join=bevel,line width=0.800pt]
  \end{scope}
  \begin{scope}[scale=1.006,draw=blue,line cap=rect,line join=bevel,line width=0.800pt]
  \end{scope}
  \begin{scope}[scale=1.006,draw=blue,line cap=rect,line join=bevel,line width=0.800pt]
  \end{scope}
  \begin{scope}[cm={{1.00588,0.0,0.0,1.00588,(232.359,238.394)}},draw=blue,line cap=rect,line join=bevel,line width=0.800pt]
  \end{scope}
  \begin{scope}[cm={{1.00588,0.0,0.0,1.00588,(232.359,238.394)}},draw=blue,line cap=rect,line join=bevel,line width=0.800pt]
  \end{scope}
  \begin{scope}[cm={{1.00588,0.0,0.0,1.00588,(232.359,238.394)}},draw=blue,line cap=rect,line join=bevel,line width=0.800pt]
  \end{scope}
  \begin{scope}[cm={{1.00588,0.0,0.0,1.00588,(232.359,238.394)}},draw=blue,line cap=rect,line join=bevel,line width=0.800pt]
  \end{scope}
  \begin{scope}[cm={{1.00588,0.0,0.0,1.00588,(232.359,238.394)}},draw=blue,line cap=rect,line join=bevel,line width=0.800pt]
  \end{scope}
  \begin{scope}[cm={{1.00588,0.0,0.0,1.00588,(232.359,238.394)}},draw=blue,line cap=rect,line join=bevel,line width=0.800pt]
  \end{scope}
  \begin{scope}[cm={{1.00588,0.0,0.0,1.00588,(130.262,337.976)}},draw=blue,line cap=rect,line join=bevel,line width=0.800pt]
  \end{scope}
  \begin{scope}[cm={{1.00588,0.0,0.0,1.00588,(130.262,337.976)}},draw=blue,line cap=rect,line join=bevel,line width=0.800pt]
  \end{scope}
  \begin{scope}[cm={{1.00588,0.0,0.0,1.00588,(130.262,337.976)}},draw=blue,line cap=rect,line join=bevel,line width=0.800pt]
  \end{scope}
  \begin{scope}[cm={{1.00588,0.0,0.0,1.00588,(130.262,337.976)}},draw=blue,line cap=rect,line join=bevel,line width=0.800pt]
  \end{scope}
  \begin{scope}[cm={{1.00588,0.0,0.0,1.00588,(130.262,337.976)}},draw=blue,line cap=rect,line join=bevel,line width=0.800pt]
  \end{scope}
  \begin{scope}[cm={{1.00588,0.0,0.0,1.00588,(-204.23807,390.01506)}},draw=blue,line cap=rect,line join=bevel,line width=0.800pt]
    \path[fill=blue] (0.0000,0.0000) node[above right] (text1412) {Time (min)};



  \end{scope}
  \begin{scope}[cm={{1.00588,0.0,0.0,1.00588,(130.262,337.976)}},draw=blue,line cap=rect,line join=bevel,line width=0.800pt]
  \end{scope}
  \begin{scope}[scale=1.006,draw=blue,line cap=rect,line join=bevel,line width=0.800pt]
  \end{scope}
  \begin{scope}[scale=1.006,draw=blue,line cap=rect,line join=bevel,line width=0.800pt]
  \end{scope}
  \begin{scope}[scale=1.006,draw=blue,line cap=rect,line join=bevel,line width=0.800pt]
  \end{scope}
  \begin{scope}[cm={{1.04164,0.0,0.0,1.04164,(-342.4953,42.14671)}},draw=blue,line cap=round,line join=round,line width=0.480pt]
    \path[draw] (56.1000,255.4000) -- (56.1000,255.4000) -- (56.3000,261.0000) -- (56.5000,262.0000) -- (56.7000,262.2000) -- (56.9000,262.4000) -- (57.1000,261.4000) -- (57.3000,260.3000) -- (57.5000,259.7000) -- (57.7000,259.7000) -- (57.9000,259.8000) -- (58.1000,259.9000) -- (58.3000,260.0000) -- (58.4000,260.0000) -- (58.6000,260.0000) -- (58.8000,260.0000) -- (59.0000,260.0000) -- (59.2000,260.0000) -- (59.4000,260.0000) -- (59.6000,260.0000) -- (59.8000,260.0000) -- (60.0000,260.0000) -- (60.2000,260.0000) -- (60.4000,260.0000) -- (60.6000,260.2000) -- (60.8000,260.7000) -- (61.0000,261.3000) -- (61.2000,262.2000) -- (61.4000,263.3000) -- (61.6000,264.6000) -- (61.8000,266.1000) -- (62.0000,267.8000) -- (62.2000,269.7000) -- (62.4000,271.8000) -- (62.6000,274.1000) -- (62.8000,276.6000) -- (63.0000,279.2000) -- (63.1000,281.9000) -- (63.3000,284.7000) -- (63.5000,287.5000) -- (63.7000,290.1000) -- (63.9000,292.8000) -- (64.1000,295.5000) -- (64.3000,297.5000) -- (64.5000,298.2000) -- (64.7000,298.2000) -- (64.9000,297.9000) -- (65.1000,297.6000) -- (65.3000,297.5000) -- (65.5000,297.4000) -- (65.7000,297.4000) -- (65.9000,297.4000) -- (66.1000,297.4000) -- (66.3000,297.5000) -- (66.5000,297.5000) -- (66.7000,297.5000) -- (66.9000,297.5000) -- (67.1000,297.4000) -- (67.3000,297.1000) -- (67.5000,297.2000) -- (67.7000,297.2000) -- (67.9000,296.1000) -- (68.0000,294.1000) -- (68.2000,291.2000) -- (68.4000,288.0000) -- (68.6000,284.5000) -- (68.8000,281.1000) -- (69.0000,277.8000) -- (69.2000,274.8000) -- (69.4000,272.0000) -- (69.6000,269.4000) -- (69.8000,267.2000) -- (70.0000,265.3000) -- (70.2000,263.7000) -- (70.4000,262.4000) -- (70.6000,261.3000) -- (70.8000,260.5000) -- (71.0000,260.0000) -- (71.2000,259.6000) -- (71.4000,259.7000) -- (71.6000,259.8000) -- (71.8000,259.9000) -- (72.0000,260.0000) -- (72.2000,260.0000) -- (72.4000,260.0000) -- (72.6000,260.0000) -- (72.7000,260.0000) -- (72.9000,260.0000) -- (73.1000,260.0000) -- (73.3000,260.0000) -- (73.5000,260.0000) -- (73.7000,260.0000) -- (73.9000,260.0000) -- (74.1000,260.0000) -- (74.3000,260.1000) -- (74.5000,260.5000) -- (74.7000,261.1000) -- (74.9000,261.9000) -- (75.1000,262.9000) -- (75.3000,264.1000) -- (75.5000,265.4000) -- (75.7000,267.0000) -- (75.9000,268.9000) -- (76.1000,270.9000) -- (76.3000,273.1000) -- (76.5000,275.5000) -- (76.7000,278.0000) -- (76.9000,280.7000) -- (77.1000,283.4000) -- (77.3000,286.2000) -- (77.4000,288.9000) -- (77.6000,291.5000) -- (77.8000,294.2000) -- (78.0000,296.7000) -- (78.2000,298.1000) -- (78.4000,298.3000) -- (78.6000,298.0000) -- (78.8000,297.7000) -- (79.0000,297.5000) -- (79.2000,297.4000) -- (79.4000,297.4000) -- (79.6000,297.4000) -- (79.8000,297.4000) -- (80.0000,297.4000) -- (80.2000,297.5000) -- (80.4000,297.5000) -- (80.6000,297.5000) -- (80.8000,297.5000) -- (81.0000,298.3000) -- (81.2000,280.4000) -- (81.4000,268.1000) -- (81.6000,268.7000) -- (81.8000,267.3000) -- (82.0000,265.2000) -- (82.2000,262.6000) -- (82.3000,259.7000) -- (82.5000,256.7000) -- (82.7000,253.7000) -- (82.9000,250.7000) -- (83.1000,247.9000) -- (83.3000,245.3000) -- (83.5000,242.9000) -- (83.7000,240.7000) -- (83.9000,238.8000) -- (84.1000,237.2000) -- (84.3000,235.7000) -- (84.5000,234.5000) -- (84.7000,233.5000) -- (84.9000,232.7000) -- (85.1000,232.2000) -- (85.3000,231.7000) -- (85.5000,231.6000) -- (85.7000,231.6000) -- (85.9000,231.7000) -- (86.1000,231.8000) -- (86.3000,231.9000) -- (86.5000,231.9000) -- (86.7000,231.9000) -- (86.9000,231.9000) -- (87.0000,231.9000) -- (87.2000,231.9000) -- (87.4000,231.9000) -- (87.6000,231.9000) -- (87.8000,231.9000) -- (88.0000,231.9000) -- (88.2000,231.9000) -- (88.4000,231.9000) -- (88.6000,232.1000) -- (88.8000,232.5000) -- (89.0000,233.1000) -- (89.2000,234.0000) -- (89.4000,235.0000) -- (89.6000,236.2000) -- (89.8000,237.7000) -- (90.0000,239.3000) -- (90.2000,241.2000) -- (90.4000,243.2000) -- (90.6000,245.4000) -- (90.8000,247.9000) -- (91.0000,250.4000) -- (91.2000,253.1000) -- (91.4000,255.8000) -- (91.6000,258.6000) -- (91.8000,261.3000) -- (91.9000,263.9000) -- (92.1000,266.6000) -- (92.3000,269.0000) -- (92.5000,270.1000) -- (92.7000,270.2000) -- (92.9000,269.8000) -- (93.1000,269.5000) -- (93.3000,269.4000) -- (93.5000,269.3000) -- (93.7000,269.3000) -- (93.9000,269.3000) -- (94.1000,269.3000) -- (94.3000,269.3000) -- (94.5000,269.3000) -- (94.7000,269.3000) -- (94.9000,269.3000) -- (95.1000,269.3000) -- (95.3000,269.1000) -- (95.5000,269.1000) -- (95.7000,269.2000) -- (95.9000,268.5000) -- (96.1000,266.9000) -- (96.3000,264.6000) -- (96.5000,261.9000) -- (96.6000,259.0000) -- (96.8000,255.9000) -- (97.0000,252.9000) -- (97.2000,250.0000) -- (97.4000,247.3000) -- (97.6000,244.7000) -- (97.8000,242.4000) -- (98.0000,240.3000) -- (98.2000,238.4000) -- (98.4000,236.8000) -- (98.6000,235.4000) -- (98.8000,234.2000) -- (99.0000,233.3000) -- (99.2000,232.6000) -- (99.4000,232.0000) -- (99.6000,231.7000) -- (99.8000,231.6000) -- (100.0000,231.7000) -- (100.2000,231.8000) -- (100.4000,231.8000) -- (100.6000,231.9000) -- (100.8000,231.9000) -- (101.0000,231.9000) -- (101.2000,231.9000) -- (101.3000,231.9000) -- (101.5000,231.9000) -- (101.7000,231.9000) -- (101.9000,231.9000) -- (102.1000,231.9000) -- (102.3000,231.9000) -- (102.5000,231.9000) -- (102.7000,231.9000) -- (102.9000,232.2000) -- (103.1000,232.7000) -- (103.3000,233.4000) -- (103.5000,234.3000) -- (103.7000,235.4000) -- (103.9000,236.7000) -- (104.1000,238.2000) -- (104.3000,239.9000) -- (104.5000,241.8000) -- (104.7000,244.0000) -- (104.9000,246.3000) -- (105.1000,248.7000) -- (105.3000,251.4000) -- (105.5000,254.1000) -- (105.7000,256.8000) -- (105.9000,259.6000) -- (106.0000,262.2000) -- (106.2000,264.9000) -- (106.4000,267.6000) -- (106.6000,269.5000) -- (106.8000,270.2000) -- (107.0000,270.0000) -- (107.2000,269.7000) -- (107.4000,269.5000) -- (107.6000,269.3000) -- (107.8000,269.3000) -- (108.0000,269.3000) -- (108.2000,269.3000) -- (108.4000,269.3000) -- (108.6000,269.3000) -- (108.8000,269.3000) -- (109.0000,269.3000) -- (109.2000,269.3000) -- (109.4000,269.2000) -- (109.6000,269.0000) -- (109.8000,269.2000) -- (110.0000,269.0000) -- (110.2000,268.0000) -- (110.4000,266.1000) -- (110.6000,263.6000) -- (110.8000,260.8000) -- (110.9000,257.8000) -- (111.1000,254.8000) -- (111.3000,251.8000) -- (111.5000,248.9000) -- (111.7000,246.2000) -- (111.9000,243.8000) -- (112.1000,241.5000) -- (112.3000,239.5000) -- (112.5000,237.7000) -- (112.7000,236.2000) -- (112.9000,234.9000) -- (113.1000,233.8000) -- (113.3000,233.0000) -- (113.5000,232.3000) -- (113.7000,231.9000) -- (113.9000,231.6000) -- (114.1000,231.6000) -- (114.3000,231.7000) -- (114.5000,231.8000) -- (114.7000,231.9000) -- (114.9000,231.9000) -- (115.1000,231.9000) -- (115.3000,231.9000) -- (115.5000,231.9000) -- (115.6000,231.9000) -- (115.8000,231.9000) -- (116.0000,231.9000) -- (116.2000,231.9000) -- (116.4000,231.9000) -- (116.6000,231.9000) -- (116.8000,231.9000) -- (117.0000,232.0000) -- (117.2000,232.3000) -- (117.4000,232.9000) -- (117.6000,233.7000) -- (117.8000,234.7000) -- (118.0000,235.9000) -- (118.2000,237.2000) -- (118.4000,238.8000) -- (118.6000,240.6000) -- (118.8000,242.6000) -- (119.0000,244.8000) -- (119.2000,247.2000) -- (119.4000,249.8000) -- (119.6000,252.4000) -- (119.8000,255.2000) -- (120.0000,257.9000) -- (120.2000,260.6000) -- (120.4000,263.2000) -- (120.5000,266.0000) -- (120.7000,268.5000) -- (120.9000,269.9000) -- (121.1000,270.2000) -- (121.3000,269.9000) -- (121.5000,269.6000) -- (121.7000,269.4000) -- (121.9000,269.3000) -- (122.1000,269.3000) -- (122.3000,269.3000) -- (122.5000,269.3000) -- (122.7000,269.3000) -- (122.9000,269.3000) -- (123.1000,269.3000) -- (123.3000,269.3000) -- (123.5000,269.3000) -- (123.7000,269.1000) -- (123.9000,269.1000) -- (124.1000,269.2000) -- (124.3000,268.7000) -- (124.5000,267.3000) -- (124.7000,265.2000) -- (124.9000,262.5000) -- (125.1000,259.6000) -- (125.2000,256.6000) -- (125.4000,253.6000) -- (125.6000,250.6000) -- (125.8000,247.8000) -- (126.0000,245.2000) -- (126.2000,242.8000) -- (126.4000,240.7000) -- (126.6000,238.8000) -- (126.8000,237.1000) -- (127.0000,235.7000) -- (127.2000,234.5000) -- (127.4000,233.5000) -- (127.6000,232.7000) -- (127.8000,232.1000) -- (128.0000,231.7000) -- (128.2000,231.6000) -- (128.4000,231.6000) -- (128.6000,231.8000) -- (128.8000,231.8000) -- (129.0000,231.9000) -- (129.2000,231.9000) -- (129.4000,231.9000) -- (129.6000,231.9000) -- (129.8000,231.9000) -- (129.9000,231.9000) -- (130.1000,231.9000) -- (130.3000,231.9000) -- (130.5000,231.9000) -- (130.7000,231.9000) -- (130.9000,231.9000) -- (131.1000,231.9000) -- (131.3000,232.1000) -- (131.5000,232.5000) -- (131.7000,233.2000) -- (131.9000,234.1000) -- (132.1000,235.1000) -- (132.3000,236.4000) -- (132.5000,237.9000) -- (132.7000,239.6000) -- (132.9000,241.4000) -- (133.1000,243.5000) -- (133.3000,245.8000) -- (133.5000,248.2000) -- (133.7000,250.8000) -- (133.9000,253.5000) -- (134.1000,256.3000) -- (134.3000,259.1000) -- (134.5000,261.7000) -- (134.7000,264.3000) -- (134.8000,267.1000) -- (135.0000,269.2000) -- (135.2000,270.1000) -- (135.4000,270.1000) -- (135.6000,269.8000) -- (135.8000,269.5000) -- (136.0000,269.3000) -- (136.2000,269.3000) -- (136.4000,269.3000) -- (136.6000,269.3000) -- (136.8000,269.3000) -- (137.0000,269.3000) -- (137.2000,269.3000) -- (137.4000,269.3000) -- (137.6000,269.3000) -- (137.8000,269.3000) -- (138.0000,269.0000) -- (138.2000,269.2000) -- (138.4000,269.1000) -- (138.6000,268.2000) -- (138.8000,266.5000) -- (139.0000,264.1000) -- (139.2000,261.3000) -- (139.4000,258.4000) -- (139.5000,255.3000) -- (139.7000,252.3000) -- (139.9000,249.4000) -- (140.1000,246.7000) -- (140.3000,244.2000) -- (140.5000,241.9000) -- (140.7000,239.9000) -- (140.9000,238.1000) -- (141.1000,236.5000) -- (141.3000,235.1000) -- (141.5000,234.0000) -- (141.7000,233.1000) -- (141.9000,232.4000) -- (142.1000,231.9000) -- (142.3000,231.6000) -- (142.5000,231.6000) -- (142.7000,231.7000) -- (142.9000,231.8000) -- (143.1000,231.8000) -- (143.3000,231.9000) -- (143.5000,231.9000) -- (143.7000,231.9000) -- (143.9000,231.9000) -- (144.1000,231.9000) -- (144.3000,231.9000) -- (144.4000,231.9000) -- (144.6000,231.9000) -- (144.8000,231.9000) -- (145.0000,231.9000) -- (145.2000,231.9000) -- (145.4000,232.0000) -- (145.6000,232.3000) -- (145.8000,232.8000) -- (146.0000,233.5000) -- (146.2000,234.5000) -- (146.4000,235.6000) -- (146.6000,237.0000) -- (146.8000,238.5000) -- (147.0000,240.3000) -- (147.2000,242.3000) -- (147.4000,244.4000) -- (147.6000,246.8000) -- (147.8000,249.3000) -- (148.0000,252.0000) -- (148.2000,254.7000) -- (148.4000,257.5000) -- (148.6000,260.2000) -- (148.8000,262.8000) -- (149.0000,265.5000) -- (149.1000,268.1000) -- (149.3000,269.8000) -- (149.5000,270.2000) -- (149.7000,270.0000) -- (149.9000,269.6000) -- (150.1000,269.4000) -- (150.3000,269.3000) -- (150.5000,269.3000) -- (150.7000,269.3000) -- (150.9000,269.3000) -- (151.1000,269.3000) -- (151.3000,269.3000) -- (151.5000,269.3000) -- (151.7000,269.3000) -- (151.9000,269.3000) -- (152.1000,269.2000) -- (152.3000,269.0000) -- (152.5000,269.2000) -- (152.7000,268.9000) -- (152.9000,267.6000) -- (153.1000,265.6000) -- (153.3000,263.0000) -- (153.5000,260.1000) -- (153.7000,257.1000) -- (153.8000,254.1000) -- (154.0000,251.1000) -- (154.2000,248.3000) -- (154.4000,245.7000) -- (154.6000,243.2000) -- (154.8000,241.0000) -- (155.0000,239.1000) -- (155.2000,237.1000) -- (155.4000,238.9000) -- (155.6000,242.0000) -- (155.8000,240.7000) -- (156.0000,239.8000) -- (156.2000,239.2000) -- (156.4000,238.8000) -- (156.6000,238.6000) -- (156.8000,238.6000) -- (157.0000,238.8000) -- (157.2000,238.9000) -- (157.4000,238.9000) -- (157.6000,238.9000) -- (157.8000,238.9000) -- (158.0000,238.9000) -- (158.2000,238.9000) -- (158.4000,238.9000) -- (158.6000,238.9000) -- (158.7000,238.9000) -- (158.9000,238.9000) -- (159.1000,238.9000) -- (159.3000,238.9000) -- (159.5000,238.9000) -- (159.7000,239.1000) -- (159.9000,239.5000) -- (160.1000,240.1000) -- (160.3000,240.9000) -- (160.5000,242.2000) -- (160.7000,241.5000) -- (160.9000,237.5000) -- (161.1000,239.2000) -- (161.3000,241.1000) -- (161.5000,243.1000) -- (161.7000,245.4000) -- (161.9000,247.8000) -- (162.1000,250.4000) -- (162.3000,253.0000) -- (162.5000,255.8000) -- (162.7000,258.6000) -- (162.9000,261.3000) -- (163.1000,263.8000) -- (163.3000,266.6000) -- (163.4000,268.9000) -- (163.6000,270.1000) -- (163.8000,270.1000) -- (164.0000,269.8000) -- (164.2000,269.5000) -- (164.4000,269.4000) -- (164.6000,269.3000) -- (164.8000,269.3000) -- (165.0000,269.3000) -- (165.2000,269.3000) -- (165.4000,269.3000) -- (165.6000,269.3000) -- (165.8000,269.3000) -- (166.0000,269.3000) -- (166.2000,269.3000) -- (166.4000,269.0000) -- (166.6000,269.1000) -- (166.8000,269.2000) -- (167.0000,268.5000) -- (167.2000,266.9000) -- (167.4000,264.6000) -- (167.6000,261.9000) -- (167.8000,258.9000) -- (168.0000,255.9000) -- (168.2000,252.9000) -- (168.3000,250.0000) -- (168.5000,246.9000) -- (168.7000,247.8000) -- (168.9000,249.7000) -- (169.1000,247.3000) -- (169.3000,245.3000) -- (169.5000,244.1000) -- (169.7000,248.5000) -- (169.9000,248.4000) -- (170.1000,247.3000) -- (170.3000,246.6000) -- (170.5000,246.1000) -- (170.7000,245.7000) -- (170.9000,245.6000) -- (171.1000,245.7000) -- (171.3000,245.8000) -- (171.5000,245.9000) -- (171.7000,245.9000) -- (171.9000,245.9000) -- (172.1000,245.9000) -- (172.3000,245.9000) -- (172.5000,245.9000) -- (172.7000,245.9000) -- (172.9000,245.9000) -- (173.0000,245.9000) -- (173.2000,245.9000) -- (173.4000,245.9000) -- (173.6000,245.9000) -- (173.8000,246.0000) -- (174.0000,246.2000) -- (174.2000,246.7000) -- (174.4000,247.4000) -- (174.6000,248.4000) -- (174.8000,249.6000) -- (175.0000,245.8000) -- (175.2000,245.0000) -- (175.4000,247.0000) -- (175.6000,249.1000) -- (175.8000,249.5000) -- (176.0000,246.3000) -- (176.2000,248.8000) -- (176.4000,251.4000) -- (176.6000,254.1000) -- (176.8000,256.9000) -- (177.0000,259.6000) -- (177.2000,262.3000) -- (177.4000,264.9000) -- (177.6000,267.7000) -- (177.7000,269.5000) -- (177.9000,270.2000) -- (178.1000,270.0000) -- (178.3000,269.7000) -- (178.5000,269.5000) -- (178.7000,269.3000) -- (178.9000,269.3000) -- (179.1000,269.3000) -- (179.3000,269.3000) -- (179.5000,269.3000) -- (179.7000,269.3000) -- (179.9000,269.3000) -- (180.1000,269.3000) -- (180.3000,269.3000) -- (180.5000,269.2000) -- (180.7000,269.0000) -- (180.9000,269.2000) -- (181.1000,269.0000) -- (181.3000,267.9000) -- (181.5000,266.1000) -- (181.7000,263.6000) -- (181.9000,260.7000) -- (182.1000,257.6000) -- (182.3000,255.1000) -- (182.4000,258.0000) -- (182.6000,255.9000) -- (182.8000,253.5000) -- (183.0000,256.9000) -- (183.2000,255.7000) -- (183.4000,253.5000) -- (183.6000,251.5000) -- (183.8000,253.1000) -- (184.0000,256.2000) -- (184.2000,254.9000) -- (184.4000,254.0000) -- (184.6000,253.4000) -- (184.8000,252.9000) -- (185.0000,252.7000) -- (185.2000,252.7000) -- (185.4000,252.8000) -- (185.6000,252.9000) -- (185.8000,252.9000) -- (186.0000,253.0000) -- (186.2000,252.9000) -- (186.4000,252.9000) -- (186.6000,252.9000) -- (186.8000,252.9000) -- (187.0000,252.9000) -- (187.2000,252.9000) -- (187.3000,252.9000) -- (187.5000,252.9000) -- (187.7000,252.9000) -- (187.9000,253.0000) -- (188.1000,253.1000) -- (188.3000,253.4000) -- (188.5000,254.0000) -- (188.7000,254.8000) -- (188.9000,256.0000) -- (189.1000,255.4000) -- (189.3000,251.3000) -- (189.5000,252.9000) -- (189.7000,254.8000) -- (189.9000,257.0000) -- (190.1000,254.2000) -- (190.3000,254.1000) -- (190.5000,257.1000) -- (190.7000,254.9000) -- (190.9000,255.0000) -- (191.1000,258.0000) -- (191.3000,260.7000) -- (191.5000,263.3000) -- (191.7000,266.0000) -- (191.9000,268.6000) -- (192.0000,269.9000) -- (192.2000,270.2000) -- (192.4000,269.9000) -- (192.6000,269.6000) -- (192.8000,269.4000) -- (193.0000,269.3000) -- (193.2000,269.3000) -- (193.4000,269.3000) -- (193.6000,269.3000) -- (193.8000,269.3000) -- (194.0000,269.3000) -- (194.2000,269.3000) -- (194.4000,269.3000) -- (194.6000,269.3000) -- (194.8000,269.1000) -- (195.0000,269.1000) -- (195.2000,269.2000) -- (195.4000,268.7000) -- (195.6000,267.3000) -- (195.8000,265.1000) -- (196.0000,262.4000) -- (196.2000,259.9000) -- (196.4000,262.5000) -- (196.6000,263.8000) -- (196.8000,264.9000) -- (196.9000,261.5000) -- (197.1000,262.2000) -- (197.3000,264.2000) -- (197.5000,261.7000) -- (197.7000,259.8000) -- (197.9000,258.4000) -- (198.1000,262.7000) -- (198.3000,262.7000) -- (198.5000,261.6000) -- (198.7000,260.8000) -- (198.9000,260.2000) -- (199.1000,259.8000) -- (199.3000,259.7000) -- (199.5000,259.8000) -- (199.7000,259.9000) -- (199.9000,260.0000) -- (200.1000,260.0000) -- (200.3000,260.0000) -- (200.5000,260.0000) -- (200.7000,260.0000) -- (200.9000,260.0000) -- (201.1000,260.0000) -- (201.3000,260.0000) -- (201.5000,260.0000) -- (201.6000,260.0000) -- (201.8000,260.0000) -- (202.0000,260.0000) -- (202.2000,260.0000) -- (202.4000,260.2000) -- (202.6000,260.7000) -- (202.8000,261.3000) -- (203.0000,262.2000) -- (203.2000,263.5000) -- (203.4000,259.7000) -- (203.6000,258.7000) -- (203.8000,260.7000) -- (204.0000,262.7000) -- (204.2000,263.3000) -- (204.4000,259.9000) -- (204.6000,262.5000) -- (204.8000,263.7000) -- (205.0000,260.7000) -- (205.2000,263.5000) -- (205.4000,265.0000) -- (205.6000,261.9000) -- (205.8000,264.3000) -- (206.0000,267.1000) -- (206.2000,269.3000) -- (206.3000,270.1000) -- (206.5000,270.1000) -- (206.7000,269.8000) -- (206.9000,269.5000) -- (207.1000,269.3000) -- (207.3000,269.3000) -- (207.5000,269.3000) -- (207.7000,269.3000) -- (207.9000,269.3000) -- (208.1000,269.3000) -- (208.3000,269.3000) -- (208.5000,269.3000) -- (208.7000,269.3000) -- (208.9000,269.3000) -- (209.1000,269.0000) -- (209.3000,269.2000) -- (209.5000,269.1000) -- (209.7000,268.2000) -- (209.9000,266.4000) -- (210.1000,264.5000) -- (210.3000,266.7000) -- (210.5000,274.9000) -- (210.7000,284.3000) -- (210.9000,280.4000) -- (211.1000,277.6000) -- (211.2000,274.8000) -- (211.4000,272.3000) -- (211.6000,270.0000) -- (211.8000,268.0000) -- (212.0000,266.2000) -- (212.2000,264.6000) -- (212.4000,263.3000) -- (212.6000,262.1000) -- (212.8000,261.2000) -- (213.0000,260.6000) -- (213.2000,260.1000) -- (213.4000,259.7000) -- (213.6000,259.7000) -- (213.8000,259.8000) -- (214.0000,259.9000) -- (214.2000,260.0000) -- (214.4000,260.0000) -- (214.6000,260.0000) -- (214.8000,260.0000) -- (215.0000,260.0000) -- (215.2000,260.0000) -- (215.4000,260.0000) -- (215.6000,260.0000) -- (215.8000,260.0000) -- (215.9000,260.0000) -- (216.1000,260.0000) -- (216.3000,260.0000) -- (216.5000,260.1000) -- (216.7000,260.4000) -- (216.9000,260.9000) -- (217.1000,261.7000) -- (217.3000,262.6000) -- (217.5000,263.8000) -- (217.7000,265.1000) -- (217.9000,266.7000) -- (218.1000,268.4000) -- (218.3000,270.6000) -- (218.5000,268.0000) -- (218.7000,267.6000) -- (218.9000,270.6000) -- (219.1000,268.6000) -- (219.3000,268.5000) -- (219.5000,271.8000) -- (219.7000,269.9000) -- (219.9000,269.5000) -- (220.1000,272.8000) -- (220.3000,271.0000) -- (220.5000,269.5000) -- (220.7000,270.2000) -- (220.8000,270.0000) -- (221.0000,269.6000) -- (221.2000,269.4000) -- (221.4000,269.3000) -- (221.6000,269.3000) -- (221.8000,269.3000) -- (222.0000,269.3000) -- (222.2000,269.3000) -- (222.4000,269.3000) -- (222.6000,269.2000) -- (222.8000,269.9000) -- (223.0000,275.8000) -- (223.2000,276.3000) -- (223.4000,276.1000) -- (223.6000,276.3000) -- (223.8000,275.9000) -- (224.0000,274.7000) -- (224.2000,271.8000) -- (224.4000,279.4000) -- (224.6000,289.1000) -- (224.8000,285.2000) -- (225.0000,282.2000) -- (225.2000,279.3000) -- (225.4000,276.4000) -- (225.5000,273.8000) -- (225.7000,271.4000) -- (225.9000,269.2000) -- (226.1000,267.2000) -- (226.3000,265.5000) -- (226.5000,264.0000) -- (226.7000,262.8000) -- (226.9000,261.8000) -- (227.1000,260.9000) -- (227.3000,260.3000) -- (227.5000,259.9000) -- (227.7000,259.7000) -- (227.9000,259.7000) -- (228.1000,259.9000) -- (228.3000,259.9000) -- (228.5000,260.0000) -- (228.7000,260.0000) -- (228.9000,260.0000) -- (229.1000,260.0000) -- (229.3000,260.0000) -- (229.5000,260.0000) -- (229.7000,260.0000) -- (229.9000,260.0000) -- (230.1000,260.0000) -- (230.3000,260.0000) -- (230.4000,260.0000) -- (230.6000,260.0000) -- (230.8000,260.2000) -- (231.0000,260.6000) -- (231.2000,261.2000) -- (231.4000,262.0000) -- (231.6000,263.0000) -- (231.8000,264.3000) -- (232.0000,265.7000) -- (232.2000,267.3000) -- (232.4000,269.2000) -- (232.6000,271.2000) -- (232.8000,273.5000) -- (233.0000,276.0000) -- (233.2000,277.3000) -- (233.4000,274.3000) -- (233.6000,277.0000) -- (233.8000,278.6000) -- (234.0000,275.5000) -- (234.2000,277.9000) -- (234.4000,279.7000) -- (234.6000,276.2000) -- (234.8000,277.0000) -- (234.9000,277.2000) -- (235.1000,276.9000) -- (235.3000,276.6000) -- (235.5000,276.4000) -- (235.7000,276.3000) -- (235.9000,276.3000) -- (236.1000,276.3000) -- (236.3000,276.4000) -- (236.5000,276.3000) -- (236.7000,276.9000) -- (236.9000,282.8000) -- (237.1000,283.5000) -- (237.3000,283.4000) -- (237.5000,283.1000) -- (237.7000,283.0000) -- (237.9000,284.0000) -- (238.1000,295.1000) -- (238.3000,295.2000) -- (238.5000,292.8000) -- (238.7000,290.0000) -- (238.9000,287.1000) -- (239.1000,284.0000) -- (239.3000,281.0000) -- (239.5000,278.1000) -- (239.7000,275.4000) -- (239.8000,272.8000) -- (240.0000,270.5000) -- (240.2000,268.4000) -- (240.4000,266.5000) -- (240.6000,264.9000) -- (240.8000,263.5000) -- (241.0000,262.3000) -- (241.2000,261.4000) -- (241.4000,260.7000) -- (241.6000,260.1000) -- (241.8000,259.8000) -- (242.0000,259.7000) -- (242.2000,259.8000) -- (242.4000,259.9000) -- (242.6000,260.0000) -- (242.8000,260.0000) -- (243.0000,260.0000) -- (243.2000,260.0000) -- (243.4000,260.0000) -- (243.6000,260.0000) -- (243.8000,260.0000) -- (244.0000,260.0000) -- (244.2000,260.0000) -- (244.4000,260.0000) -- (244.5000,260.0000) -- (244.7000,260.0000) -- (244.9000,260.0000) -- (245.1000,260.3000) -- (245.3000,260.8000) -- (245.5000,261.5000) -- (245.7000,262.4000) -- (245.9000,263.5000) -- (246.1000,264.8000) -- (246.3000,266.3000) -- (246.5000,268.0000) -- (246.7000,270.0000) -- (246.9000,272.1000) -- (247.1000,274.4000) -- (247.3000,276.9000) -- (247.5000,279.5000) -- (247.7000,282.2000) -- (247.9000,285.3000) -- (248.1000,283.6000) -- (248.3000,283.1000) -- (248.5000,286.3000) -- (248.7000,284.7000) -- (248.9000,283.3000) -- (249.1000,284.2000) -- (249.3000,284.1000) -- (249.4000,283.8000) -- (249.6000,283.5000) -- (249.8000,283.4000) -- (250.0000,283.4000) -- (250.2000,283.4000) -- (250.4000,283.4000) -- (250.6000,283.4000) -- (250.8000,283.1000) -- (251.0000,286.8000) -- (251.2000,290.7000) -- (251.4000,290.1000) -- (251.6000,293.6000) -- (251.8000,297.4000);



  \end{scope}
  \begin{scope}[scale=1.006,draw=blue,line cap=rect,line join=bevel,line width=0.800pt]
  \end{scope}
  \begin{scope}[scale=1.006,draw=blue,line cap=rect,line join=bevel,line width=0.800pt]
  \end{scope}
  \begin{scope}[scale=1.006,draw=blue,line cap=rect,line join=bevel,line width=0.800pt]
  \end{scope}
  \begin{scope}[scale=1.006,draw=blue,line cap=rect,line join=bevel,line width=0.800pt]
  \end{scope}
  \begin{scope}[cm={{1.04164,0.0,0.0,1.04164,(-342.4953,42.14671)}},draw=c00ff00,dash pattern=on 0.48pt off 0.48pt,line cap=round,line join=round,line width=0.480pt,miter limit=4.00]
    \path[draw,dash pattern=on 0.48pt off 0.48pt,line width=0.480pt,miter limit=4.00] (127.3000,221.1000) -- (127.6000,221.2000) -- (127.9000,221.4000) -- (128.2000,221.5000) -- (128.5000,221.7000) -- (128.8000,221.8000) -- (129.1000,222.0000) -- (129.4000,222.1000) -- (129.7000,222.3000) -- (130.0000,222.4000) -- (130.3000,222.6000) -- (130.6000,222.7000) -- (130.9000,222.9000) -- (131.2000,223.0000) -- (131.5000,223.2000) -- (131.8000,223.3000) -- (132.1000,223.5000) -- (132.3000,223.6000) -- (132.6000,223.8000) -- (132.9000,223.9000) -- (133.2000,224.1000) -- (133.5000,224.2000) -- (133.8000,224.4000) -- (134.1000,224.5000) -- (134.4000,224.7000) -- (134.7000,224.8000) -- (135.0000,225.0000) -- (135.3000,225.1000) -- (135.6000,225.2000) -- (135.9000,225.4000) -- (136.2000,225.5000) -- (136.5000,225.7000) -- (136.8000,225.8000) -- (137.1000,226.0000) -- (137.4000,226.1000) -- (137.7000,226.3000) -- (138.0000,226.4000) -- (138.3000,226.6000) -- (138.6000,226.7000) -- (138.9000,226.9000) -- (139.2000,227.0000) -- (139.4000,227.2000) -- (139.7000,227.3000) -- (140.0000,227.5000) -- (140.3000,227.6000) -- (140.6000,227.8000) -- (140.9000,227.9000) -- (141.2000,228.1000) -- (141.5000,228.2000) -- (141.8000,228.4000) -- (142.1000,228.5000) -- (142.4000,228.7000) -- (142.7000,228.8000) -- (143.0000,229.0000) -- (143.3000,229.1000) -- (143.6000,229.3000) -- (143.9000,229.4000) -- (144.2000,229.6000) -- (144.5000,229.7000) -- (144.8000,229.8000) -- (145.1000,230.0000) -- (145.4000,230.1000) -- (145.7000,230.3000) -- (146.0000,230.4000) -- (146.3000,230.6000) -- (146.6000,230.7000) -- (146.8000,230.9000) -- (147.1000,231.0000) -- (147.4000,231.2000) -- (147.7000,231.3000) -- (148.0000,231.5000) -- (148.3000,231.6000) -- (148.6000,231.8000) -- (148.9000,231.9000) -- (149.2000,232.1000) -- (149.5000,232.2000) -- (149.8000,232.4000) -- (150.1000,232.5000) -- (150.4000,232.7000) -- (150.7000,232.8000) -- (151.0000,233.0000) -- (151.3000,233.1000) -- (151.6000,233.3000) -- (151.9000,233.4000) -- (152.2000,233.6000) -- (152.5000,233.7000) -- (152.8000,233.9000) -- (153.1000,234.0000) -- (153.4000,234.1000) -- (153.7000,234.3000) -- (153.9000,234.4000) -- (154.2000,234.6000) -- (154.5000,234.7000) -- (154.8000,234.9000) -- (155.1000,235.0000) -- (155.4000,235.2000) -- (155.7000,235.3000) -- (156.0000,235.5000) -- (156.3000,235.6000) -- (156.6000,235.8000) -- (156.9000,235.9000) -- (157.2000,236.1000) -- (157.5000,236.2000) -- (157.8000,236.4000) -- (158.1000,236.5000) -- (158.4000,236.7000) -- (158.7000,236.8000) -- (159.0000,237.0000) -- (159.3000,237.1000) -- (159.6000,237.3000) -- (159.9000,237.4000) -- (160.2000,237.6000) -- (160.5000,237.7000) -- (160.8000,237.9000) -- (161.0000,238.0000) -- (161.3000,238.2000) -- (161.6000,238.3000) -- (161.9000,238.5000) -- (162.2000,238.6000) -- (162.5000,238.6000) -- (162.8000,238.7000) -- (163.1000,238.9000) -- (163.4000,239.0000) -- (163.7000,239.2000) -- (164.0000,239.3000) -- (164.3000,239.5000) -- (164.6000,239.6000) -- (164.9000,239.8000) -- (165.2000,239.9000) -- (165.5000,240.1000) -- (165.8000,240.2000) -- (166.1000,240.4000) -- (166.4000,240.5000) -- (166.7000,240.7000) -- (167.0000,240.8000) -- (167.3000,241.0000) -- (167.6000,241.1000) -- (167.9000,241.3000) -- (168.2000,241.4000) -- (168.4000,241.6000) -- (168.7000,241.7000) -- (169.0000,241.9000) -- (169.3000,242.0000) -- (169.6000,242.2000) -- (169.9000,242.3000) -- (170.2000,242.5000) -- (170.5000,242.6000) -- (170.8000,242.8000) -- (171.1000,242.9000) -- (171.4000,243.1000) -- (171.7000,243.2000) -- (172.0000,243.3000) -- (172.3000,243.5000) -- (172.6000,243.6000) -- (172.9000,243.8000) -- (173.2000,243.9000) -- (173.5000,244.1000) -- (173.8000,244.2000) -- (174.1000,244.4000) -- (174.4000,244.5000) -- (174.7000,244.7000) -- (175.0000,244.8000) -- (175.3000,245.0000) -- (175.5000,245.1000) -- (175.8000,245.3000) -- (176.1000,245.4000) -- (176.4000,245.6000) -- (176.7000,245.7000) -- (177.0000,245.9000) -- (177.3000,246.0000) -- (177.6000,246.2000) -- (177.9000,246.3000) -- (178.2000,246.5000) -- (178.5000,246.6000) -- (178.8000,246.8000) -- (179.1000,246.9000) -- (179.4000,247.1000) -- (179.7000,247.2000) -- (180.0000,247.4000) -- (180.3000,247.5000) -- (180.6000,247.6000) -- (180.9000,247.8000) -- (181.2000,247.9000) -- (181.5000,248.1000) -- (181.8000,248.2000) -- (182.1000,248.4000) -- (182.4000,248.5000) -- (182.7000,248.7000) -- (182.9000,248.8000) -- (183.2000,249.0000) -- (183.5000,249.1000) -- (183.8000,249.3000) -- (184.1000,249.4000) -- (184.4000,249.6000) -- (184.7000,249.7000) -- (185.0000,249.9000) -- (185.3000,250.0000) -- (185.6000,250.2000) -- (185.9000,250.3000) -- (186.2000,250.5000) -- (186.5000,250.6000) -- (186.8000,250.8000) -- (187.1000,250.9000) -- (187.4000,251.1000) -- (187.7000,251.2000) -- (188.0000,251.4000) -- (188.3000,251.5000) -- (188.6000,251.7000) -- (188.9000,251.8000) -- (189.2000,252.0000) -- (189.5000,252.1000) -- (189.8000,252.2000) -- (190.0000,252.4000) -- (190.3000,252.5000) -- (190.6000,252.7000) -- (190.9000,252.8000) -- (191.2000,253.0000) -- (191.5000,253.1000) -- (191.8000,253.3000) -- (192.1000,253.4000) -- (192.4000,253.6000) -- (192.7000,253.7000) -- (193.0000,253.9000) -- (193.3000,254.0000) -- (193.6000,254.2000) -- (193.9000,254.3000) -- (194.2000,254.5000) -- (194.5000,254.6000) -- (194.8000,254.8000) -- (195.1000,254.9000) -- (195.4000,255.1000) -- (195.7000,255.2000) -- (196.0000,255.4000) -- (196.3000,255.5000) -- (196.6000,255.7000) -- (196.9000,255.8000) -- (197.2000,256.0000) -- (197.4000,256.1000) -- (197.7000,256.3000) -- (198.0000,256.4000) -- (198.3000,256.5000) -- (198.6000,256.7000) -- (198.9000,256.8000) -- (199.2000,257.0000) -- (199.5000,257.1000) -- (199.8000,257.3000) -- (200.1000,257.4000) -- (200.4000,257.6000) -- (200.7000,257.7000) -- (201.0000,257.9000) -- (201.3000,258.0000) -- (201.6000,258.2000) -- (201.9000,258.3000) -- (202.2000,258.5000) -- (202.5000,258.6000) -- (202.8000,258.8000) -- (203.1000,258.9000) -- (203.4000,259.1000) -- (203.7000,259.2000) -- (204.0000,259.4000) -- (204.3000,259.5000) -- (204.5000,259.7000) -- (204.8000,259.8000) -- (205.1000,260.0000) -- (205.4000,260.1000) -- (205.7000,260.3000) -- (206.0000,260.4000) -- (206.3000,260.6000) -- (206.6000,260.7000) -- (206.9000,260.9000) -- (207.2000,261.0000) -- (207.5000,261.1000) -- (207.8000,261.3000) -- (208.1000,261.4000) -- (208.4000,261.6000) -- (208.7000,261.7000) -- (209.0000,261.9000) -- (209.3000,262.0000) -- (209.6000,262.2000) -- (209.9000,262.3000) -- (210.2000,262.5000) -- (210.5000,262.6000) -- (210.8000,262.8000) -- (211.1000,262.9000) -- (211.4000,263.1000) -- (211.7000,263.2000) -- (211.9000,263.4000) -- (212.2000,263.5000) -- (212.5000,263.7000) -- (212.8000,263.8000) -- (213.1000,264.0000) -- (213.4000,264.1000) -- (213.7000,264.3000) -- (214.0000,264.4000) -- (214.3000,264.6000) -- (214.6000,264.7000) -- (214.9000,264.9000) -- (215.2000,265.0000) -- (215.5000,265.2000) -- (215.8000,265.3000) -- (216.1000,265.5000) -- (216.4000,265.6000) -- (216.7000,265.7000) -- (217.0000,265.9000) -- (217.3000,266.0000) -- (217.6000,266.2000) -- (217.9000,266.3000) -- (218.2000,266.5000) -- (218.5000,266.6000) -- (218.8000,266.8000) -- (219.0000,266.9000) -- (219.3000,267.1000) -- (219.6000,267.2000) -- (219.9000,267.4000) -- (220.2000,267.5000) -- (220.5000,267.7000) -- (220.8000,267.8000) -- (221.1000,268.0000) -- (221.4000,268.1000) -- (221.7000,268.3000) -- (222.0000,268.4000) -- (222.3000,268.6000) -- (222.6000,268.7000) -- (222.9000,268.9000) -- (223.2000,269.0000) -- (223.5000,269.2000) -- (223.8000,269.3000) -- (224.1000,269.5000) -- (224.4000,269.6000) -- (224.7000,269.8000) -- (225.0000,269.9000) -- (225.3000,270.0000) -- (225.6000,270.2000) -- (225.9000,270.3000) -- (226.1000,270.5000) -- (226.4000,270.6000) -- (226.7000,270.8000) -- (227.0000,270.9000) -- (227.3000,271.1000) -- (227.6000,271.2000) -- (227.9000,271.4000) -- (228.2000,271.5000) -- (228.5000,271.7000) -- (228.8000,271.8000) -- (229.1000,272.0000) -- (229.4000,272.1000) -- (229.7000,272.3000) -- (230.0000,272.4000) -- (230.3000,272.6000) -- (230.6000,272.7000) -- (230.9000,272.9000) -- (231.2000,273.0000) -- (231.5000,273.2000) -- (231.8000,273.3000) -- (232.1000,273.5000) -- (232.4000,273.6000) -- (232.7000,273.8000) -- (233.0000,273.9000) -- (233.3000,274.1000) -- (233.5000,274.2000) -- (233.8000,274.4000) -- (234.1000,274.5000) -- (234.4000,274.6000) -- (234.7000,274.8000) -- (235.0000,274.9000) -- (235.3000,275.1000) -- (235.6000,275.2000) -- (235.9000,275.4000) -- (236.2000,275.5000) -- (236.5000,275.7000) -- (236.8000,275.8000) -- (237.1000,276.0000) -- (237.4000,276.1000) -- (237.7000,276.3000) -- (238.0000,276.4000) -- (238.3000,276.6000) -- (238.6000,276.7000) -- (238.9000,276.9000) -- (239.2000,277.0000) -- (239.5000,277.2000) -- (239.8000,277.3000) -- (240.1000,277.5000) -- (240.4000,277.6000) -- (240.6000,277.8000) -- (240.9000,277.9000) -- (241.2000,278.1000) -- (241.5000,278.2000) -- (241.8000,278.4000) -- (242.1000,278.5000) -- (242.4000,278.7000) -- (242.7000,278.8000) -- (243.0000,279.0000) -- (243.3000,279.1000) -- (243.6000,279.2000) -- (243.9000,279.4000) -- (244.2000,279.5000) -- (244.5000,279.7000) -- (244.8000,279.8000) -- (245.1000,280.0000) -- (245.4000,280.1000) -- (245.7000,280.3000) -- (246.0000,280.4000) -- (246.3000,280.6000) -- (246.6000,280.7000) -- (246.9000,280.9000) -- (247.2000,281.0000) -- (247.5000,281.2000) -- (247.8000,281.3000) -- (248.0000,281.5000) -- (248.3000,281.6000) -- (248.6000,281.8000) -- (248.9000,281.9000) -- (249.2000,282.1000) -- (249.5000,282.2000) -- (249.8000,282.4000) -- (250.1000,282.5000) -- (250.4000,282.7000) -- (250.7000,282.8000) -- (251.0000,283.0000) -- (251.3000,283.1000) -- (251.6000,283.3000) -- (251.9000,283.4000);



  \end{scope}
  \begin{scope}[scale=1.006,draw=blue,line cap=rect,line join=bevel,line width=0.800pt]
  \end{scope}
  \begin{scope}[draw=blue,line cap=rect,line join=bevel,line width=0.800pt]
  \end{scope}
  \begin{scope}[cm={{0.0,-1.00588,1.00588,0.0,(-600.28902,263.45539)}},draw=blue,line cap=rect,line join=bevel,line width=0.800pt]
    \path[fill=blue] (0.0000,0.0000) node[above right] (text344) {\rotatebox{90}{y (m)}};



  \end{scope}
  \begin{scope}[cm={{0.84173,0.0,0.0,0.84173,(-601.60573,125.64086)}},fill=cffffff]
  \end{scope}
  \begin{scope}[cm={{0.92047,0.0,0.0,0.92047,(-569.12952,57.84001)}},draw=ca0a0a4,dash pattern=on 1.96pt off 1.96pt,line cap=round,line join=round,line width=0.326pt,miter limit=4.00]
    \path[shift={(0,-5.96493)},draw,dash pattern=on 1.96pt off 1.96pt,line width=0.326pt,miter limit=4.00] (184.5000,151.5000) -- (246.5000,151.5000);



  \end{scope}
  \begin{scope}[cm={{0.92047,0.0,0.0,0.92047,(-569.12952,52.34945)}},draw=blue,line cap=round,line join=round,line width=0.480pt]
    \path[draw] (184.5000,151.5000) -- (186.5000,151.5000);



    \path[draw] (246.5000,151.5000) -- (245.5000,151.5000);



  \end{scope}
  \begin{scope}[cm={{0.92047,0.0,0.0,0.92047,(-569.12952,52.34945)}},draw=ca0a0a4,dash pattern=on 1.96pt off 1.96pt,line cap=round,line join=round,line width=0.326pt,miter limit=4.00]
    \path[draw,dash pattern=on 1.96pt off 1.96pt,line width=0.326pt,miter limit=4.00] (184.5000,132.5000) -- (246.5000,132.5000);



  \end{scope}
  \begin{scope}[cm={{0.92047,0.0,0.0,0.92047,(-569.12952,52.34945)}},draw=blue,line cap=round,line join=round,line width=0.480pt]
    \path[draw] (184.5000,132.5000) -- (186.5000,132.5000);



    \path[draw] (246.5000,132.5000) -- (245.5000,132.5000);



  \end{scope}
  \begin{scope}[cm={{0.92047,0.0,0.0,0.92047,(-569.12952,52.34945)}},draw=ca0a0a4,dash pattern=on 1.96pt off 1.96pt,line cap=round,line join=round,line width=0.326pt,miter limit=4.00]
    \path[draw,dash pattern=on 1.96pt off 1.96pt,line width=0.326pt,miter limit=4.00] (184.5000,113.5000) -- (246.5000,113.5000);



  \end{scope}
  \begin{scope}[cm={{0.92047,0.0,0.0,0.92047,(-569.12952,57.84001)}},draw=blue,line cap=round,line join=round,line width=0.480pt]
    \path[shift={(0,-5.96493)},draw] (184.5000,113.5000) -- (186.5000,113.5000);



    \path[shift={(0,-5.96493)},draw] (246.5000,113.5000) -- (245.5000,113.5000);



  \end{scope}
  \begin{scope}[cm={{0.92047,0.0,0.0,0.92047,(-569.12952,52.34945)}},draw=ca0a0a4,dash pattern=on 0.40pt off 0.80pt,line cap=round,line join=round,line width=0.400pt]
    \path[draw] (184.5000,156.5000) -- (184.5000,108.5000);



  \end{scope}
  \begin{scope}[cm={{0.92047,0.0,0.0,0.92047,(-569.12952,52.34945)}},draw=blue,line cap=round,line join=round,line width=0.480pt]
    \path[draw] (184.5000,156.5000) -- (184.5000,155.5000);



    \path[draw] (184.5000,108.5000) -- (184.5000,109.5000);



  \end{scope}
  \begin{scope}[cm={{0.92047,0.0,0.0,0.92047,(-569.12952,52.34945)}},draw=ca0a0a4,dash pattern=on 1.96pt off 1.96pt,line cap=round,line join=round,line width=0.326pt,miter limit=4.00]
    \path[draw,dash pattern=on 1.96pt off 1.96pt,line width=0.326pt,miter limit=4.00] (203.5000,156.5000) -- (203.5000,108.5000);



  \end{scope}
  \begin{scope}[cm={{0.92047,0.0,0.0,0.92047,(-569.12952,52.34945)}},draw=blue,line cap=round,line join=round,line width=0.480pt]
    \path[draw] (203.5000,156.5000) -- (203.5000,155.5000);



    \path[draw] (203.5000,108.5000) -- (203.5000,109.5000);



  \end{scope}
  \begin{scope}[cm={{0.92047,0.0,0.0,0.92047,(-569.12952,52.34945)}},draw=ca0a0a4,dash pattern=on 1.96pt off 1.96pt,line cap=round,line join=round,line width=0.326pt,miter limit=4.00]
    \path[draw,dash pattern=on 1.96pt off 1.96pt,line width=0.326pt,miter limit=4.00] (221.5000,156.5000) -- (221.5000,108.5000);



  \end{scope}
  \begin{scope}[cm={{0.92047,0.0,0.0,0.92047,(-569.12952,52.34945)}},draw=blue,line cap=round,line join=round,line width=0.480pt]
    \path[draw] (221.5000,156.5000) -- (221.5000,155.5000);



    \path[draw] (221.5000,108.5000) -- (221.5000,109.5000);



  \end{scope}
  \begin{scope}[cm={{0.92047,0.0,0.0,0.92047,(-569.12952,52.34945)}},draw=ca0a0a4,dash pattern=on 1.96pt off 1.96pt,line cap=round,line join=round,line width=0.326pt,miter limit=4.00]
    \path[draw,dash pattern=on 1.96pt off 1.96pt,line width=0.326pt,miter limit=4.00] (239.5000,156.5000) -- (239.5000,108.5000);



  \end{scope}
  \begin{scope}[cm={{0.92047,0.0,0.0,0.92047,(-569.12952,52.34945)}},draw=blue,line cap=round,line join=round,line width=0.480pt]
    \path[draw] (239.5000,156.5000) -- (239.5000,155.5000);



    \path[draw] (239.5000,108.5000) -- (239.5000,109.5000);



  \end{scope}
  \begin{scope}[cm={{0.92047,0.0,0.0,0.92047,(-569.12952,57.84001)}},draw=blue,line cap=round,line join=round,line width=0.480pt]
    \path[shift={(0,-5.96493)},draw] (246.5000,151.5000) -- (245.5000,151.5000);



  \end{scope}
  \begin{scope}[cm={{0.92047,0.0,0.0,0.92047,(-569.12952,57.84001)}},draw=blue,line cap=round,line join=round,line width=0.480pt]
    \path[shift={(0,-5.96493)},draw] (246.5000,132.5000) -- (245.5000,132.5000);



  \end{scope}
  \begin{scope}[cm={{0.92047,0.0,0.0,0.92047,(-569.12952,52.34945)}},draw=blue,line cap=round,line join=round,line width=0.480pt]
    \path[draw] (246.5000,113.5000) -- (245.5000,113.5000);



  \end{scope}
  \begin{scope}[cm={{0.92047,0.0,0.0,0.92047,(-569.12952,52.34945)}},draw=blue,line cap=round,line join=round,line width=0.480pt]
    \path[draw] (184.5000,108.5000) -- (184.5000,156.5000) -- (246.5000,156.5000) -- (246.5000,108.5000) -- (184.5000,108.5000);



  \end{scope}
  \begin{scope}[cm={{0.92047,0.0,0.0,0.92047,(-569.12952,52.34945)}},draw=blue,line cap=round,line join=round,line width=0.480pt]
    \path[draw] (184.8000,113.6000) -- (184.8000,113.6000) -- (184.9000,113.6000) -- (185.0000,113.6000) -- (185.0000,113.6000) -- (185.1000,113.6000) -- (185.1000,113.6000) -- (185.2000,113.6000) -- (185.3000,113.6000) -- (185.3000,113.6000) -- (185.4000,113.6000) -- (185.4000,113.6000) -- (185.5000,113.6000) -- (185.6000,113.6000) -- (185.6000,113.6000) -- (185.7000,113.6000) -- (185.8000,113.6000) -- (185.8000,113.6000) -- (185.9000,113.6000) -- (185.9000,113.6000) -- (186.0000,113.6000) -- (186.1000,113.6000) -- (186.1000,113.6000) -- (186.2000,113.6000) -- (186.2000,113.6000) -- (186.3000,113.6000) -- (186.4000,113.6000) -- (186.4000,113.6000) -- (186.5000,113.6000) -- (186.6000,113.6000) -- (186.6000,113.6000) -- (186.7000,113.6000) -- (186.7000,113.6000) -- (186.8000,113.6000) -- (186.9000,113.6000) -- (186.9000,113.6000) -- (187.0000,113.6000) -- (187.0000,113.6000) -- (187.1000,113.6000) -- (187.2000,113.6000) -- (187.2000,113.6000) -- (187.3000,113.6000) -- (187.4000,113.6000) -- (187.4000,113.6000) -- (187.5000,113.6000) -- (187.5000,113.6000) -- (187.6000,113.6000) -- (187.7000,113.6000) -- (187.7000,113.6000) -- (187.8000,113.6000) -- (187.8000,113.6000) -- (187.9000,113.6000) -- (188.0000,113.6000) -- (188.0000,113.6000) -- (188.1000,113.6000) -- (188.2000,113.6000) -- (188.2000,113.6000) -- (188.3000,113.6000) -- (188.3000,113.6000) -- (188.4000,113.6000) -- (188.5000,113.6000) -- (188.5000,113.6000) -- (188.6000,113.6000) -- (188.6000,113.6000) -- (188.7000,113.6000) -- (188.8000,113.6000) -- (188.8000,113.6000) -- (188.9000,113.6000) -- (189.0000,113.6000) -- (189.0000,113.6000) -- (189.1000,113.6000) -- (189.1000,113.6000) -- (189.2000,113.6000) -- (189.3000,113.6000) -- (189.3000,113.6000) -- (189.4000,113.6000) -- (189.4000,113.6000) -- (189.5000,113.6000) -- (189.6000,113.6000) -- (189.6000,113.6000) -- (189.7000,113.6000) -- (189.8000,113.6000) -- (189.8000,113.6000) -- (189.9000,113.6000) -- (189.9000,113.6000) -- (190.0000,113.6000) -- (190.1000,113.6000) -- (190.1000,113.6000) -- (190.2000,113.6000) -- (190.2000,113.6000) -- (190.3000,113.6000) -- (190.4000,113.6000) -- (190.4000,113.6000) -- (190.5000,113.6000) -- (190.6000,113.6000) -- (190.6000,113.6000) -- (190.7000,113.6000) -- (190.7000,113.6000) -- (190.8000,113.6000) -- (190.9000,113.6000) -- (190.9000,113.6000) -- (191.0000,113.6000) -- (191.0000,113.6000) -- (191.1000,113.6000) -- (191.2000,113.6000) -- (191.2000,113.6000) -- (191.3000,113.6000) -- (191.4000,113.6000) -- (191.4000,113.6000) -- (191.5000,113.6000) -- (191.5000,113.6000) -- (191.6000,113.6000) -- (191.7000,113.6000) -- (191.7000,113.6000) -- (191.8000,113.6000) -- (191.8000,113.6000) -- (191.9000,113.6000) -- (192.0000,113.6000) -- (192.0000,113.6000) -- (192.1000,113.6000) -- (192.2000,113.6000) -- (192.2000,113.6000) -- (192.3000,113.6000) -- (192.3000,113.6000) -- (192.4000,113.6000) -- (192.5000,113.6000) -- (192.5000,113.6000) -- (192.6000,113.6000) -- (192.6000,113.6000) -- (192.7000,113.6000) -- (192.8000,113.6000) -- (192.8000,113.6000) -- (192.9000,113.6000) -- (193.0000,113.6000) -- (193.0000,113.6000) -- (193.1000,113.6000) -- (193.1000,113.6000) -- (193.2000,113.6000) -- (193.3000,113.6000) -- (193.3000,113.6000) -- (193.4000,113.6000) -- (193.4000,113.6000) -- (193.5000,113.6000) -- (193.6000,113.6000) -- (193.6000,113.6000) -- (193.7000,113.6000) -- (193.8000,113.6000) -- (193.8000,113.6000) -- (193.9000,113.6000) -- (193.9000,113.6000) -- (194.0000,113.6000) -- (194.1000,113.6000) -- (194.1000,113.6000) -- (194.2000,113.6000) -- (194.2000,113.6000) -- (194.3000,113.6000) -- (194.4000,113.6000) -- (194.4000,113.6000) -- (194.5000,113.6000) -- (194.6000,113.6000) -- (194.6000,113.6000) -- (194.7000,113.6000) -- (194.7000,113.6000) -- (194.8000,113.6000) -- (194.9000,113.6000) -- (194.9000,113.6000) -- (195.0000,113.6000) -- (195.0000,113.6000) -- (195.1000,113.6000) -- (195.2000,113.6000) -- (195.2000,113.6000) -- (195.3000,113.6000) -- (195.4000,113.6000) -- (195.4000,113.6000) -- (195.5000,113.6000) -- (195.5000,113.6000) -- (195.6000,113.6000) -- (195.7000,113.6000) -- (195.7000,113.6000) -- (195.8000,113.6000) -- (195.8000,113.6000) -- (195.9000,113.6000) -- (196.0000,113.6000) -- (196.0000,113.6000) -- (196.1000,113.6000) -- (196.2000,113.6000) -- (196.2000,113.6000) -- (196.3000,113.6000) -- (196.3000,113.6000) -- (196.4000,113.6000) -- (196.5000,113.6000) -- (196.5000,113.6000) -- (196.6000,113.6000) -- (196.6000,113.6000) -- (196.7000,113.6000) -- (196.8000,113.6000) -- (196.8000,113.6000) -- (196.9000,113.6000) -- (196.9000,113.6000) -- (197.0000,113.6000) -- (197.1000,113.6000) -- (197.1000,113.6000) -- (197.2000,113.6000) -- (197.3000,113.6000) -- (197.3000,113.6000) -- (197.4000,113.6000) -- (197.4000,113.6000) -- (197.5000,113.6000) -- (197.6000,113.6000) -- (197.6000,113.6000) -- (197.7000,113.6000) -- (197.7000,113.6000) -- (197.8000,113.6000) -- (197.9000,113.6000) -- (197.9000,113.6000) -- (198.0000,113.6000) -- (198.1000,113.6000) -- (198.1000,113.6000) -- (198.2000,113.6000) -- (198.2000,113.6000) -- (198.3000,113.6000) -- (198.4000,113.6000) -- (198.4000,113.6000) -- (198.5000,113.6000) -- (198.5000,113.6000) -- (198.6000,113.6000) -- (198.7000,113.6000) -- (198.7000,113.6000) -- (198.8000,113.6000) -- (198.9000,113.6000) -- (198.9000,113.6000) -- (199.0000,113.6000) -- (199.0000,113.6000) -- (199.1000,113.6000) -- (199.2000,113.6000) -- (199.2000,113.6000) -- (199.3000,113.6000) -- (199.3000,113.6000) -- (199.4000,113.6000) -- (199.5000,113.6000) -- (199.5000,113.6000) -- (199.6000,113.6000) -- (199.7000,113.6000) -- (199.7000,113.6000) -- (199.8000,113.6000) -- (199.8000,113.6000) -- (199.9000,113.6000) -- (200.0000,113.6000) -- (200.0000,113.6000) -- (200.1000,113.6000) -- (200.1000,113.6000) -- (200.2000,113.6000) -- (200.3000,113.6000) -- (200.3000,113.6000) -- (200.4000,113.6000) -- (200.5000,113.6000) -- (200.5000,113.6000) -- (200.6000,113.6000) -- (200.6000,113.6000) -- (200.7000,113.6000) -- (200.8000,113.6000) -- (200.8000,113.6000) -- (200.9000,113.6000) -- (200.9000,113.6000) -- (201.0000,113.6000) -- (201.1000,113.6000) -- (201.1000,113.6000) -- (201.2000,113.6000) -- (201.3000,113.6000) -- (201.3000,113.6000) -- (201.4000,113.6000) -- (201.4000,113.6000) -- (201.5000,113.6000) -- (201.6000,113.6000) -- (201.6000,113.6000) -- (201.7000,113.6000) -- (201.7000,113.6000) -- (201.8000,113.6000) -- (201.9000,113.6000) -- (201.9000,113.6000) -- (202.0000,113.6000) -- (202.1000,113.6000) -- (202.1000,113.6000) -- (202.2000,113.6000) -- (202.2000,113.6000) -- (202.3000,113.6000) -- (202.4000,113.6000) -- (202.4000,113.6000) -- (202.5000,113.6000) -- (202.5000,113.6000) -- (202.6000,113.6000) -- (202.7000,113.6000) -- (202.7000,113.6000) -- (202.8000,113.6000) -- (202.9000,113.6000) -- (202.9000,113.6000) -- (203.0000,113.6000) -- (203.0000,113.6000) -- (203.1000,113.6000) -- (203.2000,113.6000) -- (203.2000,113.8000) -- (203.3000,121.5000) -- (203.3000,123.4000) -- (203.4000,123.1000) -- (203.5000,123.1000) -- (203.5000,123.1000) -- (203.6000,123.1000) -- (203.7000,123.1000) -- (203.7000,123.1000) -- (203.8000,123.1000) -- (203.8000,123.1000) -- (203.9000,123.1000) -- (204.0000,123.1000) -- (204.0000,123.1000) -- (204.1000,123.1000) -- (204.1000,123.1000) -- (204.2000,123.1000) -- (204.3000,123.1000) -- (204.3000,123.1000) -- (204.4000,123.1000) -- (204.5000,123.1000) -- (204.5000,123.1000) -- (204.6000,123.1000) -- (204.6000,123.1000) -- (204.7000,123.1000) -- (204.8000,123.1000) -- (204.8000,123.1000) -- (204.9000,122.8000) -- (204.9000,128.6000) -- (205.0000,132.8000) -- (205.1000,134.0000) -- (205.1000,141.8000) -- (205.2000,142.2000) -- (205.3000,142.1000) -- (205.3000,142.1000) -- (205.4000,142.1000) -- (205.4000,142.1000) -- (205.5000,141.8000) -- (205.6000,148.4000) -- (205.6000,152.0000) -- (205.7000,151.6000) -- (205.7000,151.6000) -- (205.8000,151.6000) -- (205.9000,151.6000) -- (205.9000,151.6000) -- (206.0000,151.6000) -- (206.1000,151.6000) -- (206.1000,151.6000) -- (206.2000,151.6000) -- (206.2000,151.9000) -- (206.3000,149.1000) -- (206.4000,142.0000) -- (206.4000,142.1000) -- (206.5000,142.1000) -- (206.5000,142.1000) -- (206.6000,142.1000) -- (206.7000,142.2000) -- (206.7000,142.2000) -- (206.8000,134.6000) -- (206.9000,132.4000) -- (206.9000,132.6000) -- (207.0000,132.6000) -- (207.0000,132.8000) -- (207.1000,140.5000) -- (207.2000,142.4000) -- (207.2000,142.1000) -- (207.3000,142.1000) -- (207.3000,142.1000) -- (207.4000,142.1000) -- (207.5000,142.1000) -- (207.5000,142.1000) -- (207.6000,142.1000) -- (207.7000,142.1000) -- (207.7000,142.1000) -- (207.8000,142.1000) -- (207.8000,142.1000) -- (207.9000,142.1000) -- (208.0000,142.1000) -- (208.0000,142.1000) -- (208.1000,142.1000) -- (208.1000,142.1000) -- (208.2000,142.1000) -- (208.3000,142.1000) -- (208.3000,142.1000) -- (208.4000,142.5000) -- (208.5000,137.1000) -- (208.5000,132.4000) -- (208.6000,131.6000) -- (208.6000,123.7000) -- (208.7000,123.1000) -- (208.8000,123.1000) -- (208.8000,123.3000) -- (208.9000,121.9000) -- (208.9000,114.0000) -- (209.0000,113.6000) -- (209.1000,113.6000) -- (209.1000,113.6000) -- (209.2000,113.6000) -- (209.3000,113.6000) -- (209.3000,113.6000) -- (209.4000,113.6000) -- (209.4000,113.4000) -- (209.5000,115.5000) -- (209.6000,123.1000) -- (209.6000,123.2000) -- (209.7000,123.1000) -- (209.7000,122.9000) -- (209.8000,125.3000) -- (209.9000,132.7000) -- (209.9000,132.5000) -- (210.0000,139.4000) -- (210.1000,142.5000) -- (210.1000,142.1000) -- (210.2000,142.1000) -- (210.2000,142.0000) -- (210.3000,149.3000) -- (210.4000,151.9000) -- (210.4000,151.6000) -- (210.5000,151.6000) -- (210.5000,151.6000) -- (210.6000,151.6000) -- (210.7000,151.6000) -- (210.7000,151.6000) -- (210.8000,151.6000) -- (210.9000,151.6000) -- (210.9000,151.6000) -- (211.0000,151.6000) -- (211.0000,151.6000) -- (211.1000,151.6000) -- (211.2000,151.6000) -- (211.2000,151.6000) -- (211.3000,151.6000) -- (211.3000,151.6000) -- (211.4000,151.6000) -- (211.5000,151.6000) -- (211.5000,151.6000) -- (211.6000,151.6000) -- (211.7000,151.6000) -- (211.7000,151.6000) -- (211.8000,151.6000) -- (211.8000,151.6000) -- (211.9000,151.6000) -- (212.0000,151.6000) -- (212.0000,151.6000) -- (212.1000,151.6000) -- (212.1000,151.6000) -- (212.2000,151.6000) -- (212.3000,151.6000) -- (212.3000,151.6000) -- (212.4000,151.6000) -- (212.5000,151.6000) -- (212.5000,151.6000) -- (212.6000,151.6000) -- (212.6000,151.6000) -- (212.7000,151.6000) -- (212.8000,151.6000) -- (212.8000,151.6000) -- (212.9000,151.6000) -- (212.9000,151.6000) -- (213.0000,151.6000) -- (213.1000,151.6000) -- (213.1000,151.6000) -- (213.2000,151.6000) -- (213.3000,151.6000) -- (213.3000,151.6000) -- (213.4000,151.6000) -- (213.4000,151.6000) -- (213.5000,151.6000) -- (213.6000,151.6000) -- (213.6000,151.6000) -- (213.7000,151.6000) -- (213.7000,151.6000) -- (213.8000,151.6000) -- (213.9000,151.6000) -- (213.9000,151.6000) -- (214.0000,151.6000) -- (214.1000,151.6000) -- (214.1000,151.6000) -- (214.2000,151.6000) -- (214.2000,151.6000) -- (214.3000,151.6000) -- (214.4000,151.6000) -- (214.4000,151.6000) -- (214.5000,151.6000) -- (214.5000,151.6000) -- (214.6000,151.6000) -- (214.7000,151.6000) -- (214.7000,151.6000) -- (214.8000,151.6000) -- (214.9000,151.6000) -- (214.9000,151.6000) -- (215.0000,151.6000) -- (215.0000,151.6000) -- (215.1000,151.6000) -- (215.2000,151.6000) -- (215.2000,151.6000) -- (215.3000,151.6000) -- (215.3000,151.6000) -- (215.4000,151.6000) -- (215.5000,151.6000) -- (215.5000,151.6000) -- (215.6000,151.6000) -- (215.7000,151.6000) -- (215.7000,151.6000) -- (215.8000,151.6000) -- (215.8000,151.6000) -- (215.9000,151.6000) -- (216.0000,151.6000) -- (216.0000,151.6000) -- (216.1000,151.6000) -- (216.1000,151.6000) -- (216.2000,151.6000) -- (216.3000,151.6000) -- (216.3000,151.6000) -- (216.4000,151.6000) -- (216.5000,151.6000) -- (216.5000,151.6000) -- (216.6000,151.6000) -- (216.6000,151.6000) -- (216.7000,151.6000) -- (216.8000,151.6000) -- (216.8000,151.6000) -- (216.9000,151.6000) -- (216.9000,151.6000) -- (217.0000,151.6000) -- (217.1000,151.6000) -- (217.1000,151.6000) -- (217.2000,151.6000) -- (217.3000,151.6000) -- (217.3000,151.6000) -- (217.4000,151.6000) -- (217.4000,151.6000) -- (217.5000,151.6000) -- (217.6000,151.6000) -- (217.6000,151.6000) -- (217.7000,151.6000) -- (217.7000,151.6000) -- (217.8000,151.6000) -- (217.9000,151.6000) -- (217.9000,151.6000) -- (218.0000,151.6000) -- (218.1000,151.6000) -- (218.1000,151.6000) -- (218.2000,151.6000) -- (218.2000,151.6000) -- (218.3000,151.6000) -- (218.4000,151.6000) -- (218.4000,151.6000) -- (218.5000,151.6000) -- (218.5000,151.6000) -- (218.6000,151.6000) -- (218.7000,151.6000) -- (218.7000,151.6000) -- (218.8000,151.6000) -- (218.9000,151.6000) -- (218.9000,151.6000) -- (219.0000,151.6000) -- (219.0000,151.6000) -- (219.1000,151.6000) -- (219.2000,151.6000) -- (219.2000,151.6000) -- (219.3000,151.6000) -- (219.3000,151.6000) -- (219.4000,151.6000) -- (219.5000,151.6000) -- (219.5000,151.6000) -- (219.6000,151.6000) -- (219.7000,151.6000) -- (219.7000,151.6000) -- (219.8000,151.6000) -- (219.8000,151.6000) -- (219.9000,151.6000) -- (220.0000,151.6000) -- (220.0000,151.6000) -- (220.1000,151.6000) -- (220.1000,151.6000) -- (220.2000,151.6000) -- (220.3000,151.6000) -- (220.3000,151.6000) -- (220.4000,151.6000) -- (220.5000,151.6000) -- (220.5000,151.6000) -- (220.6000,151.6000) -- (220.6000,151.6000) -- (220.7000,151.6000) -- (220.8000,151.6000) -- (220.8000,151.6000) -- (220.9000,151.6000) -- (220.9000,151.6000) -- (221.0000,151.6000) -- (221.1000,151.6000) -- (221.1000,151.6000) -- (221.2000,151.6000) -- (221.3000,151.6000) -- (221.3000,151.6000) -- (221.4000,151.6000) -- (221.4000,151.6000) -- (221.5000,151.6000) -- (221.6000,151.6000) -- (221.6000,151.6000) -- (221.7000,151.6000) -- (221.7000,151.6000) -- (221.8000,151.6000) -- (221.9000,151.6000) -- (221.9000,151.6000) -- (222.0000,151.6000) -- (222.1000,151.6000) -- (222.1000,151.6000) -- (222.2000,151.6000) -- (222.2000,151.6000) -- (222.3000,151.6000) -- (222.4000,151.6000) -- (222.4000,151.6000) -- (222.5000,151.6000) -- (222.5000,151.6000) -- (222.6000,151.6000) -- (222.7000,151.6000) -- (222.7000,151.6000) -- (222.8000,151.6000) -- (222.9000,151.6000) -- (222.9000,151.6000) -- (223.0000,151.6000) -- (223.0000,151.6000) -- (223.1000,151.6000) -- (223.2000,151.6000) -- (223.2000,151.6000) -- (223.3000,151.6000) -- (223.3000,151.6000) -- (223.4000,151.6000) -- (223.5000,151.6000) -- (223.5000,151.6000) -- (223.6000,151.6000) -- (223.7000,151.6000) -- (223.7000,151.6000) -- (223.8000,151.6000) -- (223.8000,151.6000) -- (223.9000,151.6000) -- (224.0000,151.6000) -- (224.0000,151.6000) -- (224.1000,151.6000) -- (224.1000,151.6000) -- (224.2000,151.6000) -- (224.3000,151.6000) -- (224.3000,151.6000) -- (224.4000,151.6000) -- (224.5000,151.6000) -- (224.5000,151.6000) -- (224.6000,151.6000) -- (224.6000,151.6000) -- (224.7000,151.6000) -- (224.8000,151.6000) -- (224.8000,151.6000) -- (224.9000,151.6000) -- (224.9000,151.6000) -- (225.0000,151.6000) -- (225.1000,151.6000) -- (225.1000,151.6000) -- (225.2000,151.6000) -- (225.3000,151.6000) -- (225.3000,151.6000) -- (225.4000,151.6000) -- (225.4000,151.6000) -- (225.5000,151.6000) -- (225.6000,151.6000) -- (225.6000,151.6000) -- (225.7000,151.6000) -- (225.7000,151.6000) -- (225.8000,151.6000) -- (225.9000,151.6000) -- (225.9000,151.6000) -- (226.0000,151.6000) -- (226.0000,151.6000) -- (226.1000,151.6000) -- (226.2000,151.6000) -- (226.2000,151.6000) -- (226.3000,151.6000) -- (226.4000,151.6000) -- (226.4000,151.6000) -- (226.5000,151.6000) -- (226.5000,151.6000) -- (226.6000,151.6000) -- (226.7000,151.6000) -- (226.7000,151.6000) -- (226.8000,151.6000) -- (226.8000,151.6000) -- (226.9000,151.6000) -- (227.0000,151.6000) -- (227.0000,151.6000) -- (227.1000,151.6000) -- (227.2000,151.6000) -- (227.2000,151.6000) -- (227.3000,151.6000) -- (227.3000,151.6000) -- (227.4000,151.6000) -- (227.5000,151.6000) -- (227.5000,151.6000) -- (227.6000,151.6000) -- (227.7000,151.6000) -- (227.7000,151.6000) -- (227.8000,151.6000) -- (227.8000,151.6000) -- (227.9000,151.6000) -- (228.0000,151.6000) -- (228.0000,151.6000) -- (228.1000,151.6000) -- (228.1000,151.6000) -- (228.2000,151.6000) -- (228.3000,151.6000) -- (228.3000,151.6000) -- (228.4000,151.6000) -- (228.5000,151.6000) -- (228.5000,151.6000) -- (228.6000,151.6000) -- (228.6000,151.6000) -- (228.7000,151.6000) -- (228.8000,151.6000) -- (228.8000,151.6000) -- (228.9000,151.6000) -- (228.9000,151.6000) -- (229.0000,151.6000) -- (229.1000,151.6000) -- (229.1000,151.6000) -- (229.2000,151.6000) -- (229.2000,151.6000) -- (229.3000,151.6000) -- (229.4000,151.6000) -- (229.4000,151.6000) -- (229.5000,151.6000) -- (229.6000,151.6000) -- (229.6000,151.6000) -- (229.7000,151.6000) -- (229.7000,151.6000) -- (229.8000,151.6000) -- (229.9000,151.6000) -- (229.9000,151.6000) -- (230.0000,151.6000) -- (230.0000,151.6000) -- (230.1000,151.6000) -- (230.2000,151.6000) -- (230.2000,151.6000) -- (230.3000,151.6000) -- (230.4000,151.6000) -- (230.4000,151.6000) -- (230.5000,151.6000) -- (230.5000,151.6000) -- (230.6000,151.6000) -- (230.7000,151.6000) -- (230.7000,151.6000) -- (230.8000,151.6000) -- (230.8000,151.6000) -- (230.9000,151.6000) -- (231.0000,151.6000) -- (231.0000,151.6000) -- (231.1000,151.6000) -- (231.2000,151.6000) -- (231.2000,151.6000) -- (231.3000,151.6000) -- (231.3000,151.6000) -- (231.4000,151.6000) -- (231.5000,151.6000) -- (231.5000,151.6000) -- (231.6000,151.6000) -- (231.6000,151.6000) -- (231.7000,151.6000) -- (231.8000,151.6000) -- (231.8000,151.6000) -- (231.9000,151.6000) -- (232.0000,151.6000) -- (232.0000,151.6000) -- (232.1000,151.6000) -- (232.1000,151.6000) -- (232.2000,151.6000) -- (232.3000,151.6000) -- (232.3000,151.6000) -- (232.4000,151.6000) -- (232.4000,151.6000) -- (232.5000,151.6000) -- (232.6000,151.6000) -- (232.6000,151.6000) -- (232.7000,151.6000) -- (232.8000,151.6000) -- (232.8000,151.6000) -- (232.9000,151.6000) -- (232.9000,151.6000) -- (233.0000,151.6000) -- (233.1000,151.6000) -- (233.1000,151.6000) -- (233.2000,151.6000) -- (233.2000,151.6000) -- (233.3000,151.6000) -- (233.4000,151.6000) -- (233.4000,151.6000) -- (233.5000,151.6000) -- (233.6000,151.6000) -- (233.6000,151.6000) -- (233.7000,151.6000) -- (233.7000,151.6000) -- (233.8000,151.6000) -- (233.9000,151.6000) -- (233.9000,151.6000) -- (234.0000,151.6000) -- (234.0000,151.6000) -- (234.1000,151.6000) -- (234.2000,151.6000) -- (234.2000,151.6000) -- (234.3000,151.6000) -- (234.4000,151.6000) -- (234.4000,151.6000) -- (234.5000,151.6000) -- (234.5000,151.6000) -- (234.6000,151.6000) -- (234.7000,151.6000) -- (234.7000,151.6000) -- (234.8000,151.6000) -- (234.8000,151.6000) -- (234.9000,151.6000) -- (235.0000,151.6000) -- (235.0000,151.6000) -- (235.1000,151.6000) -- (235.2000,151.6000) -- (235.2000,151.6000) -- (235.3000,151.6000) -- (235.3000,151.6000) -- (235.4000,151.6000) -- (235.5000,151.6000) -- (235.5000,151.6000) -- (235.6000,151.6000) -- (235.6000,151.6000) -- (235.7000,151.6000) -- (235.8000,151.6000) -- (235.8000,151.6000) -- (235.9000,151.6000) -- (236.0000,151.6000) -- (236.0000,151.6000) -- (236.1000,151.6000) -- (236.1000,151.6000) -- (236.2000,151.6000) -- (236.3000,151.6000) -- (236.3000,151.6000) -- (236.4000,151.6000) -- (236.4000,151.6000) -- (236.5000,151.6000) -- (236.6000,151.6000) -- (236.6000,151.6000) -- (236.7000,151.6000) -- (236.8000,151.6000) -- (236.8000,151.6000) -- (236.9000,151.6000) -- (236.9000,151.6000) -- (237.0000,151.6000) -- (237.1000,151.6000) -- (237.1000,151.6000) -- (237.2000,151.6000) -- (237.2000,151.6000) -- (237.3000,151.6000) -- (237.4000,151.6000) -- (237.4000,151.6000) -- (237.5000,151.6000) -- (237.6000,151.6000) -- (237.6000,151.6000) -- (237.7000,151.6000) -- (237.7000,151.6000) -- (237.8000,151.6000) -- (237.9000,151.6000) -- (237.9000,151.6000) -- (238.0000,151.6000) -- (238.0000,151.6000) -- (238.1000,151.6000) -- (238.2000,151.6000) -- (238.2000,151.6000) -- (238.3000,151.6000) -- (238.4000,151.6000) -- (238.4000,151.6000) -- (238.5000,151.6000) -- (238.5000,151.6000) -- (238.6000,151.6000) -- (238.7000,151.6000) -- (238.7000,151.6000) -- (238.8000,151.6000) -- (238.8000,151.6000) -- (238.9000,151.6000) -- (239.0000,151.6000) -- (239.0000,151.6000) -- (239.1000,151.6000) -- (239.2000,151.6000) -- (239.2000,151.6000) -- (239.3000,151.6000) -- (239.3000,151.6000) -- (239.4000,151.6000) -- (239.5000,151.6000) -- (239.5000,151.6000) -- (239.6000,151.6000) -- (239.6000,151.6000) -- (239.7000,151.6000) -- (239.8000,151.6000) -- (239.8000,151.6000) -- (239.9000,151.6000) -- (240.0000,151.6000) -- (240.0000,151.6000) -- (240.1000,151.6000) -- (240.1000,151.6000) -- (240.2000,151.6000) -- (240.3000,151.6000) -- (240.3000,151.6000) -- (240.4000,151.6000) -- (240.4000,151.6000) -- (240.5000,151.6000) -- (240.6000,151.6000) -- (240.6000,151.6000) -- (240.7000,151.6000) -- (240.8000,151.6000) -- (240.8000,151.6000) -- (240.9000,151.6000) -- (240.9000,151.6000) -- (241.0000,151.6000) -- (241.1000,151.6000) -- (241.1000,151.6000) -- (241.2000,151.6000) -- (241.2000,151.6000) -- (241.3000,151.6000) -- (241.4000,151.6000) -- (241.4000,151.6000) -- (241.5000,151.6000) -- (241.6000,151.6000) -- (241.6000,151.6000) -- (241.7000,151.6000) -- (241.7000,151.6000) -- (241.8000,151.6000) -- (241.9000,151.6000) -- (241.9000,151.6000) -- (242.0000,151.6000) -- (242.0000,151.6000) -- (242.1000,151.6000) -- (242.2000,151.6000) -- (242.2000,151.6000) -- (242.3000,151.6000) -- (242.4000,151.6000) -- (242.4000,151.6000) -- (242.5000,151.6000) -- (242.5000,151.6000) -- (242.6000,151.6000) -- (242.7000,151.6000) -- (242.7000,151.6000) -- (242.8000,151.6000) -- (242.8000,151.6000) -- (242.9000,151.6000) -- (243.0000,151.6000) -- (243.0000,151.6000) -- (243.1000,151.6000) -- (243.2000,151.6000) -- (243.2000,151.6000) -- (243.3000,151.6000) -- (243.3000,151.6000) -- (243.4000,151.6000) -- (243.5000,151.6000) -- (243.5000,151.6000) -- (243.6000,151.6000) -- (243.6000,151.6000) -- (243.7000,151.6000) -- (243.8000,151.6000) -- (243.8000,151.6000) -- (243.9000,151.6000) -- (244.0000,151.6000) -- (244.0000,151.6000) -- (244.1000,151.6000) -- (244.1000,151.6000) -- (244.2000,151.6000) -- (244.3000,151.6000) -- (244.3000,151.6000) -- (244.4000,151.6000) -- (244.4000,151.6000) -- (244.5000,151.6000) -- (244.6000,151.6000) -- (244.6000,151.6000) -- (244.7000,151.6000) -- (244.8000,151.6000) -- (244.8000,151.6000) -- (244.9000,151.6000) -- (244.9000,151.6000) -- (245.0000,151.6000) -- (245.1000,151.6000) -- (245.1000,151.6000) -- (245.2000,151.6000) -- (245.2000,151.6000) -- (245.3000,151.6000) -- (245.4000,151.6000) -- (245.4000,151.6000) -- (245.5000,151.6000) -- (245.6000,151.6000) -- (245.6000,151.6000) -- (245.7000,151.6000) -- (245.7000,151.6000) -- (245.8000,151.6000) -- (245.9000,151.6000) -- (245.9000,151.6000) -- (246.0000,151.6000) -- (246.0000,151.6000) -- (246.1000,151.6000) -- (246.2000,151.6000) -- (246.2000,151.6000) -- (246.3000,151.6000);



  \end{scope}
  \begin{scope}[cm={{0.92047,0.0,0.0,0.92047,(-569.12952,52.34945)}},draw=blue,line cap=round,line join=round,line width=0.480pt]
    \path[draw] (184.5000,108.5000) -- (184.5000,156.5000) -- (246.5000,156.5000) -- (246.5000,108.5000) -- (184.5000,108.5000);



  \end{scope}
  \begin{scope}[cm={{0.92047,0.0,0.0,0.92047,(-569.12952,52.34945)}},draw=ca0a0a4,dash pattern=on 1.96pt off 1.96pt,line cap=round,line join=round,line width=0.326pt,miter limit=4.00]
    \path[draw,dash pattern=on 1.96pt off 1.96pt,line width=0.326pt,miter limit=4.00] (184.5000,200.5000) -- (246.5000,200.5000);



  \end{scope}
  \begin{scope}[cm={{0.92047,0.0,0.0,0.92047,(-569.12952,52.34945)}},draw=blue,line cap=round,line join=round,line width=0.480pt]
    \path[draw] (184.5000,200.5000) -- (186.5000,200.5000);



    \path[draw] (246.5000,200.5000) -- (245.5000,200.5000);



  \end{scope}
  \begin{scope}[cm={{0.92047,0.0,0.0,0.92047,(-569.12952,52.34945)}},draw=ca0a0a4,dash pattern=on 1.96pt off 1.96pt,line cap=round,line join=round,line width=0.326pt,miter limit=4.00]
    \path[draw,dash pattern=on 1.96pt off 1.96pt,line width=0.326pt,miter limit=4.00] (184.5000,180.5000) -- (246.5000,180.5000);



  \end{scope}
  \begin{scope}[cm={{0.92047,0.0,0.0,0.92047,(-569.12952,52.34945)}},draw=blue,line cap=round,line join=round,line width=0.480pt]
    \path[draw] (184.5000,180.5000) -- (186.5000,180.5000);



    \path[draw] (246.5000,180.5000) -- (245.5000,180.5000);



  \end{scope}
  \begin{scope}[cm={{0.92047,0.0,0.0,0.92047,(-569.12952,52.34945)}},draw=ca0a0a4,dash pattern=on 1.96pt off 1.96pt,line cap=round,line join=round,line width=0.326pt,miter limit=4.00]
    \path[draw,dash pattern=on 1.96pt off 1.96pt,line width=0.326pt,miter limit=4.00] (184.5000,160.5000) -- (246.5000,160.5000);



  \end{scope}
  \begin{scope}[cm={{0.92047,0.0,0.0,0.92047,(-569.12952,52.34945)}},draw=blue,line cap=round,line join=round,line width=0.480pt]
    \path[draw] (184.5000,160.5000) -- (186.5000,160.5000);



    \path[draw] (246.5000,160.5000) -- (245.5000,160.5000);



  \end{scope}
  \begin{scope}[cm={{0.92047,0.0,0.0,0.92047,(-569.12952,52.34945)}},draw=ca0a0a4,dash pattern=on 0.40pt off 0.80pt,line cap=round,line join=round,line width=0.400pt]
    \path[draw] (184.5000,204.5000) -- (184.5000,156.5000);



  \end{scope}
  \begin{scope}[cm={{0.92047,0.0,0.0,0.92047,(-569.12952,52.34945)}},draw=blue,line cap=round,line join=round,line width=0.480pt]
    \path[draw] (184.5000,204.5000) -- (184.5000,203.5000);



    \path[draw] (184.5000,156.5000) -- (184.5000,157.5000);



  \end{scope}
  \begin{scope}[cm={{1.00588,0.0,0.0,1.00588,(-400.6796,252.66367)}},draw=blue,line cap=rect,line join=bevel,line width=0.800pt]
    \path[fill=blue] (0.0000,0.0000) node[above right] (text1416) {\scriptsize 0};



  \end{scope}
  \begin{scope}[cm={{0.92047,0.0,0.0,0.92047,(-569.12952,52.34945)}},draw=ca0a0a4,dash pattern=on 1.96pt off 1.96pt,line cap=round,line join=round,line width=0.326pt,miter limit=4.00]
    \path[draw,dash pattern=on 1.96pt off 1.96pt,line width=0.326pt,miter limit=4.00] (203.5000,204.5000) -- (203.5000,156.5000);



  \end{scope}
  \begin{scope}[cm={{0.92047,0.0,0.0,0.92047,(-569.12952,52.34945)}},draw=blue,line cap=round,line join=round,line width=0.480pt]
    \path[draw] (203.5000,204.5000) -- (203.5000,203.5000);



    \path[draw] (203.5000,156.5000) -- (203.5000,157.5000);



  \end{scope}
  \begin{scope}[cm={{1.00588,0.0,0.0,1.00588,(-384.0746,252.77633)}},draw=blue,line cap=rect,line join=bevel,line width=0.800pt]
    \path[fill=blue] (0.0000,0.0000) node[above right] (text1446) {\scriptsize 2};



  \end{scope}
  \begin{scope}[cm={{0.92047,0.0,0.0,0.92047,(-569.12952,52.34945)}},draw=ca0a0a4,dash pattern=on 1.96pt off 1.96pt,line cap=round,line join=round,line width=0.326pt,miter limit=4.00]
    \path[draw,dash pattern=on 1.96pt off 1.96pt,line width=0.326pt,miter limit=4.00] (221.5000,204.5000) -- (221.5000,156.5000);



  \end{scope}
  \begin{scope}[cm={{0.92047,0.0,0.0,0.92047,(-569.12952,52.34945)}},draw=blue,line cap=round,line join=round,line width=0.480pt]
    \path[draw] (221.5000,204.5000) -- (221.5000,203.5000);



    \path[draw] (221.5000,156.5000) -- (221.5000,157.5000);



  \end{scope}
  \begin{scope}[cm={{1.00588,0.0,0.0,1.00588,(-366.9656,252.77633)}},draw=blue,line cap=rect,line join=bevel,line width=0.800pt]
    \path[fill=blue] (0.0000,0.0000) node[above right] (text1476) {\scriptsize 4};



  \end{scope}
  \begin{scope}[cm={{0.92047,0.0,0.0,0.92047,(-569.12952,52.34945)}},draw=ca0a0a4,dash pattern=on 1.96pt off 1.96pt,line cap=round,line join=round,line width=0.326pt,miter limit=4.00]
    \path[draw,dash pattern=on 1.96pt off 1.96pt,line width=0.326pt,miter limit=4.00] (239.5000,204.5000) -- (239.5000,156.5000);



  \end{scope}
  \begin{scope}[cm={{0.92047,0.0,0.0,0.92047,(-569.12952,52.34945)}},draw=blue,line cap=round,line join=round,line width=0.480pt]
    \path[draw] (239.5000,204.5000) -- (239.5000,203.5000);



    \path[draw] (239.5000,156.5000) -- (239.5000,157.5000);



  \end{scope}
  \begin{scope}[cm={{1.00588,0.0,0.0,1.00588,(-351.3566,252.66367)}},draw=blue,line cap=rect,line join=bevel,line width=0.800pt]
    \path[fill=blue] (0.0000,0.0000) node[above right] (text1506) {\scriptsize 6};



  \end{scope}
  \begin{scope}[cm={{0.92047,0.0,0.0,0.92047,(-569.12952,52.34945)}},draw=blue,line cap=round,line join=round,line width=0.480pt]
    \path[draw] (246.5000,200.5000) -- (245.5000,200.5000);



  \end{scope}
  \begin{scope}[cm={{0.92047,0.0,0.0,0.92047,(-569.12952,52.34945)}},draw=blue,line cap=round,line join=round,line width=0.480pt]
    \path[draw] (246.5000,180.5000) -- (245.5000,180.5000);



  \end{scope}
  \begin{scope}[cm={{0.92047,0.0,0.0,0.92047,(-569.12952,52.34945)}},draw=blue,line cap=round,line join=round,line width=0.480pt]
    \path[draw] (246.5000,160.5000) -- (245.5000,160.5000);



  \end{scope}
  \begin{scope}[cm={{0.92047,0.0,0.0,0.92047,(-569.12952,52.34945)}},draw=blue,line cap=round,line join=round,line width=0.480pt]
    \path[draw] (184.5000,156.5000) -- (184.5000,204.5000) -- (246.5000,204.5000) -- (246.5000,156.5000) -- (184.5000,156.5000);



  \end{scope}
  \begin{scope}[cm={{0.92047,0.0,0.0,0.92047,(-569.12952,52.34945)}},draw=blue,line cap=round,line join=round,line width=0.480pt]
    \path[draw] (184.8000,160.5000) -- (184.8000,160.5000) -- (184.9000,160.5000) -- (185.0000,160.5000) -- (185.0000,160.5000) -- (185.1000,160.5000) -- (185.1000,160.5000) -- (185.2000,160.5000) -- (185.3000,160.5000) -- (185.3000,160.5000) -- (185.4000,160.5000) -- (185.4000,160.5000) -- (185.5000,160.5000) -- (185.6000,160.5000) -- (185.6000,160.5000) -- (185.7000,160.5000) -- (185.8000,160.5000) -- (185.8000,160.5000) -- (185.9000,160.5000) -- (185.9000,160.5000) -- (186.0000,160.5000) -- (186.1000,160.5000) -- (186.1000,160.5000) -- (186.2000,160.5000) -- (186.2000,160.5000) -- (186.3000,160.5000) -- (186.4000,160.5000) -- (186.4000,160.5000) -- (186.5000,160.5000) -- (186.6000,160.5000) -- (186.6000,160.5000) -- (186.7000,160.5000) -- (186.7000,160.5000) -- (186.8000,160.5000) -- (186.9000,160.5000) -- (186.9000,160.5000) -- (187.0000,160.5000) -- (187.0000,160.5000) -- (187.1000,160.5000) -- (187.2000,160.5000) -- (187.2000,160.5000) -- (187.3000,160.5000) -- (187.4000,160.5000) -- (187.4000,160.5000) -- (187.5000,160.5000) -- (187.5000,160.5000) -- (187.6000,160.5000) -- (187.7000,160.5000) -- (187.7000,160.5000) -- (187.8000,160.5000) -- (187.8000,160.5000) -- (187.9000,160.5000) -- (188.0000,160.5000) -- (188.0000,160.5000) -- (188.1000,160.5000) -- (188.2000,160.5000) -- (188.2000,160.5000) -- (188.3000,160.5000) -- (188.3000,160.5000) -- (188.4000,160.5000) -- (188.5000,160.5000) -- (188.5000,160.5000) -- (188.6000,160.5000) -- (188.6000,160.5000) -- (188.7000,160.5000) -- (188.8000,160.5000) -- (188.8000,160.5000) -- (188.9000,160.5000) -- (189.0000,160.5000) -- (189.0000,160.5000) -- (189.1000,160.5000) -- (189.1000,160.5000) -- (189.2000,160.5000) -- (189.3000,160.5000) -- (189.3000,160.5000) -- (189.4000,160.5000) -- (189.4000,160.5000) -- (189.5000,160.5000) -- (189.6000,160.5000) -- (189.6000,160.5000) -- (189.7000,160.5000) -- (189.8000,160.5000) -- (189.8000,160.5000) -- (189.9000,160.5000) -- (189.9000,160.5000) -- (190.0000,160.5000) -- (190.1000,160.5000) -- (190.1000,160.5000) -- (190.2000,160.5000) -- (190.2000,160.5000) -- (190.3000,160.5000) -- (190.4000,160.5000) -- (190.4000,160.5000) -- (190.5000,160.5000) -- (190.6000,160.5000) -- (190.6000,160.5000) -- (190.7000,160.5000) -- (190.7000,160.5000) -- (190.8000,160.5000) -- (190.9000,160.5000) -- (190.9000,160.5000) -- (191.0000,160.5000) -- (191.0000,160.5000) -- (191.1000,160.5000) -- (191.2000,160.5000) -- (191.2000,160.5000) -- (191.3000,160.5000) -- (191.4000,160.5000) -- (191.4000,160.5000) -- (191.5000,160.5000) -- (191.5000,160.5000) -- (191.6000,160.5000) -- (191.7000,160.5000) -- (191.7000,160.5000) -- (191.8000,160.5000) -- (191.8000,160.5000) -- (191.9000,160.5000) -- (192.0000,160.5000) -- (192.0000,160.5000) -- (192.1000,160.5000) -- (192.2000,160.5000) -- (192.2000,160.5000) -- (192.3000,160.5000) -- (192.3000,160.5000) -- (192.4000,160.5000) -- (192.5000,160.5000) -- (192.5000,160.5000) -- (192.6000,160.5000) -- (192.6000,160.5000) -- (192.7000,160.5000) -- (192.8000,160.5000) -- (192.8000,160.5000) -- (192.9000,160.5000) -- (193.0000,160.5000) -- (193.0000,160.5000) -- (193.1000,160.5000) -- (193.1000,160.5000) -- (193.2000,160.5000) -- (193.3000,160.5000) -- (193.3000,160.5000) -- (193.4000,160.5000) -- (193.4000,160.5000) -- (193.5000,160.5000) -- (193.6000,160.5000) -- (193.6000,160.5000) -- (193.7000,160.5000) -- (193.8000,160.5000) -- (193.8000,160.5000) -- (193.9000,160.5000) -- (193.9000,160.5000) -- (194.0000,160.5000) -- (194.1000,160.5000) -- (194.1000,160.5000) -- (194.2000,160.5000) -- (194.2000,160.5000) -- (194.3000,160.5000) -- (194.4000,160.5000) -- (194.4000,160.5000) -- (194.5000,160.5000) -- (194.6000,160.5000) -- (194.6000,160.5000) -- (194.7000,160.5000) -- (194.7000,160.5000) -- (194.8000,160.5000) -- (194.9000,160.5000) -- (194.9000,160.5000) -- (195.0000,160.5000) -- (195.0000,160.5000) -- (195.1000,160.5000) -- (195.2000,160.5000) -- (195.2000,160.5000) -- (195.3000,160.5000) -- (195.4000,160.5000) -- (195.4000,160.5000) -- (195.5000,160.5000) -- (195.5000,160.5000) -- (195.6000,160.5000) -- (195.7000,160.5000) -- (195.7000,160.5000) -- (195.8000,160.5000) -- (195.8000,160.5000) -- (195.9000,160.5000) -- (196.0000,160.5000) -- (196.0000,160.5000) -- (196.1000,160.5000) -- (196.2000,160.5000) -- (196.2000,160.5000) -- (196.3000,160.5000) -- (196.3000,160.5000) -- (196.4000,160.5000) -- (196.5000,160.5000) -- (196.5000,160.5000) -- (196.6000,160.5000) -- (196.6000,160.5000) -- (196.7000,160.5000) -- (196.8000,160.5000) -- (196.8000,160.5000) -- (196.9000,160.5000) -- (196.9000,160.5000) -- (197.0000,160.5000) -- (197.1000,160.5000) -- (197.1000,160.5000) -- (197.2000,160.5000) -- (197.3000,160.5000) -- (197.3000,160.5000) -- (197.4000,160.5000) -- (197.4000,160.5000) -- (197.5000,160.5000) -- (197.6000,160.5000) -- (197.6000,160.5000) -- (197.7000,160.5000) -- (197.7000,160.5000) -- (197.8000,160.5000) -- (197.9000,160.5000) -- (197.9000,160.5000) -- (198.0000,160.5000) -- (198.1000,160.5000) -- (198.1000,160.5000) -- (198.2000,160.5000) -- (198.2000,160.5000) -- (198.3000,160.5000) -- (198.4000,160.5000) -- (198.4000,160.5000) -- (198.5000,160.5000) -- (198.5000,160.5000) -- (198.6000,160.5000) -- (198.7000,160.5000) -- (198.7000,160.5000) -- (198.8000,160.5000) -- (198.9000,159.9000) -- (198.9000,165.0000) -- (199.0000,180.4000) -- (199.0000,180.3000) -- (199.1000,180.2000) -- (199.2000,180.2000) -- (199.2000,180.2000) -- (199.3000,180.2000) -- (199.3000,180.2000) -- (199.4000,180.2000) -- (199.5000,180.2000) -- (199.5000,180.2000) -- (199.6000,180.2000) -- (199.7000,180.2000) -- (199.7000,180.2000) -- (199.8000,180.2000) -- (199.8000,180.2000) -- (199.9000,180.2000) -- (200.0000,180.2000) -- (200.0000,180.2000) -- (200.1000,180.2000) -- (200.1000,180.2000) -- (200.2000,180.2000) -- (200.3000,180.2000) -- (200.3000,180.2000) -- (200.4000,180.2000) -- (200.5000,180.2000) -- (200.5000,180.2000) -- (200.6000,180.2000) -- (200.6000,180.2000) -- (200.7000,180.2000) -- (200.8000,180.2000) -- (200.8000,180.2000) -- (200.9000,180.2000) -- (200.9000,180.2000) -- (201.0000,180.2000) -- (201.1000,180.2000) -- (201.1000,180.2000) -- (201.2000,180.2000) -- (201.3000,180.2000) -- (201.3000,180.2000) -- (201.4000,180.2000) -- (201.4000,180.2000) -- (201.5000,180.2000) -- (201.6000,180.2000) -- (201.6000,180.2000) -- (201.7000,180.2000) -- (201.7000,180.2000) -- (201.8000,180.2000) -- (201.9000,180.2000) -- (201.9000,180.2000) -- (202.0000,180.2000) -- (202.1000,180.2000) -- (202.1000,180.2000) -- (202.2000,180.2000) -- (202.2000,180.2000) -- (202.3000,180.2000) -- (202.4000,180.2000) -- (202.4000,180.2000) -- (202.5000,180.2000) -- (202.5000,180.2000) -- (202.6000,180.2000) -- (202.7000,180.2000) -- (202.7000,180.2000) -- (202.8000,180.2000) -- (202.9000,180.2000) -- (202.9000,180.2000) -- (203.0000,180.2000) -- (203.0000,180.2000) -- (203.1000,180.2000) -- (203.2000,180.2000) -- (203.2000,180.2000) -- (203.3000,180.2000) -- (203.3000,180.2000) -- (203.4000,180.2000) -- (203.5000,180.2000) -- (203.5000,180.2000) -- (203.6000,180.2000) -- (203.7000,180.2000) -- (203.7000,180.2000) -- (203.8000,180.2000) -- (203.8000,180.2000) -- (203.9000,180.2000) -- (204.0000,180.2000) -- (204.0000,180.2000) -- (204.1000,180.2000) -- (204.1000,180.2000) -- (204.2000,180.2000) -- (204.3000,180.2000) -- (204.3000,180.2000) -- (204.4000,180.2000) -- (204.5000,180.2000) -- (204.5000,180.2000) -- (204.6000,180.2000) -- (204.6000,180.2000) -- (204.7000,180.2000) -- (204.8000,180.2000) -- (204.8000,180.2000) -- (204.9000,180.2000) -- (204.9000,180.2000) -- (205.0000,180.2000) -- (205.1000,180.2000) -- (205.1000,180.2000) -- (205.2000,180.2000) -- (205.3000,180.2000) -- (205.3000,180.2000) -- (205.4000,180.2000) -- (205.4000,180.2000) -- (205.5000,180.2000) -- (205.6000,180.2000) -- (205.6000,180.2000) -- (205.7000,180.2000) -- (205.7000,180.2000) -- (205.8000,180.2000) -- (205.9000,180.2000) -- (205.9000,180.2000) -- (206.0000,180.2000) -- (206.1000,180.2000) -- (206.1000,180.2000) -- (206.2000,180.2000) -- (206.2000,180.2000) -- (206.3000,180.2000) -- (206.4000,180.2000) -- (206.4000,180.2000) -- (206.5000,180.2000) -- (206.5000,180.2000) -- (206.6000,180.2000) -- (206.7000,180.2000) -- (206.7000,180.2000) -- (206.8000,180.2000) -- (206.9000,180.2000) -- (206.9000,180.2000) -- (207.0000,180.2000) -- (207.0000,180.2000) -- (207.1000,180.2000) -- (207.2000,180.2000) -- (207.2000,180.2000) -- (207.3000,180.2000) -- (207.3000,180.2000) -- (207.4000,180.2000) -- (207.5000,180.2000) -- (207.5000,180.2000) -- (207.6000,180.2000) -- (207.7000,180.2000) -- (207.7000,180.2000) -- (207.8000,180.2000) -- (207.8000,180.2000) -- (207.9000,180.2000) -- (208.0000,180.2000) -- (208.0000,180.2000) -- (208.1000,180.2000) -- (208.1000,180.2000) -- (208.2000,180.2000) -- (208.3000,180.2000) -- (208.3000,180.2000) -- (208.4000,180.2000) -- (208.5000,180.2000) -- (208.5000,180.2000) -- (208.6000,180.2000) -- (208.6000,180.2000) -- (208.7000,180.2000) -- (208.8000,180.2000) -- (208.8000,180.2000) -- (208.9000,180.2000) -- (208.9000,180.2000) -- (209.0000,180.2000) -- (209.1000,180.2000) -- (209.1000,180.2000) -- (209.2000,180.2000) -- (209.3000,180.2000) -- (209.3000,180.2000) -- (209.4000,180.2000) -- (209.4000,180.2000) -- (209.5000,180.2000) -- (209.6000,180.2000) -- (209.6000,180.2000) -- (209.7000,180.2000) -- (209.7000,180.2000) -- (209.8000,180.2000) -- (209.9000,180.2000) -- (209.9000,180.2000) -- (210.0000,180.2000) -- (210.1000,180.2000) -- (210.1000,180.2000) -- (210.2000,180.2000) -- (210.2000,180.2000) -- (210.3000,180.2000) -- (210.4000,180.2000) -- (210.4000,180.2000) -- (210.5000,180.2000) -- (210.5000,180.2000) -- (210.6000,180.2000) -- (210.7000,180.2000) -- (210.7000,180.2000) -- (210.8000,180.2000) -- (210.9000,180.2000) -- (210.9000,180.2000) -- (211.0000,180.2000) -- (211.0000,180.2000) -- (211.1000,180.2000) -- (211.2000,180.2000) -- (211.2000,180.2000) -- (211.3000,180.2000) -- (211.3000,180.2000) -- (211.4000,180.2000) -- (211.5000,180.2000) -- (211.5000,180.2000) -- (211.6000,180.2000) -- (211.7000,180.2000) -- (211.7000,180.2000) -- (211.8000,180.2000) -- (211.8000,180.2000) -- (211.9000,180.2000) -- (212.0000,180.2000) -- (212.0000,180.2000) -- (212.1000,180.2000) -- (212.1000,180.2000) -- (212.2000,180.2000) -- (212.3000,180.2000) -- (212.3000,180.2000) -- (212.4000,180.2000) -- (212.5000,180.2000) -- (212.5000,180.2000) -- (212.6000,180.2000) -- (212.6000,180.2000) -- (212.7000,180.2000) -- (212.8000,180.2000) -- (212.8000,180.2000) -- (212.9000,180.2000) -- (212.9000,180.2000) -- (213.0000,180.2000) -- (213.1000,180.2000) -- (213.1000,180.2000) -- (213.2000,180.2000) -- (213.3000,180.2000) -- (213.3000,180.2000) -- (213.4000,180.2000) -- (213.4000,180.2000) -- (213.5000,180.2000) -- (213.6000,180.2000) -- (213.6000,180.2000) -- (213.7000,180.2000) -- (213.7000,180.2000) -- (213.8000,180.2000) -- (213.9000,180.2000) -- (213.9000,180.2000) -- (214.0000,180.2000) -- (214.1000,180.2000) -- (214.1000,180.2000) -- (214.2000,180.2000) -- (214.2000,180.2000) -- (214.3000,180.2000) -- (214.4000,180.2000) -- (214.4000,180.2000) -- (214.5000,180.2000) -- (214.5000,180.2000) -- (214.6000,180.2000) -- (214.7000,180.2000) -- (214.7000,180.2000) -- (214.8000,180.2000) -- (214.9000,180.2000) -- (214.9000,180.2000) -- (215.0000,180.2000) -- (215.0000,180.2000) -- (215.1000,180.2000) -- (215.2000,180.2000) -- (215.2000,180.2000) -- (215.3000,180.2000) -- (215.3000,180.2000) -- (215.4000,180.2000) -- (215.5000,180.2000) -- (215.5000,180.2000) -- (215.6000,180.2000) -- (215.7000,180.2000) -- (215.7000,180.2000) -- (215.8000,180.2000) -- (215.8000,180.2000) -- (215.9000,180.2000) -- (216.0000,180.2000) -- (216.0000,180.2000) -- (216.1000,180.2000) -- (216.1000,180.2000) -- (216.2000,180.2000) -- (216.3000,180.2000) -- (216.3000,180.2000) -- (216.4000,180.2000) -- (216.5000,180.2000) -- (216.5000,180.2000) -- (216.6000,180.2000) -- (216.6000,180.2000) -- (216.7000,180.2000) -- (216.8000,180.2000) -- (216.8000,180.2000) -- (216.9000,180.2000) -- (216.9000,180.2000) -- (217.0000,180.2000) -- (217.1000,180.2000) -- (217.1000,180.2000) -- (217.2000,180.2000) -- (217.3000,180.2000) -- (217.3000,180.2000) -- (217.4000,180.2000) -- (217.4000,180.2000) -- (217.5000,180.2000) -- (217.6000,180.2000) -- (217.6000,180.2000) -- (217.7000,180.2000) -- (217.7000,180.2000) -- (217.8000,180.2000) -- (217.9000,180.2000) -- (217.9000,180.2000) -- (218.0000,180.2000) -- (218.1000,180.2000) -- (218.1000,180.2000) -- (218.2000,180.2000) -- (218.2000,180.2000) -- (218.3000,180.2000) -- (218.4000,180.2000) -- (218.4000,180.2000) -- (218.5000,180.2000) -- (218.5000,180.2000) -- (218.6000,180.2000) -- (218.7000,180.2000) -- (218.7000,180.2000) -- (218.8000,180.2000) -- (218.9000,180.2000) -- (218.9000,180.2000) -- (219.0000,180.2000) -- (219.0000,180.2000) -- (219.1000,180.2000) -- (219.2000,180.2000) -- (219.2000,180.2000) -- (219.3000,180.2000) -- (219.3000,180.2000) -- (219.4000,180.2000) -- (219.5000,180.2000) -- (219.5000,180.2000) -- (219.6000,180.2000) -- (219.7000,180.2000) -- (219.7000,180.2000) -- (219.8000,180.2000) -- (219.8000,180.2000) -- (219.9000,180.2000) -- (220.0000,180.2000) -- (220.0000,180.2000) -- (220.1000,180.2000) -- (220.1000,180.2000) -- (220.2000,180.2000) -- (220.3000,180.2000) -- (220.3000,180.2000) -- (220.4000,180.2000) -- (220.5000,180.2000) -- (220.5000,180.2000) -- (220.6000,180.2000) -- (220.6000,180.2000) -- (220.7000,180.2000) -- (220.8000,180.2000) -- (220.8000,180.2000) -- (220.9000,180.2000) -- (220.9000,180.2000) -- (221.0000,180.2000) -- (221.1000,180.2000) -- (221.1000,180.2000) -- (221.2000,180.2000) -- (221.3000,180.2000) -- (221.3000,180.2000) -- (221.4000,180.2000) -- (221.4000,180.2000) -- (221.5000,180.2000) -- (221.6000,180.2000) -- (221.6000,180.2000) -- (221.7000,180.2000) -- (221.7000,180.2000) -- (221.8000,180.2000) -- (221.9000,180.2000) -- (221.9000,180.2000) -- (222.0000,180.2000) -- (222.1000,180.2000) -- (222.1000,180.2000) -- (222.2000,180.2000) -- (222.2000,180.2000) -- (222.3000,180.2000) -- (222.4000,180.2000) -- (222.4000,180.2000) -- (222.5000,180.2000) -- (222.5000,180.2000) -- (222.6000,180.2000) -- (222.7000,180.2000) -- (222.7000,180.2000) -- (222.8000,180.2000) -- (222.9000,180.2000) -- (222.9000,180.2000) -- (223.0000,180.2000) -- (223.0000,180.2000) -- (223.1000,180.2000) -- (223.2000,180.2000) -- (223.2000,180.2000) -- (223.3000,180.2000) -- (223.3000,180.2000) -- (223.4000,180.2000) -- (223.5000,180.2000) -- (223.5000,180.2000) -- (223.6000,180.2000) -- (223.7000,180.2000) -- (223.7000,180.2000) -- (223.8000,180.2000) -- (223.8000,180.2000) -- (223.9000,180.2000) -- (224.0000,180.2000) -- (224.0000,180.2000) -- (224.1000,180.2000) -- (224.1000,180.2000) -- (224.2000,180.2000) -- (224.3000,180.2000) -- (224.3000,180.2000) -- (224.4000,180.2000) -- (224.5000,180.2000) -- (224.5000,180.2000) -- (224.6000,180.2000) -- (224.6000,180.2000) -- (224.7000,180.2000) -- (224.8000,180.2000) -- (224.8000,180.2000) -- (224.9000,180.2000) -- (224.9000,180.2000) -- (225.0000,180.2000) -- (225.1000,180.2000) -- (225.1000,180.2000) -- (225.2000,180.2000) -- (225.3000,180.2000) -- (225.3000,180.2000) -- (225.4000,180.2000) -- (225.4000,180.2000) -- (225.5000,180.2000) -- (225.6000,180.2000) -- (225.6000,180.2000) -- (225.7000,180.2000) -- (225.7000,180.2000) -- (225.8000,180.2000) -- (225.9000,179.4000) -- (225.9000,190.4000) -- (226.0000,200.9000) -- (226.0000,200.0000) -- (226.1000,200.0000) -- (226.2000,200.0000) -- (226.2000,200.0000) -- (226.3000,200.0000) -- (226.4000,200.0000) -- (226.4000,200.0000) -- (226.5000,200.0000) -- (226.5000,200.0000) -- (226.6000,200.0000) -- (226.7000,200.0000) -- (226.7000,200.0000) -- (226.8000,200.0000) -- (226.8000,200.0000) -- (226.9000,200.0000) -- (227.0000,200.0000) -- (227.0000,200.0000) -- (227.1000,200.0000) -- (227.2000,200.0000) -- (227.2000,200.0000) -- (227.3000,200.0000) -- (227.3000,200.0000) -- (227.4000,200.0000) -- (227.5000,200.0000) -- (227.5000,200.0000) -- (227.6000,200.0000) -- (227.7000,200.0000) -- (227.7000,200.0000) -- (227.8000,200.0000) -- (227.8000,200.0000) -- (227.9000,200.0000) -- (228.0000,200.0000) -- (228.0000,200.0000) -- (228.1000,200.0000) -- (228.1000,200.0000) -- (228.2000,200.0000) -- (228.3000,200.0000) -- (228.3000,200.0000) -- (228.4000,200.0000) -- (228.5000,200.0000) -- (228.5000,200.0000) -- (228.6000,200.0000) -- (228.6000,200.0000) -- (228.7000,200.0000) -- (228.8000,200.0000) -- (228.8000,200.0000) -- (228.9000,200.0000) -- (228.9000,200.0000) -- (229.0000,200.0000) -- (229.1000,200.0000) -- (229.1000,200.0000) -- (229.2000,200.0000) -- (229.2000,200.0000) -- (229.3000,200.0000) -- (229.4000,200.0000) -- (229.4000,200.0000) -- (229.5000,200.0000) -- (229.6000,200.0000) -- (229.6000,200.0000) -- (229.7000,200.0000) -- (229.7000,200.0000) -- (229.8000,200.0000) -- (229.9000,200.0000) -- (229.9000,200.0000) -- (230.0000,200.0000) -- (230.0000,200.0000) -- (230.1000,200.0000) -- (230.2000,200.0000) -- (230.2000,200.0000) -- (230.3000,200.0000) -- (230.4000,200.0000) -- (230.4000,200.0000) -- (230.5000,200.0000) -- (230.5000,200.0000) -- (230.6000,200.0000) -- (230.7000,200.0000) -- (230.7000,200.0000) -- (230.8000,200.0000) -- (230.8000,200.0000) -- (230.9000,200.0000) -- (231.0000,200.0000) -- (231.0000,200.0000) -- (231.1000,200.0000) -- (231.2000,200.0000) -- (231.2000,200.0000) -- (231.3000,200.0000) -- (231.3000,200.0000) -- (231.4000,200.0000) -- (231.5000,200.0000) -- (231.5000,200.0000) -- (231.6000,200.0000) -- (231.6000,200.0000) -- (231.7000,200.0000) -- (231.8000,200.0000) -- (231.8000,200.0000) -- (231.9000,200.0000) -- (232.0000,200.0000) -- (232.0000,200.0000) -- (232.1000,200.0000) -- (232.1000,200.0000) -- (232.2000,200.0000) -- (232.3000,200.0000) -- (232.3000,200.0000) -- (232.4000,200.0000) -- (232.4000,200.0000) -- (232.5000,200.0000) -- (232.6000,200.0000) -- (232.6000,200.0000) -- (232.7000,200.0000) -- (232.8000,200.0000) -- (232.8000,200.0000) -- (232.9000,200.0000) -- (232.9000,200.0000) -- (233.0000,200.0000) -- (233.1000,200.0000) -- (233.1000,200.0000) -- (233.2000,200.0000) -- (233.2000,200.0000) -- (233.3000,200.0000) -- (233.4000,200.0000) -- (233.4000,200.0000) -- (233.5000,200.0000) -- (233.6000,200.0000) -- (233.6000,200.0000) -- (233.7000,200.0000) -- (233.7000,200.0000) -- (233.8000,200.0000) -- (233.9000,200.0000) -- (233.9000,200.0000) -- (234.0000,200.0000) -- (234.0000,200.0000) -- (234.1000,200.0000) -- (234.2000,200.0000) -- (234.2000,200.0000) -- (234.3000,200.0000) -- (234.4000,200.0000) -- (234.4000,200.0000) -- (234.5000,200.0000) -- (234.5000,200.0000) -- (234.6000,200.0000) -- (234.7000,200.0000) -- (234.7000,200.0000) -- (234.8000,200.0000) -- (234.8000,200.0000) -- (234.9000,200.0000) -- (235.0000,200.0000) -- (235.0000,200.0000) -- (235.1000,200.0000) -- (235.2000,200.0000) -- (235.2000,200.0000) -- (235.3000,200.0000) -- (235.3000,200.0000) -- (235.4000,200.0000) -- (235.5000,200.0000) -- (235.5000,200.0000) -- (235.6000,200.0000) -- (235.6000,200.0000) -- (235.7000,200.0000) -- (235.8000,200.0000) -- (235.8000,200.0000) -- (235.9000,200.0000) -- (236.0000,200.0000) -- (236.0000,200.0000) -- (236.1000,200.0000) -- (236.1000,200.0000) -- (236.2000,200.0000) -- (236.3000,200.0000) -- (236.3000,200.0000) -- (236.4000,200.0000) -- (236.4000,200.0000) -- (236.5000,200.0000) -- (236.6000,200.0000) -- (236.6000,200.0000) -- (236.7000,200.0000) -- (236.8000,200.0000) -- (236.8000,200.0000) -- (236.9000,200.0000) -- (236.9000,200.0000) -- (237.0000,200.0000) -- (237.1000,200.0000) -- (237.1000,200.0000) -- (237.2000,200.0000) -- (237.2000,200.0000) -- (237.3000,200.0000) -- (237.4000,200.0000) -- (237.4000,200.0000) -- (237.5000,200.0000) -- (237.6000,200.0000) -- (237.6000,200.0000) -- (237.7000,200.0000) -- (237.7000,200.0000) -- (237.8000,200.0000) -- (237.9000,200.0000) -- (237.9000,200.0000) -- (238.0000,200.0000) -- (238.0000,200.0000) -- (238.1000,200.0000) -- (238.2000,200.0000) -- (238.2000,200.0000) -- (238.3000,200.0000) -- (238.4000,200.0000) -- (238.4000,200.0000) -- (238.5000,200.0000) -- (238.5000,200.0000) -- (238.6000,200.0000) -- (238.7000,200.0000) -- (238.7000,200.0000) -- (238.8000,200.0000) -- (238.8000,200.0000) -- (238.9000,200.0000) -- (239.0000,200.0000) -- (239.0000,200.0000) -- (239.1000,200.0000) -- (239.2000,200.0000) -- (239.2000,200.0000) -- (239.3000,200.0000) -- (239.3000,200.0000) -- (239.4000,200.0000) -- (239.5000,200.0000) -- (239.5000,200.0000) -- (239.6000,200.0000) -- (239.6000,200.0000) -- (239.7000,200.0000) -- (239.8000,200.0000) -- (239.8000,200.0000) -- (239.9000,200.0000) -- (240.0000,200.0000) -- (240.0000,200.0000) -- (240.1000,200.0000) -- (240.1000,200.0000) -- (240.2000,200.0000) -- (240.3000,200.0000) -- (240.3000,200.0000) -- (240.4000,200.0000) -- (240.4000,200.0000) -- (240.5000,200.0000) -- (240.6000,200.0000) -- (240.6000,200.0000) -- (240.7000,200.0000) -- (240.8000,200.0000) -- (240.8000,200.0000) -- (240.9000,200.0000) -- (240.9000,200.0000) -- (241.0000,200.0000) -- (241.1000,200.0000) -- (241.1000,200.0000) -- (241.2000,200.0000) -- (241.2000,200.0000) -- (241.3000,200.0000) -- (241.4000,200.0000) -- (241.4000,200.0000) -- (241.5000,200.0000) -- (241.6000,200.0000) -- (241.6000,200.0000) -- (241.7000,200.0000) -- (241.7000,200.0000) -- (241.8000,200.0000) -- (241.9000,200.0000) -- (241.9000,200.0000) -- (242.0000,200.0000) -- (242.0000,200.0000) -- (242.1000,200.0000) -- (242.2000,200.0000) -- (242.2000,200.0000) -- (242.3000,200.0000) -- (242.4000,200.0000) -- (242.4000,200.0000) -- (242.5000,200.0000) -- (242.5000,200.0000) -- (242.6000,200.0000) -- (242.7000,200.0000) -- (242.7000,200.0000) -- (242.8000,200.0000) -- (242.8000,200.0000) -- (242.9000,200.0000) -- (243.0000,200.0000) -- (243.0000,200.0000) -- (243.1000,200.0000) -- (243.2000,200.0000) -- (243.2000,200.0000) -- (243.3000,200.0000) -- (243.3000,200.0000) -- (243.4000,200.0000) -- (243.5000,200.0000) -- (243.5000,200.0000) -- (243.6000,200.0000) -- (243.6000,200.0000) -- (243.7000,200.0000) -- (243.8000,200.0000) -- (243.8000,200.0000) -- (243.9000,200.0000) -- (244.0000,200.0000) -- (244.0000,200.0000) -- (244.1000,200.0000) -- (244.1000,200.0000) -- (244.2000,200.0000) -- (244.3000,200.0000) -- (244.3000,200.0000) -- (244.4000,200.0000) -- (244.4000,200.0000) -- (244.5000,200.0000) -- (244.6000,200.0000) -- (244.6000,200.0000) -- (244.7000,200.0000) -- (244.8000,200.0000) -- (244.8000,200.0000) -- (244.9000,200.0000) -- (244.9000,200.0000) -- (245.0000,200.0000) -- (245.1000,200.0000) -- (245.1000,200.0000) -- (245.2000,200.0000) -- (245.2000,200.0000) -- (245.3000,200.0000) -- (245.4000,200.0000) -- (245.4000,200.0000) -- (245.5000,200.0000) -- (245.6000,200.0000) -- (245.6000,200.0000) -- (245.7000,200.0000) -- (245.7000,200.0000) -- (245.8000,200.0000) -- (245.9000,200.0000) -- (245.9000,200.0000) -- (246.0000,200.0000) -- (246.0000,200.0000) -- (246.1000,200.0000) -- (246.2000,200.0000) -- (246.2000,200.0000) -- (246.3000,200.0000);



  \end{scope}
  \begin{scope}[cm={{0.92047,0.0,0.0,0.92047,(-569.12952,52.34945)}},draw=blue,line cap=round,line join=round,line width=0.480pt]
    \path[draw] (184.5000,156.5000) -- (184.5000,204.5000) -- (246.5000,204.5000) -- (246.5000,156.5000) -- (184.5000,156.5000);



  \end{scope}
  \begin{scope}[cm={{0.92743,0.0,0.0,0.92743,(-570.4103,65.15216)}},draw=ca0a0a4,dash pattern=on 2.59pt off 2.59pt,line cap=round,line join=round,line width=0.323pt,miter limit=4.00]
    \path[draw,dash pattern=on 2.59pt off 2.59pt,line width=0.323pt,miter limit=4.00] (184.5000,267.5000) -- (246.5000,267.5000);



  \end{scope}
  \begin{scope}[cm={{0.92743,0.0,0.0,0.92743,(-570.4103,65.15216)}},draw=blue,line cap=round,line join=round,line width=0.480pt]
    \path[draw] (184.5000,267.5000) -- (186.5000,267.5000);



    \path[draw] (246.5000,267.5000) -- (245.5000,267.5000);



  \end{scope}
  \begin{scope}[cm={{0.92743,0.0,0.0,0.92743,(-570.4103,65.15216)}},draw=ca0a0a4,dash pattern=on 2.59pt off 2.59pt,line cap=round,line join=round,line width=0.323pt,miter limit=4.00]
    \path[draw,dash pattern=on 2.59pt off 2.59pt,line width=0.323pt,miter limit=4.00] (184.5000,248.5000) -- (246.5000,248.5000);



  \end{scope}
  \begin{scope}[cm={{0.92743,0.0,0.0,0.92743,(-570.4103,65.15216)}},draw=blue,line cap=round,line join=round,line width=0.480pt]
    \path[draw] (184.5000,248.5000) -- (186.5000,248.5000);



    \path[draw] (246.5000,248.5000) -- (245.5000,248.5000);



  \end{scope}
  \begin{scope}[cm={{0.92743,0.0,0.0,0.92743,(-570.4103,65.15216)}},draw=ca0a0a4,dash pattern=on 2.59pt off 2.59pt,line cap=round,line join=round,line width=0.323pt,miter limit=4.00]
    \path[draw,dash pattern=on 2.59pt off 2.59pt,line width=0.323pt,miter limit=4.00] (184.5000,229.5000) -- (246.5000,229.5000);



  \end{scope}
  \begin{scope}[cm={{0.92743,0.0,0.0,0.92743,(-570.4103,65.15216)}},draw=blue,line cap=round,line join=round,line width=0.480pt]
    \path[draw] (184.5000,229.5000) -- (186.5000,229.5000);



    \path[draw] (246.5000,229.5000) -- (245.5000,229.5000);



  \end{scope}
  \begin{scope}[cm={{0.92743,0.0,0.0,0.92743,(-570.4103,65.15216)}},draw=ca0a0a4,dash pattern=on 0.40pt off 0.80pt,line cap=round,line join=round,line width=0.400pt]
    \path[draw] (184.5000,272.5000) -- (184.5000,224.5000);



  \end{scope}
  \begin{scope}[cm={{0.92743,0.0,0.0,0.92743,(-570.4103,65.15216)}},draw=blue,line cap=round,line join=round,line width=0.480pt]
    \path[draw] (184.5000,272.5000) -- (184.5000,270.5000);



    \path[draw] (184.5000,224.5000) -- (184.5000,225.5000);



  \end{scope}
  \begin{scope}[cm={{0.92743,0.0,0.0,0.92743,(-570.4103,65.15216)}},draw=ca0a0a4,dash pattern=on 2.59pt off 2.59pt,line cap=round,line join=round,line width=0.323pt,miter limit=4.00]
    \path[draw,dash pattern=on 2.59pt off 2.59pt,line width=0.323pt,miter limit=4.00] (201.5000,272.5000) -- (201.5000,224.5000);



  \end{scope}
  \begin{scope}[cm={{0.92743,0.0,0.0,0.92743,(-570.4103,65.15216)}},draw=blue,line cap=round,line join=round,line width=0.480pt]
    \path[draw] (201.5000,272.5000) -- (201.5000,270.5000);



    \path[draw] (201.5000,224.5000) -- (201.5000,225.5000);



  \end{scope}
  \begin{scope}[cm={{0.92743,0.0,0.0,0.92743,(-570.4103,65.15216)}},draw=ca0a0a4,dash pattern=on 2.59pt off 2.59pt,line cap=round,line join=round,line width=0.323pt,miter limit=4.00]
    \path[draw,dash pattern=on 2.59pt off 2.59pt,line width=0.323pt,miter limit=4.00] (218.5000,272.5000) -- (218.5000,224.5000);



  \end{scope}
  \begin{scope}[cm={{0.92743,0.0,0.0,0.92743,(-570.4103,65.15216)}},draw=blue,line cap=round,line join=round,line width=0.480pt]
    \path[draw] (218.5000,272.5000) -- (218.5000,270.5000);



    \path[draw] (218.5000,224.5000) -- (218.5000,225.5000);



  \end{scope}
  \begin{scope}[cm={{0.92743,0.0,0.0,0.92743,(-570.4103,65.15216)}},draw=ca0a0a4,dash pattern=on 2.59pt off 2.59pt,line cap=round,line join=round,line width=0.323pt,miter limit=4.00]
    \path[draw,dash pattern=on 2.59pt off 2.59pt,line width=0.323pt,miter limit=4.00] (235.5000,272.5000) -- (235.5000,224.5000);



  \end{scope}
  \begin{scope}[cm={{0.92743,0.0,0.0,0.92743,(-570.4103,65.15216)}},draw=blue,line cap=round,line join=round,line width=0.480pt]
    \path[draw] (235.5000,272.5000) -- (235.5000,270.5000);



    \path[draw] (235.5000,224.5000) -- (235.5000,225.5000);



  \end{scope}
  \begin{scope}[cm={{0.92743,0.0,0.0,0.92743,(-570.4103,65.15216)}},draw=blue,line cap=round,line join=round,line width=0.480pt]
    \path[draw] (246.5000,267.5000) -- (245.5000,267.5000);



  \end{scope}
  \begin{scope}[cm={{1.00588,0.0,0.0,1.00588,(-407.95798,313.77387)}},draw=blue,line cap=rect,line join=bevel,line width=0.800pt]
    \path[fill=blue] (0.0000,0.0000) node[above right] (text2120) {\scriptsize 2};



  \end{scope}
  \begin{scope}[cm={{0.92743,0.0,0.0,0.92743,(-570.4103,65.15216)}},draw=blue,line cap=round,line join=round,line width=0.480pt]
    \path[draw] (246.5000,248.5000) -- (245.5000,248.5000);



  \end{scope}
  \begin{scope}[cm={{1.00588,0.0,0.0,1.00588,(-407.90165,297.66187)}},draw=blue,line cap=rect,line join=bevel,line width=0.800pt]
    \path[fill=blue] (0.0000,0.0000) node[above right] (text2144) {\scriptsize 6};



  \end{scope}
  \begin{scope}[cm={{0.92743,0.0,0.0,0.92743,(-570.4103,65.15216)}},draw=blue,line cap=round,line join=round,line width=0.480pt]
    \path[draw] (246.5000,229.5000) -- (245.5000,229.5000);



  \end{scope}
  \begin{scope}[cm={{1.00588,0.0,0.0,1.00588,(-411.98953,280.05087)}},draw=blue,line cap=rect,line join=bevel,line width=0.800pt]
    \path[fill=blue] (0.0000,0.0000) node[above right] (text2168) {\scriptsize 10};



  \end{scope}
  \begin{scope}[cm={{0.92743,0.0,0.0,0.92743,(-570.4103,65.15216)}},draw=blue,line cap=round,line join=round,line width=0.480pt]
    \path[draw] (184.5000,224.5000) -- (184.5000,272.5000) -- (246.5000,272.5000) -- (246.5000,224.5000) -- (184.5000,224.5000);



  \end{scope}
  \begin{scope}[cm={{0.0,-1.00588,1.00588,0.0,(-419.24335,303.67087)}},draw=blue,line cap=rect,line join=bevel,line width=0.800pt]
    \path[fill=blue] (0.0000,-5.9329) node[above right] (text2192) {\rotatebox{90}{\scriptsize $c_{i,2}$}};



  \end{scope}
  \begin{scope}[cm={{0.92743,0.0,0.0,0.92743,(-570.4103,65.15216)}},draw=blue,line cap=round,line join=round,line width=0.480pt]
    \path[draw] (184.8000,267.2000) -- (184.8000,267.2000) -- (184.9000,267.2000) -- (185.0000,267.2000) -- (185.0000,267.2000) -- (185.1000,267.2000) -- (185.1000,267.2000) -- (185.2000,267.2000) -- (185.3000,267.2000) -- (185.3000,267.2000) -- (185.4000,267.2000) -- (185.4000,267.2000) -- (185.5000,267.2000) -- (185.6000,267.2000) -- (185.6000,267.2000) -- (185.7000,267.2000) -- (185.8000,267.2000) -- (185.8000,267.2000) -- (185.9000,267.2000) -- (185.9000,267.2000) -- (186.0000,267.2000) -- (186.1000,267.2000) -- (186.1000,267.2000) -- (186.2000,267.2000) -- (186.2000,267.2000) -- (186.3000,267.2000) -- (186.4000,267.2000) -- (186.4000,267.2000) -- (186.5000,267.2000) -- (186.6000,267.2000) -- (186.6000,267.2000) -- (186.7000,267.2000) -- (186.7000,267.2000) -- (186.8000,267.2000) -- (186.9000,267.2000) -- (186.9000,267.2000) -- (187.0000,267.2000) -- (187.0000,267.2000) -- (187.1000,267.2000) -- (187.2000,267.2000) -- (187.2000,267.2000) -- (187.3000,267.2000) -- (187.4000,267.2000) -- (187.4000,267.2000) -- (187.5000,267.2000) -- (187.5000,267.2000) -- (187.6000,267.2000) -- (187.7000,267.2000) -- (187.7000,267.2000) -- (187.8000,267.2000) -- (187.8000,267.2000) -- (187.9000,267.2000) -- (188.0000,267.2000) -- (188.0000,267.2000) -- (188.1000,267.2000) -- (188.2000,267.2000) -- (188.2000,267.2000) -- (188.3000,267.2000) -- (188.3000,267.2000) -- (188.4000,267.2000) -- (188.5000,267.2000) -- (188.5000,267.2000) -- (188.6000,267.2000) -- (188.6000,267.2000) -- (188.7000,267.2000) -- (188.8000,267.2000) -- (188.8000,267.2000) -- (188.9000,267.2000) -- (189.0000,267.2000) -- (189.0000,267.2000) -- (189.1000,267.2000) -- (189.1000,267.2000) -- (189.2000,267.2000) -- (189.3000,267.2000) -- (189.3000,267.2000) -- (189.4000,267.2000) -- (189.4000,267.2000) -- (189.5000,267.2000) -- (189.6000,267.2000) -- (189.6000,267.2000) -- (189.7000,267.2000) -- (189.8000,267.2000) -- (189.8000,267.2000) -- (189.9000,267.2000) -- (189.9000,267.2000) -- (190.0000,267.2000) -- (190.1000,267.2000) -- (190.1000,267.2000) -- (190.2000,267.2000) -- (190.2000,267.2000) -- (190.3000,267.2000) -- (190.4000,267.2000) -- (190.4000,267.2000) -- (190.5000,267.2000) -- (190.5000,267.2000) -- (190.6000,267.2000) -- (190.7000,267.2000) -- (190.7000,267.2000) -- (190.8000,267.2000) -- (190.9000,267.2000) -- (190.9000,267.2000) -- (191.0000,267.2000) -- (191.0000,267.2000) -- (191.1000,267.2000) -- (191.2000,267.2000) -- (191.2000,267.2000) -- (191.3000,267.2000) -- (191.3000,267.2000) -- (191.4000,267.2000) -- (191.5000,267.2000) -- (191.5000,267.2000) -- (191.6000,267.2000) -- (191.7000,267.2000) -- (191.7000,267.2000) -- (191.8000,267.2000) -- (191.8000,267.2000) -- (191.9000,267.2000) -- (192.0000,267.2000) -- (192.0000,267.2000) -- (192.1000,267.2000) -- (192.1000,267.2000) -- (192.2000,267.2000) -- (192.3000,267.2000) -- (192.3000,267.2000) -- (192.4000,267.2000) -- (192.5000,267.2000) -- (192.5000,267.2000) -- (192.6000,267.3000) -- (192.6000,268.7000) -- (192.7000,244.7000) -- (192.8000,227.7000) -- (192.8000,229.2000) -- (192.9000,229.2000) -- (192.9000,229.2000) -- (193.0000,229.2000) -- (193.1000,229.2000) -- (193.1000,229.2000) -- (193.2000,229.2000) -- (193.3000,229.2000) -- (193.3000,229.2000) -- (193.4000,229.2000) -- (193.4000,229.2000) -- (193.5000,229.2000) -- (193.6000,229.2000) -- (193.6000,229.2000) -- (193.7000,229.2000) -- (193.7000,229.2000) -- (193.8000,229.2000) -- (193.9000,229.2000) -- (193.9000,229.2000) -- (194.0000,229.2000) -- (194.1000,229.2000) -- (194.1000,229.2000) -- (194.2000,229.2000) -- (194.2000,229.2000) -- (194.3000,229.2000) -- (194.4000,229.2000) -- (194.4000,229.2000) -- (194.5000,229.2000) -- (194.5000,229.2000) -- (194.6000,229.2000) -- (194.7000,229.2000) -- (194.7000,229.2000) -- (194.8000,229.2000) -- (194.9000,229.2000) -- (194.9000,229.2000) -- (195.0000,229.2000) -- (195.0000,229.2000) -- (195.1000,229.2000) -- (195.2000,229.2000) -- (195.2000,229.2000) -- (195.3000,229.2000) -- (195.3000,229.2000) -- (195.4000,229.2000) -- (195.5000,229.2000) -- (195.5000,229.2000) -- (195.6000,229.2000) -- (195.7000,229.2000) -- (195.7000,229.2000) -- (195.8000,229.2000) -- (195.8000,229.2000) -- (195.9000,229.2000) -- (196.0000,229.2000) -- (196.0000,229.2000) -- (196.1000,229.2000) -- (196.1000,229.2000) -- (196.2000,229.2000) -- (196.3000,229.2000) -- (196.3000,229.2000) -- (196.4000,229.2000) -- (196.5000,229.2000) -- (196.5000,229.2000) -- (196.6000,229.2000) -- (196.6000,229.2000) -- (196.7000,229.2000) -- (196.8000,229.2000) -- (196.8000,229.2000) -- (196.9000,229.2000) -- (196.9000,229.2000) -- (197.0000,229.2000) -- (197.1000,229.2000) -- (197.1000,229.2000) -- (197.2000,229.2000) -- (197.3000,229.2000) -- (197.3000,229.2000) -- (197.4000,229.2000) -- (197.4000,229.2000) -- (197.5000,229.2000) -- (197.6000,229.2000) -- (197.6000,229.2000) -- (197.7000,229.2000) -- (197.7000,229.2000) -- (197.8000,229.2000) -- (197.9000,229.2000) -- (197.9000,229.2000) -- (198.0000,229.2000) -- (198.1000,229.2000) -- (198.1000,229.2000) -- (198.2000,229.2000) -- (198.2000,229.2000) -- (198.3000,229.2000) -- (198.4000,229.2000) -- (198.4000,229.2000) -- (198.5000,229.2000) -- (198.5000,229.2000) -- (198.6000,229.2000) -- (198.7000,229.2000) -- (198.7000,229.2000) -- (198.8000,229.2000) -- (198.9000,229.2000) -- (198.9000,229.2000) -- (199.0000,229.2000) -- (199.0000,229.2000) -- (199.1000,229.2000) -- (199.2000,229.2000) -- (199.2000,229.2000) -- (199.3000,229.2000) -- (199.3000,229.2000) -- (199.4000,229.2000) -- (199.5000,229.2000) -- (199.5000,229.2000) -- (199.6000,229.2000) -- (199.7000,229.2000) -- (199.7000,229.2000) -- (199.8000,229.2000) -- (199.8000,229.2000) -- (199.9000,229.2000) -- (200.0000,229.2000) -- (200.0000,229.2000) -- (200.1000,229.2000) -- (200.1000,229.2000) -- (200.2000,229.2000) -- (200.3000,229.2000) -- (200.3000,229.2000) -- (200.4000,229.2000) -- (200.5000,229.2000) -- (200.5000,229.2000) -- (200.6000,229.2000) -- (200.6000,229.2000) -- (200.7000,229.2000) -- (200.8000,229.2000) -- (200.8000,229.2000) -- (200.9000,229.2000) -- (200.9000,229.2000) -- (201.0000,229.2000) -- (201.1000,229.2000) -- (201.1000,229.2000) -- (201.2000,229.2000) -- (201.3000,229.2000) -- (201.3000,229.2000) -- (201.4000,229.2000) -- (201.4000,229.2000) -- (201.5000,229.2000) -- (201.6000,229.2000) -- (201.6000,229.2000) -- (201.7000,229.2000) -- (201.7000,229.2000) -- (201.8000,229.2000) -- (201.9000,229.2000) -- (201.9000,229.2000) -- (202.0000,229.2000) -- (202.0000,229.2000) -- (202.1000,229.2000) -- (202.2000,229.2000) -- (202.2000,229.2000) -- (202.3000,229.2000) -- (202.4000,229.2000) -- (202.4000,229.2000) -- (202.5000,229.2000) -- (202.5000,229.2000) -- (202.6000,229.2000) -- (202.7000,229.2000) -- (202.7000,229.2000) -- (202.8000,229.2000) -- (202.8000,229.2000) -- (202.9000,229.2000) -- (203.0000,229.2000) -- (203.0000,229.2000) -- (203.1000,229.2000) -- (203.2000,229.2000) -- (203.2000,229.2000) -- (203.3000,229.2000) -- (203.3000,229.2000) -- (203.4000,229.2000) -- (203.5000,229.2000) -- (203.5000,229.2000) -- (203.6000,229.2000) -- (203.6000,229.2000) -- (203.7000,229.2000) -- (203.8000,229.2000) -- (203.8000,229.2000) -- (203.9000,229.2000) -- (204.0000,229.2000) -- (204.0000,229.2000) -- (204.1000,229.2000) -- (204.1000,229.2000) -- (204.2000,229.2000) -- (204.3000,229.2000) -- (204.3000,229.2000) -- (204.4000,229.2000) -- (204.4000,229.2000) -- (204.5000,229.2000) -- (204.6000,229.2000) -- (204.6000,229.2000) -- (204.7000,229.2000) -- (204.8000,229.2000) -- (204.8000,229.2000) -- (204.9000,229.2000) -- (204.9000,229.2000) -- (205.0000,229.2000) -- (205.1000,229.2000) -- (205.1000,229.2000) -- (205.2000,229.2000) -- (205.2000,229.2000) -- (205.3000,229.2000) -- (205.4000,229.2000) -- (205.4000,229.2000) -- (205.5000,229.2000) -- (205.6000,229.2000) -- (205.6000,229.2000) -- (205.7000,229.2000) -- (205.7000,229.2000) -- (205.8000,229.2000) -- (205.9000,229.2000) -- (205.9000,229.2000) -- (206.0000,229.2000) -- (206.0000,229.2000) -- (206.1000,229.2000) -- (206.2000,229.2000) -- (206.2000,229.2000) -- (206.3000,229.2000) -- (206.4000,229.2000) -- (206.4000,229.2000) -- (206.5000,229.2000) -- (206.5000,229.2000) -- (206.6000,229.2000) -- (206.7000,229.2000) -- (206.7000,229.2000) -- (206.8000,229.2000) -- (206.8000,229.2000) -- (206.9000,229.2000) -- (207.0000,229.2000) -- (207.0000,229.2000) -- (207.1000,229.2000) -- (207.2000,229.2000) -- (207.2000,229.2000) -- (207.3000,229.2000) -- (207.3000,229.2000) -- (207.4000,229.2000) -- (207.5000,229.2000) -- (207.5000,229.2000) -- (207.6000,229.2000) -- (207.6000,229.2000) -- (207.7000,229.2000) -- (207.8000,229.2000) -- (207.8000,229.2000) -- (207.9000,229.2000) -- (208.0000,229.2000) -- (208.0000,229.2000) -- (208.1000,229.2000) -- (208.1000,229.2000) -- (208.2000,229.2000) -- (208.3000,229.2000) -- (208.3000,229.2000) -- (208.4000,229.2000) -- (208.4000,229.2000) -- (208.5000,229.2000) -- (208.6000,229.2000) -- (208.6000,229.2000) -- (208.7000,229.2000) -- (208.8000,229.2000) -- (208.8000,229.2000) -- (208.9000,229.2000) -- (208.9000,229.2000) -- (209.0000,229.2000) -- (209.1000,229.2000) -- (209.1000,229.2000) -- (209.2000,229.2000) -- (209.2000,229.2000) -- (209.3000,229.2000) -- (209.4000,229.2000) -- (209.4000,229.2000) -- (209.5000,229.2000) -- (209.6000,229.2000) -- (209.6000,229.2000) -- (209.7000,229.2000) -- (209.7000,229.2000) -- (209.8000,229.2000) -- (209.9000,229.2000) -- (209.9000,229.2000) -- (210.0000,229.2000) -- (210.0000,229.2000) -- (210.1000,229.2000) -- (210.2000,229.2000) -- (210.2000,229.2000) -- (210.3000,229.2000) -- (210.4000,229.2000) -- (210.4000,229.2000) -- (210.5000,229.2000) -- (210.5000,229.2000) -- (210.6000,229.2000) -- (210.7000,229.2000) -- (210.7000,229.2000) -- (210.8000,229.2000) -- (210.8000,229.2000) -- (210.9000,229.2000) -- (211.0000,229.2000) -- (211.0000,229.2000) -- (211.1000,229.2000) -- (211.2000,229.2000) -- (211.2000,229.2000) -- (211.3000,229.2000) -- (211.3000,229.2000) -- (211.4000,229.2000) -- (211.5000,229.2000) -- (211.5000,229.2000) -- (211.6000,229.2000) -- (211.6000,229.2000) -- (211.7000,229.2000) -- (211.8000,229.2000) -- (211.8000,229.2000) -- (211.9000,229.2000) -- (212.0000,229.2000) -- (212.0000,229.2000) -- (212.1000,229.2000) -- (212.1000,229.2000) -- (212.2000,229.2000) -- (212.3000,229.2000) -- (212.3000,229.2000) -- (212.4000,229.2000) -- (212.4000,229.2000) -- (212.5000,229.2000) -- (212.6000,229.2000) -- (212.6000,229.2000) -- (212.7000,229.2000) -- (212.8000,229.2000) -- (212.8000,229.2000) -- (212.9000,229.2000) -- (212.9000,229.2000) -- (213.0000,229.2000) -- (213.1000,229.2000) -- (213.1000,229.2000) -- (213.2000,229.2000) -- (213.2000,229.2000) -- (213.3000,229.2000) -- (213.4000,229.2000) -- (213.4000,229.2000) -- (213.5000,229.2000) -- (213.6000,229.2000) -- (213.6000,229.2000) -- (213.7000,229.2000) -- (213.7000,229.2000) -- (213.8000,229.2000) -- (213.9000,229.2000) -- (213.9000,229.2000) -- (214.0000,229.2000) -- (214.0000,229.2000) -- (214.1000,229.2000) -- (214.2000,229.2000) -- (214.2000,229.2000) -- (214.3000,229.2000) -- (214.4000,229.2000) -- (214.4000,229.2000) -- (214.5000,229.2000) -- (214.5000,229.2000) -- (214.6000,229.2000) -- (214.7000,229.2000) -- (214.7000,229.2000) -- (214.8000,229.2000) -- (214.8000,229.2000) -- (214.9000,229.2000) -- (215.0000,229.2000) -- (215.0000,229.2000) -- (215.1000,229.2000) -- (215.1000,229.2000) -- (215.2000,229.2000) -- (215.3000,229.2000) -- (215.3000,229.2000) -- (215.4000,229.2000) -- (215.5000,229.2000) -- (215.5000,229.2000) -- (215.6000,229.2000) -- (215.6000,229.2000) -- (215.7000,229.2000) -- (215.8000,229.2000) -- (215.8000,229.2000) -- (215.9000,229.2000) -- (215.9000,228.9000) -- (216.0000,233.3000) -- (216.1000,239.1000) -- (216.1000,238.8000) -- (216.2000,238.7000) -- (216.3000,238.7000) -- (216.3000,238.7000) -- (216.4000,238.7000) -- (216.4000,238.7000) -- (216.5000,238.7000) -- (216.6000,238.7000) -- (216.6000,238.7000) -- (216.7000,238.7000) -- (216.7000,238.7000) -- (216.8000,238.7000) -- (216.9000,238.7000) -- (216.9000,238.7000) -- (217.0000,238.7000) -- (217.1000,238.7000) -- (217.1000,238.7000) -- (217.2000,238.7000) -- (217.2000,238.7000) -- (217.3000,238.7000) -- (217.4000,238.7000) -- (217.4000,238.7000) -- (217.5000,238.7000) -- (217.5000,238.7000) -- (217.6000,239.0000) -- (217.7000,236.4000) -- (217.7000,229.1000) -- (217.8000,229.2000) -- (217.9000,229.2000) -- (217.9000,229.2000) -- (218.0000,229.2000) -- (218.0000,229.2000) -- (218.1000,229.2000) -- (218.2000,229.2000) -- (218.2000,229.2000) -- (218.3000,229.2000) -- (218.3000,229.2000) -- (218.4000,229.2000) -- (218.5000,229.2000) -- (218.5000,229.2000) -- (218.6000,229.2000) -- (218.7000,229.2000) -- (218.7000,229.2000) -- (218.8000,229.2000) -- (218.8000,229.2000) -- (218.9000,229.2000) -- (219.0000,229.2000) -- (219.0000,229.2000) -- (219.1000,229.2000) -- (219.1000,229.2000) -- (219.2000,229.2000) -- (219.3000,229.2000) -- (219.3000,229.2000) -- (219.4000,229.2000) -- (219.5000,229.2000) -- (219.5000,229.2000) -- (219.6000,229.2000) -- (219.6000,229.2000) -- (219.7000,229.2000) -- (219.8000,229.2000) -- (219.8000,229.2000) -- (219.9000,229.2000) -- (219.9000,229.2000) -- (220.0000,229.2000) -- (220.1000,229.2000) -- (220.1000,228.9000) -- (220.2000,233.5000) -- (220.3000,239.1000) -- (220.3000,238.8000) -- (220.4000,238.7000) -- (220.4000,239.1000) -- (220.5000,247.0000) -- (220.6000,248.4000) -- (220.6000,248.2000) -- (220.7000,248.2000) -- (220.7000,248.2000) -- (220.8000,248.2000) -- (220.9000,248.2000) -- (220.9000,248.2000) -- (221.0000,248.2000) -- (221.1000,248.2000) -- (221.1000,248.2000) -- (221.2000,248.2000) -- (221.2000,248.2000) -- (221.3000,248.2000) -- (221.4000,248.2000) -- (221.4000,248.2000) -- (221.5000,248.2000) -- (221.5000,248.2000) -- (221.6000,248.2000) -- (221.7000,248.2000) -- (221.7000,248.2000) -- (221.8000,248.2000) -- (221.9000,248.2000) -- (221.9000,248.2000) -- (222.0000,248.2000) -- (222.0000,248.3000) -- (222.1000,248.5000) -- (222.2000,241.5000) -- (222.2000,238.4000) -- (222.3000,238.7000) -- (222.3000,239.0000) -- (222.4000,236.7000) -- (222.5000,229.2000) -- (222.5000,229.2000) -- (222.6000,229.2000) -- (222.7000,229.2000) -- (222.7000,229.2000) -- (222.8000,229.2000) -- (222.8000,229.2000) -- (222.9000,229.2000) -- (223.0000,229.2000) -- (223.0000,229.2000) -- (223.1000,229.2000) -- (223.1000,229.2000) -- (223.2000,229.2000) -- (223.3000,229.2000) -- (223.3000,229.2000) -- (223.4000,229.2000) -- (223.5000,229.2000) -- (223.5000,229.2000) -- (223.6000,229.2000) -- (223.6000,229.2000) -- (223.7000,229.2000) -- (223.8000,229.2000) -- (223.8000,229.2000) -- (223.9000,229.2000) -- (223.9000,229.2000) -- (224.0000,229.2000) -- (224.1000,229.2000) -- (224.1000,229.2000) -- (224.2000,229.2000) -- (224.3000,229.2000) -- (224.3000,229.2000) -- (224.4000,229.2000) -- (224.4000,229.8000) -- (224.5000,237.7000) -- (224.6000,238.8000) -- (224.6000,239.2000) -- (224.7000,247.1000) -- (224.7000,248.4000) -- (224.8000,248.2000) -- (224.9000,247.9000) -- (224.9000,252.1000) -- (225.0000,258.1000) -- (225.1000,257.8000) -- (225.1000,257.7000) -- (225.2000,257.7000) -- (225.2000,257.7000) -- (225.3000,257.7000) -- (225.4000,257.7000) -- (225.4000,257.7000) -- (225.5000,257.7000) -- (225.5000,257.7000) -- (225.6000,257.7000) -- (225.7000,257.7000) -- (225.7000,257.7000) -- (225.8000,257.7000) -- (225.9000,257.7000) -- (225.9000,257.7000) -- (226.0000,257.7000) -- (226.0000,257.7000) -- (226.1000,257.7000) -- (226.2000,257.7000) -- (226.2000,257.7000) -- (226.3000,257.7000) -- (226.3000,257.7000) -- (226.4000,257.7000) -- (226.5000,257.7000) -- (226.5000,258.0000) -- (226.6000,255.6000) -- (226.6000,248.2000) -- (226.7000,248.2000) -- (226.8000,248.3000) -- (226.8000,248.5000) -- (226.9000,241.8000) -- (227.0000,238.4000) -- (227.0000,239.1000) -- (227.1000,232.5000) -- (227.1000,228.9000) -- (227.2000,229.2000) -- (227.3000,229.2000) -- (227.3000,229.2000) -- (227.4000,229.2000) -- (227.4000,229.2000) -- (227.5000,229.2000) -- (227.6000,229.2000) -- (227.6000,229.2000) -- (227.7000,229.2000) -- (227.8000,229.2000) -- (227.8000,229.2000) -- (227.9000,229.2000) -- (227.9000,229.2000) -- (228.0000,229.2000) -- (228.1000,229.2000) -- (228.1000,229.2000) -- (228.2000,229.2000) -- (228.2000,229.2000) -- (228.3000,229.2000) -- (228.4000,229.2000) -- (228.4000,229.2000) -- (228.5000,229.2000) -- (228.6000,229.2000) -- (228.6000,229.2000) -- (228.7000,229.2000) -- (228.7000,229.2000) -- (228.8000,229.8000) -- (228.9000,237.3000) -- (228.9000,243.1000) -- (229.0000,248.6000) -- (229.0000,247.9000) -- (229.1000,252.3000) -- (229.2000,258.1000) -- (229.2000,257.8000) -- (229.3000,257.7000) -- (229.4000,258.0000) -- (229.4000,265.8000) -- (229.5000,267.4000) -- (229.5000,267.2000) -- (229.6000,267.2000) -- (229.7000,267.2000) -- (229.7000,267.2000) -- (229.8000,267.2000) -- (229.8000,267.2000) -- (229.9000,267.2000) -- (230.0000,267.2000) -- (230.0000,267.2000) -- (230.1000,267.2000) -- (230.2000,267.2000) -- (230.2000,267.2000) -- (230.3000,267.2000) -- (230.3000,267.2000) -- (230.4000,267.2000) -- (230.5000,267.2000) -- (230.5000,267.2000) -- (230.6000,267.2000) -- (230.6000,267.2000) -- (230.7000,267.2000) -- (230.8000,267.2000) -- (230.8000,267.2000) -- (230.9000,267.2000) -- (231.0000,267.3000) -- (231.0000,267.5000) -- (231.1000,260.7000) -- (231.1000,257.4000) -- (231.2000,257.7000) -- (231.3000,258.0000) -- (231.3000,255.9000) -- (231.4000,248.3000) -- (231.4000,248.4000) -- (231.5000,246.6000) -- (231.6000,238.9000) -- (231.6000,238.9000) -- (231.7000,237.2000) -- (231.8000,229.4000) -- (231.8000,229.2000) -- (231.9000,229.2000) -- (231.9000,229.2000) -- (232.0000,229.2000) -- (232.1000,229.2000) -- (232.1000,229.2000) -- (232.2000,229.2000) -- (232.2000,229.2000) -- (232.3000,229.2000) -- (232.4000,229.2000) -- (232.4000,229.2000) -- (232.5000,229.2000) -- (232.6000,229.2000) -- (232.6000,229.2000) -- (232.7000,229.2000) -- (232.7000,229.2000) -- (232.8000,229.2000) -- (232.9000,229.2000) -- (232.9000,229.2000) -- (233.0000,229.2000) -- (233.0000,229.2000) -- (233.1000,229.2000) -- (233.2000,229.8000) -- (233.2000,236.6000) -- (233.3000,251.6000) -- (233.4000,268.4000) -- (233.4000,267.3000) -- (233.5000,267.2000) -- (233.5000,267.2000) -- (233.6000,267.2000) -- (233.7000,267.2000) -- (233.7000,267.2000) -- (233.8000,267.2000) -- (233.8000,267.2000) -- (233.9000,267.2000) -- (234.0000,267.2000) -- (234.0000,267.2000) -- (234.1000,267.2000) -- (234.2000,267.2000) -- (234.2000,267.2000) -- (234.3000,267.2000) -- (234.3000,267.2000) -- (234.4000,267.2000) -- (234.5000,267.2000) -- (234.5000,267.2000) -- (234.6000,267.2000) -- (234.6000,267.2000) -- (234.7000,267.2000) -- (234.8000,267.2000) -- (234.8000,267.2000) -- (234.9000,267.2000) -- (235.0000,267.2000) -- (235.0000,267.2000) -- (235.1000,267.2000) -- (235.1000,267.2000) -- (235.2000,267.2000) -- (235.3000,267.2000) -- (235.3000,267.2000) -- (235.4000,267.2000) -- (235.4000,267.2000) -- (235.5000,267.2000) -- (235.6000,267.2000) -- (235.6000,267.2000) -- (235.7000,267.3000) -- (235.8000,267.6000) -- (235.8000,261.0000) -- (235.9000,257.4000) -- (235.9000,258.1000) -- (236.0000,251.7000) -- (236.1000,247.9000) -- (236.1000,248.6000) -- (236.2000,242.4000) -- (236.2000,238.4000) -- (236.3000,239.1000) -- (236.4000,233.1000) -- (236.4000,228.9000) -- (236.5000,229.2000) -- (236.6000,229.2000) -- (236.6000,229.2000) -- (236.7000,229.2000) -- (236.7000,229.2000) -- (236.8000,229.2000) -- (236.9000,229.2000) -- (236.9000,229.2000) -- (237.0000,229.2000) -- (237.0000,229.2000) -- (237.1000,229.1000) -- (237.2000,230.0000) -- (237.2000,238.0000) -- (237.3000,238.9000) -- (237.4000,238.7000) -- (237.4000,238.7000) -- (237.5000,238.7000) -- (237.5000,238.7000) -- (237.6000,237.6000) -- (237.7000,251.4000) -- (237.7000,268.4000) -- (237.8000,267.3000) -- (237.8000,267.2000) -- (237.9000,267.2000) -- (238.0000,267.2000) -- (238.0000,267.2000) -- (238.1000,267.2000) -- (238.1000,267.2000) -- (238.2000,267.2000) -- (238.3000,267.2000) -- (238.3000,267.2000) -- (238.4000,267.2000) -- (238.5000,267.2000) -- (238.5000,267.2000) -- (238.6000,267.2000) -- (238.6000,267.2000) -- (238.7000,267.2000) -- (238.8000,267.2000) -- (238.8000,267.2000) -- (238.9000,267.2000) -- (238.9000,267.2000) -- (239.0000,267.2000) -- (239.1000,267.2000) -- (239.1000,267.2000) -- (239.2000,267.2000) -- (239.3000,267.2000) -- (239.3000,267.2000) -- (239.4000,267.2000) -- (239.4000,267.2000) -- (239.5000,267.2000) -- (239.6000,267.2000) -- (239.6000,267.2000) -- (239.7000,267.2000) -- (239.7000,267.2000) -- (239.8000,267.2000) -- (239.9000,267.2000) -- (239.9000,267.2000) -- (240.0000,267.2000) -- (240.1000,267.2000) -- (240.1000,267.2000) -- (240.2000,267.2000) -- (240.2000,267.2000) -- (240.3000,267.2000) -- (240.4000,267.5000) -- (240.4000,265.7000) -- (240.5000,258.0000) -- (240.5000,257.9000) -- (240.6000,256.4000) -- (240.7000,248.5000) -- (240.7000,248.4000) -- (240.8000,247.0000) -- (240.9000,239.1000) -- (240.9000,238.7000) -- (241.0000,238.7000) -- (241.0000,238.7000) -- (241.1000,238.7000) -- (241.2000,238.7000) -- (241.2000,238.7000) -- (241.3000,238.7000) -- (241.3000,238.7000) -- (241.4000,238.7000) -- (241.5000,238.6000) -- (241.5000,239.5000) -- (241.6000,247.5000) -- (241.7000,248.4000) -- (241.7000,248.2000) -- (241.8000,248.2000) -- (241.8000,248.1000) -- (241.9000,249.3000) -- (242.0000,265.2000) -- (242.0000,267.6000) -- (242.1000,267.2000) -- (242.1000,267.2000) -- (242.2000,267.2000) -- (242.3000,267.2000) -- (242.3000,267.2000) -- (242.4000,267.2000) -- (242.5000,267.2000) -- (242.5000,267.2000) -- (242.6000,267.2000) -- (242.6000,267.2000) -- (242.7000,267.2000) -- (242.8000,267.2000) -- (242.8000,267.2000) -- (242.9000,267.2000) -- (242.9000,267.2000) -- (243.0000,267.2000) -- (243.1000,267.2000) -- (243.1000,267.2000) -- (243.2000,267.2000) -- (243.3000,267.2000) -- (243.3000,267.2000) -- (243.4000,267.2000) -- (243.4000,267.2000) -- (243.5000,267.2000) -- (243.6000,267.2000) -- (243.6000,267.2000) -- (243.7000,267.2000) -- (243.7000,267.2000) -- (243.8000,267.2000) -- (243.9000,267.2000) -- (243.9000,267.2000) -- (244.0000,267.2000) -- (244.1000,267.2000) -- (244.1000,267.2000) -- (244.2000,267.2000) -- (244.2000,267.2000) -- (244.3000,267.2000) -- (244.4000,267.2000) -- (244.4000,267.2000) -- (244.5000,267.2000) -- (244.5000,267.2000) -- (244.6000,267.2000) -- (244.7000,267.2000) -- (244.7000,267.2000) -- (244.8000,267.2000) -- (244.9000,267.2000) -- (244.9000,267.2000) -- (245.0000,267.3000) -- (245.0000,267.6000) -- (245.1000,261.6000) -- (245.2000,257.4000) -- (245.2000,258.1000) -- (245.3000,252.3000) -- (245.3000,247.9000) -- (245.4000,248.2000) -- (245.5000,248.2000) -- (245.5000,248.2000) -- (245.6000,248.2000) -- (245.7000,248.2000) -- (245.7000,248.2000) -- (245.8000,248.2000) -- (245.8000,248.2000) -- (245.9000,248.2000) -- (246.0000,247.9000) -- (246.0000,252.9000) -- (246.1000,258.1000) -- (246.1000,257.4000) -- (246.2000,262.2000) -- (246.3000,267.6000);



  \end{scope}
  \begin{scope}[cm={{0.92743,0.0,0.0,0.92743,(-570.4103,65.15216)}},draw=blue,line cap=round,line join=round,line width=0.480pt]
    \path[draw] (184.5000,224.5000) -- (184.5000,272.5000) -- (246.5000,272.5000) -- (246.5000,224.5000) -- (184.5000,224.5000);



  \end{scope}
  \begin{scope}[cm={{0.92743,0.0,0.0,0.92743,(-570.4103,65.15216)}},draw=ca0a0a4,dash pattern=on 2.59pt off 2.59pt,line cap=round,line join=round,line width=0.323pt,miter limit=4.00]
    \path[draw,dash pattern=on 2.59pt off 2.59pt,line width=0.323pt,miter limit=4.00] (184.5000,315.5000) -- (246.5000,315.5000);



  \end{scope}
  \begin{scope}[cm={{0.92743,0.0,0.0,0.92743,(-570.4103,65.15216)}},draw=blue,line cap=round,line join=round,line width=0.480pt]
    \path[draw] (184.5000,315.5000) -- (186.5000,315.5000);



    \path[draw] (246.5000,315.5000) -- (245.5000,315.5000);



  \end{scope}
  \begin{scope}[cm={{0.92743,0.0,0.0,0.92743,(-570.4103,65.15216)}},draw=ca0a0a4,dash pattern=on 2.59pt off 2.59pt,line cap=round,line join=round,line width=0.323pt,miter limit=4.00]
    \path[draw,dash pattern=on 2.59pt off 2.59pt,line width=0.323pt,miter limit=4.00] (184.5000,295.5000) -- (246.5000,295.5000);



  \end{scope}
  \begin{scope}[cm={{0.92743,0.0,0.0,0.92743,(-570.4103,65.15216)}},draw=blue,line cap=round,line join=round,line width=0.480pt]
    \path[draw] (184.5000,295.5000) -- (186.5000,295.5000);



    \path[draw] (246.5000,295.5000) -- (245.5000,295.5000);



  \end{scope}
  \begin{scope}[cm={{0.92743,0.0,0.0,0.92743,(-570.4103,65.15216)}},draw=ca0a0a4,dash pattern=on 2.59pt off 2.59pt,line cap=round,line join=round,line width=0.323pt,miter limit=4.00]
    \path[draw,dash pattern=on 2.59pt off 2.59pt,line width=0.323pt,miter limit=4.00] (184.5000,276.5000) -- (246.5000,276.5000);



  \end{scope}
  \begin{scope}[cm={{0.92743,0.0,0.0,0.92743,(-570.4103,65.15216)}},draw=blue,line cap=round,line join=round,line width=0.480pt]
    \path[draw] (184.5000,276.5000) -- (186.5000,276.5000);



    \path[draw] (246.5000,276.5000) -- (245.5000,276.5000);



  \end{scope}
  \begin{scope}[cm={{0.92743,0.0,0.0,0.92743,(-570.4103,65.15216)}},draw=ca0a0a4,dash pattern=on 0.40pt off 0.80pt,line cap=round,line join=round,line width=0.400pt]
    \path[draw] (184.5000,319.5000) -- (184.5000,272.5000);



  \end{scope}
  \begin{scope}[cm={{0.92743,0.0,0.0,0.92743,(-570.4103,65.15216)}},draw=blue,line cap=round,line join=round,line width=0.480pt]
    \path[draw] (184.5000,319.5000) -- (184.5000,318.5000);



    \path[draw] (184.5000,272.5000) -- (184.5000,273.5000);



  \end{scope}
  \begin{scope}[cm={{1.00588,0.0,0.0,1.00588,(-400.29035,376.888)}},draw=blue,line cap=rect,line join=bevel,line width=0.800pt]
    \path[fill=blue] (0.0000,0.0000) node[above right] (text2316) {\scriptsize 0};



  \end{scope}
  \begin{scope}[cm={{0.92743,0.0,0.0,0.92743,(-570.4103,65.15216)}},draw=ca0a0a4,dash pattern=on 2.59pt off 2.59pt,line cap=round,line join=round,line width=0.323pt,miter limit=4.00]
    \path[draw,dash pattern=on 2.59pt off 2.59pt,line width=0.323pt,miter limit=4.00] (201.5000,319.5000) -- (201.5000,272.5000);



  \end{scope}
  \begin{scope}[cm={{0.92743,0.0,0.0,0.92743,(-570.4103,65.15216)}},draw=blue,line cap=round,line join=round,line width=0.480pt]
    \path[draw] (201.5000,319.5000) -- (201.5000,318.5000);



    \path[draw] (201.5000,272.5000) -- (201.5000,273.5000);



  \end{scope}
  \begin{scope}[cm={{1.00588,0.0,0.0,1.00588,(-385.68735,376.888)}},draw=blue,line cap=rect,line join=bevel,line width=0.800pt]
    \path[fill=blue] (0.0000,0.0000) node[above right] (text2346) {\scriptsize 3};



  \end{scope}
  \begin{scope}[cm={{0.92743,0.0,0.0,0.92743,(-570.4103,65.15216)}},draw=ca0a0a4,dash pattern=on 2.59pt off 2.59pt,line cap=round,line join=round,line width=0.323pt,miter limit=4.00]
    \path[draw,dash pattern=on 2.59pt off 2.59pt,line width=0.323pt,miter limit=4.00] (218.5000,319.5000) -- (218.5000,272.5000);



  \end{scope}
  \begin{scope}[cm={{0.92743,0.0,0.0,0.92743,(-570.4103,65.15216)}},draw=blue,line cap=round,line join=round,line width=0.480pt]
    \path[draw] (218.5000,319.5000) -- (218.5000,318.5000);



    \path[draw] (218.5000,272.5000) -- (218.5000,273.5000);



  \end{scope}
  \begin{scope}[cm={{1.00588,0.0,0.0,1.00588,(-370.09635,376.888)}},draw=blue,line cap=rect,line join=bevel,line width=0.800pt]
    \path[fill=blue] (0.0000,0.0000) node[above right] (text2376) {\scriptsize 6};



  \end{scope}
  \begin{scope}[cm={{0.92743,0.0,0.0,0.92743,(-570.4103,65.15216)}},draw=ca0a0a4,dash pattern=on 2.59pt off 2.59pt,line cap=round,line join=round,line width=0.323pt,miter limit=4.00]
    \path[draw,dash pattern=on 2.59pt off 2.59pt,line width=0.323pt,miter limit=4.00] (235.5000,319.5000) -- (235.5000,272.5000);



  \end{scope}
  \begin{scope}[cm={{0.92743,0.0,0.0,0.92743,(-570.4103,65.15216)}},draw=blue,line cap=round,line join=round,line width=0.480pt]
    \path[draw] (235.5000,319.5000) -- (235.5000,318.5000);



    \path[draw] (235.5000,272.5000) -- (235.5000,273.5000);



  \end{scope}
  \begin{scope}[cm={{1.00588,0.0,0.0,1.00588,(-354.49635,376.82362)}},draw=blue,line cap=rect,line join=bevel,line width=0.800pt]
    \path[fill=blue] (0.0000,0.0000) node[above right] (text2406) {\scriptsize 9};



  \end{scope}
  \begin{scope}[cm={{0.92743,0.0,0.0,0.92743,(-570.4103,65.15216)}},draw=blue,line cap=round,line join=round,line width=0.480pt]
    \path[draw] (246.5000,315.5000) -- (245.5000,315.5000);



  \end{scope}
  \begin{scope}[cm={{1.00588,0.0,0.0,1.00588,(-422.71619,360.06187)}},draw=blue,line cap=rect,line join=bevel,line width=0.800pt]
    \path[fill=blue] (0.0000,0.0000) node[above right] (text2430) {\scriptsize -1000};



  \end{scope}
  \begin{scope}[cm={{0.92743,0.0,0.0,0.92743,(-570.4103,65.15216)}},draw=blue,line cap=round,line join=round,line width=0.480pt]
    \path[draw] (246.5000,295.5000) -- (245.5000,295.5000);



  \end{scope}
  \begin{scope}[cm={{1.00588,0.0,0.0,1.00588,(-418.69269,342.94487)}},draw=blue,line cap=rect,line join=bevel,line width=0.800pt]
    \path[fill=blue] (0.0000,0.0000) node[above right] (text2454) {\scriptsize -500};



  \end{scope}
  \begin{scope}[cm={{0.92743,0.0,0.0,0.92743,(-570.4103,65.15216)}},draw=blue,line cap=round,line join=round,line width=0.480pt]
    \path[draw] (246.5000,276.5000) -- (245.5000,276.5000);



  \end{scope}
  \begin{scope}[cm={{1.00588,0.0,0.0,1.00588,(-407.96603,324.32687)}},draw=blue,line cap=rect,line join=bevel,line width=0.800pt]
    \path[fill=blue] (0.0000,0.0000) node[above right] (text2478) {\scriptsize 0};



  \end{scope}
  \begin{scope}[cm={{0.92743,0.0,0.0,0.92743,(-570.4103,65.15216)}},draw=blue,line cap=round,line join=round,line width=0.480pt]
    \path[draw] (184.5000,272.5000) -- (184.5000,319.5000) -- (246.5000,319.5000) -- (246.5000,272.5000) -- (184.5000,272.5000);



  \end{scope}
  \begin{scope}[cm={{0.0,-1.00588,1.00588,0.0,(-425.21116,344.45387)}},draw=blue,line cap=rect,line join=bevel,line width=0.800pt]
    \path[fill=blue] (0.0000,0.0000) node[above right] (text2502) {\rotatebox{90}{\scriptsize $c_{i,1}$}};



  \end{scope}
  \begin{scope}[cm={{0.92743,0.0,0.0,0.92743,(-570.4103,65.15216)}},draw=blue,line cap=round,line join=round,line width=0.480pt]
    \path[draw] (184.8000,315.6000) -- (184.8000,315.6000) -- (184.9000,315.6000) -- (185.0000,315.6000) -- (185.0000,315.6000) -- (185.1000,315.6000) -- (185.1000,315.6000) -- (185.2000,315.6000) -- (185.3000,315.6000) -- (185.3000,315.6000) -- (185.4000,315.6000) -- (185.4000,315.6000) -- (185.5000,315.6000) -- (185.6000,315.6000) -- (185.6000,315.6000) -- (185.7000,315.6000) -- (185.8000,315.6000) -- (185.8000,315.6000) -- (185.9000,315.6000) -- (185.9000,315.6000) -- (186.0000,315.6000) -- (186.1000,315.6000) -- (186.1000,315.6000) -- (186.2000,315.6000) -- (186.2000,315.6000) -- (186.3000,315.6000) -- (186.4000,315.6000) -- (186.4000,315.6000) -- (186.5000,315.6000) -- (186.6000,315.6000) -- (186.6000,315.6000) -- (186.7000,315.6000) -- (186.7000,315.6000) -- (186.8000,315.6000) -- (186.9000,315.6000) -- (186.9000,315.6000) -- (187.0000,315.6000) -- (187.0000,315.6000) -- (187.1000,315.6000) -- (187.2000,315.6000) -- (187.2000,315.6000) -- (187.3000,315.6000) -- (187.4000,315.6000) -- (187.4000,315.6000) -- (187.5000,315.6000) -- (187.5000,315.6000) -- (187.6000,315.6000) -- (187.7000,315.6000) -- (187.7000,315.6000) -- (187.8000,315.6000) -- (187.8000,315.6000) -- (187.9000,315.6000) -- (188.0000,315.6000) -- (188.0000,315.6000) -- (188.1000,315.6000) -- (188.2000,315.6000) -- (188.2000,315.6000) -- (188.3000,315.6000) -- (188.3000,315.6000) -- (188.4000,315.6000) -- (188.5000,315.6000) -- (188.5000,315.6000) -- (188.6000,315.6000) -- (188.6000,315.6000) -- (188.7000,315.6000) -- (188.8000,315.6000) -- (188.8000,315.6000) -- (188.9000,315.6000) -- (189.0000,315.6000) -- (189.0000,315.6000) -- (189.1000,315.6000) -- (189.1000,315.6000) -- (189.2000,315.6000) -- (189.3000,315.6000) -- (189.3000,315.6000) -- (189.4000,315.6000) -- (189.4000,315.6000) -- (189.5000,315.6000) -- (189.6000,315.6000) -- (189.6000,315.6000) -- (189.7000,315.6000) -- (189.8000,315.6000) -- (189.8000,315.6000) -- (189.9000,315.6000) -- (189.9000,315.6000) -- (190.0000,315.6000) -- (190.1000,315.6000) -- (190.1000,315.6000) -- (190.2000,315.6000) -- (190.2000,315.6000) -- (190.3000,315.6000) -- (190.4000,315.6000) -- (190.4000,315.6000) -- (190.5000,315.6000) -- (190.5000,315.6000) -- (190.6000,315.6000) -- (190.7000,315.6000) -- (190.7000,315.6000) -- (190.8000,315.6000) -- (190.9000,315.6000) -- (190.9000,315.6000) -- (191.0000,315.6000) -- (191.0000,315.6000) -- (191.1000,315.6000) -- (191.2000,315.6000) -- (191.2000,315.6000) -- (191.3000,315.6000) -- (191.3000,315.6000) -- (191.4000,315.6000) -- (191.5000,315.6000) -- (191.5000,315.6000) -- (191.6000,315.6000) -- (191.7000,315.6000) -- (191.7000,315.6000) -- (191.8000,315.6000) -- (191.8000,315.6000) -- (191.9000,315.6000) -- (192.0000,315.6000) -- (192.0000,315.6000) -- (192.1000,315.6000) -- (192.1000,315.6000) -- (192.2000,315.6000) -- (192.3000,315.6000) -- (192.3000,315.6000) -- (192.4000,315.6000) -- (192.5000,315.6000) -- (192.5000,315.6000) -- (192.6000,315.6000) -- (192.6000,317.2000) -- (192.7000,292.4000) -- (192.8000,274.5000) -- (192.8000,276.1000) -- (192.9000,276.1000) -- (192.9000,276.1000) -- (193.0000,276.1000) -- (193.1000,276.1000) -- (193.1000,276.1000) -- (193.2000,276.1000) -- (193.3000,276.1000) -- (193.3000,276.1000) -- (193.4000,276.1000) -- (193.4000,276.1000) -- (193.5000,276.1000) -- (193.6000,276.1000) -- (193.6000,276.1000) -- (193.7000,276.1000) -- (193.7000,276.1000) -- (193.8000,276.1000) -- (193.9000,276.1000) -- (193.9000,276.1000) -- (194.0000,276.1000) -- (194.1000,276.1000) -- (194.1000,276.1000) -- (194.2000,276.1000) -- (194.2000,276.1000) -- (194.3000,276.1000) -- (194.4000,276.1000) -- (194.4000,276.1000) -- (194.5000,276.1000) -- (194.5000,276.1000) -- (194.6000,276.1000) -- (194.7000,276.1000) -- (194.7000,276.1000) -- (194.8000,276.1000) -- (194.9000,276.1000) -- (194.9000,276.1000) -- (195.0000,276.1000) -- (195.0000,276.1000) -- (195.1000,276.1000) -- (195.2000,276.1000) -- (195.2000,276.1000) -- (195.3000,276.1000) -- (195.3000,276.1000) -- (195.4000,276.1000) -- (195.5000,276.1000) -- (195.5000,276.1000) -- (195.6000,276.1000) -- (195.7000,276.1000) -- (195.7000,276.1000) -- (195.8000,276.1000) -- (195.8000,276.1000) -- (195.9000,276.1000) -- (196.0000,276.1000) -- (196.0000,276.1000) -- (196.1000,276.1000) -- (196.1000,276.1000) -- (196.2000,276.1000) -- (196.3000,276.1000) -- (196.3000,276.1000) -- (196.4000,276.1000) -- (196.5000,276.1000) -- (196.5000,276.1000) -- (196.6000,276.1000) -- (196.6000,276.1000) -- (196.7000,276.1000) -- (196.8000,276.1000) -- (196.8000,276.1000) -- (196.9000,276.1000) -- (196.9000,276.1000) -- (197.0000,276.1000) -- (197.1000,276.1000) -- (197.1000,276.1000) -- (197.2000,276.1000) -- (197.3000,276.1000) -- (197.3000,276.1000) -- (197.4000,276.1000) -- (197.4000,276.1000) -- (197.5000,276.1000) -- (197.6000,276.1000) -- (197.6000,276.1000) -- (197.7000,276.1000) -- (197.7000,276.1000) -- (197.8000,276.1000) -- (197.9000,276.1000) -- (197.9000,276.1000) -- (198.0000,276.1000) -- (198.1000,276.1000) -- (198.1000,276.1000) -- (198.2000,276.1000) -- (198.2000,276.1000) -- (198.3000,276.1000) -- (198.4000,276.1000) -- (198.4000,276.1000) -- (198.5000,276.1000) -- (198.5000,276.1000) -- (198.6000,276.1000) -- (198.7000,276.1000) -- (198.7000,276.1000) -- (198.8000,276.1000) -- (198.9000,276.1000) -- (198.9000,276.1000) -- (199.0000,276.1000) -- (199.0000,276.1000) -- (199.1000,276.1000) -- (199.2000,276.1000) -- (199.2000,276.1000) -- (199.3000,276.1000) -- (199.3000,276.1000) -- (199.4000,276.1000) -- (199.5000,276.1000) -- (199.5000,276.1000) -- (199.6000,276.1000) -- (199.7000,276.1000) -- (199.7000,276.1000) -- (199.8000,276.1000) -- (199.8000,276.1000) -- (199.9000,276.1000) -- (200.0000,276.1000) -- (200.0000,276.1000) -- (200.1000,276.1000) -- (200.1000,276.1000) -- (200.2000,276.1000) -- (200.3000,276.1000) -- (200.3000,276.1000) -- (200.4000,276.1000) -- (200.5000,276.1000) -- (200.5000,276.1000) -- (200.6000,276.1000) -- (200.6000,276.1000) -- (200.7000,276.1000) -- (200.8000,276.1000) -- (200.8000,276.1000) -- (200.9000,276.1000) -- (200.9000,276.1000) -- (201.0000,276.1000) -- (201.1000,276.1000) -- (201.1000,276.1000) -- (201.2000,276.1000) -- (201.3000,276.1000) -- (201.3000,276.1000) -- (201.4000,276.1000) -- (201.4000,276.1000) -- (201.5000,276.1000) -- (201.6000,276.1000) -- (201.6000,276.1000) -- (201.7000,276.1000) -- (201.7000,276.1000) -- (201.8000,276.1000) -- (201.9000,276.1000) -- (201.9000,276.1000) -- (202.0000,276.1000) -- (202.0000,276.1000) -- (202.1000,276.1000) -- (202.2000,276.1000) -- (202.2000,276.1000) -- (202.3000,276.1000) -- (202.4000,276.1000) -- (202.4000,276.1000) -- (202.5000,276.1000) -- (202.5000,276.1000) -- (202.6000,276.1000) -- (202.7000,276.1000) -- (202.7000,276.1000) -- (202.8000,276.1000) -- (202.8000,276.1000) -- (202.9000,276.1000) -- (203.0000,276.1000) -- (203.0000,276.1000) -- (203.1000,276.1000) -- (203.2000,276.1000) -- (203.2000,276.1000) -- (203.3000,276.1000) -- (203.3000,276.1000) -- (203.4000,276.1000) -- (203.5000,276.1000) -- (203.5000,276.1000) -- (203.6000,276.1000) -- (203.6000,276.1000) -- (203.7000,276.1000) -- (203.8000,276.1000) -- (203.8000,276.1000) -- (203.9000,276.1000) -- (204.0000,276.1000) -- (204.0000,276.1000) -- (204.1000,276.1000) -- (204.1000,276.1000) -- (204.2000,276.1000) -- (204.3000,276.1000) -- (204.3000,276.1000) -- (204.4000,276.1000) -- (204.4000,276.1000) -- (204.5000,276.1000) -- (204.6000,276.1000) -- (204.6000,276.1000) -- (204.7000,276.1000) -- (204.8000,276.1000) -- (204.8000,276.1000) -- (204.9000,276.1000) -- (204.9000,276.1000) -- (205.0000,276.1000) -- (205.1000,276.1000) -- (205.1000,276.1000) -- (205.2000,276.1000) -- (205.2000,276.1000) -- (205.3000,276.1000) -- (205.4000,276.1000) -- (205.4000,276.1000) -- (205.5000,276.1000) -- (205.6000,276.1000) -- (205.6000,276.1000) -- (205.7000,276.1000) -- (205.7000,276.1000) -- (205.8000,276.1000) -- (205.9000,276.1000) -- (205.9000,276.1000) -- (206.0000,276.1000) -- (206.0000,276.1000) -- (206.1000,276.1000) -- (206.2000,276.1000) -- (206.2000,276.1000) -- (206.3000,276.1000) -- (206.4000,276.1000) -- (206.4000,276.1000) -- (206.5000,276.1000) -- (206.5000,276.1000) -- (206.6000,276.1000) -- (206.7000,276.1000) -- (206.7000,276.1000) -- (206.8000,276.1000) -- (206.8000,276.1000) -- (206.9000,276.1000) -- (207.0000,276.1000) -- (207.0000,276.1000) -- (207.1000,276.1000) -- (207.2000,276.1000) -- (207.2000,276.1000) -- (207.3000,276.1000) -- (207.3000,276.1000) -- (207.4000,276.1000) -- (207.5000,276.1000) -- (207.5000,276.1000) -- (207.6000,276.1000) -- (207.6000,276.1000) -- (207.7000,276.1000) -- (207.8000,276.1000) -- (207.8000,276.1000) -- (207.9000,276.1000) -- (208.0000,276.1000) -- (208.0000,276.1000) -- (208.1000,276.1000) -- (208.1000,276.1000) -- (208.2000,276.1000) -- (208.3000,276.1000) -- (208.3000,276.1000) -- (208.4000,276.1000) -- (208.4000,276.1000) -- (208.5000,276.1000) -- (208.6000,276.1000) -- (208.6000,276.1000) -- (208.7000,276.1000) -- (208.8000,276.1000) -- (208.8000,276.1000) -- (208.9000,276.1000) -- (208.9000,276.1000) -- (209.0000,276.1000) -- (209.1000,276.1000) -- (209.1000,276.1000) -- (209.2000,276.1000) -- (209.2000,276.1000) -- (209.3000,276.1000) -- (209.4000,276.1000) -- (209.4000,276.1000) -- (209.5000,276.1000) -- (209.6000,276.1000) -- (209.6000,276.1000) -- (209.7000,276.1000) -- (209.7000,276.1000) -- (209.8000,276.1000) -- (209.9000,276.1000) -- (209.9000,276.1000) -- (210.0000,276.1000) -- (210.0000,276.1000) -- (210.1000,276.1000) -- (210.2000,276.1000) -- (210.2000,276.1000) -- (210.3000,276.1000) -- (210.4000,276.1000) -- (210.4000,276.1000) -- (210.5000,276.1000) -- (210.5000,276.1000) -- (210.6000,276.1000) -- (210.7000,276.1000) -- (210.7000,276.1000) -- (210.8000,276.1000) -- (210.8000,276.1000) -- (210.9000,276.1000) -- (211.0000,276.1000) -- (211.0000,276.1000) -- (211.1000,276.1000) -- (211.2000,276.1000) -- (211.2000,276.1000) -- (211.3000,276.1000) -- (211.3000,276.1000) -- (211.4000,276.1000) -- (211.5000,276.1000) -- (211.5000,276.1000) -- (211.6000,276.1000) -- (211.6000,276.1000) -- (211.7000,276.1000) -- (211.8000,276.1000) -- (211.8000,276.1000) -- (211.9000,276.1000) -- (212.0000,276.1000) -- (212.0000,276.1000) -- (212.1000,276.1000) -- (212.1000,276.1000) -- (212.2000,276.1000) -- (212.3000,276.1000) -- (212.3000,276.1000) -- (212.4000,276.1000) -- (212.4000,276.1000) -- (212.5000,276.1000) -- (212.6000,276.1000) -- (212.6000,276.1000) -- (212.7000,276.1000) -- (212.8000,276.1000) -- (212.8000,276.1000) -- (212.9000,276.1000) -- (212.9000,276.1000) -- (213.0000,276.1000) -- (213.1000,276.1000) -- (213.1000,276.1000) -- (213.2000,276.1000) -- (213.2000,276.1000) -- (213.3000,276.1000) -- (213.4000,276.1000) -- (213.4000,276.1000) -- (213.5000,276.1000) -- (213.6000,276.1000) -- (213.6000,276.1000) -- (213.7000,276.1000) -- (213.7000,276.1000) -- (213.8000,276.1000) -- (213.9000,276.1000) -- (213.9000,276.1000) -- (214.0000,276.1000) -- (214.0000,276.1000) -- (214.1000,276.1000) -- (214.2000,276.1000) -- (214.2000,276.1000) -- (214.3000,276.1000) -- (214.4000,276.1000) -- (214.4000,276.1000) -- (214.5000,276.1000) -- (214.5000,276.1000) -- (214.6000,276.1000) -- (214.7000,276.1000) -- (214.7000,276.1000) -- (214.8000,276.1000) -- (214.8000,276.1000) -- (214.9000,276.1000) -- (215.0000,276.1000) -- (215.0000,276.1000) -- (215.1000,276.1000) -- (215.1000,276.1000) -- (215.2000,276.1000) -- (215.3000,276.1000) -- (215.3000,276.1000) -- (215.4000,276.1000) -- (215.5000,276.1000) -- (215.5000,276.1000) -- (215.6000,276.1000) -- (215.6000,276.1000) -- (215.7000,276.1000) -- (215.8000,276.1000) -- (215.8000,276.1000) -- (215.9000,276.1000) -- (215.9000,276.1000) -- (216.0000,276.1000) -- (216.1000,276.1000) -- (216.1000,276.1000) -- (216.2000,276.1000) -- (216.3000,276.1000) -- (216.3000,276.1000) -- (216.4000,276.1000) -- (216.4000,276.1000) -- (216.5000,276.1000) -- (216.6000,276.1000) -- (216.6000,276.1000) -- (216.7000,276.1000) -- (216.7000,276.1000) -- (216.8000,276.1000) -- (216.9000,276.1000) -- (216.9000,276.1000) -- (217.0000,276.1000) -- (217.1000,276.1000) -- (217.1000,276.1000) -- (217.2000,276.1000) -- (217.2000,276.1000) -- (217.3000,276.1000) -- (217.4000,276.1000) -- (217.4000,276.1000) -- (217.5000,276.1000) -- (217.5000,276.1000) -- (217.6000,276.1000) -- (217.7000,276.1000) -- (217.7000,276.1000) -- (217.8000,276.1000) -- (217.9000,276.1000) -- (217.9000,276.1000) -- (218.0000,276.1000) -- (218.0000,276.1000) -- (218.1000,276.1000) -- (218.2000,276.1000) -- (218.2000,276.1000) -- (218.3000,276.1000) -- (218.3000,276.1000) -- (218.4000,276.1000) -- (218.5000,276.1000) -- (218.5000,276.1000) -- (218.6000,276.1000) -- (218.7000,276.1000) -- (218.7000,276.1000) -- (218.8000,276.1000) -- (218.8000,276.1000) -- (218.9000,276.1000) -- (219.0000,276.1000) -- (219.0000,276.1000) -- (219.1000,276.1000) -- (219.1000,276.1000) -- (219.2000,276.1000) -- (219.3000,276.1000) -- (219.3000,276.1000) -- (219.4000,276.1000) -- (219.5000,276.1000) -- (219.5000,276.1000) -- (219.6000,276.1000) -- (219.6000,276.1000) -- (219.7000,276.1000) -- (219.8000,276.1000) -- (219.8000,276.1000) -- (219.9000,276.1000) -- (219.9000,276.1000) -- (220.0000,276.1000) -- (220.1000,276.1000) -- (220.1000,276.1000) -- (220.2000,276.1000) -- (220.3000,276.1000) -- (220.3000,276.1000) -- (220.4000,276.1000) -- (220.4000,276.1000) -- (220.5000,276.1000) -- (220.6000,276.1000) -- (220.6000,276.1000) -- (220.7000,276.1000) -- (220.7000,276.1000) -- (220.8000,276.1000) -- (220.9000,276.1000) -- (220.9000,276.1000) -- (221.0000,276.1000) -- (221.1000,276.1000) -- (221.1000,276.1000) -- (221.2000,276.1000) -- (221.2000,276.1000) -- (221.3000,276.1000) -- (221.4000,276.1000) -- (221.4000,276.1000) -- (221.5000,276.1000) -- (221.5000,276.1000) -- (221.6000,276.1000) -- (221.7000,276.1000) -- (221.7000,276.1000) -- (221.8000,276.1000) -- (221.9000,276.1000) -- (221.9000,276.1000) -- (222.0000,276.1000) -- (222.0000,276.1000) -- (222.1000,276.1000) -- (222.2000,276.1000) -- (222.2000,276.1000) -- (222.3000,276.1000) -- (222.3000,276.1000) -- (222.4000,276.1000) -- (222.5000,276.1000) -- (222.5000,276.1000) -- (222.6000,276.1000) -- (222.7000,276.1000) -- (222.7000,276.1000) -- (222.8000,276.1000) -- (222.8000,276.1000) -- (222.9000,276.1000) -- (223.0000,276.1000) -- (223.0000,276.1000) -- (223.1000,276.1000) -- (223.1000,276.1000) -- (223.2000,276.1000) -- (223.3000,276.1000) -- (223.3000,276.1000) -- (223.4000,276.1000) -- (223.5000,276.1000) -- (223.5000,276.1000) -- (223.6000,276.1000) -- (223.6000,276.1000) -- (223.7000,276.1000) -- (223.8000,276.1000) -- (223.8000,276.1000) -- (223.9000,276.1000) -- (223.9000,276.1000) -- (224.0000,276.1000) -- (224.1000,276.1000) -- (224.1000,276.1000) -- (224.2000,276.1000) -- (224.3000,276.1000) -- (224.3000,276.1000) -- (224.4000,276.1000) -- (224.4000,276.1000) -- (224.5000,276.1000) -- (224.6000,276.1000) -- (224.6000,276.1000) -- (224.7000,276.1000) -- (224.7000,276.1000) -- (224.8000,276.1000) -- (224.9000,276.1000) -- (224.9000,276.1000) -- (225.0000,276.1000) -- (225.1000,276.1000) -- (225.1000,276.1000) -- (225.2000,276.1000) -- (225.2000,276.1000) -- (225.3000,276.1000) -- (225.4000,276.1000) -- (225.4000,276.1000) -- (225.5000,276.1000) -- (225.5000,276.1000) -- (225.6000,276.1000) -- (225.7000,276.1000) -- (225.7000,276.1000) -- (225.8000,276.1000) -- (225.9000,276.1000) -- (225.9000,276.1000) -- (226.0000,276.1000) -- (226.0000,276.1000) -- (226.1000,276.1000) -- (226.2000,276.1000) -- (226.2000,276.1000) -- (226.3000,276.1000) -- (226.3000,276.1000) -- (226.4000,276.1000) -- (226.5000,276.1000) -- (226.5000,276.1000) -- (226.6000,276.1000) -- (226.6000,276.1000) -- (226.7000,276.1000) -- (226.8000,276.1000) -- (226.8000,276.1000) -- (226.9000,276.1000) -- (227.0000,276.1000) -- (227.0000,276.1000) -- (227.1000,276.1000) -- (227.1000,276.1000) -- (227.2000,276.1000) -- (227.3000,276.1000) -- (227.3000,276.1000) -- (227.4000,276.1000) -- (227.4000,276.1000) -- (227.5000,276.1000) -- (227.6000,276.1000) -- (227.6000,276.1000) -- (227.7000,276.1000) -- (227.8000,276.1000) -- (227.8000,276.1000) -- (227.9000,276.1000) -- (227.9000,276.1000) -- (228.0000,276.1000) -- (228.1000,276.1000) -- (228.1000,276.1000) -- (228.2000,276.1000) -- (228.2000,276.1000) -- (228.3000,276.1000) -- (228.4000,276.1000) -- (228.4000,276.1000) -- (228.5000,276.1000) -- (228.6000,276.1000) -- (228.6000,276.1000) -- (228.7000,276.1000) -- (228.7000,276.1000) -- (228.8000,276.1000) -- (228.9000,276.1000) -- (228.9000,276.1000) -- (229.0000,276.1000) -- (229.0000,276.1000) -- (229.1000,276.1000) -- (229.2000,276.1000) -- (229.2000,276.1000) -- (229.3000,276.1000) -- (229.4000,276.1000) -- (229.4000,276.1000) -- (229.5000,276.1000) -- (229.5000,276.1000) -- (229.6000,276.1000) -- (229.7000,276.1000) -- (229.7000,276.1000) -- (229.8000,276.1000) -- (229.8000,276.1000) -- (229.9000,276.1000) -- (230.0000,276.1000) -- (230.0000,276.1000) -- (230.1000,276.1000) -- (230.2000,276.1000) -- (230.2000,276.1000) -- (230.3000,276.1000) -- (230.3000,276.1000) -- (230.4000,276.1000) -- (230.5000,276.1000) -- (230.5000,276.1000) -- (230.6000,276.1000) -- (230.6000,276.1000) -- (230.7000,276.1000) -- (230.8000,276.1000) -- (230.8000,276.1000) -- (230.9000,276.1000) -- (231.0000,276.1000) -- (231.0000,276.1000) -- (231.1000,276.1000) -- (231.1000,276.1000) -- (231.2000,276.1000) -- (231.3000,276.1000) -- (231.3000,276.1000) -- (231.4000,276.1000) -- (231.4000,276.1000) -- (231.5000,276.1000) -- (231.6000,276.1000) -- (231.6000,276.1000) -- (231.7000,276.1000) -- (231.8000,276.1000) -- (231.8000,276.1000) -- (231.9000,276.1000) -- (231.9000,276.1000) -- (232.0000,276.1000) -- (232.1000,276.1000) -- (232.1000,276.1000) -- (232.2000,276.1000) -- (232.2000,276.1000) -- (232.3000,276.1000) -- (232.4000,276.1000) -- (232.4000,276.1000) -- (232.5000,276.1000) -- (232.6000,276.1000) -- (232.6000,276.1000) -- (232.7000,276.1000) -- (232.7000,276.1000) -- (232.8000,276.1000) -- (232.9000,276.1000) -- (232.9000,276.1000) -- (233.0000,276.1000) -- (233.0000,276.1000) -- (233.1000,276.1000) -- (233.2000,276.1000) -- (233.2000,276.1000) -- (233.3000,276.1000) -- (233.4000,276.1000) -- (233.4000,276.1000) -- (233.5000,276.1000) -- (233.5000,276.1000) -- (233.6000,276.1000) -- (233.7000,276.1000) -- (233.7000,276.1000) -- (233.8000,276.1000) -- (233.8000,276.1000) -- (233.9000,276.1000) -- (234.0000,276.1000) -- (234.0000,276.1000) -- (234.1000,276.1000) -- (234.2000,276.1000) -- (234.2000,276.1000) -- (234.3000,276.1000) -- (234.3000,276.1000) -- (234.4000,276.1000) -- (234.5000,276.1000) -- (234.5000,276.1000) -- (234.6000,276.1000) -- (234.6000,276.1000) -- (234.7000,276.1000) -- (234.8000,276.1000) -- (234.8000,276.1000) -- (234.9000,276.1000) -- (235.0000,276.1000) -- (235.0000,276.1000) -- (235.1000,276.1000) -- (235.1000,276.1000) -- (235.2000,276.1000) -- (235.3000,276.1000) -- (235.3000,276.1000) -- (235.4000,276.1000) -- (235.4000,276.1000) -- (235.5000,276.1000) -- (235.6000,276.1000) -- (235.6000,276.1000) -- (235.7000,276.1000) -- (235.8000,276.1000) -- (235.8000,276.1000) -- (235.9000,276.1000) -- (235.9000,276.1000) -- (236.0000,276.1000) -- (236.1000,276.1000) -- (236.1000,276.1000) -- (236.2000,276.1000) -- (236.2000,276.1000) -- (236.3000,276.1000) -- (236.4000,276.1000) -- (236.4000,276.1000) -- (236.5000,276.1000) -- (236.6000,276.1000) -- (236.6000,276.1000) -- (236.7000,276.1000) -- (236.7000,276.1000) -- (236.8000,276.1000) -- (236.9000,276.1000) -- (236.9000,276.1000) -- (237.0000,276.1000) -- (237.0000,276.1000) -- (237.1000,276.1000) -- (237.2000,276.1000) -- (237.2000,276.1000) -- (237.3000,276.1000) -- (237.4000,276.1000) -- (237.4000,276.1000) -- (237.5000,276.1000) -- (237.5000,276.1000) -- (237.6000,276.1000) -- (237.7000,276.1000) -- (237.7000,276.1000) -- (237.8000,276.1000) -- (237.8000,276.1000) -- (237.9000,276.1000) -- (238.0000,276.1000) -- (238.0000,276.1000) -- (238.1000,276.1000) -- (238.1000,276.1000) -- (238.2000,276.1000) -- (238.3000,276.1000) -- (238.3000,276.1000) -- (238.4000,276.1000) -- (238.5000,276.1000) -- (238.5000,276.1000) -- (238.6000,276.1000) -- (238.6000,276.1000) -- (238.7000,276.1000) -- (238.8000,276.1000) -- (238.8000,276.1000) -- (238.9000,276.1000) -- (238.9000,276.1000) -- (239.0000,276.1000) -- (239.1000,276.1000) -- (239.1000,276.1000) -- (239.2000,276.1000) -- (239.3000,276.1000) -- (239.3000,276.1000) -- (239.4000,276.1000) -- (239.4000,276.1000) -- (239.5000,276.1000) -- (239.6000,276.1000) -- (239.6000,276.1000) -- (239.7000,276.1000) -- (239.7000,276.1000) -- (239.8000,276.1000) -- (239.9000,276.1000) -- (239.9000,276.1000) -- (240.0000,276.1000) -- (240.1000,276.1000) -- (240.1000,276.1000) -- (240.2000,276.1000) -- (240.2000,276.1000) -- (240.3000,276.1000) -- (240.4000,276.1000) -- (240.4000,276.1000) -- (240.5000,276.1000) -- (240.5000,276.1000) -- (240.6000,276.1000) -- (240.7000,276.1000) -- (240.7000,276.1000) -- (240.8000,276.1000) -- (240.9000,276.1000) -- (240.9000,276.1000) -- (241.0000,276.1000) -- (241.0000,276.1000) -- (241.1000,276.1000) -- (241.2000,276.1000) -- (241.2000,276.1000) -- (241.3000,276.1000) -- (241.3000,276.1000) -- (241.4000,276.1000) -- (241.5000,276.1000) -- (241.5000,276.1000) -- (241.6000,276.1000) -- (241.7000,276.1000) -- (241.7000,276.1000) -- (241.8000,276.1000) -- (241.8000,276.1000) -- (241.9000,276.1000) -- (242.0000,276.1000) -- (242.0000,276.1000) -- (242.1000,276.1000) -- (242.1000,276.1000) -- (242.2000,276.1000) -- (242.3000,276.1000) -- (242.3000,276.1000) -- (242.4000,276.1000) -- (242.5000,276.1000) -- (242.5000,276.1000) -- (242.6000,276.1000) -- (242.6000,276.1000) -- (242.7000,276.1000) -- (242.8000,276.1000) -- (242.8000,276.1000) -- (242.9000,276.1000) -- (242.9000,276.1000) -- (243.0000,276.1000) -- (243.1000,276.1000) -- (243.1000,276.1000) -- (243.2000,276.1000) -- (243.3000,276.1000) -- (243.3000,276.1000) -- (243.4000,276.1000) -- (243.4000,276.1000) -- (243.5000,276.1000) -- (243.6000,276.1000) -- (243.6000,276.1000) -- (243.7000,276.1000) -- (243.7000,276.1000) -- (243.8000,276.1000) -- (243.9000,276.1000) -- (243.9000,276.1000) -- (244.0000,276.1000) -- (244.1000,276.1000) -- (244.1000,276.1000) -- (244.2000,276.1000) -- (244.2000,276.1000) -- (244.3000,276.1000) -- (244.4000,276.1000) -- (244.4000,276.1000) -- (244.5000,276.1000) -- (244.5000,276.1000) -- (244.6000,276.1000) -- (244.7000,276.1000) -- (244.7000,276.1000) -- (244.8000,276.1000) -- (244.9000,276.1000) -- (244.9000,276.1000) -- (245.0000,276.1000) -- (245.0000,276.1000) -- (245.1000,276.1000) -- (245.2000,276.1000) -- (245.2000,276.1000) -- (245.3000,276.1000) -- (245.3000,276.1000) -- (245.4000,276.1000) -- (245.5000,276.1000) -- (245.5000,276.1000) -- (245.6000,276.1000) -- (245.7000,276.1000) -- (245.7000,276.1000) -- (245.8000,276.1000) -- (245.8000,276.1000) -- (245.9000,276.1000) -- (246.0000,276.1000) -- (246.0000,276.1000) -- (246.1000,276.1000) -- (246.1000,276.1000) -- (246.2000,276.1000) -- (246.3000,276.1000);



  \end{scope}
  \begin{scope}[cm={{1.00588,0.0,0.0,1.00588,(-510.7994,343.95446)}},draw=blue,line cap=rect,line join=bevel,line width=0.800pt]
    \path[fill=blue] (-4.4737,45.8089) node[above right] (text344-6) {x (m)};



    \path[fill=blue] (120.9636,45.7382) node[above right] (text344-6-4) {\scriptsize Time (min)};



    \path[fill=blue] (-49.9760,62.5230) node[above right] (text344-6-3-1) {(a) Re-planned trajectories};



    \path[fill=blue] (81.5138,62.5230) node[above right] (text344-6-3-1-6) {(b) Parameters ($c_{i,1},c_{i,2}$) evol.};



    \path[fill=blue] (277.1823,62.4148) node[above right] (text344-6-3-1-6-3) {(c) Energies and batteries evol.};



  \end{scope}
  \begin{scope}[cm={{1.00588,0.0,0.0,1.00588,(-407.7468,193.72978)}},draw=blue,line cap=rect,line join=bevel,line width=0.800pt]
    \path[fill=blue] (0.0000,0.0000) node[above right] (text2120-2) {\scriptsize 2};



  \end{scope}
  \begin{scope}[cm={{1.00588,0.0,0.0,1.00588,(-407.69047,177.61778)}},draw=blue,line cap=rect,line join=bevel,line width=0.800pt]
    \path[fill=blue] (0.0000,0.0000) node[above right] (text2144-6) {\scriptsize 6};



  \end{scope}
  \begin{scope}[cm={{1.00588,0.0,0.0,1.00588,(-411.77835,160.00678)}},draw=blue,line cap=rect,line join=bevel,line width=0.800pt]
    \path[fill=blue] (0.0000,0.0000) node[above right] (text2168-3) {\scriptsize 10};



  \end{scope}
  \begin{scope}[cm={{0.0,-1.00588,1.00588,0.0,(-419.03217,183.62678)}},draw=blue,line cap=rect,line join=bevel,line width=0.800pt]
    \path[fill=blue] (0.0000,-5.9329) node[above right] (text2192-1) {\rotatebox{90}{\scriptsize $c_{i,2}$}};



  \end{scope}
  \begin{scope}[cm={{1.00588,0.0,0.0,1.00588,(-422.50501,240.01778)}},draw=blue,line cap=rect,line join=bevel,line width=0.800pt]
    \path[fill=blue] (0.0000,0.0000) node[above right] (text2430-9) {\scriptsize -1000};



  \end{scope}
  \begin{scope}[cm={{1.00588,0.0,0.0,1.00588,(-418.48151,222.90078)}},draw=blue,line cap=rect,line join=bevel,line width=0.800pt]
    \path[fill=blue] (0.0000,0.0000) node[above right] (text2454-1) {\scriptsize -500};



  \end{scope}
  \begin{scope}[cm={{1.00588,0.0,0.0,1.00588,(-407.75485,204.28278)}},draw=blue,line cap=rect,line join=bevel,line width=0.800pt]
    \path[fill=blue] (0.0000,0.0000) node[above right] (text2478-9) {\scriptsize 0};



  \end{scope}
  \begin{scope}[cm={{0.0,-1.00588,1.00588,0.0,(-424.99998,224.40978)}},draw=blue,line cap=rect,line join=bevel,line width=0.800pt]
    \path[fill=blue] (0.0000,0.0000) node[above right] (text2502-0) {\rotatebox{90}{\scriptsize $c_{i,1}$}};



  \end{scope}
  \begin{scope}[cm={{1.00588,0.0,0.0,1.00588,(-605.07343,207.04185)}},draw=blue,line cap=rect,line join=bevel,line width=0.800pt]
    \path[fill=blue] (0.0000,0.0000) node[above right] (text154-9-8) {\Large i};



  \end{scope}
  \begin{scope}[cm={{1.00588,0.0,0.0,1.00588,(-607.56518,325.40947)}},draw=blue,line cap=rect,line join=bevel,line width=0.800pt]
    \path[fill=blue] (2.4954,0.0000) node[above right] (text154-9-0-0) {\Large ii};



  \end{scope}
  \begin{scope}[cm={{0.84173,0.0,0.0,0.84173,(-604.99975,95.12326)}},draw=blue,line cap=round,line join=round,line width=0.480pt]
    \path[draw] (61.6000,227.9000) -- (61.6000,227.9000) -- (61.8000,231.0000) -- (61.9000,233.8000) -- (60.9000,236.4000) -- (59.0000,238.8000) -- (57.6000,241.4000) -- (56.8000,244.1000) -- (56.6000,247.0000) -- (56.6000,249.9000) -- (56.5000,252.8000) -- (56.5000,255.7000) -- (56.5000,258.6000) -- (56.5000,261.5000) -- (56.5000,264.4000) -- (56.5000,267.3000) -- (56.5000,270.2000) -- (56.5000,273.1000) -- (56.5000,276.0000) -- (56.5000,278.9000) -- (56.5000,281.8000) -- (56.5000,284.6000) -- (56.5000,287.5000) -- (56.8000,290.4000) -- (57.5000,293.3000) -- (58.7000,296.0000) -- (60.2000,298.6000) -- (62.2000,301.0000) -- (64.5000,303.1000) -- (67.1000,305.0000) -- (70.1000,306.6000) -- (73.2000,307.7000) -- (76.6000,308.6000) -- (80.1000,309.0000) -- (83.6000,308.9000) -- (87.0000,308.5000) -- (90.4000,307.6000) -- (93.6000,306.4000) -- (96.6000,304.8000) -- (99.2000,302.8000) -- (101.6000,300.6000) -- (103.4000,298.1000) -- (104.8000,295.4000) -- (105.6000,292.6000) -- (106.1000,289.8000) -- (106.3000,286.9000) -- (106.5000,284.1000) -- (106.5000,281.2000) -- (106.6000,278.3000) -- (106.6000,275.4000) -- (106.6000,272.5000) -- (106.6000,269.6000) -- (106.6000,266.8000) -- (106.6000,263.9000) -- (106.6000,261.0000) -- (106.6000,258.1000) -- (106.6000,255.2000) -- (106.7000,252.3000) -- (107.2000,249.4000) -- (107.4000,246.5000) -- (107.1000,243.7000) -- (106.2000,240.9000) -- (104.9000,238.2000) -- (103.2000,235.7000) -- (100.9000,233.5000) -- (98.2000,231.6000) -- (95.2000,230.1000) -- (91.9000,229.0000) -- (88.5000,228.4000) -- (85.0000,228.3000) -- (81.5000,228.7000) -- (78.2000,229.6000) -- (75.1000,230.9000) -- (72.3000,232.6000) -- (69.9000,234.7000) -- (67.8000,237.1000) -- (66.3000,239.7000) -- (65.3000,242.4000) -- (64.9000,245.3000) -- (64.8000,248.2000) -- (64.8000,251.1000) -- (64.8000,254.0000) -- (64.8000,256.9000) -- (64.8000,259.8000) -- (64.8000,262.7000) -- (64.8000,265.6000) -- (64.8000,268.5000) -- (64.8000,271.4000) -- (64.8000,274.2000) -- (64.8000,277.1000) -- (64.8000,280.0000) -- (64.8000,282.9000) -- (64.7000,285.8000) -- (64.9000,288.7000) -- (65.4000,291.6000) -- (66.4000,294.3000) -- (67.8000,297.0000) -- (69.6000,299.4000) -- (71.8000,301.7000) -- (74.4000,303.7000) -- (77.2000,305.4000) -- (80.3000,306.7000) -- (83.5000,307.7000) -- (87.0000,308.3000) -- (90.5000,308.4000) -- (94.0000,308.2000) -- (97.4000,307.5000) -- (100.7000,306.4000) -- (103.8000,305.0000) -- (106.6000,303.2000) -- (109.1000,301.1000) -- (111.2000,298.7000) -- (112.8000,296.1000) -- (113.8000,293.3000) -- (114.4000,290.5000) -- (114.8000,287.7000) -- (114.9000,284.8000) -- (115.0000,282.0000) -- (115.1000,279.1000) -- (115.1000,276.2000) -- (115.1000,273.3000) -- (115.1000,270.4000) -- (115.1000,267.5000) -- (115.1000,264.6000) -- (115.1000,261.7000) -- (115.1000,258.9000) -- (115.1000,256.0000) -- (115.1000,253.1000) -- (115.4000,250.2000) -- (115.8000,247.3000) -- (115.7000,244.4000) -- (115.1000,241.6000) -- (114.1000,238.8000) -- (112.7000,236.2000) -- (110.8000,233.8000) -- (108.6000,231.5000) -- (105.9000,229.6000) -- (102.9000,228.0000) -- (99.7000,226.8000) -- (96.3000,226.0000) -- (92.8000,225.7000) -- (89.3000,225.8000) -- (85.9000,226.4000) -- (82.6000,227.4000) -- (79.5000,228.7000) -- (76.7000,230.5000) -- (74.3000,232.5000) -- (72.2000,234.8000) -- (70.4000,237.3000) -- (69.1000,240.0000) -- (68.4000,242.8000) -- (68.1000,245.7000) -- (68.1000,248.6000) -- (68.1000,251.5000) -- (68.1000,254.4000) -- (68.1000,257.3000) -- (68.1000,260.2000) -- (68.1000,263.1000) -- (68.1000,266.0000) -- (68.1000,268.9000) -- (68.1000,271.8000) -- (68.1000,274.6000) -- (68.1000,277.5000) -- (68.1000,280.4000) -- (68.1000,283.3000) -- (68.0000,286.2000) -- (68.3000,289.1000) -- (68.9000,291.9000) -- (70.0000,294.7000) -- (71.5000,297.3000) -- (73.4000,299.7000) -- (75.7000,301.9000) -- (78.3000,303.8000) -- (81.2000,305.4000) -- (84.3000,306.7000) -- (87.6000,307.6000) -- (91.1000,308.1000) -- (94.6000,308.2000) -- (98.1000,307.8000) -- (101.5000,307.1000) -- (104.7000,305.9000) -- (107.7000,304.4000) -- (110.5000,302.5000) -- (112.9000,300.3000) -- (114.9000,297.9000) -- (116.4000,295.3000) -- (117.4000,292.5000) -- (117.9000,289.7000) -- (118.2000,286.9000) -- (118.3000,284.0000) -- (118.4000,281.1000) -- (118.4000,278.3000) -- (118.5000,275.4000) -- (118.5000,272.5000) -- (118.5000,269.6000) -- (118.5000,266.7000) -- (118.5000,263.8000) -- (118.5000,260.9000) -- (118.5000,258.0000) -- (118.5000,255.1000) -- (118.5000,252.2000) -- (118.9000,249.4000) -- (119.1000,246.5000) -- (118.9000,243.6000) -- (118.3000,240.8000) -- (117.2000,238.0000) -- (115.6000,235.4000) -- (113.6000,233.0000) -- (111.3000,230.9000) -- (108.5000,229.0000) -- (105.5000,227.5000) -- (102.2000,226.4000) -- (98.8000,225.8000) -- (95.3000,225.6000) -- (91.8000,225.8000) -- (88.4000,226.5000) -- (85.2000,227.5000) -- (82.2000,229.0000) -- (79.4000,230.8000) -- (77.1000,232.9000) -- (75.0000,235.3000) -- (73.4000,237.8000) -- (72.3000,240.6000) -- (71.7000,243.4000) -- (71.5000,246.3000) -- (71.5000,249.2000) -- (71.5000,252.1000) -- (71.5000,255.0000) -- (71.5000,257.9000) -- (71.5000,260.8000) -- (71.5000,263.7000) -- (71.5000,266.6000) -- (71.5000,269.5000) -- (71.5000,272.4000) -- (71.5000,275.3000) -- (71.5000,278.1000) -- (71.5000,281.0000) -- (71.5000,283.9000) -- (71.5000,286.8000) -- (71.9000,289.7000) -- (72.7000,292.5000) -- (73.9000,295.2000) -- (75.5000,297.8000) -- (77.5000,300.1000) -- (79.9000,302.2000) -- (82.6000,304.1000) -- (85.6000,305.6000) -- (88.8000,306.7000) -- (92.2000,307.5000) -- (95.6000,307.8000) -- (99.1000,307.8000) -- (102.6000,307.3000) -- (106.0000,306.4000) -- (109.2000,305.1000) -- (112.1000,303.4000) -- (114.7000,301.4000) -- (117.0000,299.2000) -- (118.9000,296.7000) -- (120.2000,294.0000) -- (120.9000,291.2000) -- (121.3000,288.3000) -- (121.6000,285.5000) -- (121.7000,282.6000) -- (121.8000,279.8000) -- (121.8000,276.9000) -- (121.8000,274.0000) -- (121.8000,271.1000) -- (121.8000,268.2000) -- (121.8000,265.3000) -- (121.8000,262.4000) -- (121.8000,259.5000) -- (121.8000,256.6000) -- (121.8000,253.7000) -- (122.0000,250.9000) -- (122.4000,248.0000) -- (122.4000,245.1000) -- (122.1000,242.2000) -- (121.2000,239.4000) -- (120.0000,236.7000) -- (118.3000,234.2000) -- (116.1000,231.9000) -- (113.6000,229.9000) -- (110.8000,228.1000) -- (107.6000,226.8000) -- (104.3000,225.9000) -- (100.8000,225.4000) -- (97.3000,225.3000) -- (93.8000,225.7000) -- (90.5000,226.6000) -- (87.3000,227.8000) -- (84.4000,229.4000) -- (81.8000,231.3000) -- (79.6000,233.5000) -- (77.7000,236.0000) -- (76.3000,238.6000) -- (75.3000,241.4000) -- (75.0000,244.3000) -- (74.9000,247.2000) -- (74.9000,250.1000) -- (74.9000,253.0000) -- (74.9000,255.9000) -- (74.9000,258.8000) -- (74.9000,261.6000) -- (74.9000,264.5000) -- (74.9000,267.4000) -- (74.9000,270.3000) -- (74.9000,273.2000) -- (74.9000,276.1000) -- (74.9000,279.0000) -- (74.9000,281.9000) -- (74.9000,284.8000) -- (75.0000,287.7000) -- (75.5000,290.5000) -- (76.5000,293.3000) -- (77.8000,296.0000) -- (79.6000,298.4000) -- (81.8000,300.7000) -- (84.3000,302.7000) -- (87.1000,304.4000) -- (90.2000,305.8000) -- (93.5000,306.8000) -- (96.9000,307.4000) -- (100.4000,307.6000) -- (103.9000,307.3000) -- (107.3000,306.7000) -- (110.6000,305.6000) -- (113.7000,304.2000) -- (116.5000,302.4000) -- (119.0000,300.3000) -- (121.1000,297.9000) -- (122.8000,295.3000) -- (123.9000,292.6000) -- (124.5000,289.7000) -- (124.8000,286.9000) -- (125.0000,284.1000) -- (125.1000,281.2000) -- (125.1000,278.3000) -- (125.1000,275.4000) -- (125.2000,272.5000) -- (125.2000,269.6000) -- (125.2000,266.8000) -- (125.2000,263.9000) -- (125.2000,261.0000) -- (125.2000,258.1000) -- (125.2000,255.2000) -- (125.2000,252.3000) -- (125.5000,249.4000) -- (125.8000,246.5000) -- (125.7000,243.6000) -- (125.1000,240.8000) -- (124.1000,238.1000) -- (122.7000,235.4000) -- (120.8000,233.0000) -- (118.5000,230.8000) -- (115.9000,228.8000) -- (112.9000,227.3000) -- (109.6000,226.1000) -- (106.2000,225.3000) -- (102.7000,225.0000) -- (99.3000,225.2000) -- (95.8000,225.7000) -- (92.6000,226.7000) -- (89.5000,228.1000) -- (86.7000,229.8000) -- (84.2000,231.9000) -- (82.2000,234.2000) -- (80.5000,236.7000) -- (79.2000,239.4000) -- (78.5000,242.3000) -- (78.3000,245.2000) -- (78.2000,248.1000) -- (78.2000,251.0000) -- (78.2000,253.9000) -- (78.2000,256.7000) -- (78.2000,259.6000) -- (78.2000,262.5000) -- (78.2000,265.4000) -- (78.2000,268.3000) -- (78.2000,271.2000) -- (78.2000,274.1000) -- (78.2000,277.0000) -- (78.2000,279.9000) -- (78.2000,282.8000) -- (78.2000,285.7000) -- (78.5000,288.5000) -- (79.2000,291.4000) -- (80.3000,294.1000) -- (81.9000,296.7000) -- (83.8000,299.1000) -- (86.1000,301.3000) -- (88.8000,303.1000) -- (91.7000,304.7000) -- (94.9000,305.9000) -- (98.2000,306.8000) -- (101.7000,307.2000) -- (105.2000,307.2000) -- (108.7000,306.8000) -- (112.0000,306.0000) -- (115.3000,304.8000) -- (118.3000,303.2000) -- (120.9000,301.2000) -- (123.3000,299.0000) -- (125.2000,296.6000) -- (126.7000,293.9000) -- (127.5000,291.1000) -- (128.0000,288.3000) -- (128.2000,285.4000) -- (128.4000,282.6000) -- (128.5000,279.7000) -- (128.5000,276.8000) -- (128.5000,273.9000) -- (128.5000,271.0000) -- (128.5000,268.2000) -- (128.5000,265.3000) -- (128.5000,262.4000) -- (128.5000,259.5000) -- (128.5000,256.6000) -- (128.5000,253.7000) -- (128.6000,250.8000) -- (129.0000,247.9000) -- (129.2000,245.0000) -- (128.9000,242.2000) -- (128.1000,239.4000) -- (126.9000,236.7000) -- (125.3000,234.1000) -- (123.3000,231.7000) -- (120.8000,229.6000) -- (118.0000,227.9000) -- (114.9000,226.4000) -- (111.6000,225.4000) -- (108.2000,224.9000) -- (104.6000,224.7000) -- (101.2000,225.1000) -- (97.8000,225.8000) -- (94.6000,227.0000) -- (91.7000,228.5000) -- (89.0000,230.4000) -- (86.7000,232.5000) -- (84.8000,234.9000) -- (83.2000,237.5000) -- (82.2000,240.3000) -- (81.7000,243.2000) -- (81.6000,246.1000) -- (81.6000,249.0000) -- (81.6000,251.9000) -- (81.6000,254.8000) -- (81.6000,257.7000) -- (81.6000,260.6000) -- (81.6000,263.4000) -- (81.6000,266.3000) -- (81.6000,269.2000) -- (81.6000,272.1000) -- (81.6000,275.0000) -- (81.6000,277.9000) -- (81.6000,280.8000) -- (81.6000,283.7000) -- (81.6000,286.6000) -- (82.1000,289.4000) -- (83.0000,292.2000) -- (84.3000,294.9000) -- (86.0000,297.4000) -- (88.1000,299.7000) -- (90.6000,301.8000) -- (93.3000,303.5000) -- (96.4000,305.0000) -- (99.6000,306.0000) -- (103.0000,306.7000) -- (106.5000,307.0000) -- (110.0000,306.8000) -- (113.4000,306.2000) -- (116.8000,305.2000) -- (119.9000,303.8000) -- (122.8000,302.1000) -- (125.3000,300.1000) -- (127.5000,297.7000) -- (129.3000,295.2000) -- (130.4000,292.5000) -- (131.1000,289.6000) -- (131.5000,286.8000) -- (131.7000,284.0000) -- (131.8000,281.1000) -- (131.8000,278.2000) -- (131.9000,275.3000) -- (131.9000,272.4000) -- (131.9000,269.6000) -- (131.9000,266.7000) -- (131.9000,263.8000) -- (131.9000,260.9000) -- (131.9000,258.0000) -- (131.9000,255.1000) -- (131.9000,252.2000) -- (132.1000,249.3000) -- (132.5000,246.5000) -- (132.5000,243.6000) -- (132.0000,240.7000) -- (131.1000,237.9000) -- (129.7000,235.3000) -- (127.9000,232.8000) -- (125.7000,230.5000) -- (123.1000,228.6000) -- (120.1000,226.9000) -- (116.9000,225.7000) -- (113.5000,224.8000) -- (110.1000,224.5000) -- (106.6000,224.5000) -- (103.1000,225.0000) -- (99.8000,225.9000) -- (96.7000,227.3000) -- (93.9000,228.9000) -- (91.4000,230.9000) -- (89.2000,233.2000) -- (87.4000,235.7000) -- (86.1000,238.4000) -- (85.3000,241.2000) -- (85.0000,244.1000) -- (85.0000,247.0000) -- (84.9000,249.9000) -- (84.9000,252.8000) -- (84.9000,255.7000) -- (84.9000,258.6000) -- (84.9000,261.4000) -- (84.9000,264.3000) -- (84.9000,267.2000) -- (84.9000,270.1000) -- (84.9000,273.0000) -- (84.9000,275.9000) -- (84.9000,278.8000) -- (84.9000,281.7000) -- (84.9000,284.6000) -- (85.1000,287.5000) -- (85.8000,290.3000) -- (86.8000,293.1000) -- (88.3000,295.7000) -- (90.2000,298.1000) -- (92.4000,300.3000) -- (95.0000,302.3000) -- (97.9000,303.9000) -- (101.0000,305.2000) -- (104.3000,306.1000) -- (107.8000,306.6000) -- (111.3000,306.7000) -- (114.8000,306.3000) -- (118.2000,305.6000) -- (121.4000,304.4000) -- (124.5000,302.9000) -- (127.2000,301.0000) -- (129.6000,298.9000) -- (131.6000,296.5000) -- (133.2000,293.8000) -- (134.1000,291.0000) -- (134.7000,288.2000) -- (134.9000,285.4000) -- (135.1000,282.5000) -- (135.2000,279.7000) -- (135.2000,276.8000) -- (135.2000,273.9000) -- (135.2000,271.0000) -- (135.2000,268.1000) -- (135.2000,265.2000) -- (135.2000,262.3000) -- (135.2000,259.4000) -- (135.2000,256.5000) -- (135.2000,253.6000) -- (135.3000,250.8000) -- (135.7000,247.9000) -- (135.9000,245.0000) -- (135.7000,242.1000) -- (135.0000,239.3000) -- (133.9000,236.6000) -- (132.4000,234.0000) -- (130.4000,231.6000) -- (128.0000,229.4000) -- (125.3000,227.6000) -- (122.2000,226.1000) -- (118.9000,225.0000) -- (115.5000,224.3000) -- (112.0000,224.1000) -- (108.5000,224.4000) -- (105.1000,225.0000) -- (101.9000,226.1000) -- (98.9000,227.6000) -- (96.2000,229.4000) -- (93.8000,231.5000) -- (91.8000,233.9000) -- (90.2000,236.4000) -- (89.1000,239.2000) -- (88.5000,242.0000) -- (88.3000,244.9000) -- (88.3000,247.8000) -- (88.3000,250.7000) -- (88.3000,253.6000) -- (88.3000,256.5000) -- (88.3000,259.4000) -- (88.3000,262.3000) -- (88.3000,265.2000) -- (88.3000,268.1000) -- (88.3000,271.0000) -- (88.3000,273.9000) -- (88.3000,276.8000) -- (88.3000,279.7000) -- (88.3000,282.5000) -- (88.3000,285.4000) -- (88.7000,288.3000) -- (89.5000,291.1000) -- (90.7000,293.8000) -- (92.4000,296.4000) -- (94.4000,298.7000) -- (96.8000,300.8000) -- (99.5000,302.7000) -- (102.5000,304.1000) -- (105.7000,305.3000) -- (109.0000,306.0000) -- (112.5000,306.4000) -- (116.0000,306.3000) -- (119.5000,305.8000) -- (122.8000,304.9000) -- (126.0000,303.6000) -- (129.0000,301.9000) -- (131.6000,299.9000) -- (133.8000,297.6000) -- (135.7000,295.1000) -- (137.0000,292.4000) -- (137.7000,289.6000) -- (138.1000,286.8000) -- (138.4000,284.0000) -- (138.5000,281.1000) -- (138.5000,278.2000) -- (138.6000,275.3000) -- (138.6000,272.4000) -- (138.6000,269.5000) -- (138.6000,266.7000) -- (138.6000,263.8000) -- (138.6000,260.9000) -- (138.6000,258.0000) -- (138.6000,255.1000) -- (138.6000,252.2000) -- (138.8000,249.3000) -- (139.2000,246.4000) -- (139.2000,243.5000) -- (138.8000,240.7000) -- (138.0000,237.9000) -- (136.7000,235.2000) -- (135.0000,232.7000) -- (132.9000,230.4000) -- (130.3000,228.4000) -- (127.4000,226.6000) -- (124.3000,225.3000) -- (121.0000,224.4000) -- (117.5000,223.9000) -- (114.0000,223.9000) -- (110.5000,224.3000) -- (107.2000,225.1000) -- (104.0000,226.4000) -- (101.1000,228.0000) -- (98.6000,229.9000) -- (96.3000,232.1000) -- (94.5000,234.6000) -- (93.0000,237.2000) -- (92.1000,240.0000) -- (91.8000,242.9000) -- (91.7000,245.8000) -- (91.7000,248.7000) -- (91.7000,251.6000) -- (91.6000,254.5000) -- (91.6000,257.4000) -- (91.6000,260.3000) -- (91.6000,263.2000) -- (91.6000,266.1000) -- (91.6000,269.0000) -- (91.6000,271.8000) -- (91.6000,274.7000) -- (91.6000,277.6000) -- (91.6000,280.5000) -- (91.6000,283.4000) -- (91.8000,286.3000) -- (92.3000,289.2000) -- (93.3000,291.9000) -- (94.7000,294.6000) -- (96.5000,297.1000) -- (98.7000,299.3000) -- (101.2000,301.3000) -- (104.0000,303.0000) -- (107.1000,304.4000) -- (110.3000,305.3000) -- (113.8000,305.9000) -- (117.3000,306.1000) -- (120.8000,305.8000) -- (124.2000,305.2000) -- (127.5000,304.1000) -- (130.6000,302.6000) -- (133.4000,300.8000) -- (135.9000,298.7000) -- (138.0000,296.4000) -- (139.6000,293.8000) -- (140.7000,291.0000) -- (141.3000,288.2000) -- (141.6000,285.4000) -- (141.8000,282.5000) -- (141.9000,279.7000) -- (141.9000,276.8000) -- (141.9000,273.9000) -- (141.9000,271.0000) -- (142.0000,268.1000) -- (142.0000,265.2000) -- (142.0000,262.3000) -- (142.0000,259.4000) -- (142.0000,256.5000) -- (142.0000,253.6000) -- (142.0000,250.8000) -- (142.3000,247.9000) -- (142.6000,245.0000) -- (142.5000,242.1000) -- (141.9000,239.3000) -- (140.9000,236.5000) -- (139.4000,233.9000) -- (137.5000,231.5000) -- (135.2000,229.2000) -- (132.6000,227.3000) -- (129.6000,225.8000) -- (126.3000,224.6000) -- (122.9000,223.9000) -- (119.4000,223.6000) -- (115.9000,223.7000) -- (112.5000,224.3000) -- (109.3000,225.3000) -- (106.2000,226.7000) -- (103.4000,228.4000) -- (101.0000,230.5000) -- (98.9000,232.8000) -- (97.2000,235.3000) -- (96.0000,238.0000) -- (95.3000,240.9000) -- (95.1000,243.8000) -- (95.0000,246.7000) -- (95.0000,249.6000) -- (95.0000,252.5000) -- (95.0000,255.4000) -- (95.0000,258.3000) -- (95.0000,261.1000) -- (95.0000,264.0000) -- (95.0000,266.9000) -- (95.0000,269.8000) -- (95.0000,272.7000) -- (95.0000,275.6000) -- (95.0000,278.5000) -- (95.0000,281.4000) -- (95.0000,284.3000) -- (95.3000,287.1000) -- (96.0000,290.0000) -- (97.2000,292.7000) -- (98.7000,295.3000) -- (100.7000,297.7000) -- (103.0000,299.9000) -- (105.6000,301.7000) -- (108.6000,303.3000) -- (111.7000,304.5000) -- (115.1000,305.3000) -- (118.5000,305.8000) -- (122.0000,305.8000) -- (125.5000,305.3000) -- (128.9000,304.5000) -- (132.1000,303.3000) -- (135.1000,301.7000) -- (137.8000,299.7000) -- (140.1000,297.5000) -- (142.1000,295.1000) -- (143.5000,292.4000) -- (144.3000,289.6000) -- (144.8000,286.8000) -- (145.0000,283.9000) -- (145.2000,281.1000) -- (145.2000,278.2000) -- (145.3000,275.3000) -- (145.3000,272.4000) -- (145.3000,269.5000) -- (145.3000,266.7000) -- (145.3000,263.8000) -- (145.3000,260.9000) -- (145.3000,258.0000) -- (145.3000,255.1000) -- (145.3000,252.2000) -- (145.4000,249.3000) -- (145.8000,246.4000) -- (146.0000,243.6000) -- (145.7000,240.7000) -- (144.9000,237.9000) -- (143.7000,235.2000) -- (142.1000,232.6000) -- (140.0000,230.3000) -- (137.6000,228.2000) -- (134.8000,226.4000) -- (131.7000,225.0000) -- (128.4000,224.0000) -- (124.9000,223.4000) -- (121.4000,223.3000) -- (117.9000,223.6000) -- (114.6000,224.3000) -- (111.4000,225.5000) -- (108.4000,227.0000) -- (105.8000,228.9000) -- (103.5000,231.1000) -- (101.5000,233.5000) -- (100.0000,236.1000) -- (99.0000,238.9000) -- (98.5000,241.7000) -- (98.4000,244.6000) -- (98.4000,247.5000) -- (98.4000,250.4000) -- (98.4000,253.3000) -- (98.4000,256.2000) -- (98.4000,259.1000) -- (98.4000,262.0000) -- (98.4000,264.9000) -- (98.4000,267.8000) -- (98.4000,270.7000) -- (98.4000,273.6000) -- (98.4000,276.5000) -- (98.4000,279.3000) -- (98.4000,282.2000) -- (98.4000,285.1000) -- (98.9000,288.0000) -- (99.8000,290.8000) -- (101.1000,293.5000) -- (102.8000,296.0000) -- (104.9000,298.3000) -- (107.4000,300.3000) -- (110.1000,302.1000) -- (113.1000,303.5000) -- (116.4000,304.6000) -- (119.8000,305.3000) -- (123.3000,305.5000) -- (126.8000,305.3000) -- (130.2000,304.8000) -- (133.6000,303.8000) -- (136.7000,302.4000) -- (139.6000,300.7000) -- (142.1000,298.6000) -- (144.3000,296.3000) -- (146.1000,293.7000) -- (147.2000,291.0000) -- (147.9000,288.2000) -- (148.3000,285.4000) -- (148.5000,282.5000) -- (148.6000,279.7000) -- (148.6000,276.8000) -- (148.6000,273.9000) -- (148.7000,271.0000) -- (148.7000,268.1000) -- (148.7000,265.2000) -- (148.7000,262.3000) -- (148.7000,259.4000) -- (148.7000,256.5000) -- (148.7000,253.6000) -- (148.7000,250.8000) -- (148.9000,247.9000) -- (149.3000,245.0000) -- (149.3000,242.1000) -- (148.8000,239.3000) -- (147.8000,236.5000) -- (146.5000,233.8000) -- (144.7000,231.3000) -- (142.5000,229.1000) -- (139.9000,227.1000) -- (136.9000,225.5000) -- (133.7000,224.2000) -- (130.3000,223.4000) -- (126.9000,223.0000) -- (123.4000,223.1000) -- (119.9000,223.6000) -- (116.6000,224.5000) -- (113.5000,225.8000) -- (110.7000,227.5000) -- (108.2000,229.5000) -- (106.0000,231.7000) -- (104.2000,234.2000) -- (102.9000,236.9000) -- (102.1000,239.7000) -- (101.8000,242.6000) -- (101.7000,245.5000) -- (101.7000,248.4000) -- (101.7000,251.3000) -- (101.7000,254.2000) -- (101.7000,257.1000) -- (101.7000,260.0000) -- (101.7000,262.9000) -- (101.7000,265.8000) -- (101.7000,268.6000) -- (101.7000,271.5000) -- (101.7000,274.4000) -- (101.7000,277.3000) -- (101.7000,280.2000) -- (101.7000,283.1000) -- (101.9000,286.0000) -- (102.5000,288.8000) -- (103.6000,291.6000) -- (105.1000,294.2000) -- (107.0000,296.6000) -- (109.2000,298.9000) -- (111.8000,300.8000) -- (114.7000,302.4000) -- (117.8000,303.7000) -- (121.1000,304.6000) -- (124.5000,305.1000) -- (128.0000,305.2000) -- (131.5000,304.9000) -- (134.9000,304.1000) -- (138.2000,303.0000) -- (141.2000,301.4000) -- (144.0000,299.6000) -- (146.4000,297.4000) -- (148.4000,295.0000) -- (150.0000,292.4000) -- (150.9000,289.6000) -- (151.4000,286.8000) -- (151.7000,283.9000) -- (151.9000,281.1000) -- (152.0000,278.2000) -- (152.0000,275.3000) -- (152.0000,272.4000) -- (152.0000,269.5000) -- (152.0000,266.7000) -- (152.0000,263.8000) -- (152.0000,260.9000) -- (152.0000,258.0000) -- (152.0000,255.1000) -- (152.0000,252.2000) -- (152.1000,249.3000) -- (152.4000,246.4000) -- (152.7000,243.6000) -- (152.5000,240.7000) -- (151.8000,237.8000) -- (150.7000,235.1000) -- (149.2000,232.5000) -- (147.2000,230.1000) -- (144.8000,228.0000) -- (142.1000,226.1000) -- (139.0000,224.6000) -- (135.8000,223.5000) -- (132.3000,222.9000) -- (128.8000,222.7000) -- (125.3000,222.9000) -- (121.9000,223.6000) -- (118.7000,224.7000) -- (115.7000,226.1000) -- (113.0000,227.9000) -- (110.6000,230.0000) -- (108.6000,232.4000) -- (107.0000,235.0000) -- (105.8000,237.7000) -- (105.3000,240.6000) -- (105.1000,243.5000) -- (105.1000,246.4000) -- (105.1000,249.3000) -- (105.1000,252.2000) -- (105.1000,255.1000) -- (105.1000,257.9000) -- (105.1000,260.8000) -- (105.1000,263.7000) -- (105.1000,266.6000) -- (105.1000,269.5000) -- (105.1000,272.4000) -- (105.1000,275.3000) -- (105.1000,278.2000) -- (105.1000,281.1000) -- (105.1000,284.0000) -- (105.5000,286.8000) -- (106.3000,289.7000) -- (107.5000,292.4000) -- (109.1000,294.9000) -- (111.2000,297.3000) -- (113.6000,299.4000) -- (116.3000,301.2000) -- (119.2000,302.7000) -- (122.4000,303.8000) -- (125.8000,304.6000) -- (129.3000,304.9000) -- (132.8000,304.8000) -- (136.3000,304.3000) -- (139.6000,303.4000) -- (142.8000,302.1000) -- (145.7000,300.4000) -- (148.3000,298.5000) -- (150.6000,296.2000) -- (152.5000,293.7000) -- (153.8000,291.0000) -- (154.5000,288.2000) -- (154.9000,285.3000) -- (155.1000,282.5000) -- (155.3000,279.7000) -- (155.3000,276.8000) -- (155.4000,273.9000) -- (155.4000,271.0000) -- (155.4000,268.1000) -- (155.4000,265.2000) -- (155.4000,262.3000) -- (155.4000,259.4000) -- (155.4000,256.5000) -- (155.4000,253.6000) -- (155.4000,250.8000) -- (155.5000,247.9000) -- (155.9000,244.9000);



  \end{scope}
  \begin{scope}[cm={{1.00588,0.0,0.0,1.00588,(-526.48136,104.32102)}},draw=blue,line cap=rect,line join=bevel,line width=0.800pt]
  \end{scope}
\end{scope}
\begin{scope}[cm={{1.00588,0.0,0.0,1.00588,(10.36994,109.98172)}},draw=blue,line cap=rect,line join=bevel,line width=0.800pt]
  \path[fill=blue] (0.0000,0.0000) node[above right] (text34-9-1) {-100};



\end{scope}
\begin{scope}[cm={{1.15801,0.0,0.0,1.15801,(-19.81968,-10.65042)}},draw=blue,line cap=round,line join=round,line width=0.480pt]
  \path[draw] (44.5000,79.5000) -- (48.5000,79.5000);



  \path[draw] (142.5000,79.5000) -- (139.5000,79.5000);



\end{scope}
\begin{scope}[cm={{1.00588,0.0,0.0,1.00588,(21.10457,84.85183)}},draw=blue,line cap=rect,line join=bevel,line width=0.800pt]
  \path[fill=blue] (0.0000,0.0000) node[above right] (text64-6-5) {0};



\end{scope}
\begin{scope}[cm={{1.15801,0.0,0.0,1.15801,(-19.81968,-10.65042)}},draw=ca0a0a4,dash pattern=on 1.55pt off 1.55pt,line cap=round,line join=round,line width=0.259pt,miter limit=4.00]
  \path[draw,dash pattern=on 1.55pt off 1.55pt,line width=0.259pt,miter limit=4.00] (44.5000,57.5000) -- (142.5000,57.5000);



\end{scope}
\begin{scope}[cm={{1.15801,0.0,0.0,1.15801,(-19.81968,-10.65042)}},draw=blue,line cap=round,line join=round,line width=0.480pt]
  \path[draw] (44.5000,57.5000) -- (48.5000,57.5000);



  \path[draw] (142.5000,57.5000) -- (139.5000,57.5000);



\end{scope}
\begin{scope}[cm={{1.00588,0.0,0.0,1.00588,(13.05753,58.22243)}},draw=blue,line cap=rect,line join=bevel,line width=0.800pt]
  \path[fill=blue] (0.0000,0.0000) node[above right] (text94-7-3) {100};



\end{scope}
\begin{scope}[cm={{1.15801,0.0,0.0,1.15801,(-19.81968,-10.65042)}},draw=ca0a0a4,dash pattern=on 1.55pt off 1.55pt,line cap=round,line join=round,line width=0.259pt,miter limit=4.00]
  \path[draw,dash pattern=on 1.55pt off 1.55pt,line width=0.259pt,miter limit=4.00] (44.5000,35.5000) -- (142.5000,35.5000);



\end{scope}
\begin{scope}[cm={{1.15801,0.0,0.0,1.15801,(-19.81968,-10.65042)}},draw=blue,line cap=round,line join=round,line width=0.480pt]
  \path[draw] (44.5000,35.5000) -- (48.5000,35.5000);



  \path[draw] (142.5000,35.5000) -- (139.5000,35.5000);



\end{scope}
\begin{scope}[cm={{1.00588,0.0,0.0,1.00588,(13.05753,33.09303)}},draw=blue,line cap=rect,line join=bevel,line width=0.800pt]
  \path[fill=blue] (0.0000,0.0000) node[above right] (text124-9-9) {200};



\end{scope}
\begin{scope}[cm={{1.15801,0.0,0.0,1.15801,(-19.81968,-10.65042)}},draw=ca0a0a4,dash pattern=on 0.40pt off 0.80pt,line cap=round,line join=round,line width=0.400pt]
  \path[draw] (44.5000,13.5000) -- (142.5000,13.5000);



\end{scope}
\begin{scope}[cm={{1.15801,0.0,0.0,1.15801,(-19.81968,-10.65042)}},draw=blue,line cap=round,line join=round,line width=0.480pt]
  \path[draw] (44.5000,13.5000) -- (48.5000,13.5000);



  \path[draw] (142.5000,13.5000) -- (139.5000,13.5000);



\end{scope}
\begin{scope}[cm={{1.00588,0.0,0.0,1.00588,(13.05753,7.96363)}},draw=blue,line cap=rect,line join=bevel,line width=0.800pt]
  \path[fill=blue] (0.0000,0.0000) node[above right] (text154-93) {300};



\end{scope}
\begin{scope}[cm={{1.15801,0.0,0.0,1.15801,(-19.81968,-10.65042)}},draw=ca0a0a4,dash pattern=on 0.40pt off 0.80pt,line cap=round,line join=round,line width=0.400pt]
  \path[draw] (44.5000,102.5000) -- (44.5000,13.5000);



\end{scope}
\begin{scope}[cm={{1.15801,0.0,0.0,1.15801,(-19.81968,-10.65042)}},draw=blue,line cap=round,line join=round,line width=0.480pt]
  \path[draw] (44.5000,102.5000) -- (44.5000,99.5000);



  \path[draw] (44.5000,13.5000) -- (44.5000,16.5000);



\end{scope}
\begin{scope}[cm={{1.15801,0.0,0.0,1.15801,(-19.81968,-10.65042)}},draw=blue,line cap=round,line join=round,line width=0.480pt]
  \path[draw] (69.5000,102.5000) -- (69.5000,99.5000);



  \path[draw] (69.5000,13.5000) -- (69.5000,16.5000);



\end{scope}
\begin{scope}[cm={{1.15801,0.0,0.0,1.15801,(-19.81968,-10.65042)}},draw=ca0a0a4,dash pattern=on 1.55pt off 1.55pt,line cap=round,line join=round,line width=0.259pt,miter limit=4.00]
  \path[draw,dash pattern=on 1.55pt off 1.55pt,line width=0.259pt,miter limit=4.00] (93.5000,102.5000) -- (93.5000,27.5000);



  \path[draw,dash pattern=on 1.55pt off 1.55pt,line width=0.259pt,miter limit=4.00] (93.5000,19.5000) -- (93.5000,13.5000);



\end{scope}
\begin{scope}[cm={{1.15801,0.0,0.0,1.15801,(-19.81968,-10.65042)}},draw=blue,line cap=round,line join=round,line width=0.480pt]
  \path[draw] (93.5000,102.5000) -- (93.5000,99.5000);



  \path[draw] (93.5000,13.5000) -- (93.5000,16.5000);



\end{scope}
\begin{scope}[cm={{1.15801,0.0,0.0,1.15801,(-19.81968,-10.65042)}},draw=ca0a0a4,dash pattern=on 1.55pt off 1.55pt,line cap=round,line join=round,line width=0.259pt,miter limit=4.00]
  \path[draw,dash pattern=on 1.55pt off 1.55pt,line width=0.259pt,miter limit=4.00] (118.5000,102.5000) -- (118.5000,13.5000);



\end{scope}
\begin{scope}[cm={{1.15801,0.0,0.0,1.15801,(-19.81968,-10.65042)}},draw=blue,line cap=round,line join=round,line width=0.480pt]
  \path[draw] (118.5000,102.5000) -- (118.5000,99.5000);



  \path[draw] (118.5000,13.5000) -- (118.5000,16.5000);



\end{scope}
\begin{scope}[cm={{1.15801,0.0,0.0,1.15801,(-19.81968,-10.65042)}},draw=ca0a0a4,dash pattern=on 0.40pt off 0.80pt,line cap=round,line join=round,line width=0.400pt]
  \path[draw] (142.5000,102.5000) -- (142.5000,13.5000);



\end{scope}
\begin{scope}[cm={{1.15801,0.0,0.0,1.15801,(-19.81968,-10.65042)}},draw=blue,line cap=round,line join=round,line width=0.480pt]
  \path[draw] (142.5000,102.5000) -- (142.5000,99.5000);



  \path[draw] (142.5000,13.5000) -- (142.5000,16.5000);



\end{scope}
\begin{scope}[cm={{1.15801,0.0,0.0,1.15801,(-19.81968,-10.65042)}},draw=blue,line cap=round,line join=round,line width=0.480pt]
  \path[draw] (44.5000,13.5000) -- (44.5000,102.5000) -- (142.5000,102.5000) -- (142.5000,13.5000) -- (44.5000,13.5000);



\end{scope}
\begin{scope}[cm={{0.84029,0.0,0.0,0.84029,(41.25055,18.91027)}},draw=blue,line cap=rect,line join=bevel,line width=0.800pt]
  \path[fill=blue] (-0.9117,0.4559) node[above right] (text360-8) {\scriptsize $\mathbf{p}(t)$};



\end{scope}
\begin{scope}[cm={{1.15801,0.0,0.0,1.15801,(-27.33812,-11.90349)}},draw=blue,line cap=round,line join=round,line width=0.480pt]
  \path[draw,even odd rule] (74.5000,23.5000) -- (101.5000,23.5000);



\end{scope}
\begin{scope}[cm={{1.15801,0.0,0.0,1.15801,(-19.81968,-10.65042)}},draw=blue,line cap=round,line join=round,line width=0.480pt]
  \path[draw] (44.5000,13.5000) -- (44.5000,102.5000) -- (142.5000,102.5000) -- (142.5000,13.5000) -- (44.5000,13.5000);



\end{scope}
\begin{scope}[cm={{1.15801,0.0,0.0,1.15801,(-140.25839,106.3496)}},draw=ca0a0a4,dash pattern=on 0.40pt off 0.80pt,line cap=round,line join=round,line width=0.400pt]
  \path[draw] (148.5000,102.5000) -- (246.5000,102.5000);



\end{scope}
\begin{scope}[cm={{1.15801,0.0,0.0,1.15801,(-140.25839,106.3496)}},draw=blue,line cap=round,line join=round,line width=0.480pt]
  \path[draw] (148.5000,102.5000) -- (151.5000,102.5000);



  \path[draw] (246.5000,102.5000) -- (243.5000,102.5000);



\end{scope}
\begin{scope}[cm={{1.15801,0.0,0.0,1.15801,(-140.25839,106.3496)}},draw=ca0a0a4,dash pattern=on 1.55pt off 1.55pt,line cap=round,line join=round,line width=0.259pt,miter limit=4.00]
  \path[draw,dash pattern=on 1.55pt off 1.55pt,line width=0.259pt,miter limit=4.00] (148.5000,79.5000) -- (246.5000,79.5000);



\end{scope}
\begin{scope}[cm={{1.15801,0.0,0.0,1.15801,(-140.25839,106.3496)}},draw=blue,line cap=round,line join=round,line width=0.480pt]
  \path[draw] (148.5000,79.5000) -- (151.5000,79.5000);



  \path[draw] (246.5000,79.5000) -- (243.5000,79.5000);



\end{scope}
\begin{scope}[cm={{1.15801,0.0,0.0,1.15801,(-140.25839,106.3496)}},draw=ca0a0a4,dash pattern=on 1.55pt off 1.55pt,line cap=round,line join=round,line width=0.259pt,miter limit=4.00]
  \path[draw,dash pattern=on 1.55pt off 1.55pt,line width=0.259pt,miter limit=4.00] (148.5000,57.5000) -- (246.5000,57.5000);



\end{scope}
\begin{scope}[cm={{1.15801,0.0,0.0,1.15801,(-140.25839,106.3496)}},draw=blue,line cap=round,line join=round,line width=0.480pt]
  \path[draw] (148.5000,57.5000) -- (151.5000,57.5000);



  \path[draw] (246.5000,57.5000) -- (243.5000,57.5000);



\end{scope}
\begin{scope}[cm={{1.15801,0.0,0.0,1.15801,(-140.25839,106.3496)}},draw=ca0a0a4,dash pattern=on 1.55pt off 1.55pt,line cap=round,line join=round,line width=0.259pt,miter limit=4.00]
  \path[draw,dash pattern=on 1.55pt off 1.55pt,line width=0.259pt,miter limit=4.00] (148.5000,35.5000) -- (246.5000,35.5000);



\end{scope}
\begin{scope}[cm={{1.15801,0.0,0.0,1.15801,(-140.25839,106.3496)}},draw=blue,line cap=round,line join=round,line width=0.480pt]
  \path[draw] (148.5000,35.5000) -- (151.5000,35.5000);



  \path[draw] (246.5000,35.5000) -- (243.5000,35.5000);



\end{scope}
\begin{scope}[cm={{1.15801,0.0,0.0,1.15801,(-140.25839,106.3496)}},draw=ca0a0a4,dash pattern=on 0.40pt off 0.80pt,line cap=round,line join=round,line width=0.400pt]
  \path[draw] (148.5000,13.5000) -- (246.5000,13.5000);



\end{scope}
\begin{scope}[cm={{1.15801,0.0,0.0,1.15801,(-77.72572,106.3496)}},draw=ca0a0a4,dash pattern=on 1.55pt off 1.55pt,line cap=round,line join=round,line width=0.259pt,miter limit=4.00]
  \path[draw,dash pattern=on 1.55pt off 1.55pt,line width=0.259pt,miter limit=4.00] (118.5000,102.5000) -- (118.5000,13.5000);



\end{scope}
\begin{scope}[cm={{1.15801,0.0,0.0,1.15801,(-140.25839,106.3496)}},draw=blue,line cap=round,line join=round,line width=0.480pt]
  \path[draw] (148.5000,13.5000) -- (151.5000,13.5000);



  \path[draw] (246.5000,13.5000) -- (243.5000,13.5000);



\end{scope}
\begin{scope}[cm={{1.15801,0.0,0.0,1.15801,(-140.25839,106.3496)}},draw=ca0a0a4,dash pattern=on 0.40pt off 0.80pt,line cap=round,line join=round,line width=0.400pt]
  \path[draw] (148.5000,102.5000) -- (148.5000,13.5000);



\end{scope}
\begin{scope}[cm={{1.15801,0.0,0.0,1.15801,(-140.25839,106.3496)}},draw=blue,line cap=round,line join=round,line width=0.480pt]
  \path[draw] (148.5000,102.5000) -- (148.5000,99.5000);



  \path[draw] (148.5000,13.5000) -- (148.5000,16.5000);



\end{scope}
\begin{scope}[cm={{1.15801,0.0,0.0,1.15801,(-140.25839,106.3496)}},draw=blue,line cap=round,line join=round,line width=0.480pt]
  \path[draw] (172.5000,102.5000) -- (172.5000,99.5000);



  \path[draw] (172.5000,13.5000) -- (172.5000,16.5000);



\end{scope}
\begin{scope}[cm={{1.15801,0.0,0.0,1.15801,(-48.77541,106.3496)}},draw=ca0a0a4,dash pattern=on 1.55pt off 1.55pt,line cap=round,line join=round,line width=0.259pt,miter limit=4.00]
  \path[draw,dash pattern=on 1.55pt off 1.55pt,line width=0.259pt,miter limit=4.00] (118.5000,102.5000) -- (118.5000,13.5000);



\end{scope}
\begin{scope}[cm={{1.15801,0.0,0.0,1.15801,(-140.25839,106.3496)}},draw=blue,line cap=round,line join=round,line width=0.480pt]
  \path[draw] (197.5000,102.5000) -- (197.5000,99.5000);



  \path[draw] (197.5000,13.5000) -- (197.5000,16.5000);



\end{scope}
\begin{scope}[cm={{1.15801,0.0,0.0,1.15801,(-20.98311,106.3496)}},draw=ca0a0a4,dash pattern=on 1.55pt off 1.55pt,line cap=round,line join=round,line width=0.259pt,miter limit=4.00]
  \path[draw,dash pattern=on 1.55pt off 1.55pt,line width=0.259pt,miter limit=4.00] (118.5000,102.5000) -- (118.5000,13.5000);



\end{scope}
\begin{scope}[cm={{1.15801,0.0,0.0,1.15801,(-140.25839,106.3496)}},draw=blue,line cap=round,line join=round,line width=0.480pt]
  \path[draw] (221.5000,102.5000) -- (221.5000,99.5000);



  \path[draw] (221.5000,13.5000) -- (221.5000,16.5000);



\end{scope}
\begin{scope}[cm={{1.15801,0.0,0.0,1.15801,(-140.25839,106.3496)}},draw=ca0a0a4,dash pattern=on 0.40pt off 0.80pt,line cap=round,line join=round,line width=0.400pt]
  \path[draw] (246.5000,102.5000) -- (246.5000,27.5000);



  \path[draw] (246.5000,19.5000) -- (246.5000,13.5000);



\end{scope}
\begin{scope}[cm={{1.15801,0.0,0.0,1.15801,(-140.25839,106.3496)}},draw=blue,line cap=round,line join=round,line width=0.480pt]
  \path[draw] (246.5000,102.5000) -- (246.5000,99.5000);



  \path[draw] (246.5000,13.5000) -- (246.5000,16.5000);



\end{scope}
\begin{scope}[cm={{1.15801,0.0,0.0,1.15801,(-140.25839,106.3496)}},draw=blue,line cap=round,line join=round,line width=0.480pt]
  \path[draw] (148.5000,13.5000) -- (148.5000,102.5000) -- (246.5000,102.5000) -- (246.5000,13.5000) -- (148.5000,13.5000);



\end{scope}
\begin{scope}[cm={{0.84029,0.0,0.0,0.84029,(-93.49162,81.25858)}},fill=cd9d9d9]
  \path[rounded corners=0.0000cm] (222.0000,18.0000) rectangle (238.0000,34.0000);



\end{scope}
\begin{scope}[cm={{1.15801,0.0,0.0,1.15801,(-140.25839,106.3496)}},draw=blue,line cap=round,line join=round,line width=0.480pt]
  \path[draw] (148.5000,13.5000) -- (148.5000,102.5000) -- (246.5000,102.5000) -- (246.5000,13.5000) -- (148.5000,13.5000);



\end{scope}
\begin{scope}[cm={{1.00588,0.0,0.0,1.00588,(25.10681,236.06903)}},draw=blue,line cap=rect,line join=bevel,line width=0.800pt]
  \path[fill=blue] (0.0000,0.0000) node[above right] (text1768-1-4) {-150};



\end{scope}
\begin{scope}[cm={{1.00588,0.0,0.0,1.00588,(52.31861,236.06903)}},draw=blue,line cap=rect,line join=bevel,line width=0.800pt]
  \path[fill=blue] (0.0000,0.0000) node[above right] (text1798-7-9) {-50};



\end{scope}
\begin{scope}[cm={{1.00588,0.0,0.0,1.00588,(82.52181,236.06903)}},draw=blue,line cap=rect,line join=bevel,line width=0.800pt]
  \path[fill=blue] (0.0000,0.0000) node[above right] (text1828-9-59) {50};



\end{scope}
\begin{scope}[cm={{1.00588,0.0,0.0,1.00588,(107.70381,236.06903)}},draw=blue,line cap=rect,line join=bevel,line width=0.800pt]
  \path[fill=blue] (0.0000,0.0000) node[above right] (text1858-6-5) {150};



\end{scope}
\begin{scope}[cm={{1.00588,0.0,0.0,1.00588,(133.32142,236.06903)}},draw=blue,line cap=rect,line join=bevel,line width=0.800pt]
  \path[fill=blue] (0.0000,0.0000) node[above right] (text1858-6-6-08) {250};



\end{scope}
\begin{scope}[cm={{1.00588,0.0,0.0,1.00588,(10.64827,227.36002)}},draw=blue,line cap=rect,line join=bevel,line width=0.800pt]
  \path[fill=blue] (0.0000,0.0000) node[above right] (text34-9-6-9) {-100};



\end{scope}
\begin{scope}[cm={{1.00588,0.0,0.0,1.00588,(21.3829,202.23013)}},draw=blue,line cap=rect,line join=bevel,line width=0.800pt]
  \path[fill=blue] (0.0000,0.0000) node[above right] (text64-6-8-7) {0};



\end{scope}
\begin{scope}[cm={{1.00588,0.0,0.0,1.00588,(13.33586,175.60074)}},draw=blue,line cap=rect,line join=bevel,line width=0.800pt]
  \path[fill=blue] (0.0000,0.0000) node[above right] (text94-7-9-2) {100};



\end{scope}
\begin{scope}[cm={{1.00588,0.0,0.0,1.00588,(13.33586,150.47134)}},draw=blue,line cap=rect,line join=bevel,line width=0.800pt]
  \path[fill=blue] (0.0000,0.0000) node[above right] (text124-9-0-3) {200};



\end{scope}
\begin{scope}[cm={{1.00588,0.0,0.0,1.00588,(13.33586,125.34194)}},draw=blue,line cap=rect,line join=bevel,line width=0.800pt]
  \path[fill=blue] (0.0000,0.0000) node[above right] (text154-3-9) {300};



\end{scope}

\end{tikzpicture}


  \vspace*{-.6ex}
  {\color{blue}
  Fig.~7:~Planning-scheduling of CPP and ground patterns detections% with PedNet CNN
  , utilizing the lowest configuration {\color{red}I} as a starting point in {\color{red}i} and the highest {\color{red}II} in {\color{red}ii} while varying atmospheric (same as Fig.~{\color{red}6}) and battery conditions. In {\color{red}a} are the re-planned trajectories, in {\color{red}b} the parameters, and in {\color{red}c} the energy w.r.t. the battery.}
  \vspace*{1ex}
\end{formal}
}

{\hspace*{-4.5em}\textbf{[R1:5]}\vspace*{-1.9em}}

5. In line 26, shouldn’t the re-planning take place when $t_r>t_b$? In line 27, it is unclear how
re-planning is performed. I would suggest having more discussion on this.

{\hspace*{-4.5em}{[R1:5]}\vspace*{-1.9em}}

{\color{blue} 

{\hspace*{-4.5em}{[R1:5]}\vspace*{-1.9em}}

Thank you for noting the issue as well as for suggesting more discussion. The former is a typo, and we have fixed the if statement in Algorithm 2.

\begin{formal}
  \begin{algorithmic}[1]
    \small
      \makeatletter
      \setcounter{ALC@line}{24}
      \makeatother
      \IF{$t_r>t_b$}
        \color{blue}\STATE $c_i^{\rho}(t)\gets${ find }$c_i^{\rho}${ with }$t_r\in[0,t_b]${, otherwise take }$\underline{c}_i^\rho$\vspace*{.3ex}\label{alg:traj2}
      \color{black}\ENDIF
      \vspace*{.8ex}
    \end{algorithmic}
  \end{formal}  

  To clarify the re-planning of the path parameters, we have added explanatory text in Section IV-B.

  \begin{formal}
    \color{black} [...] The {\color{blue}path parameters and thus the} coverage is then re{\color{blue}-}planned [...] on Lines~{\color{red}24}--{\color{red}26} [...] {\color{blue}Concretely, these lines implement the greedy approach by decreasing the path parameters of a given value $\delta_i$ or similarly increasing the parameters when $t_r\leq t_b$ within the bounds (this latter analogous case is not shown explicitly in Algorithm~{\color{red}2} but implemented in Sec.~{\color{red}V})}.
    \vspace*{1ex}
  \end{formal}

  We have then detailed the actual value of $\delta_i$ in Section~V.

  \begin{formal}
    \color{black} [...] The set of parameters is unaltered through the flight, i.e, $c_i:=\begin{bmatrix}c_{i,1}&c_{i,2}\end{bmatrix}',\forall i${\color{blue}, along $\delta_i$ %utilized 
    in the greedy approach}.
    
    [...] scaling factors are derived empirically %, 
    {\color{blue}similarly to $\delta_i$ set to two hundred fifty}
    \vspace*{1ex}
  \end{formal}
}

{\hspace*{-4.5em}\textbf{[R1:6]}\vspace*{-1.9em}}

6. The following paper is related to this letter and the authors could consider referencing it.
\begin{itemize}
  \item Di Franco, Carmelo, and Giorgio Buttazzo. ``Coverage path planning for UAVs photogrammetry with energy and resolution constraints.'' Journal of Intelligent Robotic Systems 83, no. 3 (2016): 445-462.
\end{itemize}

{\color{blue} 

{\hspace*{-4.5em}{[R1:6]}\vspace*{-1.9em}}

Thank you for proposing Di Franco and Buttazzo's past work to our attention. We have considered another past study by the authors ``Energy-aware coverage path planning of UAVs'' from 2015, and the proposed work appears to be an extension. Both studies do not account for the energy contribution of the computations but interestingly propose the variability of the cover by changing the distance of the survey lines in the boustrophedon motion. Whilst similar to our Zamboni-like motion, the cover variability is not achieved in flight nor applies to constrained aerial robots such as fixed wings--which we consider due to a narrower computation-motion energy difference--but rather rotary wings (they utilize boustrophedon motion).

We have updated the introductory section to include the work comparing both the studies to ours.

\begin{formal}
  \color{black} [...] {\color{blue}In terms of aerial coverage, past work considers criteria including the completeness of the coverage and resolution [{\color{green}25}], and deals with aspects such as the quality of the cover [{\color{green}26}], but neglects the energy expenditure of computations and favors rotary-wing aerial robots rather than aerial robots broadly.} Such a state of practice has prompted us to propose the planning-scheduling approach for autonomous aerial robots [...]
  \vspace*{1ex}
\end{formal}

We have also updated Section IV-A to explicitly state that the variability of the cover is already considered for rotary wings utilizing the boustrophedon motion.

\begin{formal}
  \color{black} [...] this section details a [...] motion with a wide turning radius. It is similar to another motion in the literature, the Zamboni
  motion [{\color{green}42}], but additionally allows variable CPP [...] {\color{blue} Although cover variability is already considered in the literature [{\color{green}25}], it is limited to boustrophedon motion for rotary wings.} The novel motion is termed Zamboni-like motion [...]
  \vspace*{1ex}
\end{formal}
}


\vspace{2em}


Minor Comments:

{\hspace*{-4.5em}\textbf{[R1:7]}\vspace*{-1.9em}}

1. In Definition II.1, it would be useful to clarify the use of $j$ and $k$ in the formula of $\Gamma_i$. I assume the authors are saying $c^\rho_i=[c_{i,1},c_{i,2},\dots,c_{i,\rho}]$ and $c^\sigma_i=[c_{i,1},c_{i,2},\dots,c_{i,\sigma}]$, but the definition of $\Gamma_i$ is not quite clear.

{\color{blue} 

{\hspace*{-4.5em}{[R1:7]}\vspace*{-1.9em}}

Thank you for underlying this issue regarding the definition of stage. We are using the indices $j$ and $k$ to indicate that there might be a different constraint per each parameter and the numbers of parameters $\rho$ and $\sigma$ to indicate that there might be a different number of parameters. Nonetheless, we acknowledge the necessity to exemplify these in a better manner.

We have altered the definition for this purpose.

\begin{formal}
  \color{black} 
  \textbf{Definition}~II.1~(Stage)\textbf{.}~[...] the $i$th \emph{stage} $\Gamma_i$ %at time instant $t$ %of a plan $\Gamma$ 
    is
  \begin{equation*}\begin{split}
      \Gamma_i:=\{{\color{blue}\varphi_i(\mathbf{p}%(t)
      ,c_i^\rho)},c_i^\sigma\mid
      \,&\forall j\in\,[\rho]_{>0},\,c_{i,j}\,\,\,\,\,\,\,\in\mathcal{C}_{i,j},\,\\
        &\forall k\in[\sigma]_{>0},\,c_{i,\rho+k}\in\mathcal{S}_{i,k}\,\},
  \end{split}\end{equation*}
  where $c_i^\rho${\color{blue}$:=\{c_{i,1},c_{i,2},\dots,c_{i,\rho}\}$} and $c_i^\sigma${\color{blue}$:=\{c_{i,\rho+1},c_{i,\rho+2},\dots,$ $c_{i,\rho+\sigma}\}$} are $\rho$ \emph{path} and $\sigma$ \emph{computation parameters}{\color{blue}, e.g., $c_i^\rho:=\{c_{i,1}\}$ is a value that changes the distance of the coverage lines and $c^\sigma_i:=\{c_{i,2}\}$ the detection rate with $\rho$ and $\sigma$ being one (see Section~{\color{red}V})}. $\mathcal{C}_{i,j}:=[\underline{c}_{i,j},\overline{c}_{i,j}]\subseteq\mathbb{R}$ is the $j$th path parameter %$c_{i,j}$ 
  constraint set, and $\mathcal{S}_{i,k}:=[\underline{c}_{i,\rho+k},\overline{c}_{i,\rho+k}]\subseteq\mathbb{Z}_{\geq 0}$ is the $k$th computation parameter constraint set. {\color{blue}Indices $j$ and $k$ serves to differentiate path and computation parameters constraints and to indicate that each parameter can have a different constraint set.}
  \vspace*{1ex}
\end{formal}

}

{\hspace*{-4.5em}\textbf{[R1:8]}\vspace*{-1.9em}}

2. By definition $[\rho]=\{0,1,2,...,\rho\}$ is a $\rho+1$ tuple, implying that $c^\rho_i$, as the second variable of the path function, is a $\rho+1$ vector in Definition II.1. However, in Definition II.2, it is claimed that the second variable of the path function is a $\rho$-vector.

{\color{blue} 

{\hspace*{-4.5em}{[R1:8]}\vspace*{-1.9em}}

Thank you for noticing this inconsistency in the notation. We have utilized $[\rho]_{>0}$ with the subscript to indicate a $\rho$ tuple. We have updated the explanatory text after the definition of the stage accordingly.

\begin{formal}
  \color{black}
  [...] The notation $[x]$ denotes positive naturals up to $x$, i.e., $\{0,1,\dots,x\}$, {\color{blue}$[x]_{>0}$ strictly positive naturals, i.e., $\{1,2,\dots,x\}$,} [...]
  \vspace*{1ex}
\end{formal}
}

{\hspace*{-4.5em}\textbf{[R1:9]}\vspace*{-1.9em}}

3. I noticed a typo under Definition II.1. ``\dots is the $j$th path parameter $c_{i,j}$ constraint set'', should $c_{i,j}$ be removed?

{\color{blue} 

{\hspace*{-4.5em}{[R1:9]}\vspace*{-1.9em}}

Thank you for noticing the typo. We have removed $c_{i,j}$ from Definition II.1.}

{\hspace*{-4.5em}\textbf{[R1:10]}\vspace*{-1.9em}}

4. In Definition II.1, the use of $\mathbf{p}(t)$ is confusing. I assume $\mathbf{p}$ is an arbitrary point on the path, but $\mathbf{p}(t)$ usually refers to a trajectory.

{\color{blue} 

{\hspace*{-4.5em}{[R1:10]}\vspace*{-1.9em}}

Thank you for underlying the issue. We have not realized this possible source of additional confusion and decided to utilize only $\mathbf{p}$ to designate a point. We have altered the occurrence in the remainder of the work, for consistency. Apart from Definition II.1 that we showed in point 7, the occurrence was in the definition of path functions.

\begin{formal}
  \color{black}
  \textbf{Definition~II.2~}(Path~functions)\textbf{.}~$\varphi_i:\mathbb{R}^2\times\mathbb{R}^\rho\rightarrow\mathbb{R},\,\forall i\in\{1,2,\dots\}
  $ are \emph{path functions}, forming the path. They are a function of {\color{blue}$\mathbf{p}%(t)
  $} and path parameters $c_i^\rho%(t)
  $ and are continuous and twice differentiable.
  \vspace*{1ex}
\end{formal}

In the state-transition function.

\begin{formal}
  \color{black}
  [...] the state-transition function $s:\bigcup_i{\Gamma_i}\times\mathbb{R}^2\rightarrow\bigcup_i{\Gamma_i}$ maps a stage and a point to the next stage
  \begin{equation*}{\color{blue}s(\Gamma_i,\mathbf{p}%(t)
    )}:=\begin{cases}
    \Gamma_{i+j} & {\color{blue}\text{if }\norm{\mathbf{p}%(t)
    -\mathbf{p}_{\Gamma_i}}<\varepsilon_i,\,\exists j\in\mathbb{Z},}\\
    \Gamma_i & \text{otherwise}.
  \end{cases}\end{equation*}
\end{formal}

Finally in Algorithm 1 in lines 2 and 3.

\begin{formal}
  \begin{algorithmic}[1]
    \small
      \makeatletter
      \setcounter{ALC@line}{1}
      \makeatother
      \STATE \textbf{if} $\mathbf{p}%(t)
      =\mathbf{p}_{\Gamma_l}${ in Definition~{\color{red}II.3}} \textbf{then return }$\Gamma$\vspace*{.3ex}
      \vspace*{.8ex}
      \STATE \textbf{if} $\mathbf{p}%(t)
      =\mathbf{p}_{\Gamma_i}$ \textbf{then}
    \end{algorithmic}
  \end{formal}  

}



{\hspace*{-4.5em}\textbf{[R1:11]}\vspace*{-1.9em}}

5. I noticed a typo ``computations parameters'', I think it should be ``computation parameters''

{\color{blue} 

{\hspace*{-4.5em}{[R1:11]}\vspace*{-1.9em}}

Thank you once again for noticing the typo. We have changed the occurrences of ``computations parameters'' with ``computation parameters'', whilst keeping computations when the occurrence indicates multiple computations or when we utilize it to underline, e.g., the energy or schedule of multiple computations.}

{\hspace*{-4.5em}\textbf{[R1:12]}\vspace*{-1.9em}}

6. Could you please explain how you obtain matrix $A$ and $C$ in (6) and (7), or put some reference?

{\color{blue} 

{\hspace*{-4.5em}{[R1:12]}\vspace*{-1.9em}}

Thank you for noticing that a citation and clarification are missing regarding the energy models. The items of matrices $A$ and $C$ are constructed so that the two periodic models, the model in Equation ({\color{red}3}) and Equation ({\color{red}4}) are equal under some conditions (the value of the initial guess is described by Equation ({\color{red}9}) and the model is not perturbed by the control).

We have elaborated further in Section III-A to reflect the observation and added a reference to the corresponding author's previous work.

\begin{formal}
  {\color{black}
  [...] {\color{blue}Matrices $A$ and $C$ are constructed such that t}he %Under favorable conditions, %the 
  models in Eqs.~({\color{red}3}--{\color{red}4}) {\color{black}are equal when $\mathbf{u}$ is a {\color{black}zero vector %, matrices $A,C$ described by Eqs.~(\ref{eq:mat_A}--\ref{eq:mat_C}), 
  and an initial guess $\mathbf{q}(t_0)=\mathbf{q}_0$ at initial time instant $t_0$}}
    {\color{blue}\begin{equation}\tag{9}
    {\color{black}\mathbf{q}_0=\begin{bmatrix}a_0 & a_1/2 & b_1/2 & \cdots & a_r/2 & b_r/2\end{bmatrix}',}
    \end{equation}}
    %$h$ in Eq.~(\ref{eq:fourier}) is equal to $y$ in Eq.~(\ref{eq:state-perf}).
  %\end{lem}
  %\begin{proof}
  %The equality of the signal and output is achieved by a proper choice of the items of matrices $A,C$ and the initial guess $\mathbf{q}_0$. We refer the reader to Appendix~\ref{app:proof-eqv} for a formal proof, where we justify the choices of the items of the matrices and of the initial guess. 
  %\end{proof}
  %Appendix~\ref{app:proof-eqv} contains a proof of Lem.~\ref{lem:eqv}.
  \color{black}i.e., $h,y$ are both harmonic signals with the same frequencies{\color{blue}~[{\color{green}33}]}.}
  \vspace*{1ex}
\end{formal}
}

{\hspace*{-4.5em}\textbf{[R1:13]}\vspace*{-1.9em}}

7. In line 16 of Algorithm 2, $\mathcal{K}={t,t+h,\dots,t+N}$. It should be ${t,t+1,\dots,t+N}$?

8. The definition of $h$ in algorithm 2 is not clear with ``the sets $\mathcal{K},\mathcal{T}$ have possibly different steps $h$''. Could you clarify how $h$ is obtained? Is $h$ set to 1, or specified by users?

{\color{blue} 

{\hspace*{-4.5em}{[R1:13]}\vspace*{-1.9em}}

Thank you for underlying the necessity of further explanation for the steps in $\mathcal{K},\mathcal{T}$ in both the points 7 and 8. The step for $\mathcal{T}$ is utilized to decide the granularity of re-planning, i.e., how often does the aerial robot finds the configurations of path and computations parameters, whereas the step in $\mathcal{K}$ denotes the integration step for the models in MPC.

Practically, we use a step of one for $\mathcal{T}$, running the re-planning every second, and an integration step of 1/100 fraction of a second, since we express $t_f$ and $N$ both in seconds. We find the parameters so that the re-planning is feasible in real-time and that the numerical simulation does not diverge.

This was not clear in the text, so we have updated Section IV-B according to the observation.

\begin{formal}
  \color{black} [...] Here, the sets $\mathcal{K},\mathcal{T}$ have possibly different steps $h$ (not to be confused with the altitude){\color{blue}: 
  the set $\mathcal{K}$ is used for the numerical simulation, whereas $\mathcal{T}$ is for re-planning, meaning that $h$ tunes the precision and the frequency of re-planning in $\mathcal{K}$, $\mathcal{T}$ respectively.}
  \vspace*{1ex}
\end{formal}

We have then updated Section V.


\begin{formal}
  \color{black} [...] {\color{blue} $h$ is set to one-hundredth of a second and to one second in $\mathcal{K}, \mathcal{T}$ respectively to allow sufficient precision and re-planning online.}
  \vspace*{1ex}
\end{formal}


}

{\hspace*{-4.5em}\textbf{[R1:14]}\vspace*{-1.9em}}

9. By Definition III.2, $g(\cdot)$ has two variables, but equation (12) only has one variable. I assume a time variable is missing

{\color{blue} 

{\hspace*{-4.5em}{[R1:14]}\vspace*{-1.9em}}

[!]}

{\hspace*{-4.5em}\textbf{[R1:15]}\vspace*{-1.9em}}

10. There is a typo in the axis title of parameters in Fig. 6(a). The title should be $c_{i,1}$ instead of k$c_{i,1}$

{\color{blue} 

{\hspace*{-4.5em}{[R1:15]}\vspace*{-1.9em}}

Thank you for noticing the typo. We have fixed the axis title.
}

{\hspace*{-4.5em}\textbf{[R1:16]}\vspace*{-1.9em}}

11. The way that the authors define $\tau_{i,j}$ and $\nu_{i,j}$ in (10a) and (10b) implies path parameter $c_{i,j}$, $j\in\{1,2,\dots,\rho\}$ contributes equally to $t_s$ (defined in line 24 in Algorithm 2). It is difficult to determine if this assumption is reasonable, since $\rho=1$ in the numerical simulations.

{\color{blue} 

{\hspace*{-4.5em}{[R1:16]}\vspace*{-1.9em}}

[!]}













\newpage

{Second reviwew}

\vspace{3em}

{\hspace*{-4.5em}\textbf{[R2]}\vspace*{-1.9em}}

Contributions:

The contributions of this paper are 1) an energy model that predicts the impact of the effect of changes to the path or computation energy on the battery in the future 2) introduction of Zamboni-like motion patterns that can be replanned while on the path itself 3) an optimal control problem and solution to solve the re-planning coverage problem to decide how to change the computations and the path to optimize the energy usage

{\color{blue} 


{\hspace*{-4.5em}{[R2]}\vspace*{-1.9em}}

We would like to thank the second referee, especially for the summary of our work and the kind commentary on the relevance and difficulty of accounting for both motion and computations energies in planning for aerial robots. Indeed, we believe that planning-scheduling as a sub-topic of motion planning is still very nascent for aerial robots and hope that this work continues to build upon previous successes in planning-scheduling for mobile robots broadly. To that end, we are excited to comment on and alter the manuscript as necessary, as the second referee brings up a lot of great points.


}

\vspace{2em}

Related work:

- Good and varied connections made to situate this work in the field. 

{\hspace*{-4.5em}\textbf{[R2:1]}\vspace*{-1.9em}}

- Not made clear how this is different from other relevant works such as work cited ``C. Di Franco and G. Buttazzo, ``Energy-aware coverage path planning of UAVs,'' in Int. Conf. on Autonomous Robot Syst. and Competitions. IEEE, 2015, pp. 111–117''; especially when stating ``These studies are focused on ground-based robots [9], [19], [23], [24], yet, aerial robots are particularly affected by energy considerations, as it would be generally required to land to recharge the battery''; since this is not the first with aerial robots, what about this research is different from other work on energy aware CPP? Summarize clearly differences in related work in UAV literature.

{\color{blue} 

{\hspace*{-4.5em}{[R2:1]}\vspace*{-1.9em}}

Thank you for proposing Di Franco and Buttazzo's past study to our attention. We have considered the work in the original manuscript in terms of the coverage algorithm in Section IV-A, utilizing it as a reference for aerial coverage. We have not included it in the introductory material as the work does not account for the energy expenditure of computations, nonetheless, we agree on the necessity to clarify this in the introduction. To this end, we have added explanatory text in the introduction that--aside the proposed work--includes the follow-up work by the same set of authors proposed by the first referee.

\begin{formal}
  \color{black} [...] {\color{blue}In terms of aerial coverage, past work considers criteria including the completeness of the coverage and resolution [{\color{green}25}], and deals with aspects such as the quality of the cover [{\color{green}26}], but neglects the energy expenditure of computations and favors rotary-wing aerial robots rather than aerial robots broadly.} Such a state of practice has prompted us to propose the planning-scheduling approach for autonomous aerial robots [...]
  \vspace*{1ex}
\end{formal}

We have then explicitly stated the work in Section~IV-A
in terms of the variability of the cover.

\begin{formal}
  \color{black} [...] this section details a [...] motion with a wide turning radius. It is similar to another motion in the literature, the Zamboni
  motion [{\color{green}42}], but additionally allows variable CPP [...] {\color{blue} Although cover variability is already considered in the literature [{\color{green}25}], it is limited to boustrophedon motion for rotary wings.} The novel motion is termed Zamboni-like motion [...]
  \vspace*{1ex}
\end{formal}

}

\vspace{2em}

Strengths:

- This paper addresses a very interesting and practical issue when using UAVs: how to change the behavior of the fixed-wing UAV computations and path w.r.t. to how much battery is left to optimize energy usage.

- Clear preliminaries section helps with understanding technical detail.

- Synthesizes techniques and ideas from various different topics/domains into the challenging replanning coverage problem.

- Technical solution of optimal control problem to manage both computing workloads and path plan interesting and relevant contribution.
  
- Good simulations with modeling different wind speeds/directions.

\vspace{2em}

Weaknesses:

  
{\hspace*{-4.5em}\textbf{[R2:2]}\vspace*{-1.9em}}
  
- High-level comment: the paper would benefit from restructuring for clarity to better convey the interesting technical ideas. At present, difficult to understand the insights behind the main contributions. For example, ``the re-planning-scheduling problem is finding the optimal trajectory of parameters $c_i$ in time.'' -- optimal w.r.t. what objective function? Total energy? Mission duration? Make this clear in the problem definition section. 
  
{\color{blue} 

{\hspace*{-4.5em}{[R2:2]}\vspace*{-1.9em}}

Thank you for the high-level comment and for proposing to state clearly the objective function. Regarding the objective function, we have integrated the problem definition accordingly, whereas we have altered the text in the manuscript by both implementing the specific comments and rewriting some technical details or adding further explanatory commentary. 

\begin{formal}
  {\color{black} [...] the \emph{re-planning-scheduling problem} is finding the {\color{blue}energy-aware} trajectory of parameters $c_i$ in time{\color{blue}, optimizing battery state of charge (SoC)}.}
  \vspace*{1ex}
\end{formal}
}
  
  {\hspace*{-4.5em}\textbf{[R2:3]}\vspace*{-1.9em}}

  Also, ``The energy optimization of computations schedules can be achieved by, e.g., varying the quality of service between specific bounds [12] and frequency and voltage of the computing hardware [9], [13], [14]. We focus on the former aspect and schedule the onboard computations altering their quality while simultaneously changing the quality of the coverage'' --$>$ What are the actual computation inputs that can be changed? It is unclear throughout the paper what $c_i^\sigma$ is. 
  
  {\color{blue} 
  
  {\hspace*{-4.5em}{[R2:3]}\vspace*{-1.9em}}
  
  Thank you for underlying the need to state clearly what the computations parameters refer to. We have not realized this point was not stated explicitly in the original version earlier than in the experimental setup. Our purpose is to alter the granularity at which different computations run so that the energy consumption required by the computing hardware onboard the aerial robot increases or decreases.

  In our experimental setup, these are the detection rate--meaning we vary the frequency at which a convolutional neural network trained to detect specific hazards runs from two to ten frames per second. These two values are determined empirically, by sampling different boundary configurations and their energy consumption via the modeling tool we describe in Section III-B. We have varied other parameters as well, such as the encryption, utilizing a binary value simply indicating whether the encryption is or is not enabled, and the size of the key in a variable key-size encryption algorithm. However, we have not observed a significant energy drain changing these parameters, conversely to the parameter detection rate. Thus we utilized merely the computation parameter 'detection rate' in our current experimental setup. 

  We have updated the introductory material to state concretely what computational input is being varied.

  \begin{formal}\color{black}
    [...] The energy optimization of computations schedules can be achieved by, e.g., varying the quality of service between specific bounds~[{\color{green}10}] and frequency and voltage of the computing hardware~[{\color{green}7}], [{\color{green}11}], [{\color{green}12}]. We focus on the former aspect and schedule the onboard computations altering their quality while simultaneously changing the quality of the coverage.
    {\color{blue} Concretely, we alter how often the aerial robot detects ground patterns along with the distance of the lines that form the coverage.} [...]
    \vspace*{1ex}
  \end{formal}

  We have further included the concrete meaning of parameters in Section II.

  \begin{formal}
    \color{black} 
    \textbf{Definition}~II.1~(Stage)\textbf{.}~[...] the $i$th \emph{stage} $\Gamma_i$ %at time instant $t$ %of a plan $\Gamma$ 
      is
    \begin{equation*}\begin{split}
        \Gamma_i:=\{{\color{blue}\varphi_i(\mathbf{p}%(t)
        ,c_i^\rho)},c_i^\sigma\mid
        \,&\forall j\in\,[\rho]_{>0},\,c_{i,j}\,\,\,\,\,\,\,\in\mathcal{C}_{i,j},\,\\
          &\forall k\in[\sigma]_{>0},\,c_{i,\rho+k}\in\mathcal{S}_{i,k}\,\},
    \end{split}\end{equation*}
    where $c_i^\rho${\color{blue}$:=\{c_{i,1},c_{i,2},\dots,c_{i,\rho}\}$} and $c_i^\sigma${\color{blue}$:=\{c_{i,\rho+1},c_{i,\rho+2},\dots,$ $c_{i,\rho+\sigma}\}$} are $\rho$ \emph{path} and $\sigma$ \emph{computation parameters}{\color{blue}, e.g., $c_i^\rho:=\{c_{i,1}\}$ is a value that changes the distance of the coverage lines and $c^\sigma_i:=\{c_{i,2}\}$ the detection rate with $\rho$ and $\sigma$ being one (see Section~{\color{red}V})}. [...]
    \vspace*{1ex}
  \end{formal}



  }

  {\hspace*{-4.5em}\textbf{[R2:4]}\vspace*{-1.9em}}

  In addition, what is the performance metric we want to observe improving using the algorithm 2 in the numerical experiments (comparing i, ii to I, II baselines)? 
  
  {\color{blue} 
  
  {\hspace*{-4.5em}{[R2:4]}\vspace*{-1.9em}}

  [!]}

{\hspace*{-4.5em}\textbf{[R2:5]}\vspace*{-1.9em}}

- Paper would benefit from clear indication of what the performance metric we want to see improve. 
  
{\color{blue} 
  
{\hspace*{-4.5em}{[R2:5]}\vspace*{-1.9em}}
  
Thank you for suggesting adding a clear indication of what performance metrics are being utilized. Indeed this information was missing in the original manuscript. We have added explanatory text in Problem Formulation to address the point.

\begin{formal}\color{black}
  The problem %of planning-scheduling 
is composed of two sub-problems{\color{blue}: }%. One is 
to form a %static 
plan that visits every point in space, {\color{blue}and }%the other to
re-plan and re-schedule the plan in-flight in an energy-aware way. %{\color{blue}, whereas }
{\color{blue} The performance metric the re-planning-scheduling %in the letter 
improves is %then
the quality of the plan-schedule against battery SoC, e.g., the weighted average value of parameters divided by remaining %battery 
SoC.% at the end of the flight.
\vspace*{1ex}
}
\end{formal}
}

  {\hspace*{-4.5em}\textbf{[R2:6]}\vspace*{-1.9em}}

  Average results from more numeric experiments (e.g., average flights competed on baseline vs. with new algorithm) would also really benefit the paper. Paper would also be greatly strengthened by actual physical experiments on UAV.
  
  {\color{blue} 
  
  {\hspace*{-4.5em}{[R2:6]}\vspace*{-1.9em}}
  
  [!]}

{\hspace*{-4.5em}\textbf{[R2:7]}\vspace*{-1.9em}}

- ``the Zamboni motion [40], but additionally allows variable CPP at the very core of this work.'' -- need to clarify more why Algorithm 1/Zamboni-like motion is different from Zamboni motion. What is the   insight that allows the Zamboni motion to be recalculated/replanned at each stage that isn't possible with regular Zamboni motion in reference [40]? 

{\color{blue} 
  
{\hspace*{-4.5em}{[R2:7]}\vspace*{-1.9em}}
  
[!]}


\vspace{2em}

Specific edits/typos: 

{\hspace*{-4.5em}\textbf{[R2:8]}\vspace*{-1.9em}}

- ``Such use cases arise in, e.g., precision agriculture [4], where harvesting involves ground vehicles [5], [6], information collection prior to an operation as well as damage prevention during the operation involve aerial robots'' is unclear/grammatically incorrect. --$>$	``Such use cases arise in precision agriculture [4] where information collection prior to an harvesting operation and damage prevention during the operation involve aerial robots''
  
{\color{blue}

{\hspace*{-4.5em}{[R2:8]}\vspace*{-1.9em}}

Thank you for spotting the issue in the introductory section. We have corrected the sentence.

\begin{formal}
  \color{black} [...] Such use cases arise in precision agriculture~[{\color{green}4}] where {\color{blue}information collection prior to an harvesting operation and} damage prevention during the operation involve aerial robots~[{\color{green}5}],~[{\color{green}6}].

  \vspace*{1ex}
\end{formal}

}

{\hspace*{-4.5em}\textbf{[R2:9]}\vspace*{-1.9em}}

- Figure 1 not clear; are we supposed to be able to differentiate between i, ii, and iii paths? Not able to differentiate currently, so unable to understand key takeaway of figure. What is the meaning of the colors (green for i, red for ii/iii)?
  
{\color{blue} 

{\hspace*{-4.5em}{[R2:9]}\vspace*{-1.9em}}

[!]}

{\hspace*{-4.5em}\textbf{[R2:10]}\vspace*{-1.9em}}

- ``i.e., when the motion energy contribution far outreaches the computations or vice-versa. The occurrence frequently happens with rotary-wing aerial robots (e.g., quadrotors or quad-copters, hexacopters, etc.) and lighter-than-air aerial robots (e.g., blimps).'' --$>$ clarify which example is to which case (motion energy $>$ computing energy for quadrotor, motion energy $<$ computing energy for lighter-than-air)
  
{\color{blue} 

{\hspace*{-4.5em}{[R2:10]}\vspace*{-1.9em}}

Thank you for underlying this possible source of confusion. We have fixed the text to clearly state whether the category is in the former or latter occurrence.

\begin{formal}
  \color{black} [...] there are other classes where planning-scheduling energy awareness leads to irrelevant savings, i.e., when the motion energy contribution far outreaches the computations or vice-versa. The {\color{blue}motion outreaching computation energy} frequently happens with rotary-wing aerial robots (e.g., quadrotors or quadcopters, hexacopters, etc.){\color{blue}, whereas the opposite occurs with} lighter-than-air aerial robots (e.g., blimps). [...]
  \vspace*{1ex}
\end{formal}}

{\hspace*{-4.5em}\textbf{[R2:11]}\vspace*{-1.9em}}

- ordering of figures in Figure 6 is confusing. Consider lumping each case in a separate row (I, i, II, ii). 
  
{\color{blue} 

{\hspace*{-4.5em}{[R2:11]}\vspace*{-1.9em}}

Thank you for underlying the issue. We have attempted to optimize the space but overloaded the figure and significantly reduced its readability. We have reordered Figure 6 to address the issue. We have first split the figure into two Figures, 6 and 7. In both, we utilize one row per case. Figure 6 contains cases I and II related to static coverage at the two boundary configurations under varying wind speeds and directions.


\begin{formal}
  \footnotesize
  
% ! replace blue with black
\definecolor{cd9d9d9}{RGB}{235,235,235}
\definecolor{cffffff}{RGB}{255,255,255}
\definecolor{ca0a0a4}{RGB}{160,160,164}
\definecolor{ce10000}{RGB}{225,0,0}
\definecolor{cff0000}{RGB}{255,0,0}


\def \globalscale {0.940000}
\begin{tikzpicture}[y=0.80pt, x=0.80pt, yscale=-1*\globalscale, xscale=1*\globalscale, inner sep=0pt, outer sep=0pt]
\color{blue}
\begin{scope}[shift={(598.26327,165.72719)},draw=blue,even odd rule,line cap=rect,line join=bevel,line width=0.800pt]
  \path[draw=cd9d9d9,line cap=butt,line join=miter,line width=1.440pt,miter limit=4.00] (-302.5903,-136.2302) -- (-281.2030,-152.5408);



  \path[draw=cd9d9d9,line cap=butt,line join=miter,line width=1.440pt,miter limit=4.00] (-303.0158,-64.4976) -- (-281.5953,-47.1121);



  \path[fill=cd9d9d9,dash pattern=on 0.99pt off 0.99pt,even odd rule,line cap=round,line width=0.248pt,miter limit=4.00,rounded corners=0.0000cm] (-617.3423,-41.6805) rectangle (-187.4691,79.0929);



  \path[draw=cffffff,line cap=butt,line join=miter,line width=1.440pt,miter limit=4.00] (-302.9246,46.2884) -- (-281.1456,66.6309);



  \begin{scope}[draw=blue,line cap=rect,line join=bevel,line width=0.800pt]
  \end{scope}
  \begin{scope}[scale=1.006,draw=blue,line cap=rect,line join=bevel,line width=0.800pt]
  \end{scope}
  \begin{scope}[scale=1.006,draw=blue,line cap=rect,line join=bevel,line width=0.800pt]
  \end{scope}
  \begin{scope}[cm={{1.00588,0.0,0.0,1.00588,(39.2294,93.5471)}},draw=blue,line cap=rect,line join=bevel,line width=0.800pt]
  \end{scope}
  \begin{scope}[cm={{1.00588,0.0,0.0,1.00588,(39.2294,93.5471)}},draw=blue,line cap=rect,line join=bevel,line width=0.800pt]
  \end{scope}
  \begin{scope}[cm={{1.00588,0.0,0.0,1.00588,(39.2294,93.5471)}},draw=blue,line cap=rect,line join=bevel,line width=0.800pt]
  \end{scope}
  \begin{scope}[cm={{1.00588,0.0,0.0,1.00588,(39.2294,93.5471)}},draw=blue,line cap=rect,line join=bevel,line width=0.800pt]
  \end{scope}
  \begin{scope}[cm={{1.00588,0.0,0.0,1.00588,(39.2294,93.5471)}},draw=blue,line cap=rect,line join=bevel,line width=0.800pt]
  \end{scope}
  \begin{scope}[cm={{1.00588,0.0,0.0,1.00588,(-426.03325,-54.43023)}},draw=blue,line cap=rect,line join=bevel,line width=0.800pt]
    \path[fill=blue] (0.0000,0.0000) node[above right] (text34) {32};



  \end{scope}
  \begin{scope}[cm={{1.00588,0.0,0.0,1.00588,(39.2294,93.5471)}},draw=blue,line cap=rect,line join=bevel,line width=0.800pt]
  \end{scope}
  \begin{scope}[scale=1.006,draw=blue,line cap=rect,line join=bevel,line width=0.800pt]
  \end{scope}
  \begin{scope}[scale=1.006,draw=blue,line cap=rect,line join=bevel,line width=0.800pt]
  \end{scope}
  \begin{scope}[cm={{1.00588,0.0,0.0,1.00588,(39.2294,68.4)}},draw=blue,line cap=rect,line join=bevel,line width=0.800pt]
  \end{scope}
  \begin{scope}[cm={{1.00588,0.0,0.0,1.00588,(39.2294,68.4)}},draw=blue,line cap=rect,line join=bevel,line width=0.800pt]
  \end{scope}
  \begin{scope}[cm={{1.00588,0.0,0.0,1.00588,(39.2294,68.4)}},draw=blue,line cap=rect,line join=bevel,line width=0.800pt]
  \end{scope}
  \begin{scope}[cm={{1.00588,0.0,0.0,1.00588,(39.2294,68.4)}},draw=blue,line cap=rect,line join=bevel,line width=0.800pt]
  \end{scope}
  \begin{scope}[cm={{1.00588,0.0,0.0,1.00588,(39.2294,68.4)}},draw=blue,line cap=rect,line join=bevel,line width=0.800pt]
  \end{scope}
  \begin{scope}[cm={{1.00588,0.0,0.0,1.00588,(-425.97692,-85.57732)}},draw=blue,line cap=rect,line join=bevel,line width=0.800pt]
    \path[fill=blue] (0.0000,0.0000) node[above right] (text64) {36};



  \end{scope}
  \begin{scope}[cm={{1.00588,0.0,0.0,1.00588,(39.2294,68.4)}},draw=blue,line cap=rect,line join=bevel,line width=0.800pt]
  \end{scope}
  \begin{scope}[scale=1.006,draw=blue,line cap=rect,line join=bevel,line width=0.800pt]
  \end{scope}
  \begin{scope}[scale=1.006,draw=blue,line cap=rect,line join=bevel,line width=0.800pt]
  \end{scope}
  \begin{scope}[cm={{1.00588,0.0,0.0,1.00588,(39.2294,43.2529)}},draw=blue,line cap=rect,line join=bevel,line width=0.800pt]
  \end{scope}
  \begin{scope}[cm={{1.00588,0.0,0.0,1.00588,(39.2294,43.2529)}},draw=blue,line cap=rect,line join=bevel,line width=0.800pt]
  \end{scope}
  \begin{scope}[cm={{1.00588,0.0,0.0,1.00588,(39.2294,43.2529)}},draw=blue,line cap=rect,line join=bevel,line width=0.800pt]
  \end{scope}
  \begin{scope}[cm={{1.00588,0.0,0.0,1.00588,(39.2294,43.2529)}},draw=blue,line cap=rect,line join=bevel,line width=0.800pt]
  \end{scope}
  \begin{scope}[cm={{1.00588,0.0,0.0,1.00588,(39.2294,43.2529)}},draw=blue,line cap=rect,line join=bevel,line width=0.800pt]
  \end{scope}
  \begin{scope}[cm={{1.00588,0.0,0.0,1.00588,(-426.0413,-118.22444)}},draw=blue,line cap=rect,line join=bevel,line width=0.800pt]
    \path[fill=blue] (0.0000,0.0000) node[above right] (text94) {40};



  \end{scope}
  \begin{scope}[cm={{1.00588,0.0,0.0,1.00588,(39.2294,43.2529)}},draw=blue,line cap=rect,line join=bevel,line width=0.800pt]
  \end{scope}
  \begin{scope}[scale=1.006,draw=blue,line cap=rect,line join=bevel,line width=0.800pt]
  \end{scope}
  \begin{scope}[scale=1.006,draw=blue,line cap=rect,line join=bevel,line width=0.800pt]
  \end{scope}
  \begin{scope}[cm={{1.00588,0.0,0.0,1.00588,(53.3118,110.647)}},draw=blue,line cap=rect,line join=bevel,line width=0.800pt]
  \end{scope}
  \begin{scope}[cm={{1.00588,0.0,0.0,1.00588,(53.3118,110.647)}},draw=blue,line cap=rect,line join=bevel,line width=0.800pt]
  \end{scope}
  \begin{scope}[cm={{1.00588,0.0,0.0,1.00588,(53.3118,110.647)}},draw=blue,line cap=rect,line join=bevel,line width=0.800pt]
  \end{scope}
  \begin{scope}[cm={{1.00588,0.0,0.0,1.00588,(53.3118,110.647)}},draw=blue,line cap=rect,line join=bevel,line width=0.800pt]
  \end{scope}
  \begin{scope}[cm={{1.00588,0.0,0.0,1.00588,(53.3118,110.647)}},draw=blue,line cap=rect,line join=bevel,line width=0.800pt]
  \end{scope}
  \begin{scope}[cm={{1.00588,0.0,0.0,1.00588,(-170.37572,77.03586)}},draw=blue,line cap=rect,line join=bevel,line width=0.800pt]
    \path[fill=blue] (0.0000,0.0000) node[above right] (text124) {\scriptsize 0};



  \end{scope}
  \begin{scope}[cm={{1.00588,0.0,0.0,1.00588,(53.3118,110.647)}},draw=blue,line cap=rect,line join=bevel,line width=0.800pt]
  \end{scope}
  \begin{scope}[scale=1.006,draw=blue,line cap=rect,line join=bevel,line width=0.800pt]
  \end{scope}
  \begin{scope}[scale=1.006,draw=blue,line cap=rect,line join=bevel,line width=0.800pt]
  \end{scope}
  \begin{scope}[cm={{1.00588,0.0,0.0,1.00588,(79.4647,110.647)}},draw=blue,line cap=rect,line join=bevel,line width=0.800pt]
  \end{scope}
  \begin{scope}[cm={{1.00588,0.0,0.0,1.00588,(79.4647,110.647)}},draw=blue,line cap=rect,line join=bevel,line width=0.800pt]
  \end{scope}
  \begin{scope}[cm={{1.00588,0.0,0.0,1.00588,(79.4647,110.647)}},draw=blue,line cap=rect,line join=bevel,line width=0.800pt]
  \end{scope}
  \begin{scope}[cm={{1.00588,0.0,0.0,1.00588,(79.4647,110.647)}},draw=blue,line cap=rect,line join=bevel,line width=0.800pt]
  \end{scope}
  \begin{scope}[cm={{1.00588,0.0,0.0,1.00588,(79.4647,110.647)}},draw=blue,line cap=rect,line join=bevel,line width=0.800pt]
  \end{scope}
  \begin{scope}[cm={{1.00588,0.0,0.0,1.00588,(-144.22284,77.03586)}},draw=blue,line cap=rect,line join=bevel,line width=0.800pt]
    \path[fill=blue] (0.0000,0.0000) node[above right] (text156) {\scriptsize 1};



  \end{scope}
  \begin{scope}[cm={{1.00588,0.0,0.0,1.00588,(79.4647,110.647)}},draw=blue,line cap=rect,line join=bevel,line width=0.800pt]
  \end{scope}
  \begin{scope}[scale=1.006,draw=blue,line cap=rect,line join=bevel,line width=0.800pt]
  \end{scope}
  \begin{scope}[scale=1.006,draw=blue,line cap=rect,line join=bevel,line width=0.800pt]
  \end{scope}
  \begin{scope}[cm={{1.00588,0.0,0.0,1.00588,(105.618,110.647)}},draw=blue,line cap=rect,line join=bevel,line width=0.800pt]
  \end{scope}
  \begin{scope}[cm={{1.00588,0.0,0.0,1.00588,(105.618,110.647)}},draw=blue,line cap=rect,line join=bevel,line width=0.800pt]
  \end{scope}
  \begin{scope}[cm={{1.00588,0.0,0.0,1.00588,(105.618,110.647)}},draw=blue,line cap=rect,line join=bevel,line width=0.800pt]
  \end{scope}
  \begin{scope}[cm={{1.00588,0.0,0.0,1.00588,(105.618,110.647)}},draw=blue,line cap=rect,line join=bevel,line width=0.800pt]
  \end{scope}
  \begin{scope}[cm={{1.00588,0.0,0.0,1.00588,(105.618,110.647)}},draw=blue,line cap=rect,line join=bevel,line width=0.800pt]
  \end{scope}
  \begin{scope}[cm={{1.00588,0.0,0.0,1.00588,(-118.06953,77.03586)}},draw=blue,line cap=rect,line join=bevel,line width=0.800pt]
    \path[fill=blue] (0.0000,0.0000) node[above right] (text188) {\scriptsize 2};



  \end{scope}
  \begin{scope}[cm={{1.00588,0.0,0.0,1.00588,(105.618,110.647)}},draw=blue,line cap=rect,line join=bevel,line width=0.800pt]
  \end{scope}
  \begin{scope}[scale=1.006,draw=blue,line cap=rect,line join=bevel,line width=0.800pt]
  \end{scope}
  \begin{scope}[scale=1.006,draw=blue,line cap=rect,line join=bevel,line width=0.800pt]
  \end{scope}
  \begin{scope}[cm={{1.00588,0.0,0.0,1.00588,(132.274,110.647)}},draw=blue,line cap=rect,line join=bevel,line width=0.800pt]
  \end{scope}
  \begin{scope}[cm={{1.00588,0.0,0.0,1.00588,(132.274,110.647)}},draw=blue,line cap=rect,line join=bevel,line width=0.800pt]
  \end{scope}
  \begin{scope}[cm={{1.00588,0.0,0.0,1.00588,(132.274,110.647)}},draw=blue,line cap=rect,line join=bevel,line width=0.800pt]
  \end{scope}
  \begin{scope}[cm={{1.00588,0.0,0.0,1.00588,(132.274,110.647)}},draw=blue,line cap=rect,line join=bevel,line width=0.800pt]
  \end{scope}
  \begin{scope}[cm={{1.00588,0.0,0.0,1.00588,(132.274,110.647)}},draw=blue,line cap=rect,line join=bevel,line width=0.800pt]
  \end{scope}
  \begin{scope}[cm={{1.00588,0.0,0.0,1.00588,(-91.41353,-12.96416)}},draw=blue,line cap=rect,line join=bevel,line width=0.800pt]
    \path[fill=blue] (0.0000,89.4739) node[above right] (text218) {\scriptsize 3};



  \end{scope}
  \begin{scope}[cm={{1.00588,0.0,0.0,1.00588,(132.274,110.647)}},draw=blue,line cap=rect,line join=bevel,line width=0.800pt]
  \end{scope}
  \begin{scope}[scale=1.006,draw=blue,line cap=rect,line join=bevel,line width=0.800pt]
  \end{scope}
  \begin{scope}[scale=1.006,draw=blue,line cap=rect,line join=bevel,line width=0.800pt]
  \end{scope}
  \begin{scope}[scale=1.006,draw=blue,line cap=rect,line join=bevel,line width=0.800pt]
  \end{scope}
  \begin{scope}[scale=1.006,draw=blue,line cap=rect,line join=bevel,line width=0.800pt]
  \end{scope}
  \begin{scope}[scale=1.006,draw=blue,line cap=rect,line join=bevel,line width=0.800pt]
  \end{scope}
  \begin{scope}[scale=1.006,draw=blue,line cap=rect,line join=bevel,line width=0.800pt]
  \end{scope}
  \begin{scope}[cm={{1.00588,0.0,0.0,1.00588,(128.753,29.1706)}},draw=blue,line cap=rect,line join=bevel,line width=0.800pt]
  \end{scope}
  \begin{scope}[cm={{1.00588,0.0,0.0,1.00588,(128.753,29.1706)}},draw=blue,line cap=rect,line join=bevel,line width=0.800pt]
  \end{scope}
  \begin{scope}[cm={{1.00588,0.0,0.0,1.00588,(128.753,29.1706)}},draw=blue,line cap=rect,line join=bevel,line width=0.800pt]
  \end{scope}
  \begin{scope}[cm={{1.00588,0.0,0.0,1.00588,(128.753,29.1706)}},draw=blue,line cap=rect,line join=bevel,line width=0.800pt]
  \end{scope}
  \begin{scope}[cm={{1.00588,0.0,0.0,1.00588,(128.753,29.1706)}},draw=blue,line cap=rect,line join=bevel,line width=0.800pt]
  \end{scope}
  \begin{scope}[cm={{1.00588,0.0,0.0,1.00588,(128.753,29.1706)}},draw=blue,line cap=rect,line join=bevel,line width=0.800pt]
  \end{scope}
  \begin{scope}[cm={{0.0,-1.00588,1.00588,0.0,(29.1706,189.106)}},draw=blue,line cap=rect,line join=bevel,line width=0.800pt]
  \end{scope}
  \begin{scope}[cm={{0.0,-1.00588,1.00588,0.0,(29.1706,189.106)}},draw=blue,line cap=rect,line join=bevel,line width=0.800pt]
  \end{scope}
  \begin{scope}[cm={{0.0,-1.00588,1.00588,0.0,(29.1706,189.106)}},draw=blue,line cap=rect,line join=bevel,line width=0.800pt]
  \end{scope}
  \begin{scope}[cm={{0.0,-1.00588,1.00588,0.0,(29.1706,189.106)}},draw=blue,line cap=rect,line join=bevel,line width=0.800pt]
  \end{scope}
  \begin{scope}[cm={{0.0,-1.00588,1.00588,0.0,(29.1706,189.106)}},draw=blue,line cap=rect,line join=bevel,line width=0.800pt]
  \end{scope}
  \begin{scope}[cm={{0.0,-1.00588,1.00588,0.0,(29.1706,189.106)}},draw=blue,line cap=rect,line join=bevel,line width=0.800pt]
  \end{scope}
  \begin{scope}[cm={{1.00588,0.0,0.0,1.00588,(62.3647,28.1647)}},draw=blue,line cap=rect,line join=bevel,line width=0.800pt]
  \end{scope}
  \begin{scope}[cm={{1.00588,0.0,0.0,1.00588,(62.3647,28.1647)}},draw=blue,line cap=rect,line join=bevel,line width=0.800pt]
  \end{scope}
  \begin{scope}[cm={{1.00588,0.0,0.0,1.00588,(62.3647,28.1647)}},draw=blue,line cap=rect,line join=bevel,line width=0.800pt]
  \end{scope}
  \begin{scope}[cm={{1.00588,0.0,0.0,1.00588,(62.3647,28.1647)}},draw=blue,line cap=rect,line join=bevel,line width=0.800pt]
  \end{scope}
  \begin{scope}[cm={{1.00588,0.0,0.0,1.00588,(62.3647,28.1647)}},draw=blue,line cap=rect,line join=bevel,line width=0.800pt]
  \end{scope}
  \begin{scope}[cm={{1.00588,0.0,0.0,1.00588,(62.3647,28.1647)}},draw=blue,line cap=rect,line join=bevel,line width=0.800pt]
  \end{scope}
  \begin{scope}[scale=1.006,draw=blue,line cap=rect,line join=bevel,line width=0.800pt]
  \end{scope}
  \begin{scope}[scale=1.006,draw=blue,line cap=rect,line join=bevel,line width=0.800pt]
  \end{scope}
  \begin{scope}[scale=1.006,draw=blue,line cap=rect,line join=bevel,line width=0.800pt]
  \end{scope}
  \begin{scope}[scale=1.006,draw=blue,line cap=rect,line join=bevel,line width=0.800pt]
  \end{scope}
  \begin{scope}[scale=1.006,draw=blue,line cap=rect,line join=bevel,line width=0.800pt]
  \end{scope}
  \begin{scope}[scale=1.006,draw=blue,line cap=rect,line join=bevel,line width=0.800pt]
  \end{scope}
  \begin{scope}[cm={{1.00588,0.0,0.0,1.00588,(60.3529,36.2118)}},draw=blue,line cap=rect,line join=bevel,line width=0.800pt]
  \end{scope}
  \begin{scope}[cm={{1.00588,0.0,0.0,1.00588,(60.3529,36.2118)}},draw=blue,line cap=rect,line join=bevel,line width=0.800pt]
  \end{scope}
  \begin{scope}[cm={{1.00588,0.0,0.0,1.00588,(60.3529,36.2118)}},draw=blue,line cap=rect,line join=bevel,line width=0.800pt]
  \end{scope}
  \begin{scope}[cm={{1.00588,0.0,0.0,1.00588,(60.3529,36.2118)}},draw=blue,line cap=rect,line join=bevel,line width=0.800pt]
  \end{scope}
  \begin{scope}[cm={{1.00588,0.0,0.0,1.00588,(60.3529,36.2118)}},draw=blue,line cap=rect,line join=bevel,line width=0.800pt]
  \end{scope}
  \begin{scope}[cm={{1.00588,0.0,0.0,1.00588,(60.3529,36.2118)}},draw=blue,line cap=rect,line join=bevel,line width=0.800pt]
  \end{scope}
  \begin{scope}[scale=1.006,draw=blue,line cap=rect,line join=bevel,line width=0.800pt]
  \end{scope}
  \begin{scope}[scale=1.006,draw=blue,line cap=rect,line join=bevel,line width=0.800pt]
  \end{scope}
  \begin{scope}[scale=1.006,draw=blue,line cap=rect,line join=bevel,line width=0.800pt]
  \end{scope}
  \begin{scope}[scale=1.006,draw=blue,line cap=rect,line join=bevel,line width=0.800pt]
  \end{scope}
  \begin{scope}[scale=1.006,draw=blue,line cap=rect,line join=bevel,line width=0.800pt]
  \end{scope}
  \begin{scope}[scale=1.006,draw=blue,line cap=rect,line join=bevel,line width=0.800pt]
  \end{scope}
  \begin{scope}[scale=1.006,draw=blue,line cap=rect,line join=bevel,line width=0.800pt]
  \end{scope}
  \begin{scope}[scale=1.006,draw=blue,line cap=round,line join=round,line width=0.480pt]
    \path[cm={{1.25275,0.0,0.0,1.25275,(-479.34266,-167.80381)}},draw] (56.5000,13.5000) -- (56.5000,95.5000) -- (142.5000,95.5000) -- (142.5000,13.5000) -- (56.5000,13.5000);



    \begin{scope}[cm={{0.99623,0.0,0.0,1.3704,(-430.13329,-152.91751)}},draw=ca0a0a4,dash pattern=on 1.02pt off 1.02pt,line cap=round,line join=round,line width=0.255pt,miter limit=4.00]
      \path[shift={(110.39113,-5.49717)},draw,dash pattern=on 1.02pt off 1.02pt,line width=0.255pt,miter limit=4.00] (70.5000,164.5000) -- (70.5000,88.5000);



    \end{scope}
    \begin{scope}[cm={{0.99623,0.0,0.0,1.3704,(-320.15845,-159.02501)}},draw=ca0a0a4,dash pattern=on 1.02pt off 1.02pt,line cap=round,line join=round,line width=0.255pt,miter limit=4.00]
      \path[draw,dash pattern=on 1.02pt off 1.02pt,line width=0.255pt,miter limit=4.00] (98.5000,164.5000) -- (98.5000,88.5000);



    \end{scope}
    \begin{scope}[cm={{0.74279,0.0,0.0,1.28515,(-186.22138,-161.30028)}},draw=ca0a0a4,dash pattern=on 1.22pt off 1.22pt,line cap=round,line join=round,line width=0.305pt,miter limit=4.00]
      \path[draw,dash pattern=on 1.22pt off 1.22pt,line width=0.305pt,miter limit=4.00] (25.5000,26.5000) -- (108.5000,26.5000);



      \path[draw,dash pattern=on 1.22pt off 1.22pt,line width=0.305pt,miter limit=4.00] (137.5000,26.5000) -- (142.5000,26.5000);



    \end{scope}
    \begin{scope}[cm={{0.74279,0.0,0.0,1.28515,(-186.22138,-161.30028)}},draw=blue,line cap=round,line join=round,line width=0.480pt]
      \path[cm={{1.54975,0.0,0.0,1.0,(-13.85377,0.0)}},draw] (25.5000,26.5000) -- (28.5000,26.5000);



      \path[cm={{1.54975,0.0,0.0,1.0,(-78.53317,0.0)}},draw] (142.5000,26.5000) -- (139.5000,26.5000);



    \end{scope}
    \begin{scope}[cm={{0.95389,0.0,0.0,0.95389,(-180.91222,-124.57711)}},draw=blue,fill=ce10000,line cap=rect,line join=bevel,line width=0.800pt]
      \path[fill=ce10000] (0.0000,0.0000) node[above right] (text36) {\scriptsize 30};



    \end{scope}
    \begin{scope}[cm={{0.74279,0.0,0.0,1.28515,(-186.22138,-161.30028)}},draw=ca0a0a4,dash pattern=on 1.22pt off 1.22pt,line cap=round,line join=round,line width=0.305pt,miter limit=4.00]
      \path[draw,dash pattern=on 1.22pt off 1.22pt,line width=0.305pt,miter limit=4.00] (25.5000,11.5000) -- (142.5000,11.5000);



    \end{scope}
    \begin{scope}[cm={{0.74279,0.0,0.0,1.28515,(-186.22138,-161.30028)}},draw=blue,line cap=round,line join=round,line width=0.480pt]
      \path[cm={{1.54975,0.0,0.0,1.0,(-13.85377,0.0)}},draw] (25.5000,11.5000) -- (28.5000,11.5000);



      \path[cm={{1.54975,0.0,0.0,1.0,(-78.53317,0.0)}},draw] (142.5000,11.5000) -- (139.5000,11.5000);



    \end{scope}
    \begin{scope}[cm={{0.95389,0.0,0.0,0.95389,(-180.91222,-144.27915)}},draw=blue,fill=ce10000,line cap=rect,line join=bevel,line width=0.800pt]
      \path[fill=ce10000] (0.0000,0.0000) node[above right] (text66) {\scriptsize 40};



    \end{scope}
    \begin{scope}[cm={{0.74279,0.0,0.0,1.28515,(-186.22138,-161.30028)}},draw=ca0a0a4,dash pattern=on 0.40pt off 0.80pt,line cap=round,line join=round,line width=0.400pt]
      \path[draw] (25.5000,32.5000) -- (25.5000,8.5000);



    \end{scope}
    \begin{scope}[cm={{0.74279,0.0,0.0,1.28515,(-186.22138,-161.30028)}},draw=blue,line cap=round,line join=round,line width=0.480pt]
      \path[draw] (25.5000,32.5000) -- (25.5000,28.5000);



      \path[draw] (25.5000,8.5000) -- (25.5000,11.5000);



    \end{scope}
    \begin{scope}[cm={{0.74279,0.0,0.0,1.28515,(-186.22138,-161.30028)}},draw=ca0a0a4,dash pattern=on 1.22pt off 1.22pt,line cap=round,line join=round,line width=0.305pt,miter limit=4.00]
      \path[draw,dash pattern=on 1.22pt off 1.22pt,line width=0.305pt,miter limit=4.00] (60.5000,32.5000) -- (60.5000,8.5000);



    \end{scope}
    \begin{scope}[cm={{0.74279,0.0,0.0,1.28515,(-186.22138,-161.30028)}},draw=blue,line cap=round,line join=round,line width=0.480pt]
      \path[cm={{1.0,0.0,0.0,0.70101,(0.0,9.59194)}},draw] (60.5000,32.5000) -- (60.5000,28.5000);



      \path[cm={{1.0,0.0,0.0,0.89573,(0.0,0.855)}},draw] (60.5000,8.5000) -- (60.5000,11.5000);



    \end{scope}
    \begin{scope}[cm={{0.74279,0.0,0.0,1.28515,(-186.22138,-161.30028)}},draw=ca0a0a4,dash pattern=on 1.22pt off 1.22pt,line cap=round,line join=round,line width=0.305pt,miter limit=4.00]
      \path[draw,dash pattern=on 1.22pt off 1.22pt,line width=0.305pt,miter limit=4.00] (95.5000,32.5000) -- (95.5000,8.5000);



    \end{scope}
    \begin{scope}[cm={{0.74279,0.0,0.0,1.28515,(-186.22138,-161.30028)}},draw=blue,line cap=round,line join=round,line width=0.480pt]
      \path[cm={{1.0,0.0,0.0,0.70101,(0.0,9.59194)}},draw] (95.5000,32.5000) -- (95.5000,28.5000);



      \path[cm={{1.0,0.0,0.0,0.89573,(0.0,0.855)}},draw] (95.5000,8.5000) -- (95.5000,11.5000);



    \end{scope}
    \begin{scope}[cm={{0.74279,0.0,0.0,1.28515,(-186.22138,-161.30028)}},draw=ca0a0a4,dash pattern=on 1.22pt off 1.22pt,line cap=round,line join=round,line width=0.305pt,miter limit=4.00]
      \path[draw,dash pattern=on 1.22pt off 1.22pt,line width=0.305pt,miter limit=4.00] (130.5000,20.5000) -- (130.5000,8.5000);



    \end{scope}
    \begin{scope}[cm={{0.74279,0.0,0.0,1.28515,(-186.22138,-161.30028)}},draw=blue,line cap=round,line join=round,line width=0.480pt]
      \path[cm={{1.0,0.0,0.0,0.70101,(0.0,9.59194)}},draw] (130.5000,32.5000) -- (130.5000,28.5000);



      \path[cm={{1.0,0.0,0.0,0.89573,(0.0,0.855)}},draw] (130.5000,8.5000) -- (130.5000,11.5000);



    \end{scope}
    \begin{scope}[cm={{0.74279,0.0,0.0,1.28515,(-186.22138,-161.30028)}},draw=blue,line cap=round,line join=round,line width=0.480pt]
      \path[draw] (25.5000,8.5000) -- (25.5000,32.5000) -- (142.5000,32.5000) -- (142.5000,8.5000) -- (25.5000,8.5000);



    \end{scope}
    \begin{scope}[cm={{0.95389,0.0,0.0,0.95389,(-104.99275,-127.89091)}},draw=blue,line cap=rect,line join=bevel,line width=0.800pt]
      \path[fill=blue] (0.0000,0.0000) node[above right] (text194) {\scriptsize $\alpha_0$};



    \end{scope}
    \begin{scope}[cm={{0.74279,0.0,0.0,1.28515,(-184.07401,-166.61499)}},draw=blue,line cap=round,line join=round,line width=0.480pt]
      \path[draw,even odd rule] (123.5000,28.5000) -- (132.5000,28.5000);



    \end{scope}
    \begin{scope}[cm={{0.74279,0.0,0.0,1.28515,(-186.22138,-161.30028)}},draw=blue,line cap=round,line join=round,line width=0.480pt]
      \path[draw] (25.8000,32.0000) -- (26.2000,14.3000) -- (26.7000,17.2000) -- (27.1000,18.1000) -- (27.5000,14.6000) -- (27.9000,12.6000) -- (28.3000,14.9000) -- (28.8000,17.3000) -- (29.2000,10.3000) -- (29.6000,14.5000) -- (30.0000,19.1000) -- (30.5000,15.7000) -- (30.9000,12.7000) -- (31.3000,15.0000) -- (31.7000,18.7000) -- (32.2000,19.3000) -- (32.6000,16.6000) -- (33.0000,13.7000) -- (33.4000,12.8000) -- (33.8000,14.3000) -- (34.3000,16.9000) -- (34.7000,18.9000) -- (35.1000,19.4000) -- (35.5000,18.6000) -- (36.0000,17.2000) -- (36.4000,15.9000) -- (36.8000,15.5000) -- (37.2000,15.9000) -- (37.6000,17.1000) -- (38.1000,18.6000) -- (38.5000,20.0000) -- (38.9000,21.0000) -- (39.3000,21.3000) -- (39.8000,21.0000) -- (40.2000,20.3000) -- (40.6000,19.3000) -- (41.0000,18.3000) -- (41.5000,17.4000) -- (41.9000,16.9000) -- (42.3000,16.8000) -- (42.7000,17.1000) -- (43.1000,17.6000) -- (43.6000,18.3000) -- (44.0000,19.0000) -- (44.4000,19.7000) -- (44.8000,20.1000) -- (45.3000,20.4000) -- (45.7000,20.5000) -- (46.1000,20.4000) -- (46.5000,20.1000) -- (47.0000,19.8000) -- (47.4000,19.3000) -- (47.8000,18.5000) -- (48.2000,17.8000) -- (48.6000,17.2000) -- (49.1000,16.7000) -- (49.5000,16.5000) -- (49.9000,16.4000) -- (50.3000,16.4000) -- (50.8000,16.5000) -- (51.2000,16.7000) -- (51.6000,17.0000) -- (52.0000,17.2000) -- (52.4000,17.4000) -- (52.9000,17.5000) -- (53.3000,17.5000) -- (53.7000,17.3000) -- (54.1000,17.0000) -- (54.6000,16.6000) -- (55.0000,16.1000) -- (55.4000,15.6000) -- (55.8000,15.0000) -- (56.3000,14.4000) -- (56.7000,13.9000) -- (57.1000,13.5000) -- (57.5000,13.2000) -- (57.9000,13.2000) -- (58.4000,13.3000) -- (58.8000,13.5000) -- (59.2000,13.9000) -- (59.6000,14.3000) -- (60.1000,14.7000) -- (60.5000,15.0000) -- (60.9000,15.4000) -- (61.3000,15.7000) -- (61.8000,16.0000) -- (62.2000,16.2000) -- (62.6000,16.3000) -- (63.0000,16.2000) -- (63.4000,16.2000) -- (63.9000,16.1000) -- (64.3000,16.0000) -- (64.7000,15.9000) -- (65.1000,15.9000) -- (65.6000,15.9000) -- (66.0000,16.0000) -- (66.4000,16.2000) -- (66.8000,16.4000) -- (67.3000,16.7000) -- (67.7000,16.9000) -- (68.1000,17.2000) -- (68.5000,17.3000) -- (68.9000,17.4000) -- (69.4000,17.5000) -- (69.8000,17.7000) -- (70.2000,17.7000) -- (70.6000,17.7000) -- (71.1000,17.6000) -- (71.5000,17.5000) -- (71.9000,17.3000) -- (72.3000,17.1000) -- (72.7000,17.0000) -- (73.2000,16.9000) -- (73.6000,16.8000) -- (74.0000,16.9000) -- (74.4000,17.0000) -- (74.9000,16.9000) -- (75.3000,16.8000) -- (75.7000,16.8000) -- (76.1000,16.8000) -- (76.6000,16.9000) -- (77.0000,17.1000) -- (77.4000,17.2000) -- (77.8000,17.3000) -- (78.2000,17.4000) -- (78.7000,17.4000) -- (79.1000,17.4000) -- (79.5000,17.4000) -- (79.9000,17.4000) -- (80.4000,17.3000) -- (80.8000,17.2000) -- (81.2000,17.1000) -- (81.6000,17.1000) -- (82.1000,17.0000) -- (82.5000,17.0000) -- (82.9000,16.9000) -- (83.3000,16.7000) -- (83.7000,16.6000) -- (84.2000,16.4000) -- (84.6000,16.4000) -- (85.0000,16.3000) -- (85.4000,16.3000) -- (85.9000,16.3000) -- (86.3000,16.3000) -- (86.7000,16.2000) -- (87.1000,16.2000) -- (87.6000,16.1000) -- (88.0000,16.1000) -- (88.4000,16.2000) -- (88.8000,16.3000) -- (89.2000,16.3000) -- (89.7000,16.3000) -- (90.1000,16.3000) -- (90.5000,16.3000) -- (90.9000,16.3000) -- (91.4000,16.3000) -- (91.8000,16.3000) -- (92.2000,16.3000) -- (92.6000,16.3000) -- (93.0000,16.2000) -- (93.5000,16.2000) -- (93.9000,16.2000) -- (94.3000,16.2000) -- (94.7000,16.2000) -- (95.2000,16.1000) -- (95.6000,16.1000) -- (96.0000,16.1000) -- (96.4000,16.2000) -- (96.9000,16.4000) -- (97.3000,16.5000) -- (97.7000,16.5000) -- (98.1000,16.6000) -- (98.5000,16.6000) -- (99.0000,16.6000) -- (99.4000,16.6000) -- (99.8000,16.5000) -- (100.2000,16.5000) -- (100.7000,16.6000) -- (101.1000,16.7000) -- (101.5000,16.6000) -- (101.9000,16.5000) -- (102.3000,16.3000) -- (102.8000,16.3000) -- (103.2000,16.3000) -- (103.6000,16.3000) -- (104.0000,16.3000) -- (104.5000,16.3000) -- (104.9000,16.3000) -- (105.3000,16.3000) -- (105.7000,16.2000) -- (106.2000,16.2000) -- (106.6000,16.2000) -- (107.0000,16.2000) -- (107.4000,16.2000) -- (107.8000,16.2000) -- (108.3000,16.3000) -- (108.7000,16.3000) -- (109.1000,16.4000) -- (109.5000,16.5000) -- (110.0000,16.5000) -- (110.4000,16.5000) -- (110.8000,16.5000) -- (111.2000,16.5000) -- (111.7000,16.5000) -- (112.1000,16.6000) -- (112.5000,16.6000) -- (112.9000,16.6000) -- (113.3000,16.5000) -- (113.8000,16.5000) -- (114.2000,16.4000) -- (114.6000,16.3000) -- (115.0000,16.3000) -- (115.5000,16.4000) -- (115.9000,16.5000) -- (116.3000,16.5000) -- (116.7000,16.5000) -- (117.2000,16.5000) -- (117.6000,16.5000) -- (118.0000,16.5000) -- (118.4000,16.5000) -- (118.8000,16.6000) -- (119.3000,16.6000) -- (119.7000,16.6000) -- (120.1000,16.5000) -- (120.5000,16.5000) -- (121.0000,16.5000) -- (121.4000,16.5000) -- (121.8000,16.5000) -- (122.2000,16.3000) -- (122.6000,16.3000) -- (123.1000,16.3000) -- (123.5000,16.4000) -- (123.9000,16.4000) -- (124.3000,16.4000) -- (124.8000,16.4000) -- (125.2000,16.4000) -- (125.6000,16.3000) -- (126.0000,16.3000) -- (126.5000,16.2000) -- (126.9000,16.2000) -- (127.3000,16.3000) -- (127.7000,16.5000) -- (128.1000,16.5000) -- (128.6000,16.4000) -- (129.0000,16.3000) -- (129.4000,16.3000) -- (129.8000,16.4000) -- (130.3000,16.5000) -- (130.7000,16.5000) -- (131.1000,16.6000) -- (131.5000,16.6000) -- (131.9000,16.6000) -- (132.4000,16.6000) -- (132.8000,16.5000) -- (133.2000,16.5000) -- (133.6000,16.4000) -- (134.1000,16.4000) -- (134.5000,16.3000) -- (134.9000,16.3000) -- (135.3000,16.3000) -- (135.8000,16.4000) -- (136.2000,16.4000) -- (136.6000,16.4000) -- (137.0000,16.4000) -- (137.4000,16.4000) -- (137.9000,16.4000) -- (138.3000,16.4000) -- (138.7000,16.5000) -- (139.1000,16.5000) -- (139.6000,16.5000) -- (140.0000,16.5000) -- (140.4000,16.5000) -- (140.8000,16.4000) -- (141.3000,16.4000) -- (141.7000,16.3000) -- (142.1000,16.4000) -- (142.3000,16.4000);



    \end{scope}
    \begin{scope}[cm={{0.74279,0.0,0.0,1.28515,(-186.22138,-161.30028)}},draw=blue,line cap=round,line join=round,line width=0.480pt]
      \path[draw] (25.5000,8.5000) -- (25.5000,32.5000) -- (142.5000,32.5000) -- (142.5000,8.5000) -- (25.5000,8.5000);



    \end{scope}
    \begin{scope}[cm={{0.74279,0.0,0.0,1.28515,(-186.22138,-161.30028)}},draw=ca0a0a4,dash pattern=on 1.22pt off 1.22pt,line cap=round,line join=round,line width=0.305pt,miter limit=4.00]
      \path[draw,dash pattern=on 1.22pt off 1.22pt,line width=0.305pt,miter limit=4.00] (25.5000,46.5000) -- (108.5000,46.5000);



      \path[draw,dash pattern=on 1.22pt off 1.22pt,line width=0.305pt,miter limit=4.00] (137.5000,46.5000) -- (142.5000,46.5000);



    \end{scope}
    \begin{scope}[cm={{0.74279,0.0,0.0,1.28515,(-186.22138,-161.30028)}},draw=blue,line cap=round,line join=round,line width=0.480pt]
      \path[cm={{1.54975,0.0,0.0,1.0,(-13.85377,0.0)}},draw] (25.5000,46.5000) -- (28.5000,46.5000);



      \path[cm={{1.54975,0.0,0.0,1.0,(-78.53317,0.0)}},draw] (142.5000,46.5000) -- (139.5000,46.5000);



    \end{scope}
    \begin{scope}[cm={{0.95389,0.0,0.0,0.95389,(-183.45337,-99.34397)}},draw=blue,fill=ce10000,line cap=rect,line join=bevel,line width=0.800pt]
      \path[fill=ce10000] (0.0000,0.0000) node[above right] (text250) {\scriptsize -10};



    \end{scope}
    \begin{scope}[cm={{0.74279,0.0,0.0,1.28515,(-186.22138,-161.30028)}},draw=ca0a0a4,dash pattern=on 1.22pt off 1.22pt,line cap=round,line join=round,line width=0.305pt,miter limit=4.00]
      \path[draw,dash pattern=on 1.22pt off 1.22pt,line width=0.305pt,miter limit=4.00] (25.5000,34.5000) -- (142.5000,34.5000);



    \end{scope}
    \begin{scope}[cm={{0.74279,0.0,0.0,1.28515,(-186.22138,-161.30028)}},draw=blue,line cap=round,line join=round,line width=0.480pt]
      \path[cm={{1.54975,0.0,0.0,1.0,(-13.85377,0.0)}},draw] (25.5000,34.5000) -- (28.5000,34.5000);



      \path[cm={{1.54975,0.0,0.0,1.0,(-78.53317,0.0)}},draw] (142.5000,34.5000) -- (139.5000,34.5000);



    \end{scope}
    \begin{scope}[cm={{0.95389,0.0,0.0,0.95389,(-180.91222,-114.12005)}},draw=blue,fill=ce10000,line cap=rect,line join=bevel,line width=0.800pt]
      \path[fill=ce10000] (0.0000,0.0000) node[above right] (text280) {\scriptsize 10};



    \end{scope}
    \begin{scope}[cm={{0.74279,0.0,0.0,1.28515,(-186.22138,-161.30028)}},draw=ca0a0a4,dash pattern=on 0.40pt off 0.80pt,line cap=round,line join=round,line width=0.400pt]
      \path[draw] (25.5000,56.5000) -- (25.5000,32.5000);



    \end{scope}
    \begin{scope}[cm={{0.74279,0.0,0.0,1.28515,(-186.22138,-161.30028)}},draw=blue,line cap=round,line join=round,line width=0.480pt]
      \path[draw] (25.5000,56.5000) -- (25.5000,51.5000);



      \path[draw] (25.5000,32.5000) -- (25.5000,36.5000);



    \end{scope}
    \begin{scope}[cm={{0.74279,0.0,0.0,1.28515,(-186.22138,-161.30028)}},draw=ca0a0a4,dash pattern=on 1.22pt off 1.22pt,line cap=round,line join=round,line width=0.305pt,miter limit=4.00]
      \path[draw,dash pattern=on 1.22pt off 1.22pt,line width=0.305pt,miter limit=4.00] (60.5000,56.5000) -- (60.5000,32.5000);



    \end{scope}
    \begin{scope}[cm={{0.74279,0.0,0.0,1.28515,(-186.22138,-161.30028)}},draw=blue,line cap=round,line join=round,line width=0.480pt]
      \path[cm={{1.0,0.0,0.0,0.57583,(0.0,24.03834)}},draw] (60.5000,56.5000) -- (60.5000,51.5000);



      \path[cm={{1.0,0.0,0.0,0.70101,(0.0,9.62756)}},draw] (60.5000,32.5000) -- (60.5000,36.5000);



    \end{scope}
    \begin{scope}[cm={{0.74279,0.0,0.0,1.28515,(-186.22138,-161.30028)}},draw=ca0a0a4,dash pattern=on 1.22pt off 1.22pt,line cap=round,line join=round,line width=0.305pt,miter limit=4.00]
      \path[draw,dash pattern=on 1.22pt off 1.22pt,line width=0.305pt,miter limit=4.00] (95.5000,56.5000) -- (95.5000,32.5000);



    \end{scope}
    \begin{scope}[cm={{0.74279,0.0,0.0,1.28515,(-186.22138,-161.30028)}},draw=blue,line cap=round,line join=round,line width=0.480pt]
      \path[cm={{1.0,0.0,0.0,0.57583,(0.0,24.03834)}},draw] (95.5000,56.5000) -- (95.5000,51.5000);



      \path[cm={{1.0,0.0,0.0,0.70101,(0.0,9.62756)}},draw] (95.5000,32.5000) -- (95.5000,36.5000);



    \end{scope}
    \begin{scope}[cm={{0.74279,0.0,0.0,1.28515,(-186.22138,-161.30028)}},draw=ca0a0a4,dash pattern=on 1.22pt off 1.22pt,line cap=round,line join=round,line width=0.305pt,miter limit=4.00]
      \path[draw,dash pattern=on 1.22pt off 1.22pt,line width=0.305pt,miter limit=4.00] (130.5000,44.5000) -- (130.5000,32.5000);



    \end{scope}
    \begin{scope}[cm={{0.74279,0.0,0.0,1.28515,(-186.22138,-161.30028)}},draw=blue,line cap=round,line join=round,line width=0.480pt]
      \path[cm={{1.0,0.0,0.0,0.57583,(0.0,24.03834)}},draw] (130.5000,56.5000) -- (130.5000,51.5000);



      \path[cm={{1.0,0.0,0.0,0.70101,(0.0,9.62756)}},draw] (130.5000,32.5000) -- (130.5000,36.5000);



    \end{scope}
    \begin{scope}[cm={{0.74279,0.0,0.0,1.28515,(-186.22138,-161.30028)}},draw=blue,line cap=round,line join=round,line width=0.480pt]
      \path[draw] (25.5000,32.5000) -- (25.5000,56.5000) -- (142.5000,56.5000) -- (142.5000,32.5000) -- (25.5000,32.5000);



    \end{scope}
    \begin{scope}[cm={{0.95389,0.0,0.0,0.95389,(-104.99275,-97.16859)}},draw=blue,line cap=rect,line join=bevel,line width=0.800pt]
      \path[fill=blue] (0.0000,0.0000) node[above right] (text408) {\scriptsize $\alpha_1$};



    \end{scope}
    \begin{scope}[cm={{0.74279,0.0,0.0,1.28515,(-184.07401,-166.61499)}},draw=blue,line cap=round,line join=round,line width=0.480pt]
      \path[draw,even odd rule] (123.5000,52.5000) -- (132.5000,52.5000);



    \end{scope}
    \begin{scope}[cm={{0.74279,0.0,0.0,1.28515,(-186.22138,-161.30028)}},draw=blue,line cap=round,line join=round,line width=0.480pt]
      \path[draw] (25.8000,36.4000) -- (25.8000,36.4000) -- (26.2000,41.9000) -- (26.7000,39.4000) -- (27.1000,40.5000) -- (27.5000,40.1000) -- (27.9000,39.9000) -- (28.3000,40.6000) -- (28.8000,43.1000) -- (29.2000,39.4000) -- (29.6000,39.6000) -- (30.0000,41.8000) -- (30.5000,40.8000) -- (30.9000,39.0000) -- (31.3000,39.5000) -- (31.7000,41.2000) -- (32.2000,42.0000) -- (32.6000,41.2000) -- (33.0000,39.7000) -- (33.4000,38.9000) -- (33.8000,39.1000) -- (34.3000,40.2000) -- (34.7000,41.3000) -- (35.1000,41.9000) -- (35.5000,41.9000) -- (36.0000,41.5000) -- (36.4000,40.9000) -- (36.8000,40.4000) -- (37.2000,40.4000) -- (37.6000,40.7000) -- (38.1000,41.2000) -- (38.5000,41.8000) -- (38.9000,42.2000) -- (39.3000,42.3000) -- (39.8000,42.2000) -- (40.2000,41.8000) -- (40.6000,41.3000) -- (41.0000,40.6000) -- (41.5000,39.9000) -- (41.9000,39.3000) -- (42.3000,38.9000) -- (42.7000,38.7000) -- (43.1000,38.7000) -- (43.6000,38.8000) -- (44.0000,39.0000) -- (44.4000,39.2000) -- (44.8000,39.4000) -- (45.3000,39.6000) -- (45.7000,39.8000) -- (46.1000,39.8000) -- (46.5000,39.8000) -- (47.0000,39.8000) -- (47.4000,39.7000) -- (47.8000,39.5000) -- (48.2000,39.3000) -- (48.6000,39.1000) -- (49.1000,39.0000) -- (49.5000,38.9000) -- (49.9000,38.9000) -- (50.3000,39.0000) -- (50.8000,39.2000) -- (51.2000,39.4000) -- (51.6000,39.7000) -- (52.0000,39.9000) -- (52.4000,40.2000) -- (52.9000,40.5000) -- (53.3000,40.7000) -- (53.7000,40.8000) -- (54.1000,40.9000) -- (54.6000,40.9000) -- (55.0000,40.9000) -- (55.4000,40.8000) -- (55.8000,40.6000) -- (56.3000,40.5000) -- (56.7000,40.3000) -- (57.1000,40.1000) -- (57.5000,40.0000) -- (57.9000,40.0000) -- (58.4000,40.0000) -- (58.8000,40.1000) -- (59.2000,40.3000) -- (59.6000,40.5000) -- (60.1000,40.7000) -- (60.5000,40.9000) -- (60.9000,41.1000) -- (61.3000,41.4000) -- (61.8000,41.6000) -- (62.2000,41.7000) -- (62.6000,41.8000) -- (63.0000,41.9000) -- (63.4000,41.9000) -- (63.9000,41.8000) -- (64.3000,41.8000) -- (64.7000,41.7000) -- (65.1000,41.6000) -- (65.6000,41.6000) -- (66.0000,41.5000) -- (66.4000,41.4000) -- (66.8000,41.4000) -- (67.3000,41.4000) -- (67.7000,41.3000) -- (68.1000,41.3000) -- (68.5000,41.2000) -- (68.9000,41.1000) -- (69.4000,41.0000) -- (69.8000,40.9000) -- (70.2000,40.8000) -- (70.6000,40.7000) -- (71.1000,40.5000) -- (71.5000,40.3000) -- (71.9000,40.1000) -- (72.3000,39.9000) -- (72.7000,39.7000) -- (73.2000,39.5000) -- (73.6000,39.4000) -- (74.0000,39.3000) -- (74.4000,39.2000) -- (74.9000,39.0000) -- (75.3000,38.9000) -- (75.7000,38.9000) -- (76.1000,38.8000) -- (76.6000,38.8000) -- (77.0000,38.9000) -- (77.4000,38.9000) -- (77.8000,39.0000) -- (78.2000,39.0000) -- (78.7000,39.1000) -- (79.1000,39.2000) -- (79.5000,39.3000) -- (79.9000,39.3000) -- (80.4000,39.4000) -- (80.8000,39.5000) -- (81.2000,39.6000) -- (81.6000,39.7000) -- (82.1000,39.8000) -- (82.5000,39.9000) -- (82.9000,40.0000) -- (83.3000,40.1000) -- (83.7000,40.1000) -- (84.2000,40.2000) -- (84.6000,40.3000) -- (85.0000,40.4000) -- (85.4000,40.5000) -- (85.9000,40.7000) -- (86.3000,40.8000) -- (86.7000,40.9000) -- (87.1000,41.0000) -- (87.6000,41.1000) -- (88.0000,41.1000) -- (88.4000,41.3000) -- (88.8000,41.4000) -- (89.2000,41.5000) -- (89.7000,41.6000) -- (90.1000,41.6000) -- (90.5000,41.7000) -- (90.9000,41.7000) -- (91.4000,41.7000) -- (91.8000,41.7000) -- (92.2000,41.7000) -- (92.6000,41.7000) -- (93.0000,41.6000) -- (93.5000,41.6000) -- (93.9000,41.5000) -- (94.3000,41.5000) -- (94.7000,41.4000) -- (95.2000,41.3000) -- (95.6000,41.2000) -- (96.0000,41.1000) -- (96.4000,41.0000) -- (96.9000,41.0000) -- (97.3000,40.9000) -- (97.7000,40.8000) -- (98.1000,40.7000) -- (98.5000,40.6000) -- (99.0000,40.5000) -- (99.4000,40.4000) -- (99.8000,40.3000) -- (100.2000,40.1000) -- (100.7000,40.1000) -- (101.1000,40.0000) -- (101.5000,39.9000) -- (101.9000,39.7000) -- (102.3000,39.6000) -- (102.8000,39.5000) -- (103.2000,39.4000) -- (103.6000,39.3000) -- (104.0000,39.2000) -- (104.5000,39.2000) -- (104.9000,39.1000) -- (105.3000,39.1000) -- (105.7000,39.1000) -- (106.2000,39.1000) -- (106.6000,39.1000) -- (107.0000,39.1000) -- (107.4000,39.2000) -- (107.8000,39.2000) -- (108.3000,39.3000) -- (108.7000,39.4000) -- (109.1000,39.5000) -- (109.5000,39.7000) -- (110.0000,39.8000) -- (110.4000,39.9000) -- (110.8000,40.0000) -- (111.2000,40.1000) -- (111.7000,40.3000) -- (112.1000,40.4000) -- (112.5000,40.6000) -- (112.9000,40.7000) -- (113.3000,40.8000) -- (113.8000,40.9000) -- (114.2000,41.0000) -- (114.6000,41.1000) -- (115.0000,41.2000) -- (115.5000,41.3000) -- (115.9000,41.4000) -- (116.3000,41.5000) -- (116.7000,41.6000) -- (117.2000,41.6000) -- (117.6000,41.7000) -- (118.0000,41.7000) -- (118.4000,41.8000) -- (118.8000,41.8000) -- (119.3000,41.8000) -- (119.7000,41.8000) -- (120.1000,41.7000) -- (120.5000,41.7000) -- (121.0000,41.6000) -- (121.4000,41.6000) -- (121.8000,41.5000) -- (122.2000,41.3000) -- (122.6000,41.2000) -- (123.1000,41.1000) -- (123.5000,41.0000) -- (123.9000,40.9000) -- (124.3000,40.8000) -- (124.8000,40.7000) -- (125.2000,40.5000) -- (125.6000,40.4000) -- (126.0000,40.2000) -- (126.5000,40.1000) -- (126.9000,39.9000) -- (127.3000,39.8000) -- (127.7000,39.8000) -- (128.1000,39.7000) -- (128.6000,39.5000) -- (129.0000,39.4000) -- (129.4000,39.4000) -- (129.8000,39.3000) -- (130.3000,39.3000) -- (130.7000,39.2000) -- (131.1000,39.2000) -- (131.5000,39.2000) -- (131.9000,39.2000) -- (132.4000,39.2000) -- (132.8000,39.2000) -- (133.2000,39.2000) -- (133.6000,39.2000) -- (134.1000,39.3000) -- (134.5000,39.3000) -- (134.9000,39.4000) -- (135.3000,39.5000) -- (135.8000,39.6000) -- (136.2000,39.7000) -- (136.6000,39.8000) -- (137.0000,39.9000) -- (137.4000,40.0000) -- (137.9000,40.1000) -- (138.3000,40.3000) -- (138.7000,40.4000) -- (139.1000,40.6000) -- (139.6000,40.7000) -- (140.0000,40.8000) -- (140.4000,40.9000) -- (140.8000,41.0000) -- (141.3000,41.1000) -- (141.7000,41.2000) -- (142.1000,41.3000) -- (142.3000,41.4000);



    \end{scope}
    \begin{scope}[cm={{0.74279,0.0,0.0,1.28515,(-186.22138,-161.30028)}},draw=blue,line cap=round,line join=round,line width=0.480pt]
      \path[draw] (25.5000,32.5000) -- (25.5000,56.5000) -- (142.5000,56.5000) -- (142.5000,32.5000) -- (25.5000,32.5000);



    \end{scope}
    \begin{scope}[cm={{0.74279,0.0,0.0,1.28515,(-186.22138,-161.30028)}},draw=ca0a0a4,dash pattern=on 1.22pt off 1.22pt,line cap=round,line join=round,line width=0.305pt,miter limit=4.00]
      \path[draw,dash pattern=on 1.22pt off 1.22pt,line width=0.305pt,miter limit=4.00] (25.5000,70.5000) -- (108.5000,70.5000);



      \path[draw,dash pattern=on 1.22pt off 1.22pt,line width=0.305pt,miter limit=4.00] (137.5000,70.5000) -- (142.5000,70.5000);



    \end{scope}
    \begin{scope}[cm={{0.74279,0.0,0.0,1.28515,(-186.22138,-161.30028)}},draw=blue,line cap=round,line join=round,line width=0.480pt]
      \path[cm={{1.54975,0.0,0.0,1.0,(-13.85377,0.0)}},draw] (25.5000,70.5000) -- (28.5000,70.5000);



      \path[cm={{1.54975,0.0,0.0,1.0,(-78.53317,0.15196)}},draw] (142.5000,70.5000) -- (139.5000,70.5000);



    \end{scope}
    \begin{scope}[cm={{0.95389,0.0,0.0,0.95389,(-183.45337,-67.69359)}},draw=blue,fill=ce10000,line cap=rect,line join=bevel,line width=0.800pt]
      \path[fill=ce10000] (0.0000,0.0000) node[above right] (text464) {\scriptsize -10};



    \end{scope}
    \begin{scope}[cm={{0.74279,0.0,0.0,1.28515,(-186.22138,-161.30028)}},draw=ca0a0a4,dash pattern=on 1.22pt off 1.22pt,line cap=round,line join=round,line width=0.305pt,miter limit=4.00]
      \path[draw,dash pattern=on 1.22pt off 1.22pt,line width=0.305pt,miter limit=4.00] (25.5000,58.5000) -- (142.5000,58.5000);



    \end{scope}
    \begin{scope}[cm={{0.74279,0.0,0.0,1.28515,(-186.22138,-161.30028)}},draw=blue,line cap=round,line join=round,line width=0.480pt]
      \path[cm={{1.54975,0.0,0.0,1.0,(-13.85377,0.0)}},draw] (25.5000,58.5000) -- (28.5000,58.5000);



      \path[cm={{1.54975,0.0,0.0,1.0,(-78.53317,0.0)}},draw] (142.5000,58.5000) -- (139.5000,58.5000);



    \end{scope}
    \begin{scope}[cm={{0.95389,0.0,0.0,0.95389,(-180.91222,-84.15133)}},draw=blue,fill=ce10000,line cap=rect,line join=bevel,line width=0.800pt]
      \path[fill=ce10000] (0.0000,0.0000) node[above right] (text494) {\scriptsize 10};



    \end{scope}
    \begin{scope}[cm={{0.74279,0.0,0.0,1.28515,(-186.22138,-161.30028)}},draw=ca0a0a4,dash pattern=on 0.40pt off 0.80pt,line cap=round,line join=round,line width=0.400pt]
      \path[draw] (25.5000,80.5000) -- (25.5000,56.5000);



    \end{scope}
    \begin{scope}[cm={{0.74279,0.0,0.0,1.28515,(-186.22138,-161.30028)}},draw=blue,line cap=round,line join=round,line width=0.480pt]
      \path[draw] (25.5000,80.5000) -- (25.5000,75.5000);



      \path[draw] (25.5000,56.5000) -- (25.5000,60.5000);



    \end{scope}
    \begin{scope}[cm={{0.74279,0.0,0.0,1.28515,(-186.22138,-161.30028)}},draw=ca0a0a4,dash pattern=on 1.22pt off 1.22pt,line cap=round,line join=round,line width=0.305pt,miter limit=4.00]
      \path[draw,dash pattern=on 1.22pt off 1.22pt,line width=0.305pt,miter limit=4.00] (60.5000,80.5000) -- (60.5000,56.5000);



    \end{scope}
    \begin{scope}[cm={{0.74279,0.0,0.0,1.28515,(-186.22138,-161.30028)}},draw=blue,line cap=round,line join=round,line width=0.480pt]
      \path[cm={{1.0,0.0,0.0,0.57583,(0.0,34.21848)}},draw] (60.5000,80.5000) -- (60.5000,75.5000);



      \path[cm={{1.0,0.0,0.0,0.70101,(0.0,16.80337)}},draw] (60.5000,56.5000) -- (60.5000,60.5000);



    \end{scope}
    \begin{scope}[cm={{0.74279,0.0,0.0,1.28515,(-186.22138,-161.30028)}},draw=ca0a0a4,dash pattern=on 1.22pt off 1.22pt,line cap=round,line join=round,line width=0.305pt,miter limit=4.00]
      \path[draw,dash pattern=on 1.22pt off 1.22pt,line width=0.305pt,miter limit=4.00] (95.5000,80.5000) -- (95.5000,56.5000);



    \end{scope}
    \begin{scope}[cm={{0.74279,0.0,0.0,1.28515,(-186.22138,-161.30028)}},draw=blue,line cap=round,line join=round,line width=0.480pt]
      \path[cm={{1.0,0.0,0.0,0.57583,(0.0,34.21848)}},draw] (95.5000,80.5000) -- (95.5000,75.5000);



      \path[cm={{1.0,0.0,0.0,0.70101,(0.0,16.80337)}},draw] (95.5000,56.5000) -- (95.5000,60.5000);



    \end{scope}
    \begin{scope}[cm={{0.74279,0.0,0.0,1.28515,(-186.22138,-161.30028)}},draw=ca0a0a4,dash pattern=on 1.22pt off 1.22pt,line cap=round,line join=round,line width=0.305pt,miter limit=4.00]
      \path[draw,dash pattern=on 1.22pt off 1.22pt,line width=0.305pt,miter limit=4.00] (130.5000,68.5000) -- (130.5000,56.5000);



    \end{scope}
    \begin{scope}[cm={{0.74279,0.0,0.0,1.28515,(-186.22138,-161.30028)}},draw=blue,line cap=round,line join=round,line width=0.480pt]
      \path[cm={{1.0,0.0,0.0,0.57583,(0.0,34.21848)}},draw] (130.5000,80.5000) -- (130.5000,75.5000);



      \path[cm={{1.0,0.0,0.0,0.70101,(0.0,16.80337)}},draw] (130.5000,56.5000) -- (130.5000,60.5000);



    \end{scope}
    \begin{scope}[cm={{0.74279,0.0,0.0,1.28515,(-186.22138,-161.30028)}},draw=blue,line cap=round,line join=round,line width=0.480pt]
      \path[draw] (25.5000,56.5000) -- (25.5000,80.5000) -- (142.5000,80.5000) -- (142.5000,56.5000) -- (25.5000,56.5000);



    \end{scope}
    \begin{scope}[cm={{0.95389,0.0,0.0,0.95389,(-104.73329,-65.32779)}},draw=blue,line cap=rect,line join=bevel,line width=0.800pt]
      \path[fill=blue] (0.0000,0.0000) node[above right] (text622) {\scriptsize $\beta_1$};



    \end{scope}
    \begin{scope}[cm={{0.74279,0.0,0.0,1.28515,(-184.07401,-166.61499)}},draw=blue,line cap=round,line join=round,line width=0.480pt]
      \path[draw,even odd rule] (123.5000,76.5000) -- (132.5000,76.5000);



    \end{scope}
    \begin{scope}[cm={{0.74279,0.0,0.0,1.28515,(-186.22138,-161.30028)}},draw=blue,line cap=round,line join=round,line width=0.480pt]
      \path[draw] (25.8000,67.9000) -- (25.8000,67.9000) -- (26.2000,64.4000) -- (26.7000,63.4000) -- (27.1000,65.9000) -- (27.5000,64.8000) -- (27.9000,64.0000) -- (28.3000,65.9000) -- (28.8000,65.0000) -- (29.2000,61.7000) -- (29.6000,64.2000) -- (30.0000,65.6000) -- (30.5000,63.7000) -- (30.9000,62.9000) -- (31.3000,64.5000) -- (31.7000,65.9000) -- (32.2000,65.6000) -- (32.6000,64.1000) -- (33.0000,62.9000) -- (33.4000,63.0000) -- (33.8000,64.1000) -- (34.3000,65.3000) -- (34.7000,66.0000) -- (35.1000,65.8000) -- (35.5000,65.1000) -- (36.0000,64.3000) -- (36.4000,63.6000) -- (36.8000,63.4000) -- (37.2000,63.6000) -- (37.6000,64.0000) -- (38.1000,64.5000) -- (38.5000,64.8000) -- (38.9000,64.8000) -- (39.3000,64.6000) -- (39.8000,64.1000) -- (40.2000,63.5000) -- (40.6000,63.0000) -- (41.0000,62.5000) -- (41.5000,62.2000) -- (41.9000,62.1000) -- (42.3000,62.3000) -- (42.7000,62.7000) -- (43.1000,63.2000) -- (43.6000,63.7000) -- (44.0000,64.1000) -- (44.4000,64.6000) -- (44.8000,64.9000) -- (45.3000,65.1000) -- (45.7000,65.3000) -- (46.1000,65.3000) -- (46.5000,65.3000) -- (47.0000,65.2000) -- (47.4000,65.1000) -- (47.8000,64.9000) -- (48.2000,64.8000) -- (48.6000,64.7000) -- (49.1000,64.8000) -- (49.5000,64.8000) -- (49.9000,65.0000) -- (50.3000,65.2000) -- (50.8000,65.4000) -- (51.2000,65.6000) -- (51.6000,65.8000) -- (52.0000,66.0000) -- (52.4000,66.1000) -- (52.9000,66.2000) -- (53.3000,66.1000) -- (53.7000,66.0000) -- (54.1000,65.9000) -- (54.6000,65.7000) -- (55.0000,65.4000) -- (55.4000,65.2000) -- (55.8000,64.9000) -- (56.3000,64.7000) -- (56.7000,64.5000) -- (57.1000,64.4000) -- (57.5000,64.3000) -- (57.9000,64.4000) -- (58.4000,64.5000) -- (58.8000,64.6000) -- (59.2000,64.8000) -- (59.6000,64.9000) -- (60.1000,65.0000) -- (60.5000,65.1000) -- (60.9000,65.2000) -- (61.3000,65.2000) -- (61.8000,65.2000) -- (62.2000,65.1000) -- (62.6000,65.0000) -- (63.0000,64.8000) -- (63.4000,64.6000) -- (63.9000,64.3000) -- (64.3000,64.1000) -- (64.7000,63.9000) -- (65.1000,63.8000) -- (65.6000,63.7000) -- (66.0000,63.6000) -- (66.4000,63.5000) -- (66.8000,63.4000) -- (67.3000,63.4000) -- (67.7000,63.4000) -- (68.1000,63.4000) -- (68.5000,63.3000) -- (68.9000,63.3000) -- (69.4000,63.3000) -- (69.8000,63.3000) -- (70.2000,63.2000) -- (70.6000,63.2000) -- (71.1000,63.1000) -- (71.5000,63.1000) -- (71.9000,63.0000) -- (72.3000,63.0000) -- (72.7000,63.0000) -- (73.2000,63.1000) -- (73.6000,63.2000) -- (74.0000,63.3000) -- (74.4000,63.4000) -- (74.9000,63.5000) -- (75.3000,63.6000) -- (75.7000,63.8000) -- (76.1000,64.0000) -- (76.6000,64.2000) -- (77.0000,64.4000) -- (77.4000,64.6000) -- (77.8000,64.8000) -- (78.2000,64.9000) -- (78.7000,65.1000) -- (79.1000,65.2000) -- (79.5000,65.3000) -- (79.9000,65.4000) -- (80.4000,65.5000) -- (80.8000,65.6000) -- (81.2000,65.6000) -- (81.6000,65.7000) -- (82.1000,65.7000) -- (82.5000,65.8000) -- (82.9000,65.8000) -- (83.3000,65.8000) -- (83.7000,65.8000) -- (84.2000,65.7000) -- (84.6000,65.7000) -- (85.0000,65.7000) -- (85.4000,65.7000) -- (85.9000,65.7000) -- (86.3000,65.6000) -- (86.7000,65.6000) -- (87.1000,65.5000) -- (87.6000,65.5000) -- (88.0000,65.4000) -- (88.4000,65.3000) -- (88.8000,65.3000) -- (89.2000,65.2000) -- (89.7000,65.1000) -- (90.1000,65.0000) -- (90.5000,64.9000) -- (90.9000,64.7000) -- (91.4000,64.6000) -- (91.8000,64.5000) -- (92.2000,64.4000) -- (92.6000,64.2000) -- (93.0000,64.1000) -- (93.5000,64.0000) -- (93.9000,63.8000) -- (94.3000,63.7000) -- (94.7000,63.6000) -- (95.2000,63.5000) -- (95.6000,63.4000) -- (96.0000,63.4000) -- (96.4000,63.4000) -- (96.9000,63.4000) -- (97.3000,63.4000) -- (97.7000,63.3000) -- (98.1000,63.3000) -- (98.5000,63.3000) -- (99.0000,63.3000) -- (99.4000,63.3000) -- (99.8000,63.3000) -- (100.2000,63.3000) -- (100.7000,63.3000) -- (101.1000,63.4000) -- (101.5000,63.4000) -- (101.9000,63.4000) -- (102.3000,63.5000) -- (102.8000,63.6000) -- (103.2000,63.7000) -- (103.6000,63.8000) -- (104.0000,63.9000) -- (104.5000,64.0000) -- (104.9000,64.1000) -- (105.3000,64.3000) -- (105.7000,64.4000) -- (106.2000,64.5000) -- (106.6000,64.6000) -- (107.0000,64.8000) -- (107.4000,64.9000) -- (107.8000,65.0000) -- (108.3000,65.2000) -- (108.7000,65.3000) -- (109.1000,65.4000) -- (109.5000,65.5000) -- (110.0000,65.6000) -- (110.4000,65.7000) -- (110.8000,65.7000) -- (111.2000,65.7000) -- (111.7000,65.8000) -- (112.1000,65.8000) -- (112.5000,65.8000) -- (112.9000,65.8000) -- (113.3000,65.7000) -- (113.8000,65.7000) -- (114.2000,65.6000) -- (114.6000,65.5000) -- (115.0000,65.4000) -- (115.5000,65.4000) -- (115.9000,65.3000) -- (116.3000,65.2000) -- (116.7000,65.1000) -- (117.2000,65.0000) -- (117.6000,64.9000) -- (118.0000,64.7000) -- (118.4000,64.6000) -- (118.8000,64.5000) -- (119.3000,64.4000) -- (119.7000,64.2000) -- (120.1000,64.1000) -- (120.5000,64.0000) -- (121.0000,63.8000) -- (121.4000,63.7000) -- (121.8000,63.6000) -- (122.2000,63.4000) -- (122.6000,63.3000) -- (123.1000,63.3000) -- (123.5000,63.2000) -- (123.9000,63.2000) -- (124.3000,63.2000) -- (124.8000,63.1000) -- (125.2000,63.1000) -- (125.6000,63.1000) -- (126.0000,63.1000) -- (126.5000,63.1000) -- (126.9000,63.1000) -- (127.3000,63.2000) -- (127.7000,63.4000) -- (128.1000,63.4000) -- (128.6000,63.5000) -- (129.0000,63.5000) -- (129.4000,63.6000) -- (129.8000,63.8000) -- (130.3000,63.9000) -- (130.7000,64.1000) -- (131.1000,64.2000) -- (131.5000,64.3000) -- (131.9000,64.4000) -- (132.4000,64.6000) -- (132.8000,64.7000) -- (133.2000,64.8000) -- (133.6000,64.9000) -- (134.1000,65.0000) -- (134.5000,65.1000) -- (134.9000,65.2000) -- (135.3000,65.3000) -- (135.8000,65.4000) -- (136.2000,65.5000) -- (136.6000,65.6000) -- (137.0000,65.6000) -- (137.4000,65.6000) -- (137.9000,65.7000) -- (138.3000,65.7000) -- (138.7000,65.8000) -- (139.1000,65.8000) -- (139.6000,65.8000) -- (140.0000,65.7000) -- (140.4000,65.7000) -- (140.8000,65.6000) -- (141.3000,65.5000) -- (141.7000,65.4000) -- (142.1000,65.4000) -- (142.3000,65.3000);



    \end{scope}
    \begin{scope}[cm={{0.74279,0.0,0.0,1.28515,(-186.22138,-161.30028)}},draw=blue,line cap=round,line join=round,line width=0.480pt]
      \path[draw] (25.5000,56.5000) -- (25.5000,80.5000) -- (142.5000,80.5000) -- (142.5000,56.5000) -- (25.5000,56.5000);



    \end{scope}
    \begin{scope}[cm={{0.74279,0.0,0.0,1.28515,(-186.22138,-161.30028)}},draw=ca0a0a4,dash pattern=on 1.22pt off 1.22pt,line cap=round,line join=round,line width=0.305pt,miter limit=4.00]
      \path[draw,dash pattern=on 1.22pt off 1.22pt,line width=0.305pt,miter limit=4.00] (25.5000,94.5000) -- (108.5000,94.5000);



      \path[draw,dash pattern=on 1.22pt off 1.22pt,line width=0.305pt,miter limit=4.00] (137.5000,94.5000) -- (142.5000,94.5000);



    \end{scope}
    \begin{scope}[cm={{0.74279,0.0,0.0,1.28515,(-186.22138,-161.30028)}},draw=blue,line cap=round,line join=round,line width=0.480pt]
      \path[cm={{1.54975,0.0,0.0,1.0,(-13.85377,0.0)}},draw] (25.5000,94.5000) -- (28.5000,94.5000);



      \path[cm={{1.54975,0.0,0.0,1.0,(-78.53317,0.0)}},draw] (142.5000,94.5000) -- (139.5000,94.5000);



    \end{scope}
    \begin{scope}[cm={{0.95389,0.0,0.0,0.95389,(-183.45337,-37.34406)}},draw=blue,fill=ce10000,line cap=rect,line join=bevel,line width=0.800pt]
      \path[fill=ce10000] (0.0000,0.0000) node[above right] (text678) {\scriptsize -10};



    \end{scope}
    \begin{scope}[cm={{0.74279,0.0,0.0,1.28515,(-186.22138,-161.30028)}},draw=ca0a0a4,dash pattern=on 1.22pt off 1.22pt,line cap=round,line join=round,line width=0.305pt,miter limit=4.00]
      \path[draw,dash pattern=on 1.22pt off 1.22pt,line width=0.305pt,miter limit=4.00] (25.5000,82.5000) -- (142.5000,82.5000);



    \end{scope}
    \begin{scope}[cm={{0.74279,0.0,0.0,1.28515,(-186.22138,-161.30028)}},draw=blue,line cap=round,line join=round,line width=0.480pt]
      \path[cm={{1.54975,0.0,0.0,1.0,(-13.85377,0.0)}},draw] (25.5000,82.5000) -- (28.5000,82.5000);



      \path[cm={{1.54975,0.0,0.0,1.0,(-78.53317,0.0)}},draw] (142.5000,82.5000) -- (139.5000,82.5000);



    \end{scope}
    \begin{scope}[cm={{0.95389,0.0,0.0,0.95389,(-180.91222,-52.31054)}},draw=blue,fill=ce10000,line cap=rect,line join=bevel,line width=0.800pt]
      \path[fill=ce10000] (0.0000,0.0000) node[above right] (text708) {\scriptsize 10};



    \end{scope}
    \begin{scope}[cm={{0.74279,0.0,0.0,1.28515,(-186.22138,-161.30028)}},draw=ca0a0a4,dash pattern=on 0.40pt off 0.80pt,line cap=round,line join=round,line width=0.400pt]
      \path[draw] (25.5000,104.5000) -- (25.5000,80.5000);



    \end{scope}
    \begin{scope}[cm={{0.74279,0.0,0.0,1.28515,(-186.22138,-161.30028)}},draw=blue,line cap=round,line join=round,line width=0.480pt]
      \path[draw] (25.5000,104.5000) -- (25.5000,99.5000);



      \path[draw] (25.5000,80.5000) -- (25.5000,84.5000);



    \end{scope}
    \begin{scope}[cm={{0.74279,0.0,0.0,1.28515,(-186.22138,-161.30028)}},draw=ca0a0a4,dash pattern=on 1.22pt off 1.22pt,line cap=round,line join=round,line width=0.305pt,miter limit=4.00]
      \path[draw,dash pattern=on 1.22pt off 1.22pt,line width=0.305pt,miter limit=4.00] (60.5000,104.5000) -- (60.5000,80.5000);



    \end{scope}
    \begin{scope}[cm={{0.74279,0.0,0.0,1.28515,(-186.22138,-161.30028)}},draw=blue,line cap=round,line join=round,line width=0.480pt]
      \path[cm={{1.0,0.0,0.0,0.57583,(0.0,44.39861)}},draw] (60.5000,104.5000) -- (60.5000,99.5000);



      \path[cm={{1.0,0.0,0.0,0.70101,(0.0,23.97919)}},draw] (60.5000,80.5000) -- (60.5000,84.5000);



    \end{scope}
    \begin{scope}[cm={{0.74279,0.0,0.0,1.28515,(-186.22138,-161.30028)}},draw=ca0a0a4,dash pattern=on 1.22pt off 1.22pt,line cap=round,line join=round,line width=0.305pt,miter limit=4.00]
      \path[draw,dash pattern=on 1.22pt off 1.22pt,line width=0.305pt,miter limit=4.00] (95.5000,104.5000) -- (95.5000,80.5000);



    \end{scope}
    \begin{scope}[cm={{0.74279,0.0,0.0,1.28515,(-186.22138,-161.30028)}},draw=blue,line cap=round,line join=round,line width=0.480pt]
      \path[cm={{1.0,0.0,0.0,0.57583,(0.0,44.39861)}},draw] (95.5000,104.5000) -- (95.5000,99.5000);



      \path[cm={{1.0,0.0,0.0,0.70101,(0.0,23.97919)}},draw] (95.5000,80.5000) -- (95.5000,84.5000);



    \end{scope}
    \begin{scope}[cm={{0.74279,0.0,0.0,1.28515,(-186.22138,-161.30028)}},draw=ca0a0a4,dash pattern=on 1.22pt off 1.22pt,line cap=round,line join=round,line width=0.305pt,miter limit=4.00]
      \path[draw,dash pattern=on 1.22pt off 1.22pt,line width=0.305pt,miter limit=4.00] (130.5000,92.5000) -- (130.5000,80.5000);



    \end{scope}
    \begin{scope}[cm={{0.74279,0.0,0.0,1.28515,(-186.22138,-161.30028)}},draw=blue,line cap=round,line join=round,line width=0.480pt]
      \path[cm={{1.0,0.0,0.0,0.57583,(0.0,44.39861)}},draw] (130.5000,104.5000) -- (130.5000,99.5000);



      \path[cm={{1.0,0.0,0.0,0.70101,(0.0,23.97919)}},draw] (130.5000,80.5000) -- (130.5000,84.5000);



    \end{scope}
    \begin{scope}[cm={{0.74279,0.0,0.0,1.28515,(-186.22138,-161.30028)}},draw=blue,line cap=round,line join=round,line width=0.480pt]
      \path[draw] (25.5000,80.5000) -- (25.5000,104.5000) -- (142.5000,104.5000) -- (142.5000,80.5000) -- (25.5000,80.5000);



    \end{scope}
    \begin{scope}[cm={{0.95389,0.0,0.0,0.95389,(-104.99275,-34.97829)}},draw=blue,line cap=rect,line join=bevel,line width=0.800pt]
      \path[fill=blue] (0.0000,0.0000) node[above right] (text836) {\scriptsize $\alpha_2$};



    \end{scope}
    \begin{scope}[cm={{0.74279,0.0,0.0,1.28515,(-184.07401,-166.61499)}},draw=blue,line cap=round,line join=round,line width=0.480pt]
      \path[draw,even odd rule] (123.5000,100.5000) -- (132.5000,100.5000);



    \end{scope}
    \begin{scope}[cm={{0.74279,0.0,0.0,1.28515,(-186.22138,-161.30028)}},draw=blue,line cap=round,line join=round,line width=0.480pt]
      \path[draw] (25.8000,86.8000) -- (25.8000,86.8000) -- (26.2000,88.5000) -- (26.7000,88.7000) -- (27.1000,88.0000) -- (27.5000,88.0000) -- (27.9000,90.0000) -- (28.3000,85.4000) -- (28.8000,92.2000) -- (29.2000,86.5000) -- (29.6000,86.2000) -- (30.0000,92.2000) -- (30.5000,91.0000) -- (30.9000,85.7000) -- (31.3000,84.6000) -- (31.7000,88.3000) -- (32.2000,91.6000) -- (32.6000,91.4000) -- (33.0000,88.6000) -- (33.4000,86.0000) -- (33.8000,85.5000) -- (34.3000,87.1000) -- (34.7000,89.4000) -- (35.1000,91.2000) -- (35.5000,91.5000) -- (36.0000,90.5000) -- (36.4000,88.8000) -- (36.8000,87.2000) -- (37.2000,86.2000) -- (37.6000,86.2000) -- (38.1000,86.9000) -- (38.5000,88.0000) -- (38.9000,89.2000) -- (39.3000,90.2000) -- (39.8000,90.6000) -- (40.2000,90.5000) -- (40.6000,89.9000) -- (41.0000,89.0000) -- (41.5000,88.0000) -- (41.9000,87.1000) -- (42.3000,86.4000) -- (42.7000,86.1000) -- (43.1000,86.2000) -- (43.6000,86.6000) -- (44.0000,87.3000) -- (44.4000,88.1000) -- (44.8000,88.9000) -- (45.3000,89.6000) -- (45.7000,90.1000) -- (46.1000,90.5000) -- (46.5000,90.6000) -- (47.0000,90.6000) -- (47.4000,90.3000) -- (47.8000,89.7000) -- (48.2000,89.0000) -- (48.6000,88.3000) -- (49.1000,87.6000) -- (49.5000,86.9000) -- (49.9000,86.4000) -- (50.3000,86.0000) -- (50.8000,85.9000) -- (51.2000,86.0000) -- (51.6000,86.2000) -- (52.0000,86.6000) -- (52.4000,87.2000) -- (52.9000,87.8000) -- (53.3000,88.3000) -- (53.7000,88.9000) -- (54.1000,89.3000) -- (54.6000,89.6000) -- (55.0000,89.8000) -- (55.4000,89.8000) -- (55.8000,89.6000) -- (56.3000,89.4000) -- (56.7000,89.0000) -- (57.1000,88.6000) -- (57.5000,88.2000) -- (57.9000,87.9000) -- (58.4000,87.7000) -- (58.8000,87.6000) -- (59.2000,87.6000) -- (59.6000,87.7000) -- (60.1000,87.9000) -- (60.5000,88.2000) -- (60.9000,88.5000) -- (61.3000,88.8000) -- (61.8000,89.1000) -- (62.2000,89.4000) -- (62.6000,89.5000) -- (63.0000,89.6000) -- (63.4000,89.6000) -- (63.9000,89.4000) -- (64.3000,89.2000) -- (64.7000,89.0000) -- (65.1000,88.7000) -- (65.6000,88.5000) -- (66.0000,88.2000) -- (66.4000,88.1000) -- (66.8000,87.9000) -- (67.3000,87.9000) -- (67.7000,87.9000) -- (68.1000,87.9000) -- (68.5000,88.0000) -- (68.9000,88.1000) -- (69.4000,88.2000) -- (69.8000,88.4000) -- (70.2000,88.5000) -- (70.6000,88.6000) -- (71.1000,88.7000) -- (71.5000,88.7000) -- (71.9000,88.7000) -- (72.3000,88.6000) -- (72.7000,88.6000) -- (73.2000,88.5000) -- (73.6000,88.5000) -- (74.0000,88.5000) -- (74.4000,88.5000) -- (74.9000,88.4000) -- (75.3000,88.4000) -- (75.7000,88.3000) -- (76.1000,88.3000) -- (76.6000,88.4000) -- (77.0000,88.4000) -- (77.4000,88.4000) -- (77.8000,88.5000) -- (78.2000,88.5000) -- (78.7000,88.6000) -- (79.1000,88.6000) -- (79.5000,88.6000) -- (79.9000,88.6000) -- (80.4000,88.6000) -- (80.8000,88.5000) -- (81.2000,88.5000) -- (81.6000,88.4000) -- (82.1000,88.3000) -- (82.5000,88.3000) -- (82.9000,88.2000) -- (83.3000,88.1000) -- (83.7000,88.0000) -- (84.2000,88.0000) -- (84.6000,87.9000) -- (85.0000,87.9000) -- (85.4000,87.9000) -- (85.9000,87.9000) -- (86.3000,88.0000) -- (86.7000,88.0000) -- (87.1000,88.1000) -- (87.6000,88.2000) -- (88.0000,88.3000) -- (88.4000,88.4000) -- (88.8000,88.5000) -- (89.2000,88.7000) -- (89.7000,88.8000) -- (90.1000,88.9000) -- (90.5000,89.0000) -- (90.9000,89.0000) -- (91.4000,89.1000) -- (91.8000,89.1000) -- (92.2000,89.0000) -- (92.6000,89.0000) -- (93.0000,88.9000) -- (93.5000,88.8000) -- (93.9000,88.7000) -- (94.3000,88.6000) -- (94.7000,88.5000) -- (95.2000,88.3000) -- (95.6000,88.2000) -- (96.0000,88.1000) -- (96.4000,88.0000) -- (96.9000,88.0000) -- (97.3000,88.0000) -- (97.7000,88.0000) -- (98.1000,88.0000) -- (98.5000,88.0000) -- (99.0000,88.1000) -- (99.4000,88.1000) -- (99.8000,88.2000) -- (100.2000,88.2000) -- (100.7000,88.3000) -- (101.1000,88.5000) -- (101.5000,88.5000) -- (101.9000,88.5000) -- (102.3000,88.5000) -- (102.8000,88.6000) -- (103.2000,88.6000) -- (103.6000,88.6000) -- (104.0000,88.7000) -- (104.5000,88.7000) -- (104.9000,88.7000) -- (105.3000,88.7000) -- (105.7000,88.7000) -- (106.2000,88.6000) -- (106.6000,88.6000) -- (107.0000,88.6000) -- (107.4000,88.5000) -- (107.8000,88.5000) -- (108.3000,88.4000) -- (108.7000,88.4000) -- (109.1000,88.4000) -- (109.5000,88.4000) -- (110.0000,88.4000) -- (110.4000,88.4000) -- (110.8000,88.3000) -- (111.2000,88.3000) -- (111.7000,88.3000) -- (112.1000,88.3000) -- (112.5000,88.3000) -- (112.9000,88.3000) -- (113.3000,88.3000) -- (113.8000,88.3000) -- (114.2000,88.3000) -- (114.6000,88.3000) -- (115.0000,88.3000) -- (115.5000,88.3000) -- (115.9000,88.4000) -- (116.3000,88.4000) -- (116.7000,88.4000) -- (117.2000,88.5000) -- (117.6000,88.5000) -- (118.0000,88.6000) -- (118.4000,88.6000) -- (118.8000,88.6000) -- (119.3000,88.7000) -- (119.7000,88.7000) -- (120.1000,88.7000) -- (120.5000,88.6000) -- (121.0000,88.6000) -- (121.4000,88.6000) -- (121.8000,88.5000) -- (122.2000,88.4000) -- (122.6000,88.4000) -- (123.1000,88.3000) -- (123.5000,88.3000) -- (123.9000,88.3000) -- (124.3000,88.3000) -- (124.8000,88.2000) -- (125.2000,88.2000) -- (125.6000,88.2000) -- (126.0000,88.1000) -- (126.5000,88.1000) -- (126.9000,88.2000) -- (127.3000,88.2000) -- (127.7000,88.3000) -- (128.1000,88.4000) -- (128.6000,88.4000) -- (129.0000,88.4000) -- (129.4000,88.5000) -- (129.8000,88.6000) -- (130.3000,88.6000) -- (130.7000,88.7000) -- (131.1000,88.8000) -- (131.5000,88.8000) -- (131.9000,88.8000) -- (132.4000,88.8000) -- (132.8000,88.8000) -- (133.2000,88.7000) -- (133.6000,88.7000) -- (134.1000,88.6000) -- (134.5000,88.5000) -- (134.9000,88.4000) -- (135.3000,88.4000) -- (135.8000,88.3000) -- (136.2000,88.3000) -- (136.6000,88.2000) -- (137.0000,88.2000) -- (137.4000,88.1000) -- (137.9000,88.1000) -- (138.3000,88.1000) -- (138.7000,88.1000) -- (139.1000,88.2000) -- (139.6000,88.2000) -- (140.0000,88.3000) -- (140.4000,88.3000) -- (140.8000,88.3000) -- (141.3000,88.4000) -- (141.7000,88.4000) -- (142.1000,88.5000) -- (142.3000,88.5000);



    \end{scope}
    \begin{scope}[cm={{0.74279,0.0,0.0,1.28515,(-186.22138,-161.30028)}},draw=blue,line cap=round,line join=round,line width=0.480pt]
      \path[draw] (25.5000,80.5000) -- (25.5000,104.5000) -- (142.5000,104.5000) -- (142.5000,80.5000) -- (25.5000,80.5000);



    \end{scope}
    \begin{scope}[cm={{0.74279,0.0,0.0,1.28515,(-186.22138,-161.30028)}},draw=ca0a0a4,dash pattern=on 1.22pt off 1.22pt,line cap=round,line join=round,line width=0.305pt,miter limit=4.00]
      \path[draw,dash pattern=on 1.22pt off 1.22pt,line width=0.305pt,miter limit=4.00] (25.5000,118.5000) -- (108.5000,118.5000);



      \path[draw,dash pattern=on 1.22pt off 1.22pt,line width=0.305pt,miter limit=4.00] (137.5000,118.5000) -- (142.5000,118.5000);



    \end{scope}
    \begin{scope}[cm={{0.74279,0.0,0.0,1.28515,(-186.22138,-161.30028)}},draw=blue,line cap=round,line join=round,line width=0.480pt]
      \path[cm={{1.54975,0.0,0.0,1.0,(-13.85377,0.0)}},draw] (25.5000,118.5000) -- (28.5000,118.5000);



      \path[cm={{1.54975,0.0,0.0,1.0,(-78.53317,0.0)}},draw] (142.5000,118.5000) -- (139.5000,118.5000);



    \end{scope}
    \begin{scope}[cm={{0.95389,0.0,0.0,0.95389,(-183.45337,-6.99453)}},draw=blue,fill=ce10000,line cap=rect,line join=bevel,line width=0.800pt]
      \path[fill=ce10000] (0.0000,0.0000) node[above right] (text892) {\scriptsize -10};



    \end{scope}
    \begin{scope}[cm={{0.74279,0.0,0.0,1.28515,(-186.22138,-161.30028)}},draw=ca0a0a4,dash pattern=on 1.22pt off 1.22pt,line cap=round,line join=round,line width=0.305pt,miter limit=4.00]
      \path[draw,dash pattern=on 1.22pt off 1.22pt,line width=0.305pt,miter limit=4.00] (25.5000,106.5000) -- (142.5000,106.5000);



    \end{scope}
    \begin{scope}[cm={{0.74279,0.0,0.0,1.28515,(-186.22138,-161.30028)}},draw=blue,line cap=round,line join=round,line width=0.480pt]
      \path[cm={{1.54975,0.0,0.0,1.0,(-13.85377,0.0)}},draw] (25.5000,106.5000) -- (28.5000,106.5000);



      \path[cm={{1.54975,0.0,0.0,1.0,(-78.53317,0.0)}},draw] (142.5000,106.5000) -- (139.5000,106.5000);



    \end{scope}
    \begin{scope}[cm={{0.95389,0.0,0.0,0.95389,(-180.91222,-21.96101)}},draw=blue,fill=ce10000,line cap=rect,line join=bevel,line width=0.800pt]
      \path[fill=ce10000] (0.0000,0.0000) node[above right] (text922) {\scriptsize 10};



    \end{scope}
    \begin{scope}[cm={{0.74279,0.0,0.0,1.28515,(-186.22138,-161.30028)}},draw=ca0a0a4,dash pattern=on 0.40pt off 0.80pt,line cap=round,line join=round,line width=0.400pt]
      \path[draw] (25.5000,128.5000) -- (25.5000,104.5000);



    \end{scope}
    \begin{scope}[cm={{0.74279,0.0,0.0,1.28515,(-186.22138,-161.30028)}},draw=blue,line cap=round,line join=round,line width=0.480pt]
      \path[draw] (25.5000,128.5000) -- (25.5000,123.5000);



      \path[draw] (25.5000,104.5000) -- (25.5000,108.5000);



    \end{scope}
    \begin{scope}[cm={{0.74279,0.0,0.0,1.28515,(-186.22138,-161.30028)}},draw=ca0a0a4,dash pattern=on 1.22pt off 1.22pt,line cap=round,line join=round,line width=0.305pt,miter limit=4.00]
      \path[draw,dash pattern=on 1.22pt off 1.22pt,line width=0.305pt,miter limit=4.00] (60.5000,128.5000) -- (60.5000,104.5000);



    \end{scope}
    \begin{scope}[cm={{0.74279,0.0,0.0,1.28515,(-186.22138,-161.30028)}},draw=blue,line cap=round,line join=round,line width=0.480pt]
      \path[cm={{1.0,0.0,0.0,0.57583,(0.0,54.57875)}},draw] (60.5000,128.5000) -- (60.5000,123.5000);



      \path[cm={{1.0,0.0,0.0,0.70101,(0.0,31.15501)}},draw] (60.5000,104.5000) -- (60.5000,108.5000);



    \end{scope}
    \begin{scope}[cm={{0.74279,0.0,0.0,1.28515,(-186.22138,-161.30028)}},draw=ca0a0a4,dash pattern=on 1.22pt off 1.22pt,line cap=round,line join=round,line width=0.305pt,miter limit=4.00]
      \path[draw,dash pattern=on 1.22pt off 1.22pt,line width=0.305pt,miter limit=4.00] (95.5000,128.5000) -- (95.5000,104.5000);



    \end{scope}
    \begin{scope}[cm={{0.74279,0.0,0.0,1.28515,(-147.4494,-161.30028)}},draw=blue,line cap=round,line join=round,line width=0.480pt]
      \path[cm={{1.0,0.0,0.0,0.57583,(-52.1975,54.57875)}},draw] (95.5000,128.5000) -- (95.5000,123.5000);



      \path[cm={{1.0,0.0,0.0,0.70101,(-52.1975,31.15501)}},draw] (95.5000,104.5000) -- (95.5000,108.5000);



    \end{scope}
    \begin{scope}[cm={{0.74279,0.0,0.0,1.28515,(-186.22138,-161.30028)}},draw=ca0a0a4,dash pattern=on 1.22pt off 1.22pt,line cap=round,line join=round,line width=0.305pt,miter limit=4.00]
      \path[draw,dash pattern=on 1.22pt off 1.22pt,line width=0.305pt,miter limit=4.00] (130.5000,116.5000) -- (130.5000,104.5000);



    \end{scope}
    \begin{scope}[cm={{0.74279,0.0,0.0,1.28515,(-186.22138,-161.30028)}},draw=blue,line cap=round,line join=round,line width=0.480pt]
      \path[cm={{1.0,0.0,0.0,0.57583,(0.0,54.57875)}},draw] (130.5000,128.5000) -- (130.5000,123.5000);



      \path[cm={{1.0,0.0,0.0,0.70101,(0.0,31.15501)}},draw] (130.5000,104.5000) -- (130.5000,108.5000);



    \end{scope}
    \begin{scope}[cm={{0.74279,0.0,0.0,1.28515,(-186.22138,-161.30028)}},draw=blue,line cap=round,line join=round,line width=0.480pt]
      \path[draw] (25.5000,104.5000) -- (25.5000,128.5000) -- (142.5000,128.5000) -- (142.5000,104.5000) -- (25.5000,104.5000);



    \end{scope}
    \begin{scope}[cm={{0.95389,0.0,0.0,0.95389,(-104.73329,-4.06956)}},draw=blue,line cap=rect,line join=bevel,line width=0.800pt]
      \path[fill=blue] (0.0000,0.0000) node[above right] (text1050) {\scriptsize $\beta_2$};



    \end{scope}
    \begin{scope}[cm={{0.74279,0.0,0.0,1.28515,(-184.07401,-166.61499)}},draw=blue,line cap=round,line join=round,line width=0.480pt]
      \path[draw,even odd rule] (123.5000,124.5000) -- (132.5000,124.5000);



    \end{scope}
    \begin{scope}[cm={{0.74279,0.0,0.0,1.28515,(-186.22138,-161.30028)}},draw=blue,line cap=round,line join=round,line width=0.480pt]
      \path[draw] (25.8000,114.2000) -- (25.8000,114.2000) -- (26.2000,112.5000) -- (26.7000,112.2000) -- (27.1000,112.0000) -- (27.5000,115.1000) -- (27.9000,109.2000) -- (28.3000,114.1000) -- (28.8000,113.9000) -- (29.2000,108.7000) -- (29.6000,114.6000) -- (30.0000,115.3000) -- (30.5000,109.5000) -- (30.9000,108.5000) -- (31.3000,113.0000) -- (31.7000,116.4000) -- (32.2000,115.2000) -- (32.6000,111.5000) -- (33.0000,109.2000) -- (33.4000,109.7000) -- (33.8000,112.1000) -- (34.3000,114.6000) -- (34.7000,115.4000) -- (35.1000,114.5000) -- (35.5000,112.6000) -- (36.0000,110.6000) -- (36.4000,109.5000) -- (36.8000,109.5000) -- (37.2000,110.5000) -- (37.6000,112.0000) -- (38.1000,113.4000) -- (38.5000,114.4000) -- (38.9000,114.7000) -- (39.3000,114.3000) -- (39.8000,113.4000) -- (40.2000,112.3000) -- (40.6000,111.3000) -- (41.0000,110.6000) -- (41.5000,110.2000) -- (41.9000,110.3000) -- (42.3000,110.8000) -- (42.7000,111.6000) -- (43.1000,112.5000) -- (43.6000,113.4000) -- (44.0000,114.1000) -- (44.4000,114.5000) -- (44.8000,114.7000) -- (45.3000,114.6000) -- (45.7000,114.3000) -- (46.1000,113.7000) -- (46.5000,113.0000) -- (47.0000,112.4000) -- (47.4000,111.7000) -- (47.8000,111.0000) -- (48.2000,110.5000) -- (48.6000,110.2000) -- (49.1000,110.2000) -- (49.5000,110.4000) -- (49.9000,110.8000) -- (50.3000,111.4000) -- (50.8000,112.0000) -- (51.2000,112.7000) -- (51.6000,113.4000) -- (52.0000,114.0000) -- (52.4000,114.5000) -- (52.9000,114.8000) -- (53.3000,114.8000) -- (53.7000,114.7000) -- (54.1000,114.5000) -- (54.6000,114.0000) -- (55.0000,113.5000) -- (55.4000,113.0000) -- (55.8000,112.5000) -- (56.3000,112.0000) -- (56.7000,111.6000) -- (57.1000,111.4000) -- (57.5000,111.3000) -- (57.9000,111.3000) -- (58.4000,111.5000) -- (58.8000,111.8000) -- (59.2000,112.2000) -- (59.6000,112.5000) -- (60.1000,112.8000) -- (60.5000,113.0000) -- (60.9000,113.2000) -- (61.3000,113.2000) -- (61.8000,113.2000) -- (62.2000,113.1000) -- (62.6000,112.8000) -- (63.0000,112.6000) -- (63.4000,112.2000) -- (63.9000,111.9000) -- (64.3000,111.7000) -- (64.7000,111.5000) -- (65.1000,111.4000) -- (65.6000,111.3000) -- (66.0000,111.4000) -- (66.4000,111.5000) -- (66.8000,111.7000) -- (67.3000,112.0000) -- (67.7000,112.2000) -- (68.1000,112.4000) -- (68.5000,112.6000) -- (68.9000,112.7000) -- (69.4000,112.8000) -- (69.8000,112.9000) -- (70.2000,112.9000) -- (70.6000,112.9000) -- (71.1000,112.8000) -- (71.5000,112.7000) -- (71.9000,112.6000) -- (72.3000,112.5000) -- (72.7000,112.4000) -- (73.2000,112.3000) -- (73.6000,112.3000) -- (74.0000,112.3000) -- (74.4000,112.3000) -- (74.9000,112.3000) -- (75.3000,112.3000) -- (75.7000,112.3000) -- (76.1000,112.3000) -- (76.6000,112.4000) -- (77.0000,112.5000) -- (77.4000,112.5000) -- (77.8000,112.5000) -- (78.2000,112.5000) -- (78.7000,112.5000) -- (79.1000,112.5000) -- (79.5000,112.5000) -- (79.9000,112.4000) -- (80.4000,112.4000) -- (80.8000,112.3000) -- (81.2000,112.3000) -- (81.6000,112.3000) -- (82.1000,112.2000) -- (82.5000,112.3000) -- (82.9000,112.3000) -- (83.3000,112.3000) -- (83.7000,112.3000) -- (84.2000,112.3000) -- (84.6000,112.4000) -- (85.0000,112.5000) -- (85.4000,112.6000) -- (85.9000,112.7000) -- (86.3000,112.8000) -- (86.7000,112.8000) -- (87.1000,112.9000) -- (87.6000,112.9000) -- (88.0000,113.0000) -- (88.4000,113.0000) -- (88.8000,113.0000) -- (89.2000,113.0000) -- (89.7000,113.0000) -- (90.1000,112.9000) -- (90.5000,112.8000) -- (90.9000,112.7000) -- (91.4000,112.6000) -- (91.8000,112.4000) -- (92.2000,112.3000) -- (92.6000,112.2000) -- (93.0000,112.0000) -- (93.5000,111.9000) -- (93.9000,111.9000) -- (94.3000,111.8000) -- (94.7000,111.8000) -- (95.2000,111.8000) -- (95.6000,111.8000) -- (96.0000,111.9000) -- (96.4000,112.0000) -- (96.9000,112.2000) -- (97.3000,112.3000) -- (97.7000,112.4000) -- (98.1000,112.5000) -- (98.5000,112.6000) -- (99.0000,112.6000) -- (99.4000,112.7000) -- (99.8000,112.7000) -- (100.2000,112.8000) -- (100.7000,112.8000) -- (101.1000,112.9000) -- (101.5000,112.9000) -- (101.9000,112.8000) -- (102.3000,112.7000) -- (102.8000,112.7000) -- (103.2000,112.6000) -- (103.6000,112.6000) -- (104.0000,112.6000) -- (104.5000,112.5000) -- (104.9000,112.5000) -- (105.3000,112.4000) -- (105.7000,112.3000) -- (106.2000,112.3000) -- (106.6000,112.3000) -- (107.0000,112.2000) -- (107.4000,112.2000) -- (107.8000,112.2000) -- (108.3000,112.2000) -- (108.7000,112.2000) -- (109.1000,112.3000) -- (109.5000,112.3000) -- (110.0000,112.3000) -- (110.4000,112.3000) -- (110.8000,112.3000) -- (111.2000,112.3000) -- (111.7000,112.4000) -- (112.1000,112.4000) -- (112.5000,112.4000) -- (112.9000,112.5000) -- (113.3000,112.5000) -- (113.8000,112.5000) -- (114.2000,112.5000) -- (114.6000,112.5000) -- (115.0000,112.5000) -- (115.5000,112.6000) -- (115.9000,112.6000) -- (116.3000,112.6000) -- (116.7000,112.6000) -- (117.2000,112.6000) -- (117.6000,112.6000) -- (118.0000,112.6000) -- (118.4000,112.6000) -- (118.8000,112.6000) -- (119.3000,112.5000) -- (119.7000,112.5000) -- (120.1000,112.4000) -- (120.5000,112.4000) -- (121.0000,112.3000) -- (121.4000,112.3000) -- (121.8000,112.2000) -- (122.2000,112.2000) -- (122.6000,112.2000) -- (123.1000,112.2000) -- (123.5000,112.2000) -- (123.9000,112.3000) -- (124.3000,112.3000) -- (124.8000,112.3000) -- (125.2000,112.4000) -- (125.6000,112.4000) -- (126.0000,112.4000) -- (126.5000,112.5000) -- (126.9000,112.5000) -- (127.3000,112.6000) -- (127.7000,112.7000) -- (128.1000,112.7000) -- (128.6000,112.7000) -- (129.0000,112.7000) -- (129.4000,112.7000) -- (129.8000,112.7000) -- (130.3000,112.7000) -- (130.7000,112.7000) -- (131.1000,112.6000) -- (131.5000,112.6000) -- (131.9000,112.5000) -- (132.4000,112.4000) -- (132.8000,112.3000) -- (133.2000,112.2000) -- (133.6000,112.2000) -- (134.1000,112.1000) -- (134.5000,112.1000) -- (134.9000,112.1000) -- (135.3000,112.1000) -- (135.8000,112.1000) -- (136.2000,112.2000) -- (136.6000,112.2000) -- (137.0000,112.2000) -- (137.4000,112.3000) -- (137.9000,112.4000) -- (138.3000,112.4000) -- (138.7000,112.5000) -- (139.1000,112.6000) -- (139.6000,112.7000) -- (140.0000,112.7000) -- (140.4000,112.7000) -- (140.8000,112.7000) -- (141.3000,112.7000) -- (141.7000,112.7000) -- (142.1000,112.7000) -- (142.3000,112.7000);



    \end{scope}
    \begin{scope}[cm={{0.74279,0.0,0.0,1.28515,(-186.22138,-161.30028)}},draw=blue,line cap=round,line join=round,line width=0.480pt]
      \path[draw] (25.5000,104.5000) -- (25.5000,128.5000) -- (142.5000,128.5000) -- (142.5000,104.5000) -- (25.5000,104.5000);



    \end{scope}
    \begin{scope}[cm={{0.74279,0.0,0.0,1.28515,(-186.22138,-161.30028)}},draw=ca0a0a4,dash pattern=on 1.22pt off 1.22pt,line cap=round,line join=round,line width=0.305pt,miter limit=4.00]
      \path[draw,dash pattern=on 1.22pt off 1.22pt,line width=0.305pt,miter limit=4.00] (25.5000,144.5000) -- (108.5000,144.5000);



      \path[draw,dash pattern=on 1.22pt off 1.22pt,line width=0.305pt,miter limit=4.00] (137.5000,144.5000) -- (142.5000,144.5000);



    \end{scope}
    \begin{scope}[cm={{0.74279,0.0,0.0,1.28515,(-186.22138,-161.30028)}},draw=blue,line cap=round,line join=round,line width=0.480pt]
      \path[cm={{1.54975,0.0,0.0,1.0,(-13.85377,0.0)}},draw] (25.5000,144.5000) -- (28.5000,144.5000);



      \path[cm={{1.54975,0.0,0.0,1.0,(-78.53317,0.0)}},draw] (142.5000,144.5000) -- (139.5000,144.5000);



    \end{scope}
    \begin{scope}[cm={{0.95389,0.0,0.0,0.95389,(-183.45337,27.70791)}},draw=blue,fill=ce10000,line cap=rect,line join=bevel,line width=0.800pt]
      \path[fill=ce10000] (0.0000,0.0000) node[above right] (text1106) {\scriptsize -20};



    \end{scope}
    \begin{scope}[cm={{0.74279,0.0,0.0,1.28515,(-186.22138,-161.30028)}},draw=ca0a0a4,dash pattern=on 1.22pt off 1.22pt,line cap=round,line join=round,line width=0.305pt,miter limit=4.00]
      \path[draw,dash pattern=on 1.22pt off 1.22pt,line width=0.305pt,miter limit=4.00] (25.5000,129.5000) -- (142.5000,129.5000);



    \end{scope}
    \begin{scope}[cm={{0.74279,0.0,0.0,1.28515,(-186.22138,-161.30028)}},draw=blue,line cap=round,line join=round,line width=0.480pt]
      \path[cm={{1.54975,0.0,0.0,1.0,(-13.85377,0.0)}},draw] (25.5000,129.5000) -- (28.5000,129.5000);



      \path[cm={{1.54975,0.0,0.0,1.0,(-78.53317,0.0)}},draw] (142.5000,129.5000) -- (139.5000,129.5000);



    \end{scope}
    \begin{scope}[cm={{0.95389,0.0,0.0,0.95389,(-180.91222,7.43463)}},draw=blue,fill=ce10000,line cap=rect,line join=bevel,line width=0.800pt]
      \path[fill=ce10000] (0.0000,0.0000) node[above right] (text1136) {\scriptsize 20};



    \end{scope}
    \begin{scope}[cm={{0.74279,0.0,0.0,1.28515,(-186.22138,-161.30028)}},draw=ca0a0a4,dash pattern=on 0.40pt off 0.80pt,line cap=round,line join=round,line width=0.400pt]
      \path[draw] (25.5000,152.5000) -- (25.5000,128.5000);



    \end{scope}
    \begin{scope}[cm={{0.74279,0.0,0.0,1.28515,(-186.22138,-161.30028)}},draw=blue,line cap=round,line join=round,line width=0.480pt]
      \path[draw] (25.5000,152.5000) -- (25.5000,147.5000);



      \path[draw] (25.5000,128.5000) -- (25.5000,132.5000);



    \end{scope}
    \begin{scope}[cm={{0.74279,0.0,0.0,1.28515,(-186.22138,-161.30028)}},draw=ca0a0a4,dash pattern=on 1.22pt off 1.22pt,line cap=round,line join=round,line width=0.305pt,miter limit=4.00]
      \path[draw,dash pattern=on 1.22pt off 1.22pt,line width=0.305pt,miter limit=4.00] (60.5000,152.5000) -- (60.5000,128.5000);



    \end{scope}
    \begin{scope}[cm={{0.74279,0.0,0.0,1.28515,(-186.22138,-161.30028)}},draw=blue,line cap=round,line join=round,line width=0.480pt]
      \path[cm={{1.0,0.0,0.0,0.57583,(0.0,64.75889)}},draw] (60.5000,152.5000) -- (60.5000,147.5000);



      \path[cm={{1.0,0.0,0.0,0.70101,(0.0,38.33083)}},draw] (60.5000,128.5000) -- (60.5000,132.5000);



    \end{scope}
    \begin{scope}[cm={{0.74279,0.0,0.0,1.28515,(-186.22138,-161.30028)}},draw=ca0a0a4,dash pattern=on 1.22pt off 1.22pt,line cap=round,line join=round,line width=0.305pt,miter limit=4.00]
      \path[draw,dash pattern=on 1.22pt off 1.22pt,line width=0.305pt,miter limit=4.00] (95.5000,152.5000) -- (95.5000,128.5000);



    \end{scope}
    \begin{scope}[cm={{0.74279,0.0,0.0,1.28515,(-186.22138,-161.30028)}},draw=blue,line cap=round,line join=round,line width=0.480pt]
      \path[cm={{1.0,0.0,0.0,0.57583,(0.0,64.75889)}},draw] (95.5000,152.5000) -- (95.5000,147.5000);



      \path[cm={{1.0,0.0,0.0,0.70101,(0.0,38.33083)}},draw] (95.5000,128.5000) -- (95.5000,132.5000);



    \end{scope}
    \begin{scope}[cm={{0.74279,0.0,0.0,1.28515,(-186.22138,-161.30028)}},draw=ca0a0a4,dash pattern=on 1.22pt off 1.22pt,line cap=round,line join=round,line width=0.305pt,miter limit=4.00]
      \path[draw,dash pattern=on 1.22pt off 1.22pt,line width=0.305pt,miter limit=4.00] (130.5000,140.5000) -- (130.5000,128.5000);



    \end{scope}
    \begin{scope}[cm={{0.99623,0.0,0.0,1.3704,(-320.15815,-158.86873)}},draw=ca0a0a4,dash pattern=on 1.02pt off 1.02pt,even odd rule,line cap=round,line join=round,line width=0.255pt,miter limit=4.00]
      \path[draw,dash pattern=on 1.02pt off 1.02pt,even odd rule,line cap=round,line width=0.255pt,miter limit=4.00] (41.5000,101.5000) -- (127.5000,101.5000);



    \end{scope}
    \begin{scope}[cm={{0.95389,0.0,0.0,0.95389,(-228.06584,308.36388)}},draw=blue,line cap=rect,line join=bevel,line width=0.800pt]
      \begin{scope}[cm={{1.04485,0.0,0.0,1.39271,(-96.58638,-490.878)}},draw=ca0a0a4,dash pattern=on 1.56pt off 1.56pt,line cap=round,line join=round,line width=0.259pt,miter limit=4.00]
        \path[draw,dash pattern=on 1.56pt off 1.56pt,line width=0.259pt,miter limit=4.00] (70.5000,84.5000) -- (70.5000,30.5000);



        \path[draw,dash pattern=on 1.56pt off 1.56pt,line width=0.259pt,miter limit=4.00] (70.5000,14.5000) -- (70.5000,8.5000);



      \end{scope}
      \begin{scope}[cm={{1.04485,0.0,0.0,1.39271,(-96.58638,-490.7838)}},draw=ca0a0a4,dash pattern=on 1.56pt off 1.56pt,line cap=round,line join=round,line width=0.259pt,miter limit=4.00]
        \path[draw,dash pattern=on 1.56pt off 1.56pt,line width=0.259pt,miter limit=4.00] (98.5000,84.5000) -- (98.5000,30.5000);



        \path[draw,dash pattern=on 1.56pt off 1.56pt,line width=0.259pt,miter limit=4.00] (98.5000,14.5000) -- (98.5000,8.5000);



      \end{scope}
      \path[fill=ce10000] (46.7690,-262.2919) node[above right] (text1320) {\scriptsize -20};



      \begin{scope}[cm={{1.04439,0.0,0.0,1.41697,(-96.54445,-493.13085)}},draw=blue,line cap=round,line join=round,line width=0.480pt]
        \path[draw=cd9d9d9,line cap=rect,line join=miter,line width=2.127pt,miter limit=4.00] (41.5000,8.5000) -- (41.5000,84.5000) -- (127.5000,84.5000) -- (127.5000,8.5000) -- (41.5000,8.5000);



      \end{scope}
      \begin{scope}[cm={{1.04439,0.0,0.0,1.41697,(-96.54445,-493.13085)}},draw=ca0a0a4,dash pattern=on 1.03pt off 1.03pt,line cap=round,line join=round,line width=0.257pt,miter limit=4.00]
        \path[draw,dash pattern=on 1.03pt off 1.03pt,line width=0.257pt,miter limit=4.00] (41.5000,74.5000) -- (127.5000,74.5000);



      \end{scope}
      \begin{scope}[cm={{1.04439,0.0,0.0,1.41697,(-96.54445,-493.13085)}},draw=blue,line cap=round,line join=round,line width=0.480pt]
        \path[cm={{1.15551,0.0,0.0,1.0,(-6.40682,0.0)}},draw] (41.5000,74.5000) -- (44.5000,74.5000);



        \path[cm={{1.15551,0.0,0.0,1.0,(-19.98953,0.0)}},draw] (127.5000,74.5000) -- (124.5000,74.5000);



      \end{scope}
      \begin{scope}[cm={{1.05023,0.0,0.0,1.05023,(-193.26478,-275.39592)}},draw=blue,line cap=rect,line join=bevel,line width=0.800pt]
        \path[fill=blue] (119.0842,-104.1987) node[above right] (text34-5) {\scriptsize 32};



      \end{scope}
      \begin{scope}[cm={{1.04439,0.0,0.0,1.41697,(-96.54445,-493.13085)}},draw=ca0a0a4,dash pattern=on 1.03pt off 1.03pt,line cap=round,line join=round,line width=0.257pt,miter limit=4.00]
        \path[draw,dash pattern=on 1.03pt off 1.03pt,line width=0.257pt,miter limit=4.00] (41.5000,48.5000) -- (127.5000,48.5000);



      \end{scope}
      \begin{scope}[cm={{1.04439,0.0,0.0,1.41697,(-96.54445,-493.13085)}},draw=blue,line cap=round,line join=round,line width=0.480pt]
        \path[cm={{1.15551,0.0,0.0,1.0,(-6.40682,0.0)}},draw] (41.5000,48.5000) -- (44.5000,48.5000);



        \path[cm={{1.15551,0.0,0.0,1.0,(-19.98953,0.0)}},draw] (127.5000,48.5000) -- (124.5000,48.5000);



      \end{scope}
      \begin{scope}[cm={{1.05023,0.0,0.0,1.05023,(-68.14069,-421.00079)}},draw=blue,line cap=rect,line join=bevel,line width=0.800pt]
        \path[fill=blue] (0.0000,0.0000) node[above right] (text64-7) {\scriptsize 36};



      \end{scope}
      \begin{scope}[cm={{1.04485,0.0,0.0,1.39271,(-88.8098,-492.60283)}},draw=ca0a0a4,dash pattern=on 1.56pt off 1.56pt,line cap=round,line join=round,line width=0.259pt,miter limit=4.00]
        \path[shift={(-7.44276,0)},draw,dash pattern=on 1.56pt off 1.56pt,line width=0.259pt,miter limit=4.00] (41.5000,22.5000) -- (46.5000,22.5000);



        \path[shift={(-7.44276,0)},draw,dash pattern=on 1.56pt off 1.56pt,line width=0.259pt,miter limit=4.00] (103.5000,22.5000) -- (127.5000,22.5000);



      \end{scope}
      \begin{scope}[cm={{1.04439,0.0,0.0,1.41697,(-96.54445,-493.13085)}},draw=blue,line cap=round,line join=round,line width=0.480pt]
        \path[cm={{1.15551,0.0,0.0,1.0,(-6.40682,0.0)}},draw] (41.5000,22.5000) -- (44.5000,22.5000);



        \path[cm={{1.15551,0.0,0.0,1.0,(-19.98953,0.0)}},draw] (127.5000,22.5000) -- (124.5000,22.5000);



      \end{scope}
      \begin{scope}[cm={{1.05023,0.0,0.0,1.05023,(-68.2079,-459.24995)}},draw=blue,line cap=rect,line join=bevel,line width=0.800pt]
        \path[fill=blue] (0.0000,0.0000) node[above right] (text96) {\scriptsize 40};



      \end{scope}
      \begin{scope}[cm={{1.04439,0.0,0.0,1.41697,(-96.54445,-493.13085)}},draw=ca0a0a4,dash pattern=on 0.40pt off 0.80pt,line cap=round,line join=round,line width=0.400pt]
        \path[draw] (41.5000,84.5000) -- (41.5000,8.5000);



      \end{scope}
      \begin{scope}[cm={{1.04439,0.0,0.0,1.41697,(-96.54445,-493.13085)}},draw=blue,line cap=round,line join=round,line width=0.480pt]
        \path[draw] (41.5000,84.5000) -- (41.5000,80.5000);



        \path[draw] (41.5000,8.5000) -- (41.5000,11.5000);



      \end{scope}
      \begin{scope}[cm={{1.04439,0.0,0.0,1.41697,(-96.54445,-493.13085)}},draw=blue,line cap=round,line join=round,line width=0.480pt]
        \path[cm={{1.0,0.0,0.0,0.66652,(0.0,27.84811)}},draw] (70.5000,84.5000) -- (70.5000,80.5000);



        \path[cm={{1.0,0.0,0.0,0.85167,(0.00012,1.21632)}},draw] (70.5000,8.5000) -- (70.5000,11.5000);



      \end{scope}
      \begin{scope}[cm={{1.04439,0.0,0.0,1.41697,(-96.54445,-493.13085)}},draw=blue,line cap=round,line join=round,line width=0.480pt]
        \path[cm={{1.0,0.0,0.0,0.66652,(0.0,27.84811)}},draw] (98.5000,84.5000) -- (98.5000,80.5000);



        \path[cm={{1.0,0.0,0.0,0.85167,(0.14844,1.21632)}},draw] (98.5000,8.5000) -- (98.5000,11.5000);



      \end{scope}
      \begin{scope}[cm={{1.04439,0.0,0.0,1.41697,(-96.54445,-493.13085)}},draw=ca0a0a4,dash pattern=on 0.40pt off 0.80pt,line cap=round,line join=round,line width=0.400pt]
        \path[draw] (127.5000,84.5000) -- (127.5000,8.5000);



      \end{scope}
      \begin{scope}[cm={{1.04439,0.0,0.0,1.41697,(-96.54445,-493.13085)}},draw=blue,line cap=round,line join=round,line width=0.480pt]
        \path[draw] (127.5000,84.5000) -- (127.5000,80.5000);



        \path[draw] (127.5000,8.5000) -- (127.5000,11.5000);



      \end{scope}
      \begin{scope}[cm={{1.05016,0.0,0.0,1.02141,(-35.68793,-461.26773)}},draw=blue,line cap=rect,line join=bevel,line width=0.800pt]
        \path[fill=blue] (0.0000,0.0000) node[above right] (text276) {\scriptsize $\Upsilon(t)$};



      \end{scope}
      \begin{scope}[cm={{0.76164,0.0,0.0,1.39271,(-67.04184,-490.7838)}},draw=blue,line cap=round,line join=round,line width=0.480pt]
        \path[draw,even odd rule] (71.5000,18.5000) -- (98.5000,18.5000);



      \end{scope}
      \begin{scope}[cm={{1.04439,0.0,0.0,1.41697,(-96.54445,-493.13085)}},draw=blue,line cap=round,line join=round,line width=0.480pt]
        \path[draw] (41.6000,17.0000) -- (41.6000,17.0000) -- (41.7000,17.8000) -- (41.9000,18.6000) -- (42.0000,19.4000) -- (42.2000,20.2000) -- (42.3000,20.9000) -- (42.5000,21.6000) -- (42.6000,22.4000) -- (42.7000,23.1000) -- (42.9000,23.8000) -- (43.0000,24.5000) -- (43.2000,25.2000) -- (43.3000,25.8000) -- (43.5000,26.5000) -- (43.6000,27.1000) -- (43.7000,27.7000) -- (43.9000,28.4000) -- (44.0000,29.0000) -- (44.2000,29.6000) -- (44.3000,30.1000) -- (44.5000,30.7000) -- (44.6000,31.3000) -- (44.7000,31.8000) -- (44.9000,32.4000) -- (45.0000,32.9000) -- (45.2000,33.4000) -- (45.3000,34.0000) -- (45.5000,34.5000) -- (45.6000,35.0000) -- (45.7000,35.5000) -- (45.9000,35.9000) -- (46.0000,36.4000) -- (46.2000,36.9000) -- (46.3000,37.3000) -- (46.5000,37.8000) -- (46.6000,38.2000) -- (46.7000,38.7000) -- (46.9000,39.1000) -- (47.0000,39.5000) -- (47.2000,39.9000) -- (47.3000,40.3000) -- (47.5000,40.7000) -- (47.6000,41.1000) -- (47.7000,41.5000) -- (47.9000,41.9000) -- (48.0000,42.2000) -- (48.2000,42.6000) -- (48.3000,43.0000) -- (48.5000,43.3000) -- (48.6000,43.6000) -- (48.7000,44.0000) -- (48.9000,44.3000) -- (49.0000,44.6000) -- (49.2000,45.0000) -- (49.3000,45.3000) -- (49.5000,45.6000) -- (49.6000,45.9000) -- (49.7000,46.2000) -- (49.9000,46.5000) -- (50.0000,46.8000) -- (50.2000,47.0000) -- (50.3000,47.3000) -- (50.5000,47.6000) -- (50.6000,47.8000) -- (50.7000,48.1000) -- (50.9000,48.4000) -- (51.0000,48.6000) -- (51.2000,48.8000) -- (51.3000,49.1000) -- (51.5000,49.3000) -- (51.6000,49.5000) -- (51.7000,49.8000) -- (51.9000,50.0000) -- (52.0000,50.2000) -- (52.2000,50.4000) -- (52.3000,50.6000) -- (52.5000,50.8000) -- (52.6000,51.0000) -- (52.7000,51.2000) -- (52.9000,51.4000) -- (53.0000,51.6000) -- (53.2000,51.7000) -- (53.3000,51.9000) -- (53.5000,52.1000) -- (53.6000,52.3000) -- (53.7000,52.4000) -- (53.9000,52.6000) -- (54.0000,52.7000) -- (54.2000,52.9000) -- (54.3000,53.0000) -- (54.5000,53.1000) -- (54.6000,53.3000) -- (54.7000,53.4000) -- (54.9000,53.5000) -- (55.0000,53.7000) -- (55.2000,53.8000) -- (55.3000,53.9000) -- (55.5000,54.0000) -- (55.6000,54.1000) -- (55.7000,54.2000) -- (55.9000,54.3000) -- (56.0000,54.4000) -- (56.2000,54.5000) -- (56.3000,54.6000) -- (56.5000,54.6000) -- (56.6000,54.7000) -- (56.7000,54.8000) -- (56.9000,54.9000) -- (57.0000,54.9000) -- (57.2000,55.0000) -- (57.3000,55.0000) -- (57.5000,55.1000) -- (57.6000,55.1000) -- (57.7000,55.2000) -- (57.9000,55.2000) -- (58.0000,55.2000) -- (58.2000,55.3000) -- (58.3000,55.3000) -- (58.5000,55.3000) -- (58.6000,55.3000) -- (58.7000,55.3000) -- (58.9000,55.3000) -- (59.0000,55.3000) -- (59.2000,55.3000) -- (59.3000,55.3000) -- (59.5000,55.3000) -- (59.6000,55.2000) -- (59.7000,55.2000) -- (59.9000,55.2000) -- (60.0000,55.1000) -- (60.2000,55.1000) -- (60.3000,55.0000) -- (60.5000,54.9000) -- (60.6000,54.9000) -- (60.7000,54.8000) -- (60.9000,54.7000) -- (61.0000,54.6000) -- (61.2000,54.5000) -- (61.3000,54.4000) -- (61.5000,54.2000) -- (61.6000,54.1000) -- (61.7000,54.0000) -- (61.9000,53.8000) -- (62.0000,53.7000) -- (62.2000,53.5000) -- (62.3000,53.3000) -- (62.5000,53.1000) -- (62.6000,53.0000) -- (62.7000,52.8000) -- (62.9000,52.6000) -- (63.0000,52.4000) -- (63.2000,52.2000) -- (63.3000,52.0000) -- (63.5000,51.7000) -- (63.6000,51.5000) -- (63.7000,51.3000) -- (63.9000,51.1000) -- (64.0000,50.8000) -- (64.2000,50.6000) -- (64.3000,50.4000) -- (64.5000,50.1000) -- (64.6000,49.9000) -- (64.7000,49.7000) -- (64.9000,49.4000) -- (65.0000,49.2000) -- (65.2000,49.0000) -- (65.3000,48.7000) -- (65.5000,48.5000) -- (65.6000,48.2000) -- (65.7000,48.0000) -- (65.9000,47.8000) -- (66.0000,47.5000) -- (66.2000,47.3000) -- (66.3000,47.1000) -- (66.5000,46.8000) -- (66.6000,46.6000) -- (66.7000,46.4000) -- (66.9000,46.1000) -- (67.0000,45.9000) -- (67.2000,45.7000) -- (67.3000,45.4000) -- (67.5000,45.2000) -- (67.6000,45.0000) -- (67.7000,44.8000) -- (67.9000,44.5000) -- (68.0000,44.3000) -- (68.2000,44.1000) -- (68.3000,43.9000) -- (68.5000,43.7000) -- (68.6000,43.5000) -- (68.7000,43.2000) -- (68.9000,43.0000) -- (69.0000,42.8000) -- (69.2000,42.6000) -- (69.3000,42.4000) -- (69.5000,42.2000) -- (69.6000,42.0000) -- (69.7000,41.8000) -- (69.9000,41.6000) -- (70.0000,41.4000) -- (70.2000,41.2000) -- (70.3000,41.0000) -- (70.5000,40.9000) -- (70.6000,40.7000) -- (70.7000,40.5000) -- (70.9000,40.3000) -- (71.0000,40.1000) -- (71.2000,40.0000) -- (71.3000,39.8000) -- (71.5000,39.6000) -- (71.6000,39.4000) -- (71.7000,39.3000) -- (71.9000,39.1000) -- (72.0000,38.9000) -- (72.2000,38.8000) -- (72.3000,38.6000) -- (72.5000,38.5000) -- (72.6000,38.3000) -- (72.7000,38.1000) -- (72.9000,38.0000) -- (73.0000,37.9000) -- (73.2000,37.7000) -- (73.3000,37.6000) -- (73.5000,37.4000) -- (73.6000,37.3000) -- (73.7000,37.1000) -- (73.9000,37.0000) -- (74.0000,36.9000) -- (74.2000,36.8000) -- (74.3000,36.6000) -- (74.5000,36.5000) -- (74.6000,36.4000) -- (74.7000,36.2000) -- (74.9000,36.1000) -- (75.0000,36.0000) -- (75.2000,35.9000) -- (75.3000,35.8000) -- (75.5000,35.7000) -- (75.6000,35.6000) -- (75.7000,35.5000) -- (75.9000,35.4000) -- (76.0000,35.3000) -- (76.2000,35.2000) -- (76.3000,35.1000) -- (76.5000,35.0000) -- (76.6000,34.9000) -- (76.7000,34.8000) -- (76.9000,34.7000) -- (77.0000,34.6000) -- (77.2000,34.5000) -- (77.3000,34.4000) -- (77.5000,34.4000) -- (77.6000,34.3000) -- (77.7000,34.2000) -- (77.9000,34.1000) -- (78.0000,34.1000) -- (78.2000,34.0000) -- (78.3000,33.9000) -- (78.5000,33.9000) -- (78.6000,33.8000) -- (78.7000,33.7000) -- (78.9000,33.7000) -- (79.0000,33.6000) -- (79.2000,33.6000) -- (79.3000,33.5000) -- (79.5000,33.5000) -- (79.6000,33.4000) -- (79.7000,33.4000) -- (79.9000,33.3000) -- (80.0000,33.3000) -- (80.2000,33.2000) -- (80.3000,33.2000) -- (80.5000,33.1000) -- (80.6000,33.1000) -- (80.7000,33.1000) -- (80.9000,33.0000) -- (81.0000,33.0000) -- (81.2000,33.0000) -- (81.3000,32.9000) -- (81.5000,32.9000) -- (81.6000,32.9000) -- (81.7000,32.9000) -- (81.9000,32.8000) -- (82.0000,32.8000) -- (82.2000,32.8000) -- (82.3000,32.8000) -- (82.5000,32.8000) -- (82.6000,32.7000) -- (82.7000,32.7000) -- (82.9000,32.7000) -- (83.0000,32.7000) -- (83.2000,32.7000) -- (83.3000,32.7000) -- (83.5000,32.7000) -- (83.6000,32.7000) -- (83.7000,32.7000) -- (83.9000,32.6000) -- (84.0000,32.6000) -- (84.2000,32.6000) -- (84.3000,32.6000) -- (84.5000,32.6000) -- (84.6000,32.6000) -- (84.7000,32.6000) -- (84.9000,32.7000) -- (85.0000,32.7000) -- (85.2000,32.7000) -- (85.3000,32.7000) -- (85.4000,32.7000) -- (85.6000,32.7000) -- (85.7000,32.7000) -- (85.9000,32.7000) -- (86.0000,32.7000) -- (86.2000,32.7000) -- (86.3000,32.8000) -- (86.4000,32.8000) -- (86.6000,32.8000) -- (86.7000,32.8000) -- (86.9000,32.8000) -- (87.0000,32.8000) -- (87.2000,32.9000) -- (87.3000,32.9000) -- (87.4000,32.9000) -- (87.6000,32.9000) -- (87.7000,33.0000) -- (87.9000,33.0000) -- (88.0000,33.0000) -- (88.2000,33.0000) -- (88.3000,33.1000) -- (88.4000,33.1000) -- (88.6000,33.1000) -- (88.7000,33.1000) -- (88.9000,33.2000) -- (89.0000,33.2000) -- (89.2000,33.2000) -- (89.3000,33.3000) -- (89.4000,33.3000) -- (89.6000,33.3000) -- (89.7000,33.4000) -- (89.9000,33.4000) -- (90.0000,33.4000) -- (90.2000,33.5000) -- (90.3000,33.5000) -- (90.4000,33.6000) -- (90.6000,33.6000) -- (90.7000,33.6000) -- (90.9000,33.7000) -- (91.0000,33.7000) -- (91.2000,33.8000) -- (91.3000,33.8000) -- (91.4000,33.8000) -- (91.6000,33.9000) -- (91.7000,33.9000) -- (91.9000,34.0000) -- (92.0000,34.0000) -- (92.2000,34.1000) -- (92.3000,34.1000) -- (92.4000,34.1000) -- (92.6000,34.2000) -- (92.7000,34.2000) -- (92.9000,34.3000) -- (93.0000,34.3000) -- (93.2000,34.4000) -- (93.3000,34.4000) -- (93.4000,34.5000) -- (93.6000,34.5000) -- (93.7000,34.6000) -- (93.9000,34.6000) -- (94.0000,34.7000) -- (94.2000,34.7000) -- (94.3000,34.8000) -- (94.4000,34.8000) -- (94.6000,34.9000) -- (94.7000,34.9000) -- (94.9000,35.0000) -- (95.0000,35.0000) -- (95.2000,35.1000) -- (95.3000,35.1000) -- (95.4000,35.2000) -- (95.6000,35.2000) -- (95.7000,35.3000) -- (95.9000,35.3000) -- (96.0000,35.4000) -- (96.2000,35.4000) -- (96.3000,35.5000) -- (96.4000,35.5000) -- (96.6000,35.6000) -- (96.7000,35.6000) -- (96.9000,35.7000) -- (97.0000,35.7000) -- (97.2000,35.8000) -- (97.3000,35.9000) -- (97.4000,35.9000) -- (97.6000,36.0000) -- (97.7000,36.0000) -- (97.9000,36.1000) -- (98.0000,36.1000) -- (98.2000,36.2000) -- (98.3000,36.2000) -- (98.4000,36.3000) -- (98.6000,36.4000) -- (98.7000,36.4000) -- (98.9000,36.5000) -- (99.0000,36.6000) -- (99.2000,36.6000) -- (99.3000,36.7000) -- (99.4000,36.8000) -- (99.6000,36.8000) -- (99.7000,36.9000) -- (99.9000,37.0000) -- (100.0000,37.0000) -- (100.2000,37.1000) -- (100.3000,37.2000) -- (100.4000,37.2000) -- (100.6000,37.3000) -- (100.7000,37.4000) -- (100.9000,37.4000) -- (101.0000,37.5000) -- (101.2000,37.6000) -- (101.3000,37.6000) -- (101.4000,37.7000) -- (101.6000,37.8000) -- (101.7000,37.9000) -- (101.9000,37.9000) -- (102.0000,38.0000) -- (102.2000,38.1000) -- (102.3000,38.2000) -- (102.4000,38.2000) -- (102.6000,38.3000) -- (102.7000,38.4000) -- (102.9000,38.4000) -- (103.0000,38.5000) -- (103.2000,38.6000) -- (103.3000,38.7000) -- (103.4000,38.7000) -- (103.6000,38.8000) -- (103.7000,38.9000) -- (103.9000,38.9000) -- (104.0000,39.0000) -- (104.2000,39.1000) -- (104.3000,39.2000) -- (104.4000,39.2000) -- (104.6000,39.3000) -- (104.7000,39.4000) -- (104.9000,39.4000) -- (105.0000,39.5000) -- (105.2000,39.6000) -- (105.3000,39.7000) -- (105.4000,39.7000) -- (105.6000,39.8000) -- (105.7000,39.9000) -- (105.9000,39.9000) -- (106.0000,40.0000) -- (106.2000,40.1000) -- (106.3000,40.1000) -- (106.4000,40.2000) -- (106.6000,40.3000) -- (106.7000,40.3000) -- (106.9000,40.4000) -- (107.0000,40.5000) -- (107.2000,40.5000) -- (107.3000,40.6000) -- (107.4000,40.6000) -- (107.6000,40.7000) -- (107.7000,40.8000) -- (107.9000,40.8000) -- (108.0000,40.9000) -- (108.2000,41.0000) -- (108.3000,41.0000) -- (108.4000,41.1000) -- (108.6000,41.1000) -- (108.7000,41.2000) -- (108.9000,41.3000) -- (109.0000,41.3000) -- (109.2000,41.4000) -- (109.3000,41.4000) -- (109.4000,41.5000) -- (109.6000,41.5000) -- (109.7000,41.6000) -- (109.9000,41.6000) -- (110.0000,41.7000) -- (110.2000,41.7000) -- (110.3000,41.8000) -- (110.4000,41.9000) -- (110.6000,41.9000) -- (110.7000,42.0000) -- (110.9000,42.0000) -- (111.0000,42.1000) -- (111.2000,42.1000) -- (111.3000,42.1000) -- (111.4000,42.2000) -- (111.6000,42.2000) -- (111.7000,42.3000) -- (111.9000,42.3000) -- (112.0000,42.4000) -- (112.2000,42.4000) -- (112.3000,42.5000) -- (112.4000,42.5000) -- (112.6000,42.6000) -- (112.7000,42.6000) -- (112.9000,42.6000) -- (113.0000,42.7000) -- (113.2000,42.7000) -- (113.3000,42.8000) -- (113.4000,42.8000) -- (113.6000,42.8000) -- (113.7000,42.9000) -- (113.9000,42.9000) -- (114.0000,43.0000) -- (114.2000,43.0000) -- (114.3000,43.0000) -- (114.4000,43.1000) -- (114.6000,43.1000) -- (114.7000,43.1000) -- (114.9000,43.2000) -- (115.0000,43.2000) -- (115.2000,43.2000) -- (115.3000,43.3000) -- (115.4000,43.3000) -- (115.6000,43.3000) -- (115.7000,43.4000) -- (115.9000,43.4000) -- (116.0000,43.4000) -- (116.2000,43.4000) -- (116.3000,43.5000) -- (116.4000,43.5000) -- (116.6000,43.5000) -- (116.7000,43.6000) -- (116.9000,43.6000) -- (117.0000,43.6000) -- (117.2000,43.6000) -- (117.3000,43.7000) -- (117.4000,43.7000) -- (117.6000,43.7000) -- (117.7000,43.7000) -- (117.9000,43.7000) -- (118.0000,43.8000) -- (118.2000,43.8000) -- (118.3000,43.8000) -- (118.4000,43.8000) -- (118.6000,43.9000) -- (118.7000,43.9000) -- (118.9000,43.9000) -- (119.0000,43.9000) -- (119.2000,43.9000) -- (119.3000,44.0000) -- (119.4000,44.0000) -- (119.6000,44.0000) -- (119.7000,44.0000) -- (119.9000,44.0000) -- (120.0000,44.0000) -- (120.2000,44.0000) -- (120.3000,44.1000) -- (120.4000,44.1000) -- (120.6000,44.1000) -- (120.7000,44.1000) -- (120.9000,44.1000) -- (121.0000,44.1000) -- (121.2000,44.1000) -- (121.3000,44.2000) -- (121.4000,44.2000) -- (121.6000,44.2000) -- (121.7000,44.2000) -- (121.9000,44.2000) -- (122.0000,44.2000) -- (122.2000,44.2000) -- (122.3000,44.2000) -- (122.4000,44.2000) -- (122.6000,44.2000) -- (122.7000,44.3000) -- (122.9000,44.3000) -- (123.0000,44.3000) -- (123.2000,44.3000) -- (123.3000,44.3000) -- (123.4000,44.3000) -- (123.6000,44.3000) -- (123.7000,44.3000) -- (123.9000,44.3000) -- (124.0000,44.3000) -- (124.2000,44.3000) -- (124.3000,44.3000) -- (124.4000,44.3000) -- (124.6000,44.3000) -- (124.7000,44.3000) -- (124.9000,44.3000) -- (125.0000,44.3000) -- (125.2000,44.3000) -- (125.3000,44.3000) -- (125.4000,44.4000) -- (125.6000,44.4000) -- (125.7000,44.4000) -- (125.9000,44.4000) -- (126.0000,44.4000) -- (126.2000,44.4000) -- (126.3000,44.4000) -- (126.4000,44.4000) -- (126.6000,44.4000) -- (126.7000,44.4000) -- (126.9000,44.4000) -- (127.0000,44.4000) -- (127.2000,44.4000) -- (127.3000,44.4000);



      \end{scope}
      \begin{scope}[cm={{1.05016,0.0,0.0,1.02141,(-33.29971,-448.94892)}},draw=blue,line cap=rect,line join=bevel,line width=0.800pt]
        \path[fill=blue] (0.0000,0.0000) node[above right] (text312) {\scriptsize $\hat{y}(t)$};



      \end{scope}
      \begin{scope}[cm={{1.04485,0.0,0.0,1.39271,(-87.20649,-490.7838)}},draw=cff0000,line cap=round,line join=round,line width=0.480pt]
        \path[draw,even odd rule,line width=0.410pt] (71.4187,26.5000) -- (91.1002,26.5000);



      \end{scope}
      \begin{scope}[cm={{1.04439,0.0,0.0,1.41697,(-96.54445,-493.13085)}},draw=cff0000,line cap=round,line join=round,line width=0.480pt]
        \path[draw] (41.6000,21.8000) -- (41.6000,21.8000) -- (41.7000,17.5000) -- (41.9000,18.2000) -- (42.0000,19.1000) -- (42.2000,20.0000) -- (42.3000,20.9000) -- (42.5000,21.7000) -- (42.6000,22.5000) -- (42.7000,23.3000) -- (42.9000,24.1000) -- (43.0000,24.8000) -- (43.2000,25.5000) -- (43.3000,26.2000) -- (43.5000,26.9000) -- (43.6000,27.6000) -- (43.7000,28.2000) -- (43.9000,28.8000) -- (44.0000,29.4000) -- (44.2000,30.0000) -- (44.3000,30.5000) -- (44.5000,31.1000) -- (44.6000,31.6000) -- (44.7000,32.1000) -- (44.9000,32.7000) -- (45.0000,33.2000) -- (45.2000,33.7000) -- (45.3000,34.2000) -- (45.5000,34.6000) -- (45.6000,35.1000) -- (45.7000,35.6000) -- (45.9000,36.1000) -- (46.0000,36.5000) -- (46.2000,37.0000) -- (46.3000,37.5000) -- (46.5000,37.9000) -- (46.6000,38.4000) -- (46.7000,38.8000) -- (46.9000,39.3000) -- (47.0000,39.7000) -- (47.2000,40.1000) -- (47.3000,40.6000) -- (47.5000,41.0000) -- (47.6000,41.4000) -- (47.7000,41.8000) -- (47.9000,42.2000) -- (48.0000,42.5000) -- (48.2000,42.9000) -- (48.3000,43.3000) -- (48.5000,43.6000) -- (48.6000,43.9000) -- (48.7000,44.3000) -- (48.9000,44.6000) -- (49.0000,44.9000) -- (49.2000,45.2000) -- (49.3000,45.5000) -- (49.5000,45.8000) -- (49.6000,46.0000) -- (49.7000,46.3000) -- (49.9000,46.6000) -- (50.0000,46.8000) -- (50.2000,47.1000) -- (50.3000,47.4000) -- (50.5000,47.6000) -- (50.6000,47.9000) -- (50.7000,48.1000) -- (50.9000,48.4000) -- (51.0000,48.7000) -- (51.2000,48.9000) -- (51.3000,49.2000) -- (51.5000,49.4000) -- (51.6000,49.7000) -- (51.7000,50.0000) -- (51.9000,50.2000) -- (52.0000,50.5000) -- (52.2000,50.8000) -- (52.3000,51.0000) -- (52.5000,51.3000) -- (52.6000,51.5000) -- (52.7000,51.7000) -- (52.9000,51.9000) -- (53.0000,52.1000) -- (53.2000,52.3000) -- (53.3000,52.5000) -- (53.5000,52.7000) -- (53.6000,52.8000) -- (53.7000,52.9000) -- (53.9000,53.1000) -- (54.0000,53.2000) -- (54.2000,53.3000) -- (54.3000,53.3000) -- (54.5000,53.4000) -- (54.6000,53.5000) -- (54.7000,53.5000) -- (54.9000,53.6000) -- (55.0000,53.6000) -- (55.2000,53.6000) -- (55.3000,53.7000) -- (55.5000,53.7000) -- (55.6000,53.7000) -- (55.7000,53.8000) -- (55.9000,53.8000) -- (56.0000,53.8000) -- (56.2000,53.9000) -- (56.3000,54.0000) -- (56.5000,54.0000) -- (56.6000,54.1000) -- (56.7000,54.2000) -- (56.9000,54.3000) -- (57.0000,54.4000) -- (57.2000,54.5000) -- (57.3000,54.6000) -- (57.5000,54.7000) -- (57.6000,54.8000) -- (57.7000,54.9000) -- (57.9000,55.0000) -- (58.0000,55.1000) -- (58.2000,55.2000) -- (58.3000,55.3000) -- (58.5000,55.3000) -- (58.6000,55.4000) -- (58.7000,55.4000) -- (58.9000,55.5000) -- (59.0000,55.5000) -- (59.2000,55.5000) -- (59.3000,55.5000) -- (59.5000,55.5000) -- (59.6000,55.5000) -- (59.7000,55.5000) -- (59.9000,55.4000) -- (60.0000,55.3000) -- (60.2000,55.3000) -- (60.3000,55.2000) -- (60.5000,55.1000) -- (60.6000,55.0000) -- (60.7000,54.9000) -- (60.9000,54.8000) -- (61.0000,54.6000) -- (61.2000,54.5000) -- (61.3000,54.4000) -- (61.5000,54.2000) -- (61.6000,54.0000) -- (61.7000,53.9000) -- (61.9000,53.7000) -- (62.0000,53.5000) -- (62.2000,53.3000) -- (62.3000,53.1000) -- (62.5000,52.9000) -- (62.6000,52.7000) -- (62.7000,52.5000) -- (62.9000,52.3000) -- (63.0000,52.1000) -- (63.2000,51.9000) -- (63.3000,51.7000) -- (63.5000,51.5000) -- (63.6000,51.3000) -- (63.7000,51.1000) -- (63.9000,50.9000) -- (64.0000,50.7000) -- (64.2000,50.5000) -- (64.3000,50.3000) -- (64.5000,50.0000) -- (64.6000,49.8000) -- (64.7000,49.6000) -- (64.9000,49.4000) -- (65.0000,49.2000) -- (65.2000,49.0000) -- (65.3000,48.8000) -- (65.5000,48.6000) -- (65.6000,48.3000) -- (65.7000,48.1000) -- (65.9000,47.9000) -- (66.0000,47.7000) -- (66.2000,47.5000) -- (66.3000,47.3000) -- (66.5000,47.0000) -- (66.6000,46.8000) -- (66.7000,46.6000) -- (66.9000,46.4000) -- (67.0000,46.2000) -- (67.2000,45.9000) -- (67.3000,45.7000) -- (67.5000,45.5000) -- (67.6000,45.3000) -- (67.7000,45.0000) -- (67.9000,44.8000) -- (68.0000,44.6000) -- (68.2000,44.3000) -- (68.3000,44.1000) -- (68.5000,43.9000) -- (68.6000,43.6000) -- (68.7000,43.4000) -- (68.9000,43.2000) -- (69.0000,43.0000) -- (69.2000,42.7000) -- (69.3000,42.5000) -- (69.5000,42.3000) -- (69.6000,42.1000) -- (69.7000,41.8000) -- (69.9000,41.6000) -- (70.0000,41.4000) -- (70.2000,41.2000) -- (70.3000,41.0000) -- (70.5000,40.8000) -- (70.6000,40.6000) -- (70.7000,40.4000) -- (70.9000,40.2000) -- (71.0000,40.0000) -- (71.2000,39.8000) -- (71.3000,39.6000) -- (71.5000,39.4000) -- (71.6000,39.2000) -- (71.7000,39.0000) -- (71.9000,38.9000) -- (72.0000,38.7000) -- (72.2000,38.5000) -- (72.3000,38.3000) -- (72.5000,38.2000) -- (72.6000,38.0000) -- (72.7000,37.9000) -- (72.9000,37.7000) -- (73.0000,37.6000) -- (73.2000,37.4000) -- (73.3000,37.3000) -- (73.5000,37.2000) -- (73.6000,37.0000) -- (73.7000,36.9000) -- (73.9000,36.8000) -- (74.0000,36.7000) -- (74.2000,36.5000) -- (74.3000,36.4000) -- (74.5000,36.3000) -- (74.6000,36.2000) -- (74.7000,36.1000) -- (74.9000,36.0000) -- (75.0000,35.9000) -- (75.2000,35.8000) -- (75.3000,35.7000) -- (75.5000,35.6000) -- (75.6000,35.5000) -- (75.7000,35.4000) -- (75.9000,35.3000) -- (76.0000,35.3000) -- (76.2000,35.2000) -- (76.3000,35.1000) -- (76.5000,35.0000) -- (76.6000,34.9000) -- (76.7000,34.9000) -- (76.9000,34.8000) -- (77.0000,34.7000) -- (77.2000,34.6000) -- (77.3000,34.6000) -- (77.5000,34.5000) -- (77.6000,34.4000) -- (77.7000,34.4000) -- (77.9000,34.3000) -- (78.0000,34.3000) -- (78.2000,34.2000) -- (78.3000,34.1000) -- (78.5000,34.1000) -- (78.6000,34.0000) -- (78.7000,34.0000) -- (78.9000,33.9000) -- (79.0000,33.9000) -- (79.2000,33.8000) -- (79.3000,33.7000) -- (79.5000,33.7000) -- (79.6000,33.6000) -- (79.7000,33.6000) -- (79.9000,33.5000) -- (80.0000,33.5000) -- (80.2000,33.5000) -- (80.3000,33.4000) -- (80.5000,33.4000) -- (80.6000,33.3000) -- (80.7000,33.3000) -- (80.9000,33.2000) -- (81.0000,33.2000) -- (81.2000,33.2000) -- (81.3000,33.1000) -- (81.5000,33.1000) -- (81.6000,33.0000) -- (81.7000,33.0000) -- (81.9000,33.0000) -- (82.0000,32.9000) -- (82.2000,32.9000) -- (82.3000,32.9000) -- (82.5000,32.9000) -- (82.6000,32.8000) -- (82.7000,32.8000) -- (82.9000,32.8000) -- (83.0000,32.8000) -- (83.2000,32.7000) -- (83.3000,32.7000) -- (83.5000,32.7000) -- (83.6000,32.7000) -- (83.7000,32.7000) -- (83.9000,32.7000) -- (84.0000,32.6000) -- (84.2000,32.6000) -- (84.3000,32.6000) -- (84.5000,32.6000) -- (84.6000,32.6000) -- (84.7000,32.6000) -- (84.9000,32.6000) -- (85.0000,32.6000) -- (85.2000,32.6000) -- (85.3000,32.6000) -- (85.4000,32.6000) -- (85.6000,32.6000) -- (85.7000,32.6000) -- (85.9000,32.6000) -- (86.0000,32.6000) -- (86.2000,32.6000) -- (86.3000,32.6000) -- (86.4000,32.7000) -- (86.6000,32.7000) -- (86.7000,32.7000) -- (86.9000,32.7000) -- (87.0000,32.7000) -- (87.2000,32.7000) -- (87.3000,32.8000) -- (87.4000,32.8000) -- (87.6000,32.8000) -- (87.7000,32.8000) -- (87.9000,32.8000) -- (88.0000,32.9000) -- (88.2000,32.9000) -- (88.3000,32.9000) -- (88.4000,33.0000) -- (88.6000,33.0000) -- (88.7000,33.0000) -- (88.9000,33.1000) -- (89.0000,33.1000) -- (89.2000,33.1000) -- (89.3000,33.2000) -- (89.4000,33.2000) -- (89.6000,33.2000) -- (89.7000,33.3000) -- (89.9000,33.3000) -- (90.0000,33.4000) -- (90.2000,33.4000) -- (90.3000,33.5000) -- (90.4000,33.5000) -- (90.6000,33.5000) -- (90.7000,33.6000) -- (90.9000,33.6000) -- (91.0000,33.7000) -- (91.2000,33.7000) -- (91.3000,33.8000) -- (91.4000,33.8000) -- (91.6000,33.9000) -- (91.7000,33.9000) -- (91.9000,34.0000) -- (92.0000,34.0000) -- (92.2000,34.1000) -- (92.3000,34.1000) -- (92.4000,34.2000) -- (92.6000,34.2000) -- (92.7000,34.3000) -- (92.9000,34.3000) -- (93.0000,34.4000) -- (93.2000,34.4000) -- (93.3000,34.5000) -- (93.4000,34.5000) -- (93.6000,34.6000) -- (93.7000,34.6000) -- (93.9000,34.7000) -- (94.0000,34.7000) -- (94.2000,34.8000) -- (94.3000,34.8000) -- (94.4000,34.9000) -- (94.6000,34.9000) -- (94.7000,35.0000) -- (94.9000,35.1000) -- (95.0000,35.1000) -- (95.2000,35.2000) -- (95.3000,35.2000) -- (95.4000,35.3000) -- (95.6000,35.3000) -- (95.7000,35.4000) -- (95.9000,35.4000) -- (96.0000,35.5000) -- (96.2000,35.5000) -- (96.3000,35.6000) -- (96.4000,35.6000) -- (96.6000,35.7000) -- (96.7000,35.7000) -- (96.9000,35.8000) -- (97.0000,35.9000) -- (97.2000,35.9000) -- (97.3000,36.0000) -- (97.4000,36.0000) -- (97.6000,36.1000) -- (97.7000,36.1000) -- (97.9000,36.2000) -- (98.0000,36.2000) -- (98.2000,36.3000) -- (98.3000,36.3000) -- (98.4000,36.4000) -- (98.6000,36.5000) -- (98.7000,36.5000) -- (98.9000,36.6000) -- (99.0000,36.6000) -- (99.2000,36.7000) -- (99.3000,36.8000) -- (99.4000,36.8000) -- (99.6000,36.9000) -- (99.7000,37.0000) -- (99.9000,37.0000) -- (100.0000,37.1000) -- (100.2000,37.1000) -- (100.3000,37.2000) -- (100.4000,37.3000) -- (100.6000,37.3000) -- (100.7000,37.4000) -- (100.9000,37.5000) -- (101.0000,37.5000) -- (101.2000,37.6000) -- (101.3000,37.7000) -- (101.4000,37.7000) -- (101.6000,37.8000) -- (101.7000,37.9000) -- (101.9000,37.9000) -- (102.0000,38.0000) -- (102.2000,38.1000) -- (102.3000,38.1000) -- (102.4000,38.2000) -- (102.6000,38.3000) -- (102.7000,38.4000) -- (102.9000,38.4000) -- (103.0000,38.5000) -- (103.2000,38.6000) -- (103.3000,38.6000) -- (103.4000,38.7000) -- (103.6000,38.8000) -- (103.7000,38.8000) -- (103.9000,38.9000) -- (104.0000,39.0000) -- (104.2000,39.0000) -- (104.3000,39.1000) -- (104.4000,39.2000) -- (104.6000,39.2000) -- (104.7000,39.3000) -- (104.9000,39.4000) -- (105.0000,39.4000) -- (105.2000,39.5000) -- (105.3000,39.6000) -- (105.4000,39.6000) -- (105.6000,39.7000) -- (105.7000,39.8000) -- (105.9000,39.8000) -- (106.0000,39.9000) -- (106.2000,40.0000) -- (106.3000,40.0000) -- (106.4000,40.1000) -- (106.6000,40.2000) -- (106.7000,40.2000) -- (106.9000,40.3000) -- (107.0000,40.4000) -- (107.2000,40.4000) -- (107.3000,40.5000) -- (107.4000,40.5000) -- (107.6000,40.6000) -- (107.7000,40.7000) -- (107.9000,40.7000) -- (108.0000,40.8000) -- (108.2000,40.9000) -- (108.3000,40.9000) -- (108.4000,41.0000) -- (108.6000,41.0000) -- (108.7000,41.1000) -- (108.9000,41.2000) -- (109.0000,41.2000) -- (109.2000,41.3000) -- (109.3000,41.3000) -- (109.4000,41.4000) -- (109.6000,41.4000) -- (109.7000,41.5000) -- (109.9000,41.5000) -- (110.0000,41.6000) -- (110.2000,41.7000) -- (110.3000,41.7000) -- (110.4000,41.8000) -- (110.6000,41.8000) -- (110.7000,41.9000) -- (110.9000,41.9000) -- (111.0000,42.0000) -- (111.2000,42.0000) -- (111.3000,42.1000) -- (111.4000,42.1000) -- (111.6000,42.2000) -- (111.7000,42.2000) -- (111.9000,42.3000) -- (112.0000,42.3000) -- (112.2000,42.4000) -- (112.3000,42.4000) -- (112.4000,42.5000) -- (112.6000,42.5000) -- (112.7000,42.6000) -- (112.9000,42.6000) -- (113.0000,42.6000) -- (113.2000,42.7000) -- (113.3000,42.7000) -- (113.4000,42.8000) -- (113.6000,42.8000) -- (113.7000,42.9000) -- (113.9000,42.9000) -- (114.0000,42.9000) -- (114.2000,43.0000) -- (114.3000,43.0000) -- (114.4000,43.1000) -- (114.6000,43.1000) -- (114.7000,43.1000) -- (114.9000,43.2000) -- (115.0000,43.2000) -- (115.2000,43.2000) -- (115.3000,43.3000) -- (115.4000,43.3000) -- (115.6000,43.3000) -- (115.7000,43.4000) -- (115.9000,43.4000) -- (116.0000,43.4000) -- (116.2000,43.5000) -- (116.3000,43.5000) -- (116.4000,43.5000) -- (116.6000,43.6000) -- (116.7000,43.6000) -- (116.9000,43.6000) -- (117.0000,43.6000) -- (117.2000,43.7000) -- (117.3000,43.7000) -- (117.4000,43.7000) -- (117.6000,43.7000) -- (117.7000,43.8000) -- (117.9000,43.8000) -- (118.0000,43.8000) -- (118.2000,43.8000) -- (118.3000,43.9000) -- (118.4000,43.9000) -- (118.6000,43.9000) -- (118.7000,43.9000) -- (118.9000,44.0000) -- (119.0000,44.0000) -- (119.2000,44.0000) -- (119.3000,44.0000) -- (119.4000,44.0000) -- (119.6000,44.1000) -- (119.7000,44.1000) -- (119.9000,44.1000) -- (120.0000,44.1000) -- (120.2000,44.1000) -- (120.3000,44.1000) -- (120.4000,44.1000) -- (120.6000,44.2000) -- (120.7000,44.2000) -- (120.9000,44.2000) -- (121.0000,44.2000) -- (121.2000,44.2000) -- (121.3000,44.2000) -- (121.4000,44.2000) -- (121.6000,44.3000) -- (121.7000,44.3000) -- (121.9000,44.3000) -- (122.0000,44.3000) -- (122.2000,44.3000) -- (122.3000,44.3000) -- (122.4000,44.3000) -- (122.6000,44.3000) -- (122.7000,44.3000) -- (122.9000,44.3000) -- (123.0000,44.3000) -- (123.2000,44.4000) -- (123.3000,44.4000) -- (123.4000,44.4000) -- (123.6000,44.4000) -- (123.7000,44.4000) -- (123.9000,44.4000) -- (124.0000,44.4000) -- (124.2000,44.4000) -- (124.3000,44.4000) -- (124.4000,44.4000) -- (124.6000,44.4000) -- (124.7000,44.4000) -- (124.9000,44.4000) -- (125.0000,44.4000) -- (125.2000,44.4000) -- (125.3000,44.4000) -- (125.4000,44.4000) -- (125.6000,44.4000) -- (125.7000,44.4000) -- (125.9000,44.4000) -- (126.0000,44.4000) -- (126.2000,44.4000) -- (126.3000,44.4000) -- (126.4000,44.4000) -- (126.6000,44.4000) -- (126.7000,44.4000) -- (126.9000,44.4000) -- (127.0000,44.4000) -- (127.2000,44.4000) -- (127.3000,44.4000);



      \end{scope}
      \begin{scope}[cm={{1.04439,0.0,0.0,1.41697,(-96.54445,-495.32852)}},draw=blue,line cap=round,line join=round,line width=0.480pt]
        \path[draw=blue] (41.5000,10.0509) -- (41.5000,86.0509) -- (127.5000,86.0509) -- (127.5000,10.0509) -- (41.5000,10.0509);



      \end{scope}
      \begin{scope}[cm={{1.27491,0.0,0.0,1.41542,(-124.58597,-493.0531)}},fill=cffffff]
        \path[fill,rounded corners=0.0000cm] (81.0000,50.0000) rectangle (121.0000,78.0000);



      \end{scope}
      \begin{scope}[cm={{1.27491,0.0,0.0,1.41542,(-124.58597,-493.0531)}},draw=ca0a0a4,dash pattern=on 0.93pt off 0.93pt,line cap=round,line join=round,line width=0.233pt,miter limit=4.00]
        \path[draw,dash pattern=on 0.93pt off 0.93pt,line width=0.233pt,miter limit=4.00] (81.5000,55.5000) -- (121.5000,55.5000);



      \end{scope}
      \begin{scope}[cm={{1.27491,0.0,0.0,1.41542,(-124.58597,-493.0531)}},draw=blue,line cap=round,line join=round,line width=0.480pt]
        \path[draw] (81.5000,55.5000) -- (82.6777,55.5000);



        \path[draw] (121.5000,55.5000) -- (120.1220,55.5000);



      \end{scope}
      \begin{scope}[cm={{1.27491,0.0,0.0,1.41542,(-124.58597,-493.0531)}},draw=ca0a0a4,dash pattern=on 0.93pt off 0.93pt,line cap=round,line join=round,line width=0.233pt,miter limit=4.00]
        \path[draw,dash pattern=on 0.93pt off 0.93pt,line width=0.233pt,miter limit=4.00] (97.5000,78.5000) -- (97.5000,50.5000);



      \end{scope}
      \begin{scope}[cm={{1.27491,0.0,0.0,1.41542,(-124.58597,-493.0531)}},draw=blue,line cap=round,line join=round,line width=0.480pt]
        \path[draw] (97.5000,50.5000) -- (97.5000,50.5000) -- (97.5000,51.6564);



      \end{scope}
      \begin{scope}[cm={{1.27491,0.0,0.0,1.41542,(-124.58597,-493.0531)}},draw=blue,line cap=round,line join=round,line width=0.480pt]
        \path[draw] (81.5000,50.5000) -- (81.5000,78.5000) -- (121.5000,78.5000) -- (121.5000,50.5000) -- (81.5000,50.5000);



      \end{scope}
      \begin{scope}[cm={{1.27491,0.0,0.0,1.41542,(-124.58597,-493.0531)}},draw=blue,line cap=round,line join=round,line width=0.480pt]
        \path[draw] (81.2000,77.5000) -- (81.2000,77.5000) -- (81.2000,77.5000) -- (81.2000,77.5000) -- (81.2000,77.5000) -- (81.2000,77.5000) -- (81.2000,77.5000) -- (81.2000,77.5000) -- (81.2000,77.5000) -- (81.2000,77.5000) -- (81.2000,77.5000) -- (81.2000,77.5000) -- (81.2000,77.5000) -- (81.2000,77.5000) -- (81.2000,77.5000) -- (81.2000,77.5000) -- (81.2000,77.4000) -- (81.2000,77.4000) -- (81.2000,77.4000) -- (81.2000,77.4000) -- (81.2000,77.4000) -- (81.2000,77.4000) -- (81.2000,77.4000) -- (81.2000,77.4000) -- (81.2000,77.4000) -- (81.2000,77.4000) -- (81.3000,77.4000) -- (81.3000,77.4000) -- (81.3000,77.4000) -- (81.3000,77.4000) -- (81.3000,77.4000) -- (81.3000,77.4000) -- (81.3000,77.4000) -- (81.3000,77.4000) -- (81.3000,77.4000) -- (81.3000,77.4000) -- (81.3000,77.4000) -- (81.3000,77.3000) -- (81.3000,77.3000) -- (81.3000,77.3000) -- (81.3000,77.3000) -- (81.3000,77.3000) -- (81.3000,77.3000) -- (81.3000,77.3000) -- (81.3000,77.3000) -- (81.3000,77.3000) -- (81.3000,77.3000) -- (81.3000,77.3000) -- (81.3000,77.3000) -- (81.3000,77.3000) -- (81.3000,77.3000) -- (81.3000,77.3000) -- (81.3000,77.3000) -- (81.3000,77.3000) -- (81.3000,77.3000) -- (81.3000,77.3000) -- (81.3000,77.3000) -- (81.3000,77.2000) -- (81.3000,77.2000) -- (81.3000,77.2000) -- (81.3000,77.2000) -- (81.3000,77.2000) -- (81.3000,77.2000) -- (81.3000,77.2000) -- (81.3000,77.2000) -- (81.3000,77.2000) -- (81.3000,77.2000) -- (81.3000,77.2000) -- (81.3000,77.2000) -- (81.3000,77.2000) -- (81.3000,77.2000) -- (81.3000,77.2000) -- (81.3000,77.2000) -- (81.3000,77.2000) -- (81.3000,77.2000) -- (81.3000,77.2000) -- (81.4000,77.2000) -- (81.4000,77.2000) -- (81.4000,77.1000) -- (81.4000,77.1000) -- (81.4000,77.1000) -- (81.4000,77.1000) -- (81.4000,77.1000) -- (81.4000,77.1000) -- (81.4000,77.1000) -- (81.4000,77.1000) -- (81.4000,77.1000) -- (81.4000,77.1000) -- (81.4000,77.1000) -- (81.4000,77.1000) -- (81.4000,77.1000) -- (81.4000,77.1000) -- (81.4000,77.1000) -- (81.4000,77.1000) -- (81.4000,77.1000) -- (81.4000,77.1000) -- (81.4000,77.1000) -- (81.4000,77.1000) -- (81.4000,77.1000) -- (81.4000,77.0000) -- (81.4000,77.0000) -- (81.4000,77.0000) -- (81.4000,77.0000) -- (81.4000,77.0000) -- (81.4000,77.0000) -- (81.4000,77.0000) -- (81.4000,77.0000) -- (81.4000,77.0000) -- (81.4000,77.0000) -- (81.4000,77.0000) -- (81.4000,77.0000) -- (81.4000,77.0000) -- (81.4000,77.0000) -- (81.4000,77.0000) -- (81.4000,77.0000) -- (81.4000,77.0000) -- (81.4000,77.0000) -- (81.4000,77.0000) -- (81.4000,77.0000) -- (81.4000,77.0000) -- (81.4000,76.9000) -- (81.4000,76.9000) -- (81.4000,76.9000) -- (81.4000,76.9000) -- (81.4000,76.9000) -- (81.5000,76.9000) -- (81.5000,76.9000) -- (81.5000,76.9000) -- (81.5000,76.9000) -- (81.5000,76.9000) -- (81.5000,76.9000) -- (81.5000,76.9000) -- (81.5000,76.9000) -- (81.5000,76.9000) -- (81.5000,76.9000) -- (81.5000,76.9000) -- (81.5000,76.9000) -- (81.5000,76.9000) -- (81.5000,76.9000) -- (81.5000,76.9000) -- (81.5000,76.9000) -- (81.5000,76.8000) -- (81.5000,76.8000) -- (81.5000,76.8000) -- (81.5000,76.8000) -- (81.5000,76.8000) -- (81.5000,76.8000) -- (81.5000,76.8000) -- (81.5000,76.8000) -- (81.5000,76.8000) -- (81.5000,76.8000) -- (81.5000,76.8000) -- (81.5000,76.8000) -- (81.5000,76.8000) -- (81.5000,76.8000) -- (81.5000,76.8000) -- (81.5000,76.8000) -- (81.5000,76.8000) -- (81.5000,76.8000) -- (81.5000,76.8000) -- (81.5000,76.8000) -- (81.5000,76.7000) -- (81.5000,76.7000) -- (81.5000,76.7000) -- (81.5000,76.7000) -- (81.5000,76.7000) -- (81.5000,76.7000) -- (81.5000,76.7000) -- (81.5000,76.7000) -- (81.5000,76.7000) -- (81.5000,76.7000) -- (81.5000,76.7000) -- (81.5000,76.7000) -- (81.5000,76.7000) -- (81.5000,76.7000) -- (81.6000,76.7000) -- (81.6000,76.7000) -- (81.6000,76.7000) -- (81.6000,76.7000) -- (81.6000,76.7000) -- (81.6000,76.7000) -- (81.6000,76.7000) -- (81.6000,76.6000) -- (81.6000,76.6000) -- (81.6000,76.6000) -- (81.6000,76.6000) -- (81.6000,76.6000) -- (81.6000,76.6000) -- (81.6000,76.6000) -- (81.6000,76.6000) -- (81.6000,76.6000) -- (81.6000,76.6000) -- (81.6000,76.6000) -- (81.6000,76.6000) -- (81.6000,76.6000) -- (81.6000,76.6000) -- (81.6000,76.6000) -- (81.6000,76.6000) -- (81.6000,76.6000) -- (81.6000,76.6000) -- (81.6000,76.6000) -- (81.6000,76.6000) -- (81.6000,76.6000) -- (81.6000,76.5000) -- (81.6000,76.5000) -- (81.6000,76.5000) -- (81.6000,76.5000) -- (81.6000,76.5000) -- (81.6000,76.5000) -- (81.6000,76.5000) -- (81.6000,76.5000) -- (81.6000,76.5000) -- (81.6000,76.5000) -- (81.6000,76.5000) -- (81.6000,76.5000) -- (81.6000,76.5000) -- (81.6000,76.5000) -- (81.6000,76.5000) -- (81.6000,76.5000) -- (81.6000,76.5000) -- (81.6000,76.5000) -- (81.6000,76.5000) -- (81.6000,76.5000) -- (81.6000,76.5000) -- (81.7000,76.4000) -- (81.7000,76.4000) -- (81.7000,76.4000) -- (81.7000,76.4000) -- (81.7000,76.4000) -- (81.7000,76.4000) -- (81.7000,76.4000) -- (81.7000,76.4000) -- (81.7000,76.4000) -- (81.7000,76.4000) -- (81.7000,76.4000) -- (81.7000,76.4000) -- (81.7000,76.4000) -- (81.7000,76.4000) -- (81.7000,76.4000) -- (81.7000,76.4000) -- (81.7000,76.4000) -- (81.7000,76.4000) -- (81.7000,76.4000) -- (81.7000,76.4000) -- (81.7000,76.4000) -- (81.7000,76.3000) -- (81.7000,76.3000) -- (81.7000,76.3000) -- (81.7000,76.3000) -- (81.7000,76.3000) -- (81.7000,76.3000) -- (81.7000,76.3000) -- (81.7000,76.3000) -- (81.7000,76.3000) -- (81.7000,76.3000) -- (81.7000,76.3000) -- (81.7000,76.3000) -- (81.7000,76.3000) -- (81.7000,76.3000) -- (81.7000,76.3000) -- (81.7000,76.3000) -- (81.7000,76.3000) -- (81.7000,76.3000) -- (81.7000,76.3000) -- (81.7000,76.3000) -- (81.7000,76.2000) -- (81.7000,76.2000) -- (81.7000,76.2000) -- (81.7000,76.2000) -- (81.7000,76.2000) -- (81.7000,76.2000) -- (81.7000,76.2000) -- (81.7000,76.2000) -- (81.7000,76.2000) -- (81.8000,76.2000) -- (81.8000,76.2000) -- (81.8000,76.2000) -- (81.8000,76.2000) -- (81.8000,76.2000) -- (81.8000,76.2000) -- (81.8000,76.2000) -- (81.8000,76.2000) -- (81.8000,76.2000) -- (81.8000,76.2000) -- (81.8000,76.2000) -- (81.8000,76.2000) -- (81.8000,76.1000) -- (81.8000,76.1000) -- (81.8000,76.1000) -- (81.8000,76.1000) -- (81.8000,76.1000) -- (81.8000,76.1000) -- (81.8000,76.1000) -- (81.8000,76.1000) -- (81.8000,76.1000) -- (81.8000,76.1000) -- (81.8000,76.1000) -- (81.8000,76.1000) -- (81.8000,76.1000) -- (81.8000,76.1000) -- (81.8000,76.1000) -- (81.8000,76.1000) -- (81.8000,76.1000) -- (81.8000,76.1000) -- (81.8000,76.1000) -- (81.8000,76.1000) -- (81.8000,76.1000) -- (81.8000,76.0000) -- (81.8000,76.0000) -- (81.8000,76.0000) -- (81.8000,76.0000) -- (81.8000,76.0000) -- (81.8000,76.0000) -- (81.8000,76.0000) -- (81.8000,76.0000) -- (81.8000,76.0000) -- (81.8000,76.0000) -- (81.8000,76.0000) -- (81.8000,76.0000) -- (81.8000,76.0000) -- (81.8000,76.0000) -- (81.8000,76.0000) -- (81.8000,76.0000) -- (81.9000,76.0000) -- (81.9000,76.0000) -- (81.9000,76.0000) -- (81.9000,76.0000) -- (81.9000,76.0000) -- (81.9000,75.9000) -- (81.9000,75.9000) -- (81.9000,75.9000) -- (81.9000,75.9000) -- (81.9000,75.9000) -- (81.9000,75.9000) -- (81.9000,75.9000) -- (81.9000,75.9000) -- (81.9000,75.9000) -- (81.9000,75.9000) -- (81.9000,75.9000) -- (81.9000,75.9000) -- (81.9000,75.9000) -- (81.9000,75.9000) -- (81.9000,75.9000) -- (81.9000,75.9000) -- (81.9000,75.9000) -- (81.9000,75.9000) -- (81.9000,75.9000) -- (81.9000,75.9000) -- (81.9000,75.8000) -- (81.9000,75.8000) -- (81.9000,75.8000) -- (81.9000,75.8000) -- (81.9000,75.8000) -- (81.9000,75.8000) -- (81.9000,75.8000) -- (81.9000,75.8000) -- (81.9000,75.8000) -- (81.9000,75.8000) -- (81.9000,75.8000) -- (81.9000,75.8000) -- (81.9000,75.8000) -- (81.9000,75.8000) -- (81.9000,75.8000) -- (81.9000,75.8000) -- (81.9000,75.8000) -- (81.9000,75.8000) -- (81.9000,75.8000) -- (81.9000,75.8000) -- (81.9000,75.8000) -- (81.9000,75.7000) -- (81.9000,75.7000) -- (81.9000,75.7000) -- (81.9000,75.7000) -- (82.0000,75.7000) -- (82.0000,75.7000) -- (82.0000,75.7000) -- (82.0000,75.7000) -- (82.0000,75.7000) -- (82.0000,75.7000) -- (82.0000,75.7000) -- (82.0000,75.7000) -- (82.0000,75.7000) -- (82.0000,75.7000) -- (82.0000,75.7000) -- (82.0000,75.7000) -- (82.0000,75.7000) -- (82.0000,75.7000) -- (82.0000,75.7000) -- (82.0000,75.7000) -- (82.0000,75.7000) -- (82.0000,75.6000) -- (82.0000,75.6000) -- (82.0000,75.6000) -- (82.0000,75.6000) -- (82.0000,75.6000) -- (82.0000,75.6000) -- (82.0000,75.6000) -- (82.0000,75.6000) -- (82.0000,75.6000) -- (82.0000,75.6000) -- (82.0000,75.6000) -- (82.0000,75.6000) -- (82.0000,75.6000) -- (82.0000,75.6000) -- (82.0000,75.6000) -- (82.0000,75.6000) -- (82.0000,75.6000) -- (82.0000,75.6000) -- (82.0000,75.6000) -- (82.0000,75.6000) -- (82.0000,75.6000) -- (82.0000,75.5000) -- (82.0000,75.5000) -- (82.0000,75.5000) -- (82.0000,75.5000) -- (82.0000,75.5000) -- (82.0000,75.5000) -- (82.0000,75.5000) -- (82.0000,75.5000) -- (82.0000,75.5000) -- (82.0000,75.5000) -- (82.0000,75.5000) -- (82.1000,75.5000) -- (82.1000,75.5000) -- (82.1000,75.5000) -- (82.1000,75.5000) -- (82.1000,75.5000) -- (82.1000,75.5000) -- (82.1000,75.5000) -- (82.1000,75.5000) -- (82.1000,75.5000) -- (82.1000,75.5000) -- (82.1000,75.4000) -- (82.1000,75.4000) -- (82.1000,75.4000) -- (82.1000,75.4000) -- (82.1000,75.4000) -- (82.1000,75.4000) -- (82.1000,75.4000) -- (82.1000,75.4000) -- (82.1000,75.4000) -- (82.1000,75.4000) -- (82.1000,75.4000) -- (82.1000,75.4000) -- (82.1000,75.4000) -- (82.1000,75.4000) -- (82.1000,75.4000) -- (82.1000,75.4000) -- (82.1000,75.4000) -- (82.1000,75.4000) -- (82.1000,75.4000) -- (82.1000,75.4000) -- (82.1000,75.3000) -- (82.1000,75.3000) -- (82.1000,75.3000) -- (82.1000,75.3000) -- (82.1000,75.3000) -- (82.1000,75.3000) -- (82.1000,75.3000) -- (82.1000,75.3000) -- (82.1000,75.3000) -- (82.1000,75.3000) -- (82.1000,75.3000) -- (82.1000,75.3000) -- (82.1000,75.3000) -- (82.1000,75.3000) -- (82.1000,75.3000) -- (82.1000,75.3000) -- (82.1000,75.3000) -- (82.1000,75.3000) -- (82.1000,75.3000) -- (82.1000,75.3000) -- (82.2000,75.3000) -- (82.2000,75.2000) -- (82.2000,75.2000) -- (82.2000,75.2000) -- (82.2000,75.2000) -- (82.2000,75.2000) -- (82.2000,75.2000) -- (82.2000,75.2000) -- (82.2000,75.2000) -- (82.2000,75.2000) -- (82.2000,75.2000) -- (82.2000,75.2000) -- (82.2000,75.2000) -- (82.2000,75.2000) -- (82.2000,75.2000) -- (82.2000,75.2000) -- (82.2000,75.2000) -- (82.2000,75.2000) -- (82.2000,75.2000) -- (82.2000,75.2000) -- (82.2000,75.2000) -- (82.2000,75.2000) -- (82.2000,75.1000) -- (82.2000,75.1000) -- (82.2000,75.1000) -- (82.2000,75.1000) -- (82.2000,75.1000) -- (82.2000,75.1000) -- (82.2000,75.1000) -- (82.2000,75.1000) -- (82.2000,75.1000) -- (82.2000,75.1000) -- (82.2000,75.1000) -- (82.2000,75.1000) -- (82.2000,75.1000) -- (82.2000,75.1000) -- (82.2000,75.1000) -- (82.2000,75.1000) -- (82.2000,75.1000) -- (82.2000,75.1000) -- (82.2000,75.1000) -- (82.2000,75.1000) -- (82.2000,75.1000) -- (82.2000,75.0000) -- (82.2000,75.0000) -- (82.2000,75.0000) -- (82.2000,75.0000) -- (82.2000,75.0000) -- (82.2000,75.0000) -- (82.3000,75.0000) -- (82.3000,75.0000) -- (82.3000,75.0000) -- (82.3000,75.0000) -- (82.3000,75.0000) -- (82.3000,75.0000) -- (82.3000,75.0000) -- (82.3000,75.0000) -- (82.3000,75.0000) -- (82.3000,75.0000) -- (82.3000,75.0000) -- (82.3000,75.0000) -- (82.3000,75.0000) -- (82.3000,75.0000) -- (82.3000,75.0000) -- (82.3000,74.9000) -- (82.3000,74.9000) -- (82.3000,74.9000) -- (82.3000,74.9000) -- (82.3000,74.9000) -- (82.3000,74.9000) -- (82.3000,74.9000) -- (82.3000,74.9000) -- (82.3000,74.9000) -- (82.3000,74.9000) -- (82.3000,74.9000) -- (82.3000,74.9000) -- (82.3000,74.9000) -- (82.3000,74.9000) -- (82.3000,74.9000) -- (82.3000,74.9000) -- (82.3000,74.9000) -- (82.3000,74.9000) -- (82.3000,74.9000) -- (82.3000,74.9000) -- (82.3000,74.8000) -- (82.3000,74.8000) -- (82.3000,74.8000) -- (82.3000,74.8000) -- (82.3000,74.8000) -- (82.3000,74.8000) -- (82.3000,74.8000) -- (82.3000,74.8000) -- (82.3000,74.8000) -- (82.3000,74.8000) -- (82.3000,74.8000) -- (82.3000,74.8000) -- (82.3000,74.8000) -- (82.3000,74.8000) -- (82.3000,74.8000) -- (82.4000,74.8000) -- (82.4000,74.8000) -- (82.4000,74.8000) -- (82.4000,74.8000) -- (82.4000,74.8000) -- (82.4000,74.8000) -- (82.4000,74.7000) -- (82.4000,74.7000) -- (82.4000,74.7000) -- (82.4000,74.7000) -- (82.4000,74.7000) -- (82.4000,74.7000) -- (82.4000,74.7000) -- (82.4000,74.7000) -- (82.4000,74.7000) -- (82.4000,74.7000) -- (82.4000,74.7000) -- (82.4000,74.7000) -- (82.4000,74.7000) -- (82.4000,74.7000) -- (82.4000,74.7000) -- (82.4000,74.7000) -- (82.4000,74.7000) -- (82.4000,74.7000) -- (82.4000,74.7000) -- (82.4000,74.7000) -- (82.4000,74.7000) -- (82.4000,74.6000) -- (82.4000,74.6000) -- (82.4000,74.6000) -- (82.4000,74.6000) -- (82.4000,74.6000) -- (82.4000,74.6000) -- (82.4000,74.6000) -- (82.4000,74.6000) -- (82.4000,74.6000) -- (82.4000,74.6000) -- (82.4000,74.6000) -- (82.4000,74.6000) -- (82.4000,74.6000) -- (82.4000,74.6000) -- (82.4000,74.6000) -- (82.4000,74.6000) -- (82.4000,74.6000) -- (82.4000,74.6000) -- (82.4000,74.6000) -- (82.4000,74.6000) -- (82.4000,74.6000) -- (82.4000,74.5000) -- (82.5000,74.5000) -- (82.5000,74.5000) -- (82.5000,74.5000) -- (82.5000,74.5000) -- (82.5000,74.5000) -- (82.5000,74.5000) -- (82.5000,74.5000) -- (82.5000,74.5000) -- (82.5000,74.5000) -- (82.5000,74.5000) -- (82.5000,74.5000) -- (82.5000,74.5000) -- (82.5000,74.5000) -- (82.5000,74.5000) -- (82.5000,74.5000) -- (82.5000,74.5000) -- (82.5000,74.5000) -- (82.5000,74.5000) -- (82.5000,74.5000) -- (82.5000,74.4000) -- (82.5000,74.4000) -- (82.5000,74.4000) -- (82.5000,74.4000) -- (82.5000,74.4000) -- (82.5000,74.4000) -- (82.5000,74.4000) -- (82.5000,74.4000) -- (82.5000,74.4000) -- (82.5000,74.4000) -- (82.5000,74.4000) -- (82.5000,74.4000) -- (82.5000,74.4000) -- (82.5000,74.4000) -- (82.5000,74.4000) -- (82.5000,74.4000) -- (82.5000,74.4000) -- (82.5000,74.4000) -- (82.5000,74.4000) -- (82.5000,74.4000) -- (82.5000,74.4000) -- (82.5000,74.3000) -- (82.5000,74.3000) -- (82.5000,74.3000) -- (82.5000,74.3000) -- (82.5000,74.3000) -- (82.5000,74.3000) -- (82.5000,74.3000) -- (82.5000,74.3000) -- (82.5000,74.3000) -- (82.5000,74.3000) -- (82.6000,74.3000) -- (82.6000,74.3000) -- (82.6000,74.3000) -- (82.6000,74.3000) -- (82.6000,74.3000) -- (82.6000,74.3000) -- (82.6000,74.3000) -- (82.6000,74.3000) -- (82.6000,74.3000) -- (82.6000,74.3000) -- (82.6000,74.3000) -- (82.6000,74.2000) -- (82.6000,74.2000) -- (82.6000,74.2000) -- (82.6000,74.2000) -- (82.6000,74.2000) -- (82.6000,74.2000) -- (82.6000,74.2000) -- (82.6000,74.2000) -- (82.6000,74.2000) -- (82.6000,74.2000) -- (82.6000,74.2000) -- (82.6000,74.2000) -- (82.6000,74.2000) -- (82.6000,74.2000) -- (82.6000,74.2000) -- (82.6000,74.2000) -- (82.6000,74.2000) -- (82.6000,74.2000) -- (82.6000,74.2000) -- (82.6000,74.2000) -- (82.6000,74.2000) -- (82.6000,74.1000) -- (82.6000,74.1000) -- (82.6000,74.1000) -- (82.6000,74.1000) -- (82.6000,74.1000) -- (82.6000,74.1000) -- (82.6000,74.1000) -- (82.6000,74.1000) -- (82.6000,74.1000) -- (82.6000,74.1000) -- (82.6000,74.1000) -- (82.6000,74.1000) -- (82.6000,74.1000) -- (82.6000,74.1000) -- (82.6000,74.1000) -- (82.6000,74.1000) -- (82.6000,74.1000) -- (82.7000,74.1000) -- (82.7000,74.1000) -- (82.7000,74.1000) -- (82.7000,74.1000) -- (82.7000,74.0000) -- (82.7000,74.0000) -- (82.7000,74.0000) -- (82.7000,74.0000) -- (82.7000,74.0000) -- (82.7000,74.0000) -- (82.7000,74.0000) -- (82.7000,74.0000) -- (82.7000,74.0000) -- (82.7000,74.0000) -- (82.7000,74.0000) -- (82.7000,74.0000) -- (82.7000,74.0000) -- (82.7000,74.0000) -- (82.7000,74.0000) -- (82.7000,74.0000) -- (82.7000,74.0000) -- (82.7000,74.0000) -- (82.7000,74.0000) -- (82.7000,74.0000) -- (82.7000,73.9000) -- (82.7000,73.9000) -- (82.7000,73.9000) -- (82.7000,73.9000) -- (82.7000,73.9000) -- (82.7000,73.9000) -- (82.7000,73.9000) -- (82.7000,73.9000) -- (82.7000,73.9000) -- (82.7000,73.9000) -- (82.7000,73.9000) -- (82.7000,73.9000) -- (82.7000,73.9000) -- (82.7000,73.9000) -- (82.7000,73.9000) -- (82.7000,73.9000) -- (82.7000,73.9000) -- (82.7000,73.9000) -- (82.7000,73.9000) -- (82.7000,73.9000) -- (82.7000,73.9000) -- (82.7000,73.8000) -- (82.7000,73.8000) -- (82.7000,73.8000) -- (82.7000,73.8000) -- (82.7000,73.8000) -- (82.8000,73.8000) -- (82.8000,73.8000) -- (82.8000,73.8000) -- (82.8000,73.8000) -- (82.8000,73.8000) -- (82.8000,73.8000) -- (82.8000,73.8000) -- (82.8000,73.8000) -- (82.8000,73.8000) -- (82.8000,73.8000) -- (82.8000,73.8000) -- (82.8000,73.8000) -- (82.8000,73.8000) -- (82.8000,73.8000) -- (82.8000,73.8000) -- (82.8000,73.8000) -- (82.8000,73.7000) -- (82.8000,73.7000) -- (82.8000,73.7000) -- (82.8000,73.7000) -- (82.8000,73.7000) -- (82.8000,73.7000) -- (82.8000,73.7000) -- (82.8000,73.7000) -- (82.8000,73.7000) -- (82.8000,73.7000) -- (82.8000,73.7000) -- (82.8000,73.7000) -- (82.8000,73.7000) -- (82.8000,73.7000) -- (82.8000,73.7000) -- (82.8000,73.7000) -- (82.8000,73.7000) -- (82.8000,73.7000) -- (82.8000,73.7000) -- (82.8000,73.7000) -- (82.8000,73.7000) -- (82.8000,73.6000) -- (82.8000,73.6000) -- (82.8000,73.6000) -- (82.8000,73.6000) -- (82.8000,73.6000) -- (82.8000,73.6000) -- (82.8000,73.6000) -- (82.8000,73.6000) -- (82.8000,73.6000) -- (82.8000,73.6000) -- (82.8000,73.6000) -- (82.8000,73.6000) -- (82.9000,73.6000) -- (82.9000,73.6000) -- (82.9000,73.6000) -- (82.9000,73.6000) -- (82.9000,73.6000) -- (82.9000,73.6000) -- (82.9000,73.6000) -- (82.9000,73.6000) -- (82.9000,73.6000) -- (82.9000,73.5000) -- (82.9000,73.5000) -- (82.9000,73.5000) -- (82.9000,73.5000) -- (82.9000,73.5000) -- (82.9000,73.5000) -- (82.9000,73.5000) -- (82.9000,73.5000) -- (82.9000,73.5000) -- (82.9000,73.5000) -- (82.9000,73.5000) -- (82.9000,73.5000) -- (82.9000,73.5000) -- (82.9000,73.5000) -- (82.9000,73.5000) -- (82.9000,73.5000) -- (82.9000,73.5000) -- (82.9000,73.5000) -- (82.9000,73.5000) -- (82.9000,73.5000) -- (82.9000,73.4000) -- (82.9000,73.4000) -- (82.9000,73.4000) -- (82.9000,73.4000) -- (82.9000,73.4000) -- (82.9000,73.4000) -- (82.9000,73.4000) -- (82.9000,73.4000) -- (82.9000,73.4000) -- (82.9000,73.4000) -- (82.9000,73.4000) -- (82.9000,73.4000) -- (82.9000,73.4000) -- (82.9000,73.4000) -- (82.9000,73.4000) -- (82.9000,73.4000) -- (82.9000,73.4000) -- (82.9000,73.4000) -- (82.9000,73.4000) -- (82.9000,73.4000) -- (82.9000,73.4000) -- (83.0000,73.3000) -- (83.0000,73.3000) -- (83.0000,73.3000) -- (83.0000,73.3000) -- (83.0000,73.3000) -- (83.0000,73.3000) -- (83.0000,73.3000) -- (83.0000,73.3000) -- (83.0000,73.3000) -- (83.0000,73.3000) -- (83.0000,73.3000) -- (83.0000,73.3000) -- (83.0000,73.3000) -- (83.0000,73.3000) -- (83.0000,73.3000) -- (83.0000,73.3000) -- (83.0000,73.3000) -- (83.0000,73.3000) -- (83.0000,73.3000) -- (83.0000,73.3000) -- (83.0000,73.3000) -- (83.0000,73.2000) -- (83.0000,73.2000) -- (83.0000,73.2000) -- (83.0000,73.2000) -- (83.0000,73.2000) -- (83.0000,73.2000) -- (83.0000,73.2000) -- (83.0000,73.2000) -- (83.0000,73.2000) -- (83.0000,73.2000) -- (83.0000,73.2000) -- (83.0000,73.2000) -- (83.0000,73.2000) -- (83.0000,73.2000) -- (83.0000,73.2000) -- (83.0000,73.2000) -- (83.0000,73.2000) -- (83.0000,73.2000) -- (83.0000,73.2000) -- (83.0000,73.2000) -- (83.0000,73.2000) -- (83.0000,73.1000) -- (83.0000,73.1000) -- (83.0000,73.1000) -- (83.0000,73.1000) -- (83.0000,73.1000) -- (83.0000,73.1000) -- (83.0000,73.1000) -- (83.1000,73.1000) -- (83.1000,73.1000) -- (83.1000,73.1000) -- (83.1000,73.1000) -- (83.1000,73.1000) -- (83.1000,73.1000) -- (83.1000,73.1000) -- (83.1000,73.1000) -- (83.1000,73.1000) -- (83.1000,73.1000) -- (83.1000,73.1000) -- (83.1000,73.1000) -- (83.1000,73.1000) -- (83.1000,73.1000) -- (83.1000,73.0000) -- (83.1000,73.0000) -- (83.1000,73.0000) -- (83.1000,73.0000) -- (83.1000,73.0000) -- (83.1000,73.0000) -- (83.1000,73.0000) -- (83.1000,73.0000) -- (83.1000,73.0000) -- (83.1000,73.0000) -- (83.1000,73.0000) -- (83.1000,73.0000) -- (83.1000,73.0000) -- (83.1000,73.0000) -- (83.1000,73.0000) -- (83.1000,73.0000) -- (83.1000,73.0000) -- (83.1000,73.0000) -- (83.1000,73.0000) -- (83.1000,73.0000) -- (83.1000,72.9000) -- (83.1000,72.9000) -- (83.1000,72.9000) -- (83.1000,72.9000) -- (83.1000,72.9000) -- (83.1000,72.9000) -- (83.1000,72.9000) -- (83.1000,72.9000) -- (83.1000,72.9000) -- (83.1000,72.9000) -- (83.1000,72.9000) -- (83.1000,72.9000) -- (83.1000,72.9000) -- (83.1000,72.9000) -- (83.1000,72.9000) -- (83.1000,72.9000) -- (83.2000,72.9000) -- (83.2000,72.9000) -- (83.2000,72.9000) -- (83.2000,72.9000) -- (83.2000,72.9000) -- (83.2000,72.8000) -- (83.2000,72.8000) -- (83.2000,72.8000) -- (83.2000,72.8000) -- (83.2000,72.8000) -- (83.2000,72.8000) -- (83.2000,72.8000) -- (83.2000,72.8000) -- (83.2000,72.8000) -- (83.2000,72.8000) -- (83.2000,72.8000) -- (83.2000,72.8000) -- (83.2000,72.8000) -- (83.2000,72.8000) -- (83.2000,72.8000) -- (83.2000,72.8000) -- (83.2000,72.8000) -- (83.2000,72.8000) -- (83.2000,72.8000) -- (83.2000,72.8000) -- (83.2000,72.8000) -- (83.2000,72.7000) -- (83.2000,72.7000) -- (83.2000,72.7000) -- (83.2000,72.7000) -- (83.2000,72.7000) -- (83.2000,72.7000) -- (83.2000,72.7000) -- (83.2000,72.7000) -- (83.2000,72.7000) -- (83.2000,72.7000) -- (83.2000,72.7000) -- (83.2000,72.7000) -- (83.2000,72.7000) -- (83.2000,72.7000) -- (83.2000,72.7000) -- (83.2000,72.7000) -- (83.2000,72.7000) -- (83.2000,72.7000) -- (83.2000,72.7000) -- (83.2000,72.7000) -- (83.2000,72.7000) -- (83.2000,72.6000) -- (83.2000,72.6000) -- (83.3000,72.6000) -- (83.3000,72.6000) -- (83.3000,72.6000) -- (83.3000,72.6000) -- (83.3000,72.6000) -- (83.3000,72.6000) -- (83.3000,72.6000) -- (83.3000,72.6000) -- (83.3000,72.6000) -- (83.3000,72.6000) -- (83.3000,72.6000) -- (83.3000,72.6000) -- (83.3000,72.6000) -- (83.3000,72.6000) -- (83.3000,72.6000) -- (83.3000,72.6000) -- (83.3000,72.6000) -- (83.3000,72.6000) -- (83.3000,72.5000) -- (83.3000,72.5000) -- (83.3000,72.5000) -- (83.3000,72.5000) -- (83.3000,72.5000) -- (83.3000,72.5000) -- (83.3000,72.5000) -- (83.3000,72.5000) -- (83.3000,72.5000) -- (83.3000,72.5000) -- (83.3000,72.5000) -- (83.3000,72.5000) -- (83.3000,72.5000) -- (83.3000,72.5000) -- (83.3000,72.5000) -- (83.3000,72.5000) -- (83.3000,72.5000) -- (83.3000,72.5000) -- (83.3000,72.5000) -- (83.3000,72.5000) -- (83.3000,72.5000) -- (83.3000,72.4000) -- (83.3000,72.4000) -- (83.3000,72.4000) -- (83.3000,72.4000) -- (83.3000,72.4000) -- (83.3000,72.4000) -- (83.3000,72.4000) -- (83.3000,72.4000) -- (83.3000,72.4000) -- (83.3000,72.4000) -- (83.3000,72.4000) -- (83.4000,72.4000) -- (83.4000,72.4000) -- (83.4000,72.4000) -- (83.4000,72.4000) -- (83.4000,72.4000) -- (83.4000,72.4000) -- (83.4000,72.4000) -- (83.4000,72.4000) -- (83.4000,72.4000) -- (83.4000,72.4000) -- (83.4000,72.3000) -- (83.4000,72.3000) -- (83.4000,72.3000) -- (83.4000,72.3000) -- (83.4000,72.3000) -- (83.4000,72.3000) -- (83.4000,72.3000) -- (83.4000,72.3000) -- (83.4000,72.3000) -- (83.4000,72.3000) -- (83.4000,72.3000) -- (83.4000,72.3000) -- (83.4000,72.3000) -- (83.4000,72.3000) -- (83.4000,72.3000) -- (83.4000,72.3000) -- (83.4000,72.3000) -- (83.4000,72.3000) -- (83.4000,72.3000) -- (83.4000,72.3000) -- (83.4000,72.3000) -- (83.4000,72.2000) -- (83.4000,72.2000) -- (83.4000,72.2000) -- (83.4000,72.2000) -- (83.4000,72.2000) -- (83.4000,72.2000) -- (83.4000,72.2000) -- (83.4000,72.2000) -- (83.4000,72.2000) -- (83.4000,72.2000) -- (83.4000,72.2000) -- (83.4000,72.2000) -- (83.4000,72.2000) -- (83.4000,72.2000) -- (83.4000,72.2000) -- (83.4000,72.2000) -- (83.4000,72.2000) -- (83.4000,72.2000) -- (83.5000,72.2000) -- (83.5000,72.2000) -- (83.5000,72.2000) -- (83.5000,72.1000) -- (83.5000,72.1000) -- (83.5000,72.1000) -- (83.5000,72.1000) -- (83.5000,72.1000) -- (83.5000,72.1000) -- (83.5000,72.1000) -- (83.5000,72.1000) -- (83.5000,72.1000) -- (83.5000,72.1000) -- (83.5000,72.1000) -- (83.5000,72.1000) -- (83.5000,72.1000) -- (83.5000,72.1000) -- (83.5000,72.1000) -- (83.5000,72.1000) -- (83.5000,72.1000) -- (83.5000,72.1000) -- (83.5000,72.1000) -- (83.5000,72.1000) -- (83.5000,72.0000) -- (83.5000,72.0000) -- (83.5000,72.0000) -- (83.5000,72.0000) -- (83.5000,72.0000) -- (83.5000,72.0000) -- (83.5000,72.0000) -- (83.5000,72.0000) -- (83.5000,72.0000) -- (83.5000,72.0000) -- (83.5000,72.0000) -- (83.5000,72.0000) -- (83.5000,72.0000) -- (83.5000,72.0000) -- (83.5000,72.0000) -- (83.5000,72.0000) -- (83.5000,72.0000) -- (83.5000,72.0000) -- (83.5000,72.0000) -- (83.5000,72.0000) -- (83.5000,72.0000) -- (83.5000,71.9000) -- (83.5000,71.9000) -- (83.5000,71.9000) -- (83.5000,71.9000) -- (83.5000,71.9000) -- (83.5000,71.9000) -- (83.6000,71.9000) -- (83.6000,71.9000) -- (83.6000,71.9000) -- (83.6000,71.9000) -- (83.6000,71.9000) -- (83.6000,71.9000) -- (83.6000,71.9000) -- (83.6000,71.9000) -- (83.6000,71.9000) -- (83.6000,71.9000) -- (83.6000,71.9000) -- (83.6000,71.9000) -- (83.6000,71.9000) -- (83.6000,71.9000) -- (83.6000,71.9000) -- (83.6000,71.8000) -- (83.6000,71.8000) -- (83.6000,71.8000) -- (83.6000,71.8000) -- (83.6000,71.8000) -- (83.6000,71.8000) -- (83.6000,71.8000) -- (83.6000,71.8000) -- (83.6000,71.8000) -- (83.6000,71.8000) -- (83.6000,71.8000) -- (83.6000,71.8000) -- (83.6000,71.8000) -- (83.6000,71.8000) -- (83.6000,71.8000) -- (83.6000,71.8000) -- (83.6000,71.8000) -- (83.6000,71.8000) -- (83.6000,71.8000) -- (83.6000,71.8000) -- (83.6000,71.8000) -- (83.6000,71.7000) -- (83.6000,71.7000) -- (83.6000,71.7000) -- (83.6000,71.7000) -- (83.6000,71.7000) -- (83.6000,71.7000) -- (83.6000,71.7000) -- (83.6000,71.7000) -- (83.6000,71.7000) -- (83.6000,71.7000) -- (83.6000,71.7000) -- (83.6000,71.7000) -- (83.6000,71.7000) -- (83.7000,71.7000) -- (83.7000,71.7000) -- (83.7000,71.7000) -- (83.7000,71.7000) -- (83.7000,71.7000) -- (83.7000,71.7000) -- (83.7000,71.7000) -- (83.7000,71.7000) -- (83.7000,71.6000) -- (83.7000,71.6000) -- (83.7000,71.6000) -- (83.7000,71.6000) -- (83.7000,71.6000) -- (83.7000,71.6000) -- (83.7000,71.6000) -- (83.7000,71.6000) -- (83.7000,71.6000) -- (83.7000,71.6000) -- (83.7000,71.6000) -- (83.7000,71.6000) -- (83.7000,71.6000) -- (83.7000,71.6000) -- (83.7000,71.6000) -- (83.7000,71.6000) -- (83.7000,71.6000) -- (83.7000,71.6000) -- (83.7000,71.6000) -- (83.7000,71.6000) -- (83.7000,71.5000) -- (83.7000,71.5000) -- (83.7000,71.5000) -- (83.7000,71.5000) -- (83.7000,71.5000) -- (83.7000,71.5000) -- (83.7000,71.5000) -- (83.7000,71.5000) -- (83.7000,71.5000) -- (83.7000,71.5000) -- (83.7000,71.5000) -- (83.7000,71.5000) -- (83.7000,71.5000) -- (83.7000,71.5000) -- (83.7000,71.5000) -- (83.7000,71.5000) -- (83.7000,71.5000) -- (83.7000,71.5000) -- (83.7000,71.5000) -- (83.7000,71.5000) -- (83.7000,71.5000) -- (83.7000,71.4000) -- (83.8000,71.4000) -- (83.8000,71.4000) -- (83.8000,71.4000) -- (83.8000,71.4000) -- (83.8000,71.4000) -- (83.8000,71.4000) -- (83.8000,71.4000) -- (83.8000,71.4000) -- (83.8000,71.4000) -- (83.8000,71.4000) -- (83.8000,71.4000) -- (83.8000,71.4000) -- (83.8000,71.4000) -- (83.8000,71.4000) -- (83.8000,71.4000) -- (83.8000,71.4000) -- (83.8000,71.4000) -- (83.8000,71.4000) -- (83.8000,71.4000) -- (83.8000,71.4000) -- (83.8000,71.3000) -- (83.8000,71.3000) -- (83.8000,71.3000) -- (83.8000,71.3000) -- (83.8000,71.3000) -- (83.8000,71.3000) -- (83.8000,71.3000) -- (83.8000,71.3000) -- (83.8000,71.3000) -- (83.8000,71.3000) -- (83.8000,71.3000) -- (83.8000,71.3000) -- (83.8000,71.3000) -- (83.8000,71.3000) -- (83.8000,71.3000) -- (83.8000,71.3000) -- (83.8000,71.3000) -- (83.8000,71.3000) -- (83.8000,71.3000) -- (83.8000,71.3000) -- (83.8000,71.3000) -- (83.8000,71.2000) -- (83.8000,71.2000) -- (83.8000,71.2000) -- (83.8000,71.2000) -- (83.8000,71.2000) -- (83.8000,71.2000) -- (83.8000,71.2000) -- (83.8000,71.2000) -- (83.9000,71.2000) -- (83.9000,71.2000) -- (83.9000,71.2000) -- (83.9000,71.2000) -- (83.9000,71.2000) -- (83.9000,71.2000) -- (83.9000,71.2000) -- (83.9000,71.2000) -- (83.9000,71.2000) -- (83.9000,71.2000) -- (83.9000,71.2000) -- (83.9000,71.2000) -- (83.9000,71.2000) -- (83.9000,71.1000) -- (83.9000,71.1000) -- (83.9000,71.1000) -- (83.9000,71.1000) -- (83.9000,71.1000) -- (83.9000,71.1000) -- (83.9000,71.1000) -- (83.9000,71.1000) -- (83.9000,71.1000) -- (83.9000,71.1000) -- (83.9000,71.1000) -- (83.9000,71.1000) -- (83.9000,71.1000) -- (83.9000,71.1000) -- (83.9000,71.1000) -- (83.9000,71.1000) -- (83.9000,71.1000) -- (83.9000,71.1000) -- (83.9000,71.1000) -- (83.9000,71.1000) -- (83.9000,71.0000) -- (83.9000,71.0000) -- (83.9000,71.0000) -- (83.9000,71.0000) -- (83.9000,71.0000) -- (83.9000,71.0000) -- (83.9000,71.0000) -- (83.9000,71.0000) -- (83.9000,71.0000) -- (83.9000,71.0000) -- (83.9000,71.0000) -- (83.9000,71.0000) -- (83.9000,71.0000) -- (83.9000,71.0000) -- (83.9000,71.0000) -- (83.9000,71.0000) -- (83.9000,71.0000) -- (84.0000,71.0000) -- (84.0000,71.0000) -- (84.0000,71.0000) -- (84.0000,71.0000) -- (84.0000,70.9000) -- (84.0000,70.9000) -- (84.0000,70.9000) -- (84.0000,70.9000) -- (84.0000,70.9000) -- (84.0000,70.9000) -- (84.0000,70.9000) -- (84.0000,70.9000) -- (84.0000,70.9000) -- (84.0000,70.9000) -- (84.0000,70.9000) -- (84.0000,70.9000) -- (84.0000,70.9000) -- (84.0000,70.9000) -- (84.0000,70.9000) -- (84.0000,70.9000) -- (84.0000,70.9000) -- (84.0000,70.9000) -- (84.0000,70.9000) -- (84.0000,70.9000) -- (84.0000,70.9000) -- (84.0000,70.8000) -- (84.0000,70.8000) -- (84.0000,70.8000) -- (84.0000,70.8000) -- (84.0000,70.8000) -- (84.0000,70.8000) -- (84.0000,70.8000) -- (84.0000,70.8000) -- (84.0000,70.8000) -- (84.0000,70.8000) -- (84.0000,70.8000) -- (84.0000,70.8000) -- (84.0000,70.8000) -- (84.0000,70.8000) -- (84.0000,70.8000) -- (84.0000,70.8000) -- (84.0000,70.8000) -- (84.0000,70.8000) -- (84.0000,70.8000) -- (84.0000,70.8000) -- (84.0000,70.8000) -- (84.0000,70.7000) -- (84.0000,70.7000) -- (84.0000,70.7000) -- (84.1000,70.7000) -- (84.1000,70.7000) -- (84.1000,70.7000) -- (84.1000,70.7000) -- (84.1000,70.7000) -- (84.1000,70.7000) -- (84.1000,70.7000) -- (84.1000,70.7000) -- (84.1000,70.7000) -- (84.1000,70.7000) -- (84.1000,70.7000) -- (84.1000,70.7000) -- (84.1000,70.7000) -- (84.1000,70.7000) -- (84.1000,70.7000) -- (84.1000,70.7000) -- (84.1000,70.7000) -- (84.1000,70.6000) -- (84.1000,70.6000) -- (84.1000,70.6000) -- (84.1000,70.6000) -- (84.1000,70.6000) -- (84.1000,70.6000) -- (84.1000,70.6000) -- (84.1000,70.6000) -- (84.1000,70.6000) -- (84.1000,70.6000) -- (84.1000,70.6000) -- (84.1000,70.6000) -- (84.1000,70.6000) -- (84.1000,70.6000) -- (84.1000,70.6000) -- (84.1000,70.6000) -- (84.1000,70.6000) -- (84.1000,70.6000) -- (84.1000,70.6000) -- (84.1000,70.6000) -- (84.1000,70.6000) -- (84.1000,70.5000) -- (84.1000,70.5000) -- (84.1000,70.5000) -- (84.1000,70.5000) -- (84.1000,70.5000) -- (84.1000,70.5000) -- (84.1000,70.5000) -- (84.1000,70.5000) -- (84.1000,70.5000) -- (84.1000,70.5000) -- (84.1000,70.5000) -- (84.1000,70.5000) -- (84.2000,70.5000) -- (84.2000,70.5000) -- (84.2000,70.5000) -- (84.2000,70.5000) -- (84.2000,70.5000) -- (84.2000,70.5000) -- (84.2000,70.5000) -- (84.2000,70.5000) -- (84.2000,70.5000) -- (84.2000,70.4000) -- (84.2000,70.4000) -- (84.2000,70.4000) -- (84.2000,70.4000) -- (84.2000,70.4000) -- (84.2000,70.4000) -- (84.2000,70.4000) -- (84.2000,70.4000) -- (84.2000,70.4000) -- (84.2000,70.4000) -- (84.2000,70.4000) -- (84.2000,70.4000) -- (84.2000,70.4000) -- (84.2000,70.4000) -- (84.2000,70.4000) -- (84.2000,70.4000) -- (84.2000,70.4000) -- (84.2000,70.4000) -- (84.2000,70.4000) -- (84.2000,70.4000) -- (84.2000,70.4000) -- (84.2000,70.3000) -- (84.2000,70.3000) -- (84.2000,70.3000) -- (84.2000,70.3000) -- (84.2000,70.3000) -- (84.2000,70.3000) -- (84.2000,70.3000) -- (84.2000,70.3000) -- (84.2000,70.3000) -- (84.2000,70.3000) -- (84.2000,70.3000) -- (84.2000,70.3000) -- (84.2000,70.3000) -- (84.2000,70.3000) -- (84.2000,70.3000) -- (84.2000,70.3000) -- (84.2000,70.3000) -- (84.2000,70.3000) -- (84.2000,70.3000) -- (84.3000,70.3000) -- (84.3000,70.3000) -- (84.3000,70.2000) -- (84.3000,70.2000) -- (84.3000,70.2000) -- (84.3000,70.2000) -- (84.3000,70.2000) -- (84.3000,70.2000) -- (84.3000,70.2000) -- (84.3000,70.2000) -- (84.3000,70.2000) -- (84.3000,70.2000) -- (84.3000,70.2000) -- (84.3000,70.2000) -- (84.3000,70.2000) -- (84.3000,70.2000) -- (84.3000,70.2000) -- (84.3000,70.2000) -- (84.3000,70.2000) -- (84.3000,70.2000) -- (84.3000,70.2000) -- (84.3000,70.2000) -- (84.3000,70.1000) -- (84.3000,70.1000) -- (84.3000,70.1000) -- (84.3000,70.1000) -- (84.3000,70.1000) -- (84.3000,70.1000) -- (84.3000,70.1000) -- (84.3000,70.1000) -- (84.3000,70.1000) -- (84.3000,70.1000) -- (84.3000,70.1000) -- (84.3000,70.1000) -- (84.3000,70.1000) -- (84.3000,70.1000) -- (84.3000,70.1000) -- (84.3000,70.1000) -- (84.3000,70.1000) -- (84.3000,70.1000) -- (84.3000,70.1000) -- (84.3000,70.1000) -- (84.3000,70.1000) -- (84.3000,70.0000) -- (84.3000,70.0000) -- (84.3000,70.0000) -- (84.3000,70.0000) -- (84.3000,70.0000) -- (84.3000,70.0000) -- (84.3000,70.0000) -- (84.4000,70.0000) -- (84.4000,70.0000) -- (84.4000,70.0000) -- (84.4000,70.0000) -- (84.4000,70.0000) -- (84.4000,70.0000) -- (84.4000,70.0000) -- (84.4000,70.0000) -- (84.4000,70.0000) -- (84.4000,70.0000) -- (84.4000,70.0000) -- (84.4000,70.0000) -- (84.4000,70.0000) -- (84.4000,70.0000) -- (84.4000,69.9000) -- (84.4000,69.9000) -- (84.4000,69.9000) -- (84.4000,69.9000) -- (84.4000,69.9000) -- (84.4000,69.9000) -- (84.4000,69.9000) -- (84.4000,69.9000) -- (84.4000,69.9000) -- (84.4000,69.9000) -- (84.4000,69.9000) -- (84.4000,69.9000) -- (84.4000,69.9000) -- (84.4000,69.9000) -- (84.4000,69.9000) -- (84.4000,69.9000) -- (84.4000,69.9000) -- (84.4000,69.9000) -- (84.4000,69.9000) -- (84.4000,69.9000) -- (84.4000,69.9000) -- (84.4000,69.8000) -- (84.4000,69.8000) -- (84.4000,69.8000) -- (84.4000,69.8000) -- (84.4000,69.8000) -- (84.4000,69.8000) -- (84.4000,69.8000) -- (84.4000,69.8000) -- (84.4000,69.8000) -- (84.4000,69.8000) -- (84.4000,69.8000) -- (84.4000,69.8000) -- (84.4000,69.8000) -- (84.4000,69.8000) -- (84.5000,69.8000) -- (84.5000,69.8000) -- (84.5000,69.8000) -- (84.5000,69.8000) -- (84.5000,69.8000) -- (84.5000,69.8000) -- (84.5000,69.8000) -- (84.5000,69.7000) -- (84.5000,69.7000) -- (84.5000,69.7000) -- (84.5000,69.7000) -- (84.5000,69.7000) -- (84.5000,69.7000) -- (84.5000,69.7000) -- (84.5000,69.7000) -- (84.5000,69.7000) -- (84.5000,69.7000) -- (84.5000,69.7000) -- (84.5000,69.7000) -- (84.5000,69.7000) -- (84.5000,69.7000) -- (84.5000,69.7000) -- (84.5000,69.7000) -- (84.5000,69.7000) -- (84.5000,69.7000) -- (84.5000,69.7000) -- (84.5000,69.7000) -- (84.5000,69.6000) -- (84.5000,69.6000) -- (84.5000,69.6000) -- (84.5000,69.6000) -- (84.5000,69.6000) -- (84.5000,69.6000) -- (84.5000,69.6000) -- (84.5000,69.6000) -- (84.5000,69.6000) -- (84.5000,69.6000) -- (84.5000,69.6000) -- (84.5000,69.6000) -- (84.5000,69.6000) -- (84.5000,69.6000) -- (84.5000,69.6000) -- (84.5000,69.6000) -- (84.5000,69.6000) -- (84.5000,69.6000) -- (84.5000,69.6000) -- (84.5000,69.6000) -- (84.5000,69.6000) -- (84.5000,69.5000) -- (84.5000,69.5000) -- (84.6000,69.5000) -- (84.6000,69.5000) -- (84.6000,69.5000) -- (84.6000,69.5000) -- (84.6000,69.5000) -- (84.6000,69.5000) -- (84.6000,69.5000) -- (84.6000,69.5000) -- (84.6000,69.5000) -- (84.6000,69.5000) -- (84.6000,69.5000) -- (84.6000,69.5000) -- (84.6000,69.5000) -- (84.6000,69.5000) -- (84.6000,69.5000) -- (84.6000,69.5000) -- (84.6000,69.5000) -- (84.6000,69.5000) -- (84.6000,69.5000) -- (84.6000,69.4000) -- (84.6000,69.4000) -- (84.6000,69.4000) -- (84.6000,69.4000) -- (84.6000,69.4000) -- (84.6000,69.4000) -- (84.6000,69.4000) -- (84.6000,69.4000) -- (84.6000,69.4000) -- (84.6000,69.4000) -- (84.6000,69.4000) -- (84.6000,69.4000) -- (84.6000,69.4000) -- (84.6000,69.4000) -- (84.6000,69.4000) -- (84.6000,69.4000) -- (84.6000,69.4000) -- (84.6000,69.4000) -- (84.6000,69.4000) -- (84.6000,69.4000) -- (84.6000,69.4000) -- (84.6000,69.3000) -- (84.6000,69.3000) -- (84.6000,69.3000) -- (84.6000,69.3000) -- (84.6000,69.3000) -- (84.6000,69.3000) -- (84.6000,69.3000) -- (84.6000,69.3000) -- (84.6000,69.3000) -- (84.7000,69.3000) -- (84.7000,69.3000) -- (84.7000,69.3000) -- (84.7000,69.3000) -- (84.7000,69.3000) -- (84.7000,69.3000) -- (84.7000,69.3000) -- (84.7000,69.3000) -- (84.7000,69.3000) -- (84.7000,69.3000) -- (84.7000,69.3000) -- (84.7000,69.2000) -- (84.7000,69.2000) -- (84.7000,69.2000) -- (84.7000,69.2000) -- (84.7000,69.2000) -- (84.7000,69.2000) -- (84.7000,69.2000) -- (84.7000,69.2000) -- (84.7000,69.2000) -- (84.7000,69.2000) -- (84.7000,69.2000) -- (84.7000,69.2000) -- (84.7000,69.2000) -- (84.7000,69.2000) -- (84.7000,69.2000) -- (84.7000,69.2000) -- (84.7000,69.2000) -- (84.7000,69.2000) -- (84.7000,69.2000) -- (84.7000,69.2000) -- (84.7000,69.2000) -- (84.7000,69.1000) -- (84.7000,69.1000) -- (84.7000,69.1000) -- (84.7000,69.1000) -- (84.7000,69.1000) -- (84.7000,69.1000) -- (84.7000,69.1000) -- (84.7000,69.1000) -- (84.7000,69.1000) -- (84.7000,69.1000) -- (84.7000,69.1000) -- (84.7000,69.1000) -- (84.7000,69.1000) -- (84.7000,69.1000) -- (84.7000,69.1000) -- (84.7000,69.1000) -- (84.7000,69.1000) -- (84.7000,69.1000) -- (84.8000,69.1000) -- (84.8000,69.1000) -- (84.8000,69.1000) -- (84.8000,69.0000) -- (84.8000,69.0000) -- (84.8000,69.0000) -- (84.8000,69.0000) -- (84.8000,69.0000) -- (84.8000,69.0000) -- (84.8000,69.0000) -- (84.8000,69.0000) -- (84.8000,69.0000) -- (84.8000,69.0000) -- (84.8000,69.0000) -- (84.8000,69.0000) -- (84.8000,69.0000) -- (84.8000,69.0000) -- (84.8000,69.0000) -- (84.8000,69.0000) -- (84.8000,69.0000) -- (84.8000,69.0000) -- (84.8000,69.0000) -- (84.8000,69.0000) -- (84.8000,69.0000) -- (84.8000,68.9000) -- (84.8000,68.9000) -- (84.8000,68.9000) -- (84.8000,68.9000) -- (84.8000,68.9000) -- (84.8000,68.9000) -- (84.8000,68.9000) -- (84.8000,68.9000) -- (84.8000,68.9000) -- (84.8000,68.9000) -- (84.8000,68.9000) -- (84.8000,68.9000) -- (84.8000,68.9000) -- (84.8000,68.9000) -- (84.8000,68.9000) -- (84.8000,68.9000) -- (84.8000,68.9000) -- (84.8000,68.9000) -- (84.8000,68.9000) -- (84.8000,68.9000) -- (84.8000,68.9000) -- (84.8000,68.8000) -- (84.8000,68.8000) -- (84.8000,68.8000) -- (84.8000,68.8000) -- (84.9000,68.8000) -- (84.9000,68.8000) -- (84.9000,68.8000) -- (84.9000,68.8000) -- (84.9000,68.8000) -- (84.9000,68.8000) -- (84.9000,68.8000) -- (84.9000,68.8000) -- (84.9000,68.8000) -- (84.9000,68.8000) -- (84.9000,68.8000) -- (84.9000,68.8000) -- (84.9000,68.8000) -- (84.9000,68.8000) -- (84.9000,68.8000) -- (84.9000,68.8000) -- (84.9000,68.7000) -- (84.9000,68.7000) -- (84.9000,68.7000) -- (84.9000,68.7000) -- (84.9000,68.7000) -- (84.9000,68.7000) -- (84.9000,68.7000) -- (84.9000,68.7000) -- (84.9000,68.7000) -- (84.9000,68.7000) -- (84.9000,68.7000) -- (84.9000,68.7000) -- (84.9000,68.7000) -- (84.9000,68.7000) -- (84.9000,68.7000) -- (84.9000,68.7000) -- (84.9000,68.7000) -- (84.9000,68.7000) -- (84.9000,68.7000) -- (84.9000,68.7000) -- (84.9000,68.7000) -- (84.9000,68.6000) -- (84.9000,68.6000) -- (84.9000,68.6000) -- (84.9000,68.6000) -- (84.9000,68.6000) -- (84.9000,68.6000) -- (84.9000,68.6000) -- (84.9000,68.6000) -- (84.9000,68.6000) -- (84.9000,68.6000) -- (84.9000,68.6000) -- (84.9000,68.6000) -- (84.9000,68.6000) -- (85.0000,68.6000) -- (85.0000,68.6000) -- (85.0000,68.6000) -- (85.0000,68.6000) -- (85.0000,68.6000) -- (85.0000,68.6000) -- (85.0000,68.6000) -- (85.0000,68.6000) -- (85.0000,68.5000) -- (85.0000,68.5000) -- (85.0000,68.5000) -- (85.0000,68.5000) -- (85.0000,68.5000) -- (85.0000,68.5000) -- (85.0000,68.5000) -- (85.0000,68.5000) -- (85.0000,68.5000) -- (85.0000,68.5000) -- (85.0000,68.5000) -- (85.0000,68.5000) -- (85.0000,68.5000) -- (85.0000,68.5000) -- (85.0000,68.5000) -- (85.0000,68.5000) -- (85.0000,68.5000) -- (85.0000,68.5000) -- (85.0000,68.5000) -- (85.0000,68.5000) -- (85.0000,68.5000) -- (85.0000,68.4000) -- (85.0000,68.4000) -- (85.0000,68.4000) -- (85.0000,68.4000) -- (85.0000,68.4000) -- (85.0000,68.4000) -- (85.0000,68.4000) -- (85.0000,68.4000) -- (85.0000,68.4000) -- (85.0000,68.4000) -- (85.0000,68.4000) -- (85.0000,68.4000) -- (85.0000,68.4000) -- (85.0000,68.4000) -- (85.0000,68.4000) -- (85.0000,68.4000) -- (85.0000,68.4000) -- (85.0000,68.4000) -- (85.0000,68.4000) -- (85.0000,68.4000) -- (85.1000,68.4000) -- (85.1000,68.3000) -- (85.1000,68.3000) -- (85.1000,68.3000) -- (85.1000,68.3000) -- (85.1000,68.3000) -- (85.1000,68.3000) -- (85.1000,68.3000) -- (85.1000,68.3000) -- (85.1000,68.3000) -- (85.1000,68.3000) -- (85.1000,68.3000) -- (85.1000,68.3000) -- (85.1000,68.3000) -- (85.1000,68.3000) -- (85.1000,68.3000) -- (85.1000,68.3000) -- (85.1000,68.3000) -- (85.1000,68.3000) -- (85.1000,68.3000) -- (85.1000,68.3000) -- (85.1000,68.2000) -- (85.1000,68.2000) -- (85.1000,68.2000) -- (85.1000,68.2000) -- (85.1000,68.2000) -- (85.1000,68.2000) -- (85.1000,68.2000) -- (85.1000,68.2000) -- (85.1000,68.2000) -- (85.1000,68.2000) -- (85.1000,68.2000) -- (85.1000,68.2000) -- (85.1000,68.2000) -- (85.1000,68.2000) -- (85.1000,68.2000) -- (85.1000,68.2000) -- (85.1000,68.2000) -- (85.1000,68.2000) -- (85.1000,68.2000) -- (85.1000,68.2000) -- (85.1000,68.2000) -- (85.1000,68.1000) -- (85.1000,68.1000) -- (85.1000,68.1000) -- (85.1000,68.1000) -- (85.1000,68.1000) -- (85.1000,68.1000) -- (85.1000,68.1000) -- (85.1000,68.1000) -- (85.2000,68.1000) -- (85.2000,68.1000) -- (85.2000,68.1000) -- (85.2000,68.1000) -- (85.2000,68.1000) -- (85.2000,68.1000) -- (85.2000,68.1000) -- (85.2000,68.1000) -- (85.2000,68.1000) -- (85.2000,68.1000) -- (85.2000,68.1000) -- (85.2000,68.1000) -- (85.2000,68.1000) -- (85.2000,68.0000) -- (85.2000,68.0000) -- (85.2000,68.0000) -- (85.2000,68.0000) -- (85.2000,68.0000) -- (85.2000,68.0000) -- (85.2000,68.0000) -- (85.2000,68.0000) -- (85.2000,68.0000) -- (85.2000,68.0000) -- (85.2000,68.0000) -- (85.2000,68.0000) -- (85.2000,68.0000) -- (85.2000,68.0000) -- (85.2000,68.0000) -- (85.2000,68.0000) -- (85.2000,68.0000) -- (85.2000,68.0000) -- (85.2000,68.0000) -- (85.2000,68.0000) -- (85.2000,68.0000) -- (85.2000,67.9000) -- (85.2000,67.9000) -- (85.2000,67.9000) -- (85.2000,67.9000) -- (85.2000,67.9000) -- (85.2000,67.9000) -- (85.2000,67.9000) -- (85.2000,67.9000) -- (85.2000,67.9000) -- (85.2000,67.9000) -- (85.2000,67.9000) -- (85.2000,67.9000) -- (85.2000,67.9000) -- (85.2000,67.9000) -- (85.2000,67.9000) -- (85.3000,67.9000) -- (85.3000,67.9000) -- (85.3000,67.9000) -- (85.3000,67.9000) -- (85.3000,67.9000) -- (85.3000,67.9000) -- (85.3000,67.8000) -- (85.3000,67.8000) -- (85.3000,67.8000) -- (85.3000,67.8000) -- (85.3000,67.8000) -- (85.3000,67.8000) -- (85.3000,67.8000) -- (85.3000,67.8000) -- (85.3000,67.8000) -- (85.3000,67.8000) -- (85.3000,67.8000) -- (85.3000,67.8000) -- (85.3000,67.8000) -- (85.3000,67.8000) -- (85.3000,67.8000) -- (85.3000,67.8000) -- (85.3000,67.8000) -- (85.3000,67.8000) -- (85.3000,67.8000) -- (85.3000,67.8000) -- (85.3000,67.7000) -- (85.3000,67.7000) -- (85.3000,67.7000) -- (85.3000,67.7000) -- (85.3000,67.7000) -- (85.3000,67.7000) -- (85.3000,67.7000) -- (85.3000,67.7000) -- (85.3000,67.7000) -- (85.3000,67.7000) -- (85.3000,67.7000) -- (85.3000,67.7000) -- (85.3000,67.7000) -- (85.3000,67.7000) -- (85.3000,67.7000) -- (85.3000,67.7000) -- (85.3000,67.7000) -- (85.3000,67.7000) -- (85.3000,67.7000) -- (85.3000,67.7000) -- (85.3000,67.7000) -- (85.3000,67.6000) -- (85.3000,67.6000) -- (85.3000,67.6000) -- (85.4000,67.6000) -- (85.4000,67.6000) -- (85.4000,67.6000) -- (85.4000,67.6000) -- (85.4000,67.6000) -- (85.4000,67.6000) -- (85.4000,67.6000) -- (85.4000,67.6000) -- (85.4000,67.6000) -- (85.4000,67.6000) -- (85.4000,67.6000) -- (85.4000,67.6000) -- (85.4000,67.6000) -- (85.4000,67.6000) -- (85.4000,67.6000) -- (85.4000,67.6000) -- (85.4000,67.6000) -- (85.4000,67.6000) -- (85.4000,67.5000) -- (85.4000,67.5000) -- (85.4000,67.5000) -- (85.4000,67.5000) -- (85.4000,67.5000) -- (85.4000,67.5000) -- (85.4000,67.5000) -- (85.4000,67.5000) -- (85.4000,67.5000) -- (85.4000,67.5000) -- (85.4000,67.5000) -- (85.4000,67.5000) -- (85.4000,67.5000) -- (85.4000,67.5000) -- (85.4000,67.5000) -- (85.4000,67.5000) -- (85.4000,67.5000) -- (85.4000,67.5000) -- (85.4000,67.5000) -- (85.4000,67.5000) -- (85.4000,67.5000) -- (85.4000,67.4000) -- (85.4000,67.4000) -- (85.4000,67.4000) -- (85.4000,67.4000) -- (85.4000,67.4000) -- (85.4000,67.4000) -- (85.4000,67.4000) -- (85.4000,67.4000) -- (85.4000,67.4000) -- (85.4000,67.4000) -- (85.5000,67.4000) -- (85.5000,67.4000) -- (85.5000,67.4000) -- (85.5000,67.4000) -- (85.5000,67.4000) -- (85.5000,67.4000) -- (85.5000,67.4000) -- (85.5000,67.4000) -- (85.5000,67.4000) -- (85.5000,67.4000) -- (85.5000,67.3000) -- (85.5000,67.3000) -- (85.5000,67.3000) -- (85.5000,67.3000) -- (85.5000,67.3000) -- (85.5000,67.3000) -- (85.5000,67.3000) -- (85.5000,67.3000) -- (85.5000,67.3000) -- (85.5000,67.3000) -- (85.5000,67.3000) -- (85.5000,67.3000) -- (85.5000,67.3000) -- (85.5000,67.3000) -- (85.5000,67.3000) -- (85.5000,67.3000) -- (85.5000,67.3000) -- (85.5000,67.3000) -- (85.5000,67.3000) -- (85.5000,67.3000) -- (85.5000,67.3000) -- (85.5000,67.2000) -- (85.5000,67.2000) -- (85.5000,67.2000) -- (85.5000,67.2000) -- (85.5000,67.2000) -- (85.5000,67.2000) -- (85.5000,67.2000) -- (85.5000,67.2000) -- (85.5000,67.2000) -- (85.5000,67.2000) -- (85.5000,67.2000) -- (85.5000,67.2000) -- (85.5000,67.2000) -- (85.5000,67.2000) -- (85.5000,67.2000) -- (85.5000,67.2000) -- (85.5000,67.2000) -- (85.5000,67.2000) -- (85.5000,67.2000) -- (85.6000,67.2000) -- (85.6000,67.2000) -- (85.6000,67.1000) -- (85.6000,67.1000) -- (85.6000,67.1000) -- (85.6000,67.1000) -- (85.6000,67.1000) -- (85.6000,67.1000) -- (85.6000,67.1000) -- (85.6000,67.1000) -- (85.6000,67.1000) -- (85.6000,67.1000) -- (85.6000,67.1000) -- (85.6000,67.1000) -- (85.6000,67.1000) -- (85.6000,67.1000) -- (85.6000,67.1000) -- (85.6000,67.1000) -- (85.6000,67.1000) -- (85.6000,67.1000) -- (85.6000,67.1000) -- (85.6000,67.1000) -- (85.6000,67.1000) -- (85.6000,67.0000) -- (85.6000,67.0000) -- (85.6000,67.0000) -- (85.6000,67.0000) -- (85.6000,67.0000) -- (85.6000,67.0000) -- (85.6000,67.0000) -- (85.6000,67.0000) -- (85.6000,67.0000) -- (85.6000,67.0000) -- (85.6000,67.0000) -- (85.6000,67.0000) -- (85.6000,67.0000) -- (85.6000,67.0000) -- (85.6000,67.0000) -- (85.6000,67.0000) -- (85.6000,67.0000) -- (85.6000,67.0000) -- (85.6000,67.0000) -- (85.6000,67.0000) -- (85.6000,67.0000) -- (85.6000,66.9000) -- (85.6000,66.9000) -- (85.6000,66.9000) -- (85.6000,66.9000) -- (85.6000,66.9000) -- (85.7000,66.9000) -- (85.7000,66.9000) -- (85.7000,66.9000) -- (85.7000,66.9000) -- (85.7000,66.9000) -- (85.7000,66.9000) -- (85.7000,66.9000) -- (85.7000,66.9000) -- (85.7000,66.9000) -- (85.7000,66.9000) -- (85.7000,66.9000) -- (85.7000,66.9000) -- (85.7000,66.9000) -- (85.7000,66.9000) -- (85.7000,66.9000) -- (85.7000,66.8000) -- (85.7000,66.8000) -- (85.7000,66.8000) -- (85.7000,66.8000) -- (85.7000,66.8000) -- (85.7000,66.8000) -- (85.7000,66.8000) -- (85.7000,66.8000) -- (85.7000,66.8000) -- (85.7000,66.8000) -- (85.7000,66.8000) -- (85.7000,66.8000) -- (85.7000,66.8000) -- (85.7000,66.8000) -- (85.7000,66.8000) -- (85.7000,66.8000) -- (85.7000,66.8000) -- (85.7000,66.8000) -- (85.7000,66.8000) -- (85.7000,66.8000) -- (85.7000,66.8000) -- (85.7000,66.7000) -- (85.7000,66.7000) -- (85.7000,66.7000) -- (85.7000,66.7000) -- (85.7000,66.7000) -- (85.7000,66.7000) -- (85.7000,66.7000) -- (85.7000,66.7000) -- (85.7000,66.7000) -- (85.7000,66.7000) -- (85.7000,66.7000) -- (85.7000,66.7000) -- (85.7000,66.7000) -- (85.7000,66.7000) -- (85.8000,66.7000) -- (85.8000,66.7000) -- (85.8000,66.7000) -- (85.8000,66.7000) -- (85.8000,66.7000) -- (85.8000,66.7000) -- (85.8000,66.7000) -- (85.8000,66.6000) -- (85.8000,66.6000) -- (85.8000,66.6000) -- (85.8000,66.6000) -- (85.8000,66.6000) -- (85.8000,66.6000) -- (85.8000,66.6000) -- (85.8000,66.6000) -- (85.8000,66.6000) -- (85.8000,66.6000) -- (85.8000,66.6000) -- (85.8000,66.6000) -- (85.8000,66.6000) -- (85.8000,66.6000) -- (85.8000,66.6000) -- (85.8000,66.6000) -- (85.8000,66.6000) -- (85.8000,66.6000) -- (85.8000,66.6000) -- (85.8000,66.6000) -- (85.8000,66.6000) -- (85.8000,66.5000) -- (85.8000,66.5000) -- (85.8000,66.5000) -- (85.8000,66.5000) -- (85.8000,66.5000) -- (85.8000,66.5000) -- (85.8000,66.5000) -- (85.8000,66.5000) -- (85.8000,66.5000) -- (85.8000,66.5000) -- (85.8000,66.5000) -- (85.8000,66.5000) -- (85.8000,66.5000) -- (85.8000,66.5000) -- (85.8000,66.5000) -- (85.8000,66.5000) -- (85.8000,66.5000) -- (85.8000,66.5000) -- (85.8000,66.5000) -- (85.8000,66.5000) -- (85.8000,66.5000) -- (85.9000,66.4000) -- (85.9000,66.4000) -- (85.9000,66.4000) -- (85.9000,66.4000) -- (85.9000,66.4000) -- (85.9000,66.4000) -- (85.9000,66.4000) -- (85.9000,66.4000) -- (85.9000,66.4000) -- (85.9000,66.4000) -- (85.9000,66.4000) -- (85.9000,66.4000) -- (85.9000,66.4000) -- (85.9000,66.4000) -- (85.9000,66.4000) -- (85.9000,66.4000) -- (85.9000,66.4000) -- (85.9000,66.4000) -- (85.9000,66.4000) -- (85.9000,66.4000) -- (85.9000,66.3000) -- (85.9000,66.3000) -- (85.9000,66.3000) -- (85.9000,66.3000) -- (85.9000,66.3000) -- (85.9000,66.3000) -- (85.9000,66.3000) -- (85.9000,66.3000) -- (85.9000,66.3000) -- (85.9000,66.3000) -- (85.9000,66.3000) -- (85.9000,66.3000) -- (85.9000,66.3000) -- (85.9000,66.3000) -- (85.9000,66.3000) -- (85.9000,66.3000) -- (85.9000,66.3000) -- (85.9000,66.3000) -- (85.9000,66.3000) -- (85.9000,66.3000) -- (85.9000,66.3000) -- (85.9000,66.2000) -- (85.9000,66.2000) -- (85.9000,66.2000) -- (85.9000,66.2000) -- (85.9000,66.2000) -- (85.9000,66.2000) -- (85.9000,66.2000) -- (85.9000,66.2000) -- (85.9000,66.2000) -- (86.0000,66.2000) -- (86.0000,66.2000) -- (86.0000,66.2000) -- (86.0000,66.2000) -- (86.0000,66.2000) -- (86.0000,66.2000) -- (86.0000,66.2000) -- (86.0000,66.2000) -- (86.0000,66.2000) -- (86.0000,66.2000) -- (86.0000,66.2000) -- (86.0000,66.2000) -- (86.0000,66.1000) -- (86.0000,66.1000) -- (86.0000,66.1000) -- (86.0000,66.1000) -- (86.0000,66.1000) -- (86.0000,66.1000) -- (86.0000,66.1000) -- (86.0000,66.1000) -- (86.0000,66.1000) -- (86.0000,66.1000) -- (86.0000,66.1000) -- (86.0000,66.1000) -- (86.0000,66.1000) -- (86.0000,66.1000) -- (86.0000,66.1000) -- (86.0000,66.1000) -- (86.0000,66.1000) -- (86.0000,66.1000) -- (86.0000,66.1000) -- (86.0000,66.1000) -- (86.0000,66.1000) -- (86.0000,66.0000) -- (86.0000,66.0000) -- (86.0000,66.0000) -- (86.0000,66.0000) -- (86.0000,66.0000) -- (86.0000,66.0000) -- (86.0000,66.0000) -- (86.0000,66.0000) -- (86.0000,66.0000) -- (86.0000,66.0000) -- (86.0000,66.0000) -- (86.0000,66.0000) -- (86.0000,66.0000) -- (86.0000,66.0000) -- (86.0000,66.0000) -- (86.0000,66.0000) -- (86.1000,66.0000) -- (86.1000,66.0000) -- (86.1000,66.0000) -- (86.1000,66.0000) -- (86.1000,66.0000) -- (86.1000,65.9000) -- (86.1000,65.9000) -- (86.1000,65.9000) -- (86.1000,65.9000) -- (86.1000,65.9000) -- (86.1000,65.9000) -- (86.1000,65.9000) -- (86.1000,65.9000) -- (86.1000,65.9000) -- (86.1000,65.9000) -- (86.1000,65.9000) -- (86.1000,65.9000) -- (86.1000,65.9000) -- (86.1000,65.9000) -- (86.1000,65.9000) -- (86.1000,65.9000) -- (86.1000,65.9000) -- (86.1000,65.9000) -- (86.1000,65.9000) -- (86.1000,65.9000) -- (86.1000,65.8000) -- (86.1000,65.8000) -- (86.1000,65.8000) -- (86.1000,65.8000) -- (86.1000,65.8000) -- (86.1000,65.8000) -- (86.1000,65.8000) -- (86.1000,65.8000) -- (86.1000,65.8000) -- (86.1000,65.8000) -- (86.1000,65.8000) -- (86.1000,65.8000) -- (86.1000,65.8000) -- (86.1000,65.8000) -- (86.1000,65.8000) -- (86.1000,65.8000) -- (86.1000,65.8000) -- (86.1000,65.8000) -- (86.1000,65.8000) -- (86.1000,65.8000) -- (86.1000,65.8000) -- (86.1000,65.7000) -- (86.1000,65.7000) -- (86.1000,65.7000) -- (86.1000,65.7000) -- (86.2000,65.7000) -- (86.2000,65.7000) -- (86.2000,65.7000) -- (86.2000,65.7000) -- (86.2000,65.7000) -- (86.2000,65.7000) -- (86.2000,65.7000) -- (86.2000,65.7000) -- (86.2000,65.7000) -- (86.2000,65.7000) -- (86.2000,65.7000) -- (86.2000,65.7000) -- (86.2000,65.7000) -- (86.2000,65.7000) -- (86.2000,65.7000) -- (86.2000,65.7000) -- (86.2000,65.7000) -- (86.2000,65.6000) -- (86.2000,65.6000) -- (86.2000,65.6000) -- (86.2000,65.6000) -- (86.2000,65.6000) -- (86.2000,65.6000) -- (86.2000,65.6000) -- (86.2000,65.6000) -- (86.2000,65.6000) -- (86.2000,65.6000) -- (86.2000,65.6000) -- (86.2000,65.6000) -- (86.2000,65.6000) -- (86.2000,65.6000) -- (86.2000,65.6000) -- (86.2000,65.6000) -- (86.2000,65.6000) -- (86.2000,65.6000) -- (86.2000,65.6000) -- (86.2000,65.6000) -- (86.2000,65.6000) -- (86.2000,65.5000) -- (86.2000,65.5000) -- (86.2000,65.5000) -- (86.2000,65.5000) -- (86.2000,65.5000) -- (86.2000,65.5000) -- (86.2000,65.5000) -- (86.2000,65.5000) -- (86.2000,65.5000) -- (86.2000,65.5000) -- (86.2000,65.5000) -- (86.2000,65.5000) -- (86.3000,65.5000) -- (86.3000,65.5000) -- (86.3000,65.5000) -- (86.3000,65.5000) -- (86.3000,65.5000) -- (86.3000,65.5000) -- (86.3000,65.5000) -- (86.3000,65.5000) -- (86.3000,65.4000) -- (86.3000,65.4000) -- (86.3000,65.4000) -- (86.3000,65.4000) -- (86.3000,65.4000) -- (86.3000,65.4000) -- (86.3000,65.4000) -- (86.3000,65.4000) -- (86.3000,65.4000) -- (86.3000,65.4000) -- (86.3000,65.4000) -- (86.3000,65.4000) -- (86.3000,65.4000) -- (86.3000,65.4000) -- (86.3000,65.4000) -- (86.3000,65.4000) -- (86.3000,65.4000) -- (86.3000,65.4000) -- (86.3000,65.4000) -- (86.3000,65.4000) -- (86.3000,65.4000) -- (86.3000,65.3000) -- (86.3000,65.3000) -- (86.3000,65.3000) -- (86.3000,65.3000) -- (86.3000,65.3000) -- (86.3000,65.3000) -- (86.3000,65.3000) -- (86.3000,65.3000) -- (86.3000,65.3000) -- (86.3000,65.3000) -- (86.3000,65.3000) -- (86.3000,65.3000) -- (86.3000,65.3000) -- (86.3000,65.3000) -- (86.3000,65.3000) -- (86.3000,65.3000) -- (86.3000,65.3000) -- (86.3000,65.3000) -- (86.3000,65.3000) -- (86.3000,65.3000) -- (86.4000,65.3000) -- (86.4000,65.2000) -- (86.4000,65.2000) -- (86.4000,65.2000) -- (86.4000,65.2000) -- (86.4000,65.2000) -- (86.4000,65.2000) -- (86.4000,65.2000) -- (86.4000,65.2000) -- (86.4000,65.2000) -- (86.4000,65.2000) -- (86.4000,65.2000) -- (86.4000,65.2000) -- (86.4000,65.2000) -- (86.4000,65.2000) -- (86.4000,65.2000) -- (86.4000,65.2000) -- (86.4000,65.2000) -- (86.4000,65.2000) -- (86.4000,65.2000) -- (86.4000,65.2000) -- (86.4000,65.2000) -- (86.4000,65.1000) -- (86.4000,65.1000) -- (86.4000,65.1000) -- (86.4000,65.1000) -- (86.4000,65.1000) -- (86.4000,65.1000) -- (86.4000,65.1000) -- (86.4000,65.1000) -- (86.4000,65.1000) -- (86.4000,65.1000) -- (86.4000,65.1000) -- (86.4000,65.1000) -- (86.4000,65.1000) -- (86.4000,65.1000) -- (86.4000,65.1000) -- (86.4000,65.1000) -- (86.4000,65.1000) -- (86.4000,65.1000) -- (86.4000,65.1000) -- (86.4000,65.1000) -- (86.4000,65.1000) -- (86.4000,65.0000) -- (86.4000,65.0000) -- (86.4000,65.0000) -- (86.4000,65.0000) -- (86.4000,65.0000) -- (86.4000,65.0000) -- (86.4000,65.0000) -- (86.5000,65.0000) -- (86.5000,65.0000) -- (86.5000,65.0000) -- (86.5000,65.0000) -- (86.5000,65.0000) -- (86.5000,65.0000) -- (86.5000,65.0000) -- (86.5000,65.0000) -- (86.5000,65.0000) -- (86.5000,65.0000) -- (86.5000,65.0000) -- (86.5000,65.0000) -- (86.5000,65.0000) -- (86.5000,64.9000) -- (86.5000,64.9000) -- (86.5000,64.9000) -- (86.5000,64.9000) -- (86.5000,64.9000) -- (86.5000,64.9000) -- (86.5000,64.9000) -- (86.5000,64.9000) -- (86.5000,64.9000) -- (86.5000,64.9000) -- (86.5000,64.9000) -- (86.5000,64.9000) -- (86.5000,64.9000) -- (86.5000,64.9000) -- (86.5000,64.9000) -- (86.5000,64.9000) -- (86.5000,64.9000) -- (86.5000,64.9000) -- (86.5000,64.9000) -- (86.5000,64.9000) -- (86.5000,64.9000) -- (86.5000,64.8000) -- (86.5000,64.8000) -- (86.5000,64.8000) -- (86.5000,64.8000) -- (86.5000,64.8000) -- (86.5000,64.8000) -- (86.5000,64.8000) -- (86.5000,64.8000) -- (86.5000,64.8000) -- (86.5000,64.8000) -- (86.5000,64.8000) -- (86.5000,64.8000) -- (86.5000,64.8000) -- (86.5000,64.8000) -- (86.5000,64.8000) -- (86.6000,64.8000) -- (86.6000,64.8000) -- (86.6000,64.8000) -- (86.6000,64.8000) -- (86.6000,64.8000) -- (86.6000,64.8000) -- (86.6000,64.7000) -- (86.6000,64.7000) -- (86.6000,64.7000) -- (86.6000,64.7000) -- (86.6000,64.7000) -- (86.6000,64.7000) -- (86.6000,64.7000) -- (86.6000,64.7000) -- (86.6000,64.7000) -- (86.6000,64.7000) -- (86.6000,64.7000) -- (86.6000,64.7000) -- (86.6000,64.7000) -- (86.6000,64.7000) -- (86.6000,64.7000) -- (86.6000,64.7000) -- (86.6000,64.7000) -- (86.6000,64.7000) -- (86.6000,64.7000) -- (86.6000,64.7000) -- (86.6000,64.7000) -- (86.6000,64.6000) -- (86.6000,64.6000) -- (86.6000,64.6000) -- (86.6000,64.6000) -- (86.6000,64.6000) -- (86.6000,64.6000) -- (86.6000,64.6000) -- (86.6000,64.6000) -- (86.6000,64.6000) -- (86.6000,64.6000) -- (86.6000,64.6000) -- (86.6000,64.6000) -- (86.6000,64.6000) -- (86.6000,64.6000) -- (86.6000,64.6000) -- (86.6000,64.6000) -- (86.6000,64.6000) -- (86.6000,64.6000) -- (86.6000,64.6000) -- (86.6000,64.6000) -- (86.6000,64.6000) -- (86.6000,64.5000) -- (86.6000,64.5000) -- (86.7000,64.5000) -- (86.7000,64.5000) -- (86.7000,64.5000) -- (86.7000,64.5000) -- (86.7000,64.5000) -- (86.7000,64.5000) -- (86.7000,64.5000) -- (86.7000,64.5000) -- (86.7000,64.5000) -- (86.7000,64.5000) -- (86.7000,64.5000) -- (86.7000,64.5000) -- (86.7000,64.5000) -- (86.7000,64.5000) -- (86.7000,64.5000) -- (86.7000,64.5000) -- (86.7000,64.5000) -- (86.7000,64.5000) -- (86.7000,64.4000) -- (86.7000,64.4000) -- (86.7000,64.4000) -- (86.7000,64.4000) -- (86.7000,64.4000) -- (86.7000,64.4000) -- (86.7000,64.4000) -- (86.7000,64.4000) -- (86.7000,64.4000) -- (86.7000,64.4000) -- (86.7000,64.4000) -- (86.7000,64.4000) -- (86.7000,64.4000) -- (86.7000,64.4000) -- (86.7000,64.4000) -- (86.7000,64.4000) -- (86.7000,64.4000) -- (86.7000,64.4000) -- (86.7000,64.4000) -- (86.7000,64.4000) -- (86.7000,64.4000) -- (86.7000,64.3000) -- (86.7000,64.3000) -- (86.7000,64.3000) -- (86.7000,64.3000) -- (86.7000,64.3000) -- (86.7000,64.3000) -- (86.7000,64.3000) -- (86.7000,64.3000) -- (86.7000,64.3000) -- (86.7000,64.3000) -- (86.8000,64.3000) -- (86.8000,64.3000) -- (86.8000,64.3000) -- (86.8000,64.3000) -- (86.8000,64.3000) -- (86.8000,64.3000) -- (86.8000,64.3000) -- (86.8000,64.3000) -- (86.8000,64.3000) -- (86.8000,64.3000) -- (86.8000,64.3000) -- (86.8000,64.2000) -- (86.8000,64.2000) -- (86.8000,64.2000) -- (86.8000,64.2000) -- (86.8000,64.2000) -- (86.8000,64.2000) -- (86.8000,64.2000) -- (86.8000,64.2000) -- (86.8000,64.2000) -- (86.8000,64.2000) -- (86.8000,64.2000) -- (86.8000,64.2000) -- (86.8000,64.2000) -- (86.8000,64.2000) -- (86.8000,64.2000) -- (86.8000,64.2000) -- (86.8000,64.2000) -- (86.8000,64.2000) -- (86.8000,64.2000) -- (86.8000,64.2000) -- (86.8000,64.2000) -- (86.8000,64.1000) -- (86.8000,64.1000) -- (86.8000,64.1000) -- (86.8000,64.1000) -- (86.8000,64.1000) -- (86.8000,64.1000) -- (86.8000,64.1000) -- (86.8000,64.1000) -- (86.8000,64.1000) -- (86.8000,64.1000) -- (86.8000,64.1000) -- (86.8000,64.1000) -- (86.8000,64.1000) -- (86.8000,64.1000) -- (86.8000,64.1000) -- (86.8000,64.1000) -- (86.8000,64.1000) -- (86.8000,64.1000) -- (86.9000,64.1000) -- (86.9000,64.1000) -- (86.9000,64.0000) -- (86.9000,64.0000) -- (86.9000,64.0000) -- (86.9000,64.0000) -- (86.9000,64.0000) -- (86.9000,64.0000) -- (86.9000,64.0000) -- (86.9000,64.0000) -- (86.9000,64.0000) -- (86.9000,64.0000) -- (86.9000,64.0000) -- (86.9000,64.0000) -- (86.9000,64.0000) -- (86.9000,64.0000) -- (86.9000,64.0000) -- (86.9000,64.0000) -- (86.9000,64.0000) -- (86.9000,64.0000) -- (86.9000,64.0000) -- (86.9000,64.0000) -- (86.9000,64.0000) -- (86.9000,63.9000) -- (86.9000,63.9000) -- (86.9000,63.9000) -- (86.9000,63.9000) -- (86.9000,63.9000) -- (86.9000,63.9000) -- (86.9000,63.9000) -- (86.9000,63.9000) -- (86.9000,63.9000) -- (86.9000,63.9000) -- (86.9000,63.9000) -- (86.9000,63.9000) -- (86.9000,63.9000) -- (86.9000,63.9000) -- (86.9000,63.9000) -- (86.9000,63.9000) -- (86.9000,63.9000) -- (86.9000,63.9000) -- (86.9000,63.9000) -- (86.9000,63.9000) -- (86.9000,63.9000) -- (86.9000,63.8000) -- (86.9000,63.8000) -- (86.9000,63.8000) -- (86.9000,63.8000) -- (86.9000,63.8000) -- (87.0000,63.8000) -- (87.0000,63.8000) -- (87.0000,63.8000) -- (87.0000,63.8000) -- (87.0000,63.8000) -- (87.0000,63.8000) -- (87.0000,63.8000) -- (87.0000,63.8000) -- (87.0000,63.8000) -- (87.0000,63.8000) -- (87.0000,63.8000) -- (87.0000,63.8000) -- (87.0000,63.8000) -- (87.0000,63.8000) -- (87.0000,63.8000) -- (87.0000,63.8000) -- (87.0000,63.7000) -- (87.0000,63.7000) -- (87.0000,63.7000) -- (87.0000,63.7000) -- (87.0000,63.7000) -- (87.0000,63.7000) -- (87.0000,63.7000) -- (87.0000,63.7000) -- (87.0000,63.7000) -- (87.0000,63.7000) -- (87.0000,63.7000) -- (87.0000,63.7000) -- (87.0000,63.7000) -- (87.0000,63.7000) -- (87.0000,63.7000) -- (87.0000,63.7000) -- (87.0000,63.7000) -- (87.0000,63.7000) -- (87.0000,63.7000) -- (87.0000,63.7000) -- (87.0000,63.7000) -- (87.0000,63.6000) -- (87.0000,63.6000) -- (87.0000,63.6000) -- (87.0000,63.6000) -- (87.0000,63.6000) -- (87.0000,63.6000) -- (87.0000,63.6000) -- (87.0000,63.6000) -- (87.0000,63.6000) -- (87.0000,63.6000) -- (87.0000,63.6000) -- (87.0000,63.6000) -- (87.0000,63.6000) -- (87.1000,63.6000) -- (87.1000,63.6000) -- (87.1000,63.6000) -- (87.1000,63.6000) -- (87.1000,63.6000) -- (87.1000,63.6000) -- (87.1000,63.6000) -- (87.1000,63.5000) -- (87.1000,63.5000) -- (87.1000,63.5000) -- (87.1000,63.5000) -- (87.1000,63.5000) -- (87.1000,63.5000) -- (87.1000,63.5000) -- (87.1000,63.5000) -- (87.1000,63.5000) -- (87.1000,63.5000) -- (87.1000,63.5000) -- (87.1000,63.5000) -- (87.1000,63.5000) -- (87.1000,63.5000) -- (87.1000,63.5000) -- (87.1000,63.5000) -- (87.1000,63.5000) -- (87.1000,63.5000) -- (87.1000,63.5000) -- (87.1000,63.5000) -- (87.1000,63.5000) -- (87.1000,63.4000) -- (87.1000,63.4000) -- (87.1000,63.4000) -- (87.1000,63.4000) -- (87.1000,63.4000) -- (87.1000,63.4000) -- (87.1000,63.4000) -- (87.1000,63.4000) -- (87.1000,63.4000) -- (87.1000,63.4000) -- (87.1000,63.4000) -- (87.1000,63.4000) -- (87.1000,63.4000) -- (87.1000,63.4000) -- (87.1000,63.4000) -- (87.1000,63.4000) -- (87.1000,63.4000) -- (87.1000,63.4000) -- (87.1000,63.4000) -- (87.1000,63.4000) -- (87.1000,63.4000) -- (87.2000,63.3000) -- (87.2000,63.3000) -- (87.2000,63.3000) -- (87.2000,63.3000) -- (87.2000,63.3000) -- (87.2000,63.3000) -- (87.2000,63.3000) -- (87.2000,63.3000) -- (87.2000,63.3000) -- (87.2000,63.3000) -- (87.2000,63.3000) -- (87.2000,63.3000) -- (87.2000,63.3000) -- (87.2000,63.3000) -- (87.2000,63.3000) -- (87.2000,63.3000) -- (87.2000,63.3000) -- (87.2000,63.3000) -- (87.2000,63.3000) -- (87.2000,63.3000) -- (87.2000,63.3000) -- (87.2000,63.2000) -- (87.2000,63.2000) -- (87.2000,63.2000) -- (87.2000,63.2000) -- (87.2000,63.2000) -- (87.2000,63.2000) -- (87.2000,63.2000) -- (87.2000,63.2000) -- (87.2000,63.2000) -- (87.2000,63.2000) -- (87.2000,63.2000) -- (87.2000,63.2000) -- (87.2000,63.2000) -- (87.2000,63.2000) -- (87.2000,63.2000) -- (87.2000,63.2000) -- (87.2000,63.2000) -- (87.2000,63.2000) -- (87.2000,63.2000) -- (87.2000,63.2000) -- (87.2000,63.2000) -- (87.2000,63.1000) -- (87.2000,63.1000) -- (87.2000,63.1000) -- (87.2000,63.1000) -- (87.2000,63.1000) -- (87.2000,63.1000) -- (87.2000,63.1000) -- (87.2000,63.1000) -- (87.3000,63.1000) -- (87.3000,63.1000) -- (87.3000,63.1000) -- (87.3000,63.1000) -- (87.3000,63.1000) -- (87.3000,63.1000) -- (87.3000,63.1000) -- (87.3000,63.1000) -- (87.3000,63.1000) -- (87.3000,63.1000) -- (87.3000,63.1000) -- (87.3000,63.1000) -- (87.3000,63.0000) -- (87.3000,63.0000) -- (87.3000,63.0000) -- (87.3000,63.0000) -- (87.3000,63.0000) -- (87.3000,63.0000) -- (87.3000,63.0000) -- (87.3000,63.0000) -- (87.3000,63.0000) -- (87.3000,63.0000) -- (87.3000,63.0000) -- (87.3000,63.0000) -- (87.3000,63.0000) -- (87.3000,63.0000) -- (87.3000,63.0000) -- (87.3000,63.0000) -- (87.3000,63.0000) -- (87.3000,63.0000) -- (87.3000,63.0000) -- (87.3000,63.0000) -- (87.3000,63.0000) -- (87.3000,62.9000) -- (87.3000,62.9000) -- (87.3000,62.9000) -- (87.3000,62.9000) -- (87.3000,62.9000) -- (87.3000,62.9000) -- (87.3000,62.9000) -- (87.3000,62.9000) -- (87.3000,62.9000) -- (87.3000,62.9000) -- (87.3000,62.9000) -- (87.3000,62.9000) -- (87.3000,62.9000) -- (87.3000,62.9000) -- (87.3000,62.9000) -- (87.3000,62.9000) -- (87.4000,62.9000) -- (87.4000,62.9000) -- (87.4000,62.9000) -- (87.4000,62.9000) -- (87.4000,62.9000) -- (87.4000,62.8000) -- (87.4000,62.8000) -- (87.4000,62.8000) -- (87.4000,62.8000) -- (87.4000,62.8000) -- (87.4000,62.8000) -- (87.4000,62.8000) -- (87.4000,62.8000) -- (87.4000,62.8000) -- (87.4000,62.8000) -- (87.4000,62.8000) -- (87.4000,62.8000) -- (87.4000,62.8000) -- (87.4000,62.8000) -- (87.4000,62.8000) -- (87.4000,62.8000) -- (87.4000,62.8000) -- (87.4000,62.8000) -- (87.4000,62.8000) -- (87.4000,62.8000) -- (87.4000,62.8000) -- (87.4000,62.7000) -- (87.4000,62.7000) -- (87.4000,62.7000) -- (87.4000,62.7000) -- (87.4000,62.7000) -- (87.4000,62.7000) -- (87.4000,62.7000) -- (87.4000,62.7000) -- (87.4000,62.7000) -- (87.4000,62.7000) -- (87.4000,62.7000) -- (87.4000,62.7000) -- (87.4000,62.7000) -- (87.4000,62.7000) -- (87.4000,62.7000) -- (87.4000,62.7000) -- (87.4000,62.7000) -- (87.4000,62.7000) -- (87.4000,62.7000) -- (87.4000,62.7000) -- (87.4000,62.7000) -- (87.4000,62.6000) -- (87.4000,62.6000) -- (87.4000,62.6000) -- (87.5000,62.6000) -- (87.5000,62.6000) -- (87.5000,62.6000) -- (87.5000,62.6000) -- (87.5000,62.6000) -- (87.5000,62.6000) -- (87.5000,62.6000) -- (87.5000,62.6000) -- (87.5000,62.6000) -- (87.5000,62.6000) -- (87.5000,62.6000) -- (87.5000,62.6000) -- (87.5000,62.6000) -- (87.5000,62.6000) -- (87.5000,62.6000) -- (87.5000,62.6000) -- (87.5000,62.6000) -- (87.5000,62.5000) -- (87.5000,62.5000) -- (87.5000,62.5000) -- (87.5000,62.5000) -- (87.5000,62.5000) -- (87.5000,62.5000) -- (87.5000,62.5000) -- (87.5000,62.5000) -- (87.5000,62.5000) -- (87.5000,62.5000) -- (87.5000,62.5000) -- (87.5000,62.5000) -- (87.5000,62.5000) -- (87.5000,62.5000) -- (87.5000,62.5000) -- (87.5000,62.5000) -- (87.5000,62.5000) -- (87.5000,62.5000) -- (87.5000,62.5000) -- (87.5000,62.5000) -- (87.5000,62.5000) -- (87.5000,62.4000) -- (87.5000,62.4000) -- (87.5000,62.4000) -- (87.5000,62.4000) -- (87.5000,62.4000) -- (87.5000,62.4000) -- (87.5000,62.4000) -- (87.5000,62.4000) -- (87.5000,62.4000) -- (87.5000,62.4000) -- (87.5000,62.4000) -- (87.6000,62.4000) -- (87.6000,62.4000) -- (87.6000,62.4000) -- (87.6000,62.4000) -- (87.6000,62.4000) -- (87.6000,62.4000) -- (87.6000,62.4000) -- (87.6000,62.4000) -- (87.6000,62.4000) -- (87.6000,62.4000) -- (87.6000,62.3000) -- (87.6000,62.3000) -- (87.6000,62.3000) -- (87.6000,62.3000) -- (87.6000,62.3000) -- (87.6000,62.3000) -- (87.6000,62.3000) -- (87.6000,62.3000) -- (87.6000,62.3000) -- (87.6000,62.3000) -- (87.6000,62.3000) -- (87.6000,62.3000) -- (87.6000,62.3000) -- (87.6000,62.3000) -- (87.6000,62.3000) -- (87.6000,62.3000) -- (87.6000,62.3000) -- (87.6000,62.3000) -- (87.6000,62.3000) -- (87.6000,62.3000) -- (87.6000,62.3000) -- (87.6000,62.2000) -- (87.6000,62.2000) -- (87.6000,62.2000) -- (87.6000,62.2000) -- (87.6000,62.2000) -- (87.6000,62.2000) -- (87.6000,62.2000) -- (87.6000,62.2000) -- (87.6000,62.2000) -- (87.6000,62.2000) -- (87.6000,62.2000) -- (87.6000,62.2000) -- (87.6000,62.2000) -- (87.6000,62.2000) -- (87.6000,62.2000) -- (87.6000,62.2000) -- (87.6000,62.2000) -- (87.6000,62.2000) -- (87.6000,62.2000) -- (87.7000,62.2000) -- (87.7000,62.1000) -- (87.7000,62.1000) -- (87.7000,62.1000) -- (87.7000,62.1000) -- (87.7000,62.1000) -- (87.7000,62.1000) -- (87.7000,62.1000) -- (87.7000,62.1000) -- (87.7000,62.1000) -- (87.7000,62.1000) -- (87.7000,62.1000) -- (87.7000,62.1000) -- (87.7000,62.1000) -- (87.7000,62.1000) -- (87.7000,62.1000) -- (87.7000,62.1000) -- (87.7000,62.1000) -- (87.7000,62.1000) -- (87.7000,62.1000) -- (87.7000,62.1000) -- (87.7000,62.1000) -- (87.7000,62.0000) -- (87.7000,62.0000) -- (87.7000,62.0000) -- (87.7000,62.0000) -- (87.7000,62.0000) -- (87.7000,62.0000) -- (87.7000,62.0000) -- (87.7000,62.0000) -- (87.7000,62.0000) -- (87.7000,62.0000) -- (87.7000,62.0000) -- (87.7000,62.0000) -- (87.7000,62.0000) -- (87.7000,62.0000) -- (87.7000,62.0000) -- (87.7000,62.0000) -- (87.7000,62.0000) -- (87.7000,62.0000) -- (87.7000,62.0000) -- (87.7000,62.0000) -- (87.7000,62.0000) -- (87.7000,61.9000) -- (87.7000,61.9000) -- (87.7000,61.9000) -- (87.7000,61.9000) -- (87.7000,61.9000) -- (87.7000,61.9000) -- (87.8000,61.9000) -- (87.8000,61.9000) -- (87.8000,61.9000) -- (87.8000,61.9000) -- (87.8000,61.9000) -- (87.8000,61.9000) -- (87.8000,61.9000) -- (87.8000,61.9000) -- (87.8000,61.9000) -- (87.8000,61.9000) -- (87.8000,61.9000) -- (87.8000,61.9000) -- (87.8000,61.9000) -- (87.8000,61.9000) -- (87.8000,61.9000) -- (87.8000,61.8000) -- (87.8000,61.8000) -- (87.8000,61.8000) -- (87.8000,61.8000) -- (87.8000,61.8000) -- (87.8000,61.8000) -- (87.8000,61.8000) -- (87.8000,61.8000) -- (87.8000,61.8000) -- (87.8000,61.8000) -- (87.8000,61.8000) -- (87.8000,61.8000) -- (87.8000,61.8000) -- (87.8000,61.8000) -- (87.8000,61.8000) -- (87.8000,61.8000) -- (87.8000,61.8000) -- (87.8000,61.8000) -- (87.8000,61.8000) -- (87.8000,61.8000) -- (87.8000,61.8000) -- (87.8000,61.7000) -- (87.8000,61.7000) -- (87.8000,61.7000) -- (87.8000,61.7000) -- (87.8000,61.7000) -- (87.8000,61.7000) -- (87.8000,61.7000) -- (87.8000,61.7000) -- (87.8000,61.7000) -- (87.8000,61.7000) -- (87.8000,61.7000) -- (87.8000,61.7000) -- (87.8000,61.7000) -- (87.8000,61.7000) -- (87.9000,61.7000) -- (87.9000,61.7000) -- (87.9000,61.7000) -- (87.9000,61.7000) -- (87.9000,61.7000) -- (87.9000,61.7000) -- (87.9000,61.6000) -- (87.9000,61.6000) -- (87.9000,61.6000) -- (87.9000,61.6000) -- (87.9000,61.6000) -- (87.9000,61.6000) -- (87.9000,61.6000) -- (87.9000,61.6000) -- (87.9000,61.6000) -- (87.9000,61.6000) -- (87.9000,61.6000) -- (87.9000,61.6000) -- (87.9000,61.6000) -- (87.9000,61.6000) -- (87.9000,61.6000) -- (87.9000,61.6000) -- (87.9000,61.6000) -- (87.9000,61.6000) -- (87.9000,61.6000) -- (87.9000,61.6000) -- (87.9000,61.6000) -- (87.9000,61.5000) -- (87.9000,61.5000) -- (87.9000,61.5000) -- (87.9000,61.5000) -- (87.9000,61.5000) -- (87.9000,61.5000) -- (87.9000,61.5000) -- (87.9000,61.5000) -- (87.9000,61.5000) -- (87.9000,61.5000) -- (87.9000,61.5000) -- (87.9000,61.5000) -- (87.9000,61.5000) -- (87.9000,61.5000) -- (87.9000,61.5000) -- (87.9000,61.5000) -- (87.9000,61.5000) -- (87.9000,61.5000) -- (87.9000,61.5000) -- (87.9000,61.5000) -- (87.9000,61.5000) -- (87.9000,61.4000) -- (88.0000,61.4000) -- (88.0000,61.4000) -- (88.0000,61.4000) -- (88.0000,61.4000) -- (88.0000,61.4000) -- (88.0000,61.4000) -- (88.0000,61.4000) -- (88.0000,61.4000) -- (88.0000,61.4000) -- (88.0000,61.4000) -- (88.0000,61.4000) -- (88.0000,61.4000) -- (88.0000,61.4000) -- (88.0000,61.4000) -- (88.0000,61.4000) -- (88.0000,61.4000) -- (88.0000,61.4000) -- (88.0000,61.4000) -- (88.0000,61.4000) -- (88.0000,61.4000) -- (88.0000,61.3000) -- (88.0000,61.3000) -- (88.0000,61.3000) -- (88.0000,61.3000) -- (88.0000,61.3000) -- (88.0000,61.3000) -- (88.0000,61.3000) -- (88.0000,61.3000) -- (88.0000,61.3000) -- (88.0000,61.3000) -- (88.0000,61.3000) -- (88.0000,61.3000) -- (88.0000,61.3000) -- (88.0000,61.3000) -- (88.0000,61.3000) -- (88.0000,61.3000) -- (88.0000,61.3000) -- (88.0000,61.3000) -- (88.0000,61.3000) -- (88.0000,61.3000) -- (88.0000,61.3000) -- (88.0000,61.2000) -- (88.0000,61.2000) -- (88.0000,61.2000) -- (88.0000,61.2000) -- (88.0000,61.2000) -- (88.0000,61.2000) -- (88.0000,61.2000) -- (88.0000,61.2000) -- (88.0000,61.2000) -- (88.1000,61.2000) -- (88.1000,61.2000) -- (88.1000,61.2000) -- (88.1000,61.2000) -- (88.1000,61.2000) -- (88.1000,61.2000) -- (88.1000,61.2000) -- (88.1000,61.2000) -- (88.1000,61.2000) -- (88.1000,61.2000) -- (88.1000,61.2000) -- (88.1000,61.1000) -- (88.1000,61.1000) -- (88.1000,61.1000) -- (88.1000,61.1000) -- (88.1000,61.1000) -- (88.1000,61.1000) -- (88.1000,61.1000) -- (88.1000,61.1000) -- (88.1000,61.1000) -- (88.1000,61.1000) -- (88.1000,61.1000) -- (88.1000,61.1000) -- (88.1000,61.1000) -- (88.1000,61.1000) -- (88.1000,61.1000) -- (88.1000,61.1000) -- (88.1000,61.1000) -- (88.1000,61.1000) -- (88.1000,61.1000) -- (88.1000,61.1000) -- (88.1000,61.1000) -- (88.1000,61.0000) -- (88.1000,61.0000) -- (88.1000,61.0000) -- (88.1000,61.0000) -- (88.1000,61.0000) -- (88.1000,61.0000) -- (88.1000,61.0000) -- (88.1000,61.0000) -- (88.1000,61.0000) -- (88.1000,61.0000) -- (88.1000,61.0000) -- (88.1000,61.0000) -- (88.1000,61.0000) -- (88.1000,61.0000) -- (88.1000,61.0000) -- (88.1000,61.0000) -- (88.1000,61.0000) -- (88.2000,61.0000) -- (88.2000,61.0000) -- (88.2000,61.0000) -- (88.2000,61.0000) -- (88.2000,60.9000) -- (88.2000,60.9000) -- (88.2000,60.9000) -- (88.2000,60.9000) -- (88.2000,60.9000) -- (88.2000,60.9000) -- (88.2000,60.9000) -- (88.2000,60.9000) -- (88.2000,60.9000) -- (88.2000,60.9000) -- (88.2000,60.9000) -- (88.2000,60.9000) -- (88.2000,60.9000) -- (88.2000,60.9000) -- (88.2000,60.9000) -- (88.2000,60.9000) -- (88.2000,60.9000) -- (88.2000,60.9000) -- (88.2000,60.9000) -- (88.2000,60.9000) -- (88.2000,60.9000) -- (88.2000,60.8000) -- (88.2000,60.8000) -- (88.2000,60.8000) -- (88.2000,60.8000) -- (88.2000,60.8000) -- (88.2000,60.8000) -- (88.2000,60.8000) -- (88.2000,60.8000) -- (88.2000,60.8000) -- (88.2000,60.8000) -- (88.2000,60.8000) -- (88.2000,60.8000) -- (88.2000,60.8000) -- (88.2000,60.8000) -- (88.2000,60.8000) -- (88.2000,60.8000) -- (88.2000,60.8000) -- (88.2000,60.8000) -- (88.2000,60.8000) -- (88.2000,60.8000) -- (88.2000,60.8000) -- (88.2000,60.7000) -- (88.2000,60.7000) -- (88.2000,60.7000) -- (88.2000,60.7000) -- (88.3000,60.7000) -- (88.3000,60.7000) -- (88.3000,60.7000) -- (88.3000,60.7000) -- (88.3000,60.7000) -- (88.3000,60.7000) -- (88.3000,60.7000) -- (88.3000,60.7000) -- (88.3000,60.7000) -- (88.3000,60.7000) -- (88.3000,60.7000) -- (88.3000,60.7000) -- (88.3000,60.7000) -- (88.3000,60.7000) -- (88.3000,60.7000) -- (88.3000,60.7000) -- (88.3000,60.6000) -- (88.3000,60.6000) -- (88.3000,60.6000) -- (88.3000,60.6000) -- (88.3000,60.6000) -- (88.3000,60.6000) -- (88.3000,60.6000) -- (88.3000,60.6000) -- (88.3000,60.6000) -- (88.3000,60.6000) -- (88.3000,60.6000) -- (88.3000,60.6000) -- (88.3000,60.6000) -- (88.3000,60.6000) -- (88.3000,60.6000) -- (88.3000,60.6000) -- (88.3000,60.6000) -- (88.3000,60.6000) -- (88.3000,60.6000) -- (88.3000,60.6000) -- (88.3000,60.6000) -- (95.7000,60.5000) -- (95.7000,60.5000) -- (95.7000,60.5000) -- (95.7000,60.5000) -- (95.7000,60.5000) -- (95.7000,60.5000) -- (95.7000,60.5000) -- (95.7000,60.5000) -- (95.7000,60.5000) -- (95.7000,60.5000) -- (95.7000,60.5000) -- (95.7000,60.5000) -- (95.7000,60.5000) -- (95.7000,60.5000) -- (95.7000,60.5000) -- (95.7000,60.5000) -- (95.7000,60.5000) -- (95.7000,60.5000) -- (95.7000,60.5000) -- (95.7000,60.5000) -- (95.7000,60.5000) -- (95.7000,60.4000) -- (95.7000,60.4000) -- (95.7000,60.4000) -- (95.7000,60.4000) -- (95.7000,60.4000) -- (95.7000,60.4000) -- (95.7000,60.4000) -- (95.7000,60.4000) -- (95.7000,60.4000) -- (95.7000,60.4000) -- (95.7000,60.4000) -- (95.7000,60.4000) -- (95.7000,60.4000) -- (95.7000,60.4000) -- (95.7000,60.4000) -- (95.7000,60.4000) -- (95.7000,60.4000) -- (95.7000,60.4000) -- (95.7000,60.4000) -- (95.7000,60.4000) -- (95.7000,60.4000) -- (95.7000,60.3000) -- (95.7000,60.3000) -- (95.7000,60.3000) -- (95.7000,60.3000) -- (95.7000,60.3000) -- (95.7000,60.3000) -- (95.8000,60.3000) -- (95.8000,60.3000) -- (95.8000,60.3000) -- (95.8000,60.3000) -- (95.8000,60.3000) -- (95.8000,60.3000) -- (95.8000,60.3000) -- (95.8000,60.3000) -- (95.8000,60.3000) -- (95.8000,60.3000) -- (95.8000,60.3000) -- (95.8000,60.3000) -- (95.8000,60.3000) -- (95.8000,60.3000) -- (95.8000,60.2000) -- (95.8000,60.2000) -- (95.8000,60.2000) -- (95.8000,60.2000) -- (95.8000,60.2000) -- (95.8000,60.2000) -- (95.8000,60.2000) -- (95.8000,60.2000) -- (95.8000,60.2000) -- (95.8000,60.2000) -- (95.8000,60.2000) -- (95.8000,60.2000) -- (95.8000,60.2000) -- (95.8000,60.2000) -- (95.8000,60.2000) -- (95.8000,60.2000) -- (95.8000,60.2000) -- (95.8000,60.2000) -- (95.8000,60.2000) -- (95.8000,60.2000) -- (95.8000,60.2000) -- (95.8000,60.1000) -- (95.8000,60.1000) -- (95.8000,60.1000) -- (95.8000,60.1000) -- (95.8000,60.1000) -- (95.8000,60.1000) -- (95.8000,60.1000) -- (95.8000,60.1000) -- (95.8000,60.1000) -- (95.8000,60.1000) -- (95.8000,60.1000) -- (95.8000,60.1000) -- (95.8000,60.1000) -- (95.8000,60.1000) -- (95.8000,60.1000) -- (95.9000,60.1000) -- (95.9000,60.1000) -- (95.9000,60.1000) -- (95.9000,60.1000) -- (95.9000,60.1000) -- (95.9000,60.1000) -- (95.9000,60.0000) -- (95.9000,60.0000) -- (95.9000,60.0000) -- (95.9000,60.0000) -- (95.9000,60.0000) -- (95.9000,60.0000) -- (95.9000,60.0000) -- (95.9000,60.0000) -- (95.9000,60.0000) -- (95.9000,60.0000) -- (95.9000,60.0000) -- (95.9000,60.0000) -- (95.9000,60.0000) -- (95.9000,60.0000) -- (95.9000,60.0000) -- (95.9000,60.0000) -- (95.9000,60.0000) -- (95.9000,60.0000) -- (95.9000,60.0000) -- (95.9000,60.0000) -- (95.9000,60.0000) -- (95.9000,59.9000) -- (95.9000,59.9000) -- (95.9000,59.9000) -- (95.9000,59.9000) -- (95.9000,59.9000) -- (95.9000,59.9000) -- (95.9000,59.9000) -- (95.9000,59.9000) -- (95.9000,59.9000) -- (95.9000,59.9000) -- (95.9000,59.9000) -- (95.9000,59.9000) -- (95.9000,59.9000) -- (95.9000,59.9000) -- (95.9000,59.9000) -- (95.9000,59.9000) -- (95.9000,59.9000) -- (95.9000,59.9000) -- (95.9000,59.9000) -- (95.9000,59.9000) -- (95.9000,59.9000) -- (95.9000,59.8000) -- (96.0000,59.8000) -- (96.0000,59.8000) -- (96.0000,59.8000) -- (96.0000,59.8000) -- (96.0000,59.8000) -- (96.0000,59.8000) -- (96.0000,59.8000) -- (96.0000,59.8000) -- (96.0000,59.8000) -- (96.0000,59.8000) -- (96.0000,59.8000) -- (96.0000,59.8000) -- (96.0000,59.8000) -- (96.0000,59.8000) -- (96.0000,59.8000) -- (96.0000,59.8000) -- (96.0000,59.8000) -- (96.0000,59.8000) -- (96.0000,59.8000) -- (96.0000,59.7000) -- (96.0000,59.7000) -- (96.0000,59.7000) -- (96.0000,59.7000) -- (96.0000,59.7000) -- (96.0000,59.7000) -- (96.0000,59.7000) -- (96.0000,59.7000) -- (96.0000,59.7000) -- (96.0000,59.7000) -- (96.0000,59.7000) -- (96.0000,59.7000) -- (96.0000,59.7000) -- (96.0000,59.7000) -- (96.0000,59.7000) -- (96.0000,59.7000) -- (96.0000,59.7000) -- (96.0000,59.7000) -- (96.0000,59.7000) -- (96.0000,59.7000) -- (96.0000,59.7000) -- (96.0000,59.6000) -- (96.0000,59.6000) -- (96.0000,59.6000) -- (96.0000,59.6000) -- (96.0000,59.6000) -- (96.0000,59.6000) -- (96.0000,59.6000) -- (96.0000,59.6000) -- (96.0000,59.6000) -- (96.0000,59.6000) -- (96.1000,59.6000) -- (96.1000,59.6000) -- (96.1000,59.6000) -- (96.1000,59.6000) -- (96.1000,59.6000) -- (96.1000,59.6000) -- (96.1000,59.6000) -- (96.1000,59.6000) -- (96.1000,59.6000) -- (96.1000,59.6000) -- (96.1000,59.6000) -- (96.1000,59.5000) -- (96.1000,59.5000) -- (96.1000,59.5000) -- (96.1000,59.5000) -- (96.1000,59.5000) -- (96.1000,59.5000) -- (96.1000,59.5000) -- (96.1000,59.5000) -- (96.1000,59.5000) -- (96.1000,59.5000) -- (96.1000,59.5000) -- (96.1000,59.5000) -- (96.1000,59.5000) -- (96.1000,59.5000) -- (96.1000,59.5000) -- (96.1000,59.5000) -- (96.1000,59.5000) -- (96.1000,59.5000) -- (96.1000,59.5000) -- (96.1000,59.5000) -- (96.1000,59.5000) -- (96.1000,59.4000) -- (96.1000,59.4000) -- (96.1000,59.4000) -- (96.1000,59.4000) -- (96.1000,59.4000) -- (96.1000,59.4000) -- (96.1000,59.4000) -- (96.1000,59.4000) -- (96.1000,59.4000) -- (96.1000,59.4000) -- (96.1000,59.4000) -- (96.1000,59.4000) -- (96.1000,59.4000) -- (96.1000,59.4000) -- (96.1000,59.4000) -- (96.1000,59.4000) -- (96.1000,59.4000) -- (96.2000,59.4000) -- (96.2000,59.4000) -- (96.2000,59.4000) -- (96.2000,59.4000) -- (96.2000,59.3000) -- (96.2000,59.3000) -- (96.2000,59.3000) -- (96.2000,59.3000) -- (96.2000,59.3000) -- (96.2000,59.3000) -- (96.2000,59.3000) -- (96.2000,59.3000) -- (96.2000,59.3000) -- (96.2000,59.3000) -- (96.2000,59.3000) -- (96.2000,59.3000) -- (96.2000,59.3000) -- (96.2000,59.3000) -- (96.2000,59.3000) -- (96.2000,59.3000) -- (96.2000,59.3000) -- (96.2000,59.3000) -- (96.2000,59.3000) -- (96.2000,59.3000) -- (96.2000,59.2000) -- (96.2000,59.2000) -- (96.2000,59.2000) -- (96.2000,59.2000) -- (96.2000,59.2000) -- (96.2000,59.2000) -- (96.2000,59.2000) -- (96.2000,59.2000) -- (96.2000,59.2000) -- (96.2000,59.2000) -- (96.2000,59.2000) -- (96.2000,59.2000) -- (96.2000,59.2000) -- (96.2000,59.2000) -- (96.2000,59.2000) -- (96.2000,59.2000) -- (96.2000,59.2000) -- (96.2000,59.2000) -- (96.2000,59.2000) -- (96.2000,59.2000) -- (96.2000,59.2000) -- (96.2000,59.1000) -- (96.2000,59.1000) -- (96.2000,59.1000) -- (96.2000,59.1000) -- (96.2000,59.1000) -- (96.3000,59.1000) -- (96.3000,59.1000) -- (96.3000,59.1000) -- (96.3000,59.1000) -- (96.3000,59.1000) -- (96.3000,59.1000) -- (96.3000,59.1000) -- (96.3000,59.1000) -- (96.3000,59.1000) -- (96.3000,59.1000) -- (96.3000,59.1000) -- (96.3000,59.1000) -- (96.3000,59.1000) -- (96.3000,59.1000) -- (96.3000,59.1000) -- (96.3000,59.1000) -- (96.3000,59.0000) -- (96.3000,59.0000) -- (96.3000,59.0000) -- (96.3000,59.0000) -- (96.3000,59.0000) -- (96.3000,59.0000) -- (96.3000,59.0000) -- (96.3000,59.0000) -- (96.3000,59.0000) -- (96.3000,59.0000) -- (96.3000,59.0000) -- (96.3000,59.0000) -- (96.3000,59.0000) -- (96.3000,59.0000) -- (96.3000,59.0000) -- (96.3000,59.0000) -- (96.3000,59.0000) -- (96.3000,59.0000) -- (96.3000,59.0000) -- (96.3000,59.0000) -- (96.3000,59.0000) -- (96.3000,58.9000) -- (96.3000,58.9000) -- (96.3000,58.9000) -- (96.3000,58.9000) -- (96.3000,58.9000) -- (96.3000,58.9000) -- (96.3000,58.9000) -- (96.3000,58.9000) -- (96.3000,58.9000) -- (96.3000,58.9000) -- (96.3000,58.9000) -- (96.3000,58.9000) -- (96.3000,58.9000) -- (96.4000,58.9000) -- (96.4000,58.9000) -- (96.4000,58.9000) -- (96.4000,58.9000) -- (96.4000,58.9000) -- (96.4000,58.9000) -- (96.4000,58.9000) -- (96.4000,58.9000) -- (96.4000,58.8000) -- (96.4000,58.8000) -- (96.4000,58.8000) -- (96.4000,58.8000) -- (96.4000,58.8000) -- (96.4000,58.8000) -- (96.4000,58.8000) -- (96.4000,58.8000) -- (96.4000,58.8000) -- (96.4000,58.8000) -- (96.4000,58.8000) -- (96.4000,58.8000) -- (96.4000,58.8000) -- (96.4000,58.8000) -- (96.4000,58.8000) -- (96.4000,58.8000) -- (96.4000,58.8000) -- (96.4000,58.8000) -- (96.4000,58.8000) -- (96.4000,58.8000) -- (96.4000,58.7000) -- (96.4000,58.7000) -- (96.4000,58.7000) -- (96.4000,58.7000) -- (96.4000,58.7000) -- (96.4000,58.7000) -- (96.4000,58.7000) -- (96.4000,58.7000) -- (96.4000,58.7000) -- (96.4000,58.7000) -- (96.4000,58.7000) -- (96.4000,58.7000) -- (96.4000,58.7000) -- (96.4000,58.7000) -- (96.4000,58.7000) -- (96.4000,58.7000) -- (96.4000,58.7000) -- (96.4000,58.7000) -- (96.4000,58.7000) -- (96.4000,58.7000) -- (96.4000,58.7000) -- (96.5000,58.6000) -- (96.5000,58.6000) -- (96.5000,58.6000) -- (96.5000,58.6000) -- (96.5000,58.6000) -- (96.5000,58.6000) -- (96.5000,58.6000) -- (96.5000,58.6000) -- (96.5000,58.6000) -- (96.5000,58.6000) -- (96.5000,58.6000) -- (96.5000,58.6000) -- (96.5000,58.6000) -- (96.5000,58.6000) -- (96.5000,58.6000) -- (96.5000,58.6000) -- (96.5000,58.6000) -- (96.5000,58.6000) -- (96.5000,58.6000) -- (96.5000,58.6000) -- (96.5000,58.6000) -- (96.5000,58.5000) -- (96.5000,58.5000) -- (96.5000,58.5000) -- (96.5000,58.5000) -- (96.5000,58.5000) -- (96.5000,58.5000) -- (96.5000,58.5000) -- (96.5000,58.5000) -- (96.5000,58.5000) -- (96.5000,58.5000) -- (96.5000,58.5000) -- (96.5000,58.5000) -- (96.5000,58.5000) -- (96.5000,58.5000) -- (96.5000,58.5000) -- (96.5000,58.5000) -- (96.5000,58.5000) -- (96.5000,58.5000) -- (96.5000,58.5000) -- (96.5000,58.5000) -- (96.5000,58.5000) -- (96.5000,58.4000) -- (96.5000,58.4000) -- (96.5000,58.4000) -- (96.5000,58.4000) -- (96.5000,58.4000) -- (96.5000,58.4000) -- (96.5000,58.4000) -- (96.5000,58.4000) -- (96.6000,58.4000) -- (96.6000,58.4000) -- (96.6000,58.4000) -- (96.6000,58.4000) -- (96.6000,58.4000) -- (96.6000,58.4000) -- (96.6000,58.4000) -- (96.6000,58.4000) -- (96.6000,58.4000) -- (96.6000,58.4000) -- (96.6000,58.4000) -- (96.6000,58.4000) -- (96.6000,58.3000) -- (96.6000,58.3000) -- (96.6000,58.3000) -- (96.6000,58.3000) -- (96.6000,58.3000) -- (96.6000,58.3000) -- (96.6000,58.3000) -- (96.6000,58.3000) -- (96.6000,58.3000) -- (96.6000,58.3000) -- (96.6000,58.3000) -- (96.6000,58.3000) -- (96.6000,58.3000) -- (96.6000,58.3000) -- (96.6000,58.3000) -- (96.6000,58.3000) -- (96.6000,58.3000) -- (96.6000,58.3000) -- (96.6000,58.3000) -- (96.6000,58.3000) -- (96.6000,58.3000) -- (96.6000,58.2000) -- (96.6000,58.2000) -- (96.6000,58.2000) -- (96.6000,58.2000) -- (96.6000,58.2000) -- (96.6000,58.2000) -- (96.6000,58.2000) -- (96.6000,58.2000) -- (96.6000,58.2000) -- (96.6000,58.2000) -- (96.6000,58.2000) -- (96.6000,58.2000) -- (96.6000,58.2000) -- (96.6000,58.2000) -- (96.6000,58.2000) -- (96.6000,58.2000) -- (96.7000,58.2000) -- (96.7000,58.2000) -- (96.7000,58.2000) -- (96.7000,58.2000) -- (96.7000,58.2000) -- (96.7000,58.1000) -- (96.7000,58.1000) -- (96.7000,58.1000) -- (96.7000,58.1000) -- (96.7000,58.1000) -- (96.7000,58.1000) -- (96.7000,58.1000) -- (96.7000,58.1000) -- (96.7000,58.1000) -- (96.7000,58.1000) -- (96.7000,58.1000) -- (96.7000,58.1000) -- (96.7000,58.1000) -- (96.7000,58.1000) -- (96.7000,58.1000) -- (96.7000,58.1000) -- (96.7000,58.1000) -- (96.7000,58.1000) -- (96.7000,58.1000) -- (96.7000,58.1000) -- (96.7000,58.1000) -- (96.7000,58.0000) -- (96.7000,58.0000) -- (96.7000,58.0000) -- (96.7000,58.0000) -- (96.7000,58.0000) -- (96.7000,58.0000) -- (96.7000,58.0000) -- (96.7000,58.0000) -- (96.7000,58.0000) -- (96.7000,58.0000) -- (96.7000,58.0000) -- (96.7000,58.0000) -- (96.7000,58.0000) -- (96.7000,58.0000) -- (96.7000,58.0000) -- (96.7000,58.0000) -- (96.7000,58.0000) -- (96.7000,58.0000) -- (96.7000,58.0000) -- (96.7000,58.0000) -- (96.7000,58.0000) -- (96.7000,57.9000) -- (96.7000,57.9000) -- (96.7000,57.9000) -- (96.8000,57.9000) -- (96.8000,57.9000) -- (96.8000,57.9000) -- (96.8000,57.9000) -- (96.8000,57.9000) -- (96.8000,57.9000) -- (96.8000,57.9000) -- (96.8000,57.9000) -- (96.8000,57.9000) -- (96.8000,57.9000) -- (96.8000,57.9000) -- (96.8000,57.9000) -- (96.8000,57.9000) -- (96.8000,57.9000) -- (96.8000,57.9000) -- (96.8000,57.9000) -- (96.8000,57.9000) -- (96.8000,57.8000) -- (96.8000,57.8000) -- (96.8000,57.8000) -- (96.8000,57.8000) -- (96.8000,57.8000) -- (96.8000,57.8000) -- (96.8000,57.8000) -- (96.8000,57.8000) -- (96.8000,57.8000) -- (96.8000,57.8000) -- (96.8000,57.8000) -- (96.8000,57.8000) -- (96.8000,57.8000) -- (96.8000,57.8000) -- (96.8000,57.8000) -- (96.8000,57.8000) -- (96.8000,57.8000) -- (96.8000,57.8000) -- (96.8000,57.8000) -- (96.8000,57.8000) -- (96.8000,57.8000) -- (96.8000,57.7000) -- (96.8000,57.7000) -- (96.8000,57.7000) -- (96.8000,57.7000) -- (96.8000,57.7000) -- (96.8000,57.7000) -- (96.8000,57.7000) -- (96.8000,57.7000) -- (96.8000,57.7000) -- (96.8000,57.7000) -- (96.8000,57.7000) -- (96.9000,57.7000) -- (96.9000,57.7000) -- (96.9000,57.7000) -- (96.9000,57.7000) -- (96.9000,57.7000) -- (96.9000,57.7000) -- (96.9000,57.7000) -- (96.9000,57.7000) -- (96.9000,57.7000) -- (96.9000,57.7000) -- (96.9000,57.6000) -- (96.9000,57.6000) -- (96.9000,57.6000) -- (96.9000,57.6000) -- (96.9000,57.6000) -- (96.9000,57.6000) -- (96.9000,57.6000) -- (96.9000,57.6000) -- (96.9000,57.6000) -- (96.9000,57.6000) -- (96.9000,57.6000) -- (96.9000,57.6000) -- (96.9000,57.6000) -- (96.9000,57.6000) -- (96.9000,57.6000) -- (96.9000,57.6000) -- (96.9000,57.6000) -- (96.9000,57.6000) -- (96.9000,57.6000) -- (96.9000,57.6000) -- (96.9000,57.6000) -- (96.9000,57.5000) -- (96.9000,57.5000) -- (96.9000,57.5000) -- (96.9000,57.5000) -- (96.9000,57.5000) -- (96.9000,57.5000) -- (96.9000,57.5000) -- (96.9000,57.5000) -- (96.9000,57.5000) -- (96.9000,57.5000) -- (96.9000,57.5000) -- (96.9000,57.5000) -- (96.9000,57.5000) -- (96.9000,57.5000) -- (96.9000,57.5000) -- (96.9000,57.5000) -- (96.9000,57.5000) -- (96.9000,57.5000) -- (96.9000,57.5000) -- (97.0000,57.5000) -- (97.0000,57.5000) -- (97.0000,57.4000) -- (97.0000,57.4000) -- (97.0000,57.4000) -- (97.0000,57.4000) -- (97.0000,57.4000) -- (97.0000,57.4000) -- (97.0000,57.4000) -- (97.0000,57.4000) -- (97.0000,57.4000) -- (97.0000,57.4000) -- (97.0000,57.4000) -- (97.0000,57.4000) -- (97.0000,57.4000) -- (97.0000,57.4000) -- (97.0000,57.4000) -- (97.0000,57.4000) -- (97.0000,57.4000) -- (97.0000,57.4000) -- (97.0000,57.4000) -- (97.0000,57.4000) -- (97.0000,57.3000) -- (97.0000,57.3000) -- (97.0000,57.3000) -- (97.0000,57.3000) -- (97.0000,57.3000) -- (97.0000,57.3000) -- (97.0000,57.3000) -- (97.0000,57.3000) -- (97.0000,57.3000) -- (97.0000,57.3000) -- (97.0000,57.3000) -- (97.0000,57.3000) -- (97.0000,57.3000) -- (97.0000,57.3000) -- (97.0000,57.3000) -- (97.0000,57.3000) -- (97.0000,57.3000) -- (97.0000,57.3000) -- (97.0000,57.3000) -- (97.0000,57.3000) -- (97.0000,57.3000) -- (97.0000,57.2000) -- (97.0000,57.2000) -- (97.0000,57.2000) -- (97.0000,57.2000) -- (97.0000,57.2000) -- (97.0000,57.2000) -- (97.1000,57.2000) -- (97.1000,57.2000) -- (97.1000,57.2000) -- (97.1000,57.2000) -- (97.1000,57.2000) -- (97.1000,57.2000) -- (97.1000,57.2000) -- (97.1000,57.2000) -- (97.1000,57.2000) -- (97.1000,57.2000) -- (97.1000,57.2000) -- (97.1000,57.2000) -- (97.1000,57.2000) -- (97.1000,57.2000) -- (97.1000,57.2000) -- (97.1000,57.1000) -- (97.1000,57.1000) -- (97.1000,57.1000) -- (97.1000,57.1000) -- (97.1000,57.1000) -- (97.1000,57.1000) -- (97.1000,57.1000) -- (97.1000,57.1000) -- (97.1000,57.1000) -- (97.1000,57.1000) -- (97.1000,57.1000) -- (97.1000,57.1000) -- (97.1000,57.1000) -- (97.1000,57.1000) -- (97.1000,57.1000) -- (97.1000,57.1000) -- (97.1000,57.1000) -- (97.1000,57.1000) -- (97.1000,57.1000) -- (97.1000,57.1000) -- (97.1000,57.1000) -- (97.1000,57.0000) -- (97.1000,57.0000) -- (97.1000,57.0000) -- (97.1000,57.0000) -- (97.1000,57.0000) -- (97.1000,57.0000) -- (97.1000,57.0000) -- (97.1000,57.0000) -- (97.1000,57.0000) -- (97.1000,57.0000) -- (97.1000,57.0000) -- (97.1000,57.0000) -- (97.1000,57.0000) -- (97.1000,57.0000) -- (97.2000,57.0000) -- (97.2000,57.0000) -- (97.2000,57.0000) -- (97.2000,57.0000) -- (97.2000,57.0000) -- (97.2000,57.0000) -- (97.2000,56.9000) -- (97.2000,56.9000) -- (97.2000,56.9000) -- (97.2000,56.9000) -- (97.2000,56.9000) -- (97.2000,56.9000) -- (97.2000,56.9000) -- (97.2000,56.9000) -- (97.2000,56.9000) -- (97.2000,56.9000) -- (97.2000,56.9000) -- (97.2000,56.9000) -- (97.2000,56.9000) -- (97.2000,56.9000) -- (97.2000,56.9000) -- (97.2000,56.9000) -- (97.2000,56.9000) -- (97.2000,56.9000) -- (97.2000,56.9000) -- (97.2000,56.9000) -- (97.2000,56.9000) -- (97.2000,56.8000) -- (97.2000,56.8000) -- (97.2000,56.8000) -- (97.2000,56.8000) -- (97.2000,56.8000) -- (97.2000,56.8000) -- (97.2000,56.8000) -- (97.2000,56.8000) -- (97.2000,56.8000) -- (97.2000,56.8000) -- (97.2000,56.8000) -- (97.2000,56.8000) -- (97.2000,56.8000) -- (97.2000,56.8000) -- (97.2000,56.8000) -- (97.2000,56.8000) -- (97.2000,56.8000) -- (97.2000,56.8000) -- (97.2000,56.8000) -- (97.2000,56.8000) -- (97.2000,56.8000) -- (97.2000,56.7000) -- (97.3000,56.7000) -- (97.3000,56.7000) -- (97.3000,56.7000) -- (97.3000,56.7000) -- (97.3000,56.7000) -- (97.3000,56.7000) -- (97.3000,56.7000) -- (97.3000,56.7000) -- (97.3000,56.7000) -- (97.3000,56.7000) -- (97.3000,56.7000) -- (97.3000,56.7000) -- (97.3000,56.7000) -- (97.3000,56.7000) -- (97.3000,56.7000) -- (97.3000,56.7000) -- (97.3000,56.7000) -- (97.3000,56.7000) -- (97.3000,56.7000) -- (97.3000,56.7000) -- (97.3000,56.6000) -- (97.3000,56.6000) -- (97.3000,56.6000) -- (97.3000,56.6000) -- (97.3000,56.6000) -- (97.3000,56.6000) -- (97.3000,56.6000) -- (97.3000,56.6000) -- (97.3000,56.6000) -- (97.3000,56.6000) -- (97.3000,56.6000) -- (97.3000,56.6000) -- (97.3000,56.6000) -- (97.3000,56.6000) -- (97.3000,56.6000) -- (97.3000,56.6000) -- (97.3000,56.6000) -- (97.3000,56.6000) -- (97.3000,56.6000) -- (97.3000,56.6000) -- (97.3000,56.6000) -- (97.3000,56.5000) -- (97.3000,56.5000) -- (97.3000,56.5000) -- (97.3000,56.5000) -- (97.3000,56.5000) -- (97.3000,56.5000) -- (97.3000,56.5000) -- (97.3000,56.5000) -- (97.3000,56.5000) -- (97.4000,56.5000) -- (97.4000,56.5000) -- (97.4000,56.5000) -- (97.4000,56.5000) -- (97.4000,56.5000) -- (97.4000,56.5000) -- (97.4000,56.5000) -- (97.4000,56.5000) -- (97.4000,56.5000) -- (97.4000,56.5000) -- (97.4000,56.5000) -- (97.4000,56.4000) -- (97.4000,56.4000) -- (97.4000,56.4000) -- (97.4000,56.4000) -- (97.4000,56.4000) -- (97.4000,56.4000) -- (97.4000,56.4000) -- (97.4000,56.4000) -- (97.4000,56.4000) -- (97.4000,56.4000) -- (97.4000,56.4000) -- (97.4000,56.4000) -- (97.4000,56.4000) -- (97.4000,56.4000) -- (97.4000,56.4000) -- (97.4000,56.4000) -- (97.4000,56.4000) -- (97.4000,56.4000) -- (97.4000,56.4000) -- (97.4000,56.4000) -- (97.4000,56.4000) -- (97.4000,56.3000) -- (97.4000,56.3000) -- (97.4000,56.3000) -- (97.4000,56.3000) -- (97.4000,56.3000) -- (97.4000,56.3000) -- (97.4000,56.3000) -- (97.4000,56.3000) -- (97.4000,56.3000) -- (97.4000,56.3000) -- (97.4000,56.3000) -- (97.4000,56.3000) -- (97.4000,56.3000) -- (97.4000,56.3000) -- (97.4000,56.3000) -- (97.4000,56.3000) -- (97.4000,56.3000) -- (97.5000,56.3000) -- (97.5000,56.3000) -- (97.5000,56.3000) -- (97.5000,56.3000) -- (97.5000,56.2000) -- (97.5000,56.2000) -- (97.5000,56.2000) -- (97.5000,56.2000) -- (97.5000,56.2000) -- (97.5000,56.2000) -- (97.5000,56.2000) -- (97.5000,56.2000) -- (97.5000,56.2000) -- (97.5000,56.2000) -- (97.5000,56.2000) -- (97.5000,56.2000) -- (97.5000,56.2000) -- (97.5000,56.2000) -- (97.5000,56.2000) -- (97.5000,56.2000) -- (97.5000,56.2000) -- (97.5000,56.2000) -- (97.5000,56.2000) -- (97.5000,56.2000) -- (97.5000,56.2000) -- (97.5000,56.1000) -- (97.5000,56.1000) -- (97.5000,56.1000) -- (97.5000,56.1000) -- (97.5000,56.1000) -- (97.5000,56.1000) -- (97.5000,56.1000) -- (97.5000,56.1000) -- (97.5000,56.1000) -- (97.5000,56.1000) -- (97.5000,56.1000) -- (97.5000,56.1000) -- (97.5000,56.1000) -- (97.5000,56.1000) -- (97.5000,56.1000) -- (97.5000,56.1000) -- (97.5000,56.1000) -- (97.5000,56.1000) -- (97.5000,56.1000) -- (97.5000,56.1000) -- (97.5000,56.1000) -- (97.5000,56.0000) -- (97.5000,56.0000) -- (97.5000,56.0000) -- (97.5000,56.0000) -- (97.6000,56.0000) -- (97.6000,56.0000) -- (97.6000,56.0000) -- (97.6000,56.0000) -- (97.6000,56.0000) -- (97.6000,56.0000) -- (97.6000,56.0000) -- (97.6000,56.0000) -- (97.6000,56.0000) -- (97.6000,56.0000) -- (97.6000,56.0000) -- (97.6000,56.0000) -- (97.6000,56.0000) -- (97.6000,56.0000) -- (97.6000,56.0000) -- (97.6000,56.0000) -- (97.6000,55.9000) -- (97.6000,55.9000) -- (97.6000,55.9000) -- (97.6000,55.9000) -- (97.6000,55.9000) -- (97.6000,55.9000) -- (97.6000,55.9000) -- (97.6000,55.9000) -- (97.6000,55.9000) -- (97.6000,55.9000) -- (97.6000,55.9000) -- (97.6000,55.9000) -- (97.6000,55.9000) -- (97.6000,55.9000) -- (97.6000,55.9000) -- (97.6000,55.9000) -- (97.6000,55.9000) -- (97.6000,55.9000) -- (97.6000,55.9000) -- (97.6000,55.9000) -- (97.6000,55.9000) -- (97.6000,55.8000) -- (97.6000,55.8000) -- (97.6000,55.8000) -- (97.6000,55.8000) -- (97.6000,55.8000) -- (97.6000,55.8000) -- (97.6000,55.8000) -- (97.6000,55.8000) -- (97.6000,55.8000) -- (97.6000,55.8000) -- (97.6000,55.8000) -- (97.6000,55.8000) -- (97.7000,55.8000) -- (97.7000,55.8000) -- (97.7000,55.8000) -- (97.7000,55.8000) -- (97.7000,55.8000) -- (97.7000,55.8000) -- (97.7000,55.8000) -- (97.7000,55.8000) -- (97.7000,55.8000) -- (97.7000,55.7000) -- (97.7000,55.7000) -- (97.7000,55.7000) -- (97.7000,55.7000) -- (97.7000,55.7000) -- (97.7000,55.7000) -- (97.7000,55.7000) -- (121.4000,55.9000);



      \end{scope}
      \begin{scope}[cm={{1.27491,0.0,0.0,1.41542,(-124.58597,-493.0531)}},draw=blue,line cap=round,line join=round,line width=0.480pt]
        \path[draw] (81.5000,50.5000) -- (81.5000,78.5000) -- (121.5000,78.5000) -- (121.5000,50.5000) -- (81.5000,50.5000);



      \end{scope}
      \begin{scope}[cm={{1.05023,0.0,0.0,1.05023,(-51.68473,-410.86372)}},draw=blue,line cap=rect,line join=bevel,line width=0.800pt]
        \path[fill=blue] (0.0000,0.0000) node[above right] (text64-7-2) {\scriptsize $T$\hspace{.5ex}=\hspace{.5ex}46};



      \end{scope}
      \begin{scope}[cm={{1.05023,0.0,0.0,1.05023,(-3.79554,-425.35157)}},draw=blue,line cap=rect,line join=bevel,line width=0.800pt]
        \path[fill=blue] (0.0000,0.0000) node[above right] (text64-7-8) {\scriptsize 83};



      \end{scope}
    \end{scope}
    \begin{scope}[cm={{0.74279,0.0,0.0,1.28515,(-186.22138,-161.30028)}},draw=blue,line cap=round,line join=round,line width=0.480pt]
      \path[cm={{1.0,0.0,0.0,0.57583,(0.0,64.75889)}},draw] (130.5000,152.5000) -- (130.5000,147.5000);



      \path[cm={{1.0,0.0,0.0,0.70101,(-0.0,38.33083)}},draw] (130.5000,128.5000) -- (130.5000,132.5000);



    \end{scope}
    \begin{scope}[cm={{0.74279,0.0,0.0,1.28515,(-186.22138,-161.30028)}},draw=blue,line cap=round,line join=round,line width=0.480pt]
      \path[draw] (25.5000,128.5000) -- (25.5000,152.5000) -- (142.5000,152.5000) -- (142.5000,128.5000) -- (25.5000,128.5000);



    \end{scope}
    \begin{scope}[cm={{0.95389,0.0,0.0,0.95389,(-108.68182,28.37633)}},draw=blue,line cap=rect,line join=bevel,line width=0.800pt]
      \path[fill=blue] (3.8674,-1.6114) node[above right] (text1264) {\scriptsize $\alpha_3$};



    \end{scope}
    \begin{scope}[cm={{0.74279,0.0,0.0,1.28515,(-184.07401,-166.61499)}},draw=blue,line cap=round,line join=round,line width=0.480pt]
      \path[draw,even odd rule] (123.5000,148.5000) -- (132.5000,148.5000);



    \end{scope}
    \begin{scope}[cm={{0.74279,0.0,0.0,1.28515,(-186.22138,-161.30028)}},draw=blue,line cap=round,line join=round,line width=0.480pt]
      \path[draw] (25.8000,136.6000) -- (25.8000,136.6000) -- (26.2000,137.3000) -- (26.7000,137.7000) -- (27.1000,136.6000) -- (27.5000,137.5000) -- (27.9000,137.0000) -- (28.3000,139.3000) -- (28.8000,132.8000) -- (29.2000,140.7000) -- (29.6000,139.7000) -- (30.0000,133.1000) -- (30.5000,135.4000) -- (30.9000,140.8000) -- (31.3000,140.7000) -- (31.7000,136.2000) -- (32.2000,133.3000) -- (32.6000,134.6000) -- (33.0000,138.2000) -- (33.4000,140.7000) -- (33.8000,140.6000) -- (34.3000,138.4000) -- (34.7000,135.8000) -- (35.1000,134.3000) -- (35.5000,134.4000) -- (36.0000,136.0000) -- (36.4000,138.0000) -- (36.8000,139.6000) -- (37.2000,140.3000) -- (37.6000,140.0000) -- (38.1000,138.8000) -- (38.5000,137.3000) -- (38.9000,135.9000) -- (39.3000,135.0000) -- (39.8000,134.7000) -- (40.2000,135.0000) -- (40.6000,135.8000) -- (41.0000,136.8000) -- (41.5000,137.9000) -- (41.9000,138.8000) -- (42.3000,139.4000) -- (42.7000,139.6000) -- (43.1000,139.4000) -- (43.6000,139.0000) -- (44.0000,138.3000) -- (44.4000,137.4000) -- (44.8000,136.6000) -- (45.3000,136.0000) -- (45.7000,135.5000) -- (46.1000,135.2000) -- (46.5000,135.2000) -- (47.0000,135.5000) -- (47.4000,135.9000) -- (47.8000,136.4000) -- (48.2000,137.0000) -- (48.6000,137.6000) -- (49.1000,138.1000) -- (49.5000,138.5000) -- (49.9000,138.8000) -- (50.3000,138.9000) -- (50.8000,138.8000) -- (51.2000,138.5000) -- (51.6000,138.1000) -- (52.0000,137.6000) -- (52.4000,137.1000) -- (52.9000,136.5000) -- (53.3000,136.1000) -- (53.7000,135.7000) -- (54.1000,135.6000) -- (54.6000,135.6000) -- (55.0000,135.8000) -- (55.4000,136.1000) -- (55.8000,136.6000) -- (56.3000,137.1000) -- (56.7000,137.6000) -- (57.1000,138.0000) -- (57.5000,138.3000) -- (57.9000,138.6000) -- (58.4000,138.7000) -- (58.8000,138.6000) -- (59.2000,138.4000) -- (59.6000,138.1000) -- (60.1000,137.8000) -- (60.5000,137.3000) -- (60.9000,136.9000) -- (61.3000,136.6000) -- (61.8000,136.4000) -- (62.2000,136.3000) -- (62.6000,136.4000) -- (63.0000,136.5000) -- (63.4000,136.7000) -- (63.9000,137.0000) -- (64.3000,137.4000) -- (64.7000,137.7000) -- (65.1000,138.0000) -- (65.6000,138.2000) -- (66.0000,138.3000) -- (66.4000,138.2000) -- (66.8000,138.1000) -- (67.3000,137.9000) -- (67.7000,137.7000) -- (68.1000,137.4000) -- (68.5000,137.1000) -- (68.9000,136.8000) -- (69.4000,136.6000) -- (69.8000,136.5000) -- (70.2000,136.5000) -- (70.6000,136.6000) -- (71.1000,136.7000) -- (71.5000,136.9000) -- (71.9000,137.0000) -- (72.3000,137.2000) -- (72.7000,137.4000) -- (73.2000,137.6000) -- (73.6000,137.8000) -- (74.0000,137.9000) -- (74.4000,138.0000) -- (74.9000,138.0000) -- (75.3000,137.9000) -- (75.7000,137.8000) -- (76.1000,137.7000) -- (76.6000,137.5000) -- (77.0000,137.3000) -- (77.4000,137.2000) -- (77.8000,137.1000) -- (78.2000,136.9000) -- (78.7000,136.8000) -- (79.1000,136.8000) -- (79.5000,136.8000) -- (79.9000,136.8000) -- (80.4000,136.8000) -- (80.8000,136.9000) -- (81.2000,137.0000) -- (81.6000,137.1000) -- (82.1000,137.3000) -- (82.5000,137.4000) -- (82.9000,137.5000) -- (83.3000,137.6000) -- (83.7000,137.6000) -- (84.2000,137.6000) -- (84.6000,137.6000) -- (85.0000,137.6000) -- (85.4000,137.5000) -- (85.9000,137.4000) -- (86.3000,137.3000) -- (86.7000,137.2000) -- (87.1000,137.1000) -- (87.6000,137.0000) -- (88.0000,136.9000) -- (88.4000,136.9000) -- (88.8000,137.0000) -- (89.2000,137.0000) -- (89.7000,137.1000) -- (90.1000,137.2000) -- (90.5000,137.3000) -- (90.9000,137.4000) -- (91.4000,137.5000) -- (91.8000,137.5000) -- (92.2000,137.6000) -- (92.6000,137.6000) -- (93.0000,137.6000) -- (93.5000,137.5000) -- (93.9000,137.5000) -- (94.3000,137.4000) -- (94.7000,137.3000) -- (95.2000,137.2000) -- (95.6000,137.1000) -- (96.0000,137.1000) -- (96.4000,137.0000) -- (96.9000,137.1000) -- (97.3000,137.1000) -- (97.7000,137.1000) -- (98.1000,137.2000) -- (98.5000,137.3000) -- (99.0000,137.3000) -- (99.4000,137.4000) -- (99.8000,137.4000) -- (100.2000,137.5000) -- (100.7000,137.5000) -- (101.1000,137.6000) -- (101.5000,137.5000) -- (101.9000,137.4000) -- (102.3000,137.3000) -- (102.8000,137.3000) -- (103.2000,137.2000) -- (103.6000,137.1000) -- (104.0000,137.1000) -- (104.5000,137.0000) -- (104.9000,137.0000) -- (105.3000,137.0000) -- (105.7000,137.1000) -- (106.2000,137.1000) -- (106.6000,137.2000) -- (107.0000,137.3000) -- (107.4000,137.3000) -- (107.8000,137.4000) -- (108.3000,137.5000) -- (108.7000,137.6000) -- (109.1000,137.6000) -- (109.5000,137.7000) -- (110.0000,137.6000) -- (110.4000,137.6000) -- (110.8000,137.5000) -- (111.2000,137.4000) -- (111.7000,137.3000) -- (112.1000,137.2000) -- (112.5000,137.1000) -- (112.9000,137.0000) -- (113.3000,137.0000) -- (113.8000,136.9000) -- (114.2000,136.9000) -- (114.6000,136.9000) -- (115.0000,136.9000) -- (115.5000,137.0000) -- (115.9000,137.2000) -- (116.3000,137.3000) -- (116.7000,137.4000) -- (117.2000,137.5000) -- (117.6000,137.6000) -- (118.0000,137.7000) -- (118.4000,137.7000) -- (118.8000,137.7000) -- (119.3000,137.7000) -- (119.7000,137.6000) -- (120.1000,137.5000) -- (120.5000,137.4000) -- (121.0000,137.2000) -- (121.4000,137.1000) -- (121.8000,137.0000) -- (122.2000,136.8000) -- (122.6000,136.8000) -- (123.1000,136.8000) -- (123.5000,136.8000) -- (123.9000,136.9000) -- (124.3000,137.0000) -- (124.8000,137.2000) -- (125.2000,137.3000) -- (125.6000,137.4000) -- (126.0000,137.6000) -- (126.5000,137.7000) -- (126.9000,137.7000) -- (127.3000,137.8000) -- (127.7000,137.8000) -- (128.1000,137.8000) -- (128.6000,137.7000) -- (129.0000,137.5000) -- (129.4000,137.4000) -- (129.8000,137.3000) -- (130.3000,137.1000) -- (130.7000,137.0000) -- (131.1000,136.9000) -- (131.5000,136.9000) -- (131.9000,136.8000) -- (132.4000,136.8000) -- (132.8000,136.9000) -- (133.2000,137.0000) -- (133.6000,137.1000) -- (134.1000,137.2000) -- (134.5000,137.3000) -- (134.9000,137.4000) -- (135.3000,137.6000) -- (135.8000,137.7000) -- (136.2000,137.8000) -- (136.6000,137.8000) -- (137.0000,137.8000) -- (137.4000,137.7000) -- (137.9000,137.6000) -- (138.3000,137.5000) -- (138.7000,137.4000) -- (139.1000,137.3000) -- (139.6000,137.1000) -- (140.0000,137.0000) -- (140.4000,136.9000) -- (140.8000,136.8000) -- (141.3000,136.8000) -- (141.7000,136.8000) -- (142.1000,136.9000) -- (142.3000,136.9000);



    \end{scope}
    \begin{scope}[cm={{0.7438,0.0,0.0,0.77563,(-64.025,-157.04212)}},draw=blue,line cap=round,line join=round,line width=0.480pt]
    \end{scope}
    \begin{scope}[cm={{0.74279,0.0,0.0,1.28515,(-186.22138,-161.30028)}},draw=ca0a0a4,dash pattern=on 1.22pt off 1.22pt,line cap=round,line join=round,line width=0.305pt,miter limit=4.00]
      \path[draw,dash pattern=on 1.22pt off 1.22pt,line width=0.305pt,miter limit=4.00] (25.5000,168.5000) -- (108.5000,168.5000);



      \path[draw,dash pattern=on 1.22pt off 1.22pt,line width=0.305pt,miter limit=4.00] (137.5000,168.5000) -- (142.5000,168.5000);



    \end{scope}
    \begin{scope}[cm={{0.74279,0.0,0.0,1.28515,(-186.22138,-161.30028)}},draw=blue,line cap=round,line join=round,line width=0.480pt]
      \path[cm={{1.54975,0.0,0.0,1.0,(-13.85377,0.0)}},draw] (25.5000,168.5000) -- (28.5000,168.5000);



      \path[cm={{1.54975,0.0,0.0,1.0,(-78.53317,0.0)}},draw] (142.5000,168.5000) -- (139.5000,168.5000);



    \end{scope}
    \begin{scope}[cm={{0.74279,0.0,0.0,1.28515,(-186.22138,-161.30028)}},draw=ca0a0a4,dash pattern=on 1.22pt off 1.22pt,line cap=round,line join=round,line width=0.305pt,miter limit=4.00]
      \path[draw,dash pattern=on 1.22pt off 1.22pt,line width=0.305pt,miter limit=4.00] (25.5000,153.5000) -- (142.5000,153.5000);



    \end{scope}
    \begin{scope}[cm={{0.74279,0.0,0.0,1.28515,(-186.22138,-161.30028)}},draw=blue,line cap=round,line join=round,line width=0.480pt]
      \path[cm={{1.54975,0.0,0.0,1.0,(-13.85377,0.0)}},draw] (25.5000,153.5000) -- (28.5000,153.5000);



      \path[cm={{1.54975,0.0,0.0,1.0,(-78.53317,0.0)}},draw] (142.5000,153.5000) -- (139.5000,153.5000);



    \end{scope}
    \begin{scope}[cm={{0.95389,0.0,0.0,0.95389,(-180.91222,39.27539)}},draw=blue,fill=ce10000,line cap=rect,line join=bevel,line width=0.800pt]
      \path[fill=ce10000] (0.0000,0.0000) node[above right] (text1350) {\scriptsize 20};



    \end{scope}
    \begin{scope}[cm={{0.74279,0.0,0.0,1.28515,(-186.22138,-161.30028)}},draw=ca0a0a4,dash pattern=on 0.40pt off 0.80pt,line cap=round,line join=round,line width=0.400pt]
      \path[draw] (25.5000,176.5000) -- (25.5000,152.5000);



    \end{scope}
    \begin{scope}[cm={{0.74279,0.0,0.0,1.28515,(-186.22138,-161.30028)}},draw=blue,line cap=round,line join=round,line width=0.480pt]
      \path[draw] (25.5000,176.5000) -- (25.5000,171.5000);



      \path[draw] (25.5000,152.5000) -- (25.5000,156.5000);



    \end{scope}
    \begin{scope}[cm={{0.74279,0.0,0.0,1.28515,(-186.22138,-161.30028)}},draw=ca0a0a4,dash pattern=on 1.22pt off 1.22pt,line cap=round,line join=round,line width=0.305pt,miter limit=4.00]
      \path[draw,dash pattern=on 1.22pt off 1.22pt,line width=0.305pt,miter limit=4.00] (60.5000,176.5000) -- (60.5000,152.5000);



    \end{scope}
    \begin{scope}[cm={{0.74279,0.0,0.0,1.28515,(-186.22138,-161.30028)}},draw=blue,line cap=round,line join=round,line width=0.480pt]
      \path[cm={{1.0,0.0,0.0,0.57583,(0.0,74.93902)}},draw] (60.5000,176.5000) -- (60.5000,171.5000);



      \path[cm={{1.0,0.0,0.0,0.70101,(0.0,45.50665)}},draw] (60.5000,152.5000) -- (60.5000,156.5000);



    \end{scope}
    \begin{scope}[cm={{0.74279,0.0,0.0,1.28515,(-186.22138,-161.30028)}},draw=ca0a0a4,dash pattern=on 1.22pt off 1.22pt,line cap=round,line join=round,line width=0.305pt,miter limit=4.00]
      \path[draw,dash pattern=on 1.22pt off 1.22pt,line width=0.305pt,miter limit=4.00] (95.5000,176.5000) -- (95.5000,152.5000);



    \end{scope}
    \begin{scope}[cm={{0.74279,0.0,0.0,1.28515,(-186.22138,-161.30028)}},draw=blue,line cap=round,line join=round,line width=0.480pt]
      \path[cm={{1.0,0.0,0.0,0.57583,(0.0,74.93902)}},draw] (95.5000,176.5000) -- (95.5000,171.5000);



      \path[cm={{1.0,0.0,0.0,0.70101,(0.0,45.50665)}},draw] (95.5000,152.5000) -- (95.5000,156.5000);



    \end{scope}
    \begin{scope}[cm={{0.74279,0.0,0.0,1.28515,(-186.22138,-161.30028)}},draw=ca0a0a4,dash pattern=on 1.22pt off 1.22pt,line cap=round,line join=round,line width=0.305pt,miter limit=4.00]
      \path[draw,dash pattern=on 1.22pt off 1.22pt,line width=0.305pt,miter limit=4.00] (130.5000,164.5000) -- (130.5000,152.5000);



    \end{scope}
    \begin{scope}[cm={{0.74279,0.0,0.0,1.28515,(-186.22138,-161.30028)}},draw=blue,line cap=round,line join=round,line width=0.480pt]
      \path[cm={{1.0,0.0,0.0,0.57583,(0.0,74.93902)}},draw] (130.5000,176.5000) -- (130.5000,171.5000);



      \path[cm={{1.0,0.0,0.0,0.70101,(0.0,45.50665)}},draw] (130.5000,152.5000) -- (130.5000,156.5000);



    \end{scope}
    \begin{scope}[cm={{0.74279,0.0,0.0,1.28515,(-186.22138,-161.30028)}},draw=blue,line cap=round,line join=round,line width=0.480pt]
      \path[draw] (25.5000,152.5000) -- (25.5000,176.5000) -- (142.5000,176.5000) -- (142.5000,152.5000) -- (25.5000,152.5000);



    \end{scope}
    \begin{scope}[cm={{0.95389,0.0,0.0,0.95389,(-104.73329,57.56154)}},draw=blue,line cap=rect,line join=bevel,line width=0.800pt]
      \path[fill=blue] (0.0000,0.0000) node[above right] (text1494) {\scriptsize $\beta_3$};



    \end{scope}
    \begin{scope}[cm={{0.74279,0.0,0.0,1.28515,(-184.07401,-166.61499)}},draw=blue,line cap=round,line join=round,line width=0.480pt]
      \path[draw,even odd rule] (123.5000,172.5000) -- (132.5000,172.5000);



    \end{scope}
    \begin{scope}[cm={{0.74279,0.0,0.0,1.28515,(-186.22138,-161.30028)}},draw=blue,line cap=round,line join=round,line width=0.480pt]
      \path[draw] (25.8000,162.1000) -- (25.8000,162.1000) -- (26.2000,161.2000) -- (26.7000,161.1000) -- (27.1000,162.2000) -- (27.5000,159.4000) -- (27.9000,164.7000) -- (28.3000,157.2000) -- (28.8000,163.3000) -- (29.2000,164.4000) -- (29.6000,156.6000) -- (30.0000,159.9000) -- (30.5000,165.6000) -- (30.9000,163.5000) -- (31.3000,158.2000) -- (31.7000,157.3000) -- (32.2000,161.0000) -- (32.6000,164.7000) -- (33.0000,165.0000) -- (33.4000,162.4000) -- (33.8000,159.2000) -- (34.3000,157.7000) -- (34.7000,158.4000) -- (35.1000,160.6000) -- (35.5000,162.9000) -- (36.0000,164.3000) -- (36.4000,164.2000) -- (36.8000,163.0000) -- (37.2000,161.2000) -- (37.6000,159.5000) -- (38.1000,158.5000) -- (38.5000,158.3000) -- (38.9000,159.0000) -- (39.3000,160.2000) -- (39.8000,161.6000) -- (40.2000,162.8000) -- (40.6000,163.6000) -- (41.0000,163.9000) -- (41.5000,163.6000) -- (41.9000,163.0000) -- (42.3000,162.1000) -- (42.7000,161.1000) -- (43.1000,160.2000) -- (43.6000,159.5000) -- (44.0000,159.1000) -- (44.4000,159.0000) -- (44.8000,159.2000) -- (45.3000,159.7000) -- (45.7000,160.4000) -- (46.1000,161.1000) -- (46.5000,161.8000) -- (47.0000,162.5000) -- (47.4000,163.0000) -- (47.8000,163.2000) -- (48.2000,163.3000) -- (48.6000,163.1000) -- (49.1000,162.8000) -- (49.5000,162.4000) -- (49.9000,161.8000) -- (50.3000,161.2000) -- (50.8000,160.7000) -- (51.2000,160.2000) -- (51.6000,159.9000) -- (52.0000,159.7000) -- (52.4000,159.7000) -- (52.9000,159.9000) -- (53.3000,160.3000) -- (53.7000,160.8000) -- (54.1000,161.3000) -- (54.6000,161.9000) -- (55.0000,162.4000) -- (55.4000,162.7000) -- (55.8000,163.0000) -- (56.3000,163.0000) -- (56.7000,162.8000) -- (57.1000,162.6000) -- (57.5000,162.2000) -- (57.9000,161.7000) -- (58.4000,161.2000) -- (58.8000,160.8000) -- (59.2000,160.4000) -- (59.6000,160.1000) -- (60.1000,160.0000) -- (60.5000,160.0000) -- (60.9000,160.1000) -- (61.3000,160.4000) -- (61.8000,160.7000) -- (62.2000,161.1000) -- (62.6000,161.5000) -- (63.0000,161.8000) -- (63.4000,162.1000) -- (63.9000,162.2000) -- (64.3000,162.2000) -- (64.7000,162.1000) -- (65.1000,161.9000) -- (65.6000,161.6000) -- (66.0000,161.3000) -- (66.4000,161.0000) -- (66.8000,160.7000) -- (67.3000,160.5000) -- (67.7000,160.4000) -- (68.1000,160.3000) -- (68.5000,160.4000) -- (68.9000,160.6000) -- (69.4000,160.8000) -- (69.8000,161.1000) -- (70.2000,161.4000) -- (70.6000,161.6000) -- (71.1000,161.8000) -- (71.5000,162.0000) -- (71.9000,162.0000) -- (72.3000,162.0000) -- (72.7000,162.0000) -- (73.2000,161.9000) -- (73.6000,161.8000) -- (74.0000,161.6000) -- (74.4000,161.5000) -- (74.9000,161.2000) -- (75.3000,161.0000) -- (75.7000,160.8000) -- (76.1000,160.7000) -- (76.6000,160.7000) -- (77.0000,160.7000) -- (77.4000,160.7000) -- (77.8000,160.8000) -- (78.2000,160.9000) -- (78.7000,161.0000) -- (79.1000,161.2000) -- (79.5000,161.3000) -- (79.9000,161.5000) -- (80.4000,161.6000) -- (80.8000,161.7000) -- (81.2000,161.8000) -- (81.6000,161.8000) -- (82.1000,161.8000) -- (82.5000,161.8000) -- (82.9000,161.7000) -- (83.3000,161.6000) -- (83.7000,161.5000) -- (84.2000,161.3000) -- (84.6000,161.2000) -- (85.0000,161.1000) -- (85.4000,161.0000) -- (85.9000,161.0000) -- (86.3000,161.0000) -- (86.7000,161.0000) -- (87.1000,161.0000) -- (87.6000,161.1000) -- (88.0000,161.1000) -- (88.4000,161.3000) -- (88.8000,161.4000) -- (89.2000,161.5000) -- (89.7000,161.6000) -- (90.1000,161.6000) -- (90.5000,161.6000) -- (90.9000,161.6000) -- (91.4000,161.6000) -- (91.8000,161.5000) -- (92.2000,161.4000) -- (92.6000,161.3000) -- (93.0000,161.2000) -- (93.5000,161.1000) -- (93.9000,161.1000) -- (94.3000,161.0000) -- (94.7000,161.0000) -- (95.2000,161.0000) -- (95.6000,161.0000) -- (96.0000,161.1000) -- (96.4000,161.2000) -- (96.9000,161.3000) -- (97.3000,161.4000) -- (97.7000,161.5000) -- (98.1000,161.5000) -- (98.5000,161.5000) -- (99.0000,161.5000) -- (99.4000,161.5000) -- (99.8000,161.4000) -- (100.2000,161.4000) -- (100.7000,161.3000) -- (101.1000,161.3000) -- (101.5000,161.2000) -- (101.9000,161.1000) -- (102.3000,161.1000) -- (102.8000,161.0000) -- (103.2000,161.1000) -- (103.6000,161.1000) -- (104.0000,161.2000) -- (104.5000,161.2000) -- (104.9000,161.3000) -- (105.3000,161.4000) -- (105.7000,161.4000) -- (106.2000,161.5000) -- (106.6000,161.5000) -- (107.0000,161.6000) -- (107.4000,161.6000) -- (107.8000,161.5000) -- (108.3000,161.5000) -- (108.7000,161.4000) -- (109.1000,161.4000) -- (109.5000,161.3000) -- (110.0000,161.2000) -- (110.4000,161.1000) -- (110.8000,161.0000) -- (111.2000,161.0000) -- (111.7000,160.9000) -- (112.1000,160.9000) -- (112.5000,161.0000) -- (112.9000,161.0000) -- (113.3000,161.1000) -- (113.8000,161.2000) -- (114.2000,161.3000) -- (114.6000,161.4000) -- (115.0000,161.5000) -- (115.5000,161.6000) -- (115.9000,161.7000) -- (116.3000,161.7000) -- (116.7000,161.7000) -- (117.2000,161.6000) -- (117.6000,161.6000) -- (118.0000,161.5000) -- (118.4000,161.3000) -- (118.8000,161.2000) -- (119.3000,161.1000) -- (119.7000,161.0000) -- (120.1000,160.9000) -- (120.5000,160.8000) -- (121.0000,160.8000) -- (121.4000,160.9000) -- (121.8000,160.9000) -- (122.2000,161.0000) -- (122.6000,161.2000) -- (123.1000,161.3000) -- (123.5000,161.5000) -- (123.9000,161.6000) -- (124.3000,161.7000) -- (124.8000,161.8000) -- (125.2000,161.8000) -- (125.6000,161.8000) -- (126.0000,161.7000) -- (126.5000,161.6000) -- (126.9000,161.5000) -- (127.3000,161.4000) -- (127.7000,161.3000) -- (128.1000,161.1000) -- (128.6000,160.9000) -- (129.0000,160.8000) -- (129.4000,160.8000) -- (129.8000,160.8000) -- (130.3000,160.8000) -- (130.7000,160.9000) -- (131.1000,161.0000) -- (131.5000,161.1000) -- (131.9000,161.3000) -- (132.4000,161.4000) -- (132.8000,161.5000) -- (133.2000,161.6000) -- (133.6000,161.7000) -- (134.1000,161.7000) -- (134.5000,161.8000) -- (134.9000,161.7000) -- (135.3000,161.7000) -- (135.8000,161.6000) -- (136.2000,161.5000) -- (136.6000,161.3000) -- (137.0000,161.2000) -- (137.4000,161.0000) -- (137.9000,160.9000) -- (138.3000,160.9000) -- (138.7000,160.8000) -- (139.1000,160.8000) -- (139.6000,160.9000) -- (140.0000,160.9000) -- (140.4000,161.0000) -- (140.8000,161.1000) -- (141.3000,161.3000) -- (141.7000,161.4000) -- (142.1000,161.6000) -- (142.3000,161.6000);



    \end{scope}
    \path[draw=blue,line cap=butt,line join=miter,line width=0.747pt] (-429.4395,-113.5288) -- cycle;



    \begin{scope}[cm={{0.99667,0.0,0.0,1.34693,(-320.19845,-156.47287)}},draw=ca0a0a4,dash pattern=on 1.54pt off 1.54pt,line cap=round,line join=round,line width=0.257pt,miter limit=4.00]
      \path[draw,dash pattern=on 1.54pt off 1.54pt,line width=0.257pt,miter limit=4.00] (41.5000,153.5000) -- (127.5000,153.5000);



    \end{scope}
    \begin{scope}[cm={{0.99667,0.0,0.0,1.34693,(-320.19845,-156.47287)}},draw=cd9d9d9,line cap=rect,line join=miter,line width=2.113pt,miter limit=10.00]
      \path[draw=cffffff,line cap=rect,line join=miter,line width=2.113pt,miter limit=10.00] (41.5000,88.5000) -- (41.5000,164.5000) -- (127.5000,164.5000) -- (127.5000,88.5000) -- (41.5000,88.5000);



    \end{scope}
    \begin{scope}[cm={{0.99667,0.0,0.0,1.34693,(-320.19845,-156.47287)}},draw=blue,line cap=round,line join=round,line width=0.480pt]
      \path[cm={{1.155,0.0,0.0,1.0,(-6.38582,0.0)}},draw] (41.5000,153.5000) -- (44.5000,153.5000);



      \path[cm={{1.155,0.0,0.0,1.0,(-20.06101,0.0)}},draw] (127.5000,153.5000) -- (124.5000,153.5000);



    \end{scope}
    \begin{scope}[cm={{1.0018,0.0,0.0,1.0018,(-293.12869,52.89918)}},draw=blue,line cap=rect,line join=bevel,line width=0.800pt]
      \path[fill=blue] (0.0000,0.0000) node[above right] (text366) {\scriptsize 30};



    \end{scope}
    \begin{scope}[cm={{0.99667,0.0,0.0,1.34693,(-320.19845,-156.47287)}},draw=ca0a0a4,dash pattern=on 1.54pt off 1.54pt,line cap=round,line join=round,line width=0.257pt,miter limit=4.00]
      \path[draw,dash pattern=on 1.54pt off 1.54pt,line width=0.257pt,miter limit=4.00] (41.5000,127.5000) -- (127.5000,127.5000);



    \end{scope}
    \begin{scope}[cm={{0.99667,0.0,0.0,1.34693,(-320.19845,-156.47287)}},draw=blue,line cap=round,line join=round,line width=0.480pt]
      \path[cm={{1.155,0.0,0.0,1.0,(-6.38582,0.0)}},draw] (41.5000,127.5000) -- (44.5000,127.5000);



      \path[cm={{1.155,0.0,0.0,1.0,(-20.06101,0.0)}},draw] (127.5000,127.5000) -- (124.5000,127.5000);



    \end{scope}
    \begin{scope}[cm={{1.0018,0.0,0.0,1.0018,(-292.76805,19.17029)}},draw=blue,line cap=rect,line join=bevel,line width=0.800pt]
      \path[fill=blue] (0.0000,0.0000) node[above right] (text396) {\scriptsize 33};



    \end{scope}
    \begin{scope}[cm={{0.99667,0.0,0.0,1.34693,(-320.19845,-156.47287)}},draw=blue,line cap=round,line join=round,line width=0.480pt]
      \path[cm={{1.155,0.0,0.0,1.0,(-6.38582,0.0)}},draw] (41.5000,101.5000) -- (44.5000,101.5000);



      \path[cm={{1.155,0.0,0.0,1.0,(-20.06101,0.0)}},draw] (127.5000,101.5000) -- (124.5000,101.5000);



    \end{scope}
    \begin{scope}[cm={{1.0018,0.0,0.0,1.0018,(-293.06458,-16.50733)}},draw=blue,line cap=rect,line join=bevel,line width=0.800pt]
      \path[fill=blue] (0.0000,0.0000) node[above right] (text426) {\scriptsize 36};



    \end{scope}
    \begin{scope}[cm={{0.99667,0.0,0.0,1.34693,(-320.19845,-156.47287)}},draw=ca0a0a4,dash pattern=on 0.40pt off 0.80pt,line cap=round,line join=round,line width=0.400pt]
      \path[draw] (41.5000,164.5000) -- (41.5000,88.5000);



    \end{scope}
    \begin{scope}[cm={{0.99667,0.0,0.0,1.34693,(-320.19845,-156.47287)}},draw=blue,line cap=round,line join=round,line width=0.480pt]
      \path[draw] (41.5000,164.5000) -- (41.5000,159.5000);



      \path[draw] (41.5000,88.5000) -- (41.5000,92.5000);



    \end{scope}
    \begin{scope}[cm={{1.00174,0.0,0.0,1.01515,(-281.55885,76.66745)}},draw=blue,line cap=rect,line join=bevel,line width=0.800pt]
      \path[fill=blue] (0.0000,0.0000) node[above right] (text456) {\scriptsize 0};



    \end{scope}
    \begin{scope}[cm={{0.99667,0.0,0.0,1.34693,(-320.19845,-156.47287)}},draw=blue,line cap=round,line join=round,line width=0.480pt]
      \path[cm={{1.0,0.0,0.0,0.54942,(0.0,73.94736)}},draw] (70.5000,164.5000) -- (70.5000,159.5000);



      \path[cm={{1.0,0.0,0.0,0.66885,(0.0,29.20714)}},draw] (70.5000,88.5000) -- (70.5000,92.5000);



    \end{scope}
    \begin{scope}[cm={{1.00174,0.0,0.0,1.01515,(-253.51015,76.66745)}},draw=blue,line cap=rect,line join=bevel,line width=0.800pt]
      \path[fill=blue] (0.0000,0.0000) node[above right] (text486) {\scriptsize 2};



    \end{scope}
    \begin{scope}[cm={{0.99667,0.0,0.0,1.34693,(-320.19845,-156.47287)}},draw=blue,line cap=round,line join=round,line width=0.480pt]
      \path[cm={{1.0,0.0,0.0,0.54942,(0.0,73.94736)}},draw] (98.5000,164.5000) -- (98.5000,159.5000);



      \path[cm={{1.0,0.0,0.0,0.66885,(0.0,29.20714)}},draw] (98.5000,88.5000) -- (98.5000,92.5000);



    \end{scope}
    \begin{scope}[cm={{1.00174,0.0,0.0,1.01515,(-223.95891,76.66745)}},draw=blue,line cap=rect,line join=bevel,line width=0.800pt]
      \path[fill=blue] (0.0000,0.0000) node[above right] (text516) {\scriptsize 4};



    \end{scope}
    \begin{scope}[cm={{0.99667,0.0,0.0,1.34693,(-320.19845,-156.47287)}},draw=ca0a0a4,dash pattern=on 0.40pt off 0.80pt,line cap=round,line join=round,line width=0.400pt]
      \path[draw] (127.5000,164.5000) -- (127.5000,88.5000);



    \end{scope}
    \begin{scope}[cm={{0.99667,0.0,0.0,1.34693,(-320.19845,-156.47287)}},draw=blue,line cap=round,line join=round,line width=0.480pt]
      \path[draw] (127.5000,164.5000) -- (127.5000,159.5000);



      \path[draw] (127.5000,88.5000) -- (127.5000,92.5000);



    \end{scope}
    \begin{scope}[cm={{1.00174,0.0,0.0,1.01515,(-196.411,76.73242)}},draw=blue,line cap=rect,line join=bevel,line width=0.800pt]
      \path[fill=blue] (0.0000,0.0000) node[above right] (text546) {\scriptsize 6};



    \end{scope}
    \begin{scope}[cm={{0.99667,0.0,0.0,1.34693,(-320.19845,-156.47287)}},draw=blue,line cap=round,line join=round,line width=0.480pt]
      \path[draw] (41.6000,103.3000) -- (41.6000,103.3000) -- (41.7000,103.6000) -- (41.9000,104.0000) -- (42.0000,104.3000) -- (42.2000,104.6000) -- (42.3000,104.9000) -- (42.5000,105.1000) -- (42.6000,105.4000) -- (42.7000,105.7000) -- (42.9000,106.0000) -- (43.0000,106.3000) -- (43.2000,106.5000) -- (43.3000,106.8000) -- (43.5000,107.1000) -- (43.6000,107.3000) -- (43.7000,107.6000) -- (43.9000,107.9000) -- (44.0000,108.1000) -- (44.2000,108.4000) -- (44.3000,108.6000) -- (44.5000,108.8000) -- (44.6000,109.1000) -- (44.7000,109.3000) -- (44.9000,109.5000) -- (45.0000,109.8000) -- (45.2000,110.0000) -- (45.3000,110.2000) -- (45.5000,110.4000) -- (45.6000,110.7000) -- (45.7000,110.9000) -- (45.9000,111.1000) -- (46.0000,111.3000) -- (46.2000,111.5000) -- (46.3000,111.7000) -- (46.5000,111.9000) -- (46.6000,112.1000) -- (46.7000,112.3000) -- (46.9000,112.4000) -- (47.0000,112.6000) -- (47.2000,112.8000) -- (47.3000,113.0000) -- (47.5000,113.2000) -- (47.6000,113.3000) -- (47.7000,113.5000) -- (47.9000,113.7000) -- (48.0000,113.9000) -- (48.2000,114.0000) -- (48.3000,114.2000) -- (48.5000,114.3000) -- (48.6000,114.5000) -- (48.7000,114.7000) -- (48.9000,114.8000) -- (49.0000,115.0000) -- (49.2000,115.1000) -- (49.3000,115.2000) -- (49.5000,115.4000) -- (49.6000,115.5000) -- (49.7000,115.7000) -- (49.9000,115.8000) -- (50.0000,115.9000) -- (50.2000,116.1000) -- (50.3000,116.2000) -- (50.5000,116.3000) -- (50.6000,116.4000) -- (50.7000,116.6000) -- (50.9000,116.7000) -- (51.0000,116.8000) -- (51.2000,116.9000) -- (51.3000,117.0000) -- (51.5000,117.1000) -- (51.6000,117.2000) -- (51.7000,117.4000) -- (51.9000,117.5000) -- (52.0000,117.6000) -- (52.2000,117.7000) -- (52.3000,117.8000) -- (52.5000,117.9000) -- (52.6000,118.0000) -- (52.7000,118.0000) -- (52.9000,118.1000) -- (53.0000,118.2000) -- (53.2000,118.3000) -- (53.3000,118.4000) -- (53.5000,118.5000) -- (53.6000,118.6000) -- (53.7000,118.7000) -- (53.9000,118.7000) -- (54.0000,118.8000) -- (54.2000,118.9000) -- (54.3000,119.0000) -- (54.5000,119.0000) -- (54.6000,119.1000) -- (54.7000,119.2000) -- (54.9000,119.2000) -- (55.0000,119.3000) -- (55.2000,119.4000) -- (55.3000,119.4000) -- (55.5000,119.5000) -- (55.6000,119.6000) -- (55.7000,119.6000) -- (55.9000,119.7000) -- (56.0000,119.7000) -- (56.2000,119.8000) -- (56.3000,119.8000) -- (56.5000,119.9000) -- (56.6000,119.9000) -- (56.7000,120.0000) -- (56.9000,120.0000) -- (57.0000,120.1000) -- (57.2000,120.1000) -- (57.3000,120.2000) -- (57.5000,120.2000) -- (57.6000,120.2000) -- (57.7000,120.3000) -- (57.9000,120.3000) -- (58.0000,120.4000) -- (58.2000,120.4000) -- (58.3000,120.4000) -- (58.5000,120.5000) -- (58.6000,120.5000) -- (58.7000,120.5000) -- (58.9000,120.6000) -- (59.0000,120.6000) -- (59.2000,120.6000) -- (59.3000,120.6000) -- (59.5000,120.7000) -- (59.6000,120.7000) -- (59.7000,120.7000) -- (59.9000,120.7000) -- (60.0000,120.8000) -- (60.2000,120.8000) -- (60.3000,120.8000) -- (60.5000,120.8000) -- (60.6000,120.8000) -- (60.7000,120.9000) -- (60.9000,120.9000) -- (61.0000,120.9000) -- (61.2000,120.9000) -- (61.3000,120.9000) -- (61.5000,120.9000) -- (61.6000,120.9000) -- (61.7000,121.0000) -- (61.9000,121.0000) -- (62.0000,121.0000) -- (62.2000,121.0000) -- (62.3000,121.0000) -- (62.5000,121.0000) -- (62.6000,121.0000) -- (62.7000,121.0000) -- (62.9000,121.0000) -- (63.0000,121.0000) -- (63.2000,121.0000) -- (63.3000,121.0000) -- (63.5000,121.0000) -- (63.6000,121.0000) -- (63.7000,121.0000) -- (63.9000,121.0000) -- (64.0000,121.0000) -- (64.2000,121.0000) -- (64.3000,121.0000) -- (64.5000,121.0000) -- (64.6000,121.0000) -- (64.7000,121.0000) -- (64.9000,121.0000) -- (65.0000,121.0000) -- (65.2000,121.0000) -- (65.3000,121.0000) -- (65.5000,121.0000) -- (65.6000,121.0000) -- (65.7000,121.0000) -- (65.9000,121.0000) -- (66.0000,121.0000) -- (66.2000,121.0000) -- (66.3000,121.0000) -- (66.5000,120.9000) -- (66.6000,120.9000) -- (66.7000,120.9000) -- (66.9000,120.9000) -- (67.0000,120.9000) -- (67.2000,120.9000) -- (67.3000,120.9000) -- (67.5000,120.9000) -- (67.6000,120.9000) -- (67.7000,120.9000) -- (67.9000,120.9000) -- (68.0000,120.9000) -- (68.2000,120.8000) -- (68.3000,120.8000) -- (68.5000,120.8000) -- (68.6000,120.8000) -- (68.7000,120.8000) -- (68.9000,120.8000) -- (69.0000,120.8000) -- (69.2000,120.8000) -- (69.3000,120.8000) -- (69.5000,120.8000) -- (69.6000,120.8000) -- (69.7000,120.8000) -- (69.9000,120.8000) -- (70.0000,120.8000) -- (70.2000,120.8000) -- (70.3000,120.8000) -- (70.5000,120.8000) -- (70.6000,120.8000) -- (70.7000,120.8000) -- (70.9000,120.8000) -- (71.0000,120.8000) -- (71.2000,120.8000) -- (71.3000,120.8000) -- (71.5000,120.9000) -- (71.6000,120.9000) -- (71.7000,120.9000) -- (71.9000,120.9000) -- (72.0000,120.9000) -- (72.2000,120.9000) -- (72.3000,121.0000) -- (72.5000,121.0000) -- (72.6000,121.0000) -- (72.7000,121.0000) -- (72.9000,121.1000) -- (73.0000,121.1000) -- (73.2000,121.1000) -- (73.3000,121.2000) -- (73.5000,121.2000) -- (73.6000,121.2000) -- (73.7000,121.2000) -- (73.9000,121.3000) -- (74.0000,121.3000) -- (74.2000,121.3000) -- (74.3000,121.3000) -- (74.5000,121.4000) -- (74.6000,121.4000) -- (74.7000,121.4000) -- (74.9000,121.5000) -- (75.0000,121.5000) -- (75.2000,121.5000) -- (75.3000,121.5000) -- (75.5000,121.6000) -- (75.6000,121.6000) -- (75.7000,121.6000) -- (75.9000,121.6000) -- (76.0000,121.7000) -- (76.2000,121.7000) -- (76.3000,121.7000) -- (76.5000,121.7000) -- (76.6000,121.7000) -- (76.7000,121.7000) -- (76.9000,121.8000) -- (77.0000,121.8000) -- (77.2000,121.8000) -- (77.3000,121.8000) -- (77.5000,121.8000) -- (77.6000,121.8000) -- (77.7000,121.9000) -- (77.9000,121.9000) -- (78.0000,121.9000) -- (78.2000,121.9000) -- (78.3000,121.9000) -- (78.5000,121.9000) -- (78.6000,121.9000) -- (78.7000,121.9000) -- (78.9000,121.9000) -- (79.0000,121.9000) -- (79.2000,121.9000) -- (79.3000,121.9000) -- (79.5000,121.9000) -- (79.6000,121.9000) -- (79.7000,122.0000) -- (79.9000,122.0000) -- (80.0000,122.0000) -- (80.2000,122.0000) -- (80.3000,122.0000) -- (80.5000,121.9000) -- (80.6000,121.9000) -- (80.7000,121.9000) -- (80.9000,121.9000) -- (81.0000,121.9000) -- (81.2000,121.9000) -- (81.3000,121.9000) -- (81.5000,121.9000) -- (81.6000,121.9000) -- (81.7000,121.9000) -- (81.9000,121.9000) -- (82.0000,121.9000) -- (82.2000,121.9000) -- (82.3000,121.9000) -- (82.5000,121.9000) -- (82.6000,121.9000) -- (82.7000,121.8000) -- (82.9000,121.8000) -- (83.0000,121.8000) -- (83.2000,121.8000) -- (83.3000,121.8000) -- (83.5000,121.8000) -- (83.6000,121.8000) -- (83.7000,121.8000) -- (83.9000,121.7000) -- (84.0000,121.7000) -- (84.2000,121.7000) -- (84.3000,121.7000) -- (84.5000,121.7000) -- (84.6000,121.7000) -- (84.7000,121.6000) -- (84.9000,121.6000) -- (85.0000,121.6000) -- (85.2000,121.6000) -- (85.3000,121.6000) -- (85.5000,121.5000) -- (85.6000,121.5000) -- (85.7000,121.5000) -- (85.9000,121.5000) -- (86.0000,121.5000) -- (86.2000,121.5000) -- (86.3000,121.4000) -- (86.5000,121.4000) -- (86.6000,121.4000) -- (86.7000,121.4000) -- (86.9000,121.3000) -- (87.0000,121.3000) -- (87.2000,121.3000) -- (87.3000,121.3000) -- (87.5000,121.3000) -- (87.6000,121.2000) -- (87.7000,121.2000) -- (87.9000,121.2000) -- (88.0000,121.2000) -- (88.2000,121.1000) -- (88.3000,121.1000) -- (88.5000,121.1000) -- (88.6000,121.1000) -- (88.7000,121.0000) -- (88.9000,121.0000) -- (89.0000,121.0000) -- (89.2000,121.0000) -- (89.3000,120.9000) -- (89.5000,120.9000) -- (89.6000,120.9000) -- (89.7000,120.9000) -- (89.9000,120.8000) -- (90.0000,120.8000) -- (90.2000,120.8000) -- (90.3000,120.8000) -- (90.5000,120.7000) -- (90.6000,120.7000) -- (90.7000,120.7000) -- (90.9000,120.6000) -- (91.0000,120.6000) -- (91.2000,120.6000) -- (91.3000,120.6000) -- (91.5000,120.5000) -- (91.6000,120.5000) -- (91.7000,120.5000) -- (91.9000,120.5000) -- (92.0000,120.4000) -- (92.2000,120.4000) -- (92.3000,120.4000) -- (92.5000,120.4000) -- (92.6000,120.3000) -- (92.7000,120.3000) -- (92.9000,120.3000) -- (93.0000,120.2000) -- (93.2000,120.2000) -- (93.3000,120.2000) -- (93.4000,120.2000) -- (93.6000,120.1000) -- (93.7000,120.1000) -- (93.9000,120.1000) -- (94.0000,120.1000) -- (94.2000,120.0000) -- (94.3000,120.0000) -- (94.4000,120.0000) -- (94.6000,120.0000) -- (94.7000,119.9000) -- (94.9000,119.9000) -- (95.0000,119.9000) -- (95.2000,119.8000) -- (95.3000,119.8000) -- (95.4000,119.8000) -- (95.6000,119.8000) -- (95.7000,119.7000) -- (95.9000,119.7000) -- (96.0000,119.7000) -- (96.2000,119.7000) -- (96.3000,119.6000) -- (96.4000,119.6000) -- (96.6000,119.6000) -- (96.7000,119.6000) -- (96.9000,119.5000) -- (97.0000,119.5000) -- (97.2000,119.5000) -- (97.3000,119.5000) -- (97.4000,119.4000) -- (97.6000,119.4000) -- (97.7000,119.4000) -- (97.9000,119.3000) -- (98.0000,119.3000) -- (98.2000,119.3000) -- (98.3000,119.3000) -- (98.4000,119.2000) -- (98.6000,119.2000) -- (98.7000,119.2000) -- (98.9000,119.1000) -- (99.0000,119.1000) -- (99.2000,119.1000) -- (99.3000,119.1000) -- (99.4000,119.0000) -- (99.6000,119.0000) -- (99.7000,119.0000) -- (99.9000,119.0000) -- (100.0000,118.9000) -- (100.2000,118.9000) -- (100.3000,118.9000) -- (100.4000,118.9000) -- (100.6000,118.8000) -- (100.7000,118.8000) -- (100.9000,118.8000) -- (101.0000,118.8000) -- (101.2000,118.7000) -- (101.3000,118.7000) -- (101.4000,118.7000) -- (101.6000,118.7000) -- (101.7000,118.6000) -- (101.9000,118.6000) -- (102.0000,118.6000) -- (102.2000,118.6000) -- (102.3000,118.5000) -- (102.4000,118.5000) -- (102.6000,118.5000) -- (102.7000,118.5000) -- (102.9000,118.4000) -- (103.0000,118.4000) -- (103.2000,118.4000) -- (103.3000,118.4000) -- (103.4000,118.4000) -- (103.6000,118.3000) -- (103.7000,118.3000) -- (103.9000,118.3000) -- (104.0000,118.3000) -- (104.2000,118.3000) -- (104.3000,118.2000) -- (104.4000,118.2000) -- (104.6000,118.2000) -- (104.7000,118.2000) -- (104.9000,118.2000) -- (105.0000,118.1000) -- (105.2000,118.1000) -- (105.3000,118.1000) -- (105.4000,118.1000) -- (105.6000,118.1000) -- (105.7000,118.0000) -- (105.9000,118.0000) -- (106.0000,118.0000) -- (106.2000,118.0000) -- (106.3000,118.0000) -- (106.4000,118.0000) -- (106.6000,117.9000) -- (106.7000,117.9000) -- (106.9000,117.9000) -- (107.0000,117.9000) -- (107.2000,117.9000) -- (107.3000,117.9000) -- (107.4000,117.9000) -- (107.6000,117.8000) -- (107.7000,117.8000) -- (107.9000,117.8000) -- (108.0000,117.8000) -- (108.2000,117.8000) -- (108.3000,117.8000) -- (108.4000,117.8000) -- (108.6000,117.8000) -- (108.7000,117.7000) -- (108.9000,117.7000) -- (109.0000,117.7000) -- (109.2000,117.7000) -- (109.3000,117.7000) -- (109.4000,117.7000) -- (109.6000,117.7000) -- (109.7000,117.7000) -- (109.9000,117.7000) -- (110.0000,117.6000) -- (110.2000,117.6000) -- (110.3000,117.6000) -- (110.4000,117.6000) -- (110.6000,117.6000) -- (110.7000,117.6000) -- (110.9000,117.6000) -- (111.0000,117.6000) -- (111.2000,117.6000) -- (111.3000,117.6000) -- (111.4000,117.6000) -- (111.6000,117.6000) -- (111.7000,117.5000) -- (111.9000,117.5000) -- (112.0000,117.5000) -- (112.2000,117.5000) -- (112.3000,117.5000) -- (112.4000,117.5000) -- (112.6000,117.5000) -- (112.7000,117.5000) -- (112.9000,117.5000) -- (113.0000,117.5000) -- (113.2000,117.5000) -- (113.3000,117.5000) -- (113.4000,117.5000) -- (113.6000,117.5000) -- (113.7000,117.5000) -- (113.9000,117.5000) -- (114.0000,117.5000) -- (114.2000,117.5000) -- (114.3000,117.4000) -- (114.4000,117.4000) -- (114.6000,117.4000) -- (114.7000,117.4000) -- (114.9000,117.4000) -- (115.0000,117.4000) -- (115.2000,117.4000) -- (115.3000,117.4000) -- (115.4000,117.4000) -- (115.6000,117.4000) -- (115.7000,117.4000) -- (115.9000,117.4000) -- (116.0000,117.4000) -- (116.2000,117.4000) -- (116.3000,117.4000) -- (116.4000,117.4000) -- (116.6000,117.4000) -- (116.7000,117.4000) -- (116.9000,117.4000) -- (117.0000,117.4000) -- (117.2000,117.4000) -- (117.3000,117.4000) -- (117.4000,117.4000) -- (117.6000,117.4000) -- (117.7000,117.4000) -- (117.9000,117.4000) -- (118.0000,117.4000) -- (118.2000,117.4000) -- (118.3000,117.4000) -- (118.4000,117.4000) -- (118.6000,117.4000) -- (118.7000,117.4000) -- (118.9000,117.4000) -- (119.0000,117.4000) -- (119.2000,117.4000) -- (119.3000,117.4000) -- (119.4000,117.4000) -- (119.6000,117.4000) -- (119.7000,117.4000) -- (119.9000,117.4000) -- (120.0000,117.4000) -- (120.2000,117.4000) -- (120.3000,117.4000) -- (120.4000,117.4000) -- (120.6000,117.4000) -- (120.7000,117.4000) -- (120.9000,117.4000) -- (121.0000,117.4000) -- (121.2000,117.4000) -- (121.3000,117.4000) -- (121.4000,117.4000) -- (121.6000,117.4000) -- (121.7000,117.4000) -- (121.9000,117.4000) -- (122.0000,117.4000) -- (122.2000,117.4000) -- (122.3000,117.4000) -- (122.4000,117.4000) -- (122.6000,117.4000) -- (122.7000,117.4000) -- (122.9000,117.4000) -- (123.0000,117.4000) -- (123.2000,117.4000) -- (123.3000,117.4000) -- (123.4000,117.4000) -- (123.6000,117.4000) -- (123.7000,117.4000) -- (123.9000,117.4000) -- (124.0000,117.4000) -- (124.2000,117.4000) -- (124.3000,117.4000) -- (124.4000,117.5000) -- (124.6000,117.5000) -- (124.7000,117.5000) -- (124.9000,117.5000) -- (125.0000,117.5000) -- (125.2000,117.5000) -- (125.3000,117.5000) -- (125.4000,117.5000) -- (125.6000,117.5000) -- (125.7000,117.5000) -- (125.9000,117.5000) -- (126.0000,117.5000) -- (126.2000,117.5000) -- (126.3000,117.5000) -- (126.4000,117.5000) -- (126.6000,117.5000) -- (126.7000,117.5000) -- (126.9000,117.5000) -- (127.0000,117.5000) -- (127.2000,117.5000) -- (127.3000,117.5000);



    \end{scope}
    \begin{scope}[cm={{0.99667,0.0,0.0,1.34693,(-320.19845,-156.47287)}},draw=cff0000,line cap=round,line join=bevel,line width=0.480pt,miter limit=4.00]
      \path[draw,line cap=round,line join=round,line width=0.480pt,miter limit=4.00] (41.6000,109.8000) -- (41.6000,109.8000) -- (41.7000,103.2000) -- (41.9000,103.5000) -- (42.0000,103.9000) -- (42.2000,104.4000) -- (42.3000,104.8000) -- (42.5000,105.2000) -- (42.6000,105.6000) -- (42.7000,106.0000) -- (42.9000,106.4000) -- (43.0000,106.7000) -- (43.2000,107.1000) -- (43.3000,107.4000) -- (43.5000,107.7000) -- (43.6000,107.9000) -- (43.7000,108.2000) -- (43.9000,108.5000) -- (44.0000,108.7000) -- (44.2000,108.9000) -- (44.3000,109.1000) -- (44.5000,109.4000) -- (44.6000,109.6000) -- (44.7000,109.7000) -- (44.9000,109.9000) -- (45.0000,110.1000) -- (45.2000,110.3000) -- (45.3000,110.5000) -- (45.5000,110.7000) -- (45.6000,110.9000) -- (45.7000,111.1000) -- (45.9000,111.3000) -- (46.0000,111.5000) -- (46.2000,111.7000) -- (46.3000,111.9000) -- (46.5000,112.1000) -- (46.6000,112.3000) -- (46.7000,112.5000) -- (46.9000,112.7000) -- (47.0000,112.9000) -- (47.2000,113.1000) -- (47.3000,113.3000) -- (47.5000,113.5000) -- (47.6000,113.7000) -- (47.7000,113.9000) -- (47.9000,114.1000) -- (48.0000,114.3000) -- (48.2000,114.5000) -- (48.3000,114.6000) -- (48.5000,114.8000) -- (48.6000,114.9000) -- (48.7000,115.1000) -- (48.9000,115.2000) -- (49.0000,115.3000) -- (49.2000,115.4000) -- (49.3000,115.6000) -- (49.5000,115.7000) -- (49.6000,115.8000) -- (49.7000,115.9000) -- (49.9000,116.0000) -- (50.0000,116.1000) -- (50.2000,116.2000) -- (50.3000,116.3000) -- (50.5000,116.4000) -- (50.6000,116.5000) -- (50.7000,116.6000) -- (50.9000,116.8000) -- (51.0000,116.9000) -- (51.2000,117.0000) -- (51.3000,117.2000) -- (51.5000,117.3000) -- (51.6000,117.5000) -- (51.7000,117.7000) -- (51.9000,117.8000) -- (52.0000,118.0000) -- (52.2000,118.2000) -- (52.3000,118.3000) -- (52.5000,118.5000) -- (52.6000,118.6000) -- (52.7000,118.8000) -- (52.9000,118.9000) -- (53.0000,119.0000) -- (53.2000,119.1000) -- (53.3000,119.2000) -- (53.5000,119.3000) -- (53.6000,119.4000) -- (53.7000,119.4000) -- (53.9000,119.4000) -- (54.0000,119.5000) -- (54.2000,119.5000) -- (54.3000,119.5000) -- (54.5000,119.4000) -- (54.6000,119.4000) -- (54.7000,119.4000) -- (54.9000,119.3000) -- (55.0000,119.3000) -- (55.2000,119.2000) -- (55.3000,119.2000) -- (55.5000,119.1000) -- (55.6000,119.1000) -- (55.7000,119.1000) -- (55.9000,119.0000) -- (56.0000,119.0000) -- (56.2000,119.0000) -- (56.3000,119.1000) -- (56.5000,119.1000) -- (56.6000,119.2000) -- (56.7000,119.2000) -- (56.9000,119.3000) -- (57.0000,119.4000) -- (57.2000,119.5000) -- (57.3000,119.6000) -- (57.5000,119.7000) -- (57.6000,119.8000) -- (57.7000,119.9000) -- (57.9000,120.1000) -- (58.0000,120.2000) -- (58.2000,120.3000) -- (58.3000,120.4000) -- (58.5000,120.5000) -- (58.6000,120.6000) -- (58.7000,120.7000) -- (58.9000,120.8000) -- (59.0000,120.8000) -- (59.2000,120.9000) -- (59.3000,121.0000) -- (59.5000,121.0000) -- (59.6000,121.0000) -- (59.7000,121.1000) -- (59.9000,121.1000) -- (60.0000,121.1000) -- (60.2000,121.1000) -- (60.3000,121.1000) -- (60.5000,121.1000) -- (60.6000,121.1000) -- (60.7000,121.0000) -- (60.9000,121.0000) -- (61.0000,121.0000) -- (61.2000,121.0000) -- (61.3000,120.9000) -- (61.5000,120.9000) -- (61.6000,120.9000) -- (61.7000,120.8000) -- (61.9000,120.8000) -- (62.0000,120.8000) -- (62.2000,120.8000) -- (62.3000,120.7000) -- (62.5000,120.7000) -- (62.6000,120.7000) -- (62.7000,120.7000) -- (62.9000,120.7000) -- (63.0000,120.7000) -- (63.2000,120.7000) -- (63.3000,120.7000) -- (63.5000,120.7000) -- (63.6000,120.8000) -- (63.7000,120.8000) -- (63.9000,120.8000) -- (64.0000,120.8000) -- (64.2000,120.8000) -- (64.3000,120.9000) -- (64.5000,120.9000) -- (64.6000,120.9000) -- (64.7000,121.0000) -- (64.9000,121.0000) -- (65.0000,121.0000) -- (65.2000,121.0000) -- (65.3000,121.1000) -- (65.5000,121.1000) -- (65.6000,121.1000) -- (65.7000,121.1000) -- (65.9000,121.2000) -- (66.0000,121.2000) -- (66.2000,121.2000) -- (66.3000,121.2000) -- (66.5000,121.2000) -- (66.6000,121.2000) -- (66.7000,121.2000) -- (66.9000,121.2000) -- (67.0000,121.2000) -- (67.2000,121.2000) -- (67.3000,121.2000) -- (67.5000,121.2000) -- (67.6000,121.2000) -- (67.7000,121.2000) -- (67.9000,121.2000) -- (68.0000,121.2000) -- (68.2000,121.1000) -- (68.3000,121.1000) -- (68.5000,121.1000) -- (68.6000,121.1000) -- (68.7000,121.0000) -- (68.9000,121.0000) -- (69.0000,121.0000) -- (69.2000,120.9000) -- (69.3000,120.9000) -- (69.5000,120.9000) -- (69.6000,120.8000) -- (69.7000,120.8000) -- (69.9000,120.8000) -- (70.0000,120.7000) -- (70.2000,120.7000) -- (70.3000,120.7000) -- (70.5000,120.7000) -- (70.6000,120.6000) -- (70.7000,120.6000) -- (70.9000,120.6000) -- (71.0000,120.6000) -- (71.2000,120.6000) -- (71.3000,120.6000) -- (71.5000,120.6000) -- (71.6000,120.6000) -- (71.7000,120.6000) -- (71.9000,120.6000) -- (72.0000,120.6000) -- (72.2000,120.6000) -- (72.3000,120.6000) -- (72.5000,120.6000) -- (72.6000,120.6000) -- (72.7000,120.7000) -- (72.9000,120.7000) -- (73.0000,120.7000) -- (73.2000,120.8000) -- (73.3000,120.8000) -- (73.5000,120.8000) -- (73.6000,120.9000) -- (73.7000,120.9000) -- (73.9000,120.9000) -- (74.0000,121.0000) -- (74.2000,121.0000) -- (74.3000,121.1000) -- (74.5000,121.1000) -- (74.6000,121.2000) -- (74.7000,121.2000) -- (74.9000,121.3000) -- (75.0000,121.3000) -- (75.2000,121.3000) -- (75.3000,121.4000) -- (75.5000,121.4000) -- (75.6000,121.5000) -- (75.7000,121.5000) -- (75.9000,121.6000) -- (76.0000,121.6000) -- (76.2000,121.7000) -- (76.3000,121.7000) -- (76.5000,121.7000) -- (76.6000,121.8000) -- (76.7000,121.8000) -- (76.9000,121.9000) -- (77.0000,121.9000) -- (77.2000,121.9000) -- (77.3000,122.0000) -- (77.5000,122.0000) -- (77.6000,122.0000) -- (77.7000,122.1000) -- (77.9000,122.1000) -- (78.0000,122.1000) -- (78.2000,122.1000) -- (78.3000,122.1000) -- (78.5000,122.2000) -- (78.6000,122.2000) -- (78.7000,122.2000) -- (78.9000,122.2000) -- (79.0000,122.2000) -- (79.2000,122.2000) -- (79.3000,122.2000) -- (79.5000,122.2000) -- (79.6000,122.2000) -- (79.7000,122.2000) -- (79.9000,122.2000) -- (80.0000,122.2000) -- (80.2000,122.2000) -- (80.3000,122.2000) -- (80.5000,122.2000) -- (80.6000,122.2000) -- (80.7000,122.2000) -- (80.9000,122.2000) -- (81.0000,122.2000) -- (81.2000,122.2000) -- (81.3000,122.2000) -- (81.5000,122.1000) -- (81.6000,122.1000) -- (81.7000,122.1000) -- (81.9000,122.1000) -- (82.0000,122.1000) -- (82.2000,122.0000) -- (82.3000,122.0000) -- (82.5000,122.0000) -- (82.6000,122.0000) -- (82.7000,121.9000) -- (82.9000,121.9000) -- (83.0000,121.9000) -- (83.2000,121.9000) -- (83.3000,121.8000) -- (83.5000,121.8000) -- (83.6000,121.8000) -- (83.7000,121.8000) -- (83.9000,121.7000) -- (84.0000,121.7000) -- (84.2000,121.7000) -- (84.3000,121.7000) -- (84.5000,121.6000) -- (84.6000,121.6000) -- (84.7000,121.6000) -- (84.9000,121.5000) -- (85.0000,121.5000) -- (85.2000,121.5000) -- (85.3000,121.5000) -- (85.5000,121.4000) -- (85.6000,121.4000) -- (85.7000,121.4000) -- (85.9000,121.3000) -- (86.0000,121.3000) -- (86.2000,121.3000) -- (86.3000,121.3000) -- (86.5000,121.2000) -- (86.6000,121.2000) -- (86.7000,121.2000) -- (86.9000,121.2000) -- (87.0000,121.1000) -- (87.2000,121.1000) -- (87.3000,121.1000) -- (87.5000,121.1000) -- (87.6000,121.1000) -- (87.7000,121.0000) -- (87.9000,121.0000) -- (88.0000,121.0000) -- (88.2000,121.0000) -- (88.3000,120.9000) -- (88.5000,120.9000) -- (88.6000,120.9000) -- (88.7000,120.9000) -- (88.9000,120.9000) -- (89.0000,120.8000) -- (89.2000,120.8000) -- (89.3000,120.8000) -- (89.5000,120.8000) -- (89.6000,120.8000) -- (89.7000,120.7000) -- (89.9000,120.7000) -- (90.0000,120.7000) -- (90.2000,120.7000) -- (90.3000,120.7000) -- (90.5000,120.6000) -- (90.6000,120.6000) -- (90.7000,120.6000) -- (90.9000,120.6000) -- (91.0000,120.6000) -- (91.2000,120.5000) -- (91.3000,120.5000) -- (91.5000,120.5000) -- (91.6000,120.5000) -- (91.7000,120.5000) -- (91.9000,120.4000) -- (92.0000,120.4000) -- (92.2000,120.4000) -- (92.3000,120.4000) -- (92.5000,120.4000) -- (92.6000,120.4000) -- (92.7000,120.3000) -- (92.9000,120.3000) -- (93.0000,120.3000) -- (93.2000,120.3000) -- (93.3000,120.3000) -- (93.4000,120.2000) -- (93.6000,120.2000) -- (93.7000,120.2000) -- (93.9000,120.2000) -- (94.0000,120.2000) -- (94.2000,120.1000) -- (94.3000,120.1000) -- (94.4000,120.1000) -- (94.6000,120.1000) -- (94.7000,120.0000) -- (94.9000,120.0000) -- (95.0000,120.0000) -- (95.2000,120.0000) -- (95.3000,120.0000) -- (95.4000,119.9000) -- (95.6000,119.9000) -- (95.7000,119.9000) -- (95.9000,119.9000) -- (96.0000,119.8000) -- (96.2000,119.8000) -- (96.3000,119.8000) -- (96.4000,119.8000) -- (96.6000,119.7000) -- (96.7000,119.7000) -- (96.9000,119.7000) -- (97.0000,119.7000) -- (97.2000,119.6000) -- (97.3000,119.6000) -- (97.4000,119.6000) -- (97.6000,119.5000) -- (97.7000,119.5000) -- (97.9000,119.5000) -- (98.0000,119.5000) -- (98.2000,119.4000) -- (98.3000,119.4000) -- (98.4000,119.4000) -- (98.6000,119.3000) -- (98.7000,119.3000) -- (98.9000,119.3000) -- (99.0000,119.2000) -- (99.2000,119.2000) -- (99.3000,119.2000) -- (99.4000,119.1000) -- (99.6000,119.1000) -- (99.7000,119.1000) -- (99.9000,119.1000) -- (100.0000,119.0000) -- (100.2000,119.0000) -- (100.3000,119.0000) -- (100.4000,118.9000) -- (100.6000,118.9000) -- (100.7000,118.9000) -- (100.9000,118.8000) -- (101.0000,118.8000) -- (101.2000,118.8000) -- (101.3000,118.7000) -- (101.4000,118.7000) -- (101.6000,118.7000) -- (101.7000,118.6000) -- (101.9000,118.6000) -- (102.0000,118.6000) -- (102.2000,118.6000) -- (102.3000,118.5000) -- (102.4000,118.5000) -- (102.6000,118.5000) -- (102.7000,118.4000) -- (102.9000,118.4000) -- (103.0000,118.4000) -- (103.2000,118.4000) -- (103.3000,118.3000) -- (103.4000,118.3000) -- (103.6000,118.3000) -- (103.7000,118.3000) -- (103.9000,118.2000) -- (104.0000,118.2000) -- (104.2000,118.2000) -- (104.3000,118.2000) -- (104.4000,118.1000) -- (104.6000,118.1000) -- (104.7000,118.1000) -- (104.9000,118.1000) -- (105.0000,118.0000) -- (105.2000,118.0000) -- (105.3000,118.0000) -- (105.4000,118.0000) -- (105.6000,118.0000) -- (105.7000,117.9000) -- (105.9000,117.9000) -- (106.0000,117.9000) -- (106.2000,117.9000) -- (106.3000,117.9000) -- (106.4000,117.8000) -- (106.6000,117.8000) -- (106.7000,117.8000) -- (106.9000,117.8000) -- (107.0000,117.8000) -- (107.2000,117.8000) -- (107.3000,117.7000) -- (107.4000,117.7000) -- (107.6000,117.7000) -- (107.7000,117.7000) -- (107.9000,117.7000) -- (108.0000,117.7000) -- (108.2000,117.7000) -- (108.3000,117.6000) -- (108.4000,117.6000) -- (108.6000,117.6000) -- (108.7000,117.6000) -- (108.9000,117.6000) -- (109.0000,117.6000) -- (109.2000,117.6000) -- (109.3000,117.6000) -- (109.4000,117.6000) -- (109.6000,117.6000) -- (109.7000,117.5000) -- (109.9000,117.5000) -- (110.0000,117.5000) -- (110.2000,117.5000) -- (110.3000,117.5000) -- (110.4000,117.5000) -- (110.6000,117.5000) -- (110.7000,117.5000) -- (110.9000,117.5000) -- (111.0000,117.5000) -- (111.2000,117.5000) -- (111.3000,117.5000) -- (111.4000,117.5000) -- (111.6000,117.5000) -- (111.7000,117.5000) -- (111.9000,117.5000) -- (112.0000,117.5000) -- (112.2000,117.5000) -- (112.3000,117.4000) -- (112.4000,117.4000) -- (112.6000,117.4000) -- (112.7000,117.4000) -- (112.9000,117.4000) -- (113.0000,117.4000) -- (113.2000,117.4000) -- (113.3000,117.4000) -- (113.4000,117.4000) -- (113.6000,117.4000) -- (113.7000,117.4000) -- (113.9000,117.4000) -- (114.0000,117.4000) -- (114.2000,117.4000) -- (114.3000,117.4000) -- (114.4000,117.4000) -- (114.6000,117.4000) -- (114.7000,117.4000) -- (114.9000,117.4000) -- (115.0000,117.4000) -- (115.2000,117.4000) -- (115.3000,117.4000) -- (115.4000,117.4000) -- (115.6000,117.4000) -- (115.7000,117.4000) -- (115.9000,117.4000) -- (116.0000,117.4000) -- (116.2000,117.4000) -- (116.3000,117.4000) -- (116.4000,117.4000) -- (116.6000,117.4000) -- (116.7000,117.4000) -- (116.9000,117.5000) -- (117.0000,117.5000) -- (117.2000,117.5000) -- (117.3000,117.5000) -- (117.4000,117.5000) -- (117.6000,117.5000) -- (117.7000,117.5000) -- (117.9000,117.5000) -- (118.0000,117.5000) -- (118.2000,117.5000) -- (118.3000,117.5000) -- (118.4000,117.5000) -- (118.6000,117.5000) -- (118.7000,117.5000) -- (118.9000,117.5000) -- (119.0000,117.5000) -- (119.2000,117.5000) -- (119.3000,117.5000) -- (119.4000,117.5000) -- (119.6000,117.5000) -- (119.7000,117.5000) -- (119.9000,117.5000) -- (120.0000,117.5000) -- (120.2000,117.5000) -- (120.3000,117.5000) -- (120.4000,117.5000) -- (120.6000,117.5000) -- (120.7000,117.5000) -- (120.9000,117.5000) -- (121.0000,117.5000) -- (121.2000,117.5000) -- (121.3000,117.5000) -- (121.4000,117.5000) -- (121.6000,117.5000) -- (121.7000,117.5000) -- (121.9000,117.5000) -- (122.0000,117.5000) -- (122.2000,117.5000) -- (122.3000,117.5000) -- (122.4000,117.5000) -- (122.6000,117.5000) -- (122.7000,117.5000) -- (122.9000,117.5000) -- (123.0000,117.5000) -- (123.2000,117.5000) -- (123.3000,117.5000) -- (123.4000,117.5000) -- (123.6000,117.5000) -- (123.7000,117.5000) -- (123.9000,117.6000) -- (124.0000,117.6000) -- (124.2000,117.6000) -- (124.3000,117.6000) -- (124.4000,117.6000) -- (124.6000,117.6000) -- (124.7000,117.6000) -- (124.9000,117.6000) -- (125.0000,117.6000) -- (125.2000,117.6000) -- (125.3000,117.6000) -- (125.4000,117.6000) -- (125.6000,117.6000) -- (125.7000,117.6000) -- (125.9000,117.6000) -- (126.0000,117.6000) -- (126.2000,117.6000) -- (126.3000,117.6000) -- (126.4000,117.6000) -- (126.6000,117.6000) -- (126.7000,117.6000) -- (126.9000,117.6000) -- (127.0000,117.6000) -- (127.2000,117.6000) -- (127.3000,117.6000);



    \end{scope}
    \begin{scope}[cm={{1.21653,0.0,0.0,1.34548,(-346.94368,-156.28627)}},fill=cffffff]
      \path[fill=cd9d9d9,rounded corners=0.0000cm] (81.0000,129.0000) rectangle (121.0000,157.0000);



    \end{scope}
    \begin{scope}[cm={{1.21653,0.0,0.0,1.34548,(-346.94368,-156.28627)}},draw=ca0a0a4,dash pattern=on 1.40pt off 1.40pt,line cap=round,line join=round,line width=0.233pt,miter limit=4.00]
      \path[draw,dash pattern=on 1.40pt off 1.40pt,line width=0.233pt,miter limit=4.00] (81.5000,135.5000) -- (121.5000,135.5000);



    \end{scope}
    \begin{scope}[cm={{1.21653,0.0,0.0,1.34548,(-346.94368,-156.28627)}},draw=blue,line cap=round,line join=round,line width=0.480pt]
      \path[draw] (81.5000,135.5000) -- (82.6460,135.5000);



      \path[draw] (121.5000,135.5000) -- (120.1220,135.5000);



    \end{scope}
    \begin{scope}[cm={{1.00174,0.0,0.0,1.01515,(-276.4622,28.89387)}},draw=blue,line cap=rect,line join=bevel,line width=0.800pt]
      \path[fill=blue] (0.0000,0.0000) node[above right] (text766) {\scriptsize $T$\hspace{.5ex}=\hspace{.5ex}47};



    \end{scope}
    \begin{scope}[cm={{1.21653,0.0,0.0,1.34548,(-346.94368,-156.28627)}},draw=ca0a0a4,dash pattern=on 1.40pt off 1.40pt,line cap=round,line join=round,line width=0.233pt,miter limit=4.00]
      \path[draw,dash pattern=on 1.40pt off 1.40pt,line width=0.233pt,miter limit=4.00] (97.5000,157.5000) -- (97.5000,129.5000);



    \end{scope}
    \begin{scope}[cm={{1.21653,0.0,0.0,1.34548,(-346.94368,-156.28627)}},draw=blue,line cap=round,line join=round,line width=0.480pt]
      \path[draw] (97.5000,129.5000) -- (97.5000,129.5000) -- (97.5000,130.6564);



    \end{scope}
    \begin{scope}[cm={{1.00174,0.0,0.0,1.01515,(-232.9405,15.55865)}},draw=blue,line cap=rect,line join=bevel,line width=0.800pt]
      \path[fill=blue] (0.0000,0.0000) node[above right] (text794) {\scriptsize 84};



    \end{scope}
    \begin{scope}[cm={{1.21653,0.0,0.0,1.34548,(-346.94368,-156.28627)}},draw=blue,line cap=round,line join=round,line width=0.480pt]
      \path[draw] (81.5000,129.5000) -- (81.5000,157.5000) -- (121.5000,157.5000) -- (121.5000,129.5000) -- (81.5000,129.5000);



    \end{scope}
    \begin{scope}[cm={{1.21653,0.0,0.0,1.34548,(-346.94368,-156.28627)}},draw=blue,line cap=round,line join=round,line width=0.480pt]
      \path[draw] (81.2000,156.5000) -- (81.2000,156.5000) -- (81.2000,156.5000) -- (81.2000,156.5000) -- (81.2000,156.5000) -- (81.2000,156.5000) -- (81.2000,156.5000) -- (81.2000,156.5000) -- (81.2000,156.5000) -- (81.2000,156.5000) -- (81.2000,156.5000) -- (81.2000,156.5000) -- (81.2000,156.5000) -- (81.2000,156.5000) -- (81.2000,156.5000) -- (81.2000,156.5000) -- (81.2000,156.5000) -- (81.2000,156.5000) -- (81.2000,156.5000) -- (81.2000,156.5000) -- (81.2000,156.4000) -- (81.2000,156.4000) -- (81.2000,156.4000) -- (81.2000,156.4000) -- (81.2000,156.4000) -- (81.2000,156.4000) -- (81.3000,156.4000) -- (81.3000,156.4000) -- (81.3000,156.4000) -- (81.3000,156.4000) -- (81.3000,156.4000) -- (81.3000,156.4000) -- (81.3000,156.4000) -- (81.3000,156.4000) -- (81.3000,156.4000) -- (81.3000,156.4000) -- (81.3000,156.4000) -- (81.3000,156.4000) -- (81.3000,156.4000) -- (81.3000,156.4000) -- (81.3000,156.4000) -- (81.3000,156.3000) -- (81.3000,156.3000) -- (81.3000,156.3000) -- (81.3000,156.3000) -- (81.3000,156.3000) -- (81.3000,156.3000) -- (81.3000,156.3000) -- (81.3000,156.3000) -- (81.3000,156.3000) -- (81.3000,156.3000) -- (81.3000,156.3000) -- (81.3000,156.3000) -- (81.3000,156.3000) -- (81.3000,156.3000) -- (81.3000,156.3000) -- (81.3000,156.3000) -- (81.3000,156.3000) -- (81.3000,156.3000) -- (81.3000,156.3000) -- (81.3000,156.3000) -- (81.3000,156.3000) -- (81.3000,156.3000) -- (81.3000,156.2000) -- (81.3000,156.2000) -- (81.3000,156.2000) -- (81.3000,156.2000) -- (81.3000,156.2000) -- (81.3000,156.2000) -- (81.3000,156.2000) -- (81.3000,156.2000) -- (81.3000,156.2000) -- (81.3000,156.2000) -- (81.3000,156.2000) -- (81.3000,156.2000) -- (81.3000,156.2000) -- (81.4000,156.2000) -- (81.4000,156.2000) -- (81.4000,156.2000) -- (81.4000,156.2000) -- (81.4000,156.2000) -- (81.4000,156.2000) -- (81.4000,156.2000) -- (81.4000,156.2000) -- (81.4000,156.1000) -- (81.4000,156.1000) -- (81.4000,156.1000) -- (81.4000,156.1000) -- (81.4000,156.1000) -- (81.4000,156.1000) -- (81.4000,156.1000) -- (81.4000,156.1000) -- (81.4000,156.1000) -- (81.4000,156.1000) -- (81.4000,156.1000) -- (81.4000,156.1000) -- (81.4000,156.1000) -- (81.4000,156.1000) -- (81.4000,156.1000) -- (81.4000,156.1000) -- (81.4000,156.1000) -- (81.4000,156.1000) -- (81.4000,156.1000) -- (81.4000,156.1000) -- (81.4000,156.1000) -- (81.4000,156.1000) -- (81.4000,156.0000) -- (81.4000,156.0000) -- (81.4000,156.0000) -- (81.4000,156.0000) -- (81.4000,156.0000) -- (81.4000,156.0000) -- (81.4000,156.0000) -- (81.4000,156.0000) -- (81.4000,156.0000) -- (81.4000,156.0000) -- (81.4000,156.0000) -- (81.4000,156.0000) -- (81.4000,156.0000) -- (81.4000,156.0000) -- (81.4000,156.0000) -- (81.4000,156.0000) -- (81.4000,156.0000) -- (81.4000,156.0000) -- (81.4000,156.0000) -- (81.5000,156.0000) -- (81.5000,156.0000) -- (81.5000,155.9000) -- (81.5000,155.9000) -- (81.5000,155.9000) -- (81.5000,155.9000) -- (81.5000,155.9000) -- (81.5000,155.9000) -- (81.5000,155.9000) -- (81.5000,155.9000) -- (81.5000,155.9000) -- (81.5000,155.9000) -- (81.5000,155.9000) -- (81.5000,155.9000) -- (81.5000,155.9000) -- (81.5000,155.9000) -- (81.5000,155.9000) -- (81.5000,155.9000) -- (81.5000,155.9000) -- (81.5000,155.9000) -- (81.5000,155.9000) -- (81.5000,155.9000) -- (81.5000,155.9000) -- (81.5000,155.9000) -- (81.5000,155.8000) -- (81.5000,155.8000) -- (81.5000,155.8000) -- (81.5000,155.8000) -- (81.5000,155.8000) -- (81.5000,155.8000) -- (81.5000,155.8000) -- (81.5000,155.8000) -- (81.5000,155.8000) -- (81.5000,155.8000) -- (81.5000,155.8000) -- (81.5000,155.8000) -- (81.5000,155.8000) -- (81.5000,155.8000) -- (81.5000,155.8000) -- (81.5000,155.8000) -- (81.5000,155.8000) -- (81.5000,155.8000) -- (81.5000,155.8000) -- (81.5000,155.8000) -- (81.5000,155.8000) -- (81.5000,155.7000) -- (81.5000,155.7000) -- (81.5000,155.7000) -- (81.5000,155.7000) -- (81.5000,155.7000) -- (81.6000,155.7000) -- (81.6000,155.7000) -- (81.6000,155.7000) -- (81.6000,155.7000) -- (81.6000,155.7000) -- (81.6000,155.7000) -- (81.6000,155.7000) -- (81.6000,155.7000) -- (81.6000,155.7000) -- (81.6000,155.7000) -- (81.6000,155.7000) -- (81.6000,155.7000) -- (81.6000,155.7000) -- (81.6000,155.7000) -- (81.6000,155.7000) -- (81.6000,155.7000) -- (81.6000,155.7000) -- (81.6000,155.6000) -- (81.6000,155.6000) -- (81.6000,155.6000) -- (81.6000,155.6000) -- (81.6000,155.6000) -- (81.6000,155.6000) -- (81.6000,155.6000) -- (81.6000,155.6000) -- (81.6000,155.6000) -- (81.6000,155.6000) -- (81.6000,155.6000) -- (81.6000,155.6000) -- (81.6000,155.6000) -- (81.6000,155.6000) -- (81.6000,155.6000) -- (81.6000,155.6000) -- (81.6000,155.6000) -- (81.6000,155.6000) -- (81.6000,155.6000) -- (81.6000,155.6000) -- (81.6000,155.6000) -- (81.6000,155.5000) -- (81.6000,155.5000) -- (81.6000,155.5000) -- (81.6000,155.5000) -- (81.6000,155.5000) -- (81.6000,155.5000) -- (81.6000,155.5000) -- (81.6000,155.5000) -- (81.6000,155.5000) -- (81.6000,155.5000) -- (81.6000,155.5000) -- (81.7000,155.5000) -- (81.7000,155.5000) -- (81.7000,155.5000) -- (81.7000,155.5000) -- (81.7000,155.5000) -- (81.7000,155.5000) -- (81.7000,155.5000) -- (81.7000,155.5000) -- (81.7000,155.5000) -- (81.7000,155.5000) -- (81.7000,155.5000) -- (81.7000,155.4000) -- (81.7000,155.4000) -- (81.7000,155.4000) -- (81.7000,155.4000) -- (81.7000,155.4000) -- (81.7000,155.4000) -- (81.7000,155.4000) -- (81.7000,155.4000) -- (81.7000,155.4000) -- (81.7000,155.4000) -- (81.7000,155.4000) -- (81.7000,155.4000) -- (81.7000,155.4000) -- (81.7000,155.4000) -- (81.7000,155.4000) -- (81.7000,155.4000) -- (81.7000,155.4000) -- (81.7000,155.4000) -- (81.7000,155.4000) -- (81.7000,155.4000) -- (81.7000,155.4000) -- (81.7000,155.3000) -- (81.7000,155.3000) -- (81.7000,155.3000) -- (81.7000,155.3000) -- (81.7000,155.3000) -- (81.7000,155.3000) -- (81.7000,155.3000) -- (81.7000,155.3000) -- (81.7000,155.3000) -- (81.7000,155.3000) -- (81.7000,155.3000) -- (81.7000,155.3000) -- (81.7000,155.3000) -- (81.7000,155.3000) -- (81.7000,155.3000) -- (81.7000,155.3000) -- (81.7000,155.3000) -- (81.7000,155.3000) -- (81.8000,155.3000) -- (81.8000,155.3000) -- (81.8000,155.3000) -- (81.8000,155.3000) -- (81.8000,155.2000) -- (81.8000,155.2000) -- (81.8000,155.2000) -- (81.8000,155.2000) -- (81.8000,155.2000) -- (81.8000,155.2000) -- (81.8000,155.2000) -- (81.8000,155.2000) -- (81.8000,155.2000) -- (81.8000,155.2000) -- (81.8000,155.2000) -- (81.8000,155.2000) -- (81.8000,155.2000) -- (81.8000,155.2000) -- (81.8000,155.2000) -- (81.8000,155.2000) -- (81.8000,155.2000) -- (81.8000,155.2000) -- (81.8000,155.2000) -- (81.8000,155.2000) -- (81.8000,155.2000) -- (81.8000,155.1000) -- (81.8000,155.1000) -- (81.8000,155.1000) -- (81.8000,155.1000) -- (81.8000,155.1000) -- (81.8000,155.1000) -- (81.8000,155.1000) -- (81.8000,155.1000) -- (81.8000,155.1000) -- (81.8000,155.1000) -- (81.8000,155.1000) -- (81.8000,155.1000) -- (81.8000,155.1000) -- (81.8000,155.1000) -- (81.8000,155.1000) -- (81.8000,155.1000) -- (81.8000,155.1000) -- (81.8000,155.1000) -- (81.8000,155.1000) -- (81.8000,155.1000) -- (81.8000,155.1000) -- (81.8000,155.1000) -- (81.8000,155.0000) -- (81.8000,155.0000) -- (81.9000,155.0000) -- (81.9000,155.0000) -- (81.9000,155.0000) -- (81.9000,155.0000) -- (81.9000,155.0000) -- (81.9000,155.0000) -- (81.9000,155.0000) -- (81.9000,155.0000) -- (81.9000,155.0000) -- (81.9000,155.0000) -- (81.9000,155.0000) -- (81.9000,155.0000) -- (81.9000,155.0000) -- (81.9000,155.0000) -- (81.9000,155.0000) -- (81.9000,155.0000) -- (81.9000,155.0000) -- (81.9000,155.0000) -- (81.9000,155.0000) -- (81.9000,154.9000) -- (81.9000,154.9000) -- (81.9000,154.9000) -- (81.9000,154.9000) -- (81.9000,154.9000) -- (81.9000,154.9000) -- (81.9000,154.9000) -- (81.9000,154.9000) -- (81.9000,154.9000) -- (81.9000,154.9000) -- (81.9000,154.9000) -- (81.9000,154.9000) -- (81.9000,154.9000) -- (81.9000,154.9000) -- (81.9000,154.9000) -- (81.9000,154.9000) -- (81.9000,154.9000) -- (81.9000,154.9000) -- (81.9000,154.9000) -- (81.9000,154.9000) -- (81.9000,154.9000) -- (81.9000,154.9000) -- (81.9000,154.8000) -- (81.9000,154.8000) -- (81.9000,154.8000) -- (81.9000,154.8000) -- (81.9000,154.8000) -- (81.9000,154.8000) -- (81.9000,154.8000) -- (81.9000,154.8000) -- (81.9000,154.8000) -- (82.0000,154.8000) -- (82.0000,154.8000) -- (82.0000,154.8000) -- (82.0000,154.8000) -- (82.0000,154.8000) -- (82.0000,154.8000) -- (82.0000,154.8000) -- (82.0000,154.8000) -- (82.0000,154.8000) -- (82.0000,154.8000) -- (82.0000,154.8000) -- (82.0000,154.8000) -- (82.0000,154.7000) -- (82.0000,154.7000) -- (82.0000,154.7000) -- (82.0000,154.7000) -- (82.0000,154.7000) -- (82.0000,154.7000) -- (82.0000,154.7000) -- (82.0000,154.7000) -- (82.0000,154.7000) -- (82.0000,154.7000) -- (82.0000,154.7000) -- (82.0000,154.7000) -- (82.0000,154.7000) -- (82.0000,154.7000) -- (82.0000,154.7000) -- (82.0000,154.7000) -- (82.0000,154.7000) -- (82.0000,154.7000) -- (82.0000,154.7000) -- (82.0000,154.7000) -- (82.0000,154.7000) -- (82.0000,154.7000) -- (82.0000,154.6000) -- (82.0000,154.6000) -- (82.0000,154.6000) -- (82.0000,154.6000) -- (82.0000,154.6000) -- (82.0000,154.6000) -- (82.0000,154.6000) -- (82.0000,154.6000) -- (82.0000,154.6000) -- (82.0000,154.6000) -- (82.0000,154.6000) -- (82.0000,154.6000) -- (82.0000,154.6000) -- (82.0000,154.6000) -- (82.0000,154.6000) -- (82.1000,154.6000) -- (82.1000,154.6000) -- (82.1000,154.6000) -- (82.1000,154.6000) -- (82.1000,154.6000) -- (82.1000,154.6000) -- (82.1000,154.5000) -- (82.1000,154.5000) -- (82.1000,154.5000) -- (82.1000,154.5000) -- (82.1000,154.5000) -- (82.1000,154.5000) -- (82.1000,154.5000) -- (82.1000,154.5000) -- (82.1000,154.5000) -- (82.1000,154.5000) -- (82.1000,154.5000) -- (82.1000,154.5000) -- (82.1000,154.5000) -- (82.1000,154.5000) -- (82.1000,154.5000) -- (82.1000,154.5000) -- (82.1000,154.5000) -- (82.1000,154.5000) -- (82.1000,154.5000) -- (82.1000,154.5000) -- (82.1000,154.5000) -- (82.1000,154.5000) -- (82.1000,154.4000) -- (82.1000,154.4000) -- (82.1000,154.4000) -- (82.1000,154.4000) -- (82.1000,154.4000) -- (82.1000,154.4000) -- (82.1000,154.4000) -- (82.1000,154.4000) -- (82.1000,154.4000) -- (82.1000,154.4000) -- (82.1000,154.4000) -- (82.1000,154.4000) -- (82.1000,154.4000) -- (82.1000,154.4000) -- (82.1000,154.4000) -- (82.1000,154.4000) -- (82.1000,154.4000) -- (82.1000,154.4000) -- (82.1000,154.4000) -- (82.1000,154.4000) -- (82.1000,154.4000) -- (82.1000,154.3000) -- (82.2000,154.3000) -- (82.2000,154.3000) -- (82.2000,154.3000) -- (82.2000,154.3000) -- (82.2000,154.3000) -- (82.2000,154.3000) -- (82.2000,154.3000) -- (82.2000,154.3000) -- (82.2000,154.3000) -- (82.2000,154.3000) -- (82.2000,154.3000) -- (82.2000,154.3000) -- (82.2000,154.3000) -- (82.2000,154.3000) -- (82.2000,154.3000) -- (82.2000,154.3000) -- (82.2000,154.3000) -- (82.2000,154.3000) -- (82.2000,154.3000) -- (82.2000,154.3000) -- (82.2000,154.3000) -- (82.2000,154.2000) -- (82.2000,154.2000) -- (82.2000,154.2000) -- (82.2000,154.2000) -- (82.2000,154.2000) -- (82.2000,154.2000) -- (82.2000,154.2000) -- (82.2000,154.2000) -- (82.2000,154.2000) -- (82.2000,154.2000) -- (82.2000,154.2000) -- (82.2000,154.2000) -- (82.2000,154.2000) -- (82.2000,154.2000) -- (82.2000,154.2000) -- (82.2000,154.2000) -- (82.2000,154.2000) -- (82.2000,154.2000) -- (82.2000,154.2000) -- (82.2000,154.2000) -- (82.2000,154.2000) -- (82.2000,154.1000) -- (82.2000,154.1000) -- (82.2000,154.1000) -- (82.2000,154.1000) -- (82.2000,154.1000) -- (82.2000,154.1000) -- (82.2000,154.1000) -- (82.3000,154.1000) -- (82.3000,154.1000) -- (82.3000,154.1000) -- (82.3000,154.1000) -- (82.3000,154.1000) -- (82.3000,154.1000) -- (82.3000,154.1000) -- (82.3000,154.1000) -- (82.3000,154.1000) -- (82.3000,154.1000) -- (82.3000,154.1000) -- (82.3000,154.1000) -- (82.3000,154.1000) -- (82.3000,154.1000) -- (82.3000,154.1000) -- (82.3000,154.0000) -- (82.3000,154.0000) -- (82.3000,154.0000) -- (82.3000,154.0000) -- (82.3000,154.0000) -- (82.3000,154.0000) -- (82.3000,154.0000) -- (82.3000,154.0000) -- (82.3000,154.0000) -- (82.3000,154.0000) -- (82.3000,154.0000) -- (82.3000,154.0000) -- (82.3000,154.0000) -- (82.3000,154.0000) -- (82.3000,154.0000) -- (82.3000,154.0000) -- (82.3000,154.0000) -- (82.3000,154.0000) -- (82.3000,154.0000) -- (82.3000,154.0000) -- (82.3000,154.0000) -- (82.3000,153.9000) -- (82.3000,153.9000) -- (82.3000,153.9000) -- (82.3000,153.9000) -- (82.3000,153.9000) -- (82.3000,153.9000) -- (82.3000,153.9000) -- (82.3000,153.9000) -- (82.3000,153.9000) -- (82.3000,153.9000) -- (82.3000,153.9000) -- (82.3000,153.9000) -- (82.3000,153.9000) -- (82.3000,153.9000) -- (82.4000,153.9000) -- (82.4000,153.9000) -- (82.4000,153.9000) -- (82.4000,153.9000) -- (82.4000,153.9000) -- (82.4000,153.9000) -- (82.4000,153.9000) -- (82.4000,153.9000) -- (82.4000,153.8000) -- (82.4000,153.8000) -- (82.4000,153.8000) -- (82.4000,153.8000) -- (82.4000,153.8000) -- (82.4000,153.8000) -- (82.4000,153.8000) -- (82.4000,153.8000) -- (82.4000,153.8000) -- (82.4000,153.8000) -- (82.4000,153.8000) -- (82.4000,153.8000) -- (82.4000,153.8000) -- (82.4000,153.8000) -- (82.4000,153.8000) -- (82.4000,153.8000) -- (82.4000,153.8000) -- (82.4000,153.8000) -- (82.4000,153.8000) -- (82.4000,153.8000) -- (82.4000,153.8000) -- (82.4000,153.7000) -- (82.4000,153.7000) -- (82.4000,153.7000) -- (82.4000,153.7000) -- (82.4000,153.7000) -- (82.4000,153.7000) -- (82.4000,153.7000) -- (82.4000,153.7000) -- (82.4000,153.7000) -- (82.4000,153.7000) -- (82.4000,153.7000) -- (82.4000,153.7000) -- (82.4000,153.7000) -- (82.4000,153.7000) -- (82.4000,153.7000) -- (82.4000,153.7000) -- (82.4000,153.7000) -- (82.4000,153.7000) -- (82.4000,153.7000) -- (82.4000,153.7000) -- (82.5000,153.7000) -- (82.5000,153.7000) -- (82.5000,153.6000) -- (82.5000,153.6000) -- (82.5000,153.6000) -- (82.5000,153.6000) -- (82.5000,153.6000) -- (82.5000,153.6000) -- (82.5000,153.6000) -- (82.5000,153.6000) -- (82.5000,153.6000) -- (82.5000,153.6000) -- (82.5000,153.6000) -- (82.5000,153.6000) -- (82.5000,153.6000) -- (82.5000,153.6000) -- (82.5000,153.6000) -- (82.5000,153.6000) -- (82.5000,153.6000) -- (82.5000,153.6000) -- (82.5000,153.6000) -- (82.5000,153.6000) -- (82.5000,153.6000) -- (82.5000,153.5000) -- (82.5000,153.5000) -- (82.5000,153.5000) -- (82.5000,153.5000) -- (82.5000,153.5000) -- (82.5000,153.5000) -- (82.5000,153.5000) -- (82.5000,153.5000) -- (82.5000,153.5000) -- (82.5000,153.5000) -- (82.5000,153.5000) -- (82.5000,153.5000) -- (82.5000,153.5000) -- (82.5000,153.5000) -- (82.5000,153.5000) -- (82.5000,153.5000) -- (82.5000,153.5000) -- (82.5000,153.5000) -- (82.5000,153.5000) -- (82.5000,153.5000) -- (82.5000,153.5000) -- (82.5000,153.5000) -- (82.5000,153.4000) -- (82.5000,153.4000) -- (82.5000,153.4000) -- (82.5000,153.4000) -- (82.5000,153.4000) -- (82.6000,153.4000) -- (82.6000,153.4000) -- (82.6000,153.4000) -- (82.6000,153.4000) -- (82.6000,153.4000) -- (82.6000,153.4000) -- (82.6000,153.4000) -- (82.6000,153.4000) -- (82.6000,153.4000) -- (82.6000,153.4000) -- (82.6000,153.4000) -- (82.6000,153.4000) -- (82.6000,153.4000) -- (82.6000,153.4000) -- (82.6000,153.4000) -- (82.6000,153.4000) -- (82.6000,153.3000) -- (82.6000,153.3000) -- (82.6000,153.3000) -- (82.6000,153.3000) -- (82.6000,153.3000) -- (82.6000,153.3000) -- (82.6000,153.3000) -- (82.6000,153.3000) -- (82.6000,153.3000) -- (82.6000,153.3000) -- (82.6000,153.3000) -- (82.6000,153.3000) -- (82.6000,153.3000) -- (82.6000,153.3000) -- (82.6000,153.3000) -- (82.6000,153.3000) -- (82.6000,153.3000) -- (82.6000,153.3000) -- (82.6000,153.3000) -- (82.6000,153.3000) -- (82.6000,153.3000) -- (82.6000,153.3000) -- (82.6000,153.2000) -- (82.6000,153.2000) -- (82.6000,153.2000) -- (82.6000,153.2000) -- (82.6000,153.2000) -- (82.6000,153.2000) -- (82.6000,153.2000) -- (82.6000,153.2000) -- (82.6000,153.2000) -- (82.6000,153.2000) -- (82.6000,153.2000) -- (82.7000,153.2000) -- (82.7000,153.2000) -- (82.7000,153.2000) -- (82.7000,153.2000) -- (82.7000,153.2000) -- (82.7000,153.2000) -- (82.7000,153.2000) -- (82.7000,153.2000) -- (82.7000,153.2000) -- (82.7000,153.2000) -- (82.7000,153.1000) -- (82.7000,153.1000) -- (82.7000,153.1000) -- (82.7000,153.1000) -- (82.7000,153.1000) -- (82.7000,153.1000) -- (82.7000,153.1000) -- (82.7000,153.1000) -- (82.7000,153.1000) -- (82.7000,153.1000) -- (82.7000,153.1000) -- (82.7000,153.1000) -- (82.7000,153.1000) -- (82.7000,153.1000) -- (82.7000,153.1000) -- (82.7000,153.1000) -- (82.7000,153.1000) -- (82.7000,153.1000) -- (82.7000,153.1000) -- (82.7000,153.1000) -- (82.7000,153.1000) -- (82.7000,153.1000) -- (82.7000,153.0000) -- (82.7000,153.0000) -- (82.7000,153.0000) -- (82.7000,153.0000) -- (82.7000,153.0000) -- (82.7000,153.0000) -- (82.7000,153.0000) -- (82.7000,153.0000) -- (82.7000,153.0000) -- (82.7000,153.0000) -- (82.7000,153.0000) -- (82.7000,153.0000) -- (82.7000,153.0000) -- (82.7000,153.0000) -- (82.7000,153.0000) -- (82.7000,153.0000) -- (82.7000,153.0000) -- (82.7000,153.0000) -- (82.8000,153.0000) -- (82.8000,153.0000) -- (82.8000,153.0000) -- (82.8000,152.9000) -- (82.8000,152.9000) -- (82.8000,152.9000) -- (82.8000,152.9000) -- (82.8000,152.9000) -- (82.8000,152.9000) -- (82.8000,152.9000) -- (82.8000,152.9000) -- (82.8000,152.9000) -- (82.8000,152.9000) -- (82.8000,152.9000) -- (82.8000,152.9000) -- (82.8000,152.9000) -- (82.8000,152.9000) -- (82.8000,152.9000) -- (82.8000,152.9000) -- (82.8000,152.9000) -- (82.8000,152.9000) -- (82.8000,152.9000) -- (82.8000,152.9000) -- (82.8000,152.9000) -- (82.8000,152.9000) -- (82.8000,152.8000) -- (82.8000,152.8000) -- (82.8000,152.8000) -- (82.8000,152.8000) -- (82.8000,152.8000) -- (82.8000,152.8000) -- (82.8000,152.8000) -- (82.8000,152.8000) -- (82.8000,152.8000) -- (82.8000,152.8000) -- (82.8000,152.8000) -- (82.8000,152.8000) -- (82.8000,152.8000) -- (82.8000,152.8000) -- (82.8000,152.8000) -- (82.8000,152.8000) -- (82.8000,152.8000) -- (82.8000,152.8000) -- (82.8000,152.8000) -- (82.8000,152.8000) -- (82.8000,152.8000) -- (82.8000,152.7000) -- (82.8000,152.7000) -- (82.8000,152.7000) -- (82.9000,152.7000) -- (82.9000,152.7000) -- (82.9000,152.7000) -- (82.9000,152.7000) -- (82.9000,152.7000) -- (82.9000,152.7000) -- (82.9000,152.7000) -- (82.9000,152.7000) -- (82.9000,152.7000) -- (82.9000,152.7000) -- (82.9000,152.7000) -- (82.9000,152.7000) -- (82.9000,152.7000) -- (82.9000,152.7000) -- (82.9000,152.7000) -- (82.9000,152.7000) -- (82.9000,152.7000) -- (82.9000,152.7000) -- (82.9000,152.7000) -- (82.9000,152.6000) -- (82.9000,152.6000) -- (82.9000,152.6000) -- (82.9000,152.6000) -- (82.9000,152.6000) -- (82.9000,152.6000) -- (82.9000,152.6000) -- (82.9000,152.6000) -- (82.9000,152.6000) -- (82.9000,152.6000) -- (82.9000,152.6000) -- (82.9000,152.6000) -- (82.9000,152.6000) -- (82.9000,152.6000) -- (82.9000,152.6000) -- (82.9000,152.6000) -- (82.9000,152.6000) -- (82.9000,152.6000) -- (82.9000,152.6000) -- (82.9000,152.6000) -- (82.9000,152.6000) -- (82.9000,152.5000) -- (82.9000,152.5000) -- (82.9000,152.5000) -- (82.9000,152.5000) -- (82.9000,152.5000) -- (82.9000,152.5000) -- (82.9000,152.5000) -- (82.9000,152.5000) -- (82.9000,152.5000) -- (82.9000,152.5000) -- (83.0000,152.5000) -- (83.0000,152.5000) -- (83.0000,152.5000) -- (83.0000,152.5000) -- (83.0000,152.5000) -- (83.0000,152.5000) -- (83.0000,152.5000) -- (83.0000,152.5000) -- (83.0000,152.5000) -- (83.0000,152.5000) -- (83.0000,152.5000) -- (83.0000,152.5000) -- (83.0000,152.4000) -- (83.0000,152.4000) -- (83.0000,152.4000) -- (83.0000,152.4000) -- (83.0000,152.4000) -- (83.0000,152.4000) -- (83.0000,152.4000) -- (83.0000,152.4000) -- (83.0000,152.4000) -- (83.0000,152.4000) -- (83.0000,152.4000) -- (83.0000,152.4000) -- (83.0000,152.4000) -- (83.0000,152.4000) -- (83.0000,152.4000) -- (83.0000,152.4000) -- (83.0000,152.4000) -- (83.0000,152.4000) -- (83.0000,152.4000) -- (83.0000,152.4000) -- (83.0000,152.4000) -- (83.0000,152.3000) -- (83.0000,152.3000) -- (83.0000,152.3000) -- (83.0000,152.3000) -- (83.0000,152.3000) -- (83.0000,152.3000) -- (83.0000,152.3000) -- (83.0000,152.3000) -- (83.0000,152.3000) -- (83.0000,152.3000) -- (83.0000,152.3000) -- (83.0000,152.3000) -- (83.0000,152.3000) -- (83.0000,152.3000) -- (83.0000,152.3000) -- (83.0000,152.3000) -- (83.1000,152.3000) -- (83.1000,152.3000) -- (83.1000,152.3000) -- (83.1000,152.3000) -- (83.1000,152.3000) -- (83.1000,152.3000) -- (83.1000,152.2000) -- (83.1000,152.2000) -- (83.1000,152.2000) -- (83.1000,152.2000) -- (83.1000,152.2000) -- (83.1000,152.2000) -- (83.1000,152.2000) -- (83.1000,152.2000) -- (83.1000,152.2000) -- (83.1000,152.2000) -- (83.1000,152.2000) -- (83.1000,152.2000) -- (83.1000,152.2000) -- (83.1000,152.2000) -- (83.1000,152.2000) -- (83.1000,152.2000) -- (83.1000,152.2000) -- (83.1000,152.2000) -- (83.1000,152.2000) -- (83.1000,152.2000) -- (83.1000,152.2000) -- (83.1000,152.2000) -- (83.1000,152.1000) -- (83.1000,152.1000) -- (83.1000,152.1000) -- (83.1000,152.1000) -- (83.1000,152.1000) -- (83.1000,152.1000) -- (83.1000,152.1000) -- (83.1000,152.1000) -- (83.1000,152.1000) -- (83.1000,152.1000) -- (83.1000,152.1000) -- (83.1000,152.1000) -- (83.1000,152.1000) -- (83.1000,152.1000) -- (83.1000,152.1000) -- (83.1000,152.1000) -- (83.1000,152.1000) -- (83.1000,152.1000) -- (83.1000,152.1000) -- (83.1000,152.1000) -- (83.1000,152.1000) -- (83.1000,152.0000) -- (83.2000,152.0000) -- (83.2000,152.0000) -- (83.2000,152.0000) -- (83.2000,152.0000) -- (83.2000,152.0000) -- (83.2000,152.0000) -- (83.2000,152.0000) -- (83.2000,152.0000) -- (83.2000,152.0000) -- (83.2000,152.0000) -- (83.2000,152.0000) -- (83.2000,152.0000) -- (83.2000,152.0000) -- (83.2000,152.0000) -- (83.2000,152.0000) -- (83.2000,152.0000) -- (83.2000,152.0000) -- (83.2000,152.0000) -- (83.2000,152.0000) -- (83.2000,152.0000) -- (83.2000,152.0000) -- (83.2000,151.9000) -- (83.2000,151.9000) -- (83.2000,151.9000) -- (83.2000,151.9000) -- (83.2000,151.9000) -- (83.2000,151.9000) -- (83.2000,151.9000) -- (83.2000,151.9000) -- (83.2000,151.9000) -- (83.2000,151.9000) -- (83.2000,151.9000) -- (83.2000,151.9000) -- (83.2000,151.9000) -- (83.2000,151.9000) -- (83.2000,151.9000) -- (83.2000,151.9000) -- (83.2000,151.9000) -- (83.2000,151.9000) -- (83.2000,151.9000) -- (83.2000,151.9000) -- (83.2000,151.9000) -- (83.2000,151.8000) -- (83.2000,151.8000) -- (83.2000,151.8000) -- (83.2000,151.8000) -- (83.2000,151.8000) -- (83.2000,151.8000) -- (83.2000,151.8000) -- (83.3000,151.8000) -- (83.3000,151.8000) -- (83.3000,151.8000) -- (83.3000,151.8000) -- (83.3000,151.8000) -- (83.3000,151.8000) -- (83.3000,151.8000) -- (83.3000,151.8000) -- (83.3000,151.8000) -- (83.3000,151.8000) -- (83.3000,151.8000) -- (83.3000,151.8000) -- (83.3000,151.8000) -- (83.3000,151.8000) -- (83.3000,151.8000) -- (83.3000,151.7000) -- (83.3000,151.7000) -- (83.3000,151.7000) -- (83.3000,151.7000) -- (83.3000,151.7000) -- (83.3000,151.7000) -- (83.3000,151.7000) -- (83.3000,151.7000) -- (83.3000,151.7000) -- (83.3000,151.7000) -- (83.3000,151.7000) -- (83.3000,151.7000) -- (83.3000,151.7000) -- (83.3000,151.7000) -- (83.3000,151.7000) -- (83.3000,151.7000) -- (83.3000,151.7000) -- (83.3000,151.7000) -- (83.3000,151.7000) -- (83.3000,151.7000) -- (83.3000,151.7000) -- (83.3000,151.6000) -- (83.3000,151.6000) -- (83.3000,151.6000) -- (83.3000,151.6000) -- (83.3000,151.6000) -- (83.3000,151.6000) -- (83.3000,151.6000) -- (83.3000,151.6000) -- (83.3000,151.6000) -- (83.3000,151.6000) -- (83.3000,151.6000) -- (83.3000,151.6000) -- (83.3000,151.6000) -- (83.3000,151.6000) -- (83.4000,151.6000) -- (83.4000,151.6000) -- (83.4000,151.6000) -- (83.4000,151.6000) -- (83.4000,151.6000) -- (83.4000,151.6000) -- (83.4000,151.6000) -- (83.4000,151.6000) -- (83.4000,151.5000) -- (83.4000,151.5000) -- (83.4000,151.5000) -- (83.4000,151.5000) -- (83.4000,151.5000) -- (83.4000,151.5000) -- (83.4000,151.5000) -- (83.4000,151.5000) -- (83.4000,151.5000) -- (83.4000,151.5000) -- (83.4000,151.5000) -- (83.4000,151.5000) -- (83.4000,151.5000) -- (83.4000,151.5000) -- (83.4000,151.5000) -- (83.4000,151.5000) -- (83.4000,151.5000) -- (83.4000,151.5000) -- (83.4000,151.5000) -- (83.4000,151.5000) -- (83.4000,151.5000) -- (83.4000,151.4000) -- (83.4000,151.4000) -- (83.4000,151.4000) -- (83.4000,151.4000) -- (83.4000,151.4000) -- (83.4000,151.4000) -- (83.4000,151.4000) -- (83.4000,151.4000) -- (83.4000,151.4000) -- (83.4000,151.4000) -- (83.4000,151.4000) -- (83.4000,151.4000) -- (83.4000,151.4000) -- (83.4000,151.4000) -- (83.4000,151.4000) -- (83.4000,151.4000) -- (83.4000,151.4000) -- (83.4000,151.4000) -- (83.4000,151.4000) -- (83.4000,151.4000) -- (83.5000,151.4000) -- (83.5000,151.4000) -- (83.5000,151.3000) -- (83.5000,151.3000) -- (83.5000,151.3000) -- (83.5000,151.3000) -- (83.5000,151.3000) -- (83.5000,151.3000) -- (83.5000,151.3000) -- (83.5000,151.3000) -- (83.5000,151.3000) -- (83.5000,151.3000) -- (83.5000,151.3000) -- (83.5000,151.3000) -- (83.5000,151.3000) -- (83.5000,151.3000) -- (83.5000,151.3000) -- (83.5000,151.3000) -- (83.5000,151.3000) -- (83.5000,151.3000) -- (83.5000,151.3000) -- (83.5000,151.3000) -- (83.5000,151.3000) -- (83.5000,151.2000) -- (83.5000,151.2000) -- (83.5000,151.2000) -- (83.5000,151.2000) -- (83.5000,151.2000) -- (83.5000,151.2000) -- (83.5000,151.2000) -- (83.5000,151.2000) -- (83.5000,151.2000) -- (83.5000,151.2000) -- (83.5000,151.2000) -- (83.5000,151.2000) -- (83.5000,151.2000) -- (83.5000,151.2000) -- (83.5000,151.2000) -- (83.5000,151.2000) -- (83.5000,151.2000) -- (83.5000,151.2000) -- (83.5000,151.2000) -- (83.5000,151.2000) -- (83.5000,151.2000) -- (83.5000,151.2000) -- (83.5000,151.1000) -- (83.5000,151.1000) -- (83.5000,151.1000) -- (83.5000,151.1000) -- (83.5000,151.1000) -- (83.6000,151.1000) -- (83.6000,151.1000) -- (83.6000,151.1000) -- (83.6000,151.1000) -- (83.6000,151.1000) -- (83.6000,151.1000) -- (83.6000,151.1000) -- (83.6000,151.1000) -- (83.6000,151.1000) -- (83.6000,151.1000) -- (83.6000,151.1000) -- (83.6000,151.1000) -- (83.6000,151.1000) -- (83.6000,151.1000) -- (83.6000,151.1000) -- (83.6000,151.1000) -- (83.6000,151.0000) -- (83.6000,151.0000) -- (83.6000,151.0000) -- (83.6000,151.0000) -- (83.6000,151.0000) -- (83.6000,151.0000) -- (83.6000,151.0000) -- (83.6000,151.0000) -- (83.6000,151.0000) -- (83.6000,151.0000) -- (83.6000,151.0000) -- (83.6000,151.0000) -- (83.6000,151.0000) -- (83.6000,151.0000) -- (83.6000,151.0000) -- (83.6000,151.0000) -- (83.6000,151.0000) -- (83.6000,151.0000) -- (83.6000,151.0000) -- (83.6000,151.0000) -- (83.6000,151.0000) -- (83.6000,151.0000) -- (83.6000,150.9000) -- (83.6000,150.9000) -- (83.6000,150.9000) -- (83.6000,150.9000) -- (83.6000,150.9000) -- (83.6000,150.9000) -- (83.6000,150.9000) -- (83.6000,150.9000) -- (83.6000,150.9000) -- (83.6000,150.9000) -- (83.6000,150.9000) -- (83.7000,150.9000) -- (83.7000,150.9000) -- (83.7000,150.9000) -- (83.7000,150.9000) -- (83.7000,150.9000) -- (83.7000,150.9000) -- (83.7000,150.9000) -- (83.7000,150.9000) -- (83.7000,150.9000) -- (83.7000,150.9000) -- (83.7000,150.8000) -- (83.7000,150.8000) -- (83.7000,150.8000) -- (83.7000,150.8000) -- (83.7000,150.8000) -- (83.7000,150.8000) -- (83.7000,150.8000) -- (83.7000,150.8000) -- (83.7000,150.8000) -- (83.7000,150.8000) -- (83.7000,150.8000) -- (83.7000,150.8000) -- (83.7000,150.8000) -- (83.7000,150.8000) -- (83.7000,150.8000) -- (83.7000,150.8000) -- (83.7000,150.8000) -- (83.7000,150.8000) -- (83.7000,150.8000) -- (83.7000,150.8000) -- (83.7000,150.8000) -- (83.7000,150.8000) -- (83.7000,150.7000) -- (83.7000,150.7000) -- (83.7000,150.7000) -- (83.7000,150.7000) -- (83.7000,150.7000) -- (83.7000,150.7000) -- (83.7000,150.7000) -- (83.7000,150.7000) -- (83.7000,150.7000) -- (83.7000,150.7000) -- (83.7000,150.7000) -- (83.7000,150.7000) -- (83.7000,150.7000) -- (83.7000,150.7000) -- (83.7000,150.7000) -- (83.7000,150.7000) -- (83.7000,150.7000) -- (83.7000,150.7000) -- (83.8000,150.7000) -- (83.8000,150.7000) -- (83.8000,150.7000) -- (83.8000,150.6000) -- (83.8000,150.6000) -- (83.8000,150.6000) -- (83.8000,150.6000) -- (83.8000,150.6000) -- (83.8000,150.6000) -- (83.8000,150.6000) -- (83.8000,150.6000) -- (83.8000,150.6000) -- (83.8000,150.6000) -- (83.8000,150.6000) -- (83.8000,150.6000) -- (83.8000,150.6000) -- (83.8000,150.6000) -- (83.8000,150.6000) -- (83.8000,150.6000) -- (83.8000,150.6000) -- (83.8000,150.6000) -- (83.8000,150.6000) -- (83.8000,150.6000) -- (83.8000,150.6000) -- (83.8000,150.6000) -- (83.8000,150.5000) -- (83.8000,150.5000) -- (83.8000,150.5000) -- (83.8000,150.5000) -- (83.8000,150.5000) -- (83.8000,150.5000) -- (83.8000,150.5000) -- (83.8000,150.5000) -- (83.8000,150.5000) -- (83.8000,150.5000) -- (83.8000,150.5000) -- (83.8000,150.5000) -- (83.8000,150.5000) -- (83.8000,150.5000) -- (83.8000,150.5000) -- (83.8000,150.5000) -- (83.8000,150.5000) -- (83.8000,150.5000) -- (83.8000,150.5000) -- (83.8000,150.5000) -- (83.8000,150.5000) -- (83.8000,150.4000) -- (83.8000,150.4000) -- (83.8000,150.4000) -- (83.9000,150.4000) -- (83.9000,150.4000) -- (83.9000,150.4000) -- (83.9000,150.4000) -- (83.9000,150.4000) -- (83.9000,150.4000) -- (83.9000,150.4000) -- (83.9000,150.4000) -- (83.9000,150.4000) -- (83.9000,150.4000) -- (83.9000,150.4000) -- (83.9000,150.4000) -- (83.9000,150.4000) -- (83.9000,150.4000) -- (83.9000,150.4000) -- (83.9000,150.4000) -- (83.9000,150.4000) -- (83.9000,150.4000) -- (83.9000,150.4000) -- (83.9000,150.3000) -- (83.9000,150.3000) -- (83.9000,150.3000) -- (83.9000,150.3000) -- (83.9000,150.3000) -- (83.9000,150.3000) -- (83.9000,150.3000) -- (83.9000,150.3000) -- (83.9000,150.3000) -- (83.9000,150.3000) -- (83.9000,150.3000) -- (83.9000,150.3000) -- (83.9000,150.3000) -- (83.9000,150.3000) -- (83.9000,150.3000) -- (83.9000,150.3000) -- (83.9000,150.3000) -- (83.9000,150.3000) -- (83.9000,150.3000) -- (83.9000,150.3000) -- (83.9000,150.3000) -- (83.9000,150.2000) -- (83.9000,150.2000) -- (83.9000,150.2000) -- (83.9000,150.2000) -- (83.9000,150.2000) -- (83.9000,150.2000) -- (83.9000,150.2000) -- (83.9000,150.2000) -- (83.9000,150.2000) -- (83.9000,150.2000) -- (84.0000,150.2000) -- (84.0000,150.2000) -- (84.0000,150.2000) -- (84.0000,150.2000) -- (84.0000,150.2000) -- (84.0000,150.2000) -- (84.0000,150.2000) -- (84.0000,150.2000) -- (84.0000,150.2000) -- (84.0000,150.2000) -- (84.0000,150.2000) -- (84.0000,150.2000) -- (84.0000,150.1000) -- (84.0000,150.1000) -- (84.0000,150.1000) -- (84.0000,150.1000) -- (84.0000,150.1000) -- (84.0000,150.1000) -- (84.0000,150.1000) -- (84.0000,150.1000) -- (84.0000,150.1000) -- (84.0000,150.1000) -- (84.0000,150.1000) -- (84.0000,150.1000) -- (84.0000,150.1000) -- (84.0000,150.1000) -- (84.0000,150.1000) -- (84.0000,150.1000) -- (84.0000,150.1000) -- (84.0000,150.1000) -- (84.0000,150.1000) -- (84.0000,150.1000) -- (84.0000,150.1000) -- (84.0000,150.0000) -- (84.0000,150.0000) -- (84.0000,150.0000) -- (84.0000,150.0000) -- (84.0000,150.0000) -- (84.0000,150.0000) -- (84.0000,150.0000) -- (84.0000,150.0000) -- (84.0000,150.0000) -- (84.0000,150.0000) -- (84.0000,150.0000) -- (84.0000,150.0000) -- (84.0000,150.0000) -- (84.0000,150.0000) -- (84.0000,150.0000) -- (84.0000,150.0000) -- (84.1000,150.0000) -- (84.1000,150.0000) -- (84.1000,150.0000) -- (84.1000,150.0000) -- (84.1000,150.0000) -- (84.1000,150.0000) -- (84.1000,149.9000) -- (84.1000,149.9000) -- (84.1000,149.9000) -- (84.1000,149.9000) -- (84.1000,149.9000) -- (84.1000,149.9000) -- (84.1000,149.9000) -- (84.1000,149.9000) -- (84.1000,149.9000) -- (84.1000,149.9000) -- (84.1000,149.9000) -- (84.1000,149.9000) -- (84.1000,149.9000) -- (84.1000,149.9000) -- (84.1000,149.9000) -- (84.1000,149.9000) -- (84.1000,149.9000) -- (84.1000,149.9000) -- (84.1000,149.9000) -- (84.1000,149.9000) -- (84.1000,149.9000) -- (84.1000,149.8000) -- (84.1000,149.8000) -- (84.1000,149.8000) -- (84.1000,149.8000) -- (84.1000,149.8000) -- (84.1000,149.8000) -- (84.1000,149.8000) -- (84.1000,149.8000) -- (84.1000,149.8000) -- (84.1000,149.8000) -- (84.1000,149.8000) -- (84.1000,149.8000) -- (84.1000,149.8000) -- (84.1000,149.8000) -- (84.1000,149.8000) -- (84.1000,149.8000) -- (84.1000,149.8000) -- (84.1000,149.8000) -- (84.1000,149.8000) -- (84.1000,149.8000) -- (84.1000,149.8000) -- (84.1000,149.8000) -- (84.1000,149.7000) -- (84.2000,149.7000) -- (84.2000,149.7000) -- (84.2000,149.7000) -- (84.2000,149.7000) -- (84.2000,149.7000) -- (84.2000,149.7000) -- (84.2000,149.7000) -- (84.2000,149.7000) -- (84.2000,149.7000) -- (84.2000,149.7000) -- (84.2000,149.7000) -- (84.2000,149.7000) -- (84.2000,149.7000) -- (84.2000,149.7000) -- (84.2000,149.7000) -- (84.2000,149.7000) -- (84.2000,149.7000) -- (84.2000,149.7000) -- (84.2000,149.7000) -- (84.2000,149.7000) -- (84.2000,149.6000) -- (84.2000,149.6000) -- (84.2000,149.6000) -- (84.2000,149.6000) -- (84.2000,149.6000) -- (84.2000,149.6000) -- (84.2000,149.6000) -- (84.2000,149.6000) -- (84.2000,149.6000) -- (84.2000,149.6000) -- (84.2000,149.6000) -- (84.2000,149.6000) -- (84.2000,149.6000) -- (84.2000,149.6000) -- (84.2000,149.6000) -- (84.2000,149.6000) -- (84.2000,149.6000) -- (84.2000,149.6000) -- (84.2000,149.6000) -- (84.2000,149.6000) -- (84.2000,149.6000) -- (84.2000,149.6000) -- (84.2000,149.5000) -- (84.2000,149.5000) -- (84.2000,149.5000) -- (84.2000,149.5000) -- (84.2000,149.5000) -- (84.2000,149.5000) -- (84.2000,149.5000) -- (84.3000,149.5000) -- (84.3000,149.5000) -- (84.3000,149.5000) -- (84.3000,149.5000) -- (84.3000,149.5000) -- (84.3000,149.5000) -- (84.3000,149.5000) -- (84.3000,149.5000) -- (84.3000,149.5000) -- (84.3000,149.5000) -- (84.3000,149.5000) -- (84.3000,149.5000) -- (84.3000,149.5000) -- (84.3000,149.5000) -- (84.3000,149.4000) -- (84.3000,149.4000) -- (84.3000,149.4000) -- (84.3000,149.4000) -- (84.3000,149.4000) -- (84.3000,149.4000) -- (84.3000,149.4000) -- (84.3000,149.4000) -- (84.3000,149.4000) -- (84.3000,149.4000) -- (84.3000,149.4000) -- (84.3000,149.4000) -- (84.3000,149.4000) -- (84.3000,149.4000) -- (84.3000,149.4000) -- (84.3000,149.4000) -- (84.3000,149.4000) -- (84.3000,149.4000) -- (84.3000,149.4000) -- (84.3000,149.4000) -- (84.3000,149.4000) -- (84.3000,149.4000) -- (84.3000,149.3000) -- (84.3000,149.3000) -- (84.3000,149.3000) -- (84.3000,149.3000) -- (84.3000,149.3000) -- (84.3000,149.3000) -- (84.3000,149.3000) -- (84.3000,149.3000) -- (84.3000,149.3000) -- (84.3000,149.3000) -- (84.3000,149.3000) -- (84.3000,149.3000) -- (84.3000,149.3000) -- (84.3000,149.3000) -- (84.4000,149.3000) -- (84.4000,149.3000) -- (84.4000,149.3000) -- (84.4000,149.3000) -- (84.4000,149.3000) -- (84.4000,149.3000) -- (84.4000,149.3000) -- (84.4000,149.2000) -- (84.4000,149.2000) -- (84.4000,149.2000) -- (84.4000,149.2000) -- (84.4000,149.2000) -- (84.4000,149.2000) -- (84.4000,149.2000) -- (84.4000,149.2000) -- (84.4000,149.2000) -- (84.4000,149.2000) -- (84.4000,149.2000) -- (84.4000,149.2000) -- (84.4000,149.2000) -- (84.4000,149.2000) -- (84.4000,149.2000) -- (84.4000,149.2000) -- (84.4000,149.2000) -- (84.4000,149.2000) -- (84.4000,149.2000) -- (84.4000,149.2000) -- (84.4000,149.2000) -- (84.4000,149.2000) -- (84.4000,149.1000) -- (84.4000,149.1000) -- (84.4000,149.1000) -- (84.4000,149.1000) -- (84.4000,149.1000) -- (84.4000,149.1000) -- (84.4000,149.1000) -- (84.4000,149.1000) -- (84.4000,149.1000) -- (84.4000,149.1000) -- (84.4000,149.1000) -- (84.4000,149.1000) -- (84.4000,149.1000) -- (84.4000,149.1000) -- (84.4000,149.1000) -- (84.4000,149.1000) -- (84.4000,149.1000) -- (84.4000,149.1000) -- (84.4000,149.1000) -- (84.4000,149.1000) -- (84.5000,149.1000) -- (84.5000,149.0000) -- (84.5000,149.0000) -- (84.5000,149.0000) -- (84.5000,149.0000) -- (84.5000,149.0000) -- (84.5000,149.0000) -- (84.5000,149.0000) -- (84.5000,149.0000) -- (84.5000,149.0000) -- (84.5000,149.0000) -- (84.5000,149.0000) -- (84.5000,149.0000) -- (84.5000,149.0000) -- (84.5000,149.0000) -- (84.5000,149.0000) -- (84.5000,149.0000) -- (84.5000,149.0000) -- (84.5000,149.0000) -- (84.5000,149.0000) -- (84.5000,149.0000) -- (84.5000,149.0000) -- (84.5000,149.0000) -- (84.5000,148.9000) -- (84.5000,148.9000) -- (84.5000,148.9000) -- (84.5000,148.9000) -- (84.5000,148.9000) -- (84.5000,148.9000) -- (84.5000,148.9000) -- (84.5000,148.9000) -- (84.5000,148.9000) -- (84.5000,148.9000) -- (84.5000,148.9000) -- (84.5000,148.9000) -- (84.5000,148.9000) -- (84.5000,148.9000) -- (84.5000,148.9000) -- (84.5000,148.9000) -- (84.5000,148.9000) -- (84.5000,148.9000) -- (84.5000,148.9000) -- (84.5000,148.9000) -- (84.5000,148.9000) -- (84.5000,148.8000) -- (84.5000,148.8000) -- (84.5000,148.8000) -- (84.5000,148.8000) -- (84.5000,148.8000) -- (84.5000,148.8000) -- (84.6000,148.8000) -- (84.6000,148.8000) -- (84.6000,148.8000) -- (84.6000,148.8000) -- (84.6000,148.8000) -- (84.6000,148.8000) -- (84.6000,148.8000) -- (84.6000,148.8000) -- (84.6000,148.8000) -- (84.6000,148.8000) -- (84.6000,148.8000) -- (84.6000,148.8000) -- (84.6000,148.8000) -- (84.6000,148.8000) -- (84.6000,148.8000) -- (84.6000,148.8000) -- (84.6000,148.7000) -- (84.6000,148.7000) -- (84.6000,148.7000) -- (84.6000,148.7000) -- (84.6000,148.7000) -- (84.6000,148.7000) -- (84.6000,148.7000) -- (84.6000,148.7000) -- (84.6000,148.7000) -- (84.6000,148.7000) -- (84.6000,148.7000) -- (84.6000,148.7000) -- (84.6000,148.7000) -- (84.6000,148.7000) -- (84.6000,148.7000) -- (84.6000,148.7000) -- (84.6000,148.7000) -- (84.6000,148.7000) -- (84.6000,148.7000) -- (84.6000,148.7000) -- (84.6000,148.7000) -- (84.6000,148.6000) -- (84.6000,148.6000) -- (84.6000,148.6000) -- (84.6000,148.6000) -- (84.6000,148.6000) -- (84.6000,148.6000) -- (84.6000,148.6000) -- (84.6000,148.6000) -- (84.6000,148.6000) -- (84.6000,148.6000) -- (84.6000,148.6000) -- (84.6000,148.6000) -- (84.7000,148.6000) -- (84.7000,148.6000) -- (84.7000,148.6000) -- (84.7000,148.6000) -- (84.7000,148.6000) -- (84.7000,148.6000) -- (84.7000,148.6000) -- (84.7000,148.6000) -- (84.7000,148.6000) -- (84.7000,148.6000) -- (84.7000,148.5000) -- (84.7000,148.5000) -- (84.7000,148.5000) -- (84.7000,148.5000) -- (84.7000,148.5000) -- (84.7000,148.5000) -- (84.7000,148.5000) -- (84.7000,148.5000) -- (84.7000,148.5000) -- (84.7000,148.5000) -- (84.7000,148.5000) -- (84.7000,148.5000) -- (84.7000,148.5000) -- (84.7000,148.5000) -- (84.7000,148.5000) -- (84.7000,148.5000) -- (84.7000,148.5000) -- (84.7000,148.5000) -- (84.7000,148.5000) -- (84.7000,148.5000) -- (84.7000,148.5000) -- (84.7000,148.4000) -- (84.7000,148.4000) -- (84.7000,148.4000) -- (84.7000,148.4000) -- (84.7000,148.4000) -- (84.7000,148.4000) -- (84.7000,148.4000) -- (84.7000,148.4000) -- (84.7000,148.4000) -- (84.7000,148.4000) -- (84.7000,148.4000) -- (84.7000,148.4000) -- (84.7000,148.4000) -- (84.7000,148.4000) -- (84.7000,148.4000) -- (84.7000,148.4000) -- (84.7000,148.4000) -- (84.7000,148.4000) -- (84.7000,148.4000) -- (84.8000,148.4000) -- (84.8000,148.4000) -- (84.8000,148.4000) -- (84.8000,148.3000) -- (84.8000,148.3000) -- (84.8000,148.3000) -- (84.8000,148.3000) -- (84.8000,148.3000) -- (84.8000,148.3000) -- (84.8000,148.3000) -- (84.8000,148.3000) -- (84.8000,148.3000) -- (84.8000,148.3000) -- (84.8000,148.3000) -- (84.8000,148.3000) -- (84.8000,148.3000) -- (84.8000,148.3000) -- (84.8000,148.3000) -- (84.8000,148.3000) -- (84.8000,148.3000) -- (84.8000,148.3000) -- (84.8000,148.3000) -- (84.8000,148.3000) -- (84.8000,148.3000) -- (84.8000,148.2000) -- (84.8000,148.2000) -- (84.8000,148.2000) -- (84.8000,148.2000) -- (84.8000,148.2000) -- (84.8000,148.2000) -- (84.8000,148.2000) -- (84.8000,148.2000) -- (84.8000,148.2000) -- (84.8000,148.2000) -- (84.8000,148.2000) -- (84.8000,148.2000) -- (84.8000,148.2000) -- (84.8000,148.2000) -- (84.8000,148.2000) -- (84.8000,148.2000) -- (84.8000,148.2000) -- (84.8000,148.2000) -- (84.8000,148.2000) -- (84.8000,148.2000) -- (84.8000,148.2000) -- (84.8000,148.2000) -- (84.8000,148.1000) -- (84.8000,148.1000) -- (84.8000,148.1000) -- (84.9000,148.1000) -- (84.9000,148.1000) -- (84.9000,148.1000) -- (84.9000,148.1000) -- (84.9000,148.1000) -- (84.9000,148.1000) -- (84.9000,148.1000) -- (84.9000,148.1000) -- (84.9000,148.1000) -- (84.9000,148.1000) -- (84.9000,148.1000) -- (84.9000,148.1000) -- (84.9000,148.1000) -- (84.9000,148.1000) -- (84.9000,148.1000) -- (84.9000,148.1000) -- (84.9000,148.1000) -- (84.9000,148.1000) -- (84.9000,148.0000) -- (84.9000,148.0000) -- (84.9000,148.0000) -- (84.9000,148.0000) -- (84.9000,148.0000) -- (84.9000,148.0000) -- (84.9000,148.0000) -- (84.9000,148.0000) -- (84.9000,148.0000) -- (84.9000,148.0000) -- (84.9000,148.0000) -- (84.9000,148.0000) -- (84.9000,148.0000) -- (84.9000,148.0000) -- (84.9000,148.0000) -- (84.9000,148.0000) -- (84.9000,148.0000) -- (84.9000,148.0000) -- (84.9000,148.0000) -- (84.9000,148.0000) -- (84.9000,148.0000) -- (84.9000,148.0000) -- (84.9000,147.9000) -- (84.9000,147.9000) -- (84.9000,147.9000) -- (84.9000,147.9000) -- (84.9000,147.9000) -- (84.9000,147.9000) -- (84.9000,147.9000) -- (84.9000,147.9000) -- (84.9000,147.9000) -- (84.9000,147.9000) -- (85.0000,147.9000) -- (85.0000,147.9000) -- (85.0000,147.9000) -- (85.0000,147.9000) -- (85.0000,147.9000) -- (85.0000,147.9000) -- (85.0000,147.9000) -- (85.0000,147.9000) -- (85.0000,147.9000) -- (85.0000,147.9000) -- (85.0000,147.9000) -- (85.0000,147.8000) -- (85.0000,147.8000) -- (85.0000,147.8000) -- (85.0000,147.8000) -- (85.0000,147.8000) -- (85.0000,147.8000) -- (85.0000,147.8000) -- (85.0000,147.8000) -- (85.0000,147.8000) -- (85.0000,147.8000) -- (85.0000,147.8000) -- (85.0000,147.8000) -- (85.0000,147.8000) -- (85.0000,147.8000) -- (85.0000,147.8000) -- (85.0000,147.8000) -- (85.0000,147.8000) -- (85.0000,147.8000) -- (85.0000,147.8000) -- (85.0000,147.8000) -- (85.0000,147.8000) -- (85.0000,147.8000) -- (85.0000,147.7000) -- (85.0000,147.7000) -- (85.0000,147.7000) -- (85.0000,147.7000) -- (85.0000,147.7000) -- (85.0000,147.7000) -- (85.0000,147.7000) -- (85.0000,147.7000) -- (85.0000,147.7000) -- (85.0000,147.7000) -- (85.0000,147.7000) -- (85.0000,147.7000) -- (85.0000,147.7000) -- (85.0000,147.7000) -- (85.0000,147.7000) -- (85.0000,147.7000) -- (85.1000,147.7000) -- (85.1000,147.7000) -- (85.1000,147.7000) -- (85.1000,147.7000) -- (85.1000,147.7000) -- (85.1000,147.6000) -- (85.1000,147.6000) -- (85.1000,147.6000) -- (85.1000,147.6000) -- (85.1000,147.6000) -- (85.1000,147.6000) -- (85.1000,147.6000) -- (85.1000,147.6000) -- (85.1000,147.6000) -- (85.1000,147.6000) -- (85.1000,147.6000) -- (85.1000,147.6000) -- (85.1000,147.6000) -- (85.1000,147.6000) -- (85.1000,147.6000) -- (85.1000,147.6000) -- (85.1000,147.6000) -- (85.1000,147.6000) -- (85.1000,147.6000) -- (85.1000,147.6000) -- (85.1000,147.6000) -- (85.1000,147.6000) -- (85.1000,147.5000) -- (85.1000,147.5000) -- (85.1000,147.5000) -- (85.1000,147.5000) -- (85.1000,147.5000) -- (85.1000,147.5000) -- (85.1000,147.5000) -- (85.1000,147.5000) -- (85.1000,147.5000) -- (85.1000,147.5000) -- (85.1000,147.5000) -- (85.1000,147.5000) -- (85.1000,147.5000) -- (85.1000,147.5000) -- (85.1000,147.5000) -- (85.1000,147.5000) -- (85.1000,147.5000) -- (85.1000,147.5000) -- (85.1000,147.5000) -- (85.1000,147.5000) -- (85.1000,147.5000) -- (85.1000,147.4000) -- (85.1000,147.4000) -- (85.2000,147.4000) -- (85.2000,147.4000) -- (85.2000,147.4000) -- (85.2000,147.4000) -- (85.2000,147.4000) -- (85.2000,147.4000) -- (85.2000,147.4000) -- (85.2000,147.4000) -- (85.2000,147.4000) -- (85.2000,147.4000) -- (85.2000,147.4000) -- (85.2000,147.4000) -- (85.2000,147.4000) -- (85.2000,147.4000) -- (85.2000,147.4000) -- (85.2000,147.4000) -- (85.2000,147.4000) -- (85.2000,147.4000) -- (85.2000,147.4000) -- (85.2000,147.4000) -- (85.2000,147.3000) -- (85.2000,147.3000) -- (85.2000,147.3000) -- (85.2000,147.3000) -- (85.2000,147.3000) -- (85.2000,147.3000) -- (85.2000,147.3000) -- (85.2000,147.3000) -- (85.2000,147.3000) -- (85.2000,147.3000) -- (85.2000,147.3000) -- (85.2000,147.3000) -- (85.2000,147.3000) -- (85.2000,147.3000) -- (85.2000,147.3000) -- (85.2000,147.3000) -- (85.2000,147.3000) -- (85.2000,147.3000) -- (85.2000,147.3000) -- (85.2000,147.3000) -- (85.2000,147.3000) -- (85.2000,147.2000) -- (85.2000,147.2000) -- (85.2000,147.2000) -- (85.2000,147.2000) -- (85.2000,147.2000) -- (85.2000,147.2000) -- (85.2000,147.2000) -- (85.2000,147.2000) -- (85.3000,147.2000) -- (85.3000,147.2000) -- (85.3000,147.2000) -- (85.3000,147.2000) -- (85.3000,147.2000) -- (85.3000,147.2000) -- (85.3000,147.2000) -- (85.3000,147.2000) -- (85.3000,147.2000) -- (85.3000,147.2000) -- (85.3000,147.2000) -- (85.3000,147.2000) -- (85.3000,147.2000) -- (85.3000,147.2000) -- (85.3000,147.1000) -- (85.3000,147.1000) -- (85.3000,147.1000) -- (85.3000,147.1000) -- (85.3000,147.1000) -- (85.3000,147.1000) -- (85.3000,147.1000) -- (85.3000,147.1000) -- (85.3000,147.1000) -- (85.3000,147.1000) -- (85.3000,147.1000) -- (85.3000,147.1000) -- (85.3000,147.1000) -- (85.3000,147.1000) -- (85.3000,147.1000) -- (85.3000,147.1000) -- (85.3000,147.1000) -- (85.3000,147.1000) -- (85.3000,147.1000) -- (85.3000,147.1000) -- (85.3000,147.1000) -- (85.3000,147.0000) -- (85.3000,147.0000) -- (85.3000,147.0000) -- (85.3000,147.0000) -- (85.3000,147.0000) -- (85.3000,147.0000) -- (85.3000,147.0000) -- (85.3000,147.0000) -- (85.3000,147.0000) -- (85.3000,147.0000) -- (85.3000,147.0000) -- (85.3000,147.0000) -- (85.3000,147.0000) -- (85.3000,147.0000) -- (85.3000,147.0000) -- (85.4000,147.0000) -- (85.4000,147.0000) -- (85.4000,147.0000) -- (85.4000,147.0000) -- (85.4000,147.0000) -- (85.4000,147.0000) -- (85.4000,147.0000) -- (85.4000,146.9000) -- (85.4000,146.9000) -- (85.4000,146.9000) -- (85.4000,146.9000) -- (85.4000,146.9000) -- (85.4000,146.9000) -- (85.4000,146.9000) -- (85.4000,146.9000) -- (85.4000,146.9000) -- (85.4000,146.9000) -- (85.4000,146.9000) -- (85.4000,146.9000) -- (85.4000,146.9000) -- (85.4000,146.9000) -- (85.4000,146.9000) -- (85.4000,146.9000) -- (85.4000,146.9000) -- (85.4000,146.9000) -- (85.4000,146.9000) -- (85.4000,146.9000) -- (85.4000,146.9000) -- (85.4000,146.8000) -- (85.4000,146.8000) -- (85.4000,146.8000) -- (85.4000,146.8000) -- (85.4000,146.8000) -- (85.4000,146.8000) -- (85.4000,146.8000) -- (85.4000,146.8000) -- (85.4000,146.8000) -- (85.4000,146.8000) -- (85.4000,146.8000) -- (85.4000,146.8000) -- (85.4000,146.8000) -- (85.4000,146.8000) -- (85.4000,146.8000) -- (85.4000,146.8000) -- (85.4000,146.8000) -- (85.4000,146.8000) -- (85.4000,146.8000) -- (85.4000,146.8000) -- (85.4000,146.8000) -- (85.5000,146.8000) -- (85.5000,146.7000) -- (85.5000,146.7000) -- (85.5000,146.7000) -- (85.5000,146.7000) -- (85.5000,146.7000) -- (85.5000,146.7000) -- (85.5000,146.7000) -- (85.5000,146.7000) -- (85.5000,146.7000) -- (85.5000,146.7000) -- (85.5000,146.7000) -- (85.5000,146.7000) -- (85.5000,146.7000) -- (85.5000,146.7000) -- (85.5000,146.7000) -- (85.5000,146.7000) -- (85.5000,146.7000) -- (85.5000,146.7000) -- (85.5000,146.7000) -- (85.5000,146.7000) -- (85.5000,146.7000) -- (85.5000,146.6000) -- (85.5000,146.6000) -- (85.5000,146.6000) -- (85.5000,146.6000) -- (85.5000,146.6000) -- (85.5000,146.6000) -- (85.5000,146.6000) -- (85.5000,146.6000) -- (85.5000,146.6000) -- (85.5000,146.6000) -- (85.5000,146.6000) -- (85.5000,146.6000) -- (85.5000,146.6000) -- (85.5000,146.6000) -- (85.5000,146.6000) -- (85.5000,146.6000) -- (85.5000,146.6000) -- (85.5000,146.6000) -- (85.5000,146.6000) -- (85.5000,146.6000) -- (85.5000,146.6000) -- (85.5000,146.6000) -- (85.5000,146.5000) -- (85.5000,146.5000) -- (85.5000,146.5000) -- (85.5000,146.5000) -- (85.5000,146.5000) -- (85.5000,146.5000) -- (85.6000,146.5000) -- (85.6000,146.5000) -- (85.6000,146.5000) -- (85.6000,146.5000) -- (85.6000,146.5000) -- (85.6000,146.5000) -- (85.6000,146.5000) -- (85.6000,146.5000) -- (85.6000,146.5000) -- (85.6000,146.5000) -- (85.6000,146.5000) -- (85.6000,146.5000) -- (85.6000,146.5000) -- (85.6000,146.5000) -- (85.6000,146.5000) -- (85.6000,146.4000) -- (85.6000,146.4000) -- (85.6000,146.4000) -- (85.6000,146.4000) -- (85.6000,146.4000) -- (85.6000,146.4000) -- (85.6000,146.4000) -- (85.6000,146.4000) -- (85.6000,146.4000) -- (85.6000,146.4000) -- (85.6000,146.4000) -- (85.6000,146.4000) -- (85.6000,146.4000) -- (85.6000,146.4000) -- (85.6000,146.4000) -- (85.6000,146.4000) -- (85.6000,146.4000) -- (85.6000,146.4000) -- (85.6000,146.4000) -- (85.6000,146.4000) -- (85.6000,146.4000) -- (85.6000,146.4000) -- (85.6000,146.3000) -- (85.6000,146.3000) -- (85.6000,146.3000) -- (85.6000,146.3000) -- (85.6000,146.3000) -- (85.6000,146.3000) -- (85.6000,146.3000) -- (85.6000,146.3000) -- (85.6000,146.3000) -- (85.6000,146.3000) -- (85.6000,146.3000) -- (85.6000,146.3000) -- (85.7000,146.3000) -- (85.7000,146.3000) -- (85.7000,146.3000) -- (85.7000,146.3000) -- (85.7000,146.3000) -- (85.7000,146.3000) -- (85.7000,146.3000) -- (85.7000,146.3000) -- (85.7000,146.3000) -- (85.7000,146.2000) -- (85.7000,146.2000) -- (85.7000,146.2000) -- (85.7000,146.2000) -- (85.7000,146.2000) -- (85.7000,146.2000) -- (85.7000,146.2000) -- (85.7000,146.2000) -- (85.7000,146.2000) -- (85.7000,146.2000) -- (85.7000,146.2000) -- (85.7000,146.2000) -- (85.7000,146.2000) -- (85.7000,146.2000) -- (85.7000,146.2000) -- (85.7000,146.2000) -- (85.7000,146.2000) -- (85.7000,146.2000) -- (85.7000,146.2000) -- (85.7000,146.2000) -- (85.7000,146.2000) -- (85.7000,146.2000) -- (85.7000,146.1000) -- (85.7000,146.1000) -- (85.7000,146.1000) -- (85.7000,146.1000) -- (85.7000,146.1000) -- (85.7000,146.1000) -- (85.7000,146.1000) -- (85.7000,146.1000) -- (85.7000,146.1000) -- (85.7000,146.1000) -- (85.7000,146.1000) -- (85.7000,146.1000) -- (85.7000,146.1000) -- (85.7000,146.1000) -- (85.7000,146.1000) -- (85.7000,146.1000) -- (85.7000,146.1000) -- (85.7000,146.1000) -- (85.7000,146.1000) -- (85.8000,146.1000) -- (85.8000,146.1000) -- (85.8000,146.0000) -- (85.8000,146.0000) -- (85.8000,146.0000) -- (85.8000,146.0000) -- (85.8000,146.0000) -- (85.8000,146.0000) -- (85.8000,146.0000) -- (85.8000,146.0000) -- (85.8000,146.0000) -- (85.8000,146.0000) -- (85.8000,146.0000) -- (85.8000,146.0000) -- (85.8000,146.0000) -- (85.8000,146.0000) -- (85.8000,146.0000) -- (85.8000,146.0000) -- (85.8000,146.0000) -- (85.8000,146.0000) -- (85.8000,146.0000) -- (85.8000,146.0000) -- (85.8000,146.0000) -- (85.8000,146.0000) -- (85.8000,145.9000) -- (85.8000,145.9000) -- (85.8000,145.9000) -- (85.8000,145.9000) -- (85.8000,145.9000) -- (85.8000,145.9000) -- (85.8000,145.9000) -- (85.8000,145.9000) -- (85.8000,145.9000) -- (85.8000,145.9000) -- (85.8000,145.9000) -- (85.8000,145.9000) -- (85.8000,145.9000) -- (85.8000,145.9000) -- (85.8000,145.9000) -- (85.8000,145.9000) -- (85.8000,145.9000) -- (85.8000,145.9000) -- (85.8000,145.9000) -- (85.8000,145.9000) -- (85.8000,145.9000) -- (85.8000,145.8000) -- (85.8000,145.8000) -- (85.8000,145.8000) -- (85.8000,145.8000) -- (85.9000,145.8000) -- (85.9000,145.8000) -- (85.9000,145.8000) -- (85.9000,145.8000) -- (85.9000,145.8000) -- (85.9000,145.8000) -- (85.9000,145.8000) -- (85.9000,145.8000) -- (85.9000,145.8000) -- (85.9000,145.8000) -- (85.9000,145.8000) -- (85.9000,145.8000) -- (85.9000,145.8000) -- (85.9000,145.8000) -- (85.9000,145.8000) -- (85.9000,145.8000) -- (85.9000,145.8000) -- (85.9000,145.8000) -- (85.9000,145.7000) -- (85.9000,145.7000) -- (85.9000,145.7000) -- (85.9000,145.7000) -- (85.9000,145.7000) -- (85.9000,145.7000) -- (85.9000,145.7000) -- (85.9000,145.7000) -- (85.9000,145.7000) -- (85.9000,145.7000) -- (85.9000,145.7000) -- (85.9000,145.7000) -- (85.9000,145.7000) -- (85.9000,145.7000) -- (85.9000,145.7000) -- (85.9000,145.7000) -- (85.9000,145.7000) -- (85.9000,145.7000) -- (85.9000,145.7000) -- (85.9000,145.7000) -- (85.9000,145.7000) -- (85.9000,145.6000) -- (85.9000,145.6000) -- (85.9000,145.6000) -- (85.9000,145.6000) -- (85.9000,145.6000) -- (85.9000,145.6000) -- (85.9000,145.6000) -- (85.9000,145.6000) -- (85.9000,145.6000) -- (85.9000,145.6000) -- (85.9000,145.6000) -- (86.0000,145.6000) -- (86.0000,145.6000) -- (86.0000,145.6000) -- (86.0000,145.6000) -- (86.0000,145.6000) -- (86.0000,145.6000) -- (86.0000,145.6000) -- (86.0000,145.6000) -- (86.0000,145.6000) -- (86.0000,145.6000) -- (86.0000,145.6000) -- (86.0000,145.5000) -- (86.0000,145.5000) -- (86.0000,145.5000) -- (86.0000,145.5000) -- (86.0000,145.5000) -- (86.0000,145.5000) -- (86.0000,145.5000) -- (86.0000,145.5000) -- (86.0000,145.5000) -- (86.0000,145.5000) -- (86.0000,145.5000) -- (86.0000,145.5000) -- (86.0000,145.5000) -- (86.0000,145.5000) -- (86.0000,145.5000) -- (86.0000,145.5000) -- (86.0000,145.5000) -- (86.0000,145.5000) -- (86.0000,145.5000) -- (86.0000,145.5000) -- (86.0000,145.5000) -- (86.0000,145.4000) -- (86.0000,145.4000) -- (86.0000,145.4000) -- (86.0000,145.4000) -- (86.0000,145.4000) -- (86.0000,145.4000) -- (86.0000,145.4000) -- (86.0000,145.4000) -- (86.0000,145.4000) -- (86.0000,145.4000) -- (86.0000,145.4000) -- (86.0000,145.4000) -- (86.0000,145.4000) -- (86.0000,145.4000) -- (86.0000,145.4000) -- (86.0000,145.4000) -- (86.0000,145.4000) -- (86.1000,145.4000) -- (86.1000,145.4000) -- (86.1000,145.4000) -- (86.1000,145.4000) -- (86.1000,145.4000) -- (86.1000,145.3000) -- (86.1000,145.3000) -- (86.1000,145.3000) -- (86.1000,145.3000) -- (86.1000,145.3000) -- (86.1000,145.3000) -- (86.1000,145.3000) -- (86.1000,145.3000) -- (86.1000,145.3000) -- (86.1000,145.3000) -- (86.1000,145.3000) -- (86.1000,145.3000) -- (86.1000,145.3000) -- (86.1000,145.3000) -- (86.1000,145.3000) -- (86.1000,145.3000) -- (86.1000,145.3000) -- (86.1000,145.3000) -- (86.1000,145.3000) -- (86.1000,145.3000) -- (86.1000,145.3000) -- (86.1000,145.2000) -- (86.1000,145.2000) -- (86.1000,145.2000) -- (86.1000,145.2000) -- (86.1000,145.2000) -- (86.1000,145.2000) -- (86.1000,145.2000) -- (86.1000,145.2000) -- (86.1000,145.2000) -- (86.1000,145.2000) -- (86.1000,145.2000) -- (86.1000,145.2000) -- (86.1000,145.2000) -- (86.1000,145.2000) -- (86.1000,145.2000) -- (86.1000,145.2000) -- (86.1000,145.2000) -- (86.1000,145.2000) -- (86.1000,145.2000) -- (86.1000,145.2000) -- (86.1000,145.2000) -- (86.1000,145.2000) -- (86.1000,145.1000) -- (86.1000,145.1000) -- (86.2000,145.1000) -- (86.2000,145.1000) -- (86.2000,145.1000) -- (86.2000,145.1000) -- (86.2000,145.1000) -- (86.2000,145.1000) -- (86.2000,145.1000) -- (86.2000,145.1000) -- (86.2000,145.1000) -- (86.2000,145.1000) -- (86.2000,145.1000) -- (86.2000,145.1000) -- (86.2000,145.1000) -- (86.2000,145.1000) -- (86.2000,145.1000) -- (86.2000,145.1000) -- (86.2000,145.1000) -- (86.2000,145.1000) -- (86.2000,145.1000) -- (86.2000,145.0000) -- (86.2000,145.0000) -- (86.2000,145.0000) -- (86.2000,145.0000) -- (86.2000,145.0000) -- (86.2000,145.0000) -- (86.2000,145.0000) -- (86.2000,145.0000) -- (86.2000,145.0000) -- (86.2000,145.0000) -- (86.2000,145.0000) -- (86.2000,145.0000) -- (86.2000,145.0000) -- (86.2000,145.0000) -- (86.2000,145.0000) -- (86.2000,145.0000) -- (86.2000,145.0000) -- (86.2000,145.0000) -- (86.2000,145.0000) -- (86.2000,145.0000) -- (86.2000,145.0000) -- (86.2000,145.0000) -- (86.2000,144.9000) -- (86.2000,144.9000) -- (86.2000,144.9000) -- (86.2000,144.9000) -- (86.2000,144.9000) -- (86.2000,144.9000) -- (86.2000,144.9000) -- (86.2000,144.9000) -- (86.2000,144.9000) -- (86.3000,144.9000) -- (86.3000,144.9000) -- (86.3000,144.9000) -- (86.3000,144.9000) -- (86.3000,144.9000) -- (86.3000,144.9000) -- (86.3000,144.9000) -- (86.3000,144.9000) -- (86.3000,144.9000) -- (86.3000,144.9000) -- (86.3000,144.9000) -- (86.3000,144.9000) -- (86.3000,144.8000) -- (86.3000,144.8000) -- (86.3000,144.8000) -- (86.3000,144.8000) -- (86.3000,144.8000) -- (86.3000,144.8000) -- (86.3000,144.8000) -- (86.3000,144.8000) -- (86.3000,144.8000) -- (86.3000,144.8000) -- (86.3000,144.8000) -- (86.3000,144.8000) -- (86.3000,144.8000) -- (86.3000,144.8000) -- (86.3000,144.8000) -- (86.3000,144.8000) -- (86.3000,144.8000) -- (86.3000,144.8000) -- (86.3000,144.8000) -- (86.3000,144.8000) -- (86.3000,144.8000) -- (86.3000,144.8000) -- (86.3000,144.7000) -- (86.3000,144.7000) -- (86.3000,144.7000) -- (86.3000,144.7000) -- (86.3000,144.7000) -- (86.3000,144.7000) -- (86.3000,144.7000) -- (86.3000,144.7000) -- (86.3000,144.7000) -- (86.3000,144.7000) -- (86.3000,144.7000) -- (86.3000,144.7000) -- (86.3000,144.7000) -- (86.3000,144.7000) -- (86.3000,144.7000) -- (86.4000,144.7000) -- (86.4000,144.7000) -- (86.4000,144.7000) -- (86.4000,144.7000) -- (86.4000,144.7000) -- (86.4000,144.7000) -- (86.4000,144.6000) -- (86.4000,144.6000) -- (86.4000,144.6000) -- (86.4000,144.6000) -- (86.4000,144.6000) -- (86.4000,144.6000) -- (86.4000,144.6000) -- (86.4000,144.6000) -- (86.4000,144.6000) -- (86.4000,144.6000) -- (86.4000,144.6000) -- (86.4000,144.6000) -- (86.4000,144.6000) -- (86.4000,144.6000) -- (86.4000,144.6000) -- (86.4000,144.6000) -- (86.4000,144.6000) -- (86.4000,144.6000) -- (86.4000,144.6000) -- (86.4000,144.6000) -- (86.4000,144.6000) -- (86.4000,144.6000) -- (86.4000,144.5000) -- (86.4000,144.5000) -- (86.4000,144.5000) -- (86.4000,144.5000) -- (86.4000,144.5000) -- (86.4000,144.5000) -- (86.4000,144.5000) -- (86.4000,144.5000) -- (86.4000,144.5000) -- (86.4000,144.5000) -- (86.4000,144.5000) -- (86.4000,144.5000) -- (86.4000,144.5000) -- (86.4000,144.5000) -- (86.4000,144.5000) -- (86.4000,144.5000) -- (86.4000,144.5000) -- (86.4000,144.5000) -- (86.4000,144.5000) -- (86.4000,144.5000) -- (86.4000,144.5000) -- (86.4000,144.4000) -- (86.5000,144.4000) -- (86.5000,144.4000) -- (86.5000,144.4000) -- (86.5000,144.4000) -- (86.5000,144.4000) -- (86.5000,144.4000) -- (86.5000,144.4000) -- (86.5000,144.4000) -- (86.5000,144.4000) -- (86.5000,144.4000) -- (86.5000,144.4000) -- (86.5000,144.4000) -- (86.5000,144.4000) -- (86.5000,144.4000) -- (86.5000,144.4000) -- (86.5000,144.4000) -- (86.5000,144.4000) -- (86.5000,144.4000) -- (86.5000,144.4000) -- (86.5000,144.4000) -- (86.5000,144.4000) -- (86.5000,144.3000) -- (86.5000,144.3000) -- (86.5000,144.3000) -- (86.5000,144.3000) -- (86.5000,144.3000) -- (86.5000,144.3000) -- (86.5000,144.3000) -- (86.5000,144.3000) -- (86.5000,144.3000) -- (86.5000,144.3000) -- (86.5000,144.3000) -- (86.5000,144.3000) -- (86.5000,144.3000) -- (86.5000,144.3000) -- (86.5000,144.3000) -- (86.5000,144.3000) -- (86.5000,144.3000) -- (86.5000,144.3000) -- (86.5000,144.3000) -- (86.5000,144.3000) -- (86.5000,144.3000) -- (86.5000,144.2000) -- (86.5000,144.2000) -- (86.5000,144.2000) -- (86.5000,144.2000) -- (86.5000,144.2000) -- (86.5000,144.2000) -- (86.5000,144.2000) -- (86.6000,144.2000) -- (86.6000,144.2000) -- (86.6000,144.2000) -- (86.6000,144.2000) -- (86.6000,144.2000) -- (86.6000,144.2000) -- (86.6000,144.2000) -- (86.6000,144.2000) -- (86.6000,144.2000) -- (86.6000,144.2000) -- (86.6000,144.2000) -- (86.6000,144.2000) -- (86.6000,144.2000) -- (86.6000,144.2000) -- (86.6000,144.2000) -- (86.6000,144.1000) -- (86.6000,144.1000) -- (86.6000,144.1000) -- (86.6000,144.1000) -- (86.6000,144.1000) -- (86.6000,144.1000) -- (86.6000,144.1000) -- (86.6000,144.1000) -- (86.6000,144.1000) -- (86.6000,144.1000) -- (86.6000,144.1000) -- (86.6000,144.1000) -- (86.6000,144.1000) -- (86.6000,144.1000) -- (86.6000,144.1000) -- (86.6000,144.1000) -- (86.6000,144.1000) -- (86.6000,144.1000) -- (86.6000,144.1000) -- (86.6000,144.1000) -- (86.6000,144.1000) -- (86.6000,144.0000) -- (86.6000,144.0000) -- (86.6000,144.0000) -- (86.6000,144.0000) -- (86.6000,144.0000) -- (86.6000,144.0000) -- (86.6000,144.0000) -- (86.6000,144.0000) -- (86.6000,144.0000) -- (86.6000,144.0000) -- (86.6000,144.0000) -- (86.6000,144.0000) -- (86.6000,144.0000) -- (86.6000,144.0000) -- (86.7000,144.0000) -- (86.7000,144.0000) -- (86.7000,144.0000) -- (86.7000,144.0000) -- (86.7000,144.0000) -- (86.7000,144.0000) -- (86.7000,144.0000) -- (86.7000,144.0000) -- (86.7000,143.9000) -- (86.7000,143.9000) -- (86.7000,143.9000) -- (86.7000,143.9000) -- (86.7000,143.9000) -- (86.7000,143.9000) -- (86.7000,143.9000) -- (86.7000,143.9000) -- (86.7000,143.9000) -- (86.7000,143.9000) -- (86.7000,143.9000) -- (86.7000,143.9000) -- (86.7000,143.9000) -- (86.7000,143.9000) -- (86.7000,143.9000) -- (86.7000,143.9000) -- (86.7000,143.9000) -- (86.7000,143.9000) -- (86.7000,143.9000) -- (86.7000,143.9000) -- (86.7000,143.9000) -- (86.7000,143.8000) -- (86.7000,143.8000) -- (86.7000,143.8000) -- (86.7000,143.8000) -- (86.7000,143.8000) -- (86.7000,143.8000) -- (86.7000,143.8000) -- (86.7000,143.8000) -- (86.7000,143.8000) -- (86.7000,143.8000) -- (86.7000,143.8000) -- (86.7000,143.8000) -- (86.7000,143.8000) -- (86.7000,143.8000) -- (86.7000,143.8000) -- (86.7000,143.8000) -- (86.7000,143.8000) -- (86.7000,143.8000) -- (86.7000,143.8000) -- (86.7000,143.8000) -- (86.8000,143.8000) -- (86.8000,143.8000) -- (86.8000,143.7000) -- (86.8000,143.7000) -- (86.8000,143.7000) -- (86.8000,143.7000) -- (86.8000,143.7000) -- (86.8000,143.7000) -- (86.8000,143.7000) -- (86.8000,143.7000) -- (86.8000,143.7000) -- (86.8000,143.7000) -- (86.8000,143.7000) -- (86.8000,143.7000) -- (86.8000,143.7000) -- (86.8000,143.7000) -- (86.8000,143.7000) -- (86.8000,143.7000) -- (86.8000,143.7000) -- (86.8000,143.7000) -- (86.8000,143.7000) -- (86.8000,143.7000) -- (86.8000,143.7000) -- (86.8000,143.6000) -- (86.8000,143.6000) -- (86.8000,143.6000) -- (86.8000,143.6000) -- (86.8000,143.6000) -- (86.8000,143.6000) -- (86.8000,143.6000) -- (86.8000,143.6000) -- (86.8000,143.6000) -- (86.8000,143.6000) -- (86.8000,143.6000) -- (86.8000,143.6000) -- (86.8000,143.6000) -- (86.8000,143.6000) -- (86.8000,143.6000) -- (86.8000,143.6000) -- (86.8000,143.6000) -- (86.8000,143.6000) -- (86.8000,143.6000) -- (86.8000,143.6000) -- (86.8000,143.6000) -- (86.8000,143.6000) -- (86.8000,143.5000) -- (86.8000,143.5000) -- (86.8000,143.5000) -- (86.8000,143.5000) -- (86.8000,143.5000) -- (86.9000,143.5000) -- (86.9000,143.5000) -- (86.9000,143.5000) -- (86.9000,143.5000) -- (86.9000,143.5000) -- (86.9000,143.5000) -- (86.9000,143.5000) -- (86.9000,143.5000) -- (86.9000,143.5000) -- (86.9000,143.5000) -- (86.9000,143.5000) -- (86.9000,143.5000) -- (86.9000,143.5000) -- (86.9000,143.5000) -- (86.9000,143.5000) -- (86.9000,143.5000) -- (86.9000,143.4000) -- (86.9000,143.4000) -- (86.9000,143.4000) -- (86.9000,143.4000) -- (86.9000,143.4000) -- (86.9000,143.4000) -- (86.9000,143.4000) -- (86.9000,143.4000) -- (86.9000,143.4000) -- (86.9000,143.4000) -- (86.9000,143.4000) -- (86.9000,143.4000) -- (86.9000,143.4000) -- (86.9000,143.4000) -- (86.9000,143.4000) -- (86.9000,143.4000) -- (86.9000,143.4000) -- (86.9000,143.4000) -- (86.9000,143.4000) -- (86.9000,143.4000) -- (86.9000,143.4000) -- (86.9000,143.4000) -- (86.9000,143.3000) -- (86.9000,143.3000) -- (86.9000,143.3000) -- (86.9000,143.3000) -- (86.9000,143.3000) -- (86.9000,143.3000) -- (86.9000,143.3000) -- (86.9000,143.3000) -- (86.9000,143.3000) -- (86.9000,143.3000) -- (86.9000,143.3000) -- (87.0000,143.3000) -- (87.0000,143.3000) -- (87.0000,143.3000) -- (87.0000,143.3000) -- (87.0000,143.3000) -- (87.0000,143.3000) -- (87.0000,143.3000) -- (87.0000,143.3000) -- (87.0000,143.3000) -- (87.0000,143.3000) -- (87.0000,143.2000) -- (87.0000,143.2000) -- (87.0000,143.2000) -- (87.0000,143.2000) -- (87.0000,143.2000) -- (87.0000,143.2000) -- (87.0000,143.2000) -- (87.0000,143.2000) -- (87.0000,143.2000) -- (87.0000,143.2000) -- (87.0000,143.2000) -- (87.0000,143.2000) -- (87.0000,143.2000) -- (87.0000,143.2000) -- (87.0000,143.2000) -- (87.0000,143.2000) -- (87.0000,143.2000) -- (87.0000,143.2000) -- (87.0000,143.2000) -- (87.0000,143.2000) -- (87.0000,143.2000) -- (87.0000,143.2000) -- (87.0000,143.1000) -- (87.0000,143.1000) -- (87.0000,143.1000) -- (87.0000,143.1000) -- (87.0000,143.1000) -- (87.0000,143.1000) -- (87.0000,143.1000) -- (87.0000,143.1000) -- (87.0000,143.1000) -- (87.0000,143.1000) -- (87.0000,143.1000) -- (87.0000,143.1000) -- (87.0000,143.1000) -- (87.0000,143.1000) -- (87.0000,143.1000) -- (87.0000,143.1000) -- (87.0000,143.1000) -- (87.0000,143.1000) -- (87.1000,143.1000) -- (87.1000,143.1000) -- (87.1000,143.1000) -- (87.1000,143.0000) -- (87.1000,143.0000) -- (87.1000,143.0000) -- (87.1000,143.0000) -- (87.1000,143.0000) -- (87.1000,143.0000) -- (87.1000,143.0000) -- (87.1000,143.0000) -- (87.1000,143.0000) -- (87.1000,143.0000) -- (87.1000,143.0000) -- (87.1000,143.0000) -- (87.1000,143.0000) -- (87.1000,143.0000) -- (87.1000,143.0000) -- (87.1000,143.0000) -- (87.1000,143.0000) -- (87.1000,143.0000) -- (87.1000,143.0000) -- (87.1000,143.0000) -- (87.1000,143.0000) -- (87.1000,143.0000) -- (87.1000,142.9000) -- (87.1000,142.9000) -- (87.1000,142.9000) -- (87.1000,142.9000) -- (87.1000,142.9000) -- (87.1000,142.9000) -- (87.1000,142.9000) -- (87.1000,142.9000) -- (87.1000,142.9000) -- (87.1000,142.9000) -- (87.1000,142.9000) -- (87.1000,142.9000) -- (87.1000,142.9000) -- (87.1000,142.9000) -- (87.1000,142.9000) -- (87.1000,142.9000) -- (87.1000,142.9000) -- (87.1000,142.9000) -- (87.1000,142.9000) -- (87.1000,142.9000) -- (87.1000,142.9000) -- (87.1000,142.9000) -- (87.1000,142.8000) -- (87.1000,142.8000) -- (87.2000,142.8000) -- (87.2000,142.8000) -- (87.2000,142.8000) -- (87.2000,142.8000) -- (87.2000,142.8000) -- (87.2000,142.8000) -- (87.2000,142.8000) -- (87.2000,142.8000) -- (87.2000,142.8000) -- (87.2000,142.8000) -- (87.2000,142.8000) -- (87.2000,142.8000) -- (87.2000,142.8000) -- (87.2000,142.8000) -- (87.2000,142.8000) -- (87.2000,142.8000) -- (87.2000,142.8000) -- (87.2000,142.8000) -- (87.2000,142.8000) -- (87.2000,142.7000) -- (87.2000,142.7000) -- (87.2000,142.7000) -- (87.2000,142.7000) -- (87.2000,142.7000) -- (87.2000,142.7000) -- (87.2000,142.7000) -- (87.2000,142.7000) -- (87.2000,142.7000) -- (87.2000,142.7000) -- (87.2000,142.7000) -- (87.2000,142.7000) -- (87.2000,142.7000) -- (87.2000,142.7000) -- (87.2000,142.7000) -- (87.2000,142.7000) -- (87.2000,142.7000) -- (87.2000,142.7000) -- (87.2000,142.7000) -- (87.2000,142.7000) -- (87.2000,142.7000) -- (87.2000,142.7000) -- (87.2000,142.6000) -- (87.2000,142.6000) -- (87.2000,142.6000) -- (87.2000,142.6000) -- (87.2000,142.6000) -- (87.2000,142.6000) -- (87.2000,142.6000) -- (87.2000,142.6000) -- (87.2000,142.6000) -- (87.3000,142.6000) -- (87.3000,142.6000) -- (87.3000,142.6000) -- (87.3000,142.6000) -- (87.3000,142.6000) -- (87.3000,142.6000) -- (87.3000,142.6000) -- (87.3000,142.6000) -- (87.3000,142.6000) -- (87.3000,142.6000) -- (87.3000,142.6000) -- (87.3000,142.6000) -- (87.3000,142.5000) -- (87.3000,142.5000) -- (87.3000,142.5000) -- (87.3000,142.5000) -- (87.3000,142.5000) -- (87.3000,142.5000) -- (87.3000,142.5000) -- (87.3000,142.5000) -- (87.3000,142.5000) -- (87.3000,142.5000) -- (87.3000,142.5000) -- (87.3000,142.5000) -- (87.3000,142.5000) -- (87.3000,142.5000) -- (87.3000,142.5000) -- (87.3000,142.5000) -- (87.3000,142.5000) -- (87.3000,142.5000) -- (87.3000,142.5000) -- (87.3000,142.5000) -- (87.3000,142.5000) -- (87.3000,142.5000) -- (87.3000,142.4000) -- (87.3000,142.4000) -- (87.3000,142.4000) -- (87.3000,142.4000) -- (87.3000,142.4000) -- (87.3000,142.4000) -- (87.3000,142.4000) -- (87.3000,142.4000) -- (87.3000,142.4000) -- (87.3000,142.4000) -- (87.3000,142.4000) -- (87.3000,142.4000) -- (87.3000,142.4000) -- (87.3000,142.4000) -- (87.3000,142.4000) -- (87.4000,142.4000) -- (87.4000,142.4000) -- (87.4000,142.4000) -- (87.4000,142.4000) -- (87.4000,142.4000) -- (87.4000,142.4000) -- (87.4000,142.3000) -- (87.4000,142.3000) -- (87.4000,142.3000) -- (87.4000,142.3000) -- (87.4000,142.3000) -- (87.4000,142.3000) -- (87.4000,142.3000) -- (87.4000,142.3000) -- (87.4000,142.3000) -- (87.4000,142.3000) -- (87.4000,142.3000) -- (87.4000,142.3000) -- (87.4000,142.3000) -- (87.4000,142.3000) -- (87.4000,142.3000) -- (87.4000,142.3000) -- (87.4000,142.3000) -- (87.4000,142.3000) -- (87.4000,142.3000) -- (87.4000,142.3000) -- (87.4000,142.3000) -- (87.4000,142.3000) -- (87.4000,142.2000) -- (87.4000,142.2000) -- (87.4000,142.2000) -- (87.4000,142.2000) -- (87.4000,142.2000) -- (87.4000,142.2000) -- (87.4000,142.2000) -- (87.4000,142.2000) -- (87.4000,142.2000) -- (87.4000,142.2000) -- (87.4000,142.2000) -- (87.4000,142.2000) -- (87.4000,142.2000) -- (87.4000,142.2000) -- (87.4000,142.2000) -- (87.4000,142.2000) -- (87.4000,142.2000) -- (87.4000,142.2000) -- (87.4000,142.2000) -- (87.4000,142.2000) -- (87.4000,142.2000) -- (87.4000,142.1000) -- (87.5000,142.1000) -- (87.5000,142.1000) -- (87.5000,142.1000) -- (87.5000,142.1000) -- (87.5000,142.1000) -- (87.5000,142.1000) -- (87.5000,142.1000) -- (87.5000,142.1000) -- (87.5000,142.1000) -- (87.5000,142.1000) -- (87.5000,142.1000) -- (87.5000,142.1000) -- (87.5000,142.1000) -- (87.5000,142.1000) -- (87.5000,142.1000) -- (87.5000,142.1000) -- (87.5000,142.1000) -- (87.5000,142.1000) -- (87.5000,142.1000) -- (87.5000,142.1000) -- (87.5000,142.1000) -- (87.5000,142.0000) -- (87.5000,142.0000) -- (87.5000,142.0000) -- (87.5000,142.0000) -- (87.5000,142.0000) -- (87.5000,142.0000) -- (87.5000,142.0000) -- (87.5000,142.0000) -- (87.5000,142.0000) -- (87.5000,142.0000) -- (87.5000,142.0000) -- (87.5000,142.0000) -- (87.5000,142.0000) -- (87.5000,142.0000) -- (87.5000,142.0000) -- (87.5000,142.0000) -- (87.5000,142.0000) -- (87.5000,142.0000) -- (87.5000,142.0000) -- (87.5000,142.0000) -- (87.5000,142.0000) -- (87.5000,141.9000) -- (87.5000,141.9000) -- (87.5000,141.9000) -- (87.5000,141.9000) -- (87.5000,141.9000) -- (87.5000,141.9000) -- (87.5000,141.9000) -- (87.6000,141.9000) -- (87.6000,141.9000) -- (87.6000,141.9000) -- (87.6000,141.9000) -- (87.6000,141.9000) -- (87.6000,141.9000) -- (87.6000,141.9000) -- (87.6000,141.9000) -- (87.6000,141.9000) -- (87.6000,141.9000) -- (87.6000,141.9000) -- (87.6000,141.9000) -- (87.6000,141.9000) -- (87.6000,141.9000) -- (87.6000,141.9000) -- (87.6000,141.8000) -- (87.6000,141.8000) -- (87.6000,141.8000) -- (87.6000,141.8000) -- (87.6000,141.8000) -- (87.6000,141.8000) -- (87.6000,141.8000) -- (87.6000,141.8000) -- (87.6000,141.8000) -- (87.6000,141.8000) -- (87.6000,141.8000) -- (87.6000,141.8000) -- (87.6000,141.8000) -- (87.6000,141.8000) -- (87.6000,141.8000) -- (87.6000,141.8000) -- (87.6000,141.8000) -- (87.6000,141.8000) -- (87.6000,141.8000) -- (87.6000,141.8000) -- (87.6000,141.8000) -- (87.6000,141.7000) -- (87.6000,141.7000) -- (87.6000,141.7000) -- (87.6000,141.7000) -- (87.6000,141.7000) -- (87.6000,141.7000) -- (87.6000,141.7000) -- (87.6000,141.7000) -- (87.6000,141.7000) -- (87.6000,141.7000) -- (87.6000,141.7000) -- (87.6000,141.7000) -- (87.6000,141.7000) -- (87.6000,141.7000) -- (87.7000,141.7000) -- (87.7000,141.7000) -- (87.7000,141.7000) -- (87.7000,141.7000) -- (87.7000,141.7000) -- (87.7000,141.7000) -- (87.7000,141.7000) -- (87.7000,141.7000) -- (87.7000,141.6000) -- (87.7000,141.6000) -- (87.7000,141.6000) -- (87.7000,141.6000) -- (87.7000,141.6000) -- (87.7000,141.6000) -- (87.7000,141.6000) -- (87.7000,141.6000) -- (87.7000,141.6000) -- (87.7000,141.6000) -- (87.7000,141.6000) -- (87.7000,141.6000) -- (87.7000,141.6000) -- (87.7000,141.6000) -- (87.7000,141.6000) -- (87.7000,141.6000) -- (87.7000,141.6000) -- (87.7000,141.6000) -- (87.7000,141.6000) -- (87.7000,141.6000) -- (87.7000,141.6000) -- (87.7000,141.5000) -- (87.7000,141.5000) -- (87.7000,141.5000) -- (87.7000,141.5000) -- (87.7000,141.5000) -- (87.7000,141.5000) -- (87.7000,141.5000) -- (87.7000,141.5000) -- (87.7000,141.5000) -- (87.7000,141.5000) -- (87.7000,141.5000) -- (87.7000,141.5000) -- (87.7000,141.5000) -- (87.7000,141.5000) -- (87.7000,141.5000) -- (87.7000,141.5000) -- (87.7000,141.5000) -- (87.7000,141.5000) -- (87.7000,141.5000) -- (87.7000,141.5000) -- (87.8000,141.5000) -- (87.8000,141.5000) -- (87.8000,141.4000) -- (87.8000,141.4000) -- (87.8000,141.4000) -- (87.8000,141.4000) -- (87.8000,141.4000) -- (87.8000,141.4000) -- (87.8000,141.4000) -- (87.8000,141.4000) -- (87.8000,141.4000) -- (87.8000,141.4000) -- (87.8000,141.4000) -- (87.8000,141.4000) -- (87.8000,141.4000) -- (87.8000,141.4000) -- (87.8000,141.4000) -- (87.8000,141.4000) -- (87.8000,141.4000) -- (87.8000,141.4000) -- (87.8000,141.4000) -- (87.8000,141.4000) -- (87.8000,141.4000) -- (87.8000,141.3000) -- (87.8000,141.3000) -- (87.8000,141.3000) -- (87.8000,141.3000) -- (87.8000,141.3000) -- (87.8000,141.3000) -- (87.8000,141.3000) -- (87.8000,141.3000) -- (87.8000,141.3000) -- (87.8000,141.3000) -- (87.8000,141.3000) -- (87.8000,141.3000) -- (87.8000,141.3000) -- (87.8000,141.3000) -- (87.8000,141.3000) -- (87.8000,141.3000) -- (87.8000,141.3000) -- (87.8000,141.3000) -- (87.8000,141.3000) -- (87.8000,141.3000) -- (87.8000,141.3000) -- (87.8000,141.3000) -- (87.8000,141.2000) -- (87.8000,141.2000) -- (87.8000,141.2000) -- (87.8000,141.2000) -- (87.8000,141.2000) -- (87.9000,141.2000) -- (87.9000,141.2000) -- (87.9000,141.2000) -- (87.9000,141.2000) -- (87.9000,141.2000) -- (87.9000,141.2000) -- (87.9000,141.2000) -- (87.9000,141.2000) -- (87.9000,141.2000) -- (87.9000,141.2000) -- (87.9000,141.2000) -- (87.9000,141.2000) -- (87.9000,141.2000) -- (87.9000,141.2000) -- (87.9000,141.2000) -- (87.9000,141.2000) -- (87.9000,141.1000) -- (87.9000,141.1000) -- (87.9000,141.1000) -- (87.9000,141.1000) -- (87.9000,141.1000) -- (87.9000,141.1000) -- (87.9000,141.1000) -- (87.9000,141.1000) -- (87.9000,141.1000) -- (87.9000,141.1000) -- (87.9000,141.1000) -- (87.9000,141.1000) -- (87.9000,141.1000) -- (87.9000,141.1000) -- (87.9000,141.1000) -- (87.9000,141.1000) -- (87.9000,141.1000) -- (87.9000,141.1000) -- (87.9000,141.1000) -- (87.9000,141.1000) -- (87.9000,141.1000) -- (87.9000,141.1000) -- (87.9000,141.0000) -- (87.9000,141.0000) -- (87.9000,141.0000) -- (87.9000,141.0000) -- (87.9000,141.0000) -- (87.9000,141.0000) -- (87.9000,141.0000) -- (87.9000,141.0000) -- (87.9000,141.0000) -- (87.9000,141.0000) -- (87.9000,141.0000) -- (88.0000,141.0000) -- (88.0000,141.0000) -- (88.0000,141.0000) -- (88.0000,141.0000) -- (88.0000,141.0000) -- (88.0000,141.0000) -- (88.0000,141.0000) -- (88.0000,141.0000) -- (88.0000,141.0000) -- (88.0000,141.0000) -- (88.0000,140.9000) -- (88.0000,140.9000) -- (88.0000,140.9000) -- (88.0000,140.9000) -- (88.0000,140.9000) -- (88.0000,140.9000) -- (88.0000,140.9000) -- (88.0000,140.9000) -- (88.0000,140.9000) -- (88.0000,140.9000) -- (88.0000,140.9000) -- (88.0000,140.9000) -- (88.0000,140.9000) -- (88.0000,140.9000) -- (88.0000,140.9000) -- (88.0000,140.9000) -- (88.0000,140.9000) -- (88.0000,140.9000) -- (88.0000,140.9000) -- (88.0000,140.9000) -- (88.0000,140.9000) -- (88.0000,140.9000) -- (88.0000,140.8000) -- (88.0000,140.8000) -- (88.0000,140.8000) -- (88.0000,140.8000) -- (88.0000,140.8000) -- (88.0000,140.8000) -- (88.0000,140.8000) -- (88.0000,140.8000) -- (88.0000,140.8000) -- (88.0000,140.8000) -- (88.0000,140.8000) -- (88.0000,140.8000) -- (88.0000,140.8000) -- (88.0000,140.8000) -- (88.0000,140.8000) -- (88.0000,140.8000) -- (88.0000,140.8000) -- (88.0000,140.8000) -- (88.1000,140.8000) -- (88.1000,140.8000) -- (88.1000,140.8000) -- (88.1000,140.7000) -- (88.1000,140.7000) -- (88.1000,140.7000) -- (88.1000,140.7000) -- (88.1000,140.7000) -- (88.1000,140.7000) -- (88.1000,140.7000) -- (88.1000,140.7000) -- (88.1000,140.7000) -- (88.1000,140.7000) -- (88.1000,140.7000) -- (88.1000,140.7000) -- (88.1000,140.7000) -- (88.1000,140.7000) -- (88.1000,140.7000) -- (88.1000,140.7000) -- (88.1000,140.7000) -- (88.1000,140.7000) -- (88.1000,140.7000) -- (88.1000,140.7000) -- (88.1000,140.7000) -- (88.1000,140.7000) -- (88.1000,140.6000) -- (88.1000,140.6000) -- (88.1000,140.6000) -- (88.1000,140.6000) -- (88.1000,140.6000) -- (88.1000,140.6000) -- (88.1000,140.6000) -- (88.1000,140.6000) -- (88.1000,140.6000) -- (88.1000,140.6000) -- (88.1000,140.6000) -- (88.1000,140.6000) -- (88.1000,140.6000) -- (88.1000,140.6000) -- (88.1000,140.6000) -- (88.1000,140.6000) -- (88.1000,140.6000) -- (88.1000,140.6000) -- (88.1000,140.6000) -- (88.1000,140.6000) -- (88.1000,140.6000) -- (88.1000,140.5000) -- (88.1000,140.5000) -- (88.1000,140.5000) -- (88.2000,140.5000) -- (88.2000,140.5000) -- (88.2000,140.5000) -- (88.2000,140.5000) -- (88.2000,140.5000) -- (88.2000,140.5000) -- (88.2000,140.5000) -- (88.2000,140.5000) -- (88.2000,140.5000) -- (88.2000,140.5000) -- (88.2000,140.5000) -- (88.2000,140.5000) -- (88.2000,140.5000) -- (88.2000,140.5000) -- (88.2000,140.5000) -- (88.2000,140.5000) -- (88.2000,140.5000) -- (88.2000,140.5000) -- (88.2000,140.5000) -- (88.2000,140.4000) -- (88.2000,140.4000) -- (88.2000,140.4000) -- (88.2000,140.4000) -- (88.2000,140.4000) -- (88.2000,140.4000) -- (88.2000,140.4000) -- (88.2000,140.4000) -- (88.2000,140.4000) -- (88.2000,140.4000) -- (88.2000,140.4000) -- (88.2000,140.4000) -- (88.2000,140.4000) -- (88.2000,140.4000) -- (88.2000,140.4000) -- (88.2000,140.4000) -- (88.2000,140.4000) -- (88.2000,140.4000) -- (88.2000,140.4000) -- (88.2000,140.4000) -- (88.2000,140.4000) -- (88.2000,140.3000) -- (88.2000,140.3000) -- (88.2000,140.3000) -- (88.2000,140.3000) -- (88.2000,140.3000) -- (88.2000,140.3000) -- (88.2000,140.3000) -- (88.2000,140.3000) -- (88.2000,140.3000) -- (88.2000,140.3000) -- (88.3000,140.3000) -- (88.3000,140.3000) -- (88.3000,140.3000) -- (88.3000,140.3000) -- (88.3000,140.3000) -- (88.3000,140.3000) -- (88.3000,140.3000) -- (88.3000,140.3000) -- (88.3000,140.3000) -- (88.3000,140.3000) -- (88.3000,140.3000) -- (88.3000,140.3000) -- (88.3000,140.2000) -- (88.3000,140.2000) -- (88.3000,140.2000) -- (88.3000,140.2000) -- (88.3000,140.2000) -- (88.3000,140.2000) -- (88.3000,140.2000) -- (88.3000,140.2000) -- (88.3000,140.2000) -- (88.3000,140.2000) -- (88.3000,140.2000) -- (88.3000,140.2000) -- (88.3000,140.2000) -- (88.3000,140.2000) -- (88.3000,140.2000) -- (88.3000,140.2000) -- (88.3000,140.2000) -- (88.3000,140.2000) -- (88.3000,140.2000) -- (88.3000,140.2000) -- (88.3000,140.2000) -- (88.3000,140.1000) -- (88.3000,140.1000) -- (88.3000,140.1000) -- (88.3000,140.1000) -- (88.3000,140.1000) -- (88.3000,140.1000) -- (88.3000,140.1000) -- (88.3000,140.1000) -- (88.3000,140.1000) -- (88.3000,140.1000) -- (88.3000,140.1000) -- (88.3000,140.1000) -- (88.3000,140.1000) -- (88.3000,140.1000) -- (88.3000,140.1000) -- (88.3000,140.1000) -- (88.4000,140.1000) -- (88.4000,140.1000) -- (88.4000,140.1000) -- (88.4000,140.1000) -- (88.4000,140.1000) -- (88.4000,140.1000) -- (88.4000,140.0000) -- (88.4000,140.0000) -- (88.4000,140.0000) -- (88.4000,140.0000) -- (88.4000,140.0000) -- (88.4000,140.0000) -- (88.4000,140.0000) -- (88.4000,140.0000) -- (88.4000,140.0000) -- (88.4000,140.0000) -- (88.4000,140.0000) -- (88.4000,140.0000) -- (88.4000,140.0000) -- (88.4000,140.0000) -- (88.4000,140.0000) -- (88.4000,140.0000) -- (88.4000,140.0000) -- (88.4000,140.0000) -- (88.4000,140.0000) -- (88.4000,140.0000) -- (88.4000,140.0000) -- (88.4000,139.9000) -- (88.4000,139.9000) -- (88.4000,139.9000) -- (88.4000,139.9000) -- (88.4000,139.9000) -- (88.4000,139.9000) -- (88.4000,139.9000) -- (88.4000,139.9000) -- (88.4000,139.9000) -- (88.4000,139.9000) -- (88.4000,139.9000) -- (88.4000,139.9000) -- (88.4000,139.9000) -- (88.4000,139.9000) -- (88.4000,139.9000) -- (88.4000,139.9000) -- (88.4000,139.9000) -- (88.4000,139.9000) -- (88.4000,139.9000) -- (88.4000,139.9000) -- (88.4000,139.9000) -- (88.4000,139.9000) -- (88.4000,139.8000) -- (88.5000,139.8000) -- (88.5000,139.8000) -- (88.5000,139.8000) -- (88.5000,139.8000) -- (88.5000,139.8000) -- (88.5000,139.8000) -- (88.5000,139.8000) -- (88.5000,139.8000) -- (88.5000,139.8000) -- (88.5000,139.8000) -- (88.5000,139.8000) -- (88.5000,139.8000) -- (88.5000,139.8000) -- (88.5000,139.8000) -- (88.5000,139.8000) -- (88.5000,139.8000) -- (88.5000,139.8000) -- (96.0000,139.8000) -- (96.0000,139.8000) -- (96.0000,139.8000) -- (96.0000,139.7000) -- (96.0000,139.7000) -- (96.0000,139.7000) -- (96.0000,139.7000) -- (96.0000,139.7000) -- (96.0000,139.7000) -- (96.0000,139.7000) -- (96.0000,139.7000) -- (96.0000,139.7000) -- (96.0000,139.7000) -- (96.0000,139.7000) -- (96.0000,139.7000) -- (96.0000,139.7000) -- (96.0000,139.7000) -- (96.0000,139.7000) -- (96.0000,139.7000) -- (96.0000,139.7000) -- (96.0000,139.7000) -- (96.0000,139.7000) -- (96.0000,139.7000) -- (96.0000,139.7000) -- (96.0000,139.7000) -- (96.0000,139.6000) -- (96.0000,139.6000) -- (96.0000,139.6000) -- (96.0000,139.6000) -- (96.0000,139.6000) -- (96.0000,139.6000) -- (96.0000,139.6000) -- (96.0000,139.6000) -- (96.0000,139.6000) -- (96.0000,139.6000) -- (96.0000,139.6000) -- (96.0000,139.6000) -- (96.0000,139.6000) -- (96.0000,139.6000) -- (96.1000,139.6000) -- (96.1000,139.6000) -- (96.1000,139.6000) -- (96.1000,139.6000) -- (96.1000,139.6000) -- (96.1000,139.6000) -- (96.1000,139.6000) -- (96.1000,139.5000) -- (96.1000,139.5000) -- (96.1000,139.5000) -- (96.1000,139.5000) -- (96.1000,139.5000) -- (96.1000,139.5000) -- (96.1000,139.5000) -- (96.1000,139.5000) -- (96.1000,139.5000) -- (96.1000,139.5000) -- (96.1000,139.5000) -- (96.1000,139.5000) -- (96.1000,139.5000) -- (96.1000,139.5000) -- (96.1000,139.5000) -- (96.1000,139.5000) -- (96.1000,139.5000) -- (96.1000,139.5000) -- (96.1000,139.5000) -- (96.1000,139.5000) -- (96.1000,139.5000) -- (96.1000,139.5000) -- (96.1000,139.4000) -- (96.1000,139.4000) -- (96.1000,139.4000) -- (96.1000,139.4000) -- (96.1000,139.4000) -- (96.1000,139.4000) -- (96.1000,139.4000) -- (96.1000,139.4000) -- (96.1000,139.4000) -- (96.1000,139.4000) -- (96.1000,139.4000) -- (96.1000,139.4000) -- (96.1000,139.4000) -- (96.1000,139.4000) -- (96.1000,139.4000) -- (96.1000,139.4000) -- (96.1000,139.4000) -- (96.1000,139.4000) -- (96.1000,139.4000) -- (96.1000,139.4000) -- (96.2000,139.4000) -- (96.2000,139.3000) -- (96.2000,139.3000) -- (96.2000,139.3000) -- (96.2000,139.3000) -- (96.2000,139.3000) -- (96.2000,139.3000) -- (96.2000,139.3000) -- (96.2000,139.3000) -- (96.2000,139.3000) -- (96.2000,139.3000) -- (96.2000,139.3000) -- (96.2000,139.3000) -- (96.2000,139.3000) -- (96.2000,139.3000) -- (96.2000,139.3000) -- (96.2000,139.3000) -- (96.2000,139.3000) -- (96.2000,139.3000) -- (96.2000,139.3000) -- (96.2000,139.3000) -- (96.2000,139.3000) -- (96.2000,139.3000) -- (96.2000,139.2000) -- (96.2000,139.2000) -- (96.2000,139.2000) -- (96.2000,139.2000) -- (96.2000,139.2000) -- (96.2000,139.2000) -- (96.2000,139.2000) -- (96.2000,139.2000) -- (96.2000,139.2000) -- (96.2000,139.2000) -- (96.2000,139.2000) -- (96.2000,139.2000) -- (96.2000,139.2000) -- (96.2000,139.2000) -- (96.2000,139.2000) -- (96.2000,139.2000) -- (96.2000,139.2000) -- (96.2000,139.2000) -- (96.2000,139.2000) -- (96.2000,139.2000) -- (96.2000,139.2000) -- (96.2000,139.1000) -- (96.2000,139.1000) -- (96.2000,139.1000) -- (96.2000,139.1000) -- (96.2000,139.1000) -- (96.2000,139.1000) -- (96.3000,139.1000) -- (96.3000,139.1000) -- (96.3000,139.1000) -- (96.3000,139.1000) -- (96.3000,139.1000) -- (96.3000,139.1000) -- (96.3000,139.1000) -- (96.3000,139.1000) -- (96.3000,139.1000) -- (96.3000,139.1000) -- (96.3000,139.1000) -- (96.3000,139.1000) -- (96.3000,139.1000) -- (96.3000,139.1000) -- (96.3000,139.1000) -- (96.3000,139.1000) -- (96.3000,139.0000) -- (96.3000,139.0000) -- (96.3000,139.0000) -- (96.3000,139.0000) -- (96.3000,139.0000) -- (96.3000,139.0000) -- (96.3000,139.0000) -- (96.3000,139.0000) -- (96.3000,139.0000) -- (96.3000,139.0000) -- (96.3000,139.0000) -- (96.3000,139.0000) -- (96.3000,139.0000) -- (96.3000,139.0000) -- (96.3000,139.0000) -- (96.3000,139.0000) -- (96.3000,139.0000) -- (96.3000,139.0000) -- (96.3000,139.0000) -- (96.3000,139.0000) -- (96.3000,139.0000) -- (96.3000,138.9000) -- (96.3000,138.9000) -- (96.3000,138.9000) -- (96.3000,138.9000) -- (96.3000,138.9000) -- (96.3000,138.9000) -- (96.3000,138.9000) -- (96.3000,138.9000) -- (96.3000,138.9000) -- (96.3000,138.9000) -- (96.3000,138.9000) -- (96.3000,138.9000) -- (96.3000,138.9000) -- (96.4000,138.9000) -- (96.4000,138.9000) -- (96.4000,138.9000) -- (96.4000,138.9000) -- (96.4000,138.9000) -- (96.4000,138.9000) -- (96.4000,138.9000) -- (96.4000,138.9000) -- (96.4000,138.9000) -- (96.4000,138.8000) -- (96.4000,138.8000) -- (96.4000,138.8000) -- (96.4000,138.8000) -- (96.4000,138.8000) -- (96.4000,138.8000) -- (96.4000,138.8000) -- (96.4000,138.8000) -- (96.4000,138.8000) -- (96.4000,138.8000) -- (96.4000,138.8000) -- (96.4000,138.8000) -- (96.4000,138.8000) -- (96.4000,138.8000) -- (96.4000,138.8000) -- (96.4000,138.8000) -- (96.4000,138.8000) -- (96.4000,138.8000) -- (96.4000,138.8000) -- (96.4000,138.8000) -- (96.4000,138.8000) -- (96.4000,138.7000) -- (96.4000,138.7000) -- (96.4000,138.7000) -- (96.4000,138.7000) -- (96.4000,138.7000) -- (96.4000,138.7000) -- (96.4000,138.7000) -- (96.4000,138.7000) -- (96.4000,138.7000) -- (96.4000,138.7000) -- (96.4000,138.7000) -- (96.4000,138.7000) -- (96.4000,138.7000) -- (96.4000,138.7000) -- (96.4000,138.7000) -- (96.4000,138.7000) -- (96.4000,138.7000) -- (96.4000,138.7000) -- (96.4000,138.7000) -- (96.5000,138.7000) -- (96.5000,138.7000) -- (96.5000,138.7000) -- (96.5000,138.6000) -- (96.5000,138.6000) -- (96.5000,138.6000) -- (96.5000,138.6000) -- (96.5000,138.6000) -- (96.5000,138.6000) -- (96.5000,138.6000) -- (96.5000,138.6000) -- (96.5000,138.6000) -- (96.5000,138.6000) -- (96.5000,138.6000) -- (96.5000,138.6000) -- (96.5000,138.6000) -- (96.5000,138.6000) -- (96.5000,138.6000) -- (96.5000,138.6000) -- (96.5000,138.6000) -- (96.5000,138.6000) -- (96.5000,138.6000) -- (96.5000,138.6000) -- (96.5000,138.6000) -- (96.5000,138.5000) -- (96.5000,138.5000) -- (96.5000,138.5000) -- (96.5000,138.5000) -- (96.5000,138.5000) -- (96.5000,138.5000) -- (96.5000,138.5000) -- (96.5000,138.5000) -- (96.5000,138.5000) -- (96.5000,138.5000) -- (96.5000,138.5000) -- (96.5000,138.5000) -- (96.5000,138.5000) -- (96.5000,138.5000) -- (96.5000,138.5000) -- (96.5000,138.5000) -- (96.5000,138.5000) -- (96.5000,138.5000) -- (96.5000,138.5000) -- (96.5000,138.5000) -- (96.5000,138.5000) -- (96.5000,138.5000) -- (96.5000,138.4000) -- (96.5000,138.4000) -- (96.5000,138.4000) -- (96.5000,138.4000) -- (96.6000,138.4000) -- (96.6000,138.4000) -- (96.6000,138.4000) -- (96.6000,138.4000) -- (96.6000,138.4000) -- (96.6000,138.4000) -- (96.6000,138.4000) -- (96.6000,138.4000) -- (96.6000,138.4000) -- (96.6000,138.4000) -- (96.6000,138.4000) -- (96.6000,138.4000) -- (96.6000,138.4000) -- (96.6000,138.4000) -- (96.6000,138.4000) -- (96.6000,138.4000) -- (96.6000,138.4000) -- (96.6000,138.3000) -- (96.6000,138.3000) -- (96.6000,138.3000) -- (96.6000,138.3000) -- (96.6000,138.3000) -- (96.6000,138.3000) -- (96.6000,138.3000) -- (96.6000,138.3000) -- (96.6000,138.3000) -- (96.6000,138.3000) -- (96.6000,138.3000) -- (96.6000,138.3000) -- (96.6000,138.3000) -- (96.6000,138.3000) -- (96.6000,138.3000) -- (96.6000,138.3000) -- (96.6000,138.3000) -- (96.6000,138.3000) -- (96.6000,138.3000) -- (96.6000,138.3000) -- (96.6000,138.3000) -- (96.6000,138.3000) -- (96.6000,138.2000) -- (96.6000,138.2000) -- (96.6000,138.2000) -- (96.6000,138.2000) -- (96.6000,138.2000) -- (96.6000,138.2000) -- (96.6000,138.2000) -- (96.6000,138.2000) -- (96.6000,138.2000) -- (96.6000,138.2000) -- (96.7000,138.2000) -- (96.7000,138.2000) -- (96.7000,138.2000) -- (96.7000,138.2000) -- (96.7000,138.2000) -- (96.7000,138.2000) -- (96.7000,138.2000) -- (96.7000,138.2000) -- (96.7000,138.2000) -- (96.7000,138.2000) -- (96.7000,138.2000) -- (96.7000,138.1000) -- (96.7000,138.1000) -- (96.7000,138.1000) -- (96.7000,138.1000) -- (96.7000,138.1000) -- (96.7000,138.1000) -- (96.7000,138.1000) -- (96.7000,138.1000) -- (96.7000,138.1000) -- (96.7000,138.1000) -- (96.7000,138.1000) -- (96.7000,138.1000) -- (96.7000,138.1000) -- (96.7000,138.1000) -- (96.7000,138.1000) -- (96.7000,138.1000) -- (96.7000,138.1000) -- (96.7000,138.1000) -- (96.7000,138.1000) -- (96.7000,138.1000) -- (96.7000,138.1000) -- (96.7000,138.1000) -- (96.7000,138.0000) -- (96.7000,138.0000) -- (96.7000,138.0000) -- (96.7000,138.0000) -- (96.7000,138.0000) -- (96.7000,138.0000) -- (96.7000,138.0000) -- (96.7000,138.0000) -- (96.7000,138.0000) -- (96.7000,138.0000) -- (96.7000,138.0000) -- (96.7000,138.0000) -- (96.7000,138.0000) -- (96.7000,138.0000) -- (96.7000,138.0000) -- (96.7000,138.0000) -- (96.7000,138.0000) -- (96.8000,138.0000) -- (96.8000,138.0000) -- (96.8000,138.0000) -- (96.8000,138.0000) -- (96.8000,137.9000) -- (96.8000,137.9000) -- (96.8000,137.9000) -- (96.8000,137.9000) -- (96.8000,137.9000) -- (96.8000,137.9000) -- (96.8000,137.9000) -- (96.8000,137.9000) -- (96.8000,137.9000) -- (96.8000,137.9000) -- (96.8000,137.9000) -- (96.8000,137.9000) -- (96.8000,137.9000) -- (96.8000,137.9000) -- (96.8000,137.9000) -- (96.8000,137.9000) -- (96.8000,137.9000) -- (96.8000,137.9000) -- (96.8000,137.9000) -- (96.8000,137.9000) -- (96.8000,137.9000) -- (96.8000,137.9000) -- (96.8000,137.8000) -- (96.8000,137.8000) -- (96.8000,137.8000) -- (96.8000,137.8000) -- (96.8000,137.8000) -- (96.8000,137.8000) -- (96.8000,137.8000) -- (96.8000,137.8000) -- (96.8000,137.8000) -- (96.8000,137.8000) -- (96.8000,137.8000) -- (96.8000,137.8000) -- (96.8000,137.8000) -- (96.8000,137.8000) -- (96.8000,137.8000) -- (96.8000,137.8000) -- (96.8000,137.8000) -- (96.8000,137.8000) -- (96.8000,137.8000) -- (96.8000,137.8000) -- (96.8000,137.8000) -- (96.8000,137.7000) -- (96.8000,137.7000) -- (96.9000,137.7000) -- (96.9000,137.7000) -- (96.9000,137.7000) -- (96.9000,137.7000) -- (96.9000,137.7000) -- (96.9000,137.7000) -- (96.9000,137.7000) -- (96.9000,137.7000) -- (96.9000,137.7000) -- (96.9000,137.7000) -- (96.9000,137.7000) -- (96.9000,137.7000) -- (96.9000,137.7000) -- (96.9000,137.7000) -- (96.9000,137.7000) -- (96.9000,137.7000) -- (96.9000,137.7000) -- (96.9000,137.7000) -- (96.9000,137.7000) -- (96.9000,137.7000) -- (96.9000,137.6000) -- (96.9000,137.6000) -- (96.9000,137.6000) -- (96.9000,137.6000) -- (96.9000,137.6000) -- (96.9000,137.6000) -- (96.9000,137.6000) -- (96.9000,137.6000) -- (96.9000,137.6000) -- (96.9000,137.6000) -- (96.9000,137.6000) -- (96.9000,137.6000) -- (96.9000,137.6000) -- (96.9000,137.6000) -- (96.9000,137.6000) -- (96.9000,137.6000) -- (96.9000,137.6000) -- (96.9000,137.6000) -- (96.9000,137.6000) -- (96.9000,137.6000) -- (96.9000,137.6000) -- (96.9000,137.5000) -- (96.9000,137.5000) -- (96.9000,137.5000) -- (96.9000,137.5000) -- (96.9000,137.5000) -- (96.9000,137.5000) -- (96.9000,137.5000) -- (96.9000,137.5000) -- (96.9000,137.5000) -- (97.0000,137.5000) -- (97.0000,137.5000) -- (97.0000,137.5000) -- (97.0000,137.5000) -- (97.0000,137.5000) -- (97.0000,137.5000) -- (97.0000,137.5000) -- (97.0000,137.5000) -- (97.0000,137.5000) -- (97.0000,137.5000) -- (97.0000,137.5000) -- (97.0000,137.5000) -- (97.0000,137.5000) -- (97.0000,137.4000) -- (97.0000,137.4000) -- (97.0000,137.4000) -- (97.0000,137.4000) -- (97.0000,137.4000) -- (97.0000,137.4000) -- (97.0000,137.4000) -- (97.0000,137.4000) -- (97.0000,137.4000) -- (97.0000,137.4000) -- (97.0000,137.4000) -- (97.0000,137.4000) -- (97.0000,137.4000) -- (97.0000,137.4000) -- (97.0000,137.4000) -- (97.0000,137.4000) -- (97.0000,137.4000) -- (97.0000,137.4000) -- (97.0000,137.4000) -- (97.0000,137.4000) -- (97.0000,137.4000) -- (97.0000,137.3000) -- (97.0000,137.3000) -- (97.0000,137.3000) -- (97.0000,137.3000) -- (97.0000,137.3000) -- (97.0000,137.3000) -- (97.0000,137.3000) -- (97.0000,137.3000) -- (97.0000,137.3000) -- (97.0000,137.3000) -- (97.0000,137.3000) -- (97.0000,137.3000) -- (97.0000,137.3000) -- (97.0000,137.3000) -- (97.0000,137.3000) -- (97.1000,137.3000) -- (97.1000,137.3000) -- (97.1000,137.3000) -- (97.1000,137.3000) -- (97.1000,137.3000) -- (97.1000,137.3000) -- (97.1000,137.3000) -- (97.1000,137.2000) -- (97.1000,137.2000) -- (97.1000,137.2000) -- (97.1000,137.2000) -- (97.1000,137.2000) -- (97.1000,137.2000) -- (97.1000,137.2000) -- (97.1000,137.2000) -- (97.1000,137.2000) -- (97.1000,137.2000) -- (97.1000,137.2000) -- (97.1000,137.2000) -- (97.1000,137.2000) -- (97.1000,137.2000) -- (97.1000,137.2000) -- (97.1000,137.2000) -- (97.1000,137.2000) -- (97.1000,137.2000) -- (97.1000,137.2000) -- (97.1000,137.2000) -- (97.1000,137.2000) -- (97.1000,137.1000) -- (97.1000,137.1000) -- (97.1000,137.1000) -- (97.1000,137.1000) -- (97.1000,137.1000) -- (97.1000,137.1000) -- (97.1000,137.1000) -- (97.1000,137.1000) -- (97.1000,137.1000) -- (97.1000,137.1000) -- (97.1000,137.1000) -- (97.1000,137.1000) -- (97.1000,137.1000) -- (97.1000,137.1000) -- (97.1000,137.1000) -- (97.1000,137.1000) -- (97.1000,137.1000) -- (97.1000,137.1000) -- (97.1000,137.1000) -- (97.1000,137.1000) -- (97.1000,137.1000) -- (97.1000,137.1000) -- (97.2000,137.0000) -- (97.2000,137.0000) -- (97.2000,137.0000) -- (97.2000,137.0000) -- (97.2000,137.0000) -- (97.2000,137.0000) -- (97.2000,137.0000) -- (97.2000,137.0000) -- (97.2000,137.0000) -- (97.2000,137.0000) -- (97.2000,137.0000) -- (97.2000,137.0000) -- (97.2000,137.0000) -- (97.2000,137.0000) -- (97.2000,137.0000) -- (97.2000,137.0000) -- (97.2000,137.0000) -- (97.2000,137.0000) -- (97.2000,137.0000) -- (97.2000,137.0000) -- (97.2000,137.0000) -- (97.2000,136.9000) -- (97.2000,136.9000) -- (97.2000,136.9000) -- (97.2000,136.9000) -- (97.2000,136.9000) -- (97.2000,136.9000) -- (97.2000,136.9000) -- (97.2000,136.9000) -- (97.2000,136.9000) -- (97.2000,136.9000) -- (97.2000,136.9000) -- (97.2000,136.9000) -- (97.2000,136.9000) -- (97.2000,136.9000) -- (97.2000,136.9000) -- (97.2000,136.9000) -- (97.2000,136.9000) -- (97.2000,136.9000) -- (97.2000,136.9000) -- (97.2000,136.9000) -- (97.2000,136.9000) -- (97.2000,136.9000) -- (97.2000,136.8000) -- (97.2000,136.8000) -- (97.2000,136.8000) -- (97.2000,136.8000) -- (97.2000,136.8000) -- (97.2000,136.8000) -- (97.3000,136.8000) -- (97.3000,136.8000) -- (97.3000,136.8000) -- (97.3000,136.8000) -- (97.3000,136.8000) -- (97.3000,136.8000) -- (97.3000,136.8000) -- (97.3000,136.8000) -- (97.3000,136.8000) -- (97.3000,136.8000) -- (97.3000,136.8000) -- (97.3000,136.8000) -- (97.3000,136.8000) -- (97.3000,136.8000) -- (97.3000,136.8000) -- (97.3000,136.7000) -- (97.3000,136.7000) -- (97.3000,136.7000) -- (97.3000,136.7000) -- (97.3000,136.7000) -- (97.3000,136.7000) -- (97.3000,136.7000) -- (97.3000,136.7000) -- (97.3000,136.7000) -- (97.3000,136.7000) -- (97.3000,136.7000) -- (97.3000,136.7000) -- (97.3000,136.7000) -- (97.3000,136.7000) -- (97.3000,136.7000) -- (97.3000,136.7000) -- (97.3000,136.7000) -- (97.3000,136.7000) -- (97.3000,136.7000) -- (97.3000,136.7000) -- (97.3000,136.7000) -- (97.3000,136.7000) -- (97.3000,136.6000) -- (97.3000,136.6000) -- (97.3000,136.6000) -- (97.3000,136.6000) -- (97.3000,136.6000) -- (97.3000,136.6000) -- (97.3000,136.6000) -- (97.3000,136.6000) -- (97.3000,136.6000) -- (97.3000,136.6000) -- (97.3000,136.6000) -- (97.3000,136.6000) -- (97.3000,136.6000) -- (97.4000,136.6000) -- (97.4000,136.6000) -- (97.4000,136.6000) -- (97.4000,136.6000) -- (97.4000,136.6000) -- (97.4000,136.6000) -- (97.4000,136.6000) -- (97.4000,136.6000) -- (97.4000,136.5000) -- (97.4000,136.5000) -- (97.4000,136.5000) -- (97.4000,136.5000) -- (97.4000,136.5000) -- (97.4000,136.5000) -- (97.4000,136.5000) -- (97.4000,136.5000) -- (97.4000,136.5000) -- (97.4000,136.5000) -- (97.4000,136.5000) -- (97.4000,136.5000) -- (97.4000,136.5000) -- (97.4000,136.5000) -- (97.4000,136.5000) -- (97.4000,136.5000) -- (97.4000,136.5000) -- (97.4000,136.5000) -- (97.4000,136.5000) -- (97.4000,136.5000) -- (97.4000,136.5000) -- (97.4000,136.5000) -- (97.4000,136.4000) -- (97.4000,136.4000) -- (97.4000,136.4000) -- (97.4000,136.4000) -- (97.4000,136.4000) -- (97.4000,136.4000) -- (97.4000,136.4000) -- (97.4000,136.4000) -- (97.4000,136.4000) -- (97.4000,136.4000) -- (97.4000,136.4000) -- (97.4000,136.4000) -- (97.4000,136.4000) -- (97.4000,136.4000) -- (97.4000,136.4000) -- (97.4000,136.4000) -- (97.4000,136.4000) -- (97.4000,136.4000) -- (97.4000,136.4000) -- (97.5000,136.4000) -- (97.5000,136.4000) -- (97.5000,136.3000) -- (97.5000,136.3000) -- (97.5000,136.3000) -- (97.5000,136.3000) -- (97.5000,136.3000) -- (97.5000,136.3000) -- (97.5000,136.3000) -- (97.5000,136.3000) -- (97.5000,136.3000) -- (97.5000,136.3000) -- (97.5000,136.3000) -- (97.5000,136.3000) -- (97.5000,136.3000) -- (97.5000,136.3000) -- (97.5000,136.3000) -- (97.5000,136.3000) -- (97.5000,136.3000) -- (97.5000,136.3000) -- (97.5000,136.3000) -- (97.5000,136.3000) -- (97.5000,136.3000) -- (97.5000,136.3000) -- (97.5000,136.2000) -- (97.5000,136.2000) -- (97.5000,136.2000) -- (97.5000,136.2000) -- (97.5000,136.2000) -- (97.5000,136.2000) -- (97.5000,136.2000) -- (97.5000,136.2000) -- (97.5000,136.2000) -- (97.5000,136.2000) -- (97.5000,136.2000) -- (97.5000,136.2000) -- (97.5000,136.2000) -- (97.5000,136.2000) -- (97.5000,136.2000) -- (97.5000,136.2000) -- (97.5000,136.2000) -- (97.5000,136.2000) -- (97.5000,136.2000) -- (97.5000,136.2000) -- (97.5000,136.2000) -- (97.5000,136.1000) -- (97.5000,136.1000) -- (97.5000,136.1000) -- (97.5000,136.1000) -- (97.5000,136.1000) -- (97.6000,136.1000) -- (97.6000,136.1000) -- (97.6000,136.1000) -- (97.6000,136.1000) -- (97.6000,136.1000) -- (97.6000,136.1000) -- (97.6000,136.1000) -- (97.6000,136.1000) -- (97.6000,136.1000) -- (97.6000,136.1000) -- (97.6000,136.1000) -- (97.6000,136.1000) -- (97.6000,136.1000) -- (97.6000,136.1000) -- (97.6000,136.1000) -- (97.6000,136.1000) -- (97.6000,136.1000) -- (97.6000,136.0000) -- (97.6000,136.0000) -- (97.6000,136.0000) -- (97.6000,136.0000) -- (97.6000,136.0000) -- (97.6000,136.0000) -- (97.6000,136.0000) -- (97.6000,136.0000) -- (97.6000,136.0000) -- (97.6000,136.0000) -- (97.6000,136.0000) -- (97.6000,136.0000) -- (97.6000,136.0000) -- (97.6000,136.0000) -- (97.6000,136.0000) -- (97.6000,136.0000) -- (97.6000,136.0000) -- (97.6000,136.0000) -- (97.6000,136.0000) -- (97.6000,136.0000) -- (97.6000,136.0000) -- (97.6000,135.9000) -- (97.6000,135.9000) -- (97.6000,135.9000) -- (97.6000,135.9000) -- (97.6000,135.9000) -- (97.6000,135.9000) -- (97.6000,135.9000) -- (97.6000,135.9000) -- (97.6000,135.9000) -- (97.6000,135.9000) -- (97.6000,135.9000) -- (97.7000,135.9000) -- (97.7000,135.9000) -- (97.7000,135.9000) -- (97.7000,135.9000) -- (97.7000,135.9000) -- (97.7000,135.9000) -- (97.7000,135.9000) -- (97.7000,135.9000) -- (97.7000,135.9000) -- (97.7000,135.9000) -- (97.7000,135.9000) -- (97.7000,135.8000) -- (97.7000,135.8000) -- (97.7000,135.8000) -- (97.7000,135.8000) -- (97.7000,135.8000) -- (97.7000,135.8000) -- (97.7000,135.8000) -- (97.7000,135.8000) -- (97.7000,135.8000) -- (97.7000,135.8000) -- (97.7000,135.8000) -- (97.7000,135.8000) -- (97.7000,135.8000) -- (97.7000,135.8000) -- (97.7000,135.8000) -- (97.7000,135.8000) -- (97.7000,135.8000) -- (97.7000,135.8000) -- (97.7000,135.8000) -- (97.7000,135.8000) -- (97.7000,135.8000) -- (97.7000,135.7000) -- (97.7000,135.7000) -- (97.7000,135.7000) -- (97.7000,135.7000) -- (97.7000,135.7000) -- (97.7000,135.7000) -- (97.7000,135.7000) -- (97.7000,135.7000) -- (97.7000,135.7000) -- (97.7000,135.7000) -- (97.7000,135.7000) -- (97.7000,135.7000) -- (97.7000,135.7000) -- (97.7000,135.7000) -- (97.7000,135.7000) -- (97.7000,135.7000) -- (97.7000,135.7000) -- (97.7000,135.7000) -- (97.8000,135.7000) -- (97.8000,135.7000) -- (97.8000,135.7000) -- (97.8000,135.7000) -- (97.8000,135.6000) -- (97.8000,135.6000) -- (97.8000,135.6000) -- (97.8000,135.6000) -- (97.8000,135.6000) -- (97.8000,135.6000) -- (97.8000,135.6000) -- (97.8000,135.6000) -- (97.8000,135.6000) -- (97.8000,135.6000) -- (97.8000,135.6000) -- (97.8000,135.6000) -- (97.8000,135.6000) -- (97.8000,135.6000) -- (97.8000,135.6000) -- (97.8000,135.6000) -- (97.8000,135.6000) -- (97.8000,135.6000) -- (97.8000,135.6000) -- (97.8000,135.6000) -- (97.8000,135.6000) -- (97.8000,135.5000) -- (97.8000,135.5000) -- (97.8000,135.5000) -- (97.8000,135.5000) -- (97.8000,135.5000) -- (97.8000,135.5000) -- (97.8000,135.5000) -- (97.8000,135.5000) -- (97.8000,135.5000) -- (97.8000,135.5000) -- (97.8000,135.5000) -- (97.8000,135.5000) -- (97.8000,135.5000) -- (97.8000,135.5000) -- (97.8000,135.5000) -- (97.8000,135.5000) -- (97.8000,135.5000) -- (97.8000,135.5000) -- (97.8000,135.5000) -- (97.8000,135.5000) -- (97.8000,135.5000) -- (97.8000,135.5000) -- (97.8000,135.4000) -- (97.8000,135.4000) -- (97.9000,135.4000) -- (97.9000,135.4000) -- (97.9000,135.4000) -- (97.9000,135.4000) -- (97.9000,135.4000) -- (97.9000,135.4000) -- (97.9000,135.4000) -- (97.9000,135.4000) -- (97.9000,135.4000) -- (97.9000,135.4000) -- (97.9000,135.4000) -- (97.9000,135.4000) -- (97.9000,135.4000) -- (97.9000,135.4000) -- (97.9000,135.4000) -- (97.9000,135.4000) -- (97.9000,135.4000) -- (97.9000,135.4000) -- (97.9000,135.4000) -- (97.9000,135.3000) -- (97.9000,135.3000) -- (97.9000,135.3000) -- (97.9000,135.3000) -- (97.9000,135.3000) -- (97.9000,135.3000) -- (97.9000,135.3000) -- (97.9000,135.3000) -- (97.9000,135.3000) -- (97.9000,135.3000) -- (97.9000,135.3000) -- (97.9000,135.3000) -- (97.9000,135.3000) -- (107.3000,135.3000) -- (107.3000,135.3000) -- (107.3000,135.3000) -- (107.4000,135.3000) -- (121.2657,135.3000);



    \end{scope}
    \begin{scope}[cm={{1.21653,0.0,0.0,1.34548,(-346.94368,-156.28627)}},draw=blue,line cap=round,line join=round,line width=0.480pt]
      \path[draw] (81.5000,129.5000) -- (81.5000,157.5000) -- (121.5000,157.5000) -- (121.5000,129.5000) -- (81.5000,129.5000);



    \end{scope}
    \path[cm={{0.99667,0.0,0.0,1.34693,(-320.19845,-156.47287)}},draw=blue] (41.5000,88.5000) -- (41.5000,164.5000) -- (127.5000,164.5000) -- (127.5000,88.5000) -- (41.5000,88.5000);



    \path[draw=cffffff,line cap=butt,line join=miter,line width=1.312pt,miter limit=4.00] (-300.8820,-23.4840) -- (-279.6507,-38.3822);



  \end{scope}
  \begin{scope}[cm={{1.26012,0.0,0.0,1.26012,(-619.61892,-54.78689)}},draw=ca0a0a4,dash pattern=on 0.95pt off 0.95pt,line cap=round,line join=round,line width=0.238pt,miter limit=4.00]
    \path[draw,dash pattern=on 0.95pt off 0.95pt,line width=0.238pt,miter limit=4.00] (165.5000,88.5000) -- (251.5000,88.5000);



  \end{scope}
  \begin{scope}[cm={{1.26012,0.0,0.0,1.26012,(-619.61892,-54.78689)}},draw=blue,line cap=round,line join=round,line width=0.480pt]
    \path[cm={{0.9189,0.0,0.0,1.0,(13.39849,0.0)}},draw] (165.5000,88.5000) -- (168.5000,88.5000);



    \path[cm={{0.9189,0.0,0.0,1.0,(20.28216,0.0)}},draw] (251.5000,88.5000) -- (248.5000,88.5000);



  \end{scope}
  \begin{scope}[scale=1.006,draw=blue,line cap=rect,line join=bevel,line width=0.800pt]
  \end{scope}
  \begin{scope}[cm={{1.00588,0.0,0.0,1.00588,(148.871,93.5471)}},draw=blue,line cap=rect,line join=bevel,line width=0.800pt]
  \end{scope}
  \begin{scope}[cm={{1.00588,0.0,0.0,1.00588,(148.871,93.5471)}},draw=blue,line cap=rect,line join=bevel,line width=0.800pt]
  \end{scope}
  \begin{scope}[cm={{1.00588,0.0,0.0,1.00588,(148.871,93.5471)}},draw=blue,line cap=rect,line join=bevel,line width=0.800pt]
  \end{scope}
  \begin{scope}[cm={{1.00588,0.0,0.0,1.00588,(148.871,93.5471)}},draw=blue,line cap=rect,line join=bevel,line width=0.800pt]
  \end{scope}
  \begin{scope}[cm={{1.00588,0.0,0.0,1.00588,(148.871,93.5471)}},draw=blue,line cap=rect,line join=bevel,line width=0.800pt]
  \end{scope}
  \begin{scope}[cm={{1.00588,0.0,0.0,1.00588,(-425.82403,58.61615)}},draw=blue,line cap=rect,line join=bevel,line width=0.800pt]
    \path[fill=blue] (0.0000,0.0000) node[above right] (text380) {27};



  \end{scope}
  \begin{scope}[cm={{1.00588,0.0,0.0,1.00588,(148.871,93.5471)}},draw=blue,line cap=rect,line join=bevel,line width=0.800pt]
  \end{scope}
  \begin{scope}[scale=1.006,draw=blue,line cap=rect,line join=bevel,line width=0.800pt]
  \end{scope}
  \begin{scope}[cm={{1.26012,0.0,0.0,1.26012,(-619.61892,-54.78689)}},draw=ca0a0a4,dash pattern=on 0.95pt off 0.95pt,line cap=round,line join=round,line width=0.238pt,miter limit=4.00]
    \path[draw,dash pattern=on 0.95pt off 0.95pt,line width=0.238pt,miter limit=4.00] (165.5000,63.5000) -- (251.5000,63.5000);



  \end{scope}
  \begin{scope}[cm={{1.26012,0.0,0.0,1.26012,(-619.61892,-54.78689)}},draw=blue,line cap=round,line join=round,line width=0.480pt]
    \path[cm={{0.9189,0.0,0.0,1.0,(13.39849,0.0)}},draw] (165.5000,63.5000) -- (168.5000,63.5000);



    \path[cm={{0.9189,0.0,0.0,1.0,(20.28216,0.0)}},draw] (251.5000,63.5000) -- (248.5000,63.5000);



  \end{scope}
  \begin{scope}[scale=1.006,draw=blue,line cap=rect,line join=bevel,line width=0.800pt]
  \end{scope}
  \begin{scope}[cm={{1.00588,0.0,0.0,1.00588,(149.876,68.4)}},draw=blue,line cap=rect,line join=bevel,line width=0.800pt]
  \end{scope}
  \begin{scope}[cm={{1.00588,0.0,0.0,1.00588,(149.876,68.4)}},draw=blue,line cap=rect,line join=bevel,line width=0.800pt]
  \end{scope}
  \begin{scope}[cm={{1.00588,0.0,0.0,1.00588,(149.876,68.4)}},draw=blue,line cap=rect,line join=bevel,line width=0.800pt]
  \end{scope}
  \begin{scope}[cm={{1.00588,0.0,0.0,1.00588,(149.876,68.4)}},draw=blue,line cap=rect,line join=bevel,line width=0.800pt]
  \end{scope}
  \begin{scope}[cm={{1.00588,0.0,0.0,1.00588,(149.876,68.4)}},draw=blue,line cap=rect,line join=bevel,line width=0.800pt]
  \end{scope}
  \begin{scope}[cm={{1.00588,0.0,0.0,1.00588,(-425.38144,27.46905)}},draw=blue,line cap=rect,line join=bevel,line width=0.800pt]
    \path[fill=blue] (0.0000,0.0000) node[above right] (text410) {31};



  \end{scope}
  \begin{scope}[cm={{1.00588,0.0,0.0,1.00588,(149.876,68.4)}},draw=blue,line cap=rect,line join=bevel,line width=0.800pt]
  \end{scope}
  \begin{scope}[scale=1.006,draw=blue,line cap=rect,line join=bevel,line width=0.800pt]
  \end{scope}
  \begin{scope}[cm={{1.26012,0.0,0.0,1.26012,(-482.17027,-168.78691)}},draw=ca0a0a4,dash pattern=on 0.95pt off 0.95pt,line cap=round,line join=round,line width=0.238pt,miter limit=4.00]
    \path[draw,dash pattern=on 0.95pt off 0.95pt,line width=0.238pt,miter limit=4.00] (108.5000,95.5000) -- (108.5000,36.5000);



    \path[draw,dash pattern=on 0.95pt off 0.95pt,line width=0.238pt,miter limit=4.00] (108.5000,20.5000) -- (108.5000,13.5000);



  \end{scope}
  \begin{scope}[cm={{1.26012,0.0,0.0,1.26012,(-619.61892,-54.78689)}},draw=ca0a0a4,dash pattern=on 0.95pt off 0.95pt,line cap=round,line join=round,line width=0.238pt,miter limit=4.00]
    \path[draw,dash pattern=on 0.95pt off 0.95pt,line width=0.238pt,miter limit=4.00] (165.5000,38.5000) -- (251.5000,38.5000);



  \end{scope}
  \begin{scope}[cm={{1.26012,0.0,0.0,1.26012,(-619.61892,-54.78689)}},draw=blue,line cap=round,line join=round,line width=0.480pt]
    \path[cm={{0.9189,0.0,0.0,1.0,(13.39849,0.0)}},draw] (165.5000,38.5000) -- (168.5000,38.5000);



    \path[cm={{0.9189,0.0,0.0,1.0,(20.28216,0.0)}},draw] (251.5000,38.5000) -- (248.5000,38.5000);



  \end{scope}
  \begin{scope}[scale=1.006,draw=blue,line cap=rect,line join=bevel,line width=0.800pt]
  \end{scope}
  \begin{scope}[cm={{1.00588,0.0,0.0,1.00588,(149.876,43.2529)}},draw=blue,line cap=rect,line join=bevel,line width=0.800pt]
  \end{scope}
  \begin{scope}[cm={{1.00588,0.0,0.0,1.00588,(149.876,43.2529)}},draw=blue,line cap=rect,line join=bevel,line width=0.800pt]
  \end{scope}
  \begin{scope}[cm={{1.00588,0.0,0.0,1.00588,(149.876,43.2529)}},draw=blue,line cap=rect,line join=bevel,line width=0.800pt]
  \end{scope}
  \begin{scope}[cm={{1.00588,0.0,0.0,1.00588,(149.876,43.2529)}},draw=blue,line cap=rect,line join=bevel,line width=0.800pt]
  \end{scope}
  \begin{scope}[cm={{1.00588,0.0,0.0,1.00588,(149.876,43.2529)}},draw=blue,line cap=rect,line join=bevel,line width=0.800pt]
  \end{scope}
  \begin{scope}[cm={{1.00588,0.0,0.0,1.00588,(-425.73551,-5.17805)}},draw=blue,line cap=rect,line join=bevel,line width=0.800pt]
    \path[fill=blue] (0.0000,0.0000) node[above right] (text440) {35};



  \end{scope}
  \begin{scope}[cm={{1.00588,0.0,0.0,1.00588,(149.876,43.2529)}},draw=blue,line cap=rect,line join=bevel,line width=0.800pt]
  \end{scope}
  \begin{scope}[scale=1.006,draw=blue,line cap=rect,line join=bevel,line width=0.800pt]
  \end{scope}
  \begin{scope}[cm={{1.26012,0.0,0.0,1.26012,(-619.61892,-54.78689)}},draw=ca0a0a4,dash pattern=on 0.40pt off 0.80pt,line cap=round,line join=round,line width=0.400pt]
    \path[draw] (165.5000,95.5000) -- (165.5000,13.5000);



  \end{scope}
  \begin{scope}[cm={{1.26012,0.0,0.0,1.26012,(-619.61892,-54.78689)}},draw=blue,line cap=round,line join=round,line width=0.480pt]
    \path[draw] (165.5000,95.5000) -- (165.5000,92.5000);



    \path[draw] (165.5000,13.5000) -- (165.5000,16.5000);



  \end{scope}
  \begin{scope}[scale=1.006,draw=blue,line cap=rect,line join=bevel,line width=0.800pt]
  \end{scope}
  \begin{scope}[cm={{1.00588,0.0,0.0,1.00588,(162.953,110.647)}},draw=blue,line cap=rect,line join=bevel,line width=0.800pt]
  \end{scope}
  \begin{scope}[cm={{1.00588,0.0,0.0,1.00588,(162.953,110.647)}},draw=blue,line cap=rect,line join=bevel,line width=0.800pt]
  \end{scope}
  \begin{scope}[cm={{1.00588,0.0,0.0,1.00588,(162.953,110.647)}},draw=blue,line cap=rect,line join=bevel,line width=0.800pt]
  \end{scope}
  \begin{scope}[cm={{1.00588,0.0,0.0,1.00588,(162.953,110.647)}},draw=blue,line cap=rect,line join=bevel,line width=0.800pt]
  \end{scope}
  \begin{scope}[cm={{1.00588,0.0,0.0,1.00588,(162.953,110.647)}},draw=blue,line cap=rect,line join=bevel,line width=0.800pt]
  \end{scope}
  \begin{scope}[cm={{1.00588,0.0,0.0,1.00588,(-413.68137,77.03586)}},draw=blue,line cap=rect,line join=bevel,line width=0.800pt]
    \path[fill=blue] (0.0000,0.0000) node[above right] (text470) {0};



  \end{scope}
  \begin{scope}[cm={{1.00588,0.0,0.0,1.00588,(162.953,110.647)}},draw=blue,line cap=rect,line join=bevel,line width=0.800pt]
  \end{scope}
  \begin{scope}[scale=1.006,draw=blue,line cap=rect,line join=bevel,line width=0.800pt]
  \end{scope}
  \begin{scope}[cm={{1.26012,0.0,0.0,1.26012,(-619.6494,-55.17358)}},draw=ca0a0a4,dash pattern=on 1.03pt off 1.03pt,line cap=round,line join=round,line width=0.257pt,miter limit=4.00]
    \path[draw,dash pattern=on 1.03pt off 1.03pt,line width=0.257pt,miter limit=4.00] (191.5000,95.5000) -- (191.5000,13.5000);



  \end{scope}
  \begin{scope}[cm={{1.26012,0.0,0.0,1.26012,(-482.17027,-168.78691)}},draw=ca0a0a4,dash pattern=on 0.95pt off 0.95pt,line cap=round,line join=round,line width=0.238pt,miter limit=4.00]
    \path[draw,dash pattern=on 0.95pt off 0.95pt,line width=0.238pt,miter limit=4.00] (82.5000,95.5000) -- (82.5000,36.5000);



    \path[draw,dash pattern=on 0.95pt off 0.95pt,line width=0.238pt,miter limit=4.00] (82.5000,20.5000) -- (82.5000,13.5000);



  \end{scope}
  \begin{scope}[cm={{1.26012,0.0,0.0,1.26012,(-619.61892,-54.78689)}},draw=blue,line cap=round,line join=round,line width=0.480pt]
    \path[cm={{1.0,0.0,0.0,0.9189,(0.0,7.78184)}},draw] (191.5000,95.5000) -- (191.5000,92.5000);



    \path[cm={{1.0,0.0,0.0,0.9189,(0.0,1.07058)}},draw] (191.5000,13.5000) -- (191.5000,16.5000);



  \end{scope}
  \begin{scope}[scale=1.006,draw=blue,line cap=rect,line join=bevel,line width=0.800pt]
  \end{scope}
  \begin{scope}[cm={{1.00588,0.0,0.0,1.00588,(189.106,110.647)}},draw=blue,line cap=rect,line join=bevel,line width=0.800pt]
  \end{scope}
  \begin{scope}[cm={{1.00588,0.0,0.0,1.00588,(189.106,110.647)}},draw=blue,line cap=rect,line join=bevel,line width=0.800pt]
  \end{scope}
  \begin{scope}[cm={{1.00588,0.0,0.0,1.00588,(189.106,110.647)}},draw=blue,line cap=rect,line join=bevel,line width=0.800pt]
  \end{scope}
  \begin{scope}[cm={{1.00588,0.0,0.0,1.00588,(189.106,110.647)}},draw=blue,line cap=rect,line join=bevel,line width=0.800pt]
  \end{scope}
  \begin{scope}[cm={{1.00588,0.0,0.0,1.00588,(189.106,110.647)}},draw=blue,line cap=rect,line join=bevel,line width=0.800pt]
  \end{scope}
  \begin{scope}[cm={{1.00588,0.0,0.0,1.00588,(-380.02837,77.03586)}},draw=blue,line cap=rect,line join=bevel,line width=0.800pt]
    \path[fill=blue] (0.0000,0.0000) node[above right] (text500) {1};



  \end{scope}
  \begin{scope}[cm={{1.00588,0.0,0.0,1.00588,(189.106,110.647)}},draw=blue,line cap=rect,line join=bevel,line width=0.800pt]
  \end{scope}
  \begin{scope}[scale=1.006,draw=blue,line cap=rect,line join=bevel,line width=0.800pt]
  \end{scope}
  \begin{scope}[cm={{1.26012,0.0,0.0,1.26012,(-619.6494,-55.17358)}},draw=ca0a0a4,dash pattern=on 1.03pt off 1.03pt,line cap=round,line join=round,line width=0.257pt,miter limit=4.00]
    \path[draw,dash pattern=on 1.03pt off 1.03pt,line width=0.257pt,miter limit=4.00] (217.5000,95.5000) -- (217.5000,13.5000);



  \end{scope}
  \begin{scope}[cm={{1.26012,0.0,0.0,1.26012,(-619.61892,-54.78689)}},draw=blue,line cap=round,line join=round,line width=0.480pt]
    \path[cm={{1.0,0.0,0.0,0.9189,(0.0,7.78184)}},draw] (217.5000,95.5000) -- (217.5000,92.5000);



    \path[cm={{1.0,0.0,0.0,0.9189,(0.0,1.07058)}},draw] (217.5000,13.5000) -- (217.5000,16.5000);



  \end{scope}
  \begin{scope}[scale=1.006,draw=blue,line cap=rect,line join=bevel,line width=0.800pt]
  \end{scope}
  \begin{scope}[cm={{1.00588,0.0,0.0,1.00588,(215.259,110.647)}},draw=blue,line cap=rect,line join=bevel,line width=0.800pt]
  \end{scope}
  \begin{scope}[cm={{1.00588,0.0,0.0,1.00588,(215.259,110.647)}},draw=blue,line cap=rect,line join=bevel,line width=0.800pt]
  \end{scope}
  \begin{scope}[cm={{1.00588,0.0,0.0,1.00588,(215.259,110.647)}},draw=blue,line cap=rect,line join=bevel,line width=0.800pt]
  \end{scope}
  \begin{scope}[cm={{1.00588,0.0,0.0,1.00588,(215.259,110.647)}},draw=blue,line cap=rect,line join=bevel,line width=0.800pt]
  \end{scope}
  \begin{scope}[cm={{1.00588,0.0,0.0,1.00588,(215.259,110.647)}},draw=blue,line cap=rect,line join=bevel,line width=0.800pt]
  \end{scope}
  \begin{scope}[cm={{1.00588,0.0,0.0,1.00588,(-347.87533,77.03586)}},draw=blue,line cap=rect,line join=bevel,line width=0.800pt]
    \path[fill=blue] (0.0000,0.0000) node[above right] (text530) {2};



  \end{scope}
  \begin{scope}[cm={{1.00588,0.0,0.0,1.00588,(215.259,110.647)}},draw=blue,line cap=rect,line join=bevel,line width=0.800pt]
  \end{scope}
  \begin{scope}[scale=1.006,draw=blue,line cap=rect,line join=bevel,line width=0.800pt]
  \end{scope}
  \begin{scope}[cm={{1.26012,0.0,0.0,1.26012,(-619.61892,-55.16996)}},draw=ca0a0a4,dash pattern=on 0.95pt off 0.95pt,line cap=round,line join=round,line width=0.238pt,miter limit=4.00]
    \path[draw,dash pattern=on 0.95pt off 0.95pt,line width=0.238pt,miter limit=4.00] (243.5000,95.5000) -- (243.5000,13.5000);



  \end{scope}
  \begin{scope}[cm={{1.26012,0.0,0.0,1.26012,(-619.61892,-54.78689)}},draw=blue,line cap=round,line join=round,line width=0.480pt]
    \path[cm={{1.0,0.0,0.0,0.9189,(0.0,7.78184)}},draw] (243.5000,95.5000) -- (243.5000,92.5000);



    \path[cm={{1.0,0.0,0.0,0.9189,(0.0,1.07058)}},draw] (243.5000,13.5000) -- (243.5000,16.5000);



  \end{scope}
  \begin{scope}[scale=1.006,draw=blue,line cap=rect,line join=bevel,line width=0.800pt]
  \end{scope}
  \begin{scope}[cm={{1.00588,0.0,0.0,1.00588,(241.915,110.647)}},draw=blue,line cap=rect,line join=bevel,line width=0.800pt]
  \end{scope}
  \begin{scope}[cm={{1.00588,0.0,0.0,1.00588,(241.915,110.647)}},draw=blue,line cap=rect,line join=bevel,line width=0.800pt]
  \end{scope}
  \begin{scope}[cm={{1.00588,0.0,0.0,1.00588,(241.915,110.647)}},draw=blue,line cap=rect,line join=bevel,line width=0.800pt]
  \end{scope}
  \begin{scope}[cm={{1.00588,0.0,0.0,1.00588,(241.915,110.647)}},draw=blue,line cap=rect,line join=bevel,line width=0.800pt]
  \end{scope}
  \begin{scope}[cm={{1.00588,0.0,0.0,1.00588,(241.915,110.647)}},draw=blue,line cap=rect,line join=bevel,line width=0.800pt]
  \end{scope}
  \begin{scope}[cm={{1.00588,0.0,0.0,1.00588,(-315.21933,77.03586)}},draw=blue,line cap=rect,line join=bevel,line width=0.800pt]
    \path[fill=blue] (0.0000,0.0000) node[above right] (text560) {3};



  \end{scope}
  \begin{scope}[cm={{1.00588,0.0,0.0,1.00588,(241.915,110.647)}},draw=blue,line cap=rect,line join=bevel,line width=0.800pt]
  \end{scope}
  \begin{scope}[scale=1.006,draw=blue,line cap=rect,line join=bevel,line width=0.800pt]
  \end{scope}
  \begin{scope}[cm={{1.26012,0.0,0.0,1.26012,(-619.61892,-54.78689)}},draw=blue,line cap=round,line join=round,line width=0.480pt]
    \path[draw] (165.5000,13.5000) -- (165.5000,95.5000) -- (251.5000,95.5000) -- (251.5000,13.5000) -- (165.5000,13.5000);



  \end{scope}
  \begin{scope}[scale=1.006,draw=blue,line cap=rect,line join=bevel,line width=0.800pt]
  \end{scope}
  \begin{scope}[scale=1.006,draw=blue,line cap=rect,line join=bevel,line width=0.800pt]
  \end{scope}
  \begin{scope}[scale=1.006,draw=blue,line cap=rect,line join=bevel,line width=0.800pt]
  \end{scope}
  \begin{scope}[scale=1.006,draw=blue,line cap=rect,line join=bevel,line width=0.800pt]
  \end{scope}
  \begin{scope}[scale=1.006,draw=blue,line cap=rect,line join=bevel,line width=0.800pt]
  \end{scope}
  \begin{scope}[cm={{1.00588,0.0,0.0,1.00588,(235.376,29.1706)}},draw=blue,line cap=rect,line join=bevel,line width=0.800pt]
  \end{scope}
  \begin{scope}[cm={{1.00588,0.0,0.0,1.00588,(235.376,29.1706)}},draw=blue,line cap=rect,line join=bevel,line width=0.800pt]
  \end{scope}
  \begin{scope}[cm={{1.00588,0.0,0.0,1.00588,(235.376,29.1706)}},draw=blue,line cap=rect,line join=bevel,line width=0.800pt]
  \end{scope}
  \begin{scope}[cm={{1.00588,0.0,0.0,1.00588,(235.376,29.1706)}},draw=blue,line cap=rect,line join=bevel,line width=0.800pt]
  \end{scope}
  \begin{scope}[cm={{1.00588,0.0,0.0,1.00588,(235.376,29.1706)}},draw=blue,line cap=rect,line join=bevel,line width=0.800pt]
  \end{scope}
  \begin{scope}[cm={{1.00588,0.0,0.0,1.00588,(235.376,29.1706)}},draw=blue,line cap=rect,line join=bevel,line width=0.800pt]
  \end{scope}
  \begin{scope}[scale=1.006,draw=blue,line cap=rect,line join=bevel,line width=0.800pt]
  \end{scope}
  \begin{scope}[scale=1.006,draw=blue,line cap=rect,line join=bevel,line width=0.800pt]
  \end{scope}
  \begin{scope}[scale=1.006,draw=blue,line cap=rect,line join=bevel,line width=0.800pt]
  \end{scope}
  \begin{scope}[cm={{1.26012,0.0,0.0,1.26012,(-619.61892,-54.78689)}},draw=blue,line cap=round,line join=round,line width=0.480pt]
    \path[draw] (165.3000,36.9000) -- (165.3000,36.9000) -- (165.4000,41.9000) -- (165.5000,44.7000) -- (165.7000,46.2000) -- (165.8000,46.7000) -- (166.0000,46.7000) -- (166.1000,46.7000) -- (166.2000,47.2000) -- (166.4000,47.4000) -- (166.5000,47.2000) -- (166.6000,46.7000) -- (166.8000,46.1000) -- (166.9000,45.5000) -- (167.1000,44.9000) -- (167.2000,44.5000) -- (167.3000,44.2000) -- (167.5000,44.2000) -- (167.6000,44.2000) -- (167.8000,44.2000) -- (167.9000,44.3000) -- (168.0000,44.4000) -- (168.2000,44.4000) -- (168.3000,44.5000) -- (168.4000,44.5000) -- (168.6000,44.5000) -- (168.7000,44.5000) -- (168.9000,44.5000) -- (169.0000,44.5000) -- (169.1000,44.5000) -- (169.3000,44.5000) -- (169.4000,44.5000) -- (169.5000,44.5000) -- (169.7000,44.5000) -- (169.8000,44.5000) -- (170.0000,44.5000) -- (170.1000,44.5000) -- (170.2000,44.5000) -- (170.4000,44.5000) -- (170.5000,44.5000) -- (170.7000,44.5000) -- (170.8000,44.5000) -- (170.9000,44.5000) -- (171.1000,44.5000) -- (171.2000,44.5000) -- (171.3000,44.5000) -- (171.5000,44.5000) -- (171.6000,44.5000) -- (171.8000,44.6000) -- (171.9000,44.7000) -- (172.0000,44.9000) -- (172.2000,45.2000) -- (172.3000,45.5000) -- (172.5000,45.8000) -- (172.6000,46.3000) -- (172.7000,46.7000) -- (172.9000,47.3000) -- (173.0000,47.8000) -- (173.1000,48.5000) -- (173.3000,49.1000) -- (173.4000,49.9000) -- (173.6000,50.7000) -- (173.7000,51.5000) -- (173.8000,52.4000) -- (174.0000,53.4000) -- (174.1000,54.4000) -- (174.2000,55.4000) -- (174.4000,56.5000) -- (174.5000,57.7000) -- (174.7000,58.9000) -- (174.8000,60.2000) -- (174.9000,61.5000) -- (175.1000,62.8000) -- (175.2000,64.2000) -- (175.4000,65.6000) -- (175.5000,67.1000) -- (175.6000,68.5000) -- (175.8000,70.0000) -- (175.9000,71.5000) -- (176.0000,73.0000) -- (176.2000,74.5000) -- (176.3000,76.0000) -- (176.5000,77.4000) -- (176.6000,78.8000) -- (176.7000,80.2000) -- (176.9000,81.6000) -- (177.0000,83.1000) -- (177.1000,84.5000) -- (177.3000,85.7000) -- (177.4000,86.5000) -- (177.6000,87.0000) -- (177.7000,87.2000) -- (177.8000,87.2000) -- (178.0000,87.1000) -- (178.1000,86.9000) -- (178.3000,86.7000) -- (178.4000,86.6000) -- (178.5000,86.5000) -- (178.7000,86.4000) -- (178.8000,86.3000) -- (178.9000,86.3000) -- (179.1000,86.3000) -- (179.2000,86.3000) -- (179.4000,86.3000) -- (179.5000,86.3000) -- (179.6000,86.3000) -- (179.8000,86.3000) -- (179.9000,86.3000) -- (180.0000,86.3000) -- (180.2000,86.3000) -- (180.3000,86.3000) -- (180.5000,86.3000) -- (180.6000,86.3000) -- (180.7000,86.3000) -- (180.9000,86.3000) -- (181.0000,86.3000) -- (181.2000,86.3000) -- (181.3000,86.3000) -- (181.4000,86.2000) -- (181.6000,86.0000) -- (181.7000,85.8000) -- (181.8000,85.9000) -- (182.0000,86.1000) -- (182.1000,86.2000) -- (182.3000,86.0000) -- (182.4000,85.7000) -- (182.5000,85.0000) -- (182.7000,84.0000) -- (182.8000,82.8000) -- (183.0000,81.4000) -- (183.1000,79.9000) -- (183.2000,78.2000) -- (183.4000,76.4000) -- (183.5000,74.6000) -- (183.6000,72.8000) -- (183.8000,70.9000) -- (183.9000,69.1000) -- (184.1000,67.3000) -- (184.2000,65.5000) -- (184.3000,63.8000) -- (184.5000,62.1000) -- (184.6000,60.5000) -- (184.7000,59.0000) -- (184.9000,57.6000) -- (185.0000,56.2000) -- (185.2000,54.9000) -- (185.3000,53.7000) -- (185.4000,52.5000) -- (185.6000,51.5000) -- (185.7000,50.5000) -- (185.9000,49.6000) -- (186.0000,48.7000) -- (186.1000,48.0000) -- (186.3000,47.3000) -- (186.4000,46.7000) -- (186.5000,46.1000) -- (186.7000,45.6000) -- (186.8000,45.2000) -- (187.0000,44.9000) -- (187.1000,44.6000) -- (187.2000,44.3000) -- (187.4000,44.1000) -- (187.5000,44.1000) -- (187.6000,44.1000) -- (187.8000,44.2000) -- (187.9000,44.3000) -- (188.1000,44.4000) -- (188.2000,44.4000) -- (188.3000,44.5000) -- (188.5000,44.5000) -- (188.6000,44.5000) -- (188.8000,44.5000) -- (188.9000,44.5000) -- (189.0000,44.5000) -- (189.2000,44.5000) -- (189.3000,44.5000) -- (189.4000,44.5000) -- (189.6000,44.5000) -- (189.7000,44.5000) -- (189.9000,44.5000) -- (190.0000,44.5000) -- (190.1000,44.5000) -- (190.3000,44.5000) -- (190.4000,44.5000) -- (190.6000,44.5000) -- (190.7000,44.5000) -- (190.8000,44.5000) -- (191.0000,44.5000) -- (191.1000,44.5000) -- (191.2000,44.5000) -- (191.4000,44.5000) -- (191.5000,44.5000) -- (191.7000,44.5000) -- (191.8000,44.5000) -- (191.9000,44.6000) -- (192.1000,44.7000) -- (192.2000,44.9000) -- (192.3000,45.1000) -- (192.5000,45.4000) -- (192.6000,45.8000) -- (192.8000,46.2000) -- (192.9000,46.6000) -- (193.0000,47.2000) -- (193.2000,47.7000) -- (193.3000,48.3000) -- (193.5000,49.0000) -- (193.6000,49.7000) -- (193.7000,50.5000) -- (193.9000,51.3000) -- (194.0000,52.2000) -- (194.1000,53.1000) -- (194.3000,54.1000) -- (194.4000,55.1000) -- (194.6000,56.2000) -- (194.7000,57.3000) -- (194.8000,58.5000) -- (195.0000,59.8000) -- (195.1000,61.0000) -- (195.2000,62.4000) -- (195.4000,63.7000) -- (195.5000,65.1000) -- (195.7000,66.5000) -- (195.8000,68.0000) -- (195.9000,69.5000) -- (196.1000,71.0000) -- (196.2000,72.4000) -- (196.4000,73.9000) -- (196.5000,75.4000) -- (196.6000,76.8000) -- (196.8000,78.2000) -- (196.9000,79.6000) -- (197.0000,81.0000) -- (197.2000,82.5000) -- (197.3000,84.0000) -- (197.5000,85.3000) -- (197.6000,86.3000) -- (197.7000,86.9000) -- (197.9000,87.2000) -- (198.0000,87.3000) -- (198.1000,87.2000) -- (198.3000,87.0000) -- (198.4000,86.8000) -- (198.6000,86.6000) -- (198.7000,86.5000) -- (198.8000,86.4000) -- (199.0000,86.3000) -- (199.1000,86.3000) -- (199.3000,86.3000) -- (199.4000,86.3000) -- (199.5000,86.3000) -- (199.7000,86.3000) -- (199.8000,86.3000) -- (199.9000,86.3000) -- (200.1000,86.3000) -- (200.2000,86.3000) -- (200.4000,86.3000) -- (200.5000,86.3000) -- (200.6000,86.3000) -- (200.8000,86.3000) -- (200.9000,86.3000) -- (201.1000,86.3000) -- (201.2000,86.3000) -- (201.3000,86.3000) -- (201.5000,86.3000) -- (201.6000,86.3000) -- (201.7000,86.1000) -- (201.9000,85.9000) -- (202.0000,85.9000) -- (202.2000,86.0000) -- (202.3000,86.1000) -- (202.4000,86.1000) -- (202.6000,85.9000) -- (202.7000,85.3000) -- (202.8000,84.5000) -- (203.0000,83.5000) -- (203.1000,82.2000) -- (203.3000,80.7000) -- (203.4000,79.1000) -- (203.5000,77.3000) -- (203.7000,75.5000) -- (203.8000,73.7000) -- (204.0000,71.9000) -- (204.1000,70.0000) -- (204.2000,68.2000) -- (204.4000,66.4000) -- (204.5000,64.7000) -- (204.6000,63.0000) -- (204.8000,61.4000) -- (204.9000,59.8000) -- (205.1000,58.3000) -- (205.2000,56.9000) -- (205.3000,55.5000) -- (205.5000,54.3000) -- (205.6000,53.1000) -- (205.7000,52.0000) -- (205.9000,51.0000) -- (206.0000,50.0000) -- (206.2000,49.1000) -- (206.3000,48.4000) -- (206.4000,47.6000) -- (206.6000,47.0000) -- (206.7000,46.4000) -- (206.9000,45.9000) -- (207.0000,45.4000) -- (207.1000,45.1000) -- (207.3000,44.8000) -- (207.4000,44.5000) -- (207.5000,44.2000) -- (207.7000,44.1000) -- (207.8000,44.1000) -- (208.0000,44.1000) -- (208.1000,44.2000) -- (208.2000,44.3000) -- (208.4000,44.4000) -- (208.5000,44.4000) -- (208.6000,44.5000) -- (208.8000,44.5000) -- (208.9000,44.5000) -- (209.1000,44.5000) -- (209.2000,44.5000) -- (209.3000,44.5000) -- (209.5000,44.5000) -- (209.6000,44.5000) -- (209.8000,44.5000) -- (209.9000,44.5000) -- (210.0000,44.5000) -- (210.2000,44.5000) -- (210.3000,44.5000) -- (210.4000,44.5000) -- (210.6000,44.5000) -- (210.7000,44.5000) -- (210.9000,44.5000) -- (211.0000,44.5000) -- (211.1000,44.5000) -- (211.3000,44.5000) -- (211.4000,44.5000) -- (211.5000,44.5000) -- (211.7000,44.5000) -- (211.8000,44.5000) -- (212.0000,44.5000) -- (212.1000,44.6000) -- (212.2000,44.7000) -- (212.4000,44.8000) -- (212.5000,45.0000) -- (212.7000,45.3000) -- (212.8000,45.6000) -- (212.9000,45.9000) -- (213.1000,46.4000) -- (213.2000,46.9000) -- (213.3000,47.4000) -- (213.5000,48.0000) -- (213.6000,48.6000) -- (213.8000,49.3000) -- (213.9000,50.0000) -- (214.0000,50.8000) -- (214.2000,51.7000) -- (214.3000,52.6000) -- (214.5000,53.5000) -- (214.6000,54.5000) -- (214.7000,55.6000) -- (214.9000,56.7000) -- (215.0000,57.8000) -- (215.1000,59.0000) -- (215.3000,60.3000) -- (215.4000,61.6000) -- (215.6000,62.9000) -- (215.7000,64.3000) -- (215.8000,65.7000) -- (216.0000,67.1000) -- (216.1000,68.6000) -- (216.2000,70.1000) -- (216.4000,71.6000) -- (216.5000,73.0000) -- (216.7000,74.5000) -- (216.8000,76.0000) -- (216.9000,77.4000) -- (217.1000,78.8000) -- (217.2000,80.1000) -- (217.4000,81.6000) -- (217.5000,83.1000) -- (217.6000,84.6000) -- (217.8000,85.8000) -- (217.9000,86.6000) -- (218.0000,87.0000) -- (218.2000,87.2000) -- (218.3000,87.2000) -- (218.5000,87.1000) -- (218.6000,86.9000) -- (218.7000,86.7000) -- (218.9000,86.6000) -- (219.0000,86.5000) -- (219.1000,86.4000) -- (219.3000,86.3000) -- (219.4000,86.3000) -- (219.6000,86.3000) -- (219.7000,86.3000) -- (219.8000,86.3000) -- (220.0000,86.3000) -- (220.1000,86.3000) -- (220.3000,86.3000) -- (220.4000,86.3000) -- (220.5000,86.3000) -- (220.7000,86.3000) -- (220.8000,86.3000) -- (220.9000,86.3000) -- (221.1000,86.3000) -- (221.2000,86.3000) -- (221.4000,86.3000) -- (221.5000,86.3000) -- (221.6000,86.3000) -- (221.8000,86.3000) -- (221.9000,86.2000) -- (222.0000,86.0000) -- (222.2000,85.9000) -- (222.3000,85.9000) -- (222.5000,86.1000) -- (222.6000,86.2000) -- (222.7000,86.1000) -- (222.9000,85.7000) -- (223.0000,85.1000) -- (223.2000,84.1000) -- (223.3000,83.0000) -- (223.4000,81.6000) -- (223.6000,80.1000) -- (223.7000,78.4000) -- (223.8000,76.6000) -- (224.0000,74.8000) -- (224.1000,73.0000) -- (224.3000,71.2000) -- (224.4000,69.3000) -- (224.5000,67.5000) -- (224.7000,65.7000) -- (224.8000,64.0000) -- (225.0000,62.3000) -- (225.1000,60.7000) -- (225.2000,59.2000) -- (225.4000,57.7000) -- (225.5000,56.3000) -- (225.6000,55.0000) -- (225.8000,53.8000) -- (225.9000,52.7000) -- (226.1000,51.6000) -- (226.2000,50.6000) -- (226.3000,49.7000) -- (226.5000,48.8000) -- (226.6000,48.1000) -- (226.7000,47.4000) -- (226.9000,46.7000) -- (227.0000,46.2000) -- (227.2000,45.7000) -- (227.3000,45.3000) -- (227.4000,44.9000) -- (227.6000,44.7000) -- (227.7000,44.4000) -- (227.9000,44.2000) -- (228.0000,44.1000) -- (228.1000,44.1000) -- (228.3000,44.2000) -- (228.4000,44.3000) -- (228.5000,44.3000) -- (228.7000,44.4000) -- (228.8000,44.5000) -- (229.0000,44.5000) -- (229.1000,44.5000) -- (229.2000,44.5000) -- (229.4000,44.5000) -- (229.5000,44.5000) -- (229.6000,44.5000) -- (229.8000,44.5000) -- (229.9000,44.5000) -- (230.1000,44.5000) -- (230.2000,44.5000) -- (230.3000,44.5000) -- (230.5000,44.5000) -- (230.6000,44.5000) -- (230.8000,44.5000) -- (230.9000,44.5000) -- (231.0000,44.5000) -- (231.2000,44.5000) -- (231.3000,44.5000) -- (231.4000,44.5000) -- (231.6000,44.5000) -- (231.7000,44.5000) -- (231.9000,44.5000) -- (232.0000,44.5000) -- (232.1000,44.5000) -- (232.3000,44.5000) -- (232.4000,44.6000) -- (232.6000,44.7000) -- (232.7000,44.9000) -- (232.8000,45.1000) -- (233.0000,45.4000) -- (233.1000,45.7000) -- (233.2000,46.1000) -- (233.4000,46.6000) -- (233.5000,47.1000) -- (233.7000,47.6000) -- (233.8000,48.2000) -- (233.9000,48.9000) -- (234.1000,49.6000) -- (234.2000,50.3000) -- (234.3000,51.1000) -- (234.5000,52.0000) -- (234.6000,52.9000) -- (234.8000,53.9000) -- (234.9000,54.9000) -- (235.0000,56.0000) -- (235.2000,57.1000) -- (235.3000,58.3000) -- (235.5000,59.5000) -- (235.6000,60.8000) -- (235.7000,62.1000) -- (235.9000,63.4000) -- (236.0000,64.8000) -- (236.1000,66.3000) -- (236.3000,67.7000) -- (236.4000,69.2000) -- (236.6000,70.6000) -- (236.7000,72.1000) -- (236.8000,73.6000) -- (237.0000,75.1000) -- (237.1000,76.5000) -- (237.2000,77.9000) -- (237.4000,79.3000) -- (237.5000,80.7000) -- (237.7000,82.2000) -- (237.8000,83.7000) -- (237.9000,85.1000) -- (238.1000,86.1000) -- (238.2000,86.8000) -- (238.4000,87.2000) -- (238.5000,87.3000) -- (238.6000,87.2000) -- (238.8000,87.0000) -- (238.9000,86.9000) -- (239.0000,86.7000) -- (239.2000,86.5000) -- (239.3000,86.4000) -- (239.5000,86.3000) -- (239.6000,86.3000) -- (239.7000,86.3000) -- (239.9000,86.3000) -- (240.0000,86.3000) -- (240.2000,86.3000) -- (240.3000,86.3000) -- (240.4000,86.3000) -- (240.6000,86.3000) -- (240.7000,86.3000) -- (240.8000,86.3000) -- (241.0000,86.3000) -- (241.1000,86.3000) -- (241.3000,86.3000) -- (241.4000,86.3000) -- (241.5000,86.3000) -- (241.7000,86.3000) -- (241.8000,86.3000) -- (241.9000,86.3000) -- (242.1000,86.3000) -- (242.2000,86.2000) -- (242.4000,86.0000) -- (242.5000,85.9000) -- (242.6000,86.0000) -- (242.8000,86.1000) -- (242.9000,86.1000) -- (243.1000,85.9000) -- (243.2000,85.5000) -- (243.3000,84.7000) -- (243.5000,83.7000) -- (243.6000,82.5000) -- (243.7000,81.0000) -- (243.9000,79.4000) -- (244.0000,77.7000) -- (244.2000,75.9000) -- (244.3000,74.1000) -- (244.4000,72.3000) -- (244.6000,70.4000) -- (244.7000,68.6000) -- (244.8000,66.8000) -- (245.0000,65.1000) -- (245.1000,63.4000) -- (245.3000,61.7000) -- (245.4000,60.1000) -- (245.5000,58.6000) -- (245.7000,57.2000) -- (245.8000,55.8000) -- (246.0000,54.6000) -- (246.1000,53.3000) -- (246.2000,52.2000) -- (246.4000,51.2000) -- (246.5000,50.2000) -- (246.6000,49.3000) -- (246.8000,48.5000) -- (246.9000,47.8000) -- (247.1000,47.1000) -- (247.2000,46.5000) -- (247.3000,46.0000) -- (247.5000,45.5000) -- (247.6000,45.1000) -- (247.7000,44.8000) -- (247.9000,44.5000) -- (248.0000,44.3000) -- (248.2000,44.1000) -- (248.3000,44.1000) -- (248.4000,44.1000) -- (248.6000,44.2000) -- (248.7000,44.3000) -- (248.9000,44.4000) -- (249.0000,44.4000) -- (249.1000,44.5000) -- (249.3000,44.5000) -- (249.4000,44.5000) -- (249.5000,44.5000) -- (249.7000,44.5000) -- (249.8000,44.5000) -- (250.0000,44.5000) -- (250.1000,44.5000) -- (250.2000,44.5000) -- (250.4000,44.5000) -- (250.5000,44.5000) -- (250.7000,44.5000) -- (250.8000,44.5000) -- (250.9000,44.5000) -- (251.1000,44.5000) -- (251.2000,44.5000) -- (251.3000,44.5000) -- (251.5000,44.5000) -- (251.6000,44.5000) -- (251.8000,44.5000) -- (251.9000,44.5000) -- (251.9000,44.5000);



  \end{scope}
  \begin{scope}[scale=1.006,draw=blue,line cap=rect,line join=bevel,line width=0.800pt]
  \end{scope}
  \begin{scope}[scale=1.006,draw=blue,line cap=rect,line join=bevel,line width=0.800pt]
  \end{scope}
  \begin{scope}[scale=1.006,draw=blue,line cap=rect,line join=bevel,line width=0.800pt]
  \end{scope}
  \begin{scope}[scale=1.006,draw=blue,line cap=rect,line join=bevel,line width=0.800pt]
  \end{scope}
  \begin{scope}[cm={{1.26012,0.0,0.0,1.26012,(-619.61892,-54.78689)}},draw=cff0000,line cap=round,line join=round,line width=0.480pt]
    \path[draw] (165.2000,51.8000) -- (165.2000,51.8000) -- (165.3000,51.0000) -- (165.4000,50.3000) -- (165.5000,49.6000) -- (165.5000,48.9000) -- (165.6000,48.3000) -- (165.7000,47.8000) -- (165.8000,47.2000) -- (165.9000,46.8000) -- (166.0000,46.4000) -- (166.1000,46.0000) -- (166.2000,45.7000) -- (166.2000,45.4000) -- (166.3000,45.2000) -- (166.4000,45.0000) -- (166.5000,44.8000) -- (166.6000,44.7000) -- (166.7000,44.7000) -- (166.8000,44.7000) -- (166.8000,44.7000) -- (166.9000,44.7000) -- (167.0000,44.8000) -- (167.1000,44.9000) -- (167.2000,45.1000) -- (167.3000,45.3000) -- (167.4000,45.4000) -- (167.5000,45.6000) -- (167.5000,45.9000) -- (167.6000,46.1000) -- (167.7000,46.4000) -- (167.8000,46.6000) -- (167.9000,46.9000) -- (168.0000,47.1000) -- (168.1000,47.4000) -- (168.2000,47.6000) -- (168.2000,47.8000) -- (168.3000,48.0000) -- (168.4000,48.2000) -- (168.5000,48.4000) -- (168.6000,48.6000) -- (168.7000,48.8000) -- (168.8000,48.9000) -- (168.8000,49.0000) -- (168.9000,49.1000) -- (169.0000,49.1000) -- (169.1000,49.1000) -- (169.2000,49.1000) -- (169.3000,49.1000) -- (169.4000,49.0000) -- (169.5000,48.9000) -- (169.5000,48.8000) -- (169.6000,48.6000) -- (169.7000,48.5000) -- (169.8000,48.3000) -- (169.9000,48.0000) -- (170.0000,47.8000) -- (170.1000,47.5000) -- (170.1000,47.2000) -- (170.2000,46.9000) -- (170.3000,46.6000) -- (170.4000,46.2000) -- (170.5000,45.9000) -- (170.6000,45.5000) -- (170.7000,45.1000) -- (170.8000,44.8000) -- (170.8000,44.4000) -- (170.9000,44.0000) -- (171.0000,43.7000) -- (171.1000,43.4000) -- (171.2000,43.0000) -- (171.3000,42.7000) -- (171.4000,42.5000) -- (171.4000,42.2000) -- (171.5000,42.0000) -- (171.6000,41.8000) -- (171.7000,41.6000) -- (171.8000,41.5000) -- (171.9000,41.4000) -- (172.0000,41.3000) -- (172.1000,41.3000) -- (172.1000,41.3000) -- (172.2000,41.4000) -- (172.3000,41.6000) -- (172.4000,41.7000) -- (172.5000,42.0000) -- (172.6000,42.3000) -- (172.7000,42.6000) -- (172.8000,43.0000) -- (172.8000,43.4000) -- (172.9000,43.9000) -- (173.0000,44.5000) -- (173.1000,45.1000) -- (173.2000,45.7000) -- (173.3000,46.4000) -- (173.4000,47.1000) -- (173.4000,47.9000) -- (173.5000,48.8000) -- (173.6000,49.6000) -- (173.7000,50.5000) -- (173.8000,51.5000) -- (173.9000,52.5000) -- (174.0000,53.5000) -- (174.1000,54.5000) -- (174.1000,55.6000) -- (174.2000,56.6000) -- (174.3000,57.7000) -- (174.4000,58.9000) -- (174.5000,60.0000) -- (174.6000,61.1000) -- (174.7000,62.2000) -- (174.7000,63.4000) -- (174.8000,64.5000) -- (174.9000,65.6000) -- (175.0000,66.8000) -- (175.1000,67.8000) -- (175.2000,68.9000) -- (175.3000,70.0000) -- (175.4000,71.0000) -- (175.4000,72.1000) -- (175.5000,73.0000) -- (175.6000,74.0000) -- (175.7000,74.9000) -- (175.8000,75.8000) -- (175.9000,76.7000) -- (176.0000,77.5000) -- (176.0000,78.3000) -- (176.1000,79.0000) -- (176.2000,79.7000) -- (176.3000,80.3000) -- (176.4000,80.9000) -- (176.5000,81.5000) -- (176.6000,82.0000) -- (176.7000,82.5000) -- (176.7000,83.0000) -- (176.8000,83.4000) -- (176.9000,83.7000) -- (177.0000,84.0000) -- (177.1000,84.3000) -- (177.2000,84.6000) -- (177.3000,84.8000) -- (177.4000,85.0000) -- (177.4000,85.1000) -- (177.5000,85.3000) -- (177.6000,85.3000) -- (177.7000,85.4000) -- (177.8000,85.5000) -- (177.9000,85.5000) -- (178.0000,85.5000) -- (178.0000,85.5000) -- (178.1000,85.5000) -- (178.2000,85.5000) -- (178.3000,85.4000) -- (178.4000,85.4000) -- (178.5000,85.4000) -- (178.6000,85.3000) -- (178.7000,85.3000) -- (178.7000,85.2000) -- (178.8000,85.2000) -- (178.9000,85.2000) -- (179.0000,85.1000) -- (179.1000,85.1000) -- (179.2000,85.1000) -- (179.3000,85.1000) -- (179.3000,85.1000) -- (179.4000,85.1000) -- (179.5000,85.1000) -- (179.6000,85.2000) -- (179.7000,85.2000) -- (179.8000,85.3000) -- (179.9000,85.4000) -- (180.0000,85.4000) -- (180.0000,85.5000) -- (180.1000,85.6000) -- (180.2000,85.7000) -- (180.3000,85.8000) -- (180.4000,85.9000) -- (180.5000,86.0000) -- (180.6000,86.1000) -- (180.6000,86.2000) -- (180.7000,86.3000) -- (180.8000,86.4000) -- (180.9000,86.4000) -- (181.0000,86.5000) -- (181.1000,86.5000) -- (181.2000,86.6000) -- (181.3000,86.6000) -- (181.3000,86.6000) -- (181.4000,86.5000) -- (181.5000,86.4000) -- (181.6000,86.4000) -- (181.7000,86.2000) -- (181.8000,86.1000) -- (181.9000,85.9000) -- (182.0000,85.6000) -- (182.0000,85.4000) -- (182.1000,85.1000) -- (182.2000,84.7000) -- (182.3000,84.3000) -- (182.4000,83.9000) -- (182.5000,83.5000) -- (182.6000,82.9000) -- (182.6000,82.4000) -- (182.7000,81.8000) -- (182.8000,81.2000) -- (182.9000,80.5000) -- (183.0000,79.8000) -- (183.1000,79.0000) -- (183.2000,78.2000) -- (183.3000,77.4000) -- (183.3000,76.5000) -- (183.4000,75.6000) -- (183.5000,74.7000) -- (183.6000,73.7000) -- (183.7000,72.7000) -- (183.8000,71.7000) -- (183.9000,70.7000) -- (183.9000,69.6000) -- (184.0000,68.6000) -- (184.1000,67.5000) -- (184.2000,66.4000) -- (184.3000,65.3000) -- (184.4000,64.2000) -- (184.5000,63.1000) -- (184.6000,62.1000) -- (184.6000,61.0000) -- (184.7000,59.9000) -- (184.8000,58.9000) -- (184.9000,57.9000) -- (185.0000,56.9000) -- (185.1000,55.9000) -- (185.2000,54.9000) -- (185.2000,54.0000) -- (185.3000,53.1000) -- (185.4000,52.3000) -- (185.5000,51.5000) -- (185.6000,50.7000) -- (185.7000,50.0000) -- (185.8000,49.3000) -- (185.9000,48.7000) -- (185.9000,48.1000) -- (186.0000,47.5000) -- (186.1000,47.0000) -- (186.2000,46.6000) -- (186.3000,46.2000) -- (186.4000,45.8000) -- (186.5000,45.5000) -- (186.5000,45.3000) -- (186.6000,45.0000) -- (186.7000,44.9000) -- (186.8000,44.8000) -- (186.9000,44.7000) -- (187.0000,44.6000) -- (187.1000,44.6000) -- (187.2000,44.7000) -- (187.2000,44.7000) -- (187.3000,44.8000) -- (187.4000,45.0000) -- (187.5000,45.1000) -- (187.6000,45.3000) -- (187.7000,45.5000) -- (187.8000,45.7000) -- (187.9000,46.0000) -- (187.9000,46.2000) -- (188.0000,46.4000) -- (188.1000,46.7000) -- (188.2000,47.0000) -- (188.3000,47.2000) -- (188.4000,47.4000) -- (188.5000,47.7000) -- (188.5000,47.9000) -- (188.6000,48.1000) -- (188.7000,48.3000) -- (188.8000,48.5000) -- (188.9000,48.7000) -- (189.0000,48.8000) -- (189.1000,48.9000) -- (189.2000,49.0000) -- (189.2000,49.1000) -- (189.3000,49.1000) -- (189.4000,49.1000) -- (189.5000,49.1000) -- (189.6000,49.1000) -- (189.7000,49.0000) -- (189.8000,48.9000) -- (189.8000,48.8000) -- (189.9000,48.6000) -- (190.0000,48.4000) -- (190.1000,48.2000) -- (190.2000,47.9000) -- (190.3000,47.7000) -- (190.4000,47.4000) -- (190.5000,47.1000) -- (190.5000,46.8000) -- (190.6000,46.4000) -- (190.7000,46.1000) -- (190.8000,45.7000) -- (190.9000,45.4000) -- (191.0000,45.0000) -- (191.1000,44.6000) -- (191.1000,44.3000) -- (191.2000,43.9000) -- (191.3000,43.6000) -- (191.4000,43.2000) -- (191.5000,42.9000) -- (191.6000,42.6000) -- (191.7000,42.3000) -- (191.8000,42.1000) -- (191.8000,41.9000) -- (191.9000,41.7000) -- (192.0000,41.5000) -- (192.1000,41.4000) -- (192.2000,41.3000) -- (192.3000,41.3000) -- (192.4000,41.3000) -- (192.5000,41.3000) -- (192.5000,41.4000) -- (192.6000,41.6000) -- (192.7000,41.8000) -- (192.8000,42.0000) -- (192.9000,42.4000) -- (193.0000,42.7000) -- (193.1000,43.1000) -- (193.1000,43.6000) -- (193.2000,44.1000) -- (193.3000,44.7000) -- (193.4000,45.3000) -- (193.5000,45.9000) -- (193.6000,46.7000) -- (193.7000,47.4000) -- (193.8000,48.2000) -- (193.8000,49.1000) -- (193.9000,50.0000) -- (194.0000,50.9000) -- (194.1000,51.8000) -- (194.2000,52.8000) -- (194.3000,53.9000) -- (194.4000,54.9000) -- (194.4000,56.0000) -- (194.5000,57.1000) -- (194.6000,58.2000) -- (194.7000,59.3000) -- (194.8000,60.4000) -- (194.9000,61.6000) -- (195.0000,62.7000) -- (195.1000,63.8000) -- (195.1000,65.0000) -- (195.2000,66.1000) -- (195.3000,67.2000) -- (195.4000,68.3000) -- (195.5000,69.4000) -- (195.6000,70.4000) -- (195.7000,71.5000) -- (195.7000,72.5000) -- (195.8000,73.4000) -- (195.9000,74.4000) -- (196.0000,75.3000) -- (196.1000,76.2000) -- (196.2000,77.0000) -- (196.3000,77.8000) -- (196.4000,78.6000) -- (196.4000,79.3000) -- (196.5000,80.0000) -- (196.6000,80.6000) -- (196.7000,81.2000) -- (196.8000,81.7000) -- (196.9000,82.3000) -- (197.0000,82.7000) -- (197.1000,83.2000) -- (197.1000,83.5000) -- (197.2000,83.9000) -- (197.3000,84.2000) -- (197.4000,84.5000) -- (197.5000,84.7000) -- (197.6000,84.9000) -- (197.7000,85.1000) -- (197.7000,85.2000) -- (197.8000,85.3000) -- (197.9000,85.4000) -- (198.0000,85.5000) -- (198.1000,85.5000) -- (198.2000,85.5000) -- (198.3000,85.5000) -- (198.4000,85.5000) -- (198.4000,85.5000) -- (198.5000,85.5000) -- (198.6000,85.4000) -- (198.7000,85.4000) -- (198.8000,85.3000) -- (198.9000,85.3000) -- (199.0000,85.2000) -- (199.0000,85.2000) -- (199.1000,85.2000) -- (199.2000,85.1000) -- (199.3000,85.1000) -- (199.4000,85.1000) -- (199.5000,85.1000) -- (199.6000,85.1000) -- (199.7000,85.1000) -- (199.7000,85.1000) -- (199.8000,85.2000) -- (199.9000,85.2000) -- (200.0000,85.2000) -- (200.1000,85.3000) -- (200.2000,85.4000) -- (200.3000,85.5000) -- (200.3000,85.5000) -- (200.4000,85.6000) -- (200.5000,85.7000) -- (200.6000,85.8000) -- (200.7000,85.9000) -- (200.8000,86.0000) -- (200.9000,86.1000) -- (201.0000,86.2000) -- (201.0000,86.3000) -- (201.1000,86.4000) -- (201.2000,86.5000) -- (201.3000,86.5000) -- (201.4000,86.6000) -- (201.5000,86.6000) -- (201.6000,86.6000) -- (201.7000,86.6000) -- (201.7000,86.5000) -- (201.8000,86.4000) -- (201.9000,86.3000) -- (202.0000,86.2000) -- (202.1000,86.0000) -- (202.2000,85.8000) -- (202.3000,85.6000) -- (202.3000,85.3000) -- (202.4000,85.0000) -- (202.5000,84.6000) -- (202.6000,84.2000) -- (202.7000,83.8000) -- (202.8000,83.3000) -- (202.9000,82.8000) -- (203.0000,82.2000) -- (203.0000,81.6000) -- (203.1000,80.9000) -- (203.2000,80.2000) -- (203.3000,79.5000) -- (203.4000,78.7000) -- (203.5000,77.9000) -- (203.6000,77.0000) -- (203.6000,76.2000) -- (203.7000,75.2000) -- (203.8000,74.3000) -- (203.9000,73.3000) -- (204.0000,72.3000) -- (204.1000,71.3000) -- (204.2000,70.3000) -- (204.3000,69.2000) -- (204.3000,68.1000) -- (204.4000,67.1000) -- (204.5000,66.0000) -- (204.6000,64.9000) -- (204.7000,63.8000) -- (204.8000,62.7000) -- (204.9000,61.6000) -- (204.9000,60.6000) -- (205.0000,59.5000) -- (205.1000,58.5000) -- (205.2000,57.5000) -- (205.3000,56.5000) -- (205.4000,55.5000) -- (205.5000,54.6000) -- (205.6000,53.7000) -- (205.6000,52.8000) -- (205.7000,51.9000) -- (205.8000,51.1000) -- (205.9000,50.4000) -- (206.0000,49.7000) -- (206.1000,49.0000) -- (206.2000,48.4000) -- (206.3000,47.8000) -- (206.3000,47.3000) -- (206.4000,46.8000) -- (206.5000,46.4000) -- (206.6000,46.0000) -- (206.7000,45.7000) -- (206.8000,45.4000) -- (206.9000,45.1000) -- (206.9000,45.0000) -- (207.0000,44.8000) -- (207.1000,44.7000) -- (207.2000,44.6000) -- (207.3000,44.6000) -- (207.4000,44.6000) -- (207.5000,44.7000) -- (207.6000,44.8000) -- (207.6000,44.9000) -- (207.7000,45.0000) -- (207.8000,45.2000) -- (207.9000,45.4000) -- (208.0000,45.6000) -- (208.1000,45.8000) -- (208.2000,46.0000) -- (208.2000,46.3000) -- (208.3000,46.5000) -- (208.4000,46.8000) -- (208.5000,47.1000) -- (208.6000,47.3000) -- (208.7000,47.5000) -- (208.8000,47.8000) -- (208.9000,48.0000) -- (208.9000,48.2000) -- (209.0000,48.4000) -- (209.1000,48.6000) -- (209.2000,48.8000) -- (209.3000,48.9000) -- (209.4000,49.0000) -- (209.5000,49.1000) -- (209.5000,49.1000) -- (209.6000,49.2000) -- (209.7000,49.2000) -- (209.8000,49.1000) -- (209.9000,49.1000) -- (210.0000,49.0000) -- (210.1000,48.9000) -- (210.2000,48.7000) -- (210.2000,48.5000) -- (210.3000,48.3000) -- (210.4000,48.1000) -- (210.5000,47.9000) -- (210.6000,47.6000) -- (210.7000,47.3000) -- (210.8000,47.0000) -- (210.9000,46.6000) -- (210.9000,46.3000) -- (211.0000,45.9000) -- (211.1000,45.6000) -- (211.2000,45.2000) -- (211.3000,44.8000) -- (211.4000,44.5000) -- (211.5000,44.1000) -- (211.5000,43.8000) -- (211.6000,43.4000) -- (211.7000,43.1000) -- (211.8000,42.8000) -- (211.9000,42.5000) -- (212.0000,42.2000) -- (212.1000,42.0000) -- (212.2000,41.8000) -- (212.2000,41.6000) -- (212.3000,41.4000) -- (212.4000,41.3000) -- (212.5000,41.3000) -- (212.6000,41.3000) -- (212.7000,41.3000) -- (212.8000,41.4000) -- (212.8000,41.5000) -- (212.9000,41.6000) -- (213.0000,41.9000) -- (213.1000,42.1000) -- (213.2000,42.5000) -- (213.3000,42.8000) -- (213.4000,43.3000) -- (213.5000,43.7000) -- (213.5000,44.3000) -- (213.6000,44.9000) -- (213.7000,45.5000) -- (213.8000,46.2000) -- (213.9000,46.9000) -- (214.0000,47.7000) -- (214.1000,48.5000) -- (214.1000,49.4000) -- (214.2000,50.3000) -- (214.3000,51.2000) -- (214.4000,52.2000) -- (214.5000,53.2000) -- (214.6000,54.2000) -- (214.7000,55.3000) -- (214.8000,56.4000) -- (214.8000,57.5000) -- (214.9000,58.6000) -- (215.0000,59.7000) -- (215.1000,60.9000) -- (215.2000,62.0000) -- (215.3000,63.1000) -- (215.4000,64.3000) -- (215.5000,65.4000) -- (215.5000,66.5000) -- (215.6000,67.6000) -- (215.7000,68.7000) -- (215.8000,69.8000) -- (215.9000,70.8000) -- (216.0000,71.9000) -- (216.1000,72.9000) -- (216.1000,73.8000) -- (216.2000,74.8000) -- (216.3000,75.7000) -- (216.4000,76.5000) -- (216.5000,77.4000) -- (216.6000,78.1000) -- (216.7000,78.9000) -- (216.8000,79.6000) -- (216.8000,80.2000) -- (216.9000,80.9000) -- (217.0000,81.4000) -- (217.1000,82.0000) -- (217.2000,82.5000) -- (217.3000,82.9000) -- (217.4000,83.3000) -- (217.4000,83.7000) -- (217.5000,84.0000) -- (217.6000,84.3000) -- (217.7000,84.6000) -- (217.8000,84.8000) -- (217.9000,85.0000) -- (218.0000,85.1000) -- (218.1000,85.3000) -- (218.1000,85.4000) -- (218.2000,85.4000) -- (218.3000,85.5000) -- (218.4000,85.5000) -- (218.5000,85.5000) -- (218.6000,85.5000) -- (218.7000,85.5000) -- (218.7000,85.5000) -- (218.8000,85.5000) -- (218.9000,85.4000) -- (219.0000,85.4000) -- (219.1000,85.3000) -- (219.2000,85.3000) -- (219.3000,85.2000) -- (219.4000,85.2000) -- (219.4000,85.2000) -- (219.5000,85.1000) -- (219.6000,85.1000) -- (219.7000,85.1000) -- (219.8000,85.1000) -- (219.9000,85.1000) -- (220.0000,85.1000) -- (220.1000,85.1000) -- (220.1000,85.2000) -- (220.2000,85.2000) -- (220.3000,85.3000) -- (220.4000,85.3000) -- (220.5000,85.4000) -- (220.6000,85.5000) -- (220.7000,85.6000) -- (220.7000,85.7000) -- (220.8000,85.8000) -- (220.9000,85.9000) -- (221.0000,86.0000) -- (221.1000,86.1000) -- (221.2000,86.2000) -- (221.3000,86.3000) -- (221.4000,86.4000) -- (221.4000,86.4000) -- (221.5000,86.5000) -- (221.6000,86.5000) -- (221.7000,86.6000) -- (221.8000,86.6000) -- (221.9000,86.6000) -- (222.0000,86.6000) -- (222.0000,86.5000) -- (222.1000,86.4000) -- (222.2000,86.3000) -- (222.3000,86.1000) -- (222.4000,86.0000) -- (222.5000,85.7000) -- (222.6000,85.5000) -- (222.7000,85.2000) -- (222.7000,84.9000) -- (222.8000,84.5000) -- (222.9000,84.1000) -- (223.0000,83.6000) -- (223.1000,83.1000) -- (223.2000,82.6000) -- (223.3000,82.0000) -- (223.3000,81.3000) -- (223.4000,80.7000) -- (223.5000,80.0000) -- (223.6000,79.2000) -- (223.7000,78.4000) -- (223.8000,77.6000) -- (223.9000,76.7000) -- (224.0000,75.8000) -- (224.0000,74.9000) -- (224.1000,73.9000) -- (224.2000,73.0000) -- (224.3000,71.9000) -- (224.4000,70.9000) -- (224.5000,69.9000) -- (224.6000,68.8000) -- (224.7000,67.7000) -- (224.7000,66.6000) -- (224.8000,65.6000) -- (224.9000,64.5000) -- (225.0000,63.4000) -- (225.1000,62.3000) -- (225.2000,61.2000) -- (225.3000,60.1000) -- (225.3000,59.1000) -- (225.4000,58.1000) -- (225.5000,57.1000) -- (225.6000,56.1000) -- (225.7000,55.1000) -- (225.8000,54.2000) -- (225.9000,53.3000) -- (226.0000,52.4000) -- (226.0000,51.6000) -- (226.1000,50.8000) -- (226.2000,50.1000) -- (226.3000,49.4000) -- (226.4000,48.7000) -- (226.5000,48.1000) -- (226.6000,47.6000) -- (226.6000,47.1000) -- (226.7000,46.6000) -- (226.8000,46.2000) -- (226.9000,45.8000) -- (227.0000,45.5000) -- (227.1000,45.3000) -- (227.2000,45.0000) -- (227.3000,44.9000) -- (227.3000,44.7000) -- (227.4000,44.6000) -- (227.5000,44.6000) -- (227.6000,44.6000) -- (227.7000,44.6000) -- (227.8000,44.7000) -- (227.9000,44.8000) -- (227.9000,44.9000) -- (228.0000,45.1000) -- (228.1000,45.2000) -- (228.2000,45.4000) -- (228.3000,45.7000) -- (228.4000,45.9000) -- (228.5000,46.1000) -- (228.6000,46.4000) -- (228.6000,46.6000) -- (228.7000,46.9000) -- (228.8000,47.2000) -- (228.9000,47.4000) -- (229.0000,47.6000) -- (229.1000,47.9000) -- (229.2000,48.1000) -- (229.2000,48.3000) -- (229.3000,48.5000) -- (229.4000,48.7000) -- (229.5000,48.8000) -- (229.6000,48.9000) -- (229.7000,49.0000) -- (229.8000,49.1000) -- (229.9000,49.2000) -- (229.9000,49.2000) -- (230.0000,49.2000) -- (230.1000,49.1000) -- (230.2000,49.1000) -- (230.3000,49.0000) -- (230.4000,48.8000) -- (230.5000,48.7000) -- (230.6000,48.5000) -- (230.6000,48.3000) -- (230.7000,48.0000) -- (230.8000,47.8000) -- (230.9000,47.5000) -- (231.0000,47.2000) -- (231.1000,46.9000) -- (231.2000,46.5000) -- (231.2000,46.2000) -- (231.3000,45.8000) -- (231.4000,45.4000) -- (231.5000,45.1000) -- (231.6000,44.7000) -- (231.7000,44.3000) -- (231.8000,44.0000) -- (231.9000,43.6000) -- (231.9000,43.3000) -- (232.0000,43.0000) -- (232.1000,42.6000) -- (232.2000,42.4000) -- (232.3000,42.1000) -- (232.4000,41.9000) -- (232.5000,41.7000) -- (232.5000,41.5000) -- (232.6000,41.4000) -- (232.7000,41.3000) -- (232.8000,41.2000) -- (232.9000,41.2000) -- (233.0000,41.3000) -- (233.1000,41.4000) -- (233.2000,41.5000) -- (233.2000,41.7000) -- (233.3000,41.9000) -- (233.4000,42.2000) -- (233.5000,42.6000) -- (233.6000,43.0000) -- (233.7000,43.4000) -- (233.8000,43.9000) -- (233.8000,44.5000) -- (233.9000,45.1000) -- (234.0000,45.7000) -- (234.1000,46.4000) -- (234.2000,47.2000) -- (234.3000,48.0000) -- (234.4000,48.8000) -- (234.5000,49.7000) -- (234.5000,50.6000) -- (234.6000,51.6000) -- (234.7000,52.6000) -- (234.8000,53.6000) -- (234.9000,54.6000) -- (235.0000,55.7000) -- (235.1000,56.8000) -- (235.1000,57.9000) -- (235.2000,59.0000) -- (235.3000,60.2000) -- (235.4000,61.3000) -- (235.5000,62.4000) -- (235.6000,63.6000) -- (235.7000,64.7000) -- (235.8000,65.8000) -- (235.8000,67.0000) -- (235.9000,68.1000) -- (236.0000,69.2000) -- (236.1000,70.2000) -- (236.2000,71.3000) -- (236.3000,72.3000) -- (236.4000,73.3000) -- (236.5000,74.2000) -- (236.5000,75.1000) -- (236.6000,76.0000) -- (236.7000,76.9000) -- (236.8000,77.7000) -- (236.9000,78.5000) -- (237.0000,79.2000) -- (237.1000,79.9000) -- (237.1000,80.5000) -- (237.2000,81.1000) -- (237.3000,81.7000) -- (237.4000,82.2000) -- (237.5000,82.7000) -- (237.6000,83.1000) -- (237.7000,83.5000) -- (237.8000,83.9000) -- (237.8000,84.2000) -- (237.9000,84.4000) -- (238.0000,84.7000) -- (238.1000,84.9000) -- (238.2000,85.1000) -- (238.3000,85.2000) -- (238.4000,85.3000) -- (238.4000,85.4000) -- (238.5000,85.5000) -- (238.6000,85.5000) -- (238.7000,85.5000) -- (238.8000,85.5000) -- (238.9000,85.5000) -- (239.0000,85.5000) -- (239.1000,85.5000) -- (239.1000,85.4000) -- (239.2000,85.4000) -- (239.3000,85.4000) -- (239.4000,85.3000) -- (239.5000,85.3000) -- (239.6000,85.2000) -- (239.7000,85.2000) -- (239.7000,85.1000) -- (239.8000,85.1000) -- (239.9000,85.1000) -- (240.0000,85.1000) -- (240.1000,85.1000) -- (240.2000,85.1000) -- (240.3000,85.1000) -- (240.4000,85.1000) -- (240.4000,85.2000) -- (240.5000,85.2000) -- (240.6000,85.3000) -- (240.7000,85.4000) -- (240.8000,85.4000) -- (240.9000,85.5000) -- (241.0000,85.6000) -- (241.0000,85.7000) -- (241.1000,85.8000) -- (241.2000,85.9000) -- (241.3000,86.0000) -- (241.4000,86.1000) -- (241.5000,86.2000) -- (241.6000,86.3000) -- (241.7000,86.4000) -- (241.7000,86.5000) -- (241.8000,86.5000) -- (241.9000,86.6000) -- (242.0000,86.6000) -- (242.1000,86.6000) -- (242.2000,86.6000) -- (242.3000,86.5000) -- (242.4000,86.5000) -- (242.4000,86.4000) -- (242.5000,86.3000) -- (242.6000,86.1000) -- (242.7000,85.9000) -- (242.8000,85.7000) -- (242.9000,85.4000) -- (243.0000,85.1000) -- (243.0000,84.7000) -- (243.1000,84.3000) -- (243.2000,83.9000) -- (243.3000,83.4000) -- (243.4000,82.9000) -- (243.5000,82.4000) -- (243.6000,81.8000) -- (243.7000,81.1000) -- (243.7000,80.4000) -- (243.8000,79.7000) -- (243.9000,78.9000) -- (244.0000,78.1000) -- (244.1000,77.3000) -- (244.2000,76.4000) -- (244.3000,75.5000) -- (244.3000,74.5000) -- (244.4000,73.6000) -- (244.5000,72.6000) -- (244.6000,71.6000) -- (244.7000,70.5000) -- (244.8000,69.5000) -- (244.9000,68.4000) -- (245.0000,67.3000) -- (245.0000,66.2000) -- (245.1000,65.1000) -- (245.2000,64.0000) -- (245.3000,62.9000) -- (245.4000,61.9000) -- (245.5000,60.8000) -- (245.6000,59.7000) -- (245.7000,58.7000) -- (245.7000,57.7000) -- (245.8000,56.7000) -- (245.9000,55.7000) -- (246.0000,54.7000) -- (246.1000,53.8000) -- (246.2000,52.9000) -- (246.3000,52.1000) -- (246.3000,51.3000) -- (246.4000,50.5000) -- (246.5000,49.8000) -- (246.6000,49.1000) -- (246.7000,48.5000) -- (246.8000,47.9000) -- (246.9000,47.3000) -- (247.0000,46.9000) -- (247.0000,46.4000) -- (247.1000,46.0000) -- (247.2000,45.7000) -- (247.3000,45.4000) -- (247.4000,45.1000) -- (247.5000,44.9000) -- (247.6000,44.8000) -- (247.6000,44.7000) -- (247.7000,44.6000) -- (247.8000,44.6000) -- (247.9000,44.6000) -- (248.0000,44.6000) -- (248.1000,44.7000) -- (248.2000,44.8000) -- (248.3000,45.0000) -- (248.3000,45.1000) -- (248.4000,45.3000) -- (248.5000,45.5000) -- (248.6000,45.7000) -- (248.7000,46.0000) -- (248.8000,46.2000) -- (248.9000,46.5000) -- (248.9000,46.7000) -- (249.0000,47.0000) -- (249.1000,47.3000) -- (249.2000,47.5000) -- (249.3000,47.7000) -- (249.4000,48.0000) -- (249.5000,48.2000) -- (249.6000,48.4000) -- (249.6000,48.6000) -- (249.7000,48.8000) -- (249.8000,48.9000) -- (249.9000,49.0000) -- (250.0000,49.1000) -- (250.1000,49.2000) -- (250.2000,49.2000) -- (250.3000,49.2000) -- (250.3000,49.2000) -- (250.4000,49.1000) -- (250.5000,49.0000) -- (250.6000,48.9000) -- (250.7000,48.8000) -- (250.8000,48.6000) -- (250.9000,48.4000) -- (250.9000,48.2000) -- (251.0000,47.9000) -- (251.1000,47.7000) -- (251.2000,47.4000) -- (251.3000,47.1000) -- (251.4000,46.7000) -- (251.5000,46.4000) -- (251.6000,46.0000) -- (251.6000,45.7000) -- (251.7000,45.3000) -- (251.8000,44.9000) -- (251.9000,44.5000);



  \end{scope}
  \begin{scope}[scale=1.006,draw=blue,line cap=rect,line join=bevel,line width=0.800pt]
  \end{scope}
  \begin{scope}[scale=1.006,draw=blue,line cap=rect,line join=bevel,line width=0.800pt]
  \end{scope}
  \begin{scope}[cm={{1.26012,0.0,0.0,1.26012,(-619.61892,-54.78689)}},draw=blue,line cap=round,line join=round,line width=0.480pt]
    \path[draw] (165.5000,13.5000) -- (165.5000,95.5000) -- (251.5000,95.5000) -- (251.5000,13.5000) -- (165.5000,13.5000);



  \end{scope}
  \begin{scope}[scale=1.006,draw=blue,line cap=rect,line join=bevel,line width=0.800pt]
  \end{scope}
  \begin{scope}[cm={{1.00588,0.0,0.0,1.00588,(39.2294,199.165)}},draw=blue,line cap=rect,line join=bevel,line width=0.800pt]
  \end{scope}
  \begin{scope}[cm={{1.00588,0.0,0.0,1.00588,(39.2294,199.165)}},draw=blue,line cap=rect,line join=bevel,line width=0.800pt]
  \end{scope}
  \begin{scope}[cm={{1.00588,0.0,0.0,1.00588,(39.2294,199.165)}},draw=blue,line cap=rect,line join=bevel,line width=0.800pt]
  \end{scope}
  \begin{scope}[cm={{1.00588,0.0,0.0,1.00588,(39.2294,199.165)}},draw=blue,line cap=rect,line join=bevel,line width=0.800pt]
  \end{scope}
  \begin{scope}[cm={{1.00588,0.0,0.0,1.00588,(39.2294,199.165)}},draw=blue,line cap=rect,line join=bevel,line width=0.800pt]
  \end{scope}
  \begin{scope}[cm={{1.00588,0.0,0.0,1.00588,(39.2294,199.165)}},draw=blue,line cap=rect,line join=bevel,line width=0.800pt]
  \end{scope}
  \begin{scope}[scale=1.006,draw=blue,line cap=rect,line join=bevel,line width=0.800pt]
  \end{scope}
  \begin{scope}[scale=1.006,draw=blue,line cap=rect,line join=bevel,line width=0.800pt]
  \end{scope}
  \begin{scope}[cm={{1.00588,0.0,0.0,1.00588,(40.2353,177.035)}},draw=blue,line cap=rect,line join=bevel,line width=0.800pt]
  \end{scope}
  \begin{scope}[cm={{1.00588,0.0,0.0,1.00588,(40.2353,177.035)}},draw=blue,line cap=rect,line join=bevel,line width=0.800pt]
  \end{scope}
  \begin{scope}[cm={{1.00588,0.0,0.0,1.00588,(40.2353,177.035)}},draw=blue,line cap=rect,line join=bevel,line width=0.800pt]
  \end{scope}
  \begin{scope}[cm={{1.00588,0.0,0.0,1.00588,(40.2353,177.035)}},draw=blue,line cap=rect,line join=bevel,line width=0.800pt]
  \end{scope}
  \begin{scope}[cm={{1.00588,0.0,0.0,1.00588,(40.2353,177.035)}},draw=blue,line cap=rect,line join=bevel,line width=0.800pt]
  \end{scope}
  \begin{scope}[cm={{1.00588,0.0,0.0,1.00588,(40.2353,177.035)}},draw=blue,line cap=rect,line join=bevel,line width=0.800pt]
  \end{scope}
  \begin{scope}[scale=1.006,draw=blue,line cap=rect,line join=bevel,line width=0.800pt]
  \end{scope}
  \begin{scope}[scale=1.006,draw=blue,line cap=rect,line join=bevel,line width=0.800pt]
  \end{scope}
  \begin{scope}[cm={{1.00588,0.0,0.0,1.00588,(40.2353,153.9)}},draw=blue,line cap=rect,line join=bevel,line width=0.800pt]
  \end{scope}
  \begin{scope}[cm={{1.00588,0.0,0.0,1.00588,(40.2353,153.9)}},draw=blue,line cap=rect,line join=bevel,line width=0.800pt]
  \end{scope}
  \begin{scope}[cm={{1.00588,0.0,0.0,1.00588,(40.2353,153.9)}},draw=blue,line cap=rect,line join=bevel,line width=0.800pt]
  \end{scope}
  \begin{scope}[cm={{1.00588,0.0,0.0,1.00588,(40.2353,153.9)}},draw=blue,line cap=rect,line join=bevel,line width=0.800pt]
  \end{scope}
  \begin{scope}[cm={{1.00588,0.0,0.0,1.00588,(40.2353,153.9)}},draw=blue,line cap=rect,line join=bevel,line width=0.800pt]
  \end{scope}
  \begin{scope}[cm={{1.00588,0.0,0.0,1.00588,(40.2353,153.9)}},draw=blue,line cap=rect,line join=bevel,line width=0.800pt]
  \end{scope}
  \begin{scope}[scale=1.006,draw=blue,line cap=rect,line join=bevel,line width=0.800pt]
  \end{scope}
  \begin{scope}[scale=1.006,draw=blue,line cap=rect,line join=bevel,line width=0.800pt]
  \end{scope}
  \begin{scope}[cm={{1.00588,0.0,0.0,1.00588,(39.2294,131.771)}},draw=blue,line cap=rect,line join=bevel,line width=0.800pt]
  \end{scope}
  \begin{scope}[cm={{1.00588,0.0,0.0,1.00588,(39.2294,131.771)}},draw=blue,line cap=rect,line join=bevel,line width=0.800pt]
  \end{scope}
  \begin{scope}[cm={{1.00588,0.0,0.0,1.00588,(39.2294,131.771)}},draw=blue,line cap=rect,line join=bevel,line width=0.800pt]
  \end{scope}
  \begin{scope}[cm={{1.00588,0.0,0.0,1.00588,(39.2294,131.771)}},draw=blue,line cap=rect,line join=bevel,line width=0.800pt]
  \end{scope}
  \begin{scope}[cm={{1.00588,0.0,0.0,1.00588,(39.2294,131.771)}},draw=blue,line cap=rect,line join=bevel,line width=0.800pt]
  \end{scope}
  \begin{scope}[cm={{1.00588,0.0,0.0,1.00588,(39.2294,131.771)}},draw=blue,line cap=rect,line join=bevel,line width=0.800pt]
  \end{scope}
  \begin{scope}[scale=1.006,draw=blue,line cap=rect,line join=bevel,line width=0.800pt]
  \end{scope}
  \begin{scope}[scale=1.006,draw=blue,line cap=rect,line join=bevel,line width=0.800pt]
  \end{scope}
  \begin{scope}[cm={{1.00588,0.0,0.0,1.00588,(53.3118,217.271)}},draw=blue,line cap=rect,line join=bevel,line width=0.800pt]
  \end{scope}
  \begin{scope}[cm={{1.00588,0.0,0.0,1.00588,(53.3118,217.271)}},draw=blue,line cap=rect,line join=bevel,line width=0.800pt]
  \end{scope}
  \begin{scope}[cm={{1.00588,0.0,0.0,1.00588,(53.3118,217.271)}},draw=blue,line cap=rect,line join=bevel,line width=0.800pt]
  \end{scope}
  \begin{scope}[cm={{1.00588,0.0,0.0,1.00588,(53.3118,217.271)}},draw=blue,line cap=rect,line join=bevel,line width=0.800pt]
  \end{scope}
  \begin{scope}[cm={{1.00588,0.0,0.0,1.00588,(53.3118,217.271)}},draw=blue,line cap=rect,line join=bevel,line width=0.800pt]
  \end{scope}
  \begin{scope}[cm={{1.00588,0.0,0.0,1.00588,(53.3118,217.271)}},draw=blue,line cap=rect,line join=bevel,line width=0.800pt]
  \end{scope}
  \begin{scope}[scale=1.006,draw=blue,line cap=rect,line join=bevel,line width=0.800pt]
  \end{scope}
  \begin{scope}[scale=1.006,draw=blue,line cap=rect,line join=bevel,line width=0.800pt]
  \end{scope}
  \begin{scope}[cm={{1.00588,0.0,0.0,1.00588,(82.4824,217.271)}},draw=blue,line cap=rect,line join=bevel,line width=0.800pt]
  \end{scope}
  \begin{scope}[cm={{1.00588,0.0,0.0,1.00588,(82.4824,217.271)}},draw=blue,line cap=rect,line join=bevel,line width=0.800pt]
  \end{scope}
  \begin{scope}[cm={{1.00588,0.0,0.0,1.00588,(82.4824,217.271)}},draw=blue,line cap=rect,line join=bevel,line width=0.800pt]
  \end{scope}
  \begin{scope}[cm={{1.00588,0.0,0.0,1.00588,(82.4824,217.271)}},draw=blue,line cap=rect,line join=bevel,line width=0.800pt]
  \end{scope}
  \begin{scope}[cm={{1.00588,0.0,0.0,1.00588,(82.4824,217.271)}},draw=blue,line cap=rect,line join=bevel,line width=0.800pt]
  \end{scope}
  \begin{scope}[cm={{1.00588,0.0,0.0,1.00588,(82.4824,217.271)}},draw=blue,line cap=rect,line join=bevel,line width=0.800pt]
  \end{scope}
  \begin{scope}[scale=1.006,draw=blue,line cap=rect,line join=bevel,line width=0.800pt]
  \end{scope}
  \begin{scope}[scale=1.006,draw=blue,line cap=rect,line join=bevel,line width=0.800pt]
  \end{scope}
  \begin{scope}[cm={{1.00588,0.0,0.0,1.00588,(111.653,217.271)}},draw=blue,line cap=rect,line join=bevel,line width=0.800pt]
  \end{scope}
  \begin{scope}[cm={{1.00588,0.0,0.0,1.00588,(111.653,217.271)}},draw=blue,line cap=rect,line join=bevel,line width=0.800pt]
  \end{scope}
  \begin{scope}[cm={{1.00588,0.0,0.0,1.00588,(111.653,217.271)}},draw=blue,line cap=rect,line join=bevel,line width=0.800pt]
  \end{scope}
  \begin{scope}[cm={{1.00588,0.0,0.0,1.00588,(111.653,217.271)}},draw=blue,line cap=rect,line join=bevel,line width=0.800pt]
  \end{scope}
  \begin{scope}[cm={{1.00588,0.0,0.0,1.00588,(111.653,217.271)}},draw=blue,line cap=rect,line join=bevel,line width=0.800pt]
  \end{scope}
  \begin{scope}[cm={{1.00588,0.0,0.0,1.00588,(111.653,217.271)}},draw=blue,line cap=rect,line join=bevel,line width=0.800pt]
  \end{scope}
  \begin{scope}[scale=1.006,draw=blue,line cap=rect,line join=bevel,line width=0.800pt]
  \end{scope}
  \begin{scope}[scale=1.006,draw=blue,line cap=rect,line join=bevel,line width=0.800pt]
  \end{scope}
  \begin{scope}[cm={{1.00588,0.0,0.0,1.00588,(142.332,217.271)}},draw=blue,line cap=rect,line join=bevel,line width=0.800pt]
  \end{scope}
  \begin{scope}[cm={{1.00588,0.0,0.0,1.00588,(142.332,217.271)}},draw=blue,line cap=rect,line join=bevel,line width=0.800pt]
  \end{scope}
  \begin{scope}[cm={{1.00588,0.0,0.0,1.00588,(142.332,217.271)}},draw=blue,line cap=rect,line join=bevel,line width=0.800pt]
  \end{scope}
  \begin{scope}[cm={{1.00588,0.0,0.0,1.00588,(142.332,217.271)}},draw=blue,line cap=rect,line join=bevel,line width=0.800pt]
  \end{scope}
  \begin{scope}[cm={{1.00588,0.0,0.0,1.00588,(142.332,217.271)}},draw=blue,line cap=rect,line join=bevel,line width=0.800pt]
  \end{scope}
  \begin{scope}[cm={{1.00588,0.0,0.0,1.00588,(142.332,217.271)}},draw=blue,line cap=rect,line join=bevel,line width=0.800pt]
  \end{scope}
  \begin{scope}[scale=1.006,draw=blue,line cap=rect,line join=bevel,line width=0.800pt]
  \end{scope}
  \begin{scope}[scale=1.006,draw=blue,line cap=rect,line join=bevel,line width=0.800pt]
  \end{scope}
  \begin{scope}[cm={{1.00588,0.0,0.0,1.00588,(171.503,217.271)}},draw=blue,line cap=rect,line join=bevel,line width=0.800pt]
  \end{scope}
  \begin{scope}[cm={{1.00588,0.0,0.0,1.00588,(171.503,217.271)}},draw=blue,line cap=rect,line join=bevel,line width=0.800pt]
  \end{scope}
  \begin{scope}[cm={{1.00588,0.0,0.0,1.00588,(171.503,217.271)}},draw=blue,line cap=rect,line join=bevel,line width=0.800pt]
  \end{scope}
  \begin{scope}[cm={{1.00588,0.0,0.0,1.00588,(171.503,217.271)}},draw=blue,line cap=rect,line join=bevel,line width=0.800pt]
  \end{scope}
  \begin{scope}[cm={{1.00588,0.0,0.0,1.00588,(171.503,217.271)}},draw=blue,line cap=rect,line join=bevel,line width=0.800pt]
  \end{scope}
  \begin{scope}[cm={{1.00588,0.0,0.0,1.00588,(171.503,217.271)}},draw=blue,line cap=rect,line join=bevel,line width=0.800pt]
  \end{scope}
  \begin{scope}[scale=1.006,draw=blue,line cap=rect,line join=bevel,line width=0.800pt]
  \end{scope}
  \begin{scope}[scale=1.006,draw=blue,line cap=rect,line join=bevel,line width=0.800pt]
  \end{scope}
  \begin{scope}[cm={{1.00588,0.0,0.0,1.00588,(200.674,217.271)}},draw=blue,line cap=rect,line join=bevel,line width=0.800pt]
  \end{scope}
  \begin{scope}[cm={{1.00588,0.0,0.0,1.00588,(200.674,217.271)}},draw=blue,line cap=rect,line join=bevel,line width=0.800pt]
  \end{scope}
  \begin{scope}[cm={{1.00588,0.0,0.0,1.00588,(200.674,217.271)}},draw=blue,line cap=rect,line join=bevel,line width=0.800pt]
  \end{scope}
  \begin{scope}[cm={{1.00588,0.0,0.0,1.00588,(200.674,217.271)}},draw=blue,line cap=rect,line join=bevel,line width=0.800pt]
  \end{scope}
  \begin{scope}[cm={{1.00588,0.0,0.0,1.00588,(200.674,217.271)}},draw=blue,line cap=rect,line join=bevel,line width=0.800pt]
  \end{scope}
  \begin{scope}[cm={{1.00588,0.0,0.0,1.00588,(200.674,217.271)}},draw=blue,line cap=rect,line join=bevel,line width=0.800pt]
  \end{scope}
  \begin{scope}[scale=1.006,draw=blue,line cap=rect,line join=bevel,line width=0.800pt]
  \end{scope}
  \begin{scope}[scale=1.006,draw=blue,line cap=rect,line join=bevel,line width=0.800pt]
  \end{scope}
  \begin{scope}[cm={{1.00588,0.0,0.0,1.00588,(229.341,217.271)}},draw=blue,line cap=rect,line join=bevel,line width=0.800pt]
  \end{scope}
  \begin{scope}[cm={{1.00588,0.0,0.0,1.00588,(229.341,217.271)}},draw=blue,line cap=rect,line join=bevel,line width=0.800pt]
  \end{scope}
  \begin{scope}[cm={{1.00588,0.0,0.0,1.00588,(229.341,217.271)}},draw=blue,line cap=rect,line join=bevel,line width=0.800pt]
  \end{scope}
  \begin{scope}[cm={{1.00588,0.0,0.0,1.00588,(229.341,217.271)}},draw=blue,line cap=rect,line join=bevel,line width=0.800pt]
  \end{scope}
  \begin{scope}[cm={{1.00588,0.0,0.0,1.00588,(229.341,217.271)}},draw=blue,line cap=rect,line join=bevel,line width=0.800pt]
  \end{scope}
  \begin{scope}[cm={{1.00588,0.0,0.0,1.00588,(229.341,217.271)}},draw=blue,line cap=rect,line join=bevel,line width=0.800pt]
  \end{scope}
  \begin{scope}[scale=1.006,draw=blue,line cap=rect,line join=bevel,line width=0.800pt]
  \end{scope}
  \begin{scope}[scale=1.006,draw=blue,line cap=rect,line join=bevel,line width=0.800pt]
  \end{scope}
  \begin{scope}[scale=1.006,draw=blue,line cap=rect,line join=bevel,line width=0.800pt]
  \end{scope}
  \begin{scope}[scale=1.006,draw=blue,line cap=rect,line join=bevel,line width=0.800pt]
  \end{scope}
  \begin{scope}[scale=1.006,draw=blue,line cap=rect,line join=bevel,line width=0.800pt]
  \end{scope}
  \begin{scope}[scale=1.006,draw=blue,line cap=rect,line join=bevel,line width=0.800pt]
  \end{scope}
  \begin{scope}[cm={{1.00588,0.0,0.0,1.00588,(236.382,132.776)}},draw=blue,line cap=rect,line join=bevel,line width=0.800pt]
  \end{scope}
  \begin{scope}[cm={{1.00588,0.0,0.0,1.00588,(236.382,132.776)}},draw=blue,line cap=rect,line join=bevel,line width=0.800pt]
  \end{scope}
  \begin{scope}[cm={{1.00588,0.0,0.0,1.00588,(236.382,132.776)}},draw=blue,line cap=rect,line join=bevel,line width=0.800pt]
  \end{scope}
  \begin{scope}[cm={{1.00588,0.0,0.0,1.00588,(236.382,132.776)}},draw=blue,line cap=rect,line join=bevel,line width=0.800pt]
  \end{scope}
  \begin{scope}[cm={{1.00588,0.0,0.0,1.00588,(236.382,132.776)}},draw=blue,line cap=rect,line join=bevel,line width=0.800pt]
  \end{scope}
  \begin{scope}[cm={{1.00588,0.0,0.0,1.00588,(236.382,132.776)}},draw=blue,line cap=rect,line join=bevel,line width=0.800pt]
  \end{scope}
  \begin{scope}[scale=1.006,draw=blue,line cap=rect,line join=bevel,line width=0.800pt]
  \end{scope}
  \begin{scope}[scale=1.006,draw=blue,line cap=rect,line join=bevel,line width=0.800pt]
  \end{scope}
  \begin{scope}[scale=1.006,draw=blue,line cap=rect,line join=bevel,line width=0.800pt]
  \end{scope}
  \begin{scope}[scale=1.006,draw=blue,line cap=rect,line join=bevel,line width=0.800pt]
  \end{scope}
  \begin{scope}[cm={{1.00588,0.0,0.0,1.00588,(197.153,129.759)}},draw=blue,line cap=rect,line join=bevel,line width=0.800pt]
  \end{scope}
  \begin{scope}[cm={{1.00588,0.0,0.0,1.00588,(197.153,129.759)}},draw=blue,line cap=rect,line join=bevel,line width=0.800pt]
  \end{scope}
  \begin{scope}[cm={{1.00588,0.0,0.0,1.00588,(197.153,129.759)}},draw=blue,line cap=rect,line join=bevel,line width=0.800pt]
  \end{scope}
  \begin{scope}[cm={{1.00588,0.0,0.0,1.00588,(197.153,129.759)}},draw=blue,line cap=rect,line join=bevel,line width=0.800pt]
  \end{scope}
  \begin{scope}[cm={{1.00588,0.0,0.0,1.00588,(197.153,129.759)}},draw=blue,line cap=rect,line join=bevel,line width=0.800pt]
  \end{scope}
  \begin{scope}[cm={{1.00588,0.0,0.0,1.00588,(197.153,129.759)}},draw=blue,line cap=rect,line join=bevel,line width=0.800pt]
  \end{scope}
  \begin{scope}[scale=1.006,draw=blue,line cap=rect,line join=bevel,line width=0.800pt]
  \end{scope}
  \begin{scope}[scale=1.006,draw=blue,line cap=rect,line join=bevel,line width=0.800pt]
  \end{scope}
  \begin{scope}[scale=1.006,draw=blue,line cap=rect,line join=bevel,line width=0.800pt]
  \end{scope}
  \begin{scope}[scale=1.006,draw=blue,line cap=rect,line join=bevel,line width=0.800pt]
  \end{scope}
  \begin{scope}[scale=1.006,draw=blue,line cap=rect,line join=bevel,line width=0.800pt]
  \end{scope}
  \begin{scope}[scale=1.006,draw=blue,line cap=rect,line join=bevel,line width=0.800pt]
  \end{scope}
  \begin{scope}[scale=1.006,draw=blue,line cap=rect,line join=bevel,line width=0.800pt]
  \end{scope}
  \begin{scope}[scale=1.006,draw=blue,line cap=rect,line join=bevel,line width=0.800pt]
  \end{scope}
  \begin{scope}[cm={{1.00588,0.0,0.0,1.00588,(39.2294,305.788)}},draw=blue,line cap=rect,line join=bevel,line width=0.800pt]
  \end{scope}
  \begin{scope}[cm={{1.00588,0.0,0.0,1.00588,(39.2294,305.788)}},draw=blue,line cap=rect,line join=bevel,line width=0.800pt]
  \end{scope}
  \begin{scope}[cm={{1.00588,0.0,0.0,1.00588,(39.2294,305.788)}},draw=blue,line cap=rect,line join=bevel,line width=0.800pt]
  \end{scope}
  \begin{scope}[cm={{1.00588,0.0,0.0,1.00588,(39.2294,305.788)}},draw=blue,line cap=rect,line join=bevel,line width=0.800pt]
  \end{scope}
  \begin{scope}[cm={{1.00588,0.0,0.0,1.00588,(39.2294,305.788)}},draw=blue,line cap=rect,line join=bevel,line width=0.800pt]
  \end{scope}
  \begin{scope}[cm={{1.00588,0.0,0.0,1.00588,(39.2294,305.788)}},draw=blue,line cap=rect,line join=bevel,line width=0.800pt]
  \end{scope}
  \begin{scope}[scale=1.006,draw=blue,line cap=rect,line join=bevel,line width=0.800pt]
  \end{scope}
  \begin{scope}[scale=1.006,draw=blue,line cap=rect,line join=bevel,line width=0.800pt]
  \end{scope}
  \begin{scope}[cm={{1.00588,0.0,0.0,1.00588,(40.2353,282.653)}},draw=blue,line cap=rect,line join=bevel,line width=0.800pt]
  \end{scope}
  \begin{scope}[cm={{1.00588,0.0,0.0,1.00588,(40.2353,282.653)}},draw=blue,line cap=rect,line join=bevel,line width=0.800pt]
  \end{scope}
  \begin{scope}[cm={{1.00588,0.0,0.0,1.00588,(40.2353,282.653)}},draw=blue,line cap=rect,line join=bevel,line width=0.800pt]
  \end{scope}
  \begin{scope}[cm={{1.00588,0.0,0.0,1.00588,(40.2353,282.653)}},draw=blue,line cap=rect,line join=bevel,line width=0.800pt]
  \end{scope}
  \begin{scope}[cm={{1.00588,0.0,0.0,1.00588,(40.2353,282.653)}},draw=blue,line cap=rect,line join=bevel,line width=0.800pt]
  \end{scope}
  \begin{scope}[cm={{1.00588,0.0,0.0,1.00588,(40.2353,282.653)}},draw=blue,line cap=rect,line join=bevel,line width=0.800pt]
  \end{scope}
  \begin{scope}[scale=1.006,draw=blue,line cap=rect,line join=bevel,line width=0.800pt]
  \end{scope}
  \begin{scope}[scale=1.006,draw=blue,line cap=rect,line join=bevel,line width=0.800pt]
  \end{scope}
  \begin{scope}[cm={{1.00588,0.0,0.0,1.00588,(40.2353,260.524)}},draw=blue,line cap=rect,line join=bevel,line width=0.800pt]
  \end{scope}
  \begin{scope}[cm={{1.00588,0.0,0.0,1.00588,(40.2353,260.524)}},draw=blue,line cap=rect,line join=bevel,line width=0.800pt]
  \end{scope}
  \begin{scope}[cm={{1.00588,0.0,0.0,1.00588,(40.2353,260.524)}},draw=blue,line cap=rect,line join=bevel,line width=0.800pt]
  \end{scope}
  \begin{scope}[cm={{1.00588,0.0,0.0,1.00588,(40.2353,260.524)}},draw=blue,line cap=rect,line join=bevel,line width=0.800pt]
  \end{scope}
  \begin{scope}[cm={{1.00588,0.0,0.0,1.00588,(40.2353,260.524)}},draw=blue,line cap=rect,line join=bevel,line width=0.800pt]
  \end{scope}
  \begin{scope}[cm={{1.00588,0.0,0.0,1.00588,(40.2353,260.524)}},draw=blue,line cap=rect,line join=bevel,line width=0.800pt]
  \end{scope}
  \begin{scope}[scale=1.006,draw=blue,line cap=rect,line join=bevel,line width=0.800pt]
  \end{scope}
  \begin{scope}[scale=1.006,draw=blue,line cap=rect,line join=bevel,line width=0.800pt]
  \end{scope}
  \begin{scope}[cm={{1.00588,0.0,0.0,1.00588,(39.2294,237.388)}},draw=blue,line cap=rect,line join=bevel,line width=0.800pt]
  \end{scope}
  \begin{scope}[cm={{1.00588,0.0,0.0,1.00588,(39.2294,237.388)}},draw=blue,line cap=rect,line join=bevel,line width=0.800pt]
  \end{scope}
  \begin{scope}[cm={{1.00588,0.0,0.0,1.00588,(39.2294,237.388)}},draw=blue,line cap=rect,line join=bevel,line width=0.800pt]
  \end{scope}
  \begin{scope}[cm={{1.00588,0.0,0.0,1.00588,(39.2294,237.388)}},draw=blue,line cap=rect,line join=bevel,line width=0.800pt]
  \end{scope}
  \begin{scope}[cm={{1.00588,0.0,0.0,1.00588,(39.2294,237.388)}},draw=blue,line cap=rect,line join=bevel,line width=0.800pt]
  \end{scope}
  \begin{scope}[cm={{1.00588,0.0,0.0,1.00588,(39.2294,237.388)}},draw=blue,line cap=rect,line join=bevel,line width=0.800pt]
  \end{scope}
  \begin{scope}[scale=1.006,draw=blue,line cap=rect,line join=bevel,line width=0.800pt]
  \end{scope}
  \begin{scope}[scale=1.006,draw=blue,line cap=rect,line join=bevel,line width=0.800pt]
  \end{scope}
  \begin{scope}[cm={{1.00588,0.0,0.0,1.00588,(53.3118,322.888)}},draw=blue,line cap=rect,line join=bevel,line width=0.800pt]
  \end{scope}
  \begin{scope}[cm={{1.00588,0.0,0.0,1.00588,(53.3118,322.888)}},draw=blue,line cap=rect,line join=bevel,line width=0.800pt]
  \end{scope}
  \begin{scope}[cm={{1.00588,0.0,0.0,1.00588,(53.3118,322.888)}},draw=blue,line cap=rect,line join=bevel,line width=0.800pt]
  \end{scope}
  \begin{scope}[cm={{1.00588,0.0,0.0,1.00588,(53.3118,322.888)}},draw=blue,line cap=rect,line join=bevel,line width=0.800pt]
  \end{scope}
  \begin{scope}[cm={{1.00588,0.0,0.0,1.00588,(53.3118,322.888)}},draw=blue,line cap=rect,line join=bevel,line width=0.800pt]
  \end{scope}
  \begin{scope}[cm={{1.00588,0.0,0.0,1.00588,(53.3118,322.888)}},draw=blue,line cap=rect,line join=bevel,line width=0.800pt]
  \end{scope}
  \begin{scope}[scale=1.006,draw=blue,line cap=rect,line join=bevel,line width=0.800pt]
  \end{scope}
  \begin{scope}[scale=1.006,draw=blue,line cap=rect,line join=bevel,line width=0.800pt]
  \end{scope}
  \begin{scope}[cm={{1.00588,0.0,0.0,1.00588,(89.5235,322.888)}},draw=blue,line cap=rect,line join=bevel,line width=0.800pt]
  \end{scope}
  \begin{scope}[cm={{1.00588,0.0,0.0,1.00588,(89.5235,322.888)}},draw=blue,line cap=rect,line join=bevel,line width=0.800pt]
  \end{scope}
  \begin{scope}[cm={{1.00588,0.0,0.0,1.00588,(89.5235,322.888)}},draw=blue,line cap=rect,line join=bevel,line width=0.800pt]
  \end{scope}
  \begin{scope}[cm={{1.00588,0.0,0.0,1.00588,(89.5235,322.888)}},draw=blue,line cap=rect,line join=bevel,line width=0.800pt]
  \end{scope}
  \begin{scope}[cm={{1.00588,0.0,0.0,1.00588,(89.5235,322.888)}},draw=blue,line cap=rect,line join=bevel,line width=0.800pt]
  \end{scope}
  \begin{scope}[cm={{1.00588,0.0,0.0,1.00588,(89.5235,322.888)}},draw=blue,line cap=rect,line join=bevel,line width=0.800pt]
  \end{scope}
  \begin{scope}[scale=1.006,draw=blue,line cap=rect,line join=bevel,line width=0.800pt]
  \end{scope}
  \begin{scope}[scale=1.006,draw=blue,line cap=rect,line join=bevel,line width=0.800pt]
  \end{scope}
  \begin{scope}[cm={{1.00588,0.0,0.0,1.00588,(125.232,322.888)}},draw=blue,line cap=rect,line join=bevel,line width=0.800pt]
  \end{scope}
  \begin{scope}[cm={{1.00588,0.0,0.0,1.00588,(125.232,322.888)}},draw=blue,line cap=rect,line join=bevel,line width=0.800pt]
  \end{scope}
  \begin{scope}[cm={{1.00588,0.0,0.0,1.00588,(125.232,322.888)}},draw=blue,line cap=rect,line join=bevel,line width=0.800pt]
  \end{scope}
  \begin{scope}[cm={{1.00588,0.0,0.0,1.00588,(125.232,322.888)}},draw=blue,line cap=rect,line join=bevel,line width=0.800pt]
  \end{scope}
  \begin{scope}[cm={{1.00588,0.0,0.0,1.00588,(125.232,322.888)}},draw=blue,line cap=rect,line join=bevel,line width=0.800pt]
  \end{scope}
  \begin{scope}[cm={{1.00588,0.0,0.0,1.00588,(125.232,322.888)}},draw=blue,line cap=rect,line join=bevel,line width=0.800pt]
  \end{scope}
  \begin{scope}[scale=1.006,draw=blue,line cap=rect,line join=bevel,line width=0.800pt]
  \end{scope}
  \begin{scope}[scale=1.006,draw=blue,line cap=rect,line join=bevel,line width=0.800pt]
  \end{scope}
  \begin{scope}[cm={{1.00588,0.0,0.0,1.00588,(160.941,322.888)}},draw=blue,line cap=rect,line join=bevel,line width=0.800pt]
  \end{scope}
  \begin{scope}[cm={{1.00588,0.0,0.0,1.00588,(160.941,322.888)}},draw=blue,line cap=rect,line join=bevel,line width=0.800pt]
  \end{scope}
  \begin{scope}[cm={{1.00588,0.0,0.0,1.00588,(160.941,322.888)}},draw=blue,line cap=rect,line join=bevel,line width=0.800pt]
  \end{scope}
  \begin{scope}[cm={{1.00588,0.0,0.0,1.00588,(160.941,322.888)}},draw=blue,line cap=rect,line join=bevel,line width=0.800pt]
  \end{scope}
  \begin{scope}[cm={{1.00588,0.0,0.0,1.00588,(160.941,322.888)}},draw=blue,line cap=rect,line join=bevel,line width=0.800pt]
  \end{scope}
  \begin{scope}[cm={{1.00588,0.0,0.0,1.00588,(160.941,322.888)}},draw=blue,line cap=rect,line join=bevel,line width=0.800pt]
  \end{scope}
  \begin{scope}[scale=1.006,draw=blue,line cap=rect,line join=bevel,line width=0.800pt]
  \end{scope}
  \begin{scope}[scale=1.006,draw=blue,line cap=rect,line join=bevel,line width=0.800pt]
  \end{scope}
  \begin{scope}[cm={{1.00588,0.0,0.0,1.00588,(196.147,322.888)}},draw=blue,line cap=rect,line join=bevel,line width=0.800pt]
  \end{scope}
  \begin{scope}[cm={{1.00588,0.0,0.0,1.00588,(196.147,322.888)}},draw=blue,line cap=rect,line join=bevel,line width=0.800pt]
  \end{scope}
  \begin{scope}[cm={{1.00588,0.0,0.0,1.00588,(196.147,322.888)}},draw=blue,line cap=rect,line join=bevel,line width=0.800pt]
  \end{scope}
  \begin{scope}[cm={{1.00588,0.0,0.0,1.00588,(196.147,322.888)}},draw=blue,line cap=rect,line join=bevel,line width=0.800pt]
  \end{scope}
  \begin{scope}[cm={{1.00588,0.0,0.0,1.00588,(196.147,322.888)}},draw=blue,line cap=rect,line join=bevel,line width=0.800pt]
  \end{scope}
  \begin{scope}[cm={{1.00588,0.0,0.0,1.00588,(196.147,322.888)}},draw=blue,line cap=rect,line join=bevel,line width=0.800pt]
  \end{scope}
  \begin{scope}[scale=1.006,draw=blue,line cap=rect,line join=bevel,line width=0.800pt]
  \end{scope}
  \begin{scope}[scale=1.006,draw=blue,line cap=rect,line join=bevel,line width=0.800pt]
  \end{scope}
  \begin{scope}[cm={{1.00588,0.0,0.0,1.00588,(228.838,322.888)}},draw=blue,line cap=rect,line join=bevel,line width=0.800pt]
  \end{scope}
  \begin{scope}[cm={{1.00588,0.0,0.0,1.00588,(228.838,322.888)}},draw=blue,line cap=rect,line join=bevel,line width=0.800pt]
  \end{scope}
  \begin{scope}[cm={{1.00588,0.0,0.0,1.00588,(228.838,322.888)}},draw=blue,line cap=rect,line join=bevel,line width=0.800pt]
  \end{scope}
  \begin{scope}[cm={{1.00588,0.0,0.0,1.00588,(228.838,322.888)}},draw=blue,line cap=rect,line join=bevel,line width=0.800pt]
  \end{scope}
  \begin{scope}[cm={{1.00588,0.0,0.0,1.00588,(228.838,322.888)}},draw=blue,line cap=rect,line join=bevel,line width=0.800pt]
  \end{scope}
  \begin{scope}[cm={{1.00588,0.0,0.0,1.00588,(228.838,322.888)}},draw=blue,line cap=rect,line join=bevel,line width=0.800pt]
  \end{scope}
  \begin{scope}[scale=1.006,draw=blue,line cap=rect,line join=bevel,line width=0.800pt]
  \end{scope}
  \begin{scope}[scale=1.006,draw=blue,line cap=rect,line join=bevel,line width=0.800pt]
  \end{scope}
  \begin{scope}[scale=1.006,draw=blue,line cap=rect,line join=bevel,line width=0.800pt]
  \end{scope}
  \begin{scope}[scale=1.006,draw=blue,line cap=rect,line join=bevel,line width=0.800pt]
  \end{scope}
  \begin{scope}[scale=1.006,draw=blue,line cap=rect,line join=bevel,line width=0.800pt]
  \end{scope}
  \begin{scope}[scale=1.006,draw=blue,line cap=rect,line join=bevel,line width=0.800pt]
  \end{scope}
  \begin{scope}[cm={{1.00588,0.0,0.0,1.00588,(232.359,238.394)}},draw=blue,line cap=rect,line join=bevel,line width=0.800pt]
  \end{scope}
  \begin{scope}[cm={{1.00588,0.0,0.0,1.00588,(232.359,238.394)}},draw=blue,line cap=rect,line join=bevel,line width=0.800pt]
  \end{scope}
  \begin{scope}[cm={{1.00588,0.0,0.0,1.00588,(232.359,238.394)}},draw=blue,line cap=rect,line join=bevel,line width=0.800pt]
  \end{scope}
  \begin{scope}[cm={{1.00588,0.0,0.0,1.00588,(232.359,238.394)}},draw=blue,line cap=rect,line join=bevel,line width=0.800pt]
  \end{scope}
  \begin{scope}[cm={{1.00588,0.0,0.0,1.00588,(232.359,238.394)}},draw=blue,line cap=rect,line join=bevel,line width=0.800pt]
  \end{scope}
  \begin{scope}[cm={{1.00588,0.0,0.0,1.00588,(232.359,238.394)}},draw=blue,line cap=rect,line join=bevel,line width=0.800pt]
  \end{scope}
  \begin{scope}[cm={{1.00588,0.0,0.0,1.00588,(130.262,337.976)}},draw=blue,line cap=rect,line join=bevel,line width=0.800pt]
  \end{scope}
  \begin{scope}[cm={{1.00588,0.0,0.0,1.00588,(130.262,337.976)}},draw=blue,line cap=rect,line join=bevel,line width=0.800pt]
  \end{scope}
  \begin{scope}[cm={{1.00588,0.0,0.0,1.00588,(130.262,337.976)}},draw=blue,line cap=rect,line join=bevel,line width=0.800pt]
  \end{scope}
  \begin{scope}[cm={{1.00588,0.0,0.0,1.00588,(130.262,337.976)}},draw=blue,line cap=rect,line join=bevel,line width=0.800pt]
  \end{scope}
  \begin{scope}[cm={{1.00588,0.0,0.0,1.00588,(130.262,337.976)}},draw=blue,line cap=rect,line join=bevel,line width=0.800pt]
  \end{scope}
  \begin{scope}[cm={{1.00588,0.0,0.0,1.00588,(130.262,337.976)}},draw=blue,line cap=rect,line join=bevel,line width=0.800pt]
  \end{scope}
  \begin{scope}[scale=1.006,draw=blue,line cap=rect,line join=bevel,line width=0.800pt]
  \end{scope}
  \begin{scope}[scale=1.006,draw=blue,line cap=rect,line join=bevel,line width=0.800pt]
  \end{scope}
  \begin{scope}[scale=1.006,draw=blue,line cap=rect,line join=bevel,line width=0.800pt]
  \end{scope}
  \begin{scope}[scale=1.006,draw=blue,line cap=rect,line join=bevel,line width=0.800pt]
  \end{scope}
  \begin{scope}[scale=1.006,draw=blue,line cap=rect,line join=bevel,line width=0.800pt]
  \end{scope}
  \begin{scope}[scale=1.006,draw=blue,line cap=rect,line join=bevel,line width=0.800pt]
  \end{scope}
  \begin{scope}[scale=1.006,draw=blue,line cap=rect,line join=bevel,line width=0.800pt]
  \end{scope}
  \begin{scope}[scale=1.006,draw=blue,line cap=rect,line join=bevel,line width=0.800pt]
  \end{scope}
  \begin{scope}[draw=blue,line cap=rect,line join=bevel,line width=0.800pt]
  \end{scope}
  \begin{scope}[cm={{1.26012,0.0,0.0,1.26012,(-482.16108,-168.79053)}},draw=ca0a0a4,dash pattern=on 1.03pt off 1.03pt,line cap=round,line join=round,line width=0.257pt,miter limit=4.00]
    \path[draw,dash pattern=on 1.03pt off 1.03pt,line width=0.257pt,miter limit=4.00] (56.5000,88.5000) -- (142.5000,88.5000);



  \end{scope}
  \begin{scope}[cm={{1.26012,0.0,0.0,1.26012,(-482.16108,-168.79053)}},draw=blue,line cap=round,line join=round,line width=0.480pt]
    \path[cm={{0.9189,0.0,0.0,1.0,(4.55807,0.0)}},draw] (56.5000,88.5000) -- (59.5000,88.5000);



    \path[cm={{0.9189,0.0,0.0,1.0,(11.5047,0.0)}},draw] (142.5000,88.5000) -- (139.5000,88.5000);



  \end{scope}
  \begin{scope}[cm={{1.26012,0.0,0.0,1.26012,(-482.16108,-168.79053)}},draw=ca0a0a4,dash pattern=on 1.03pt off 1.03pt,line cap=round,line join=round,line width=0.257pt,miter limit=4.00]
    \path[draw,dash pattern=on 1.03pt off 1.03pt,line width=0.257pt,miter limit=4.00] (56.5000,63.5000) -- (142.5000,63.5000);



  \end{scope}
  \begin{scope}[cm={{1.26012,0.0,0.0,1.26012,(-482.16108,-168.79053)}},draw=blue,line cap=round,line join=round,line width=0.480pt]
    \path[cm={{0.9189,0.0,0.0,1.0,(4.55807,-3e-05)}},draw] (56.5000,63.5000) -- (59.5000,63.5000);



    \path[cm={{0.9189,0.0,0.0,1.0,(11.5047,0.0)}},draw] (142.5000,63.5000) -- (139.5000,63.5000);



  \end{scope}
  \begin{scope}[cm={{1.26012,0.0,0.0,1.26012,(-482.16108,-168.79053)}},draw=ca0a0a4,dash pattern=on 1.03pt off 1.03pt,line cap=round,line join=round,line width=0.257pt,miter limit=4.00]
    \path[draw,dash pattern=on 1.03pt off 1.03pt,line width=0.257pt,miter limit=4.00] (56.5000,38.5000) -- (142.5000,38.5000);



  \end{scope}
  \begin{scope}[cm={{1.26012,0.0,0.0,1.26012,(-482.16108,-168.79053)}},draw=blue,line cap=round,line join=round,line width=0.480pt]
    \path[cm={{0.9189,0.0,0.0,1.0,(4.55807,-3e-05)}},draw] (56.5000,38.5000) -- (59.5000,38.5000);



    \path[cm={{0.9189,0.0,0.0,1.0,(11.5047,0.0)}},draw] (142.5000,38.5000) -- (139.5000,38.5000);



  \end{scope}
  \begin{scope}[cm={{1.26012,0.0,0.0,1.26012,(-482.16108,-168.79053)}},draw=ca0a0a4,dash pattern=on 0.40pt off 0.80pt,line cap=round,line join=round,line width=0.400pt]
    \path[draw] (56.5000,95.5000) -- (56.5000,13.5000);



  \end{scope}
  \begin{scope}[cm={{1.26012,0.0,0.0,1.26012,(-482.16108,-168.79053)}},draw=blue,line cap=round,line join=round,line width=0.480pt]
    \path[draw] (56.5000,95.5000) -- (56.5000,92.5000);



    \path[draw] (56.5000,13.5000) -- (56.5000,16.5000);



  \end{scope}
  \begin{scope}[cm={{1.26012,0.0,0.0,1.26012,(-482.16108,-168.79053)}},draw=blue,line cap=round,line join=round,line width=0.480pt]
    \path[cm={{1.0,0.0,0.0,0.9189,(0.0,7.47783)}},draw] (82.5000,95.5000) -- (82.5000,92.5000);



    \path[cm={{1.0,0.0,0.0,0.9189,(0.0,1.07058)}},draw] (82.5000,13.5000) -- (82.5000,16.5000);



  \end{scope}
  \begin{scope}[cm={{1.26012,0.0,0.0,1.26012,(-482.16108,-168.79053)}},draw=blue,line cap=round,line join=round,line width=0.480pt]
    \path[cm={{1.0,0.0,0.0,0.9189,(0.0,7.47783)}},draw] (108.5000,95.5000) -- (108.5000,92.5000);



    \path[cm={{1.0,0.0,0.0,0.9189,(0.0,1.07058)}},draw] (108.5000,13.5000) -- (108.5000,16.5000);



  \end{scope}
  \begin{scope}[cm={{1.26012,0.0,0.0,1.26012,(-482.16108,-168.79053)}},draw=ca0a0a4,dash pattern=on 1.03pt off 1.03pt,line cap=round,line join=round,line width=0.257pt,miter limit=4.00]
    \path[draw,dash pattern=on 1.03pt off 1.03pt,line width=0.257pt,miter limit=4.00] (134.5000,95.5000) -- (134.5000,13.5000);



  \end{scope}
  \begin{scope}[cm={{1.26012,0.0,0.0,1.26012,(-482.16108,-168.79053)}},draw=blue,line cap=round,line join=round,line width=0.480pt]
    \path[cm={{1.0,0.0,0.0,0.9189,(0.0,7.47783)}},draw] (134.5000,95.5000) -- (134.5000,92.5000);



    \path[cm={{1.0,0.0,0.0,0.9189,(0.0,1.07058)}},draw] (134.5000,13.5000) -- (134.5000,16.5000);



  \end{scope}
  \begin{scope}[cm={{1.00588,0.0,0.0,1.00588,(-400.2762,-132.7966)}},draw=blue,line cap=rect,line join=bevel,line width=0.800pt]
    \path[fill=blue] (0.0000,-1.4912) node[above right] (text290) {\scriptsize $\Upsilon(t)$};



  \end{scope}
  \begin{scope}[cm={{1.26012,0.0,0.0,1.26012,(-487.73989,-169.45128)}},draw=blue,line cap=round,line join=round,line width=0.480pt]
    \path[draw,even odd rule] (86.5000,24.5000) -- (112.5000,24.5000);



  \end{scope}
  \begin{scope}[cm={{1.26012,0.0,0.0,1.26012,(-482.16108,-168.79053)}},draw=blue,line cap=round,line join=round,line width=0.480pt]
    \path[draw] (56.1000,52.3000) -- (56.1000,52.3000) -- (56.5000,69.8000) -- (56.8000,61.5000) -- (57.1000,50.5000) -- (57.4000,49.2000) -- (57.7000,52.5000) -- (58.0000,57.0000) -- (58.4000,59.6000) -- (58.7000,60.0000) -- (59.0000,59.6000) -- (59.3000,59.2000) -- (59.6000,59.1000) -- (59.9000,59.0000) -- (60.2000,59.0000) -- (60.6000,59.1000) -- (60.9000,59.1000) -- (61.2000,59.1000) -- (61.5000,59.1000) -- (61.8000,59.0000) -- (62.1000,59.5000) -- (62.4000,62.4000) -- (62.8000,66.1000) -- (63.1000,69.9000) -- (63.4000,73.4000) -- (63.7000,76.7000) -- (64.0000,79.8000) -- (64.3000,82.6000) -- (64.6000,84.9000) -- (65.0000,86.4000) -- (65.3000,87.1000) -- (65.6000,86.7000) -- (65.9000,85.4000) -- (66.2000,83.2000) -- (66.5000,80.3000) -- (66.8000,77.1000) -- (67.2000,73.7000) -- (67.5000,70.1000) -- (67.8000,65.0000) -- (68.1000,60.3000) -- (68.4000,58.1000) -- (68.7000,58.0000) -- (69.0000,58.5000) -- (69.4000,58.9000) -- (69.7000,59.1000) -- (70.0000,59.1000) -- (70.3000,59.1000) -- (70.6000,59.1000) -- (70.9000,59.1000) -- (71.2000,59.1000) -- (71.5000,59.0000) -- (71.9000,61.2000) -- (72.2000,62.2000) -- (72.5000,59.0000) -- (72.8000,55.3000) -- (73.1000,52.3000) -- (73.4000,50.0000) -- (73.7000,48.2000) -- (74.1000,46.8000) -- (74.4000,45.7000) -- (74.7000,44.9000) -- (75.0000,44.4000) -- (75.3000,44.2000) -- (75.6000,44.3000) -- (75.9000,44.7000) -- (76.3000,45.4000) -- (76.6000,46.4000) -- (76.9000,47.7000) -- (77.2000,49.3000) -- (77.5000,51.3000) -- (77.8000,53.6000) -- (78.1000,56.7000) -- (78.5000,59.2000) -- (78.8000,59.9000) -- (79.1000,59.6000) -- (79.4000,59.3000) -- (79.7000,59.1000) -- (80.0000,59.0000) -- (80.3000,59.0000) -- (80.7000,59.1000) -- (81.0000,59.1000) -- (81.3000,59.1000) -- (81.6000,59.1000) -- (81.9000,59.0000) -- (82.2000,59.1000) -- (82.5000,61.3000) -- (82.9000,65.2000) -- (83.2000,69.1000) -- (83.5000,72.7000) -- (83.8000,76.1000) -- (84.1000,79.3000) -- (84.4000,82.1000) -- (84.7000,84.6000) -- (85.1000,86.3000) -- (85.4000,87.1000) -- (85.7000,86.8000) -- (86.0000,85.5000) -- (86.3000,83.3000) -- (86.6000,80.4000) -- (86.9000,77.1000) -- (87.3000,73.6000) -- (87.6000,69.9000) -- (87.9000,65.0000) -- (88.2000,60.5000) -- (88.5000,58.3000) -- (88.8000,58.1000) -- (89.1000,58.5000) -- (89.5000,58.9000) -- (89.8000,59.1000) -- (90.1000,59.1000) -- (90.4000,59.1000) -- (90.7000,59.1000) -- (91.0000,59.1000) -- (91.3000,59.1000) -- (91.7000,59.0000) -- (92.0000,61.4000) -- (92.3000,62.0000) -- (92.6000,58.6000) -- (92.9000,55.0000) -- (93.2000,52.0000) -- (93.5000,49.7000) -- (93.9000,47.9000) -- (94.2000,46.6000) -- (94.5000,45.5000) -- (94.8000,44.8000) -- (95.1000,44.3000) -- (95.4000,44.2000) -- (95.7000,44.4000) -- (96.1000,44.8000) -- (96.4000,45.6000) -- (96.7000,46.7000) -- (97.0000,48.2000) -- (97.3000,50.0000) -- (97.6000,52.1000) -- (97.9000,54.7000) -- (98.3000,57.7000) -- (98.6000,59.4000) -- (98.9000,59.7000) -- (99.2000,59.5000) -- (99.5000,59.2000) -- (99.8000,59.1000) -- (100.1000,59.0000) -- (100.5000,59.1000) -- (100.8000,59.1000) -- (101.1000,59.1000) -- (101.4000,59.1000) -- (101.7000,59.1000) -- (102.0000,59.0000) -- (102.3000,59.5000) -- (102.7000,62.6000) -- (103.0000,66.8000) -- (103.3000,70.7000) -- (103.6000,74.2000) -- (103.9000,77.5000) -- (104.2000,80.5000) -- (104.5000,83.2000) -- (104.9000,85.3000) -- (105.2000,86.7000) -- (105.5000,87.1000) -- (105.8000,86.3000) -- (106.1000,84.5000) -- (106.4000,81.8000) -- (106.7000,78.6000) -- (107.0000,75.1000) -- (107.4000,71.6000) -- (107.7000,67.2000) -- (108.0000,62.3000) -- (108.3000,59.0000) -- (108.6000,58.0000) -- (108.9000,58.3000) -- (109.3000,58.7000) -- (109.6000,59.0000) -- (109.9000,59.1000) -- (110.2000,59.1000) -- (110.5000,59.1000) -- (110.8000,59.1000) -- (111.1000,59.1000) -- (111.4000,59.0000) -- (111.8000,60.0000) -- (112.1000,62.4000) -- (112.4000,60.2000) -- (112.7000,56.4000) -- (113.0000,53.1000) -- (113.3000,50.5000) -- (113.6000,48.6000) -- (114.0000,47.1000) -- (114.3000,45.9000) -- (114.6000,45.0000) -- (114.9000,44.5000) -- (115.2000,44.2000) -- (115.5000,44.3000) -- (115.8000,44.6000) -- (116.2000,45.3000) -- (116.5000,46.3000) -- (116.8000,47.6000) -- (117.1000,49.4000) -- (117.4000,51.4000) -- (117.7000,53.9000) -- (118.0000,56.8000) -- (118.4000,59.0000) -- (118.7000,59.7000) -- (119.0000,59.6000) -- (119.3000,59.3000) -- (119.6000,59.1000) -- (119.9000,59.1000) -- (120.2000,59.1000) -- (120.6000,59.1000) -- (120.9000,59.1000) -- (121.2000,59.1000) -- (121.5000,59.1000) -- (121.8000,59.0000) -- (122.1000,59.2000) -- (122.4000,61.3000) -- (122.8000,65.4000) -- (123.1000,69.6000) -- (123.4000,73.3000) -- (123.7000,76.6000) -- (124.0000,79.6000) -- (124.3000,82.4000) -- (124.6000,84.7000) -- (125.0000,86.4000) -- (125.3000,87.1000) -- (125.6000,86.7000) -- (125.9000,85.1000) -- (126.2000,82.5000) -- (126.5000,79.4000) -- (126.8000,76.0000) -- (127.2000,72.5000) -- (127.5000,68.4000) -- (127.8000,63.4000) -- (128.1000,59.6000) -- (128.4000,58.1000) -- (128.7000,58.2000) -- (129.0000,58.6000) -- (129.4000,59.0000) -- (129.7000,59.1000) -- (130.0000,59.1000) -- (130.3000,59.1000) -- (130.6000,59.1000) -- (130.9000,59.1000) -- (131.2000,59.0000) -- (131.6000,59.4000) -- (131.9000,62.2000) -- (132.2000,61.0000) -- (132.5000,57.3000) -- (132.8000,53.8000) -- (133.1000,51.1000) -- (133.4000,49.0000) -- (133.8000,47.4000) -- (134.1000,46.1000) -- (134.4000,45.2000) -- (134.7000,44.6000) -- (135.0000,44.2000) -- (135.3000,44.2000) -- (135.6000,44.5000) -- (136.0000,45.1000) -- (136.3000,46.0000) -- (136.6000,47.3000) -- (136.9000,49.0000) -- (137.2000,51.0000) -- (137.5000,53.3000) -- (137.8000,56.2000) -- (138.2000,58.6000) -- (138.5000,59.6000) -- (138.8000,59.6000) -- (139.1000,59.3000) -- (139.4000,59.1000) -- (139.7000,59.1000) -- (140.0000,59.1000) -- (140.4000,59.1000) -- (140.7000,59.1000) -- (141.0000,59.1000) -- (141.3000,59.1000) -- (141.6000,59.1000) -- (141.9000,59.0000) -- (142.2000,60.6000) -- (142.6000,64.5000) -- (142.7000,66.5000);



  \end{scope}
  \begin{scope}[cm={{1.00588,0.0,0.0,1.00588,(-399.05137,-123.74949)}},draw=blue,line cap=rect,line join=bevel,line width=0.800pt]
    \path[fill=blue] (0.0000,0.0000) node[above right] (text326) {\scriptsize $y(t)$};



  \end{scope}
  \begin{scope}[cm={{1.26012,0.0,0.0,1.26012,(-205.87067,-45.42881)}},draw=cff0000,line cap=round,line join=round,line width=0.480pt]
    \path[draw,even odd rule] (-137.1848,-65.9213) -- (-111.1848,-65.9213);



  \end{scope}
  \begin{scope}[cm={{1.26012,0.0,0.0,1.26012,(-482.16108,-168.79053)}},draw=cff0000,line cap=round,line join=round,line width=0.480pt]
    \path[draw] (56.0000,44.3000) -- (56.0000,44.3000) -- (56.1000,44.4000) -- (56.2000,44.6000) -- (56.3000,44.8000) -- (56.3000,45.1000) -- (56.4000,45.4000) -- (56.5000,45.7000) -- (56.6000,46.1000) -- (56.7000,46.5000) -- (56.8000,47.0000) -- (56.9000,47.4000) -- (57.0000,47.9000) -- (57.0000,48.4000) -- (57.1000,48.9000) -- (57.2000,49.5000) -- (57.3000,50.0000) -- (57.4000,50.6000) -- (57.5000,51.2000) -- (57.6000,51.8000) -- (57.6000,52.3000) -- (57.7000,52.9000) -- (57.8000,53.5000) -- (57.9000,54.0000) -- (58.0000,54.6000) -- (58.1000,55.1000) -- (58.2000,55.6000) -- (58.3000,56.1000) -- (58.3000,56.6000) -- (58.4000,57.1000) -- (58.5000,57.5000) -- (58.6000,57.9000) -- (58.7000,58.2000) -- (58.8000,58.6000) -- (58.9000,58.9000) -- (59.0000,59.1000) -- (59.0000,59.4000) -- (59.1000,59.6000) -- (59.2000,59.8000) -- (59.3000,59.9000) -- (59.4000,60.0000) -- (59.5000,60.1000) -- (59.6000,60.1000) -- (59.6000,60.1000) -- (59.7000,60.1000) -- (59.8000,60.1000) -- (59.9000,60.0000) -- (60.0000,60.0000) -- (60.1000,59.9000) -- (60.2000,59.7000) -- (60.3000,59.6000) -- (60.3000,59.5000) -- (60.4000,59.3000) -- (60.5000,59.2000) -- (60.6000,59.0000) -- (60.7000,58.9000) -- (60.8000,58.8000) -- (60.9000,58.6000) -- (60.9000,58.5000) -- (61.0000,58.4000) -- (61.1000,58.3000) -- (61.2000,58.2000) -- (61.3000,58.2000) -- (61.4000,58.2000) -- (61.5000,58.2000) -- (61.6000,58.3000) -- (61.6000,58.3000) -- (61.7000,58.5000) -- (61.8000,58.6000) -- (61.9000,58.8000) -- (62.0000,59.1000) -- (62.1000,59.3000) -- (62.2000,59.7000) -- (62.2000,60.0000) -- (62.3000,60.4000) -- (62.4000,60.9000) -- (62.5000,61.4000) -- (62.6000,61.9000) -- (62.7000,62.5000) -- (62.8000,63.1000) -- (62.9000,63.8000) -- (62.9000,64.4000) -- (63.0000,65.2000) -- (63.1000,65.9000) -- (63.2000,66.7000) -- (63.3000,67.5000) -- (63.4000,68.3000) -- (63.5000,69.2000) -- (63.6000,70.0000) -- (63.6000,70.9000) -- (63.7000,71.8000) -- (63.8000,72.7000) -- (63.9000,73.6000) -- (64.0000,74.4000) -- (64.1000,75.3000) -- (64.2000,76.2000) -- (64.2000,77.0000) -- (64.3000,77.8000) -- (64.4000,78.6000) -- (64.5000,79.4000) -- (64.6000,80.1000) -- (64.7000,80.8000) -- (64.8000,81.5000) -- (64.9000,82.1000) -- (64.9000,82.7000) -- (65.0000,83.2000) -- (65.1000,83.7000) -- (65.2000,84.1000) -- (65.3000,84.4000) -- (65.4000,84.7000) -- (65.5000,85.0000) -- (65.5000,85.2000) -- (65.6000,85.3000) -- (65.7000,85.3000) -- (65.8000,85.3000) -- (65.9000,85.3000) -- (66.0000,85.1000) -- (66.1000,84.9000) -- (66.2000,84.7000) -- (66.2000,84.3000) -- (66.3000,84.0000) -- (66.4000,83.5000) -- (66.5000,83.1000) -- (66.6000,82.5000) -- (66.7000,81.9000) -- (66.8000,81.3000) -- (66.8000,80.6000) -- (66.9000,79.9000) -- (67.0000,79.2000) -- (67.1000,78.4000) -- (67.2000,77.6000) -- (67.3000,76.7000) -- (67.4000,75.9000) -- (67.5000,75.0000) -- (67.5000,74.1000) -- (67.6000,73.2000) -- (67.7000,72.3000) -- (67.8000,71.4000) -- (67.9000,70.5000) -- (68.0000,69.6000) -- (68.1000,68.7000) -- (68.2000,67.9000) -- (68.2000,67.0000) -- (68.3000,66.2000) -- (68.4000,65.4000) -- (68.5000,64.7000) -- (68.6000,63.9000) -- (68.7000,63.2000) -- (68.8000,62.5000) -- (68.8000,61.9000) -- (68.9000,61.3000) -- (69.0000,60.8000) -- (69.1000,60.2000) -- (69.2000,59.8000) -- (69.3000,59.4000) -- (69.4000,59.0000) -- (69.5000,58.6000) -- (69.5000,58.3000) -- (69.6000,58.1000) -- (69.7000,57.8000) -- (69.8000,57.7000) -- (69.9000,57.5000) -- (70.0000,57.4000) -- (70.1000,57.3000) -- (70.1000,57.3000) -- (70.2000,57.3000) -- (70.3000,57.3000) -- (70.4000,57.3000) -- (70.5000,57.4000) -- (70.6000,57.5000) -- (70.7000,57.6000) -- (70.8000,57.7000) -- (70.8000,57.8000) -- (70.9000,57.9000) -- (71.0000,58.0000) -- (71.1000,58.2000) -- (71.2000,58.3000) -- (71.3000,58.4000) -- (71.4000,58.5000) -- (71.4000,58.6000) -- (71.5000,58.7000) -- (71.6000,58.7000) -- (71.7000,58.8000) -- (71.8000,58.8000) -- (71.9000,58.8000) -- (72.0000,58.8000) -- (72.1000,58.7000) -- (72.1000,58.6000) -- (72.2000,58.5000) -- (72.3000,58.3000) -- (72.4000,58.2000) -- (72.5000,57.9000) -- (72.6000,57.7000) -- (72.7000,57.4000) -- (72.8000,57.1000) -- (72.8000,56.8000) -- (72.9000,56.4000) -- (73.0000,56.0000) -- (73.1000,55.6000) -- (73.2000,55.2000) -- (73.3000,54.7000) -- (73.4000,54.3000) -- (73.4000,53.8000) -- (73.5000,53.3000) -- (73.6000,52.7000) -- (73.7000,52.2000) -- (73.8000,51.7000) -- (73.9000,51.1000) -- (74.0000,50.6000) -- (74.1000,50.0000) -- (74.1000,49.5000) -- (74.2000,49.0000) -- (74.3000,48.5000) -- (74.4000,48.0000) -- (74.5000,47.5000) -- (74.6000,47.1000) -- (74.7000,46.6000) -- (74.7000,46.2000) -- (74.8000,45.8000) -- (74.9000,45.5000) -- (75.0000,45.2000) -- (75.1000,44.9000) -- (75.2000,44.7000) -- (75.3000,44.5000) -- (75.4000,44.3000) -- (75.4000,44.2000) -- (75.5000,44.1000) -- (75.6000,44.1000) -- (75.7000,44.1000) -- (75.8000,44.2000) -- (75.9000,44.3000) -- (76.0000,44.4000) -- (76.0000,44.6000) -- (76.1000,44.8000) -- (76.2000,45.1000) -- (76.3000,45.4000) -- (76.4000,45.8000) -- (76.5000,46.1000) -- (76.6000,46.5000) -- (76.7000,47.0000) -- (76.7000,47.4000) -- (76.8000,47.9000) -- (76.9000,48.5000) -- (77.0000,49.0000) -- (77.1000,49.5000) -- (77.2000,50.1000) -- (77.3000,50.7000) -- (77.3000,51.2000) -- (77.4000,51.8000) -- (77.5000,52.4000) -- (77.6000,53.0000) -- (77.7000,53.5000) -- (77.8000,54.1000) -- (77.9000,54.6000) -- (78.0000,55.2000) -- (78.0000,55.7000) -- (78.1000,56.2000) -- (78.2000,56.7000) -- (78.3000,57.1000) -- (78.4000,57.5000) -- (78.5000,57.9000) -- (78.6000,58.3000) -- (78.7000,58.6000) -- (78.7000,58.9000) -- (78.8000,59.2000) -- (78.9000,59.4000) -- (79.0000,59.6000) -- (79.1000,59.8000) -- (79.2000,59.9000) -- (79.3000,60.0000) -- (79.3000,60.1000) -- (79.4000,60.1000) -- (79.5000,60.2000) -- (79.6000,60.1000) -- (79.7000,60.1000) -- (79.8000,60.0000) -- (79.9000,60.0000) -- (80.0000,59.9000) -- (80.0000,59.7000) -- (80.1000,59.6000) -- (80.2000,59.5000) -- (80.3000,59.3000) -- (80.4000,59.2000) -- (80.5000,59.0000) -- (80.6000,58.9000) -- (80.6000,58.7000) -- (80.7000,58.6000) -- (80.8000,58.5000) -- (80.9000,58.4000) -- (81.0000,58.3000) -- (81.1000,58.2000) -- (81.2000,58.2000) -- (81.3000,58.2000) -- (81.3000,58.2000) -- (81.4000,58.2000) -- (81.5000,58.3000) -- (81.6000,58.5000) -- (81.7000,58.6000) -- (81.8000,58.8000) -- (81.9000,59.1000) -- (81.9000,59.3000) -- (82.0000,59.7000) -- (82.1000,60.0000) -- (82.2000,60.4000) -- (82.3000,60.9000) -- (82.4000,61.4000) -- (82.5000,61.9000) -- (82.6000,62.5000) -- (82.6000,63.1000) -- (82.7000,63.8000) -- (82.8000,64.5000) -- (82.9000,65.2000) -- (83.0000,66.0000) -- (83.1000,66.8000) -- (83.2000,67.6000) -- (83.3000,68.4000) -- (83.3000,69.3000) -- (83.4000,70.1000) -- (83.5000,71.0000) -- (83.6000,71.9000) -- (83.7000,72.8000) -- (83.8000,73.7000) -- (83.9000,74.5000) -- (83.9000,75.4000) -- (84.0000,76.3000) -- (84.1000,77.1000) -- (84.2000,77.9000) -- (84.3000,78.7000) -- (84.4000,79.5000) -- (84.5000,80.2000) -- (84.6000,80.9000) -- (84.6000,81.6000) -- (84.7000,82.2000) -- (84.8000,82.8000) -- (84.9000,83.3000) -- (85.0000,83.7000) -- (85.1000,84.2000) -- (85.2000,84.5000) -- (85.2000,84.8000) -- (85.3000,85.0000) -- (85.4000,85.2000) -- (85.5000,85.3000) -- (85.6000,85.4000) -- (85.7000,85.4000) -- (85.8000,85.3000) -- (85.9000,85.1000) -- (85.9000,84.9000) -- (86.0000,84.7000) -- (86.1000,84.4000) -- (86.2000,84.0000) -- (86.3000,83.5000) -- (86.4000,83.0000) -- (86.5000,82.5000) -- (86.5000,81.9000) -- (86.6000,81.3000) -- (86.7000,80.6000) -- (86.8000,79.9000) -- (86.9000,79.1000) -- (87.0000,78.3000) -- (87.1000,77.5000) -- (87.2000,76.7000) -- (87.2000,75.8000) -- (87.3000,74.9000) -- (87.4000,74.0000) -- (87.5000,73.1000) -- (87.6000,72.2000) -- (87.7000,71.3000) -- (87.8000,70.4000) -- (87.9000,69.5000) -- (87.9000,68.7000) -- (88.0000,67.8000) -- (88.1000,67.0000) -- (88.2000,66.1000) -- (88.3000,65.3000) -- (88.4000,64.6000) -- (88.5000,63.8000) -- (88.5000,63.1000) -- (88.6000,62.5000) -- (88.7000,61.8000) -- (88.8000,61.2000) -- (88.9000,60.7000) -- (89.0000,60.2000) -- (89.1000,59.7000) -- (89.2000,59.3000) -- (89.2000,58.9000) -- (89.3000,58.6000) -- (89.4000,58.3000) -- (89.5000,58.0000) -- (89.6000,57.8000) -- (89.7000,57.6000) -- (89.8000,57.5000) -- (89.8000,57.4000) -- (89.9000,57.3000) -- (90.0000,57.3000) -- (90.1000,57.3000) -- (90.2000,57.3000) -- (90.3000,57.3000) -- (90.4000,57.4000) -- (90.5000,57.5000) -- (90.5000,57.6000) -- (90.6000,57.7000) -- (90.7000,57.8000) -- (90.8000,57.9000) -- (90.9000,58.1000) -- (91.0000,58.2000) -- (91.1000,58.3000) -- (91.1000,58.4000) -- (91.2000,58.5000) -- (91.3000,58.6000) -- (91.4000,58.7000) -- (91.5000,58.8000) -- (91.6000,58.8000) -- (91.7000,58.8000) -- (91.8000,58.8000) -- (91.8000,58.8000) -- (91.9000,58.7000) -- (92.0000,58.6000) -- (92.1000,58.5000) -- (92.2000,58.3000) -- (92.3000,58.2000) -- (92.4000,57.9000) -- (92.5000,57.7000) -- (92.5000,57.4000) -- (92.6000,57.1000) -- (92.7000,56.8000) -- (92.8000,56.4000) -- (92.9000,56.0000) -- (93.0000,55.6000) -- (93.1000,55.2000) -- (93.1000,54.7000) -- (93.2000,54.2000) -- (93.3000,53.7000) -- (93.4000,53.2000) -- (93.5000,52.7000) -- (93.6000,52.2000) -- (93.7000,51.6000) -- (93.8000,51.1000) -- (93.8000,50.5000) -- (93.9000,50.0000) -- (94.0000,49.5000) -- (94.1000,48.9000) -- (94.2000,48.4000) -- (94.3000,47.9000) -- (94.4000,47.5000) -- (94.4000,47.0000) -- (94.5000,46.6000) -- (94.6000,46.2000) -- (94.7000,45.8000) -- (94.8000,45.4000) -- (94.9000,45.1000) -- (95.0000,44.9000) -- (95.1000,44.6000) -- (95.1000,44.4000) -- (95.2000,44.3000) -- (95.3000,44.2000) -- (95.4000,44.1000) -- (95.5000,44.1000) -- (95.6000,44.1000) -- (95.7000,44.2000) -- (95.7000,44.3000) -- (95.8000,44.4000) -- (95.9000,44.6000) -- (96.0000,44.8000) -- (96.1000,45.1000) -- (96.2000,45.4000) -- (96.3000,45.8000) -- (96.4000,46.2000) -- (96.4000,46.6000) -- (96.5000,47.0000) -- (96.6000,47.5000) -- (96.7000,48.0000) -- (96.8000,48.5000) -- (96.9000,49.0000) -- (97.0000,49.6000) -- (97.1000,50.1000) -- (97.1000,50.7000) -- (97.2000,51.3000) -- (97.3000,51.9000) -- (97.4000,52.4000) -- (97.5000,53.0000) -- (97.6000,53.6000) -- (97.7000,54.2000) -- (97.7000,54.7000) -- (97.8000,55.2000) -- (97.9000,55.8000) -- (98.0000,56.2000) -- (98.1000,56.7000) -- (98.2000,57.2000) -- (98.3000,57.6000) -- (98.4000,58.0000) -- (98.4000,58.3000) -- (98.5000,58.7000) -- (98.6000,59.0000) -- (98.7000,59.2000) -- (98.8000,59.5000) -- (98.9000,59.7000) -- (99.0000,59.8000) -- (99.0000,60.0000) -- (99.1000,60.1000) -- (99.2000,60.1000) -- (99.3000,60.2000) -- (99.4000,60.2000) -- (99.5000,60.2000) -- (99.6000,60.1000) -- (99.7000,60.0000) -- (99.7000,60.0000) -- (99.8000,59.9000) -- (99.9000,59.7000) -- (100.0000,59.6000) -- (100.1000,59.5000) -- (100.2000,59.3000) -- (100.3000,59.2000) -- (100.3000,59.0000) -- (100.4000,58.9000) -- (100.5000,58.7000) -- (100.6000,58.6000) -- (100.7000,58.5000) -- (100.8000,58.4000) -- (100.9000,58.3000) -- (101.0000,58.2000) -- (101.0000,58.2000) -- (101.1000,58.2000) -- (101.2000,58.2000) -- (101.3000,58.2000) -- (101.4000,58.3000) -- (101.5000,58.4000) -- (101.6000,58.6000) -- (101.7000,58.8000) -- (101.7000,59.1000) -- (101.8000,59.3000) -- (101.9000,59.7000) -- (102.0000,60.0000) -- (102.1000,60.5000) -- (102.2000,60.9000) -- (102.3000,61.4000) -- (102.3000,62.0000) -- (102.4000,62.6000) -- (102.5000,63.2000) -- (102.6000,63.8000) -- (102.7000,64.5000) -- (102.8000,65.3000) -- (102.9000,66.0000) -- (103.0000,66.8000) -- (103.0000,67.6000) -- (103.1000,68.5000) -- (103.2000,69.3000) -- (103.3000,70.2000) -- (103.4000,71.1000) -- (103.5000,72.0000) -- (103.6000,72.8000) -- (103.6000,73.7000) -- (103.7000,74.6000) -- (103.8000,75.5000) -- (103.9000,76.4000) -- (104.0000,77.2000) -- (104.1000,78.0000) -- (104.2000,78.8000) -- (104.3000,79.6000) -- (104.3000,80.3000) -- (104.4000,81.0000) -- (104.5000,81.7000) -- (104.6000,82.3000) -- (104.7000,82.8000) -- (104.8000,83.4000) -- (104.9000,83.8000) -- (104.9000,84.2000) -- (105.0000,84.6000) -- (105.1000,84.9000) -- (105.2000,85.1000) -- (105.3000,85.3000) -- (105.4000,85.4000) -- (105.5000,85.4000) -- (105.6000,85.4000) -- (105.6000,85.3000) -- (105.7000,85.2000) -- (105.8000,85.0000) -- (105.9000,84.7000) -- (106.0000,84.4000) -- (106.1000,84.0000) -- (106.2000,83.5000) -- (106.3000,83.0000) -- (106.3000,82.5000) -- (106.4000,81.9000) -- (106.5000,81.2000) -- (106.6000,80.5000) -- (106.7000,79.8000) -- (106.8000,79.1000) -- (106.9000,78.3000) -- (106.9000,77.4000) -- (107.0000,76.6000) -- (107.1000,75.7000) -- (107.2000,74.8000) -- (107.3000,73.9000) -- (107.4000,73.0000) -- (107.5000,72.1000) -- (107.6000,71.2000) -- (107.6000,70.3000) -- (107.7000,69.4000) -- (107.8000,68.6000) -- (107.9000,67.7000) -- (108.0000,66.9000) -- (108.1000,66.0000) -- (108.2000,65.2000) -- (108.2000,64.5000) -- (108.3000,63.7000) -- (108.4000,63.0000) -- (108.5000,62.4000) -- (108.6000,61.8000) -- (108.7000,61.2000) -- (108.8000,60.6000) -- (108.9000,60.1000) -- (108.9000,59.6000) -- (109.0000,59.2000) -- (109.1000,58.9000) -- (109.2000,58.5000) -- (109.3000,58.2000) -- (109.4000,58.0000) -- (109.5000,57.8000) -- (109.5000,57.6000) -- (109.6000,57.5000) -- (109.7000,57.4000) -- (109.8000,57.3000) -- (109.9000,57.3000) -- (110.0000,57.3000) -- (110.1000,57.3000) -- (110.2000,57.3000) -- (110.2000,57.4000) -- (110.3000,57.5000) -- (110.4000,57.6000) -- (110.5000,57.7000) -- (110.6000,57.8000) -- (110.7000,57.9000) -- (110.8000,58.1000) -- (110.9000,58.2000) -- (110.9000,58.3000) -- (111.0000,58.4000) -- (111.1000,58.5000) -- (111.2000,58.6000) -- (111.3000,58.7000) -- (111.4000,58.8000) -- (111.5000,58.8000) -- (111.5000,58.8000) -- (111.6000,58.8000) -- (111.7000,58.8000) -- (111.8000,58.7000) -- (111.9000,58.6000) -- (112.0000,58.5000) -- (112.1000,58.3000) -- (112.2000,58.2000) -- (112.2000,57.9000) -- (112.3000,57.7000) -- (112.4000,57.4000) -- (112.5000,57.1000) -- (112.6000,56.8000) -- (112.7000,56.4000) -- (112.8000,56.0000) -- (112.8000,55.6000) -- (112.9000,55.1000) -- (113.0000,54.7000) -- (113.1000,54.2000) -- (113.2000,53.7000) -- (113.3000,53.2000) -- (113.4000,52.7000) -- (113.5000,52.1000) -- (113.5000,51.6000) -- (113.6000,51.0000) -- (113.7000,50.5000) -- (113.8000,49.9000) -- (113.9000,49.4000) -- (114.0000,48.9000) -- (114.1000,48.4000) -- (114.1000,47.9000) -- (114.2000,47.4000) -- (114.3000,46.9000) -- (114.4000,46.5000) -- (114.5000,46.1000) -- (114.6000,45.7000) -- (114.7000,45.4000) -- (114.8000,45.1000) -- (114.8000,44.8000) -- (114.9000,44.6000) -- (115.0000,44.4000) -- (115.1000,44.3000) -- (115.2000,44.1000) -- (115.3000,44.1000) -- (115.4000,44.1000) -- (115.5000,44.1000) -- (115.5000,44.1000) -- (115.6000,44.3000) -- (115.7000,44.4000) -- (115.8000,44.6000) -- (115.9000,44.8000) -- (116.0000,45.1000) -- (116.1000,45.4000) -- (116.1000,45.8000) -- (116.2000,46.2000) -- (116.3000,46.6000) -- (116.4000,47.0000) -- (116.5000,47.5000) -- (116.6000,48.0000) -- (116.7000,48.5000) -- (116.8000,49.1000) -- (116.8000,49.6000) -- (116.9000,50.2000) -- (117.0000,50.8000) -- (117.1000,51.3000) -- (117.2000,51.9000) -- (117.3000,52.5000) -- (117.4000,53.1000) -- (117.4000,53.6000) -- (117.5000,54.2000) -- (117.6000,54.8000) -- (117.7000,55.3000) -- (117.8000,55.8000) -- (117.9000,56.3000) -- (118.0000,56.8000) -- (118.1000,57.2000) -- (118.1000,57.6000) -- (118.2000,58.0000) -- (118.3000,58.4000) -- (118.4000,58.7000) -- (118.5000,59.0000) -- (118.6000,59.3000) -- (118.7000,59.5000) -- (118.7000,59.7000) -- (118.8000,59.9000) -- (118.9000,60.0000) -- (119.0000,60.1000) -- (119.1000,60.1000) -- (119.2000,60.2000) -- (119.3000,60.2000) -- (119.4000,60.2000) -- (119.4000,60.1000) -- (119.5000,60.1000) -- (119.6000,60.0000) -- (119.7000,59.9000) -- (119.8000,59.7000) -- (119.9000,59.6000) -- (120.0000,59.5000) -- (120.0000,59.3000) -- (120.1000,59.1000) -- (120.2000,59.0000) -- (120.3000,58.8000) -- (120.4000,58.7000) -- (120.5000,58.6000) -- (120.6000,58.4000) -- (120.7000,58.3000) -- (120.7000,58.2000) -- (120.8000,58.2000) -- (120.9000,58.2000) -- (121.0000,58.1000) -- (121.1000,58.2000) -- (121.2000,58.2000) -- (121.3000,58.3000) -- (121.4000,58.4000) -- (121.4000,58.6000) -- (121.5000,58.8000) -- (121.6000,59.1000) -- (121.7000,59.3000) -- (121.8000,59.7000) -- (121.9000,60.1000) -- (122.0000,60.5000) -- (122.0000,60.9000) -- (122.1000,61.5000) -- (122.2000,62.0000) -- (122.3000,62.6000) -- (122.4000,63.2000) -- (122.5000,63.9000) -- (122.6000,64.6000) -- (122.7000,65.3000) -- (122.7000,66.1000) -- (122.8000,66.9000) -- (122.9000,67.7000) -- (123.0000,68.5000) -- (123.1000,69.4000) -- (123.2000,70.3000) -- (123.3000,71.1000) -- (123.3000,72.0000) -- (123.4000,72.9000) -- (123.5000,73.8000) -- (123.6000,74.7000) -- (123.7000,75.6000) -- (123.8000,76.4000) -- (123.9000,77.3000) -- (124.0000,78.1000) -- (124.0000,78.9000) -- (124.1000,79.7000) -- (124.2000,80.4000) -- (124.3000,81.1000) -- (124.4000,81.7000) -- (124.5000,82.4000) -- (124.6000,82.9000) -- (124.6000,83.4000) -- (124.7000,83.9000) -- (124.8000,84.3000) -- (124.9000,84.6000) -- (125.0000,84.9000) -- (125.1000,85.2000) -- (125.2000,85.3000) -- (125.3000,85.4000) -- (125.3000,85.5000) -- (125.4000,85.4000) -- (125.5000,85.3000) -- (125.6000,85.2000) -- (125.7000,85.0000) -- (125.8000,84.7000) -- (125.9000,84.4000) -- (125.9000,84.0000) -- (126.0000,83.5000) -- (126.1000,83.0000) -- (126.2000,82.4000) -- (126.3000,81.8000) -- (126.4000,81.2000) -- (126.5000,80.5000) -- (126.6000,79.8000) -- (126.6000,79.0000) -- (126.7000,78.2000) -- (126.8000,77.4000) -- (126.9000,76.5000) -- (127.0000,75.7000) -- (127.1000,74.8000) -- (127.2000,73.9000) -- (127.3000,73.0000) -- (127.3000,72.1000) -- (127.4000,71.2000) -- (127.5000,70.3000) -- (127.6000,69.4000) -- (127.7000,68.5000) -- (127.8000,67.6000) -- (127.9000,66.8000) -- (127.9000,66.0000) -- (128.0000,65.2000) -- (128.1000,64.4000) -- (128.2000,63.7000) -- (128.3000,63.0000) -- (128.4000,62.3000) -- (128.5000,61.7000) -- (128.6000,61.1000) -- (128.6000,60.5000) -- (128.7000,60.0000) -- (128.8000,59.6000) -- (128.9000,59.2000) -- (129.0000,58.8000) -- (129.1000,58.5000) -- (129.2000,58.2000) -- (129.2000,57.9000) -- (129.3000,57.7000) -- (129.4000,57.6000) -- (129.5000,57.4000) -- (129.6000,57.3000) -- (129.7000,57.3000) -- (129.8000,57.2000) -- (129.9000,57.3000) -- (129.9000,57.3000) -- (130.0000,57.3000) -- (130.1000,57.4000) -- (130.2000,57.5000) -- (130.3000,57.6000) -- (130.4000,57.7000) -- (130.5000,57.8000) -- (130.5000,58.0000) -- (130.6000,58.1000) -- (130.7000,58.2000) -- (130.8000,58.3000) -- (130.9000,58.5000) -- (131.0000,58.6000) -- (131.1000,58.7000) -- (131.2000,58.7000) -- (131.2000,58.8000) -- (131.3000,58.8000) -- (131.4000,58.8000) -- (131.5000,58.8000) -- (131.6000,58.8000) -- (131.7000,58.7000) -- (131.8000,58.6000) -- (131.8000,58.5000) -- (131.9000,58.3000) -- (132.0000,58.2000) -- (132.1000,57.9000) -- (132.2000,57.7000) -- (132.3000,57.4000) -- (132.4000,57.1000) -- (132.5000,56.8000) -- (132.5000,56.4000) -- (132.6000,56.0000) -- (132.7000,55.6000) -- (132.8000,55.1000) -- (132.9000,54.6000) -- (133.0000,54.2000) -- (133.1000,53.7000) -- (133.2000,53.1000) -- (133.2000,52.6000) -- (133.3000,52.1000) -- (133.4000,51.5000) -- (133.5000,51.0000) -- (133.6000,50.4000) -- (133.7000,49.9000) -- (133.8000,49.4000) -- (133.8000,48.8000) -- (133.9000,48.3000) -- (134.0000,47.8000) -- (134.1000,47.3000) -- (134.2000,46.9000) -- (134.3000,46.5000) -- (134.4000,46.1000) -- (134.5000,45.7000) -- (134.5000,45.3000) -- (134.6000,45.0000) -- (134.7000,44.8000) -- (134.8000,44.6000) -- (134.9000,44.4000) -- (135.0000,44.2000) -- (135.1000,44.1000) -- (135.1000,44.1000) -- (135.2000,44.0000) -- (135.3000,44.1000) -- (135.4000,44.1000) -- (135.5000,44.2000) -- (135.6000,44.4000) -- (135.7000,44.6000) -- (135.8000,44.8000) -- (135.8000,45.1000) -- (135.9000,45.4000) -- (136.0000,45.8000) -- (136.1000,46.2000) -- (136.2000,46.6000) -- (136.3000,47.1000) -- (136.4000,47.5000) -- (136.5000,48.0000) -- (136.5000,48.6000) -- (136.6000,49.1000) -- (136.7000,49.7000) -- (136.8000,50.2000) -- (136.9000,50.8000) -- (137.0000,51.4000) -- (137.1000,52.0000) -- (137.1000,52.6000) -- (137.2000,53.1000) -- (137.3000,53.7000) -- (137.4000,54.3000) -- (137.5000,54.8000) -- (137.6000,55.4000) -- (137.7000,55.9000) -- (137.8000,56.4000) -- (137.8000,56.8000) -- (137.9000,57.3000) -- (138.0000,57.7000) -- (138.1000,58.1000) -- (138.2000,58.4000) -- (138.3000,58.8000) -- (138.4000,59.1000) -- (138.4000,59.3000) -- (138.5000,59.5000) -- (138.6000,59.7000) -- (138.7000,59.9000) -- (138.8000,60.0000) -- (138.9000,60.1000) -- (139.0000,60.2000) -- (139.1000,60.2000) -- (139.1000,60.2000) -- (139.2000,60.2000) -- (139.3000,60.1000) -- (139.4000,60.1000) -- (139.5000,60.0000) -- (139.6000,59.9000) -- (139.7000,59.7000) -- (139.7000,59.6000) -- (139.8000,59.4000) -- (139.9000,59.3000) -- (140.0000,59.1000) -- (140.1000,59.0000) -- (140.2000,58.8000) -- (140.3000,58.7000) -- (140.4000,58.5000) -- (140.4000,58.4000) -- (140.5000,58.3000) -- (140.6000,58.2000) -- (140.7000,58.2000) -- (140.8000,58.1000) -- (140.9000,58.1000) -- (141.0000,58.1000) -- (141.1000,58.2000) -- (141.1000,58.3000) -- (141.2000,58.4000) -- (141.3000,58.6000) -- (141.4000,58.8000) -- (141.5000,59.1000) -- (141.6000,59.4000) -- (141.7000,59.7000) -- (141.7000,60.1000) -- (141.8000,60.5000) -- (141.9000,61.0000) -- (142.0000,61.5000) -- (142.1000,62.0000) -- (142.2000,62.6000) -- (142.3000,63.3000) -- (142.4000,63.9000) -- (142.4000,64.6000) -- (142.5000,65.4000) -- (142.6000,66.1000) -- (142.7000,67.0000);



  \end{scope}
  \begin{scope}[cm={{1.26012,0.0,0.0,1.26012,(-482.16108,-168.79053)}},draw=blue,line cap=round,line join=round,line width=0.480pt]
    \path[draw] (56.5000,13.5000) -- (56.5000,95.5000) -- (142.5000,95.5000) -- (142.5000,13.5000) -- (56.5000,13.5000);



  \end{scope}
  \begin{scope}[cm={{1.15801,0.0,0.0,1.15801,(-615.96866,-167.12877)}},draw=ca0a0a4,dash pattern=on 0.40pt off 0.80pt,line cap=round,line join=round,line width=0.400pt]
    \path[draw] (44.5000,102.5000) -- (142.5000,102.5000);



  \end{scope}
  \begin{scope}[cm={{1.15801,0.0,0.0,1.15801,(-615.96866,-167.12877)}},draw=blue,line cap=round,line join=round,line width=0.480pt]
    \path[draw] (44.5000,102.5000) -- (48.5000,102.5000);



    \path[draw] (142.5000,102.5000) -- (139.5000,102.5000);



  \end{scope}
  \begin{scope}[cm={{1.00588,0.0,0.0,1.00588,(-585.77904,-46.49663)}},draw=blue,line cap=rect,line join=bevel,line width=0.800pt]
    \path[fill=blue] (0.0000,0.0000) node[above right] (text34-9) {-100};



  \end{scope}
  \begin{scope}[cm={{1.15801,0.0,0.0,1.15801,(-615.96866,-167.12877)}},draw=ca0a0a4,dash pattern=on 1.55pt off 1.55pt,line cap=round,line join=round,line width=0.259pt,miter limit=4.00]
    \path[draw,dash pattern=on 1.55pt off 1.55pt,line width=0.259pt,miter limit=4.00] (44.5000,79.5000) -- (142.5000,79.5000);



  \end{scope}
  \begin{scope}[cm={{1.15801,0.0,0.0,1.15801,(-615.96866,-167.12877)}},draw=blue,line cap=round,line join=round,line width=0.480pt]
    \path[draw] (44.5000,79.5000) -- (48.5000,79.5000);



    \path[draw] (142.5000,79.5000) -- (139.5000,79.5000);



  \end{scope}
  \begin{scope}[cm={{1.00588,0.0,0.0,1.00588,(-575.04441,-71.62653)}},draw=blue,line cap=rect,line join=bevel,line width=0.800pt]
    \path[fill=blue] (0.0000,0.0000) node[above right] (text64-6) {0};



  \end{scope}
  \begin{scope}[cm={{1.15801,0.0,0.0,1.15801,(-615.96866,-167.12877)}},draw=ca0a0a4,dash pattern=on 1.55pt off 1.55pt,line cap=round,line join=round,line width=0.259pt,miter limit=4.00]
    \path[draw,dash pattern=on 1.55pt off 1.55pt,line width=0.259pt,miter limit=4.00] (44.5000,57.5000) -- (142.5000,57.5000);



  \end{scope}
  \begin{scope}[cm={{1.15801,0.0,0.0,1.15801,(-615.96866,-167.12877)}},draw=blue,line cap=round,line join=round,line width=0.480pt]
    \path[draw] (44.5000,57.5000) -- (48.5000,57.5000);



    \path[draw] (142.5000,57.5000) -- (139.5000,57.5000);



  \end{scope}
  \begin{scope}[cm={{1.00588,0.0,0.0,1.00588,(-583.09145,-98.25592)}},draw=blue,line cap=rect,line join=bevel,line width=0.800pt]
    \path[fill=blue] (0.0000,0.0000) node[above right] (text94-7) {100};



  \end{scope}
  \begin{scope}[cm={{1.15801,0.0,0.0,1.15801,(-615.96866,-167.12877)}},draw=ca0a0a4,dash pattern=on 1.55pt off 1.55pt,line cap=round,line join=round,line width=0.259pt,miter limit=4.00]
    \path[draw,dash pattern=on 1.55pt off 1.55pt,line width=0.259pt,miter limit=4.00] (44.5000,35.5000) -- (142.5000,35.5000);



  \end{scope}
  \begin{scope}[cm={{1.15801,0.0,0.0,1.15801,(-615.96866,-167.12877)}},draw=blue,line cap=round,line join=round,line width=0.480pt]
    \path[draw] (44.5000,35.5000) -- (48.5000,35.5000);



    \path[draw] (142.5000,35.5000) -- (139.5000,35.5000);



  \end{scope}
  \begin{scope}[cm={{1.00588,0.0,0.0,1.00588,(-583.09145,-123.38532)}},draw=blue,line cap=rect,line join=bevel,line width=0.800pt]
    \path[fill=blue] (0.0000,0.0000) node[above right] (text124-9) {200};



  \end{scope}
  \begin{scope}[cm={{1.15801,0.0,0.0,1.15801,(-615.96866,-167.12877)}},draw=ca0a0a4,dash pattern=on 0.40pt off 0.80pt,line cap=round,line join=round,line width=0.400pt]
    \path[draw] (44.5000,13.5000) -- (142.5000,13.5000);



  \end{scope}
  \begin{scope}[cm={{1.15801,0.0,0.0,1.15801,(-615.96866,-167.12877)}},draw=blue,line cap=round,line join=round,line width=0.480pt]
    \path[draw] (44.5000,13.5000) -- (48.5000,13.5000);



    \path[draw] (142.5000,13.5000) -- (139.5000,13.5000);



  \end{scope}
  \begin{scope}[cm={{1.00588,0.0,0.0,1.00588,(-583.09145,-148.51472)}},draw=blue,line cap=rect,line join=bevel,line width=0.800pt]
    \path[fill=blue] (0.0000,0.0000) node[above right] (text154) {300};



  \end{scope}
  \begin{scope}[cm={{1.15801,0.0,0.0,1.15801,(-615.96866,-167.12877)}},draw=ca0a0a4,dash pattern=on 0.40pt off 0.80pt,line cap=round,line join=round,line width=0.400pt]
    \path[draw] (44.5000,102.5000) -- (44.5000,13.5000);



  \end{scope}
  \begin{scope}[cm={{1.15801,0.0,0.0,1.15801,(-615.96866,-167.12877)}},draw=blue,line cap=round,line join=round,line width=0.480pt]
    \path[draw] (44.5000,102.5000) -- (44.5000,99.5000);



    \path[draw] (44.5000,13.5000) -- (44.5000,16.5000);



  \end{scope}
  \begin{scope}[cm={{1.15801,0.0,0.0,1.15801,(-615.96866,-167.12877)}},draw=ca0a0a4,dash pattern=on 1.55pt off 1.55pt,line cap=round,line join=round,line width=0.259pt,miter limit=4.00]
    \path[draw,dash pattern=on 1.55pt off 1.55pt,line width=0.259pt,miter limit=4.00] (69.5000,102.5000) -- (69.5000,27.5000);



    \path[draw,dash pattern=on 1.55pt off 1.55pt,line width=0.259pt,miter limit=4.00] (69.5000,19.5000) -- (69.5000,13.5000);



  \end{scope}
  \begin{scope}[cm={{1.15801,0.0,0.0,1.15801,(-615.96866,-167.12877)}},draw=blue,line cap=round,line join=round,line width=0.480pt]
    \path[draw] (69.5000,102.5000) -- (69.5000,99.5000);



    \path[draw] (69.5000,13.5000) -- (69.5000,16.5000);



  \end{scope}
  \begin{scope}[cm={{1.15801,0.0,0.0,1.15801,(-615.96866,-167.12877)}},draw=ca0a0a4,dash pattern=on 1.55pt off 1.55pt,line cap=round,line join=round,line width=0.259pt,miter limit=4.00]
    \path[draw,dash pattern=on 1.55pt off 1.55pt,line width=0.259pt,miter limit=4.00] (93.5000,102.5000) -- (93.5000,27.5000);



    \path[draw,dash pattern=on 1.55pt off 1.55pt,line width=0.259pt,miter limit=4.00] (93.5000,19.5000) -- (93.5000,13.5000);



  \end{scope}
  \begin{scope}[cm={{1.15801,0.0,0.0,1.15801,(-615.96866,-167.12877)}},draw=blue,line cap=round,line join=round,line width=0.480pt]
    \path[draw] (93.5000,102.5000) -- (93.5000,99.5000);



    \path[draw] (93.5000,13.5000) -- (93.5000,16.5000);



  \end{scope}
  \begin{scope}[cm={{1.15801,0.0,0.0,1.15801,(-615.96866,-167.12877)}},draw=ca0a0a4,dash pattern=on 1.55pt off 1.55pt,line cap=round,line join=round,line width=0.259pt,miter limit=4.00]
    \path[draw,dash pattern=on 1.55pt off 1.55pt,line width=0.259pt,miter limit=4.00] (118.5000,102.5000) -- (118.5000,13.5000);



  \end{scope}
  \begin{scope}[cm={{1.15801,0.0,0.0,1.15801,(-615.96866,-167.12877)}},draw=blue,line cap=round,line join=round,line width=0.480pt]
    \path[draw] (118.5000,102.5000) -- (118.5000,99.5000);



    \path[draw] (118.5000,13.5000) -- (118.5000,16.5000);



  \end{scope}
  \begin{scope}[cm={{1.15801,0.0,0.0,1.15801,(-615.96866,-167.12877)}},draw=ca0a0a4,dash pattern=on 0.40pt off 0.80pt,line cap=round,line join=round,line width=0.400pt]
    \path[draw] (142.5000,102.5000) -- (142.5000,13.5000);



  \end{scope}
  \begin{scope}[cm={{1.15801,0.0,0.0,1.15801,(-615.96866,-167.12877)}},draw=blue,line cap=round,line join=round,line width=0.480pt]
    \path[draw] (142.5000,102.5000) -- (142.5000,99.5000);



    \path[draw] (142.5000,13.5000) -- (142.5000,16.5000);



  \end{scope}
  \begin{scope}[cm={{1.15801,0.0,0.0,1.15801,(-615.96866,-167.12877)}},draw=blue,line cap=round,line join=round,line width=0.480pt]
    \path[draw] (44.5000,13.5000) -- (44.5000,102.5000) -- (142.5000,102.5000) -- (142.5000,13.5000) -- (44.5000,13.5000);



  \end{scope}
  \begin{scope}[cm={{0.84029,0.0,0.0,0.84029,(-557.89843,-136.56808)}},draw=blue,line cap=rect,line join=bevel,line width=0.800pt]
    \path[fill=blue] (0.0000,0.0000) node[above right] (text360) {\scriptsize $\mathbf{p}(t)$};



  \end{scope}
  \begin{scope}[cm={{1.15801,0.0,0.0,1.15801,(-623.4871,-168.38184)}},draw=blue,line cap=round,line join=round,line width=0.480pt]
    \path[draw,even odd rule] (74.5000,23.5000) -- (101.5000,23.5000);



  \end{scope}
  \begin{scope}[cm={{1.15801,0.0,0.0,1.15801,(-615.96866,-167.12877)}},draw=blue,line cap=round,line join=round,line width=0.480pt]
    \path[draw] (57.5000,32.8000) -- (57.5000,32.8000) -- (58.3000,35.4000) -- (57.8000,37.2000) -- (56.3000,38.9000) -- (54.9000,41.1000) -- (53.9000,43.5000) -- (53.5000,46.1000) -- (53.4000,48.6000) -- (53.4000,51.1000) -- (53.4000,53.6000) -- (53.4000,56.1000) -- (53.4000,58.6000) -- (53.4000,61.1000) -- (53.4000,63.6000) -- (53.4000,66.1000) -- (53.4000,68.7000) -- (53.4000,71.2000) -- (53.4000,73.7000) -- (53.4000,76.2000) -- (53.4000,78.7000) -- (53.9000,81.3000) -- (54.7000,83.7000) -- (56.0000,86.0000) -- (57.6000,88.1000) -- (59.5000,89.9000) -- (61.7000,91.4000) -- (64.1000,92.6000) -- (66.6000,93.4000) -- (69.2000,93.9000) -- (71.8000,94.1000) -- (74.4000,93.9000) -- (77.0000,93.4000) -- (79.5000,92.6000) -- (81.8000,91.4000) -- (84.0000,90.0000) -- (85.9000,88.2000) -- (87.5000,86.1000) -- (88.6000,83.8000) -- (89.1000,81.3000) -- (89.2000,78.8000) -- (89.3000,76.2000) -- (89.3000,73.7000) -- (89.3000,71.1000) -- (89.3000,68.6000) -- (89.3000,66.1000) -- (89.3000,63.6000) -- (89.3000,61.0000) -- (89.3000,58.5000) -- (89.3000,56.0000) -- (89.3000,53.5000) -- (89.6000,51.0000) -- (90.0000,48.5000) -- (89.9000,46.0000) -- (89.4000,43.6000) -- (88.5000,41.3000) -- (87.2000,39.1000) -- (85.5000,37.2000) -- (83.5000,35.5000) -- (81.3000,34.1000) -- (78.9000,33.1000) -- (76.3000,32.5000) -- (73.6000,32.2000) -- (71.0000,32.3000) -- (68.3000,32.8000) -- (65.8000,33.7000) -- (63.5000,34.9000) -- (61.4000,36.5000) -- (59.6000,38.3000) -- (58.2000,40.4000) -- (57.1000,42.7000) -- (56.5000,45.2000) -- (56.3000,47.7000) -- (56.3000,50.1000) -- (56.3000,52.6000) -- (56.3000,55.1000) -- (56.3000,57.6000) -- (56.3000,60.2000) -- (56.3000,62.7000) -- (56.3000,65.2000) -- (56.3000,67.7000) -- (56.3000,70.2000) -- (56.3000,72.7000) -- (56.3000,75.3000) -- (56.3000,77.8000) -- (56.6000,80.3000) -- (57.3000,82.8000) -- (58.5000,85.1000) -- (60.1000,87.2000) -- (61.9000,89.1000) -- (64.1000,90.7000) -- (66.4000,91.9000) -- (68.9000,92.8000) -- (71.5000,93.4000) -- (74.1000,93.6000) -- (76.7000,93.5000) -- (79.3000,93.0000) -- (81.8000,92.2000) -- (84.1000,91.0000) -- (86.3000,89.5000) -- (88.2000,87.7000) -- (89.8000,85.7000) -- (90.9000,83.4000) -- (91.4000,80.9000) -- (91.6000,78.3000) -- (91.6000,75.8000) -- (91.6000,73.2000) -- (91.6000,70.7000) -- (91.6000,68.2000) -- (91.6000,65.6000) -- (91.6000,63.1000) -- (91.6000,60.6000) -- (91.6000,58.1000) -- (91.6000,55.6000) -- (91.6000,53.0000) -- (91.9000,50.5000) -- (92.3000,48.0000) -- (92.2000,45.6000) -- (91.7000,43.2000) -- (90.7000,40.9000) -- (89.4000,38.7000) -- (87.7000,36.8000) -- (85.6000,35.2000) -- (83.3000,33.9000) -- (80.9000,32.9000) -- (78.3000,32.4000) -- (75.6000,32.2000) -- (72.9000,32.4000) -- (70.3000,33.0000) -- (67.9000,33.9000) -- (65.6000,35.2000) -- (63.6000,36.9000) -- (61.9000,38.8000) -- (60.6000,41.0000) -- (59.7000,43.4000) -- (59.3000,45.9000) -- (59.2000,48.3000) -- (59.1000,50.8000) -- (59.1000,53.3000) -- (59.1000,55.8000) -- (59.1000,58.3000) -- (59.1000,60.9000) -- (59.1000,63.4000) -- (59.1000,65.9000) -- (59.1000,68.4000) -- (59.1000,70.9000) -- (59.1000,73.4000) -- (59.1000,76.0000) -- (59.2000,78.5000) -- (59.6000,81.0000) -- (60.6000,83.5000) -- (61.9000,85.7000) -- (63.6000,87.7000) -- (65.6000,89.4000) -- (67.8000,90.9000) -- (70.2000,92.0000) -- (72.7000,92.8000) -- (75.3000,93.2000) -- (78.0000,93.2000) -- (80.6000,92.9000) -- (83.1000,92.3000) -- (85.5000,91.3000) -- (87.8000,89.9000) -- (89.8000,88.3000) -- (91.6000,86.3000) -- (92.9000,84.1000) -- (93.6000,81.7000) -- (93.9000,79.2000) -- (94.0000,76.6000) -- (94.0000,74.0000) -- (94.1000,71.5000) -- (94.1000,69.0000) -- (94.1000,66.4000) -- (94.1000,63.9000) -- (94.1000,61.4000) -- (94.1000,58.9000) -- (94.1000,56.4000) -- (94.1000,53.9000) -- (94.2000,51.3000) -- (94.6000,48.8000) -- (94.7000,46.4000) -- (94.4000,43.9000) -- (93.6000,41.6000) -- (92.4000,39.3000) -- (90.8000,37.3000) -- (88.9000,35.6000) -- (86.7000,34.2000) -- (84.3000,33.1000) -- (81.7000,32.4000) -- (79.1000,32.0000) -- (76.4000,32.1000) -- (73.8000,32.5000) -- (71.3000,33.4000) -- (68.9000,34.6000) -- (66.8000,36.1000) -- (65.0000,37.9000) -- (63.6000,40.1000) -- (62.6000,42.4000) -- (62.0000,44.8000) -- (61.8000,47.3000) -- (61.8000,49.8000) -- (61.7000,52.3000) -- (61.7000,54.8000) -- (61.7000,57.3000) -- (61.7000,59.8000) -- (61.7000,62.4000) -- (61.7000,64.9000) -- (61.7000,67.4000) -- (61.7000,69.9000) -- (61.7000,72.4000) -- (61.7000,75.0000) -- (61.7000,77.5000) -- (62.0000,80.0000) -- (62.8000,82.5000) -- (64.0000,84.8000) -- (65.6000,86.9000) -- (67.5000,88.7000) -- (69.7000,90.2000) -- (72.1000,91.4000) -- (74.6000,92.3000) -- (77.1000,92.8000) -- (79.8000,93.0000) -- (82.4000,92.8000) -- (84.9000,92.2000) -- (87.4000,91.3000) -- (89.7000,90.0000) -- (91.8000,88.5000) -- (93.6000,86.6000) -- (95.1000,84.4000) -- (95.9000,82.1000) -- (96.3000,79.6000) -- (96.4000,77.0000) -- (96.5000,74.4000) -- (96.5000,71.9000) -- (96.5000,69.3000) -- (96.5000,66.8000) -- (96.5000,64.3000) -- (96.5000,61.8000) -- (96.5000,59.3000) -- (96.5000,56.7000) -- (96.5000,54.2000) -- (96.5000,51.7000) -- (97.0000,49.2000) -- (97.2000,46.7000) -- (96.9000,44.3000) -- (96.3000,41.9000) -- (95.1000,39.6000) -- (93.6000,37.6000) -- (91.8000,35.8000) -- (89.7000,34.3000) -- (87.3000,33.1000) -- (84.8000,32.3000) -- (82.1000,31.9000) -- (79.5000,31.8000) -- (76.8000,32.2000) -- (74.3000,32.9000) -- (71.9000,34.1000) -- (69.7000,35.5000) -- (67.9000,37.3000) -- (66.4000,39.4000) -- (65.2000,41.7000) -- (64.6000,44.1000) -- (64.3000,46.6000) -- (64.3000,49.1000) -- (64.2000,51.6000) -- (64.2000,54.1000) -- (64.2000,56.6000) -- (64.2000,59.1000) -- (64.2000,61.6000) -- (64.2000,64.1000) -- (64.2000,66.6000) -- (64.2000,69.2000) -- (64.2000,71.7000) -- (64.2000,74.2000) -- (64.2000,76.7000) -- (64.4000,79.3000) -- (65.1000,81.8000) -- (66.2000,84.1000) -- (67.8000,86.2000) -- (69.6000,88.1000) -- (71.7000,89.7000) -- (74.0000,91.0000) -- (76.5000,91.9000) -- (79.1000,92.5000) -- (81.7000,92.7000) -- (84.3000,92.6000) -- (86.9000,92.1000) -- (89.4000,91.3000) -- (91.7000,90.1000) -- (93.9000,88.6000) -- (95.7000,86.7000) -- (97.3000,84.6000) -- (98.2000,82.3000) -- (98.7000,79.8000) -- (98.9000,77.3000) -- (98.9000,74.7000) -- (98.9000,72.1000) -- (99.0000,69.6000) -- (99.0000,67.1000) -- (99.0000,64.6000) -- (99.0000,62.0000) -- (99.0000,59.5000) -- (99.0000,57.0000) -- (99.0000,54.5000) -- (99.0000,52.0000) -- (99.3000,49.4000) -- (99.6000,47.0000) -- (99.5000,44.5000) -- (98.9000,42.1000) -- (97.8000,39.8000) -- (96.4000,37.7000) -- (94.6000,35.9000) -- (92.5000,34.3000) -- (90.2000,33.1000) -- (87.7000,32.2000) -- (85.1000,31.7000) -- (82.4000,31.6000) -- (79.7000,31.9000) -- (77.2000,32.6000) -- (74.8000,33.6000) -- (72.6000,35.1000) -- (70.6000,36.8000) -- (69.1000,38.8000) -- (67.9000,41.1000) -- (67.2000,43.5000) -- (66.9000,46.0000) -- (66.8000,48.5000) -- (66.8000,51.0000) -- (66.7000,53.5000) -- (66.7000,56.0000) -- (66.7000,58.5000) -- (66.7000,61.0000) -- (66.7000,63.5000) -- (66.7000,66.0000) -- (66.7000,68.6000) -- (66.7000,71.1000) -- (66.7000,73.6000) -- (66.7000,76.1000) -- (66.9000,78.7000) -- (67.5000,81.2000) -- (68.6000,83.5000) -- (70.0000,85.7000) -- (71.8000,87.6000) -- (73.9000,89.2000) -- (76.2000,90.6000) -- (78.7000,91.6000) -- (81.2000,92.2000) -- (83.8000,92.5000) -- (86.5000,92.4000) -- (89.0000,92.0000) -- (91.5000,91.2000) -- (93.9000,90.0000) -- (96.1000,88.5000) -- (98.0000,86.7000) -- (99.5000,84.7000) -- (100.6000,82.4000) -- (101.1000,79.9000) -- (101.3000,77.3000) -- (101.4000,74.8000) -- (101.4000,72.2000) -- (101.4000,69.7000) -- (101.4000,67.2000) -- (101.4000,64.6000) -- (101.4000,62.1000) -- (101.4000,59.6000) -- (101.4000,57.1000) -- (101.4000,54.6000) -- (101.4000,52.0000) -- (101.7000,49.5000) -- (102.1000,47.0000) -- (102.0000,44.6000) -- (101.4000,42.2000) -- (100.4000,39.9000) -- (99.0000,37.7000) -- (97.3000,35.9000) -- (95.2000,34.2000) -- (92.9000,33.0000) -- (90.4000,32.1000) -- (87.8000,31.5000) -- (85.2000,31.4000) -- (82.5000,31.6000) -- (79.9000,32.3000) -- (77.5000,33.3000) -- (75.3000,34.6000) -- (73.3000,36.3000) -- (71.7000,38.3000) -- (70.5000,40.6000) -- (69.7000,43.0000) -- (69.3000,45.5000) -- (69.2000,47.9000) -- (69.2000,50.4000) -- (69.2000,52.9000) -- (69.2000,55.4000) -- (69.2000,57.9000) -- (69.2000,60.5000) -- (69.2000,63.0000) -- (69.2000,65.5000) -- (69.2000,68.0000) -- (69.2000,70.5000) -- (69.2000,73.1000) -- (69.2000,75.6000) -- (69.3000,78.1000) -- (69.8000,80.6000) -- (70.8000,83.0000) -- (72.3000,85.2000) -- (74.0000,87.2000) -- (76.1000,88.8000) -- (78.3000,90.2000) -- (80.8000,91.2000) -- (83.3000,91.9000) -- (85.9000,92.3000) -- (88.6000,92.2000) -- (91.2000,91.8000) -- (93.7000,91.1000) -- (96.0000,90.0000) -- (98.2000,88.5000) -- (100.2000,86.8000) -- (101.8000,84.7000) -- (102.9000,82.5000) -- (103.5000,80.0000) -- (103.7000,77.4000) -- (103.8000,74.9000) -- (103.8000,72.3000) -- (103.8000,69.8000) -- (103.8000,67.3000) -- (103.8000,64.7000) -- (103.9000,62.2000) -- (103.8000,59.7000) -- (103.8000,57.2000) -- (103.8000,54.7000) -- (103.8000,52.1000) -- (104.1000,49.6000) -- (104.5000,47.1000) -- (104.4000,44.7000) -- (103.9000,42.3000) -- (103.0000,39.9000) -- (101.7000,37.8000) -- (100.0000,35.9000) -- (98.0000,34.2000) -- (95.7000,32.9000) -- (93.2000,31.9000) -- (90.6000,31.4000) -- (88.0000,31.2000) -- (85.3000,31.3000) -- (82.7000,31.9000) -- (80.2000,32.9000) -- (78.0000,34.2000) -- (76.0000,35.9000) -- (74.3000,37.8000) -- (73.0000,40.0000) -- (72.2000,42.4000) -- (71.8000,44.9000) -- (71.7000,47.4000) -- (71.6000,49.9000) -- (71.6000,52.4000) -- (71.6000,54.9000) -- (71.6000,57.4000) -- (71.6000,59.9000) -- (71.6000,62.4000) -- (71.6000,65.0000) -- (71.6000,67.5000) -- (71.6000,70.0000) -- (71.6000,72.5000) -- (71.6000,75.0000) -- (71.7000,77.6000) -- (72.2000,80.1000) -- (73.1000,82.5000) -- (74.5000,84.7000) -- (76.2000,86.7000) -- (78.2000,88.4000) -- (80.5000,89.8000) -- (82.9000,90.9000) -- (85.4000,91.6000) -- (88.0000,92.0000) -- (90.7000,92.0000) -- (93.3000,91.7000) -- (95.8000,91.0000) -- (98.2000,89.9000) -- (100.4000,88.5000) -- (102.4000,86.8000) -- (104.1000,84.8000) -- (105.3000,82.6000) -- (105.9000,80.1000) -- (106.2000,77.6000) -- (106.2000,75.0000) -- (106.3000,72.4000) -- (106.3000,69.9000) -- (106.3000,67.4000) -- (106.3000,64.9000) -- (106.3000,62.3000) -- (106.3000,59.8000) -- (106.3000,57.3000) -- (106.3000,54.8000) -- (106.3000,52.3000) -- (106.5000,49.7000) -- (106.9000,47.2000) -- (106.9000,44.8000) -- (106.5000,42.4000) -- (105.6000,40.0000) -- (104.3000,37.8000) -- (102.7000,35.9000) -- (100.7000,34.2000) -- (98.4000,32.8000) -- (96.0000,31.8000) -- (93.4000,31.2000) -- (90.8000,30.9000) -- (88.1000,31.1000) -- (85.5000,31.6000) -- (83.0000,32.5000) -- (80.7000,33.8000) -- (78.7000,35.4000) -- (77.0000,37.4000) -- (75.6000,39.5000) -- (74.8000,41.9000) -- (74.3000,44.4000) -- (74.2000,46.9000) -- (74.1000,49.3000) -- (74.1000,51.8000) -- (74.1000,54.3000) -- (74.1000,56.9000) -- (74.1000,59.4000) -- (74.1000,61.9000) -- (74.1000,64.4000) -- (74.1000,66.9000) -- (74.1000,69.4000) -- (74.1000,72.0000) -- (74.1000,74.5000) -- (74.1000,77.0000) -- (74.5000,79.6000) -- (75.4000,82.0000) -- (76.8000,84.2000) -- (78.5000,86.2000) -- (80.4000,88.0000) -- (82.6000,89.4000) -- (85.0000,90.6000) -- (87.6000,91.3000) -- (90.2000,91.8000) -- (92.8000,91.8000) -- (95.4000,91.5000) -- (97.9000,90.9000) -- (100.3000,89.8000) -- (102.6000,88.5000) -- (104.6000,86.8000) -- (106.3000,84.8000) -- (107.6000,82.6000) -- (108.3000,80.2000) -- (108.6000,77.7000) -- (108.7000,75.1000) -- (108.7000,72.5000) -- (108.7000,70.0000) -- (108.7000,67.5000) -- (108.7000,64.9000) -- (108.7000,62.4000) -- (108.7000,59.9000) -- (108.7000,57.4000) -- (108.7000,54.9000) -- (108.7000,52.3000) -- (108.9000,49.8000) -- (109.3000,47.3000) -- (109.4000,44.9000) -- (109.0000,42.4000) -- (108.2000,40.1000) -- (106.9000,37.9000) -- (105.3000,35.9000) -- (103.4000,34.2000) -- (101.2000,32.8000) -- (98.8000,31.7000) -- (96.2000,31.0000) -- (93.5000,30.7000) -- (90.9000,30.8000) -- (88.2000,31.3000) -- (85.7000,32.2000) -- (83.4000,33.4000) -- (81.4000,35.0000) -- (79.6000,36.9000) -- (78.2000,39.0000) -- (77.3000,41.4000) -- (76.8000,43.8000) -- (76.6000,46.3000) -- (76.5000,48.8000) -- (76.5000,51.3000) -- (76.5000,53.8000) -- (76.5000,56.3000) -- (76.5000,58.8000) -- (76.5000,61.4000) -- (76.5000,63.9000) -- (76.5000,66.4000) -- (76.5000,68.9000) -- (76.5000,71.4000) -- (76.5000,73.9000) -- (76.5000,76.5000) -- (76.9000,79.0000) -- (77.7000,81.5000) -- (79.0000,83.7000) -- (80.7000,85.8000) -- (82.6000,87.6000) -- (84.8000,89.0000) -- (87.2000,90.2000) -- (89.7000,91.0000) -- (92.3000,91.5000) -- (94.9000,91.6000) -- (97.5000,91.4000) -- (100.1000,90.7000) -- (102.5000,89.8000) -- (104.8000,88.4000) -- (106.8000,86.8000) -- (108.6000,84.9000) -- (109.9000,82.7000) -- (110.7000,80.3000) -- (111.0000,77.8000) -- (111.1000,75.2000) -- (111.2000,72.6000) -- (111.2000,70.1000) -- (111.2000,67.6000) -- (111.2000,65.0000) -- (111.2000,62.5000) -- (111.2000,60.0000) -- (111.2000,57.5000) -- (111.2000,55.0000) -- (111.2000,52.4000) -- (111.3000,49.9000) -- (111.7000,47.4000) -- (111.9000,45.0000) -- (111.5000,42.5000) -- (110.8000,40.2000) -- (109.6000,37.9000) -- (108.0000,35.9000) -- (106.1000,34.1000) -- (103.9000,32.7000) -- (101.5000,31.6000) -- (99.0000,30.9000) -- (96.3000,30.5000) -- (93.6000,30.6000) -- (91.0000,31.0000) -- (88.5000,31.8000) -- (86.2000,33.0000) -- (84.1000,34.6000) -- (82.3000,36.4000) -- (80.8000,38.5000) -- (79.8000,40.8000) -- (79.3000,43.3000) -- (79.1000,45.8000) -- (79.0000,48.3000) -- (79.0000,50.8000) -- (79.0000,53.3000) -- (79.0000,55.8000) -- (79.0000,58.3000) -- (79.0000,60.8000) -- (79.0000,63.3000) -- (79.0000,65.9000) -- (79.0000,68.4000) -- (79.0000,70.9000) -- (79.0000,73.4000) -- (79.0000,75.9000) -- (79.3000,78.5000) -- (80.1000,81.0000) -- (81.3000,83.2000) -- (82.9000,85.3000) -- (84.8000,87.1000) -- (87.0000,88.6000) -- (89.3000,89.9000) -- (91.8000,90.7000) -- (94.4000,91.3000) -- (97.0000,91.4000) -- (99.6000,91.2000) -- (102.2000,90.6000) -- (104.6000,89.7000) -- (106.9000,88.4000) -- (109.0000,86.8000) -- (110.8000,84.9000) -- (112.2000,82.8000) -- (113.1000,80.4000) -- (113.4000,77.9000) -- (113.6000,75.3000) -- (113.6000,72.7000) -- (113.6000,70.2000) -- (113.6000,67.6000) -- (113.6000,65.1000) -- (113.6000,62.6000) -- (113.6000,60.1000) -- (113.6000,57.6000) -- (113.6000,55.1000) -- (113.6000,52.5000) -- (113.7000,50.0000) -- (114.1000,47.5000) -- (114.3000,45.0000) -- (114.0000,42.6000) -- (113.3000,40.2000) -- (112.2000,38.0000) -- (110.7000,35.9000) -- (108.8000,34.1000) -- (106.6000,32.6000) -- (104.3000,31.5000) -- (101.7000,30.7000) -- (99.1000,30.3000) -- (96.4000,30.3000) -- (93.8000,30.7000) -- (91.2000,31.5000) -- (88.9000,32.6000) -- (86.7000,34.1000) -- (84.9000,36.0000) -- (83.4000,38.0000) -- (82.4000,40.3000) -- (81.8000,42.8000) -- (81.6000,45.3000) -- (81.5000,47.8000) -- (81.4000,50.3000) -- (81.4000,52.8000) -- (81.4000,55.3000) -- (81.4000,57.8000) -- (81.4000,60.3000) -- (81.4000,62.8000) -- (81.4000,65.4000) -- (81.4000,67.9000) -- (81.4000,70.4000) -- (81.4000,72.9000) -- (81.4000,75.4000) -- (81.7000,78.0000) -- (82.4000,80.5000) -- (83.6000,82.8000) -- (85.2000,84.9000) -- (87.0000,86.7000) -- (89.2000,88.3000) -- (91.5000,89.5000) -- (94.0000,90.4000) -- (96.6000,91.0000) -- (99.2000,91.2000) -- (101.8000,91.0000) -- (104.4000,90.5000) -- (106.8000,89.6000) -- (109.2000,88.3000) -- (111.3000,86.8000) -- (113.1000,84.9000) -- (114.6000,82.8000) -- (115.5000,80.4000) -- (115.9000,77.9000) -- (116.0000,75.3000) -- (116.1000,72.8000) -- (116.1000,70.2000) -- (116.1000,67.7000) -- (116.1000,65.2000) -- (116.1000,62.7000) -- (116.1000,60.1000) -- (116.1000,57.6000) -- (116.1000,55.1000) -- (116.1000,52.6000) -- (116.1000,50.1000) -- (116.5000,47.5000) -- (116.8000,45.1000) -- (116.5000,42.6000) -- (115.9000,40.2000) -- (114.8000,38.0000) -- (113.3000,35.9000) -- (111.5000,34.1000) -- (109.3000,32.6000) -- (107.0000,31.4000) -- (104.5000,30.6000) -- (101.8000,30.1000) -- (99.2000,30.1000) -- (96.5000,30.4000) -- (94.0000,31.2000) -- (91.6000,32.3000) -- (89.4000,33.7000) -- (87.5000,35.5000) -- (86.0000,37.6000) -- (84.9000,39.9000) -- (84.3000,42.3000) -- (84.0000,44.8000) -- (83.9000,47.3000) -- (83.9000,49.8000) -- (83.9000,52.3000) -- (83.9000,54.8000) -- (83.9000,57.3000) -- (83.9000,59.8000) -- (83.9000,62.3000) -- (83.9000,64.8000) -- (83.9000,67.4000) -- (83.9000,69.9000) -- (83.9000,72.4000) -- (83.9000,74.9000) -- (84.0000,77.5000) -- (84.7000,80.0000) -- (85.9000,82.3000) -- (87.4000,84.4000) -- (89.2000,86.3000) -- (91.3000,87.9000) -- (93.7000,89.2000) -- (96.1000,90.1000) -- (98.7000,90.7000) -- (101.3000,91.0000) -- (103.9000,90.8000) -- (106.5000,90.3000) -- (109.0000,89.5000) -- (111.3000,88.3000) -- (113.5000,86.8000) -- (115.3000,84.9000) -- (116.9000,82.8000) -- (117.8000,80.5000) -- (118.3000,78.0000) -- (118.4000,75.4000) -- (118.5000,72.9000) -- (118.5000,70.3000) -- (118.5000,67.8000) -- (118.5000,65.3000) -- (118.5000,62.7000) -- (118.5000,60.2000) -- (118.5000,57.7000) -- (118.5000,55.2000) -- (118.5000,52.7000) -- (118.5000,50.2000) -- (118.9000,47.6000) -- (119.2000,45.1000) -- (119.0000,42.7000) -- (118.4000,40.3000) -- (117.4000,38.0000) -- (115.9000,35.9000) -- (114.1000,34.1000) -- (112.1000,32.5000) -- (109.7000,31.3000) -- (107.2000,30.4000) -- (104.6000,29.9000) -- (101.9000,29.9000) -- (99.3000,30.2000) -- (96.7000,30.8000) -- (94.3000,31.9000) -- (92.1000,33.3000) -- (90.2000,35.1000) -- (88.6000,37.1000) -- (87.5000,39.4000) -- (86.8000,41.8000) -- (86.5000,44.3000) -- (86.4000,46.8000) -- (86.3000,49.3000) -- (86.3000,51.8000) -- (86.3000,54.3000) -- (86.3000,56.8000) -- (86.3000,59.3000) -- (86.3000,61.8000) -- (86.3000,64.3000) -- (86.3000,66.8000) -- (86.3000,69.4000) -- (86.3000,71.9000) -- (86.3000,74.4000) -- (86.5000,76.9000) -- (87.1000,79.5000) -- (88.2000,81.8000) -- (89.7000,84.0000) -- (91.5000,85.9000) -- (93.6000,87.5000) -- (95.8000,88.8000) -- (98.3000,89.8000) -- (100.9000,90.5000) -- (103.5000,90.7000) -- (106.1000,90.6000) -- (108.7000,90.2000) -- (111.2000,89.4000) -- (113.5000,88.2000) -- (115.7000,86.7000) -- (117.6000,84.9000) -- (119.2000,82.9000) -- (120.2000,80.5000) -- (120.7000,78.1000) -- (120.9000,75.5000) -- (120.9000,72.9000) -- (121.0000,70.4000) -- (121.0000,67.8000) -- (121.0000,65.3000) -- (121.0000,62.8000) -- (121.0000,60.3000) -- (121.0000,57.8000) -- (121.0000,55.2000) -- (121.0000,52.7000) -- (121.0000,50.2000) -- (121.3000,47.7000) -- (121.6000,45.2000) -- (121.5000,42.8000) -- (121.0000,40.4000) -- (120.0000,38.0000) -- (118.6000,35.9000) -- (116.8000,34.0000) -- (114.8000,32.4000) -- (112.4000,31.2000) -- (110.0000,30.3000) -- (107.4000,29.7000) -- (104.7000,29.6000) -- (102.0000,29.9000) -- (99.4000,30.5000) -- (97.0000,31.5000) -- (94.8000,32.9000) -- (92.8000,34.6000) -- (91.2000,36.6000) -- (90.0000,38.8000) -- (89.2000,41.2000) -- (88.9000,43.7000) -- (88.8000,46.2000) -- (88.7000,48.7000) -- (88.7000,51.2000) -- (88.7000,53.7000) -- (88.7000,56.2000) -- (88.7000,58.7000) -- (88.7000,61.2000) -- (88.7000,63.8000) -- (88.7000,66.3000) -- (88.7000,68.8000) -- (88.7000,71.3000) -- (88.7000,73.8000) -- (88.8000,76.4000) -- (89.4000,78.9000) -- (90.4000,81.3000) -- (91.8000,83.5000) -- (93.6000,85.4000) -- (95.6000,87.1000) -- (97.9000,88.4000) -- (100.3000,89.5000) -- (102.9000,90.2000) -- (105.5000,90.5000) -- (108.1000,90.5000) -- (110.7000,90.1000) -- (113.2000,89.3000) -- (115.6000,88.2000) -- (117.8000,86.8000) -- (119.8000,85.0000) -- (121.4000,83.0000) -- (122.5000,80.7000) -- (123.1000,78.3000) -- (123.3000,75.7000) -- (123.4000,73.1000) -- (123.4000,70.6000) -- (123.4000,68.0000) -- (123.4000,65.5000) -- (123.4000,63.0000) -- (123.4000,60.5000) -- (123.4000,58.0000) -- (123.4000,55.4000) -- (123.4000,52.9000) -- (123.4000,50.4000) -- (123.7000,47.9000) -- (124.1000,45.4000) -- (124.0000,42.9000) -- (123.5000,40.5000) -- (122.6000,38.2000) -- (121.3000,36.0000) -- (119.6000,34.1000) -- (117.6000,32.5000) -- (115.3000,31.1000) -- (112.8000,30.2000) -- (110.2000,29.6000) -- (107.6000,29.4000) -- (104.9000,29.6000) -- (102.3000,30.1000) -- (99.9000,31.1000) -- (97.6000,32.4000) -- (95.6000,34.1000) -- (93.9000,36.0000) -- (92.6000,38.2000) -- (91.8000,40.6000) -- (91.4000,43.1000) -- (91.2000,45.6000) -- (91.2000,48.0000) -- (91.2000,50.5000) -- (91.2000,53.0000) -- (91.2000,55.6000) -- (91.2000,58.1000) -- (91.2000,60.6000) -- (91.2000,63.1000) -- (91.2000,65.6000) -- (91.2000,68.1000) -- (91.2000,70.7000) -- (91.2000,73.2000) -- (91.2000,75.7000) -- (91.7000,78.3000) -- (92.6000,80.7000) -- (94.0000,82.9000) -- (95.7000,84.9000) -- (97.7000,86.6000) -- (99.9000,88.0000) -- (102.4000,89.1000) -- (104.9000,89.9000) -- (107.5000,90.3000) -- (110.1000,90.3000) -- (112.7000,90.0000) -- (115.3000,89.3000) -- (117.7000,88.2000) -- (119.9000,86.8000) -- (121.9000,85.1000) -- (123.6000,83.1000) -- (124.8000,80.9000) -- (125.5000,78.5000) -- (125.7000,75.9000) -- (125.8000,73.4000) -- (125.9000,70.8000) -- (125.9000,68.2000) -- (125.9000,65.7000) -- (125.9000,63.2000) -- (125.9000,60.7000) -- (125.9000,58.2000) -- (125.9000,55.7000) -- (125.9000,53.1000) -- (125.9000,50.6000) -- (126.0000,48.1000) -- (126.5000,45.5000);



  \end{scope}
  \begin{scope}[cm={{1.15801,0.0,0.0,1.15801,(-615.96866,-167.12877)}},draw=blue,line cap=round,line join=round,line width=0.480pt]
    \path[draw] (44.5000,13.5000) -- (44.5000,102.5000) -- (142.5000,102.5000) -- (142.5000,13.5000) -- (44.5000,13.5000);



  \end{scope}
  \begin{scope}[cm={{1.15801,0.0,0.0,1.15801,(-736.40737,-53.12875)}},draw=ca0a0a4,dash pattern=on 0.40pt off 0.80pt,line cap=round,line join=round,line width=0.400pt]
    \path[draw] (148.5000,102.5000) -- (246.5000,102.5000);



  \end{scope}
  \begin{scope}[cm={{1.15801,0.0,0.0,1.15801,(-736.40737,-53.12875)}},draw=blue,line cap=round,line join=round,line width=0.480pt]
    \path[draw] (148.5000,102.5000) -- (151.5000,102.5000);



    \path[draw] (246.5000,102.5000) -- (243.5000,102.5000);



  \end{scope}
  \begin{scope}[cm={{1.15801,0.0,0.0,1.15801,(-736.40737,-53.12875)}},draw=ca0a0a4,dash pattern=on 1.55pt off 1.55pt,line cap=round,line join=round,line width=0.259pt,miter limit=4.00]
    \path[draw,dash pattern=on 1.55pt off 1.55pt,line width=0.259pt,miter limit=4.00] (148.5000,79.5000) -- (246.5000,79.5000);



  \end{scope}
  \begin{scope}[cm={{1.15801,0.0,0.0,1.15801,(-736.40737,-53.12875)}},draw=blue,line cap=round,line join=round,line width=0.480pt]
    \path[draw] (148.5000,79.5000) -- (151.5000,79.5000);



    \path[draw] (246.5000,79.5000) -- (243.5000,79.5000);



  \end{scope}
  \begin{scope}[cm={{1.15801,0.0,0.0,1.15801,(-736.40737,-53.12875)}},draw=ca0a0a4,dash pattern=on 1.55pt off 1.55pt,line cap=round,line join=round,line width=0.259pt,miter limit=4.00]
    \path[draw,dash pattern=on 1.55pt off 1.55pt,line width=0.259pt,miter limit=4.00] (148.5000,57.5000) -- (246.5000,57.5000);



  \end{scope}
  \begin{scope}[cm={{1.15801,0.0,0.0,1.15801,(-736.40737,-53.12875)}},draw=blue,line cap=round,line join=round,line width=0.480pt]
    \path[draw] (148.5000,57.5000) -- (151.5000,57.5000);



    \path[draw] (246.5000,57.5000) -- (243.5000,57.5000);



  \end{scope}
  \begin{scope}[cm={{1.15801,0.0,0.0,1.15801,(-736.40737,-53.12875)}},draw=ca0a0a4,dash pattern=on 1.55pt off 1.55pt,line cap=round,line join=round,line width=0.259pt,miter limit=4.00]
    \path[draw,dash pattern=on 1.55pt off 1.55pt,line width=0.259pt,miter limit=4.00] (148.5000,35.5000) -- (246.5000,35.5000);



  \end{scope}
  \begin{scope}[cm={{1.15801,0.0,0.0,1.15801,(-736.40737,-53.12875)}},draw=blue,line cap=round,line join=round,line width=0.480pt]
    \path[draw] (148.5000,35.5000) -- (151.5000,35.5000);



    \path[draw] (246.5000,35.5000) -- (243.5000,35.5000);



  \end{scope}
  \begin{scope}[cm={{1.15801,0.0,0.0,1.15801,(-736.40737,-53.12875)}},draw=ca0a0a4,dash pattern=on 0.40pt off 0.80pt,line cap=round,line join=round,line width=0.400pt]
    \path[draw] (148.5000,13.5000) -- (246.5000,13.5000);



  \end{scope}
  \begin{scope}[cm={{1.15801,0.0,0.0,1.15801,(-673.8747,-53.12875)}},draw=ca0a0a4,dash pattern=on 1.55pt off 1.55pt,line cap=round,line join=round,line width=0.259pt,miter limit=4.00]
    \path[draw,dash pattern=on 1.55pt off 1.55pt,line width=0.259pt,miter limit=4.00] (118.5000,102.5000) -- (118.5000,13.5000);



  \end{scope}
  \begin{scope}[cm={{1.15801,0.0,0.0,1.15801,(-736.40737,-53.12875)}},draw=blue,line cap=round,line join=round,line width=0.480pt]
    \path[draw] (148.5000,13.5000) -- (151.5000,13.5000);



    \path[draw] (246.5000,13.5000) -- (243.5000,13.5000);



  \end{scope}
  \begin{scope}[cm={{1.15801,0.0,0.0,1.15801,(-736.40737,-53.12875)}},draw=ca0a0a4,dash pattern=on 0.40pt off 0.80pt,line cap=round,line join=round,line width=0.400pt]
    \path[draw] (148.5000,102.5000) -- (148.5000,13.5000);



  \end{scope}
  \begin{scope}[cm={{1.15801,0.0,0.0,1.15801,(-736.40737,-53.12875)}},draw=blue,line cap=round,line join=round,line width=0.480pt]
    \path[draw] (148.5000,102.5000) -- (148.5000,99.5000);



    \path[draw] (148.5000,13.5000) -- (148.5000,16.5000);



  \end{scope}
  \begin{scope}[cm={{1.15801,0.0,0.0,1.15801,(-736.40737,-53.12875)}},draw=blue,line cap=round,line join=round,line width=0.480pt]
    \path[draw] (172.5000,102.5000) -- (172.5000,99.5000);



    \path[draw] (172.5000,13.5000) -- (172.5000,16.5000);



  \end{scope}
  \begin{scope}[cm={{1.15801,0.0,0.0,1.15801,(-644.92439,-53.12875)}},draw=ca0a0a4,dash pattern=on 1.55pt off 1.55pt,line cap=round,line join=round,line width=0.259pt,miter limit=4.00]
    \path[draw,dash pattern=on 1.55pt off 1.55pt,line width=0.259pt,miter limit=4.00] (118.5000,102.5000) -- (118.5000,13.5000);



  \end{scope}
  \begin{scope}[cm={{1.15801,0.0,0.0,1.15801,(-736.40737,-53.12875)}},draw=blue,line cap=round,line join=round,line width=0.480pt]
    \path[draw] (197.5000,102.5000) -- (197.5000,99.5000);



    \path[draw] (197.5000,13.5000) -- (197.5000,16.5000);



  \end{scope}
  \begin{scope}[cm={{1.15801,0.0,0.0,1.15801,(-617.13209,-53.12875)}},draw=ca0a0a4,dash pattern=on 1.55pt off 1.55pt,line cap=round,line join=round,line width=0.259pt,miter limit=4.00]
    \path[draw,dash pattern=on 1.55pt off 1.55pt,line width=0.259pt,miter limit=4.00] (118.5000,102.5000) -- (118.5000,13.5000);



  \end{scope}
  \begin{scope}[cm={{1.15801,0.0,0.0,1.15801,(-736.40737,-53.12875)}},draw=blue,line cap=round,line join=round,line width=0.480pt]
    \path[draw] (221.5000,102.5000) -- (221.5000,99.5000);



    \path[draw] (221.5000,13.5000) -- (221.5000,16.5000);



  \end{scope}
  \begin{scope}[cm={{1.15801,0.0,0.0,1.15801,(-736.40737,-53.12875)}},draw=ca0a0a4,dash pattern=on 0.40pt off 0.80pt,line cap=round,line join=round,line width=0.400pt]
    \path[draw] (246.5000,102.5000) -- (246.5000,27.5000);



    \path[draw] (246.5000,19.5000) -- (246.5000,13.5000);



  \end{scope}
  \begin{scope}[cm={{1.15801,0.0,0.0,1.15801,(-736.40737,-53.12875)}},draw=blue,line cap=round,line join=round,line width=0.480pt]
    \path[draw] (246.5000,102.5000) -- (246.5000,99.5000);



    \path[draw] (246.5000,13.5000) -- (246.5000,16.5000);



  \end{scope}
  \begin{scope}[cm={{1.15801,0.0,0.0,1.15801,(-736.40737,-53.12875)}},draw=blue,line cap=round,line join=round,line width=0.480pt]
    \path[draw] (148.5000,13.5000) -- (148.5000,102.5000) -- (246.5000,102.5000) -- (246.5000,13.5000) -- (148.5000,13.5000);



  \end{scope}
  \begin{scope}[cm={{0.84029,0.0,0.0,0.84029,(-689.6406,-75.21977)}},fill=cd9d9d9]
    \path[rounded corners=0.0000cm] (222.0000,18.0000) rectangle (238.0000,34.0000);



  \end{scope}
  \begin{scope}[cm={{1.15801,0.0,0.0,1.15801,(-736.40737,-53.12875)}},draw=blue,line cap=round,line join=round,line width=0.480pt]
    \path[draw] (160.6000,31.9000) -- (160.6000,31.9000) -- (160.7000,33.1000) -- (160.7000,34.2000) -- (160.8000,35.2000) -- (160.9000,36.2000) -- (160.9000,37.2000) -- (160.6000,38.2000) -- (159.9000,39.1000) -- (159.3000,40.0000) -- (158.7000,40.9000) -- (158.2000,41.8000) -- (157.7000,42.8000) -- (157.4000,43.8000) -- (157.2000,44.8000) -- (157.1000,45.8000) -- (157.0000,46.9000) -- (157.0000,48.0000) -- (157.0000,49.0000) -- (157.0000,50.1000) -- (157.0000,51.1000) -- (157.0000,52.2000) -- (157.0000,53.3000) -- (157.0000,54.3000) -- (157.0000,55.4000) -- (157.0000,56.4000) -- (157.0000,57.5000) -- (157.0000,58.5000) -- (157.0000,59.6000) -- (157.0000,60.6000) -- (157.0000,61.7000) -- (157.0000,62.8000) -- (157.0000,63.8000) -- (157.0000,64.9000) -- (157.0000,65.9000) -- (157.0000,67.0000) -- (157.0000,68.0000) -- (157.0000,69.1000) -- (157.0000,70.1000) -- (157.0000,71.2000) -- (157.0000,72.2000) -- (157.0000,73.3000) -- (157.0000,74.4000) -- (157.0000,75.4000) -- (157.0000,76.5000) -- (157.0000,77.5000) -- (157.0000,78.6000) -- (157.1000,79.6000) -- (157.3000,80.7000) -- (157.6000,81.7000) -- (157.9000,82.7000) -- (158.3000,83.7000) -- (158.8000,84.7000) -- (159.4000,85.6000) -- (160.0000,86.5000) -- (160.7000,87.4000) -- (161.4000,88.3000) -- (162.2000,89.0000) -- (163.1000,89.8000) -- (164.0000,90.5000) -- (165.0000,91.1000) -- (166.0000,91.7000) -- (167.1000,92.2000) -- (168.2000,92.7000) -- (169.3000,93.1000) -- (170.5000,93.4000) -- (171.7000,93.6000) -- (172.9000,93.8000) -- (174.1000,93.9000) -- (175.3000,94.0000) -- (176.6000,93.9000) -- (177.8000,93.8000) -- (179.0000,93.6000) -- (180.2000,93.4000) -- (181.4000,93.0000) -- (182.5000,92.6000) -- (183.7000,92.1000) -- (184.7000,91.6000) -- (185.8000,91.0000) -- (186.7000,90.3000) -- (187.7000,89.6000) -- (188.5000,88.8000) -- (189.4000,88.0000) -- (190.1000,87.2000) -- (190.8000,86.3000) -- (191.4000,85.3000) -- (191.9000,84.4000) -- (192.3000,83.3000) -- (192.6000,82.3000) -- (192.8000,81.3000) -- (193.0000,80.3000) -- (193.1000,79.2000) -- (193.2000,78.2000) -- (193.3000,77.2000) -- (193.3000,76.1000) -- (193.4000,75.1000) -- (193.4000,74.0000) -- (193.4000,73.0000) -- (193.4000,71.9000) -- (193.4000,70.9000) -- (193.4000,69.8000) -- (193.4000,68.8000) -- (193.5000,67.7000) -- (193.5000,66.6000) -- (193.5000,65.6000) -- (193.5000,64.5000) -- (193.5000,63.5000) -- (193.5000,62.4000) -- (193.5000,61.4000) -- (193.5000,60.3000) -- (193.5000,59.3000) -- (193.5000,58.2000) -- (193.5000,57.2000) -- (193.5000,56.1000) -- (193.5000,55.0000) -- (193.5000,54.0000) -- (193.5000,52.9000) -- (193.5000,51.9000) -- (193.5000,50.8000) -- (193.7000,49.8000) -- (193.9000,48.7000) -- (194.0000,47.7000) -- (194.0000,46.6000) -- (193.9000,45.6000) -- (193.8000,44.5000) -- (193.6000,43.5000) -- (193.2000,42.5000) -- (192.8000,41.5000) -- (192.4000,40.5000) -- (191.8000,39.6000) -- (191.2000,38.7000) -- (190.4000,37.8000) -- (189.6000,37.0000) -- (188.8000,36.3000) -- (187.8000,35.6000) -- (186.8000,34.9000) -- (185.8000,34.4000) -- (184.7000,33.9000) -- (183.5000,33.5000) -- (182.3000,33.2000) -- (181.1000,33.0000) -- (179.9000,32.8000) -- (178.7000,32.8000) -- (177.4000,32.8000) -- (176.2000,32.9000) -- (175.0000,33.1000) -- (173.8000,33.4000) -- (172.7000,33.8000) -- (171.6000,34.2000) -- (170.5000,34.8000) -- (169.5000,35.3000) -- (168.5000,36.0000) -- (167.6000,36.7000) -- (166.8000,37.5000) -- (166.0000,38.3000) -- (165.4000,39.2000) -- (164.7000,40.1000) -- (164.2000,41.1000) -- (163.8000,42.1000) -- (163.4000,43.1000) -- (163.2000,44.1000) -- (163.1000,45.2000) -- (163.0000,46.2000) -- (163.0000,47.3000) -- (163.0000,48.4000) -- (163.0000,49.4000) -- (163.0000,50.5000) -- (163.0000,51.5000) -- (163.0000,52.6000) -- (163.0000,53.6000) -- (163.0000,54.7000) -- (162.9000,55.7000) -- (162.9000,56.8000) -- (162.9000,57.9000) -- (162.9000,58.9000) -- (162.9000,60.0000) -- (162.9000,61.0000) -- (162.9000,62.1000) -- (162.9000,63.1000) -- (162.9000,64.2000) -- (162.9000,65.2000) -- (162.9000,66.3000) -- (162.9000,67.4000) -- (162.9000,68.4000) -- (162.9000,69.5000) -- (162.9000,70.5000) -- (162.9000,71.6000) -- (162.9000,72.6000) -- (162.9000,73.7000) -- (162.9000,74.7000) -- (162.9000,75.8000) -- (162.9000,76.9000) -- (163.0000,77.9000) -- (163.1000,79.0000) -- (163.3000,80.0000) -- (163.5000,81.0000) -- (163.9000,82.1000) -- (164.3000,83.1000) -- (164.7000,84.0000) -- (165.3000,85.0000) -- (165.9000,85.9000) -- (166.6000,86.8000) -- (167.3000,87.6000) -- (168.1000,88.4000) -- (168.9000,89.2000) -- (169.9000,89.9000) -- (170.8000,90.5000) -- (171.8000,91.1000) -- (172.9000,91.7000) -- (174.0000,92.1000) -- (175.1000,92.5000) -- (176.3000,92.9000) -- (177.4000,93.2000) -- (178.6000,93.4000) -- (179.9000,93.5000) -- (181.1000,93.5000) -- (182.3000,93.5000) -- (183.6000,93.4000) -- (184.8000,93.3000) -- (186.0000,93.0000) -- (187.2000,92.7000) -- (188.3000,92.3000) -- (189.5000,91.9000) -- (190.6000,91.4000) -- (191.6000,90.8000) -- (192.6000,90.1000) -- (193.6000,89.4000) -- (194.5000,88.7000) -- (195.3000,87.9000) -- (196.1000,87.1000) -- (196.8000,86.2000) -- (197.4000,85.2000) -- (197.9000,84.3000) -- (198.4000,83.3000) -- (198.7000,82.3000) -- (198.9000,81.2000) -- (199.1000,80.2000) -- (199.3000,79.2000) -- (199.4000,78.1000) -- (199.4000,77.1000) -- (199.5000,76.1000) -- (199.5000,75.0000) -- (199.6000,74.0000) -- (199.6000,72.9000) -- (199.6000,71.9000) -- (199.6000,70.8000) -- (199.6000,69.8000) -- (199.6000,68.7000) -- (199.6000,67.7000) -- (199.6000,66.6000) -- (199.6000,65.5000) -- (199.7000,64.5000) -- (199.7000,63.4000) -- (199.7000,62.4000) -- (199.7000,61.3000) -- (199.7000,60.3000) -- (199.7000,59.2000) -- (199.7000,58.2000) -- (199.7000,57.1000) -- (199.7000,56.0000) -- (199.7000,55.0000) -- (199.7000,53.9000) -- (199.7000,52.9000) -- (199.7000,51.8000) -- (199.7000,50.8000) -- (199.8000,49.7000) -- (200.0000,48.7000) -- (200.1000,47.6000) -- (200.2000,46.6000) -- (200.1000,45.5000) -- (200.0000,44.5000) -- (199.8000,43.4000) -- (199.6000,42.4000) -- (199.2000,41.4000) -- (198.8000,40.4000) -- (198.3000,39.5000) -- (197.6000,38.5000) -- (197.0000,37.7000) -- (196.2000,36.8000) -- (195.4000,36.1000) -- (194.5000,35.3000) -- (193.5000,34.7000) -- (192.5000,34.1000) -- (191.4000,33.5000) -- (190.2000,33.1000) -- (189.1000,32.7000) -- (187.9000,32.5000) -- (186.7000,32.3000) -- (185.4000,32.2000) -- (184.2000,32.2000) -- (183.0000,32.2000) -- (181.8000,32.4000) -- (180.6000,32.7000) -- (179.4000,33.0000) -- (178.3000,33.4000) -- (177.2000,33.9000) -- (176.1000,34.4000) -- (175.2000,35.1000) -- (174.2000,35.8000) -- (173.4000,36.5000) -- (172.6000,37.3000) -- (171.8000,38.2000) -- (171.2000,39.0000) -- (170.6000,40.0000) -- (170.1000,41.0000) -- (169.7000,42.0000) -- (169.4000,43.0000) -- (169.2000,44.0000) -- (169.2000,45.1000) -- (169.1000,46.1000) -- (169.1000,47.2000) -- (169.1000,48.3000) -- (169.1000,49.3000) -- (169.1000,50.4000) -- (169.1000,51.4000) -- (169.1000,52.5000) -- (169.1000,53.6000) -- (169.1000,54.6000) -- (169.1000,55.7000) -- (169.1000,56.7000) -- (169.1000,57.8000) -- (169.1000,58.8000) -- (169.1000,59.9000) -- (169.1000,60.9000) -- (169.1000,62.0000) -- (169.1000,63.1000) -- (169.1000,64.1000) -- (169.1000,65.2000) -- (169.1000,66.2000) -- (169.1000,67.3000) -- (169.1000,68.3000) -- (169.1000,69.4000) -- (169.1000,70.4000) -- (169.1000,71.5000) -- (169.1000,72.6000) -- (169.1000,73.6000) -- (169.1000,74.7000) -- (169.1000,75.7000) -- (169.1000,76.8000) -- (169.1000,77.8000) -- (169.3000,78.9000) -- (169.5000,79.9000) -- (169.8000,80.9000) -- (170.1000,82.0000) -- (170.6000,82.9000) -- (171.1000,83.9000) -- (171.7000,84.8000) -- (172.3000,85.7000) -- (173.0000,86.6000) -- (173.8000,87.4000) -- (174.6000,88.2000) -- (175.5000,88.9000) -- (176.4000,89.6000) -- (177.4000,90.2000) -- (178.4000,90.8000) -- (179.5000,91.3000) -- (180.6000,91.8000) -- (181.7000,92.2000) -- (182.9000,92.5000) -- (184.1000,92.7000) -- (185.3000,92.9000) -- (186.5000,93.0000) -- (187.8000,93.0000) -- (189.0000,93.0000) -- (190.2000,92.8000) -- (191.4000,92.6000) -- (192.6000,92.4000) -- (193.8000,92.0000) -- (195.0000,91.6000) -- (196.1000,91.1000) -- (197.2000,90.6000) -- (198.2000,90.0000) -- (199.2000,89.3000) -- (200.1000,88.6000) -- (201.0000,87.8000) -- (201.8000,87.0000) -- (202.5000,86.2000) -- (203.2000,85.3000) -- (203.8000,84.3000) -- (204.3000,83.3000) -- (204.7000,82.3000) -- (205.0000,81.3000) -- (205.2000,80.3000) -- (205.3000,79.2000) -- (205.5000,78.2000) -- (205.6000,77.2000) -- (205.6000,76.1000) -- (205.7000,75.1000) -- (205.7000,74.0000) -- (205.7000,73.0000) -- (205.8000,71.9000) -- (205.8000,70.9000) -- (205.8000,69.8000) -- (205.8000,68.8000) -- (205.8000,67.7000) -- (205.8000,66.7000) -- (205.8000,65.6000) -- (205.8000,64.6000) -- (205.8000,63.5000) -- (205.8000,62.5000) -- (205.8000,61.4000) -- (205.8000,60.3000) -- (205.8000,59.3000) -- (205.8000,58.2000) -- (205.8000,57.2000) -- (205.8000,56.1000) -- (205.8000,55.1000) -- (205.8000,54.0000) -- (205.8000,53.0000) -- (205.8000,51.9000) -- (205.8000,50.8000) -- (205.9000,49.8000) -- (206.0000,48.7000) -- (206.2000,47.7000) -- (206.3000,46.6000) -- (206.3000,45.6000) -- (206.3000,44.5000) -- (206.1000,43.5000) -- (205.9000,42.5000) -- (205.6000,41.4000) -- (205.2000,40.4000) -- (204.7000,39.5000) -- (204.2000,38.5000) -- (203.5000,37.6000) -- (202.8000,36.8000) -- (202.0000,36.0000) -- (201.2000,35.2000) -- (200.2000,34.5000) -- (199.2000,33.9000) -- (198.2000,33.3000) -- (197.1000,32.8000) -- (195.9000,32.4000) -- (194.8000,32.1000) -- (193.5000,31.8000) -- (192.3000,31.7000) -- (191.1000,31.6000) -- (189.9000,31.6000) -- (188.6000,31.7000) -- (187.4000,31.9000) -- (186.3000,32.2000) -- (185.1000,32.6000) -- (184.0000,33.0000) -- (182.9000,33.5000) -- (181.9000,34.1000) -- (180.9000,34.8000) -- (180.0000,35.5000) -- (179.2000,36.2000) -- (178.4000,37.1000) -- (177.7000,37.9000) -- (177.1000,38.9000) -- (176.6000,39.8000) -- (176.1000,40.8000) -- (175.7000,41.8000) -- (175.5000,42.8000) -- (175.3000,43.9000) -- (175.3000,44.9000) -- (175.3000,46.0000) -- (175.2000,47.1000) -- (175.2000,48.1000) -- (175.2000,49.2000) -- (175.2000,50.2000) -- (175.2000,51.3000) -- (175.2000,52.4000) -- (175.2000,53.4000) -- (175.2000,54.5000) -- (175.2000,55.5000) -- (175.2000,56.6000) -- (175.2000,57.6000) -- (175.2000,58.7000) -- (175.2000,59.7000) -- (175.2000,60.8000) -- (175.2000,61.9000) -- (175.2000,62.9000) -- (175.2000,64.0000) -- (175.2000,65.0000) -- (175.2000,66.1000) -- (175.2000,67.1000) -- (175.2000,68.2000) -- (175.2000,69.2000) -- (175.2000,70.3000) -- (175.2000,71.4000) -- (175.2000,72.4000) -- (175.2000,73.5000) -- (175.2000,74.5000) -- (175.2000,75.6000) -- (175.2000,76.6000) -- (175.3000,77.7000) -- (175.5000,78.7000) -- (175.7000,79.8000) -- (176.1000,80.8000) -- (176.5000,81.8000) -- (176.9000,82.8000) -- (177.4000,83.7000) -- (178.0000,84.6000) -- (178.7000,85.5000) -- (179.4000,86.4000) -- (180.2000,87.2000) -- (181.1000,87.9000) -- (182.0000,88.7000) -- (182.9000,89.3000) -- (183.9000,89.9000) -- (185.0000,90.5000) -- (186.0000,91.0000) -- (187.2000,91.4000) -- (188.3000,91.7000) -- (189.5000,92.0000) -- (190.7000,92.2000) -- (191.9000,92.4000) -- (193.1000,92.4000) -- (194.4000,92.4000) -- (195.6000,92.4000) -- (196.8000,92.2000) -- (198.0000,92.0000) -- (199.2000,91.7000) -- (200.4000,91.3000) -- (201.5000,90.9000) -- (202.6000,90.4000) -- (203.7000,89.8000) -- (204.7000,89.2000) -- (205.7000,88.5000) -- (206.6000,87.8000) -- (207.4000,87.0000) -- (208.2000,86.1000) -- (208.9000,85.3000) -- (209.6000,84.3000) -- (210.1000,83.4000) -- (210.6000,82.4000) -- (210.9000,81.4000) -- (211.2000,80.3000) -- (211.4000,79.3000) -- (211.5000,78.3000) -- (211.6000,77.3000) -- (211.7000,76.2000) -- (211.8000,75.2000) -- (211.8000,74.1000) -- (211.9000,73.1000) -- (211.9000,72.0000) -- (211.9000,71.0000) -- (211.9000,69.9000) -- (211.9000,68.9000) -- (211.9000,67.8000) -- (211.9000,66.8000) -- (211.9000,65.7000) -- (211.9000,64.7000) -- (211.9000,63.6000) -- (211.9000,62.5000) -- (211.9000,61.5000) -- (211.9000,60.4000) -- (211.9000,59.4000) -- (211.9000,58.3000) -- (211.9000,57.3000) -- (211.9000,56.2000) -- (211.9000,55.2000) -- (211.9000,54.1000) -- (211.9000,53.0000) -- (211.9000,52.0000) -- (211.9000,50.9000) -- (212.0000,49.9000) -- (212.1000,48.8000) -- (212.2000,47.8000) -- (212.4000,46.7000) -- (212.5000,45.7000) -- (212.4000,44.6000) -- (212.4000,43.6000) -- (212.2000,42.5000) -- (211.9000,41.5000) -- (211.6000,40.5000) -- (211.2000,39.5000) -- (210.7000,38.6000) -- (210.1000,37.6000) -- (209.4000,36.7000) -- (208.7000,35.9000) -- (207.8000,35.1000) -- (206.9000,34.4000) -- (206.0000,33.7000) -- (205.0000,33.1000) -- (203.9000,32.5000) -- (202.8000,32.1000) -- (201.6000,31.7000) -- (200.4000,31.4000) -- (199.2000,31.2000) -- (198.0000,31.1000) -- (196.8000,31.1000) -- (195.5000,31.1000) -- (194.3000,31.3000) -- (193.1000,31.5000) -- (191.9000,31.8000) -- (190.8000,32.2000) -- (189.7000,32.7000) -- (188.7000,33.2000) -- (187.7000,33.8000) -- (186.7000,34.5000) -- (185.8000,35.2000) -- (185.0000,36.0000) -- (184.3000,36.9000) -- (183.6000,37.7000) -- (183.0000,38.7000) -- (182.5000,39.6000) -- (182.1000,40.6000) -- (181.7000,41.6000) -- (181.6000,42.7000) -- (181.5000,43.7000) -- (181.4000,44.8000) -- (181.4000,45.9000) -- (181.4000,46.9000) -- (181.4000,48.0000) -- (181.4000,49.0000) -- (181.4000,50.1000) -- (181.4000,51.2000) -- (181.4000,52.2000) -- (181.4000,53.3000) -- (181.4000,54.3000) -- (181.4000,55.4000) -- (181.4000,56.4000) -- (181.4000,57.5000) -- (181.4000,58.5000) -- (181.4000,59.6000) -- (181.4000,60.7000) -- (181.4000,61.7000) -- (181.4000,62.8000) -- (181.4000,63.8000) -- (181.4000,64.9000) -- (181.4000,65.9000) -- (181.4000,67.0000) -- (181.4000,68.0000) -- (181.4000,69.1000) -- (181.4000,70.2000) -- (181.4000,71.2000) -- (181.4000,72.3000) -- (181.4000,73.3000) -- (181.4000,74.4000) -- (181.4000,75.4000) -- (181.4000,76.5000) -- (181.5000,77.5000) -- (181.7000,78.6000) -- (182.0000,79.6000) -- (182.3000,80.6000) -- (182.8000,81.6000) -- (183.3000,82.6000) -- (183.8000,83.5000) -- (184.4000,84.4000) -- (185.1000,85.3000) -- (185.9000,86.1000) -- (186.7000,86.9000) -- (187.6000,87.7000) -- (188.5000,88.4000) -- (189.4000,89.0000) -- (190.5000,89.6000) -- (191.5000,90.1000) -- (192.6000,90.6000) -- (193.8000,91.0000) -- (194.9000,91.3000) -- (196.1000,91.6000) -- (197.3000,91.7000) -- (198.5000,91.9000) -- (199.8000,91.9000) -- (201.0000,91.9000) -- (202.2000,91.8000) -- (203.5000,91.6000) -- (204.7000,91.3000) -- (205.8000,91.0000) -- (207.0000,90.6000) -- (208.1000,90.1000) -- (209.2000,89.6000) -- (210.2000,89.0000) -- (211.2000,88.4000) -- (212.2000,87.7000) -- (213.1000,86.9000) -- (213.9000,86.1000) -- (214.6000,85.2000) -- (215.3000,84.3000) -- (215.9000,83.4000) -- (216.5000,82.4000) -- (216.9000,81.4000) -- (217.2000,80.4000) -- (217.4000,79.4000) -- (217.6000,78.4000) -- (217.7000,77.3000) -- (217.8000,76.3000) -- (217.9000,75.3000) -- (217.9000,74.2000) -- (218.0000,73.2000) -- (218.0000,72.1000) -- (218.0000,71.1000) -- (218.1000,70.0000) -- (218.1000,69.0000) -- (218.1000,67.9000) -- (218.1000,66.9000) -- (218.1000,65.8000) -- (218.1000,64.7000) -- (218.1000,63.7000) -- (218.1000,62.6000) -- (218.1000,61.6000) -- (218.1000,60.5000) -- (218.1000,59.5000) -- (218.1000,58.4000) -- (218.1000,57.4000) -- (218.1000,56.3000) -- (218.1000,55.2000) -- (218.1000,54.2000) -- (218.1000,53.1000) -- (218.1000,52.1000) -- (218.1000,51.0000) -- (218.1000,50.0000) -- (218.1000,48.9000) -- (218.3000,47.9000) -- (218.4000,46.8000) -- (218.6000,45.8000) -- (218.6000,44.7000) -- (218.6000,43.7000) -- (218.4000,42.6000) -- (218.2000,41.6000) -- (217.9000,40.6000) -- (217.6000,39.6000) -- (217.1000,38.6000) -- (216.6000,37.6000) -- (216.0000,36.7000) -- (215.3000,35.9000) -- (214.5000,35.0000) -- (213.6000,34.3000) -- (212.7000,33.5000) -- (211.7000,32.9000) -- (210.7000,32.3000) -- (209.6000,31.8000) -- (208.5000,31.4000) -- (207.3000,31.0000) -- (206.1000,30.8000) -- (204.9000,30.6000) -- (203.6000,30.5000) -- (202.4000,30.5000) -- (201.2000,30.6000) -- (200.0000,30.8000) -- (198.8000,31.1000) -- (197.6000,31.4000) -- (196.5000,31.8000) -- (195.4000,32.3000) -- (194.4000,32.9000) -- (193.4000,33.5000) -- (192.5000,34.2000) -- (191.6000,35.0000) -- (190.9000,35.8000) -- (190.1000,36.6000) -- (189.5000,37.5000) -- (188.9000,38.5000) -- (188.5000,39.5000) -- (188.1000,40.5000) -- (187.8000,41.5000) -- (187.7000,42.6000) -- (187.6000,43.6000) -- (187.5000,44.7000) -- (187.5000,45.7000) -- (187.5000,46.8000) -- (187.5000,47.9000) -- (187.5000,48.9000) -- (187.5000,50.0000) -- (187.5000,51.0000) -- (187.5000,52.1000) -- (187.5000,53.1000) -- (187.5000,54.2000) -- (187.5000,55.2000) -- (187.5000,56.3000) -- (187.5000,57.4000) -- (187.5000,58.4000) -- (187.5000,59.5000) -- (187.5000,60.5000) -- (187.5000,61.6000) -- (187.5000,62.6000) -- (187.5000,63.7000) -- (187.5000,64.7000) -- (187.5000,65.8000) -- (187.5000,66.9000) -- (187.5000,67.9000) -- (187.5000,69.0000) -- (187.5000,70.0000) -- (187.5000,71.1000) -- (187.5000,72.1000) -- (187.5000,73.2000) -- (187.5000,74.2000) -- (187.5000,75.3000) -- (187.6000,76.4000) -- (187.7000,77.4000) -- (188.0000,78.4000) -- (188.3000,79.5000) -- (188.7000,80.5000) -- (189.1000,81.5000) -- (189.6000,82.4000) -- (190.2000,83.3000) -- (190.8000,84.2000) -- (191.6000,85.1000) -- (192.3000,85.9000) -- (193.2000,86.7000) -- (194.1000,87.4000) -- (195.0000,88.1000) -- (196.0000,88.7000) -- (197.0000,89.2000) -- (198.1000,89.7000) -- (199.2000,90.2000) -- (200.4000,90.6000) -- (201.5000,90.9000) -- (202.7000,91.1000) -- (204.0000,91.2000) -- (205.2000,91.3000) -- (206.4000,91.3000) -- (207.6000,91.3000) -- (208.9000,91.1000) -- (210.1000,90.9000) -- (211.3000,90.6000) -- (212.5000,90.3000) -- (213.6000,89.9000) -- (214.7000,89.4000) -- (215.8000,88.8000) -- (216.8000,88.2000) -- (217.8000,87.5000) -- (218.7000,86.8000) -- (219.5000,86.0000) -- (220.3000,85.2000) -- (221.1000,84.3000) -- (221.7000,83.4000) -- (222.3000,82.5000) -- (222.8000,81.5000) -- (223.1000,80.5000) -- (223.4000,79.4000) -- (223.6000,78.4000) -- (223.8000,77.4000) -- (223.9000,76.4000) -- (224.0000,75.3000) -- (224.1000,74.3000) -- (224.1000,73.2000) -- (224.1000,72.2000) -- (224.2000,71.1000) -- (224.2000,70.1000) -- (224.2000,69.0000) -- (224.2000,68.0000) -- (224.2000,66.9000) -- (224.2000,65.9000) -- (224.2000,64.8000) -- (224.2000,63.8000) -- (224.2000,62.7000) -- (224.2000,61.7000) -- (224.2000,60.6000) -- (224.2000,59.5000) -- (224.2000,58.5000) -- (224.2000,57.4000) -- (224.2000,56.4000) -- (224.2000,55.3000) -- (224.2000,54.3000) -- (224.2000,53.2000) -- (224.2000,52.2000) -- (224.2000,51.1000) -- (224.2000,50.0000) -- (224.2000,49.0000) -- (224.3000,47.9000) -- (224.5000,46.9000) -- (224.6000,45.8000) -- (224.7000,44.8000) -- (224.7000,43.7000) -- (224.7000,42.7000) -- (224.5000,41.6000) -- (224.3000,40.6000) -- (223.9000,39.6000) -- (223.5000,38.6000) -- (223.1000,37.6000) -- (222.5000,36.7000) -- (221.8000,35.8000) -- (221.1000,35.0000) -- (220.3000,34.2000) -- (219.4000,33.4000) -- (218.5000,32.7000) -- (217.5000,32.1000) -- (216.4000,31.5000) -- (215.3000,31.1000) -- (214.1000,30.7000) -- (213.0000,30.4000) -- (211.7000,30.1000) -- (210.5000,30.0000) -- (209.3000,29.9000) -- (208.1000,30.0000) -- (206.8000,30.1000) -- (205.6000,30.3000) -- (204.5000,30.6000) -- (203.3000,31.0000) -- (202.2000,31.5000) -- (201.2000,32.0000) -- (200.1000,32.6000) -- (199.2000,33.2000) -- (198.3000,34.0000) -- (197.5000,34.7000) -- (196.7000,35.6000) -- (196.0000,36.4000) -- (195.4000,37.4000) -- (194.9000,38.3000) -- (194.4000,39.3000) -- (194.1000,40.3000) -- (193.9000,41.4000) -- (193.8000,42.4000) -- (193.7000,43.5000) -- (193.7000,44.5000) -- (193.7000,45.6000) -- (193.7000,46.7000) -- (193.6000,47.7000) -- (193.6000,48.8000) -- (193.6000,49.8000) -- (193.6000,50.9000) -- (193.6000,51.9000) -- (193.6000,53.0000) -- (193.6000,54.1000) -- (193.6000,55.1000) -- (193.6000,56.2000) -- (193.6000,57.2000) -- (193.6000,58.3000) -- (193.6000,59.3000) -- (193.6000,60.4000) -- (193.6000,61.4000) -- (193.6000,62.5000) -- (193.6000,63.6000) -- (193.6000,64.6000) -- (193.6000,65.7000) -- (193.6000,66.7000) -- (193.6000,67.8000) -- (193.6000,68.8000) -- (193.6000,69.9000) -- (193.6000,70.9000) -- (193.6000,72.0000) -- (193.6000,73.1000) -- (193.6000,74.1000) -- (193.7000,75.2000) -- (193.8000,76.2000) -- (194.0000,77.3000) -- (194.2000,78.3000) -- (194.6000,79.3000) -- (195.0000,80.3000) -- (195.4000,81.3000) -- (196.0000,82.2000) -- (196.6000,83.1000) -- (197.3000,84.0000) -- (198.0000,84.9000) -- (198.8000,85.7000) -- (199.7000,86.4000) -- (200.6000,87.1000) -- (201.5000,87.8000) -- (202.5000,88.4000) -- (203.6000,88.9000) -- (204.7000,89.4000) -- (205.8000,89.8000) -- (207.0000,90.1000) -- (208.2000,90.4000) -- (209.4000,90.6000) -- (210.6000,90.7000) -- (211.8000,90.8000) -- (213.0000,90.8000) -- (214.3000,90.7000) -- (215.5000,90.5000) -- (216.7000,90.3000) -- (217.9000,90.0000) -- (219.0000,89.6000) -- (220.2000,89.1000) -- (221.3000,88.6000) -- (222.3000,88.0000) -- (223.3000,87.4000) -- (224.3000,86.7000) -- (225.2000,85.9000) -- (226.0000,85.2000) -- (226.8000,84.3000) -- (227.5000,83.4000) -- (228.1000,82.5000) -- (228.6000,81.5000) -- (229.1000,80.5000) -- (229.4000,79.5000) -- (229.7000,78.5000) -- (229.9000,77.5000) -- (230.0000,76.4000) -- (230.1000,75.4000) -- (230.2000,74.4000) -- (230.2000,73.3000) -- (230.3000,72.3000) -- (230.3000,71.2000) -- (230.3000,70.2000) -- (230.3000,69.1000) -- (230.3000,68.1000) -- (230.4000,67.0000) -- (230.4000,66.0000) -- (230.4000,64.9000) -- (230.4000,63.9000) -- (230.4000,62.8000) -- (230.4000,61.7000) -- (230.4000,60.7000) -- (230.4000,59.6000) -- (230.4000,58.6000) -- (230.4000,57.5000) -- (230.4000,56.5000) -- (230.4000,55.4000) -- (230.4000,54.4000) -- (230.4000,53.3000) -- (230.4000,52.2000) -- (230.4000,51.2000) -- (230.4000,50.1000) -- (230.4000,49.1000) -- (230.4000,48.0000) -- (230.5000,47.0000) -- (230.7000,45.9000) -- (230.8000,44.8000);



  \end{scope}
  \begin{scope}[cm={{1.15801,0.0,0.0,1.15801,(-736.40737,-53.12875)}},draw=blue,line cap=round,line join=round,line width=0.480pt]
    \path[draw] (148.5000,13.5000) -- (148.5000,102.5000) -- (246.5000,102.5000) -- (246.5000,13.5000) -- (148.5000,13.5000);



  \end{scope}
  \begin{scope}[cm={{0.84173,0.0,0.0,0.84173,(-601.60573,125.64086)}},fill=cffffff]
  \end{scope}
  \begin{scope}[cm={{1.00588,0.0,0.0,1.00588,(-513.7994,343.95446)}},draw=blue,line cap=rect,line join=bevel,line width=0.800pt]
    \begin{scope}[rotate around={-90.0:(-229.13106,-146.12949)},draw=blue,line cap=rect,line join=bevel,line width=0.800pt]
      \path[fill=blue] (0.0000,0.0000) node[above right] (text344-9) {\rotatebox{90}{y (m)}};



    \end{scope}
    \path[fill=blue] (-6.4779,-252.9639) node[above right] (text344-6-3) {x (m)};



    \path[fill=blue] (-76.6396,-237.2505) node[above right] (text344-6-3-1) {(a) Trajectories without re-planning};



    \path[fill=blue] (313 .5506,-237.2505) node[above right] (text344-6-3-1-1-9) {(c) State $(\alpha_0\dots\in\mathbf{q})$ evol. for {\hyperref[fig:trajs-I-static]{\color{red}I}}};



    \path[fill=blue] (100.2713,-237.2505) node[above right] (text344-6-3-1-6) {(b) Energies, detail of first instants, periods $T$};



  \end{scope}
  \begin{scope}[cm={{1.00588,0.0,0.0,1.00588,(-571.04217,77.13242)}},draw=blue,line cap=rect,line join=bevel,line width=0.800pt]
    \path[fill=blue] (0.0000,0.0000) node[above right] (text1768-1) {-150};



  \end{scope}
  \begin{scope}[cm={{1.00588,0.0,0.0,1.00588,(-543.83037,77.13242)}},draw=blue,line cap=rect,line join=bevel,line width=0.800pt]
    \path[fill=blue] (0.0000,0.0000) node[above right] (text1798-7) {-50};



  \end{scope}
  \begin{scope}[cm={{1.00588,0.0,0.0,1.00588,(-513.62717,77.13242)}},draw=blue,line cap=rect,line join=bevel,line width=0.800pt]
    \path[fill=blue] (0.0000,0.0000) node[above right] (text1828-9) {50};



  \end{scope}
  \begin{scope}[cm={{1.00588,0.0,0.0,1.00588,(-488.44517,77.13242)}},draw=blue,line cap=rect,line join=bevel,line width=0.800pt]
    \path[fill=blue] (0.0000,0.0000) node[above right] (text1858-6) {150};



  \end{scope}
  \begin{scope}[cm={{1.00588,0.0,0.0,1.00588,(-462.82756,77.13242)}},draw=blue,line cap=rect,line join=bevel,line width=0.800pt]
    \path[fill=blue] (0.0000,0.0000) node[above right] (text1858-6-6) {250};



  \end{scope}
  \begin{scope}[cm={{1.00588,0.0,0.0,1.00588,(-585.50071,67.88167)}},draw=blue,line cap=rect,line join=bevel,line width=0.800pt]
    \path[fill=blue] (0.0000,0.0000) node[above right] (text34-9-6) {-100};



  \end{scope}
  \begin{scope}[cm={{1.00588,0.0,0.0,1.00588,(-574.76608,42.75178)}},draw=blue,line cap=rect,line join=bevel,line width=0.800pt]
    \path[fill=blue] (0.0000,0.0000) node[above right] (text64-6-8) {0};



  \end{scope}
  \begin{scope}[cm={{1.00588,0.0,0.0,1.00588,(-582.81312,16.12239)}},draw=blue,line cap=rect,line join=bevel,line width=0.800pt]
    \path[fill=blue] (0.0000,0.0000) node[above right] (text94-7-9) {100};



  \end{scope}
  \begin{scope}[cm={{1.00588,0.0,0.0,1.00588,(-582.81312,-9.00701)}},draw=blue,line cap=rect,line join=bevel,line width=0.800pt]
    \path[fill=blue] (0.0000,0.0000) node[above right] (text124-9-0) {200};



  \end{scope}
  \begin{scope}[cm={{1.00588,0.0,0.0,1.00588,(-582.81312,-34.13641)}},draw=blue,line cap=rect,line join=bevel,line width=0.800pt]
    \path[fill=blue] (0.0000,0.0000) node[above right] (text154-3) {300};



  \end{scope}
  \begin{scope}[cm={{1.00769,0.0,0.0,1.00769,(-255.48702,89.55166)}},draw=blue,line cap=rect,line join=bevel,line width=0.800pt]
    \path[fill=blue] (0.0000,0.0000) node[above right] (text602-2) {\scriptsize Time (sec)};



  \end{scope}
  \begin{scope}[cm={{1.00588,0.0,0.0,1.00588,(-140.21331,89.54177)}},draw=blue,line cap=rect,line join=bevel,line width=0.800pt]
    \path[fill=blue] (0.0000,0.0000) node[above right] (text1412-73) {\scriptsize Time (min)};



  \end{scope}
  \begin{scope}[cm={{1.00588,0.0,0.0,1.00588,(-376.86267,89.54177)}},draw=blue,line cap=rect,line join=bevel,line width=0.800pt]
    \path[fill=blue] (0.0000,0.0000) node[above right] (text1412-73-1) {Time (min)};



  \end{scope}
  \begin{scope}[cm={{1.00588,0.0,0.0,1.00588,(-605.35515,-93.86128)}},draw=blue,line cap=rect,line join=bevel,line width=0.800pt]
    \path[fill=blue] (0.0000,0.0000) node[above right] (text154-9) {\Large{\label{fig:ener:static-I}\label{fig:trajs-I-static}I}};



  \end{scope}
  \begin{scope}[cm={{1.00588,0.0,0.0,1.00588,(-607.8469,20.77248)}},draw=blue,line cap=rect,line join=bevel,line width=0.800pt]
    \path[fill=blue] (2.4954,0.0000) node[above right] (text154-9-0) {\Large{\label{fig:ener:static-II}\label{fig:trajs-II-static}II}};



  \end{scope}
  \begin{scope}[cm={{0.0,-1.00588,1.00588,0.0,(154.28002,10.40301)}},draw=blue,line cap=rect,line join=bevel,line width=0.800pt]
    \path[fill=blue] (35.7896,-587.5454) node[above right] (text274-1) {\rotatebox{90}{Power (W)}};



  \end{scope}
  \begin{scope}[cm={{1.00588,0.0,0.0,1.00588,(-526.48136,104.32102)}},draw=blue,line cap=rect,line join=bevel,line width=0.800pt]
  \end{scope}
\end{scope}

\end{tikzpicture}


  {\color{blue}Fig.~6:~\color{blue}CPP with Zamboni-like motion using two boundary configurations. In \hyperref[fig:stat]{a} are the trajectories of the coverage--the highest in {\color{red}I} and the lowest in {\color{red}II}. In {\color{red}b} are the energy and the period evolutions for both {\color{red}I} and {\color{red}II} with different atmospheric conditions. In {\color{red}c} are the states that compose the energy model for {\color{red}I}.}
\end{formal}

Figure 7 contains cases i and ii that start from I and II but employ the planning-scheduling in the letter varying battery conditions. 

\begin{formal}
  \footnotesize
  
\definecolor{ca0a0a4}{RGB}{160,160,164}
\definecolor{cd9d9d9}{RGB}{235,235,235}
\definecolor{c00ff00}{RGB}{0,255,0}
\definecolor{cffffff}{RGB}{255,255,255}


\def \globalscale {1.000000}
\begin{tikzpicture}[y=0.80pt, x=0.80pt, yscale=-1.03*\globalscale, xscale=1.18*\globalscale, inner sep=0pt, outer sep=0pt]
\color{blue}
\begin{scope}[shift={(598.44106,-141.22792)},draw=blue,even odd rule,line cap=rect,line join=bevel,line width=0.800pt]
  \begin{scope}[cm={{0.84173,0.0,0.0,0.84173,(-603.85556,69.88496)}},draw=ca0a0a4,dash pattern=on 2.14pt off 2.14pt,line cap=round,line join=round,line width=0.356pt,miter limit=4.00]
    \path[draw,dash pattern=on 2.14pt off 2.14pt,line width=0.356pt,miter limit=4.00] (44.5000,181.5000) -- (179.5000,181.5000);



  \end{scope}
  \begin{scope}[cm={{0.84173,0.0,0.0,0.84173,(-603.85556,69.88496)}},draw=ca0a0a4,dash pattern=on 2.14pt off 2.14pt,line cap=round,line join=round,line width=0.356pt,miter limit=4.00]
    \path[draw,dash pattern=on 2.14pt off 2.14pt,line width=0.356pt,miter limit=4.00] (78.5000,210.5000) -- (78.5000,108.5000);



  \end{scope}
  \begin{scope}[cm={{0.84173,0.0,0.0,0.84173,(-603.85556,69.88496)}},draw=blue,line cap=round,line join=round,line width=0.480pt]
    \path[draw] (61.9000,118.7000) -- (61.9000,118.7000) -- (62.6000,120.9000) -- (63.3000,122.6000) -- (63.5000,124.1000) -- (62.7000,125.3000) -- (61.5000,126.5000) -- (60.3000,127.9000) -- (59.2000,129.5000) -- (58.3000,131.2000) -- (57.5000,133.0000) -- (57.0000,134.8000) -- (56.8000,136.7000) -- (56.7000,138.6000) -- (56.6000,140.4000) -- (56.6000,142.2000) -- (56.6000,144.1000) -- (56.6000,145.9000) -- (56.6000,147.7000) -- (56.6000,149.5000) -- (56.6000,151.4000) -- (56.6000,153.2000) -- (56.6000,155.1000) -- (56.6000,156.9000) -- (56.6000,158.8000) -- (56.6000,160.6000) -- (56.6000,162.5000) -- (56.6000,164.3000) -- (56.6000,166.2000) -- (56.6000,168.0000) -- (56.6000,169.9000) -- (56.6000,171.7000) -- (56.6000,173.6000) -- (56.6000,175.4000) -- (56.6000,177.2000) -- (56.6000,179.1000) -- (56.7000,181.0000) -- (57.1000,182.8000) -- (57.6000,184.7000) -- (58.3000,186.4000) -- (59.2000,188.2000) -- (60.2000,189.8000) -- (61.4000,191.3000) -- (62.7000,192.8000) -- (64.2000,194.1000) -- (65.7000,195.3000) -- (67.4000,196.4000) -- (69.1000,197.4000) -- (71.0000,198.2000) -- (72.9000,198.9000) -- (74.8000,199.5000) -- (76.8000,199.9000) -- (78.8000,200.2000) -- (80.8000,200.4000) -- (82.8000,200.4000) -- (84.8000,200.2000) -- (86.7000,200.0000) -- (88.7000,199.6000) -- (90.6000,199.0000) -- (92.5000,198.4000) -- (94.3000,197.5000) -- (96.1000,196.6000) -- (97.8000,195.5000) -- (99.3000,194.3000) -- (100.8000,193.0000) -- (102.1000,191.6000) -- (103.3000,190.0000) -- (104.2000,188.4000) -- (104.9000,186.7000) -- (105.3000,184.9000) -- (105.6000,183.0000) -- (105.7000,181.1000) -- (105.8000,179.2000) -- (105.8000,177.4000) -- (105.8000,175.5000) -- (105.8000,173.6000) -- (105.8000,171.7000) -- (105.8000,169.9000) -- (105.9000,168.0000) -- (105.9000,166.2000) -- (105.9000,164.3000) -- (105.9000,162.5000) -- (105.9000,160.6000) -- (105.9000,158.8000) -- (105.9000,157.0000) -- (105.9000,155.1000) -- (105.9000,153.3000) -- (105.9000,151.4000) -- (105.9000,149.6000) -- (105.9000,147.7000) -- (105.8000,145.9000) -- (106.1000,144.0000) -- (106.5000,142.1000) -- (106.8000,140.3000) -- (106.8000,138.6000) -- (106.7000,136.8000) -- (106.4000,135.0000) -- (105.9000,133.2000) -- (105.2000,131.5000) -- (104.3000,129.8000) -- (103.3000,128.3000) -- (102.1000,126.8000) -- (100.8000,125.4000) -- (99.3000,124.1000) -- (97.7000,122.9000) -- (96.0000,121.9000) -- (94.3000,121.0000) -- (92.4000,120.3000) -- (90.5000,119.7000) -- (88.5000,119.2000) -- (86.5000,118.9000) -- (84.5000,118.8000) -- (82.4000,118.8000) -- (80.4000,119.0000) -- (78.4000,119.3000) -- (76.4000,119.8000) -- (74.5000,120.4000) -- (72.6000,121.2000) -- (70.9000,122.1000) -- (69.2000,123.1000) -- (67.6000,124.3000) -- (66.2000,125.6000) -- (64.9000,127.0000) -- (63.8000,128.5000) -- (62.8000,130.1000) -- (62.0000,131.8000) -- (61.3000,133.6000) -- (60.9000,135.4000) -- (60.7000,137.2000) -- (60.6000,139.1000) -- (60.6000,140.9000) -- (60.6000,142.7000) -- (60.6000,144.5000) -- (60.6000,146.4000) -- (60.6000,148.2000) -- (60.6000,150.0000) -- (60.6000,151.9000) -- (60.6000,153.7000) -- (60.5000,155.6000) -- (60.5000,157.4000) -- (60.5000,159.3000) -- (60.5000,161.1000) -- (60.5000,163.0000) -- (60.5000,164.8000) -- (60.5000,166.7000) -- (60.5000,168.5000) -- (60.6000,170.4000) -- (60.6000,172.2000) -- (60.6000,174.1000) -- (60.6000,175.9000) -- (60.5000,177.7000) -- (60.6000,179.6000) -- (60.8000,181.5000) -- (61.2000,183.3000) -- (61.9000,185.1000) -- (62.7000,186.9000) -- (63.6000,188.6000) -- (64.7000,190.1000) -- (66.0000,191.6000) -- (67.4000,193.0000) -- (68.9000,194.3000) -- (70.6000,195.4000) -- (72.3000,196.4000) -- (74.1000,197.3000) -- (76.0000,198.1000) -- (77.9000,198.7000) -- (79.8000,199.2000) -- (81.8000,199.5000) -- (83.8000,199.7000) -- (85.8000,199.7000) -- (87.8000,199.6000) -- (89.8000,199.4000) -- (91.8000,199.0000) -- (93.7000,198.5000) -- (95.6000,197.8000) -- (97.4000,197.0000) -- (99.2000,196.1000) -- (100.8000,195.0000) -- (102.4000,193.8000) -- (103.9000,192.5000) -- (105.2000,191.1000) -- (106.4000,189.5000) -- (107.3000,187.9000) -- (108.0000,186.2000) -- (108.5000,184.4000) -- (108.7000,182.5000) -- (108.9000,180.6000) -- (109.0000,178.7000) -- (109.0000,176.9000) -- (109.0000,175.0000) -- (109.0000,173.1000) -- (109.0000,171.2000) -- (109.1000,169.4000) -- (109.1000,167.5000) -- (109.1000,165.7000) -- (109.1000,163.8000) -- (109.1000,162.0000) -- (109.1000,160.1000) -- (109.1000,158.3000) -- (109.1000,156.5000) -- (109.1000,154.6000) -- (109.1000,152.8000) -- (109.1000,150.9000) -- (109.1000,149.1000) -- (109.1000,147.2000) -- (109.1000,145.4000) -- (109.3000,143.5000) -- (109.7000,141.6000) -- (110.0000,139.8000) -- (110.0000,138.1000) -- (109.8000,136.3000) -- (109.5000,134.5000) -- (109.0000,132.7000) -- (108.3000,131.0000) -- (107.4000,129.4000) -- (106.3000,127.8000) -- (105.1000,126.3000) -- (103.7000,125.0000) -- (102.3000,123.7000) -- (100.6000,122.6000) -- (98.9000,121.6000) -- (97.1000,120.7000) -- (95.2000,120.0000) -- (93.3000,119.5000) -- (91.3000,119.1000) -- (89.3000,118.8000) -- (87.3000,118.7000) -- (85.2000,118.8000) -- (83.2000,119.0000) -- (81.2000,119.4000) -- (79.3000,119.9000) -- (77.4000,120.6000) -- (75.6000,121.5000) -- (73.9000,122.6000) -- (72.4000,123.8000) -- (70.9000,125.1000) -- (69.6000,126.5000) -- (68.5000,128.0000) -- (67.5000,129.6000) -- (66.7000,131.3000) -- (66.0000,133.0000) -- (65.7000,134.9000) -- (65.5000,136.7000) -- (65.4000,138.6000) -- (65.4000,140.4000) -- (65.4000,142.2000) -- (65.4000,144.0000) -- (65.3000,145.8000) -- (65.3000,147.7000) -- (65.3000,149.5000) -- (65.3000,151.4000) -- (65.3000,153.2000) -- (65.3000,155.1000) -- (65.3000,156.9000) -- (65.3000,158.8000) -- (65.3000,160.6000) -- (65.3000,162.4000) -- (65.3000,164.3000) -- (65.3000,166.1000) -- (65.3000,168.0000) -- (65.3000,169.8000) -- (65.3000,171.7000) -- (65.3000,173.5000) -- (65.3000,175.4000) -- (65.3000,177.2000) -- (65.4000,179.1000) -- (65.6000,181.0000) -- (66.0000,182.8000) -- (66.6000,184.6000) -- (67.4000,186.4000) -- (68.3000,188.1000) -- (69.4000,189.7000) -- (70.6000,191.2000) -- (71.9000,192.6000) -- (73.4000,193.9000) -- (75.0000,195.1000) -- (76.7000,196.2000) -- (78.5000,197.1000) -- (80.3000,198.0000) -- (82.2000,198.7000) -- (84.1000,199.2000) -- (86.1000,199.6000) -- (88.1000,199.9000) -- (90.1000,200.1000) -- (92.1000,200.2000) -- (94.1000,200.1000) -- (96.1000,199.8000) -- (98.0000,199.5000) -- (100.0000,199.0000) -- (101.9000,198.4000) -- (103.8000,197.7000) -- (105.6000,196.9000) -- (107.3000,195.9000) -- (109.0000,194.8000) -- (110.6000,193.6000) -- (112.0000,192.3000) -- (113.4000,190.9000) -- (114.6000,189.4000) -- (115.4000,187.7000) -- (115.9000,185.9000) -- (116.1000,184.1000) -- (116.2000,182.2000) -- (116.2000,180.3000) -- (116.3000,178.4000) -- (116.3000,176.5000) -- (116.3000,174.6000) -- (116.3000,172.8000) -- (116.3000,170.9000) -- (116.3000,169.1000) -- (116.3000,167.2000) -- (116.3000,165.4000) -- (116.3000,163.5000) -- (116.3000,161.7000) -- (116.3000,159.8000) -- (116.3000,158.0000) -- (116.3000,156.1000) -- (116.3000,154.3000) -- (116.3000,152.4000) -- (116.3000,150.6000) -- (116.3000,148.8000) -- (116.3000,146.9000) -- (116.3000,145.1000) -- (116.3000,143.2000) -- (116.5000,141.3000) -- (116.7000,139.5000) -- (116.7000,137.7000) -- (116.6000,135.9000) -- (116.3000,134.1000) -- (115.8000,132.3000) -- (115.1000,130.6000) -- (114.2000,129.0000) -- (113.2000,127.4000) -- (112.0000,125.9000) -- (110.6000,124.6000) -- (109.1000,123.3000) -- (107.5000,122.2000) -- (105.8000,121.2000) -- (104.0000,120.3000) -- (102.1000,119.6000) -- (100.2000,119.0000) -- (98.2000,118.6000) -- (96.2000,118.4000) -- (94.1000,118.3000) -- (92.1000,118.4000) -- (90.1000,118.6000) -- (88.1000,119.0000) -- (86.1000,119.5000) -- (84.2000,120.2000) -- (82.4000,121.1000) -- (80.7000,122.0000) -- (79.1000,123.2000) -- (77.6000,124.4000) -- (76.2000,125.8000) -- (75.0000,127.3000) -- (74.0000,128.8000) -- (73.1000,130.5000) -- (72.4000,132.2000) -- (72.0000,134.0000) -- (71.8000,135.9000) -- (71.7000,137.7000) -- (71.6000,139.5000) -- (71.6000,141.4000) -- (71.6000,143.2000) -- (71.6000,145.0000) -- (71.6000,146.8000) -- (71.6000,148.7000) -- (71.6000,150.5000) -- (71.6000,152.4000) -- (71.6000,154.2000) -- (71.6000,156.1000) -- (71.6000,157.9000) -- (71.6000,159.8000) -- (71.6000,161.6000) -- (71.6000,163.5000) -- (71.6000,165.3000) -- (71.6000,167.2000) -- (71.6000,169.0000) -- (71.6000,170.8000) -- (71.6000,172.7000) -- (71.6000,174.5000) -- (71.6000,176.4000) -- (71.6000,178.2000) -- (71.8000,180.1000) -- (72.2000,182.0000) -- (72.7000,183.8000) -- (73.5000,185.6000) -- (74.4000,187.3000) -- (75.5000,188.9000) -- (76.7000,190.4000) -- (78.0000,191.8000) -- (79.5000,193.1000) -- (81.1000,194.3000) -- (82.8000,195.4000) -- (84.6000,196.3000) -- (86.4000,197.1000) -- (88.3000,197.8000) -- (90.3000,198.3000) -- (92.2000,198.7000) -- (94.2000,199.0000) -- (96.2000,199.1000) -- (98.2000,199.1000) -- (100.2000,198.9000) -- (102.2000,198.6000) -- (104.2000,198.2000) -- (106.1000,197.6000) -- (107.9000,196.9000) -- (109.7000,196.0000) -- (111.5000,195.1000) -- (113.1000,194.0000) -- (114.7000,192.7000) -- (116.1000,191.4000) -- (117.4000,189.9000) -- (118.5000,188.4000) -- (119.4000,186.7000) -- (120.0000,185.0000) -- (120.4000,183.2000) -- (120.6000,181.3000) -- (120.7000,179.4000) -- (120.8000,177.5000) -- (120.8000,175.6000) -- (120.8000,173.8000) -- (120.8000,171.9000) -- (120.8000,170.0000) -- (120.9000,168.2000) -- (120.9000,166.3000) -- (120.9000,164.5000) -- (120.9000,162.6000) -- (120.9000,160.8000) -- (120.9000,158.9000) -- (120.9000,157.1000) -- (120.9000,155.2000) -- (120.9000,153.4000) -- (120.9000,151.5000) -- (120.9000,149.7000) -- (120.9000,147.8000) -- (120.9000,146.0000) -- (120.9000,144.1000) -- (121.2000,142.3000) -- (121.7000,140.4000) -- (121.9000,138.6000) -- (121.9000,136.9000) -- (121.8000,135.1000) -- (121.4000,133.3000) -- (120.9000,131.6000) -- (120.1000,129.9000) -- (119.2000,128.2000) -- (118.1000,126.7000) -- (116.9000,125.2000) -- (115.5000,123.9000) -- (114.0000,122.7000) -- (112.3000,121.6000) -- (110.6000,120.6000) -- (108.7000,119.8000) -- (106.8000,119.2000) -- (104.9000,118.7000) -- (102.8000,118.4000) -- (100.8000,118.2000) -- (98.8000,118.2000) -- (96.7000,118.4000) -- (94.7000,118.7000) -- (92.8000,119.2000) -- (90.9000,119.8000) -- (89.0000,120.6000) -- (87.3000,121.6000) -- (85.6000,122.6000) -- (84.1000,123.9000) -- (82.7000,125.2000) -- (81.5000,126.7000) -- (80.4000,128.2000) -- (79.5000,129.9000) -- (78.7000,131.6000) -- (78.3000,133.4000) -- (78.0000,135.2000) -- (77.9000,137.1000) -- (77.9000,138.9000) -- (77.8000,140.7000) -- (77.8000,142.5000) -- (77.8000,144.4000) -- (77.8000,146.2000) -- (77.8000,148.0000) -- (77.8000,149.9000) -- (77.8000,151.7000) -- (77.8000,153.6000) -- (77.8000,155.4000) -- (77.8000,157.3000) -- (77.8000,159.1000) -- (77.8000,161.0000) -- (77.8000,162.8000) -- (77.8000,164.6000) -- (77.8000,166.5000) -- (77.8000,168.3000) -- (77.8000,170.2000) -- (77.8000,172.0000) -- (77.8000,173.9000) -- (77.8000,175.7000) -- (77.8000,177.6000) -- (78.0000,179.5000) -- (78.4000,181.3000) -- (78.9000,183.1000) -- (79.7000,184.9000) -- (80.6000,186.6000) -- (81.7000,188.2000) -- (82.9000,189.7000) -- (84.2000,191.1000) -- (85.7000,192.5000) -- (87.3000,193.6000) -- (89.0000,194.7000) -- (90.8000,195.6000) -- (92.6000,196.4000) -- (94.5000,197.1000) -- (96.5000,197.6000) -- (98.5000,198.0000) -- (100.5000,198.2000) -- (102.5000,198.3000) -- (104.5000,198.3000) -- (106.5000,198.1000) -- (108.4000,197.7000) -- (110.4000,197.3000) -- (112.3000,196.7000) -- (114.1000,195.9000) -- (115.9000,195.0000) -- (117.6000,194.0000) -- (119.2000,192.9000) -- (120.8000,191.6000) -- (122.1000,190.2000) -- (123.4000,188.7000) -- (124.5000,187.1000) -- (125.3000,185.5000) -- (125.8000,183.7000) -- (126.1000,181.8000) -- (126.3000,180.0000) -- (126.4000,178.1000) -- (126.5000,176.2000) -- (126.5000,174.3000) -- (126.5000,172.4000) -- (126.5000,170.6000) -- (126.5000,168.7000) -- (126.5000,166.9000) -- (126.5000,165.0000) -- (126.5000,163.2000) -- (126.6000,161.3000) -- (126.6000,159.5000) -- (126.6000,157.6000) -- (126.6000,155.8000) -- (126.6000,153.9000) -- (126.6000,152.1000) -- (126.5000,150.2000) -- (126.5000,148.4000) -- (126.5000,146.5000) -- (126.5000,144.7000) -- (126.6000,142.8000) -- (127.1000,141.0000) -- (127.5000,139.2000) -- (127.6000,137.4000) -- (127.6000,135.6000) -- (127.3000,133.8000) -- (126.9000,132.0000) -- (126.3000,130.3000) -- (125.5000,128.6000) -- (124.5000,127.0000) -- (123.3000,125.5000) -- (122.0000,124.1000) -- (120.5000,122.8000) -- (118.9000,121.7000) -- (117.2000,120.7000) -- (115.4000,119.8000) -- (113.5000,119.1000) -- (111.6000,118.5000) -- (109.6000,118.1000) -- (107.6000,117.9000) -- (105.6000,117.8000) -- (103.5000,117.9000) -- (101.5000,118.2000) -- (99.5000,118.6000) -- (97.6000,119.2000) -- (95.7000,119.9000) -- (93.9000,120.8000) -- (92.2000,121.8000) -- (90.7000,123.0000) -- (89.2000,124.3000) -- (87.9000,125.7000) -- (86.8000,127.2000) -- (85.8000,128.8000) -- (85.0000,130.5000) -- (84.5000,132.3000) -- (84.1000,134.1000) -- (84.0000,136.0000) -- (83.9000,137.8000) -- (83.9000,139.6000) -- (83.9000,141.4000) -- (83.9000,143.3000) -- (83.8000,145.1000) -- (83.8000,146.9000) -- (83.8000,148.8000) -- (83.8000,150.6000) -- (83.8000,152.5000) -- (83.8000,154.3000) -- (83.8000,156.2000) -- (83.8000,158.0000) -- (83.8000,159.9000) -- (83.8000,161.7000) -- (83.8000,163.5000) -- (83.8000,165.4000) -- (83.8000,167.2000) -- (83.8000,169.1000) -- (83.8000,170.9000) -- (83.8000,172.8000) -- (83.8000,174.6000) -- (83.8000,176.5000) -- (83.9000,178.3000) -- (84.2000,180.2000) -- (84.8000,182.1000) -- (85.5000,183.8000) -- (86.3000,185.6000) -- (87.3000,187.2000) -- (88.5000,188.7000) -- (89.9000,190.2000) -- (91.3000,191.5000) -- (92.9000,192.7000) -- (94.5000,193.8000) -- (96.3000,194.8000) -- (98.1000,195.6000) -- (100.0000,196.3000) -- (102.0000,196.9000) -- (103.9000,197.3000) -- (105.9000,197.6000) -- (107.9000,197.7000) -- (109.9000,197.7000) -- (111.9000,197.5000) -- (113.9000,197.2000) -- (115.8000,196.8000) -- (117.7000,196.2000) -- (119.6000,195.5000) -- (121.4000,194.6000) -- (123.1000,193.6000) -- (124.8000,192.5000) -- (126.3000,191.3000) -- (127.7000,189.9000) -- (129.0000,188.4000) -- (130.1000,186.9000) -- (130.9000,185.2000) -- (131.5000,183.4000) -- (131.9000,181.6000) -- (132.1000,179.7000) -- (132.2000,177.8000) -- (132.3000,176.0000) -- (132.3000,174.1000) -- (132.3000,172.2000) -- (132.3000,170.3000) -- (132.3000,168.5000) -- (132.3000,166.6000) -- (132.3000,164.8000) -- (132.3000,162.9000) -- (132.3000,161.1000) -- (132.3000,159.2000) -- (132.3000,157.4000) -- (132.3000,155.5000) -- (132.3000,153.7000) -- (132.3000,151.8000) -- (132.3000,150.0000) -- (132.3000,148.1000) -- (132.3000,146.3000) -- (132.3000,144.4000) -- (132.4000,142.6000) -- (132.9000,140.7000) -- (133.2000,138.9000) -- (133.4000,137.1000) -- (133.4000,135.3000) -- (133.2000,133.5000) -- (132.8000,131.8000) -- (132.2000,130.0000) -- (131.3000,128.4000) -- (130.3000,126.8000) -- (129.1000,125.3000) -- (127.8000,123.9000) -- (126.3000,122.7000) -- (124.7000,121.6000) -- (122.9000,120.6000) -- (121.1000,119.8000) -- (119.2000,119.2000) -- (117.2000,118.7000) -- (115.2000,118.4000) -- (113.1000,118.3000) -- (111.1000,118.3000) -- (109.1000,118.6000) -- (107.1000,119.0000) -- (105.2000,119.6000) -- (103.3000,120.3000) -- (101.5000,121.2000) -- (99.9000,122.3000) -- (98.3000,123.5000) -- (96.9000,124.8000) -- (95.7000,126.3000) -- (94.6000,127.9000) -- (93.8000,129.5000) -- (93.1000,131.3000) -- (92.7000,133.1000) -- (92.6000,134.9000) -- (92.5000,136.8000) -- (92.4000,138.6000) -- (92.4000,140.4000) -- (92.4000,142.2000) -- (92.4000,144.1000) -- (92.4000,145.9000) -- (92.4000,147.7000) -- (92.4000,149.6000) -- (92.4000,151.4000) -- (92.4000,153.3000) -- (92.4000,155.1000) -- (92.4000,157.0000) -- (92.4000,158.8000) -- (92.4000,160.7000) -- (92.4000,162.5000) -- (92.4000,164.4000) -- (92.4000,166.2000) -- (92.4000,168.1000) -- (92.4000,169.9000) -- (92.4000,171.8000) -- (92.4000,173.6000) -- (92.4000,175.4000) -- (92.4000,177.3000) -- (92.7000,179.2000) -- (93.2000,181.0000) -- (93.8000,182.8000) -- (94.7000,184.5000) -- (95.7000,186.2000) -- (96.9000,187.8000) -- (98.2000,189.2000) -- (99.6000,190.6000) -- (101.1000,191.8000) -- (102.8000,192.9000) -- (104.5000,193.9000) -- (106.4000,194.8000) -- (108.2000,195.5000) -- (110.2000,196.0000) -- (112.1000,196.5000) -- (114.1000,196.8000) -- (116.1000,196.9000) -- (118.1000,196.9000) -- (120.1000,196.8000) -- (122.1000,196.5000) -- (124.1000,196.1000) -- (126.0000,195.5000) -- (127.8000,194.8000) -- (129.7000,194.0000) -- (131.4000,193.0000) -- (133.0000,191.9000) -- (134.6000,190.6000) -- (136.0000,189.3000) -- (137.3000,187.8000) -- (138.4000,186.3000) -- (139.3000,184.6000) -- (139.9000,182.8000) -- (140.3000,181.0000) -- (140.5000,179.2000) -- (140.6000,177.3000) -- (140.7000,175.4000) -- (140.7000,173.5000) -- (140.7000,171.6000) -- (140.7000,169.8000) -- (140.8000,167.9000) -- (140.8000,166.0000) -- (140.8000,164.2000) -- (140.8000,162.3000) -- (140.8000,160.5000) -- (140.8000,158.6000) -- (140.8000,156.8000) -- (140.8000,154.9000) -- (140.8000,153.1000) -- (140.8000,151.3000) -- (140.8000,149.4000) -- (140.8000,147.6000) -- (140.8000,145.7000) -- (140.8000,143.9000) -- (140.8000,142.0000) -- (141.3000,140.1000) -- (141.7000,138.3000) -- (142.0000,136.5000) -- (142.0000,134.8000) -- (141.8000,133.0000) -- (141.4000,131.2000) -- (140.8000,129.5000) -- (140.0000,127.8000) -- (139.0000,126.2000) -- (137.9000,124.7000) -- (136.5000,123.3000) -- (135.1000,122.0000) -- (133.5000,120.9000) -- (131.7000,119.9000) -- (129.9000,119.1000) -- (128.0000,118.4000) -- (126.1000,117.9000) -- (124.0000,117.6000) -- (122.0000,117.4000) -- (120.0000,117.4000) -- (117.9000,117.6000) -- (115.9000,118.0000) -- (114.0000,118.5000) -- (112.1000,119.2000) -- (110.3000,120.1000) -- (108.6000,121.1000) -- (107.0000,122.2000) -- (105.6000,123.5000) -- (104.3000,124.9000) -- (103.2000,126.5000) -- (102.2000,128.1000) -- (101.5000,129.8000) -- (101.0000,131.6000) -- (100.8000,133.5000) -- (100.6000,135.3000) -- (100.6000,137.1000) -- (100.6000,138.9000) -- (100.6000,140.8000) -- (100.5000,142.6000) -- (100.5000,144.4000) -- (100.5000,146.3000) -- (100.5000,148.1000) -- (100.5000,149.9000) -- (100.5000,151.8000) -- (100.5000,153.6000) -- (100.5000,155.5000) -- (100.5000,157.3000) -- (100.5000,159.2000) -- (100.5000,161.0000) -- (100.5000,162.9000) -- (100.5000,164.7000) -- (100.5000,166.6000) -- (100.5000,168.4000) -- (100.5000,170.3000) -- (100.5000,172.1000) -- (100.5000,174.0000) -- (100.5000,175.8000) -- (100.7000,177.7000) -- (101.1000,179.6000) -- (101.7000,181.4000) -- (102.5000,183.1000) -- (103.4000,184.8000) -- (104.5000,186.4000) -- (105.8000,187.9000) -- (107.1000,189.3000) -- (108.6000,190.6000) -- (110.2000,191.8000) -- (111.9000,192.8000) -- (113.7000,193.7000) -- (115.6000,194.5000) -- (117.5000,195.2000) -- (119.4000,195.7000) -- (121.4000,196.0000) -- (123.4000,196.2000) -- (125.4000,196.3000) -- (127.4000,196.2000) -- (129.4000,196.0000) -- (131.4000,195.7000) -- (133.3000,195.2000) -- (135.2000,194.6000) -- (137.1000,193.8000) -- (138.8000,192.9000) -- (140.5000,191.9000) -- (142.1000,190.7000) -- (143.6000,189.4000) -- (145.0000,188.0000) -- (146.2000,186.5000) -- (147.3000,184.9000) -- (148.0000,183.2000) -- (148.5000,181.4000) -- (148.8000,179.6000) -- (149.0000,177.7000) -- (149.1000,175.8000) -- (149.1000,173.9000) -- (149.2000,172.1000) -- (149.2000,170.2000) -- (149.2000,168.3000) -- (149.2000,166.5000) -- (149.2000,164.6000) -- (149.2000,162.8000) -- (149.2000,160.9000) -- (149.2000,159.1000) -- (149.2000,157.2000) -- (149.2000,155.4000) -- (149.2000,153.5000) -- (149.2000,151.7000) -- (149.2000,149.8000) -- (149.2000,148.0000) -- (149.2000,146.1000) -- (149.2000,144.3000) -- (149.2000,142.4000) -- (149.4000,140.6000) -- (149.9000,138.7000) -- (150.3000,136.9000) -- (150.4000,135.1000) -- (150.4000,133.4000) -- (150.1000,131.6000) -- (149.7000,129.8000) -- (149.0000,128.1000) -- (148.1000,126.5000) -- (147.1000,124.9000) -- (145.9000,123.4000) -- (144.5000,122.1000) -- (142.9000,120.8000) -- (141.3000,119.8000) -- (139.5000,118.8000) -- (137.7000,118.0000) -- (135.8000,117.4000) -- (133.8000,117.0000) -- (131.8000,116.7000) -- (129.7000,116.6000) -- (127.7000,116.7000) -- (125.7000,117.0000) -- (123.7000,117.4000) -- (121.7000,118.0000) -- (119.9000,118.7000) -- (118.1000,119.6000) -- (116.4000,120.7000) -- (114.9000,121.9000) -- (113.5000,123.2000) -- (112.3000,124.7000) -- (111.2000,126.3000) -- (110.3000,127.9000) -- (109.7000,129.7000) -- (109.3000,131.5000) -- (109.1000,133.3000) -- (109.0000,135.2000) -- (109.0000,137.0000) -- (109.0000,138.8000) -- (108.9000,140.6000) -- (108.9000,142.5000) -- (108.9000,144.3000) -- (108.9000,146.1000) -- (108.9000,148.0000) -- (108.9000,149.8000) -- (108.9000,151.7000) -- (108.9000,153.5000) -- (108.9000,155.4000) -- (108.9000,157.2000) -- (108.9000,159.1000) -- (108.9000,160.9000) -- (108.9000,162.8000) -- (108.9000,164.6000) -- (108.9000,166.5000) -- (108.9000,168.3000) -- (108.9000,170.1000) -- (108.9000,172.0000) -- (108.9000,173.8000) -- (109.0000,175.7000) -- (109.2000,177.6000) -- (109.7000,179.4000) -- (110.4000,181.2000) -- (111.2000,183.0000) -- (112.2000,184.6000) -- (113.3000,186.2000) -- (114.6000,187.7000) -- (116.0000,189.0000) -- (117.5000,190.3000) -- (119.2000,191.4000) -- (120.9000,192.4000) -- (122.7000,193.3000) -- (124.6000,194.0000) -- (126.5000,194.6000) -- (128.5000,195.1000) -- (130.5000,195.4000) -- (132.5000,195.6000) -- (134.5000,195.6000) -- (136.5000,195.5000) -- (138.5000,195.2000) -- (140.4000,194.8000) -- (142.3000,194.3000) -- (144.2000,193.6000) -- (146.0000,192.8000) -- (147.8000,191.9000) -- (149.5000,190.8000) -- (151.0000,189.6000) -- (152.5000,188.3000) -- (153.8000,186.9000) -- (155.0000,185.3000) -- (156.0000,183.7000) -- (156.6000,182.0000) -- (157.1000,180.2000) -- (157.3000,178.3000) -- (157.5000,176.4000) -- (157.6000,174.5000) -- (157.6000,172.7000) -- (157.6000,170.8000) -- (157.6000,168.9000) -- (157.6000,167.0000) -- (157.6000,165.2000) -- (157.6000,163.3000) -- (157.7000,161.5000) -- (157.7000,159.6000) -- (157.7000,157.8000) -- (157.7000,155.9000) -- (157.7000,154.1000) -- (157.7000,152.2000) -- (157.7000,150.4000) -- (157.7000,148.6000) -- (157.7000,146.7000) -- (157.7000,144.9000) -- (157.6000,143.0000) -- (157.6000,141.2000) -- (158.0000,139.3000) -- (158.5000,137.4000) -- (158.8000,135.6000);



  \end{scope}
  \path[fill=cd9d9d9,dash pattern=on 1.12pt off 1.12pt,even odd rule,line cap=round,line width=0.281pt,miter limit=4.00,rounded corners=0.0000cm] (-617.3423,257.0234) rectangle (-74.5817,379.8000);



  \begin{scope}[cm={{1.04177,0.0,0.0,1.04177,(-342.63646,42.1635)}},draw=ca0a0a4,dash pattern=on 1.73pt off 1.73pt,line cap=round,line join=round,line width=0.288pt,miter limit=4.00]
    \path[draw,dash pattern=on 1.73pt off 1.73pt,line width=0.288pt,miter limit=4.00] (56.5000,232.5000) -- (251.5000,232.5000);



  \end{scope}
  \begin{scope}[draw=blue,line cap=rect,line join=bevel,line width=0.800pt]
  \end{scope}
  \begin{scope}[scale=1.006,draw=blue,line cap=rect,line join=bevel,line width=0.800pt]
  \end{scope}
  \begin{scope}[scale=1.006,draw=blue,line cap=rect,line join=bevel,line width=0.800pt]
  \end{scope}
  \begin{scope}[cm={{1.00588,0.0,0.0,1.00588,(39.2294,93.5471)}},draw=blue,line cap=rect,line join=bevel,line width=0.800pt]
  \end{scope}
  \begin{scope}[cm={{1.00588,0.0,0.0,1.00588,(39.2294,93.5471)}},draw=blue,line cap=rect,line join=bevel,line width=0.800pt]
  \end{scope}
  \begin{scope}[cm={{1.00588,0.0,0.0,1.00588,(39.2294,93.5471)}},draw=blue,line cap=rect,line join=bevel,line width=0.800pt]
  \end{scope}
  \begin{scope}[cm={{1.00588,0.0,0.0,1.00588,(39.2294,93.5471)}},draw=blue,line cap=rect,line join=bevel,line width=0.800pt]
  \end{scope}
  \begin{scope}[cm={{1.00588,0.0,0.0,1.00588,(39.2294,93.5471)}},draw=blue,line cap=rect,line join=bevel,line width=0.800pt]
  \end{scope}
  \begin{scope}[cm={{1.00588,0.0,0.0,1.00588,(39.2294,93.5471)}},draw=blue,line cap=rect,line join=bevel,line width=0.800pt]
  \end{scope}
  \begin{scope}[scale=1.006,draw=blue,line cap=rect,line join=bevel,line width=0.800pt]
  \end{scope}
  \begin{scope}[scale=1.006,draw=blue,line cap=rect,line join=bevel,line width=0.800pt]
  \end{scope}
  \begin{scope}[cm={{1.00588,0.0,0.0,1.00588,(39.2294,68.4)}},draw=blue,line cap=rect,line join=bevel,line width=0.800pt]
  \end{scope}
  \begin{scope}[cm={{1.00588,0.0,0.0,1.00588,(39.2294,68.4)}},draw=blue,line cap=rect,line join=bevel,line width=0.800pt]
  \end{scope}
  \begin{scope}[cm={{1.00588,0.0,0.0,1.00588,(39.2294,68.4)}},draw=blue,line cap=rect,line join=bevel,line width=0.800pt]
  \end{scope}
  \begin{scope}[cm={{1.00588,0.0,0.0,1.00588,(39.2294,68.4)}},draw=blue,line cap=rect,line join=bevel,line width=0.800pt]
  \end{scope}
  \begin{scope}[cm={{1.00588,0.0,0.0,1.00588,(39.2294,68.4)}},draw=blue,line cap=rect,line join=bevel,line width=0.800pt]
  \end{scope}
  \begin{scope}[cm={{1.00588,0.0,0.0,1.00588,(39.2294,68.4)}},draw=blue,line cap=rect,line join=bevel,line width=0.800pt]
  \end{scope}
  \begin{scope}[scale=1.006,draw=blue,line cap=rect,line join=bevel,line width=0.800pt]
  \end{scope}
  \begin{scope}[scale=1.006,draw=blue,line cap=rect,line join=bevel,line width=0.800pt]
  \end{scope}
  \begin{scope}[cm={{1.00588,0.0,0.0,1.00588,(39.2294,43.2529)}},draw=blue,line cap=rect,line join=bevel,line width=0.800pt]
  \end{scope}
  \begin{scope}[cm={{1.00588,0.0,0.0,1.00588,(39.2294,43.2529)}},draw=blue,line cap=rect,line join=bevel,line width=0.800pt]
  \end{scope}
  \begin{scope}[cm={{1.00588,0.0,0.0,1.00588,(39.2294,43.2529)}},draw=blue,line cap=rect,line join=bevel,line width=0.800pt]
  \end{scope}
  \begin{scope}[cm={{1.00588,0.0,0.0,1.00588,(39.2294,43.2529)}},draw=blue,line cap=rect,line join=bevel,line width=0.800pt]
  \end{scope}
  \begin{scope}[cm={{1.00588,0.0,0.0,1.00588,(39.2294,43.2529)}},draw=blue,line cap=rect,line join=bevel,line width=0.800pt]
  \end{scope}
  \begin{scope}[cm={{1.00588,0.0,0.0,1.00588,(39.2294,43.2529)}},draw=blue,line cap=rect,line join=bevel,line width=0.800pt]
  \end{scope}
  \begin{scope}[scale=1.006,draw=blue,line cap=rect,line join=bevel,line width=0.800pt]
  \end{scope}
  \begin{scope}[scale=1.006,draw=blue,line cap=rect,line join=bevel,line width=0.800pt]
  \end{scope}
  \begin{scope}[cm={{1.00588,0.0,0.0,1.00588,(53.3118,110.647)}},draw=blue,line cap=rect,line join=bevel,line width=0.800pt]
  \end{scope}
  \begin{scope}[cm={{1.00588,0.0,0.0,1.00588,(53.3118,110.647)}},draw=blue,line cap=rect,line join=bevel,line width=0.800pt]
  \end{scope}
  \begin{scope}[cm={{1.00588,0.0,0.0,1.00588,(53.3118,110.647)}},draw=blue,line cap=rect,line join=bevel,line width=0.800pt]
  \end{scope}
  \begin{scope}[cm={{1.00588,0.0,0.0,1.00588,(53.3118,110.647)}},draw=blue,line cap=rect,line join=bevel,line width=0.800pt]
  \end{scope}
  \begin{scope}[cm={{1.00588,0.0,0.0,1.00588,(53.3118,110.647)}},draw=blue,line cap=rect,line join=bevel,line width=0.800pt]
  \end{scope}
  \begin{scope}[cm={{1.00588,0.0,0.0,1.00588,(53.3118,110.647)}},draw=blue,line cap=rect,line join=bevel,line width=0.800pt]
  \end{scope}
  \begin{scope}[scale=1.006,draw=blue,line cap=rect,line join=bevel,line width=0.800pt]
  \end{scope}
  \begin{scope}[scale=1.006,draw=blue,line cap=rect,line join=bevel,line width=0.800pt]
  \end{scope}
  \begin{scope}[cm={{1.00588,0.0,0.0,1.00588,(79.4647,110.647)}},draw=blue,line cap=rect,line join=bevel,line width=0.800pt]
  \end{scope}
  \begin{scope}[cm={{1.00588,0.0,0.0,1.00588,(79.4647,110.647)}},draw=blue,line cap=rect,line join=bevel,line width=0.800pt]
  \end{scope}
  \begin{scope}[cm={{1.00588,0.0,0.0,1.00588,(79.4647,110.647)}},draw=blue,line cap=rect,line join=bevel,line width=0.800pt]
  \end{scope}
  \begin{scope}[cm={{1.00588,0.0,0.0,1.00588,(79.4647,110.647)}},draw=blue,line cap=rect,line join=bevel,line width=0.800pt]
  \end{scope}
  \begin{scope}[cm={{1.00588,0.0,0.0,1.00588,(79.4647,110.647)}},draw=blue,line cap=rect,line join=bevel,line width=0.800pt]
  \end{scope}
  \begin{scope}[cm={{1.00588,0.0,0.0,1.00588,(79.4647,110.647)}},draw=blue,line cap=rect,line join=bevel,line width=0.800pt]
  \end{scope}
  \begin{scope}[scale=1.006,draw=blue,line cap=rect,line join=bevel,line width=0.800pt]
  \end{scope}
  \begin{scope}[scale=1.006,draw=blue,line cap=rect,line join=bevel,line width=0.800pt]
  \end{scope}
  \begin{scope}[cm={{1.00588,0.0,0.0,1.00588,(105.618,110.647)}},draw=blue,line cap=rect,line join=bevel,line width=0.800pt]
  \end{scope}
  \begin{scope}[cm={{1.00588,0.0,0.0,1.00588,(105.618,110.647)}},draw=blue,line cap=rect,line join=bevel,line width=0.800pt]
  \end{scope}
  \begin{scope}[cm={{1.00588,0.0,0.0,1.00588,(105.618,110.647)}},draw=blue,line cap=rect,line join=bevel,line width=0.800pt]
  \end{scope}
  \begin{scope}[cm={{1.00588,0.0,0.0,1.00588,(105.618,110.647)}},draw=blue,line cap=rect,line join=bevel,line width=0.800pt]
  \end{scope}
  \begin{scope}[cm={{1.00588,0.0,0.0,1.00588,(105.618,110.647)}},draw=blue,line cap=rect,line join=bevel,line width=0.800pt]
  \end{scope}
  \begin{scope}[cm={{1.00588,0.0,0.0,1.00588,(105.618,110.647)}},draw=blue,line cap=rect,line join=bevel,line width=0.800pt]
  \end{scope}
  \begin{scope}[scale=1.006,draw=blue,line cap=rect,line join=bevel,line width=0.800pt]
  \end{scope}
  \begin{scope}[scale=1.006,draw=blue,line cap=rect,line join=bevel,line width=0.800pt]
  \end{scope}
  \begin{scope}[cm={{1.00588,0.0,0.0,1.00588,(132.274,110.647)}},draw=blue,line cap=rect,line join=bevel,line width=0.800pt]
  \end{scope}
  \begin{scope}[cm={{1.00588,0.0,0.0,1.00588,(132.274,110.647)}},draw=blue,line cap=rect,line join=bevel,line width=0.800pt]
  \end{scope}
  \begin{scope}[cm={{1.00588,0.0,0.0,1.00588,(132.274,110.647)}},draw=blue,line cap=rect,line join=bevel,line width=0.800pt]
  \end{scope}
  \begin{scope}[cm={{1.00588,0.0,0.0,1.00588,(132.274,110.647)}},draw=blue,line cap=rect,line join=bevel,line width=0.800pt]
  \end{scope}
  \begin{scope}[cm={{1.00588,0.0,0.0,1.00588,(132.274,110.647)}},draw=blue,line cap=rect,line join=bevel,line width=0.800pt]
  \end{scope}
  \begin{scope}[cm={{1.00588,0.0,0.0,1.00588,(132.274,110.647)}},draw=blue,line cap=rect,line join=bevel,line width=0.800pt]
  \end{scope}
  \begin{scope}[scale=1.006,draw=blue,line cap=rect,line join=bevel,line width=0.800pt]
  \end{scope}
  \begin{scope}[scale=1.006,draw=blue,line cap=rect,line join=bevel,line width=0.800pt]
  \end{scope}
  \begin{scope}[scale=1.006,draw=blue,line cap=rect,line join=bevel,line width=0.800pt]
  \end{scope}
  \begin{scope}[scale=1.006,draw=blue,line cap=rect,line join=bevel,line width=0.800pt]
  \end{scope}
  \begin{scope}[scale=1.006,draw=blue,line cap=rect,line join=bevel,line width=0.800pt]
  \end{scope}
  \begin{scope}[scale=1.006,draw=blue,line cap=rect,line join=bevel,line width=0.800pt]
  \end{scope}
  \begin{scope}[cm={{1.00588,0.0,0.0,1.00588,(128.753,29.1706)}},draw=blue,line cap=rect,line join=bevel,line width=0.800pt]
  \end{scope}
  \begin{scope}[cm={{1.00588,0.0,0.0,1.00588,(128.753,29.1706)}},draw=blue,line cap=rect,line join=bevel,line width=0.800pt]
  \end{scope}
  \begin{scope}[cm={{1.00588,0.0,0.0,1.00588,(128.753,29.1706)}},draw=blue,line cap=rect,line join=bevel,line width=0.800pt]
  \end{scope}
  \begin{scope}[cm={{1.00588,0.0,0.0,1.00588,(128.753,29.1706)}},draw=blue,line cap=rect,line join=bevel,line width=0.800pt]
  \end{scope}
  \begin{scope}[cm={{1.00588,0.0,0.0,1.00588,(128.753,29.1706)}},draw=blue,line cap=rect,line join=bevel,line width=0.800pt]
  \end{scope}
  \begin{scope}[cm={{1.00588,0.0,0.0,1.00588,(128.753,29.1706)}},draw=blue,line cap=rect,line join=bevel,line width=0.800pt]
  \end{scope}
  \begin{scope}[cm={{0.0,-1.00588,1.00588,0.0,(29.1706,189.106)}},draw=blue,line cap=rect,line join=bevel,line width=0.800pt]
  \end{scope}
  \begin{scope}[cm={{0.0,-1.00588,1.00588,0.0,(29.1706,189.106)}},draw=blue,line cap=rect,line join=bevel,line width=0.800pt]
  \end{scope}
  \begin{scope}[cm={{0.0,-1.00588,1.00588,0.0,(29.1706,189.106)}},draw=blue,line cap=rect,line join=bevel,line width=0.800pt]
  \end{scope}
  \begin{scope}[cm={{0.0,-1.00588,1.00588,0.0,(29.1706,189.106)}},draw=blue,line cap=rect,line join=bevel,line width=0.800pt]
  \end{scope}
  \begin{scope}[cm={{0.0,-1.00588,1.00588,0.0,(29.1706,189.106)}},draw=blue,line cap=rect,line join=bevel,line width=0.800pt]
  \end{scope}
  \begin{scope}[cm={{0.0,-1.00588,1.00588,0.0,(281.91864,312.23446)}},draw=blue,line cap=rect,line join=bevel,line width=0.800pt]
    \path[fill=blue] (32.7896,-587.5454) node[above right] (text274) {\rotatebox{90}{Power (W)}};



  \end{scope}
  \begin{scope}[cm={{0.0,-1.00588,1.00588,0.0,(29.1706,189.106)}},draw=blue,line cap=rect,line join=bevel,line width=0.800pt]
  \end{scope}
  \begin{scope}[cm={{1.00588,0.0,0.0,1.00588,(62.3647,28.1647)}},draw=blue,line cap=rect,line join=bevel,line width=0.800pt]
  \end{scope}
  \begin{scope}[cm={{1.00588,0.0,0.0,1.00588,(62.3647,28.1647)}},draw=blue,line cap=rect,line join=bevel,line width=0.800pt]
  \end{scope}
  \begin{scope}[cm={{1.00588,0.0,0.0,1.00588,(62.3647,28.1647)}},draw=blue,line cap=rect,line join=bevel,line width=0.800pt]
  \end{scope}
  \begin{scope}[cm={{1.00588,0.0,0.0,1.00588,(62.3647,28.1647)}},draw=blue,line cap=rect,line join=bevel,line width=0.800pt]
  \end{scope}
  \begin{scope}[cm={{1.00588,0.0,0.0,1.00588,(62.3647,28.1647)}},draw=blue,line cap=rect,line join=bevel,line width=0.800pt]
  \end{scope}
  \begin{scope}[cm={{1.00588,0.0,0.0,1.00588,(62.3647,28.1647)}},draw=blue,line cap=rect,line join=bevel,line width=0.800pt]
  \end{scope}
  \begin{scope}[scale=1.006,draw=blue,line cap=rect,line join=bevel,line width=0.800pt]
  \end{scope}
  \begin{scope}[scale=1.006,draw=blue,line cap=rect,line join=bevel,line width=0.800pt]
  \end{scope}
  \begin{scope}[scale=1.006,draw=blue,line cap=rect,line join=bevel,line width=0.800pt]
  \end{scope}
  \begin{scope}[scale=1.006,draw=blue,line cap=rect,line join=bevel,line width=0.800pt]
  \end{scope}
  \begin{scope}[scale=1.006,draw=blue,line cap=rect,line join=bevel,line width=0.800pt]
  \end{scope}
  \begin{scope}[scale=1.006,draw=blue,line cap=rect,line join=bevel,line width=0.800pt]
  \end{scope}
  \begin{scope}[cm={{1.00588,0.0,0.0,1.00588,(60.3529,36.2118)}},draw=blue,line cap=rect,line join=bevel,line width=0.800pt]
  \end{scope}
  \begin{scope}[cm={{1.00588,0.0,0.0,1.00588,(60.3529,36.2118)}},draw=blue,line cap=rect,line join=bevel,line width=0.800pt]
  \end{scope}
  \begin{scope}[cm={{1.00588,0.0,0.0,1.00588,(60.3529,36.2118)}},draw=blue,line cap=rect,line join=bevel,line width=0.800pt]
  \end{scope}
  \begin{scope}[cm={{1.00588,0.0,0.0,1.00588,(60.3529,36.2118)}},draw=blue,line cap=rect,line join=bevel,line width=0.800pt]
  \end{scope}
  \begin{scope}[cm={{1.00588,0.0,0.0,1.00588,(60.3529,36.2118)}},draw=blue,line cap=rect,line join=bevel,line width=0.800pt]
  \end{scope}
  \begin{scope}[cm={{1.00588,0.0,0.0,1.00588,(60.3529,36.2118)}},draw=blue,line cap=rect,line join=bevel,line width=0.800pt]
  \end{scope}
  \begin{scope}[scale=1.006,draw=blue,line cap=rect,line join=bevel,line width=0.800pt]
  \end{scope}
  \begin{scope}[scale=1.006,draw=blue,line cap=rect,line join=bevel,line width=0.800pt]
  \end{scope}
  \begin{scope}[scale=1.006,draw=blue,line cap=rect,line join=bevel,line width=0.800pt]
  \end{scope}
  \begin{scope}[scale=1.006,draw=blue,line cap=rect,line join=bevel,line width=0.800pt]
  \end{scope}
  \begin{scope}[scale=1.006,draw=blue,line cap=rect,line join=bevel,line width=0.800pt]
  \end{scope}
  \begin{scope}[scale=1.006,draw=blue,line cap=rect,line join=bevel,line width=0.800pt]
  \end{scope}
  \begin{scope}[scale=1.006,draw=blue,line cap=rect,line join=bevel,line width=0.800pt]
  \end{scope}
  \begin{scope}[scale=1.006,draw=blue,line cap=rect,line join=bevel,line width=0.800pt]
  \end{scope}
  \begin{scope}[cm={{1.00588,0.0,0.0,1.00588,(148.871,93.5471)}},draw=blue,line cap=rect,line join=bevel,line width=0.800pt]
  \end{scope}
  \begin{scope}[cm={{1.00588,0.0,0.0,1.00588,(148.871,93.5471)}},draw=blue,line cap=rect,line join=bevel,line width=0.800pt]
  \end{scope}
  \begin{scope}[cm={{1.00588,0.0,0.0,1.00588,(148.871,93.5471)}},draw=blue,line cap=rect,line join=bevel,line width=0.800pt]
  \end{scope}
  \begin{scope}[cm={{1.00588,0.0,0.0,1.00588,(148.871,93.5471)}},draw=blue,line cap=rect,line join=bevel,line width=0.800pt]
  \end{scope}
  \begin{scope}[cm={{1.00588,0.0,0.0,1.00588,(148.871,93.5471)}},draw=blue,line cap=rect,line join=bevel,line width=0.800pt]
  \end{scope}
  \begin{scope}[cm={{1.00588,0.0,0.0,1.00588,(148.871,93.5471)}},draw=blue,line cap=rect,line join=bevel,line width=0.800pt]
  \end{scope}
  \begin{scope}[scale=1.006,draw=blue,line cap=rect,line join=bevel,line width=0.800pt]
  \end{scope}
  \begin{scope}[scale=1.006,draw=blue,line cap=rect,line join=bevel,line width=0.800pt]
  \end{scope}
  \begin{scope}[cm={{1.00588,0.0,0.0,1.00588,(149.876,68.4)}},draw=blue,line cap=rect,line join=bevel,line width=0.800pt]
  \end{scope}
  \begin{scope}[cm={{1.00588,0.0,0.0,1.00588,(149.876,68.4)}},draw=blue,line cap=rect,line join=bevel,line width=0.800pt]
  \end{scope}
  \begin{scope}[cm={{1.00588,0.0,0.0,1.00588,(149.876,68.4)}},draw=blue,line cap=rect,line join=bevel,line width=0.800pt]
  \end{scope}
  \begin{scope}[cm={{1.00588,0.0,0.0,1.00588,(149.876,68.4)}},draw=blue,line cap=rect,line join=bevel,line width=0.800pt]
  \end{scope}
  \begin{scope}[cm={{1.00588,0.0,0.0,1.00588,(149.876,68.4)}},draw=blue,line cap=rect,line join=bevel,line width=0.800pt]
  \end{scope}
  \begin{scope}[cm={{1.00588,0.0,0.0,1.00588,(149.876,68.4)}},draw=blue,line cap=rect,line join=bevel,line width=0.800pt]
  \end{scope}
  \begin{scope}[scale=1.006,draw=blue,line cap=rect,line join=bevel,line width=0.800pt]
  \end{scope}
  \begin{scope}[scale=1.006,draw=blue,line cap=rect,line join=bevel,line width=0.800pt]
  \end{scope}
  \begin{scope}[cm={{1.00588,0.0,0.0,1.00588,(149.876,43.2529)}},draw=blue,line cap=rect,line join=bevel,line width=0.800pt]
  \end{scope}
  \begin{scope}[cm={{1.00588,0.0,0.0,1.00588,(149.876,43.2529)}},draw=blue,line cap=rect,line join=bevel,line width=0.800pt]
  \end{scope}
  \begin{scope}[cm={{1.00588,0.0,0.0,1.00588,(149.876,43.2529)}},draw=blue,line cap=rect,line join=bevel,line width=0.800pt]
  \end{scope}
  \begin{scope}[cm={{1.00588,0.0,0.0,1.00588,(149.876,43.2529)}},draw=blue,line cap=rect,line join=bevel,line width=0.800pt]
  \end{scope}
  \begin{scope}[cm={{1.00588,0.0,0.0,1.00588,(149.876,43.2529)}},draw=blue,line cap=rect,line join=bevel,line width=0.800pt]
  \end{scope}
  \begin{scope}[cm={{1.00588,0.0,0.0,1.00588,(149.876,43.2529)}},draw=blue,line cap=rect,line join=bevel,line width=0.800pt]
  \end{scope}
  \begin{scope}[scale=1.006,draw=blue,line cap=rect,line join=bevel,line width=0.800pt]
  \end{scope}
  \begin{scope}[scale=1.006,draw=blue,line cap=rect,line join=bevel,line width=0.800pt]
  \end{scope}
  \begin{scope}[cm={{1.00588,0.0,0.0,1.00588,(162.953,110.647)}},draw=blue,line cap=rect,line join=bevel,line width=0.800pt]
  \end{scope}
  \begin{scope}[cm={{1.00588,0.0,0.0,1.00588,(162.953,110.647)}},draw=blue,line cap=rect,line join=bevel,line width=0.800pt]
  \end{scope}
  \begin{scope}[cm={{1.00588,0.0,0.0,1.00588,(162.953,110.647)}},draw=blue,line cap=rect,line join=bevel,line width=0.800pt]
  \end{scope}
  \begin{scope}[cm={{1.00588,0.0,0.0,1.00588,(162.953,110.647)}},draw=blue,line cap=rect,line join=bevel,line width=0.800pt]
  \end{scope}
  \begin{scope}[cm={{1.00588,0.0,0.0,1.00588,(162.953,110.647)}},draw=blue,line cap=rect,line join=bevel,line width=0.800pt]
  \end{scope}
  \begin{scope}[cm={{1.00588,0.0,0.0,1.00588,(162.953,110.647)}},draw=blue,line cap=rect,line join=bevel,line width=0.800pt]
  \end{scope}
  \begin{scope}[scale=1.006,draw=blue,line cap=rect,line join=bevel,line width=0.800pt]
  \end{scope}
  \begin{scope}[scale=1.006,draw=blue,line cap=rect,line join=bevel,line width=0.800pt]
  \end{scope}
  \begin{scope}[cm={{1.00588,0.0,0.0,1.00588,(189.106,110.647)}},draw=blue,line cap=rect,line join=bevel,line width=0.800pt]
  \end{scope}
  \begin{scope}[cm={{1.00588,0.0,0.0,1.00588,(189.106,110.647)}},draw=blue,line cap=rect,line join=bevel,line width=0.800pt]
  \end{scope}
  \begin{scope}[cm={{1.00588,0.0,0.0,1.00588,(189.106,110.647)}},draw=blue,line cap=rect,line join=bevel,line width=0.800pt]
  \end{scope}
  \begin{scope}[cm={{1.00588,0.0,0.0,1.00588,(189.106,110.647)}},draw=blue,line cap=rect,line join=bevel,line width=0.800pt]
  \end{scope}
  \begin{scope}[cm={{1.00588,0.0,0.0,1.00588,(189.106,110.647)}},draw=blue,line cap=rect,line join=bevel,line width=0.800pt]
  \end{scope}
  \begin{scope}[cm={{1.00588,0.0,0.0,1.00588,(189.106,110.647)}},draw=blue,line cap=rect,line join=bevel,line width=0.800pt]
  \end{scope}
  \begin{scope}[scale=1.006,draw=blue,line cap=rect,line join=bevel,line width=0.800pt]
  \end{scope}
  \begin{scope}[scale=1.006,draw=blue,line cap=rect,line join=bevel,line width=0.800pt]
  \end{scope}
  \begin{scope}[cm={{1.00588,0.0,0.0,1.00588,(215.259,110.647)}},draw=blue,line cap=rect,line join=bevel,line width=0.800pt]
  \end{scope}
  \begin{scope}[cm={{1.00588,0.0,0.0,1.00588,(215.259,110.647)}},draw=blue,line cap=rect,line join=bevel,line width=0.800pt]
  \end{scope}
  \begin{scope}[cm={{1.00588,0.0,0.0,1.00588,(215.259,110.647)}},draw=blue,line cap=rect,line join=bevel,line width=0.800pt]
  \end{scope}
  \begin{scope}[cm={{1.00588,0.0,0.0,1.00588,(215.259,110.647)}},draw=blue,line cap=rect,line join=bevel,line width=0.800pt]
  \end{scope}
  \begin{scope}[cm={{1.00588,0.0,0.0,1.00588,(215.259,110.647)}},draw=blue,line cap=rect,line join=bevel,line width=0.800pt]
  \end{scope}
  \begin{scope}[cm={{1.00588,0.0,0.0,1.00588,(215.259,110.647)}},draw=blue,line cap=rect,line join=bevel,line width=0.800pt]
  \end{scope}
  \begin{scope}[scale=1.006,draw=blue,line cap=rect,line join=bevel,line width=0.800pt]
  \end{scope}
  \begin{scope}[scale=1.006,draw=blue,line cap=rect,line join=bevel,line width=0.800pt]
  \end{scope}
  \begin{scope}[cm={{1.00588,0.0,0.0,1.00588,(241.915,110.647)}},draw=blue,line cap=rect,line join=bevel,line width=0.800pt]
  \end{scope}
  \begin{scope}[cm={{1.00588,0.0,0.0,1.00588,(241.915,110.647)}},draw=blue,line cap=rect,line join=bevel,line width=0.800pt]
  \end{scope}
  \begin{scope}[cm={{1.00588,0.0,0.0,1.00588,(241.915,110.647)}},draw=blue,line cap=rect,line join=bevel,line width=0.800pt]
  \end{scope}
  \begin{scope}[cm={{1.00588,0.0,0.0,1.00588,(241.915,110.647)}},draw=blue,line cap=rect,line join=bevel,line width=0.800pt]
  \end{scope}
  \begin{scope}[cm={{1.00588,0.0,0.0,1.00588,(241.915,110.647)}},draw=blue,line cap=rect,line join=bevel,line width=0.800pt]
  \end{scope}
  \begin{scope}[cm={{1.00588,0.0,0.0,1.00588,(241.915,110.647)}},draw=blue,line cap=rect,line join=bevel,line width=0.800pt]
  \end{scope}
  \begin{scope}[scale=1.006,draw=blue,line cap=rect,line join=bevel,line width=0.800pt]
  \end{scope}
  \begin{scope}[scale=1.006,draw=blue,line cap=rect,line join=bevel,line width=0.800pt]
  \end{scope}
  \begin{scope}[scale=1.006,draw=blue,line cap=rect,line join=bevel,line width=0.800pt]
  \end{scope}
  \begin{scope}[scale=1.006,draw=blue,line cap=rect,line join=bevel,line width=0.800pt]
  \end{scope}
  \begin{scope}[scale=1.006,draw=blue,line cap=rect,line join=bevel,line width=0.800pt]
  \end{scope}
  \begin{scope}[scale=1.006,draw=blue,line cap=rect,line join=bevel,line width=0.800pt]
  \end{scope}
  \begin{scope}[cm={{1.00588,0.0,0.0,1.00588,(235.376,29.1706)}},draw=blue,line cap=rect,line join=bevel,line width=0.800pt]
  \end{scope}
  \begin{scope}[cm={{1.00588,0.0,0.0,1.00588,(235.376,29.1706)}},draw=blue,line cap=rect,line join=bevel,line width=0.800pt]
  \end{scope}
  \begin{scope}[cm={{1.00588,0.0,0.0,1.00588,(235.376,29.1706)}},draw=blue,line cap=rect,line join=bevel,line width=0.800pt]
  \end{scope}
  \begin{scope}[cm={{1.00588,0.0,0.0,1.00588,(235.376,29.1706)}},draw=blue,line cap=rect,line join=bevel,line width=0.800pt]
  \end{scope}
  \begin{scope}[cm={{1.00588,0.0,0.0,1.00588,(235.376,29.1706)}},draw=blue,line cap=rect,line join=bevel,line width=0.800pt]
  \end{scope}
  \begin{scope}[cm={{1.00588,0.0,0.0,1.00588,(235.376,29.1706)}},draw=blue,line cap=rect,line join=bevel,line width=0.800pt]
  \end{scope}
  \begin{scope}[scale=1.006,draw=blue,line cap=rect,line join=bevel,line width=0.800pt]
  \end{scope}
  \begin{scope}[scale=1.006,draw=blue,line cap=rect,line join=bevel,line width=0.800pt]
  \end{scope}
  \begin{scope}[scale=1.006,draw=blue,line cap=rect,line join=bevel,line width=0.800pt]
  \end{scope}
  \begin{scope}[scale=1.006,draw=blue,line cap=rect,line join=bevel,line width=0.800pt]
  \end{scope}
  \begin{scope}[scale=1.006,draw=blue,line cap=rect,line join=bevel,line width=0.800pt]
  \end{scope}
  \begin{scope}[scale=1.006,draw=blue,line cap=rect,line join=bevel,line width=0.800pt]
  \end{scope}
  \begin{scope}[scale=1.006,draw=blue,line cap=rect,line join=bevel,line width=0.800pt]
  \end{scope}
  \begin{scope}[scale=1.006,draw=blue,line cap=rect,line join=bevel,line width=0.800pt]
  \end{scope}
  \begin{scope}[scale=1.006,draw=blue,line cap=rect,line join=bevel,line width=0.800pt]
  \end{scope}
  \begin{scope}[cm={{1.04177,0.0,0.0,1.04177,(-342.44493,31.82443)}},draw=ca0a0a4,dash pattern=on 1.73pt off 1.73pt,line cap=round,line join=round,line width=0.288pt,miter limit=4.00]
    \path[draw,dash pattern=on 1.73pt off 1.73pt,line width=0.288pt,miter limit=4.00] (56.5000,194.5000) -- (251.5000,194.5000);



  \end{scope}
  \begin{scope}[cm={{1.04177,0.0,0.0,1.04177,(-342.44493,31.82443)}},draw=blue,line cap=round,line join=round,line width=0.480pt]
    \path[cm={{1.1115,0.0,0.0,1.0,(-6.26603,0.0)}},draw] (56.5000,194.5000) -- (59.5000,194.5000);



    \path[cm={{1.1115,0.0,0.0,1.0,(-28.19312,0.0)}},draw] (251.5000,194.5000) -- (248.5000,194.5000);



  \end{scope}
  \begin{scope}[scale=1.006,draw=blue,line cap=rect,line join=bevel,line width=0.800pt]
  \end{scope}
  \begin{scope}[cm={{1.00588,0.0,0.0,1.00588,(39.2294,199.165)}},draw=blue,line cap=rect,line join=bevel,line width=0.800pt]
  \end{scope}
  \begin{scope}[cm={{1.00588,0.0,0.0,1.00588,(39.2294,199.165)}},draw=blue,line cap=rect,line join=bevel,line width=0.800pt]
  \end{scope}
  \begin{scope}[cm={{1.00588,0.0,0.0,1.00588,(39.2294,199.165)}},draw=blue,line cap=rect,line join=bevel,line width=0.800pt]
  \end{scope}
  \begin{scope}[cm={{1.00588,0.0,0.0,1.00588,(39.2294,199.165)}},draw=blue,line cap=rect,line join=bevel,line width=0.800pt]
  \end{scope}
  \begin{scope}[cm={{1.00588,0.0,0.0,1.00588,(39.2294,199.165)}},draw=blue,line cap=rect,line join=bevel,line width=0.800pt]
  \end{scope}
  \begin{scope}[cm={{1.00588,0.0,0.0,1.00588,(-298.32402,235.165)}},draw=blue,line cap=rect,line join=bevel,line width=0.800pt]
    \path[fill=blue] (0.0000,0.0000) node[above right] (text658) {27};



  \end{scope}
  \begin{scope}[cm={{1.00588,0.0,0.0,1.00588,(39.2294,199.165)}},draw=blue,line cap=rect,line join=bevel,line width=0.800pt]
  \end{scope}
  \begin{scope}[scale=1.006,draw=blue,line cap=rect,line join=bevel,line width=0.800pt]
  \end{scope}
  \begin{scope}[cm={{1.04177,0.0,0.0,1.04177,(-342.44493,31.82443)}},draw=ca0a0a4,dash pattern=on 1.73pt off 1.73pt,line cap=round,line join=round,line width=0.288pt,miter limit=4.00]
    \path[draw,dash pattern=on 1.73pt off 1.73pt,line width=0.288pt,miter limit=4.00] (56.5000,171.5000) -- (251.5000,171.5000);



  \end{scope}
  \begin{scope}[cm={{1.04177,0.0,0.0,1.04177,(-342.44493,31.82443)}},draw=blue,line cap=round,line join=round,line width=0.480pt]
    \path[cm={{1.1115,0.0,0.0,1.0,(-6.26603,0.0)}},draw] (56.5000,171.5000) -- (59.5000,171.5000);



    \path[cm={{1.1115,0.0,0.0,1.0,(-28.19312,0.0)}},draw] (251.5000,171.5000) -- (248.5000,171.5000);



  \end{scope}
  \begin{scope}[scale=1.006,draw=blue,line cap=rect,line join=bevel,line width=0.800pt]
  \end{scope}
  \begin{scope}[cm={{1.00588,0.0,0.0,1.00588,(40.2353,177.035)}},draw=blue,line cap=rect,line join=bevel,line width=0.800pt]
  \end{scope}
  \begin{scope}[cm={{1.00588,0.0,0.0,1.00588,(40.2353,177.035)}},draw=blue,line cap=rect,line join=bevel,line width=0.800pt]
  \end{scope}
  \begin{scope}[cm={{1.00588,0.0,0.0,1.00588,(40.2353,177.035)}},draw=blue,line cap=rect,line join=bevel,line width=0.800pt]
  \end{scope}
  \begin{scope}[cm={{1.00588,0.0,0.0,1.00588,(40.2353,177.035)}},draw=blue,line cap=rect,line join=bevel,line width=0.800pt]
  \end{scope}
  \begin{scope}[cm={{1.00588,0.0,0.0,1.00588,(40.2353,177.035)}},draw=blue,line cap=rect,line join=bevel,line width=0.800pt]
  \end{scope}
  \begin{scope}[cm={{1.00588,0.0,0.0,1.00588,(-298.41069,213.035)}},draw=blue,line cap=rect,line join=bevel,line width=0.800pt]
    \path[fill=blue] (0.5262,0.0000) node[above right] (text688) {31};



  \end{scope}
  \begin{scope}[cm={{1.00588,0.0,0.0,1.00588,(40.2353,177.035)}},draw=blue,line cap=rect,line join=bevel,line width=0.800pt]
  \end{scope}
  \begin{scope}[scale=1.006,draw=blue,line cap=rect,line join=bevel,line width=0.800pt]
  \end{scope}
  \begin{scope}[cm={{1.04177,0.0,0.0,1.04177,(-342.44493,31.82443)}},draw=ca0a0a4,dash pattern=on 1.73pt off 1.73pt,line cap=round,line join=round,line width=0.288pt,miter limit=4.00]
    \path[draw,dash pattern=on 1.73pt off 1.73pt,line width=0.288pt,miter limit=4.00] (56.5000,149.5000) -- (251.5000,149.5000);



  \end{scope}
  \begin{scope}[cm={{1.04177,0.0,0.0,1.04177,(-342.44493,31.82443)}},draw=blue,line cap=round,line join=round,line width=0.480pt]
    \path[cm={{1.1115,0.0,0.0,1.0,(-6.26603,0.0)}},draw] (56.5000,149.5000) -- (59.5000,149.5000);



    \path[cm={{1.1115,0.0,0.0,1.0,(-28.19312,0.0)}},draw] (251.5000,149.5000) -- (248.5000,149.5000);



  \end{scope}
  \begin{scope}[scale=1.006,draw=blue,line cap=rect,line join=bevel,line width=0.800pt]
  \end{scope}
  \begin{scope}[cm={{1.00588,0.0,0.0,1.00588,(40.2353,153.9)}},draw=blue,line cap=rect,line join=bevel,line width=0.800pt]
  \end{scope}
  \begin{scope}[cm={{1.00588,0.0,0.0,1.00588,(40.2353,153.9)}},draw=blue,line cap=rect,line join=bevel,line width=0.800pt]
  \end{scope}
  \begin{scope}[cm={{1.00588,0.0,0.0,1.00588,(40.2353,153.9)}},draw=blue,line cap=rect,line join=bevel,line width=0.800pt]
  \end{scope}
  \begin{scope}[cm={{1.00588,0.0,0.0,1.00588,(40.2353,153.9)}},draw=blue,line cap=rect,line join=bevel,line width=0.800pt]
  \end{scope}
  \begin{scope}[cm={{1.00588,0.0,0.0,1.00588,(40.2353,153.9)}},draw=blue,line cap=rect,line join=bevel,line width=0.800pt]
  \end{scope}
  \begin{scope}[cm={{1.00588,0.0,0.0,1.00588,(-298.23551,191.4)}},draw=blue,line cap=rect,line join=bevel,line width=0.800pt]
    \path[fill=blue] (0.0000,0.0000) node[above right] (text718) {35};



  \end{scope}
  \begin{scope}[cm={{1.00588,0.0,0.0,1.00588,(40.2353,153.9)}},draw=blue,line cap=rect,line join=bevel,line width=0.800pt]
  \end{scope}
  \begin{scope}[scale=1.006,draw=blue,line cap=rect,line join=bevel,line width=0.800pt]
  \end{scope}
  \begin{scope}[cm={{1.04164,0.0,0.0,1.04164,(-342.30377,31.82628)}},draw=ca0a0a4,dash pattern=on 1.73pt off 1.73pt,line cap=round,line join=round,line width=0.288pt,miter limit=4.00]
    \path[draw,dash pattern=on 1.73pt off 1.73pt,line width=0.288pt,miter limit=4.00] (56.5000,126.5000) -- (195.5000,126.5000);



    \path[draw,dash pattern=on 1.73pt off 1.73pt,line width=0.288pt,miter limit=4.00] (246.5000,126.5000) -- (251.5000,126.5000);



  \end{scope}
  \begin{scope}[cm={{1.04177,0.0,0.0,1.04177,(-342.44493,31.82443)}},draw=blue,line cap=round,line join=round,line width=0.480pt]
    \path[cm={{1.1115,0.0,0.0,1.0,(-6.26603,0.0)}},draw] (56.5000,126.5000) -- (59.5000,126.5000);



    \path[cm={{1.1115,0.0,0.0,1.0,(-28.19312,0.0)}},draw] (251.5000,126.5000) -- (248.5000,126.5000);



  \end{scope}
  \begin{scope}[scale=1.006,draw=blue,line cap=rect,line join=bevel,line width=0.800pt]
  \end{scope}
  \begin{scope}[cm={{1.00588,0.0,0.0,1.00588,(39.2294,131.771)}},draw=blue,line cap=rect,line join=bevel,line width=0.800pt]
  \end{scope}
  \begin{scope}[cm={{1.00588,0.0,0.0,1.00588,(39.2294,131.771)}},draw=blue,line cap=rect,line join=bevel,line width=0.800pt]
  \end{scope}
  \begin{scope}[cm={{1.00588,0.0,0.0,1.00588,(39.2294,131.771)}},draw=blue,line cap=rect,line join=bevel,line width=0.800pt]
  \end{scope}
  \begin{scope}[cm={{1.00588,0.0,0.0,1.00588,(39.2294,131.771)}},draw=blue,line cap=rect,line join=bevel,line width=0.800pt]
  \end{scope}
  \begin{scope}[cm={{1.00588,0.0,0.0,1.00588,(39.2294,131.771)}},draw=blue,line cap=rect,line join=bevel,line width=0.800pt]
  \end{scope}
  \begin{scope}[cm={{1.00588,0.0,0.0,1.00588,(-298.4045,166.271)}},draw=blue,line cap=rect,line join=bevel,line width=0.800pt]
    \path[fill=blue] (0.0000,0.0000) node[above right] (text750) {39};



  \end{scope}
  \begin{scope}[cm={{1.00588,0.0,0.0,1.00588,(39.2294,131.771)}},draw=blue,line cap=rect,line join=bevel,line width=0.800pt]
  \end{scope}
  \begin{scope}[scale=1.006,draw=blue,line cap=rect,line join=bevel,line width=0.800pt]
  \end{scope}
  \begin{scope}[cm={{1.04177,0.0,0.0,1.04177,(-342.44493,31.82443)}},draw=ca0a0a4,dash pattern=on 0.40pt off 0.80pt,line cap=round,line join=round,line width=0.400pt]
    \path[draw] (56.5000,200.5000) -- (56.5000,115.5000);



  \end{scope}
  \begin{scope}[cm={{1.04177,0.0,0.0,1.04177,(-342.44493,31.82443)}},draw=blue,line cap=round,line join=round,line width=0.480pt]
    \path[draw] (56.5000,200.5000) -- (56.5000,198.5000);



    \path[draw] (56.5000,115.5000) -- (56.5000,118.5000);



  \end{scope}
  \begin{scope}[scale=1.006,draw=blue,line cap=rect,line join=bevel,line width=0.800pt]
  \end{scope}
  \begin{scope}[cm={{1.00588,0.0,0.0,1.00588,(53.3118,217.271)}},draw=blue,line cap=rect,line join=bevel,line width=0.800pt]
  \end{scope}
  \begin{scope}[cm={{1.00588,0.0,0.0,1.00588,(53.3118,217.271)}},draw=blue,line cap=rect,line join=bevel,line width=0.800pt]
  \end{scope}
  \begin{scope}[cm={{1.00588,0.0,0.0,1.00588,(53.3118,217.271)}},draw=blue,line cap=rect,line join=bevel,line width=0.800pt]
  \end{scope}
  \begin{scope}[cm={{1.00588,0.0,0.0,1.00588,(53.3118,217.271)}},draw=blue,line cap=rect,line join=bevel,line width=0.800pt]
  \end{scope}
  \begin{scope}[cm={{1.00588,0.0,0.0,1.00588,(53.3118,217.271)}},draw=blue,line cap=rect,line join=bevel,line width=0.800pt]
  \end{scope}
  \begin{scope}[cm={{1.00588,0.0,0.0,1.00588,(-282.68826,252.63148)}},draw=blue,line cap=rect,line join=bevel,line width=0.800pt]
    \path[fill=blue] (0.0000,0.0000) node[above right] (text780) {0};



  \end{scope}
  \begin{scope}[cm={{1.00588,0.0,0.0,1.00588,(53.3118,217.271)}},draw=blue,line cap=rect,line join=bevel,line width=0.800pt]
  \end{scope}
  \begin{scope}[scale=1.006,draw=blue,line cap=rect,line join=bevel,line width=0.800pt]
  \end{scope}
  \begin{scope}[cm={{1.04177,0.0,0.0,1.04177,(-342.44493,31.82443)}},draw=ca0a0a4,dash pattern=on 1.73pt off 1.73pt,line cap=round,line join=round,line width=0.288pt,miter limit=4.00]
    \path[draw,dash pattern=on 1.73pt off 1.73pt,line width=0.288pt,miter limit=4.00] (85.5000,200.5000) -- (85.5000,115.5000);



  \end{scope}
  \begin{scope}[cm={{1.04177,0.0,0.0,1.04177,(-342.44493,31.82443)}},draw=blue,line cap=round,line join=round,line width=0.480pt]
    \path[cm={{1.0,0.0,0.0,1.53899,(0.0,-108.26833)}},draw] (85.5000,200.5000) -- (85.5000,198.5000);



    \path[cm={{1.0,0.0,0.0,1.1115,(0.0,-12.84424)}},draw] (85.5000,115.5000) -- (85.5000,118.5000);



  \end{scope}
  \begin{scope}[scale=1.006,draw=blue,line cap=rect,line join=bevel,line width=0.800pt]
  \end{scope}
  \begin{scope}[cm={{1.00588,0.0,0.0,1.00588,(82.4824,217.271)}},draw=blue,line cap=rect,line join=bevel,line width=0.800pt]
  \end{scope}
  \begin{scope}[cm={{1.00588,0.0,0.0,1.00588,(82.4824,217.271)}},draw=blue,line cap=rect,line join=bevel,line width=0.800pt]
  \end{scope}
  \begin{scope}[cm={{1.00588,0.0,0.0,1.00588,(82.4824,217.271)}},draw=blue,line cap=rect,line join=bevel,line width=0.800pt]
  \end{scope}
  \begin{scope}[cm={{1.00588,0.0,0.0,1.00588,(82.4824,217.271)}},draw=blue,line cap=rect,line join=bevel,line width=0.800pt]
  \end{scope}
  \begin{scope}[cm={{1.00588,0.0,0.0,1.00588,(82.4824,217.271)}},draw=blue,line cap=rect,line join=bevel,line width=0.800pt]
  \end{scope}
  \begin{scope}[cm={{1.00588,0.0,0.0,1.00588,(-253.51767,252.74414)}},draw=blue,line cap=rect,line join=bevel,line width=0.800pt]
    \path[fill=blue] (0.0000,0.0000) node[above right] (text810) {1};



  \end{scope}
  \begin{scope}[cm={{1.00588,0.0,0.0,1.00588,(82.4824,217.271)}},draw=blue,line cap=rect,line join=bevel,line width=0.800pt]
  \end{scope}
  \begin{scope}[scale=1.006,draw=blue,line cap=rect,line join=bevel,line width=0.800pt]
  \end{scope}
  \begin{scope}[cm={{1.04177,0.0,0.0,1.04177,(-342.44493,31.82443)}},draw=ca0a0a4,dash pattern=on 1.73pt off 1.73pt,line cap=round,line join=round,line width=0.288pt,miter limit=4.00]
    \path[draw,dash pattern=on 1.73pt off 1.73pt,line width=0.288pt,miter limit=4.00] (114.5000,200.5000) -- (114.5000,115.5000);



  \end{scope}
  \begin{scope}[cm={{1.04177,0.0,0.0,1.04177,(-342.44493,31.82443)}},draw=blue,line cap=round,line join=round,line width=0.480pt]
    \path[cm={{1.0,0.0,0.0,1.53899,(0.0,-108.26833)}},draw] (114.5000,200.5000) -- (114.5000,198.5000);



    \path[cm={{1.0,0.0,0.0,1.1115,(0.0,-12.84424)}},draw] (114.5000,115.5000) -- (114.5000,118.5000);



  \end{scope}
  \begin{scope}[scale=1.006,draw=blue,line cap=rect,line join=bevel,line width=0.800pt]
  \end{scope}
  \begin{scope}[cm={{1.00588,0.0,0.0,1.00588,(111.653,217.271)}},draw=blue,line cap=rect,line join=bevel,line width=0.800pt]
  \end{scope}
  \begin{scope}[cm={{1.00588,0.0,0.0,1.00588,(111.653,217.271)}},draw=blue,line cap=rect,line join=bevel,line width=0.800pt]
  \end{scope}
  \begin{scope}[cm={{1.00588,0.0,0.0,1.00588,(111.653,217.271)}},draw=blue,line cap=rect,line join=bevel,line width=0.800pt]
  \end{scope}
  \begin{scope}[cm={{1.00588,0.0,0.0,1.00588,(111.653,217.271)}},draw=blue,line cap=rect,line join=bevel,line width=0.800pt]
  \end{scope}
  \begin{scope}[cm={{1.00588,0.0,0.0,1.00588,(111.653,217.271)}},draw=blue,line cap=rect,line join=bevel,line width=0.800pt]
  \end{scope}
  \begin{scope}[cm={{1.00588,0.0,0.0,1.00588,(-224.34706,252.74414)}},draw=blue,line cap=rect,line join=bevel,line width=0.800pt]
    \path[fill=blue] (0.0000,0.0000) node[above right] (text840) {2};



  \end{scope}
  \begin{scope}[cm={{1.00588,0.0,0.0,1.00588,(111.653,217.271)}},draw=blue,line cap=rect,line join=bevel,line width=0.800pt]
  \end{scope}
  \begin{scope}[scale=1.006,draw=blue,line cap=rect,line join=bevel,line width=0.800pt]
  \end{scope}
  \begin{scope}[cm={{1.04177,0.0,0.0,1.04177,(-342.44493,31.82443)}},draw=ca0a0a4,dash pattern=on 1.73pt off 1.73pt,line cap=round,line join=round,line width=0.288pt,miter limit=4.00]
    \path[draw,dash pattern=on 1.73pt off 1.73pt,line width=0.288pt,miter limit=4.00] (143.5000,200.5000) -- (143.5000,115.5000);



  \end{scope}
  \begin{scope}[cm={{1.04177,0.0,0.0,1.04177,(-342.44493,31.82443)}},draw=blue,line cap=round,line join=round,line width=0.480pt]
    \path[cm={{1.0,0.0,0.0,1.53899,(0.0,-108.26833)}},draw] (143.5000,200.5000) -- (143.5000,198.5000);



    \path[cm={{1.0,0.0,0.0,1.1115,(0.0,-12.84424)}},draw] (143.5000,115.5000) -- (143.5000,118.5000);



  \end{scope}
  \begin{scope}[scale=1.006,draw=blue,line cap=rect,line join=bevel,line width=0.800pt]
  \end{scope}
  \begin{scope}[cm={{1.00588,0.0,0.0,1.00588,(142.332,217.271)}},draw=blue,line cap=rect,line join=bevel,line width=0.800pt]
  \end{scope}
  \begin{scope}[cm={{1.00588,0.0,0.0,1.00588,(142.332,217.271)}},draw=blue,line cap=rect,line join=bevel,line width=0.800pt]
  \end{scope}
  \begin{scope}[cm={{1.00588,0.0,0.0,1.00588,(142.332,217.271)}},draw=blue,line cap=rect,line join=bevel,line width=0.800pt]
  \end{scope}
  \begin{scope}[cm={{1.00588,0.0,0.0,1.00588,(142.332,217.271)}},draw=blue,line cap=rect,line join=bevel,line width=0.800pt]
  \end{scope}
  \begin{scope}[cm={{1.00588,0.0,0.0,1.00588,(142.332,217.271)}},draw=blue,line cap=rect,line join=bevel,line width=0.800pt]
  \end{scope}
  \begin{scope}[cm={{1.00588,0.0,0.0,1.00588,(-193.66806,252.63148)}},draw=blue,line cap=rect,line join=bevel,line width=0.800pt]
    \path[fill=blue] (0.0000,0.0000) node[above right] (text870) {3};



  \end{scope}
  \begin{scope}[cm={{1.00588,0.0,0.0,1.00588,(142.332,217.271)}},draw=blue,line cap=rect,line join=bevel,line width=0.800pt]
  \end{scope}
  \begin{scope}[scale=1.006,draw=blue,line cap=rect,line join=bevel,line width=0.800pt]
  \end{scope}
  \begin{scope}[cm={{1.04177,0.0,0.0,1.04177,(-342.44493,31.82443)}},draw=ca0a0a4,dash pattern=on 1.73pt off 1.73pt,line cap=round,line join=round,line width=0.288pt,miter limit=4.00]
    \path[draw,dash pattern=on 1.73pt off 1.73pt,line width=0.288pt,miter limit=4.00] (172.5000,200.5000) -- (172.5000,115.5000);



  \end{scope}
  \begin{scope}[cm={{1.04177,0.0,0.0,1.04177,(-342.44493,31.82443)}},draw=blue,line cap=round,line join=round,line width=0.480pt]
    \path[cm={{1.0,0.0,0.0,1.53899,(0.0,-108.26833)}},draw] (172.5000,200.5000) -- (172.5000,198.5000);



    \path[cm={{1.0,0.0,0.0,1.1115,(0.0,-12.84424)}},draw] (172.5000,115.5000) -- (172.5000,118.5000);



  \end{scope}
  \begin{scope}[scale=1.006,draw=blue,line cap=rect,line join=bevel,line width=0.800pt]
  \end{scope}
  \begin{scope}[cm={{1.00588,0.0,0.0,1.00588,(171.503,217.271)}},draw=blue,line cap=rect,line join=bevel,line width=0.800pt]
  \end{scope}
  \begin{scope}[cm={{1.00588,0.0,0.0,1.00588,(171.503,217.271)}},draw=blue,line cap=rect,line join=bevel,line width=0.800pt]
  \end{scope}
  \begin{scope}[cm={{1.00588,0.0,0.0,1.00588,(171.503,217.271)}},draw=blue,line cap=rect,line join=bevel,line width=0.800pt]
  \end{scope}
  \begin{scope}[cm={{1.00588,0.0,0.0,1.00588,(171.503,217.271)}},draw=blue,line cap=rect,line join=bevel,line width=0.800pt]
  \end{scope}
  \begin{scope}[cm={{1.00588,0.0,0.0,1.00588,(171.503,217.271)}},draw=blue,line cap=rect,line join=bevel,line width=0.800pt]
  \end{scope}
  \begin{scope}[cm={{1.00588,0.0,0.0,1.00588,(-164.49705,252.74414)}},draw=blue,line cap=rect,line join=bevel,line width=0.800pt]
    \path[fill=blue] (0.0000,0.0000) node[above right] (text900) {4};



  \end{scope}
  \begin{scope}[cm={{1.00588,0.0,0.0,1.00588,(171.503,217.271)}},draw=blue,line cap=rect,line join=bevel,line width=0.800pt]
  \end{scope}
  \begin{scope}[scale=1.006,draw=blue,line cap=rect,line join=bevel,line width=0.800pt]
  \end{scope}
  \begin{scope}[cm={{1.04164,0.0,0.0,1.04164,(-342.42871,32.01781)}},draw=ca0a0a4,dash pattern=on 1.73pt off 1.73pt,line cap=round,line join=round,line width=0.288pt,miter limit=4.00]
    \path[draw,dash pattern=on 1.73pt off 1.73pt,line width=0.288pt,miter limit=4.00] (201.5000,200.5000) -- (201.5000,129.5000);



    \path[draw,dash pattern=on 1.73pt off 1.73pt,line width=0.288pt,miter limit=4.00] (201.5000,121.5000) -- (201.5000,115.5000);



  \end{scope}
  \begin{scope}[cm={{1.04177,0.0,0.0,1.04177,(-342.51116,41.97197)}},draw=ca0a0a4,dash pattern=on 1.73pt off 1.73pt,line cap=round,line join=round,line width=0.288pt,miter limit=4.00]
    \path[draw,dash pattern=on 1.73pt off 1.73pt,line width=0.288pt,miter limit=4.00] (198.5000,306.5000) -- (198.5000,221.5000);



  \end{scope}
  \begin{scope}[cm={{1.04177,0.0,0.0,1.04177,(-342.44493,31.82445)}},draw=blue,line cap=round,line join=round,line width=0.480pt]
    \path[cm={{1.0,0.0,0.0,1.53899,(0.0,-108.26833)}},draw] (201.5000,200.5000) -- (201.5000,198.5000);



    \path[cm={{1.0,0.0,0.0,1.1115,(0.0,-12.84424)}},draw] (201.5000,115.5000) -- (201.5000,118.5000);



  \end{scope}
  \begin{scope}[scale=1.006,draw=blue,line cap=rect,line join=bevel,line width=0.800pt]
  \end{scope}
  \begin{scope}[cm={{1.00588,0.0,0.0,1.00588,(200.674,217.271)}},draw=blue,line cap=rect,line join=bevel,line width=0.800pt]
  \end{scope}
  \begin{scope}[cm={{1.00588,0.0,0.0,1.00588,(200.674,217.271)}},draw=blue,line cap=rect,line join=bevel,line width=0.800pt]
  \end{scope}
  \begin{scope}[cm={{1.00588,0.0,0.0,1.00588,(200.674,217.271)}},draw=blue,line cap=rect,line join=bevel,line width=0.800pt]
  \end{scope}
  \begin{scope}[cm={{1.00588,0.0,0.0,1.00588,(200.674,217.271)}},draw=blue,line cap=rect,line join=bevel,line width=0.800pt]
  \end{scope}
  \begin{scope}[cm={{1.00588,0.0,0.0,1.00588,(200.674,217.271)}},draw=blue,line cap=rect,line join=bevel,line width=0.800pt]
  \end{scope}
  \begin{scope}[cm={{1.00588,0.0,0.0,1.00588,(-136.82605,252.63148)}},draw=blue,line cap=rect,line join=bevel,line width=0.800pt]
    \path[fill=blue] (1.4912,0.0000) node[above right] (text932) {5};



  \end{scope}
  \begin{scope}[cm={{1.00588,0.0,0.0,1.00588,(200.674,217.271)}},draw=blue,line cap=rect,line join=bevel,line width=0.800pt]
  \end{scope}
  \begin{scope}[scale=1.006,draw=blue,line cap=rect,line join=bevel,line width=0.800pt]
  \end{scope}
  \begin{scope}[cm={{1.04164,0.0,0.0,1.04164,(-342.48662,32.01785)}},draw=ca0a0a4,dash pattern=on 1.73pt off 1.73pt,line cap=round,line join=round,line width=0.288pt,miter limit=4.00]
    \path[draw,dash pattern=on 1.73pt off 1.73pt,line width=0.288pt,miter limit=4.00] (231.5000,200.5000) -- (231.5000,129.5000);



    \path[draw,dash pattern=on 1.73pt off 1.73pt,line width=0.288pt,miter limit=4.00] (231.5000,121.5000) -- (231.5000,115.5000);



  \end{scope}
  \begin{scope}[cm={{1.04177,0.0,0.0,1.04177,(-342.45387,41.97197)}},draw=ca0a0a4,dash pattern=on 1.73pt off 1.73pt,line cap=round,line join=round,line width=0.288pt,miter limit=4.00]
    \path[draw,dash pattern=on 1.73pt off 1.73pt,line width=0.288pt,miter limit=4.00] (233.5000,306.5000) -- (233.5000,221.5000);



  \end{scope}
  \begin{scope}[cm={{1.04177,0.0,0.0,1.04177,(-342.44493,31.82443)}},draw=blue,line cap=round,line join=round,line width=0.480pt]
    \path[cm={{1.0,0.0,0.0,1.53899,(0.0,-108.26833)}},draw] (231.5000,200.5000) -- (231.5000,198.5000);



    \path[cm={{1.0,0.0,0.0,1.1115,(0.0,-12.84424)}},draw] (231.5000,115.5000) -- (231.5000,118.5000);



  \end{scope}
  \begin{scope}[scale=1.006,draw=blue,line cap=rect,line join=bevel,line width=0.800pt]
  \end{scope}
  \begin{scope}[cm={{1.00588,0.0,0.0,1.00588,(229.341,217.271)}},draw=blue,line cap=rect,line join=bevel,line width=0.800pt]
  \end{scope}
  \begin{scope}[cm={{1.00588,0.0,0.0,1.00588,(229.341,217.271)}},draw=blue,line cap=rect,line join=bevel,line width=0.800pt]
  \end{scope}
  \begin{scope}[cm={{1.00588,0.0,0.0,1.00588,(229.341,217.271)}},draw=blue,line cap=rect,line join=bevel,line width=0.800pt]
  \end{scope}
  \begin{scope}[cm={{1.00588,0.0,0.0,1.00588,(229.341,217.271)}},draw=blue,line cap=rect,line join=bevel,line width=0.800pt]
  \end{scope}
  \begin{scope}[cm={{1.00588,0.0,0.0,1.00588,(229.341,217.271)}},draw=blue,line cap=rect,line join=bevel,line width=0.800pt]
  \end{scope}
  \begin{scope}[cm={{1.00588,0.0,0.0,1.00588,(-106.65906,250.271)}},draw=blue,line cap=rect,line join=bevel,line width=0.800pt]
    \path[fill=blue] (0.0000,2.3467) node[above right] (text964) {6};



  \end{scope}
  \begin{scope}[cm={{1.00588,0.0,0.0,1.00588,(229.341,217.271)}},draw=blue,line cap=rect,line join=bevel,line width=0.800pt]
  \end{scope}
  \begin{scope}[scale=1.006,draw=blue,line cap=rect,line join=bevel,line width=0.800pt]
  \end{scope}
  \begin{scope}[cm={{1.04177,0.0,0.0,1.04177,(-342.44493,31.82443)}},draw=blue,line cap=round,line join=round,line width=0.480pt]
    \path[draw] (56.5000,115.5000) -- (56.5000,200.5000) -- (251.5000,200.5000) -- (251.5000,115.5000) -- (56.5000,115.5000);



  \end{scope}
  \begin{scope}[scale=1.006,draw=blue,line cap=rect,line join=bevel,line width=0.800pt]
  \end{scope}
  \begin{scope}[scale=1.006,draw=blue,line cap=rect,line join=bevel,line width=0.800pt]
  \end{scope}
  \begin{scope}[cm={{1.04164,0.0,0.0,1.04164,(-345.64402,156.32537)}},draw=c00ff00,line cap=round,line join=round,line width=0.480pt]
    \path[draw,even odd rule] (215.5000,6.2015) -- (241.5000,6.2015);



  \end{scope}
  \begin{scope}[scale=1.006,draw=blue,line cap=rect,line join=bevel,line width=0.800pt]
  \end{scope}
  \begin{scope}[scale=1.006,draw=blue,line cap=rect,line join=bevel,line width=0.800pt]
  \end{scope}
  \begin{scope}[scale=1.006,draw=blue,line cap=rect,line join=bevel,line width=0.800pt]
  \end{scope}
  \begin{scope}[cm={{1.00588,0.0,0.0,1.00588,(236.382,132.776)}},draw=blue,line cap=rect,line join=bevel,line width=0.800pt]
  \end{scope}
  \begin{scope}[cm={{1.00588,0.0,0.0,1.00588,(236.382,132.776)}},draw=blue,line cap=rect,line join=bevel,line width=0.800pt]
  \end{scope}
  \begin{scope}[cm={{1.00588,0.0,0.0,1.00588,(236.382,132.776)}},draw=blue,line cap=rect,line join=bevel,line width=0.800pt]
  \end{scope}
  \begin{scope}[cm={{1.00588,0.0,0.0,1.00588,(236.382,132.776)}},draw=blue,line cap=rect,line join=bevel,line width=0.800pt]
  \end{scope}
  \begin{scope}[cm={{1.00588,0.0,0.0,1.00588,(236.382,132.776)}},draw=blue,line cap=rect,line join=bevel,line width=0.800pt]
  \end{scope}
  \begin{scope}[cm={{1.00588,0.0,0.0,1.00588,(236.382,132.776)}},draw=blue,line cap=rect,line join=bevel,line width=0.800pt]
  \end{scope}
  \begin{scope}[scale=1.006,draw=blue,line cap=rect,line join=bevel,line width=0.800pt]
  \end{scope}
  \begin{scope}[scale=1.006,draw=blue,line cap=rect,line join=bevel,line width=0.800pt]
  \end{scope}
  \begin{scope}[scale=1.006,draw=blue,line cap=rect,line join=bevel,line width=0.800pt]
  \end{scope}
  \begin{scope}[cm={{1.04177,0.0,0.0,1.04177,(-342.44493,31.82443)}},draw=blue,line cap=round,line join=round,line width=0.480pt]
    \path[draw] (56.1000,127.5000) -- (56.1000,127.5000) -- (56.3000,142.6000) -- (56.5000,149.0000) -- (56.7000,148.4000) -- (56.9000,141.3000) -- (57.1000,134.7000) -- (57.3000,131.3000) -- (57.5000,130.4000) -- (57.7000,131.2000) -- (57.9000,132.8000) -- (58.1000,135.0000) -- (58.3000,137.3000) -- (58.4000,139.1000) -- (58.6000,140.1000) -- (58.8000,140.4000) -- (59.0000,140.4000) -- (59.2000,140.1000) -- (59.4000,139.9000) -- (59.6000,139.7000) -- (59.8000,139.6000) -- (60.0000,139.5000) -- (60.2000,139.5000) -- (60.4000,139.5000) -- (60.6000,139.5000) -- (60.8000,139.5000) -- (61.0000,139.5000) -- (61.2000,139.5000) -- (61.4000,139.5000) -- (61.6000,139.5000) -- (61.8000,139.5000) -- (62.0000,139.5000) -- (62.2000,139.5000) -- (62.4000,139.5000) -- (62.6000,139.5000) -- (62.8000,139.6000) -- (63.0000,140.3000) -- (63.1000,141.9000) -- (63.3000,143.7000) -- (63.5000,145.6000) -- (63.7000,147.5000) -- (63.9000,149.3000) -- (64.1000,151.1000) -- (64.3000,152.8000) -- (64.5000,154.5000) -- (64.7000,156.1000) -- (64.9000,157.6000) -- (65.1000,159.1000) -- (65.3000,160.5000) -- (65.5000,161.7000) -- (65.7000,162.7000) -- (65.9000,163.6000) -- (66.1000,164.2000) -- (66.3000,164.6000) -- (66.5000,164.7000) -- (66.7000,164.5000) -- (66.9000,164.0000) -- (67.1000,163.2000) -- (67.3000,162.2000) -- (67.5000,161.0000) -- (67.7000,159.7000) -- (67.9000,158.2000) -- (68.0000,156.6000) -- (68.2000,154.9000) -- (68.4000,153.2000) -- (68.6000,151.5000) -- (68.8000,149.7000) -- (69.0000,147.3000) -- (69.2000,144.6000) -- (69.4000,142.0000) -- (69.6000,140.1000) -- (69.8000,139.0000) -- (70.0000,138.5000) -- (70.2000,138.5000) -- (70.4000,138.7000) -- (70.6000,139.0000) -- (70.8000,139.2000) -- (71.0000,139.4000) -- (71.2000,139.5000) -- (71.4000,139.5000) -- (71.6000,139.6000) -- (71.8000,139.5000) -- (72.0000,139.5000) -- (72.2000,139.5000) -- (72.4000,139.5000) -- (72.6000,139.5000) -- (72.7000,139.5000) -- (72.9000,139.5000) -- (73.1000,139.5000) -- (73.3000,139.5000) -- (73.5000,139.4000) -- (73.7000,140.8000) -- (73.9000,142.7000) -- (74.1000,142.4000) -- (74.3000,141.0000) -- (74.5000,139.2000) -- (74.7000,137.3000) -- (74.9000,135.6000) -- (75.1000,134.1000) -- (75.3000,132.8000) -- (75.5000,131.6000) -- (75.7000,130.7000) -- (75.9000,129.8000) -- (76.1000,129.1000) -- (76.3000,128.4000) -- (76.5000,127.9000) -- (76.7000,127.4000) -- (76.9000,127.0000) -- (77.1000,126.7000) -- (77.3000,126.4000) -- (77.5000,126.3000) -- (77.6000,126.2000) -- (77.8000,126.2000) -- (78.0000,126.3000) -- (78.2000,126.5000) -- (78.4000,126.7000) -- (78.6000,127.1000) -- (78.8000,127.5000) -- (79.0000,128.0000) -- (79.2000,128.6000) -- (79.4000,129.3000) -- (79.6000,130.0000) -- (79.8000,130.9000) -- (80.0000,131.9000) -- (80.2000,132.9000) -- (80.4000,134.1000) -- (80.6000,135.3000) -- (80.8000,137.0000) -- (81.0000,138.6000) -- (81.2000,139.6000) -- (81.4000,140.1000) -- (81.6000,140.2000) -- (81.8000,140.1000) -- (82.0000,140.0000) -- (82.2000,139.8000) -- (82.4000,139.6000) -- (82.5000,139.6000) -- (82.7000,139.5000) -- (82.9000,139.5000) -- (83.1000,139.5000) -- (83.3000,139.5000) -- (83.5000,139.5000) -- (83.7000,139.5000) -- (83.9000,139.5000) -- (84.1000,139.5000) -- (84.3000,139.5000) -- (84.5000,139.5000) -- (84.7000,139.5000) -- (84.9000,139.5000) -- (85.1000,139.5000) -- (85.3000,139.5000) -- (85.5000,139.7000) -- (85.7000,140.9000) -- (85.9000,142.7000) -- (86.1000,144.6000) -- (86.3000,146.6000) -- (86.5000,148.5000) -- (86.7000,150.3000) -- (86.9000,152.1000) -- (87.1000,153.8000) -- (87.2000,155.5000) -- (87.4000,157.0000) -- (87.6000,158.6000) -- (87.8000,160.0000) -- (88.0000,161.3000) -- (88.2000,162.4000) -- (88.4000,163.3000) -- (88.6000,164.1000) -- (88.8000,164.5000) -- (89.0000,164.7000) -- (89.2000,164.6000) -- (89.4000,164.1000) -- (89.6000,163.4000) -- (89.8000,162.4000) -- (90.0000,161.2000) -- (90.2000,159.8000) -- (90.4000,158.3000) -- (90.6000,156.6000) -- (90.8000,154.9000) -- (91.0000,153.2000) -- (91.2000,151.5000) -- (91.4000,149.6000) -- (91.6000,147.3000) -- (91.8000,144.8000) -- (92.0000,142.3000) -- (92.1000,140.4000) -- (92.3000,139.2000) -- (92.5000,138.7000) -- (92.7000,138.6000) -- (92.9000,138.7000) -- (93.1000,139.0000) -- (93.3000,139.2000) -- (93.5000,139.4000) -- (93.7000,139.5000) -- (93.9000,139.5000) -- (94.1000,139.5000) -- (94.3000,139.5000) -- (94.5000,139.5000) -- (94.7000,139.5000) -- (94.9000,139.5000) -- (95.1000,139.5000) -- (95.3000,139.5000) -- (95.5000,139.5000) -- (95.7000,139.5000) -- (95.9000,139.5000) -- (96.1000,139.5000) -- (96.3000,141.0000) -- (96.5000,142.6000) -- (96.7000,142.3000) -- (96.8000,140.8000) -- (97.0000,139.0000) -- (97.2000,137.1000) -- (97.4000,135.4000) -- (97.6000,133.9000) -- (97.8000,132.6000) -- (98.0000,131.4000) -- (98.2000,130.5000) -- (98.4000,129.6000) -- (98.6000,128.9000) -- (98.8000,128.3000) -- (99.0000,127.7000) -- (99.2000,127.3000) -- (99.4000,126.9000) -- (99.6000,126.6000) -- (99.8000,126.4000) -- (100.0000,126.2000) -- (100.2000,126.2000) -- (100.4000,126.2000) -- (100.6000,126.3000) -- (100.8000,126.6000) -- (101.0000,126.8000) -- (101.2000,127.3000) -- (101.4000,127.9000) -- (101.6000,128.6000) -- (101.7000,129.3000) -- (101.9000,130.1000) -- (102.1000,130.9000) -- (102.3000,131.9000) -- (102.5000,132.9000) -- (102.7000,134.0000) -- (102.9000,135.3000) -- (103.1000,137.1000) -- (103.3000,138.7000) -- (103.5000,139.8000) -- (103.7000,140.2000) -- (103.9000,140.3000) -- (104.1000,140.2000) -- (104.3000,140.0000) -- (104.5000,139.8000) -- (104.7000,139.6000) -- (104.9000,139.5000) -- (105.1000,139.5000) -- (105.3000,139.5000) -- (105.5000,139.5000) -- (105.7000,139.5000) -- (105.9000,139.5000) -- (106.1000,139.5000) -- (106.3000,139.5000) -- (106.4000,139.5000) -- (106.6000,139.5000) -- (106.8000,139.5000) -- (107.0000,139.5000) -- (107.2000,139.5000) -- (107.4000,139.5000) -- (107.6000,139.5000) -- (107.8000,139.7000) -- (108.0000,140.8000) -- (108.2000,142.5000) -- (108.4000,144.4000) -- (108.6000,146.3000) -- (108.8000,148.1000) -- (109.0000,149.8000) -- (109.2000,151.5000) -- (109.4000,153.2000) -- (109.6000,154.8000) -- (109.8000,156.4000) -- (110.0000,157.9000) -- (110.2000,159.3000) -- (110.4000,160.6000) -- (110.6000,161.7000) -- (110.8000,162.7000) -- (111.0000,163.5000) -- (111.1000,164.1000) -- (111.3000,164.5000) -- (111.5000,164.6000) -- (111.7000,164.5000) -- (111.9000,164.1000) -- (112.1000,163.5000) -- (112.3000,162.7000) -- (112.5000,161.7000) -- (112.7000,160.6000) -- (112.9000,159.3000) -- (113.1000,157.9000) -- (113.3000,156.4000) -- (113.5000,154.9000) -- (113.7000,153.3000) -- (113.9000,151.7000) -- (114.1000,149.9000) -- (114.3000,146.7000) -- (114.5000,143.0000) -- (114.7000,140.3000) -- (114.9000,144.4000) -- (115.1000,145.2000) -- (115.3000,145.1000) -- (115.5000,145.4000) -- (115.7000,145.8000) -- (115.9000,146.2000) -- (116.0000,146.4000) -- (116.2000,146.5000) -- (116.4000,146.6000) -- (116.6000,146.6000) -- (116.8000,146.6000) -- (117.0000,146.6000) -- (117.2000,146.6000) -- (117.4000,146.6000) -- (117.6000,146.5000) -- (117.8000,146.5000) -- (118.0000,146.5000) -- (118.2000,146.5000) -- (118.4000,146.5000) -- (118.6000,146.5000) -- (118.8000,146.5000) -- (119.0000,146.7000) -- (119.2000,147.4000) -- (119.4000,147.9000) -- (119.6000,147.5000) -- (119.8000,146.2000) -- (120.0000,144.3000) -- (120.2000,146.9000) -- (120.4000,148.4000) -- (120.6000,147.9000) -- (120.8000,152.5000) -- (120.9000,151.7000) -- (121.1000,150.8000) -- (121.3000,150.0000) -- (121.5000,149.4000) -- (121.7000,148.8000) -- (121.9000,148.1000) -- (122.1000,152.5000) -- (122.3000,154.9000) -- (122.5000,154.5000) -- (122.7000,154.3000) -- (122.9000,154.3000) -- (123.1000,154.3000) -- (123.3000,154.5000) -- (123.5000,154.7000) -- (123.7000,155.0000) -- (123.9000,155.4000) -- (124.1000,155.8000) -- (124.3000,156.6000) -- (124.5000,155.2000) -- (124.7000,150.6000) -- (124.9000,151.6000) -- (125.1000,152.5000) -- (125.3000,153.6000) -- (125.5000,154.7000) -- (125.7000,156.0000) -- (125.8000,157.7000) -- (126.0000,153.9000) -- (126.2000,153.5000) -- (126.4000,154.2000) -- (126.6000,154.4000) -- (126.8000,154.4000) -- (127.0000,159.8000) -- (127.2000,161.1000) -- (127.4000,160.7000) -- (127.6000,160.6000) -- (127.8000,160.6000) -- (128.0000,160.6000) -- (128.2000,160.6000) -- (128.4000,160.6000) -- (128.6000,160.6000) -- (128.8000,160.6000) -- (129.0000,160.6000) -- (129.2000,160.6000) -- (129.4000,160.6000) -- (129.6000,160.6000) -- (129.8000,160.6000) -- (130.0000,160.6000) -- (130.2000,160.6000) -- (130.4000,160.6000) -- (130.5000,160.7000) -- (130.7000,161.7000) -- (130.9000,163.4000) -- (131.1000,165.5000) -- (131.3000,163.4000) -- (131.5000,161.8000) -- (131.7000,163.0000) -- (131.9000,158.9000) -- (132.1000,160.1000) -- (132.3000,161.9000) -- (132.5000,163.6000) -- (132.7000,164.1000) -- (132.9000,159.7000) -- (133.1000,160.7000) -- (133.3000,161.9000) -- (133.5000,162.9000) -- (133.7000,163.7000) -- (133.9000,164.3000) -- (134.1000,164.6000) -- (134.3000,164.6000) -- (134.5000,164.2000) -- (134.7000,165.2000) -- (134.9000,170.0000) -- (135.1000,169.0000) -- (135.2000,167.8000) -- (135.4000,166.2000) -- (135.6000,166.4000) -- (135.8000,170.3000) -- (136.0000,168.4000) -- (136.2000,171.9000) -- (136.4000,172.4000) -- (136.6000,170.3000) -- (136.8000,167.8000) -- (137.0000,165.0000) -- (137.2000,167.9000) -- (137.4000,168.1000) -- (137.6000,166.9000) -- (137.8000,166.6000) -- (138.0000,166.7000) -- (138.2000,166.9000) -- (138.4000,167.2000) -- (138.6000,167.4000) -- (138.8000,167.5000) -- (139.0000,167.6000) -- (139.2000,167.7000) -- (139.4000,167.7000) -- (139.6000,167.7000) -- (139.8000,167.7000) -- (140.0000,167.6000) -- (140.1000,167.6000) -- (140.3000,167.6000) -- (140.5000,167.6000) -- (140.7000,167.6000) -- (140.9000,167.6000) -- (141.1000,167.6000) -- (141.3000,167.7000) -- (141.5000,169.6000) -- (141.7000,171.2000) -- (141.9000,170.6000) -- (142.1000,169.0000) -- (142.3000,167.1000) -- (142.5000,165.1000) -- (142.7000,163.3000) -- (142.9000,161.8000) -- (143.1000,160.5000) -- (143.3000,159.3000) -- (143.5000,158.3000) -- (143.7000,157.5000) -- (143.9000,156.8000) -- (144.1000,156.1000) -- (144.3000,155.6000) -- (144.5000,155.2000) -- (144.7000,154.8000) -- (144.9000,154.6000) -- (145.0000,154.4000) -- (145.2000,154.3000) -- (145.4000,154.3000) -- (145.6000,154.4000) -- (145.8000,154.6000) -- (146.0000,154.8000) -- (146.2000,155.2000) -- (146.4000,155.7000) -- (146.6000,156.2000) -- (146.8000,156.9000) -- (147.0000,157.6000) -- (147.2000,158.4000) -- (147.4000,159.4000) -- (147.6000,160.4000) -- (147.8000,161.6000) -- (148.0000,162.8000) -- (148.2000,164.5000) -- (148.4000,166.2000) -- (148.6000,167.5000) -- (148.8000,168.2000) -- (149.0000,168.4000) -- (149.2000,168.4000) -- (149.4000,168.2000) -- (149.6000,168.0000) -- (149.8000,167.8000) -- (149.9000,167.7000) -- (150.1000,167.6000) -- (150.3000,167.6000) -- (150.5000,167.6000) -- (150.7000,167.6000) -- (150.9000,167.6000) -- (151.1000,167.6000) -- (151.3000,167.6000) -- (151.5000,167.6000) -- (151.7000,167.6000) -- (151.9000,167.6000) -- (152.1000,167.6000) -- (152.3000,167.6000) -- (152.5000,167.6000) -- (152.7000,167.6000) -- (152.9000,167.7000) -- (153.1000,168.6000) -- (153.3000,170.3000) -- (153.5000,172.2000) -- (153.7000,174.1000) -- (153.9000,176.0000) -- (154.1000,177.9000) -- (154.3000,179.6000) -- (154.5000,181.4000) -- (154.6000,183.0000) -- (154.8000,184.6000) -- (155.0000,186.2000) -- (155.2000,187.6000) -- (155.4000,189.0000) -- (155.6000,190.2000) -- (155.8000,191.2000) -- (156.0000,192.0000) -- (156.2000,192.5000) -- (156.4000,192.8000) -- (156.6000,192.8000) -- (156.8000,192.4000) -- (157.0000,191.8000) -- (157.2000,190.9000) -- (157.4000,189.8000) -- (157.6000,188.5000) -- (157.8000,187.0000) -- (158.0000,185.5000) -- (158.2000,183.8000) -- (158.4000,182.1000) -- (158.6000,180.3000) -- (158.8000,178.6000) -- (159.0000,176.5000) -- (159.2000,174.0000) -- (159.3000,171.4000) -- (159.5000,169.2000) -- (159.7000,167.7000) -- (159.9000,166.9000) -- (160.1000,166.7000) -- (160.3000,166.8000) -- (160.5000,167.0000) -- (160.7000,167.2000) -- (160.9000,167.4000) -- (161.1000,167.6000) -- (161.3000,167.6000) -- (161.5000,167.7000) -- (161.7000,167.7000) -- (161.9000,167.7000) -- (162.1000,167.6000) -- (162.3000,167.6000) -- (162.5000,167.6000) -- (162.7000,167.6000) -- (162.9000,167.6000) -- (163.1000,167.6000) -- (163.3000,167.6000) -- (163.5000,167.6000) -- (163.7000,168.2000) -- (163.9000,170.6000) -- (164.1000,171.2000) -- (164.2000,170.0000) -- (164.4000,168.2000) -- (164.6000,166.2000) -- (164.8000,164.3000) -- (165.0000,162.6000) -- (165.2000,161.2000) -- (165.4000,159.9000) -- (165.6000,158.8000) -- (165.8000,157.9000) -- (166.0000,157.1000) -- (166.2000,156.5000) -- (166.4000,155.9000) -- (166.6000,155.4000) -- (166.8000,155.0000) -- (167.0000,154.7000) -- (167.2000,154.5000) -- (167.4000,154.3000) -- (167.6000,154.3000) -- (167.8000,154.3000) -- (168.0000,154.5000) -- (168.2000,154.7000) -- (168.4000,155.0000) -- (168.6000,155.5000) -- (168.8000,156.0000) -- (169.0000,156.6000) -- (169.1000,157.3000) -- (169.3000,158.1000) -- (169.5000,159.0000) -- (169.7000,160.0000) -- (169.9000,161.2000) -- (170.1000,162.4000) -- (170.3000,163.9000) -- (170.5000,165.7000) -- (170.7000,167.1000) -- (170.9000,168.0000) -- (171.1000,168.4000) -- (171.3000,168.4000) -- (171.5000,168.2000) -- (171.7000,168.0000) -- (171.9000,167.8000) -- (172.1000,167.7000) -- (172.3000,167.7000) -- (172.5000,167.6000) -- (172.7000,167.6000) -- (172.9000,167.6000) -- (173.1000,167.6000) -- (173.3000,167.6000) -- (173.5000,167.6000) -- (173.7000,167.6000) -- (173.9000,167.6000) -- (174.0000,167.6000) -- (174.2000,167.6000) -- (174.4000,167.6000) -- (174.6000,167.6000) -- (174.8000,167.6000) -- (175.0000,167.6000) -- (175.2000,168.2000) -- (175.4000,169.6000) -- (175.6000,171.5000) -- (175.8000,173.5000) -- (176.0000,175.5000) -- (176.2000,177.4000) -- (176.4000,179.2000) -- (176.6000,180.9000) -- (176.8000,182.6000) -- (177.0000,184.2000) -- (177.2000,185.8000) -- (177.4000,187.2000) -- (177.6000,188.6000) -- (177.8000,189.8000) -- (178.0000,190.9000) -- (178.2000,191.8000) -- (178.4000,192.4000) -- (178.6000,192.7000) -- (178.7000,192.8000) -- (178.9000,192.5000) -- (179.1000,192.0000) -- (179.3000,191.2000) -- (179.5000,190.1000) -- (179.7000,188.8000) -- (179.9000,187.3000) -- (180.1000,185.7000) -- (180.3000,184.1000) -- (180.5000,182.4000) -- (180.7000,180.6000) -- (180.9000,178.9000) -- (181.1000,176.9000) -- (181.3000,174.5000) -- (181.5000,171.9000) -- (181.7000,169.6000) -- (181.9000,168.0000) -- (182.1000,167.0000) -- (182.3000,166.7000) -- (182.5000,166.8000) -- (182.7000,167.0000) -- (182.9000,167.2000) -- (183.1000,167.4000) -- (183.3000,167.5000) -- (183.4000,167.6000) -- (183.6000,167.7000) -- (183.8000,167.7000) -- (184.0000,167.7000) -- (184.2000,167.7000) -- (184.4000,167.6000) -- (184.6000,167.6000) -- (184.8000,167.6000) -- (185.0000,167.6000) -- (185.2000,167.6000) -- (185.4000,167.6000) -- (185.6000,167.6000) -- (185.8000,168.0000) -- (186.0000,170.4000) -- (186.2000,171.2000) -- (186.4000,170.3000) -- (186.6000,168.5000) -- (186.8000,166.5000) -- (187.0000,164.6000) -- (187.2000,162.8000) -- (187.4000,161.2000) -- (187.6000,159.9000) -- (187.8000,158.7000) -- (188.0000,157.7000) -- (188.2000,156.9000) -- (188.3000,156.2000) -- (188.5000,155.6000) -- (188.7000,155.1000) -- (188.9000,154.7000) -- (189.1000,154.5000) -- (189.3000,154.3000) -- (189.5000,154.2000) -- (189.7000,154.3000) -- (189.9000,154.4000) -- (190.1000,154.7000) -- (190.3000,155.0000) -- (190.5000,155.5000) -- (190.7000,156.0000) -- (190.9000,156.7000) -- (191.1000,157.5000) -- (191.3000,158.4000) -- (191.5000,159.4000) -- (191.7000,160.6000) -- (191.9000,161.8000) -- (192.1000,163.3000) -- (192.3000,165.2000) -- (192.5000,166.8000) -- (192.7000,167.8000) -- (192.9000,168.3000) -- (193.1000,168.4000) -- (193.2000,168.3000) -- (193.4000,168.1000) -- (193.6000,167.9000) -- (193.8000,167.8000) -- (194.0000,167.7000) -- (194.2000,167.6000) -- (194.4000,167.6000) -- (194.6000,167.6000) -- (194.8000,167.6000) -- (195.0000,167.6000) -- (195.2000,167.6000) -- (195.4000,167.6000) -- (195.6000,167.6000) -- (195.8000,167.6000) -- (196.0000,167.6000) -- (196.2000,167.6000) -- (196.4000,167.6000) -- (196.6000,167.6000) -- (196.8000,167.6000) -- (197.0000,168.0000) -- (197.2000,169.3000) -- (197.4000,171.2000) -- (197.6000,173.2000) -- (197.8000,175.2000) -- (197.9000,177.1000) -- (198.1000,178.9000) -- (198.3000,180.7000) -- (198.5000,182.3000) -- (198.7000,184.0000) -- (198.9000,185.5000) -- (199.1000,187.0000) -- (199.3000,188.4000) -- (199.5000,189.7000) -- (199.7000,190.8000) -- (199.9000,191.7000) -- (200.1000,192.3000) -- (200.3000,192.7000) -- (200.5000,192.8000) -- (200.7000,192.6000) -- (200.9000,192.1000) -- (201.1000,191.3000) -- (201.3000,190.2000) -- (201.5000,188.9000) -- (201.7000,187.5000) -- (201.9000,185.9000) -- (202.1000,184.3000) -- (202.3000,182.5000) -- (202.5000,180.8000) -- (202.7000,179.0000) -- (202.8000,177.1000) -- (203.0000,174.7000) -- (203.2000,172.1000) -- (203.4000,169.8000) -- (203.6000,168.1000) -- (203.8000,167.1000) -- (204.0000,166.7000) -- (204.2000,166.7000) -- (204.4000,166.9000) -- (204.6000,167.2000) -- (204.8000,167.4000) -- (205.0000,167.5000) -- (205.2000,167.6000) -- (205.4000,167.7000) -- (205.6000,167.7000) -- (205.8000,167.7000) -- (206.0000,167.7000) -- (206.2000,167.6000) -- (206.4000,167.6000) -- (206.6000,167.6000) -- (206.8000,167.6000) -- (207.0000,167.6000) -- (207.2000,167.6000) -- (207.4000,167.6000) -- (207.5000,167.8000) -- (207.7000,170.3000) -- (207.9000,171.6000) -- (208.1000,170.8000) -- (208.3000,169.0000) -- (208.5000,166.9000) -- (208.7000,164.8000) -- (208.9000,162.9000) -- (209.1000,161.3000) -- (209.3000,160.0000) -- (209.5000,158.8000) -- (209.7000,157.8000) -- (209.9000,157.0000) -- (210.1000,156.3000) -- (210.3000,155.7000) -- (210.5000,155.2000) -- (210.7000,154.8000) -- (210.9000,154.5000) -- (211.1000,154.3000) -- (211.3000,154.3000) -- (211.5000,154.3000) -- (211.7000,154.4000) -- (211.9000,154.6000) -- (212.1000,154.9000) -- (212.3000,155.3000) -- (212.4000,155.9000) -- (212.6000,156.5000) -- (212.8000,157.2000) -- (213.0000,158.1000) -- (213.2000,159.0000) -- (213.4000,160.1000) -- (213.6000,161.3000) -- (213.8000,162.7000) -- (214.0000,164.4000) -- (214.2000,166.3000) -- (214.4000,167.6000) -- (214.6000,168.3000) -- (214.8000,168.5000) -- (215.0000,168.4000) -- (215.2000,168.2000) -- (215.4000,168.0000) -- (215.6000,167.8000) -- (215.8000,167.7000) -- (216.0000,167.6000) -- (216.2000,167.6000) -- (216.4000,167.6000) -- (216.6000,167.6000) -- (216.8000,167.6000) -- (217.0000,167.6000) -- (217.1000,167.6000) -- (217.3000,167.6000) -- (217.5000,167.6000) -- (217.7000,167.6000) -- (217.9000,167.6000) -- (218.1000,167.6000) -- (218.3000,167.6000) -- (218.5000,167.6000) -- (218.7000,167.7000) -- (218.9000,168.7000) -- (219.1000,170.4000) -- (219.3000,172.4000) -- (219.5000,174.3000) -- (219.7000,176.2000) -- (219.9000,178.1000) -- (220.1000,179.8000) -- (220.3000,181.6000) -- (220.5000,183.2000) -- (220.7000,184.8000) -- (220.9000,186.4000) -- (221.1000,187.8000) -- (221.3000,189.1000) -- (221.5000,190.3000) -- (221.7000,191.3000) -- (221.9000,192.0000) -- (222.0000,192.5000) -- (222.2000,192.8000) -- (222.4000,192.7000) -- (222.6000,192.4000) -- (222.8000,191.7000) -- (223.0000,190.8000) -- (223.2000,189.7000) -- (223.4000,188.3000) -- (223.6000,186.8000) -- (223.8000,185.2000) -- (224.0000,183.6000) -- (224.2000,181.8000) -- (224.4000,180.1000) -- (224.6000,178.3000) -- (224.8000,176.2000) -- (225.0000,173.7000) -- (225.2000,171.1000) -- (225.4000,169.0000) -- (225.6000,167.6000) -- (225.8000,166.9000) -- (226.0000,166.7000) -- (226.2000,166.8000) -- (226.4000,167.0000) -- (226.6000,167.3000) -- (226.8000,167.4000) -- (226.9000,167.6000) -- (227.1000,167.6000) -- (227.3000,167.7000) -- (227.5000,167.7000) -- (227.7000,167.7000) -- (227.9000,167.6000) -- (228.1000,167.6000) -- (228.3000,167.6000) -- (228.5000,167.6000) -- (228.7000,167.6000) -- (228.9000,167.6000) -- (229.1000,167.6000) -- (229.3000,167.5000) -- (229.5000,168.5000) -- (229.7000,171.2000) -- (229.9000,171.5000) -- (230.1000,170.2000) -- (230.3000,168.2000) -- (230.5000,166.1000) -- (230.7000,164.1000) -- (230.9000,162.3000) -- (231.1000,160.8000) -- (231.3000,159.5000) -- (231.4000,158.4000) -- (231.6000,157.5000) -- (231.8000,156.7000) -- (232.0000,156.1000) -- (232.2000,155.5000) -- (232.4000,155.1000) -- (232.6000,154.7000) -- (232.8000,154.5000) -- (233.0000,154.3000) -- (233.2000,154.2000) -- (233.4000,154.3000) -- (233.6000,154.4000) -- (233.8000,154.7000) -- (234.0000,155.0000) -- (234.2000,155.5000) -- (234.4000,156.1000) -- (234.6000,156.7000) -- (234.8000,157.5000) -- (235.0000,158.4000) -- (235.2000,159.4000) -- (235.4000,160.5000) -- (235.6000,161.7000) -- (235.8000,163.2000) -- (236.0000,165.1000) -- (236.2000,166.8000) -- (236.3000,167.9000) -- (236.5000,168.4000) -- (236.7000,168.5000) -- (236.9000,168.3000) -- (237.1000,168.1000) -- (237.3000,167.9000) -- (237.5000,167.8000) -- (237.7000,167.7000) -- (237.9000,167.6000) -- (238.1000,167.6000) -- (238.3000,167.6000) -- (238.5000,167.6000) -- (238.7000,167.6000) -- (238.9000,167.6000) -- (239.1000,167.6000) -- (239.3000,167.6000) -- (239.5000,167.6000) -- (239.7000,167.6000) -- (239.9000,167.6000) -- (240.1000,167.6000) -- (240.3000,167.6000) -- (240.5000,167.6000) -- (240.7000,168.0000) -- (240.9000,169.3000) -- (241.1000,171.1000) -- (241.2000,173.0000) -- (241.4000,174.9000) -- (241.6000,176.8000) -- (241.8000,178.6000) -- (242.0000,180.4000) -- (242.2000,182.1000) -- (242.4000,183.7000) -- (242.6000,185.3000) -- (242.8000,186.8000) -- (243.0000,188.2000) -- (243.2000,189.5000) -- (243.4000,190.6000) -- (243.6000,191.5000) -- (243.8000,192.2000) -- (244.0000,192.7000) -- (244.2000,192.8000) -- (244.4000,192.6000) -- (244.6000,192.2000) -- (244.8000,191.5000) -- (245.0000,190.5000) -- (245.2000,189.3000) -- (245.4000,187.9000) -- (245.6000,186.4000) -- (245.8000,184.7000) -- (246.0000,183.0000) -- (246.1000,181.3000) -- (246.3000,179.6000) -- (246.5000,177.8000) -- (246.7000,175.5000) -- (246.9000,172.9000) -- (247.1000,170.4000) -- (247.3000,168.5000) -- (247.5000,167.3000) -- (247.7000,166.8000) -- (247.9000,166.7000) -- (248.1000,166.9000) -- (248.3000,167.1000) -- (248.5000,167.3000) -- (248.7000,167.5000) -- (248.9000,167.6000) -- (249.1000,167.6000) -- (249.3000,167.7000) -- (249.5000,167.7000) -- (249.7000,167.7000) -- (249.9000,167.6000) -- (250.1000,167.6000) -- (250.3000,167.6000) -- (250.5000,167.6000) -- (250.7000,167.6000) -- (250.9000,167.6000) -- (251.0000,167.6000) -- (251.2000,167.6000) -- (251.4000,169.4000) -- (251.6000,171.5000) -- (251.8000,171.2000);



  \end{scope}
  \begin{scope}[scale=1.006,draw=blue,line cap=rect,line join=bevel,line width=0.800pt]
  \end{scope}
  \begin{scope}[cm={{1.00588,0.0,0.0,1.00588,(197.153,129.759)}},draw=blue,line cap=rect,line join=bevel,line width=0.800pt]
  \end{scope}
  \begin{scope}[cm={{1.00588,0.0,0.0,1.00588,(197.153,129.759)}},draw=blue,line cap=rect,line join=bevel,line width=0.800pt]
  \end{scope}
  \begin{scope}[cm={{1.00588,0.0,0.0,1.00588,(197.153,129.759)}},draw=blue,line cap=rect,line join=bevel,line width=0.800pt]
  \end{scope}
  \begin{scope}[cm={{1.00588,0.0,0.0,1.00588,(197.153,129.759)}},draw=blue,line cap=rect,line join=bevel,line width=0.800pt]
  \end{scope}
  \begin{scope}[cm={{1.00588,0.0,0.0,1.00588,(197.153,129.759)}},draw=blue,line cap=rect,line join=bevel,line width=0.800pt]
  \end{scope}
  \begin{scope}[cm={{1.04164,0.0,0.0,1.04164,(-142.48887,167.27602)}},draw=blue,line cap=rect,line join=bevel,line width=0.800pt]
    \path[fill=blue] (0.0000,0.0000) node[above right] (text1032) {\scriptsize $b_0(t)$};



  \end{scope}
  \begin{scope}[cm={{1.00588,0.0,0.0,1.00588,(197.153,129.759)}},draw=blue,line cap=rect,line join=bevel,line width=0.800pt]
  \end{scope}
  \begin{scope}[scale=1.006,draw=blue,line cap=rect,line join=bevel,line width=0.800pt]
  \end{scope}
  \begin{scope}[scale=1.006,draw=blue,line cap=rect,line join=bevel,line width=0.800pt]
  \end{scope}
  \begin{scope}[scale=1.006,draw=blue,line cap=rect,line join=bevel,line width=0.800pt]
  \end{scope}
  \begin{scope}[scale=1.006,draw=blue,line cap=rect,line join=bevel,line width=0.800pt]
  \end{scope}
  \begin{scope}[scale=1.006,draw=blue,line cap=rect,line join=bevel,line width=0.800pt]
  \end{scope}
  \begin{scope}[cm={{1.04177,0.0,0.0,1.04177,(-342.44493,31.82443)}},draw=c00ff00,line cap=round,line join=round,line width=0.480pt]
    \path[draw] (101.0000,115.7000) -- (101.2000,122.5000) -- (101.7000,123.1000) -- (102.2000,123.7000) -- (102.7000,124.3000) -- (103.2000,124.9000) -- (103.6000,125.5000) -- (104.1000,126.1000) -- (104.6000,126.7000) -- (105.1000,127.3000) -- (105.6000,127.9000) -- (106.1000,128.4000) -- (106.6000,129.0000) -- (107.0000,129.6000) -- (107.5000,130.2000) -- (108.0000,130.8000) -- (108.5000,131.4000) -- (109.0000,132.0000) -- (109.5000,132.6000) -- (110.0000,133.2000) -- (110.4000,133.8000) -- (110.9000,134.3000) -- (111.4000,134.9000) -- (111.9000,135.5000) -- (112.4000,136.1000) -- (112.9000,136.7000) -- (113.4000,137.3000) -- (113.9000,137.9000) -- (114.3000,138.5000) -- (114.8000,139.1000) -- (115.3000,139.7000) -- (115.8000,140.2000) -- (116.3000,140.8000) -- (116.8000,141.4000) -- (117.3000,142.0000) -- (117.7000,142.6000) -- (118.2000,143.2000) -- (118.7000,143.8000) -- (119.2000,144.4000) -- (119.7000,145.0000) -- (120.2000,145.6000) -- (120.7000,146.1000) -- (121.1000,146.7000) -- (121.6000,147.3000) -- (122.1000,147.9000) -- (122.6000,148.5000) -- (123.1000,149.1000) -- (123.6000,149.7000) -- (124.1000,150.3000) -- (124.5000,150.9000) -- (125.0000,151.5000) -- (125.5000,152.0000) -- (126.0000,152.6000) -- (126.5000,153.2000) -- (127.0000,153.8000) -- (127.5000,154.4000) -- (128.0000,155.0000) -- (128.4000,155.6000) -- (128.9000,156.2000) -- (129.4000,156.8000) -- (129.9000,157.4000) -- (130.4000,157.9000) -- (130.9000,158.5000) -- (131.4000,159.1000) -- (131.8000,159.7000) -- (132.3000,160.3000) -- (132.8000,160.9000) -- (133.3000,161.5000) -- (133.8000,162.1000) -- (134.3000,162.7000) -- (134.8000,163.3000) -- (135.2000,163.9000) -- (135.7000,164.4000) -- (136.2000,165.0000) -- (136.7000,165.6000) -- (137.2000,166.2000) -- (137.7000,166.8000) -- (138.2000,167.4000) -- (138.6000,168.0000) -- (139.1000,168.6000) -- (139.6000,169.2000) -- (140.1000,169.7000) -- (140.6000,170.3000) -- (141.1000,170.9000) -- (141.6000,171.5000) -- (142.1000,172.1000) -- (142.5000,172.7000) -- (143.0000,173.3000) -- (143.5000,173.9000) -- (144.0000,174.5000) -- (144.5000,175.1000) -- (145.0000,175.7000) -- (145.5000,176.2000) -- (145.9000,176.8000) -- (146.4000,177.4000) -- (146.9000,178.0000) -- (147.4000,178.6000) -- (147.9000,179.2000) -- (148.4000,179.8000) -- (148.9000,180.4000) -- (149.3000,181.0000) -- (149.8000,181.6000) -- (150.3000,182.1000) -- (150.8000,182.7000) -- (151.3000,183.3000) -- (151.8000,183.9000) -- (152.3000,184.5000) -- (152.7000,185.1000) -- (153.2000,185.7000) -- (153.7000,186.3000) -- (154.2000,186.9000) -- (154.7000,187.5000) -- (155.2000,188.0000) -- (155.7000,188.6000) -- (156.2000,189.2000) -- (156.6000,189.8000) -- (157.1000,190.4000) -- (157.6000,191.0000) -- (158.1000,191.6000) -- (158.6000,192.2000) -- (159.1000,192.8000) -- (159.6000,193.4000) -- (160.0000,193.9000) -- (160.5000,194.5000) -- (161.0000,195.1000) -- (161.5000,195.7000) -- (162.0000,196.3000) -- (162.5000,196.9000) -- (163.0000,197.5000) -- (163.4000,198.1000) -- (163.9000,198.7000) -- (164.4000,199.3000) -- (164.9000,199.8000) -- (165.4000,200.4000) -- (165.5000,200.6000);



  \end{scope}
  \begin{scope}[scale=1.006,draw=blue,line cap=rect,line join=bevel,line width=0.800pt]
  \end{scope}
  \begin{scope}[scale=1.006,draw=blue,line cap=rect,line join=bevel,line width=0.800pt]
  \end{scope}
  \begin{scope}[cm={{1.04177,0.0,0.0,1.04177,(-342.44493,30.31402)}},draw=blue,line cap=round,line join=round,line width=0.480pt]
    \path[shift={(0,1.44988)},draw] (56.5000,115.5000) -- (56.5000,200.5000) -- (251.5000,200.5000) -- (251.5000,115.5000) -- (56.5000,115.5000);



  \end{scope}
  \begin{scope}[cm={{1.04164,0.0,0.0,1.04164,(-342.4953,42.14671)}},draw=ca0a0a4,dash pattern=on 1.73pt off 1.73pt,line cap=round,line join=round,line width=0.288pt,miter limit=4.00]
    \path[draw,dash pattern=on 1.73pt off 1.73pt,line width=0.288pt,miter limit=4.00] (56.5000,299.5000) -- (251.5000,299.5000);



  \end{scope}
  \begin{scope}[cm={{1.04164,0.0,0.0,1.04164,(-342.4953,42.14671)}},draw=blue,line cap=round,line join=round,line width=0.480pt]
    \path[cm={{1.11163,0.0,0.0,1.0,(-6.27372,0.0)}},draw] (56.5000,299.5000) -- (59.5000,299.5000);



    \path[cm={{1.11163,0.0,0.0,1.0,(-28.25871,0.0)}},draw] (251.5000,299.5000) -- (248.5000,299.5000);



  \end{scope}
  \begin{scope}[scale=1.006,draw=blue,line cap=rect,line join=bevel,line width=0.800pt]
  \end{scope}
  \begin{scope}[cm={{1.00588,0.0,0.0,1.00588,(39.2294,305.788)}},draw=blue,line cap=rect,line join=bevel,line width=0.800pt]
  \end{scope}
  \begin{scope}[cm={{1.00588,0.0,0.0,1.00588,(39.2294,305.788)}},draw=blue,line cap=rect,line join=bevel,line width=0.800pt]
  \end{scope}
  \begin{scope}[cm={{1.00588,0.0,0.0,1.00588,(39.2294,305.788)}},draw=blue,line cap=rect,line join=bevel,line width=0.800pt]
  \end{scope}
  \begin{scope}[cm={{1.00588,0.0,0.0,1.00588,(39.2294,305.788)}},draw=blue,line cap=rect,line join=bevel,line width=0.800pt]
  \end{scope}
  \begin{scope}[cm={{1.00588,0.0,0.0,1.00588,(39.2294,305.788)}},draw=blue,line cap=rect,line join=bevel,line width=0.800pt]
  \end{scope}
  \begin{scope}[cm={{1.00588,0.0,0.0,1.00588,(-298.32402,357.32974)}},draw=blue,line cap=rect,line join=bevel,line width=0.800pt]
    \path[fill=blue] (0.0000,0.0000) node[above right] (text1086) {27};



  \end{scope}
  \begin{scope}[cm={{1.00588,0.0,0.0,1.00588,(39.2294,305.788)}},draw=blue,line cap=rect,line join=bevel,line width=0.800pt]
  \end{scope}
  \begin{scope}[scale=1.006,draw=blue,line cap=rect,line join=bevel,line width=0.800pt]
  \end{scope}
  \begin{scope}[cm={{1.04164,0.0,0.0,1.04164,(-342.4953,42.14671)}},draw=ca0a0a4,dash pattern=on 1.73pt off 1.73pt,line cap=round,line join=round,line width=0.288pt,miter limit=4.00]
    \path[draw,dash pattern=on 1.73pt off 1.73pt,line width=0.288pt,miter limit=4.00] (56.5000,277.5000) -- (251.5000,277.5000);



  \end{scope}
  \begin{scope}[cm={{1.04164,0.0,0.0,1.04164,(-342.4953,42.14671)}},draw=blue,line cap=round,line join=round,line width=0.480pt]
    \path[cm={{1.11163,0.0,0.0,1.0,(-6.27372,0.0)}},draw] (56.5000,277.5000) -- (59.5000,277.5000);



    \path[cm={{1.11163,0.0,0.0,1.0,(-28.25871,0.0)}},draw] (251.5000,277.5000) -- (248.5000,277.5000);



  \end{scope}
  \begin{scope}[scale=1.006,draw=blue,line cap=rect,line join=bevel,line width=0.800pt]
  \end{scope}
  \begin{scope}[cm={{1.00588,0.0,0.0,1.00588,(40.2353,282.653)}},draw=blue,line cap=rect,line join=bevel,line width=0.800pt]
  \end{scope}
  \begin{scope}[cm={{1.00588,0.0,0.0,1.00588,(40.2353,282.653)}},draw=blue,line cap=rect,line join=bevel,line width=0.800pt]
  \end{scope}
  \begin{scope}[cm={{1.00588,0.0,0.0,1.00588,(40.2353,282.653)}},draw=blue,line cap=rect,line join=bevel,line width=0.800pt]
  \end{scope}
  \begin{scope}[cm={{1.00588,0.0,0.0,1.00588,(40.2353,282.653)}},draw=blue,line cap=rect,line join=bevel,line width=0.800pt]
  \end{scope}
  \begin{scope}[cm={{1.00588,0.0,0.0,1.00588,(40.2353,282.653)}},draw=blue,line cap=rect,line join=bevel,line width=0.800pt]
  \end{scope}
  \begin{scope}[cm={{1.00588,0.0,0.0,1.00588,(-298.23551,334.19474)}},draw=blue,line cap=rect,line join=bevel,line width=0.800pt]
    \path[fill=blue] (0.3520,0.0000) node[above right] (text1116) {31};



  \end{scope}
  \begin{scope}[cm={{1.00588,0.0,0.0,1.00588,(40.2353,282.653)}},draw=blue,line cap=rect,line join=bevel,line width=0.800pt]
  \end{scope}
  \begin{scope}[scale=1.006,draw=blue,line cap=rect,line join=bevel,line width=0.800pt]
  \end{scope}
  \begin{scope}[cm={{1.04164,0.0,0.0,1.04164,(-342.4953,42.14671)}},draw=ca0a0a4,dash pattern=on 1.73pt off 1.73pt,line cap=round,line join=round,line width=0.288pt,miter limit=4.00]
    \path[draw,dash pattern=on 1.73pt off 1.73pt,line width=0.288pt,miter limit=4.00] (56.5000,254.5000) -- (251.5000,254.5000);



  \end{scope}
  \begin{scope}[cm={{1.04164,0.0,0.0,1.04164,(-342.4953,42.14671)}},draw=blue,line cap=round,line join=round,line width=0.480pt]
    \path[cm={{1.11163,0.0,0.0,1.0,(-6.27372,0.0)}},draw] (56.5000,254.5000) -- (59.5000,254.5000);



    \path[cm={{1.11163,0.0,0.0,1.0,(-28.25871,0.0)}},draw] (251.5000,254.5000) -- (248.5000,254.5000);



  \end{scope}
  \begin{scope}[scale=1.006,draw=blue,line cap=rect,line join=bevel,line width=0.800pt]
  \end{scope}
  \begin{scope}[cm={{1.00588,0.0,0.0,1.00588,(40.2353,260.524)}},draw=blue,line cap=rect,line join=bevel,line width=0.800pt]
  \end{scope}
  \begin{scope}[cm={{1.00588,0.0,0.0,1.00588,(40.2353,260.524)}},draw=blue,line cap=rect,line join=bevel,line width=0.800pt]
  \end{scope}
  \begin{scope}[cm={{1.00588,0.0,0.0,1.00588,(40.2353,260.524)}},draw=blue,line cap=rect,line join=bevel,line width=0.800pt]
  \end{scope}
  \begin{scope}[cm={{1.00588,0.0,0.0,1.00588,(40.2353,260.524)}},draw=blue,line cap=rect,line join=bevel,line width=0.800pt]
  \end{scope}
  \begin{scope}[cm={{1.00588,0.0,0.0,1.00588,(40.2353,260.524)}},draw=blue,line cap=rect,line join=bevel,line width=0.800pt]
  \end{scope}
  \begin{scope}[cm={{1.00588,0.0,0.0,1.00588,(-298.23551,309.06574)}},draw=blue,line cap=rect,line join=bevel,line width=0.800pt]
    \path[fill=blue] (0.0000,0.0000) node[above right] (text1146) {35};



  \end{scope}
  \begin{scope}[cm={{1.00588,0.0,0.0,1.00588,(40.2353,260.524)}},draw=blue,line cap=rect,line join=bevel,line width=0.800pt]
  \end{scope}
  \begin{scope}[scale=1.006,draw=blue,line cap=rect,line join=bevel,line width=0.800pt]
  \end{scope}
  \begin{scope}[cm={{1.04164,0.0,0.0,1.04164,(-342.4953,42.14671)}},draw=blue,line cap=round,line join=round,line width=0.480pt]
    \path[cm={{1.11163,0.0,0.0,1.0,(-6.27372,0.0)}},draw] (56.5000,232.5000) -- (59.5000,232.5000);



    \path[cm={{1.11163,0.0,0.0,1.0,(-28.25871,0.0)}},draw] (251.5000,232.5000) -- (248.5000,232.5000);



  \end{scope}
  \begin{scope}[scale=1.006,draw=blue,line cap=rect,line join=bevel,line width=0.800pt]
  \end{scope}
  \begin{scope}[cm={{1.00588,0.0,0.0,1.00588,(39.2294,237.388)}},draw=blue,line cap=rect,line join=bevel,line width=0.800pt]
  \end{scope}
  \begin{scope}[cm={{1.00588,0.0,0.0,1.00588,(39.2294,237.388)}},draw=blue,line cap=rect,line join=bevel,line width=0.800pt]
  \end{scope}
  \begin{scope}[cm={{1.00588,0.0,0.0,1.00588,(39.2294,237.388)}},draw=blue,line cap=rect,line join=bevel,line width=0.800pt]
  \end{scope}
  \begin{scope}[cm={{1.00588,0.0,0.0,1.00588,(39.2294,237.388)}},draw=blue,line cap=rect,line join=bevel,line width=0.800pt]
  \end{scope}
  \begin{scope}[cm={{1.00588,0.0,0.0,1.00588,(39.2294,237.388)}},draw=blue,line cap=rect,line join=bevel,line width=0.800pt]
  \end{scope}
  \begin{scope}[cm={{1.00588,0.0,0.0,1.00588,(-298.4045,287.42974)}},draw=blue,line cap=rect,line join=bevel,line width=0.800pt]
    \path[fill=blue] (0.0000,0.0000) node[above right] (text1176) {39};



  \end{scope}
  \begin{scope}[cm={{1.00588,0.0,0.0,1.00588,(39.2294,237.388)}},draw=blue,line cap=rect,line join=bevel,line width=0.800pt]
  \end{scope}
  \begin{scope}[scale=1.006,draw=blue,line cap=rect,line join=bevel,line width=0.800pt]
  \end{scope}
  \begin{scope}[cm={{1.04164,0.0,0.0,1.04164,(-342.4953,42.14671)}},draw=ca0a0a4,dash pattern=on 0.40pt off 0.80pt,line cap=round,line join=round,line width=0.400pt]
    \path[draw] (56.5000,306.5000) -- (56.5000,221.5000);



  \end{scope}
  \begin{scope}[cm={{1.04164,0.0,0.0,1.04164,(-342.4953,42.14671)}},draw=blue,line cap=round,line join=round,line width=0.480pt]
    \path[draw] (56.5000,306.5000) -- (56.5000,303.5000);



    \path[draw] (56.5000,221.5000) -- (56.5000,223.5000);



  \end{scope}
  \begin{scope}[scale=1.006,draw=blue,line cap=rect,line join=bevel,line width=0.800pt]
  \end{scope}
  \begin{scope}[cm={{1.00588,0.0,0.0,1.00588,(53.3118,322.888)}},draw=blue,line cap=rect,line join=bevel,line width=0.800pt]
  \end{scope}
  \begin{scope}[cm={{1.00588,0.0,0.0,1.00588,(53.3118,322.888)}},draw=blue,line cap=rect,line join=bevel,line width=0.800pt]
  \end{scope}
  \begin{scope}[cm={{1.00588,0.0,0.0,1.00588,(53.3118,322.888)}},draw=blue,line cap=rect,line join=bevel,line width=0.800pt]
  \end{scope}
  \begin{scope}[cm={{1.00588,0.0,0.0,1.00588,(53.3118,322.888)}},draw=blue,line cap=rect,line join=bevel,line width=0.800pt]
  \end{scope}
  \begin{scope}[cm={{1.00588,0.0,0.0,1.00588,(53.3118,322.888)}},draw=blue,line cap=rect,line join=bevel,line width=0.800pt]
  \end{scope}
  \begin{scope}[cm={{1.00588,0.0,0.0,1.00588,(-285.68826,376.888)}},draw=blue,line cap=rect,line join=bevel,line width=0.800pt]
    \path[fill=blue] (0.0000,0.0000) node[above right] (text1206) {0};



  \end{scope}
  \begin{scope}[cm={{1.00588,0.0,0.0,1.00588,(53.3118,322.888)}},draw=blue,line cap=rect,line join=bevel,line width=0.800pt]
  \end{scope}
  \begin{scope}[scale=1.006,draw=blue,line cap=rect,line join=bevel,line width=0.800pt]
  \end{scope}
  \begin{scope}[cm={{1.04164,0.0,0.0,1.04164,(-342.4953,42.14671)}},draw=ca0a0a4,dash pattern=on 1.73pt off 1.73pt,line cap=round,line join=round,line width=0.288pt,miter limit=4.00]
    \path[draw,dash pattern=on 1.73pt off 1.73pt,line width=0.288pt,miter limit=4.00] (91.5000,306.5000) -- (91.5000,221.5000);



  \end{scope}
  \begin{scope}[cm={{1.04164,0.0,0.0,1.04164,(-342.4953,42.14671)}},draw=blue,line cap=round,line join=round,line width=0.480pt]
    \path[cm={{1.0,0.0,0.0,1.11163,(0.0,-34.30653)}},draw] (91.5000,306.5000) -- (91.5000,303.5000);



    \path[cm={{1.0,0.0,0.0,1.53918,(0.0,-119.26725)}},draw] (91.5000,221.5000) -- (91.5000,223.5000);



  \end{scope}
  \begin{scope}[scale=1.006,draw=blue,line cap=rect,line join=bevel,line width=0.800pt]
  \end{scope}
  \begin{scope}[cm={{1.00588,0.0,0.0,1.00588,(89.5235,322.888)}},draw=blue,line cap=rect,line join=bevel,line width=0.800pt]
  \end{scope}
  \begin{scope}[cm={{1.00588,0.0,0.0,1.00588,(89.5235,322.888)}},draw=blue,line cap=rect,line join=bevel,line width=0.800pt]
  \end{scope}
  \begin{scope}[cm={{1.00588,0.0,0.0,1.00588,(89.5235,322.888)}},draw=blue,line cap=rect,line join=bevel,line width=0.800pt]
  \end{scope}
  \begin{scope}[cm={{1.00588,0.0,0.0,1.00588,(89.5235,322.888)}},draw=blue,line cap=rect,line join=bevel,line width=0.800pt]
  \end{scope}
  \begin{scope}[cm={{1.00588,0.0,0.0,1.00588,(89.5235,322.888)}},draw=blue,line cap=rect,line join=bevel,line width=0.800pt]
  \end{scope}
  \begin{scope}[cm={{1.00588,0.0,0.0,1.00588,(-247.97657,377.00066)}},draw=blue,line cap=rect,line join=bevel,line width=0.800pt]
    \path[fill=blue] (0.0000,0.0000) node[above right] (text1236) {2};



  \end{scope}
  \begin{scope}[cm={{1.00588,0.0,0.0,1.00588,(89.5235,322.888)}},draw=blue,line cap=rect,line join=bevel,line width=0.800pt]
  \end{scope}
  \begin{scope}[scale=1.006,draw=blue,line cap=rect,line join=bevel,line width=0.800pt]
  \end{scope}
  \begin{scope}[cm={{1.04164,0.0,0.0,1.04164,(-342.4953,42.14671)}},draw=ca0a0a4,dash pattern=on 1.73pt off 1.73pt,line cap=round,line join=round,line width=0.288pt,miter limit=4.00]
    \path[draw,dash pattern=on 1.73pt off 1.73pt,line width=0.288pt,miter limit=4.00] (127.5000,306.5000) -- (127.5000,221.5000);



  \end{scope}
  \begin{scope}[cm={{1.04164,0.0,0.0,1.04164,(-342.4953,42.14671)}},draw=blue,line cap=round,line join=round,line width=0.480pt]
    \path[cm={{1.0,0.0,0.0,1.11163,(0.0,-34.30653)}},draw] (127.5000,306.5000) -- (127.5000,303.5000);



    \path[cm={{1.0,0.0,0.0,1.53918,(0.0,-119.26725)}},draw] (127.5000,221.5000) -- (127.5000,223.5000);



  \end{scope}
  \begin{scope}[scale=1.006,draw=blue,line cap=rect,line join=bevel,line width=0.800pt]
  \end{scope}
  \begin{scope}[cm={{1.00588,0.0,0.0,1.00588,(125.232,322.888)}},draw=blue,line cap=rect,line join=bevel,line width=0.800pt]
  \end{scope}
  \begin{scope}[cm={{1.00588,0.0,0.0,1.00588,(125.232,322.888)}},draw=blue,line cap=rect,line join=bevel,line width=0.800pt]
  \end{scope}
  \begin{scope}[cm={{1.00588,0.0,0.0,1.00588,(125.232,322.888)}},draw=blue,line cap=rect,line join=bevel,line width=0.800pt]
  \end{scope}
  \begin{scope}[cm={{1.00588,0.0,0.0,1.00588,(125.232,322.888)}},draw=blue,line cap=rect,line join=bevel,line width=0.800pt]
  \end{scope}
  \begin{scope}[cm={{1.00588,0.0,0.0,1.00588,(125.232,322.888)}},draw=blue,line cap=rect,line join=bevel,line width=0.800pt]
  \end{scope}
  \begin{scope}[cm={{1.00588,0.0,0.0,1.00588,(-212.26806,376.888)}},draw=blue,line cap=rect,line join=bevel,line width=0.800pt]
    \path[fill=blue] (0.0000,0.1120) node[above right] (text1266) {4};



  \end{scope}
  \begin{scope}[cm={{1.00588,0.0,0.0,1.00588,(125.232,322.888)}},draw=blue,line cap=rect,line join=bevel,line width=0.800pt]
  \end{scope}
  \begin{scope}[scale=1.006,draw=blue,line cap=rect,line join=bevel,line width=0.800pt]
  \end{scope}
  \begin{scope}[cm={{1.04164,0.0,0.0,1.04164,(-342.4953,42.14671)}},draw=ca0a0a4,dash pattern=on 1.73pt off 1.73pt,line cap=round,line join=round,line width=0.288pt,miter limit=4.00]
    \path[draw,dash pattern=on 1.73pt off 1.73pt,line width=0.288pt,miter limit=4.00] (162.5000,306.5000) -- (162.5000,221.5000);



  \end{scope}
  \begin{scope}[cm={{1.04164,0.0,0.0,1.04164,(-342.4953,42.14671)}},draw=blue,line cap=round,line join=round,line width=0.480pt]
    \path[cm={{1.0,0.0,0.0,1.11163,(0.0,-34.30653)}},draw] (162.5000,306.5000) -- (162.5000,303.5000);



    \path[cm={{1.0,0.0,0.0,1.53918,(0.0,-119.26725)}},draw] (162.5000,221.5000) -- (162.5000,223.5000);



  \end{scope}
  \begin{scope}[scale=1.006,draw=blue,line cap=rect,line join=bevel,line width=0.800pt]
  \end{scope}
  \begin{scope}[cm={{1.00588,0.0,0.0,1.00588,(160.941,322.888)}},draw=blue,line cap=rect,line join=bevel,line width=0.800pt]
  \end{scope}
  \begin{scope}[cm={{1.00588,0.0,0.0,1.00588,(160.941,322.888)}},draw=blue,line cap=rect,line join=bevel,line width=0.800pt]
  \end{scope}
  \begin{scope}[cm={{1.00588,0.0,0.0,1.00588,(160.941,322.888)}},draw=blue,line cap=rect,line join=bevel,line width=0.800pt]
  \end{scope}
  \begin{scope}[cm={{1.00588,0.0,0.0,1.00588,(160.941,322.888)}},draw=blue,line cap=rect,line join=bevel,line width=0.800pt]
  \end{scope}
  \begin{scope}[cm={{1.00588,0.0,0.0,1.00588,(160.941,322.888)}},draw=blue,line cap=rect,line join=bevel,line width=0.800pt]
  \end{scope}
  \begin{scope}[cm={{1.00588,0.0,0.0,1.00588,(-173.55905,376.888)}},draw=blue,line cap=rect,line join=bevel,line width=0.800pt]
    \path[fill=blue] (0.0000,0.0000) node[above right] (text1296) {6};



  \end{scope}
  \begin{scope}[cm={{1.00588,0.0,0.0,1.00588,(160.941,322.888)}},draw=blue,line cap=rect,line join=bevel,line width=0.800pt]
  \end{scope}
  \begin{scope}[scale=1.006,draw=blue,line cap=rect,line join=bevel,line width=0.800pt]
  \end{scope}
  \begin{scope}[cm={{1.04164,0.0,0.0,1.04164,(-342.4953,42.14671)}},draw=blue,line cap=round,line join=round,line width=0.480pt]
    \path[cm={{1.0,0.0,0.0,1.11163,(0.0,-34.30653)}},draw] (198.5000,306.5000) -- (198.5000,303.5000);



    \path[cm={{1.0,0.0,0.0,1.53918,(0.0,-119.26725)}},draw] (198.5000,221.5000) -- (198.5000,223.5000);



  \end{scope}
  \begin{scope}[scale=1.006,draw=blue,line cap=rect,line join=bevel,line width=0.800pt]
  \end{scope}
  \begin{scope}[cm={{1.00588,0.0,0.0,1.00588,(196.147,322.888)}},draw=blue,line cap=rect,line join=bevel,line width=0.800pt]
  \end{scope}
  \begin{scope}[cm={{1.00588,0.0,0.0,1.00588,(196.147,322.888)}},draw=blue,line cap=rect,line join=bevel,line width=0.800pt]
  \end{scope}
  \begin{scope}[cm={{1.00588,0.0,0.0,1.00588,(196.147,322.888)}},draw=blue,line cap=rect,line join=bevel,line width=0.800pt]
  \end{scope}
  \begin{scope}[cm={{1.00588,0.0,0.0,1.00588,(196.147,322.888)}},draw=blue,line cap=rect,line join=bevel,line width=0.800pt]
  \end{scope}
  \begin{scope}[cm={{1.00588,0.0,0.0,1.00588,(196.147,322.888)}},draw=blue,line cap=rect,line join=bevel,line width=0.800pt]
  \end{scope}
  \begin{scope}[cm={{1.00588,0.0,0.0,1.00588,(-136.85305,376.888)}},draw=blue,line cap=rect,line join=bevel,line width=0.800pt]
    \path[fill=blue] (0.0000,0.0000) node[above right] (text1326) {8};



  \end{scope}
  \begin{scope}[cm={{1.00588,0.0,0.0,1.00588,(196.147,322.888)}},draw=blue,line cap=rect,line join=bevel,line width=0.800pt]
  \end{scope}
  \begin{scope}[scale=1.006,draw=blue,line cap=rect,line join=bevel,line width=0.800pt]
  \end{scope}
  \begin{scope}[cm={{1.04164,0.0,0.0,1.04164,(-342.4953,42.14671)}},draw=blue,line cap=round,line join=round,line width=0.480pt]
    \path[cm={{1.0,0.0,0.0,1.11163,(0.0,-34.30653)}},draw] (233.5000,306.5000) -- (233.5000,303.5000);



    \path[cm={{1.0,0.0,0.0,1.53918,(0.0,-119.26725)}},draw] (233.5000,221.5000) -- (233.5000,223.5000);



  \end{scope}
  \begin{scope}[scale=1.006,draw=blue,line cap=rect,line join=bevel,line width=0.800pt]
  \end{scope}
  \begin{scope}[cm={{1.00588,0.0,0.0,1.00588,(228.838,322.888)}},draw=blue,line cap=rect,line join=bevel,line width=0.800pt]
  \end{scope}
  \begin{scope}[cm={{1.00588,0.0,0.0,1.00588,(228.838,322.888)}},draw=blue,line cap=rect,line join=bevel,line width=0.800pt]
  \end{scope}
  \begin{scope}[cm={{1.00588,0.0,0.0,1.00588,(228.838,322.888)}},draw=blue,line cap=rect,line join=bevel,line width=0.800pt]
  \end{scope}
  \begin{scope}[cm={{1.00588,0.0,0.0,1.00588,(228.838,322.888)}},draw=blue,line cap=rect,line join=bevel,line width=0.800pt]
  \end{scope}
  \begin{scope}[cm={{1.00588,0.0,0.0,1.00588,(228.838,322.888)}},draw=blue,line cap=rect,line join=bevel,line width=0.800pt]
  \end{scope}
  \begin{scope}[cm={{1.00588,0.0,0.0,1.00588,(-102.66205,376.888)}},draw=blue,line cap=rect,line join=bevel,line width=0.800pt]
    \path[fill=blue] (0.0000,0.0000) node[above right] (text1356) {10};



  \end{scope}
  \begin{scope}[cm={{1.00588,0.0,0.0,1.00588,(228.838,322.888)}},draw=blue,line cap=rect,line join=bevel,line width=0.800pt]
  \end{scope}
  \begin{scope}[scale=1.006,draw=blue,line cap=rect,line join=bevel,line width=0.800pt]
  \end{scope}
  \begin{scope}[cm={{1.04164,0.0,0.0,1.04164,(-342.4953,42.14671)}},draw=blue,line cap=round,line join=round,line width=0.480pt]
    \path[draw] (56.5000,221.5000) -- (56.5000,306.5000) -- (251.5000,306.5000) -- (251.5000,221.5000) -- (56.5000,221.5000);



  \end{scope}
  \begin{scope}[scale=1.006,draw=blue,line cap=rect,line join=bevel,line width=0.800pt]
  \end{scope}
  \begin{scope}[scale=1.006,draw=blue,line cap=rect,line join=bevel,line width=0.800pt]
  \end{scope}
  \begin{scope}[scale=1.006,draw=blue,line cap=rect,line join=bevel,line width=0.800pt]
  \end{scope}
  \begin{scope}[scale=1.006,draw=blue,line cap=rect,line join=bevel,line width=0.800pt]
  \end{scope}
  \begin{scope}[scale=1.006,draw=blue,line cap=rect,line join=bevel,line width=0.800pt]
  \end{scope}
  \begin{scope}[cm={{1.00588,0.0,0.0,1.00588,(232.359,238.394)}},draw=blue,line cap=rect,line join=bevel,line width=0.800pt]
  \end{scope}
  \begin{scope}[cm={{1.00588,0.0,0.0,1.00588,(232.359,238.394)}},draw=blue,line cap=rect,line join=bevel,line width=0.800pt]
  \end{scope}
  \begin{scope}[cm={{1.00588,0.0,0.0,1.00588,(232.359,238.394)}},draw=blue,line cap=rect,line join=bevel,line width=0.800pt]
  \end{scope}
  \begin{scope}[cm={{1.00588,0.0,0.0,1.00588,(232.359,238.394)}},draw=blue,line cap=rect,line join=bevel,line width=0.800pt]
  \end{scope}
  \begin{scope}[cm={{1.00588,0.0,0.0,1.00588,(232.359,238.394)}},draw=blue,line cap=rect,line join=bevel,line width=0.800pt]
  \end{scope}
  \begin{scope}[cm={{1.00588,0.0,0.0,1.00588,(232.359,238.394)}},draw=blue,line cap=rect,line join=bevel,line width=0.800pt]
  \end{scope}
  \begin{scope}[cm={{1.00588,0.0,0.0,1.00588,(130.262,337.976)}},draw=blue,line cap=rect,line join=bevel,line width=0.800pt]
  \end{scope}
  \begin{scope}[cm={{1.00588,0.0,0.0,1.00588,(130.262,337.976)}},draw=blue,line cap=rect,line join=bevel,line width=0.800pt]
  \end{scope}
  \begin{scope}[cm={{1.00588,0.0,0.0,1.00588,(130.262,337.976)}},draw=blue,line cap=rect,line join=bevel,line width=0.800pt]
  \end{scope}
  \begin{scope}[cm={{1.00588,0.0,0.0,1.00588,(130.262,337.976)}},draw=blue,line cap=rect,line join=bevel,line width=0.800pt]
  \end{scope}
  \begin{scope}[cm={{1.00588,0.0,0.0,1.00588,(130.262,337.976)}},draw=blue,line cap=rect,line join=bevel,line width=0.800pt]
  \end{scope}
  \begin{scope}[cm={{1.00588,0.0,0.0,1.00588,(-204.23807,390.01506)}},draw=blue,line cap=rect,line join=bevel,line width=0.800pt]
    \path[fill=blue] (0.0000,0.0000) node[above right] (text1412) {Time (min)};



  \end{scope}
  \begin{scope}[cm={{1.00588,0.0,0.0,1.00588,(130.262,337.976)}},draw=blue,line cap=rect,line join=bevel,line width=0.800pt]
  \end{scope}
  \begin{scope}[scale=1.006,draw=blue,line cap=rect,line join=bevel,line width=0.800pt]
  \end{scope}
  \begin{scope}[scale=1.006,draw=blue,line cap=rect,line join=bevel,line width=0.800pt]
  \end{scope}
  \begin{scope}[scale=1.006,draw=blue,line cap=rect,line join=bevel,line width=0.800pt]
  \end{scope}
  \begin{scope}[cm={{1.04164,0.0,0.0,1.04164,(-342.4953,42.14671)}},draw=blue,line cap=round,line join=round,line width=0.480pt]
    \path[draw] (56.1000,255.4000) -- (56.1000,255.4000) -- (56.3000,261.0000) -- (56.5000,262.0000) -- (56.7000,262.2000) -- (56.9000,262.4000) -- (57.1000,261.4000) -- (57.3000,260.3000) -- (57.5000,259.7000) -- (57.7000,259.7000) -- (57.9000,259.8000) -- (58.1000,259.9000) -- (58.3000,260.0000) -- (58.4000,260.0000) -- (58.6000,260.0000) -- (58.8000,260.0000) -- (59.0000,260.0000) -- (59.2000,260.0000) -- (59.4000,260.0000) -- (59.6000,260.0000) -- (59.8000,260.0000) -- (60.0000,260.0000) -- (60.2000,260.0000) -- (60.4000,260.0000) -- (60.6000,260.2000) -- (60.8000,260.7000) -- (61.0000,261.3000) -- (61.2000,262.2000) -- (61.4000,263.3000) -- (61.6000,264.6000) -- (61.8000,266.1000) -- (62.0000,267.8000) -- (62.2000,269.7000) -- (62.4000,271.8000) -- (62.6000,274.1000) -- (62.8000,276.6000) -- (63.0000,279.2000) -- (63.1000,281.9000) -- (63.3000,284.7000) -- (63.5000,287.5000) -- (63.7000,290.1000) -- (63.9000,292.8000) -- (64.1000,295.5000) -- (64.3000,297.5000) -- (64.5000,298.2000) -- (64.7000,298.2000) -- (64.9000,297.9000) -- (65.1000,297.6000) -- (65.3000,297.5000) -- (65.5000,297.4000) -- (65.7000,297.4000) -- (65.9000,297.4000) -- (66.1000,297.4000) -- (66.3000,297.5000) -- (66.5000,297.5000) -- (66.7000,297.5000) -- (66.9000,297.5000) -- (67.1000,297.4000) -- (67.3000,297.1000) -- (67.5000,297.2000) -- (67.7000,297.2000) -- (67.9000,296.1000) -- (68.0000,294.1000) -- (68.2000,291.2000) -- (68.4000,288.0000) -- (68.6000,284.5000) -- (68.8000,281.1000) -- (69.0000,277.8000) -- (69.2000,274.8000) -- (69.4000,272.0000) -- (69.6000,269.4000) -- (69.8000,267.2000) -- (70.0000,265.3000) -- (70.2000,263.7000) -- (70.4000,262.4000) -- (70.6000,261.3000) -- (70.8000,260.5000) -- (71.0000,260.0000) -- (71.2000,259.6000) -- (71.4000,259.7000) -- (71.6000,259.8000) -- (71.8000,259.9000) -- (72.0000,260.0000) -- (72.2000,260.0000) -- (72.4000,260.0000) -- (72.6000,260.0000) -- (72.7000,260.0000) -- (72.9000,260.0000) -- (73.1000,260.0000) -- (73.3000,260.0000) -- (73.5000,260.0000) -- (73.7000,260.0000) -- (73.9000,260.0000) -- (74.1000,260.0000) -- (74.3000,260.1000) -- (74.5000,260.5000) -- (74.7000,261.1000) -- (74.9000,261.9000) -- (75.1000,262.9000) -- (75.3000,264.1000) -- (75.5000,265.4000) -- (75.7000,267.0000) -- (75.9000,268.9000) -- (76.1000,270.9000) -- (76.3000,273.1000) -- (76.5000,275.5000) -- (76.7000,278.0000) -- (76.9000,280.7000) -- (77.1000,283.4000) -- (77.3000,286.2000) -- (77.4000,288.9000) -- (77.6000,291.5000) -- (77.8000,294.2000) -- (78.0000,296.7000) -- (78.2000,298.1000) -- (78.4000,298.3000) -- (78.6000,298.0000) -- (78.8000,297.7000) -- (79.0000,297.5000) -- (79.2000,297.4000) -- (79.4000,297.4000) -- (79.6000,297.4000) -- (79.8000,297.4000) -- (80.0000,297.4000) -- (80.2000,297.5000) -- (80.4000,297.5000) -- (80.6000,297.5000) -- (80.8000,297.5000) -- (81.0000,298.3000) -- (81.2000,280.4000) -- (81.4000,268.1000) -- (81.6000,268.7000) -- (81.8000,267.3000) -- (82.0000,265.2000) -- (82.2000,262.6000) -- (82.3000,259.7000) -- (82.5000,256.7000) -- (82.7000,253.7000) -- (82.9000,250.7000) -- (83.1000,247.9000) -- (83.3000,245.3000) -- (83.5000,242.9000) -- (83.7000,240.7000) -- (83.9000,238.8000) -- (84.1000,237.2000) -- (84.3000,235.7000) -- (84.5000,234.5000) -- (84.7000,233.5000) -- (84.9000,232.7000) -- (85.1000,232.2000) -- (85.3000,231.7000) -- (85.5000,231.6000) -- (85.7000,231.6000) -- (85.9000,231.7000) -- (86.1000,231.8000) -- (86.3000,231.9000) -- (86.5000,231.9000) -- (86.7000,231.9000) -- (86.9000,231.9000) -- (87.0000,231.9000) -- (87.2000,231.9000) -- (87.4000,231.9000) -- (87.6000,231.9000) -- (87.8000,231.9000) -- (88.0000,231.9000) -- (88.2000,231.9000) -- (88.4000,231.9000) -- (88.6000,232.1000) -- (88.8000,232.5000) -- (89.0000,233.1000) -- (89.2000,234.0000) -- (89.4000,235.0000) -- (89.6000,236.2000) -- (89.8000,237.7000) -- (90.0000,239.3000) -- (90.2000,241.2000) -- (90.4000,243.2000) -- (90.6000,245.4000) -- (90.8000,247.9000) -- (91.0000,250.4000) -- (91.2000,253.1000) -- (91.4000,255.8000) -- (91.6000,258.6000) -- (91.8000,261.3000) -- (91.9000,263.9000) -- (92.1000,266.6000) -- (92.3000,269.0000) -- (92.5000,270.1000) -- (92.7000,270.2000) -- (92.9000,269.8000) -- (93.1000,269.5000) -- (93.3000,269.4000) -- (93.5000,269.3000) -- (93.7000,269.3000) -- (93.9000,269.3000) -- (94.1000,269.3000) -- (94.3000,269.3000) -- (94.5000,269.3000) -- (94.7000,269.3000) -- (94.9000,269.3000) -- (95.1000,269.3000) -- (95.3000,269.1000) -- (95.5000,269.1000) -- (95.7000,269.2000) -- (95.9000,268.5000) -- (96.1000,266.9000) -- (96.3000,264.6000) -- (96.5000,261.9000) -- (96.6000,259.0000) -- (96.8000,255.9000) -- (97.0000,252.9000) -- (97.2000,250.0000) -- (97.4000,247.3000) -- (97.6000,244.7000) -- (97.8000,242.4000) -- (98.0000,240.3000) -- (98.2000,238.4000) -- (98.4000,236.8000) -- (98.6000,235.4000) -- (98.8000,234.2000) -- (99.0000,233.3000) -- (99.2000,232.6000) -- (99.4000,232.0000) -- (99.6000,231.7000) -- (99.8000,231.6000) -- (100.0000,231.7000) -- (100.2000,231.8000) -- (100.4000,231.8000) -- (100.6000,231.9000) -- (100.8000,231.9000) -- (101.0000,231.9000) -- (101.2000,231.9000) -- (101.3000,231.9000) -- (101.5000,231.9000) -- (101.7000,231.9000) -- (101.9000,231.9000) -- (102.1000,231.9000) -- (102.3000,231.9000) -- (102.5000,231.9000) -- (102.7000,231.9000) -- (102.9000,232.2000) -- (103.1000,232.7000) -- (103.3000,233.4000) -- (103.5000,234.3000) -- (103.7000,235.4000) -- (103.9000,236.7000) -- (104.1000,238.2000) -- (104.3000,239.9000) -- (104.5000,241.8000) -- (104.7000,244.0000) -- (104.9000,246.3000) -- (105.1000,248.7000) -- (105.3000,251.4000) -- (105.5000,254.1000) -- (105.7000,256.8000) -- (105.9000,259.6000) -- (106.0000,262.2000) -- (106.2000,264.9000) -- (106.4000,267.6000) -- (106.6000,269.5000) -- (106.8000,270.2000) -- (107.0000,270.0000) -- (107.2000,269.7000) -- (107.4000,269.5000) -- (107.6000,269.3000) -- (107.8000,269.3000) -- (108.0000,269.3000) -- (108.2000,269.3000) -- (108.4000,269.3000) -- (108.6000,269.3000) -- (108.8000,269.3000) -- (109.0000,269.3000) -- (109.2000,269.3000) -- (109.4000,269.2000) -- (109.6000,269.0000) -- (109.8000,269.2000) -- (110.0000,269.0000) -- (110.2000,268.0000) -- (110.4000,266.1000) -- (110.6000,263.6000) -- (110.8000,260.8000) -- (110.9000,257.8000) -- (111.1000,254.8000) -- (111.3000,251.8000) -- (111.5000,248.9000) -- (111.7000,246.2000) -- (111.9000,243.8000) -- (112.1000,241.5000) -- (112.3000,239.5000) -- (112.5000,237.7000) -- (112.7000,236.2000) -- (112.9000,234.9000) -- (113.1000,233.8000) -- (113.3000,233.0000) -- (113.5000,232.3000) -- (113.7000,231.9000) -- (113.9000,231.6000) -- (114.1000,231.6000) -- (114.3000,231.7000) -- (114.5000,231.8000) -- (114.7000,231.9000) -- (114.9000,231.9000) -- (115.1000,231.9000) -- (115.3000,231.9000) -- (115.5000,231.9000) -- (115.6000,231.9000) -- (115.8000,231.9000) -- (116.0000,231.9000) -- (116.2000,231.9000) -- (116.4000,231.9000) -- (116.6000,231.9000) -- (116.8000,231.9000) -- (117.0000,232.0000) -- (117.2000,232.3000) -- (117.4000,232.9000) -- (117.6000,233.7000) -- (117.8000,234.7000) -- (118.0000,235.9000) -- (118.2000,237.2000) -- (118.4000,238.8000) -- (118.6000,240.6000) -- (118.8000,242.6000) -- (119.0000,244.8000) -- (119.2000,247.2000) -- (119.4000,249.8000) -- (119.6000,252.4000) -- (119.8000,255.2000) -- (120.0000,257.9000) -- (120.2000,260.6000) -- (120.4000,263.2000) -- (120.5000,266.0000) -- (120.7000,268.5000) -- (120.9000,269.9000) -- (121.1000,270.2000) -- (121.3000,269.9000) -- (121.5000,269.6000) -- (121.7000,269.4000) -- (121.9000,269.3000) -- (122.1000,269.3000) -- (122.3000,269.3000) -- (122.5000,269.3000) -- (122.7000,269.3000) -- (122.9000,269.3000) -- (123.1000,269.3000) -- (123.3000,269.3000) -- (123.5000,269.3000) -- (123.7000,269.1000) -- (123.9000,269.1000) -- (124.1000,269.2000) -- (124.3000,268.7000) -- (124.5000,267.3000) -- (124.7000,265.2000) -- (124.9000,262.5000) -- (125.1000,259.6000) -- (125.2000,256.6000) -- (125.4000,253.6000) -- (125.6000,250.6000) -- (125.8000,247.8000) -- (126.0000,245.2000) -- (126.2000,242.8000) -- (126.4000,240.7000) -- (126.6000,238.8000) -- (126.8000,237.1000) -- (127.0000,235.7000) -- (127.2000,234.5000) -- (127.4000,233.5000) -- (127.6000,232.7000) -- (127.8000,232.1000) -- (128.0000,231.7000) -- (128.2000,231.6000) -- (128.4000,231.6000) -- (128.6000,231.8000) -- (128.8000,231.8000) -- (129.0000,231.9000) -- (129.2000,231.9000) -- (129.4000,231.9000) -- (129.6000,231.9000) -- (129.8000,231.9000) -- (129.9000,231.9000) -- (130.1000,231.9000) -- (130.3000,231.9000) -- (130.5000,231.9000) -- (130.7000,231.9000) -- (130.9000,231.9000) -- (131.1000,231.9000) -- (131.3000,232.1000) -- (131.5000,232.5000) -- (131.7000,233.2000) -- (131.9000,234.1000) -- (132.1000,235.1000) -- (132.3000,236.4000) -- (132.5000,237.9000) -- (132.7000,239.6000) -- (132.9000,241.4000) -- (133.1000,243.5000) -- (133.3000,245.8000) -- (133.5000,248.2000) -- (133.7000,250.8000) -- (133.9000,253.5000) -- (134.1000,256.3000) -- (134.3000,259.1000) -- (134.5000,261.7000) -- (134.7000,264.3000) -- (134.8000,267.1000) -- (135.0000,269.2000) -- (135.2000,270.1000) -- (135.4000,270.1000) -- (135.6000,269.8000) -- (135.8000,269.5000) -- (136.0000,269.3000) -- (136.2000,269.3000) -- (136.4000,269.3000) -- (136.6000,269.3000) -- (136.8000,269.3000) -- (137.0000,269.3000) -- (137.2000,269.3000) -- (137.4000,269.3000) -- (137.6000,269.3000) -- (137.8000,269.3000) -- (138.0000,269.0000) -- (138.2000,269.2000) -- (138.4000,269.1000) -- (138.6000,268.2000) -- (138.8000,266.5000) -- (139.0000,264.1000) -- (139.2000,261.3000) -- (139.4000,258.4000) -- (139.5000,255.3000) -- (139.7000,252.3000) -- (139.9000,249.4000) -- (140.1000,246.7000) -- (140.3000,244.2000) -- (140.5000,241.9000) -- (140.7000,239.9000) -- (140.9000,238.1000) -- (141.1000,236.5000) -- (141.3000,235.1000) -- (141.5000,234.0000) -- (141.7000,233.1000) -- (141.9000,232.4000) -- (142.1000,231.9000) -- (142.3000,231.6000) -- (142.5000,231.6000) -- (142.7000,231.7000) -- (142.9000,231.8000) -- (143.1000,231.8000) -- (143.3000,231.9000) -- (143.5000,231.9000) -- (143.7000,231.9000) -- (143.9000,231.9000) -- (144.1000,231.9000) -- (144.3000,231.9000) -- (144.4000,231.9000) -- (144.6000,231.9000) -- (144.8000,231.9000) -- (145.0000,231.9000) -- (145.2000,231.9000) -- (145.4000,232.0000) -- (145.6000,232.3000) -- (145.8000,232.8000) -- (146.0000,233.5000) -- (146.2000,234.5000) -- (146.4000,235.6000) -- (146.6000,237.0000) -- (146.8000,238.5000) -- (147.0000,240.3000) -- (147.2000,242.3000) -- (147.4000,244.4000) -- (147.6000,246.8000) -- (147.8000,249.3000) -- (148.0000,252.0000) -- (148.2000,254.7000) -- (148.4000,257.5000) -- (148.6000,260.2000) -- (148.8000,262.8000) -- (149.0000,265.5000) -- (149.1000,268.1000) -- (149.3000,269.8000) -- (149.5000,270.2000) -- (149.7000,270.0000) -- (149.9000,269.6000) -- (150.1000,269.4000) -- (150.3000,269.3000) -- (150.5000,269.3000) -- (150.7000,269.3000) -- (150.9000,269.3000) -- (151.1000,269.3000) -- (151.3000,269.3000) -- (151.5000,269.3000) -- (151.7000,269.3000) -- (151.9000,269.3000) -- (152.1000,269.2000) -- (152.3000,269.0000) -- (152.5000,269.2000) -- (152.7000,268.9000) -- (152.9000,267.6000) -- (153.1000,265.6000) -- (153.3000,263.0000) -- (153.5000,260.1000) -- (153.7000,257.1000) -- (153.8000,254.1000) -- (154.0000,251.1000) -- (154.2000,248.3000) -- (154.4000,245.7000) -- (154.6000,243.2000) -- (154.8000,241.0000) -- (155.0000,239.1000) -- (155.2000,237.1000) -- (155.4000,238.9000) -- (155.6000,242.0000) -- (155.8000,240.7000) -- (156.0000,239.8000) -- (156.2000,239.2000) -- (156.4000,238.8000) -- (156.6000,238.6000) -- (156.8000,238.6000) -- (157.0000,238.8000) -- (157.2000,238.9000) -- (157.4000,238.9000) -- (157.6000,238.9000) -- (157.8000,238.9000) -- (158.0000,238.9000) -- (158.2000,238.9000) -- (158.4000,238.9000) -- (158.6000,238.9000) -- (158.7000,238.9000) -- (158.9000,238.9000) -- (159.1000,238.9000) -- (159.3000,238.9000) -- (159.5000,238.9000) -- (159.7000,239.1000) -- (159.9000,239.5000) -- (160.1000,240.1000) -- (160.3000,240.9000) -- (160.5000,242.2000) -- (160.7000,241.5000) -- (160.9000,237.5000) -- (161.1000,239.2000) -- (161.3000,241.1000) -- (161.5000,243.1000) -- (161.7000,245.4000) -- (161.9000,247.8000) -- (162.1000,250.4000) -- (162.3000,253.0000) -- (162.5000,255.8000) -- (162.7000,258.6000) -- (162.9000,261.3000) -- (163.1000,263.8000) -- (163.3000,266.6000) -- (163.4000,268.9000) -- (163.6000,270.1000) -- (163.8000,270.1000) -- (164.0000,269.8000) -- (164.2000,269.5000) -- (164.4000,269.4000) -- (164.6000,269.3000) -- (164.8000,269.3000) -- (165.0000,269.3000) -- (165.2000,269.3000) -- (165.4000,269.3000) -- (165.6000,269.3000) -- (165.8000,269.3000) -- (166.0000,269.3000) -- (166.2000,269.3000) -- (166.4000,269.0000) -- (166.6000,269.1000) -- (166.8000,269.2000) -- (167.0000,268.5000) -- (167.2000,266.9000) -- (167.4000,264.6000) -- (167.6000,261.9000) -- (167.8000,258.9000) -- (168.0000,255.9000) -- (168.2000,252.9000) -- (168.3000,250.0000) -- (168.5000,246.9000) -- (168.7000,247.8000) -- (168.9000,249.7000) -- (169.1000,247.3000) -- (169.3000,245.3000) -- (169.5000,244.1000) -- (169.7000,248.5000) -- (169.9000,248.4000) -- (170.1000,247.3000) -- (170.3000,246.6000) -- (170.5000,246.1000) -- (170.7000,245.7000) -- (170.9000,245.6000) -- (171.1000,245.7000) -- (171.3000,245.8000) -- (171.5000,245.9000) -- (171.7000,245.9000) -- (171.9000,245.9000) -- (172.1000,245.9000) -- (172.3000,245.9000) -- (172.5000,245.9000) -- (172.7000,245.9000) -- (172.9000,245.9000) -- (173.0000,245.9000) -- (173.2000,245.9000) -- (173.4000,245.9000) -- (173.6000,245.9000) -- (173.8000,246.0000) -- (174.0000,246.2000) -- (174.2000,246.7000) -- (174.4000,247.4000) -- (174.6000,248.4000) -- (174.8000,249.6000) -- (175.0000,245.8000) -- (175.2000,245.0000) -- (175.4000,247.0000) -- (175.6000,249.1000) -- (175.8000,249.5000) -- (176.0000,246.3000) -- (176.2000,248.8000) -- (176.4000,251.4000) -- (176.6000,254.1000) -- (176.8000,256.9000) -- (177.0000,259.6000) -- (177.2000,262.3000) -- (177.4000,264.9000) -- (177.6000,267.7000) -- (177.7000,269.5000) -- (177.9000,270.2000) -- (178.1000,270.0000) -- (178.3000,269.7000) -- (178.5000,269.5000) -- (178.7000,269.3000) -- (178.9000,269.3000) -- (179.1000,269.3000) -- (179.3000,269.3000) -- (179.5000,269.3000) -- (179.7000,269.3000) -- (179.9000,269.3000) -- (180.1000,269.3000) -- (180.3000,269.3000) -- (180.5000,269.2000) -- (180.7000,269.0000) -- (180.9000,269.2000) -- (181.1000,269.0000) -- (181.3000,267.9000) -- (181.5000,266.1000) -- (181.7000,263.6000) -- (181.9000,260.7000) -- (182.1000,257.6000) -- (182.3000,255.1000) -- (182.4000,258.0000) -- (182.6000,255.9000) -- (182.8000,253.5000) -- (183.0000,256.9000) -- (183.2000,255.7000) -- (183.4000,253.5000) -- (183.6000,251.5000) -- (183.8000,253.1000) -- (184.0000,256.2000) -- (184.2000,254.9000) -- (184.4000,254.0000) -- (184.6000,253.4000) -- (184.8000,252.9000) -- (185.0000,252.7000) -- (185.2000,252.7000) -- (185.4000,252.8000) -- (185.6000,252.9000) -- (185.8000,252.9000) -- (186.0000,253.0000) -- (186.2000,252.9000) -- (186.4000,252.9000) -- (186.6000,252.9000) -- (186.8000,252.9000) -- (187.0000,252.9000) -- (187.2000,252.9000) -- (187.3000,252.9000) -- (187.5000,252.9000) -- (187.7000,252.9000) -- (187.9000,253.0000) -- (188.1000,253.1000) -- (188.3000,253.4000) -- (188.5000,254.0000) -- (188.7000,254.8000) -- (188.9000,256.0000) -- (189.1000,255.4000) -- (189.3000,251.3000) -- (189.5000,252.9000) -- (189.7000,254.8000) -- (189.9000,257.0000) -- (190.1000,254.2000) -- (190.3000,254.1000) -- (190.5000,257.1000) -- (190.7000,254.9000) -- (190.9000,255.0000) -- (191.1000,258.0000) -- (191.3000,260.7000) -- (191.5000,263.3000) -- (191.7000,266.0000) -- (191.9000,268.6000) -- (192.0000,269.9000) -- (192.2000,270.2000) -- (192.4000,269.9000) -- (192.6000,269.6000) -- (192.8000,269.4000) -- (193.0000,269.3000) -- (193.2000,269.3000) -- (193.4000,269.3000) -- (193.6000,269.3000) -- (193.8000,269.3000) -- (194.0000,269.3000) -- (194.2000,269.3000) -- (194.4000,269.3000) -- (194.6000,269.3000) -- (194.8000,269.1000) -- (195.0000,269.1000) -- (195.2000,269.2000) -- (195.4000,268.7000) -- (195.6000,267.3000) -- (195.8000,265.1000) -- (196.0000,262.4000) -- (196.2000,259.9000) -- (196.4000,262.5000) -- (196.6000,263.8000) -- (196.8000,264.9000) -- (196.9000,261.5000) -- (197.1000,262.2000) -- (197.3000,264.2000) -- (197.5000,261.7000) -- (197.7000,259.8000) -- (197.9000,258.4000) -- (198.1000,262.7000) -- (198.3000,262.7000) -- (198.5000,261.6000) -- (198.7000,260.8000) -- (198.9000,260.2000) -- (199.1000,259.8000) -- (199.3000,259.7000) -- (199.5000,259.8000) -- (199.7000,259.9000) -- (199.9000,260.0000) -- (200.1000,260.0000) -- (200.3000,260.0000) -- (200.5000,260.0000) -- (200.7000,260.0000) -- (200.9000,260.0000) -- (201.1000,260.0000) -- (201.3000,260.0000) -- (201.5000,260.0000) -- (201.6000,260.0000) -- (201.8000,260.0000) -- (202.0000,260.0000) -- (202.2000,260.0000) -- (202.4000,260.2000) -- (202.6000,260.7000) -- (202.8000,261.3000) -- (203.0000,262.2000) -- (203.2000,263.5000) -- (203.4000,259.7000) -- (203.6000,258.7000) -- (203.8000,260.7000) -- (204.0000,262.7000) -- (204.2000,263.3000) -- (204.4000,259.9000) -- (204.6000,262.5000) -- (204.8000,263.7000) -- (205.0000,260.7000) -- (205.2000,263.5000) -- (205.4000,265.0000) -- (205.6000,261.9000) -- (205.8000,264.3000) -- (206.0000,267.1000) -- (206.2000,269.3000) -- (206.3000,270.1000) -- (206.5000,270.1000) -- (206.7000,269.8000) -- (206.9000,269.5000) -- (207.1000,269.3000) -- (207.3000,269.3000) -- (207.5000,269.3000) -- (207.7000,269.3000) -- (207.9000,269.3000) -- (208.1000,269.3000) -- (208.3000,269.3000) -- (208.5000,269.3000) -- (208.7000,269.3000) -- (208.9000,269.3000) -- (209.1000,269.0000) -- (209.3000,269.2000) -- (209.5000,269.1000) -- (209.7000,268.2000) -- (209.9000,266.4000) -- (210.1000,264.5000) -- (210.3000,266.7000) -- (210.5000,274.9000) -- (210.7000,284.3000) -- (210.9000,280.4000) -- (211.1000,277.6000) -- (211.2000,274.8000) -- (211.4000,272.3000) -- (211.6000,270.0000) -- (211.8000,268.0000) -- (212.0000,266.2000) -- (212.2000,264.6000) -- (212.4000,263.3000) -- (212.6000,262.1000) -- (212.8000,261.2000) -- (213.0000,260.6000) -- (213.2000,260.1000) -- (213.4000,259.7000) -- (213.6000,259.7000) -- (213.8000,259.8000) -- (214.0000,259.9000) -- (214.2000,260.0000) -- (214.4000,260.0000) -- (214.6000,260.0000) -- (214.8000,260.0000) -- (215.0000,260.0000) -- (215.2000,260.0000) -- (215.4000,260.0000) -- (215.6000,260.0000) -- (215.8000,260.0000) -- (215.9000,260.0000) -- (216.1000,260.0000) -- (216.3000,260.0000) -- (216.5000,260.1000) -- (216.7000,260.4000) -- (216.9000,260.9000) -- (217.1000,261.7000) -- (217.3000,262.6000) -- (217.5000,263.8000) -- (217.7000,265.1000) -- (217.9000,266.7000) -- (218.1000,268.4000) -- (218.3000,270.6000) -- (218.5000,268.0000) -- (218.7000,267.6000) -- (218.9000,270.6000) -- (219.1000,268.6000) -- (219.3000,268.5000) -- (219.5000,271.8000) -- (219.7000,269.9000) -- (219.9000,269.5000) -- (220.1000,272.8000) -- (220.3000,271.0000) -- (220.5000,269.5000) -- (220.7000,270.2000) -- (220.8000,270.0000) -- (221.0000,269.6000) -- (221.2000,269.4000) -- (221.4000,269.3000) -- (221.6000,269.3000) -- (221.8000,269.3000) -- (222.0000,269.3000) -- (222.2000,269.3000) -- (222.4000,269.3000) -- (222.6000,269.2000) -- (222.8000,269.9000) -- (223.0000,275.8000) -- (223.2000,276.3000) -- (223.4000,276.1000) -- (223.6000,276.3000) -- (223.8000,275.9000) -- (224.0000,274.7000) -- (224.2000,271.8000) -- (224.4000,279.4000) -- (224.6000,289.1000) -- (224.8000,285.2000) -- (225.0000,282.2000) -- (225.2000,279.3000) -- (225.4000,276.4000) -- (225.5000,273.8000) -- (225.7000,271.4000) -- (225.9000,269.2000) -- (226.1000,267.2000) -- (226.3000,265.5000) -- (226.5000,264.0000) -- (226.7000,262.8000) -- (226.9000,261.8000) -- (227.1000,260.9000) -- (227.3000,260.3000) -- (227.5000,259.9000) -- (227.7000,259.7000) -- (227.9000,259.7000) -- (228.1000,259.9000) -- (228.3000,259.9000) -- (228.5000,260.0000) -- (228.7000,260.0000) -- (228.9000,260.0000) -- (229.1000,260.0000) -- (229.3000,260.0000) -- (229.5000,260.0000) -- (229.7000,260.0000) -- (229.9000,260.0000) -- (230.1000,260.0000) -- (230.3000,260.0000) -- (230.4000,260.0000) -- (230.6000,260.0000) -- (230.8000,260.2000) -- (231.0000,260.6000) -- (231.2000,261.2000) -- (231.4000,262.0000) -- (231.6000,263.0000) -- (231.8000,264.3000) -- (232.0000,265.7000) -- (232.2000,267.3000) -- (232.4000,269.2000) -- (232.6000,271.2000) -- (232.8000,273.5000) -- (233.0000,276.0000) -- (233.2000,277.3000) -- (233.4000,274.3000) -- (233.6000,277.0000) -- (233.8000,278.6000) -- (234.0000,275.5000) -- (234.2000,277.9000) -- (234.4000,279.7000) -- (234.6000,276.2000) -- (234.8000,277.0000) -- (234.9000,277.2000) -- (235.1000,276.9000) -- (235.3000,276.6000) -- (235.5000,276.4000) -- (235.7000,276.3000) -- (235.9000,276.3000) -- (236.1000,276.3000) -- (236.3000,276.4000) -- (236.5000,276.3000) -- (236.7000,276.9000) -- (236.9000,282.8000) -- (237.1000,283.5000) -- (237.3000,283.4000) -- (237.5000,283.1000) -- (237.7000,283.0000) -- (237.9000,284.0000) -- (238.1000,295.1000) -- (238.3000,295.2000) -- (238.5000,292.8000) -- (238.7000,290.0000) -- (238.9000,287.1000) -- (239.1000,284.0000) -- (239.3000,281.0000) -- (239.5000,278.1000) -- (239.7000,275.4000) -- (239.8000,272.8000) -- (240.0000,270.5000) -- (240.2000,268.4000) -- (240.4000,266.5000) -- (240.6000,264.9000) -- (240.8000,263.5000) -- (241.0000,262.3000) -- (241.2000,261.4000) -- (241.4000,260.7000) -- (241.6000,260.1000) -- (241.8000,259.8000) -- (242.0000,259.7000) -- (242.2000,259.8000) -- (242.4000,259.9000) -- (242.6000,260.0000) -- (242.8000,260.0000) -- (243.0000,260.0000) -- (243.2000,260.0000) -- (243.4000,260.0000) -- (243.6000,260.0000) -- (243.8000,260.0000) -- (244.0000,260.0000) -- (244.2000,260.0000) -- (244.4000,260.0000) -- (244.5000,260.0000) -- (244.7000,260.0000) -- (244.9000,260.0000) -- (245.1000,260.3000) -- (245.3000,260.8000) -- (245.5000,261.5000) -- (245.7000,262.4000) -- (245.9000,263.5000) -- (246.1000,264.8000) -- (246.3000,266.3000) -- (246.5000,268.0000) -- (246.7000,270.0000) -- (246.9000,272.1000) -- (247.1000,274.4000) -- (247.3000,276.9000) -- (247.5000,279.5000) -- (247.7000,282.2000) -- (247.9000,285.3000) -- (248.1000,283.6000) -- (248.3000,283.1000) -- (248.5000,286.3000) -- (248.7000,284.7000) -- (248.9000,283.3000) -- (249.1000,284.2000) -- (249.3000,284.1000) -- (249.4000,283.8000) -- (249.6000,283.5000) -- (249.8000,283.4000) -- (250.0000,283.4000) -- (250.2000,283.4000) -- (250.4000,283.4000) -- (250.6000,283.4000) -- (250.8000,283.1000) -- (251.0000,286.8000) -- (251.2000,290.7000) -- (251.4000,290.1000) -- (251.6000,293.6000) -- (251.8000,297.4000);



  \end{scope}
  \begin{scope}[scale=1.006,draw=blue,line cap=rect,line join=bevel,line width=0.800pt]
  \end{scope}
  \begin{scope}[scale=1.006,draw=blue,line cap=rect,line join=bevel,line width=0.800pt]
  \end{scope}
  \begin{scope}[scale=1.006,draw=blue,line cap=rect,line join=bevel,line width=0.800pt]
  \end{scope}
  \begin{scope}[scale=1.006,draw=blue,line cap=rect,line join=bevel,line width=0.800pt]
  \end{scope}
  \begin{scope}[cm={{1.04164,0.0,0.0,1.04164,(-342.4953,42.14671)}},draw=c00ff00,dash pattern=on 0.48pt off 0.48pt,line cap=round,line join=round,line width=0.480pt,miter limit=4.00]
    \path[draw,dash pattern=on 0.48pt off 0.48pt,line width=0.480pt,miter limit=4.00] (127.3000,221.1000) -- (127.6000,221.2000) -- (127.9000,221.4000) -- (128.2000,221.5000) -- (128.5000,221.7000) -- (128.8000,221.8000) -- (129.1000,222.0000) -- (129.4000,222.1000) -- (129.7000,222.3000) -- (130.0000,222.4000) -- (130.3000,222.6000) -- (130.6000,222.7000) -- (130.9000,222.9000) -- (131.2000,223.0000) -- (131.5000,223.2000) -- (131.8000,223.3000) -- (132.1000,223.5000) -- (132.3000,223.6000) -- (132.6000,223.8000) -- (132.9000,223.9000) -- (133.2000,224.1000) -- (133.5000,224.2000) -- (133.8000,224.4000) -- (134.1000,224.5000) -- (134.4000,224.7000) -- (134.7000,224.8000) -- (135.0000,225.0000) -- (135.3000,225.1000) -- (135.6000,225.2000) -- (135.9000,225.4000) -- (136.2000,225.5000) -- (136.5000,225.7000) -- (136.8000,225.8000) -- (137.1000,226.0000) -- (137.4000,226.1000) -- (137.7000,226.3000) -- (138.0000,226.4000) -- (138.3000,226.6000) -- (138.6000,226.7000) -- (138.9000,226.9000) -- (139.2000,227.0000) -- (139.4000,227.2000) -- (139.7000,227.3000) -- (140.0000,227.5000) -- (140.3000,227.6000) -- (140.6000,227.8000) -- (140.9000,227.9000) -- (141.2000,228.1000) -- (141.5000,228.2000) -- (141.8000,228.4000) -- (142.1000,228.5000) -- (142.4000,228.7000) -- (142.7000,228.8000) -- (143.0000,229.0000) -- (143.3000,229.1000) -- (143.6000,229.3000) -- (143.9000,229.4000) -- (144.2000,229.6000) -- (144.5000,229.7000) -- (144.8000,229.8000) -- (145.1000,230.0000) -- (145.4000,230.1000) -- (145.7000,230.3000) -- (146.0000,230.4000) -- (146.3000,230.6000) -- (146.6000,230.7000) -- (146.8000,230.9000) -- (147.1000,231.0000) -- (147.4000,231.2000) -- (147.7000,231.3000) -- (148.0000,231.5000) -- (148.3000,231.6000) -- (148.6000,231.8000) -- (148.9000,231.9000) -- (149.2000,232.1000) -- (149.5000,232.2000) -- (149.8000,232.4000) -- (150.1000,232.5000) -- (150.4000,232.7000) -- (150.7000,232.8000) -- (151.0000,233.0000) -- (151.3000,233.1000) -- (151.6000,233.3000) -- (151.9000,233.4000) -- (152.2000,233.6000) -- (152.5000,233.7000) -- (152.8000,233.9000) -- (153.1000,234.0000) -- (153.4000,234.1000) -- (153.7000,234.3000) -- (153.9000,234.4000) -- (154.2000,234.6000) -- (154.5000,234.7000) -- (154.8000,234.9000) -- (155.1000,235.0000) -- (155.4000,235.2000) -- (155.7000,235.3000) -- (156.0000,235.5000) -- (156.3000,235.6000) -- (156.6000,235.8000) -- (156.9000,235.9000) -- (157.2000,236.1000) -- (157.5000,236.2000) -- (157.8000,236.4000) -- (158.1000,236.5000) -- (158.4000,236.7000) -- (158.7000,236.8000) -- (159.0000,237.0000) -- (159.3000,237.1000) -- (159.6000,237.3000) -- (159.9000,237.4000) -- (160.2000,237.6000) -- (160.5000,237.7000) -- (160.8000,237.9000) -- (161.0000,238.0000) -- (161.3000,238.2000) -- (161.6000,238.3000) -- (161.9000,238.5000) -- (162.2000,238.6000) -- (162.5000,238.6000) -- (162.8000,238.7000) -- (163.1000,238.9000) -- (163.4000,239.0000) -- (163.7000,239.2000) -- (164.0000,239.3000) -- (164.3000,239.5000) -- (164.6000,239.6000) -- (164.9000,239.8000) -- (165.2000,239.9000) -- (165.5000,240.1000) -- (165.8000,240.2000) -- (166.1000,240.4000) -- (166.4000,240.5000) -- (166.7000,240.7000) -- (167.0000,240.8000) -- (167.3000,241.0000) -- (167.6000,241.1000) -- (167.9000,241.3000) -- (168.2000,241.4000) -- (168.4000,241.6000) -- (168.7000,241.7000) -- (169.0000,241.9000) -- (169.3000,242.0000) -- (169.6000,242.2000) -- (169.9000,242.3000) -- (170.2000,242.5000) -- (170.5000,242.6000) -- (170.8000,242.8000) -- (171.1000,242.9000) -- (171.4000,243.1000) -- (171.7000,243.2000) -- (172.0000,243.3000) -- (172.3000,243.5000) -- (172.6000,243.6000) -- (172.9000,243.8000) -- (173.2000,243.9000) -- (173.5000,244.1000) -- (173.8000,244.2000) -- (174.1000,244.4000) -- (174.4000,244.5000) -- (174.7000,244.7000) -- (175.0000,244.8000) -- (175.3000,245.0000) -- (175.5000,245.1000) -- (175.8000,245.3000) -- (176.1000,245.4000) -- (176.4000,245.6000) -- (176.7000,245.7000) -- (177.0000,245.9000) -- (177.3000,246.0000) -- (177.6000,246.2000) -- (177.9000,246.3000) -- (178.2000,246.5000) -- (178.5000,246.6000) -- (178.8000,246.8000) -- (179.1000,246.9000) -- (179.4000,247.1000) -- (179.7000,247.2000) -- (180.0000,247.4000) -- (180.3000,247.5000) -- (180.6000,247.6000) -- (180.9000,247.8000) -- (181.2000,247.9000) -- (181.5000,248.1000) -- (181.8000,248.2000) -- (182.1000,248.4000) -- (182.4000,248.5000) -- (182.7000,248.7000) -- (182.9000,248.8000) -- (183.2000,249.0000) -- (183.5000,249.1000) -- (183.8000,249.3000) -- (184.1000,249.4000) -- (184.4000,249.6000) -- (184.7000,249.7000) -- (185.0000,249.9000) -- (185.3000,250.0000) -- (185.6000,250.2000) -- (185.9000,250.3000) -- (186.2000,250.5000) -- (186.5000,250.6000) -- (186.8000,250.8000) -- (187.1000,250.9000) -- (187.4000,251.1000) -- (187.7000,251.2000) -- (188.0000,251.4000) -- (188.3000,251.5000) -- (188.6000,251.7000) -- (188.9000,251.8000) -- (189.2000,252.0000) -- (189.5000,252.1000) -- (189.8000,252.2000) -- (190.0000,252.4000) -- (190.3000,252.5000) -- (190.6000,252.7000) -- (190.9000,252.8000) -- (191.2000,253.0000) -- (191.5000,253.1000) -- (191.8000,253.3000) -- (192.1000,253.4000) -- (192.4000,253.6000) -- (192.7000,253.7000) -- (193.0000,253.9000) -- (193.3000,254.0000) -- (193.6000,254.2000) -- (193.9000,254.3000) -- (194.2000,254.5000) -- (194.5000,254.6000) -- (194.8000,254.8000) -- (195.1000,254.9000) -- (195.4000,255.1000) -- (195.7000,255.2000) -- (196.0000,255.4000) -- (196.3000,255.5000) -- (196.6000,255.7000) -- (196.9000,255.8000) -- (197.2000,256.0000) -- (197.4000,256.1000) -- (197.7000,256.3000) -- (198.0000,256.4000) -- (198.3000,256.5000) -- (198.6000,256.7000) -- (198.9000,256.8000) -- (199.2000,257.0000) -- (199.5000,257.1000) -- (199.8000,257.3000) -- (200.1000,257.4000) -- (200.4000,257.6000) -- (200.7000,257.7000) -- (201.0000,257.9000) -- (201.3000,258.0000) -- (201.6000,258.2000) -- (201.9000,258.3000) -- (202.2000,258.5000) -- (202.5000,258.6000) -- (202.8000,258.8000) -- (203.1000,258.9000) -- (203.4000,259.1000) -- (203.7000,259.2000) -- (204.0000,259.4000) -- (204.3000,259.5000) -- (204.5000,259.7000) -- (204.8000,259.8000) -- (205.1000,260.0000) -- (205.4000,260.1000) -- (205.7000,260.3000) -- (206.0000,260.4000) -- (206.3000,260.6000) -- (206.6000,260.7000) -- (206.9000,260.9000) -- (207.2000,261.0000) -- (207.5000,261.1000) -- (207.8000,261.3000) -- (208.1000,261.4000) -- (208.4000,261.6000) -- (208.7000,261.7000) -- (209.0000,261.9000) -- (209.3000,262.0000) -- (209.6000,262.2000) -- (209.9000,262.3000) -- (210.2000,262.5000) -- (210.5000,262.6000) -- (210.8000,262.8000) -- (211.1000,262.9000) -- (211.4000,263.1000) -- (211.7000,263.2000) -- (211.9000,263.4000) -- (212.2000,263.5000) -- (212.5000,263.7000) -- (212.8000,263.8000) -- (213.1000,264.0000) -- (213.4000,264.1000) -- (213.7000,264.3000) -- (214.0000,264.4000) -- (214.3000,264.6000) -- (214.6000,264.7000) -- (214.9000,264.9000) -- (215.2000,265.0000) -- (215.5000,265.2000) -- (215.8000,265.3000) -- (216.1000,265.5000) -- (216.4000,265.6000) -- (216.7000,265.7000) -- (217.0000,265.9000) -- (217.3000,266.0000) -- (217.6000,266.2000) -- (217.9000,266.3000) -- (218.2000,266.5000) -- (218.5000,266.6000) -- (218.8000,266.8000) -- (219.0000,266.9000) -- (219.3000,267.1000) -- (219.6000,267.2000) -- (219.9000,267.4000) -- (220.2000,267.5000) -- (220.5000,267.7000) -- (220.8000,267.8000) -- (221.1000,268.0000) -- (221.4000,268.1000) -- (221.7000,268.3000) -- (222.0000,268.4000) -- (222.3000,268.6000) -- (222.6000,268.7000) -- (222.9000,268.9000) -- (223.2000,269.0000) -- (223.5000,269.2000) -- (223.8000,269.3000) -- (224.1000,269.5000) -- (224.4000,269.6000) -- (224.7000,269.8000) -- (225.0000,269.9000) -- (225.3000,270.0000) -- (225.6000,270.2000) -- (225.9000,270.3000) -- (226.1000,270.5000) -- (226.4000,270.6000) -- (226.7000,270.8000) -- (227.0000,270.9000) -- (227.3000,271.1000) -- (227.6000,271.2000) -- (227.9000,271.4000) -- (228.2000,271.5000) -- (228.5000,271.7000) -- (228.8000,271.8000) -- (229.1000,272.0000) -- (229.4000,272.1000) -- (229.7000,272.3000) -- (230.0000,272.4000) -- (230.3000,272.6000) -- (230.6000,272.7000) -- (230.9000,272.9000) -- (231.2000,273.0000) -- (231.5000,273.2000) -- (231.8000,273.3000) -- (232.1000,273.5000) -- (232.4000,273.6000) -- (232.7000,273.8000) -- (233.0000,273.9000) -- (233.3000,274.1000) -- (233.5000,274.2000) -- (233.8000,274.4000) -- (234.1000,274.5000) -- (234.4000,274.6000) -- (234.7000,274.8000) -- (235.0000,274.9000) -- (235.3000,275.1000) -- (235.6000,275.2000) -- (235.9000,275.4000) -- (236.2000,275.5000) -- (236.5000,275.7000) -- (236.8000,275.8000) -- (237.1000,276.0000) -- (237.4000,276.1000) -- (237.7000,276.3000) -- (238.0000,276.4000) -- (238.3000,276.6000) -- (238.6000,276.7000) -- (238.9000,276.9000) -- (239.2000,277.0000) -- (239.5000,277.2000) -- (239.8000,277.3000) -- (240.1000,277.5000) -- (240.4000,277.6000) -- (240.6000,277.8000) -- (240.9000,277.9000) -- (241.2000,278.1000) -- (241.5000,278.2000) -- (241.8000,278.4000) -- (242.1000,278.5000) -- (242.4000,278.7000) -- (242.7000,278.8000) -- (243.0000,279.0000) -- (243.3000,279.1000) -- (243.6000,279.2000) -- (243.9000,279.4000) -- (244.2000,279.5000) -- (244.5000,279.7000) -- (244.8000,279.8000) -- (245.1000,280.0000) -- (245.4000,280.1000) -- (245.7000,280.3000) -- (246.0000,280.4000) -- (246.3000,280.6000) -- (246.6000,280.7000) -- (246.9000,280.9000) -- (247.2000,281.0000) -- (247.5000,281.2000) -- (247.8000,281.3000) -- (248.0000,281.5000) -- (248.3000,281.6000) -- (248.6000,281.8000) -- (248.9000,281.9000) -- (249.2000,282.1000) -- (249.5000,282.2000) -- (249.8000,282.4000) -- (250.1000,282.5000) -- (250.4000,282.7000) -- (250.7000,282.8000) -- (251.0000,283.0000) -- (251.3000,283.1000) -- (251.6000,283.3000) -- (251.9000,283.4000);



  \end{scope}
  \begin{scope}[scale=1.006,draw=blue,line cap=rect,line join=bevel,line width=0.800pt]
  \end{scope}
  \begin{scope}[draw=blue,line cap=rect,line join=bevel,line width=0.800pt]
  \end{scope}
  \begin{scope}[cm={{0.0,-1.00588,1.00588,0.0,(-600.28902,263.45539)}},draw=blue,line cap=rect,line join=bevel,line width=0.800pt]
    \path[fill=blue] (0.0000,0.0000) node[above right] (text344) {\rotatebox{90}{y (m)}};



  \end{scope}
  \begin{scope}[cm={{0.84173,0.0,0.0,0.84173,(-601.60573,125.64086)}},fill=cffffff]
  \end{scope}
  \begin{scope}[cm={{0.92047,0.0,0.0,0.92047,(-569.12952,57.84001)}},draw=ca0a0a4,dash pattern=on 1.96pt off 1.96pt,line cap=round,line join=round,line width=0.326pt,miter limit=4.00]
    \path[shift={(0,-5.96493)},draw,dash pattern=on 1.96pt off 1.96pt,line width=0.326pt,miter limit=4.00] (184.5000,151.5000) -- (246.5000,151.5000);



  \end{scope}
  \begin{scope}[cm={{0.92047,0.0,0.0,0.92047,(-569.12952,52.34945)}},draw=blue,line cap=round,line join=round,line width=0.480pt]
    \path[draw] (184.5000,151.5000) -- (186.5000,151.5000);



    \path[draw] (246.5000,151.5000) -- (245.5000,151.5000);



  \end{scope}
  \begin{scope}[cm={{0.92047,0.0,0.0,0.92047,(-569.12952,52.34945)}},draw=ca0a0a4,dash pattern=on 1.96pt off 1.96pt,line cap=round,line join=round,line width=0.326pt,miter limit=4.00]
    \path[draw,dash pattern=on 1.96pt off 1.96pt,line width=0.326pt,miter limit=4.00] (184.5000,132.5000) -- (246.5000,132.5000);



  \end{scope}
  \begin{scope}[cm={{0.92047,0.0,0.0,0.92047,(-569.12952,52.34945)}},draw=blue,line cap=round,line join=round,line width=0.480pt]
    \path[draw] (184.5000,132.5000) -- (186.5000,132.5000);



    \path[draw] (246.5000,132.5000) -- (245.5000,132.5000);



  \end{scope}
  \begin{scope}[cm={{0.92047,0.0,0.0,0.92047,(-569.12952,52.34945)}},draw=ca0a0a4,dash pattern=on 1.96pt off 1.96pt,line cap=round,line join=round,line width=0.326pt,miter limit=4.00]
    \path[draw,dash pattern=on 1.96pt off 1.96pt,line width=0.326pt,miter limit=4.00] (184.5000,113.5000) -- (246.5000,113.5000);



  \end{scope}
  \begin{scope}[cm={{0.92047,0.0,0.0,0.92047,(-569.12952,57.84001)}},draw=blue,line cap=round,line join=round,line width=0.480pt]
    \path[shift={(0,-5.96493)},draw] (184.5000,113.5000) -- (186.5000,113.5000);



    \path[shift={(0,-5.96493)},draw] (246.5000,113.5000) -- (245.5000,113.5000);



  \end{scope}
  \begin{scope}[cm={{0.92047,0.0,0.0,0.92047,(-569.12952,52.34945)}},draw=ca0a0a4,dash pattern=on 0.40pt off 0.80pt,line cap=round,line join=round,line width=0.400pt]
    \path[draw] (184.5000,156.5000) -- (184.5000,108.5000);



  \end{scope}
  \begin{scope}[cm={{0.92047,0.0,0.0,0.92047,(-569.12952,52.34945)}},draw=blue,line cap=round,line join=round,line width=0.480pt]
    \path[draw] (184.5000,156.5000) -- (184.5000,155.5000);



    \path[draw] (184.5000,108.5000) -- (184.5000,109.5000);



  \end{scope}
  \begin{scope}[cm={{0.92047,0.0,0.0,0.92047,(-569.12952,52.34945)}},draw=ca0a0a4,dash pattern=on 1.96pt off 1.96pt,line cap=round,line join=round,line width=0.326pt,miter limit=4.00]
    \path[draw,dash pattern=on 1.96pt off 1.96pt,line width=0.326pt,miter limit=4.00] (203.5000,156.5000) -- (203.5000,108.5000);



  \end{scope}
  \begin{scope}[cm={{0.92047,0.0,0.0,0.92047,(-569.12952,52.34945)}},draw=blue,line cap=round,line join=round,line width=0.480pt]
    \path[draw] (203.5000,156.5000) -- (203.5000,155.5000);



    \path[draw] (203.5000,108.5000) -- (203.5000,109.5000);



  \end{scope}
  \begin{scope}[cm={{0.92047,0.0,0.0,0.92047,(-569.12952,52.34945)}},draw=ca0a0a4,dash pattern=on 1.96pt off 1.96pt,line cap=round,line join=round,line width=0.326pt,miter limit=4.00]
    \path[draw,dash pattern=on 1.96pt off 1.96pt,line width=0.326pt,miter limit=4.00] (221.5000,156.5000) -- (221.5000,108.5000);



  \end{scope}
  \begin{scope}[cm={{0.92047,0.0,0.0,0.92047,(-569.12952,52.34945)}},draw=blue,line cap=round,line join=round,line width=0.480pt]
    \path[draw] (221.5000,156.5000) -- (221.5000,155.5000);



    \path[draw] (221.5000,108.5000) -- (221.5000,109.5000);



  \end{scope}
  \begin{scope}[cm={{0.92047,0.0,0.0,0.92047,(-569.12952,52.34945)}},draw=ca0a0a4,dash pattern=on 1.96pt off 1.96pt,line cap=round,line join=round,line width=0.326pt,miter limit=4.00]
    \path[draw,dash pattern=on 1.96pt off 1.96pt,line width=0.326pt,miter limit=4.00] (239.5000,156.5000) -- (239.5000,108.5000);



  \end{scope}
  \begin{scope}[cm={{0.92047,0.0,0.0,0.92047,(-569.12952,52.34945)}},draw=blue,line cap=round,line join=round,line width=0.480pt]
    \path[draw] (239.5000,156.5000) -- (239.5000,155.5000);



    \path[draw] (239.5000,108.5000) -- (239.5000,109.5000);



  \end{scope}
  \begin{scope}[cm={{0.92047,0.0,0.0,0.92047,(-569.12952,57.84001)}},draw=blue,line cap=round,line join=round,line width=0.480pt]
    \path[shift={(0,-5.96493)},draw] (246.5000,151.5000) -- (245.5000,151.5000);



  \end{scope}
  \begin{scope}[cm={{0.92047,0.0,0.0,0.92047,(-569.12952,57.84001)}},draw=blue,line cap=round,line join=round,line width=0.480pt]
    \path[shift={(0,-5.96493)},draw] (246.5000,132.5000) -- (245.5000,132.5000);



  \end{scope}
  \begin{scope}[cm={{0.92047,0.0,0.0,0.92047,(-569.12952,52.34945)}},draw=blue,line cap=round,line join=round,line width=0.480pt]
    \path[draw] (246.5000,113.5000) -- (245.5000,113.5000);



  \end{scope}
  \begin{scope}[cm={{0.92047,0.0,0.0,0.92047,(-569.12952,52.34945)}},draw=blue,line cap=round,line join=round,line width=0.480pt]
    \path[draw] (184.5000,108.5000) -- (184.5000,156.5000) -- (246.5000,156.5000) -- (246.5000,108.5000) -- (184.5000,108.5000);



  \end{scope}
  \begin{scope}[cm={{0.92047,0.0,0.0,0.92047,(-569.12952,52.34945)}},draw=blue,line cap=round,line join=round,line width=0.480pt]
    \path[draw] (184.8000,113.6000) -- (184.8000,113.6000) -- (184.9000,113.6000) -- (185.0000,113.6000) -- (185.0000,113.6000) -- (185.1000,113.6000) -- (185.1000,113.6000) -- (185.2000,113.6000) -- (185.3000,113.6000) -- (185.3000,113.6000) -- (185.4000,113.6000) -- (185.4000,113.6000) -- (185.5000,113.6000) -- (185.6000,113.6000) -- (185.6000,113.6000) -- (185.7000,113.6000) -- (185.8000,113.6000) -- (185.8000,113.6000) -- (185.9000,113.6000) -- (185.9000,113.6000) -- (186.0000,113.6000) -- (186.1000,113.6000) -- (186.1000,113.6000) -- (186.2000,113.6000) -- (186.2000,113.6000) -- (186.3000,113.6000) -- (186.4000,113.6000) -- (186.4000,113.6000) -- (186.5000,113.6000) -- (186.6000,113.6000) -- (186.6000,113.6000) -- (186.7000,113.6000) -- (186.7000,113.6000) -- (186.8000,113.6000) -- (186.9000,113.6000) -- (186.9000,113.6000) -- (187.0000,113.6000) -- (187.0000,113.6000) -- (187.1000,113.6000) -- (187.2000,113.6000) -- (187.2000,113.6000) -- (187.3000,113.6000) -- (187.4000,113.6000) -- (187.4000,113.6000) -- (187.5000,113.6000) -- (187.5000,113.6000) -- (187.6000,113.6000) -- (187.7000,113.6000) -- (187.7000,113.6000) -- (187.8000,113.6000) -- (187.8000,113.6000) -- (187.9000,113.6000) -- (188.0000,113.6000) -- (188.0000,113.6000) -- (188.1000,113.6000) -- (188.2000,113.6000) -- (188.2000,113.6000) -- (188.3000,113.6000) -- (188.3000,113.6000) -- (188.4000,113.6000) -- (188.5000,113.6000) -- (188.5000,113.6000) -- (188.6000,113.6000) -- (188.6000,113.6000) -- (188.7000,113.6000) -- (188.8000,113.6000) -- (188.8000,113.6000) -- (188.9000,113.6000) -- (189.0000,113.6000) -- (189.0000,113.6000) -- (189.1000,113.6000) -- (189.1000,113.6000) -- (189.2000,113.6000) -- (189.3000,113.6000) -- (189.3000,113.6000) -- (189.4000,113.6000) -- (189.4000,113.6000) -- (189.5000,113.6000) -- (189.6000,113.6000) -- (189.6000,113.6000) -- (189.7000,113.6000) -- (189.8000,113.6000) -- (189.8000,113.6000) -- (189.9000,113.6000) -- (189.9000,113.6000) -- (190.0000,113.6000) -- (190.1000,113.6000) -- (190.1000,113.6000) -- (190.2000,113.6000) -- (190.2000,113.6000) -- (190.3000,113.6000) -- (190.4000,113.6000) -- (190.4000,113.6000) -- (190.5000,113.6000) -- (190.6000,113.6000) -- (190.6000,113.6000) -- (190.7000,113.6000) -- (190.7000,113.6000) -- (190.8000,113.6000) -- (190.9000,113.6000) -- (190.9000,113.6000) -- (191.0000,113.6000) -- (191.0000,113.6000) -- (191.1000,113.6000) -- (191.2000,113.6000) -- (191.2000,113.6000) -- (191.3000,113.6000) -- (191.4000,113.6000) -- (191.4000,113.6000) -- (191.5000,113.6000) -- (191.5000,113.6000) -- (191.6000,113.6000) -- (191.7000,113.6000) -- (191.7000,113.6000) -- (191.8000,113.6000) -- (191.8000,113.6000) -- (191.9000,113.6000) -- (192.0000,113.6000) -- (192.0000,113.6000) -- (192.1000,113.6000) -- (192.2000,113.6000) -- (192.2000,113.6000) -- (192.3000,113.6000) -- (192.3000,113.6000) -- (192.4000,113.6000) -- (192.5000,113.6000) -- (192.5000,113.6000) -- (192.6000,113.6000) -- (192.6000,113.6000) -- (192.7000,113.6000) -- (192.8000,113.6000) -- (192.8000,113.6000) -- (192.9000,113.6000) -- (193.0000,113.6000) -- (193.0000,113.6000) -- (193.1000,113.6000) -- (193.1000,113.6000) -- (193.2000,113.6000) -- (193.3000,113.6000) -- (193.3000,113.6000) -- (193.4000,113.6000) -- (193.4000,113.6000) -- (193.5000,113.6000) -- (193.6000,113.6000) -- (193.6000,113.6000) -- (193.7000,113.6000) -- (193.8000,113.6000) -- (193.8000,113.6000) -- (193.9000,113.6000) -- (193.9000,113.6000) -- (194.0000,113.6000) -- (194.1000,113.6000) -- (194.1000,113.6000) -- (194.2000,113.6000) -- (194.2000,113.6000) -- (194.3000,113.6000) -- (194.4000,113.6000) -- (194.4000,113.6000) -- (194.5000,113.6000) -- (194.6000,113.6000) -- (194.6000,113.6000) -- (194.7000,113.6000) -- (194.7000,113.6000) -- (194.8000,113.6000) -- (194.9000,113.6000) -- (194.9000,113.6000) -- (195.0000,113.6000) -- (195.0000,113.6000) -- (195.1000,113.6000) -- (195.2000,113.6000) -- (195.2000,113.6000) -- (195.3000,113.6000) -- (195.4000,113.6000) -- (195.4000,113.6000) -- (195.5000,113.6000) -- (195.5000,113.6000) -- (195.6000,113.6000) -- (195.7000,113.6000) -- (195.7000,113.6000) -- (195.8000,113.6000) -- (195.8000,113.6000) -- (195.9000,113.6000) -- (196.0000,113.6000) -- (196.0000,113.6000) -- (196.1000,113.6000) -- (196.2000,113.6000) -- (196.2000,113.6000) -- (196.3000,113.6000) -- (196.3000,113.6000) -- (196.4000,113.6000) -- (196.5000,113.6000) -- (196.5000,113.6000) -- (196.6000,113.6000) -- (196.6000,113.6000) -- (196.7000,113.6000) -- (196.8000,113.6000) -- (196.8000,113.6000) -- (196.9000,113.6000) -- (196.9000,113.6000) -- (197.0000,113.6000) -- (197.1000,113.6000) -- (197.1000,113.6000) -- (197.2000,113.6000) -- (197.3000,113.6000) -- (197.3000,113.6000) -- (197.4000,113.6000) -- (197.4000,113.6000) -- (197.5000,113.6000) -- (197.6000,113.6000) -- (197.6000,113.6000) -- (197.7000,113.6000) -- (197.7000,113.6000) -- (197.8000,113.6000) -- (197.9000,113.6000) -- (197.9000,113.6000) -- (198.0000,113.6000) -- (198.1000,113.6000) -- (198.1000,113.6000) -- (198.2000,113.6000) -- (198.2000,113.6000) -- (198.3000,113.6000) -- (198.4000,113.6000) -- (198.4000,113.6000) -- (198.5000,113.6000) -- (198.5000,113.6000) -- (198.6000,113.6000) -- (198.7000,113.6000) -- (198.7000,113.6000) -- (198.8000,113.6000) -- (198.9000,113.6000) -- (198.9000,113.6000) -- (199.0000,113.6000) -- (199.0000,113.6000) -- (199.1000,113.6000) -- (199.2000,113.6000) -- (199.2000,113.6000) -- (199.3000,113.6000) -- (199.3000,113.6000) -- (199.4000,113.6000) -- (199.5000,113.6000) -- (199.5000,113.6000) -- (199.6000,113.6000) -- (199.7000,113.6000) -- (199.7000,113.6000) -- (199.8000,113.6000) -- (199.8000,113.6000) -- (199.9000,113.6000) -- (200.0000,113.6000) -- (200.0000,113.6000) -- (200.1000,113.6000) -- (200.1000,113.6000) -- (200.2000,113.6000) -- (200.3000,113.6000) -- (200.3000,113.6000) -- (200.4000,113.6000) -- (200.5000,113.6000) -- (200.5000,113.6000) -- (200.6000,113.6000) -- (200.6000,113.6000) -- (200.7000,113.6000) -- (200.8000,113.6000) -- (200.8000,113.6000) -- (200.9000,113.6000) -- (200.9000,113.6000) -- (201.0000,113.6000) -- (201.1000,113.6000) -- (201.1000,113.6000) -- (201.2000,113.6000) -- (201.3000,113.6000) -- (201.3000,113.6000) -- (201.4000,113.6000) -- (201.4000,113.6000) -- (201.5000,113.6000) -- (201.6000,113.6000) -- (201.6000,113.6000) -- (201.7000,113.6000) -- (201.7000,113.6000) -- (201.8000,113.6000) -- (201.9000,113.6000) -- (201.9000,113.6000) -- (202.0000,113.6000) -- (202.1000,113.6000) -- (202.1000,113.6000) -- (202.2000,113.6000) -- (202.2000,113.6000) -- (202.3000,113.6000) -- (202.4000,113.6000) -- (202.4000,113.6000) -- (202.5000,113.6000) -- (202.5000,113.6000) -- (202.6000,113.6000) -- (202.7000,113.6000) -- (202.7000,113.6000) -- (202.8000,113.6000) -- (202.9000,113.6000) -- (202.9000,113.6000) -- (203.0000,113.6000) -- (203.0000,113.6000) -- (203.1000,113.6000) -- (203.2000,113.6000) -- (203.2000,113.8000) -- (203.3000,121.5000) -- (203.3000,123.4000) -- (203.4000,123.1000) -- (203.5000,123.1000) -- (203.5000,123.1000) -- (203.6000,123.1000) -- (203.7000,123.1000) -- (203.7000,123.1000) -- (203.8000,123.1000) -- (203.8000,123.1000) -- (203.9000,123.1000) -- (204.0000,123.1000) -- (204.0000,123.1000) -- (204.1000,123.1000) -- (204.1000,123.1000) -- (204.2000,123.1000) -- (204.3000,123.1000) -- (204.3000,123.1000) -- (204.4000,123.1000) -- (204.5000,123.1000) -- (204.5000,123.1000) -- (204.6000,123.1000) -- (204.6000,123.1000) -- (204.7000,123.1000) -- (204.8000,123.1000) -- (204.8000,123.1000) -- (204.9000,122.8000) -- (204.9000,128.6000) -- (205.0000,132.8000) -- (205.1000,134.0000) -- (205.1000,141.8000) -- (205.2000,142.2000) -- (205.3000,142.1000) -- (205.3000,142.1000) -- (205.4000,142.1000) -- (205.4000,142.1000) -- (205.5000,141.8000) -- (205.6000,148.4000) -- (205.6000,152.0000) -- (205.7000,151.6000) -- (205.7000,151.6000) -- (205.8000,151.6000) -- (205.9000,151.6000) -- (205.9000,151.6000) -- (206.0000,151.6000) -- (206.1000,151.6000) -- (206.1000,151.6000) -- (206.2000,151.6000) -- (206.2000,151.9000) -- (206.3000,149.1000) -- (206.4000,142.0000) -- (206.4000,142.1000) -- (206.5000,142.1000) -- (206.5000,142.1000) -- (206.6000,142.1000) -- (206.7000,142.2000) -- (206.7000,142.2000) -- (206.8000,134.6000) -- (206.9000,132.4000) -- (206.9000,132.6000) -- (207.0000,132.6000) -- (207.0000,132.8000) -- (207.1000,140.5000) -- (207.2000,142.4000) -- (207.2000,142.1000) -- (207.3000,142.1000) -- (207.3000,142.1000) -- (207.4000,142.1000) -- (207.5000,142.1000) -- (207.5000,142.1000) -- (207.6000,142.1000) -- (207.7000,142.1000) -- (207.7000,142.1000) -- (207.8000,142.1000) -- (207.8000,142.1000) -- (207.9000,142.1000) -- (208.0000,142.1000) -- (208.0000,142.1000) -- (208.1000,142.1000) -- (208.1000,142.1000) -- (208.2000,142.1000) -- (208.3000,142.1000) -- (208.3000,142.1000) -- (208.4000,142.5000) -- (208.5000,137.1000) -- (208.5000,132.4000) -- (208.6000,131.6000) -- (208.6000,123.7000) -- (208.7000,123.1000) -- (208.8000,123.1000) -- (208.8000,123.3000) -- (208.9000,121.9000) -- (208.9000,114.0000) -- (209.0000,113.6000) -- (209.1000,113.6000) -- (209.1000,113.6000) -- (209.2000,113.6000) -- (209.3000,113.6000) -- (209.3000,113.6000) -- (209.4000,113.6000) -- (209.4000,113.4000) -- (209.5000,115.5000) -- (209.6000,123.1000) -- (209.6000,123.2000) -- (209.7000,123.1000) -- (209.7000,122.9000) -- (209.8000,125.3000) -- (209.9000,132.7000) -- (209.9000,132.5000) -- (210.0000,139.4000) -- (210.1000,142.5000) -- (210.1000,142.1000) -- (210.2000,142.1000) -- (210.2000,142.0000) -- (210.3000,149.3000) -- (210.4000,151.9000) -- (210.4000,151.6000) -- (210.5000,151.6000) -- (210.5000,151.6000) -- (210.6000,151.6000) -- (210.7000,151.6000) -- (210.7000,151.6000) -- (210.8000,151.6000) -- (210.9000,151.6000) -- (210.9000,151.6000) -- (211.0000,151.6000) -- (211.0000,151.6000) -- (211.1000,151.6000) -- (211.2000,151.6000) -- (211.2000,151.6000) -- (211.3000,151.6000) -- (211.3000,151.6000) -- (211.4000,151.6000) -- (211.5000,151.6000) -- (211.5000,151.6000) -- (211.6000,151.6000) -- (211.7000,151.6000) -- (211.7000,151.6000) -- (211.8000,151.6000) -- (211.8000,151.6000) -- (211.9000,151.6000) -- (212.0000,151.6000) -- (212.0000,151.6000) -- (212.1000,151.6000) -- (212.1000,151.6000) -- (212.2000,151.6000) -- (212.3000,151.6000) -- (212.3000,151.6000) -- (212.4000,151.6000) -- (212.5000,151.6000) -- (212.5000,151.6000) -- (212.6000,151.6000) -- (212.6000,151.6000) -- (212.7000,151.6000) -- (212.8000,151.6000) -- (212.8000,151.6000) -- (212.9000,151.6000) -- (212.9000,151.6000) -- (213.0000,151.6000) -- (213.1000,151.6000) -- (213.1000,151.6000) -- (213.2000,151.6000) -- (213.3000,151.6000) -- (213.3000,151.6000) -- (213.4000,151.6000) -- (213.4000,151.6000) -- (213.5000,151.6000) -- (213.6000,151.6000) -- (213.6000,151.6000) -- (213.7000,151.6000) -- (213.7000,151.6000) -- (213.8000,151.6000) -- (213.9000,151.6000) -- (213.9000,151.6000) -- (214.0000,151.6000) -- (214.1000,151.6000) -- (214.1000,151.6000) -- (214.2000,151.6000) -- (214.2000,151.6000) -- (214.3000,151.6000) -- (214.4000,151.6000) -- (214.4000,151.6000) -- (214.5000,151.6000) -- (214.5000,151.6000) -- (214.6000,151.6000) -- (214.7000,151.6000) -- (214.7000,151.6000) -- (214.8000,151.6000) -- (214.9000,151.6000) -- (214.9000,151.6000) -- (215.0000,151.6000) -- (215.0000,151.6000) -- (215.1000,151.6000) -- (215.2000,151.6000) -- (215.2000,151.6000) -- (215.3000,151.6000) -- (215.3000,151.6000) -- (215.4000,151.6000) -- (215.5000,151.6000) -- (215.5000,151.6000) -- (215.6000,151.6000) -- (215.7000,151.6000) -- (215.7000,151.6000) -- (215.8000,151.6000) -- (215.8000,151.6000) -- (215.9000,151.6000) -- (216.0000,151.6000) -- (216.0000,151.6000) -- (216.1000,151.6000) -- (216.1000,151.6000) -- (216.2000,151.6000) -- (216.3000,151.6000) -- (216.3000,151.6000) -- (216.4000,151.6000) -- (216.5000,151.6000) -- (216.5000,151.6000) -- (216.6000,151.6000) -- (216.6000,151.6000) -- (216.7000,151.6000) -- (216.8000,151.6000) -- (216.8000,151.6000) -- (216.9000,151.6000) -- (216.9000,151.6000) -- (217.0000,151.6000) -- (217.1000,151.6000) -- (217.1000,151.6000) -- (217.2000,151.6000) -- (217.3000,151.6000) -- (217.3000,151.6000) -- (217.4000,151.6000) -- (217.4000,151.6000) -- (217.5000,151.6000) -- (217.6000,151.6000) -- (217.6000,151.6000) -- (217.7000,151.6000) -- (217.7000,151.6000) -- (217.8000,151.6000) -- (217.9000,151.6000) -- (217.9000,151.6000) -- (218.0000,151.6000) -- (218.1000,151.6000) -- (218.1000,151.6000) -- (218.2000,151.6000) -- (218.2000,151.6000) -- (218.3000,151.6000) -- (218.4000,151.6000) -- (218.4000,151.6000) -- (218.5000,151.6000) -- (218.5000,151.6000) -- (218.6000,151.6000) -- (218.7000,151.6000) -- (218.7000,151.6000) -- (218.8000,151.6000) -- (218.9000,151.6000) -- (218.9000,151.6000) -- (219.0000,151.6000) -- (219.0000,151.6000) -- (219.1000,151.6000) -- (219.2000,151.6000) -- (219.2000,151.6000) -- (219.3000,151.6000) -- (219.3000,151.6000) -- (219.4000,151.6000) -- (219.5000,151.6000) -- (219.5000,151.6000) -- (219.6000,151.6000) -- (219.7000,151.6000) -- (219.7000,151.6000) -- (219.8000,151.6000) -- (219.8000,151.6000) -- (219.9000,151.6000) -- (220.0000,151.6000) -- (220.0000,151.6000) -- (220.1000,151.6000) -- (220.1000,151.6000) -- (220.2000,151.6000) -- (220.3000,151.6000) -- (220.3000,151.6000) -- (220.4000,151.6000) -- (220.5000,151.6000) -- (220.5000,151.6000) -- (220.6000,151.6000) -- (220.6000,151.6000) -- (220.7000,151.6000) -- (220.8000,151.6000) -- (220.8000,151.6000) -- (220.9000,151.6000) -- (220.9000,151.6000) -- (221.0000,151.6000) -- (221.1000,151.6000) -- (221.1000,151.6000) -- (221.2000,151.6000) -- (221.3000,151.6000) -- (221.3000,151.6000) -- (221.4000,151.6000) -- (221.4000,151.6000) -- (221.5000,151.6000) -- (221.6000,151.6000) -- (221.6000,151.6000) -- (221.7000,151.6000) -- (221.7000,151.6000) -- (221.8000,151.6000) -- (221.9000,151.6000) -- (221.9000,151.6000) -- (222.0000,151.6000) -- (222.1000,151.6000) -- (222.1000,151.6000) -- (222.2000,151.6000) -- (222.2000,151.6000) -- (222.3000,151.6000) -- (222.4000,151.6000) -- (222.4000,151.6000) -- (222.5000,151.6000) -- (222.5000,151.6000) -- (222.6000,151.6000) -- (222.7000,151.6000) -- (222.7000,151.6000) -- (222.8000,151.6000) -- (222.9000,151.6000) -- (222.9000,151.6000) -- (223.0000,151.6000) -- (223.0000,151.6000) -- (223.1000,151.6000) -- (223.2000,151.6000) -- (223.2000,151.6000) -- (223.3000,151.6000) -- (223.3000,151.6000) -- (223.4000,151.6000) -- (223.5000,151.6000) -- (223.5000,151.6000) -- (223.6000,151.6000) -- (223.7000,151.6000) -- (223.7000,151.6000) -- (223.8000,151.6000) -- (223.8000,151.6000) -- (223.9000,151.6000) -- (224.0000,151.6000) -- (224.0000,151.6000) -- (224.1000,151.6000) -- (224.1000,151.6000) -- (224.2000,151.6000) -- (224.3000,151.6000) -- (224.3000,151.6000) -- (224.4000,151.6000) -- (224.5000,151.6000) -- (224.5000,151.6000) -- (224.6000,151.6000) -- (224.6000,151.6000) -- (224.7000,151.6000) -- (224.8000,151.6000) -- (224.8000,151.6000) -- (224.9000,151.6000) -- (224.9000,151.6000) -- (225.0000,151.6000) -- (225.1000,151.6000) -- (225.1000,151.6000) -- (225.2000,151.6000) -- (225.3000,151.6000) -- (225.3000,151.6000) -- (225.4000,151.6000) -- (225.4000,151.6000) -- (225.5000,151.6000) -- (225.6000,151.6000) -- (225.6000,151.6000) -- (225.7000,151.6000) -- (225.7000,151.6000) -- (225.8000,151.6000) -- (225.9000,151.6000) -- (225.9000,151.6000) -- (226.0000,151.6000) -- (226.0000,151.6000) -- (226.1000,151.6000) -- (226.2000,151.6000) -- (226.2000,151.6000) -- (226.3000,151.6000) -- (226.4000,151.6000) -- (226.4000,151.6000) -- (226.5000,151.6000) -- (226.5000,151.6000) -- (226.6000,151.6000) -- (226.7000,151.6000) -- (226.7000,151.6000) -- (226.8000,151.6000) -- (226.8000,151.6000) -- (226.9000,151.6000) -- (227.0000,151.6000) -- (227.0000,151.6000) -- (227.1000,151.6000) -- (227.2000,151.6000) -- (227.2000,151.6000) -- (227.3000,151.6000) -- (227.3000,151.6000) -- (227.4000,151.6000) -- (227.5000,151.6000) -- (227.5000,151.6000) -- (227.6000,151.6000) -- (227.7000,151.6000) -- (227.7000,151.6000) -- (227.8000,151.6000) -- (227.8000,151.6000) -- (227.9000,151.6000) -- (228.0000,151.6000) -- (228.0000,151.6000) -- (228.1000,151.6000) -- (228.1000,151.6000) -- (228.2000,151.6000) -- (228.3000,151.6000) -- (228.3000,151.6000) -- (228.4000,151.6000) -- (228.5000,151.6000) -- (228.5000,151.6000) -- (228.6000,151.6000) -- (228.6000,151.6000) -- (228.7000,151.6000) -- (228.8000,151.6000) -- (228.8000,151.6000) -- (228.9000,151.6000) -- (228.9000,151.6000) -- (229.0000,151.6000) -- (229.1000,151.6000) -- (229.1000,151.6000) -- (229.2000,151.6000) -- (229.2000,151.6000) -- (229.3000,151.6000) -- (229.4000,151.6000) -- (229.4000,151.6000) -- (229.5000,151.6000) -- (229.6000,151.6000) -- (229.6000,151.6000) -- (229.7000,151.6000) -- (229.7000,151.6000) -- (229.8000,151.6000) -- (229.9000,151.6000) -- (229.9000,151.6000) -- (230.0000,151.6000) -- (230.0000,151.6000) -- (230.1000,151.6000) -- (230.2000,151.6000) -- (230.2000,151.6000) -- (230.3000,151.6000) -- (230.4000,151.6000) -- (230.4000,151.6000) -- (230.5000,151.6000) -- (230.5000,151.6000) -- (230.6000,151.6000) -- (230.7000,151.6000) -- (230.7000,151.6000) -- (230.8000,151.6000) -- (230.8000,151.6000) -- (230.9000,151.6000) -- (231.0000,151.6000) -- (231.0000,151.6000) -- (231.1000,151.6000) -- (231.2000,151.6000) -- (231.2000,151.6000) -- (231.3000,151.6000) -- (231.3000,151.6000) -- (231.4000,151.6000) -- (231.5000,151.6000) -- (231.5000,151.6000) -- (231.6000,151.6000) -- (231.6000,151.6000) -- (231.7000,151.6000) -- (231.8000,151.6000) -- (231.8000,151.6000) -- (231.9000,151.6000) -- (232.0000,151.6000) -- (232.0000,151.6000) -- (232.1000,151.6000) -- (232.1000,151.6000) -- (232.2000,151.6000) -- (232.3000,151.6000) -- (232.3000,151.6000) -- (232.4000,151.6000) -- (232.4000,151.6000) -- (232.5000,151.6000) -- (232.6000,151.6000) -- (232.6000,151.6000) -- (232.7000,151.6000) -- (232.8000,151.6000) -- (232.8000,151.6000) -- (232.9000,151.6000) -- (232.9000,151.6000) -- (233.0000,151.6000) -- (233.1000,151.6000) -- (233.1000,151.6000) -- (233.2000,151.6000) -- (233.2000,151.6000) -- (233.3000,151.6000) -- (233.4000,151.6000) -- (233.4000,151.6000) -- (233.5000,151.6000) -- (233.6000,151.6000) -- (233.6000,151.6000) -- (233.7000,151.6000) -- (233.7000,151.6000) -- (233.8000,151.6000) -- (233.9000,151.6000) -- (233.9000,151.6000) -- (234.0000,151.6000) -- (234.0000,151.6000) -- (234.1000,151.6000) -- (234.2000,151.6000) -- (234.2000,151.6000) -- (234.3000,151.6000) -- (234.4000,151.6000) -- (234.4000,151.6000) -- (234.5000,151.6000) -- (234.5000,151.6000) -- (234.6000,151.6000) -- (234.7000,151.6000) -- (234.7000,151.6000) -- (234.8000,151.6000) -- (234.8000,151.6000) -- (234.9000,151.6000) -- (235.0000,151.6000) -- (235.0000,151.6000) -- (235.1000,151.6000) -- (235.2000,151.6000) -- (235.2000,151.6000) -- (235.3000,151.6000) -- (235.3000,151.6000) -- (235.4000,151.6000) -- (235.5000,151.6000) -- (235.5000,151.6000) -- (235.6000,151.6000) -- (235.6000,151.6000) -- (235.7000,151.6000) -- (235.8000,151.6000) -- (235.8000,151.6000) -- (235.9000,151.6000) -- (236.0000,151.6000) -- (236.0000,151.6000) -- (236.1000,151.6000) -- (236.1000,151.6000) -- (236.2000,151.6000) -- (236.3000,151.6000) -- (236.3000,151.6000) -- (236.4000,151.6000) -- (236.4000,151.6000) -- (236.5000,151.6000) -- (236.6000,151.6000) -- (236.6000,151.6000) -- (236.7000,151.6000) -- (236.8000,151.6000) -- (236.8000,151.6000) -- (236.9000,151.6000) -- (236.9000,151.6000) -- (237.0000,151.6000) -- (237.1000,151.6000) -- (237.1000,151.6000) -- (237.2000,151.6000) -- (237.2000,151.6000) -- (237.3000,151.6000) -- (237.4000,151.6000) -- (237.4000,151.6000) -- (237.5000,151.6000) -- (237.6000,151.6000) -- (237.6000,151.6000) -- (237.7000,151.6000) -- (237.7000,151.6000) -- (237.8000,151.6000) -- (237.9000,151.6000) -- (237.9000,151.6000) -- (238.0000,151.6000) -- (238.0000,151.6000) -- (238.1000,151.6000) -- (238.2000,151.6000) -- (238.2000,151.6000) -- (238.3000,151.6000) -- (238.4000,151.6000) -- (238.4000,151.6000) -- (238.5000,151.6000) -- (238.5000,151.6000) -- (238.6000,151.6000) -- (238.7000,151.6000) -- (238.7000,151.6000) -- (238.8000,151.6000) -- (238.8000,151.6000) -- (238.9000,151.6000) -- (239.0000,151.6000) -- (239.0000,151.6000) -- (239.1000,151.6000) -- (239.2000,151.6000) -- (239.2000,151.6000) -- (239.3000,151.6000) -- (239.3000,151.6000) -- (239.4000,151.6000) -- (239.5000,151.6000) -- (239.5000,151.6000) -- (239.6000,151.6000) -- (239.6000,151.6000) -- (239.7000,151.6000) -- (239.8000,151.6000) -- (239.8000,151.6000) -- (239.9000,151.6000) -- (240.0000,151.6000) -- (240.0000,151.6000) -- (240.1000,151.6000) -- (240.1000,151.6000) -- (240.2000,151.6000) -- (240.3000,151.6000) -- (240.3000,151.6000) -- (240.4000,151.6000) -- (240.4000,151.6000) -- (240.5000,151.6000) -- (240.6000,151.6000) -- (240.6000,151.6000) -- (240.7000,151.6000) -- (240.8000,151.6000) -- (240.8000,151.6000) -- (240.9000,151.6000) -- (240.9000,151.6000) -- (241.0000,151.6000) -- (241.1000,151.6000) -- (241.1000,151.6000) -- (241.2000,151.6000) -- (241.2000,151.6000) -- (241.3000,151.6000) -- (241.4000,151.6000) -- (241.4000,151.6000) -- (241.5000,151.6000) -- (241.6000,151.6000) -- (241.6000,151.6000) -- (241.7000,151.6000) -- (241.7000,151.6000) -- (241.8000,151.6000) -- (241.9000,151.6000) -- (241.9000,151.6000) -- (242.0000,151.6000) -- (242.0000,151.6000) -- (242.1000,151.6000) -- (242.2000,151.6000) -- (242.2000,151.6000) -- (242.3000,151.6000) -- (242.4000,151.6000) -- (242.4000,151.6000) -- (242.5000,151.6000) -- (242.5000,151.6000) -- (242.6000,151.6000) -- (242.7000,151.6000) -- (242.7000,151.6000) -- (242.8000,151.6000) -- (242.8000,151.6000) -- (242.9000,151.6000) -- (243.0000,151.6000) -- (243.0000,151.6000) -- (243.1000,151.6000) -- (243.2000,151.6000) -- (243.2000,151.6000) -- (243.3000,151.6000) -- (243.3000,151.6000) -- (243.4000,151.6000) -- (243.5000,151.6000) -- (243.5000,151.6000) -- (243.6000,151.6000) -- (243.6000,151.6000) -- (243.7000,151.6000) -- (243.8000,151.6000) -- (243.8000,151.6000) -- (243.9000,151.6000) -- (244.0000,151.6000) -- (244.0000,151.6000) -- (244.1000,151.6000) -- (244.1000,151.6000) -- (244.2000,151.6000) -- (244.3000,151.6000) -- (244.3000,151.6000) -- (244.4000,151.6000) -- (244.4000,151.6000) -- (244.5000,151.6000) -- (244.6000,151.6000) -- (244.6000,151.6000) -- (244.7000,151.6000) -- (244.8000,151.6000) -- (244.8000,151.6000) -- (244.9000,151.6000) -- (244.9000,151.6000) -- (245.0000,151.6000) -- (245.1000,151.6000) -- (245.1000,151.6000) -- (245.2000,151.6000) -- (245.2000,151.6000) -- (245.3000,151.6000) -- (245.4000,151.6000) -- (245.4000,151.6000) -- (245.5000,151.6000) -- (245.6000,151.6000) -- (245.6000,151.6000) -- (245.7000,151.6000) -- (245.7000,151.6000) -- (245.8000,151.6000) -- (245.9000,151.6000) -- (245.9000,151.6000) -- (246.0000,151.6000) -- (246.0000,151.6000) -- (246.1000,151.6000) -- (246.2000,151.6000) -- (246.2000,151.6000) -- (246.3000,151.6000);



  \end{scope}
  \begin{scope}[cm={{0.92047,0.0,0.0,0.92047,(-569.12952,52.34945)}},draw=blue,line cap=round,line join=round,line width=0.480pt]
    \path[draw] (184.5000,108.5000) -- (184.5000,156.5000) -- (246.5000,156.5000) -- (246.5000,108.5000) -- (184.5000,108.5000);



  \end{scope}
  \begin{scope}[cm={{0.92047,0.0,0.0,0.92047,(-569.12952,52.34945)}},draw=ca0a0a4,dash pattern=on 1.96pt off 1.96pt,line cap=round,line join=round,line width=0.326pt,miter limit=4.00]
    \path[draw,dash pattern=on 1.96pt off 1.96pt,line width=0.326pt,miter limit=4.00] (184.5000,200.5000) -- (246.5000,200.5000);



  \end{scope}
  \begin{scope}[cm={{0.92047,0.0,0.0,0.92047,(-569.12952,52.34945)}},draw=blue,line cap=round,line join=round,line width=0.480pt]
    \path[draw] (184.5000,200.5000) -- (186.5000,200.5000);



    \path[draw] (246.5000,200.5000) -- (245.5000,200.5000);



  \end{scope}
  \begin{scope}[cm={{0.92047,0.0,0.0,0.92047,(-569.12952,52.34945)}},draw=ca0a0a4,dash pattern=on 1.96pt off 1.96pt,line cap=round,line join=round,line width=0.326pt,miter limit=4.00]
    \path[draw,dash pattern=on 1.96pt off 1.96pt,line width=0.326pt,miter limit=4.00] (184.5000,180.5000) -- (246.5000,180.5000);



  \end{scope}
  \begin{scope}[cm={{0.92047,0.0,0.0,0.92047,(-569.12952,52.34945)}},draw=blue,line cap=round,line join=round,line width=0.480pt]
    \path[draw] (184.5000,180.5000) -- (186.5000,180.5000);



    \path[draw] (246.5000,180.5000) -- (245.5000,180.5000);



  \end{scope}
  \begin{scope}[cm={{0.92047,0.0,0.0,0.92047,(-569.12952,52.34945)}},draw=ca0a0a4,dash pattern=on 1.96pt off 1.96pt,line cap=round,line join=round,line width=0.326pt,miter limit=4.00]
    \path[draw,dash pattern=on 1.96pt off 1.96pt,line width=0.326pt,miter limit=4.00] (184.5000,160.5000) -- (246.5000,160.5000);



  \end{scope}
  \begin{scope}[cm={{0.92047,0.0,0.0,0.92047,(-569.12952,52.34945)}},draw=blue,line cap=round,line join=round,line width=0.480pt]
    \path[draw] (184.5000,160.5000) -- (186.5000,160.5000);



    \path[draw] (246.5000,160.5000) -- (245.5000,160.5000);



  \end{scope}
  \begin{scope}[cm={{0.92047,0.0,0.0,0.92047,(-569.12952,52.34945)}},draw=ca0a0a4,dash pattern=on 0.40pt off 0.80pt,line cap=round,line join=round,line width=0.400pt]
    \path[draw] (184.5000,204.5000) -- (184.5000,156.5000);



  \end{scope}
  \begin{scope}[cm={{0.92047,0.0,0.0,0.92047,(-569.12952,52.34945)}},draw=blue,line cap=round,line join=round,line width=0.480pt]
    \path[draw] (184.5000,204.5000) -- (184.5000,203.5000);



    \path[draw] (184.5000,156.5000) -- (184.5000,157.5000);



  \end{scope}
  \begin{scope}[cm={{1.00588,0.0,0.0,1.00588,(-400.6796,252.66367)}},draw=blue,line cap=rect,line join=bevel,line width=0.800pt]
    \path[fill=blue] (0.0000,0.0000) node[above right] (text1416) {\scriptsize 0};



  \end{scope}
  \begin{scope}[cm={{0.92047,0.0,0.0,0.92047,(-569.12952,52.34945)}},draw=ca0a0a4,dash pattern=on 1.96pt off 1.96pt,line cap=round,line join=round,line width=0.326pt,miter limit=4.00]
    \path[draw,dash pattern=on 1.96pt off 1.96pt,line width=0.326pt,miter limit=4.00] (203.5000,204.5000) -- (203.5000,156.5000);



  \end{scope}
  \begin{scope}[cm={{0.92047,0.0,0.0,0.92047,(-569.12952,52.34945)}},draw=blue,line cap=round,line join=round,line width=0.480pt]
    \path[draw] (203.5000,204.5000) -- (203.5000,203.5000);



    \path[draw] (203.5000,156.5000) -- (203.5000,157.5000);



  \end{scope}
  \begin{scope}[cm={{1.00588,0.0,0.0,1.00588,(-384.0746,252.77633)}},draw=blue,line cap=rect,line join=bevel,line width=0.800pt]
    \path[fill=blue] (0.0000,0.0000) node[above right] (text1446) {\scriptsize 2};



  \end{scope}
  \begin{scope}[cm={{0.92047,0.0,0.0,0.92047,(-569.12952,52.34945)}},draw=ca0a0a4,dash pattern=on 1.96pt off 1.96pt,line cap=round,line join=round,line width=0.326pt,miter limit=4.00]
    \path[draw,dash pattern=on 1.96pt off 1.96pt,line width=0.326pt,miter limit=4.00] (221.5000,204.5000) -- (221.5000,156.5000);



  \end{scope}
  \begin{scope}[cm={{0.92047,0.0,0.0,0.92047,(-569.12952,52.34945)}},draw=blue,line cap=round,line join=round,line width=0.480pt]
    \path[draw] (221.5000,204.5000) -- (221.5000,203.5000);



    \path[draw] (221.5000,156.5000) -- (221.5000,157.5000);



  \end{scope}
  \begin{scope}[cm={{1.00588,0.0,0.0,1.00588,(-366.9656,252.77633)}},draw=blue,line cap=rect,line join=bevel,line width=0.800pt]
    \path[fill=blue] (0.0000,0.0000) node[above right] (text1476) {\scriptsize 4};



  \end{scope}
  \begin{scope}[cm={{0.92047,0.0,0.0,0.92047,(-569.12952,52.34945)}},draw=ca0a0a4,dash pattern=on 1.96pt off 1.96pt,line cap=round,line join=round,line width=0.326pt,miter limit=4.00]
    \path[draw,dash pattern=on 1.96pt off 1.96pt,line width=0.326pt,miter limit=4.00] (239.5000,204.5000) -- (239.5000,156.5000);



  \end{scope}
  \begin{scope}[cm={{0.92047,0.0,0.0,0.92047,(-569.12952,52.34945)}},draw=blue,line cap=round,line join=round,line width=0.480pt]
    \path[draw] (239.5000,204.5000) -- (239.5000,203.5000);



    \path[draw] (239.5000,156.5000) -- (239.5000,157.5000);



  \end{scope}
  \begin{scope}[cm={{1.00588,0.0,0.0,1.00588,(-351.3566,252.66367)}},draw=blue,line cap=rect,line join=bevel,line width=0.800pt]
    \path[fill=blue] (0.0000,0.0000) node[above right] (text1506) {\scriptsize 6};



  \end{scope}
  \begin{scope}[cm={{0.92047,0.0,0.0,0.92047,(-569.12952,52.34945)}},draw=blue,line cap=round,line join=round,line width=0.480pt]
    \path[draw] (246.5000,200.5000) -- (245.5000,200.5000);



  \end{scope}
  \begin{scope}[cm={{0.92047,0.0,0.0,0.92047,(-569.12952,52.34945)}},draw=blue,line cap=round,line join=round,line width=0.480pt]
    \path[draw] (246.5000,180.5000) -- (245.5000,180.5000);



  \end{scope}
  \begin{scope}[cm={{0.92047,0.0,0.0,0.92047,(-569.12952,52.34945)}},draw=blue,line cap=round,line join=round,line width=0.480pt]
    \path[draw] (246.5000,160.5000) -- (245.5000,160.5000);



  \end{scope}
  \begin{scope}[cm={{0.92047,0.0,0.0,0.92047,(-569.12952,52.34945)}},draw=blue,line cap=round,line join=round,line width=0.480pt]
    \path[draw] (184.5000,156.5000) -- (184.5000,204.5000) -- (246.5000,204.5000) -- (246.5000,156.5000) -- (184.5000,156.5000);



  \end{scope}
  \begin{scope}[cm={{0.92047,0.0,0.0,0.92047,(-569.12952,52.34945)}},draw=blue,line cap=round,line join=round,line width=0.480pt]
    \path[draw] (184.8000,160.5000) -- (184.8000,160.5000) -- (184.9000,160.5000) -- (185.0000,160.5000) -- (185.0000,160.5000) -- (185.1000,160.5000) -- (185.1000,160.5000) -- (185.2000,160.5000) -- (185.3000,160.5000) -- (185.3000,160.5000) -- (185.4000,160.5000) -- (185.4000,160.5000) -- (185.5000,160.5000) -- (185.6000,160.5000) -- (185.6000,160.5000) -- (185.7000,160.5000) -- (185.8000,160.5000) -- (185.8000,160.5000) -- (185.9000,160.5000) -- (185.9000,160.5000) -- (186.0000,160.5000) -- (186.1000,160.5000) -- (186.1000,160.5000) -- (186.2000,160.5000) -- (186.2000,160.5000) -- (186.3000,160.5000) -- (186.4000,160.5000) -- (186.4000,160.5000) -- (186.5000,160.5000) -- (186.6000,160.5000) -- (186.6000,160.5000) -- (186.7000,160.5000) -- (186.7000,160.5000) -- (186.8000,160.5000) -- (186.9000,160.5000) -- (186.9000,160.5000) -- (187.0000,160.5000) -- (187.0000,160.5000) -- (187.1000,160.5000) -- (187.2000,160.5000) -- (187.2000,160.5000) -- (187.3000,160.5000) -- (187.4000,160.5000) -- (187.4000,160.5000) -- (187.5000,160.5000) -- (187.5000,160.5000) -- (187.6000,160.5000) -- (187.7000,160.5000) -- (187.7000,160.5000) -- (187.8000,160.5000) -- (187.8000,160.5000) -- (187.9000,160.5000) -- (188.0000,160.5000) -- (188.0000,160.5000) -- (188.1000,160.5000) -- (188.2000,160.5000) -- (188.2000,160.5000) -- (188.3000,160.5000) -- (188.3000,160.5000) -- (188.4000,160.5000) -- (188.5000,160.5000) -- (188.5000,160.5000) -- (188.6000,160.5000) -- (188.6000,160.5000) -- (188.7000,160.5000) -- (188.8000,160.5000) -- (188.8000,160.5000) -- (188.9000,160.5000) -- (189.0000,160.5000) -- (189.0000,160.5000) -- (189.1000,160.5000) -- (189.1000,160.5000) -- (189.2000,160.5000) -- (189.3000,160.5000) -- (189.3000,160.5000) -- (189.4000,160.5000) -- (189.4000,160.5000) -- (189.5000,160.5000) -- (189.6000,160.5000) -- (189.6000,160.5000) -- (189.7000,160.5000) -- (189.8000,160.5000) -- (189.8000,160.5000) -- (189.9000,160.5000) -- (189.9000,160.5000) -- (190.0000,160.5000) -- (190.1000,160.5000) -- (190.1000,160.5000) -- (190.2000,160.5000) -- (190.2000,160.5000) -- (190.3000,160.5000) -- (190.4000,160.5000) -- (190.4000,160.5000) -- (190.5000,160.5000) -- (190.6000,160.5000) -- (190.6000,160.5000) -- (190.7000,160.5000) -- (190.7000,160.5000) -- (190.8000,160.5000) -- (190.9000,160.5000) -- (190.9000,160.5000) -- (191.0000,160.5000) -- (191.0000,160.5000) -- (191.1000,160.5000) -- (191.2000,160.5000) -- (191.2000,160.5000) -- (191.3000,160.5000) -- (191.4000,160.5000) -- (191.4000,160.5000) -- (191.5000,160.5000) -- (191.5000,160.5000) -- (191.6000,160.5000) -- (191.7000,160.5000) -- (191.7000,160.5000) -- (191.8000,160.5000) -- (191.8000,160.5000) -- (191.9000,160.5000) -- (192.0000,160.5000) -- (192.0000,160.5000) -- (192.1000,160.5000) -- (192.2000,160.5000) -- (192.2000,160.5000) -- (192.3000,160.5000) -- (192.3000,160.5000) -- (192.4000,160.5000) -- (192.5000,160.5000) -- (192.5000,160.5000) -- (192.6000,160.5000) -- (192.6000,160.5000) -- (192.7000,160.5000) -- (192.8000,160.5000) -- (192.8000,160.5000) -- (192.9000,160.5000) -- (193.0000,160.5000) -- (193.0000,160.5000) -- (193.1000,160.5000) -- (193.1000,160.5000) -- (193.2000,160.5000) -- (193.3000,160.5000) -- (193.3000,160.5000) -- (193.4000,160.5000) -- (193.4000,160.5000) -- (193.5000,160.5000) -- (193.6000,160.5000) -- (193.6000,160.5000) -- (193.7000,160.5000) -- (193.8000,160.5000) -- (193.8000,160.5000) -- (193.9000,160.5000) -- (193.9000,160.5000) -- (194.0000,160.5000) -- (194.1000,160.5000) -- (194.1000,160.5000) -- (194.2000,160.5000) -- (194.2000,160.5000) -- (194.3000,160.5000) -- (194.4000,160.5000) -- (194.4000,160.5000) -- (194.5000,160.5000) -- (194.6000,160.5000) -- (194.6000,160.5000) -- (194.7000,160.5000) -- (194.7000,160.5000) -- (194.8000,160.5000) -- (194.9000,160.5000) -- (194.9000,160.5000) -- (195.0000,160.5000) -- (195.0000,160.5000) -- (195.1000,160.5000) -- (195.2000,160.5000) -- (195.2000,160.5000) -- (195.3000,160.5000) -- (195.4000,160.5000) -- (195.4000,160.5000) -- (195.5000,160.5000) -- (195.5000,160.5000) -- (195.6000,160.5000) -- (195.7000,160.5000) -- (195.7000,160.5000) -- (195.8000,160.5000) -- (195.8000,160.5000) -- (195.9000,160.5000) -- (196.0000,160.5000) -- (196.0000,160.5000) -- (196.1000,160.5000) -- (196.2000,160.5000) -- (196.2000,160.5000) -- (196.3000,160.5000) -- (196.3000,160.5000) -- (196.4000,160.5000) -- (196.5000,160.5000) -- (196.5000,160.5000) -- (196.6000,160.5000) -- (196.6000,160.5000) -- (196.7000,160.5000) -- (196.8000,160.5000) -- (196.8000,160.5000) -- (196.9000,160.5000) -- (196.9000,160.5000) -- (197.0000,160.5000) -- (197.1000,160.5000) -- (197.1000,160.5000) -- (197.2000,160.5000) -- (197.3000,160.5000) -- (197.3000,160.5000) -- (197.4000,160.5000) -- (197.4000,160.5000) -- (197.5000,160.5000) -- (197.6000,160.5000) -- (197.6000,160.5000) -- (197.7000,160.5000) -- (197.7000,160.5000) -- (197.8000,160.5000) -- (197.9000,160.5000) -- (197.9000,160.5000) -- (198.0000,160.5000) -- (198.1000,160.5000) -- (198.1000,160.5000) -- (198.2000,160.5000) -- (198.2000,160.5000) -- (198.3000,160.5000) -- (198.4000,160.5000) -- (198.4000,160.5000) -- (198.5000,160.5000) -- (198.5000,160.5000) -- (198.6000,160.5000) -- (198.7000,160.5000) -- (198.7000,160.5000) -- (198.8000,160.5000) -- (198.9000,159.9000) -- (198.9000,165.0000) -- (199.0000,180.4000) -- (199.0000,180.3000) -- (199.1000,180.2000) -- (199.2000,180.2000) -- (199.2000,180.2000) -- (199.3000,180.2000) -- (199.3000,180.2000) -- (199.4000,180.2000) -- (199.5000,180.2000) -- (199.5000,180.2000) -- (199.6000,180.2000) -- (199.7000,180.2000) -- (199.7000,180.2000) -- (199.8000,180.2000) -- (199.8000,180.2000) -- (199.9000,180.2000) -- (200.0000,180.2000) -- (200.0000,180.2000) -- (200.1000,180.2000) -- (200.1000,180.2000) -- (200.2000,180.2000) -- (200.3000,180.2000) -- (200.3000,180.2000) -- (200.4000,180.2000) -- (200.5000,180.2000) -- (200.5000,180.2000) -- (200.6000,180.2000) -- (200.6000,180.2000) -- (200.7000,180.2000) -- (200.8000,180.2000) -- (200.8000,180.2000) -- (200.9000,180.2000) -- (200.9000,180.2000) -- (201.0000,180.2000) -- (201.1000,180.2000) -- (201.1000,180.2000) -- (201.2000,180.2000) -- (201.3000,180.2000) -- (201.3000,180.2000) -- (201.4000,180.2000) -- (201.4000,180.2000) -- (201.5000,180.2000) -- (201.6000,180.2000) -- (201.6000,180.2000) -- (201.7000,180.2000) -- (201.7000,180.2000) -- (201.8000,180.2000) -- (201.9000,180.2000) -- (201.9000,180.2000) -- (202.0000,180.2000) -- (202.1000,180.2000) -- (202.1000,180.2000) -- (202.2000,180.2000) -- (202.2000,180.2000) -- (202.3000,180.2000) -- (202.4000,180.2000) -- (202.4000,180.2000) -- (202.5000,180.2000) -- (202.5000,180.2000) -- (202.6000,180.2000) -- (202.7000,180.2000) -- (202.7000,180.2000) -- (202.8000,180.2000) -- (202.9000,180.2000) -- (202.9000,180.2000) -- (203.0000,180.2000) -- (203.0000,180.2000) -- (203.1000,180.2000) -- (203.2000,180.2000) -- (203.2000,180.2000) -- (203.3000,180.2000) -- (203.3000,180.2000) -- (203.4000,180.2000) -- (203.5000,180.2000) -- (203.5000,180.2000) -- (203.6000,180.2000) -- (203.7000,180.2000) -- (203.7000,180.2000) -- (203.8000,180.2000) -- (203.8000,180.2000) -- (203.9000,180.2000) -- (204.0000,180.2000) -- (204.0000,180.2000) -- (204.1000,180.2000) -- (204.1000,180.2000) -- (204.2000,180.2000) -- (204.3000,180.2000) -- (204.3000,180.2000) -- (204.4000,180.2000) -- (204.5000,180.2000) -- (204.5000,180.2000) -- (204.6000,180.2000) -- (204.6000,180.2000) -- (204.7000,180.2000) -- (204.8000,180.2000) -- (204.8000,180.2000) -- (204.9000,180.2000) -- (204.9000,180.2000) -- (205.0000,180.2000) -- (205.1000,180.2000) -- (205.1000,180.2000) -- (205.2000,180.2000) -- (205.3000,180.2000) -- (205.3000,180.2000) -- (205.4000,180.2000) -- (205.4000,180.2000) -- (205.5000,180.2000) -- (205.6000,180.2000) -- (205.6000,180.2000) -- (205.7000,180.2000) -- (205.7000,180.2000) -- (205.8000,180.2000) -- (205.9000,180.2000) -- (205.9000,180.2000) -- (206.0000,180.2000) -- (206.1000,180.2000) -- (206.1000,180.2000) -- (206.2000,180.2000) -- (206.2000,180.2000) -- (206.3000,180.2000) -- (206.4000,180.2000) -- (206.4000,180.2000) -- (206.5000,180.2000) -- (206.5000,180.2000) -- (206.6000,180.2000) -- (206.7000,180.2000) -- (206.7000,180.2000) -- (206.8000,180.2000) -- (206.9000,180.2000) -- (206.9000,180.2000) -- (207.0000,180.2000) -- (207.0000,180.2000) -- (207.1000,180.2000) -- (207.2000,180.2000) -- (207.2000,180.2000) -- (207.3000,180.2000) -- (207.3000,180.2000) -- (207.4000,180.2000) -- (207.5000,180.2000) -- (207.5000,180.2000) -- (207.6000,180.2000) -- (207.7000,180.2000) -- (207.7000,180.2000) -- (207.8000,180.2000) -- (207.8000,180.2000) -- (207.9000,180.2000) -- (208.0000,180.2000) -- (208.0000,180.2000) -- (208.1000,180.2000) -- (208.1000,180.2000) -- (208.2000,180.2000) -- (208.3000,180.2000) -- (208.3000,180.2000) -- (208.4000,180.2000) -- (208.5000,180.2000) -- (208.5000,180.2000) -- (208.6000,180.2000) -- (208.6000,180.2000) -- (208.7000,180.2000) -- (208.8000,180.2000) -- (208.8000,180.2000) -- (208.9000,180.2000) -- (208.9000,180.2000) -- (209.0000,180.2000) -- (209.1000,180.2000) -- (209.1000,180.2000) -- (209.2000,180.2000) -- (209.3000,180.2000) -- (209.3000,180.2000) -- (209.4000,180.2000) -- (209.4000,180.2000) -- (209.5000,180.2000) -- (209.6000,180.2000) -- (209.6000,180.2000) -- (209.7000,180.2000) -- (209.7000,180.2000) -- (209.8000,180.2000) -- (209.9000,180.2000) -- (209.9000,180.2000) -- (210.0000,180.2000) -- (210.1000,180.2000) -- (210.1000,180.2000) -- (210.2000,180.2000) -- (210.2000,180.2000) -- (210.3000,180.2000) -- (210.4000,180.2000) -- (210.4000,180.2000) -- (210.5000,180.2000) -- (210.5000,180.2000) -- (210.6000,180.2000) -- (210.7000,180.2000) -- (210.7000,180.2000) -- (210.8000,180.2000) -- (210.9000,180.2000) -- (210.9000,180.2000) -- (211.0000,180.2000) -- (211.0000,180.2000) -- (211.1000,180.2000) -- (211.2000,180.2000) -- (211.2000,180.2000) -- (211.3000,180.2000) -- (211.3000,180.2000) -- (211.4000,180.2000) -- (211.5000,180.2000) -- (211.5000,180.2000) -- (211.6000,180.2000) -- (211.7000,180.2000) -- (211.7000,180.2000) -- (211.8000,180.2000) -- (211.8000,180.2000) -- (211.9000,180.2000) -- (212.0000,180.2000) -- (212.0000,180.2000) -- (212.1000,180.2000) -- (212.1000,180.2000) -- (212.2000,180.2000) -- (212.3000,180.2000) -- (212.3000,180.2000) -- (212.4000,180.2000) -- (212.5000,180.2000) -- (212.5000,180.2000) -- (212.6000,180.2000) -- (212.6000,180.2000) -- (212.7000,180.2000) -- (212.8000,180.2000) -- (212.8000,180.2000) -- (212.9000,180.2000) -- (212.9000,180.2000) -- (213.0000,180.2000) -- (213.1000,180.2000) -- (213.1000,180.2000) -- (213.2000,180.2000) -- (213.3000,180.2000) -- (213.3000,180.2000) -- (213.4000,180.2000) -- (213.4000,180.2000) -- (213.5000,180.2000) -- (213.6000,180.2000) -- (213.6000,180.2000) -- (213.7000,180.2000) -- (213.7000,180.2000) -- (213.8000,180.2000) -- (213.9000,180.2000) -- (213.9000,180.2000) -- (214.0000,180.2000) -- (214.1000,180.2000) -- (214.1000,180.2000) -- (214.2000,180.2000) -- (214.2000,180.2000) -- (214.3000,180.2000) -- (214.4000,180.2000) -- (214.4000,180.2000) -- (214.5000,180.2000) -- (214.5000,180.2000) -- (214.6000,180.2000) -- (214.7000,180.2000) -- (214.7000,180.2000) -- (214.8000,180.2000) -- (214.9000,180.2000) -- (214.9000,180.2000) -- (215.0000,180.2000) -- (215.0000,180.2000) -- (215.1000,180.2000) -- (215.2000,180.2000) -- (215.2000,180.2000) -- (215.3000,180.2000) -- (215.3000,180.2000) -- (215.4000,180.2000) -- (215.5000,180.2000) -- (215.5000,180.2000) -- (215.6000,180.2000) -- (215.7000,180.2000) -- (215.7000,180.2000) -- (215.8000,180.2000) -- (215.8000,180.2000) -- (215.9000,180.2000) -- (216.0000,180.2000) -- (216.0000,180.2000) -- (216.1000,180.2000) -- (216.1000,180.2000) -- (216.2000,180.2000) -- (216.3000,180.2000) -- (216.3000,180.2000) -- (216.4000,180.2000) -- (216.5000,180.2000) -- (216.5000,180.2000) -- (216.6000,180.2000) -- (216.6000,180.2000) -- (216.7000,180.2000) -- (216.8000,180.2000) -- (216.8000,180.2000) -- (216.9000,180.2000) -- (216.9000,180.2000) -- (217.0000,180.2000) -- (217.1000,180.2000) -- (217.1000,180.2000) -- (217.2000,180.2000) -- (217.3000,180.2000) -- (217.3000,180.2000) -- (217.4000,180.2000) -- (217.4000,180.2000) -- (217.5000,180.2000) -- (217.6000,180.2000) -- (217.6000,180.2000) -- (217.7000,180.2000) -- (217.7000,180.2000) -- (217.8000,180.2000) -- (217.9000,180.2000) -- (217.9000,180.2000) -- (218.0000,180.2000) -- (218.1000,180.2000) -- (218.1000,180.2000) -- (218.2000,180.2000) -- (218.2000,180.2000) -- (218.3000,180.2000) -- (218.4000,180.2000) -- (218.4000,180.2000) -- (218.5000,180.2000) -- (218.5000,180.2000) -- (218.6000,180.2000) -- (218.7000,180.2000) -- (218.7000,180.2000) -- (218.8000,180.2000) -- (218.9000,180.2000) -- (218.9000,180.2000) -- (219.0000,180.2000) -- (219.0000,180.2000) -- (219.1000,180.2000) -- (219.2000,180.2000) -- (219.2000,180.2000) -- (219.3000,180.2000) -- (219.3000,180.2000) -- (219.4000,180.2000) -- (219.5000,180.2000) -- (219.5000,180.2000) -- (219.6000,180.2000) -- (219.7000,180.2000) -- (219.7000,180.2000) -- (219.8000,180.2000) -- (219.8000,180.2000) -- (219.9000,180.2000) -- (220.0000,180.2000) -- (220.0000,180.2000) -- (220.1000,180.2000) -- (220.1000,180.2000) -- (220.2000,180.2000) -- (220.3000,180.2000) -- (220.3000,180.2000) -- (220.4000,180.2000) -- (220.5000,180.2000) -- (220.5000,180.2000) -- (220.6000,180.2000) -- (220.6000,180.2000) -- (220.7000,180.2000) -- (220.8000,180.2000) -- (220.8000,180.2000) -- (220.9000,180.2000) -- (220.9000,180.2000) -- (221.0000,180.2000) -- (221.1000,180.2000) -- (221.1000,180.2000) -- (221.2000,180.2000) -- (221.3000,180.2000) -- (221.3000,180.2000) -- (221.4000,180.2000) -- (221.4000,180.2000) -- (221.5000,180.2000) -- (221.6000,180.2000) -- (221.6000,180.2000) -- (221.7000,180.2000) -- (221.7000,180.2000) -- (221.8000,180.2000) -- (221.9000,180.2000) -- (221.9000,180.2000) -- (222.0000,180.2000) -- (222.1000,180.2000) -- (222.1000,180.2000) -- (222.2000,180.2000) -- (222.2000,180.2000) -- (222.3000,180.2000) -- (222.4000,180.2000) -- (222.4000,180.2000) -- (222.5000,180.2000) -- (222.5000,180.2000) -- (222.6000,180.2000) -- (222.7000,180.2000) -- (222.7000,180.2000) -- (222.8000,180.2000) -- (222.9000,180.2000) -- (222.9000,180.2000) -- (223.0000,180.2000) -- (223.0000,180.2000) -- (223.1000,180.2000) -- (223.2000,180.2000) -- (223.2000,180.2000) -- (223.3000,180.2000) -- (223.3000,180.2000) -- (223.4000,180.2000) -- (223.5000,180.2000) -- (223.5000,180.2000) -- (223.6000,180.2000) -- (223.7000,180.2000) -- (223.7000,180.2000) -- (223.8000,180.2000) -- (223.8000,180.2000) -- (223.9000,180.2000) -- (224.0000,180.2000) -- (224.0000,180.2000) -- (224.1000,180.2000) -- (224.1000,180.2000) -- (224.2000,180.2000) -- (224.3000,180.2000) -- (224.3000,180.2000) -- (224.4000,180.2000) -- (224.5000,180.2000) -- (224.5000,180.2000) -- (224.6000,180.2000) -- (224.6000,180.2000) -- (224.7000,180.2000) -- (224.8000,180.2000) -- (224.8000,180.2000) -- (224.9000,180.2000) -- (224.9000,180.2000) -- (225.0000,180.2000) -- (225.1000,180.2000) -- (225.1000,180.2000) -- (225.2000,180.2000) -- (225.3000,180.2000) -- (225.3000,180.2000) -- (225.4000,180.2000) -- (225.4000,180.2000) -- (225.5000,180.2000) -- (225.6000,180.2000) -- (225.6000,180.2000) -- (225.7000,180.2000) -- (225.7000,180.2000) -- (225.8000,180.2000) -- (225.9000,179.4000) -- (225.9000,190.4000) -- (226.0000,200.9000) -- (226.0000,200.0000) -- (226.1000,200.0000) -- (226.2000,200.0000) -- (226.2000,200.0000) -- (226.3000,200.0000) -- (226.4000,200.0000) -- (226.4000,200.0000) -- (226.5000,200.0000) -- (226.5000,200.0000) -- (226.6000,200.0000) -- (226.7000,200.0000) -- (226.7000,200.0000) -- (226.8000,200.0000) -- (226.8000,200.0000) -- (226.9000,200.0000) -- (227.0000,200.0000) -- (227.0000,200.0000) -- (227.1000,200.0000) -- (227.2000,200.0000) -- (227.2000,200.0000) -- (227.3000,200.0000) -- (227.3000,200.0000) -- (227.4000,200.0000) -- (227.5000,200.0000) -- (227.5000,200.0000) -- (227.6000,200.0000) -- (227.7000,200.0000) -- (227.7000,200.0000) -- (227.8000,200.0000) -- (227.8000,200.0000) -- (227.9000,200.0000) -- (228.0000,200.0000) -- (228.0000,200.0000) -- (228.1000,200.0000) -- (228.1000,200.0000) -- (228.2000,200.0000) -- (228.3000,200.0000) -- (228.3000,200.0000) -- (228.4000,200.0000) -- (228.5000,200.0000) -- (228.5000,200.0000) -- (228.6000,200.0000) -- (228.6000,200.0000) -- (228.7000,200.0000) -- (228.8000,200.0000) -- (228.8000,200.0000) -- (228.9000,200.0000) -- (228.9000,200.0000) -- (229.0000,200.0000) -- (229.1000,200.0000) -- (229.1000,200.0000) -- (229.2000,200.0000) -- (229.2000,200.0000) -- (229.3000,200.0000) -- (229.4000,200.0000) -- (229.4000,200.0000) -- (229.5000,200.0000) -- (229.6000,200.0000) -- (229.6000,200.0000) -- (229.7000,200.0000) -- (229.7000,200.0000) -- (229.8000,200.0000) -- (229.9000,200.0000) -- (229.9000,200.0000) -- (230.0000,200.0000) -- (230.0000,200.0000) -- (230.1000,200.0000) -- (230.2000,200.0000) -- (230.2000,200.0000) -- (230.3000,200.0000) -- (230.4000,200.0000) -- (230.4000,200.0000) -- (230.5000,200.0000) -- (230.5000,200.0000) -- (230.6000,200.0000) -- (230.7000,200.0000) -- (230.7000,200.0000) -- (230.8000,200.0000) -- (230.8000,200.0000) -- (230.9000,200.0000) -- (231.0000,200.0000) -- (231.0000,200.0000) -- (231.1000,200.0000) -- (231.2000,200.0000) -- (231.2000,200.0000) -- (231.3000,200.0000) -- (231.3000,200.0000) -- (231.4000,200.0000) -- (231.5000,200.0000) -- (231.5000,200.0000) -- (231.6000,200.0000) -- (231.6000,200.0000) -- (231.7000,200.0000) -- (231.8000,200.0000) -- (231.8000,200.0000) -- (231.9000,200.0000) -- (232.0000,200.0000) -- (232.0000,200.0000) -- (232.1000,200.0000) -- (232.1000,200.0000) -- (232.2000,200.0000) -- (232.3000,200.0000) -- (232.3000,200.0000) -- (232.4000,200.0000) -- (232.4000,200.0000) -- (232.5000,200.0000) -- (232.6000,200.0000) -- (232.6000,200.0000) -- (232.7000,200.0000) -- (232.8000,200.0000) -- (232.8000,200.0000) -- (232.9000,200.0000) -- (232.9000,200.0000) -- (233.0000,200.0000) -- (233.1000,200.0000) -- (233.1000,200.0000) -- (233.2000,200.0000) -- (233.2000,200.0000) -- (233.3000,200.0000) -- (233.4000,200.0000) -- (233.4000,200.0000) -- (233.5000,200.0000) -- (233.6000,200.0000) -- (233.6000,200.0000) -- (233.7000,200.0000) -- (233.7000,200.0000) -- (233.8000,200.0000) -- (233.9000,200.0000) -- (233.9000,200.0000) -- (234.0000,200.0000) -- (234.0000,200.0000) -- (234.1000,200.0000) -- (234.2000,200.0000) -- (234.2000,200.0000) -- (234.3000,200.0000) -- (234.4000,200.0000) -- (234.4000,200.0000) -- (234.5000,200.0000) -- (234.5000,200.0000) -- (234.6000,200.0000) -- (234.7000,200.0000) -- (234.7000,200.0000) -- (234.8000,200.0000) -- (234.8000,200.0000) -- (234.9000,200.0000) -- (235.0000,200.0000) -- (235.0000,200.0000) -- (235.1000,200.0000) -- (235.2000,200.0000) -- (235.2000,200.0000) -- (235.3000,200.0000) -- (235.3000,200.0000) -- (235.4000,200.0000) -- (235.5000,200.0000) -- (235.5000,200.0000) -- (235.6000,200.0000) -- (235.6000,200.0000) -- (235.7000,200.0000) -- (235.8000,200.0000) -- (235.8000,200.0000) -- (235.9000,200.0000) -- (236.0000,200.0000) -- (236.0000,200.0000) -- (236.1000,200.0000) -- (236.1000,200.0000) -- (236.2000,200.0000) -- (236.3000,200.0000) -- (236.3000,200.0000) -- (236.4000,200.0000) -- (236.4000,200.0000) -- (236.5000,200.0000) -- (236.6000,200.0000) -- (236.6000,200.0000) -- (236.7000,200.0000) -- (236.8000,200.0000) -- (236.8000,200.0000) -- (236.9000,200.0000) -- (236.9000,200.0000) -- (237.0000,200.0000) -- (237.1000,200.0000) -- (237.1000,200.0000) -- (237.2000,200.0000) -- (237.2000,200.0000) -- (237.3000,200.0000) -- (237.4000,200.0000) -- (237.4000,200.0000) -- (237.5000,200.0000) -- (237.6000,200.0000) -- (237.6000,200.0000) -- (237.7000,200.0000) -- (237.7000,200.0000) -- (237.8000,200.0000) -- (237.9000,200.0000) -- (237.9000,200.0000) -- (238.0000,200.0000) -- (238.0000,200.0000) -- (238.1000,200.0000) -- (238.2000,200.0000) -- (238.2000,200.0000) -- (238.3000,200.0000) -- (238.4000,200.0000) -- (238.4000,200.0000) -- (238.5000,200.0000) -- (238.5000,200.0000) -- (238.6000,200.0000) -- (238.7000,200.0000) -- (238.7000,200.0000) -- (238.8000,200.0000) -- (238.8000,200.0000) -- (238.9000,200.0000) -- (239.0000,200.0000) -- (239.0000,200.0000) -- (239.1000,200.0000) -- (239.2000,200.0000) -- (239.2000,200.0000) -- (239.3000,200.0000) -- (239.3000,200.0000) -- (239.4000,200.0000) -- (239.5000,200.0000) -- (239.5000,200.0000) -- (239.6000,200.0000) -- (239.6000,200.0000) -- (239.7000,200.0000) -- (239.8000,200.0000) -- (239.8000,200.0000) -- (239.9000,200.0000) -- (240.0000,200.0000) -- (240.0000,200.0000) -- (240.1000,200.0000) -- (240.1000,200.0000) -- (240.2000,200.0000) -- (240.3000,200.0000) -- (240.3000,200.0000) -- (240.4000,200.0000) -- (240.4000,200.0000) -- (240.5000,200.0000) -- (240.6000,200.0000) -- (240.6000,200.0000) -- (240.7000,200.0000) -- (240.8000,200.0000) -- (240.8000,200.0000) -- (240.9000,200.0000) -- (240.9000,200.0000) -- (241.0000,200.0000) -- (241.1000,200.0000) -- (241.1000,200.0000) -- (241.2000,200.0000) -- (241.2000,200.0000) -- (241.3000,200.0000) -- (241.4000,200.0000) -- (241.4000,200.0000) -- (241.5000,200.0000) -- (241.6000,200.0000) -- (241.6000,200.0000) -- (241.7000,200.0000) -- (241.7000,200.0000) -- (241.8000,200.0000) -- (241.9000,200.0000) -- (241.9000,200.0000) -- (242.0000,200.0000) -- (242.0000,200.0000) -- (242.1000,200.0000) -- (242.2000,200.0000) -- (242.2000,200.0000) -- (242.3000,200.0000) -- (242.4000,200.0000) -- (242.4000,200.0000) -- (242.5000,200.0000) -- (242.5000,200.0000) -- (242.6000,200.0000) -- (242.7000,200.0000) -- (242.7000,200.0000) -- (242.8000,200.0000) -- (242.8000,200.0000) -- (242.9000,200.0000) -- (243.0000,200.0000) -- (243.0000,200.0000) -- (243.1000,200.0000) -- (243.2000,200.0000) -- (243.2000,200.0000) -- (243.3000,200.0000) -- (243.3000,200.0000) -- (243.4000,200.0000) -- (243.5000,200.0000) -- (243.5000,200.0000) -- (243.6000,200.0000) -- (243.6000,200.0000) -- (243.7000,200.0000) -- (243.8000,200.0000) -- (243.8000,200.0000) -- (243.9000,200.0000) -- (244.0000,200.0000) -- (244.0000,200.0000) -- (244.1000,200.0000) -- (244.1000,200.0000) -- (244.2000,200.0000) -- (244.3000,200.0000) -- (244.3000,200.0000) -- (244.4000,200.0000) -- (244.4000,200.0000) -- (244.5000,200.0000) -- (244.6000,200.0000) -- (244.6000,200.0000) -- (244.7000,200.0000) -- (244.8000,200.0000) -- (244.8000,200.0000) -- (244.9000,200.0000) -- (244.9000,200.0000) -- (245.0000,200.0000) -- (245.1000,200.0000) -- (245.1000,200.0000) -- (245.2000,200.0000) -- (245.2000,200.0000) -- (245.3000,200.0000) -- (245.4000,200.0000) -- (245.4000,200.0000) -- (245.5000,200.0000) -- (245.6000,200.0000) -- (245.6000,200.0000) -- (245.7000,200.0000) -- (245.7000,200.0000) -- (245.8000,200.0000) -- (245.9000,200.0000) -- (245.9000,200.0000) -- (246.0000,200.0000) -- (246.0000,200.0000) -- (246.1000,200.0000) -- (246.2000,200.0000) -- (246.2000,200.0000) -- (246.3000,200.0000);



  \end{scope}
  \begin{scope}[cm={{0.92047,0.0,0.0,0.92047,(-569.12952,52.34945)}},draw=blue,line cap=round,line join=round,line width=0.480pt]
    \path[draw] (184.5000,156.5000) -- (184.5000,204.5000) -- (246.5000,204.5000) -- (246.5000,156.5000) -- (184.5000,156.5000);



  \end{scope}
  \begin{scope}[cm={{0.92743,0.0,0.0,0.92743,(-570.4103,65.15216)}},draw=ca0a0a4,dash pattern=on 2.59pt off 2.59pt,line cap=round,line join=round,line width=0.323pt,miter limit=4.00]
    \path[draw,dash pattern=on 2.59pt off 2.59pt,line width=0.323pt,miter limit=4.00] (184.5000,267.5000) -- (246.5000,267.5000);



  \end{scope}
  \begin{scope}[cm={{0.92743,0.0,0.0,0.92743,(-570.4103,65.15216)}},draw=blue,line cap=round,line join=round,line width=0.480pt]
    \path[draw] (184.5000,267.5000) -- (186.5000,267.5000);



    \path[draw] (246.5000,267.5000) -- (245.5000,267.5000);



  \end{scope}
  \begin{scope}[cm={{0.92743,0.0,0.0,0.92743,(-570.4103,65.15216)}},draw=ca0a0a4,dash pattern=on 2.59pt off 2.59pt,line cap=round,line join=round,line width=0.323pt,miter limit=4.00]
    \path[draw,dash pattern=on 2.59pt off 2.59pt,line width=0.323pt,miter limit=4.00] (184.5000,248.5000) -- (246.5000,248.5000);



  \end{scope}
  \begin{scope}[cm={{0.92743,0.0,0.0,0.92743,(-570.4103,65.15216)}},draw=blue,line cap=round,line join=round,line width=0.480pt]
    \path[draw] (184.5000,248.5000) -- (186.5000,248.5000);



    \path[draw] (246.5000,248.5000) -- (245.5000,248.5000);



  \end{scope}
  \begin{scope}[cm={{0.92743,0.0,0.0,0.92743,(-570.4103,65.15216)}},draw=ca0a0a4,dash pattern=on 2.59pt off 2.59pt,line cap=round,line join=round,line width=0.323pt,miter limit=4.00]
    \path[draw,dash pattern=on 2.59pt off 2.59pt,line width=0.323pt,miter limit=4.00] (184.5000,229.5000) -- (246.5000,229.5000);



  \end{scope}
  \begin{scope}[cm={{0.92743,0.0,0.0,0.92743,(-570.4103,65.15216)}},draw=blue,line cap=round,line join=round,line width=0.480pt]
    \path[draw] (184.5000,229.5000) -- (186.5000,229.5000);



    \path[draw] (246.5000,229.5000) -- (245.5000,229.5000);



  \end{scope}
  \begin{scope}[cm={{0.92743,0.0,0.0,0.92743,(-570.4103,65.15216)}},draw=ca0a0a4,dash pattern=on 0.40pt off 0.80pt,line cap=round,line join=round,line width=0.400pt]
    \path[draw] (184.5000,272.5000) -- (184.5000,224.5000);



  \end{scope}
  \begin{scope}[cm={{0.92743,0.0,0.0,0.92743,(-570.4103,65.15216)}},draw=blue,line cap=round,line join=round,line width=0.480pt]
    \path[draw] (184.5000,272.5000) -- (184.5000,270.5000);



    \path[draw] (184.5000,224.5000) -- (184.5000,225.5000);



  \end{scope}
  \begin{scope}[cm={{0.92743,0.0,0.0,0.92743,(-570.4103,65.15216)}},draw=ca0a0a4,dash pattern=on 2.59pt off 2.59pt,line cap=round,line join=round,line width=0.323pt,miter limit=4.00]
    \path[draw,dash pattern=on 2.59pt off 2.59pt,line width=0.323pt,miter limit=4.00] (201.5000,272.5000) -- (201.5000,224.5000);



  \end{scope}
  \begin{scope}[cm={{0.92743,0.0,0.0,0.92743,(-570.4103,65.15216)}},draw=blue,line cap=round,line join=round,line width=0.480pt]
    \path[draw] (201.5000,272.5000) -- (201.5000,270.5000);



    \path[draw] (201.5000,224.5000) -- (201.5000,225.5000);



  \end{scope}
  \begin{scope}[cm={{0.92743,0.0,0.0,0.92743,(-570.4103,65.15216)}},draw=ca0a0a4,dash pattern=on 2.59pt off 2.59pt,line cap=round,line join=round,line width=0.323pt,miter limit=4.00]
    \path[draw,dash pattern=on 2.59pt off 2.59pt,line width=0.323pt,miter limit=4.00] (218.5000,272.5000) -- (218.5000,224.5000);



  \end{scope}
  \begin{scope}[cm={{0.92743,0.0,0.0,0.92743,(-570.4103,65.15216)}},draw=blue,line cap=round,line join=round,line width=0.480pt]
    \path[draw] (218.5000,272.5000) -- (218.5000,270.5000);



    \path[draw] (218.5000,224.5000) -- (218.5000,225.5000);



  \end{scope}
  \begin{scope}[cm={{0.92743,0.0,0.0,0.92743,(-570.4103,65.15216)}},draw=ca0a0a4,dash pattern=on 2.59pt off 2.59pt,line cap=round,line join=round,line width=0.323pt,miter limit=4.00]
    \path[draw,dash pattern=on 2.59pt off 2.59pt,line width=0.323pt,miter limit=4.00] (235.5000,272.5000) -- (235.5000,224.5000);



  \end{scope}
  \begin{scope}[cm={{0.92743,0.0,0.0,0.92743,(-570.4103,65.15216)}},draw=blue,line cap=round,line join=round,line width=0.480pt]
    \path[draw] (235.5000,272.5000) -- (235.5000,270.5000);



    \path[draw] (235.5000,224.5000) -- (235.5000,225.5000);



  \end{scope}
  \begin{scope}[cm={{0.92743,0.0,0.0,0.92743,(-570.4103,65.15216)}},draw=blue,line cap=round,line join=round,line width=0.480pt]
    \path[draw] (246.5000,267.5000) -- (245.5000,267.5000);



  \end{scope}
  \begin{scope}[cm={{1.00588,0.0,0.0,1.00588,(-407.95798,313.77387)}},draw=blue,line cap=rect,line join=bevel,line width=0.800pt]
    \path[fill=blue] (0.0000,0.0000) node[above right] (text2120) {\scriptsize 2};



  \end{scope}
  \begin{scope}[cm={{0.92743,0.0,0.0,0.92743,(-570.4103,65.15216)}},draw=blue,line cap=round,line join=round,line width=0.480pt]
    \path[draw] (246.5000,248.5000) -- (245.5000,248.5000);



  \end{scope}
  \begin{scope}[cm={{1.00588,0.0,0.0,1.00588,(-407.90165,297.66187)}},draw=blue,line cap=rect,line join=bevel,line width=0.800pt]
    \path[fill=blue] (0.0000,0.0000) node[above right] (text2144) {\scriptsize 6};



  \end{scope}
  \begin{scope}[cm={{0.92743,0.0,0.0,0.92743,(-570.4103,65.15216)}},draw=blue,line cap=round,line join=round,line width=0.480pt]
    \path[draw] (246.5000,229.5000) -- (245.5000,229.5000);



  \end{scope}
  \begin{scope}[cm={{1.00588,0.0,0.0,1.00588,(-411.98953,280.05087)}},draw=blue,line cap=rect,line join=bevel,line width=0.800pt]
    \path[fill=blue] (0.0000,0.0000) node[above right] (text2168) {\scriptsize 10};



  \end{scope}
  \begin{scope}[cm={{0.92743,0.0,0.0,0.92743,(-570.4103,65.15216)}},draw=blue,line cap=round,line join=round,line width=0.480pt]
    \path[draw] (184.5000,224.5000) -- (184.5000,272.5000) -- (246.5000,272.5000) -- (246.5000,224.5000) -- (184.5000,224.5000);



  \end{scope}
  \begin{scope}[cm={{0.0,-1.00588,1.00588,0.0,(-419.24335,303.67087)}},draw=blue,line cap=rect,line join=bevel,line width=0.800pt]
    \path[fill=blue] (0.0000,-5.9329) node[above right] (text2192) {\rotatebox{90}{\scriptsize $c_{i,2}$}};



  \end{scope}
  \begin{scope}[cm={{0.92743,0.0,0.0,0.92743,(-570.4103,65.15216)}},draw=blue,line cap=round,line join=round,line width=0.480pt]
    \path[draw] (184.8000,267.2000) -- (184.8000,267.2000) -- (184.9000,267.2000) -- (185.0000,267.2000) -- (185.0000,267.2000) -- (185.1000,267.2000) -- (185.1000,267.2000) -- (185.2000,267.2000) -- (185.3000,267.2000) -- (185.3000,267.2000) -- (185.4000,267.2000) -- (185.4000,267.2000) -- (185.5000,267.2000) -- (185.6000,267.2000) -- (185.6000,267.2000) -- (185.7000,267.2000) -- (185.8000,267.2000) -- (185.8000,267.2000) -- (185.9000,267.2000) -- (185.9000,267.2000) -- (186.0000,267.2000) -- (186.1000,267.2000) -- (186.1000,267.2000) -- (186.2000,267.2000) -- (186.2000,267.2000) -- (186.3000,267.2000) -- (186.4000,267.2000) -- (186.4000,267.2000) -- (186.5000,267.2000) -- (186.6000,267.2000) -- (186.6000,267.2000) -- (186.7000,267.2000) -- (186.7000,267.2000) -- (186.8000,267.2000) -- (186.9000,267.2000) -- (186.9000,267.2000) -- (187.0000,267.2000) -- (187.0000,267.2000) -- (187.1000,267.2000) -- (187.2000,267.2000) -- (187.2000,267.2000) -- (187.3000,267.2000) -- (187.4000,267.2000) -- (187.4000,267.2000) -- (187.5000,267.2000) -- (187.5000,267.2000) -- (187.6000,267.2000) -- (187.7000,267.2000) -- (187.7000,267.2000) -- (187.8000,267.2000) -- (187.8000,267.2000) -- (187.9000,267.2000) -- (188.0000,267.2000) -- (188.0000,267.2000) -- (188.1000,267.2000) -- (188.2000,267.2000) -- (188.2000,267.2000) -- (188.3000,267.2000) -- (188.3000,267.2000) -- (188.4000,267.2000) -- (188.5000,267.2000) -- (188.5000,267.2000) -- (188.6000,267.2000) -- (188.6000,267.2000) -- (188.7000,267.2000) -- (188.8000,267.2000) -- (188.8000,267.2000) -- (188.9000,267.2000) -- (189.0000,267.2000) -- (189.0000,267.2000) -- (189.1000,267.2000) -- (189.1000,267.2000) -- (189.2000,267.2000) -- (189.3000,267.2000) -- (189.3000,267.2000) -- (189.4000,267.2000) -- (189.4000,267.2000) -- (189.5000,267.2000) -- (189.6000,267.2000) -- (189.6000,267.2000) -- (189.7000,267.2000) -- (189.8000,267.2000) -- (189.8000,267.2000) -- (189.9000,267.2000) -- (189.9000,267.2000) -- (190.0000,267.2000) -- (190.1000,267.2000) -- (190.1000,267.2000) -- (190.2000,267.2000) -- (190.2000,267.2000) -- (190.3000,267.2000) -- (190.4000,267.2000) -- (190.4000,267.2000) -- (190.5000,267.2000) -- (190.5000,267.2000) -- (190.6000,267.2000) -- (190.7000,267.2000) -- (190.7000,267.2000) -- (190.8000,267.2000) -- (190.9000,267.2000) -- (190.9000,267.2000) -- (191.0000,267.2000) -- (191.0000,267.2000) -- (191.1000,267.2000) -- (191.2000,267.2000) -- (191.2000,267.2000) -- (191.3000,267.2000) -- (191.3000,267.2000) -- (191.4000,267.2000) -- (191.5000,267.2000) -- (191.5000,267.2000) -- (191.6000,267.2000) -- (191.7000,267.2000) -- (191.7000,267.2000) -- (191.8000,267.2000) -- (191.8000,267.2000) -- (191.9000,267.2000) -- (192.0000,267.2000) -- (192.0000,267.2000) -- (192.1000,267.2000) -- (192.1000,267.2000) -- (192.2000,267.2000) -- (192.3000,267.2000) -- (192.3000,267.2000) -- (192.4000,267.2000) -- (192.5000,267.2000) -- (192.5000,267.2000) -- (192.6000,267.3000) -- (192.6000,268.7000) -- (192.7000,244.7000) -- (192.8000,227.7000) -- (192.8000,229.2000) -- (192.9000,229.2000) -- (192.9000,229.2000) -- (193.0000,229.2000) -- (193.1000,229.2000) -- (193.1000,229.2000) -- (193.2000,229.2000) -- (193.3000,229.2000) -- (193.3000,229.2000) -- (193.4000,229.2000) -- (193.4000,229.2000) -- (193.5000,229.2000) -- (193.6000,229.2000) -- (193.6000,229.2000) -- (193.7000,229.2000) -- (193.7000,229.2000) -- (193.8000,229.2000) -- (193.9000,229.2000) -- (193.9000,229.2000) -- (194.0000,229.2000) -- (194.1000,229.2000) -- (194.1000,229.2000) -- (194.2000,229.2000) -- (194.2000,229.2000) -- (194.3000,229.2000) -- (194.4000,229.2000) -- (194.4000,229.2000) -- (194.5000,229.2000) -- (194.5000,229.2000) -- (194.6000,229.2000) -- (194.7000,229.2000) -- (194.7000,229.2000) -- (194.8000,229.2000) -- (194.9000,229.2000) -- (194.9000,229.2000) -- (195.0000,229.2000) -- (195.0000,229.2000) -- (195.1000,229.2000) -- (195.2000,229.2000) -- (195.2000,229.2000) -- (195.3000,229.2000) -- (195.3000,229.2000) -- (195.4000,229.2000) -- (195.5000,229.2000) -- (195.5000,229.2000) -- (195.6000,229.2000) -- (195.7000,229.2000) -- (195.7000,229.2000) -- (195.8000,229.2000) -- (195.8000,229.2000) -- (195.9000,229.2000) -- (196.0000,229.2000) -- (196.0000,229.2000) -- (196.1000,229.2000) -- (196.1000,229.2000) -- (196.2000,229.2000) -- (196.3000,229.2000) -- (196.3000,229.2000) -- (196.4000,229.2000) -- (196.5000,229.2000) -- (196.5000,229.2000) -- (196.6000,229.2000) -- (196.6000,229.2000) -- (196.7000,229.2000) -- (196.8000,229.2000) -- (196.8000,229.2000) -- (196.9000,229.2000) -- (196.9000,229.2000) -- (197.0000,229.2000) -- (197.1000,229.2000) -- (197.1000,229.2000) -- (197.2000,229.2000) -- (197.3000,229.2000) -- (197.3000,229.2000) -- (197.4000,229.2000) -- (197.4000,229.2000) -- (197.5000,229.2000) -- (197.6000,229.2000) -- (197.6000,229.2000) -- (197.7000,229.2000) -- (197.7000,229.2000) -- (197.8000,229.2000) -- (197.9000,229.2000) -- (197.9000,229.2000) -- (198.0000,229.2000) -- (198.1000,229.2000) -- (198.1000,229.2000) -- (198.2000,229.2000) -- (198.2000,229.2000) -- (198.3000,229.2000) -- (198.4000,229.2000) -- (198.4000,229.2000) -- (198.5000,229.2000) -- (198.5000,229.2000) -- (198.6000,229.2000) -- (198.7000,229.2000) -- (198.7000,229.2000) -- (198.8000,229.2000) -- (198.9000,229.2000) -- (198.9000,229.2000) -- (199.0000,229.2000) -- (199.0000,229.2000) -- (199.1000,229.2000) -- (199.2000,229.2000) -- (199.2000,229.2000) -- (199.3000,229.2000) -- (199.3000,229.2000) -- (199.4000,229.2000) -- (199.5000,229.2000) -- (199.5000,229.2000) -- (199.6000,229.2000) -- (199.7000,229.2000) -- (199.7000,229.2000) -- (199.8000,229.2000) -- (199.8000,229.2000) -- (199.9000,229.2000) -- (200.0000,229.2000) -- (200.0000,229.2000) -- (200.1000,229.2000) -- (200.1000,229.2000) -- (200.2000,229.2000) -- (200.3000,229.2000) -- (200.3000,229.2000) -- (200.4000,229.2000) -- (200.5000,229.2000) -- (200.5000,229.2000) -- (200.6000,229.2000) -- (200.6000,229.2000) -- (200.7000,229.2000) -- (200.8000,229.2000) -- (200.8000,229.2000) -- (200.9000,229.2000) -- (200.9000,229.2000) -- (201.0000,229.2000) -- (201.1000,229.2000) -- (201.1000,229.2000) -- (201.2000,229.2000) -- (201.3000,229.2000) -- (201.3000,229.2000) -- (201.4000,229.2000) -- (201.4000,229.2000) -- (201.5000,229.2000) -- (201.6000,229.2000) -- (201.6000,229.2000) -- (201.7000,229.2000) -- (201.7000,229.2000) -- (201.8000,229.2000) -- (201.9000,229.2000) -- (201.9000,229.2000) -- (202.0000,229.2000) -- (202.0000,229.2000) -- (202.1000,229.2000) -- (202.2000,229.2000) -- (202.2000,229.2000) -- (202.3000,229.2000) -- (202.4000,229.2000) -- (202.4000,229.2000) -- (202.5000,229.2000) -- (202.5000,229.2000) -- (202.6000,229.2000) -- (202.7000,229.2000) -- (202.7000,229.2000) -- (202.8000,229.2000) -- (202.8000,229.2000) -- (202.9000,229.2000) -- (203.0000,229.2000) -- (203.0000,229.2000) -- (203.1000,229.2000) -- (203.2000,229.2000) -- (203.2000,229.2000) -- (203.3000,229.2000) -- (203.3000,229.2000) -- (203.4000,229.2000) -- (203.5000,229.2000) -- (203.5000,229.2000) -- (203.6000,229.2000) -- (203.6000,229.2000) -- (203.7000,229.2000) -- (203.8000,229.2000) -- (203.8000,229.2000) -- (203.9000,229.2000) -- (204.0000,229.2000) -- (204.0000,229.2000) -- (204.1000,229.2000) -- (204.1000,229.2000) -- (204.2000,229.2000) -- (204.3000,229.2000) -- (204.3000,229.2000) -- (204.4000,229.2000) -- (204.4000,229.2000) -- (204.5000,229.2000) -- (204.6000,229.2000) -- (204.6000,229.2000) -- (204.7000,229.2000) -- (204.8000,229.2000) -- (204.8000,229.2000) -- (204.9000,229.2000) -- (204.9000,229.2000) -- (205.0000,229.2000) -- (205.1000,229.2000) -- (205.1000,229.2000) -- (205.2000,229.2000) -- (205.2000,229.2000) -- (205.3000,229.2000) -- (205.4000,229.2000) -- (205.4000,229.2000) -- (205.5000,229.2000) -- (205.6000,229.2000) -- (205.6000,229.2000) -- (205.7000,229.2000) -- (205.7000,229.2000) -- (205.8000,229.2000) -- (205.9000,229.2000) -- (205.9000,229.2000) -- (206.0000,229.2000) -- (206.0000,229.2000) -- (206.1000,229.2000) -- (206.2000,229.2000) -- (206.2000,229.2000) -- (206.3000,229.2000) -- (206.4000,229.2000) -- (206.4000,229.2000) -- (206.5000,229.2000) -- (206.5000,229.2000) -- (206.6000,229.2000) -- (206.7000,229.2000) -- (206.7000,229.2000) -- (206.8000,229.2000) -- (206.8000,229.2000) -- (206.9000,229.2000) -- (207.0000,229.2000) -- (207.0000,229.2000) -- (207.1000,229.2000) -- (207.2000,229.2000) -- (207.2000,229.2000) -- (207.3000,229.2000) -- (207.3000,229.2000) -- (207.4000,229.2000) -- (207.5000,229.2000) -- (207.5000,229.2000) -- (207.6000,229.2000) -- (207.6000,229.2000) -- (207.7000,229.2000) -- (207.8000,229.2000) -- (207.8000,229.2000) -- (207.9000,229.2000) -- (208.0000,229.2000) -- (208.0000,229.2000) -- (208.1000,229.2000) -- (208.1000,229.2000) -- (208.2000,229.2000) -- (208.3000,229.2000) -- (208.3000,229.2000) -- (208.4000,229.2000) -- (208.4000,229.2000) -- (208.5000,229.2000) -- (208.6000,229.2000) -- (208.6000,229.2000) -- (208.7000,229.2000) -- (208.8000,229.2000) -- (208.8000,229.2000) -- (208.9000,229.2000) -- (208.9000,229.2000) -- (209.0000,229.2000) -- (209.1000,229.2000) -- (209.1000,229.2000) -- (209.2000,229.2000) -- (209.2000,229.2000) -- (209.3000,229.2000) -- (209.4000,229.2000) -- (209.4000,229.2000) -- (209.5000,229.2000) -- (209.6000,229.2000) -- (209.6000,229.2000) -- (209.7000,229.2000) -- (209.7000,229.2000) -- (209.8000,229.2000) -- (209.9000,229.2000) -- (209.9000,229.2000) -- (210.0000,229.2000) -- (210.0000,229.2000) -- (210.1000,229.2000) -- (210.2000,229.2000) -- (210.2000,229.2000) -- (210.3000,229.2000) -- (210.4000,229.2000) -- (210.4000,229.2000) -- (210.5000,229.2000) -- (210.5000,229.2000) -- (210.6000,229.2000) -- (210.7000,229.2000) -- (210.7000,229.2000) -- (210.8000,229.2000) -- (210.8000,229.2000) -- (210.9000,229.2000) -- (211.0000,229.2000) -- (211.0000,229.2000) -- (211.1000,229.2000) -- (211.2000,229.2000) -- (211.2000,229.2000) -- (211.3000,229.2000) -- (211.3000,229.2000) -- (211.4000,229.2000) -- (211.5000,229.2000) -- (211.5000,229.2000) -- (211.6000,229.2000) -- (211.6000,229.2000) -- (211.7000,229.2000) -- (211.8000,229.2000) -- (211.8000,229.2000) -- (211.9000,229.2000) -- (212.0000,229.2000) -- (212.0000,229.2000) -- (212.1000,229.2000) -- (212.1000,229.2000) -- (212.2000,229.2000) -- (212.3000,229.2000) -- (212.3000,229.2000) -- (212.4000,229.2000) -- (212.4000,229.2000) -- (212.5000,229.2000) -- (212.6000,229.2000) -- (212.6000,229.2000) -- (212.7000,229.2000) -- (212.8000,229.2000) -- (212.8000,229.2000) -- (212.9000,229.2000) -- (212.9000,229.2000) -- (213.0000,229.2000) -- (213.1000,229.2000) -- (213.1000,229.2000) -- (213.2000,229.2000) -- (213.2000,229.2000) -- (213.3000,229.2000) -- (213.4000,229.2000) -- (213.4000,229.2000) -- (213.5000,229.2000) -- (213.6000,229.2000) -- (213.6000,229.2000) -- (213.7000,229.2000) -- (213.7000,229.2000) -- (213.8000,229.2000) -- (213.9000,229.2000) -- (213.9000,229.2000) -- (214.0000,229.2000) -- (214.0000,229.2000) -- (214.1000,229.2000) -- (214.2000,229.2000) -- (214.2000,229.2000) -- (214.3000,229.2000) -- (214.4000,229.2000) -- (214.4000,229.2000) -- (214.5000,229.2000) -- (214.5000,229.2000) -- (214.6000,229.2000) -- (214.7000,229.2000) -- (214.7000,229.2000) -- (214.8000,229.2000) -- (214.8000,229.2000) -- (214.9000,229.2000) -- (215.0000,229.2000) -- (215.0000,229.2000) -- (215.1000,229.2000) -- (215.1000,229.2000) -- (215.2000,229.2000) -- (215.3000,229.2000) -- (215.3000,229.2000) -- (215.4000,229.2000) -- (215.5000,229.2000) -- (215.5000,229.2000) -- (215.6000,229.2000) -- (215.6000,229.2000) -- (215.7000,229.2000) -- (215.8000,229.2000) -- (215.8000,229.2000) -- (215.9000,229.2000) -- (215.9000,228.9000) -- (216.0000,233.3000) -- (216.1000,239.1000) -- (216.1000,238.8000) -- (216.2000,238.7000) -- (216.3000,238.7000) -- (216.3000,238.7000) -- (216.4000,238.7000) -- (216.4000,238.7000) -- (216.5000,238.7000) -- (216.6000,238.7000) -- (216.6000,238.7000) -- (216.7000,238.7000) -- (216.7000,238.7000) -- (216.8000,238.7000) -- (216.9000,238.7000) -- (216.9000,238.7000) -- (217.0000,238.7000) -- (217.1000,238.7000) -- (217.1000,238.7000) -- (217.2000,238.7000) -- (217.2000,238.7000) -- (217.3000,238.7000) -- (217.4000,238.7000) -- (217.4000,238.7000) -- (217.5000,238.7000) -- (217.5000,238.7000) -- (217.6000,239.0000) -- (217.7000,236.4000) -- (217.7000,229.1000) -- (217.8000,229.2000) -- (217.9000,229.2000) -- (217.9000,229.2000) -- (218.0000,229.2000) -- (218.0000,229.2000) -- (218.1000,229.2000) -- (218.2000,229.2000) -- (218.2000,229.2000) -- (218.3000,229.2000) -- (218.3000,229.2000) -- (218.4000,229.2000) -- (218.5000,229.2000) -- (218.5000,229.2000) -- (218.6000,229.2000) -- (218.7000,229.2000) -- (218.7000,229.2000) -- (218.8000,229.2000) -- (218.8000,229.2000) -- (218.9000,229.2000) -- (219.0000,229.2000) -- (219.0000,229.2000) -- (219.1000,229.2000) -- (219.1000,229.2000) -- (219.2000,229.2000) -- (219.3000,229.2000) -- (219.3000,229.2000) -- (219.4000,229.2000) -- (219.5000,229.2000) -- (219.5000,229.2000) -- (219.6000,229.2000) -- (219.6000,229.2000) -- (219.7000,229.2000) -- (219.8000,229.2000) -- (219.8000,229.2000) -- (219.9000,229.2000) -- (219.9000,229.2000) -- (220.0000,229.2000) -- (220.1000,229.2000) -- (220.1000,228.9000) -- (220.2000,233.5000) -- (220.3000,239.1000) -- (220.3000,238.8000) -- (220.4000,238.7000) -- (220.4000,239.1000) -- (220.5000,247.0000) -- (220.6000,248.4000) -- (220.6000,248.2000) -- (220.7000,248.2000) -- (220.7000,248.2000) -- (220.8000,248.2000) -- (220.9000,248.2000) -- (220.9000,248.2000) -- (221.0000,248.2000) -- (221.1000,248.2000) -- (221.1000,248.2000) -- (221.2000,248.2000) -- (221.2000,248.2000) -- (221.3000,248.2000) -- (221.4000,248.2000) -- (221.4000,248.2000) -- (221.5000,248.2000) -- (221.5000,248.2000) -- (221.6000,248.2000) -- (221.7000,248.2000) -- (221.7000,248.2000) -- (221.8000,248.2000) -- (221.9000,248.2000) -- (221.9000,248.2000) -- (222.0000,248.2000) -- (222.0000,248.3000) -- (222.1000,248.5000) -- (222.2000,241.5000) -- (222.2000,238.4000) -- (222.3000,238.7000) -- (222.3000,239.0000) -- (222.4000,236.7000) -- (222.5000,229.2000) -- (222.5000,229.2000) -- (222.6000,229.2000) -- (222.7000,229.2000) -- (222.7000,229.2000) -- (222.8000,229.2000) -- (222.8000,229.2000) -- (222.9000,229.2000) -- (223.0000,229.2000) -- (223.0000,229.2000) -- (223.1000,229.2000) -- (223.1000,229.2000) -- (223.2000,229.2000) -- (223.3000,229.2000) -- (223.3000,229.2000) -- (223.4000,229.2000) -- (223.5000,229.2000) -- (223.5000,229.2000) -- (223.6000,229.2000) -- (223.6000,229.2000) -- (223.7000,229.2000) -- (223.8000,229.2000) -- (223.8000,229.2000) -- (223.9000,229.2000) -- (223.9000,229.2000) -- (224.0000,229.2000) -- (224.1000,229.2000) -- (224.1000,229.2000) -- (224.2000,229.2000) -- (224.3000,229.2000) -- (224.3000,229.2000) -- (224.4000,229.2000) -- (224.4000,229.8000) -- (224.5000,237.7000) -- (224.6000,238.8000) -- (224.6000,239.2000) -- (224.7000,247.1000) -- (224.7000,248.4000) -- (224.8000,248.2000) -- (224.9000,247.9000) -- (224.9000,252.1000) -- (225.0000,258.1000) -- (225.1000,257.8000) -- (225.1000,257.7000) -- (225.2000,257.7000) -- (225.2000,257.7000) -- (225.3000,257.7000) -- (225.4000,257.7000) -- (225.4000,257.7000) -- (225.5000,257.7000) -- (225.5000,257.7000) -- (225.6000,257.7000) -- (225.7000,257.7000) -- (225.7000,257.7000) -- (225.8000,257.7000) -- (225.9000,257.7000) -- (225.9000,257.7000) -- (226.0000,257.7000) -- (226.0000,257.7000) -- (226.1000,257.7000) -- (226.2000,257.7000) -- (226.2000,257.7000) -- (226.3000,257.7000) -- (226.3000,257.7000) -- (226.4000,257.7000) -- (226.5000,257.7000) -- (226.5000,258.0000) -- (226.6000,255.6000) -- (226.6000,248.2000) -- (226.7000,248.2000) -- (226.8000,248.3000) -- (226.8000,248.5000) -- (226.9000,241.8000) -- (227.0000,238.4000) -- (227.0000,239.1000) -- (227.1000,232.5000) -- (227.1000,228.9000) -- (227.2000,229.2000) -- (227.3000,229.2000) -- (227.3000,229.2000) -- (227.4000,229.2000) -- (227.4000,229.2000) -- (227.5000,229.2000) -- (227.6000,229.2000) -- (227.6000,229.2000) -- (227.7000,229.2000) -- (227.8000,229.2000) -- (227.8000,229.2000) -- (227.9000,229.2000) -- (227.9000,229.2000) -- (228.0000,229.2000) -- (228.1000,229.2000) -- (228.1000,229.2000) -- (228.2000,229.2000) -- (228.2000,229.2000) -- (228.3000,229.2000) -- (228.4000,229.2000) -- (228.4000,229.2000) -- (228.5000,229.2000) -- (228.6000,229.2000) -- (228.6000,229.2000) -- (228.7000,229.2000) -- (228.7000,229.2000) -- (228.8000,229.8000) -- (228.9000,237.3000) -- (228.9000,243.1000) -- (229.0000,248.6000) -- (229.0000,247.9000) -- (229.1000,252.3000) -- (229.2000,258.1000) -- (229.2000,257.8000) -- (229.3000,257.7000) -- (229.4000,258.0000) -- (229.4000,265.8000) -- (229.5000,267.4000) -- (229.5000,267.2000) -- (229.6000,267.2000) -- (229.7000,267.2000) -- (229.7000,267.2000) -- (229.8000,267.2000) -- (229.8000,267.2000) -- (229.9000,267.2000) -- (230.0000,267.2000) -- (230.0000,267.2000) -- (230.1000,267.2000) -- (230.2000,267.2000) -- (230.2000,267.2000) -- (230.3000,267.2000) -- (230.3000,267.2000) -- (230.4000,267.2000) -- (230.5000,267.2000) -- (230.5000,267.2000) -- (230.6000,267.2000) -- (230.6000,267.2000) -- (230.7000,267.2000) -- (230.8000,267.2000) -- (230.8000,267.2000) -- (230.9000,267.2000) -- (231.0000,267.3000) -- (231.0000,267.5000) -- (231.1000,260.7000) -- (231.1000,257.4000) -- (231.2000,257.7000) -- (231.3000,258.0000) -- (231.3000,255.9000) -- (231.4000,248.3000) -- (231.4000,248.4000) -- (231.5000,246.6000) -- (231.6000,238.9000) -- (231.6000,238.9000) -- (231.7000,237.2000) -- (231.8000,229.4000) -- (231.8000,229.2000) -- (231.9000,229.2000) -- (231.9000,229.2000) -- (232.0000,229.2000) -- (232.1000,229.2000) -- (232.1000,229.2000) -- (232.2000,229.2000) -- (232.2000,229.2000) -- (232.3000,229.2000) -- (232.4000,229.2000) -- (232.4000,229.2000) -- (232.5000,229.2000) -- (232.6000,229.2000) -- (232.6000,229.2000) -- (232.7000,229.2000) -- (232.7000,229.2000) -- (232.8000,229.2000) -- (232.9000,229.2000) -- (232.9000,229.2000) -- (233.0000,229.2000) -- (233.0000,229.2000) -- (233.1000,229.2000) -- (233.2000,229.8000) -- (233.2000,236.6000) -- (233.3000,251.6000) -- (233.4000,268.4000) -- (233.4000,267.3000) -- (233.5000,267.2000) -- (233.5000,267.2000) -- (233.6000,267.2000) -- (233.7000,267.2000) -- (233.7000,267.2000) -- (233.8000,267.2000) -- (233.8000,267.2000) -- (233.9000,267.2000) -- (234.0000,267.2000) -- (234.0000,267.2000) -- (234.1000,267.2000) -- (234.2000,267.2000) -- (234.2000,267.2000) -- (234.3000,267.2000) -- (234.3000,267.2000) -- (234.4000,267.2000) -- (234.5000,267.2000) -- (234.5000,267.2000) -- (234.6000,267.2000) -- (234.6000,267.2000) -- (234.7000,267.2000) -- (234.8000,267.2000) -- (234.8000,267.2000) -- (234.9000,267.2000) -- (235.0000,267.2000) -- (235.0000,267.2000) -- (235.1000,267.2000) -- (235.1000,267.2000) -- (235.2000,267.2000) -- (235.3000,267.2000) -- (235.3000,267.2000) -- (235.4000,267.2000) -- (235.4000,267.2000) -- (235.5000,267.2000) -- (235.6000,267.2000) -- (235.6000,267.2000) -- (235.7000,267.3000) -- (235.8000,267.6000) -- (235.8000,261.0000) -- (235.9000,257.4000) -- (235.9000,258.1000) -- (236.0000,251.7000) -- (236.1000,247.9000) -- (236.1000,248.6000) -- (236.2000,242.4000) -- (236.2000,238.4000) -- (236.3000,239.1000) -- (236.4000,233.1000) -- (236.4000,228.9000) -- (236.5000,229.2000) -- (236.6000,229.2000) -- (236.6000,229.2000) -- (236.7000,229.2000) -- (236.7000,229.2000) -- (236.8000,229.2000) -- (236.9000,229.2000) -- (236.9000,229.2000) -- (237.0000,229.2000) -- (237.0000,229.2000) -- (237.1000,229.1000) -- (237.2000,230.0000) -- (237.2000,238.0000) -- (237.3000,238.9000) -- (237.4000,238.7000) -- (237.4000,238.7000) -- (237.5000,238.7000) -- (237.5000,238.7000) -- (237.6000,237.6000) -- (237.7000,251.4000) -- (237.7000,268.4000) -- (237.8000,267.3000) -- (237.8000,267.2000) -- (237.9000,267.2000) -- (238.0000,267.2000) -- (238.0000,267.2000) -- (238.1000,267.2000) -- (238.1000,267.2000) -- (238.2000,267.2000) -- (238.3000,267.2000) -- (238.3000,267.2000) -- (238.4000,267.2000) -- (238.5000,267.2000) -- (238.5000,267.2000) -- (238.6000,267.2000) -- (238.6000,267.2000) -- (238.7000,267.2000) -- (238.8000,267.2000) -- (238.8000,267.2000) -- (238.9000,267.2000) -- (238.9000,267.2000) -- (239.0000,267.2000) -- (239.1000,267.2000) -- (239.1000,267.2000) -- (239.2000,267.2000) -- (239.3000,267.2000) -- (239.3000,267.2000) -- (239.4000,267.2000) -- (239.4000,267.2000) -- (239.5000,267.2000) -- (239.6000,267.2000) -- (239.6000,267.2000) -- (239.7000,267.2000) -- (239.7000,267.2000) -- (239.8000,267.2000) -- (239.9000,267.2000) -- (239.9000,267.2000) -- (240.0000,267.2000) -- (240.1000,267.2000) -- (240.1000,267.2000) -- (240.2000,267.2000) -- (240.2000,267.2000) -- (240.3000,267.2000) -- (240.4000,267.5000) -- (240.4000,265.7000) -- (240.5000,258.0000) -- (240.5000,257.9000) -- (240.6000,256.4000) -- (240.7000,248.5000) -- (240.7000,248.4000) -- (240.8000,247.0000) -- (240.9000,239.1000) -- (240.9000,238.7000) -- (241.0000,238.7000) -- (241.0000,238.7000) -- (241.1000,238.7000) -- (241.2000,238.7000) -- (241.2000,238.7000) -- (241.3000,238.7000) -- (241.3000,238.7000) -- (241.4000,238.7000) -- (241.5000,238.6000) -- (241.5000,239.5000) -- (241.6000,247.5000) -- (241.7000,248.4000) -- (241.7000,248.2000) -- (241.8000,248.2000) -- (241.8000,248.1000) -- (241.9000,249.3000) -- (242.0000,265.2000) -- (242.0000,267.6000) -- (242.1000,267.2000) -- (242.1000,267.2000) -- (242.2000,267.2000) -- (242.3000,267.2000) -- (242.3000,267.2000) -- (242.4000,267.2000) -- (242.5000,267.2000) -- (242.5000,267.2000) -- (242.6000,267.2000) -- (242.6000,267.2000) -- (242.7000,267.2000) -- (242.8000,267.2000) -- (242.8000,267.2000) -- (242.9000,267.2000) -- (242.9000,267.2000) -- (243.0000,267.2000) -- (243.1000,267.2000) -- (243.1000,267.2000) -- (243.2000,267.2000) -- (243.3000,267.2000) -- (243.3000,267.2000) -- (243.4000,267.2000) -- (243.4000,267.2000) -- (243.5000,267.2000) -- (243.6000,267.2000) -- (243.6000,267.2000) -- (243.7000,267.2000) -- (243.7000,267.2000) -- (243.8000,267.2000) -- (243.9000,267.2000) -- (243.9000,267.2000) -- (244.0000,267.2000) -- (244.1000,267.2000) -- (244.1000,267.2000) -- (244.2000,267.2000) -- (244.2000,267.2000) -- (244.3000,267.2000) -- (244.4000,267.2000) -- (244.4000,267.2000) -- (244.5000,267.2000) -- (244.5000,267.2000) -- (244.6000,267.2000) -- (244.7000,267.2000) -- (244.7000,267.2000) -- (244.8000,267.2000) -- (244.9000,267.2000) -- (244.9000,267.2000) -- (245.0000,267.3000) -- (245.0000,267.6000) -- (245.1000,261.6000) -- (245.2000,257.4000) -- (245.2000,258.1000) -- (245.3000,252.3000) -- (245.3000,247.9000) -- (245.4000,248.2000) -- (245.5000,248.2000) -- (245.5000,248.2000) -- (245.6000,248.2000) -- (245.7000,248.2000) -- (245.7000,248.2000) -- (245.8000,248.2000) -- (245.8000,248.2000) -- (245.9000,248.2000) -- (246.0000,247.9000) -- (246.0000,252.9000) -- (246.1000,258.1000) -- (246.1000,257.4000) -- (246.2000,262.2000) -- (246.3000,267.6000);



  \end{scope}
  \begin{scope}[cm={{0.92743,0.0,0.0,0.92743,(-570.4103,65.15216)}},draw=blue,line cap=round,line join=round,line width=0.480pt]
    \path[draw] (184.5000,224.5000) -- (184.5000,272.5000) -- (246.5000,272.5000) -- (246.5000,224.5000) -- (184.5000,224.5000);



  \end{scope}
  \begin{scope}[cm={{0.92743,0.0,0.0,0.92743,(-570.4103,65.15216)}},draw=ca0a0a4,dash pattern=on 2.59pt off 2.59pt,line cap=round,line join=round,line width=0.323pt,miter limit=4.00]
    \path[draw,dash pattern=on 2.59pt off 2.59pt,line width=0.323pt,miter limit=4.00] (184.5000,315.5000) -- (246.5000,315.5000);



  \end{scope}
  \begin{scope}[cm={{0.92743,0.0,0.0,0.92743,(-570.4103,65.15216)}},draw=blue,line cap=round,line join=round,line width=0.480pt]
    \path[draw] (184.5000,315.5000) -- (186.5000,315.5000);



    \path[draw] (246.5000,315.5000) -- (245.5000,315.5000);



  \end{scope}
  \begin{scope}[cm={{0.92743,0.0,0.0,0.92743,(-570.4103,65.15216)}},draw=ca0a0a4,dash pattern=on 2.59pt off 2.59pt,line cap=round,line join=round,line width=0.323pt,miter limit=4.00]
    \path[draw,dash pattern=on 2.59pt off 2.59pt,line width=0.323pt,miter limit=4.00] (184.5000,295.5000) -- (246.5000,295.5000);



  \end{scope}
  \begin{scope}[cm={{0.92743,0.0,0.0,0.92743,(-570.4103,65.15216)}},draw=blue,line cap=round,line join=round,line width=0.480pt]
    \path[draw] (184.5000,295.5000) -- (186.5000,295.5000);



    \path[draw] (246.5000,295.5000) -- (245.5000,295.5000);



  \end{scope}
  \begin{scope}[cm={{0.92743,0.0,0.0,0.92743,(-570.4103,65.15216)}},draw=ca0a0a4,dash pattern=on 2.59pt off 2.59pt,line cap=round,line join=round,line width=0.323pt,miter limit=4.00]
    \path[draw,dash pattern=on 2.59pt off 2.59pt,line width=0.323pt,miter limit=4.00] (184.5000,276.5000) -- (246.5000,276.5000);



  \end{scope}
  \begin{scope}[cm={{0.92743,0.0,0.0,0.92743,(-570.4103,65.15216)}},draw=blue,line cap=round,line join=round,line width=0.480pt]
    \path[draw] (184.5000,276.5000) -- (186.5000,276.5000);



    \path[draw] (246.5000,276.5000) -- (245.5000,276.5000);



  \end{scope}
  \begin{scope}[cm={{0.92743,0.0,0.0,0.92743,(-570.4103,65.15216)}},draw=ca0a0a4,dash pattern=on 0.40pt off 0.80pt,line cap=round,line join=round,line width=0.400pt]
    \path[draw] (184.5000,319.5000) -- (184.5000,272.5000);



  \end{scope}
  \begin{scope}[cm={{0.92743,0.0,0.0,0.92743,(-570.4103,65.15216)}},draw=blue,line cap=round,line join=round,line width=0.480pt]
    \path[draw] (184.5000,319.5000) -- (184.5000,318.5000);



    \path[draw] (184.5000,272.5000) -- (184.5000,273.5000);



  \end{scope}
  \begin{scope}[cm={{1.00588,0.0,0.0,1.00588,(-400.29035,376.888)}},draw=blue,line cap=rect,line join=bevel,line width=0.800pt]
    \path[fill=blue] (0.0000,0.0000) node[above right] (text2316) {\scriptsize 0};



  \end{scope}
  \begin{scope}[cm={{0.92743,0.0,0.0,0.92743,(-570.4103,65.15216)}},draw=ca0a0a4,dash pattern=on 2.59pt off 2.59pt,line cap=round,line join=round,line width=0.323pt,miter limit=4.00]
    \path[draw,dash pattern=on 2.59pt off 2.59pt,line width=0.323pt,miter limit=4.00] (201.5000,319.5000) -- (201.5000,272.5000);



  \end{scope}
  \begin{scope}[cm={{0.92743,0.0,0.0,0.92743,(-570.4103,65.15216)}},draw=blue,line cap=round,line join=round,line width=0.480pt]
    \path[draw] (201.5000,319.5000) -- (201.5000,318.5000);



    \path[draw] (201.5000,272.5000) -- (201.5000,273.5000);



  \end{scope}
  \begin{scope}[cm={{1.00588,0.0,0.0,1.00588,(-385.68735,376.888)}},draw=blue,line cap=rect,line join=bevel,line width=0.800pt]
    \path[fill=blue] (0.0000,0.0000) node[above right] (text2346) {\scriptsize 3};



  \end{scope}
  \begin{scope}[cm={{0.92743,0.0,0.0,0.92743,(-570.4103,65.15216)}},draw=ca0a0a4,dash pattern=on 2.59pt off 2.59pt,line cap=round,line join=round,line width=0.323pt,miter limit=4.00]
    \path[draw,dash pattern=on 2.59pt off 2.59pt,line width=0.323pt,miter limit=4.00] (218.5000,319.5000) -- (218.5000,272.5000);



  \end{scope}
  \begin{scope}[cm={{0.92743,0.0,0.0,0.92743,(-570.4103,65.15216)}},draw=blue,line cap=round,line join=round,line width=0.480pt]
    \path[draw] (218.5000,319.5000) -- (218.5000,318.5000);



    \path[draw] (218.5000,272.5000) -- (218.5000,273.5000);



  \end{scope}
  \begin{scope}[cm={{1.00588,0.0,0.0,1.00588,(-370.09635,376.888)}},draw=blue,line cap=rect,line join=bevel,line width=0.800pt]
    \path[fill=blue] (0.0000,0.0000) node[above right] (text2376) {\scriptsize 6};



  \end{scope}
  \begin{scope}[cm={{0.92743,0.0,0.0,0.92743,(-570.4103,65.15216)}},draw=ca0a0a4,dash pattern=on 2.59pt off 2.59pt,line cap=round,line join=round,line width=0.323pt,miter limit=4.00]
    \path[draw,dash pattern=on 2.59pt off 2.59pt,line width=0.323pt,miter limit=4.00] (235.5000,319.5000) -- (235.5000,272.5000);



  \end{scope}
  \begin{scope}[cm={{0.92743,0.0,0.0,0.92743,(-570.4103,65.15216)}},draw=blue,line cap=round,line join=round,line width=0.480pt]
    \path[draw] (235.5000,319.5000) -- (235.5000,318.5000);



    \path[draw] (235.5000,272.5000) -- (235.5000,273.5000);



  \end{scope}
  \begin{scope}[cm={{1.00588,0.0,0.0,1.00588,(-354.49635,376.82362)}},draw=blue,line cap=rect,line join=bevel,line width=0.800pt]
    \path[fill=blue] (0.0000,0.0000) node[above right] (text2406) {\scriptsize 9};



  \end{scope}
  \begin{scope}[cm={{0.92743,0.0,0.0,0.92743,(-570.4103,65.15216)}},draw=blue,line cap=round,line join=round,line width=0.480pt]
    \path[draw] (246.5000,315.5000) -- (245.5000,315.5000);



  \end{scope}
  \begin{scope}[cm={{1.00588,0.0,0.0,1.00588,(-422.71619,360.06187)}},draw=blue,line cap=rect,line join=bevel,line width=0.800pt]
    \path[fill=blue] (0.0000,0.0000) node[above right] (text2430) {\scriptsize -1000};



  \end{scope}
  \begin{scope}[cm={{0.92743,0.0,0.0,0.92743,(-570.4103,65.15216)}},draw=blue,line cap=round,line join=round,line width=0.480pt]
    \path[draw] (246.5000,295.5000) -- (245.5000,295.5000);



  \end{scope}
  \begin{scope}[cm={{1.00588,0.0,0.0,1.00588,(-418.69269,342.94487)}},draw=blue,line cap=rect,line join=bevel,line width=0.800pt]
    \path[fill=blue] (0.0000,0.0000) node[above right] (text2454) {\scriptsize -500};



  \end{scope}
  \begin{scope}[cm={{0.92743,0.0,0.0,0.92743,(-570.4103,65.15216)}},draw=blue,line cap=round,line join=round,line width=0.480pt]
    \path[draw] (246.5000,276.5000) -- (245.5000,276.5000);



  \end{scope}
  \begin{scope}[cm={{1.00588,0.0,0.0,1.00588,(-407.96603,324.32687)}},draw=blue,line cap=rect,line join=bevel,line width=0.800pt]
    \path[fill=blue] (0.0000,0.0000) node[above right] (text2478) {\scriptsize 0};



  \end{scope}
  \begin{scope}[cm={{0.92743,0.0,0.0,0.92743,(-570.4103,65.15216)}},draw=blue,line cap=round,line join=round,line width=0.480pt]
    \path[draw] (184.5000,272.5000) -- (184.5000,319.5000) -- (246.5000,319.5000) -- (246.5000,272.5000) -- (184.5000,272.5000);



  \end{scope}
  \begin{scope}[cm={{0.0,-1.00588,1.00588,0.0,(-425.21116,344.45387)}},draw=blue,line cap=rect,line join=bevel,line width=0.800pt]
    \path[fill=blue] (0.0000,0.0000) node[above right] (text2502) {\rotatebox{90}{\scriptsize $c_{i,1}$}};



  \end{scope}
  \begin{scope}[cm={{0.92743,0.0,0.0,0.92743,(-570.4103,65.15216)}},draw=blue,line cap=round,line join=round,line width=0.480pt]
    \path[draw] (184.8000,315.6000) -- (184.8000,315.6000) -- (184.9000,315.6000) -- (185.0000,315.6000) -- (185.0000,315.6000) -- (185.1000,315.6000) -- (185.1000,315.6000) -- (185.2000,315.6000) -- (185.3000,315.6000) -- (185.3000,315.6000) -- (185.4000,315.6000) -- (185.4000,315.6000) -- (185.5000,315.6000) -- (185.6000,315.6000) -- (185.6000,315.6000) -- (185.7000,315.6000) -- (185.8000,315.6000) -- (185.8000,315.6000) -- (185.9000,315.6000) -- (185.9000,315.6000) -- (186.0000,315.6000) -- (186.1000,315.6000) -- (186.1000,315.6000) -- (186.2000,315.6000) -- (186.2000,315.6000) -- (186.3000,315.6000) -- (186.4000,315.6000) -- (186.4000,315.6000) -- (186.5000,315.6000) -- (186.6000,315.6000) -- (186.6000,315.6000) -- (186.7000,315.6000) -- (186.7000,315.6000) -- (186.8000,315.6000) -- (186.9000,315.6000) -- (186.9000,315.6000) -- (187.0000,315.6000) -- (187.0000,315.6000) -- (187.1000,315.6000) -- (187.2000,315.6000) -- (187.2000,315.6000) -- (187.3000,315.6000) -- (187.4000,315.6000) -- (187.4000,315.6000) -- (187.5000,315.6000) -- (187.5000,315.6000) -- (187.6000,315.6000) -- (187.7000,315.6000) -- (187.7000,315.6000) -- (187.8000,315.6000) -- (187.8000,315.6000) -- (187.9000,315.6000) -- (188.0000,315.6000) -- (188.0000,315.6000) -- (188.1000,315.6000) -- (188.2000,315.6000) -- (188.2000,315.6000) -- (188.3000,315.6000) -- (188.3000,315.6000) -- (188.4000,315.6000) -- (188.5000,315.6000) -- (188.5000,315.6000) -- (188.6000,315.6000) -- (188.6000,315.6000) -- (188.7000,315.6000) -- (188.8000,315.6000) -- (188.8000,315.6000) -- (188.9000,315.6000) -- (189.0000,315.6000) -- (189.0000,315.6000) -- (189.1000,315.6000) -- (189.1000,315.6000) -- (189.2000,315.6000) -- (189.3000,315.6000) -- (189.3000,315.6000) -- (189.4000,315.6000) -- (189.4000,315.6000) -- (189.5000,315.6000) -- (189.6000,315.6000) -- (189.6000,315.6000) -- (189.7000,315.6000) -- (189.8000,315.6000) -- (189.8000,315.6000) -- (189.9000,315.6000) -- (189.9000,315.6000) -- (190.0000,315.6000) -- (190.1000,315.6000) -- (190.1000,315.6000) -- (190.2000,315.6000) -- (190.2000,315.6000) -- (190.3000,315.6000) -- (190.4000,315.6000) -- (190.4000,315.6000) -- (190.5000,315.6000) -- (190.5000,315.6000) -- (190.6000,315.6000) -- (190.7000,315.6000) -- (190.7000,315.6000) -- (190.8000,315.6000) -- (190.9000,315.6000) -- (190.9000,315.6000) -- (191.0000,315.6000) -- (191.0000,315.6000) -- (191.1000,315.6000) -- (191.2000,315.6000) -- (191.2000,315.6000) -- (191.3000,315.6000) -- (191.3000,315.6000) -- (191.4000,315.6000) -- (191.5000,315.6000) -- (191.5000,315.6000) -- (191.6000,315.6000) -- (191.7000,315.6000) -- (191.7000,315.6000) -- (191.8000,315.6000) -- (191.8000,315.6000) -- (191.9000,315.6000) -- (192.0000,315.6000) -- (192.0000,315.6000) -- (192.1000,315.6000) -- (192.1000,315.6000) -- (192.2000,315.6000) -- (192.3000,315.6000) -- (192.3000,315.6000) -- (192.4000,315.6000) -- (192.5000,315.6000) -- (192.5000,315.6000) -- (192.6000,315.6000) -- (192.6000,317.2000) -- (192.7000,292.4000) -- (192.8000,274.5000) -- (192.8000,276.1000) -- (192.9000,276.1000) -- (192.9000,276.1000) -- (193.0000,276.1000) -- (193.1000,276.1000) -- (193.1000,276.1000) -- (193.2000,276.1000) -- (193.3000,276.1000) -- (193.3000,276.1000) -- (193.4000,276.1000) -- (193.4000,276.1000) -- (193.5000,276.1000) -- (193.6000,276.1000) -- (193.6000,276.1000) -- (193.7000,276.1000) -- (193.7000,276.1000) -- (193.8000,276.1000) -- (193.9000,276.1000) -- (193.9000,276.1000) -- (194.0000,276.1000) -- (194.1000,276.1000) -- (194.1000,276.1000) -- (194.2000,276.1000) -- (194.2000,276.1000) -- (194.3000,276.1000) -- (194.4000,276.1000) -- (194.4000,276.1000) -- (194.5000,276.1000) -- (194.5000,276.1000) -- (194.6000,276.1000) -- (194.7000,276.1000) -- (194.7000,276.1000) -- (194.8000,276.1000) -- (194.9000,276.1000) -- (194.9000,276.1000) -- (195.0000,276.1000) -- (195.0000,276.1000) -- (195.1000,276.1000) -- (195.2000,276.1000) -- (195.2000,276.1000) -- (195.3000,276.1000) -- (195.3000,276.1000) -- (195.4000,276.1000) -- (195.5000,276.1000) -- (195.5000,276.1000) -- (195.6000,276.1000) -- (195.7000,276.1000) -- (195.7000,276.1000) -- (195.8000,276.1000) -- (195.8000,276.1000) -- (195.9000,276.1000) -- (196.0000,276.1000) -- (196.0000,276.1000) -- (196.1000,276.1000) -- (196.1000,276.1000) -- (196.2000,276.1000) -- (196.3000,276.1000) -- (196.3000,276.1000) -- (196.4000,276.1000) -- (196.5000,276.1000) -- (196.5000,276.1000) -- (196.6000,276.1000) -- (196.6000,276.1000) -- (196.7000,276.1000) -- (196.8000,276.1000) -- (196.8000,276.1000) -- (196.9000,276.1000) -- (196.9000,276.1000) -- (197.0000,276.1000) -- (197.1000,276.1000) -- (197.1000,276.1000) -- (197.2000,276.1000) -- (197.3000,276.1000) -- (197.3000,276.1000) -- (197.4000,276.1000) -- (197.4000,276.1000) -- (197.5000,276.1000) -- (197.6000,276.1000) -- (197.6000,276.1000) -- (197.7000,276.1000) -- (197.7000,276.1000) -- (197.8000,276.1000) -- (197.9000,276.1000) -- (197.9000,276.1000) -- (198.0000,276.1000) -- (198.1000,276.1000) -- (198.1000,276.1000) -- (198.2000,276.1000) -- (198.2000,276.1000) -- (198.3000,276.1000) -- (198.4000,276.1000) -- (198.4000,276.1000) -- (198.5000,276.1000) -- (198.5000,276.1000) -- (198.6000,276.1000) -- (198.7000,276.1000) -- (198.7000,276.1000) -- (198.8000,276.1000) -- (198.9000,276.1000) -- (198.9000,276.1000) -- (199.0000,276.1000) -- (199.0000,276.1000) -- (199.1000,276.1000) -- (199.2000,276.1000) -- (199.2000,276.1000) -- (199.3000,276.1000) -- (199.3000,276.1000) -- (199.4000,276.1000) -- (199.5000,276.1000) -- (199.5000,276.1000) -- (199.6000,276.1000) -- (199.7000,276.1000) -- (199.7000,276.1000) -- (199.8000,276.1000) -- (199.8000,276.1000) -- (199.9000,276.1000) -- (200.0000,276.1000) -- (200.0000,276.1000) -- (200.1000,276.1000) -- (200.1000,276.1000) -- (200.2000,276.1000) -- (200.3000,276.1000) -- (200.3000,276.1000) -- (200.4000,276.1000) -- (200.5000,276.1000) -- (200.5000,276.1000) -- (200.6000,276.1000) -- (200.6000,276.1000) -- (200.7000,276.1000) -- (200.8000,276.1000) -- (200.8000,276.1000) -- (200.9000,276.1000) -- (200.9000,276.1000) -- (201.0000,276.1000) -- (201.1000,276.1000) -- (201.1000,276.1000) -- (201.2000,276.1000) -- (201.3000,276.1000) -- (201.3000,276.1000) -- (201.4000,276.1000) -- (201.4000,276.1000) -- (201.5000,276.1000) -- (201.6000,276.1000) -- (201.6000,276.1000) -- (201.7000,276.1000) -- (201.7000,276.1000) -- (201.8000,276.1000) -- (201.9000,276.1000) -- (201.9000,276.1000) -- (202.0000,276.1000) -- (202.0000,276.1000) -- (202.1000,276.1000) -- (202.2000,276.1000) -- (202.2000,276.1000) -- (202.3000,276.1000) -- (202.4000,276.1000) -- (202.4000,276.1000) -- (202.5000,276.1000) -- (202.5000,276.1000) -- (202.6000,276.1000) -- (202.7000,276.1000) -- (202.7000,276.1000) -- (202.8000,276.1000) -- (202.8000,276.1000) -- (202.9000,276.1000) -- (203.0000,276.1000) -- (203.0000,276.1000) -- (203.1000,276.1000) -- (203.2000,276.1000) -- (203.2000,276.1000) -- (203.3000,276.1000) -- (203.3000,276.1000) -- (203.4000,276.1000) -- (203.5000,276.1000) -- (203.5000,276.1000) -- (203.6000,276.1000) -- (203.6000,276.1000) -- (203.7000,276.1000) -- (203.8000,276.1000) -- (203.8000,276.1000) -- (203.9000,276.1000) -- (204.0000,276.1000) -- (204.0000,276.1000) -- (204.1000,276.1000) -- (204.1000,276.1000) -- (204.2000,276.1000) -- (204.3000,276.1000) -- (204.3000,276.1000) -- (204.4000,276.1000) -- (204.4000,276.1000) -- (204.5000,276.1000) -- (204.6000,276.1000) -- (204.6000,276.1000) -- (204.7000,276.1000) -- (204.8000,276.1000) -- (204.8000,276.1000) -- (204.9000,276.1000) -- (204.9000,276.1000) -- (205.0000,276.1000) -- (205.1000,276.1000) -- (205.1000,276.1000) -- (205.2000,276.1000) -- (205.2000,276.1000) -- (205.3000,276.1000) -- (205.4000,276.1000) -- (205.4000,276.1000) -- (205.5000,276.1000) -- (205.6000,276.1000) -- (205.6000,276.1000) -- (205.7000,276.1000) -- (205.7000,276.1000) -- (205.8000,276.1000) -- (205.9000,276.1000) -- (205.9000,276.1000) -- (206.0000,276.1000) -- (206.0000,276.1000) -- (206.1000,276.1000) -- (206.2000,276.1000) -- (206.2000,276.1000) -- (206.3000,276.1000) -- (206.4000,276.1000) -- (206.4000,276.1000) -- (206.5000,276.1000) -- (206.5000,276.1000) -- (206.6000,276.1000) -- (206.7000,276.1000) -- (206.7000,276.1000) -- (206.8000,276.1000) -- (206.8000,276.1000) -- (206.9000,276.1000) -- (207.0000,276.1000) -- (207.0000,276.1000) -- (207.1000,276.1000) -- (207.2000,276.1000) -- (207.2000,276.1000) -- (207.3000,276.1000) -- (207.3000,276.1000) -- (207.4000,276.1000) -- (207.5000,276.1000) -- (207.5000,276.1000) -- (207.6000,276.1000) -- (207.6000,276.1000) -- (207.7000,276.1000) -- (207.8000,276.1000) -- (207.8000,276.1000) -- (207.9000,276.1000) -- (208.0000,276.1000) -- (208.0000,276.1000) -- (208.1000,276.1000) -- (208.1000,276.1000) -- (208.2000,276.1000) -- (208.3000,276.1000) -- (208.3000,276.1000) -- (208.4000,276.1000) -- (208.4000,276.1000) -- (208.5000,276.1000) -- (208.6000,276.1000) -- (208.6000,276.1000) -- (208.7000,276.1000) -- (208.8000,276.1000) -- (208.8000,276.1000) -- (208.9000,276.1000) -- (208.9000,276.1000) -- (209.0000,276.1000) -- (209.1000,276.1000) -- (209.1000,276.1000) -- (209.2000,276.1000) -- (209.2000,276.1000) -- (209.3000,276.1000) -- (209.4000,276.1000) -- (209.4000,276.1000) -- (209.5000,276.1000) -- (209.6000,276.1000) -- (209.6000,276.1000) -- (209.7000,276.1000) -- (209.7000,276.1000) -- (209.8000,276.1000) -- (209.9000,276.1000) -- (209.9000,276.1000) -- (210.0000,276.1000) -- (210.0000,276.1000) -- (210.1000,276.1000) -- (210.2000,276.1000) -- (210.2000,276.1000) -- (210.3000,276.1000) -- (210.4000,276.1000) -- (210.4000,276.1000) -- (210.5000,276.1000) -- (210.5000,276.1000) -- (210.6000,276.1000) -- (210.7000,276.1000) -- (210.7000,276.1000) -- (210.8000,276.1000) -- (210.8000,276.1000) -- (210.9000,276.1000) -- (211.0000,276.1000) -- (211.0000,276.1000) -- (211.1000,276.1000) -- (211.2000,276.1000) -- (211.2000,276.1000) -- (211.3000,276.1000) -- (211.3000,276.1000) -- (211.4000,276.1000) -- (211.5000,276.1000) -- (211.5000,276.1000) -- (211.6000,276.1000) -- (211.6000,276.1000) -- (211.7000,276.1000) -- (211.8000,276.1000) -- (211.8000,276.1000) -- (211.9000,276.1000) -- (212.0000,276.1000) -- (212.0000,276.1000) -- (212.1000,276.1000) -- (212.1000,276.1000) -- (212.2000,276.1000) -- (212.3000,276.1000) -- (212.3000,276.1000) -- (212.4000,276.1000) -- (212.4000,276.1000) -- (212.5000,276.1000) -- (212.6000,276.1000) -- (212.6000,276.1000) -- (212.7000,276.1000) -- (212.8000,276.1000) -- (212.8000,276.1000) -- (212.9000,276.1000) -- (212.9000,276.1000) -- (213.0000,276.1000) -- (213.1000,276.1000) -- (213.1000,276.1000) -- (213.2000,276.1000) -- (213.2000,276.1000) -- (213.3000,276.1000) -- (213.4000,276.1000) -- (213.4000,276.1000) -- (213.5000,276.1000) -- (213.6000,276.1000) -- (213.6000,276.1000) -- (213.7000,276.1000) -- (213.7000,276.1000) -- (213.8000,276.1000) -- (213.9000,276.1000) -- (213.9000,276.1000) -- (214.0000,276.1000) -- (214.0000,276.1000) -- (214.1000,276.1000) -- (214.2000,276.1000) -- (214.2000,276.1000) -- (214.3000,276.1000) -- (214.4000,276.1000) -- (214.4000,276.1000) -- (214.5000,276.1000) -- (214.5000,276.1000) -- (214.6000,276.1000) -- (214.7000,276.1000) -- (214.7000,276.1000) -- (214.8000,276.1000) -- (214.8000,276.1000) -- (214.9000,276.1000) -- (215.0000,276.1000) -- (215.0000,276.1000) -- (215.1000,276.1000) -- (215.1000,276.1000) -- (215.2000,276.1000) -- (215.3000,276.1000) -- (215.3000,276.1000) -- (215.4000,276.1000) -- (215.5000,276.1000) -- (215.5000,276.1000) -- (215.6000,276.1000) -- (215.6000,276.1000) -- (215.7000,276.1000) -- (215.8000,276.1000) -- (215.8000,276.1000) -- (215.9000,276.1000) -- (215.9000,276.1000) -- (216.0000,276.1000) -- (216.1000,276.1000) -- (216.1000,276.1000) -- (216.2000,276.1000) -- (216.3000,276.1000) -- (216.3000,276.1000) -- (216.4000,276.1000) -- (216.4000,276.1000) -- (216.5000,276.1000) -- (216.6000,276.1000) -- (216.6000,276.1000) -- (216.7000,276.1000) -- (216.7000,276.1000) -- (216.8000,276.1000) -- (216.9000,276.1000) -- (216.9000,276.1000) -- (217.0000,276.1000) -- (217.1000,276.1000) -- (217.1000,276.1000) -- (217.2000,276.1000) -- (217.2000,276.1000) -- (217.3000,276.1000) -- (217.4000,276.1000) -- (217.4000,276.1000) -- (217.5000,276.1000) -- (217.5000,276.1000) -- (217.6000,276.1000) -- (217.7000,276.1000) -- (217.7000,276.1000) -- (217.8000,276.1000) -- (217.9000,276.1000) -- (217.9000,276.1000) -- (218.0000,276.1000) -- (218.0000,276.1000) -- (218.1000,276.1000) -- (218.2000,276.1000) -- (218.2000,276.1000) -- (218.3000,276.1000) -- (218.3000,276.1000) -- (218.4000,276.1000) -- (218.5000,276.1000) -- (218.5000,276.1000) -- (218.6000,276.1000) -- (218.7000,276.1000) -- (218.7000,276.1000) -- (218.8000,276.1000) -- (218.8000,276.1000) -- (218.9000,276.1000) -- (219.0000,276.1000) -- (219.0000,276.1000) -- (219.1000,276.1000) -- (219.1000,276.1000) -- (219.2000,276.1000) -- (219.3000,276.1000) -- (219.3000,276.1000) -- (219.4000,276.1000) -- (219.5000,276.1000) -- (219.5000,276.1000) -- (219.6000,276.1000) -- (219.6000,276.1000) -- (219.7000,276.1000) -- (219.8000,276.1000) -- (219.8000,276.1000) -- (219.9000,276.1000) -- (219.9000,276.1000) -- (220.0000,276.1000) -- (220.1000,276.1000) -- (220.1000,276.1000) -- (220.2000,276.1000) -- (220.3000,276.1000) -- (220.3000,276.1000) -- (220.4000,276.1000) -- (220.4000,276.1000) -- (220.5000,276.1000) -- (220.6000,276.1000) -- (220.6000,276.1000) -- (220.7000,276.1000) -- (220.7000,276.1000) -- (220.8000,276.1000) -- (220.9000,276.1000) -- (220.9000,276.1000) -- (221.0000,276.1000) -- (221.1000,276.1000) -- (221.1000,276.1000) -- (221.2000,276.1000) -- (221.2000,276.1000) -- (221.3000,276.1000) -- (221.4000,276.1000) -- (221.4000,276.1000) -- (221.5000,276.1000) -- (221.5000,276.1000) -- (221.6000,276.1000) -- (221.7000,276.1000) -- (221.7000,276.1000) -- (221.8000,276.1000) -- (221.9000,276.1000) -- (221.9000,276.1000) -- (222.0000,276.1000) -- (222.0000,276.1000) -- (222.1000,276.1000) -- (222.2000,276.1000) -- (222.2000,276.1000) -- (222.3000,276.1000) -- (222.3000,276.1000) -- (222.4000,276.1000) -- (222.5000,276.1000) -- (222.5000,276.1000) -- (222.6000,276.1000) -- (222.7000,276.1000) -- (222.7000,276.1000) -- (222.8000,276.1000) -- (222.8000,276.1000) -- (222.9000,276.1000) -- (223.0000,276.1000) -- (223.0000,276.1000) -- (223.1000,276.1000) -- (223.1000,276.1000) -- (223.2000,276.1000) -- (223.3000,276.1000) -- (223.3000,276.1000) -- (223.4000,276.1000) -- (223.5000,276.1000) -- (223.5000,276.1000) -- (223.6000,276.1000) -- (223.6000,276.1000) -- (223.7000,276.1000) -- (223.8000,276.1000) -- (223.8000,276.1000) -- (223.9000,276.1000) -- (223.9000,276.1000) -- (224.0000,276.1000) -- (224.1000,276.1000) -- (224.1000,276.1000) -- (224.2000,276.1000) -- (224.3000,276.1000) -- (224.3000,276.1000) -- (224.4000,276.1000) -- (224.4000,276.1000) -- (224.5000,276.1000) -- (224.6000,276.1000) -- (224.6000,276.1000) -- (224.7000,276.1000) -- (224.7000,276.1000) -- (224.8000,276.1000) -- (224.9000,276.1000) -- (224.9000,276.1000) -- (225.0000,276.1000) -- (225.1000,276.1000) -- (225.1000,276.1000) -- (225.2000,276.1000) -- (225.2000,276.1000) -- (225.3000,276.1000) -- (225.4000,276.1000) -- (225.4000,276.1000) -- (225.5000,276.1000) -- (225.5000,276.1000) -- (225.6000,276.1000) -- (225.7000,276.1000) -- (225.7000,276.1000) -- (225.8000,276.1000) -- (225.9000,276.1000) -- (225.9000,276.1000) -- (226.0000,276.1000) -- (226.0000,276.1000) -- (226.1000,276.1000) -- (226.2000,276.1000) -- (226.2000,276.1000) -- (226.3000,276.1000) -- (226.3000,276.1000) -- (226.4000,276.1000) -- (226.5000,276.1000) -- (226.5000,276.1000) -- (226.6000,276.1000) -- (226.6000,276.1000) -- (226.7000,276.1000) -- (226.8000,276.1000) -- (226.8000,276.1000) -- (226.9000,276.1000) -- (227.0000,276.1000) -- (227.0000,276.1000) -- (227.1000,276.1000) -- (227.1000,276.1000) -- (227.2000,276.1000) -- (227.3000,276.1000) -- (227.3000,276.1000) -- (227.4000,276.1000) -- (227.4000,276.1000) -- (227.5000,276.1000) -- (227.6000,276.1000) -- (227.6000,276.1000) -- (227.7000,276.1000) -- (227.8000,276.1000) -- (227.8000,276.1000) -- (227.9000,276.1000) -- (227.9000,276.1000) -- (228.0000,276.1000) -- (228.1000,276.1000) -- (228.1000,276.1000) -- (228.2000,276.1000) -- (228.2000,276.1000) -- (228.3000,276.1000) -- (228.4000,276.1000) -- (228.4000,276.1000) -- (228.5000,276.1000) -- (228.6000,276.1000) -- (228.6000,276.1000) -- (228.7000,276.1000) -- (228.7000,276.1000) -- (228.8000,276.1000) -- (228.9000,276.1000) -- (228.9000,276.1000) -- (229.0000,276.1000) -- (229.0000,276.1000) -- (229.1000,276.1000) -- (229.2000,276.1000) -- (229.2000,276.1000) -- (229.3000,276.1000) -- (229.4000,276.1000) -- (229.4000,276.1000) -- (229.5000,276.1000) -- (229.5000,276.1000) -- (229.6000,276.1000) -- (229.7000,276.1000) -- (229.7000,276.1000) -- (229.8000,276.1000) -- (229.8000,276.1000) -- (229.9000,276.1000) -- (230.0000,276.1000) -- (230.0000,276.1000) -- (230.1000,276.1000) -- (230.2000,276.1000) -- (230.2000,276.1000) -- (230.3000,276.1000) -- (230.3000,276.1000) -- (230.4000,276.1000) -- (230.5000,276.1000) -- (230.5000,276.1000) -- (230.6000,276.1000) -- (230.6000,276.1000) -- (230.7000,276.1000) -- (230.8000,276.1000) -- (230.8000,276.1000) -- (230.9000,276.1000) -- (231.0000,276.1000) -- (231.0000,276.1000) -- (231.1000,276.1000) -- (231.1000,276.1000) -- (231.2000,276.1000) -- (231.3000,276.1000) -- (231.3000,276.1000) -- (231.4000,276.1000) -- (231.4000,276.1000) -- (231.5000,276.1000) -- (231.6000,276.1000) -- (231.6000,276.1000) -- (231.7000,276.1000) -- (231.8000,276.1000) -- (231.8000,276.1000) -- (231.9000,276.1000) -- (231.9000,276.1000) -- (232.0000,276.1000) -- (232.1000,276.1000) -- (232.1000,276.1000) -- (232.2000,276.1000) -- (232.2000,276.1000) -- (232.3000,276.1000) -- (232.4000,276.1000) -- (232.4000,276.1000) -- (232.5000,276.1000) -- (232.6000,276.1000) -- (232.6000,276.1000) -- (232.7000,276.1000) -- (232.7000,276.1000) -- (232.8000,276.1000) -- (232.9000,276.1000) -- (232.9000,276.1000) -- (233.0000,276.1000) -- (233.0000,276.1000) -- (233.1000,276.1000) -- (233.2000,276.1000) -- (233.2000,276.1000) -- (233.3000,276.1000) -- (233.4000,276.1000) -- (233.4000,276.1000) -- (233.5000,276.1000) -- (233.5000,276.1000) -- (233.6000,276.1000) -- (233.7000,276.1000) -- (233.7000,276.1000) -- (233.8000,276.1000) -- (233.8000,276.1000) -- (233.9000,276.1000) -- (234.0000,276.1000) -- (234.0000,276.1000) -- (234.1000,276.1000) -- (234.2000,276.1000) -- (234.2000,276.1000) -- (234.3000,276.1000) -- (234.3000,276.1000) -- (234.4000,276.1000) -- (234.5000,276.1000) -- (234.5000,276.1000) -- (234.6000,276.1000) -- (234.6000,276.1000) -- (234.7000,276.1000) -- (234.8000,276.1000) -- (234.8000,276.1000) -- (234.9000,276.1000) -- (235.0000,276.1000) -- (235.0000,276.1000) -- (235.1000,276.1000) -- (235.1000,276.1000) -- (235.2000,276.1000) -- (235.3000,276.1000) -- (235.3000,276.1000) -- (235.4000,276.1000) -- (235.4000,276.1000) -- (235.5000,276.1000) -- (235.6000,276.1000) -- (235.6000,276.1000) -- (235.7000,276.1000) -- (235.8000,276.1000) -- (235.8000,276.1000) -- (235.9000,276.1000) -- (235.9000,276.1000) -- (236.0000,276.1000) -- (236.1000,276.1000) -- (236.1000,276.1000) -- (236.2000,276.1000) -- (236.2000,276.1000) -- (236.3000,276.1000) -- (236.4000,276.1000) -- (236.4000,276.1000) -- (236.5000,276.1000) -- (236.6000,276.1000) -- (236.6000,276.1000) -- (236.7000,276.1000) -- (236.7000,276.1000) -- (236.8000,276.1000) -- (236.9000,276.1000) -- (236.9000,276.1000) -- (237.0000,276.1000) -- (237.0000,276.1000) -- (237.1000,276.1000) -- (237.2000,276.1000) -- (237.2000,276.1000) -- (237.3000,276.1000) -- (237.4000,276.1000) -- (237.4000,276.1000) -- (237.5000,276.1000) -- (237.5000,276.1000) -- (237.6000,276.1000) -- (237.7000,276.1000) -- (237.7000,276.1000) -- (237.8000,276.1000) -- (237.8000,276.1000) -- (237.9000,276.1000) -- (238.0000,276.1000) -- (238.0000,276.1000) -- (238.1000,276.1000) -- (238.1000,276.1000) -- (238.2000,276.1000) -- (238.3000,276.1000) -- (238.3000,276.1000) -- (238.4000,276.1000) -- (238.5000,276.1000) -- (238.5000,276.1000) -- (238.6000,276.1000) -- (238.6000,276.1000) -- (238.7000,276.1000) -- (238.8000,276.1000) -- (238.8000,276.1000) -- (238.9000,276.1000) -- (238.9000,276.1000) -- (239.0000,276.1000) -- (239.1000,276.1000) -- (239.1000,276.1000) -- (239.2000,276.1000) -- (239.3000,276.1000) -- (239.3000,276.1000) -- (239.4000,276.1000) -- (239.4000,276.1000) -- (239.5000,276.1000) -- (239.6000,276.1000) -- (239.6000,276.1000) -- (239.7000,276.1000) -- (239.7000,276.1000) -- (239.8000,276.1000) -- (239.9000,276.1000) -- (239.9000,276.1000) -- (240.0000,276.1000) -- (240.1000,276.1000) -- (240.1000,276.1000) -- (240.2000,276.1000) -- (240.2000,276.1000) -- (240.3000,276.1000) -- (240.4000,276.1000) -- (240.4000,276.1000) -- (240.5000,276.1000) -- (240.5000,276.1000) -- (240.6000,276.1000) -- (240.7000,276.1000) -- (240.7000,276.1000) -- (240.8000,276.1000) -- (240.9000,276.1000) -- (240.9000,276.1000) -- (241.0000,276.1000) -- (241.0000,276.1000) -- (241.1000,276.1000) -- (241.2000,276.1000) -- (241.2000,276.1000) -- (241.3000,276.1000) -- (241.3000,276.1000) -- (241.4000,276.1000) -- (241.5000,276.1000) -- (241.5000,276.1000) -- (241.6000,276.1000) -- (241.7000,276.1000) -- (241.7000,276.1000) -- (241.8000,276.1000) -- (241.8000,276.1000) -- (241.9000,276.1000) -- (242.0000,276.1000) -- (242.0000,276.1000) -- (242.1000,276.1000) -- (242.1000,276.1000) -- (242.2000,276.1000) -- (242.3000,276.1000) -- (242.3000,276.1000) -- (242.4000,276.1000) -- (242.5000,276.1000) -- (242.5000,276.1000) -- (242.6000,276.1000) -- (242.6000,276.1000) -- (242.7000,276.1000) -- (242.8000,276.1000) -- (242.8000,276.1000) -- (242.9000,276.1000) -- (242.9000,276.1000) -- (243.0000,276.1000) -- (243.1000,276.1000) -- (243.1000,276.1000) -- (243.2000,276.1000) -- (243.3000,276.1000) -- (243.3000,276.1000) -- (243.4000,276.1000) -- (243.4000,276.1000) -- (243.5000,276.1000) -- (243.6000,276.1000) -- (243.6000,276.1000) -- (243.7000,276.1000) -- (243.7000,276.1000) -- (243.8000,276.1000) -- (243.9000,276.1000) -- (243.9000,276.1000) -- (244.0000,276.1000) -- (244.1000,276.1000) -- (244.1000,276.1000) -- (244.2000,276.1000) -- (244.2000,276.1000) -- (244.3000,276.1000) -- (244.4000,276.1000) -- (244.4000,276.1000) -- (244.5000,276.1000) -- (244.5000,276.1000) -- (244.6000,276.1000) -- (244.7000,276.1000) -- (244.7000,276.1000) -- (244.8000,276.1000) -- (244.9000,276.1000) -- (244.9000,276.1000) -- (245.0000,276.1000) -- (245.0000,276.1000) -- (245.1000,276.1000) -- (245.2000,276.1000) -- (245.2000,276.1000) -- (245.3000,276.1000) -- (245.3000,276.1000) -- (245.4000,276.1000) -- (245.5000,276.1000) -- (245.5000,276.1000) -- (245.6000,276.1000) -- (245.7000,276.1000) -- (245.7000,276.1000) -- (245.8000,276.1000) -- (245.8000,276.1000) -- (245.9000,276.1000) -- (246.0000,276.1000) -- (246.0000,276.1000) -- (246.1000,276.1000) -- (246.1000,276.1000) -- (246.2000,276.1000) -- (246.3000,276.1000);



  \end{scope}
  \begin{scope}[cm={{1.00588,0.0,0.0,1.00588,(-510.7994,343.95446)}},draw=blue,line cap=rect,line join=bevel,line width=0.800pt]
    \path[fill=blue] (-4.4737,45.8089) node[above right] (text344-6) {x (m)};



    \path[fill=blue] (120.9636,45.7382) node[above right] (text344-6-4) {\scriptsize Time (min)};



    \path[fill=blue] (-49.9760,62.5230) node[above right] (text344-6-3-1) {(a) Re-planned trajectories};



    \path[fill=blue] (81.5138,62.5230) node[above right] (text344-6-3-1-6) {(b) Parameters ($c_{i,1},c_{i,2}$) evol.};



    \path[fill=blue] (277.1823,62.4148) node[above right] (text344-6-3-1-6-3) {(c) Energies and batteries evol.};



  \end{scope}
  \begin{scope}[cm={{1.00588,0.0,0.0,1.00588,(-407.7468,193.72978)}},draw=blue,line cap=rect,line join=bevel,line width=0.800pt]
    \path[fill=blue] (0.0000,0.0000) node[above right] (text2120-2) {\scriptsize 2};



  \end{scope}
  \begin{scope}[cm={{1.00588,0.0,0.0,1.00588,(-407.69047,177.61778)}},draw=blue,line cap=rect,line join=bevel,line width=0.800pt]
    \path[fill=blue] (0.0000,0.0000) node[above right] (text2144-6) {\scriptsize 6};



  \end{scope}
  \begin{scope}[cm={{1.00588,0.0,0.0,1.00588,(-411.77835,160.00678)}},draw=blue,line cap=rect,line join=bevel,line width=0.800pt]
    \path[fill=blue] (0.0000,0.0000) node[above right] (text2168-3) {\scriptsize 10};



  \end{scope}
  \begin{scope}[cm={{0.0,-1.00588,1.00588,0.0,(-419.03217,183.62678)}},draw=blue,line cap=rect,line join=bevel,line width=0.800pt]
    \path[fill=blue] (0.0000,-5.9329) node[above right] (text2192-1) {\rotatebox{90}{\scriptsize $c_{i,2}$}};



  \end{scope}
  \begin{scope}[cm={{1.00588,0.0,0.0,1.00588,(-422.50501,240.01778)}},draw=blue,line cap=rect,line join=bevel,line width=0.800pt]
    \path[fill=blue] (0.0000,0.0000) node[above right] (text2430-9) {\scriptsize -1000};



  \end{scope}
  \begin{scope}[cm={{1.00588,0.0,0.0,1.00588,(-418.48151,222.90078)}},draw=blue,line cap=rect,line join=bevel,line width=0.800pt]
    \path[fill=blue] (0.0000,0.0000) node[above right] (text2454-1) {\scriptsize -500};



  \end{scope}
  \begin{scope}[cm={{1.00588,0.0,0.0,1.00588,(-407.75485,204.28278)}},draw=blue,line cap=rect,line join=bevel,line width=0.800pt]
    \path[fill=blue] (0.0000,0.0000) node[above right] (text2478-9) {\scriptsize 0};



  \end{scope}
  \begin{scope}[cm={{0.0,-1.00588,1.00588,0.0,(-424.99998,224.40978)}},draw=blue,line cap=rect,line join=bevel,line width=0.800pt]
    \path[fill=blue] (0.0000,0.0000) node[above right] (text2502-0) {\rotatebox{90}{\scriptsize $c_{i,1}$}};



  \end{scope}
  \begin{scope}[cm={{1.00588,0.0,0.0,1.00588,(-605.07343,207.04185)}},draw=blue,line cap=rect,line join=bevel,line width=0.800pt]
    \path[fill=blue] (0.0000,0.0000) node[above right] (text154-9-8) {\Large i};



  \end{scope}
  \begin{scope}[cm={{1.00588,0.0,0.0,1.00588,(-607.56518,325.40947)}},draw=blue,line cap=rect,line join=bevel,line width=0.800pt]
    \path[fill=blue] (2.4954,0.0000) node[above right] (text154-9-0-0) {\Large ii};



  \end{scope}
  \begin{scope}[cm={{0.84173,0.0,0.0,0.84173,(-604.99975,95.12326)}},draw=blue,line cap=round,line join=round,line width=0.480pt]
    \path[draw] (61.6000,227.9000) -- (61.6000,227.9000) -- (61.8000,231.0000) -- (61.9000,233.8000) -- (60.9000,236.4000) -- (59.0000,238.8000) -- (57.6000,241.4000) -- (56.8000,244.1000) -- (56.6000,247.0000) -- (56.6000,249.9000) -- (56.5000,252.8000) -- (56.5000,255.7000) -- (56.5000,258.6000) -- (56.5000,261.5000) -- (56.5000,264.4000) -- (56.5000,267.3000) -- (56.5000,270.2000) -- (56.5000,273.1000) -- (56.5000,276.0000) -- (56.5000,278.9000) -- (56.5000,281.8000) -- (56.5000,284.6000) -- (56.5000,287.5000) -- (56.8000,290.4000) -- (57.5000,293.3000) -- (58.7000,296.0000) -- (60.2000,298.6000) -- (62.2000,301.0000) -- (64.5000,303.1000) -- (67.1000,305.0000) -- (70.1000,306.6000) -- (73.2000,307.7000) -- (76.6000,308.6000) -- (80.1000,309.0000) -- (83.6000,308.9000) -- (87.0000,308.5000) -- (90.4000,307.6000) -- (93.6000,306.4000) -- (96.6000,304.8000) -- (99.2000,302.8000) -- (101.6000,300.6000) -- (103.4000,298.1000) -- (104.8000,295.4000) -- (105.6000,292.6000) -- (106.1000,289.8000) -- (106.3000,286.9000) -- (106.5000,284.1000) -- (106.5000,281.2000) -- (106.6000,278.3000) -- (106.6000,275.4000) -- (106.6000,272.5000) -- (106.6000,269.6000) -- (106.6000,266.8000) -- (106.6000,263.9000) -- (106.6000,261.0000) -- (106.6000,258.1000) -- (106.6000,255.2000) -- (106.7000,252.3000) -- (107.2000,249.4000) -- (107.4000,246.5000) -- (107.1000,243.7000) -- (106.2000,240.9000) -- (104.9000,238.2000) -- (103.2000,235.7000) -- (100.9000,233.5000) -- (98.2000,231.6000) -- (95.2000,230.1000) -- (91.9000,229.0000) -- (88.5000,228.4000) -- (85.0000,228.3000) -- (81.5000,228.7000) -- (78.2000,229.6000) -- (75.1000,230.9000) -- (72.3000,232.6000) -- (69.9000,234.7000) -- (67.8000,237.1000) -- (66.3000,239.7000) -- (65.3000,242.4000) -- (64.9000,245.3000) -- (64.8000,248.2000) -- (64.8000,251.1000) -- (64.8000,254.0000) -- (64.8000,256.9000) -- (64.8000,259.8000) -- (64.8000,262.7000) -- (64.8000,265.6000) -- (64.8000,268.5000) -- (64.8000,271.4000) -- (64.8000,274.2000) -- (64.8000,277.1000) -- (64.8000,280.0000) -- (64.8000,282.9000) -- (64.7000,285.8000) -- (64.9000,288.7000) -- (65.4000,291.6000) -- (66.4000,294.3000) -- (67.8000,297.0000) -- (69.6000,299.4000) -- (71.8000,301.7000) -- (74.4000,303.7000) -- (77.2000,305.4000) -- (80.3000,306.7000) -- (83.5000,307.7000) -- (87.0000,308.3000) -- (90.5000,308.4000) -- (94.0000,308.2000) -- (97.4000,307.5000) -- (100.7000,306.4000) -- (103.8000,305.0000) -- (106.6000,303.2000) -- (109.1000,301.1000) -- (111.2000,298.7000) -- (112.8000,296.1000) -- (113.8000,293.3000) -- (114.4000,290.5000) -- (114.8000,287.7000) -- (114.9000,284.8000) -- (115.0000,282.0000) -- (115.1000,279.1000) -- (115.1000,276.2000) -- (115.1000,273.3000) -- (115.1000,270.4000) -- (115.1000,267.5000) -- (115.1000,264.6000) -- (115.1000,261.7000) -- (115.1000,258.9000) -- (115.1000,256.0000) -- (115.1000,253.1000) -- (115.4000,250.2000) -- (115.8000,247.3000) -- (115.7000,244.4000) -- (115.1000,241.6000) -- (114.1000,238.8000) -- (112.7000,236.2000) -- (110.8000,233.8000) -- (108.6000,231.5000) -- (105.9000,229.6000) -- (102.9000,228.0000) -- (99.7000,226.8000) -- (96.3000,226.0000) -- (92.8000,225.7000) -- (89.3000,225.8000) -- (85.9000,226.4000) -- (82.6000,227.4000) -- (79.5000,228.7000) -- (76.7000,230.5000) -- (74.3000,232.5000) -- (72.2000,234.8000) -- (70.4000,237.3000) -- (69.1000,240.0000) -- (68.4000,242.8000) -- (68.1000,245.7000) -- (68.1000,248.6000) -- (68.1000,251.5000) -- (68.1000,254.4000) -- (68.1000,257.3000) -- (68.1000,260.2000) -- (68.1000,263.1000) -- (68.1000,266.0000) -- (68.1000,268.9000) -- (68.1000,271.8000) -- (68.1000,274.6000) -- (68.1000,277.5000) -- (68.1000,280.4000) -- (68.1000,283.3000) -- (68.0000,286.2000) -- (68.3000,289.1000) -- (68.9000,291.9000) -- (70.0000,294.7000) -- (71.5000,297.3000) -- (73.4000,299.7000) -- (75.7000,301.9000) -- (78.3000,303.8000) -- (81.2000,305.4000) -- (84.3000,306.7000) -- (87.6000,307.6000) -- (91.1000,308.1000) -- (94.6000,308.2000) -- (98.1000,307.8000) -- (101.5000,307.1000) -- (104.7000,305.9000) -- (107.7000,304.4000) -- (110.5000,302.5000) -- (112.9000,300.3000) -- (114.9000,297.9000) -- (116.4000,295.3000) -- (117.4000,292.5000) -- (117.9000,289.7000) -- (118.2000,286.9000) -- (118.3000,284.0000) -- (118.4000,281.1000) -- (118.4000,278.3000) -- (118.5000,275.4000) -- (118.5000,272.5000) -- (118.5000,269.6000) -- (118.5000,266.7000) -- (118.5000,263.8000) -- (118.5000,260.9000) -- (118.5000,258.0000) -- (118.5000,255.1000) -- (118.5000,252.2000) -- (118.9000,249.4000) -- (119.1000,246.5000) -- (118.9000,243.6000) -- (118.3000,240.8000) -- (117.2000,238.0000) -- (115.6000,235.4000) -- (113.6000,233.0000) -- (111.3000,230.9000) -- (108.5000,229.0000) -- (105.5000,227.5000) -- (102.2000,226.4000) -- (98.8000,225.8000) -- (95.3000,225.6000) -- (91.8000,225.8000) -- (88.4000,226.5000) -- (85.2000,227.5000) -- (82.2000,229.0000) -- (79.4000,230.8000) -- (77.1000,232.9000) -- (75.0000,235.3000) -- (73.4000,237.8000) -- (72.3000,240.6000) -- (71.7000,243.4000) -- (71.5000,246.3000) -- (71.5000,249.2000) -- (71.5000,252.1000) -- (71.5000,255.0000) -- (71.5000,257.9000) -- (71.5000,260.8000) -- (71.5000,263.7000) -- (71.5000,266.6000) -- (71.5000,269.5000) -- (71.5000,272.4000) -- (71.5000,275.3000) -- (71.5000,278.1000) -- (71.5000,281.0000) -- (71.5000,283.9000) -- (71.5000,286.8000) -- (71.9000,289.7000) -- (72.7000,292.5000) -- (73.9000,295.2000) -- (75.5000,297.8000) -- (77.5000,300.1000) -- (79.9000,302.2000) -- (82.6000,304.1000) -- (85.6000,305.6000) -- (88.8000,306.7000) -- (92.2000,307.5000) -- (95.6000,307.8000) -- (99.1000,307.8000) -- (102.6000,307.3000) -- (106.0000,306.4000) -- (109.2000,305.1000) -- (112.1000,303.4000) -- (114.7000,301.4000) -- (117.0000,299.2000) -- (118.9000,296.7000) -- (120.2000,294.0000) -- (120.9000,291.2000) -- (121.3000,288.3000) -- (121.6000,285.5000) -- (121.7000,282.6000) -- (121.8000,279.8000) -- (121.8000,276.9000) -- (121.8000,274.0000) -- (121.8000,271.1000) -- (121.8000,268.2000) -- (121.8000,265.3000) -- (121.8000,262.4000) -- (121.8000,259.5000) -- (121.8000,256.6000) -- (121.8000,253.7000) -- (122.0000,250.9000) -- (122.4000,248.0000) -- (122.4000,245.1000) -- (122.1000,242.2000) -- (121.2000,239.4000) -- (120.0000,236.7000) -- (118.3000,234.2000) -- (116.1000,231.9000) -- (113.6000,229.9000) -- (110.8000,228.1000) -- (107.6000,226.8000) -- (104.3000,225.9000) -- (100.8000,225.4000) -- (97.3000,225.3000) -- (93.8000,225.7000) -- (90.5000,226.6000) -- (87.3000,227.8000) -- (84.4000,229.4000) -- (81.8000,231.3000) -- (79.6000,233.5000) -- (77.7000,236.0000) -- (76.3000,238.6000) -- (75.3000,241.4000) -- (75.0000,244.3000) -- (74.9000,247.2000) -- (74.9000,250.1000) -- (74.9000,253.0000) -- (74.9000,255.9000) -- (74.9000,258.8000) -- (74.9000,261.6000) -- (74.9000,264.5000) -- (74.9000,267.4000) -- (74.9000,270.3000) -- (74.9000,273.2000) -- (74.9000,276.1000) -- (74.9000,279.0000) -- (74.9000,281.9000) -- (74.9000,284.8000) -- (75.0000,287.7000) -- (75.5000,290.5000) -- (76.5000,293.3000) -- (77.8000,296.0000) -- (79.6000,298.4000) -- (81.8000,300.7000) -- (84.3000,302.7000) -- (87.1000,304.4000) -- (90.2000,305.8000) -- (93.5000,306.8000) -- (96.9000,307.4000) -- (100.4000,307.6000) -- (103.9000,307.3000) -- (107.3000,306.7000) -- (110.6000,305.6000) -- (113.7000,304.2000) -- (116.5000,302.4000) -- (119.0000,300.3000) -- (121.1000,297.9000) -- (122.8000,295.3000) -- (123.9000,292.6000) -- (124.5000,289.7000) -- (124.8000,286.9000) -- (125.0000,284.1000) -- (125.1000,281.2000) -- (125.1000,278.3000) -- (125.1000,275.4000) -- (125.2000,272.5000) -- (125.2000,269.6000) -- (125.2000,266.8000) -- (125.2000,263.9000) -- (125.2000,261.0000) -- (125.2000,258.1000) -- (125.2000,255.2000) -- (125.2000,252.3000) -- (125.5000,249.4000) -- (125.8000,246.5000) -- (125.7000,243.6000) -- (125.1000,240.8000) -- (124.1000,238.1000) -- (122.7000,235.4000) -- (120.8000,233.0000) -- (118.5000,230.8000) -- (115.9000,228.8000) -- (112.9000,227.3000) -- (109.6000,226.1000) -- (106.2000,225.3000) -- (102.7000,225.0000) -- (99.3000,225.2000) -- (95.8000,225.7000) -- (92.6000,226.7000) -- (89.5000,228.1000) -- (86.7000,229.8000) -- (84.2000,231.9000) -- (82.2000,234.2000) -- (80.5000,236.7000) -- (79.2000,239.4000) -- (78.5000,242.3000) -- (78.3000,245.2000) -- (78.2000,248.1000) -- (78.2000,251.0000) -- (78.2000,253.9000) -- (78.2000,256.7000) -- (78.2000,259.6000) -- (78.2000,262.5000) -- (78.2000,265.4000) -- (78.2000,268.3000) -- (78.2000,271.2000) -- (78.2000,274.1000) -- (78.2000,277.0000) -- (78.2000,279.9000) -- (78.2000,282.8000) -- (78.2000,285.7000) -- (78.5000,288.5000) -- (79.2000,291.4000) -- (80.3000,294.1000) -- (81.9000,296.7000) -- (83.8000,299.1000) -- (86.1000,301.3000) -- (88.8000,303.1000) -- (91.7000,304.7000) -- (94.9000,305.9000) -- (98.2000,306.8000) -- (101.7000,307.2000) -- (105.2000,307.2000) -- (108.7000,306.8000) -- (112.0000,306.0000) -- (115.3000,304.8000) -- (118.3000,303.2000) -- (120.9000,301.2000) -- (123.3000,299.0000) -- (125.2000,296.6000) -- (126.7000,293.9000) -- (127.5000,291.1000) -- (128.0000,288.3000) -- (128.2000,285.4000) -- (128.4000,282.6000) -- (128.5000,279.7000) -- (128.5000,276.8000) -- (128.5000,273.9000) -- (128.5000,271.0000) -- (128.5000,268.2000) -- (128.5000,265.3000) -- (128.5000,262.4000) -- (128.5000,259.5000) -- (128.5000,256.6000) -- (128.5000,253.7000) -- (128.6000,250.8000) -- (129.0000,247.9000) -- (129.2000,245.0000) -- (128.9000,242.2000) -- (128.1000,239.4000) -- (126.9000,236.7000) -- (125.3000,234.1000) -- (123.3000,231.7000) -- (120.8000,229.6000) -- (118.0000,227.9000) -- (114.9000,226.4000) -- (111.6000,225.4000) -- (108.2000,224.9000) -- (104.6000,224.7000) -- (101.2000,225.1000) -- (97.8000,225.8000) -- (94.6000,227.0000) -- (91.7000,228.5000) -- (89.0000,230.4000) -- (86.7000,232.5000) -- (84.8000,234.9000) -- (83.2000,237.5000) -- (82.2000,240.3000) -- (81.7000,243.2000) -- (81.6000,246.1000) -- (81.6000,249.0000) -- (81.6000,251.9000) -- (81.6000,254.8000) -- (81.6000,257.7000) -- (81.6000,260.6000) -- (81.6000,263.4000) -- (81.6000,266.3000) -- (81.6000,269.2000) -- (81.6000,272.1000) -- (81.6000,275.0000) -- (81.6000,277.9000) -- (81.6000,280.8000) -- (81.6000,283.7000) -- (81.6000,286.6000) -- (82.1000,289.4000) -- (83.0000,292.2000) -- (84.3000,294.9000) -- (86.0000,297.4000) -- (88.1000,299.7000) -- (90.6000,301.8000) -- (93.3000,303.5000) -- (96.4000,305.0000) -- (99.6000,306.0000) -- (103.0000,306.7000) -- (106.5000,307.0000) -- (110.0000,306.8000) -- (113.4000,306.2000) -- (116.8000,305.2000) -- (119.9000,303.8000) -- (122.8000,302.1000) -- (125.3000,300.1000) -- (127.5000,297.7000) -- (129.3000,295.2000) -- (130.4000,292.5000) -- (131.1000,289.6000) -- (131.5000,286.8000) -- (131.7000,284.0000) -- (131.8000,281.1000) -- (131.8000,278.2000) -- (131.9000,275.3000) -- (131.9000,272.4000) -- (131.9000,269.6000) -- (131.9000,266.7000) -- (131.9000,263.8000) -- (131.9000,260.9000) -- (131.9000,258.0000) -- (131.9000,255.1000) -- (131.9000,252.2000) -- (132.1000,249.3000) -- (132.5000,246.5000) -- (132.5000,243.6000) -- (132.0000,240.7000) -- (131.1000,237.9000) -- (129.7000,235.3000) -- (127.9000,232.8000) -- (125.7000,230.5000) -- (123.1000,228.6000) -- (120.1000,226.9000) -- (116.9000,225.7000) -- (113.5000,224.8000) -- (110.1000,224.5000) -- (106.6000,224.5000) -- (103.1000,225.0000) -- (99.8000,225.9000) -- (96.7000,227.3000) -- (93.9000,228.9000) -- (91.4000,230.9000) -- (89.2000,233.2000) -- (87.4000,235.7000) -- (86.1000,238.4000) -- (85.3000,241.2000) -- (85.0000,244.1000) -- (85.0000,247.0000) -- (84.9000,249.9000) -- (84.9000,252.8000) -- (84.9000,255.7000) -- (84.9000,258.6000) -- (84.9000,261.4000) -- (84.9000,264.3000) -- (84.9000,267.2000) -- (84.9000,270.1000) -- (84.9000,273.0000) -- (84.9000,275.9000) -- (84.9000,278.8000) -- (84.9000,281.7000) -- (84.9000,284.6000) -- (85.1000,287.5000) -- (85.8000,290.3000) -- (86.8000,293.1000) -- (88.3000,295.7000) -- (90.2000,298.1000) -- (92.4000,300.3000) -- (95.0000,302.3000) -- (97.9000,303.9000) -- (101.0000,305.2000) -- (104.3000,306.1000) -- (107.8000,306.6000) -- (111.3000,306.7000) -- (114.8000,306.3000) -- (118.2000,305.6000) -- (121.4000,304.4000) -- (124.5000,302.9000) -- (127.2000,301.0000) -- (129.6000,298.9000) -- (131.6000,296.5000) -- (133.2000,293.8000) -- (134.1000,291.0000) -- (134.7000,288.2000) -- (134.9000,285.4000) -- (135.1000,282.5000) -- (135.2000,279.7000) -- (135.2000,276.8000) -- (135.2000,273.9000) -- (135.2000,271.0000) -- (135.2000,268.1000) -- (135.2000,265.2000) -- (135.2000,262.3000) -- (135.2000,259.4000) -- (135.2000,256.5000) -- (135.2000,253.6000) -- (135.3000,250.8000) -- (135.7000,247.9000) -- (135.9000,245.0000) -- (135.7000,242.1000) -- (135.0000,239.3000) -- (133.9000,236.6000) -- (132.4000,234.0000) -- (130.4000,231.6000) -- (128.0000,229.4000) -- (125.3000,227.6000) -- (122.2000,226.1000) -- (118.9000,225.0000) -- (115.5000,224.3000) -- (112.0000,224.1000) -- (108.5000,224.4000) -- (105.1000,225.0000) -- (101.9000,226.1000) -- (98.9000,227.6000) -- (96.2000,229.4000) -- (93.8000,231.5000) -- (91.8000,233.9000) -- (90.2000,236.4000) -- (89.1000,239.2000) -- (88.5000,242.0000) -- (88.3000,244.9000) -- (88.3000,247.8000) -- (88.3000,250.7000) -- (88.3000,253.6000) -- (88.3000,256.5000) -- (88.3000,259.4000) -- (88.3000,262.3000) -- (88.3000,265.2000) -- (88.3000,268.1000) -- (88.3000,271.0000) -- (88.3000,273.9000) -- (88.3000,276.8000) -- (88.3000,279.7000) -- (88.3000,282.5000) -- (88.3000,285.4000) -- (88.7000,288.3000) -- (89.5000,291.1000) -- (90.7000,293.8000) -- (92.4000,296.4000) -- (94.4000,298.7000) -- (96.8000,300.8000) -- (99.5000,302.7000) -- (102.5000,304.1000) -- (105.7000,305.3000) -- (109.0000,306.0000) -- (112.5000,306.4000) -- (116.0000,306.3000) -- (119.5000,305.8000) -- (122.8000,304.9000) -- (126.0000,303.6000) -- (129.0000,301.9000) -- (131.6000,299.9000) -- (133.8000,297.6000) -- (135.7000,295.1000) -- (137.0000,292.4000) -- (137.7000,289.6000) -- (138.1000,286.8000) -- (138.4000,284.0000) -- (138.5000,281.1000) -- (138.5000,278.2000) -- (138.6000,275.3000) -- (138.6000,272.4000) -- (138.6000,269.5000) -- (138.6000,266.7000) -- (138.6000,263.8000) -- (138.6000,260.9000) -- (138.6000,258.0000) -- (138.6000,255.1000) -- (138.6000,252.2000) -- (138.8000,249.3000) -- (139.2000,246.4000) -- (139.2000,243.5000) -- (138.8000,240.7000) -- (138.0000,237.9000) -- (136.7000,235.2000) -- (135.0000,232.7000) -- (132.9000,230.4000) -- (130.3000,228.4000) -- (127.4000,226.6000) -- (124.3000,225.3000) -- (121.0000,224.4000) -- (117.5000,223.9000) -- (114.0000,223.9000) -- (110.5000,224.3000) -- (107.2000,225.1000) -- (104.0000,226.4000) -- (101.1000,228.0000) -- (98.6000,229.9000) -- (96.3000,232.1000) -- (94.5000,234.6000) -- (93.0000,237.2000) -- (92.1000,240.0000) -- (91.8000,242.9000) -- (91.7000,245.8000) -- (91.7000,248.7000) -- (91.7000,251.6000) -- (91.6000,254.5000) -- (91.6000,257.4000) -- (91.6000,260.3000) -- (91.6000,263.2000) -- (91.6000,266.1000) -- (91.6000,269.0000) -- (91.6000,271.8000) -- (91.6000,274.7000) -- (91.6000,277.6000) -- (91.6000,280.5000) -- (91.6000,283.4000) -- (91.8000,286.3000) -- (92.3000,289.2000) -- (93.3000,291.9000) -- (94.7000,294.6000) -- (96.5000,297.1000) -- (98.7000,299.3000) -- (101.2000,301.3000) -- (104.0000,303.0000) -- (107.1000,304.4000) -- (110.3000,305.3000) -- (113.8000,305.9000) -- (117.3000,306.1000) -- (120.8000,305.8000) -- (124.2000,305.2000) -- (127.5000,304.1000) -- (130.6000,302.6000) -- (133.4000,300.8000) -- (135.9000,298.7000) -- (138.0000,296.4000) -- (139.6000,293.8000) -- (140.7000,291.0000) -- (141.3000,288.2000) -- (141.6000,285.4000) -- (141.8000,282.5000) -- (141.9000,279.7000) -- (141.9000,276.8000) -- (141.9000,273.9000) -- (141.9000,271.0000) -- (142.0000,268.1000) -- (142.0000,265.2000) -- (142.0000,262.3000) -- (142.0000,259.4000) -- (142.0000,256.5000) -- (142.0000,253.6000) -- (142.0000,250.8000) -- (142.3000,247.9000) -- (142.6000,245.0000) -- (142.5000,242.1000) -- (141.9000,239.3000) -- (140.9000,236.5000) -- (139.4000,233.9000) -- (137.5000,231.5000) -- (135.2000,229.2000) -- (132.6000,227.3000) -- (129.6000,225.8000) -- (126.3000,224.6000) -- (122.9000,223.9000) -- (119.4000,223.6000) -- (115.9000,223.7000) -- (112.5000,224.3000) -- (109.3000,225.3000) -- (106.2000,226.7000) -- (103.4000,228.4000) -- (101.0000,230.5000) -- (98.9000,232.8000) -- (97.2000,235.3000) -- (96.0000,238.0000) -- (95.3000,240.9000) -- (95.1000,243.8000) -- (95.0000,246.7000) -- (95.0000,249.6000) -- (95.0000,252.5000) -- (95.0000,255.4000) -- (95.0000,258.3000) -- (95.0000,261.1000) -- (95.0000,264.0000) -- (95.0000,266.9000) -- (95.0000,269.8000) -- (95.0000,272.7000) -- (95.0000,275.6000) -- (95.0000,278.5000) -- (95.0000,281.4000) -- (95.0000,284.3000) -- (95.3000,287.1000) -- (96.0000,290.0000) -- (97.2000,292.7000) -- (98.7000,295.3000) -- (100.7000,297.7000) -- (103.0000,299.9000) -- (105.6000,301.7000) -- (108.6000,303.3000) -- (111.7000,304.5000) -- (115.1000,305.3000) -- (118.5000,305.8000) -- (122.0000,305.8000) -- (125.5000,305.3000) -- (128.9000,304.5000) -- (132.1000,303.3000) -- (135.1000,301.7000) -- (137.8000,299.7000) -- (140.1000,297.5000) -- (142.1000,295.1000) -- (143.5000,292.4000) -- (144.3000,289.6000) -- (144.8000,286.8000) -- (145.0000,283.9000) -- (145.2000,281.1000) -- (145.2000,278.2000) -- (145.3000,275.3000) -- (145.3000,272.4000) -- (145.3000,269.5000) -- (145.3000,266.7000) -- (145.3000,263.8000) -- (145.3000,260.9000) -- (145.3000,258.0000) -- (145.3000,255.1000) -- (145.3000,252.2000) -- (145.4000,249.3000) -- (145.8000,246.4000) -- (146.0000,243.6000) -- (145.7000,240.7000) -- (144.9000,237.9000) -- (143.7000,235.2000) -- (142.1000,232.6000) -- (140.0000,230.3000) -- (137.6000,228.2000) -- (134.8000,226.4000) -- (131.7000,225.0000) -- (128.4000,224.0000) -- (124.9000,223.4000) -- (121.4000,223.3000) -- (117.9000,223.6000) -- (114.6000,224.3000) -- (111.4000,225.5000) -- (108.4000,227.0000) -- (105.8000,228.9000) -- (103.5000,231.1000) -- (101.5000,233.5000) -- (100.0000,236.1000) -- (99.0000,238.9000) -- (98.5000,241.7000) -- (98.4000,244.6000) -- (98.4000,247.5000) -- (98.4000,250.4000) -- (98.4000,253.3000) -- (98.4000,256.2000) -- (98.4000,259.1000) -- (98.4000,262.0000) -- (98.4000,264.9000) -- (98.4000,267.8000) -- (98.4000,270.7000) -- (98.4000,273.6000) -- (98.4000,276.5000) -- (98.4000,279.3000) -- (98.4000,282.2000) -- (98.4000,285.1000) -- (98.9000,288.0000) -- (99.8000,290.8000) -- (101.1000,293.5000) -- (102.8000,296.0000) -- (104.9000,298.3000) -- (107.4000,300.3000) -- (110.1000,302.1000) -- (113.1000,303.5000) -- (116.4000,304.6000) -- (119.8000,305.3000) -- (123.3000,305.5000) -- (126.8000,305.3000) -- (130.2000,304.8000) -- (133.6000,303.8000) -- (136.7000,302.4000) -- (139.6000,300.7000) -- (142.1000,298.6000) -- (144.3000,296.3000) -- (146.1000,293.7000) -- (147.2000,291.0000) -- (147.9000,288.2000) -- (148.3000,285.4000) -- (148.5000,282.5000) -- (148.6000,279.7000) -- (148.6000,276.8000) -- (148.6000,273.9000) -- (148.7000,271.0000) -- (148.7000,268.1000) -- (148.7000,265.2000) -- (148.7000,262.3000) -- (148.7000,259.4000) -- (148.7000,256.5000) -- (148.7000,253.6000) -- (148.7000,250.8000) -- (148.9000,247.9000) -- (149.3000,245.0000) -- (149.3000,242.1000) -- (148.8000,239.3000) -- (147.8000,236.5000) -- (146.5000,233.8000) -- (144.7000,231.3000) -- (142.5000,229.1000) -- (139.9000,227.1000) -- (136.9000,225.5000) -- (133.7000,224.2000) -- (130.3000,223.4000) -- (126.9000,223.0000) -- (123.4000,223.1000) -- (119.9000,223.6000) -- (116.6000,224.5000) -- (113.5000,225.8000) -- (110.7000,227.5000) -- (108.2000,229.5000) -- (106.0000,231.7000) -- (104.2000,234.2000) -- (102.9000,236.9000) -- (102.1000,239.7000) -- (101.8000,242.6000) -- (101.7000,245.5000) -- (101.7000,248.4000) -- (101.7000,251.3000) -- (101.7000,254.2000) -- (101.7000,257.1000) -- (101.7000,260.0000) -- (101.7000,262.9000) -- (101.7000,265.8000) -- (101.7000,268.6000) -- (101.7000,271.5000) -- (101.7000,274.4000) -- (101.7000,277.3000) -- (101.7000,280.2000) -- (101.7000,283.1000) -- (101.9000,286.0000) -- (102.5000,288.8000) -- (103.6000,291.6000) -- (105.1000,294.2000) -- (107.0000,296.6000) -- (109.2000,298.9000) -- (111.8000,300.8000) -- (114.7000,302.4000) -- (117.8000,303.7000) -- (121.1000,304.6000) -- (124.5000,305.1000) -- (128.0000,305.2000) -- (131.5000,304.9000) -- (134.9000,304.1000) -- (138.2000,303.0000) -- (141.2000,301.4000) -- (144.0000,299.6000) -- (146.4000,297.4000) -- (148.4000,295.0000) -- (150.0000,292.4000) -- (150.9000,289.6000) -- (151.4000,286.8000) -- (151.7000,283.9000) -- (151.9000,281.1000) -- (152.0000,278.2000) -- (152.0000,275.3000) -- (152.0000,272.4000) -- (152.0000,269.5000) -- (152.0000,266.7000) -- (152.0000,263.8000) -- (152.0000,260.9000) -- (152.0000,258.0000) -- (152.0000,255.1000) -- (152.0000,252.2000) -- (152.1000,249.3000) -- (152.4000,246.4000) -- (152.7000,243.6000) -- (152.5000,240.7000) -- (151.8000,237.8000) -- (150.7000,235.1000) -- (149.2000,232.5000) -- (147.2000,230.1000) -- (144.8000,228.0000) -- (142.1000,226.1000) -- (139.0000,224.6000) -- (135.8000,223.5000) -- (132.3000,222.9000) -- (128.8000,222.7000) -- (125.3000,222.9000) -- (121.9000,223.6000) -- (118.7000,224.7000) -- (115.7000,226.1000) -- (113.0000,227.9000) -- (110.6000,230.0000) -- (108.6000,232.4000) -- (107.0000,235.0000) -- (105.8000,237.7000) -- (105.3000,240.6000) -- (105.1000,243.5000) -- (105.1000,246.4000) -- (105.1000,249.3000) -- (105.1000,252.2000) -- (105.1000,255.1000) -- (105.1000,257.9000) -- (105.1000,260.8000) -- (105.1000,263.7000) -- (105.1000,266.6000) -- (105.1000,269.5000) -- (105.1000,272.4000) -- (105.1000,275.3000) -- (105.1000,278.2000) -- (105.1000,281.1000) -- (105.1000,284.0000) -- (105.5000,286.8000) -- (106.3000,289.7000) -- (107.5000,292.4000) -- (109.1000,294.9000) -- (111.2000,297.3000) -- (113.6000,299.4000) -- (116.3000,301.2000) -- (119.2000,302.7000) -- (122.4000,303.8000) -- (125.8000,304.6000) -- (129.3000,304.9000) -- (132.8000,304.8000) -- (136.3000,304.3000) -- (139.6000,303.4000) -- (142.8000,302.1000) -- (145.7000,300.4000) -- (148.3000,298.5000) -- (150.6000,296.2000) -- (152.5000,293.7000) -- (153.8000,291.0000) -- (154.5000,288.2000) -- (154.9000,285.3000) -- (155.1000,282.5000) -- (155.3000,279.7000) -- (155.3000,276.8000) -- (155.4000,273.9000) -- (155.4000,271.0000) -- (155.4000,268.1000) -- (155.4000,265.2000) -- (155.4000,262.3000) -- (155.4000,259.4000) -- (155.4000,256.5000) -- (155.4000,253.6000) -- (155.4000,250.8000) -- (155.5000,247.9000) -- (155.9000,244.9000);



  \end{scope}
  \begin{scope}[cm={{1.00588,0.0,0.0,1.00588,(-526.48136,104.32102)}},draw=blue,line cap=rect,line join=bevel,line width=0.800pt]
  \end{scope}
\end{scope}
\begin{scope}[cm={{1.00588,0.0,0.0,1.00588,(10.36994,109.98172)}},draw=blue,line cap=rect,line join=bevel,line width=0.800pt]
  \path[fill=blue] (0.0000,0.0000) node[above right] (text34-9-1) {-100};



\end{scope}
\begin{scope}[cm={{1.15801,0.0,0.0,1.15801,(-19.81968,-10.65042)}},draw=blue,line cap=round,line join=round,line width=0.480pt]
  \path[draw] (44.5000,79.5000) -- (48.5000,79.5000);



  \path[draw] (142.5000,79.5000) -- (139.5000,79.5000);



\end{scope}
\begin{scope}[cm={{1.00588,0.0,0.0,1.00588,(21.10457,84.85183)}},draw=blue,line cap=rect,line join=bevel,line width=0.800pt]
  \path[fill=blue] (0.0000,0.0000) node[above right] (text64-6-5) {0};



\end{scope}
\begin{scope}[cm={{1.15801,0.0,0.0,1.15801,(-19.81968,-10.65042)}},draw=ca0a0a4,dash pattern=on 1.55pt off 1.55pt,line cap=round,line join=round,line width=0.259pt,miter limit=4.00]
  \path[draw,dash pattern=on 1.55pt off 1.55pt,line width=0.259pt,miter limit=4.00] (44.5000,57.5000) -- (142.5000,57.5000);



\end{scope}
\begin{scope}[cm={{1.15801,0.0,0.0,1.15801,(-19.81968,-10.65042)}},draw=blue,line cap=round,line join=round,line width=0.480pt]
  \path[draw] (44.5000,57.5000) -- (48.5000,57.5000);



  \path[draw] (142.5000,57.5000) -- (139.5000,57.5000);



\end{scope}
\begin{scope}[cm={{1.00588,0.0,0.0,1.00588,(13.05753,58.22243)}},draw=blue,line cap=rect,line join=bevel,line width=0.800pt]
  \path[fill=blue] (0.0000,0.0000) node[above right] (text94-7-3) {100};



\end{scope}
\begin{scope}[cm={{1.15801,0.0,0.0,1.15801,(-19.81968,-10.65042)}},draw=ca0a0a4,dash pattern=on 1.55pt off 1.55pt,line cap=round,line join=round,line width=0.259pt,miter limit=4.00]
  \path[draw,dash pattern=on 1.55pt off 1.55pt,line width=0.259pt,miter limit=4.00] (44.5000,35.5000) -- (142.5000,35.5000);



\end{scope}
\begin{scope}[cm={{1.15801,0.0,0.0,1.15801,(-19.81968,-10.65042)}},draw=blue,line cap=round,line join=round,line width=0.480pt]
  \path[draw] (44.5000,35.5000) -- (48.5000,35.5000);



  \path[draw] (142.5000,35.5000) -- (139.5000,35.5000);



\end{scope}
\begin{scope}[cm={{1.00588,0.0,0.0,1.00588,(13.05753,33.09303)}},draw=blue,line cap=rect,line join=bevel,line width=0.800pt]
  \path[fill=blue] (0.0000,0.0000) node[above right] (text124-9-9) {200};



\end{scope}
\begin{scope}[cm={{1.15801,0.0,0.0,1.15801,(-19.81968,-10.65042)}},draw=ca0a0a4,dash pattern=on 0.40pt off 0.80pt,line cap=round,line join=round,line width=0.400pt]
  \path[draw] (44.5000,13.5000) -- (142.5000,13.5000);



\end{scope}
\begin{scope}[cm={{1.15801,0.0,0.0,1.15801,(-19.81968,-10.65042)}},draw=blue,line cap=round,line join=round,line width=0.480pt]
  \path[draw] (44.5000,13.5000) -- (48.5000,13.5000);



  \path[draw] (142.5000,13.5000) -- (139.5000,13.5000);



\end{scope}
\begin{scope}[cm={{1.00588,0.0,0.0,1.00588,(13.05753,7.96363)}},draw=blue,line cap=rect,line join=bevel,line width=0.800pt]
  \path[fill=blue] (0.0000,0.0000) node[above right] (text154-93) {300};



\end{scope}
\begin{scope}[cm={{1.15801,0.0,0.0,1.15801,(-19.81968,-10.65042)}},draw=ca0a0a4,dash pattern=on 0.40pt off 0.80pt,line cap=round,line join=round,line width=0.400pt]
  \path[draw] (44.5000,102.5000) -- (44.5000,13.5000);



\end{scope}
\begin{scope}[cm={{1.15801,0.0,0.0,1.15801,(-19.81968,-10.65042)}},draw=blue,line cap=round,line join=round,line width=0.480pt]
  \path[draw] (44.5000,102.5000) -- (44.5000,99.5000);



  \path[draw] (44.5000,13.5000) -- (44.5000,16.5000);



\end{scope}
\begin{scope}[cm={{1.15801,0.0,0.0,1.15801,(-19.81968,-10.65042)}},draw=blue,line cap=round,line join=round,line width=0.480pt]
  \path[draw] (69.5000,102.5000) -- (69.5000,99.5000);



  \path[draw] (69.5000,13.5000) -- (69.5000,16.5000);



\end{scope}
\begin{scope}[cm={{1.15801,0.0,0.0,1.15801,(-19.81968,-10.65042)}},draw=ca0a0a4,dash pattern=on 1.55pt off 1.55pt,line cap=round,line join=round,line width=0.259pt,miter limit=4.00]
  \path[draw,dash pattern=on 1.55pt off 1.55pt,line width=0.259pt,miter limit=4.00] (93.5000,102.5000) -- (93.5000,27.5000);



  \path[draw,dash pattern=on 1.55pt off 1.55pt,line width=0.259pt,miter limit=4.00] (93.5000,19.5000) -- (93.5000,13.5000);



\end{scope}
\begin{scope}[cm={{1.15801,0.0,0.0,1.15801,(-19.81968,-10.65042)}},draw=blue,line cap=round,line join=round,line width=0.480pt]
  \path[draw] (93.5000,102.5000) -- (93.5000,99.5000);



  \path[draw] (93.5000,13.5000) -- (93.5000,16.5000);



\end{scope}
\begin{scope}[cm={{1.15801,0.0,0.0,1.15801,(-19.81968,-10.65042)}},draw=ca0a0a4,dash pattern=on 1.55pt off 1.55pt,line cap=round,line join=round,line width=0.259pt,miter limit=4.00]
  \path[draw,dash pattern=on 1.55pt off 1.55pt,line width=0.259pt,miter limit=4.00] (118.5000,102.5000) -- (118.5000,13.5000);



\end{scope}
\begin{scope}[cm={{1.15801,0.0,0.0,1.15801,(-19.81968,-10.65042)}},draw=blue,line cap=round,line join=round,line width=0.480pt]
  \path[draw] (118.5000,102.5000) -- (118.5000,99.5000);



  \path[draw] (118.5000,13.5000) -- (118.5000,16.5000);



\end{scope}
\begin{scope}[cm={{1.15801,0.0,0.0,1.15801,(-19.81968,-10.65042)}},draw=ca0a0a4,dash pattern=on 0.40pt off 0.80pt,line cap=round,line join=round,line width=0.400pt]
  \path[draw] (142.5000,102.5000) -- (142.5000,13.5000);



\end{scope}
\begin{scope}[cm={{1.15801,0.0,0.0,1.15801,(-19.81968,-10.65042)}},draw=blue,line cap=round,line join=round,line width=0.480pt]
  \path[draw] (142.5000,102.5000) -- (142.5000,99.5000);



  \path[draw] (142.5000,13.5000) -- (142.5000,16.5000);



\end{scope}
\begin{scope}[cm={{1.15801,0.0,0.0,1.15801,(-19.81968,-10.65042)}},draw=blue,line cap=round,line join=round,line width=0.480pt]
  \path[draw] (44.5000,13.5000) -- (44.5000,102.5000) -- (142.5000,102.5000) -- (142.5000,13.5000) -- (44.5000,13.5000);



\end{scope}
\begin{scope}[cm={{0.84029,0.0,0.0,0.84029,(41.25055,18.91027)}},draw=blue,line cap=rect,line join=bevel,line width=0.800pt]
  \path[fill=blue] (-0.9117,0.4559) node[above right] (text360-8) {\scriptsize $\mathbf{p}(t)$};



\end{scope}
\begin{scope}[cm={{1.15801,0.0,0.0,1.15801,(-27.33812,-11.90349)}},draw=blue,line cap=round,line join=round,line width=0.480pt]
  \path[draw,even odd rule] (74.5000,23.5000) -- (101.5000,23.5000);



\end{scope}
\begin{scope}[cm={{1.15801,0.0,0.0,1.15801,(-19.81968,-10.65042)}},draw=blue,line cap=round,line join=round,line width=0.480pt]
  \path[draw] (44.5000,13.5000) -- (44.5000,102.5000) -- (142.5000,102.5000) -- (142.5000,13.5000) -- (44.5000,13.5000);



\end{scope}
\begin{scope}[cm={{1.15801,0.0,0.0,1.15801,(-140.25839,106.3496)}},draw=ca0a0a4,dash pattern=on 0.40pt off 0.80pt,line cap=round,line join=round,line width=0.400pt]
  \path[draw] (148.5000,102.5000) -- (246.5000,102.5000);



\end{scope}
\begin{scope}[cm={{1.15801,0.0,0.0,1.15801,(-140.25839,106.3496)}},draw=blue,line cap=round,line join=round,line width=0.480pt]
  \path[draw] (148.5000,102.5000) -- (151.5000,102.5000);



  \path[draw] (246.5000,102.5000) -- (243.5000,102.5000);



\end{scope}
\begin{scope}[cm={{1.15801,0.0,0.0,1.15801,(-140.25839,106.3496)}},draw=ca0a0a4,dash pattern=on 1.55pt off 1.55pt,line cap=round,line join=round,line width=0.259pt,miter limit=4.00]
  \path[draw,dash pattern=on 1.55pt off 1.55pt,line width=0.259pt,miter limit=4.00] (148.5000,79.5000) -- (246.5000,79.5000);



\end{scope}
\begin{scope}[cm={{1.15801,0.0,0.0,1.15801,(-140.25839,106.3496)}},draw=blue,line cap=round,line join=round,line width=0.480pt]
  \path[draw] (148.5000,79.5000) -- (151.5000,79.5000);



  \path[draw] (246.5000,79.5000) -- (243.5000,79.5000);



\end{scope}
\begin{scope}[cm={{1.15801,0.0,0.0,1.15801,(-140.25839,106.3496)}},draw=ca0a0a4,dash pattern=on 1.55pt off 1.55pt,line cap=round,line join=round,line width=0.259pt,miter limit=4.00]
  \path[draw,dash pattern=on 1.55pt off 1.55pt,line width=0.259pt,miter limit=4.00] (148.5000,57.5000) -- (246.5000,57.5000);



\end{scope}
\begin{scope}[cm={{1.15801,0.0,0.0,1.15801,(-140.25839,106.3496)}},draw=blue,line cap=round,line join=round,line width=0.480pt]
  \path[draw] (148.5000,57.5000) -- (151.5000,57.5000);



  \path[draw] (246.5000,57.5000) -- (243.5000,57.5000);



\end{scope}
\begin{scope}[cm={{1.15801,0.0,0.0,1.15801,(-140.25839,106.3496)}},draw=ca0a0a4,dash pattern=on 1.55pt off 1.55pt,line cap=round,line join=round,line width=0.259pt,miter limit=4.00]
  \path[draw,dash pattern=on 1.55pt off 1.55pt,line width=0.259pt,miter limit=4.00] (148.5000,35.5000) -- (246.5000,35.5000);



\end{scope}
\begin{scope}[cm={{1.15801,0.0,0.0,1.15801,(-140.25839,106.3496)}},draw=blue,line cap=round,line join=round,line width=0.480pt]
  \path[draw] (148.5000,35.5000) -- (151.5000,35.5000);



  \path[draw] (246.5000,35.5000) -- (243.5000,35.5000);



\end{scope}
\begin{scope}[cm={{1.15801,0.0,0.0,1.15801,(-140.25839,106.3496)}},draw=ca0a0a4,dash pattern=on 0.40pt off 0.80pt,line cap=round,line join=round,line width=0.400pt]
  \path[draw] (148.5000,13.5000) -- (246.5000,13.5000);



\end{scope}
\begin{scope}[cm={{1.15801,0.0,0.0,1.15801,(-77.72572,106.3496)}},draw=ca0a0a4,dash pattern=on 1.55pt off 1.55pt,line cap=round,line join=round,line width=0.259pt,miter limit=4.00]
  \path[draw,dash pattern=on 1.55pt off 1.55pt,line width=0.259pt,miter limit=4.00] (118.5000,102.5000) -- (118.5000,13.5000);



\end{scope}
\begin{scope}[cm={{1.15801,0.0,0.0,1.15801,(-140.25839,106.3496)}},draw=blue,line cap=round,line join=round,line width=0.480pt]
  \path[draw] (148.5000,13.5000) -- (151.5000,13.5000);



  \path[draw] (246.5000,13.5000) -- (243.5000,13.5000);



\end{scope}
\begin{scope}[cm={{1.15801,0.0,0.0,1.15801,(-140.25839,106.3496)}},draw=ca0a0a4,dash pattern=on 0.40pt off 0.80pt,line cap=round,line join=round,line width=0.400pt]
  \path[draw] (148.5000,102.5000) -- (148.5000,13.5000);



\end{scope}
\begin{scope}[cm={{1.15801,0.0,0.0,1.15801,(-140.25839,106.3496)}},draw=blue,line cap=round,line join=round,line width=0.480pt]
  \path[draw] (148.5000,102.5000) -- (148.5000,99.5000);



  \path[draw] (148.5000,13.5000) -- (148.5000,16.5000);



\end{scope}
\begin{scope}[cm={{1.15801,0.0,0.0,1.15801,(-140.25839,106.3496)}},draw=blue,line cap=round,line join=round,line width=0.480pt]
  \path[draw] (172.5000,102.5000) -- (172.5000,99.5000);



  \path[draw] (172.5000,13.5000) -- (172.5000,16.5000);



\end{scope}
\begin{scope}[cm={{1.15801,0.0,0.0,1.15801,(-48.77541,106.3496)}},draw=ca0a0a4,dash pattern=on 1.55pt off 1.55pt,line cap=round,line join=round,line width=0.259pt,miter limit=4.00]
  \path[draw,dash pattern=on 1.55pt off 1.55pt,line width=0.259pt,miter limit=4.00] (118.5000,102.5000) -- (118.5000,13.5000);



\end{scope}
\begin{scope}[cm={{1.15801,0.0,0.0,1.15801,(-140.25839,106.3496)}},draw=blue,line cap=round,line join=round,line width=0.480pt]
  \path[draw] (197.5000,102.5000) -- (197.5000,99.5000);



  \path[draw] (197.5000,13.5000) -- (197.5000,16.5000);



\end{scope}
\begin{scope}[cm={{1.15801,0.0,0.0,1.15801,(-20.98311,106.3496)}},draw=ca0a0a4,dash pattern=on 1.55pt off 1.55pt,line cap=round,line join=round,line width=0.259pt,miter limit=4.00]
  \path[draw,dash pattern=on 1.55pt off 1.55pt,line width=0.259pt,miter limit=4.00] (118.5000,102.5000) -- (118.5000,13.5000);



\end{scope}
\begin{scope}[cm={{1.15801,0.0,0.0,1.15801,(-140.25839,106.3496)}},draw=blue,line cap=round,line join=round,line width=0.480pt]
  \path[draw] (221.5000,102.5000) -- (221.5000,99.5000);



  \path[draw] (221.5000,13.5000) -- (221.5000,16.5000);



\end{scope}
\begin{scope}[cm={{1.15801,0.0,0.0,1.15801,(-140.25839,106.3496)}},draw=ca0a0a4,dash pattern=on 0.40pt off 0.80pt,line cap=round,line join=round,line width=0.400pt]
  \path[draw] (246.5000,102.5000) -- (246.5000,27.5000);



  \path[draw] (246.5000,19.5000) -- (246.5000,13.5000);



\end{scope}
\begin{scope}[cm={{1.15801,0.0,0.0,1.15801,(-140.25839,106.3496)}},draw=blue,line cap=round,line join=round,line width=0.480pt]
  \path[draw] (246.5000,102.5000) -- (246.5000,99.5000);



  \path[draw] (246.5000,13.5000) -- (246.5000,16.5000);



\end{scope}
\begin{scope}[cm={{1.15801,0.0,0.0,1.15801,(-140.25839,106.3496)}},draw=blue,line cap=round,line join=round,line width=0.480pt]
  \path[draw] (148.5000,13.5000) -- (148.5000,102.5000) -- (246.5000,102.5000) -- (246.5000,13.5000) -- (148.5000,13.5000);



\end{scope}
\begin{scope}[cm={{0.84029,0.0,0.0,0.84029,(-93.49162,81.25858)}},fill=cd9d9d9]
  \path[rounded corners=0.0000cm] (222.0000,18.0000) rectangle (238.0000,34.0000);



\end{scope}
\begin{scope}[cm={{1.15801,0.0,0.0,1.15801,(-140.25839,106.3496)}},draw=blue,line cap=round,line join=round,line width=0.480pt]
  \path[draw] (148.5000,13.5000) -- (148.5000,102.5000) -- (246.5000,102.5000) -- (246.5000,13.5000) -- (148.5000,13.5000);



\end{scope}
\begin{scope}[cm={{1.00588,0.0,0.0,1.00588,(25.10681,236.06903)}},draw=blue,line cap=rect,line join=bevel,line width=0.800pt]
  \path[fill=blue] (0.0000,0.0000) node[above right] (text1768-1-4) {-150};



\end{scope}
\begin{scope}[cm={{1.00588,0.0,0.0,1.00588,(52.31861,236.06903)}},draw=blue,line cap=rect,line join=bevel,line width=0.800pt]
  \path[fill=blue] (0.0000,0.0000) node[above right] (text1798-7-9) {-50};



\end{scope}
\begin{scope}[cm={{1.00588,0.0,0.0,1.00588,(82.52181,236.06903)}},draw=blue,line cap=rect,line join=bevel,line width=0.800pt]
  \path[fill=blue] (0.0000,0.0000) node[above right] (text1828-9-59) {50};



\end{scope}
\begin{scope}[cm={{1.00588,0.0,0.0,1.00588,(107.70381,236.06903)}},draw=blue,line cap=rect,line join=bevel,line width=0.800pt]
  \path[fill=blue] (0.0000,0.0000) node[above right] (text1858-6-5) {150};



\end{scope}
\begin{scope}[cm={{1.00588,0.0,0.0,1.00588,(133.32142,236.06903)}},draw=blue,line cap=rect,line join=bevel,line width=0.800pt]
  \path[fill=blue] (0.0000,0.0000) node[above right] (text1858-6-6-08) {250};



\end{scope}
\begin{scope}[cm={{1.00588,0.0,0.0,1.00588,(10.64827,227.36002)}},draw=blue,line cap=rect,line join=bevel,line width=0.800pt]
  \path[fill=blue] (0.0000,0.0000) node[above right] (text34-9-6-9) {-100};



\end{scope}
\begin{scope}[cm={{1.00588,0.0,0.0,1.00588,(21.3829,202.23013)}},draw=blue,line cap=rect,line join=bevel,line width=0.800pt]
  \path[fill=blue] (0.0000,0.0000) node[above right] (text64-6-8-7) {0};



\end{scope}
\begin{scope}[cm={{1.00588,0.0,0.0,1.00588,(13.33586,175.60074)}},draw=blue,line cap=rect,line join=bevel,line width=0.800pt]
  \path[fill=blue] (0.0000,0.0000) node[above right] (text94-7-9-2) {100};



\end{scope}
\begin{scope}[cm={{1.00588,0.0,0.0,1.00588,(13.33586,150.47134)}},draw=blue,line cap=rect,line join=bevel,line width=0.800pt]
  \path[fill=blue] (0.0000,0.0000) node[above right] (text124-9-0-3) {200};



\end{scope}
\begin{scope}[cm={{1.00588,0.0,0.0,1.00588,(13.33586,125.34194)}},draw=blue,line cap=rect,line join=bevel,line width=0.800pt]
  \path[fill=blue] (0.0000,0.0000) node[above right] (text154-3-9) {300};



\end{scope}

\end{tikzpicture}


  \vspace*{-.6ex}
  {\color{blue}
  Fig.~7:~Planning-scheduling of CPP and ground patterns detections% with PedNet CNN
  , utilizing the lowest configuration {\color{red}I} as a starting point in {\color{red}i} and the highest {\color{red}II} in {\color{red}ii} while varying atmospheric (same as Fig.~{\color{red}6}) and battery conditions. In {\color{red}a} are the re-planned trajectories, in {\color{red}b} the parameters, and in {\color{red}c} the energy w.r.t. the battery.}
  \vspace*{1ex}
\end{formal}
}

\vspace*{2em}

{\hspace*{-4.5em}\textbf{[R2:12]}\vspace*{-1.9em}}

Overall, the paper addresses a very interesting topic and brings together many different techniques to solve a challenging problem. It could be strengthened with more attention to clarity to convey the technical insights and more experiments showing an improvement in a performance metric of interest.

{\color{blue} 

{\hspace*{-4.5em}{[R2:12]}\vspace*{-1.9em}}

[!]}


\vspace{1em}



\vspace{2em}


\end{document}

