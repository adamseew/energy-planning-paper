
%%
%% Energy-Aware Planning-Scheduling for Autonomous Aerial Robots 
%%

\newcommand{\CLASSINPUTtoptextmargin}{19mm}
\newcommand{\CLASSINPUTbottomtextmargin}{19mm}
\newcommand{\CLASSINPUTinnersidemargin}{19mm}
\newcommand{\CLASSINPUToutersidemargin}{19mm}

\documentclass[letterpaper,10pt,conference,twoside]{IEEEtran}
\IEEEoverridecommandlockouts 

\newcommand{\stt}[1]{{\small\tt #1}} %\small\tt too small here
\newcommand{\powprof}{\stt{powprofiler}}
\newcommand{\figpath}{./figures}
\newcommand{\iu}{{i\mkern1mu}}
\let\labelindent\relax

\usepackage[inline]{enumitem}
\usepackage{booktabs}
%\usepackage{flushend}
\usepackage{tikz}
\usetikzlibrary{automata,positioning,decorations.pathreplacing,backgrounds}
%\usepackage[left=1in,right=1in,top=1in,bottom=1in]{geometry} 
\usepackage{pifont}

%% citation package
\usepackage{cite}

%% figures package
%\usepackage[pdftex]{graphicx}
%\graphicspath{{figures/}}
%\DeclareGraphicsExtensions{.pdf,.jpeg,.png}

%% math package
\usepackage{amsmath}
\makeatletter
\def\maketag@@@#1{\hbox{\m@th\normalfont\normalsize#1}}
\makeatother

\usepackage{mathtools}
\usepackage{amssymb}
\usepackage{arydshln}
\DeclarePairedDelimiter{\ceil}{\lceil}{\rceil}

\let\proof\relax
\let\endproof\relax
\usepackage{amsthm}

%% pseudocode package
\usepackage{algorithm}
\usepackage[noend]{algorithmic}

% counter to maintain the line numbering
\newcommand{\setalglineno}[1]{%
  \setcounter{ALC@line}{\numexpr#1-1}}

%% packages for alignment
%\usepackage{array}
%\usepackage{mdwmath}
%\usepackage{mdwtab}
%\usepackage{eqparbox}

%% packages for subfigures (eventually)
%\usepackage[tight,footnotesize]{subfigure}
\usepackage[font=footnotesize]{caption}
\usepackage[font=footnotesize]{subcaption}
%\usepackage[caption=false]{caption}
%\usepackage[font=footnotesize]{subfig}
%\usepackage[caption=false,font=footnotesize]{subfig}

%% package for urls
\usepackage{url}

%% hyperref
% and an override to make hyperref work with ieeeconf.cls
\makeatletter
\let\NAT@parse\undefined
\makeatother
\usepackage[pagebackref=true,breaklinks=true,colorlinks,bookmarks=false]{hyperref}
\makeatletter
\newcommand*{\textlabel}[2]{%
  \edef\@currentlabel{#1}% Set target label
  \phantomsection% Correct hyper reference link
  #1\label{#2}% Print and store label
}
\makeatother

\usepackage{textpos}

\DeclarePairedDelimiter\abs{\lvert}{\rvert}%
\DeclarePairedDelimiter\norm{\lVert}{\rVert}%

%% correct bad hyphenation here
\hyphenation{}

\renewcommand{\qedsymbol}{$\blacksquare$}

\theoremstyle{definition}
\newtheorem{thm}{Theorem}[section]
\newtheorem{lem}[thm]{Lemma}
\newtheorem{prop}[thm]{Proposition}
\newtheorem{assm}[thm]{Assumption}
\newtheorem{cor}{Corollary}
\newtheorem{conj}{Conjecture}[section]
\newtheorem{defn}{Definition}[section]
\newtheorem{exmp}{Example}[section]
\newtheorem*{pb}{Problem}%[section]
\newtheorem{rem}{Remark}
\newtheorem{obs}{Observation}
\newtheorem*{ctb}{Contribution}

\DeclareMathOperator*{\argmax}{arg\,max}
\DeclareMathOperator*{\argmin}{arg\,min}

% for blank page

%\usepackage{afterpage}

% for bibliography (double column balances)

%\usepackage{balance}

%% references (generates a bib file for bibtex)
%\begin{filecontents}{\jobname.bib}

%\end{filecontents}

\definecolor{c00FFFF}{RGB}{0,255,255}

\begin{document}
\bstctlcite{IEEEexample:BSTcontrol}

\title{\vspace{6mm}\bfseries\LARGE Energy-Aware Planning-Scheduling for Autonomous Aerial Robots}

%% author names and affiliations
\author{
  Adam Seewald$^{\text{1}}$, H\'ector Garc\'ia de Marina$^{\text{2}}$, Henrik Skov Midtiby$^{\text{3}}$, and Ulrik Pagh Schultz$^{\text{3}}$
  \thanks{The work was partly funded by EU grant \textnumero 779882 (TeamPlay). The work for H.\hspace*{.4ex}G. is supported by the Ramon y Cajal grant \textnumero RYC2020-030090-I.}
  \thanks{$^{\text{1}}$A.\hspace*{.4ex}S. is with the Department of Mechanical Engineering and Material Science, Yale University, CT, USA, but the work was performed while affiliated with the\hspace*{.6ex}SDU\hspace*{.4ex}UAS. Email: {\tt\footnotesize \href{mailto:adam.seewald@yale.edu}{adam.seewald@yale.edu}};}
  \thanks{$^{\text{2}}$H.\hspace*{.4ex}G. is with the Department of Computer Architecture and Technology and with CITIC, University of Granada, Spain;} 
  \thanks{$^{\text{3}}$H.\hspace*{.3ex}S.\hspace*{.3ex}M.,\hspace*{.5ex}U.\hspace*{.3ex}P.\hspace*{.3ex}S. are with the\hspace*{.5ex}SDU\hspace*{.3ex}UAS, %M{\ae}rsk Mc-Kinney M{\o}ller Institute, 
  University of Southern Denmark.}
}


%% make the title area
\maketitle

\vspace*{-.6ex}
\begin{abstract}
  In this paper, we present an online planning-scheduling approach for battery-powered autonomous aerial robots. The approach consists of simultaneously planning a coverage path and scheduling onboard computational tasks. We further derive a novel variable coverage motion robust to airborne constraints and an empirically motivated energy model. The model includes the energy contribution of the schedule based on an automatic computational energy modeling tool. Our experiments show how an initial flight plan is adjusted online as a function of the available battery, accounting for uncertainty. Our approach %furthermore 
  remedies possible in-flight failure in case of unexpected battery drops, e.g., due to adverse atmospheric conditions, and increases the overall fault tolerance.
\end{abstract}

%\vspace*{-.5ex}
%\begin{IEEEkeywords}
  %Aerial Systems: Perception and Autonomy, %Optimization and Optimal Control, 
  %Planning, %Scheduling and Coordination, 
  %Planning under Uncertainty
%  Motion and Path Planning, 
%  Energy and Environ- ment-Aware Automation
%\end{IEEEkeywords}

% For peer review papers, you can put extra information on the cover
% page as needed:
% \ifCLASSOPTIONpeerreview
% \begin{center} \bfseries EDICS Category: 3-BBND \end{center}
% \fi
%
% For peerreview papers, this IEEEtran command inserts a page break and
% creates the second title. It will be ignored for other modes.
%\IEEEpeerreviewmaketitle


\vspace*{-.6ex}
%%%%%%%%%%%%%%%%%%%%%%%%%
\section{Introduction}  %
\label{sec:intro}       %
                        %
Use cases involving aerial robots span broadly. They comprise diverse planning and scheduling strategies and often require high autonomy under strict energy budgets. 
One such use case is coverage path planning (CPP)~\cite{choset2001coverage,galceran2013survey}, which consists of, e.g., an aerial robot visiting every point in a given space~\cite{cabreira2019survey} while running assigned computational tasks. Here, the aerial robot might detect ground patterns and notify other ground-based actors. %with little human interaction. 
Such use cases arise in %, e.g., 
precision agriculture~\cite{hajjaj2014review} 
where 
{\color{black}information collection prior to a harvesting operation and}
%harvesting involves ground vehicles~\cite{%qingchun2012study,
%dong2011development, de2011design%,aljanobi2010setup, 
%li2008analysis, 
%edan2000robotic
%}, information collection prior to an operation as well as 
damage prevention during the operation involve aerial robots~\cite{puri2017agriculture, daponte2019review}.
Microcontrollers and heterogeneous computing hardware~\cite{mei2005case} (i.e., with CPUs and GPUs) running power-demanding computational tasks are frequently mounted onto the robots in these and many other scenarios~\cite{william2019aerial%, peng2019evaluating,
%wang2020yolo
,alexey2021autonomous}.
%In these and many others, the robot frequently mounts microcontroller and heterogeneous computing hardware~\cite{mei2005case} (i.e., with CPUs and GPUs) running power-demanding computational tasks~\cite{william2019aerial, peng2019evaluating,wang2020yolo,alexey2021autonomous}. 
We refer to onboard computational tasks that can be scheduled with an energy impact as \emph{computations}. We are interested in the energy optimization of motion plans and computations schedules in-flight and refer to it as energy-aware \emph{planning-scheduling}. 
The energy optimization of computations schedules can be achieved by, e.g., varying the quality of service between specific bounds~\cite{ho2019qos} and frequency and voltage of the computing hardware~\cite{mei2005case,brateman2006energy,zhang2007low}. We focus on the former aspect and schedule the onboard computations altering their quality while simultaneously changing the quality of the coverage.
{\color{black} Concretely, we alter how often the aerial robot detects ground patterns along with the distance of the lines that form the coverage.}
Fig.~\ref{fig:il-abs} illustrates the intuition: an aerial robot flies a plan with maximal coverage and schedule~(\ref{sth:i}), that is optimized during flight to respect the battery state~(\ref{sth:ii}), and altered due to, e.g., %unexpected 
battery defects~(\ref{sth:iii}).

\begin{figure}[t]
  \centering
  \vspace*{-5ex}
 %\begin{tikzpicture}
 %  \node[inner sep=0pt] () at (0,0)
        {\color{black}\scriptsize 
\definecolor{ce9e9e9}{RGB}{233,233,233}
\definecolor{cff0000}{RGB}{255,0,0}
\definecolor{c00ff00}{RGB}{0,255,0}
\definecolor{c7e7e7e}{RGB}{126,126,126}
\definecolor{cffffff}{RGB}{255,255,255}
\definecolor{ca0a0a4}{RGB}{160,160,164}


\def \globalscale {0.980000}
\begin{tikzpicture}[y=0.80pt, x=0.80pt, yscale=-\globalscale, xscale=\globalscale, inner sep=0pt, outer sep=0pt]
\path[fill=ce9e9e9,line join=round,line width=0.160pt] (68.3886,228.7210) -- (68.8450,228.3540) -- (66.6933,224.9770) -- (66.4207,222.6910) -- (69.0874,226.5580) -- (70.0737,229.4550) .. controls (70.9241,232.1810) and (69.6541,232.8230) .. (69.6541,232.8230) -- (69.4683,232.9720) -- (49.7967,249.0540) -- (49.5637,251.2080) -- (49.0845,252.9960) .. controls (48.9731,253.2670) and (48.4526,253.8860) .. (48.1747,254.1350) .. controls (47.8967,254.3830) and (47.1529,254.4130) .. (46.5706,253.6430) .. controls (45.9884,252.8730) and (45.7178,251.8140) .. (45.7178,251.8140) .. controls (45.7178,251.8140) and (45.2522,250.4730) .. (45.1547,249.0600) -- (34.4799,243.8570) .. controls (34.4799,243.8570) and (34.1329,243.7130) .. (33.9091,243.3820) -- (30.6477,238.6210) -- (31.2184,236.7630) -- (33.8588,239.6060) .. controls (33.8588,239.6060) and (34.4955,240.1680) .. (34.7985,240.1670) -- (43.5261,240.2490) -- (43.9526,238.7440) -- (46.5333,240.0820) -- (47.3127,238.9230) -- (47.4349,237.4340) -- (47.4392,237.4000) .. controls (47.3552,237.4640) and (46.5555,237.3000) .. (46.5555,237.3000) .. controls (46.5555,237.3000) and (47.0376,237.1070) .. (47.6699,236.5890) .. controls (47.6699,236.5890) and (47.9190,235.8840) .. (48.4668,235.6390) -- (48.7424,235.7340) -- (48.8942,236.1030) .. controls (48.8942,236.1030) and (49.7624,235.3070) .. (50.7197,235.6110) .. controls (50.7197,235.6110) and (50.3981,235.6550) .. (49.5019,236.3140) -- (49.0791,236.4260) .. controls (49.1951,237.3060) and (49.2692,237.7870) .. (49.2692,237.7870) -- (49.5106,236.9830) -- (51.0369,238.6600) -- (52.1359,238.3620) -- cycle;



\path[draw=cff0000,line join=round,line width=0.512pt] (40.6194,232.3590) -- (66.3142,232.4260) -- (57.6437,255.4710) -- (32.2484,255.4190) -- cycle;



\path[draw=cff0000,line join=round,line width=0.512pt] (66.3388,232.5190) -- (53.8771,153.1020);



\path[draw=cff0000,line join=round,line width=0.512pt] (51.7178,158.9010) -- (57.4022,255.5120);



\path[draw=cff0000,line join=round,line width=0.512pt] (40.6398,232.4770) -- (47.5944,153.1300);



\path[draw=cff0000,line join=round,line width=0.512pt] (32.3132,255.3660) -- (45.4889,158.8260);



\path[draw=cff0000,line join=round,line width=0.256pt] (44.8941,255.3291) -- (53.0140,232.5691);



\path[draw=cff0000,line join=round,line width=0.256pt] (61.6178,244.0490) -- (36.5576,244.0490);



\path[fill=ce9e9e9,line join=round,line width=0.160pt] (154.2343,215.1710) -- (154.4683,215.6890) -- (151.1743,218.3530) -- (149.9593,220.5490) -- (153.9053,217.5670) -- (155.9933,214.9970) .. controls (157.8923,212.5530) and (157.0803,211.4880) .. (157.0803,211.4880) -- (156.9843,211.2780) -- (147.0263,188.6530) -- (147.7503,186.4210) -- (148.1073,184.4730) .. controls (148.1283,184.1650) and (147.9483,183.3720) .. (147.8163,183.0320) .. controls (147.6843,182.6910) and (147.0603,182.4130) .. (146.2303,182.9900) .. controls (145.4013,183.5670) and (144.7143,184.5360) .. (144.7143,184.5360) .. controls (144.7143,184.5360) and (143.7403,185.7230) .. (143.0493,187.1040) -- (131.6673,188.7610) .. controls (131.6673,188.7610) and (131.3083,188.7890) .. (130.9743,189.0460) -- (126.1353,192.7240) -- (125.8273,194.7730) -- (129.3103,192.8070) .. controls (129.3103,192.8070) and (130.0973,192.4550) .. (130.3573,192.5570) -- (137.8723,195.3770) -- (137.5923,197.0230) -- (140.3783,196.5430) -- (140.5493,197.9610) -- (140.0143,199.4910) -- (140.0033,199.5270) .. controls (139.9593,199.4350) and (139.2033,199.3330) .. (139.2033,199.3330) .. controls (139.2033,199.3330) and (139.5333,199.6860) .. (139.8533,200.4150) .. controls (139.8533,200.4150) and (139.7643,201.2030) .. (140.1283,201.6290) -- (140.4053,201.6260) -- (140.6943,201.3070) .. controls (140.6943,201.3070) and (141.0963,202.3920) .. (142.0473,202.4060) .. controls (142.0473,202.4060) and (141.7903,202.2560) .. (141.3053,201.2990) -- (140.9913,201.0460) .. controls (141.4683,200.2050) and (141.7383,199.7480) .. (141.7383,199.7480) -- (141.6003,200.6320) -- (143.6283,199.4620) -- (144.4423,200.1260) -- cycle;



\path[draw=cff0000,line join=round,line width=0.512pt] (132.1743,189.0530) -- (157.8693,189.1200) -- (149.1983,212.1650) -- (123.8033,212.1140) -- cycle;



\path[draw=cff0000,line join=round,line width=0.512pt] (157.8933,189.2130) -- (145.4313,109.7960);



\path[draw=cff0000,line join=round,line width=0.512pt] (143.2723,115.5950) -- (148.9563,212.2060);



\path[draw=cff0000,line join=round,line width=0.512pt] (132.1942,189.1710) -- (139.1482,109.8240);



\path[draw=cff0000,line join=round,line width=0.512pt] (123.8672,212.0600) -- (137.0433,115.5200);



\path[draw=cff0000,line join=round,line width=0.256pt] (136.4483,212.0230) -- (144.5683,189.2630);



\path[draw=cff0000,line join=round,line width=0.256pt] (153.1723,200.7430) -- (128.1122,200.7430);



\path[fill=ce9e9e9,line join=round,line width=0.160pt] (133.1043,103.0940) -- (133.8143,103.1630) -- (135.4273,100.4380) -- (137.4373,99.2210) -- (135.7473,102.4480) -- (133.6973,104.3300) .. controls (131.7083,106.0600) and (130.1003,105.6900) .. (130.1003,105.6900) -- (129.8123,105.6610) -- (98.9459,102.8020) -- (96.6733,103.6900) -- (94.5652,104.2760) .. controls (94.2151,104.3450) and (93.2105,104.3600) .. (92.7539,104.3300) .. controls (92.2972,104.2990) and (91.6912,103.9200) .. (91.9880,103.2500) .. controls (92.2849,102.5800) and (93.1040,101.9400) .. (93.1040,101.9400) .. controls (93.1040,101.9400) and (94.0457,101.0640) .. (95.3433,100.3480) -- (92.1246,92.2560) .. controls (92.1246,92.2560) and (91.9952,92.0050) .. (92.1437,91.7310) -- (94.2403,87.7700) -- (96.4880,87.1990) -- (95.7727,89.9310) .. controls (95.7727,89.9310) and (95.7194,90.5330) .. (95.9553,90.6920) -- (102.6403,95.3490) -- (104.4313,94.8690) -- (105.1313,96.8630) -- (106.8623,96.7310) -- (108.4023,96.0960) -- (108.4393,96.0820) .. controls (108.3123,96.0680) and (107.8513,95.5680) .. (107.8513,95.5680) .. controls (107.8513,95.5680) and (108.4123,95.7320) .. (109.4063,95.8240) .. controls (109.4063,95.8240) and (110.2833,95.6250) .. (110.9453,95.8000) -- (111.0673,95.9900) -- (110.8263,96.2440) .. controls (110.8263,96.2440) and (112.2723,96.3290) .. (112.7193,96.9790) .. controls (112.7193,96.9790) and (112.4273,96.8290) .. (111.0933,96.6640) -- (110.6563,96.4930) .. controls (109.8913,96.9680) and (109.4813,97.2330) .. (109.4813,97.2330) -- (110.4493,96.9830) -- (110.0033,98.5780) -- (111.1453,99.0200) -- cycle;



\path[draw=c00ff00,line join=round,line width=0.512pt] (97.1528,89.9640) -- (122.8473,90.0310) -- (114.1773,113.0760) -- (88.7817,113.0250) -- cycle;



\path[draw=c00ff00,line join=round,line width=0.512pt] (88.8460,112.9710) -- (102.0213,16.4309);



\path[draw=c00ff00,line join=round,line width=0.512pt] (122.8712,90.1250) -- (110.4103,10.7072);



\path[draw=c00ff00,line join=round,line width=0.512pt] (108.2502,16.5058) -- (113.9352,113.1180);



\path[draw=c00ff00,line join=round,line width=0.512pt] (97.1726,90.0830) -- (104.1273,10.7353);



\path[draw=c00ff00,line join=round,line width=0.256pt] (97.1312,112.8840) -- (105.9843,89.9510);



\path[draw=c00ff00,line join=round,line width=0.256pt] (105.7042,112.7110) -- (114.5583,89.7780);



\path[draw=c00ff00,line join=round,line width=0.256pt] (91.9046,104.9910) -- (117.1582,105.3110);



\path[draw=c00ff00,line join=round,line width=0.256pt] (94.6780,96.8180) -- (119.9312,97.1380);



\begin{scope}[shift={(0.22227,-34.242)}]
  \path[fill=c7e7e7e,line join=round,line width=0.256pt] (208.7930,292.6290) -- (208.5270,292.6290) -- (208.5270,292.4690) -- (208.7930,292.4690) -- cycle(207.9930,292.6290) -- (207.7270,292.6290) -- (207.7270,292.4690) -- (207.9930,292.4690) -- cycle(207.1930,292.6290) -- (206.9270,292.6290) -- (206.9270,292.4690) -- (207.1930,292.4690) -- cycle(206.3930,292.6290) -- (206.1270,292.6290) -- (206.1270,292.4690) -- (206.3930,292.4690) -- cycle(205.5930,292.6290) -- (205.3270,292.6290) -- (205.3270,292.4690) -- (205.5930,292.4690) -- cycle(204.7930,292.6290) -- (204.5270,292.6290) -- (204.5270,292.4690) -- (204.7930,292.4690) -- cycle(203.9930,292.6290) -- (203.7270,292.6290) -- (203.7270,292.4690) -- (203.9930,292.4690) -- cycle(203.1930,292.6290) -- (202.9270,292.6290) -- (202.9270,292.4690) -- (203.1930,292.4690) -- cycle(202.3930,292.6290) -- (202.1270,292.6290) -- (202.1270,292.4690) -- (202.3930,292.4690) -- cycle(201.5930,292.6290) -- (201.3270,292.6290) -- (201.3270,292.4690) -- (201.5930,292.4690) -- cycle(200.7930,292.6290) -- (200.5270,292.6290) -- (200.5270,292.4690) -- (200.7930,292.4690) -- cycle(199.9930,292.6290) -- (199.7270,292.6290) -- (199.7270,292.4690) -- (199.9930,292.4690) -- cycle(199.1930,292.6290) -- (198.9270,292.6290) -- (198.9270,292.4690) -- (199.1930,292.4690) -- cycle(198.3930,292.6290) -- (198.1270,292.6290) -- (198.1270,292.4690) -- (198.3930,292.4690) -- cycle(197.5930,292.6290) -- (197.3270,292.6290) -- (197.3270,292.4690) -- (197.5930,292.4690) -- cycle(196.7930,292.6290) -- (196.5270,292.6290) -- (196.5270,292.4690) -- (196.7930,292.4690) -- cycle(195.9930,292.6290) -- (195.7270,292.6290) -- (195.7270,292.4690) -- (195.9930,292.4690) -- cycle(195.1930,292.6290) -- (194.9270,292.6290) -- (194.9270,292.4690) -- (195.1930,292.4690) -- cycle(194.3930,292.6290) -- (194.1270,292.6290) -- (194.1270,292.4690) -- (194.3930,292.4690) -- cycle(193.5930,292.6290) -- (193.3270,292.6290) -- (193.3270,292.4690) -- (193.5930,292.4690) -- cycle(192.7930,292.6290) -- (192.5270,292.6290) -- (192.5270,292.4690) -- (192.7930,292.4690) -- cycle(191.9930,292.6290) -- (191.7270,292.6290) -- (191.7270,292.4690) -- (191.9930,292.4690) -- cycle(191.1930,292.6290) -- (190.9270,292.6290) -- (190.9270,292.4690) -- (191.1930,292.4690) -- cycle(190.3930,292.6290) -- (190.1270,292.6290) -- (190.1270,292.4690) -- (190.3930,292.4690) -- cycle(189.5930,292.6290) -- (189.3270,292.6290) -- (189.3270,292.4690) -- (189.5930,292.4690) -- cycle(188.7930,292.6290) -- (188.5270,292.6290) -- (188.5270,292.4690) -- (188.7930,292.4690) -- cycle(187.9930,292.6290) -- (187.7270,292.6290) -- (187.7270,292.4690) -- (187.9930,292.4690) -- cycle(187.1930,292.6290) -- (186.9270,292.6290) -- (186.9270,292.4690) -- (187.1930,292.4690) -- cycle(186.3930,292.6290) -- (186.1270,292.6290) -- (186.1270,292.4690) -- (186.3930,292.4690) -- cycle(185.5930,292.6290) -- (185.3270,292.6290) -- (185.3270,292.4690) -- (185.5930,292.4690) -- cycle(184.7930,292.6290) -- (184.5270,292.6290) -- (184.5270,292.4690) -- (184.7930,292.4690) -- cycle(183.9930,292.6290) -- (183.7270,292.6290) -- (183.7270,292.4690) -- (183.9930,292.4690) -- cycle(183.1930,292.6290) -- (182.9270,292.6290) -- (182.9270,292.4690) -- (183.1930,292.4690) -- cycle(182.3930,292.6290) -- (182.1270,292.6290) -- (182.1270,292.4690) -- (182.3930,292.4690) -- cycle(181.5930,292.6290) -- (181.3270,292.6290) -- (181.3270,292.4690) -- (181.5930,292.4690) -- cycle(180.7930,292.6290) -- (180.5270,292.6290) -- (180.5270,292.4690) -- (180.7930,292.4690) -- cycle(179.9930,292.6290) -- (179.7270,292.6290) -- (179.7270,292.4690) -- (179.9930,292.4690) -- cycle(179.1930,292.6290) -- (178.9270,292.6290) -- (178.9270,292.4690) -- (179.1930,292.4690) -- cycle(178.3930,292.6290) -- (178.1270,292.6290) -- (178.1270,292.4690) -- (178.3930,292.4690) -- cycle(177.5930,292.6290) -- (177.3270,292.6290) -- (177.3270,292.4690) -- (177.5930,292.4690) -- cycle(176.7930,292.6290) -- (176.5270,292.6290) -- (176.5270,292.4690) -- (176.7930,292.4690) -- cycle(175.9930,292.6290) -- (175.7270,292.6290) -- (175.7270,292.4690) -- (175.9930,292.4690) -- cycle(175.1930,292.6290) -- (174.9270,292.6290) -- (174.9270,292.4690) -- (175.1930,292.4690) -- cycle(174.3930,292.6290) -- (174.1270,292.6290) -- (174.1270,292.4690) -- (174.3930,292.4690) -- cycle(173.5930,292.6290) -- (173.3270,292.6290) -- (173.3270,292.4690) -- (173.5930,292.4690) -- cycle(172.7930,292.6290) -- (172.5270,292.6290) -- (172.5270,292.4690) -- (172.7930,292.4690) -- cycle(171.9930,292.6290) -- (171.7270,292.6290) -- (171.7270,292.4690) -- (171.9930,292.4690) -- cycle(171.1930,292.6290) -- (170.9270,292.6290) -- (170.9270,292.4690) -- (171.1930,292.4690) -- cycle(170.3930,292.6290) -- (170.1270,292.6290) -- (170.1270,292.4690) -- (170.3930,292.4690) -- cycle(169.5930,292.6290) -- (169.3270,292.6290) -- (169.3270,292.4690) -- (169.5930,292.4690) -- cycle(168.7930,292.6290) -- (168.5270,292.6290) -- (168.5270,292.4690) -- (168.7930,292.4690) -- cycle(167.9930,292.6290) -- (167.7270,292.6290) -- (167.7270,292.4690) -- (167.9930,292.4690) -- cycle(167.1930,292.6290) -- (166.9270,292.6290) -- (166.9270,292.4690) -- (167.1930,292.4690) -- cycle(166.3930,292.6290) -- (166.1270,292.6290) -- (166.1270,292.4690) -- (166.3930,292.4690) -- cycle(165.5930,292.6290) -- (165.3270,292.6290) -- (165.3270,292.4690) -- (165.5930,292.4690) -- cycle(164.7930,292.6290) -- (164.5270,292.6290) -- (164.5270,292.4690) -- (164.7930,292.4690) -- cycle(163.9930,292.6290) -- (163.7270,292.6290) -- (163.7270,292.4690) -- (163.9930,292.4690) -- cycle(163.1930,292.6290) -- (162.9270,292.6290) -- (162.9270,292.4690) -- (163.1930,292.4690) -- cycle(162.3930,292.6290) -- (162.1270,292.6290) -- (162.1270,292.4690) -- (162.3930,292.4690) -- cycle(161.5930,292.6290) -- (161.3270,292.6290) -- (161.3270,292.4690) -- (161.5930,292.4690) -- cycle(160.7930,292.6290) -- (160.5270,292.6290) -- (160.5270,292.4690) -- (160.7930,292.4690) -- cycle(159.9930,292.6290) -- (159.7270,292.6290) -- (159.7270,292.4690) -- (159.9930,292.4690) -- cycle(159.1930,292.6290) -- (158.9270,292.6290) -- (158.9270,292.4690) -- (159.1930,292.4690) -- cycle(158.3930,292.6290) -- (158.1270,292.6290) -- (158.1270,292.4690) -- (158.3930,292.4690) -- cycle(157.5930,292.6290) -- (157.3270,292.6290) -- (157.3270,292.4690) -- (157.5930,292.4690) -- cycle(156.7930,292.6290) -- (156.5270,292.6290) -- (156.5270,292.4690) -- (156.7930,292.4690) -- cycle(155.9930,292.6290) -- (155.7270,292.6290) -- (155.7270,292.4690) -- (155.9930,292.4690) -- cycle(155.1930,292.6290) -- (154.9270,292.6290) -- (154.9270,292.4690) -- (155.1930,292.4690) -- cycle(154.3930,292.6290) -- (154.1270,292.6290) -- (154.1270,292.4690) -- (154.3930,292.4690) -- cycle(153.5930,292.6290) -- (153.3270,292.6290) -- (153.3270,292.4690) -- (153.5930,292.4690) -- cycle(152.7930,292.6290) -- (152.5270,292.6290) -- (152.5270,292.4690) -- (152.7930,292.4690) -- cycle(151.9930,292.6290) -- (151.7270,292.6290) -- (151.7270,292.4690) -- (151.9930,292.4690) -- cycle(151.1930,292.6290) -- (150.9270,292.6290) -- (150.9270,292.4690) -- (151.1930,292.4690) -- cycle(150.3930,292.6290) -- (150.1270,292.6290) -- (150.1270,292.4690) -- (150.3930,292.4690) -- cycle(149.5930,292.6290) -- (149.3270,292.6290) -- (149.3270,292.4690) -- (149.5930,292.4690) -- cycle(148.7930,292.6290) -- (148.5270,292.6290) -- (148.5270,292.4690) -- (148.7930,292.4690) -- cycle(147.9930,292.6290) -- (147.7270,292.6290) -- (147.7270,292.4690) -- (147.9930,292.4690) -- cycle(147.1930,292.6290) -- (146.9270,292.6290) -- (146.9270,292.4690) -- (147.1930,292.4690) -- cycle(146.3930,292.6290) -- (146.1270,292.6290) -- (146.1270,292.4690) -- (146.3930,292.4690) -- cycle(145.5930,292.6290) -- (145.3270,292.6290) -- (145.3270,292.4690) -- (145.5930,292.4690) -- cycle(144.7930,292.6290) -- (144.5270,292.6290) -- (144.5270,292.4690) -- (144.7930,292.4690) -- cycle(143.9930,292.6290) -- (143.7270,292.6290) -- (143.7270,292.4690) -- (143.9930,292.4690) -- cycle(143.1930,292.6290) -- (142.9270,292.6290) -- (142.9270,292.4690) -- (143.1930,292.4690) -- cycle(142.3930,292.6290) -- (142.1270,292.6290) -- (142.1270,292.4690) -- (142.3930,292.4690) -- cycle(141.5930,292.6290) -- (141.3270,292.6290) -- (141.3270,292.4690) -- (141.5930,292.4690) -- cycle(140.7940,292.6290) -- (140.5270,292.6290) -- (140.5270,292.4690) -- (140.7940,292.4690) -- cycle(139.9930,292.6290) -- (139.7270,292.6290) -- (139.7270,292.4690) -- (139.9930,292.4690) -- cycle(139.1930,292.6290) -- (138.9270,292.6290) -- (138.9270,292.4690) -- (139.1930,292.4690) -- cycle(138.3930,292.6290) -- (138.1270,292.6290) -- (138.1270,292.4690) -- (138.3930,292.4690) -- cycle(137.5930,292.6290) -- (137.3270,292.6290) -- (137.3270,292.4690) -- (137.5930,292.4690) -- cycle(136.7940,292.6290) -- (136.5270,292.6290) -- (136.5270,292.4690) -- (136.7940,292.4690) -- cycle(135.9930,292.6290) -- (135.7270,292.6290) -- (135.7270,292.4690) -- (135.9930,292.4690) -- cycle(135.1930,292.6290) -- (134.9270,292.6290) -- (134.9270,292.4690) -- (135.1930,292.4690) -- cycle(134.3940,292.6290) -- (134.1270,292.6290) -- (134.1270,292.4690) -- (134.3940,292.4690) -- cycle(133.5940,292.6290) -- (133.3270,292.6290) -- (133.3270,292.4690) -- (133.5940,292.4690) -- cycle(132.7940,292.6290) -- (132.5270,292.6290) -- (132.5270,292.4690) -- (132.7940,292.4690) -- cycle(131.9930,292.6290) -- (131.7270,292.6290) -- (131.7270,292.4690) -- (131.9930,292.4690) -- cycle(131.1930,292.6290) -- (130.9270,292.6290) -- (130.9270,292.4690) -- (131.1930,292.4690) -- cycle(130.3940,292.6290) -- (130.1270,292.6290) -- (130.1270,292.4690) -- (130.3940,292.4690) -- cycle(129.5940,292.6290) -- (129.3270,292.6290) -- (129.3270,292.4690) -- (129.5940,292.4690) -- cycle(128.7940,292.6290) -- (128.5270,292.6290) -- (128.5270,292.4690) -- (128.7940,292.4690) -- cycle(127.9940,292.6290) -- (127.7270,292.6290) -- (127.7270,292.4690) -- (127.9940,292.4690) -- cycle(127.1940,292.6290) -- (126.9270,292.6290) -- (126.9270,292.4690) -- (127.1940,292.4690) -- cycle(126.3940,292.6290) -- (126.1270,292.6290) -- (126.1270,292.4690) -- (126.3940,292.4690) -- cycle(125.5940,292.6290) -- (125.3270,292.6290) -- (125.3270,292.4690) -- (125.5940,292.4690) -- cycle(124.7940,292.6290) -- (124.5270,292.6290) -- (124.5270,292.4690) -- (124.7940,292.4690) -- cycle(123.9940,292.6290) -- (123.7270,292.6290) -- (123.7270,292.4690) -- (123.9940,292.4690) -- cycle(123.1940,292.6290) -- (122.9270,292.6290) -- (122.9270,292.4690) -- (123.1940,292.4690) -- cycle(122.3940,292.6290) -- (122.1270,292.6290) -- (122.1270,292.4690) -- (122.3940,292.4690) -- cycle(121.5940,292.6290) -- (121.3270,292.6290) -- (121.3270,292.4690) -- (121.5940,292.4690) -- cycle(120.7940,292.6290) -- (120.5270,292.6290) -- (120.5270,292.4690) -- (120.7940,292.4690) -- cycle(119.9940,292.6290) -- (119.7270,292.6290) -- (119.7270,292.4690) -- (119.9940,292.4690) -- cycle(119.1940,292.6290) -- (118.9270,292.6290) -- (118.9270,292.4690) -- (119.1940,292.4690) -- cycle(118.3940,292.6290) -- (118.1270,292.6290) -- (118.1270,292.4690) -- (118.3940,292.4690) -- cycle(117.5940,292.6290) -- (117.3270,292.6290) -- (117.3270,292.4690) -- (117.5940,292.4690) -- cycle(116.7940,292.6290) -- (116.5270,292.6290) -- (116.5270,292.4690) -- (116.7940,292.4690) -- cycle(115.9940,292.6290) -- (115.7270,292.6290) -- (115.7270,292.4690) -- (115.9940,292.4690) -- cycle(115.1940,292.6290) -- (114.9270,292.6290) -- (114.9270,292.4690) -- (115.1940,292.4690) -- cycle(114.3940,292.6290) -- (114.1270,292.6290) -- (114.1270,292.4690) -- (114.3940,292.4690) -- cycle(113.5940,292.6290) -- (113.3270,292.6290) -- (113.3270,292.4690) -- (113.5940,292.4690) -- cycle(112.7940,292.6290) -- (112.5270,292.6290) -- (112.5270,292.4690) -- (112.7940,292.4690) -- cycle(111.9940,292.6290) -- (111.7270,292.6290) -- (111.7270,292.4690) -- (111.9940,292.4690) -- cycle(111.1940,292.6290) -- (110.9270,292.6290) -- (110.9270,292.4690) -- (111.1940,292.4690) -- cycle(110.3940,292.6290) -- (110.1270,292.6290) -- (110.1270,292.4690) -- (110.3940,292.4690) -- cycle(109.5940,292.6290) -- (109.3270,292.6290) -- (109.3270,292.4690) -- (109.5940,292.4690) -- cycle(108.7940,292.6290) -- (108.5270,292.6290) -- (108.5270,292.4690) -- (108.7940,292.4690) -- cycle(107.9940,292.6290) -- (107.7270,292.6290) -- (107.7270,292.4690) -- (107.9940,292.4690) -- cycle(107.1940,292.6290) -- (106.9270,292.6290) -- (106.9270,292.4690) -- (107.1940,292.4690) -- cycle(106.3940,292.6290) -- (106.1270,292.6290) -- (106.1270,292.4690) -- (106.3940,292.4690) -- cycle(105.5940,292.6290) -- (105.3270,292.6290) -- (105.3270,292.4690) -- (105.5940,292.4690) -- cycle(104.7940,292.6290) -- (104.5270,292.6290) -- (104.5270,292.4690) -- (104.7940,292.4690) -- cycle(103.9940,292.6290) -- (103.7270,292.6290) -- (103.7270,292.4690) -- (103.9940,292.4690) -- cycle(103.1940,292.6290) -- (102.9270,292.6290) -- (102.9270,292.4690) -- (103.1940,292.4690) -- cycle(102.3940,292.6290) -- (102.1270,292.6290) -- (102.1270,292.4690) -- (102.3940,292.4690) -- cycle(101.5940,292.6290) -- (101.3270,292.6290) -- (101.3270,292.4690) -- (101.5940,292.4690) -- cycle(100.7940,292.6290) -- (100.5270,292.6290) -- (100.5270,292.4690) -- (100.7940,292.4690) -- cycle(99.9936,292.6290) -- (99.7269,292.6290) -- (99.7269,292.4690) -- (99.9936,292.4690) -- cycle(99.1936,292.6290) -- (98.9269,292.6290) -- (98.9269,292.4690) -- (99.1936,292.4690) -- cycle(98.3936,292.6290) -- (98.1269,292.6290) -- (98.1269,292.4690) -- (98.3936,292.4690) -- cycle(97.5936,292.6290) -- (97.3270,292.6290) -- (97.3270,292.4690) -- (97.5936,292.4690) -- cycle(96.7936,292.6290) -- (96.5269,292.6290) -- (96.5269,292.4690) -- (96.7936,292.4690) -- cycle(95.9936,292.6290) -- (95.7269,292.6290) -- (95.7269,292.4690) -- (95.9936,292.4690) -- cycle(95.1936,292.6290) -- (94.9269,292.6290) -- (94.9269,292.4690) -- (95.1936,292.4690) -- cycle(94.3936,292.6290) -- (94.1269,292.6290) -- (94.1269,292.4690) -- (94.3936,292.4690) -- cycle(93.5936,292.6290) -- (93.3270,292.6290) -- (93.3270,292.4690) -- (93.5936,292.4690) -- cycle(92.7936,292.6290) -- (92.5269,292.6290) -- (92.5269,292.4690) -- (92.7936,292.4690) -- cycle(91.9936,292.6290) -- (91.7269,292.6290) -- (91.7269,292.4690) -- (91.9936,292.4690) -- cycle(91.1936,292.6290) -- (90.9270,292.6290) -- (90.9270,292.4690) -- (91.1936,292.4690) -- cycle(90.3936,292.6290) -- (90.1270,292.6290) -- (90.1270,292.4690) -- (90.3936,292.4690) -- cycle(89.5936,292.6290) -- (89.3270,292.6290) -- (89.3270,292.4690) -- (89.5936,292.4690) -- cycle(88.7936,292.6290) -- (88.5269,292.6290) -- (88.5269,292.4690) -- (88.7936,292.4690) -- cycle(87.9936,292.6290) -- (87.7269,292.6290) -- (87.7269,292.4690) -- (87.9936,292.4690) -- cycle(87.1936,292.6290) -- (86.9270,292.6290) -- (86.9270,292.4690) -- (87.1936,292.4690) -- cycle(86.3936,292.6290) -- (86.1270,292.6290) -- (86.1270,292.4690) -- (86.3936,292.4690) -- cycle(85.5936,292.6290) -- (85.3270,292.6290) -- (85.3270,292.4690) -- (85.5936,292.4690) -- cycle(84.7936,292.6290) -- (84.5270,292.6290) -- (84.5270,292.4690) -- (84.7936,292.4690) -- cycle(83.9936,292.6290) -- (83.7270,292.6290) -- (83.7270,292.4690) -- (83.9936,292.4690) -- cycle(83.1937,292.6290) -- (82.9270,292.6290) -- (82.9270,292.4690) -- (83.1937,292.4690) -- cycle(82.3936,292.6290) -- (82.1270,292.6290) -- (82.1270,292.4690) -- (82.3936,292.4690) -- cycle(81.5936,292.6290) -- (81.3270,292.6290) -- (81.3270,292.4690) -- (81.5936,292.4690) -- cycle(80.7936,292.6290) -- (80.5270,292.6290) -- (80.5270,292.4690) -- (80.7936,292.4690) -- cycle(79.9936,292.6290) -- (79.7270,292.6290) -- (79.7270,292.4690) -- (79.9936,292.4690) -- cycle(79.1937,292.6290) -- (78.9270,292.6290) -- (78.9270,292.4690) -- (79.1937,292.4690) -- cycle(78.3936,292.6290) -- (78.1270,292.6290) -- (78.1270,292.4690) -- (78.3936,292.4690) -- cycle(77.5936,292.6290) -- (77.3270,292.6290) -- (77.3270,292.4690) -- (77.5936,292.4690) -- cycle(76.7937,292.6290) -- (76.5270,292.6290) -- (76.5270,292.4690) -- (76.7937,292.4690) -- cycle(75.9937,292.6290) -- (75.7270,292.6290) -- (75.7270,292.4690) -- (75.9937,292.4690) -- cycle(75.1937,292.6290) -- (74.9270,292.6290) -- (74.9270,292.4690) -- (75.1937,292.4690) -- cycle(74.3936,292.6290) -- (74.1270,292.6290) -- (74.1270,292.4690) -- (74.3936,292.4690) -- cycle(73.5936,292.6290) -- (73.3270,292.6290) -- (73.3270,292.4690) -- (73.5936,292.4690) -- cycle(72.7937,292.6290) -- (72.5270,292.6290) -- (72.5270,292.4690) -- (72.7937,292.4690) -- cycle(71.9937,292.6290) -- (71.7270,292.6290) -- (71.7270,292.4690) -- (71.9937,292.4690) -- cycle(71.1937,292.6290) -- (70.9270,292.6290) -- (70.9270,292.4690) -- (71.1937,292.4690) -- cycle(70.3937,292.6290) -- (70.1270,292.6290) -- (70.1270,292.4690) -- (70.3937,292.4690) -- cycle(69.5937,292.6290) -- (69.3270,292.6290) -- (69.3270,292.4690) -- (69.5937,292.4690) -- cycle(68.7937,292.6290) -- (68.5270,292.6290) -- (68.5270,292.4690) -- (68.7937,292.4690) -- cycle(67.9937,292.6290) -- (67.7270,292.6290) -- (67.7270,292.4690) -- (67.9937,292.4690) -- cycle(67.1937,292.6290) -- (66.9270,292.6290) -- (66.9270,292.4690) -- (67.1937,292.4690) -- cycle(66.3937,292.6290) -- (66.1270,292.6290) -- (66.1270,292.4690) -- (66.3937,292.4690) -- cycle(65.5937,292.6290) -- (65.3270,292.6290) -- (65.3270,292.4690) -- (65.5937,292.4690) -- cycle(64.7937,292.6290) -- (64.5270,292.6290) -- (64.5270,292.4690) -- (64.7937,292.4690) -- cycle(63.9937,292.6290) -- (63.7270,292.6290) -- (63.7270,292.4690) -- (63.9937,292.4690) -- cycle(63.1937,292.6290) -- (62.9270,292.6290) -- (62.9270,292.4690) -- (63.1937,292.4690) -- cycle(62.3937,292.6290) -- (62.1270,292.6290) -- (62.1270,292.4690) -- (62.3937,292.4690) -- cycle(61.5937,292.6290) -- (61.3270,292.6290) -- (61.3270,292.4690) -- (61.5937,292.4690) -- cycle(60.7937,292.6290) -- (60.5270,292.6290) -- (60.5270,292.4690) -- (60.7937,292.4690) -- cycle(59.9937,292.6290) -- (59.7270,292.6290) -- (59.7270,292.4690) -- (59.9937,292.4690) -- cycle(59.1937,292.6290) -- (58.9270,292.6290) -- (58.9270,292.4690) -- (59.1937,292.4690) -- cycle(58.3937,292.6290) -- (58.1270,292.6290) -- (58.1270,292.4690) -- (58.3937,292.4690) -- cycle(57.5937,292.6290) -- (57.3270,292.6290) -- (57.3270,292.4690) -- (57.5937,292.4690) -- cycle(56.7937,292.6290) -- (56.5270,292.6290) -- (56.5270,292.4690) -- (56.7937,292.4690) -- cycle(55.9937,292.6290) -- (55.7271,292.6290) -- (55.7271,292.4690) -- (55.9937,292.4690) -- cycle(55.1937,292.6290) -- (54.9271,292.6290) -- (54.9271,292.4690) -- (55.1937,292.4690) -- cycle(54.3937,292.6290) -- (54.1270,292.6290) -- (54.1270,292.4690) -- (54.3937,292.4690) -- cycle(53.5937,292.6290) -- (53.3270,292.6290) -- (53.3270,292.4690) -- (53.5937,292.4690) -- cycle(52.7937,292.6290) -- (52.5270,292.6290) -- (52.5270,292.4690) -- (52.7937,292.4690) -- cycle(51.9937,292.6290) -- (51.7271,292.6290) -- (51.7271,292.4690) -- (51.9937,292.4690) -- cycle(51.1937,292.6290) -- (50.9271,292.6290) -- (50.9271,292.4690) -- (51.1937,292.4690) -- cycle(50.3937,292.6290) -- (50.1270,292.6290) -- (50.1270,292.4690) -- (50.3937,292.4690) -- cycle(49.5937,292.6290) -- (49.3271,292.6290) -- (49.3271,292.4690) -- (49.5937,292.4690) -- cycle(48.7937,292.6290) -- (48.5271,292.6290) -- (48.5271,292.4690) -- (48.7937,292.4690) -- cycle(47.9937,292.6290) -- (47.7271,292.6290) -- (47.7271,292.4690) -- (47.9937,292.4690) -- cycle(47.1937,292.6290) -- (46.9271,292.6290) -- (46.9271,292.4690) -- (47.1937,292.4690) -- cycle(46.3937,292.6290) -- (46.1271,292.6290) -- (46.1271,292.4690) -- (46.3937,292.4690) -- cycle(45.5937,292.6290) -- (45.3271,292.6290) -- (45.3271,292.4690) -- (45.5937,292.4690) -- cycle(44.7937,292.6290) -- (44.5271,292.6290) -- (44.5271,292.4690) -- (44.7937,292.4690) -- cycle(43.9937,292.6290) -- (43.7271,292.6290) -- (43.7271,292.4690) -- (43.9937,292.4690) -- cycle(43.1937,292.6290) -- (42.9271,292.6290) -- (42.9271,292.4690) -- (43.1937,292.4690) -- cycle(42.3937,292.6290) -- (42.1271,292.6290) -- (42.1271,292.4690) -- (42.3937,292.4690) -- cycle(41.5938,292.6290) -- (41.3271,292.6290) -- (41.3271,292.4690) -- (41.5938,292.4690) -- cycle(40.7938,292.6290) -- (40.5271,292.6290) -- (40.5271,292.4690) -- (40.7938,292.4690) -- cycle(39.9937,292.6290) -- (39.7271,292.6290) -- (39.7271,292.4690) -- (39.9937,292.4690) -- cycle(39.1937,292.6290) -- (38.9271,292.6290) -- (38.9271,292.4690) -- (39.1937,292.4690) -- cycle(38.3937,292.6290) -- (38.1271,292.6290) -- (38.1271,292.4690) -- (38.3937,292.4690) -- cycle(37.5938,292.6290) -- (37.3271,292.6290) -- (37.3271,292.4690) -- (37.5938,292.4690) -- cycle(36.7938,292.6290) -- (36.5271,292.6290) -- (36.5271,292.4690) -- (36.7938,292.4690) -- cycle(35.9937,292.6290) -- (35.7271,292.6290) -- (35.7271,292.4690) -- (35.9937,292.4690) -- cycle(35.1938,292.6290) -- (34.9271,292.6290) -- (34.9271,292.4690) -- (35.1938,292.4690) -- cycle(34.3938,292.6290) -- (34.1271,292.6290) -- (34.1271,292.4690) -- (34.3938,292.4690) -- cycle(33.5938,292.6290) -- (33.3271,292.6290) -- (33.3271,292.4690) -- (33.5938,292.4690) -- cycle(32.7938,292.6290) -- (32.5271,292.6290) -- (32.5271,292.4690) -- (32.7938,292.4690) -- cycle(31.9938,292.6290) -- (31.7271,292.6290) -- (31.7271,292.4690) -- (31.9938,292.4690) -- cycle(31.1938,292.6290) -- (30.9271,292.6290) -- (30.9271,292.4690) -- (31.1938,292.4690) -- cycle(30.3938,292.6290) -- (30.1271,292.6290) -- (30.1271,292.4690) -- (30.3938,292.4690) -- cycle(29.5938,292.6290) -- (29.3271,292.6290) -- (29.3271,292.4690) -- (29.5938,292.4690) -- cycle(28.7938,292.6290) -- (28.5271,292.6290) -- (28.5271,292.4690) -- (28.7938,292.4690) -- cycle(27.9938,292.6290) -- (27.7271,292.6290) -- (27.7271,292.4690) -- (27.9938,292.4690) -- cycle(27.1938,292.6290) -- (26.9271,292.6290) -- (26.9271,292.4690) -- (27.1938,292.4690) -- cycle(26.3938,292.6290) -- (26.1271,292.6290) -- (26.1271,292.4690) -- (26.3938,292.4690) -- cycle(25.5938,292.6290) -- (25.3271,292.6290) -- (25.3271,292.4690) -- (25.5938,292.4690) -- cycle(24.7938,292.6290) -- (24.5271,292.6290) -- (24.5271,292.4690) -- (24.7938,292.4690) -- cycle(23.9938,292.6290) -- (23.7271,292.6290) -- (23.7271,292.4690) -- (23.9938,292.4690) -- cycle(23.1938,292.6290) -- (22.9271,292.6290) -- (22.9271,292.4690) -- (23.1938,292.4690) -- cycle(22.3938,292.6290) -- (22.1271,292.6290) -- (22.1271,292.4690) -- (22.3938,292.4690) -- cycle(21.5938,292.6290) -- (21.3271,292.6290) -- (21.3271,292.4690) -- (21.5938,292.4690) -- cycle(20.7938,292.6290) -- (20.5271,292.6290) -- (20.5271,292.4690) -- (20.7938,292.4690) -- cycle(19.9938,292.6290) -- (19.7271,292.6290) -- (19.7271,292.4690) -- (19.9938,292.4690) -- cycle(19.1938,292.6290) -- (18.9271,292.6290) -- (18.9271,292.4690) -- (19.1938,292.4690) -- cycle(18.3938,292.6290) -- (18.1271,292.6290) -- (18.1271,292.4690) -- (18.3938,292.4690) -- cycle(17.5938,292.6290) -- (17.3271,292.6290) -- (17.3271,292.4690) -- (17.5938,292.4690) -- cycle(16.7938,292.6290) -- (16.5272,292.6290) -- (16.5272,292.4690) -- (16.7938,292.4690) -- cycle(15.9938,292.6290) -- (15.7271,292.6290) -- (15.7271,292.4690) -- (15.9938,292.4690) -- cycle(15.1938,292.6290) -- (14.9271,292.6290) -- (14.9271,292.4690) -- (15.1938,292.4690) -- cycle(14.3938,292.6290) -- (14.1271,292.6290) -- (14.1271,292.4690) -- (14.3938,292.4690) -- cycle(13.5938,292.6290) -- (13.3271,292.6290) -- (13.3271,292.4690) -- (13.5938,292.4690) -- cycle(12.7938,292.6290) -- (12.5272,292.6290) -- (12.5272,292.4690) -- (12.7938,292.4690) -- cycle(11.9938,292.6290) -- (11.7272,292.6290) -- (11.7272,292.4690) -- (11.9938,292.4690) -- cycle(11.1938,292.6290) -- (10.9272,292.6290) -- (10.9272,292.4690) -- (11.1938,292.4690) -- cycle(10.3938,292.6290) -- (10.1272,292.6290) -- (10.1272,292.4690) -- (10.3938,292.4690) -- cycle(9.5938,292.6290) -- (9.3271,292.6290) -- (9.3271,292.4690) -- (9.5938,292.4690) -- cycle(8.7938,292.6290) -- (8.5272,292.6290) -- (8.5272,292.4690) -- (8.7938,292.4690) -- cycle(7.9938,292.6290) -- (7.7272,292.6290) -- (7.7272,292.4690) -- (7.9938,292.4690) -- cycle(7.1938,292.6290) -- (6.9272,292.6290) -- (6.9272,292.4690) -- (7.1938,292.4690) -- cycle(6.3938,292.6290) -- (6.1272,292.6290) -- (6.1272,292.4690) -- (6.3938,292.4690) -- cycle(5.5938,292.6290) -- (5.3271,292.6290) -- (5.3271,292.4690) -- (5.5938,292.4690) -- cycle(4.7938,292.6290) -- (4.5272,292.6290) -- (4.5272,292.4690) -- (4.7938,292.4690) -- cycle(3.9938,292.6290) -- (3.9292,292.6290) -- (3.9196,292.6280) -- (3.9102,292.6270) -- (3.9011,292.6240) -- (3.8923,292.6200) -- (3.8841,292.6150) -- (3.8765,292.6090) -- (3.8697,292.6020) -- (3.8637,292.5950) -- (3.8587,292.5870) -- (3.8546,292.5780) -- (3.8517,292.5690) -- (3.8499,292.5590) -- (3.8492,292.5500) -- (3.8497,292.5400) -- (3.8513,292.5310) -- (3.8541,292.5220) -- (3.9232,292.3320) -- (4.0735,292.3860) -- (4.0044,292.5760) -- (3.9292,292.4690) -- (3.9938,292.4690) -- cycle(4.1056,291.8310) -- (4.1968,291.5800) -- (4.3471,291.6350) -- (4.2559,291.8850) -- cycle(4.3792,291.0790) -- (4.4704,290.8280) -- (4.6207,290.8830) -- (4.5295,291.1340) -- cycle(4.6528,290.3270) -- (4.7440,290.0770) -- (4.8943,290.1310) -- (4.8032,290.3820) -- cycle(4.9264,289.5750) -- (5.0176,289.3250) -- (5.1680,289.3790) -- (5.0768,289.6300) -- cycle(5.2000,288.8240) -- (5.2913,288.5730) -- (5.4416,288.6280) -- (5.3504,288.8780) -- cycle(5.4736,288.0720) -- (5.5648,287.8210) -- (5.7152,287.8760) -- (5.6240,288.1270) -- cycle(5.7472,287.3200) -- (5.8385,287.0690) -- (5.9888,287.1240) -- (5.8976,287.3750) -- cycle(6.0209,286.5680) -- (6.1121,286.3180) -- (6.2624,286.3720) -- (6.1712,286.6230) -- cycle(6.2945,285.8160) -- (6.3857,285.5660) -- (6.5360,285.6210) -- (6.4448,285.8710) -- cycle(6.5681,285.0650) -- (6.6593,284.8140) -- (6.8097,284.8690) -- (6.7184,285.1190) -- cycle(6.8417,284.3130) -- (6.9329,284.0620) -- (7.0833,284.1170) -- (6.9921,284.3680) -- cycle(7.1154,283.5610) -- (7.2065,283.3110) -- (7.3569,283.3650) -- (7.2657,283.6160) -- cycle(7.3890,282.8100) -- (7.4802,282.5590) -- (7.6305,282.6140) -- (7.5393,282.8640) -- cycle(7.6626,282.0580) -- (7.7538,281.8070) -- (7.9041,281.8620) -- (7.8129,282.1130) -- cycle(7.9362,281.3060) -- (8.0274,281.0560) -- (8.1777,281.1100) -- (8.0865,281.3610) -- cycle(8.2098,280.5540) -- (8.3010,280.3040) -- (8.4513,280.3580) -- (8.3602,280.6090) -- cycle(8.4834,279.8030) -- (8.5746,279.5520) -- (8.7250,279.6070) -- (8.6338,279.8570) -- cycle(8.7570,279.0510) -- (8.8482,278.8000) -- (8.9986,278.8550) -- (8.9074,279.1050) -- cycle(9.0306,278.2990) -- (9.1219,278.0480) -- (9.2722,278.1030) -- (9.1810,278.3540) -- cycle(9.3043,277.5470) -- (9.3955,277.2970) -- (9.5458,277.3510) -- (9.4546,277.6020) -- cycle(9.5779,276.7960) -- (9.6691,276.5450) -- (9.8194,276.6000) -- (9.7282,276.8500) -- cycle(9.8515,276.0440) -- (9.9427,275.7930) -- (10.0930,275.8480) -- (10.0018,276.0980) -- cycle(10.1251,275.2920) -- (10.2163,275.0410) -- (10.3667,275.0960) -- (10.2755,275.3470) -- cycle(10.3987,274.5400) -- (10.4899,274.2900) -- (10.6403,274.3440) -- (10.5491,274.5950) -- cycle(10.6724,273.7890) -- (10.7635,273.5380) -- (10.9139,273.5930) -- (10.8227,273.8430) -- cycle(10.9460,273.0370) -- (11.0372,272.7860) -- (11.1875,272.8410) -- (11.0963,273.0910) -- cycle(11.2196,272.2850) -- (11.3108,272.0340) -- (11.4611,272.0890) -- (11.3699,272.3400) -- cycle(11.4932,271.5330) -- (11.5844,271.2830) -- (11.7347,271.3370) -- (11.6436,271.5880) -- cycle(11.7668,270.7810) -- (11.8580,270.5310) -- (12.0084,270.5860) -- (11.9172,270.8360) -- cycle(12.0404,270.0300) -- (12.1316,269.7790) -- (12.2820,269.8340) -- (12.1908,270.0840) -- cycle(12.3140,269.2780) -- (12.4052,269.0270) -- (12.5556,269.0820) -- (12.4644,269.3330) -- cycle(12.5876,268.5260) -- (12.6789,268.2760) -- (12.8292,268.3300) -- (12.7380,268.5810) -- cycle(12.8613,267.7750) -- (12.9525,267.5240) -- (13.1028,267.5790) -- (13.0116,267.8290) -- cycle(13.1349,267.0230) -- (13.2261,266.7720) -- (13.3764,266.8270) -- (13.2852,267.0770) -- cycle(13.4085,266.2710) -- (13.4997,266.0200) -- (13.6501,266.0750) -- (13.5588,266.3260) -- cycle(13.6821,265.5190) -- (13.7733,265.2690) -- (13.9237,265.3230) -- (13.8325,265.5740) -- cycle(13.9557,264.7680) -- (14.0469,264.5170) -- (14.1973,264.5720) -- (14.1061,264.8220) -- cycle(14.2293,264.0160) -- (14.3206,263.7650) -- (14.4709,263.8200) -- (14.3797,264.0700) -- cycle(14.5030,263.2640) -- (14.5942,263.0130) -- (14.7445,263.0680) -- (14.6533,263.3190) -- cycle(14.7766,262.5120) -- (14.8678,262.2620) -- (15.0181,262.3160) -- (14.9269,262.5670) -- cycle(15.0502,261.7600) -- (15.1414,261.5100) -- (15.2917,261.5650) -- (15.2005,261.8150) -- cycle(15.3238,261.0090) -- (15.4150,260.7580) -- (15.5654,260.8130) -- (15.4742,261.0630) -- cycle(15.5974,260.2570) -- (15.6886,260.0060) -- (15.8390,260.0610) -- (15.7478,260.3120) -- cycle(15.8710,259.5050) -- (15.9622,259.2550) -- (16.1126,259.3090) -- (16.0214,259.5600) -- cycle(16.1447,258.7540) -- (16.2358,258.5030) -- (16.3862,258.5580) -- (16.2950,258.8080) -- cycle(16.4183,258.0020) -- (16.5095,257.7510) -- (16.6598,257.8060) -- (16.5686,258.0560) -- cycle(16.6919,257.2500) -- (16.7831,256.9990) -- (16.9334,257.0540) -- (16.8422,257.3050) -- cycle(16.9655,256.4980) -- (17.0567,256.2480) -- (17.2071,256.3020) -- (17.1158,256.5530) -- cycle(17.2391,255.7460) -- (17.3303,255.4960) -- (17.4807,255.5510) -- (17.3895,255.8010) -- cycle(17.5127,254.9950) -- (17.6039,254.7440) -- (17.7543,254.7990) -- (17.6631,255.0490) -- cycle(17.7863,254.2430) -- (17.8776,253.9920) -- (18.0279,254.0470) -- (17.9367,254.2980) -- cycle(18.0600,253.4910) -- (18.1512,253.2410) -- (18.3015,253.2950) -- (18.2103,253.5460) -- cycle(18.3336,252.7390) -- (18.4248,252.4890) -- (18.5751,252.5440) -- (18.4839,252.7940) -- cycle(18.6072,251.9880) -- (18.6984,251.7370) -- (18.8488,251.7920) -- (18.7575,252.0420) -- cycle(18.8808,251.2360) -- (18.9720,250.9850) -- (19.1224,251.0400) -- (19.0312,251.2910) -- cycle(19.1544,250.4840) -- (19.2456,250.2340) -- (19.3960,250.2880) -- (19.3048,250.5390) -- cycle(19.4280,249.7320) -- (19.5192,249.4820) -- (19.6696,249.5370) -- (19.5784,249.7870) -- cycle(19.7017,248.9810) -- (19.7928,248.7300) -- (19.9432,248.7850) -- (19.8520,249.0350) -- cycle(19.9753,248.2290) -- (20.0665,247.9780) -- (20.2168,248.0330) -- (20.1256,248.2840) -- cycle(20.2489,247.4770) -- (20.3401,247.2270) -- (20.4904,247.2810) -- (20.3992,247.5320) -- cycle(20.5225,246.7250) -- (20.6137,246.4750) -- (20.7640,246.5300) -- (20.6729,246.7800) -- cycle(20.7961,245.9740) -- (20.8873,245.7230) -- (21.0377,245.7780) -- (20.9465,246.0280) -- cycle(21.0697,245.2220) -- (21.1609,244.9710) -- (21.3113,245.0260) -- (21.2201,245.2770) -- cycle(21.3434,244.4700) -- (21.4346,244.2200) -- (21.5849,244.2740) -- (21.4937,244.5250) -- cycle(21.6169,243.7180) -- (21.7082,243.4680) -- (21.8585,243.5230) -- (21.7673,243.7730) -- cycle(21.8906,242.9670) -- (21.9818,242.7160) -- (22.1321,242.7710) -- (22.0409,243.0210) -- cycle(22.1642,242.2150) -- (22.2554,241.9640) -- (22.4058,242.0190) -- (22.3145,242.2700) -- cycle(22.4378,241.4630) -- (22.5290,241.2130) -- (22.6794,241.2670) -- (22.5882,241.5180) -- cycle(22.7114,240.7110) -- (22.8026,240.4610) -- (22.9530,240.5160) -- (22.8618,240.7660) -- cycle(22.9850,239.9600) -- (23.0762,239.7090) -- (23.2266,239.7640) -- (23.1354,240.0140) -- cycle(23.2587,239.2080) -- (23.3499,238.9570) -- (23.5002,239.0120) -- (23.4090,239.2630) -- cycle(23.5323,238.4560) -- (23.6235,238.2060) -- (23.7738,238.2600) -- (23.6826,238.5110) -- cycle(23.8059,237.7040) -- (23.8971,237.4540) -- (24.0474,237.5090) -- (23.9562,237.7590) -- cycle(24.0795,236.9530) -- (24.1707,236.7020) -- (24.3210,236.7570) -- (24.2299,237.0070) -- cycle(24.3531,236.2010) -- (24.4443,235.9500) -- (24.5947,236.0050) -- (24.5035,236.2560) -- cycle(24.6267,235.4490) -- (24.7179,235.1990) -- (24.8683,235.2530) -- (24.7771,235.5040) -- cycle(24.9003,234.6970) -- (24.9915,234.4470) -- (25.1419,234.5010) -- (25.0507,234.7520) -- cycle(25.1740,233.9460) -- (25.2652,233.6950) -- (25.4155,233.7500) -- (25.3243,234.0000) -- cycle(25.4476,233.1940) -- (25.5388,232.9430) -- (25.6891,232.9980) -- (25.5979,233.2490) -- cycle(25.7212,232.4420) -- (25.8124,232.1920) -- (25.9627,232.2460) -- (25.8716,232.4970) -- cycle(25.9948,231.6900) -- (26.0860,231.4400) -- (26.2364,231.4950) -- (26.1451,231.7450) -- cycle(26.2684,230.9390) -- (26.3596,230.6880) -- (26.5100,230.7430) -- (26.4188,230.9930) -- cycle(26.5421,230.1870) -- (26.6332,229.9360) -- (26.7836,229.9910) -- (26.6924,230.2420) -- cycle(26.8157,229.4350) -- (26.9069,229.1850) -- (27.0572,229.2390) -- (26.9660,229.4900) -- cycle(27.0893,228.6830) -- (27.1805,228.4330) -- (27.3308,228.4870) -- (27.2396,228.7380) -- cycle(27.3629,227.9320) -- (27.4541,227.6810) -- (27.6044,227.7360) -- (27.5132,227.9860) -- cycle(27.6365,227.1800) -- (27.7277,226.9290) -- (27.8781,226.9840) -- (27.7869,227.2350) -- cycle(27.9101,226.4280) -- (28.0013,226.1780) -- (28.1517,226.2320) -- (28.0605,226.4830) -- cycle(28.1837,225.6760) -- (28.2749,225.4260) -- (28.4253,225.4800) -- (28.3341,225.7310) -- cycle(28.4573,224.9250) -- (28.5486,224.6740) -- (28.6989,224.7290) -- (28.6077,224.9790) -- cycle(28.7310,224.1730) -- (28.8222,223.9220) -- (28.9725,223.9770) -- (28.8813,224.2280) -- cycle(29.0046,223.4210) -- (29.0958,223.1710) -- (29.2461,223.2250) -- (29.1549,223.4760) -- cycle(29.2782,222.6690) -- (29.3694,222.4190) -- (29.5197,222.4730) -- (29.4285,222.7240) -- cycle(29.5518,221.9180) -- (29.6430,221.6670) -- (29.7934,221.7220) -- (29.7021,221.9720) -- cycle(29.8254,221.1660) -- (29.9166,220.9150) -- (30.0670,220.9700) -- (29.9758,221.2210) -- cycle(30.0991,220.4140) -- (30.1902,220.1640) -- (30.3406,220.2180) -- (30.2494,220.4690) -- cycle(30.3727,219.6620) -- (30.4639,219.4120) -- (30.6142,219.4660) -- (30.5230,219.7170) -- cycle(30.6463,218.9110) -- (30.7375,218.6600) -- (30.8878,218.7150) -- (30.7966,218.9650) -- cycle(30.9199,218.1590) -- (31.0111,217.9080) -- (31.1614,217.9630) -- (31.0703,218.2140) -- cycle(31.1935,217.4070) -- (31.2847,217.1570) -- (31.4351,217.2110) -- (31.3439,217.4620) -- cycle(31.4671,216.6550) -- (31.5583,216.4050) -- (31.7087,216.4600) -- (31.6175,216.7100) -- cycle(31.7407,215.9040) -- (31.8319,215.6530) -- (31.9823,215.7080) -- (31.8911,215.9580) -- cycle(32.0143,215.1520) -- (32.1056,214.9010) -- (32.2559,214.9560) -- (32.1647,215.2070) -- cycle(32.2880,214.4000) -- (32.3792,214.1500) -- (32.5295,214.2040) -- (32.4383,214.4550) -- cycle(32.5616,213.6480) -- (32.6528,213.3980) -- (32.8031,213.4520) -- (32.7119,213.7030) -- cycle(32.8352,212.8970) -- (32.9264,212.6460) -- (33.0768,212.7010) -- (32.9855,212.9510) -- cycle(33.1088,212.1450) -- (33.2000,211.8940) -- (33.3504,211.9490) -- (33.2592,212.1990) -- cycle(33.3824,211.3930) -- (33.4736,211.1420) -- (33.6240,211.1970) -- (33.5328,211.4480) -- cycle(33.6560,210.6410) -- (33.7473,210.3910) -- (33.8976,210.4450) -- (33.8064,210.6960) -- cycle(33.9297,209.8900) -- (34.0209,209.6390) -- (34.1712,209.6940) -- (34.0800,209.9440) -- cycle(34.2033,209.1380) -- (34.2945,208.8870) -- (34.4448,208.9420) -- (34.3536,209.1930) -- cycle(34.4769,208.3860) -- (34.5681,208.1360) -- (34.7184,208.1900) -- (34.6272,208.4410) -- cycle(34.7505,207.6340) -- (34.8417,207.3840) -- (34.9921,207.4380) -- (34.9008,207.6890) -- cycle(35.0241,206.8830) -- (35.1153,206.6320) -- (35.2657,206.6870) -- (35.1745,206.9370) -- cycle(35.2977,206.1310) -- (35.3889,205.8800) -- (35.5393,205.9350) -- (35.4481,206.1860) -- cycle(35.5714,205.3790) -- (35.6625,205.1290) -- (35.8129,205.1830) -- (35.7217,205.4340) -- cycle(35.8450,204.6270) -- (35.9362,204.3770) -- (36.0865,204.4310) -- (35.9953,204.6820) -- cycle(36.1186,203.8760) -- (36.2098,203.6250) -- (36.3601,203.6800) -- (36.2689,203.9300) -- cycle(36.3922,203.1240) -- (36.4834,202.8730) -- (36.6338,202.9280) -- (36.5425,203.1780) -- cycle(36.6658,202.3720) -- (36.7570,202.1210) -- (36.9073,202.1760) -- (36.8162,202.4270) -- cycle(36.9394,201.6200) -- (37.0306,201.3700) -- (37.1810,201.4240) -- (37.0898,201.6750) -- cycle(37.2130,200.8690) -- (37.3043,200.6180) -- (37.4546,200.6730) -- (37.3634,200.9230) -- cycle(37.4867,200.1170) -- (37.5779,199.8660) -- (37.7282,199.9210) -- (37.6370,200.1710) -- cycle(37.7603,199.3650) -- (37.8515,199.1140) -- (38.0018,199.1690) -- (37.9106,199.4200) -- cycle(38.0339,198.6130) -- (38.1251,198.3630) -- (38.2755,198.4170) -- (38.1842,198.6680) -- cycle(38.3075,197.8620) -- (38.3987,197.6110) -- (38.5491,197.6660) -- (38.4579,197.9160) -- cycle(38.5811,197.1100) -- (38.6723,196.8590) -- (38.8227,196.9140) -- (38.7315,197.1640) -- cycle(38.8547,196.3580) -- (38.9459,196.1070) -- (39.0963,196.1620) -- (39.0051,196.4130) -- cycle(39.1284,195.6060) -- (39.2195,195.3560) -- (39.3699,195.4100) -- (39.2787,195.6610) -- cycle(39.4020,194.8550) -- (39.4932,194.6040) -- (39.6435,194.6590) -- (39.5523,194.9090) -- cycle(39.6756,194.1030) -- (39.7668,193.8520) -- (39.9171,193.9070) -- (39.8259,194.1580) -- cycle(39.9492,193.3510) -- (40.0404,193.1000) -- (40.1907,193.1550) -- (40.0995,193.4060) -- cycle(40.2228,192.5990) -- (40.3140,192.3490) -- (40.4644,192.4030) -- (40.3732,192.6540) -- cycle(40.4964,191.8480) -- (40.5876,191.5970) -- (40.7380,191.6520) -- (40.6468,191.9020) -- cycle(40.7701,191.0960) -- (40.8612,190.8450) -- (41.0116,190.9000) -- (40.9204,191.1500) -- cycle(41.0436,190.3440) -- (41.1349,190.0930) -- (41.2852,190.1480) -- (41.1940,190.3990) -- cycle(41.3173,189.5920) -- (41.4085,189.3420) -- (41.5588,189.3960) -- (41.4676,189.6470) -- cycle(41.5909,188.8410) -- (41.6821,188.5900) -- (41.8325,188.6450) -- (41.7412,188.8950) -- cycle(41.8645,188.0890) -- (41.9557,187.8380) -- (42.1060,187.8930) -- (42.0149,188.1430) -- cycle(42.1381,187.3370) -- (42.2293,187.0860) -- (42.3797,187.1410) -- (42.2885,187.3920) -- cycle(42.4117,186.5850) -- (42.5029,186.3350) -- (42.6533,186.3890) -- (42.5621,186.6400) -- cycle(42.6854,185.8340) -- (42.7766,185.5830) -- (42.9269,185.6380) -- (42.8357,185.8880) -- cycle(42.9590,185.0820) -- (43.0502,184.8310) -- (43.2005,184.8860) -- (43.1093,185.1360) -- cycle(43.2326,184.3300) -- (43.3238,184.0790) -- (43.4741,184.1340) -- (43.3829,184.3850) -- cycle(43.5062,183.5780) -- (43.5974,183.3280) -- (43.7477,183.3820) -- (43.6566,183.6330) -- cycle(43.7798,182.8270) -- (43.8710,182.5760) -- (44.0214,182.6310) -- (43.9302,182.8810) -- cycle(44.0534,182.0750) -- (44.1446,181.8240) -- (44.2950,181.8790) -- (44.2038,182.1290) -- cycle(44.3270,181.3230) -- (44.4182,181.0720) -- (44.5686,181.1270) -- (44.4774,181.3780) -- cycle(44.6006,180.5710) -- (44.6919,180.3210) -- (44.8422,180.3750) -- (44.7510,180.6260) -- cycle(44.8743,179.8190) -- (44.9655,179.5690) -- (45.1158,179.6240) -- (45.0246,179.8740) -- cycle(45.1479,179.0680) -- (45.2391,178.8170) -- (45.3894,178.8720) -- (45.2982,179.1220) -- cycle(45.4215,178.3160) -- (45.5127,178.0650) -- (45.6631,178.1200) -- (45.5718,178.3710) -- cycle(45.6951,177.5640) -- (45.7863,177.3140) -- (45.9367,177.3680) -- (45.8455,177.6190) -- cycle(45.9688,176.8130) -- (46.0599,176.5620) -- (46.2103,176.6170) -- (46.1191,176.8670) -- cycle(46.2424,176.0610) -- (46.3336,175.8100) -- (46.4839,175.8650) -- (46.3927,176.1150) -- cycle(46.5160,175.3090) -- (46.6072,175.0580) -- (46.7575,175.1130) -- (46.6663,175.3640) -- cycle(46.7896,174.5570) -- (46.8808,174.3070) -- (47.0311,174.3610) -- (46.9399,174.6120) -- cycle(47.0632,173.8050) -- (47.1544,173.5550) -- (47.3047,173.6100) -- (47.2136,173.8600) -- cycle(47.3368,173.0540) -- (47.4280,172.8030) -- (47.5784,172.8580) -- (47.4872,173.1080) -- cycle(47.6104,172.3020) -- (47.7016,172.0510) -- (47.8520,172.1060) -- (47.7608,172.3570) -- cycle(47.8840,171.5500) -- (47.9753,171.3000) -- (48.1256,171.3540) -- (48.0344,171.6050) -- cycle(48.1577,170.7980) -- (48.2489,170.5480) -- (48.3992,170.6030) -- (48.3080,170.8530) -- cycle(48.4313,170.0470) -- (48.5225,169.7960) -- (48.6728,169.8510) -- (48.5816,170.1010) -- cycle(48.7049,169.2950) -- (48.7961,169.0440) -- (48.9464,169.0990) -- (48.8552,169.3500) -- cycle(48.9785,168.5430) -- (49.0697,168.2930) -- (49.2201,168.3470) -- (49.1288,168.5980) -- cycle(49.2521,167.7910) -- (49.3433,167.5410) -- (49.4937,167.5960) -- (49.4025,167.8460) -- cycle(49.5257,167.0400) -- (49.6169,166.7890) -- (49.7673,166.8440) -- (49.6761,167.0940) -- cycle(49.7993,166.2880) -- (49.8906,166.0370) -- (50.0409,166.0920) -- (49.9497,166.3430) -- cycle(50.0730,165.5360) -- (50.1642,165.2860) -- (50.3145,165.3400) -- (50.2233,165.5910) -- cycle(50.3466,164.7840) -- (50.4378,164.5340) -- (50.5881,164.5890) -- (50.4969,164.8390) -- cycle(50.6202,164.0330) -- (50.7114,163.7820) -- (50.8618,163.8370) -- (50.7705,164.0870) -- cycle(50.8938,163.2810) -- (50.9850,163.0300) -- (51.1354,163.0850) -- (51.0442,163.3360) -- cycle(51.1674,162.5290) -- (51.2586,162.2790) -- (51.4090,162.3330) -- (51.3178,162.5840) -- cycle(51.4410,161.7770) -- (51.5323,161.5270) -- (51.6826,161.5820) -- (51.5914,161.8320) -- cycle(51.7147,161.0260) -- (51.8058,160.7750) -- (51.9562,160.8300) -- (51.8650,161.0800) -- cycle(51.9883,160.2740) -- (52.0795,160.0230) -- (52.2298,160.0780) -- (52.1386,160.3290) -- cycle(52.2619,159.5220) -- (52.3531,159.2720) -- (52.5034,159.3260) -- (52.4122,159.5770) -- cycle(52.5355,158.7700) -- (52.6267,158.5200) -- (52.7771,158.5750) -- (52.6859,158.8250) -- cycle(52.8091,158.0190) -- (52.9003,157.7680) -- (53.0507,157.8230) -- (52.9595,158.0730) -- cycle(53.0827,157.2670) -- (53.1740,157.0160) -- (53.3243,157.0710) -- (53.2331,157.3220) -- cycle(53.3564,156.5150) -- (53.4476,156.2650) -- (53.5979,156.3190) -- (53.5067,156.5700) -- cycle(53.6300,155.7630) -- (53.7212,155.5130) -- (53.8715,155.5680) -- (53.7803,155.8180) -- cycle(53.9036,155.0120) -- (53.9948,154.7610) -- (54.1451,154.8160) -- (54.0539,155.0660) -- cycle(54.2954,154.2690) -- (54.5621,154.2690) -- (54.5621,154.4290) -- (54.2954,154.4290) -- cycle(55.0954,154.2690) -- (55.3621,154.2690) -- (55.3621,154.4290) -- (55.0954,154.4290) -- cycle(55.8954,154.2690) -- (56.1621,154.2690) -- (56.1621,154.4290) -- (55.8954,154.4290) -- cycle(56.6954,154.2690) -- (56.9621,154.2690) -- (56.9621,154.4290) -- (56.6954,154.4290) -- cycle(57.4954,154.2690) -- (57.7621,154.2690) -- (57.7621,154.4290) -- (57.4954,154.4290) -- cycle(58.2954,154.2690) -- (58.5621,154.2690) -- (58.5621,154.4290) -- (58.2954,154.4290) -- cycle(59.0954,154.2690) -- (59.3621,154.2690) -- (59.3621,154.4290) -- (59.0954,154.4290) -- cycle(59.8954,154.2690) -- (60.1621,154.2690) -- (60.1621,154.4290) -- (59.8954,154.4290) -- cycle(60.6954,154.2690) -- (60.9621,154.2690) -- (60.9621,154.4290) -- (60.6954,154.4290) -- cycle(61.4954,154.2690) -- (61.7621,154.2690) -- (61.7621,154.4290) -- (61.4954,154.4290) -- cycle(62.2954,154.2690) -- (62.5621,154.2690) -- (62.5621,154.4290) -- (62.2954,154.4290) -- cycle(63.0954,154.2690) -- (63.3621,154.2690) -- (63.3621,154.4290) -- (63.0954,154.4290) -- cycle(63.8954,154.2690) -- (64.1621,154.2690) -- (64.1621,154.4290) -- (63.8954,154.4290) -- cycle(64.6954,154.2690) -- (64.9621,154.2690) -- (64.9621,154.4290) -- (64.6954,154.4290) -- cycle(65.4954,154.2690) -- (65.7621,154.2690) -- (65.7621,154.4290) -- (65.4954,154.4290) -- cycle(66.2954,154.2690) -- (66.5621,154.2690) -- (66.5621,154.4290) -- (66.2954,154.4290) -- cycle(67.0954,154.2690) -- (67.3621,154.2690) -- (67.3621,154.4290) -- (67.0954,154.4290) -- cycle(67.8954,154.2690) -- (68.1621,154.2690) -- (68.1621,154.4290) -- (67.8954,154.4290) -- cycle(68.6954,154.2690) -- (68.9620,154.2690) -- (68.9620,154.4290) -- (68.6954,154.4290) -- cycle(69.4954,154.2690) -- (69.7621,154.2690) -- (69.7621,154.4290) -- (69.4954,154.4290) -- cycle(70.2954,154.2690) -- (70.5621,154.2690) -- (70.5621,154.4290) -- (70.2954,154.4290) -- cycle(71.0954,154.2690) -- (71.3621,154.2690) -- (71.3621,154.4290) -- (71.0954,154.4290) -- cycle(71.8954,154.2690) -- (72.1620,154.2690) -- (72.1620,154.4290) -- (71.8954,154.4290) -- cycle(72.6954,154.2690) -- (72.9620,154.2690) -- (72.9620,154.4290) -- (72.6954,154.4290) -- cycle(73.4954,154.2690) -- (73.7621,154.2690) -- (73.7621,154.4290) -- (73.4954,154.4290) -- cycle(74.2954,154.2690) -- (74.5621,154.2690) -- (74.5621,154.4290) -- (74.2954,154.4290) -- cycle(75.0954,154.2690) -- (75.3620,154.2690) -- (75.3620,154.4290) -- (75.0954,154.4290) -- cycle(75.8954,154.2690) -- (76.1620,154.2690) -- (76.1620,154.4290) -- (75.8954,154.4290) -- cycle(76.6954,154.2690) -- (76.9620,154.2690) -- (76.9620,154.4290) -- (76.6954,154.4290) -- cycle(77.4954,154.2690) -- (77.7621,154.2690) -- (77.7621,154.4290) -- (77.4954,154.4290) -- cycle(78.2954,154.2690) -- (78.5620,154.2690) -- (78.5620,154.4290) -- (78.2954,154.4290) -- cycle(79.0954,154.2690) -- (79.3620,154.2690) -- (79.3620,154.4290) -- (79.0954,154.4290) -- cycle(79.8954,154.2690) -- (80.1620,154.2690) -- (80.1620,154.4290) -- (79.8954,154.4290) -- cycle(80.6954,154.2690) -- (80.9620,154.2690) -- (80.9620,154.4290) -- (80.6954,154.4290) -- cycle(81.4954,154.2690) -- (81.7620,154.2690) -- (81.7620,154.4290) -- (81.4954,154.4290) -- cycle(82.2954,154.2690) -- (82.5620,154.2690) -- (82.5620,154.4290) -- (82.2954,154.4290) -- cycle(83.0953,154.2690) -- (83.3620,154.2690) -- (83.3620,154.4290) -- (83.0953,154.4290) -- cycle(83.8954,154.2690) -- (84.1620,154.2690) -- (84.1620,154.4290) -- (83.8954,154.4290) -- cycle(84.6954,154.2690) -- (84.9620,154.2690) -- (84.9620,154.4290) -- (84.6954,154.4290) -- cycle(85.4954,154.2690) -- (85.7620,154.2690) -- (85.7620,154.4290) -- (85.4954,154.4290) -- cycle(86.2953,154.2690) -- (86.5620,154.2690) -- (86.5620,154.4290) -- (86.2953,154.4290) -- cycle(87.0953,154.2690) -- (87.3620,154.2690) -- (87.3620,154.4290) -- (87.0953,154.4290) -- cycle(87.8954,154.2690) -- (88.1620,154.2690) -- (88.1620,154.4290) -- (87.8954,154.4290) -- cycle(88.6954,154.2690) -- (88.9620,154.2690) -- (88.9620,154.4290) -- (88.6954,154.4290) -- cycle(89.4953,154.2690) -- (89.7620,154.2690) -- (89.7620,154.4290) -- (89.4953,154.4290) -- cycle(90.2953,154.2690) -- (90.5620,154.2690) -- (90.5620,154.4290) -- (90.2953,154.4290) -- cycle(91.0953,154.2690) -- (91.3620,154.2690) -- (91.3620,154.4290) -- (91.0953,154.4290) -- cycle(91.8954,154.2690) -- (92.1620,154.2690) -- (92.1620,154.4290) -- (91.8954,154.4290) -- cycle(92.6953,154.2690) -- (92.9620,154.2690) -- (92.9620,154.4290) -- (92.6953,154.4290) -- cycle(93.4953,154.2690) -- (93.7620,154.2690) -- (93.7620,154.4290) -- (93.4953,154.4290) -- cycle(94.2953,154.2690) -- (94.5620,154.2690) -- (94.5620,154.4290) -- (94.2953,154.4290) -- cycle(95.0953,154.2690) -- (95.3620,154.2690) -- (95.3620,154.4290) -- (95.0953,154.4290) -- cycle(95.8953,154.2690) -- (96.1620,154.2690) -- (96.1620,154.4290) -- (95.8953,154.4290) -- cycle(96.6953,154.2690) -- (96.9620,154.2690) -- (96.9620,154.4290) -- (96.6953,154.4290) -- cycle(97.4953,154.2690) -- (97.7620,154.2690) -- (97.7620,154.4290) -- (97.4953,154.4290) -- cycle(98.2953,154.2690) -- (98.5620,154.2690) -- (98.5620,154.4290) -- (98.2953,154.4290) -- cycle(99.0953,154.2690) -- (99.3620,154.2690) -- (99.3620,154.4290) -- (99.0953,154.4290) -- cycle(99.8953,154.2690) -- (100.1620,154.2690) -- (100.1620,154.4290) -- (99.8953,154.4290) -- cycle(100.6950,154.2690) -- (100.9620,154.2690) -- (100.9620,154.4290) -- (100.6950,154.4290) -- cycle(101.4950,154.2690) -- (101.7620,154.2690) -- (101.7620,154.4290) -- (101.4950,154.4290) -- cycle(102.2950,154.2690) -- (102.5620,154.2690) -- (102.5620,154.4290) -- (102.2950,154.4290) -- cycle(103.0950,154.2690) -- (103.3620,154.2690) -- (103.3620,154.4290) -- (103.0950,154.4290) -- cycle(103.8950,154.2690) -- (104.1620,154.2690) -- (104.1620,154.4290) -- (103.8950,154.4290) -- cycle(104.6950,154.2690) -- (104.9620,154.2690) -- (104.9620,154.4290) -- (104.6950,154.4290) -- cycle(105.4950,154.2690) -- (105.7620,154.2690) -- (105.7620,154.4290) -- (105.4950,154.4290) -- cycle(106.2950,154.2690) -- (106.5620,154.2690) -- (106.5620,154.4290) -- (106.2950,154.4290) -- cycle(107.0950,154.2690) -- (107.3620,154.2690) -- (107.3620,154.4290) -- (107.0950,154.4290) -- cycle(107.8950,154.2690) -- (108.1620,154.2690) -- (108.1620,154.4290) -- (107.8950,154.4290) -- cycle(108.6950,154.2690) -- (108.9620,154.2690) -- (108.9620,154.4290) -- (108.6950,154.4290) -- cycle(109.4950,154.2690) -- (109.7620,154.2690) -- (109.7620,154.4290) -- (109.4950,154.4290) -- cycle(110.2950,154.2690) -- (110.5620,154.2690) -- (110.5620,154.4290) -- (110.2950,154.4290) -- cycle(111.0950,154.2690) -- (111.3620,154.2690) -- (111.3620,154.4290) -- (111.0950,154.4290) -- cycle(111.8950,154.2690) -- (112.1620,154.2690) -- (112.1620,154.4290) -- (111.8950,154.4290) -- cycle(112.6950,154.2690) -- (112.9620,154.2690) -- (112.9620,154.4290) -- (112.6950,154.4290) -- cycle(113.4950,154.2690) -- (113.7620,154.2690) -- (113.7620,154.4290) -- (113.4950,154.4290) -- cycle(114.2950,154.2690) -- (114.5620,154.2690) -- (114.5620,154.4290) -- (114.2950,154.4290) -- cycle(115.0950,154.2690) -- (115.3620,154.2690) -- (115.3620,154.4290) -- (115.0950,154.4290) -- cycle(115.8950,154.2690) -- (116.1620,154.2690) -- (116.1620,154.4290) -- (115.8950,154.4290) -- cycle(116.6950,154.2690) -- (116.9620,154.2690) -- (116.9620,154.4290) -- (116.6950,154.4290) -- cycle(117.4950,154.2690) -- (117.7620,154.2690) -- (117.7620,154.4290) -- (117.4950,154.4290) -- cycle(118.2950,154.2690) -- (118.5620,154.2690) -- (118.5620,154.4290) -- (118.2950,154.4290) -- cycle(119.0950,154.2690) -- (119.3620,154.2690) -- (119.3620,154.4290) -- (119.0950,154.4290) -- cycle(119.8950,154.2690) -- (120.1620,154.2690) -- (120.1620,154.4290) -- (119.8950,154.4290) -- cycle(120.6950,154.2690) -- (120.9620,154.2690) -- (120.9620,154.4290) -- (120.6950,154.4290) -- cycle(121.4950,154.2690) -- (121.7620,154.2690) -- (121.7620,154.4290) -- (121.4950,154.4290) -- cycle(122.2950,154.2690) -- (122.5620,154.2690) -- (122.5620,154.4290) -- (122.2950,154.4290) -- cycle(123.0950,154.2690) -- (123.3620,154.2690) -- (123.3620,154.4290) -- (123.0950,154.4290) -- cycle(123.8950,154.2690) -- (124.1620,154.2690) -- (124.1620,154.4290) -- (123.8950,154.4290) -- cycle(124.6950,154.2690) -- (124.9620,154.2690) -- (124.9620,154.4290) -- (124.6950,154.4290) -- cycle(125.4950,154.2690) -- (125.7620,154.2690) -- (125.7620,154.4290) -- (125.4950,154.4290) -- cycle(126.2950,154.2690) -- (126.5620,154.2690) -- (126.5620,154.4290) -- (126.2950,154.4290) -- cycle(127.0950,154.2690) -- (127.3620,154.2690) -- (127.3620,154.4290) -- (127.0950,154.4290) -- cycle(127.8950,154.2690) -- (128.1620,154.2690) -- (128.1620,154.4290) -- (127.8950,154.4290) -- cycle(128.6950,154.2690) -- (128.9620,154.2690) -- (128.9620,154.4290) -- (128.6950,154.4290) -- cycle(129.4950,154.2690) -- (129.7620,154.2690) -- (129.7620,154.4290) -- (129.4950,154.4290) -- cycle(130.2950,154.2690) -- (130.5620,154.2690) -- (130.5620,154.4290) -- (130.2950,154.4290) -- cycle(131.0950,154.2690) -- (131.3620,154.2690) -- (131.3620,154.4290) -- (131.0950,154.4290) -- cycle(131.8950,154.2690) -- (132.1620,154.2690) -- (132.1620,154.4290) -- (131.8950,154.4290) -- cycle(132.6950,154.2690) -- (132.9620,154.2690) -- (132.9620,154.4290) -- (132.6950,154.4290) -- cycle(133.4950,154.2690) -- (133.7620,154.2690) -- (133.7620,154.4290) -- (133.4950,154.4290) -- cycle(134.2950,154.2690) -- (134.5620,154.2690) -- (134.5620,154.4290) -- (134.2950,154.4290) -- cycle(135.0950,154.2690) -- (135.3620,154.2690) -- (135.3620,154.4290) -- (135.0950,154.4290) -- cycle(135.8950,154.2690) -- (136.1620,154.2690) -- (136.1620,154.4290) -- (135.8950,154.4290) -- cycle(136.6950,154.2690) -- (136.9620,154.2690) -- (136.9620,154.4290) -- (136.6950,154.4290) -- cycle(137.4950,154.2690) -- (137.7620,154.2690) -- (137.7620,154.4290) -- (137.4950,154.4290) -- cycle(138.2950,154.2690) -- (138.5620,154.2690) -- (138.5620,154.4290) -- (138.2950,154.4290) -- cycle(139.0950,154.2690) -- (139.3620,154.2690) -- (139.3620,154.4290) -- (139.0950,154.4290) -- cycle(139.8950,154.2690) -- (140.1620,154.2690) -- (140.1620,154.4290) -- (139.8950,154.4290) -- cycle(140.6950,154.2690) -- (140.9620,154.2690) -- (140.9620,154.4290) -- (140.6950,154.4290) -- cycle(141.4950,154.2690) -- (141.7620,154.2690) -- (141.7620,154.4290) -- (141.4950,154.4290) -- cycle(142.2950,154.2690) -- (142.5620,154.2690) -- (142.5620,154.4290) -- (142.2950,154.4290) -- cycle(143.0950,154.2690) -- (143.3620,154.2690) -- (143.3620,154.4290) -- (143.0950,154.4290) -- cycle(143.8950,154.2690) -- (144.1620,154.2690) -- (144.1620,154.4290) -- (143.8950,154.4290) -- cycle(144.6950,154.2690) -- (144.9620,154.2690) -- (144.9620,154.4290) -- (144.6950,154.4290) -- cycle(145.4950,154.2690) -- (145.7620,154.2690) -- (145.7620,154.4290) -- (145.4950,154.4290) -- cycle(146.2950,154.2690) -- (146.5620,154.2690) -- (146.5620,154.4290) -- (146.2950,154.4290) -- cycle(147.0950,154.2690) -- (147.3620,154.2690) -- (147.3620,154.4290) -- (147.0950,154.4290) -- cycle(147.8950,154.2690) -- (148.1620,154.2690) -- (148.1620,154.4290) -- (147.8950,154.4290) -- cycle(148.6950,154.2690) -- (148.9620,154.2690) -- (148.9620,154.4290) -- (148.6950,154.4290) -- cycle(149.4950,154.2690) -- (149.7620,154.2690) -- (149.7620,154.4290) -- (149.4950,154.4290) -- cycle(150.2950,154.2690) -- (150.5620,154.2690) -- (150.5620,154.4290) -- (150.2950,154.4290) -- cycle(151.0950,154.2690) -- (151.3620,154.2690) -- (151.3620,154.4290) -- (151.0950,154.4290) -- cycle(151.8950,154.2690) -- (152.1620,154.2690) -- (152.1620,154.4290) -- (151.8950,154.4290) -- cycle(152.6950,154.2690) -- (152.9620,154.2690) -- (152.9620,154.4290) -- (152.6950,154.4290) -- cycle(153.4950,154.2690) -- (153.7620,154.2690) -- (153.7620,154.4290) -- (153.4950,154.4290) -- cycle(154.2950,154.2690) -- (154.5620,154.2690) -- (154.5620,154.4290) -- (154.2950,154.4290) -- cycle(155.0950,154.2690) -- (155.3620,154.2690) -- (155.3620,154.4290) -- (155.0950,154.4290) -- cycle(155.8950,154.2690) -- (156.1620,154.2690) -- (156.1620,154.4290) -- (155.8950,154.4290) -- cycle(156.6950,154.2690) -- (156.9620,154.2690) -- (156.9620,154.4290) -- (156.6950,154.4290) -- cycle(157.4950,154.2690) -- (157.7620,154.2690) -- (157.7620,154.4290) -- (157.4950,154.4290) -- cycle(158.2950,154.2690) -- (158.5620,154.2690) -- (158.5620,154.4290) -- (158.2950,154.4290) -- cycle(159.0950,154.2690) -- (159.3620,154.2690) -- (159.3620,154.4290) -- (159.0950,154.4290) -- cycle(159.8950,154.2690) -- (160.1620,154.2690) -- (160.1620,154.4290) -- (159.8950,154.4290) -- cycle(160.6950,154.2690) -- (160.9620,154.2690) -- (160.9620,154.4290) -- (160.6950,154.4290) -- cycle(161.4950,154.2690) -- (161.7620,154.2690) -- (161.7620,154.4290) -- (161.4950,154.4290) -- cycle(162.2950,154.2690) -- (162.5620,154.2690) -- (162.5620,154.4290) -- (162.2950,154.4290) -- cycle(163.0950,154.2690) -- (163.3620,154.2690) -- (163.3620,154.4290) -- (163.0950,154.4290) -- cycle(163.8950,154.2690) -- (164.1620,154.2690) -- (164.1620,154.4290) -- (163.8950,154.4290) -- cycle(164.6950,154.2690) -- (164.9620,154.2690) -- (164.9620,154.4290) -- (164.6950,154.4290) -- cycle(165.4950,154.2690) -- (165.7620,154.2690) -- (165.7620,154.4290) -- (165.4950,154.4290) -- cycle(166.2950,154.2690) -- (166.5620,154.2690) -- (166.5620,154.4290) -- (166.2950,154.4290) -- cycle(167.0950,154.2690) -- (167.3620,154.2690) -- (167.3620,154.4290) -- (167.0950,154.4290) -- cycle(167.8950,154.2690) -- (168.1620,154.2690) -- (168.1620,154.4290) -- (167.8950,154.4290) -- cycle(168.6950,154.2690) -- (168.9620,154.2690) -- (168.9620,154.4290) -- (168.6950,154.4290) -- cycle(169.4950,154.2690) -- (169.7620,154.2690) -- (169.7620,154.4290) -- (169.4950,154.4290) -- cycle(170.2950,154.2690) -- (170.5620,154.2690) -- (170.5620,154.4290) -- (170.2950,154.4290) -- cycle(171.0950,154.2690) -- (171.3620,154.2690) -- (171.3620,154.4290) -- (171.0950,154.4290) -- cycle(171.8950,154.2690) -- (172.1620,154.2690) -- (172.1620,154.4290) -- (171.8950,154.4290) -- cycle(172.6950,154.2690) -- (172.9620,154.2690) -- (172.9620,154.4290) -- (172.6950,154.4290) -- cycle(173.4950,154.2690) -- (173.7620,154.2690) -- (173.7620,154.4290) -- (173.4950,154.4290) -- cycle(174.2950,154.2690) -- (174.5620,154.2690) -- (174.5620,154.4290) -- (174.2950,154.4290) -- cycle(175.0950,154.2690) -- (175.3620,154.2690) -- (175.3620,154.4290) -- (175.0950,154.4290) -- cycle(175.8950,154.2690) -- (176.1620,154.2690) -- (176.1620,154.4290) -- (175.8950,154.4290) -- cycle(176.6950,154.2690) -- (176.9620,154.2690) -- (176.9620,154.4290) -- (176.6950,154.4290) -- cycle(177.4950,154.2690) -- (177.7620,154.2690) -- (177.7620,154.4290) -- (177.4950,154.4290) -- cycle(178.2950,154.2690) -- (178.5620,154.2690) -- (178.5620,154.4290) -- (178.2950,154.4290) -- cycle(179.0950,154.2690) -- (179.3620,154.2690) -- (179.3620,154.4290) -- (179.0950,154.4290) -- cycle(179.8950,154.2690) -- (180.1620,154.2690) -- (180.1620,154.4290) -- (179.8950,154.4290) -- cycle(180.6950,154.2690) -- (180.9620,154.2690) -- (180.9620,154.4290) -- (180.6950,154.4290) -- cycle(181.4950,154.2690) -- (181.7620,154.2690) -- (181.7620,154.4290) -- (181.4950,154.4290) -- cycle(182.2950,154.2690) -- (182.5620,154.2690) -- (182.5620,154.4290) -- (182.2950,154.4290) -- cycle(183.0950,154.2690) -- (183.3620,154.2690) -- (183.3620,154.4290) -- (183.0950,154.4290) -- cycle(183.8950,154.2690) -- (184.1620,154.2690) -- (184.1620,154.4290) -- (183.8950,154.4290) -- cycle(184.6950,154.2690) -- (184.9620,154.2690) -- (184.9620,154.4290) -- (184.6950,154.4290) -- cycle(185.4950,154.2690) -- (185.7620,154.2690) -- (185.7620,154.4290) -- (185.4950,154.4290) -- cycle(186.2950,154.2690) -- (186.5620,154.2690) -- (186.5620,154.4290) -- (186.2950,154.4290) -- cycle(187.0950,154.2690) -- (187.3620,154.2690) -- (187.3620,154.4290) -- (187.0950,154.4290) -- cycle(187.8950,154.2690) -- (188.1620,154.2690) -- (188.1620,154.4290) -- (187.8950,154.4290) -- cycle(188.6950,154.2690) -- (188.9620,154.2690) -- (188.9620,154.4290) -- (188.6950,154.4290) -- cycle(189.4950,154.2690) -- (189.7620,154.2690) -- (189.7620,154.4290) -- (189.4950,154.4290) -- cycle(190.2950,154.2690) -- (190.5620,154.2690) -- (190.5620,154.4290) -- (190.2950,154.4290) -- cycle(191.0950,154.2690) -- (191.3620,154.2690) -- (191.3620,154.4290) -- (191.0950,154.4290) -- cycle(191.8950,154.2690) -- (192.1620,154.2690) -- (192.1620,154.4290) -- (191.8950,154.4290) -- cycle(192.6950,154.2690) -- (192.9620,154.2690) -- (192.9620,154.4290) -- (192.6950,154.4290) -- cycle(193.4950,154.2690) -- (193.7620,154.2690) -- (193.7620,154.4290) -- (193.4950,154.4290) -- cycle(194.2950,154.2690) -- (194.5620,154.2690) -- (194.5620,154.4290) -- (194.2950,154.4290) -- cycle(195.0950,154.2690) -- (195.3620,154.2690) -- (195.3620,154.4290) -- (195.0950,154.4290) -- cycle(195.8950,154.2690) -- (196.1620,154.2690) -- (196.1620,154.4290) -- (195.8950,154.4290) -- cycle(196.6950,154.2690) -- (196.9620,154.2690) -- (196.9620,154.4290) -- (196.6950,154.4290) -- cycle(197.4950,154.2690) -- (197.7620,154.2690) -- (197.7620,154.4290) -- (197.4950,154.4290) -- cycle(198.2950,154.2690) -- (198.5620,154.2690) -- (198.5620,154.4290) -- (198.2950,154.4290) -- cycle(199.0950,154.2690) -- (199.3620,154.2690) -- (199.3620,154.4290) -- (199.0950,154.4290) -- cycle(199.8950,154.2690) -- (200.1620,154.2690) -- (200.1620,154.4290) -- (199.8950,154.4290) -- cycle(200.6950,154.2690) -- (200.9620,154.2690) -- (200.9620,154.4290) -- (200.6950,154.4290) -- cycle(201.4950,154.2690) -- (201.7620,154.2690) -- (201.7620,154.4290) -- (201.4950,154.4290) -- cycle(202.2950,154.2690) -- (202.5620,154.2690) -- (202.5620,154.4290) -- (202.2950,154.4290) -- cycle(203.0950,154.2690) -- (203.3620,154.2690) -- (203.3620,154.4290) -- (203.0950,154.4290) -- cycle(203.8950,154.2690) -- (204.1620,154.2690) -- (204.1620,154.4290) -- (203.8950,154.4290) -- cycle(204.6950,154.2690) -- (204.9620,154.2690) -- (204.9620,154.4290) -- (204.6950,154.4290) -- cycle(205.4950,154.2690) -- (205.7620,154.2690) -- (205.7620,154.4290) -- (205.4950,154.4290) -- cycle(206.2950,154.2690) -- (206.5620,154.2690) -- (206.5620,154.4290) -- (206.2950,154.4290) -- cycle(207.0950,154.2690) -- (207.3620,154.2690) -- (207.3620,154.4290) -- (207.0950,154.4290) -- cycle(207.8950,154.2690) -- (208.1620,154.2690) -- (208.1620,154.4290) -- (207.8950,154.4290) -- cycle(208.6950,154.2690) -- (208.9620,154.2690) -- (208.9620,154.4290) -- (208.6950,154.4290) -- cycle(209.4950,154.2690) -- (209.7620,154.2690) -- (209.7620,154.4290) -- (209.4950,154.4290) -- cycle(210.2950,154.2690) -- (210.5620,154.2690) -- (210.5620,154.4290) -- (210.2950,154.4290) -- cycle(211.0950,154.2690) -- (211.3620,154.2690) -- (211.3620,154.4290) -- (211.0950,154.4290) -- cycle(211.8950,154.2690) -- (212.1620,154.2690) -- (212.1620,154.4290) -- (211.8950,154.4290) -- cycle(212.6950,154.2690) -- (212.9620,154.2690) -- (212.9620,154.4290) -- (212.6950,154.4290) -- cycle(213.4950,154.2690) -- (213.7620,154.2690) -- (213.7620,154.4290) -- (213.4950,154.4290) -- cycle(214.2950,154.2690) -- (214.5620,154.2690) -- (214.5620,154.4290) -- (214.2950,154.4290) -- cycle(215.0950,154.2690) -- (215.3620,154.2690) -- (215.3620,154.4290) -- (215.0950,154.4290) -- cycle(215.8950,154.2690) -- (216.1620,154.2690) -- (216.1620,154.4290) -- (215.8950,154.4290) -- cycle(216.6950,154.2690) -- (216.9620,154.2690) -- (216.9620,154.4290) -- (216.6950,154.4290) -- cycle(217.4950,154.2690) -- (217.7620,154.2690) -- (217.7620,154.4290) -- (217.4950,154.4290) -- cycle(218.2950,154.2690) -- (218.5620,154.2690) -- (218.5620,154.4290) -- (218.2950,154.4290) -- cycle(219.0950,154.2690) -- (219.3620,154.2690) -- (219.3620,154.4290) -- (219.0950,154.4290) -- cycle(219.8950,154.2690) -- (220.1620,154.2690) -- (220.1620,154.4290) -- (219.8950,154.4290) -- cycle(220.6950,154.2690) -- (220.9620,154.2690) -- (220.9620,154.4290) -- (220.6950,154.4290) -- cycle(221.4950,154.2690) -- (221.7620,154.2690) -- (221.7620,154.4290) -- (221.4950,154.4290) -- cycle(222.2950,154.2690) -- (222.5620,154.2690) -- (222.5620,154.4290) -- (222.2950,154.4290) -- cycle(223.0950,154.2690) -- (223.3620,154.2690) -- (223.3620,154.4290) -- (223.0950,154.4290) -- cycle(223.8950,154.2690) -- (224.1620,154.2690) -- (224.1620,154.4290) -- (223.8950,154.4290) -- cycle(224.6950,154.2690) -- (224.9620,154.2690) -- (224.9620,154.4290) -- (224.6950,154.4290) -- cycle(225.4950,154.2690) -- (225.7620,154.2690) -- (225.7620,154.4290) -- (225.4950,154.4290) -- cycle(226.2950,154.2690) -- (226.5620,154.2690) -- (226.5620,154.4290) -- (226.2950,154.4290) -- cycle(227.0950,154.2690) -- (227.3620,154.2690) -- (227.3620,154.4290) -- (227.0950,154.4290) -- cycle(227.8950,154.2690) -- (228.1620,154.2690) -- (228.1620,154.4290) -- (227.8950,154.4290) -- cycle(228.6950,154.2690) -- (228.9620,154.2690) -- (228.9620,154.4290) -- (228.6950,154.4290) -- cycle(229.4950,154.2690) -- (229.7620,154.2690) -- (229.7620,154.4290) -- (229.4950,154.4290) -- cycle(230.2950,154.2690) -- (230.5620,154.2690) -- (230.5620,154.4290) -- (230.2950,154.4290) -- cycle(231.0950,154.2690) -- (231.3620,154.2690) -- (231.3620,154.4290) -- (231.0950,154.4290) -- cycle(231.8950,154.2690) -- (232.1620,154.2690) -- (232.1620,154.4290) -- (231.8950,154.4290) -- cycle(232.6950,154.2690) -- (232.9620,154.2690) -- (232.9620,154.4290) -- (232.6950,154.4290) -- cycle(233.4950,154.2690) -- (233.7620,154.2690) -- (233.7620,154.4290) -- (233.4950,154.4290) -- cycle(234.2950,154.2690) -- (234.5620,154.2690) -- (234.5620,154.4290) -- (234.2950,154.4290) -- cycle(235.0950,154.2690) -- (235.3620,154.2690) -- (235.3620,154.4290) -- (235.0950,154.4290) -- cycle(235.8950,154.2690) -- (236.1620,154.2690) -- (236.1620,154.4290) -- (235.8950,154.4290) -- cycle(236.6950,154.2690) -- (236.9620,154.2690) -- (236.9620,154.4290) -- (236.6950,154.4290) -- cycle(237.4950,154.2690) -- (237.7620,154.2690) -- (237.7620,154.4290) -- (237.4950,154.4290) -- cycle(238.2950,154.2690) -- (238.5620,154.2690) -- (238.5620,154.4290) -- (238.2950,154.4290) -- cycle(239.0950,154.2690) -- (239.3620,154.2690) -- (239.3620,154.4290) -- (239.0950,154.4290) -- cycle(239.8950,154.2690) -- (240.1620,154.2690) -- (240.1620,154.4290) -- (239.8950,154.4290) -- cycle(240.6950,154.2690) -- (240.9620,154.2690) -- (240.9620,154.4290) -- (240.6950,154.4290) -- cycle(241.4950,154.2690) -- (241.7620,154.2690) -- (241.7620,154.4290) -- (241.4950,154.4290) -- cycle(242.2950,154.2690) -- (242.5620,154.2690) -- (242.5620,154.4290) -- (242.2950,154.4290) -- cycle(243.0950,154.2690) -- (243.3620,154.2690) -- (243.3620,154.4290) -- (243.0950,154.4290) -- cycle(243.8950,154.2690) -- (244.1620,154.2690) -- (244.1620,154.4290) -- (243.8950,154.4290) -- cycle(244.6950,154.2690) -- (244.9620,154.2690) -- (244.9620,154.4290) -- (244.6950,154.4290) -- cycle(245.4950,154.2690) -- (245.7620,154.2690) -- (245.7620,154.4290) -- (245.4950,154.4290) -- cycle(246.2950,154.2690) -- (246.5620,154.2690) -- (246.5620,154.4290) -- (246.2950,154.4290) -- cycle(247.0950,154.2690) -- (247.3620,154.2690) -- (247.3620,154.4290) -- (247.0950,154.4290) -- cycle(247.8950,154.2690) -- (248.1620,154.2690) -- (248.1620,154.4290) -- (247.8950,154.4290) -- cycle(248.6950,154.2690) -- (248.9620,154.2690) -- (248.9620,154.4290) -- (248.6950,154.4290) -- cycle(249.4950,154.2690) -- (249.7620,154.2690) -- (249.7620,154.4290) -- (249.4950,154.4290) -- cycle(250.2950,154.2690) -- (250.5620,154.2690) -- (250.5620,154.4290) -- (250.2950,154.4290) -- cycle(251.0950,154.2690) -- (251.3620,154.2690) -- (251.3620,154.4290) -- (251.0950,154.4290) -- cycle(251.8950,154.2690) -- (252.1620,154.2690) -- (252.1620,154.4290) -- (251.8950,154.4290) -- cycle(252.6950,154.2690) -- (252.9620,154.2690) -- (252.9620,154.4290) -- (252.6950,154.4290) -- cycle(253.4950,154.2690) -- (253.7620,154.2690) -- (253.7620,154.4290) -- (253.4950,154.4290) -- cycle(254.2950,154.2690) -- (254.5620,154.2690) -- (254.5620,154.4290) -- (254.2950,154.4290) -- cycle(255.0950,154.2690) -- (255.3620,154.2690) -- (255.3620,154.4290) -- (255.0950,154.4290) -- cycle(255.8950,154.2690) -- (256.1620,154.2690) -- (256.1620,154.4290) -- (255.8950,154.4290) -- cycle(256.6950,154.2690) -- (256.9620,154.2690) -- (256.9620,154.4290) -- (256.6950,154.4290) -- cycle(257.4950,154.2690) -- (257.7620,154.2690) -- (257.7620,154.4290) -- (257.4950,154.4290) -- cycle(258.2950,154.2690) -- (258.5620,154.2690) -- (258.5620,154.4290) -- (258.2950,154.4290) -- cycle(259.0950,154.2690) -- (259.3620,154.2690) -- (259.3620,154.4290) -- (259.0950,154.4290) -- cycle(259.9690,154.3770) -- (259.8780,154.6270) -- (259.7270,154.5730) -- (259.8190,154.3220) -- cycle(259.6950,155.1290) -- (259.6040,155.3790) -- (259.4540,155.3240) -- (259.5450,155.0740) -- cycle(259.4220,155.8800) -- (259.3310,156.1310) -- (259.1800,156.0760) -- (259.2710,155.8260) -- cycle(259.1480,156.6320) -- (259.0570,156.8830) -- (258.9070,156.8280) -- (258.9980,156.5770) -- cycle(258.8750,157.3840) -- (258.7830,157.6340) -- (258.6330,157.5800) -- (258.7240,157.3290) -- cycle(258.6010,158.1360) -- (258.5100,158.3860) -- (258.3590,158.3320) -- (258.4510,158.0810) -- cycle(258.3270,158.8870) -- (258.2360,159.1380) -- (258.0860,159.0830) -- (258.1770,158.8330) -- cycle(258.0540,159.6390) -- (257.9620,159.8900) -- (257.8120,159.8350) -- (257.9030,159.5840) -- cycle(257.7800,160.3910) -- (257.6890,160.6420) -- (257.5380,160.5870) -- (257.6300,160.3360) -- cycle(257.5070,161.1430) -- (257.4150,161.3930) -- (257.2650,161.3390) -- (257.3560,161.0880) -- cycle(257.2330,161.8940) -- (257.1420,162.1450) -- (256.9910,162.0900) -- (257.0820,161.8400) -- cycle(256.9590,162.6460) -- (256.8680,162.8970) -- (256.7180,162.8420) -- (256.8090,162.5910) -- cycle(256.6860,163.3980) -- (256.5940,163.6480) -- (256.4440,163.5940) -- (256.5350,163.3430) -- cycle(256.4120,164.1500) -- (256.3210,164.4000) -- (256.1700,164.3460) -- (256.2620,164.0950) -- cycle(256.1380,164.9010) -- (256.0470,165.1520) -- (255.8970,165.0970) -- (255.9880,164.8470) -- cycle(255.8650,165.6530) -- (255.7740,165.9040) -- (255.6230,165.8490) -- (255.7140,165.5980) -- cycle(255.5910,166.4050) -- (255.5000,166.6550) -- (255.3500,166.6010) -- (255.4410,166.3500) -- cycle(255.3180,167.1570) -- (255.2260,167.4070) -- (255.0760,167.3530) -- (255.1670,167.1020) -- cycle(255.0440,167.9080) -- (254.9530,168.1590) -- (254.8020,168.1040) -- (254.8940,167.8540) -- cycle(254.7700,168.6600) -- (254.6790,168.9110) -- (254.5290,168.8560) -- (254.6200,168.6050) -- cycle(254.4970,169.4120) -- (254.4050,169.6630) -- (254.2550,169.6080) -- (254.3460,169.3570) -- cycle(254.2230,170.1640) -- (254.1320,170.4140) -- (253.9810,170.3600) -- (254.0730,170.1090) -- cycle(253.9490,170.9150) -- (253.8580,171.1660) -- (253.7080,171.1110) -- (253.7990,170.8610) -- cycle(253.6760,171.6670) -- (253.5850,171.9180) -- (253.4340,171.8630) -- (253.5250,171.6120) -- cycle(253.4020,172.4190) -- (253.3110,172.6700) -- (253.1610,172.6150) -- (253.2520,172.3640) -- cycle(253.1290,173.1710) -- (253.0370,173.4210) -- (252.8870,173.3670) -- (252.9780,173.1160) -- cycle(252.8550,173.9220) -- (252.7640,174.1730) -- (252.6130,174.1180) -- (252.7050,173.8680) -- cycle(252.5810,174.6740) -- (252.4900,174.9250) -- (252.3400,174.8700) -- (252.4310,174.6190) -- cycle(252.3080,175.4260) -- (252.2170,175.6770) -- (252.0660,175.6220) -- (252.1570,175.3710) -- cycle(252.0340,176.1780) -- (251.9430,176.4280) -- (251.7930,176.3740) -- (251.8840,176.1230) -- cycle(251.7610,176.9290) -- (251.6690,177.1800) -- (251.5190,177.1250) -- (251.6100,176.8750) -- cycle(251.4870,177.6810) -- (251.3960,177.9320) -- (251.2450,177.8770) -- (251.3370,177.6260) -- cycle(251.2130,178.4330) -- (251.1220,178.6840) -- (250.9720,178.6290) -- (251.0630,178.3780) -- cycle(250.9400,179.1850) -- (250.8480,179.4350) -- (250.6980,179.3810) -- (250.7890,179.1300) -- cycle(250.6660,179.9360) -- (250.5750,180.1870) -- (250.4240,180.1320) -- (250.5160,179.8820) -- cycle(250.3920,180.6880) -- (250.3010,180.9390) -- (250.1510,180.8840) -- (250.2420,180.6340) -- cycle(250.1190,181.4400) -- (250.0280,181.6910) -- (249.8770,181.6360) -- (249.9680,181.3850) -- cycle(249.8450,182.1920) -- (249.7540,182.4420) -- (249.6040,182.3880) -- (249.6950,182.1370) -- cycle(249.5720,182.9430) -- (249.4800,183.1940) -- (249.3300,183.1390) -- (249.4210,182.8890) -- cycle(249.2980,183.6950) -- (249.2070,183.9460) -- (249.0560,183.8910) -- (249.1480,183.6410) -- cycle(249.0240,184.4470) -- (248.9330,184.6980) -- (248.7830,184.6430) -- (248.8740,184.3920) -- cycle(248.7510,185.1990) -- (248.6600,185.4490) -- (248.5090,185.3950) -- (248.6000,185.1440) -- cycle(248.4770,185.9500) -- (248.3860,186.2010) -- (248.2360,186.1460) -- (248.3270,185.8960) -- cycle(248.2040,186.7020) -- (248.1120,186.9530) -- (247.9620,186.8980) -- (248.0530,186.6470) -- cycle(247.9300,187.4540) -- (247.8390,187.7050) -- (247.6880,187.6500) -- (247.7800,187.3990) -- cycle(247.6560,188.2060) -- (247.5650,188.4560) -- (247.4150,188.4020) -- (247.5060,188.1510) -- cycle(247.3830,188.9570) -- (247.2910,189.2080) -- (247.1410,189.1530) -- (247.2320,188.9030) -- cycle(247.1090,189.7090) -- (247.0180,189.9600) -- (246.8670,189.9050) -- (246.9590,189.6550) -- cycle(246.8350,190.4610) -- (246.7440,190.7120) -- (246.5940,190.6570) -- (246.6850,190.4060) -- cycle(246.5620,191.2130) -- (246.4710,191.4630) -- (246.3200,191.4090) -- (246.4110,191.1580) -- cycle(246.2880,191.9650) -- (246.1970,192.2150) -- (246.0470,192.1600) -- (246.1380,191.9100) -- cycle(246.0150,192.7160) -- (245.9230,192.9670) -- (245.7730,192.9120) -- (245.8640,192.6620) -- cycle(245.7410,193.4680) -- (245.6500,193.7190) -- (245.4990,193.6640) -- (245.5910,193.4130) -- cycle(245.4670,194.2200) -- (245.3760,194.4700) -- (245.2260,194.4160) -- (245.3170,194.1650) -- cycle(245.1940,194.9710) -- (245.1030,195.2220) -- (244.9520,195.1670) -- (245.0430,194.9170) -- cycle(244.9200,195.7230) -- (244.8290,195.9740) -- (244.6790,195.9190) -- (244.7700,195.6690) -- cycle(244.6470,196.4750) -- (244.5550,196.7260) -- (244.4050,196.6710) -- (244.4960,196.4200) -- cycle(244.3730,197.2270) -- (244.2820,197.4770) -- (244.1310,197.4230) -- (244.2230,197.1720) -- cycle(244.0990,197.9780) -- (244.0080,198.2290) -- (243.8580,198.1740) -- (243.9490,197.9240) -- cycle(243.8260,198.7300) -- (243.7350,198.9810) -- (243.5840,198.9260) -- (243.6750,198.6760) -- cycle(243.5520,199.4820) -- (243.4610,199.7330) -- (243.3100,199.6780) -- (243.4020,199.4270) -- cycle(243.2780,200.2340) -- (243.1870,200.4840) -- (243.0370,200.4300) -- (243.1280,200.1790) -- cycle(243.0050,200.9860) -- (242.9140,201.2360) -- (242.7630,201.1810) -- (242.8540,200.9310) -- cycle(242.7310,201.7370) -- (242.6400,201.9880) -- (242.4900,201.9330) -- (242.5810,201.6830) -- cycle(242.4580,202.4890) -- (242.3660,202.7400) -- (242.2160,202.6850) -- (242.3070,202.4340) -- cycle(242.1840,203.2410) -- (242.0930,203.4910) -- (241.9420,203.4370) -- (242.0340,203.1860) -- cycle(241.9100,203.9930) -- (241.8190,204.2430) -- (241.6690,204.1880) -- (241.7600,203.9380) -- cycle(241.6370,204.7440) -- (241.5460,204.9950) -- (241.3950,204.9400) -- (241.4860,204.6900) -- cycle(241.3630,205.4960) -- (241.2720,205.7470) -- (241.1220,205.6920) -- (241.2130,205.4410) -- cycle(241.0900,206.2480) -- (240.9980,206.4980) -- (240.8480,206.4440) -- (240.9390,206.1930) -- cycle(240.8160,207.0000) -- (240.7250,207.2500) -- (240.5740,207.1950) -- (240.6660,206.9450) -- cycle(240.5420,207.7510) -- (240.4510,208.0020) -- (240.3010,207.9470) -- (240.3920,207.6970) -- cycle(240.2690,208.5030) -- (240.1780,208.7540) -- (240.0270,208.6990) -- (240.1180,208.4480) -- cycle(239.9950,209.2550) -- (239.9040,209.5050) -- (239.7530,209.4510) -- (239.8450,209.2000) -- cycle(239.7210,210.0070) -- (239.6300,210.2570) -- (239.4800,210.2020) -- (239.5710,209.9520) -- cycle(239.4480,210.7580) -- (239.3570,211.0090) -- (239.2060,210.9540) -- (239.2970,210.7040) -- cycle(239.1740,211.5100) -- (239.0830,211.7610) -- (238.9330,211.7060) -- (239.0240,211.4550) -- cycle(238.9010,212.2620) -- (238.8090,212.5120) -- (238.6590,212.4580) -- (238.7500,212.2070) -- cycle(238.6270,213.0140) -- (238.5360,213.2640) -- (238.3850,213.2090) -- (238.4770,212.9590) -- cycle(238.3530,213.7650) -- (238.2620,214.0160) -- (238.1120,213.9610) -- (238.2030,213.7110) -- cycle(238.0800,214.5170) -- (237.9890,214.7680) -- (237.8380,214.7130) -- (237.9290,214.4620) -- cycle(237.8060,215.2690) -- (237.7150,215.5190) -- (237.5650,215.4650) -- (237.6560,215.2140) -- cycle(237.5330,216.0210) -- (237.4410,216.2710) -- (237.2910,216.2160) -- (237.3820,215.9660) -- cycle(237.2590,216.7720) -- (237.1680,217.0230) -- (237.0170,216.9680) -- (237.1090,216.7180) -- cycle(236.9850,217.5240) -- (236.8940,217.7750) -- (236.7440,217.7200) -- (236.8350,217.4690) -- cycle(236.7120,218.2760) -- (236.6200,218.5260) -- (236.4700,218.4720) -- (236.5610,218.2210) -- cycle(236.4380,219.0280) -- (236.3470,219.2780) -- (236.1960,219.2230) -- (236.2880,218.9730) -- cycle(236.1640,219.7790) -- (236.0730,220.0300) -- (235.9230,219.9750) -- (236.0140,219.7250) -- cycle(235.8910,220.5310) -- (235.8000,220.7820) -- (235.6490,220.7270) -- (235.7400,220.4760) -- cycle(235.6170,221.2830) -- (235.5260,221.5330) -- (235.3760,221.4790) -- (235.4670,221.2280) -- cycle(235.3440,222.0350) -- (235.2520,222.2850) -- (235.1020,222.2300) -- (235.1930,221.9800) -- cycle(235.0700,222.7860) -- (234.9790,223.0370) -- (234.8280,222.9820) -- (234.9200,222.7320) -- cycle(234.7960,223.5380) -- (234.7050,223.7890) -- (234.5550,223.7340) -- (234.6460,223.4830) -- cycle(234.5230,224.2900) -- (234.4320,224.5400) -- (234.2810,224.4860) -- (234.3720,224.2350) -- cycle(234.2490,225.0420) -- (234.1580,225.2920) -- (234.0080,225.2370) -- (234.0990,224.9870) -- cycle(233.9760,225.7930) -- (233.8840,226.0440) -- (233.7340,225.9890) -- (233.8250,225.7390) -- cycle(233.7020,226.5450) -- (233.6110,226.7960) -- (233.4600,226.7410) -- (233.5520,226.4900) -- cycle(233.4280,227.2970) -- (233.3370,227.5470) -- (233.1870,227.4930) -- (233.2780,227.2420) -- cycle(233.1550,228.0490) -- (233.0630,228.2990) -- (232.9130,228.2440) -- (233.0040,227.9940) -- cycle(232.8810,228.8000) -- (232.7900,229.0510) -- (232.6390,228.9960) -- (232.7310,228.7460) -- cycle(232.6070,229.5520) -- (232.5160,229.8030) -- (232.3660,229.7480) -- (232.4570,229.4970) -- cycle(232.3340,230.3040) -- (232.2430,230.5540) -- (232.0920,230.5000) -- (232.1830,230.2490) -- cycle(232.0600,231.0560) -- (231.9690,231.3060) -- (231.8190,231.2510) -- (231.9100,231.0010) -- cycle(231.7870,231.8070) -- (231.6950,232.0580) -- (231.5450,232.0030) -- (231.6360,231.7530) -- cycle(231.5130,232.5590) -- (231.4220,232.8100) -- (231.2710,232.7550) -- (231.3630,232.5040) -- cycle(231.2390,233.3110) -- (231.1480,233.5610) -- (230.9980,233.5070) -- (231.0890,233.2560) -- cycle(230.9660,234.0630) -- (230.8750,234.3130) -- (230.7240,234.2590) -- (230.8150,234.0080) -- cycle(230.6920,234.8140) -- (230.6010,235.0650) -- (230.4510,235.0100) -- (230.5420,234.7600) -- cycle(230.4190,235.5660) -- (230.3270,235.8170) -- (230.1770,235.7620) -- (230.2680,235.5110) -- cycle(230.1450,236.3180) -- (230.0540,236.5680) -- (229.9030,236.5140) -- (229.9950,236.2630) -- cycle(229.8710,237.0700) -- (229.7800,237.3200) -- (229.6300,237.2660) -- (229.7210,237.0150) -- cycle(229.5980,237.8210) -- (229.5060,238.0720) -- (229.3560,238.0170) -- (229.4470,237.7670) -- cycle(229.3240,238.5730) -- (229.2330,238.8240) -- (229.0820,238.7690) -- (229.1740,238.5180) -- cycle(229.0500,239.3250) -- (228.9590,239.5750) -- (228.8090,239.5210) -- (228.9000,239.2700) -- cycle(228.7770,240.0770) -- (228.6860,240.3270) -- (228.5350,240.2720) -- (228.6260,240.0220) -- cycle(228.5030,240.8280) -- (228.4120,241.0790) -- (228.2620,241.0240) -- (228.3530,240.7740) -- cycle(228.2300,241.5800) -- (228.1380,241.8310) -- (227.9880,241.7760) -- (228.0790,241.5250) -- cycle(227.9560,242.3320) -- (227.8650,242.5820) -- (227.7140,242.5280) -- (227.8060,242.2770) -- cycle(227.6820,243.0840) -- (227.5910,243.3340) -- (227.4410,243.2800) -- (227.5320,243.0290) -- cycle(227.4090,243.8350) -- (227.3180,244.0860) -- (227.1670,244.0310) -- (227.2580,243.7810) -- cycle(227.1350,244.5870) -- (227.0440,244.8380) -- (226.8940,244.7830) -- (226.9850,244.5320) -- cycle(226.8620,245.3390) -- (226.7700,245.5890) -- (226.6200,245.5350) -- (226.7110,245.2840) -- cycle(226.5880,246.0910) -- (226.4970,246.3410) -- (226.3460,246.2870) -- (226.4380,246.0360) -- cycle(226.3140,246.8420) -- (226.2230,247.0930) -- (226.0730,247.0380) -- (226.1640,246.7880) -- cycle(226.0410,247.5940) -- (225.9490,247.8450) -- (225.7990,247.7900) -- (225.8900,247.5390) -- cycle(225.7670,248.3460) -- (225.6760,248.5970) -- (225.5250,248.5420) -- (225.6170,248.2910) -- cycle(225.4930,249.0980) -- (225.4020,249.3480) -- (225.2520,249.2940) -- (225.3430,249.0430) -- cycle(225.2200,249.8490) -- (225.1290,250.1000) -- (224.9780,250.0450) -- (225.0690,249.7950) -- cycle(224.9460,250.6010) -- (224.8550,250.8520) -- (224.7050,250.7970) -- (224.7960,250.5460) -- cycle(224.6730,251.3530) -- (224.5810,251.6030) -- (224.4310,251.5490) -- (224.5220,251.2980) -- cycle(224.3990,252.1050) -- (224.3080,252.3550) -- (224.1570,252.3010) -- (224.2490,252.0500) -- cycle(224.1250,252.8560) -- (224.0340,253.1070) -- (223.8840,253.0520) -- (223.9750,252.8020) -- cycle(223.8520,253.6080) -- (223.7610,253.8590) -- (223.6100,253.8040) -- (223.7010,253.5530) -- cycle(223.5780,254.3600) -- (223.4870,254.6100) -- (223.3370,254.5560) -- (223.4280,254.3050) -- cycle(223.3050,255.1120) -- (223.2130,255.3620) -- (223.0630,255.3080) -- (223.1540,255.0570) -- cycle(223.0310,255.8630) -- (222.9400,256.1140) -- (222.7890,256.0590) -- (222.8800,255.8090) -- cycle(222.7570,256.6150) -- (222.6660,256.8660) -- (222.5160,256.8110) -- (222.6070,256.5610) -- cycle(222.4840,257.3670) -- (222.3920,257.6180) -- (222.2420,257.5630) -- (222.3330,257.3120) -- cycle(222.2100,258.1190) -- (222.1190,258.3690) -- (221.9680,258.3150) -- (222.0600,258.0640) -- cycle(221.9360,258.8700) -- (221.8450,259.1210) -- (221.6950,259.0660) -- (221.7860,258.8160) -- cycle(221.6630,259.6220) -- (221.5720,259.8730) -- (221.4210,259.8180) -- (221.5120,259.5670) -- cycle(221.3890,260.3740) -- (221.2980,260.6240) -- (221.1480,260.5700) -- (221.2390,260.3190) -- cycle(221.1160,261.1260) -- (221.0240,261.3760) -- (220.8740,261.3220) -- (220.9650,261.0710) -- cycle(220.8420,261.8770) -- (220.7510,262.1280) -- (220.6000,262.0730) -- (220.6920,261.8230) -- cycle(220.5680,262.6290) -- (220.4770,262.8800) -- (220.3270,262.8250) -- (220.4180,262.5750) -- cycle(220.2950,263.3810) -- (220.2040,263.6320) -- (220.0530,263.5770) -- (220.1440,263.3260) -- cycle(220.0210,264.1330) -- (219.9300,264.3830) -- (219.7800,264.3290) -- (219.8710,264.0780) -- cycle(219.7480,264.8840) -- (219.6560,265.1350) -- (219.5060,265.0800) -- (219.5970,264.8300) -- cycle(219.4740,265.6360) -- (219.3830,265.8870) -- (219.2320,265.8320) -- (219.3240,265.5820) -- cycle(219.2000,266.3880) -- (219.1090,266.6390) -- (218.9590,266.5840) -- (219.0500,266.3330) -- cycle(218.9270,267.1400) -- (218.8350,267.3900) -- (218.6850,267.3360) -- (218.7760,267.0850) -- cycle(218.6530,267.8910) -- (218.5620,268.1420) -- (218.4110,268.0870) -- (218.5030,267.8370) -- cycle(218.3790,268.6430) -- (218.2880,268.8940) -- (218.1380,268.8390) -- (218.2290,268.5880) -- cycle(218.1060,269.3950) -- (218.0150,269.6450) -- (217.8640,269.5910) -- (217.9550,269.3400) -- cycle(217.8320,270.1470) -- (217.7410,270.3970) -- (217.5910,270.3430) -- (217.6820,270.0920) -- cycle(217.5590,270.8980) -- (217.4670,271.1490) -- (217.3170,271.0940) -- (217.4080,270.8440) -- cycle(217.2850,271.6500) -- (217.1940,271.9010) -- (217.0430,271.8460) -- (217.1350,271.5960) -- cycle(217.0110,272.4020) -- (216.9200,272.6530) -- (216.7700,272.5980) -- (216.8610,272.3470) -- cycle(216.7380,273.1540) -- (216.6470,273.4040) -- (216.4960,273.3500) -- (216.5870,273.0990) -- cycle(216.4640,273.9050) -- (216.3730,274.1560) -- (216.2230,274.1010) -- (216.3140,273.8510) -- cycle(216.1910,274.6570) -- (216.0990,274.9080) -- (215.9490,274.8530) -- (216.0400,274.6030) -- cycle(215.9170,275.4090) -- (215.8260,275.6600) -- (215.6750,275.6050) -- (215.7670,275.3540) -- cycle(215.6430,276.1610) -- (215.5520,276.4110) -- (215.4020,276.3570) -- (215.4930,276.1060) -- cycle(215.3700,276.9120) -- (215.2790,277.1630) -- (215.1280,277.1080) -- (215.2190,276.8580) -- cycle(215.0960,277.6640) -- (215.0050,277.9150) -- (214.8540,277.8600) -- (214.9460,277.6100) -- cycle(214.8220,278.4160) -- (214.7310,278.6670) -- (214.5810,278.6120) -- (214.6720,278.3610) -- cycle(214.5490,279.1680) -- (214.4580,279.4180) -- (214.3070,279.3640) -- (214.3980,279.1130) -- cycle(214.2750,279.9200) -- (214.1840,280.1700) -- (214.0340,280.1150) -- (214.1250,279.8650) -- cycle(214.0020,280.6710) -- (213.9100,280.9220) -- (213.7600,280.8670) -- (213.8510,280.6170) -- cycle(213.7280,281.4230) -- (213.6370,281.6740) -- (213.4860,281.6190) -- (213.5780,281.3680) -- cycle(213.4540,282.1750) -- (213.3630,282.4250) -- (213.2130,282.3710) -- (213.3040,282.1200) -- cycle(213.1810,282.9260) -- (213.0900,283.1770) -- (212.9390,283.1220) -- (213.0300,282.8720) -- cycle(212.9070,283.6780) -- (212.8160,283.9290) -- (212.6660,283.8740) -- (212.7570,283.6240) -- cycle(212.6340,284.4300) -- (212.5420,284.6810) -- (212.3920,284.6260) -- (212.4830,284.3750) -- cycle(212.3600,285.1820) -- (212.2690,285.4320) -- (212.1180,285.3780) -- (212.2100,285.1270) -- cycle(212.0860,285.9330) -- (211.9950,286.1840) -- (211.8450,286.1290) -- (211.9360,285.8790) -- cycle(211.8130,286.6850) -- (211.7210,286.9360) -- (211.5710,286.8810) -- (211.6620,286.6310) -- cycle(211.5390,287.4370) -- (211.4480,287.6880) -- (211.2970,287.6330) -- (211.3890,287.3820) -- cycle(211.2650,288.1890) -- (211.1740,288.4390) -- (211.0240,288.3850) -- (211.1150,288.1340) -- cycle(210.9920,288.9410) -- (210.9010,289.1910) -- (210.7500,289.1360) -- (210.8410,288.8860) -- cycle(210.7180,289.6920) -- (210.6270,289.9430) -- (210.4770,289.8880) -- (210.5680,289.6380) -- cycle(210.4450,290.4440) -- (210.3530,290.6950) -- (210.2030,290.6400) -- (210.2940,290.3890) -- cycle(210.1710,291.1960) -- (210.0800,291.4460) -- (209.9290,291.3920) -- (210.0210,291.1410) -- cycle(209.8970,291.9470) -- (209.8060,292.1980) -- (209.6560,292.1430) -- (209.7470,291.8930) -- cycle(209.5930,292.6290) -- (209.3270,292.6290) -- (209.3270,292.4690) -- (209.5930,292.4690) -- cycle;



  \path[fill=c7e7e7e,line join=round,line width=0.256pt] (29.7897,291.7650) -- (29.8809,291.5140) -- (30.0313,291.5690) -- (29.9401,291.8200) -- cycle(30.0634,291.0130) -- (30.1545,290.7630) -- (30.3049,290.8170) -- (30.2137,291.0680) -- cycle(30.3370,290.2620) -- (30.4282,290.0110) -- (30.5785,290.0660) -- (30.4873,290.3160) -- cycle(30.6106,289.5100) -- (30.7018,289.2590) -- (30.8521,289.3140) -- (30.7609,289.5640) -- cycle(30.8842,288.7580) -- (30.9754,288.5070) -- (31.1257,288.5620) -- (31.0345,288.8130) -- cycle(31.1578,288.0060) -- (31.2490,287.7560) -- (31.3994,287.8100) -- (31.3082,288.0610) -- cycle(31.4314,287.2540) -- (31.5226,287.0040) -- (31.6730,287.0590) -- (31.5818,287.3090) -- cycle(31.7050,286.5030) -- (31.7962,286.2520) -- (31.9466,286.3070) -- (31.8554,286.5570) -- cycle(31.9786,285.7510) -- (32.0699,285.5000) -- (32.2202,285.5550) -- (32.1290,285.8060) -- cycle(32.2523,284.9990) -- (32.3434,284.7490) -- (32.4938,284.8030) -- (32.4026,285.0540) -- cycle(32.5259,284.2480) -- (32.6171,283.9970) -- (32.7674,284.0520) -- (32.6762,284.3020) -- cycle(32.7995,283.4960) -- (32.8907,283.2450) -- (33.0410,283.3000) -- (32.9498,283.5500) -- cycle(33.0731,282.7440) -- (33.1643,282.4930) -- (33.3146,282.5480) -- (33.2234,282.7990) -- cycle(33.3467,281.9920) -- (33.4379,281.7420) -- (33.5883,281.7960) -- (33.4971,282.0470) -- cycle(33.6203,281.2400) -- (33.7115,280.9900) -- (33.8619,281.0450) -- (33.7707,281.2950) -- cycle(33.8939,280.4890) -- (33.9851,280.2380) -- (34.1355,280.2930) -- (34.0443,280.5430) -- cycle(34.1675,279.7370) -- (34.2588,279.4860) -- (34.4091,279.5410) -- (34.3179,279.7920) -- cycle(34.4412,278.9850) -- (34.5323,278.7350) -- (34.6827,278.7890) -- (34.5915,279.0400) -- cycle(34.7148,278.2330) -- (34.8060,277.9830) -- (34.9563,278.0380) -- (34.8651,278.2880) -- cycle(34.9884,277.4820) -- (35.0796,277.2310) -- (35.2299,277.2860) -- (35.1387,277.5360) -- cycle(35.2620,276.7300) -- (35.3532,276.4790) -- (35.5035,276.5340) -- (35.4124,276.7850) -- cycle(35.5356,275.9780) -- (35.6268,275.7280) -- (35.7772,275.7820) -- (35.6860,276.0330) -- cycle(35.8092,275.2260) -- (35.9004,274.9760) -- (36.0508,275.0310) -- (35.9596,275.2810) -- cycle(36.0828,274.4750) -- (36.1740,274.2240) -- (36.3244,274.2790) -- (36.2332,274.5290) -- cycle(36.3564,273.7230) -- (36.4477,273.4720) -- (36.5980,273.5270) -- (36.5068,273.7780) -- cycle(36.6301,272.9710) -- (36.7213,272.7210) -- (36.8716,272.7750) -- (36.7804,273.0260) -- cycle(36.9037,272.2190) -- (36.9949,271.9690) -- (37.1452,272.0240) -- (37.0540,272.2740) -- cycle(37.1773,271.4680) -- (37.2685,271.2170) -- (37.4189,271.2720) -- (37.3276,271.5220) -- cycle(37.4509,270.7160) -- (37.5421,270.4650) -- (37.6924,270.5200) -- (37.6013,270.7710) -- cycle(37.7245,269.9640) -- (37.8157,269.7140) -- (37.9661,269.7680) -- (37.8749,270.0190) -- cycle(37.9981,269.2120) -- (38.0893,268.9620) -- (38.2397,269.0170) -- (38.1485,269.2670) -- cycle(38.2717,268.4610) -- (38.3629,268.2100) -- (38.5133,268.2650) -- (38.4221,268.5150) -- cycle(38.5453,267.7090) -- (38.6366,267.4580) -- (38.7869,267.5130) -- (38.6957,267.7640) -- cycle(38.8190,266.9570) -- (38.9102,266.7070) -- (39.0605,266.7610) -- (38.9693,267.0120) -- cycle(39.0926,266.2050) -- (39.1838,265.9550) -- (39.3341,266.0090) -- (39.2429,266.2600) -- cycle(39.3662,265.4540) -- (39.4574,265.2030) -- (39.6078,265.2580) -- (39.5165,265.5080) -- cycle(39.6398,264.7020) -- (39.7310,264.4510) -- (39.8813,264.5060) -- (39.7902,264.7570) -- cycle(39.9134,263.9500) -- (40.0046,263.7000) -- (40.1550,263.7540) -- (40.0638,264.0050) -- cycle(40.1870,263.1980) -- (40.2783,262.9480) -- (40.4286,263.0030) -- (40.3374,263.2530) -- cycle(40.4606,262.4470) -- (40.5518,262.1960) -- (40.7022,262.2510) -- (40.6110,262.5010) -- cycle(40.7343,261.6950) -- (40.8255,261.4440) -- (40.9758,261.4990) -- (40.8846,261.7500) -- cycle(41.0079,260.9430) -- (41.0991,260.6930) -- (41.2494,260.7470) -- (41.1582,260.9980) -- cycle(41.2815,260.1910) -- (41.3727,259.9410) -- (41.5230,259.9950) -- (41.4318,260.2460) -- cycle(41.5551,259.4400) -- (41.6463,259.1890) -- (41.7967,259.2440) -- (41.7054,259.4940) -- cycle(41.8287,258.6880) -- (41.9199,258.4370) -- (42.0703,258.4920) -- (41.9791,258.7430) -- cycle(42.1023,257.9360) -- (42.1935,257.6860) -- (42.3439,257.7400) -- (42.2527,257.9910) -- cycle(42.3759,257.1840) -- (42.4672,256.9340) -- (42.6175,256.9880) -- (42.5263,257.2390) -- cycle(42.6495,256.4330) -- (42.7408,256.1820) -- (42.8911,256.2370) -- (42.7999,256.4870) -- cycle(42.9232,255.6810) -- (43.0144,255.4300) -- (43.1647,255.4850) -- (43.0735,255.7360) -- cycle(43.1968,254.9290) -- (43.2880,254.6790) -- (43.4384,254.7330) -- (43.3471,254.9840) -- cycle(43.4704,254.1770) -- (43.5616,253.9270) -- (43.7119,253.9820) -- (43.6207,254.2320) -- cycle(43.7440,253.4260) -- (43.8352,253.1750) -- (43.9856,253.2300) -- (43.8943,253.4800) -- cycle(44.0176,252.6740) -- (44.1088,252.4230) -- (44.2592,252.4780) -- (44.1680,252.7290) -- cycle(44.2912,251.9220) -- (44.3824,251.6710) -- (44.5328,251.7260) -- (44.4416,251.9770) -- cycle(44.5648,251.1700) -- (44.6561,250.9200) -- (44.8064,250.9740) -- (44.7152,251.2250) -- cycle(44.8385,250.4190) -- (44.9297,250.1680) -- (45.0800,250.2230) -- (44.9888,250.4730) -- cycle(45.1121,249.6670) -- (45.2033,249.4160) -- (45.3536,249.4710) -- (45.2624,249.7220) -- cycle(45.3857,248.9150) -- (45.4769,248.6650) -- (45.6273,248.7190) -- (45.5360,248.9700) -- cycle(45.6593,248.1630) -- (45.7505,247.9130) -- (45.9008,247.9670) -- (45.8096,248.2180) -- cycle(45.9329,247.4120) -- (46.0241,247.1610) -- (46.1745,247.2160) -- (46.0833,247.4660) -- cycle(46.2065,246.6600) -- (46.2977,246.4090) -- (46.4481,246.4640) -- (46.3569,246.7150) -- cycle(46.4801,245.9080) -- (46.5714,245.6580) -- (46.7217,245.7120) -- (46.6305,245.9630) -- cycle(46.7538,245.1560) -- (46.8450,244.9060) -- (46.9953,244.9600) -- (46.9041,245.2110) -- cycle(47.0274,244.4050) -- (47.1186,244.1540) -- (47.2689,244.2090) -- (47.1777,244.4590) -- cycle(47.3010,243.6530) -- (47.3922,243.4020) -- (47.5425,243.4570) -- (47.4513,243.7070) -- cycle(47.5746,242.9010) -- (47.6658,242.6500) -- (47.8162,242.7050) -- (47.7249,242.9560) -- cycle(47.8482,242.1490) -- (47.9394,241.8990) -- (48.0898,241.9530) -- (47.9985,242.2040) -- cycle(48.1218,241.3980) -- (48.2130,241.1470) -- (48.3634,241.2020) -- (48.2722,241.4520) -- cycle(48.3954,240.6460) -- (48.4866,240.3950) -- (48.6370,240.4500) -- (48.5458,240.7000) -- cycle(48.6690,239.8940) -- (48.7603,239.6430) -- (48.9106,239.6980) -- (48.8194,239.9490) -- cycle(48.9427,239.1420) -- (49.0339,238.8920) -- (49.1842,238.9460) -- (49.0930,239.1970) -- cycle(49.2163,238.3910) -- (49.3075,238.1400) -- (49.4578,238.1950) -- (49.3666,238.4450) -- cycle(49.4899,237.6390) -- (49.5811,237.3880) -- (49.7314,237.4430) -- (49.6402,237.6940) -- cycle(49.7635,236.8870) -- (49.8547,236.6370) -- (50.0051,236.6910) -- (49.9138,236.9420) -- cycle(50.0371,236.1350) -- (50.1283,235.8850) -- (50.2787,235.9390) -- (50.1875,236.1900) -- cycle(50.3107,235.3840) -- (50.4019,235.1330) -- (50.5523,235.1880) -- (50.4611,235.4380) -- cycle(50.5844,234.6320) -- (50.6755,234.3810) -- (50.8259,234.4360) -- (50.7347,234.6860) -- cycle(50.8579,233.8800) -- (50.9492,233.6290) -- (51.0995,233.6840) -- (51.0083,233.9350) -- cycle(51.1316,233.1280) -- (51.2228,232.8780) -- (51.3731,232.9320) -- (51.2819,233.1830) -- cycle(51.4052,232.3760) -- (51.4964,232.1260) -- (51.6467,232.1810) -- (51.5555,232.4310) -- cycle(51.6788,231.6250) -- (51.7700,231.3740) -- (51.9203,231.4290) -- (51.8291,231.6790) -- cycle(51.9524,230.8730) -- (52.0436,230.6220) -- (52.1940,230.6770) -- (52.1028,230.9280) -- cycle(52.2260,230.1210) -- (52.3172,229.8710) -- (52.4676,229.9250) -- (52.3764,230.1760) -- cycle(52.4996,229.3700) -- (52.5908,229.1190) -- (52.7412,229.1740) -- (52.6500,229.4240) -- cycle(52.7733,228.6180) -- (52.8644,228.3670) -- (53.0148,228.4220) -- (52.9236,228.6720) -- cycle(53.0469,227.8660) -- (53.1381,227.6150) -- (53.2884,227.6700) -- (53.1972,227.9210) -- cycle(53.3205,227.1140) -- (53.4117,226.8640) -- (53.5620,226.9180) -- (53.4708,227.1690) -- cycle(53.5941,226.3630) -- (53.6853,226.1120) -- (53.8356,226.1670) -- (53.7444,226.4170) -- cycle(53.8677,225.6110) -- (53.9589,225.3600) -- (54.1093,225.4150) -- (54.0180,225.6650) -- cycle(54.1413,224.8590) -- (54.2325,224.6080) -- (54.3829,224.6630) -- (54.2917,224.9140) -- cycle(54.4149,224.1070) -- (54.5061,223.8570) -- (54.6565,223.9110) -- (54.5653,224.1620) -- cycle(54.6885,223.3550) -- (54.7797,223.1050) -- (54.9301,223.1600) -- (54.8389,223.4100) -- cycle(54.9622,222.6040) -- (55.0533,222.3530) -- (55.2037,222.4080) -- (55.1125,222.6580) -- cycle(55.2358,221.8520) -- (55.3270,221.6010) -- (55.4773,221.6560) -- (55.3861,221.9070) -- cycle(55.5094,221.1000) -- (55.6006,220.8500) -- (55.7509,220.9040) -- (55.6597,221.1550) -- cycle(55.7830,220.3480) -- (55.8742,220.0980) -- (56.0245,220.1530) -- (55.9333,220.4030) -- cycle(56.0566,219.5970) -- (56.1478,219.3460) -- (56.2982,219.4010) -- (56.2069,219.6510) -- cycle(56.3302,218.8450) -- (56.4214,218.5940) -- (56.5718,218.6490) -- (56.4806,218.9000) -- cycle(56.6038,218.0930) -- (56.6950,217.8430) -- (56.8454,217.8970) -- (56.7542,218.1480) -- cycle(56.8774,217.3410) -- (56.9686,217.0910) -- (57.1190,217.1460) -- (57.0278,217.3960) -- cycle(57.1511,216.5900) -- (57.2422,216.3390) -- (57.3926,216.3940) -- (57.3014,216.6440) -- cycle(57.4247,215.8380) -- (57.5159,215.5870) -- (57.6662,215.6420) -- (57.5750,215.8930) -- cycle(57.6983,215.0860) -- (57.7895,214.8360) -- (57.9398,214.8900) -- (57.8486,215.1410) -- cycle(57.9719,214.3340) -- (58.0631,214.0840) -- (58.2134,214.1390) -- (58.1223,214.3890) -- cycle(58.2455,213.5830) -- (58.3367,213.3320) -- (58.4871,213.3870) -- (58.3959,213.6370) -- cycle(58.5191,212.8310) -- (58.6103,212.5800) -- (58.7607,212.6350) -- (58.6695,212.8860) -- cycle(58.7927,212.0790) -- (58.8839,211.8290) -- (59.0343,211.8830) -- (58.9431,212.1340) -- cycle(59.0663,211.3270) -- (59.1575,211.0770) -- (59.3079,211.1320) -- (59.2167,211.3820) -- cycle(59.3400,210.5760) -- (59.4312,210.3250) -- (59.5815,210.3800) -- (59.4903,210.6300) -- cycle(59.6136,209.8240) -- (59.7048,209.5730) -- (59.8551,209.6280) -- (59.7639,209.8790) -- cycle(59.8872,209.0720) -- (59.9784,208.8220) -- (60.1287,208.8760) -- (60.0375,209.1270) -- cycle(60.1608,208.3200) -- (60.2520,208.0700) -- (60.4023,208.1250) -- (60.3112,208.3750) -- cycle(60.4344,207.5690) -- (60.5256,207.3180) -- (60.6760,207.3730) -- (60.5848,207.6230) -- cycle(60.7080,206.8170) -- (60.7992,206.5660) -- (60.9496,206.6210) -- (60.8584,206.8720) -- cycle(60.9816,206.0650) -- (61.0728,205.8150) -- (61.2232,205.8690) -- (61.1320,206.1200) -- cycle(61.2552,205.3130) -- (61.3464,205.0630) -- (61.4968,205.1180) -- (61.4056,205.3680) -- cycle(61.5289,204.5620) -- (61.6201,204.3110) -- (61.7704,204.3660) -- (61.6792,204.6160) -- cycle(61.8025,203.8100) -- (61.8937,203.5590) -- (62.0440,203.6140) -- (61.9528,203.8650) -- cycle(62.0761,203.0580) -- (62.1673,202.8080) -- (62.3176,202.8620) -- (62.2264,203.1130) -- cycle(62.3497,202.3060) -- (62.4409,202.0560) -- (62.5912,202.1110) -- (62.5001,202.3610) -- cycle(62.6233,201.5550) -- (62.7145,201.3040) -- (62.8649,201.3590) -- (62.7737,201.6090) -- cycle(62.8969,200.8030) -- (62.9881,200.5520) -- (63.1385,200.6070) -- (63.0473,200.8580) -- cycle(63.1705,200.0510) -- (63.2617,199.8010) -- (63.4121,199.8550) -- (63.3209,200.1060) -- cycle(63.4442,199.2990) -- (63.5353,199.0490) -- (63.6857,199.1040) -- (63.5945,199.3540) -- cycle(63.7178,198.5480) -- (63.8090,198.2970) -- (63.9593,198.3520) -- (63.8681,198.6020) -- cycle(63.9914,197.7960) -- (64.0826,197.5450) -- (64.2329,197.6000) -- (64.1417,197.8510) -- cycle(64.2650,197.0440) -- (64.3562,196.7940) -- (64.5065,196.8480) -- (64.4153,197.0990) -- cycle(64.5386,196.2920) -- (64.6298,196.0420) -- (64.7802,196.0960) -- (64.6890,196.3470) -- cycle(64.8122,195.5410) -- (64.9034,195.2900) -- (65.0538,195.3450) -- (64.9626,195.5950) -- cycle(65.0858,194.7890) -- (65.1770,194.5380) -- (65.3274,194.5930) -- (65.2362,194.8440) -- cycle(65.3594,194.0370) -- (65.4507,193.7870) -- (65.6010,193.8410) -- (65.5098,194.0920) -- cycle(65.6331,193.2850) -- (65.7243,193.0350) -- (65.8746,193.0900) -- (65.7834,193.3400) -- cycle(65.9067,192.5340) -- (65.9979,192.2830) -- (66.1482,192.3380) -- (66.0570,192.5880) -- cycle(66.1803,191.7820) -- (66.2715,191.5310) -- (66.4218,191.5860) -- (66.3307,191.8370) -- cycle(66.4539,191.0300) -- (66.5451,190.7800) -- (66.6955,190.8340) -- (66.6042,191.0850) -- cycle(66.7275,190.2780) -- (66.8187,190.0280) -- (66.9691,190.0820) -- (66.8779,190.3330) -- cycle(67.0011,189.5270) -- (67.0923,189.2760) -- (67.2427,189.3310) -- (67.1515,189.5810) -- cycle(67.2747,188.7750) -- (67.3659,188.5240) -- (67.5163,188.5790) -- (67.4251,188.8300) -- cycle(67.5483,188.0230) -- (67.6395,187.7720) -- (67.7899,187.8270) -- (67.6987,188.0780) -- cycle(67.8220,187.2710) -- (67.9131,187.0210) -- (68.0635,187.0750) -- (67.9724,187.3260) -- cycle(68.0956,186.5200) -- (68.1868,186.2690) -- (68.3372,186.3240) -- (68.2459,186.5740) -- cycle(68.3692,185.7680) -- (68.4604,185.5170) -- (68.6108,185.5720) -- (68.5196,185.8230) -- cycle(68.6428,185.0160) -- (68.7340,184.7660) -- (68.8844,184.8200) -- (68.7932,185.0710) -- cycle(68.9164,184.2640) -- (69.0076,184.0140) -- (69.1580,184.0680) -- (69.0668,184.3190) -- cycle(69.1900,183.5130) -- (69.2812,183.2620) -- (69.4316,183.3170) -- (69.3404,183.5670) -- cycle(69.4636,182.7610) -- (69.5548,182.5100) -- (69.7052,182.5650) -- (69.6140,182.8160) -- cycle(69.7372,182.0090) -- (69.8285,181.7590) -- (69.9789,181.8130) -- (69.8876,182.0640) -- cycle(70.0108,181.2570) -- (70.1021,181.0070) -- (70.2524,181.0610) -- (70.1612,181.3120) -- cycle(70.2845,180.5060) -- (70.3757,180.2550) -- (70.5261,180.3100) -- (70.4349,180.5600) -- cycle(70.5581,179.7540) -- (70.6493,179.5030) -- (70.7997,179.5580) -- (70.7085,179.8090) -- cycle(70.8317,179.0020) -- (70.9229,178.7510) -- (71.0733,178.8060) -- (70.9821,179.0570) -- cycle(71.1053,178.2500) -- (71.1965,178.0000) -- (71.3469,178.0540) -- (71.2557,178.3050) -- cycle(71.3789,177.4990) -- (71.4701,177.2480) -- (71.6205,177.3030) -- (71.5293,177.5530) -- cycle(71.6525,176.7470) -- (71.7437,176.4960) -- (71.8941,176.5510) -- (71.8029,176.8010) -- cycle(71.9261,175.9950) -- (72.0173,175.7440) -- (72.1677,175.7990) -- (72.0765,176.0500) -- cycle(72.1998,175.2430) -- (72.2910,174.9930) -- (72.4413,175.0470) -- (72.3502,175.2980) -- cycle(72.4734,174.4920) -- (72.5646,174.2410) -- (72.7150,174.2960) -- (72.6237,174.5460) -- cycle(72.7470,173.7400) -- (72.8382,173.4890) -- (72.9886,173.5440) -- (72.8974,173.7940) -- cycle(73.0206,172.9880) -- (73.1118,172.7370) -- (73.2622,172.7920) -- (73.1710,173.0430) -- cycle(73.2942,172.2360) -- (73.3854,171.9860) -- (73.5358,172.0400) -- (73.4446,172.2910) -- cycle(73.5678,171.4850) -- (73.6590,171.2340) -- (73.8094,171.2890) -- (73.7182,171.5390) -- cycle(73.8414,170.7330) -- (73.9326,170.4820) -- (74.0830,170.5370) -- (73.9918,170.7880) -- cycle(74.1151,169.9810) -- (74.2063,169.7300) -- (74.3567,169.7850) -- (74.2654,170.0360) -- cycle(74.3886,169.2290) -- (74.4799,168.9790) -- (74.6302,169.0330) -- (74.5390,169.2840) -- cycle(74.6623,168.4780) -- (74.7535,168.2270) -- (74.9039,168.2820) -- (74.8127,168.5320) -- cycle(74.9359,167.7260) -- (75.0271,167.4750) -- (75.1775,167.5300) -- (75.0863,167.7800) -- cycle(75.2095,166.9740) -- (75.3007,166.7230) -- (75.4511,166.7780) -- (75.3599,167.0290) -- cycle(75.4831,166.2220) -- (75.5743,165.9720) -- (75.7247,166.0260) -- (75.6335,166.2770) -- cycle(75.7567,165.4700) -- (75.8479,165.2200) -- (75.9983,165.2750) -- (75.9071,165.5250) -- cycle(76.0303,164.7190) -- (76.1216,164.4680) -- (76.2719,164.5230) -- (76.1807,164.7730) -- cycle(76.3040,163.9670) -- (76.3951,163.7160) -- (76.5455,163.7710) -- (76.4543,164.0220) -- cycle(76.5776,163.2150) -- (76.6688,162.9650) -- (76.8192,163.0190) -- (76.7280,163.2700) -- cycle(76.8512,162.4640) -- (76.9424,162.2130) -- (77.0928,162.2680) -- (77.0016,162.5180) -- cycle(77.1248,161.7120) -- (77.2160,161.4610) -- (77.3664,161.5160) -- (77.2752,161.7660) -- cycle(77.3984,160.9600) -- (77.4896,160.7090) -- (77.6400,160.7640) -- (77.5488,161.0150) -- cycle(77.6720,160.2080) -- (77.7632,159.9580) -- (77.9136,160.0120) -- (77.8224,160.2630) -- cycle(77.9456,159.4570) -- (78.0368,159.2060) -- (78.1872,159.2610) -- (78.0960,159.5110) -- cycle(78.2192,158.7050) -- (78.3104,158.4540) -- (78.4608,158.5090) -- (78.3696,158.7590) -- cycle(78.4929,157.9530) -- (78.5841,157.7020) -- (78.7345,157.7570) -- (78.6432,158.0080) -- cycle(78.7665,157.2010) -- (78.8577,156.9510) -- (79.0081,157.0050) -- (78.9169,157.2560) -- cycle(79.0401,156.4490) -- (79.1313,156.1990) -- (79.2817,156.2540) -- (79.1905,156.5040) -- cycle(79.3137,155.6980) -- (79.4049,155.4470) -- (79.5553,155.5020) -- (79.4641,155.7520) -- cycle(79.5873,154.9460) -- (79.6785,154.6950) -- (79.8289,154.7500) -- (79.7377,155.0010) -- cycle(29.5161,292.5170) -- (29.6073,292.2660) -- (29.7577,292.3210) -- (29.6665,292.5720) -- cycle;



  \path[fill=c7e7e7e,line join=round,line width=0.256pt] (55.4310,291.7500) -- (55.5222,291.5000) -- (55.6725,291.5550) -- (55.5813,291.8050) -- cycle(55.7046,290.9990) -- (55.7958,290.7480) -- (55.9461,290.8030) -- (55.8549,291.0530) -- cycle(55.9782,290.2470) -- (56.0694,289.9960) -- (56.2198,290.0510) -- (56.1285,290.3020) -- cycle(56.2518,289.4950) -- (56.3430,289.2450) -- (56.4933,289.2990) -- (56.4022,289.5500) -- cycle(56.5254,288.7430) -- (56.6166,288.4930) -- (56.7670,288.5480) -- (56.6758,288.7980) -- cycle(56.7990,287.9920) -- (56.8903,287.7410) -- (57.0406,287.7960) -- (56.9494,288.0460) -- cycle(57.0727,287.2400) -- (57.1638,286.9890) -- (57.3142,287.0440) -- (57.2230,287.2950) -- cycle(57.3463,286.4880) -- (57.4375,286.2380) -- (57.5878,286.2920) -- (57.4966,286.5430) -- cycle(57.6199,285.7360) -- (57.7111,285.4860) -- (57.8615,285.5410) -- (57.7702,285.7910) -- cycle(57.8935,284.9850) -- (57.9847,284.7340) -- (58.1350,284.7890) -- (58.0438,285.0390) -- cycle(58.1671,284.2330) -- (58.2583,283.9820) -- (58.4087,284.0370) -- (58.3174,284.2880) -- cycle(58.4407,283.4810) -- (58.5319,283.2310) -- (58.6823,283.2850) -- (58.5911,283.5360) -- cycle(58.7143,282.7290) -- (58.8055,282.4790) -- (58.9559,282.5340) -- (58.8647,282.7840) -- cycle(58.9879,281.9780) -- (59.0792,281.7270) -- (59.2295,281.7820) -- (59.1383,282.0320) -- cycle(59.2616,281.2260) -- (59.3528,280.9750) -- (59.5031,281.0300) -- (59.4119,281.2810) -- cycle(59.5352,280.4740) -- (59.6264,280.2240) -- (59.7767,280.2780) -- (59.6855,280.5290) -- cycle(59.8088,279.7230) -- (59.9000,279.4720) -- (60.0504,279.5270) -- (59.9591,279.7770) -- cycle(60.0824,278.9710) -- (60.1736,278.7200) -- (60.3239,278.7750) -- (60.2328,279.0250) -- cycle(60.3560,278.2190) -- (60.4472,277.9680) -- (60.5976,278.0230) -- (60.5063,278.2740) -- cycle(60.6296,277.4670) -- (60.7208,277.2170) -- (60.8712,277.2710) -- (60.7800,277.5220) -- cycle(60.9033,276.7150) -- (60.9944,276.4650) -- (61.1448,276.5200) -- (61.0536,276.7700) -- cycle(61.1768,275.9640) -- (61.2681,275.7130) -- (61.4184,275.7680) -- (61.3272,276.0180) -- cycle(61.4505,275.2120) -- (61.5417,274.9610) -- (61.6920,275.0160) -- (61.6008,275.2670) -- cycle(61.7241,274.4600) -- (61.8153,274.2100) -- (61.9656,274.2640) -- (61.8744,274.5150) -- cycle(61.9977,273.7080) -- (62.0889,273.4580) -- (62.2393,273.5130) -- (62.1480,273.7630) -- cycle(62.2713,272.9570) -- (62.3625,272.7060) -- (62.5129,272.7610) -- (62.4217,273.0110) -- cycle(62.5449,272.2050) -- (62.6361,271.9540) -- (62.7865,272.0090) -- (62.6953,272.2600) -- cycle(62.8185,271.4530) -- (62.9097,271.2030) -- (63.0601,271.2570) -- (62.9689,271.5080) -- cycle(63.0922,270.7010) -- (63.1833,270.4510) -- (63.3337,270.5050) -- (63.2425,270.7560) -- cycle(63.3658,269.9500) -- (63.4570,269.6990) -- (63.6073,269.7540) -- (63.5161,270.0040) -- cycle(63.6394,269.1980) -- (63.7306,268.9470) -- (63.8809,269.0020) -- (63.7897,269.2530) -- cycle(63.9130,268.4460) -- (64.0042,268.1960) -- (64.1545,268.2500) -- (64.0634,268.5010) -- cycle(64.1866,267.6940) -- (64.2778,267.4440) -- (64.4282,267.4990) -- (64.3369,267.7490) -- cycle(64.4602,266.9430) -- (64.5514,266.6920) -- (64.7018,266.7470) -- (64.6106,266.9970) -- cycle(64.7338,266.1910) -- (64.8250,265.9400) -- (64.9754,265.9950) -- (64.8842,266.2460) -- cycle(65.0074,265.4390) -- (65.0986,265.1890) -- (65.2490,265.2430) -- (65.1578,265.4940) -- cycle(65.2811,264.6870) -- (65.3723,264.4370) -- (65.5226,264.4920) -- (65.4314,264.7420) -- cycle(65.5547,263.9360) -- (65.6459,263.6850) -- (65.7962,263.7400) -- (65.7050,263.9900) -- cycle(65.8283,263.1840) -- (65.9195,262.9330) -- (66.0698,262.9880) -- (65.9786,263.2390) -- cycle(66.1019,262.4320) -- (66.1931,262.1820) -- (66.3435,262.2360) -- (66.2523,262.4870) -- cycle(66.3755,261.6800) -- (66.4667,261.4300) -- (66.6171,261.4840) -- (66.5259,261.7350) -- cycle(66.6491,260.9290) -- (66.7403,260.6780) -- (66.8907,260.7330) -- (66.7995,260.9830) -- cycle(66.9227,260.1770) -- (67.0139,259.9260) -- (67.1643,259.9810) -- (67.0731,260.2320) -- cycle(67.1964,259.4250) -- (67.2875,259.1750) -- (67.4379,259.2290) -- (67.3467,259.4800) -- cycle(67.4699,258.6730) -- (67.5611,258.4230) -- (67.7115,258.4780) -- (67.6203,258.7280) -- cycle(67.7435,257.9220) -- (67.8348,257.6710) -- (67.9852,257.7260) -- (67.8939,257.9760) -- cycle(68.0172,257.1700) -- (68.1084,256.9190) -- (68.2588,256.9740) -- (68.1676,257.2250) -- cycle(68.2908,256.4180) -- (68.3820,256.1680) -- (68.5323,256.2220) -- (68.4412,256.4730) -- cycle(68.5644,255.6660) -- (68.6556,255.4160) -- (68.8060,255.4700) -- (68.7148,255.7210) -- cycle(68.8380,254.9150) -- (68.9292,254.6640) -- (69.0796,254.7190) -- (68.9884,254.9690) -- cycle(69.1116,254.1630) -- (69.2028,253.9120) -- (69.3532,253.9670) -- (69.2620,254.2170) -- cycle(69.3852,253.4110) -- (69.4764,253.1600) -- (69.6268,253.2150) -- (69.5356,253.4660) -- cycle(69.6589,252.6590) -- (69.7501,252.4090) -- (69.9005,252.4630) -- (69.8093,252.7140) -- cycle(69.9325,251.9080) -- (70.0237,251.6570) -- (70.1740,251.7120) -- (70.0829,251.9620) -- cycle(70.2061,251.1560) -- (70.2973,250.9050) -- (70.4477,250.9600) -- (70.3564,251.2110) -- cycle(70.4797,250.4040) -- (70.5709,250.1540) -- (70.7213,250.2080) -- (70.6301,250.4590) -- cycle(70.7533,249.6520) -- (70.8445,249.4020) -- (70.9949,249.4560) -- (70.9037,249.7070) -- cycle(71.0269,248.9010) -- (71.1181,248.6500) -- (71.2685,248.7050) -- (71.1773,248.9550) -- cycle(71.3005,248.1490) -- (71.3917,247.8980) -- (71.5421,247.9530) -- (71.4509,248.2030) -- cycle(71.5742,247.3970) -- (71.6653,247.1460) -- (71.8157,247.2010) -- (71.7245,247.4520) -- cycle(71.8477,246.6450) -- (71.9390,246.3950) -- (72.0894,246.4490) -- (71.9981,246.7000) -- cycle(72.1213,245.8940) -- (72.2126,245.6430) -- (72.3630,245.6980) -- (72.2717,245.9480) -- cycle(72.3950,245.1420) -- (72.4862,244.8910) -- (72.6366,244.9460) -- (72.5454,245.1970) -- cycle(72.6686,244.3900) -- (72.7598,244.1390) -- (72.9102,244.1940) -- (72.8190,244.4450) -- cycle(72.9422,243.6380) -- (73.0334,243.3880) -- (73.1838,243.4420) -- (73.0926,243.6930) -- cycle(73.2158,242.8870) -- (73.3070,242.6360) -- (73.4574,242.6910) -- (73.3662,242.9410) -- cycle(73.4894,242.1350) -- (73.5806,241.8840) -- (73.7310,241.9390) -- (73.6398,242.1900) -- cycle(73.7630,241.3830) -- (73.8542,241.1320) -- (74.0046,241.1870) -- (73.9134,241.4380) -- cycle(74.0367,240.6310) -- (74.1279,240.3810) -- (74.2783,240.4350) -- (74.1871,240.6860) -- cycle(74.3103,239.8800) -- (74.4015,239.6290) -- (74.5518,239.6840) -- (74.4607,239.9340) -- cycle(74.5839,239.1280) -- (74.6751,238.8770) -- (74.8255,238.9320) -- (74.7343,239.1820) -- cycle(74.8575,238.3760) -- (74.9487,238.1250) -- (75.0991,238.1800) -- (75.0079,238.4310) -- cycle(75.1311,237.6240) -- (75.2223,237.3740) -- (75.3727,237.4280) -- (75.2815,237.6790) -- cycle(75.4047,236.8730) -- (75.4959,236.6220) -- (75.6463,236.6770) -- (75.5551,236.9270) -- cycle(75.6783,236.1210) -- (75.7695,235.8700) -- (75.9199,235.9250) -- (75.8287,236.1750) -- cycle(75.9520,235.3690) -- (76.0432,235.1180) -- (76.1935,235.1730) -- (76.1024,235.4240) -- cycle(76.2256,234.6170) -- (76.3168,234.3670) -- (76.4672,234.4210) -- (76.3759,234.6720) -- cycle(76.4992,233.8660) -- (76.5904,233.6150) -- (76.7408,233.6700) -- (76.6496,233.9200) -- cycle(76.7728,233.1140) -- (76.8640,232.8630) -- (77.0144,232.9180) -- (76.9232,233.1680) -- cycle(77.0464,232.3620) -- (77.1376,232.1110) -- (77.2880,232.1660) -- (77.1968,232.4170) -- cycle(77.3200,231.6100) -- (77.4112,231.3600) -- (77.5616,231.4140) -- (77.4704,231.6650) -- cycle(77.5936,230.8580) -- (77.6848,230.6080) -- (77.8352,230.6630) -- (77.7440,230.9130) -- cycle(77.8672,230.1070) -- (77.9585,229.8560) -- (78.1089,229.9110) -- (78.0176,230.1610) -- cycle(78.1409,229.3550) -- (78.2321,229.1040) -- (78.3824,229.1590) -- (78.2913,229.4100) -- cycle(78.4145,228.6030) -- (78.5057,228.3530) -- (78.6561,228.4070) -- (78.5649,228.6580) -- cycle(78.6881,227.8510) -- (78.7793,227.6010) -- (78.9297,227.6560) -- (78.8385,227.9060) -- cycle(78.9617,227.1000) -- (79.0529,226.8490) -- (79.2033,226.9040) -- (79.1121,227.1540) -- cycle(79.2353,226.3480) -- (79.3265,226.0970) -- (79.4769,226.1520) -- (79.3857,226.4030) -- cycle(79.5089,225.5960) -- (79.6002,225.3460) -- (79.7505,225.4000) -- (79.6593,225.6510) -- cycle(79.7825,224.8450) -- (79.8737,224.5940) -- (80.0241,224.6490) -- (79.9329,224.8990) -- cycle(80.0562,224.0930) -- (80.1473,223.8420) -- (80.2977,223.8970) -- (80.2065,224.1470) -- cycle(80.3298,223.3410) -- (80.4210,223.0900) -- (80.5714,223.1450) -- (80.4802,223.3960) -- cycle(80.6034,222.5890) -- (80.6946,222.3390) -- (80.8450,222.3930) -- (80.7538,222.6440) -- cycle(80.8770,221.8370) -- (80.9682,221.5870) -- (81.1186,221.6420) -- (81.0274,221.8920) -- cycle(81.1506,221.0860) -- (81.2418,220.8350) -- (81.3922,220.8900) -- (81.3010,221.1400) -- cycle(81.4242,220.3340) -- (81.5154,220.0830) -- (81.6658,220.1380) -- (81.5746,220.3890) -- cycle(81.6978,219.5820) -- (81.7890,219.3320) -- (81.9394,219.3860) -- (81.8482,219.6370) -- cycle(81.9714,218.8300) -- (82.0626,218.5800) -- (82.2130,218.6350) -- (82.1218,218.8850) -- cycle(82.2451,218.0790) -- (82.3363,217.8280) -- (82.4867,217.8830) -- (82.3954,218.1330) -- cycle(82.5187,217.3270) -- (82.6099,217.0760) -- (82.7603,217.1310) -- (82.6691,217.3820) -- cycle(82.7923,216.5750) -- (82.8835,216.3250) -- (83.0339,216.3790) -- (82.9427,216.6300) -- cycle(83.0659,215.8230) -- (83.1571,215.5730) -- (83.3075,215.6280) -- (83.2163,215.8780) -- cycle(83.3395,215.0720) -- (83.4307,214.8210) -- (83.5811,214.8760) -- (83.4899,215.1260) -- cycle(83.6131,214.3200) -- (83.7043,214.0690) -- (83.8547,214.1240) -- (83.7635,214.3750) -- cycle(83.8867,213.5680) -- (83.9780,213.3180) -- (84.1284,213.3720) -- (84.0371,213.6230) -- cycle(84.1604,212.8160) -- (84.2516,212.5660) -- (84.4019,212.6210) -- (84.3108,212.8710) -- cycle(84.4340,212.0650) -- (84.5251,211.8140) -- (84.6755,211.8690) -- (84.5844,212.1190) -- cycle(84.7076,211.3130) -- (84.7988,211.0620) -- (84.9492,211.1170) -- (84.8580,211.3680) -- cycle(84.9812,210.5610) -- (85.0724,210.3110) -- (85.2228,210.3650) -- (85.1316,210.6160) -- cycle(85.2548,209.8090) -- (85.3460,209.5590) -- (85.4964,209.6140) -- (85.4052,209.8640) -- cycle(85.5284,209.0580) -- (85.6196,208.8070) -- (85.7700,208.8620) -- (85.6788,209.1120) -- cycle(85.8021,208.3060) -- (85.8932,208.0550) -- (86.0436,208.1100) -- (85.9525,208.3610) -- cycle(86.0757,207.5540) -- (86.1668,207.3040) -- (86.3172,207.3580) -- (86.2260,207.6090) -- cycle(86.3492,206.8020) -- (86.4405,206.5520) -- (86.5909,206.6070) -- (86.4996,206.8570) -- cycle(86.6229,206.0510) -- (86.7141,205.8000) -- (86.8645,205.8550) -- (86.7733,206.1050) -- cycle(86.8965,205.2990) -- (86.9877,205.0480) -- (87.1381,205.1030) -- (87.0469,205.3540) -- cycle(87.1701,204.5470) -- (87.2613,204.2970) -- (87.4117,204.3510) -- (87.3205,204.6020) -- cycle(87.4437,203.7950) -- (87.5349,203.5450) -- (87.6853,203.6000) -- (87.5941,203.8500) -- cycle(87.7173,203.0440) -- (87.8085,202.7930) -- (87.9589,202.8480) -- (87.8677,203.0980) -- cycle(87.9909,202.2920) -- (88.0822,202.0410) -- (88.2325,202.0960) -- (88.1413,202.3470) -- cycle(88.2645,201.5400) -- (88.3558,201.2900) -- (88.5062,201.3440) -- (88.4149,201.5950) -- cycle(88.5382,200.7880) -- (88.6294,200.5380) -- (88.7798,200.5920) -- (88.6886,200.8430) -- cycle(88.8118,200.0370) -- (88.9030,199.7860) -- (89.0533,199.8410) -- (88.9622,200.0910) -- cycle(89.0854,199.2850) -- (89.1766,199.0340) -- (89.3270,199.0890) -- (89.2358,199.3400) -- cycle(89.3590,198.5330) -- (89.4502,198.2830) -- (89.6006,198.3370) -- (89.5094,198.5880) -- cycle(89.6326,197.7810) -- (89.7238,197.5310) -- (89.8742,197.5850) -- (89.7830,197.8360) -- cycle(89.9062,197.0300) -- (89.9974,196.7790) -- (90.1478,196.8340) -- (90.0566,197.0840) -- cycle(90.1799,196.2780) -- (90.2711,196.0270) -- (90.4214,196.0820) -- (90.3303,196.3330) -- cycle(90.4535,195.5260) -- (90.5446,195.2760) -- (90.6950,195.3300) -- (90.6039,195.5810) -- cycle(90.7271,194.7740) -- (90.8183,194.5240) -- (90.9687,194.5780) -- (90.8774,194.8290) -- cycle(91.0007,194.0230) -- (91.0919,193.7720) -- (91.2423,193.8270) -- (91.1511,194.0770) -- cycle(91.2743,193.2710) -- (91.3655,193.0200) -- (91.5159,193.0750) -- (91.4247,193.3260) -- cycle(91.5479,192.5190) -- (91.6391,192.2690) -- (91.7895,192.3230) -- (91.6983,192.5740) -- cycle(91.8215,191.7670) -- (91.9127,191.5170) -- (92.0631,191.5710) -- (91.9719,191.8220) -- cycle(92.0952,191.0160) -- (92.1863,190.7650) -- (92.3367,190.8200) -- (92.2455,191.0700) -- cycle(92.3687,190.2640) -- (92.4600,190.0130) -- (92.6104,190.0680) -- (92.5191,190.3190) -- cycle(92.6424,189.5120) -- (92.7336,189.2620) -- (92.8840,189.3160) -- (92.7928,189.5670) -- cycle(92.9160,188.7600) -- (93.0072,188.5100) -- (93.1576,188.5640) -- (93.0664,188.8150) -- cycle(93.1896,188.0090) -- (93.2808,187.7580) -- (93.4312,187.8130) -- (93.3400,188.0630) -- cycle(93.4632,187.2570) -- (93.5544,187.0060) -- (93.7048,187.0610) -- (93.6136,187.3120) -- cycle(93.7368,186.5050) -- (93.8280,186.2540) -- (93.9784,186.3090) -- (93.8872,186.5600) -- cycle(94.0104,185.7530) -- (94.1017,185.5030) -- (94.2520,185.5570) -- (94.1608,185.8080) -- cycle(94.2841,185.0020) -- (94.3752,184.7510) -- (94.5256,184.8060) -- (94.4344,185.0560) -- cycle(94.5577,184.2500) -- (94.6489,183.9990) -- (94.7993,184.0540) -- (94.7081,184.3040) -- cycle(94.8313,183.4980) -- (94.9225,183.2470) -- (95.0729,183.3020) -- (94.9817,183.5530) -- cycle(95.1049,182.7460) -- (95.1961,182.4960) -- (95.3465,182.5500) -- (95.2552,182.8010) -- cycle(95.3785,181.9950) -- (95.4697,181.7440) -- (95.6201,181.7990) -- (95.5289,182.0490) -- cycle(95.6521,181.2430) -- (95.7433,180.9920) -- (95.8937,181.0470) -- (95.8025,181.2980) -- cycle(95.9257,180.4910) -- (96.0169,180.2410) -- (96.1673,180.2950) -- (96.0761,180.5460) -- cycle(96.1993,179.7390) -- (96.2905,179.4890) -- (96.4409,179.5430) -- (96.3497,179.7940) -- cycle(96.4730,178.9880) -- (96.5642,178.7370) -- (96.7146,178.7920) -- (96.6234,179.0420) -- cycle(96.7466,178.2360) -- (96.8378,177.9850) -- (96.9882,178.0400) -- (96.8969,178.2900) -- cycle(97.0202,177.4840) -- (97.1114,177.2330) -- (97.2618,177.2880) -- (97.1706,177.5390) -- cycle(97.2938,176.7320) -- (97.3850,176.4820) -- (97.5354,176.5360) -- (97.4442,176.7870) -- cycle(97.5674,175.9810) -- (97.6586,175.7300) -- (97.8090,175.7850) -- (97.7178,176.0350) -- cycle(97.8410,175.2290) -- (97.9322,174.9780) -- (98.0826,175.0330) -- (97.9914,175.2830) -- cycle(98.1146,174.4770) -- (98.2058,174.2260) -- (98.3562,174.2810) -- (98.2650,174.5320) -- cycle(98.3882,173.7250) -- (98.4795,173.4750) -- (98.6299,173.5290) -- (98.5386,173.7800) -- cycle(98.6619,172.9740) -- (98.7531,172.7230) -- (98.9034,172.7780) -- (98.8123,173.0280) -- cycle(98.9355,172.2220) -- (99.0267,171.9710) -- (99.1771,172.0260) -- (99.0859,172.2760) -- cycle(99.2091,171.4700) -- (99.3003,171.2190) -- (99.4507,171.2740) -- (99.3595,171.5250) -- cycle(99.4827,170.7180) -- (99.5739,170.4680) -- (99.7243,170.5220) -- (99.6331,170.7730) -- cycle(99.7563,169.9670) -- (99.8475,169.7160) -- (99.9979,169.7710) -- (99.9067,170.0210) -- cycle(100.0300,169.2150) -- (100.1210,168.9640) -- (100.2720,169.0190) -- (100.1800,169.2690) -- cycle(100.3040,168.4630) -- (100.3950,168.2120) -- (100.5450,168.2670) -- (100.4540,168.5180) -- cycle(100.5770,167.7110) -- (100.6680,167.4610) -- (100.8190,167.5150) -- (100.7280,167.7660) -- cycle(100.8510,166.9600) -- (100.9420,166.7090) -- (101.0920,166.7640) -- (101.0010,167.0140) -- cycle(101.1240,166.2080) -- (101.2160,165.9570) -- (101.3660,166.0120) -- (101.2750,166.2620) -- cycle(101.3980,165.4560) -- (101.4890,165.2050) -- (101.6400,165.2600) -- (101.5480,165.5110) -- cycle(101.6720,164.7040) -- (101.7630,164.4540) -- (101.9130,164.5080) -- (101.8220,164.7590) -- cycle(101.9450,163.9520) -- (102.0360,163.7020) -- (102.1870,163.7570) -- (102.0960,164.0070) -- cycle(102.2190,163.2010) -- (102.3100,162.9500) -- (102.4600,163.0050) -- (102.3690,163.2550) -- cycle(102.4920,162.4490) -- (102.5840,162.1980) -- (102.7340,162.2530) -- (102.6430,162.5040) -- cycle(102.7660,161.6970) -- (102.8570,161.4470) -- (103.0080,161.5010) -- (102.9160,161.7520) -- cycle(103.0400,160.9450) -- (103.1310,160.6950) -- (103.2810,160.7500) -- (103.1900,161.0000) -- cycle(103.3130,160.1940) -- (103.4050,159.9430) -- (103.5550,159.9980) -- (103.4640,160.2480) -- cycle(103.5870,159.4420) -- (103.6780,159.1910) -- (103.8280,159.2460) -- (103.7370,159.4970) -- cycle(103.8610,158.6900) -- (103.9520,158.4400) -- (104.1020,158.4940) -- (104.0110,158.7450) -- cycle(104.1340,157.9390) -- (104.2250,157.6880) -- (104.3760,157.7430) -- (104.2850,157.9930) -- cycle(104.4080,157.1870) -- (104.4990,156.9360) -- (104.6490,156.9910) -- (104.5580,157.2410) -- cycle(104.6810,156.4350) -- (104.7730,156.1840) -- (104.9230,156.2390) -- (104.8320,156.4900) -- cycle(104.9550,155.6830) -- (105.0460,155.4330) -- (105.1970,155.4870) -- (105.1050,155.7380) -- cycle(105.2290,154.9310) -- (105.3200,154.6810) -- (105.4700,154.7360) -- (105.3790,154.9860) -- cycle(55.1573,292.5020) -- (55.2486,292.2520) -- (55.3989,292.3060) -- (55.3077,292.5570) -- cycle;



  \path[fill=c7e7e7e,line join=round,line width=0.256pt] (81.1125,291.7590) -- (81.2037,291.5090) -- (81.3541,291.5630) -- (81.2629,291.8140) -- cycle(81.3862,291.0080) -- (81.4774,290.7570) -- (81.6277,290.8120) -- (81.5366,291.0620) -- cycle(81.6598,290.2560) -- (81.7509,290.0050) -- (81.9013,290.0600) -- (81.8102,290.3110) -- cycle(81.9333,289.5040) -- (82.0246,289.2540) -- (82.1750,289.3080) -- (82.0837,289.5590) -- cycle(82.2070,288.7520) -- (82.2982,288.5020) -- (82.4486,288.5560) -- (82.3574,288.8070) -- cycle(82.4806,288.0010) -- (82.5718,287.7500) -- (82.7222,287.8050) -- (82.6310,288.0550) -- cycle(82.7542,287.2490) -- (82.8454,286.9980) -- (82.9958,287.0530) -- (82.9046,287.3030) -- cycle(83.0278,286.4970) -- (83.1190,286.2460) -- (83.2694,286.3010) -- (83.1782,286.5520) -- cycle(83.3015,285.7450) -- (83.3926,285.4950) -- (83.5430,285.5490) -- (83.4518,285.8000) -- cycle(83.5750,284.9930) -- (83.6663,284.7430) -- (83.8167,284.7980) -- (83.7254,285.0480) -- cycle(83.8487,284.2420) -- (83.9399,283.9910) -- (84.0903,284.0460) -- (83.9991,284.2960) -- cycle(84.1223,283.4900) -- (84.2135,283.2390) -- (84.3639,283.2940) -- (84.2727,283.5450) -- cycle(84.3959,282.7380) -- (84.4871,282.4880) -- (84.6375,282.5420) -- (84.5463,282.7930) -- cycle(84.6695,281.9870) -- (84.7607,281.7360) -- (84.9111,281.7910) -- (84.8199,282.0410) -- cycle(84.9431,281.2350) -- (85.0343,280.9840) -- (85.1847,281.0390) -- (85.0935,281.2900) -- cycle(85.2167,280.4830) -- (85.3080,280.2320) -- (85.4583,280.2870) -- (85.3671,280.5380) -- cycle(85.4903,279.7310) -- (85.5815,279.4810) -- (85.7319,279.5350) -- (85.6407,279.7860) -- cycle(85.7640,278.9800) -- (85.8552,278.7290) -- (86.0056,278.7840) -- (85.9144,279.0340) -- cycle(86.0376,278.2280) -- (86.1288,277.9770) -- (86.2791,278.0320) -- (86.1880,278.2820) -- cycle(86.3112,277.4760) -- (86.4024,277.2250) -- (86.5528,277.2800) -- (86.4615,277.5310) -- cycle(86.5848,276.7240) -- (86.6760,276.4740) -- (86.8264,276.5280) -- (86.7352,276.7790) -- cycle(86.8584,275.9730) -- (86.9496,275.7220) -- (87.1000,275.7770) -- (87.0088,276.0270) -- cycle(87.1320,275.2210) -- (87.2232,274.9700) -- (87.3736,275.0250) -- (87.2824,275.2750) -- cycle(87.4056,274.4690) -- (87.4968,274.2180) -- (87.6472,274.2730) -- (87.5560,274.5240) -- cycle(87.6793,273.7170) -- (87.7704,273.4670) -- (87.9208,273.5210) -- (87.8297,273.7720) -- cycle(87.9529,272.9660) -- (88.0441,272.7150) -- (88.1945,272.7700) -- (88.1032,273.0200) -- cycle(88.2265,272.2140) -- (88.3177,271.9630) -- (88.4681,272.0180) -- (88.3769,272.2680) -- cycle(88.5001,271.4620) -- (88.5913,271.2110) -- (88.7417,271.2660) -- (88.6505,271.5170) -- cycle(88.7737,270.7100) -- (88.8649,270.4600) -- (89.0153,270.5140) -- (88.9241,270.7650) -- cycle(89.0473,269.9580) -- (89.1385,269.7080) -- (89.2889,269.7630) -- (89.1977,270.0130) -- cycle(89.3209,269.2070) -- (89.4121,268.9560) -- (89.5625,269.0110) -- (89.4713,269.2610) -- cycle(89.5945,268.4550) -- (89.6858,268.2040) -- (89.8362,268.2590) -- (89.7449,268.5100) -- cycle(89.8682,267.7030) -- (89.9594,267.4530) -- (90.1097,267.5070) -- (90.0186,267.7580) -- cycle(90.1418,266.9520) -- (90.2330,266.7010) -- (90.3834,266.7560) -- (90.2922,267.0060) -- cycle(90.4154,266.2000) -- (90.5066,265.9490) -- (90.6570,266.0040) -- (90.5658,266.2540) -- cycle(90.6890,265.4480) -- (90.7802,265.1970) -- (90.9306,265.2520) -- (90.8394,265.5030) -- cycle(90.9626,264.6960) -- (91.0538,264.4460) -- (91.2042,264.5000) -- (91.1130,264.7510) -- cycle(91.2362,263.9440) -- (91.3275,263.6940) -- (91.4778,263.7490) -- (91.3866,263.9990) -- cycle(91.5098,263.1930) -- (91.6010,262.9420) -- (91.7514,262.9970) -- (91.6602,263.2470) -- cycle(91.7834,262.4410) -- (91.8746,262.1900) -- (92.0250,262.2450) -- (91.9338,262.4960) -- cycle(92.0571,261.6890) -- (92.1483,261.4390) -- (92.2987,261.4930) -- (92.2075,261.7440) -- cycle(92.3307,260.9370) -- (92.4219,260.6870) -- (92.5723,260.7420) -- (92.4810,260.9920) -- cycle(92.6043,260.1860) -- (92.6955,259.9350) -- (92.8459,259.9900) -- (92.7547,260.2400) -- cycle(92.8779,259.4340) -- (92.9691,259.1830) -- (93.1195,259.2380) -- (93.0283,259.4890) -- cycle(93.1515,258.6820) -- (93.2427,258.4320) -- (93.3931,258.4860) -- (93.3019,258.7370) -- cycle(93.4251,257.9300) -- (93.5163,257.6800) -- (93.6667,257.7350) -- (93.5755,257.9850) -- cycle(93.6987,257.1790) -- (93.7899,256.9280) -- (93.9403,256.9830) -- (93.8491,257.2330) -- cycle(93.9724,256.4270) -- (94.0636,256.1760) -- (94.2140,256.2310) -- (94.1227,256.4820) -- cycle(94.2460,255.6750) -- (94.3372,255.4250) -- (94.4875,255.4790) -- (94.3964,255.7300) -- cycle(94.5196,254.9230) -- (94.6108,254.6730) -- (94.7612,254.7280) -- (94.6700,254.9780) -- cycle(94.7932,254.1720) -- (94.8844,253.9210) -- (95.0348,253.9760) -- (94.9436,254.2260) -- cycle(95.0668,253.4200) -- (95.1580,253.1690) -- (95.3084,253.2240) -- (95.2172,253.4750) -- cycle(95.3404,252.6680) -- (95.4316,252.4180) -- (95.5820,252.4720) -- (95.4908,252.7230) -- cycle(95.6140,251.9160) -- (95.7053,251.6660) -- (95.8557,251.7210) -- (95.7644,251.9710) -- cycle(95.8877,251.1650) -- (95.9789,250.9140) -- (96.1292,250.9690) -- (96.0381,251.2190) -- cycle(96.1613,250.4130) -- (96.2524,250.1620) -- (96.4028,250.2170) -- (96.3116,250.4680) -- cycle(96.4349,249.6610) -- (96.5261,249.4110) -- (96.6765,249.4650) -- (96.5853,249.7160) -- cycle(96.7085,248.9090) -- (96.7997,248.6590) -- (96.9501,248.7130) -- (96.8589,248.9640) -- cycle(96.9821,248.1580) -- (97.0733,247.9070) -- (97.2237,247.9620) -- (97.1325,248.2120) -- cycle(97.2557,247.4060) -- (97.3469,247.1550) -- (97.4973,247.2100) -- (97.4061,247.4610) -- cycle(97.5293,246.6540) -- (97.6205,246.4040) -- (97.7709,246.4580) -- (97.6797,246.7090) -- cycle(97.8029,245.9020) -- (97.8941,245.6520) -- (98.0445,245.7070) -- (97.9533,245.9570) -- cycle(98.0765,245.1510) -- (98.1678,244.9000) -- (98.3182,244.9550) -- (98.2269,245.2050) -- cycle(98.3502,244.3990) -- (98.4414,244.1480) -- (98.5918,244.2030) -- (98.5005,244.4540) -- cycle(98.6238,243.6470) -- (98.7150,243.3970) -- (98.8654,243.4510) -- (98.7742,243.7020) -- cycle(98.8974,242.8950) -- (98.9886,242.6450) -- (99.1390,242.6990) -- (99.0478,242.9500) -- cycle(99.1710,242.1440) -- (99.2622,241.8930) -- (99.4126,241.9480) -- (99.3214,242.1980) -- cycle(99.4446,241.3920) -- (99.5358,241.1410) -- (99.6862,241.1960) -- (99.5950,241.4470) -- cycle(99.7182,240.6400) -- (99.8094,240.3900) -- (99.9598,240.4440) -- (99.8686,240.6950) -- cycle(99.9918,239.8880) -- (100.0830,239.6380) -- (100.2330,239.6920) -- (100.1420,239.9430) -- cycle(100.2650,239.1370) -- (100.3570,238.8860) -- (100.5070,238.9410) -- (100.4160,239.1910) -- cycle(100.5390,238.3850) -- (100.6300,238.1340) -- (100.7810,238.1890) -- (100.6890,238.4400) -- cycle(100.8130,237.6330) -- (100.9040,237.3830) -- (101.0540,237.4370) -- (100.9630,237.6880) -- cycle(101.0860,236.8810) -- (101.1770,236.6310) -- (101.3280,236.6850) -- (101.2370,236.9360) -- cycle(101.3600,236.1300) -- (101.4510,235.8790) -- (101.6020,235.9340) -- (101.5100,236.1840) -- cycle(101.6340,235.3780) -- (101.7250,235.1270) -- (101.8750,235.1820) -- (101.7840,235.4330) -- cycle(101.9070,234.6260) -- (101.9980,234.3760) -- (102.1490,234.4300) -- (102.0580,234.6810) -- cycle(102.1810,233.8740) -- (102.2720,233.6240) -- (102.4220,233.6780) -- (102.3310,233.9290) -- cycle(102.4540,233.1230) -- (102.5460,232.8720) -- (102.6960,232.9270) -- (102.6050,233.1770) -- cycle(102.7280,232.3710) -- (102.8190,232.1200) -- (102.9700,232.1750) -- (102.8780,232.4260) -- cycle(103.0020,231.6190) -- (103.0930,231.3680) -- (103.2430,231.4230) -- (103.1520,231.6740) -- cycle(103.2750,230.8670) -- (103.3660,230.6170) -- (103.5170,230.6710) -- (103.4260,230.9220) -- cycle(103.5490,230.1160) -- (103.6400,229.8650) -- (103.7900,229.9200) -- (103.6990,230.1700) -- cycle(103.8220,229.3640) -- (103.9140,229.1130) -- (104.0640,229.1680) -- (103.9730,229.4180) -- cycle(104.0960,228.6120) -- (104.1870,228.3610) -- (104.3380,228.4160) -- (104.2460,228.6670) -- cycle(104.3700,227.8600) -- (104.4610,227.6100) -- (104.6110,227.6640) -- (104.5200,227.9150) -- cycle(104.6430,227.1090) -- (104.7340,226.8580) -- (104.8850,226.9130) -- (104.7940,227.1630) -- cycle(104.9170,226.3570) -- (105.0080,226.1060) -- (105.1590,226.1610) -- (105.0670,226.4120) -- cycle(105.1900,225.6050) -- (105.2820,225.3550) -- (105.4320,225.4090) -- (105.3410,225.6600) -- cycle(105.4640,224.8530) -- (105.5550,224.6030) -- (105.7060,224.6570) -- (105.6150,224.9080) -- cycle(105.7380,224.1020) -- (105.8290,223.8510) -- (105.9790,223.9060) -- (105.8880,224.1560) -- cycle(106.0110,223.3500) -- (106.1030,223.0990) -- (106.2530,223.1540) -- (106.1620,223.4050) -- cycle(106.2850,222.5980) -- (106.3760,222.3470) -- (106.5270,222.4020) -- (106.4350,222.6530) -- cycle(106.5590,221.8460) -- (106.6500,221.5960) -- (106.8000,221.6500) -- (106.7090,221.9010) -- cycle(106.8320,221.0950) -- (106.9230,220.8440) -- (107.0740,220.8990) -- (106.9830,221.1490) -- cycle(107.1060,220.3430) -- (107.1970,220.0920) -- (107.3470,220.1470) -- (107.2560,220.3970) -- cycle(107.3790,219.5910) -- (107.4710,219.3400) -- (107.6210,219.3950) -- (107.5300,219.6460) -- cycle(107.6530,218.8390) -- (107.7440,218.5890) -- (107.8950,218.6430) -- (107.8030,218.8940) -- cycle(107.9270,218.0880) -- (108.0180,217.8370) -- (108.1680,217.8920) -- (108.0770,218.1420) -- cycle(108.2000,217.3360) -- (108.2910,217.0850) -- (108.4420,217.1400) -- (108.3510,217.3900) -- cycle(108.4740,216.5840) -- (108.5650,216.3330) -- (108.7150,216.3880) -- (108.6240,216.6390) -- cycle(108.7470,215.8320) -- (108.8390,215.5820) -- (108.9890,215.6360) -- (108.8980,215.8870) -- cycle(109.0210,215.0810) -- (109.1120,214.8300) -- (109.2630,214.8850) -- (109.1710,215.1350) -- cycle(109.2950,214.3290) -- (109.3860,214.0780) -- (109.5360,214.1330) -- (109.4450,214.3830) -- cycle(109.5680,213.5770) -- (109.6600,213.3260) -- (109.8100,213.3810) -- (109.7190,213.6320) -- cycle(109.8420,212.8250) -- (109.9330,212.5750) -- (110.0830,212.6290) -- (109.9920,212.8800) -- cycle(110.1160,212.0740) -- (110.2070,211.8230) -- (110.3570,211.8780) -- (110.2660,212.1280) -- cycle(110.3890,211.3220) -- (110.4800,211.0710) -- (110.6310,211.1260) -- (110.5400,211.3760) -- cycle(110.6630,210.5700) -- (110.7540,210.3190) -- (110.9040,210.3740) -- (110.8130,210.6250) -- cycle(110.9360,209.8180) -- (111.0280,209.5680) -- (111.1780,209.6220) -- (111.0870,209.8730) -- cycle(111.2100,209.0660) -- (111.3010,208.8160) -- (111.4520,208.8710) -- (111.3600,209.1210) -- cycle(111.4840,208.3150) -- (111.5750,208.0640) -- (111.7250,208.1190) -- (111.6340,208.3690) -- cycle(111.7570,207.5630) -- (111.8480,207.3120) -- (111.9990,207.3670) -- (111.9080,207.6180) -- cycle(112.0310,206.8110) -- (112.1220,206.5610) -- (112.2720,206.6150) -- (112.1810,206.8660) -- cycle(112.3040,206.0590) -- (112.3960,205.8090) -- (112.5460,205.8640) -- (112.4550,206.1140) -- cycle(112.5780,205.3080) -- (112.6690,205.0570) -- (112.8200,205.1120) -- (112.7280,205.3620) -- cycle(112.8520,204.5560) -- (112.9430,204.3050) -- (113.0930,204.3600) -- (113.0020,204.6110) -- cycle(113.1250,203.8040) -- (113.2160,203.5540) -- (113.3670,203.6080) -- (113.2760,203.8590) -- cycle(113.3990,203.0520) -- (113.4900,202.8020) -- (113.6400,202.8570) -- (113.5490,203.1070) -- cycle(113.6720,202.3010) -- (113.7640,202.0500) -- (113.9140,202.1050) -- (113.8230,202.3550) -- cycle(113.9460,201.5490) -- (114.0370,201.2980) -- (114.1880,201.3530) -- (114.0970,201.6040) -- cycle(114.2200,200.7970) -- (114.3110,200.5470) -- (114.4610,200.6010) -- (114.3700,200.8520) -- cycle(114.4930,200.0450) -- (114.5850,199.7950) -- (114.7350,199.8500) -- (114.6440,200.1000) -- cycle(114.7670,199.2940) -- (114.8580,199.0430) -- (115.0090,199.0980) -- (114.9170,199.3480) -- cycle(115.0410,198.5420) -- (115.1320,198.2910) -- (115.2820,198.3460) -- (115.1910,198.5970) -- cycle(115.3140,197.7900) -- (115.4050,197.5400) -- (115.5560,197.5940) -- (115.4650,197.8450) -- cycle(115.5880,197.0380) -- (115.6790,196.7880) -- (115.8290,196.8430) -- (115.7380,197.0930) -- cycle(115.8610,196.2870) -- (115.9530,196.0360) -- (116.1030,196.0910) -- (116.0120,196.3410) -- cycle(116.1350,195.5350) -- (116.2260,195.2840) -- (116.3770,195.3390) -- (116.2850,195.5900) -- cycle(116.4090,194.7830) -- (116.5000,194.5330) -- (116.6500,194.5870) -- (116.5590,194.8380) -- cycle(116.6820,194.0310) -- (116.7730,193.7810) -- (116.9240,193.8360) -- (116.8330,194.0860) -- cycle(116.9560,193.2800) -- (117.0470,193.0290) -- (117.1970,193.0840) -- (117.1060,193.3340) -- cycle(117.2290,192.5280) -- (117.3210,192.2770) -- (117.4710,192.3320) -- (117.3800,192.5830) -- cycle(117.5030,191.7760) -- (117.5940,191.5260) -- (117.7450,191.5800) -- (117.6540,191.8310) -- cycle(117.7770,191.0240) -- (117.8680,190.7740) -- (118.0180,190.8290) -- (117.9270,191.0790) -- cycle(118.0500,190.2730) -- (118.1420,190.0220) -- (118.2920,190.0770) -- (118.2010,190.3270) -- cycle(118.3240,189.5210) -- (118.4150,189.2700) -- (118.5660,189.3250) -- (118.4740,189.5760) -- cycle(118.5980,188.7690) -- (118.6890,188.5190) -- (118.8390,188.5730) -- (118.7480,188.8240) -- cycle(118.8710,188.0170) -- (118.9620,187.7670) -- (119.1130,187.8220) -- (119.0220,188.0720) -- cycle(119.1450,187.2660) -- (119.2360,187.0150) -- (119.3860,187.0700) -- (119.2950,187.3200) -- cycle(119.4180,186.5140) -- (119.5100,186.2630) -- (119.6600,186.3180) -- (119.5690,186.5690) -- cycle(119.6920,185.7620) -- (119.7830,185.5120) -- (119.9340,185.5660) -- (119.8420,185.8170) -- cycle(119.9660,185.0100) -- (120.0570,184.7600) -- (120.2070,184.8140) -- (120.1160,185.0650) -- cycle(120.2390,184.2590) -- (120.3300,184.0080) -- (120.4810,184.0630) -- (120.3900,184.3130) -- cycle(120.5130,183.5070) -- (120.6040,183.2560) -- (120.7540,183.3110) -- (120.6630,183.5620) -- cycle(120.7860,182.7550) -- (120.8780,182.5050) -- (121.0280,182.5590) -- (120.9370,182.8100) -- cycle(121.0600,182.0030) -- (121.1510,181.7530) -- (121.3020,181.8080) -- (121.2100,182.0580) -- cycle(121.3340,181.2520) -- (121.4250,181.0010) -- (121.5750,181.0560) -- (121.4840,181.3060) -- cycle(121.6070,180.5000) -- (121.6980,180.2490) -- (121.8490,180.3040) -- (121.7580,180.5550) -- cycle(121.8810,179.7480) -- (121.9720,179.4980) -- (122.1220,179.5520) -- (122.0310,179.8030) -- cycle(122.1550,178.9960) -- (122.2460,178.7460) -- (122.3960,178.8000) -- (122.3050,179.0510) -- cycle(122.4280,178.2450) -- (122.5190,177.9940) -- (122.6700,178.0490) -- (122.5790,178.2990) -- cycle(122.7020,177.4930) -- (122.7930,177.2420) -- (122.9430,177.2970) -- (122.8520,177.5480) -- cycle(122.9750,176.7410) -- (123.0670,176.4910) -- (123.2170,176.5450) -- (123.1260,176.7960) -- cycle(123.2490,175.9890) -- (123.3400,175.7390) -- (123.4910,175.7930) -- (123.3990,176.0440) -- cycle(123.5230,175.2380) -- (123.6140,174.9870) -- (123.7640,175.0420) -- (123.6730,175.2920) -- cycle(123.7960,174.4860) -- (123.8870,174.2350) -- (124.0380,174.2900) -- (123.9470,174.5410) -- cycle(124.0700,173.7340) -- (124.1610,173.4840) -- (124.3110,173.5380) -- (124.2200,173.7890) -- cycle(124.3430,172.9820) -- (124.4350,172.7320) -- (124.5850,172.7860) -- (124.4940,173.0370) -- cycle(124.6170,172.2310) -- (124.7080,171.9800) -- (124.8590,172.0350) -- (124.7670,172.2850) -- cycle(124.8910,171.4790) -- (124.9820,171.2280) -- (125.1320,171.2830) -- (125.0410,171.5340) -- cycle(125.1640,170.7270) -- (125.2550,170.4770) -- (125.4060,170.5310) -- (125.3150,170.7820) -- cycle(125.4380,169.9750) -- (125.5290,169.7250) -- (125.6790,169.7790) -- (125.5880,170.0300) -- cycle(125.7110,169.2240) -- (125.8030,168.9730) -- (125.9530,169.0280) -- (125.8620,169.2780) -- cycle(125.9850,168.4720) -- (126.0760,168.2210) -- (126.2270,168.2760) -- (126.1350,168.5270) -- cycle(126.2590,167.7200) -- (126.3500,167.4700) -- (126.5000,167.5240) -- (126.4090,167.7750) -- cycle(126.5320,166.9680) -- (126.6240,166.7180) -- (126.7740,166.7720) -- (126.6830,167.0230) -- cycle(126.8060,166.2170) -- (126.8970,165.9660) -- (127.0480,166.0210) -- (126.9560,166.2710) -- cycle(127.0800,165.4650) -- (127.1710,165.2140) -- (127.3210,165.2690) -- (127.2300,165.5200) -- cycle(127.3530,164.7130) -- (127.4440,164.4620) -- (127.5950,164.5170) -- (127.5040,164.7680) -- cycle(127.6270,163.9610) -- (127.7180,163.7110) -- (127.8680,163.7650) -- (127.7770,164.0160) -- cycle(127.9000,163.2100) -- (127.9920,162.9590) -- (128.1420,163.0140) -- (128.0510,163.2640) -- cycle(128.1740,162.4580) -- (128.2650,162.2070) -- (128.4160,162.2620) -- (128.3240,162.5120) -- cycle(128.4480,161.7060) -- (128.5390,161.4550) -- (128.6890,161.5100) -- (128.5980,161.7610) -- cycle(128.7210,160.9540) -- (128.8120,160.7040) -- (128.9630,160.7580) -- (128.8720,161.0090) -- cycle(128.9950,160.2030) -- (129.0860,159.9520) -- (129.2360,160.0070) -- (129.1450,160.2570) -- cycle(129.2680,159.4510) -- (129.3600,159.2000) -- (129.5100,159.2550) -- (129.4190,159.5060) -- cycle(129.5420,158.6990) -- (129.6330,158.4480) -- (129.7840,158.5030) -- (129.6920,158.7540) -- cycle(129.8160,157.9470) -- (129.9070,157.6970) -- (130.0570,157.7510) -- (129.9660,158.0020) -- cycle(130.0890,157.1960) -- (130.1810,156.9450) -- (130.3310,157.0000) -- (130.2400,157.2500) -- cycle(130.3630,156.4440) -- (130.4540,156.1930) -- (130.6040,156.2480) -- (130.5130,156.4980) -- cycle(130.6370,155.6920) -- (130.7280,155.4410) -- (130.8780,155.4960) -- (130.7870,155.7470) -- cycle(130.9100,154.9400) -- (131.0010,154.6900) -- (131.1520,154.7440) -- (131.0610,154.9950) -- cycle(80.8389,292.5110) -- (80.9301,292.2600) -- (81.0805,292.3150) -- (80.9893,292.5660) -- cycle;



  \path[fill=c7e7e7e,line join=round,line width=0.256pt] (106.8240,291.7620) -- (106.9160,291.5110) -- (107.0660,291.5660) -- (106.9750,291.8170) -- cycle(107.0980,291.0100) -- (107.1890,290.7600) -- (107.3400,290.8140) -- (107.2480,291.0650) -- cycle(107.3720,290.2580) -- (107.4630,290.0080) -- (107.6130,290.0620) -- (107.5220,290.3130) -- cycle(107.6450,289.5070) -- (107.7370,289.2560) -- (107.8870,289.3110) -- (107.7960,289.5610) -- cycle(107.9190,288.7550) -- (108.0100,288.5040) -- (108.1600,288.5590) -- (108.0690,288.8100) -- cycle(108.1930,288.0030) -- (108.2840,287.7530) -- (108.4340,287.8070) -- (108.3430,288.0580) -- cycle(108.4660,287.2510) -- (108.5570,287.0010) -- (108.7080,287.0560) -- (108.6170,287.3060) -- cycle(108.7400,286.5000) -- (108.8310,286.2490) -- (108.9810,286.3040) -- (108.8900,286.5540) -- cycle(109.0130,285.7480) -- (109.1050,285.4970) -- (109.2550,285.5520) -- (109.1640,285.8030) -- cycle(109.2870,284.9960) -- (109.3780,284.7460) -- (109.5290,284.8000) -- (109.4370,285.0510) -- cycle(109.5610,284.2440) -- (109.6520,283.9940) -- (109.8020,284.0490) -- (109.7110,284.2990) -- cycle(109.8340,283.4930) -- (109.9250,283.2420) -- (110.0760,283.2970) -- (109.9850,283.5470) -- cycle(110.1080,282.7410) -- (110.1990,282.4900) -- (110.3490,282.5450) -- (110.2580,282.7960) -- cycle(110.3810,281.9890) -- (110.4730,281.7390) -- (110.6230,281.7930) -- (110.5320,282.0440) -- cycle(110.6550,281.2370) -- (110.7460,280.9870) -- (110.8970,281.0410) -- (110.8050,281.2920) -- cycle(110.9290,280.4860) -- (111.0200,280.2350) -- (111.1700,280.2900) -- (111.0790,280.5400) -- cycle(111.2020,279.7340) -- (111.2930,279.4830) -- (111.4440,279.5380) -- (111.3530,279.7890) -- cycle(111.4760,278.9820) -- (111.5670,278.7320) -- (111.7170,278.7860) -- (111.6260,279.0370) -- cycle(111.7490,278.2300) -- (111.8410,277.9800) -- (111.9910,278.0350) -- (111.9000,278.2850) -- cycle(112.0230,277.4790) -- (112.1140,277.2280) -- (112.2650,277.2830) -- (112.1730,277.5330) -- cycle(112.2970,276.7270) -- (112.3880,276.4760) -- (112.5380,276.5310) -- (112.4470,276.7820) -- cycle(112.5700,275.9750) -- (112.6610,275.7250) -- (112.8120,275.7790) -- (112.7210,276.0300) -- cycle(112.8440,275.2230) -- (112.9350,274.9730) -- (113.0860,275.0270) -- (112.9940,275.2780) -- cycle(113.1180,274.4720) -- (113.2090,274.2210) -- (113.3590,274.2760) -- (113.2680,274.5260) -- cycle(113.3910,273.7200) -- (113.4820,273.4690) -- (113.6330,273.5240) -- (113.5420,273.7750) -- cycle(113.6650,272.9680) -- (113.7560,272.7170) -- (113.9060,272.7720) -- (113.8150,273.0230) -- cycle(113.9380,272.2160) -- (114.0300,271.9660) -- (114.1800,272.0200) -- (114.0890,272.2710) -- cycle(114.2120,271.4650) -- (114.3030,271.2140) -- (114.4540,271.2690) -- (114.3620,271.5190) -- cycle(114.4860,270.7130) -- (114.5770,270.4620) -- (114.7270,270.5170) -- (114.6360,270.7680) -- cycle(114.7590,269.9610) -- (114.8500,269.7110) -- (115.0010,269.7650) -- (114.9100,270.0160) -- cycle(115.0330,269.2090) -- (115.1240,268.9590) -- (115.2740,269.0130) -- (115.1830,269.2640) -- cycle(115.3060,268.4580) -- (115.3980,268.2070) -- (115.5480,268.2620) -- (115.4570,268.5120) -- cycle(115.5800,267.7060) -- (115.6710,267.4550) -- (115.8220,267.5100) -- (115.7300,267.7600) -- cycle(115.8540,266.9540) -- (115.9450,266.7030) -- (116.0950,266.7580) -- (116.0040,267.0090) -- cycle(116.1270,266.2020) -- (116.2190,265.9520) -- (116.3690,266.0060) -- (116.2780,266.2570) -- cycle(116.4010,265.4510) -- (116.4920,265.2000) -- (116.6430,265.2550) -- (116.5510,265.5050) -- cycle(116.6750,264.6990) -- (116.7660,264.4480) -- (116.9160,264.5030) -- (116.8250,264.7540) -- cycle(116.9480,263.9470) -- (117.0390,263.6970) -- (117.1900,263.7510) -- (117.0990,264.0020) -- cycle(117.2220,263.1950) -- (117.3130,262.9450) -- (117.4630,262.9990) -- (117.3720,263.2500) -- cycle(117.4950,262.4440) -- (117.5870,262.1930) -- (117.7370,262.2480) -- (117.6460,262.4980) -- cycle(117.7690,261.6920) -- (117.8600,261.4410) -- (118.0110,261.4960) -- (117.9190,261.7470) -- cycle(118.0430,260.9400) -- (118.1340,260.6890) -- (118.2840,260.7440) -- (118.1930,260.9950) -- cycle(118.3160,260.1880) -- (118.4070,259.9380) -- (118.5580,259.9920) -- (118.4670,260.2430) -- cycle(118.5900,259.4360) -- (118.6810,259.1860) -- (118.8310,259.2410) -- (118.7400,259.4910) -- cycle(118.8630,258.6850) -- (118.9550,258.4340) -- (119.1050,258.4890) -- (119.0140,258.7390) -- cycle(119.1370,257.9330) -- (119.2280,257.6820) -- (119.3790,257.7370) -- (119.2870,257.9880) -- cycle(119.4110,257.1810) -- (119.5020,256.9310) -- (119.6520,256.9850) -- (119.5610,257.2360) -- cycle(119.6840,256.4300) -- (119.7750,256.1790) -- (119.9260,256.2340) -- (119.8350,256.4840) -- cycle(119.9580,255.6780) -- (120.0490,255.4270) -- (120.1990,255.4820) -- (120.1080,255.7330) -- cycle(120.2320,254.9260) -- (120.3230,254.6760) -- (120.4730,254.7300) -- (120.3820,254.9810) -- cycle(120.5050,254.1740) -- (120.5960,253.9240) -- (120.7470,253.9780) -- (120.6550,254.2290) -- cycle(120.7790,253.4230) -- (120.8700,253.1720) -- (121.0200,253.2270) -- (120.9290,253.4770) -- cycle(121.0520,252.6710) -- (121.1440,252.4200) -- (121.2940,252.4750) -- (121.2030,252.7250) -- cycle(121.3260,251.9190) -- (121.4170,251.6680) -- (121.5680,251.7230) -- (121.4760,251.9740) -- cycle(121.6000,251.1670) -- (121.6910,250.9170) -- (121.8410,250.9710) -- (121.7500,251.2220) -- cycle(121.8730,250.4150) -- (121.9640,250.1650) -- (122.1150,250.2200) -- (122.0240,250.4700) -- cycle(122.1470,249.6640) -- (122.2380,249.4130) -- (122.3880,249.4680) -- (122.2970,249.7180) -- cycle(122.4200,248.9120) -- (122.5120,248.6610) -- (122.6620,248.7160) -- (122.5710,248.9670) -- cycle(122.6940,248.1600) -- (122.7850,247.9100) -- (122.9360,247.9640) -- (122.8440,248.2150) -- cycle(122.9680,247.4090) -- (123.0590,247.1580) -- (123.2090,247.2130) -- (123.1180,247.4630) -- cycle(123.2410,246.6570) -- (123.3320,246.4060) -- (123.4830,246.4610) -- (123.3920,246.7110) -- cycle(123.5150,245.9050) -- (123.6060,245.6540) -- (123.7560,245.7090) -- (123.6650,245.9600) -- cycle(123.7880,245.1530) -- (123.8800,244.9030) -- (124.0300,244.9570) -- (123.9390,245.2080) -- cycle(124.0620,244.4010) -- (124.1530,244.1510) -- (124.3040,244.2060) -- (124.2120,244.4560) -- cycle(124.3360,243.6500) -- (124.4270,243.3990) -- (124.5770,243.4540) -- (124.4860,243.7040) -- cycle(124.6090,242.8980) -- (124.7010,242.6470) -- (124.8510,242.7020) -- (124.7600,242.9530) -- cycle(124.8830,242.1460) -- (124.9740,241.8960) -- (125.1250,241.9500) -- (125.0330,242.2010) -- cycle(125.1570,241.3940) -- (125.2480,241.1440) -- (125.3980,241.1990) -- (125.3070,241.4490) -- cycle(125.4300,240.6430) -- (125.5210,240.3920) -- (125.6720,240.4470) -- (125.5810,240.6970) -- cycle(125.7040,239.8910) -- (125.7950,239.6400) -- (125.9450,239.6950) -- (125.8540,239.9460) -- cycle(125.9770,239.1390) -- (126.0690,238.8890) -- (126.2190,238.9430) -- (126.1280,239.1940) -- cycle(126.2510,238.3870) -- (126.3420,238.1370) -- (126.4930,238.1920) -- (126.4010,238.4420) -- cycle(126.5250,237.6360) -- (126.6160,237.3850) -- (126.7660,237.4400) -- (126.6750,237.6900) -- cycle(126.7980,236.8840) -- (126.8890,236.6330) -- (127.0400,236.6880) -- (126.9490,236.9390) -- cycle(127.0720,236.1320) -- (127.1630,235.8820) -- (127.3130,235.9360) -- (127.2220,236.1870) -- cycle(127.3450,235.3800) -- (127.4370,235.1300) -- (127.5870,235.1850) -- (127.4960,235.4350) -- cycle(127.6190,234.6290) -- (127.7100,234.3780) -- (127.8610,234.4330) -- (127.7690,234.6830) -- cycle(127.8930,233.8770) -- (127.9840,233.6260) -- (128.1340,233.6810) -- (128.0430,233.9320) -- cycle(128.1660,233.1250) -- (128.2580,232.8750) -- (128.4080,232.9290) -- (128.3170,233.1800) -- cycle(128.4400,232.3730) -- (128.5310,232.1230) -- (128.6810,232.1780) -- (128.5900,232.4280) -- cycle(128.7140,231.6220) -- (128.8050,231.3710) -- (128.9550,231.4260) -- (128.8640,231.6760) -- cycle(128.9870,230.8700) -- (129.0780,230.6190) -- (129.2290,230.6740) -- (129.1380,230.9250) -- cycle(129.2610,230.1180) -- (129.3520,229.8680) -- (129.5020,229.9220) -- (129.4110,230.1730) -- cycle(129.5340,229.3660) -- (129.6260,229.1160) -- (129.7760,229.1710) -- (129.6850,229.4210) -- cycle(129.8080,228.6150) -- (129.8990,228.3640) -- (130.0500,228.4190) -- (129.9580,228.6690) -- cycle(130.0820,227.8630) -- (130.1730,227.6120) -- (130.3230,227.6670) -- (130.2320,227.9180) -- cycle(130.3550,227.1110) -- (130.4460,226.8610) -- (130.5970,226.9150) -- (130.5060,227.1660) -- cycle(130.6290,226.3590) -- (130.7200,226.1090) -- (130.8700,226.1640) -- (130.7790,226.4140) -- cycle(130.9020,225.6080) -- (130.9940,225.3570) -- (131.1440,225.4120) -- (131.0530,225.6620) -- cycle(131.1760,224.8560) -- (131.2670,224.6050) -- (131.4180,224.6600) -- (131.3260,224.9110) -- cycle(131.4500,224.1040) -- (131.5410,223.8540) -- (131.6910,223.9080) -- (131.6000,224.1590) -- cycle(131.7230,223.3520) -- (131.8140,223.1020) -- (131.9650,223.1570) -- (131.8740,223.4070) -- cycle(131.9970,222.6010) -- (132.0880,222.3500) -- (132.2380,222.4050) -- (132.1470,222.6550) -- cycle(132.2700,221.8490) -- (132.3620,221.5980) -- (132.5120,221.6530) -- (132.4210,221.9040) -- cycle(132.5440,221.0970) -- (132.6350,220.8470) -- (132.7860,220.9010) -- (132.6940,221.1520) -- cycle(132.8180,220.3450) -- (132.9090,220.0950) -- (133.0590,220.1500) -- (132.9680,220.4000) -- cycle(133.0910,219.5940) -- (133.1820,219.3430) -- (133.3330,219.3980) -- (133.2420,219.6480) -- cycle(133.3650,218.8420) -- (133.4560,218.5910) -- (133.6070,218.6460) -- (133.5150,218.8970) -- cycle(133.6390,218.0900) -- (133.7300,217.8400) -- (133.8800,217.8940) -- (133.7890,218.1450) -- cycle(133.9120,217.3380) -- (134.0030,217.0880) -- (134.1540,217.1420) -- (134.0630,217.3930) -- cycle(134.1860,216.5870) -- (134.2770,216.3360) -- (134.4270,216.3910) -- (134.3360,216.6410) -- cycle(134.4590,215.8350) -- (134.5510,215.5840) -- (134.7010,215.6390) -- (134.6100,215.8900) -- cycle(134.7330,215.0830) -- (134.8240,214.8330) -- (134.9750,214.8870) -- (134.8830,215.1380) -- cycle(135.0070,214.3310) -- (135.0980,214.0810) -- (135.2480,214.1350) -- (135.1570,214.3860) -- cycle(135.2800,213.5800) -- (135.3710,213.3290) -- (135.5220,213.3840) -- (135.4310,213.6340) -- cycle(135.5540,212.8280) -- (135.6450,212.5770) -- (135.7950,212.6320) -- (135.7040,212.8830) -- cycle(135.8270,212.0760) -- (135.9190,211.8260) -- (136.0690,211.8800) -- (135.9780,212.1310) -- cycle(136.1010,211.3240) -- (136.1920,211.0740) -- (136.3430,211.1280) -- (136.2510,211.3790) -- cycle(136.3750,210.5730) -- (136.4660,210.3220) -- (136.6160,210.3770) -- (136.5250,210.6270) -- cycle(136.6480,209.8210) -- (136.7400,209.5700) -- (136.8900,209.6250) -- (136.7990,209.8760) -- cycle(136.9220,209.0690) -- (137.0130,208.8190) -- (137.1640,208.8730) -- (137.0720,209.1240) -- cycle(137.1960,208.3170) -- (137.2870,208.0670) -- (137.4370,208.1210) -- (137.3460,208.3720) -- cycle(137.4690,207.5660) -- (137.5600,207.3150) -- (137.7110,207.3700) -- (137.6200,207.6200) -- cycle(137.7430,206.8140) -- (137.8340,206.5630) -- (137.9840,206.6180) -- (137.8930,206.8690) -- cycle(138.0160,206.0620) -- (138.1080,205.8110) -- (138.2580,205.8660) -- (138.1670,206.1170) -- cycle(138.2900,205.3100) -- (138.3810,205.0600) -- (138.5320,205.1140) -- (138.4400,205.3650) -- cycle(138.5640,204.5590) -- (138.6550,204.3080) -- (138.8050,204.3630) -- (138.7140,204.6130) -- cycle(138.8370,203.8070) -- (138.9280,203.5560) -- (139.0790,203.6110) -- (138.9880,203.8610) -- cycle(139.1110,203.0550) -- (139.2020,202.8040) -- (139.3520,202.8590) -- (139.2610,203.1100) -- cycle(139.3840,202.3030) -- (139.4760,202.0530) -- (139.6260,202.1070) -- (139.5350,202.3580) -- cycle(139.6580,201.5520) -- (139.7490,201.3010) -- (139.9000,201.3560) -- (139.8080,201.6060) -- cycle(139.9320,200.8000) -- (140.0230,200.5490) -- (140.1730,200.6040) -- (140.0820,200.8550) -- cycle(140.2050,200.0480) -- (140.2960,199.7980) -- (140.4470,199.8520) -- (140.3560,200.1030) -- cycle(140.4790,199.2960) -- (140.5700,199.0460) -- (140.7200,199.1000) -- (140.6290,199.3510) -- cycle(140.7530,198.5450) -- (140.8440,198.2940) -- (140.9940,198.3490) -- (140.9030,198.5990) -- cycle(141.0260,197.7930) -- (141.1170,197.5420) -- (141.2680,197.5970) -- (141.1760,197.8480) -- cycle(141.3000,197.0410) -- (141.3910,196.7900) -- (141.5410,196.8450) -- (141.4500,197.0960) -- cycle(141.5730,196.2890) -- (141.6650,196.0390) -- (141.8150,196.0930) -- (141.7240,196.3440) -- cycle(141.8470,195.5380) -- (141.9380,195.2870) -- (142.0890,195.3420) -- (141.9970,195.5920) -- cycle(142.1210,194.7860) -- (142.2120,194.5350) -- (142.3620,194.5900) -- (142.2710,194.8400) -- cycle(142.3940,194.0340) -- (142.4850,193.7830) -- (142.6360,193.8380) -- (142.5450,194.0890) -- cycle(142.6680,193.2820) -- (142.7590,193.0320) -- (142.9090,193.0860) -- (142.8180,193.3370) -- cycle(142.9410,192.5310) -- (143.0330,192.2800) -- (143.1830,192.3350) -- (143.0920,192.5850) -- cycle(143.2150,191.7790) -- (143.3060,191.5280) -- (143.4570,191.5830) -- (143.3650,191.8330) -- cycle(143.4890,191.0270) -- (143.5800,190.7760) -- (143.7300,190.8310) -- (143.6390,191.0820) -- cycle(143.7620,190.2750) -- (143.8530,190.0250) -- (144.0040,190.0790) -- (143.9130,190.3300) -- cycle(144.0360,189.5240) -- (144.1270,189.2730) -- (144.2770,189.3280) -- (144.1860,189.5780) -- cycle(144.3090,188.7720) -- (144.4010,188.5210) -- (144.5510,188.5760) -- (144.4600,188.8260) -- cycle(144.5830,188.0200) -- (144.6740,187.7690) -- (144.8250,187.8240) -- (144.7330,188.0750) -- cycle(144.8570,187.2680) -- (144.9480,187.0180) -- (145.0980,187.0720) -- (145.0070,187.3230) -- cycle(145.1300,186.5170) -- (145.2210,186.2660) -- (145.3720,186.3210) -- (145.2810,186.5710) -- cycle(145.4040,185.7650) -- (145.4950,185.5140) -- (145.6460,185.5690) -- (145.5540,185.8190) -- cycle(145.6780,185.0130) -- (145.7690,184.7620) -- (145.9190,184.8170) -- (145.8280,185.0680) -- cycle(145.9510,184.2610) -- (146.0420,184.0110) -- (146.1930,184.0650) -- (146.1020,184.3160) -- cycle(146.2250,183.5090) -- (146.3160,183.2590) -- (146.4660,183.3140) -- (146.3750,183.5640) -- cycle(146.4980,182.7580) -- (146.5900,182.5070) -- (146.7400,182.5620) -- (146.6490,182.8120) -- cycle(146.7720,182.0060) -- (146.8630,181.7550) -- (147.0140,181.8100) -- (146.9220,182.0610) -- cycle(147.0460,181.2540) -- (147.1370,181.0040) -- (147.2870,181.0580) -- (147.1960,181.3090) -- cycle(147.3190,180.5020) -- (147.4100,180.2520) -- (147.5610,180.3070) -- (147.4700,180.5570) -- cycle(147.5930,179.7510) -- (147.6840,179.5000) -- (147.8340,179.5550) -- (147.7430,179.8050) -- cycle(147.8660,178.9990) -- (147.9580,178.7480) -- (148.1080,178.8030) -- (148.0170,179.0540) -- cycle(148.1400,178.2470) -- (148.2310,177.9970) -- (148.3820,178.0510) -- (148.2900,178.3020) -- cycle(148.4140,177.4960) -- (148.5050,177.2450) -- (148.6550,177.3000) -- (148.5640,177.5500) -- cycle(148.6870,176.7440) -- (148.7790,176.4930) -- (148.9290,176.5480) -- (148.8380,176.7980) -- cycle(148.9610,175.9920) -- (149.0520,175.7410) -- (149.2020,175.7960) -- (149.1110,176.0470) -- cycle(149.2340,175.2400) -- (149.3260,174.9900) -- (149.4760,175.0440) -- (149.3850,175.2950) -- cycle(149.5080,174.4880) -- (149.5990,174.2380) -- (149.7500,174.2930) -- (149.6590,174.5430) -- cycle(149.7820,173.7370) -- (149.8730,173.4860) -- (150.0230,173.5410) -- (149.9320,173.7910) -- cycle(150.0550,172.9850) -- (150.1470,172.7340) -- (150.2970,172.7890) -- (150.2060,173.0400) -- cycle(150.3290,172.2330) -- (150.4200,171.9830) -- (150.5710,172.0370) -- (150.4790,172.2880) -- cycle(150.6030,171.4810) -- (150.6940,171.2310) -- (150.8440,171.2860) -- (150.7530,171.5360) -- cycle(150.8760,170.7300) -- (150.9670,170.4790) -- (151.1180,170.5340) -- (151.0270,170.7840) -- cycle(151.1500,169.9780) -- (151.2410,169.7270) -- (151.3910,169.7820) -- (151.3000,170.0330) -- cycle(151.4230,169.2260) -- (151.5150,168.9760) -- (151.6650,169.0300) -- (151.5740,169.2810) -- cycle(151.6970,168.4740) -- (151.7880,168.2240) -- (151.9390,168.2790) -- (151.8470,168.5290) -- cycle(151.9710,167.7230) -- (152.0620,167.4720) -- (152.2120,167.5270) -- (152.1210,167.7770) -- cycle(152.2440,166.9710) -- (152.3350,166.7200) -- (152.4860,166.7750) -- (152.3950,167.0260) -- cycle(152.5180,166.2190) -- (152.6090,165.9690) -- (152.7590,166.0230) -- (152.6680,166.2740) -- cycle(152.7910,165.4670) -- (152.8830,165.2170) -- (153.0330,165.2720) -- (152.9420,165.5220) -- cycle(153.0650,164.7160) -- (153.1560,164.4650) -- (153.3070,164.5200) -- (153.2150,164.7700) -- cycle(153.3390,163.9640) -- (153.4300,163.7130) -- (153.5800,163.7680) -- (153.4890,164.0190) -- cycle(153.6120,163.2120) -- (153.7030,162.9620) -- (153.8540,163.0160) -- (153.7630,163.2670) -- cycle(153.8860,162.4600) -- (153.9770,162.2100) -- (154.1280,162.2650) -- (154.0360,162.5150) -- cycle(154.1600,161.7090) -- (154.2510,161.4580) -- (154.4010,161.5130) -- (154.3100,161.7630) -- cycle(154.4330,160.9570) -- (154.5240,160.7060) -- (154.6750,160.7610) -- (154.5840,161.0120) -- cycle(154.7070,160.2050) -- (154.7980,159.9550) -- (154.9480,160.0090) -- (154.8570,160.2600) -- cycle(154.9800,159.4530) -- (155.0720,159.2030) -- (155.2220,159.2570) -- (155.1310,159.5080) -- cycle(155.2540,158.7020) -- (155.3450,158.4510) -- (155.4960,158.5060) -- (155.4040,158.7560) -- cycle(155.5280,157.9500) -- (155.6190,157.6990) -- (155.7690,157.7540) -- (155.6780,158.0050) -- cycle(155.8010,157.1980) -- (155.8920,156.9480) -- (156.0430,157.0020) -- (155.9520,157.2530) -- cycle(156.0750,156.4460) -- (156.1660,156.1960) -- (156.3160,156.2510) -- (156.2250,156.5010) -- cycle(156.3480,155.6950) -- (156.4400,155.4440) -- (156.5900,155.4990) -- (156.4990,155.7490) -- cycle(156.6220,154.9430) -- (156.7130,154.6920) -- (156.8640,154.7470) -- (156.7720,154.9980) -- cycle(106.5510,292.5140) -- (106.6420,292.2630) -- (106.7920,292.3180) -- (106.7010,292.5680) -- cycle;



  \path[fill=c7e7e7e,line join=round,line width=0.256pt] (132.4010,291.7210) -- (132.4930,291.4710) -- (132.6430,291.5250) -- (132.5520,291.7760) -- cycle(132.6750,290.9690) -- (132.7660,290.7190) -- (132.9170,290.7740) -- (132.8250,291.0240) -- cycle(132.9490,290.2180) -- (133.0400,289.9670) -- (133.1900,290.0220) -- (133.0990,290.2720) -- cycle(133.2220,289.4660) -- (133.3130,289.2150) -- (133.4640,289.2700) -- (133.3730,289.5210) -- cycle(133.4960,288.7140) -- (133.5870,288.4640) -- (133.7370,288.5180) -- (133.6460,288.7690) -- cycle(133.7700,287.9620) -- (133.8610,287.7120) -- (134.0110,287.7670) -- (133.9200,288.0170) -- cycle(134.0430,287.2110) -- (134.1340,286.9600) -- (134.2850,287.0150) -- (134.1930,287.2650) -- cycle(134.3170,286.4590) -- (134.4080,286.2080) -- (134.5580,286.2630) -- (134.4670,286.5140) -- cycle(134.5900,285.7070) -- (134.6820,285.4570) -- (134.8320,285.5110) -- (134.7410,285.7620) -- cycle(134.8640,284.9550) -- (134.9550,284.7050) -- (135.1060,284.7600) -- (135.0140,285.0100) -- cycle(135.1380,284.2040) -- (135.2290,283.9530) -- (135.3790,284.0080) -- (135.2880,284.2580) -- cycle(135.4110,283.4520) -- (135.5020,283.2010) -- (135.6530,283.2560) -- (135.5620,283.5070) -- cycle(135.6850,282.7000) -- (135.7760,282.4500) -- (135.9260,282.5040) -- (135.8350,282.7550) -- cycle(135.9580,281.9480) -- (136.0500,281.6980) -- (136.2000,281.7530) -- (136.1090,282.0030) -- cycle(136.2320,281.1970) -- (136.3230,280.9460) -- (136.4740,281.0010) -- (136.3820,281.2510) -- cycle(136.5060,280.4450) -- (136.5970,280.1940) -- (136.7470,280.2490) -- (136.6560,280.5000) -- cycle(136.7790,279.6930) -- (136.8700,279.4430) -- (137.0210,279.4970) -- (136.9300,279.7480) -- cycle(137.0530,278.9410) -- (137.1440,278.6910) -- (137.2940,278.7460) -- (137.2030,278.9960) -- cycle(137.3260,278.1900) -- (137.4180,277.9390) -- (137.5680,277.9940) -- (137.4770,278.2440) -- cycle(137.6000,277.4380) -- (137.6910,277.1870) -- (137.8420,277.2420) -- (137.7500,277.4930) -- cycle(137.8740,276.6860) -- (137.9650,276.4360) -- (138.1150,276.4900) -- (138.0240,276.7410) -- cycle(138.1470,275.9340) -- (138.2380,275.6840) -- (138.3890,275.7380) -- (138.2980,275.9890) -- cycle(138.4210,275.1830) -- (138.5120,274.9320) -- (138.6630,274.9870) -- (138.5710,275.2370) -- cycle(138.6950,274.4310) -- (138.7860,274.1800) -- (138.9360,274.2350) -- (138.8450,274.4860) -- cycle(138.9680,273.6790) -- (139.0590,273.4290) -- (139.2100,273.4830) -- (139.1190,273.7340) -- cycle(139.2420,272.9270) -- (139.3330,272.6770) -- (139.4830,272.7320) -- (139.3920,272.9820) -- cycle(139.5150,272.1760) -- (139.6070,271.9250) -- (139.7570,271.9800) -- (139.6660,272.2300) -- cycle(139.7890,271.4240) -- (139.8800,271.1730) -- (140.0310,271.2280) -- (139.9390,271.4790) -- cycle(140.0630,270.6720) -- (140.1540,270.4220) -- (140.3040,270.4760) -- (140.2130,270.7270) -- cycle(140.3360,269.9200) -- (140.4270,269.6700) -- (140.5780,269.7240) -- (140.4870,269.9750) -- cycle(140.6100,269.1690) -- (140.7010,268.9180) -- (140.8510,268.9730) -- (140.7600,269.2230) -- cycle(140.8830,268.4170) -- (140.9750,268.1660) -- (141.1250,268.2210) -- (141.0340,268.4720) -- cycle(141.1570,267.6650) -- (141.2480,267.4150) -- (141.3990,267.4690) -- (141.3070,267.7200) -- cycle(141.4310,266.9130) -- (141.5220,266.6630) -- (141.6720,266.7170) -- (141.5810,266.9680) -- cycle(141.7040,266.1620) -- (141.7960,265.9110) -- (141.9460,265.9660) -- (141.8550,266.2160) -- cycle(141.9780,265.4100) -- (142.0690,265.1590) -- (142.2190,265.2140) -- (142.1280,265.4650) -- cycle(142.2520,264.6580) -- (142.3430,264.4080) -- (142.4930,264.4620) -- (142.4020,264.7130) -- cycle(142.5250,263.9060) -- (142.6160,263.6560) -- (142.7670,263.7110) -- (142.6760,263.9610) -- cycle(142.7990,263.1550) -- (142.8900,262.9040) -- (143.0400,262.9590) -- (142.9490,263.2090) -- cycle(143.0720,262.4030) -- (143.1640,262.1520) -- (143.3140,262.2070) -- (143.2230,262.4580) -- cycle(143.3460,261.6510) -- (143.4370,261.4000) -- (143.5880,261.4550) -- (143.4960,261.7060) -- cycle(143.6200,260.8990) -- (143.7110,260.6490) -- (143.8610,260.7030) -- (143.7700,260.9540) -- cycle(143.8930,260.1480) -- (143.9840,259.8970) -- (144.1350,259.9520) -- (144.0440,260.2020) -- cycle(144.1670,259.3960) -- (144.2580,259.1450) -- (144.4080,259.2000) -- (144.3170,259.4510) -- cycle(144.4400,258.6440) -- (144.5320,258.3940) -- (144.6820,258.4480) -- (144.5910,258.6990) -- cycle(144.7140,257.8920) -- (144.8050,257.6420) -- (144.9560,257.6970) -- (144.8640,257.9470) -- cycle(144.9880,257.1410) -- (145.0790,256.8900) -- (145.2290,256.9450) -- (145.1380,257.1950) -- cycle(145.2610,256.3890) -- (145.3520,256.1380) -- (145.5030,256.1930) -- (145.4120,256.4440) -- cycle(145.5350,255.6370) -- (145.6260,255.3870) -- (145.7760,255.4410) -- (145.6850,255.6920) -- cycle(145.8080,254.8850) -- (145.9000,254.6350) -- (146.0500,254.6890) -- (145.9590,254.9400) -- cycle(146.0820,254.1340) -- (146.1730,253.8830) -- (146.3240,253.9380) -- (146.2320,254.1880) -- cycle(146.3560,253.3820) -- (146.4470,253.1310) -- (146.5970,253.1860) -- (146.5060,253.4360) -- cycle(146.6290,252.6300) -- (146.7210,252.3790) -- (146.8710,252.4340) -- (146.7800,252.6850) -- cycle(146.9030,251.8780) -- (146.9940,251.6280) -- (147.1450,251.6820) -- (147.0530,251.9330) -- cycle(147.1770,251.1270) -- (147.2680,250.8760) -- (147.4180,250.9310) -- (147.3270,251.1810) -- cycle(147.4500,250.3750) -- (147.5410,250.1240) -- (147.6920,250.1790) -- (147.6010,250.4300) -- cycle(147.7240,249.6230) -- (147.8150,249.3730) -- (147.9650,249.4270) -- (147.8740,249.6780) -- cycle(147.9970,248.8710) -- (148.0890,248.6210) -- (148.2390,248.6750) -- (148.1480,248.9260) -- cycle(148.2710,248.1200) -- (148.3620,247.8690) -- (148.5130,247.9240) -- (148.4210,248.1740) -- cycle(148.5450,247.3680) -- (148.6360,247.1170) -- (148.7860,247.1720) -- (148.6950,247.4220) -- cycle(148.8180,246.6160) -- (148.9090,246.3650) -- (149.0600,246.4200) -- (148.9690,246.6710) -- cycle(149.0920,245.8640) -- (149.1830,245.6140) -- (149.3330,245.6680) -- (149.2420,245.9190) -- cycle(149.3650,245.1130) -- (149.4570,244.8620) -- (149.6070,244.9170) -- (149.5160,245.1670) -- cycle(149.6390,244.3610) -- (149.7300,244.1100) -- (149.8810,244.1650) -- (149.7890,244.4150) -- cycle(149.9130,243.6090) -- (150.0040,243.3580) -- (150.1540,243.4130) -- (150.0630,243.6640) -- cycle(150.1860,242.8570) -- (150.2770,242.6070) -- (150.4280,242.6610) -- (150.3370,242.9120) -- cycle(150.4600,242.1060) -- (150.5510,241.8550) -- (150.7020,241.9100) -- (150.6100,242.1600) -- cycle(150.7340,241.3540) -- (150.8250,241.1030) -- (150.9750,241.1580) -- (150.8840,241.4080) -- cycle(151.0070,240.6020) -- (151.0980,240.3510) -- (151.2490,240.4060) -- (151.1580,240.6570) -- cycle(151.2810,239.8500) -- (151.3720,239.6000) -- (151.5220,239.6540) -- (151.4310,239.9050) -- cycle(151.5540,239.0990) -- (151.6460,238.8480) -- (151.7960,238.9030) -- (151.7050,239.1530) -- cycle(151.8280,238.3470) -- (151.9190,238.0960) -- (152.0700,238.1510) -- (151.9780,238.4010) -- cycle(152.1020,237.5950) -- (152.1930,237.3440) -- (152.3430,237.3990) -- (152.2520,237.6500) -- cycle(152.3750,236.8430) -- (152.4660,236.5930) -- (152.6170,236.6470) -- (152.5260,236.8980) -- cycle(152.6490,236.0920) -- (152.7400,235.8410) -- (152.8900,235.8960) -- (152.7990,236.1460) -- cycle(152.9220,235.3400) -- (153.0140,235.0890) -- (153.1640,235.1440) -- (153.0730,235.3940) -- cycle(153.1960,234.5880) -- (153.2870,234.3370) -- (153.4380,234.3920) -- (153.3460,234.6430) -- cycle(153.4700,233.8360) -- (153.5610,233.5860) -- (153.7110,233.6400) -- (153.6200,233.8910) -- cycle(153.7430,233.0850) -- (153.8340,232.8340) -- (153.9850,232.8890) -- (153.8940,233.1390) -- cycle(154.0170,232.3330) -- (154.1080,232.0820) -- (154.2580,232.1370) -- (154.1670,232.3870) -- cycle(154.2900,231.5810) -- (154.3820,231.3300) -- (154.5320,231.3850) -- (154.4410,231.6360) -- cycle(154.5640,230.8290) -- (154.6550,230.5790) -- (154.8060,230.6330) -- (154.7140,230.8840) -- cycle(154.8380,230.0770) -- (154.9290,229.8270) -- (155.0790,229.8820) -- (154.9880,230.1320) -- cycle(155.1110,229.3260) -- (155.2030,229.0750) -- (155.3530,229.1300) -- (155.2620,229.3800) -- cycle(155.3850,228.5740) -- (155.4760,228.3230) -- (155.6270,228.3780) -- (155.5350,228.6290) -- cycle(155.6590,227.8220) -- (155.7500,227.5720) -- (155.9000,227.6260) -- (155.8090,227.8770) -- cycle(155.9320,227.0700) -- (156.0230,226.8200) -- (156.1740,226.8750) -- (156.0830,227.1250) -- cycle(156.2060,226.3190) -- (156.2970,226.0680) -- (156.4470,226.1230) -- (156.3560,226.3730) -- cycle(156.4790,225.5670) -- (156.5710,225.3160) -- (156.7210,225.3710) -- (156.6300,225.6220) -- cycle(156.7530,224.8150) -- (156.8440,224.5650) -- (156.9950,224.6190) -- (156.9030,224.8700) -- cycle(157.0270,224.0640) -- (157.1180,223.8130) -- (157.2680,223.8680) -- (157.1770,224.1180) -- cycle(157.3000,223.3120) -- (157.3910,223.0610) -- (157.5420,223.1160) -- (157.4510,223.3660) -- cycle(157.5740,222.5600) -- (157.6650,222.3090) -- (157.8150,222.3640) -- (157.7240,222.6150) -- cycle(157.8470,221.8080) -- (157.9390,221.5580) -- (158.0890,221.6120) -- (157.9980,221.8630) -- cycle(158.1210,221.0560) -- (158.2120,220.8060) -- (158.3630,220.8610) -- (158.2710,221.1110) -- cycle(158.3950,220.3050) -- (158.4860,220.0540) -- (158.6360,220.1090) -- (158.5450,220.3590) -- cycle(158.6680,219.5530) -- (158.7590,219.3020) -- (158.9100,219.3570) -- (158.8190,219.6080) -- cycle(158.9420,218.8010) -- (159.0330,218.5510) -- (159.1840,218.6050) -- (159.0920,218.8560) -- cycle(159.2160,218.0490) -- (159.3070,217.7990) -- (159.4570,217.8540) -- (159.3660,218.1040) -- cycle(159.4890,217.2980) -- (159.5800,217.0470) -- (159.7310,217.1020) -- (159.6400,217.3520) -- cycle(159.7630,216.5460) -- (159.8540,216.2950) -- (160.0040,216.3500) -- (159.9130,216.6010) -- cycle(160.0360,215.7940) -- (160.1280,215.5440) -- (160.2780,215.5980) -- (160.1870,215.8490) -- cycle(160.3100,215.0420) -- (160.4010,214.7920) -- (160.5520,214.8470) -- (160.4600,215.0970) -- cycle(160.5840,214.2910) -- (160.6750,214.0400) -- (160.8250,214.0950) -- (160.7340,214.3450) -- cycle(160.8570,213.5390) -- (160.9480,213.2880) -- (161.0990,213.3430) -- (161.0080,213.5940) -- cycle(161.1310,212.7870) -- (161.2220,212.5370) -- (161.3720,212.5910) -- (161.2810,212.8420) -- cycle(161.4040,212.0350) -- (161.4960,211.7850) -- (161.6460,211.8400) -- (161.5550,212.0900) -- cycle(161.6780,211.2840) -- (161.7690,211.0330) -- (161.9200,211.0880) -- (161.8280,211.3380) -- cycle(161.9520,210.5320) -- (162.0430,210.2810) -- (162.1930,210.3360) -- (162.1020,210.5870) -- cycle(162.2250,209.7800) -- (162.3160,209.5300) -- (162.4670,209.5840) -- (162.3760,209.8350) -- cycle(162.4990,209.0280) -- (162.5900,208.7780) -- (162.7400,208.8330) -- (162.6490,209.0830) -- cycle(162.7730,208.2770) -- (162.8640,208.0260) -- (163.0140,208.0810) -- (162.9230,208.3310) -- cycle(163.0460,207.5250) -- (163.1370,207.2740) -- (163.2880,207.3290) -- (163.1970,207.5800) -- cycle(163.3200,206.7730) -- (163.4110,206.5230) -- (163.5610,206.5770) -- (163.4700,206.8280) -- cycle(163.5930,206.0210) -- (163.6850,205.7710) -- (163.8350,205.8250) -- (163.7440,206.0760) -- cycle(163.8670,205.2700) -- (163.9580,205.0190) -- (164.1090,205.0740) -- (164.0170,205.3240) -- cycle(164.1410,204.5180) -- (164.2320,204.2670) -- (164.3820,204.3220) -- (164.2910,204.5730) -- cycle(164.4140,203.7660) -- (164.5050,203.5160) -- (164.6560,203.5700) -- (164.5650,203.8210) -- cycle(164.6880,203.0140) -- (164.7790,202.7640) -- (164.9290,202.8190) -- (164.8380,203.0690) -- cycle(164.9610,202.2630) -- (165.0530,202.0120) -- (165.2030,202.0670) -- (165.1120,202.3170) -- cycle(165.2350,201.5110) -- (165.3260,201.2600) -- (165.4770,201.3150) -- (165.3850,201.5660) -- cycle(165.5090,200.7590) -- (165.6000,200.5090) -- (165.7500,200.5630) -- (165.6590,200.8140) -- cycle(165.7820,200.0070) -- (165.8730,199.7570) -- (166.0240,199.8110) -- (165.9330,200.0620) -- cycle(166.0560,199.2560) -- (166.1470,199.0050) -- (166.2970,199.0600) -- (166.2060,199.3100) -- cycle(166.3290,198.5040) -- (166.4210,198.2530) -- (166.5710,198.3080) -- (166.4800,198.5590) -- cycle(166.6030,197.7520) -- (166.6940,197.5020) -- (166.8450,197.5560) -- (166.7530,197.8070) -- cycle(166.8770,197.0000) -- (166.9680,196.7500) -- (167.1180,196.8040) -- (167.0270,197.0550) -- cycle(167.1500,196.2490) -- (167.2420,195.9980) -- (167.3920,196.0530) -- (167.3010,196.3030) -- cycle(167.4240,195.4970) -- (167.5150,195.2460) -- (167.6660,195.3010) -- (167.5740,195.5520) -- cycle(167.6980,194.7450) -- (167.7890,194.4950) -- (167.9390,194.5490) -- (167.8480,194.8000) -- cycle(167.9710,193.9930) -- (168.0620,193.7430) -- (168.2130,193.7970) -- (168.1220,194.0480) -- cycle(168.2450,193.2420) -- (168.3360,192.9910) -- (168.4860,193.0460) -- (168.3950,193.2960) -- cycle(168.5180,192.4900) -- (168.6100,192.2390) -- (168.7600,192.2940) -- (168.6690,192.5450) -- cycle(168.7920,191.7380) -- (168.8830,191.4880) -- (169.0340,191.5420) -- (168.9420,191.7930) -- cycle(169.0660,190.9860) -- (169.1570,190.7360) -- (169.3070,190.7900) -- (169.2160,191.0410) -- cycle(169.3390,190.2350) -- (169.4300,189.9840) -- (169.5810,190.0390) -- (169.4900,190.2890) -- cycle(169.6130,189.4830) -- (169.7040,189.2320) -- (169.8540,189.2870) -- (169.7630,189.5380) -- cycle(169.8860,188.7310) -- (169.9780,188.4800) -- (170.1280,188.5350) -- (170.0370,188.7860) -- cycle(170.1600,187.9790) -- (170.2510,187.7290) -- (170.4020,187.7830) -- (170.3100,188.0340) -- cycle(170.4340,187.2280) -- (170.5250,186.9770) -- (170.6750,187.0320) -- (170.5840,187.2820) -- cycle(170.7070,186.4760) -- (170.7980,186.2250) -- (170.9490,186.2800) -- (170.8580,186.5310) -- cycle(170.9810,185.7240) -- (171.0720,185.4730) -- (171.2230,185.5280) -- (171.1310,185.7790) -- cycle(171.2550,184.9720) -- (171.3460,184.7220) -- (171.4960,184.7760) -- (171.4050,185.0270) -- cycle(171.5280,184.2210) -- (171.6190,183.9700) -- (171.7700,184.0250) -- (171.6790,184.2750) -- cycle(171.8020,183.4690) -- (171.8930,183.2180) -- (172.0430,183.2730) -- (171.9520,183.5230) -- cycle(172.0750,182.7170) -- (172.1670,182.4660) -- (172.3170,182.5210) -- (172.2260,182.7720) -- cycle(172.3490,181.9650) -- (172.4400,181.7150) -- (172.5910,181.7690) -- (172.4990,182.0200) -- cycle(172.6230,181.2140) -- (172.7140,180.9630) -- (172.8640,181.0180) -- (172.7730,181.2680) -- cycle(172.8960,180.4620) -- (172.9870,180.2110) -- (173.1380,180.2660) -- (173.0470,180.5170) -- cycle(173.1700,179.7100) -- (173.2610,179.4600) -- (173.4110,179.5140) -- (173.3200,179.7650) -- cycle(173.4430,178.9580) -- (173.5350,178.7080) -- (173.6850,178.7620) -- (173.5940,179.0130) -- cycle(173.7170,178.2070) -- (173.8080,177.9560) -- (173.9590,178.0110) -- (173.8670,178.2610) -- cycle(173.9910,177.4550) -- (174.0820,177.2040) -- (174.2320,177.2590) -- (174.1410,177.5090) -- cycle(174.2640,176.7030) -- (174.3550,176.4520) -- (174.5060,176.5070) -- (174.4150,176.7580) -- cycle(174.5380,175.9510) -- (174.6290,175.7010) -- (174.7790,175.7550) -- (174.6880,176.0060) -- cycle(174.8110,175.1990) -- (174.9030,174.9490) -- (175.0530,175.0040) -- (174.9620,175.2540) -- cycle(175.0850,174.4480) -- (175.1760,174.1970) -- (175.3270,174.2520) -- (175.2350,174.5020) -- cycle(175.3590,173.6960) -- (175.4500,173.4450) -- (175.6000,173.5000) -- (175.5090,173.7510) -- cycle(175.6320,172.9440) -- (175.7240,172.6940) -- (175.8740,172.7480) -- (175.7830,172.9990) -- cycle(175.9060,172.1930) -- (175.9970,171.9420) -- (176.1480,171.9970) -- (176.0560,172.2470) -- cycle(176.1800,171.4410) -- (176.2710,171.1900) -- (176.4210,171.2450) -- (176.3300,171.4950) -- cycle(176.4530,170.6890) -- (176.5440,170.4380) -- (176.6950,170.4930) -- (176.6040,170.7440) -- cycle(176.7270,169.9370) -- (176.8180,169.6870) -- (176.9680,169.7410) -- (176.8770,169.9920) -- cycle(177.0000,169.1860) -- (177.0920,168.9350) -- (177.2420,168.9900) -- (177.1510,169.2400) -- cycle(177.2740,168.4340) -- (177.3650,168.1830) -- (177.5160,168.2380) -- (177.4240,168.4880) -- cycle(177.5480,167.6820) -- (177.6390,167.4310) -- (177.7890,167.4860) -- (177.6980,167.7370) -- cycle(177.8210,166.9300) -- (177.9120,166.6800) -- (178.0630,166.7340) -- (177.9720,166.9850) -- cycle(178.0950,166.1780) -- (178.1860,165.9280) -- (178.3360,165.9830) -- (178.2450,166.2330) -- cycle(178.3680,165.4270) -- (178.4600,165.1760) -- (178.6100,165.2310) -- (178.5190,165.4810) -- cycle(178.6420,164.6750) -- (178.7330,164.4240) -- (178.8840,164.4790) -- (178.7920,164.7300) -- cycle(178.9160,163.9230) -- (179.0070,163.6730) -- (179.1570,163.7270) -- (179.0660,163.9780) -- cycle(179.1890,163.1710) -- (179.2800,162.9210) -- (179.4310,162.9760) -- (179.3400,163.2260) -- cycle(179.4630,162.4200) -- (179.5540,162.1690) -- (179.7050,162.2240) -- (179.6130,162.4740) -- cycle(179.7370,161.6680) -- (179.8280,161.4170) -- (179.9780,161.4720) -- (179.8870,161.7230) -- cycle(180.0100,160.9160) -- (180.1010,160.6660) -- (180.2520,160.7200) -- (180.1610,160.9710) -- cycle(180.2840,160.1640) -- (180.3750,159.9140) -- (180.5250,159.9690) -- (180.4340,160.2190) -- cycle(180.5570,159.4130) -- (180.6490,159.1620) -- (180.7990,159.2170) -- (180.7080,159.4670) -- cycle(180.8310,158.6610) -- (180.9220,158.4100) -- (181.0730,158.4650) -- (180.9810,158.7160) -- cycle(181.1050,157.9090) -- (181.1960,157.6590) -- (181.3460,157.7130) -- (181.2550,157.9640) -- cycle(181.3780,157.1570) -- (181.4690,156.9070) -- (181.6200,156.9620) -- (181.5290,157.2120) -- cycle(181.6520,156.4060) -- (181.7430,156.1550) -- (181.8930,156.2100) -- (181.8020,156.4600) -- cycle(181.9250,155.6540) -- (182.0170,155.4030) -- (182.1670,155.4580) -- (182.0760,155.7090) -- cycle(182.1990,154.9020) -- (182.2900,154.6520) -- (182.4410,154.7060) -- (182.3490,154.9570) -- cycle(132.1280,292.4730) -- (132.2190,292.2220) -- (132.3690,292.2770) -- (132.2780,292.5280) -- cycle;



  \path[fill=c7e7e7e,line join=round,line width=0.256pt] (158.0830,291.7280) -- (158.1740,291.4770) -- (158.3250,291.5320) -- (158.2340,291.7820) -- cycle(158.3570,290.9760) -- (158.4480,290.7250) -- (158.5980,290.7800) -- (158.5070,291.0310) -- cycle(158.6300,290.2240) -- (158.7220,289.9730) -- (158.8720,290.0280) -- (158.7810,290.2790) -- cycle(158.9040,289.4720) -- (158.9950,289.2220) -- (159.1460,289.2760) -- (159.0540,289.5270) -- cycle(159.1780,288.7210) -- (159.2690,288.4700) -- (159.4190,288.5250) -- (159.3280,288.7750) -- cycle(159.4510,287.9690) -- (159.5420,287.7180) -- (159.6930,287.7730) -- (159.6020,288.0240) -- cycle(159.7250,287.2170) -- (159.8160,286.9670) -- (159.9660,287.0210) -- (159.8750,287.2720) -- cycle(159.9980,286.4650) -- (160.0900,286.2150) -- (160.2400,286.2700) -- (160.1490,286.5200) -- cycle(160.2720,285.7140) -- (160.3630,285.4630) -- (160.5140,285.5180) -- (160.4220,285.7680) -- cycle(160.5460,284.9620) -- (160.6370,284.7110) -- (160.7870,284.7660) -- (160.6960,285.0170) -- cycle(160.8190,284.2100) -- (160.9100,283.9600) -- (161.0610,284.0140) -- (160.9700,284.2650) -- cycle(161.0930,283.4580) -- (161.1840,283.2080) -- (161.3340,283.2620) -- (161.2430,283.5130) -- cycle(161.3670,282.7070) -- (161.4580,282.4560) -- (161.6080,282.5110) -- (161.5170,282.7610) -- cycle(161.6400,281.9550) -- (161.7310,281.7040) -- (161.8820,281.7590) -- (161.7900,282.0090) -- cycle(161.9140,281.2030) -- (162.0050,280.9520) -- (162.1550,281.0070) -- (162.0640,281.2580) -- cycle(162.1870,280.4510) -- (162.2790,280.2010) -- (162.4290,280.2550) -- (162.3380,280.5060) -- cycle(162.4610,279.7000) -- (162.5520,279.4490) -- (162.7030,279.5040) -- (162.6110,279.7540) -- cycle(162.7350,278.9480) -- (162.8260,278.6970) -- (162.9760,278.7520) -- (162.8850,279.0030) -- cycle(163.0080,278.1960) -- (163.0990,277.9460) -- (163.2500,278.0000) -- (163.1590,278.2510) -- cycle(163.2820,277.4440) -- (163.3730,277.1940) -- (163.5230,277.2480) -- (163.4320,277.4990) -- cycle(163.5550,276.6930) -- (163.6470,276.4420) -- (163.7970,276.4970) -- (163.7060,276.7470) -- cycle(163.8290,275.9410) -- (163.9200,275.6900) -- (164.0710,275.7450) -- (163.9790,275.9950) -- cycle(164.1030,275.1890) -- (164.1940,274.9380) -- (164.3440,274.9930) -- (164.2530,275.2440) -- cycle(164.3760,274.4370) -- (164.4670,274.1870) -- (164.6180,274.2410) -- (164.5270,274.4920) -- cycle(164.6500,273.6860) -- (164.7410,273.4350) -- (164.8910,273.4900) -- (164.8000,273.7400) -- cycle(164.9230,272.9340) -- (165.0150,272.6830) -- (165.1650,272.7380) -- (165.0740,272.9880) -- cycle(165.1970,272.1820) -- (165.2880,271.9310) -- (165.4390,271.9860) -- (165.3480,272.2370) -- cycle(165.4710,271.4300) -- (165.5620,271.1800) -- (165.7120,271.2340) -- (165.6210,271.4850) -- cycle(165.7440,270.6790) -- (165.8360,270.4280) -- (165.9860,270.4830) -- (165.8950,270.7330) -- cycle(166.0180,269.9270) -- (166.1090,269.6760) -- (166.2600,269.7310) -- (166.1680,269.9820) -- cycle(166.2920,269.1750) -- (166.3830,268.9240) -- (166.5330,268.9790) -- (166.4420,269.2300) -- cycle(166.5650,268.4230) -- (166.6560,268.1730) -- (166.8070,268.2270) -- (166.7160,268.4780) -- cycle(166.8390,267.6710) -- (166.9300,267.4210) -- (167.0800,267.4760) -- (166.9890,267.7260) -- cycle(167.1120,266.9200) -- (167.2040,266.6690) -- (167.3540,266.7240) -- (167.2630,266.9740) -- cycle(167.3860,266.1680) -- (167.4770,265.9170) -- (167.6280,265.9720) -- (167.5360,266.2230) -- cycle(167.6600,265.4160) -- (167.7510,265.1660) -- (167.9010,265.2200) -- (167.8100,265.4710) -- cycle(167.9330,264.6650) -- (168.0240,264.4140) -- (168.1750,264.4690) -- (168.0840,264.7190) -- cycle(168.2070,263.9130) -- (168.2980,263.6620) -- (168.4480,263.7170) -- (168.3570,263.9670) -- cycle(168.4800,263.1610) -- (168.5720,262.9100) -- (168.7220,262.9650) -- (168.6310,263.2160) -- cycle(168.7540,262.4090) -- (168.8450,262.1590) -- (168.9960,262.2130) -- (168.9040,262.4640) -- cycle(169.0280,261.6580) -- (169.1190,261.4070) -- (169.2690,261.4620) -- (169.1780,261.7120) -- cycle(169.3010,260.9060) -- (169.3930,260.6550) -- (169.5430,260.7100) -- (169.4520,260.9600) -- cycle(169.5750,260.1540) -- (169.6660,259.9030) -- (169.8160,259.9580) -- (169.7250,260.2090) -- cycle(169.8490,259.4020) -- (169.9400,259.1520) -- (170.0900,259.2060) -- (169.9990,259.4570) -- cycle(170.1220,258.6500) -- (170.2130,258.4000) -- (170.3640,258.4550) -- (170.2730,258.7050) -- cycle(170.3960,257.8990) -- (170.4870,257.6480) -- (170.6370,257.7030) -- (170.5460,257.9530) -- cycle(170.6690,257.1470) -- (170.7610,256.8960) -- (170.9110,256.9510) -- (170.8200,257.2020) -- cycle(170.9430,256.3950) -- (171.0340,256.1450) -- (171.1850,256.1990) -- (171.0930,256.4500) -- cycle(171.2170,255.6440) -- (171.3080,255.3930) -- (171.4580,255.4480) -- (171.3670,255.6980) -- cycle(171.4900,254.8920) -- (171.5810,254.6410) -- (171.7320,254.6960) -- (171.6410,254.9460) -- cycle(171.7640,254.1400) -- (171.8550,253.8890) -- (172.0050,253.9440) -- (171.9140,254.1950) -- cycle(172.0370,253.3880) -- (172.1290,253.1380) -- (172.2790,253.1920) -- (172.1880,253.4430) -- cycle(172.3110,252.6360) -- (172.4020,252.3860) -- (172.5530,252.4410) -- (172.4610,252.6910) -- cycle(172.5850,251.8850) -- (172.6760,251.6340) -- (172.8260,251.6890) -- (172.7350,251.9390) -- cycle(172.8580,251.1330) -- (172.9490,250.8820) -- (173.1000,250.9370) -- (173.0090,251.1880) -- cycle(173.1320,250.3810) -- (173.2230,250.1310) -- (173.3730,250.1850) -- (173.2820,250.4360) -- cycle(173.4050,249.6290) -- (173.4970,249.3790) -- (173.6470,249.4340) -- (173.5560,249.6840) -- cycle(173.6790,248.8780) -- (173.7700,248.6270) -- (173.9210,248.6820) -- (173.8300,248.9320) -- cycle(173.9530,248.1260) -- (174.0440,247.8750) -- (174.1940,247.9300) -- (174.1030,248.1810) -- cycle(174.2260,247.3740) -- (174.3180,247.1240) -- (174.4680,247.1780) -- (174.3770,247.4290) -- cycle(174.5000,246.6220) -- (174.5910,246.3720) -- (174.7420,246.4270) -- (174.6500,246.6770) -- cycle(174.7740,245.8710) -- (174.8650,245.6200) -- (175.0150,245.6750) -- (174.9240,245.9250) -- cycle(175.0470,245.1190) -- (175.1380,244.8680) -- (175.2890,244.9230) -- (175.1980,245.1740) -- cycle(175.3210,244.3670) -- (175.4120,244.1170) -- (175.5620,244.1710) -- (175.4710,244.4220) -- cycle(175.5940,243.6150) -- (175.6860,243.3650) -- (175.8360,243.4200) -- (175.7450,243.6700) -- cycle(175.8680,242.8640) -- (175.9590,242.6130) -- (176.1100,242.6680) -- (176.0180,242.9180) -- cycle(176.1420,242.1120) -- (176.2330,241.8610) -- (176.3830,241.9160) -- (176.2920,242.1670) -- cycle(176.4150,241.3600) -- (176.5060,241.1100) -- (176.6570,241.1640) -- (176.5660,241.4150) -- cycle(176.6890,240.6080) -- (176.7800,240.3580) -- (176.9300,240.4130) -- (176.8390,240.6630) -- cycle(176.9620,239.8570) -- (177.0540,239.6060) -- (177.2040,239.6610) -- (177.1130,239.9110) -- cycle(177.2360,239.1050) -- (177.3270,238.8540) -- (177.4780,238.9090) -- (177.3860,239.1600) -- cycle(177.5100,238.3530) -- (177.6010,238.1030) -- (177.7510,238.1570) -- (177.6600,238.4080) -- cycle(177.7830,237.6010) -- (177.8750,237.3510) -- (178.0250,237.4060) -- (177.9340,237.6560) -- cycle(178.0570,236.8500) -- (178.1480,236.5990) -- (178.2990,236.6540) -- (178.2070,236.9040) -- cycle(178.3310,236.0980) -- (178.4220,235.8470) -- (178.5720,235.9020) -- (178.4810,236.1530) -- cycle(178.6040,235.3460) -- (178.6950,235.0960) -- (178.8460,235.1500) -- (178.7550,235.4010) -- cycle(178.8780,234.5940) -- (178.9690,234.3440) -- (179.1190,234.3980) -- (179.0280,234.6490) -- cycle(179.1510,233.8430) -- (179.2430,233.5920) -- (179.3930,233.6470) -- (179.3020,233.8970) -- cycle(179.4250,233.0910) -- (179.5160,232.8400) -- (179.6670,232.8950) -- (179.5750,233.1460) -- cycle(179.6990,232.3390) -- (179.7900,232.0890) -- (179.9400,232.1430) -- (179.8490,232.3940) -- cycle(179.9720,231.5870) -- (180.0630,231.3370) -- (180.2140,231.3920) -- (180.1230,231.6420) -- cycle(180.2460,230.8360) -- (180.3370,230.5850) -- (180.4870,230.6400) -- (180.3960,230.8900) -- cycle(180.5190,230.0840) -- (180.6110,229.8330) -- (180.7610,229.8880) -- (180.6700,230.1390) -- cycle(180.7930,229.3320) -- (180.8840,229.0820) -- (181.0350,229.1360) -- (180.9430,229.3870) -- cycle(181.0670,228.5800) -- (181.1580,228.3300) -- (181.3080,228.3840) -- (181.2170,228.6350) -- cycle(181.3400,227.8290) -- (181.4310,227.5780) -- (181.5820,227.6330) -- (181.4910,227.8830) -- cycle(181.6140,227.0770) -- (181.7050,226.8260) -- (181.8550,226.8810) -- (181.7640,227.1320) -- cycle(181.8880,226.3250) -- (181.9790,226.0750) -- (182.1290,226.1290) -- (182.0380,226.3800) -- cycle(182.1610,225.5730) -- (182.2520,225.3230) -- (182.4030,225.3770) -- (182.3110,225.6280) -- cycle(182.4350,224.8220) -- (182.5260,224.5710) -- (182.6760,224.6260) -- (182.5850,224.8760) -- cycle(182.7080,224.0700) -- (182.8000,223.8190) -- (182.9500,223.8740) -- (182.8590,224.1250) -- cycle(182.9820,223.3180) -- (183.0730,223.0680) -- (183.2240,223.1220) -- (183.1320,223.3730) -- cycle(183.2560,222.5660) -- (183.3470,222.3160) -- (183.4970,222.3700) -- (183.4060,222.6210) -- cycle(183.5290,221.8150) -- (183.6200,221.5640) -- (183.7710,221.6190) -- (183.6800,221.8690) -- cycle(183.8030,221.0630) -- (183.8940,220.8120) -- (184.0440,220.8670) -- (183.9530,221.1180) -- cycle(184.0760,220.3110) -- (184.1680,220.0610) -- (184.3180,220.1150) -- (184.2270,220.3660) -- cycle(184.3500,219.5590) -- (184.4410,219.3090) -- (184.5920,219.3630) -- (184.5000,219.6140) -- cycle(184.6240,218.8080) -- (184.7150,218.5570) -- (184.8650,218.6120) -- (184.7740,218.8620) -- cycle(184.8970,218.0560) -- (184.9880,217.8050) -- (185.1390,217.8600) -- (185.0480,218.1110) -- cycle(185.1710,217.3040) -- (185.2620,217.0530) -- (185.4120,217.1080) -- (185.3210,217.3590) -- cycle(185.4440,216.5520) -- (185.5360,216.3020) -- (185.6860,216.3560) -- (185.5950,216.6070) -- cycle(185.7180,215.8010) -- (185.8090,215.5500) -- (185.9600,215.6050) -- (185.8680,215.8550) -- cycle(185.9920,215.0490) -- (186.0830,214.7980) -- (186.2330,214.8530) -- (186.1420,215.1040) -- cycle(186.2650,214.2970) -- (186.3570,214.0460) -- (186.5070,214.1010) -- (186.4160,214.3520) -- cycle(186.5390,213.5450) -- (186.6300,213.2950) -- (186.7810,213.3490) -- (186.6890,213.6000) -- cycle(186.8130,212.7940) -- (186.9040,212.5430) -- (187.0540,212.5980) -- (186.9630,212.8480) -- cycle(187.0860,212.0420) -- (187.1770,211.7910) -- (187.3280,211.8460) -- (187.2370,212.0960) -- cycle(187.3600,211.2900) -- (187.4510,211.0390) -- (187.6010,211.0940) -- (187.5100,211.3450) -- cycle(187.6330,210.5380) -- (187.7250,210.2880) -- (187.8750,210.3420) -- (187.7840,210.5930) -- cycle(187.9070,209.7870) -- (187.9980,209.5360) -- (188.1490,209.5910) -- (188.0570,209.8410) -- cycle(188.1810,209.0350) -- (188.2720,208.7840) -- (188.4220,208.8390) -- (188.3310,209.0900) -- cycle(188.4540,208.2830) -- (188.5450,208.0330) -- (188.6960,208.0870) -- (188.6050,208.3380) -- cycle(188.7280,207.5310) -- (188.8190,207.2810) -- (188.9690,207.3350) -- (188.8780,207.5860) -- cycle(189.0010,206.7800) -- (189.0930,206.5290) -- (189.2430,206.5840) -- (189.1520,206.8340) -- cycle(189.2750,206.0280) -- (189.3660,205.7770) -- (189.5170,205.8320) -- (189.4250,206.0820) -- cycle(189.5490,205.2760) -- (189.6400,205.0250) -- (189.7900,205.0800) -- (189.6990,205.3310) -- cycle(189.8220,204.5240) -- (189.9140,204.2740) -- (190.0640,204.3280) -- (189.9730,204.5790) -- cycle(190.0960,203.7720) -- (190.1870,203.5220) -- (190.3370,203.5770) -- (190.2460,203.8270) -- cycle(190.3700,203.0210) -- (190.4610,202.7700) -- (190.6110,202.8250) -- (190.5200,203.0750) -- cycle(190.6430,202.2690) -- (190.7340,202.0180) -- (190.8850,202.0730) -- (190.7940,202.3240) -- cycle(190.9170,201.5170) -- (191.0080,201.2670) -- (191.1580,201.3210) -- (191.0670,201.5720) -- cycle(191.1900,200.7660) -- (191.2820,200.5150) -- (191.4320,200.5700) -- (191.3410,200.8200) -- cycle(191.4640,200.0140) -- (191.5550,199.7630) -- (191.7060,199.8180) -- (191.6140,200.0680) -- cycle(191.7380,199.2620) -- (191.8290,199.0110) -- (191.9790,199.0660) -- (191.8880,199.3170) -- cycle(192.0110,198.5100) -- (192.1020,198.2600) -- (192.2530,198.3140) -- (192.1620,198.5650) -- cycle(192.2850,197.7590) -- (192.3760,197.5080) -- (192.5260,197.5630) -- (192.4350,197.8130) -- cycle(192.5580,197.0070) -- (192.6500,196.7560) -- (192.8000,196.8110) -- (192.7090,197.0610) -- cycle(192.8320,196.2550) -- (192.9230,196.0040) -- (193.0740,196.0590) -- (192.9820,196.3100) -- cycle(193.1060,195.5030) -- (193.1970,195.2530) -- (193.3470,195.3070) -- (193.2560,195.5580) -- cycle(193.3790,194.7510) -- (193.4700,194.5010) -- (193.6210,194.5560) -- (193.5300,194.8060) -- cycle(193.6530,194.0000) -- (193.7440,193.7490) -- (193.8940,193.8040) -- (193.8030,194.0540) -- cycle(193.9260,193.2480) -- (194.0180,192.9970) -- (194.1680,193.0520) -- (194.0770,193.3030) -- cycle(194.2000,192.4960) -- (194.2910,192.2460) -- (194.4420,192.3000) -- (194.3500,192.5510) -- cycle(194.4740,191.7440) -- (194.5650,191.4940) -- (194.7150,191.5490) -- (194.6240,191.7990) -- cycle(194.7470,190.9930) -- (194.8390,190.7420) -- (194.9890,190.7970) -- (194.8980,191.0470) -- cycle(195.0210,190.2410) -- (195.1120,189.9900) -- (195.2630,190.0450) -- (195.1710,190.2960) -- cycle(195.2950,189.4890) -- (195.3860,189.2390) -- (195.5360,189.2930) -- (195.4450,189.5440) -- cycle(195.5680,188.7370) -- (195.6590,188.4870) -- (195.8100,188.5420) -- (195.7190,188.7920) -- cycle(195.8420,187.9860) -- (195.9330,187.7350) -- (196.0830,187.7900) -- (195.9920,188.0400) -- cycle(196.1150,187.2340) -- (196.2070,186.9830) -- (196.3570,187.0380) -- (196.2660,187.2890) -- cycle(196.3890,186.4820) -- (196.4800,186.2320) -- (196.6310,186.2860) -- (196.5390,186.5370) -- cycle(196.6630,185.7300) -- (196.7540,185.4800) -- (196.9040,185.5350) -- (196.8130,185.7850) -- cycle(196.9360,184.9790) -- (197.0270,184.7280) -- (197.1780,184.7830) -- (197.0870,185.0330) -- cycle(197.2100,184.2270) -- (197.3010,183.9760) -- (197.4510,184.0310) -- (197.3600,184.2820) -- cycle(197.4830,183.4750) -- (197.5750,183.2250) -- (197.7250,183.2790) -- (197.6340,183.5300) -- cycle(197.7570,182.7230) -- (197.8480,182.4730) -- (197.9990,182.5280) -- (197.9070,182.7780) -- cycle(198.0310,181.9720) -- (198.1220,181.7210) -- (198.2720,181.7760) -- (198.1810,182.0260) -- cycle(198.3040,181.2200) -- (198.3960,180.9690) -- (198.5460,181.0240) -- (198.4550,181.2750) -- cycle(198.5780,180.4680) -- (198.6690,180.2180) -- (198.8200,180.2720) -- (198.7280,180.5230) -- cycle(198.8520,179.7160) -- (198.9430,179.4660) -- (199.0930,179.5210) -- (199.0020,179.7710) -- cycle(199.1250,178.9650) -- (199.2160,178.7140) -- (199.3670,178.7690) -- (199.2760,179.0190) -- cycle(199.3990,178.2130) -- (199.4900,177.9620) -- (199.6400,178.0170) -- (199.5490,178.2680) -- cycle(199.6720,177.4610) -- (199.7640,177.2110) -- (199.9140,177.2650) -- (199.8230,177.5160) -- cycle(199.9460,176.7090) -- (200.0370,176.4590) -- (200.1880,176.5140) -- (200.0960,176.7640) -- cycle(200.2200,175.9580) -- (200.3110,175.7070) -- (200.4610,175.7620) -- (200.3700,176.0120) -- cycle(200.4930,175.2060) -- (200.5840,174.9550) -- (200.7350,175.0100) -- (200.6440,175.2610) -- cycle(200.7670,174.4540) -- (200.8580,174.2040) -- (201.0080,174.2580) -- (200.9170,174.5090) -- cycle(201.0400,173.7020) -- (201.1320,173.4520) -- (201.2820,173.5070) -- (201.1910,173.7570) -- cycle(201.3140,172.9510) -- (201.4050,172.7000) -- (201.5560,172.7550) -- (201.4640,173.0050) -- cycle(201.5880,172.1990) -- (201.6790,171.9480) -- (201.8290,172.0030) -- (201.7380,172.2540) -- cycle(201.8610,171.4470) -- (201.9520,171.1970) -- (202.1030,171.2510) -- (202.0120,171.5020) -- cycle(202.1350,170.6950) -- (202.2260,170.4450) -- (202.3760,170.5000) -- (202.2850,170.7500) -- cycle(202.4090,169.9440) -- (202.5000,169.6930) -- (202.6500,169.7480) -- (202.5590,169.9980) -- cycle(202.6820,169.1920) -- (202.7730,168.9410) -- (202.9240,168.9960) -- (202.8320,169.2470) -- cycle(202.9560,168.4400) -- (203.0470,168.1900) -- (203.1970,168.2440) -- (203.1060,168.4950) -- cycle(203.2290,167.6880) -- (203.3210,167.4380) -- (203.4710,167.4920) -- (203.3800,167.7430) -- cycle(203.5030,166.9370) -- (203.5940,166.6860) -- (203.7450,166.7410) -- (203.6530,166.9910) -- cycle(203.7770,166.1850) -- (203.8680,165.9340) -- (204.0180,165.9890) -- (203.9270,166.2400) -- cycle(204.0500,165.4330) -- (204.1410,165.1830) -- (204.2920,165.2370) -- (204.2010,165.4880) -- cycle(204.3240,164.6810) -- (204.4150,164.4310) -- (204.5650,164.4860) -- (204.4740,164.7360) -- cycle(204.5970,163.9300) -- (204.6890,163.6790) -- (204.8390,163.7340) -- (204.7480,163.9840) -- cycle(204.8710,163.1780) -- (204.9620,162.9270) -- (205.1130,162.9820) -- (205.0210,163.2330) -- cycle(205.1450,162.4260) -- (205.2360,162.1760) -- (205.3860,162.2300) -- (205.2950,162.4810) -- cycle(205.4180,161.6740) -- (205.5090,161.4240) -- (205.6600,161.4780) -- (205.5690,161.7290) -- cycle(205.6920,160.9230) -- (205.7830,160.6720) -- (205.9330,160.7270) -- (205.8420,160.9770) -- cycle(205.9650,160.1710) -- (206.0570,159.9200) -- (206.2070,159.9750) -- (206.1160,160.2260) -- cycle(206.2390,159.4190) -- (206.3300,159.1680) -- (206.4810,159.2230) -- (206.3890,159.4740) -- cycle(206.5130,158.6670) -- (206.6040,158.4170) -- (206.7540,158.4710) -- (206.6630,158.7220) -- cycle(206.7860,157.9160) -- (206.8780,157.6650) -- (207.0280,157.7200) -- (206.9370,157.9700) -- cycle(207.0600,157.1640) -- (207.1510,156.9130) -- (207.3020,156.9680) -- (207.2100,157.2190) -- cycle(207.3340,156.4120) -- (207.4250,156.1620) -- (207.5750,156.2160) -- (207.4840,156.4670) -- cycle(207.6070,155.6600) -- (207.6980,155.4100) -- (207.8490,155.4640) -- (207.7580,155.7150) -- cycle(207.8810,154.9090) -- (207.9720,154.6580) -- (208.1220,154.7130) -- (208.0310,154.9630) -- cycle(157.8100,292.4790) -- (157.9010,292.2290) -- (158.0510,292.2840) -- (157.9600,292.5340) -- cycle;



  \path[fill=c7e7e7e,line join=round,line width=0.256pt] (183.8090,291.7520) -- (183.9000,291.5020) -- (184.0510,291.5560) -- (183.9600,291.8070) -- cycle(184.0830,291.0000) -- (184.1740,290.7500) -- (184.3240,290.8050) -- (184.2330,291.0550) -- cycle(184.3560,290.2490) -- (184.4480,289.9980) -- (184.5980,290.0530) -- (184.5070,290.3030) -- cycle(184.6300,289.4970) -- (184.7210,289.2460) -- (184.8720,289.3010) -- (184.7800,289.5520) -- cycle(184.9040,288.7450) -- (184.9950,288.4950) -- (185.1450,288.5490) -- (185.0540,288.8000) -- cycle(185.1770,287.9930) -- (185.2680,287.7430) -- (185.4190,287.7970) -- (185.3280,288.0480) -- cycle(185.4510,287.2420) -- (185.5420,286.9910) -- (185.6930,287.0460) -- (185.6010,287.2960) -- cycle(185.7250,286.4900) -- (185.8160,286.2390) -- (185.9660,286.2940) -- (185.8750,286.5450) -- cycle(185.9980,285.7380) -- (186.0890,285.4880) -- (186.2400,285.5420) -- (186.1490,285.7930) -- cycle(186.2720,284.9860) -- (186.3630,284.7360) -- (186.5130,284.7900) -- (186.4220,285.0410) -- cycle(186.5450,284.2350) -- (186.6370,283.9840) -- (186.7870,284.0390) -- (186.6960,284.2890) -- cycle(186.8190,283.4830) -- (186.9100,283.2320) -- (187.0610,283.2870) -- (186.9690,283.5380) -- cycle(187.0930,282.7310) -- (187.1840,282.4810) -- (187.3340,282.5350) -- (187.2430,282.7860) -- cycle(187.3660,281.9790) -- (187.4570,281.7290) -- (187.6080,281.7840) -- (187.5170,282.0340) -- cycle(187.6400,281.2280) -- (187.7310,280.9770) -- (187.8810,281.0320) -- (187.7900,281.2820) -- cycle(187.9130,280.4760) -- (188.0050,280.2250) -- (188.1550,280.2800) -- (188.0640,280.5310) -- cycle(188.1870,279.7240) -- (188.2780,279.4730) -- (188.4290,279.5280) -- (188.3370,279.7790) -- cycle(188.4610,278.9720) -- (188.5520,278.7220) -- (188.7020,278.7760) -- (188.6110,279.0270) -- cycle(188.7340,278.2210) -- (188.8250,277.9700) -- (188.9760,278.0250) -- (188.8850,278.2750) -- cycle(189.0080,277.4690) -- (189.0990,277.2180) -- (189.2490,277.2730) -- (189.1580,277.5240) -- cycle(189.2810,276.7170) -- (189.3730,276.4670) -- (189.5230,276.5210) -- (189.4320,276.7720) -- cycle(189.5550,275.9650) -- (189.6460,275.7150) -- (189.7970,275.7700) -- (189.7060,276.0200) -- cycle(189.8290,275.2140) -- (189.9200,274.9630) -- (190.0700,275.0180) -- (189.9790,275.2680) -- cycle(190.1020,274.4620) -- (190.1940,274.2110) -- (190.3440,274.2660) -- (190.2530,274.5170) -- cycle(190.3760,273.7100) -- (190.4670,273.4600) -- (190.6180,273.5140) -- (190.5260,273.7650) -- cycle(190.6500,272.9580) -- (190.7410,272.7080) -- (190.8910,272.7620) -- (190.8000,273.0130) -- cycle(190.9230,272.2070) -- (191.0140,271.9560) -- (191.1650,272.0110) -- (191.0740,272.2610) -- cycle(191.1970,271.4550) -- (191.2880,271.2040) -- (191.4380,271.2590) -- (191.3470,271.5090) -- cycle(191.4700,270.7030) -- (191.5620,270.4520) -- (191.7120,270.5070) -- (191.6210,270.7580) -- cycle(191.7440,269.9510) -- (191.8350,269.7010) -- (191.9860,269.7550) -- (191.8940,270.0060) -- cycle(192.0180,269.2000) -- (192.1090,268.9490) -- (192.2590,269.0040) -- (192.1680,269.2540) -- cycle(192.2910,268.4480) -- (192.3820,268.1970) -- (192.5330,268.2520) -- (192.4420,268.5030) -- cycle(192.5650,267.6960) -- (192.6560,267.4460) -- (192.8060,267.5000) -- (192.7150,267.7510) -- cycle(192.8380,266.9440) -- (192.9300,266.6940) -- (193.0800,266.7480) -- (192.9890,266.9990) -- cycle(193.1120,266.1930) -- (193.2030,265.9420) -- (193.3540,265.9970) -- (193.2620,266.2470) -- cycle(193.3860,265.4410) -- (193.4770,265.1900) -- (193.6270,265.2450) -- (193.5360,265.4950) -- cycle(193.6590,264.6890) -- (193.7500,264.4380) -- (193.9010,264.4930) -- (193.8100,264.7440) -- cycle(193.9330,263.9370) -- (194.0240,263.6870) -- (194.1750,263.7410) -- (194.0830,263.9920) -- cycle(194.2070,263.1860) -- (194.2980,262.9350) -- (194.4480,262.9900) -- (194.3570,263.2400) -- cycle(194.4800,262.4340) -- (194.5710,262.1830) -- (194.7220,262.2380) -- (194.6310,262.4880) -- cycle(194.7540,261.6820) -- (194.8450,261.4310) -- (194.9950,261.4860) -- (194.9040,261.7370) -- cycle(195.0270,260.9300) -- (195.1190,260.6800) -- (195.2690,260.7340) -- (195.1780,260.9850) -- cycle(195.3010,260.1790) -- (195.3920,259.9280) -- (195.5430,259.9830) -- (195.4510,260.2330) -- cycle(195.5750,259.4270) -- (195.6660,259.1760) -- (195.8160,259.2310) -- (195.7250,259.4810) -- cycle(195.8480,258.6750) -- (195.9390,258.4240) -- (196.0900,258.4790) -- (195.9990,258.7300) -- cycle(196.1220,257.9230) -- (196.2130,257.6730) -- (196.3630,257.7270) -- (196.2720,257.9780) -- cycle(196.3950,257.1710) -- (196.4870,256.9210) -- (196.6370,256.9760) -- (196.5460,257.2260) -- cycle(196.6690,256.4200) -- (196.7600,256.1690) -- (196.9110,256.2240) -- (196.8190,256.4740) -- cycle(196.9430,255.6680) -- (197.0340,255.4170) -- (197.1840,255.4720) -- (197.0930,255.7230) -- cycle(197.2160,254.9160) -- (197.3070,254.6660) -- (197.4580,254.7200) -- (197.3670,254.9710) -- cycle(197.4900,254.1650) -- (197.5810,253.9140) -- (197.7320,253.9690) -- (197.6400,254.2190) -- cycle(197.7640,253.4130) -- (197.8550,253.1620) -- (198.0050,253.2170) -- (197.9140,253.4670) -- cycle(198.0370,252.6610) -- (198.1280,252.4100) -- (198.2790,252.4650) -- (198.1880,252.7160) -- cycle(198.3110,251.9090) -- (198.4020,251.6590) -- (198.5520,251.7130) -- (198.4610,251.9640) -- cycle(198.5840,251.1580) -- (198.6760,250.9070) -- (198.8260,250.9620) -- (198.7350,251.2120) -- cycle(198.8580,250.4060) -- (198.9490,250.1550) -- (199.1000,250.2100) -- (199.0080,250.4600) -- cycle(199.1320,249.6540) -- (199.2230,249.4030) -- (199.3730,249.4580) -- (199.2820,249.7090) -- cycle(199.4050,248.9020) -- (199.4960,248.6520) -- (199.6470,248.7060) -- (199.5560,248.9570) -- cycle(199.6790,248.1500) -- (199.7700,247.9000) -- (199.9200,247.9550) -- (199.8290,248.2050) -- cycle(199.9520,247.3990) -- (200.0440,247.1480) -- (200.1940,247.2030) -- (200.1030,247.4530) -- cycle(200.2260,246.6470) -- (200.3170,246.3960) -- (200.4680,246.4510) -- (200.3760,246.7020) -- cycle(200.5000,245.8950) -- (200.5910,245.6450) -- (200.7410,245.6990) -- (200.6500,245.9500) -- cycle(200.7730,245.1440) -- (200.8640,244.8930) -- (201.0150,244.9480) -- (200.9240,245.1980) -- cycle(201.0470,244.3920) -- (201.1380,244.1410) -- (201.2880,244.1960) -- (201.1970,244.4460) -- cycle(201.3200,243.6400) -- (201.4120,243.3890) -- (201.5620,243.4440) -- (201.4710,243.6950) -- cycle(201.5940,242.8880) -- (201.6850,242.6380) -- (201.8360,242.6920) -- (201.7440,242.9430) -- cycle(201.8680,242.1360) -- (201.9590,241.8860) -- (202.1090,241.9410) -- (202.0180,242.1910) -- cycle(202.1410,241.3850) -- (202.2330,241.1340) -- (202.3830,241.1890) -- (202.2920,241.4390) -- cycle(202.4150,240.6330) -- (202.5060,240.3820) -- (202.6560,240.4370) -- (202.5650,240.6880) -- cycle(202.6890,239.8810) -- (202.7800,239.6310) -- (202.9300,239.6850) -- (202.8390,239.9360) -- cycle(202.9620,239.1290) -- (203.0530,238.8790) -- (203.2040,238.9340) -- (203.1130,239.1840) -- cycle(203.2360,238.3780) -- (203.3270,238.1270) -- (203.4770,238.1820) -- (203.3860,238.4320) -- cycle(203.5090,237.6260) -- (203.6010,237.3750) -- (203.7510,237.4300) -- (203.6600,237.6810) -- cycle(203.7830,236.8740) -- (203.8740,236.6240) -- (204.0250,236.6780) -- (203.9330,236.9290) -- cycle(204.0570,236.1220) -- (204.1480,235.8720) -- (204.2980,235.9270) -- (204.2070,236.1770) -- cycle(204.3300,235.3710) -- (204.4210,235.1200) -- (204.5720,235.1750) -- (204.4810,235.4250) -- cycle(204.6040,234.6190) -- (204.6950,234.3680) -- (204.8450,234.4230) -- (204.7540,234.6740) -- cycle(204.8770,233.8670) -- (204.9690,233.6170) -- (205.1190,233.6710) -- (205.0280,233.9220) -- cycle(205.1510,233.1150) -- (205.2420,232.8650) -- (205.3930,232.9200) -- (205.3010,233.1700) -- cycle(205.4250,232.3640) -- (205.5160,232.1130) -- (205.6660,232.1680) -- (205.5750,232.4180) -- cycle(205.6980,231.6120) -- (205.7890,231.3610) -- (205.9400,231.4160) -- (205.8490,231.6670) -- cycle(205.9720,230.8600) -- (206.0630,230.6100) -- (206.2140,230.6640) -- (206.1220,230.9150) -- cycle(206.2460,230.1080) -- (206.3370,229.8580) -- (206.4870,229.9130) -- (206.3960,230.1630) -- cycle(206.5190,229.3570) -- (206.6100,229.1060) -- (206.7610,229.1610) -- (206.6700,229.4110) -- cycle(206.7930,228.6050) -- (206.8840,228.3540) -- (207.0340,228.4090) -- (206.9430,228.6600) -- cycle(207.0660,227.8530) -- (207.1580,227.6030) -- (207.3080,227.6570) -- (207.2170,227.9080) -- cycle(207.3400,227.1010) -- (207.4310,226.8510) -- (207.5820,226.9060) -- (207.4900,227.1560) -- cycle(207.6140,226.3500) -- (207.7050,226.0990) -- (207.8550,226.1540) -- (207.7640,226.4040) -- cycle(207.8870,225.5980) -- (207.9780,225.3470) -- (208.1290,225.4020) -- (208.0380,225.6530) -- cycle(208.1610,224.8460) -- (208.2520,224.5960) -- (208.4020,224.6500) -- (208.3110,224.9010) -- cycle(208.4340,224.0940) -- (208.5260,223.8440) -- (208.6760,223.8980) -- (208.5850,224.1490) -- cycle(208.7080,223.3430) -- (208.7990,223.0920) -- (208.9500,223.1470) -- (208.8580,223.3970) -- cycle(208.9820,222.5910) -- (209.0730,222.3400) -- (209.2230,222.3950) -- (209.1320,222.6460) -- cycle(209.2550,221.8390) -- (209.3460,221.5890) -- (209.4970,221.6430) -- (209.4060,221.8940) -- cycle(209.5290,221.0870) -- (209.6200,220.8370) -- (209.7700,220.8920) -- (209.6790,221.1420) -- cycle(209.8020,220.3360) -- (209.8940,220.0850) -- (210.0440,220.1400) -- (209.9530,220.3900) -- cycle(210.0760,219.5840) -- (210.1670,219.3330) -- (210.3180,219.3880) -- (210.2270,219.6390) -- cycle(210.3500,218.8320) -- (210.4410,218.5820) -- (210.5910,218.6360) -- (210.5000,218.8870) -- cycle(210.6230,218.0800) -- (210.7150,217.8300) -- (210.8650,217.8840) -- (210.7740,218.1350) -- cycle(210.8970,217.3290) -- (210.9880,217.0780) -- (211.1390,217.1330) -- (211.0470,217.3830) -- cycle(211.1710,216.5770) -- (211.2620,216.3260) -- (211.4120,216.3810) -- (211.3210,216.6320) -- cycle(211.4440,215.8250) -- (211.5350,215.5740) -- (211.6860,215.6290) -- (211.5950,215.8800) -- cycle(211.7180,215.0730) -- (211.8090,214.8230) -- (211.9590,214.8770) -- (211.8680,215.1280) -- cycle(211.9910,214.3220) -- (212.0830,214.0710) -- (212.2330,214.1260) -- (212.1420,214.3760) -- cycle(212.2650,213.5700) -- (212.3560,213.3190) -- (212.5070,213.3740) -- (212.4150,213.6250) -- cycle(212.5390,212.8180) -- (212.6300,212.5680) -- (212.7800,212.6220) -- (212.6890,212.8730) -- cycle(212.8120,212.0660) -- (212.9030,211.8160) -- (213.0540,211.8700) -- (212.9630,212.1210) -- cycle(213.0860,211.3150) -- (213.1770,211.0640) -- (213.3270,211.1190) -- (213.2360,211.3690) -- cycle(213.3590,210.5630) -- (213.4510,210.3120) -- (213.6010,210.3670) -- (213.5100,210.6180) -- cycle(213.6330,209.8110) -- (213.7240,209.5610) -- (213.8750,209.6150) -- (213.7830,209.8660) -- cycle(213.9070,209.0590) -- (213.9980,208.8090) -- (214.1480,208.8630) -- (214.0570,209.1140) -- cycle(214.1800,208.3080) -- (214.2710,208.0570) -- (214.4220,208.1120) -- (214.3310,208.3620) -- cycle(214.4540,207.5560) -- (214.5450,207.3050) -- (214.6960,207.3600) -- (214.6040,207.6110) -- cycle(214.7280,206.8040) -- (214.8190,206.5530) -- (214.9690,206.6080) -- (214.8780,206.8590) -- cycle(215.0010,206.0520) -- (215.0920,205.8020) -- (215.2430,205.8560) -- (215.1520,206.1070) -- cycle(215.2750,205.3010) -- (215.3660,205.0500) -- (215.5160,205.1050) -- (215.4250,205.3550) -- cycle(215.5480,204.5490) -- (215.6400,204.2980) -- (215.7900,204.3530) -- (215.6990,204.6040) -- cycle(215.8220,203.7970) -- (215.9130,203.5460) -- (216.0640,203.6010) -- (215.9720,203.8520) -- cycle(216.0960,203.0450) -- (216.1870,202.7950) -- (216.3370,202.8490) -- (216.2460,203.1000) -- cycle(216.3690,202.2940) -- (216.4600,202.0430) -- (216.6110,202.0980) -- (216.5200,202.3480) -- cycle(216.6430,201.5420) -- (216.7340,201.2910) -- (216.8840,201.3460) -- (216.7930,201.5960) -- cycle(216.9160,200.7900) -- (217.0080,200.5390) -- (217.1580,200.5940) -- (217.0670,200.8450) -- cycle(217.1900,200.0380) -- (217.2810,199.7880) -- (217.4320,199.8420) -- (217.3400,200.0930) -- cycle(217.4640,199.2870) -- (217.5550,199.0360) -- (217.7050,199.0910) -- (217.6140,199.3410) -- cycle(217.7370,198.5350) -- (217.8280,198.2840) -- (217.9790,198.3390) -- (217.8880,198.5900) -- cycle(218.0110,197.7830) -- (218.1020,197.5330) -- (218.2530,197.5870) -- (218.1610,197.8380) -- cycle(218.2850,197.0310) -- (218.3760,196.7810) -- (218.5260,196.8350) -- (218.4350,197.0860) -- cycle(218.5580,196.2800) -- (218.6490,196.0290) -- (218.8000,196.0840) -- (218.7080,196.3340) -- cycle(218.8320,195.5280) -- (218.9230,195.2770) -- (219.0730,195.3320) -- (218.9820,195.5820) -- cycle(219.1050,194.7760) -- (219.1970,194.5250) -- (219.3470,194.5800) -- (219.2560,194.8310) -- cycle(219.3790,194.0240) -- (219.4700,193.7740) -- (219.6210,193.8280) -- (219.5290,194.0790) -- cycle(219.6530,193.2720) -- (219.7440,193.0220) -- (219.8940,193.0770) -- (219.8030,193.3270) -- cycle(219.9260,192.5210) -- (220.0170,192.2700) -- (220.1680,192.3250) -- (220.0770,192.5750) -- cycle(220.2000,191.7690) -- (220.2910,191.5180) -- (220.4410,191.5730) -- (220.3500,191.8240) -- cycle(220.4730,191.0170) -- (220.5650,190.7670) -- (220.7150,190.8210) -- (220.6240,191.0720) -- cycle(220.7470,190.2660) -- (220.8380,190.0150) -- (220.9890,190.0700) -- (220.8970,190.3200) -- cycle(221.0210,189.5140) -- (221.1120,189.2630) -- (221.2620,189.3180) -- (221.1710,189.5680) -- cycle(221.2940,188.7620) -- (221.3850,188.5110) -- (221.5360,188.5660) -- (221.4450,188.8170) -- cycle(221.5680,188.0100) -- (221.6590,187.7600) -- (221.8090,187.8140) -- (221.7180,188.0650) -- cycle(221.8410,187.2590) -- (221.9330,187.0080) -- (222.0830,187.0630) -- (221.9920,187.3130) -- cycle(222.1150,186.5070) -- (222.2060,186.2560) -- (222.3570,186.3110) -- (222.2650,186.5610) -- cycle(222.3890,185.7550) -- (222.4800,185.5040) -- (222.6300,185.5590) -- (222.5390,185.8100) -- cycle(222.6620,185.0030) -- (222.7540,184.7530) -- (222.9040,184.8070) -- (222.8130,185.0580) -- cycle(222.9360,184.2510) -- (223.0270,184.0010) -- (223.1770,184.0560) -- (223.0860,184.3060) -- cycle(223.2100,183.5000) -- (223.3010,183.2490) -- (223.4510,183.3040) -- (223.3600,183.5540) -- cycle(223.4830,182.7480) -- (223.5740,182.4970) -- (223.7250,182.5520) -- (223.6340,182.8030) -- cycle(223.7570,181.9960) -- (223.8480,181.7460) -- (223.9980,181.8000) -- (223.9070,182.0510) -- cycle(224.0300,181.2440) -- (224.1220,180.9940) -- (224.2720,181.0490) -- (224.1810,181.2990) -- cycle(224.3040,180.4930) -- (224.3950,180.2420) -- (224.5460,180.2970) -- (224.4540,180.5470) -- cycle(224.5780,179.7410) -- (224.6690,179.4900) -- (224.8190,179.5450) -- (224.7280,179.7960) -- cycle(224.8510,178.9890) -- (224.9420,178.7390) -- (225.0930,178.7930) -- (225.0020,179.0440) -- cycle(225.1250,178.2370) -- (225.2160,177.9870) -- (225.3660,178.0420) -- (225.2750,178.2920) -- cycle(225.3980,177.4860) -- (225.4900,177.2350) -- (225.6400,177.2900) -- (225.5490,177.5400) -- cycle(225.6720,176.7340) -- (225.7630,176.4830) -- (225.9140,176.5380) -- (225.8220,176.7890) -- cycle(225.9460,175.9820) -- (226.0370,175.7320) -- (226.1870,175.7860) -- (226.0960,176.0370) -- cycle(226.2190,175.2300) -- (226.3100,174.9800) -- (226.4610,175.0350) -- (226.3700,175.2850) -- cycle(226.4930,174.4790) -- (226.5840,174.2280) -- (226.7340,174.2830) -- (226.6430,174.5330) -- cycle(226.7660,173.7270) -- (226.8580,173.4760) -- (227.0080,173.5310) -- (226.9170,173.7820) -- cycle(227.0400,172.9750) -- (227.1310,172.7250) -- (227.2820,172.7790) -- (227.1910,173.0300) -- cycle(227.3140,172.2230) -- (227.4050,171.9730) -- (227.5550,172.0280) -- (227.4640,172.2780) -- cycle(227.5870,171.4720) -- (227.6790,171.2210) -- (227.8290,171.2760) -- (227.7380,171.5260) -- cycle(227.8610,170.7200) -- (227.9520,170.4690) -- (228.1030,170.5240) -- (228.0110,170.7750) -- cycle(228.1350,169.9680) -- (228.2260,169.7180) -- (228.3760,169.7720) -- (228.2850,170.0230) -- cycle(228.4080,169.2160) -- (228.4990,168.9660) -- (228.6500,169.0210) -- (228.5590,169.2710) -- cycle(228.6820,168.4650) -- (228.7730,168.2140) -- (228.9230,168.2690) -- (228.8320,168.5190) -- cycle(228.9550,167.7130) -- (229.0470,167.4620) -- (229.1970,167.5170) -- (229.1060,167.7680) -- cycle(229.2290,166.9610) -- (229.3200,166.7110) -- (229.4710,166.7650) -- (229.3790,167.0160) -- cycle(229.5030,166.2090) -- (229.5940,165.9590) -- (229.7440,166.0140) -- (229.6530,166.2640) -- cycle(229.7760,165.4580) -- (229.8670,165.2070) -- (230.0180,165.2620) -- (229.9270,165.5120) -- cycle(230.0500,164.7060) -- (230.1410,164.4550) -- (230.2910,164.5100) -- (230.2000,164.7610) -- cycle(230.3230,163.9540) -- (230.4150,163.7040) -- (230.5650,163.7580) -- (230.4740,164.0090) -- cycle(230.5970,163.2020) -- (230.6880,162.9520) -- (230.8390,163.0070) -- (230.7470,163.2570) -- cycle(230.8710,162.4510) -- (230.9620,162.2000) -- (231.1120,162.2550) -- (231.0210,162.5050) -- cycle(231.1440,161.6990) -- (231.2360,161.4480) -- (231.3860,161.5030) -- (231.2950,161.7540) -- cycle(231.4180,160.9470) -- (231.5090,160.6970) -- (231.6600,160.7510) -- (231.5680,161.0020) -- cycle(231.6920,160.1950) -- (231.7830,159.9450) -- (231.9330,159.9990) -- (231.8420,160.2500) -- cycle(231.9650,159.4440) -- (232.0560,159.1930) -- (232.2070,159.2480) -- (232.1160,159.4980) -- cycle(232.2390,158.6920) -- (232.3300,158.4410) -- (232.4800,158.4960) -- (232.3890,158.7470) -- cycle(232.5120,157.9400) -- (232.6040,157.6900) -- (232.7540,157.7440) -- (232.6630,157.9950) -- cycle(232.7860,157.1880) -- (232.8770,156.9380) -- (233.0280,156.9920) -- (232.9360,157.2430) -- cycle(233.0600,156.4370) -- (233.1510,156.1860) -- (233.3010,156.2410) -- (233.2100,156.4910) -- cycle(233.3330,155.6850) -- (233.4240,155.4340) -- (233.5750,155.4890) -- (233.4840,155.7400) -- cycle(233.6070,154.9330) -- (233.6980,154.6830) -- (233.8480,154.7370) -- (233.7570,154.9880) -- cycle(183.5360,292.5040) -- (183.6270,292.2530) -- (183.7770,292.3080) -- (183.6860,292.5590) -- cycle;



  \path[fill=c7e7e7e,line join=round,line width=0.256pt] (46.6111,177.2300) -- (46.8777,177.2300) -- (46.8777,177.3900) -- (46.6111,177.3900) -- cycle(47.4111,177.2300) -- (47.6777,177.2300) -- (47.6777,177.3900) -- (47.4111,177.3900) -- cycle(48.2111,177.2300) -- (48.4777,177.2300) -- (48.4777,177.3900) -- (48.2111,177.3900) -- cycle(49.0111,177.2300) -- (49.2777,177.2300) -- (49.2777,177.3900) -- (49.0111,177.3900) -- cycle(49.8111,177.2300) -- (50.0777,177.2300) -- (50.0777,177.3900) -- (49.8111,177.3900) -- cycle(50.6111,177.2300) -- (50.8777,177.2300) -- (50.8777,177.3900) -- (50.6111,177.3900) -- cycle(51.4111,177.2300) -- (51.6777,177.2300) -- (51.6777,177.3900) -- (51.4111,177.3900) -- cycle(52.2111,177.2300) -- (52.4777,177.2300) -- (52.4777,177.3900) -- (52.2111,177.3900) -- cycle(53.0110,177.2300) -- (53.2777,177.2300) -- (53.2777,177.3900) -- (53.0110,177.3900) -- cycle(53.8111,177.2300) -- (54.0777,177.2300) -- (54.0777,177.3900) -- (53.8111,177.3900) -- cycle(54.6111,177.2300) -- (54.8777,177.2300) -- (54.8777,177.3900) -- (54.6111,177.3900) -- cycle(55.4111,177.2300) -- (55.6777,177.2300) -- (55.6777,177.3900) -- (55.4111,177.3900) -- cycle(56.2111,177.2300) -- (56.4777,177.2300) -- (56.4777,177.3900) -- (56.2111,177.3900) -- cycle(57.0110,177.2300) -- (57.2777,177.2300) -- (57.2777,177.3900) -- (57.0110,177.3900) -- cycle(57.8110,177.2300) -- (58.0777,177.2300) -- (58.0777,177.3900) -- (57.8110,177.3900) -- cycle(58.6111,177.2300) -- (58.8777,177.2300) -- (58.8777,177.3900) -- (58.6111,177.3900) -- cycle(59.4110,177.2300) -- (59.6777,177.2300) -- (59.6777,177.3900) -- (59.4110,177.3900) -- cycle(60.2111,177.2300) -- (60.4777,177.2300) -- (60.4777,177.3900) -- (60.2111,177.3900) -- cycle(61.0110,177.2300) -- (61.2777,177.2300) -- (61.2777,177.3900) -- (61.0110,177.3900) -- cycle(61.8110,177.2300) -- (62.0777,177.2300) -- (62.0777,177.3900) -- (61.8110,177.3900) -- cycle(62.6110,177.2300) -- (62.8777,177.2300) -- (62.8777,177.3900) -- (62.6110,177.3900) -- cycle(63.4110,177.2300) -- (63.6777,177.2300) -- (63.6777,177.3900) -- (63.4110,177.3900) -- cycle(64.2110,177.2300) -- (64.4777,177.2300) -- (64.4777,177.3900) -- (64.2110,177.3900) -- cycle(65.0110,177.2300) -- (65.2777,177.2300) -- (65.2777,177.3900) -- (65.0110,177.3900) -- cycle(65.8110,177.2300) -- (66.0777,177.2300) -- (66.0777,177.3900) -- (65.8110,177.3900) -- cycle(66.6110,177.2300) -- (66.8777,177.2300) -- (66.8777,177.3900) -- (66.6110,177.3900) -- cycle(67.4110,177.2300) -- (67.6777,177.2300) -- (67.6777,177.3900) -- (67.4110,177.3900) -- cycle(68.2110,177.2300) -- (68.4777,177.2300) -- (68.4777,177.3900) -- (68.2110,177.3900) -- cycle(69.0110,177.2300) -- (69.2777,177.2300) -- (69.2777,177.3900) -- (69.0110,177.3900) -- cycle(69.8110,177.2300) -- (70.0777,177.2300) -- (70.0777,177.3900) -- (69.8110,177.3900) -- cycle(70.6110,177.2300) -- (70.8777,177.2300) -- (70.8777,177.3900) -- (70.6110,177.3900) -- cycle(71.4110,177.2300) -- (71.6777,177.2300) -- (71.6777,177.3900) -- (71.4110,177.3900) -- cycle(72.2110,177.2300) -- (72.4777,177.2300) -- (72.4777,177.3900) -- (72.2110,177.3900) -- cycle(73.0110,177.2300) -- (73.2777,177.2300) -- (73.2777,177.3900) -- (73.0110,177.3900) -- cycle(73.8110,177.2300) -- (74.0777,177.2300) -- (74.0777,177.3900) -- (73.8110,177.3900) -- cycle(74.6110,177.2300) -- (74.8777,177.2300) -- (74.8777,177.3900) -- (74.6110,177.3900) -- cycle(75.4110,177.2300) -- (75.6777,177.2300) -- (75.6777,177.3900) -- (75.4110,177.3900) -- cycle(76.2110,177.2300) -- (76.4777,177.2300) -- (76.4777,177.3900) -- (76.2110,177.3900) -- cycle(77.0110,177.2300) -- (77.2777,177.2300) -- (77.2777,177.3900) -- (77.0110,177.3900) -- cycle(77.8110,177.2300) -- (78.0776,177.2300) -- (78.0776,177.3900) -- (77.8110,177.3900) -- cycle(78.6110,177.2300) -- (78.8777,177.2300) -- (78.8777,177.3900) -- (78.6110,177.3900) -- cycle(79.4110,177.2300) -- (79.6777,177.2300) -- (79.6777,177.3900) -- (79.4110,177.3900) -- cycle(80.2110,177.2300) -- (80.4777,177.2300) -- (80.4777,177.3900) -- (80.2110,177.3900) -- cycle(81.0110,177.2300) -- (81.2776,177.2300) -- (81.2776,177.3900) -- (81.0110,177.3900) -- cycle(81.8110,177.2300) -- (82.0776,177.2300) -- (82.0776,177.3900) -- (81.8110,177.3900) -- cycle(82.6110,177.2300) -- (82.8777,177.2300) -- (82.8777,177.3900) -- (82.6110,177.3900) -- cycle(83.4110,177.2300) -- (83.6777,177.2300) -- (83.6777,177.3900) -- (83.4110,177.3900) -- cycle(84.2110,177.2300) -- (84.4776,177.2300) -- (84.4776,177.3900) -- (84.2110,177.3900) -- cycle(85.0110,177.2300) -- (85.2776,177.2300) -- (85.2776,177.3900) -- (85.0110,177.3900) -- cycle(85.8110,177.2300) -- (86.0776,177.2300) -- (86.0776,177.3900) -- (85.8110,177.3900) -- cycle(86.6110,177.2300) -- (86.8777,177.2300) -- (86.8777,177.3900) -- (86.6110,177.3900) -- cycle(87.4110,177.2300) -- (87.6776,177.2300) -- (87.6776,177.3900) -- (87.4110,177.3900) -- cycle(88.2110,177.2300) -- (88.4776,177.2300) -- (88.4776,177.3900) -- (88.2110,177.3900) -- cycle(89.0110,177.2300) -- (89.2776,177.2300) -- (89.2776,177.3900) -- (89.0110,177.3900) -- cycle(89.8110,177.2300) -- (90.0776,177.2300) -- (90.0776,177.3900) -- (89.8110,177.3900) -- cycle(90.6110,177.2300) -- (90.8776,177.2300) -- (90.8776,177.3900) -- (90.6110,177.3900) -- cycle(91.4110,177.2300) -- (91.6776,177.2300) -- (91.6776,177.3900) -- (91.4110,177.3900) -- cycle(92.2109,177.2300) -- (92.4776,177.2300) -- (92.4776,177.3900) -- (92.2109,177.3900) -- cycle(93.0110,177.2300) -- (93.2776,177.2300) -- (93.2776,177.3900) -- (93.0110,177.3900) -- cycle(93.8110,177.2300) -- (94.0776,177.2300) -- (94.0776,177.3900) -- (93.8110,177.3900) -- cycle(94.6110,177.2300) -- (94.8776,177.2300) -- (94.8776,177.3900) -- (94.6110,177.3900) -- cycle(95.4109,177.2300) -- (95.6776,177.2300) -- (95.6776,177.3900) -- (95.4109,177.3900) -- cycle(96.2109,177.2300) -- (96.4776,177.2300) -- (96.4776,177.3900) -- (96.2109,177.3900) -- cycle(97.0110,177.2300) -- (97.2776,177.2300) -- (97.2776,177.3900) -- (97.0110,177.3900) -- cycle(97.8110,177.2300) -- (98.0776,177.2300) -- (98.0776,177.3900) -- (97.8110,177.3900) -- cycle(98.6109,177.2300) -- (98.8776,177.2300) -- (98.8776,177.3900) -- (98.6109,177.3900) -- cycle(99.4109,177.2300) -- (99.6776,177.2300) -- (99.6776,177.3900) -- (99.4109,177.3900) -- cycle(100.2110,177.2300) -- (100.4780,177.2300) -- (100.4780,177.3900) -- (100.2110,177.3900) -- cycle(101.0110,177.2300) -- (101.2780,177.2300) -- (101.2780,177.3900) -- (101.0110,177.3900) -- cycle(101.8110,177.2300) -- (102.0780,177.2300) -- (102.0780,177.3900) -- (101.8110,177.3900) -- cycle(102.6110,177.2300) -- (102.8780,177.2300) -- (102.8780,177.3900) -- (102.6110,177.3900) -- cycle(103.4110,177.2300) -- (103.6780,177.2300) -- (103.6780,177.3900) -- (103.4110,177.3900) -- cycle(104.2110,177.2300) -- (104.4780,177.2300) -- (104.4780,177.3900) -- (104.2110,177.3900) -- cycle(105.0110,177.2300) -- (105.2780,177.2300) -- (105.2780,177.3900) -- (105.0110,177.3900) -- cycle(105.8110,177.2300) -- (106.0780,177.2300) -- (106.0780,177.3900) -- (105.8110,177.3900) -- cycle(106.6110,177.2300) -- (106.8780,177.2300) -- (106.8780,177.3900) -- (106.6110,177.3900) -- cycle(107.4110,177.2300) -- (107.6780,177.2300) -- (107.6780,177.3900) -- (107.4110,177.3900) -- cycle(108.2110,177.2300) -- (108.4780,177.2300) -- (108.4780,177.3900) -- (108.2110,177.3900) -- cycle(109.0110,177.2300) -- (109.2780,177.2300) -- (109.2780,177.3900) -- (109.0110,177.3900) -- cycle(109.8110,177.2300) -- (110.0780,177.2300) -- (110.0780,177.3900) -- (109.8110,177.3900) -- cycle(110.6110,177.2300) -- (110.8780,177.2300) -- (110.8780,177.3900) -- (110.6110,177.3900) -- cycle(111.4110,177.2300) -- (111.6780,177.2300) -- (111.6780,177.3900) -- (111.4110,177.3900) -- cycle(112.2110,177.2300) -- (112.4780,177.2300) -- (112.4780,177.3900) -- (112.2110,177.3900) -- cycle(113.0110,177.2300) -- (113.2780,177.2300) -- (113.2780,177.3900) -- (113.0110,177.3900) -- cycle(113.8110,177.2300) -- (114.0780,177.2300) -- (114.0780,177.3900) -- (113.8110,177.3900) -- cycle(114.6110,177.2300) -- (114.8780,177.2300) -- (114.8780,177.3900) -- (114.6110,177.3900) -- cycle(115.4110,177.2300) -- (115.6780,177.2300) -- (115.6780,177.3900) -- (115.4110,177.3900) -- cycle(116.2110,177.2300) -- (116.4780,177.2300) -- (116.4780,177.3900) -- (116.2110,177.3900) -- cycle(117.0110,177.2300) -- (117.2780,177.2300) -- (117.2780,177.3900) -- (117.0110,177.3900) -- cycle(117.8110,177.2300) -- (118.0780,177.2300) -- (118.0780,177.3900) -- (117.8110,177.3900) -- cycle(118.6110,177.2300) -- (118.8780,177.2300) -- (118.8780,177.3900) -- (118.6110,177.3900) -- cycle(119.4110,177.2300) -- (119.6780,177.2300) -- (119.6780,177.3900) -- (119.4110,177.3900) -- cycle(120.2110,177.2300) -- (120.4780,177.2300) -- (120.4780,177.3900) -- (120.2110,177.3900) -- cycle(121.0110,177.2300) -- (121.2780,177.2300) -- (121.2780,177.3900) -- (121.0110,177.3900) -- cycle(121.8110,177.2300) -- (122.0780,177.2300) -- (122.0780,177.3900) -- (121.8110,177.3900) -- cycle(122.6110,177.2300) -- (122.8780,177.2300) -- (122.8780,177.3900) -- (122.6110,177.3900) -- cycle(123.4110,177.2300) -- (123.6780,177.2300) -- (123.6780,177.3900) -- (123.4110,177.3900) -- cycle(124.2110,177.2300) -- (124.4780,177.2300) -- (124.4780,177.3900) -- (124.2110,177.3900) -- cycle(125.0110,177.2300) -- (125.2780,177.2300) -- (125.2780,177.3900) -- (125.0110,177.3900) -- cycle(125.8110,177.2300) -- (126.0780,177.2300) -- (126.0780,177.3900) -- (125.8110,177.3900) -- cycle(126.6110,177.2300) -- (126.8780,177.2300) -- (126.8780,177.3900) -- (126.6110,177.3900) -- cycle(127.4110,177.2300) -- (127.6780,177.2300) -- (127.6780,177.3900) -- (127.4110,177.3900) -- cycle(128.2110,177.2300) -- (128.4780,177.2300) -- (128.4780,177.3900) -- (128.2110,177.3900) -- cycle(129.0110,177.2300) -- (129.2780,177.2300) -- (129.2780,177.3900) -- (129.0110,177.3900) -- cycle(129.8110,177.2300) -- (130.0780,177.2300) -- (130.0780,177.3900) -- (129.8110,177.3900) -- cycle(130.6110,177.2300) -- (130.8780,177.2300) -- (130.8780,177.3900) -- (130.6110,177.3900) -- cycle(131.4110,177.2300) -- (131.6780,177.2300) -- (131.6780,177.3900) -- (131.4110,177.3900) -- cycle(132.2110,177.2300) -- (132.4780,177.2300) -- (132.4780,177.3900) -- (132.2110,177.3900) -- cycle(133.0110,177.2300) -- (133.2780,177.2300) -- (133.2780,177.3900) -- (133.0110,177.3900) -- cycle(133.8110,177.2300) -- (134.0780,177.2300) -- (134.0780,177.3900) -- (133.8110,177.3900) -- cycle(134.6110,177.2300) -- (134.8780,177.2300) -- (134.8780,177.3900) -- (134.6110,177.3900) -- cycle(135.4110,177.2300) -- (135.6770,177.2300) -- (135.6770,177.3900) -- (135.4110,177.3900) -- cycle(136.2110,177.2300) -- (136.4780,177.2300) -- (136.4780,177.3900) -- (136.2110,177.3900) -- cycle(137.0110,177.2300) -- (137.2780,177.2300) -- (137.2780,177.3900) -- (137.0110,177.3900) -- cycle(137.8110,177.2300) -- (138.0780,177.2300) -- (138.0780,177.3900) -- (137.8110,177.3900) -- cycle(138.6110,177.2300) -- (138.8780,177.2300) -- (138.8780,177.3900) -- (138.6110,177.3900) -- cycle(139.4110,177.2300) -- (139.6770,177.2300) -- (139.6770,177.3900) -- (139.4110,177.3900) -- cycle(140.2110,177.2300) -- (140.4780,177.2300) -- (140.4780,177.3900) -- (140.2110,177.3900) -- cycle(141.0110,177.2300) -- (141.2780,177.2300) -- (141.2780,177.3900) -- (141.0110,177.3900) -- cycle(141.8110,177.2300) -- (142.0770,177.2300) -- (142.0770,177.3900) -- (141.8110,177.3900) -- cycle(142.6110,177.2300) -- (142.8780,177.2300) -- (142.8780,177.3900) -- (142.6110,177.3900) -- cycle(143.4110,177.2300) -- (143.6770,177.2300) -- (143.6770,177.3900) -- (143.4110,177.3900) -- cycle(144.2110,177.2300) -- (144.4780,177.2300) -- (144.4780,177.3900) -- (144.2110,177.3900) -- cycle(145.0110,177.2300) -- (145.2770,177.2300) -- (145.2770,177.3900) -- (145.0110,177.3900) -- cycle(145.8110,177.2300) -- (146.0770,177.2300) -- (146.0770,177.3900) -- (145.8110,177.3900) -- cycle(146.6110,177.2300) -- (146.8780,177.2300) -- (146.8780,177.3900) -- (146.6110,177.3900) -- cycle(147.4110,177.2300) -- (147.6770,177.2300) -- (147.6770,177.3900) -- (147.4110,177.3900) -- cycle(148.2110,177.2300) -- (148.4770,177.2300) -- (148.4770,177.3900) -- (148.2110,177.3900) -- cycle(149.0110,177.2300) -- (149.2770,177.2300) -- (149.2770,177.3900) -- (149.0110,177.3900) -- cycle(149.8110,177.2300) -- (150.0770,177.2300) -- (150.0770,177.3900) -- (149.8110,177.3900) -- cycle(150.6110,177.2300) -- (150.8780,177.2300) -- (150.8780,177.3900) -- (150.6110,177.3900) -- cycle(151.4110,177.2300) -- (151.6770,177.2300) -- (151.6770,177.3900) -- (151.4110,177.3900) -- cycle(152.2110,177.2300) -- (152.4770,177.2300) -- (152.4770,177.3900) -- (152.2110,177.3900) -- cycle(153.0110,177.2300) -- (153.2770,177.2300) -- (153.2770,177.3900) -- (153.0110,177.3900) -- cycle(153.8110,177.2300) -- (154.0770,177.2300) -- (154.0770,177.3900) -- (153.8110,177.3900) -- cycle(154.6110,177.2300) -- (154.8770,177.2300) -- (154.8770,177.3900) -- (154.6110,177.3900) -- cycle(155.4110,177.2300) -- (155.6770,177.2300) -- (155.6770,177.3900) -- (155.4110,177.3900) -- cycle(156.2110,177.2300) -- (156.4770,177.2300) -- (156.4770,177.3900) -- (156.2110,177.3900) -- cycle(157.0110,177.2300) -- (157.2770,177.2300) -- (157.2770,177.3900) -- (157.0110,177.3900) -- cycle(157.8110,177.2300) -- (158.0770,177.2300) -- (158.0770,177.3900) -- (157.8110,177.3900) -- cycle(158.6110,177.2300) -- (158.8770,177.2300) -- (158.8770,177.3900) -- (158.6110,177.3900) -- cycle(159.4110,177.2300) -- (159.6770,177.2300) -- (159.6770,177.3900) -- (159.4110,177.3900) -- cycle(160.2110,177.2300) -- (160.4770,177.2300) -- (160.4770,177.3900) -- (160.2110,177.3900) -- cycle(161.0110,177.2300) -- (161.2770,177.2300) -- (161.2770,177.3900) -- (161.0110,177.3900) -- cycle(161.8110,177.2300) -- (162.0770,177.2300) -- (162.0770,177.3900) -- (161.8110,177.3900) -- cycle(162.6110,177.2300) -- (162.8770,177.2300) -- (162.8770,177.3900) -- (162.6110,177.3900) -- cycle(163.4110,177.2300) -- (163.6770,177.2300) -- (163.6770,177.3900) -- (163.4110,177.3900) -- cycle(164.2110,177.2300) -- (164.4770,177.2300) -- (164.4770,177.3900) -- (164.2110,177.3900) -- cycle(165.0110,177.2300) -- (165.2770,177.2300) -- (165.2770,177.3900) -- (165.0110,177.3900) -- cycle(165.8110,177.2300) -- (166.0770,177.2300) -- (166.0770,177.3900) -- (165.8110,177.3900) -- cycle(166.6110,177.2300) -- (166.8770,177.2300) -- (166.8770,177.3900) -- (166.6110,177.3900) -- cycle(167.4110,177.2300) -- (167.6770,177.2300) -- (167.6770,177.3900) -- (167.4110,177.3900) -- cycle(168.2110,177.2300) -- (168.4770,177.2300) -- (168.4770,177.3900) -- (168.2110,177.3900) -- cycle(169.0110,177.2300) -- (169.2770,177.2300) -- (169.2770,177.3900) -- (169.0110,177.3900) -- cycle(169.8110,177.2300) -- (170.0770,177.2300) -- (170.0770,177.3900) -- (169.8110,177.3900) -- cycle(170.6110,177.2300) -- (170.8770,177.2300) -- (170.8770,177.3900) -- (170.6110,177.3900) -- cycle(171.4110,177.2300) -- (171.6770,177.2300) -- (171.6770,177.3900) -- (171.4110,177.3900) -- cycle(172.2110,177.2300) -- (172.4770,177.2300) -- (172.4770,177.3900) -- (172.2110,177.3900) -- cycle(173.0110,177.2300) -- (173.2770,177.2300) -- (173.2770,177.3900) -- (173.0110,177.3900) -- cycle(173.8110,177.2300) -- (174.0770,177.2300) -- (174.0770,177.3900) -- (173.8110,177.3900) -- cycle(174.6110,177.2300) -- (174.8770,177.2300) -- (174.8770,177.3900) -- (174.6110,177.3900) -- cycle(175.4110,177.2300) -- (175.6770,177.2300) -- (175.6770,177.3900) -- (175.4110,177.3900) -- cycle(176.2110,177.2300) -- (176.4770,177.2300) -- (176.4770,177.3900) -- (176.2110,177.3900) -- cycle(177.0110,177.2300) -- (177.2770,177.2300) -- (177.2770,177.3900) -- (177.0110,177.3900) -- cycle(177.8110,177.2300) -- (178.0770,177.2300) -- (178.0770,177.3900) -- (177.8110,177.3900) -- cycle(178.6110,177.2300) -- (178.8770,177.2300) -- (178.8770,177.3900) -- (178.6110,177.3900) -- cycle(179.4110,177.2300) -- (179.6770,177.2300) -- (179.6770,177.3900) -- (179.4110,177.3900) -- cycle(180.2110,177.2300) -- (180.4770,177.2300) -- (180.4770,177.3900) -- (180.2110,177.3900) -- cycle(181.0110,177.2300) -- (181.2770,177.2300) -- (181.2770,177.3900) -- (181.0110,177.3900) -- cycle(181.8110,177.2300) -- (182.0770,177.2300) -- (182.0770,177.3900) -- (181.8110,177.3900) -- cycle(182.6110,177.2300) -- (182.8770,177.2300) -- (182.8770,177.3900) -- (182.6110,177.3900) -- cycle(183.4110,177.2300) -- (183.6770,177.2300) -- (183.6770,177.3900) -- (183.4110,177.3900) -- cycle(184.2110,177.2300) -- (184.4770,177.2300) -- (184.4770,177.3900) -- (184.2110,177.3900) -- cycle(185.0110,177.2300) -- (185.2770,177.2300) -- (185.2770,177.3900) -- (185.0110,177.3900) -- cycle(185.8110,177.2300) -- (186.0770,177.2300) -- (186.0770,177.3900) -- (185.8110,177.3900) -- cycle(186.6110,177.2300) -- (186.8770,177.2300) -- (186.8770,177.3900) -- (186.6110,177.3900) -- cycle(187.4110,177.2300) -- (187.6770,177.2300) -- (187.6770,177.3900) -- (187.4110,177.3900) -- cycle(188.2110,177.2300) -- (188.4770,177.2300) -- (188.4770,177.3900) -- (188.2110,177.3900) -- cycle(189.0110,177.2300) -- (189.2770,177.2300) -- (189.2770,177.3900) -- (189.0110,177.3900) -- cycle(189.8110,177.2300) -- (190.0770,177.2300) -- (190.0770,177.3900) -- (189.8110,177.3900) -- cycle(190.6110,177.2300) -- (190.8770,177.2300) -- (190.8770,177.3900) -- (190.6110,177.3900) -- cycle(191.4110,177.2300) -- (191.6770,177.2300) -- (191.6770,177.3900) -- (191.4110,177.3900) -- cycle(192.2110,177.2300) -- (192.4770,177.2300) -- (192.4770,177.3900) -- (192.2110,177.3900) -- cycle(193.0110,177.2300) -- (193.2770,177.2300) -- (193.2770,177.3900) -- (193.0110,177.3900) -- cycle(193.8110,177.2300) -- (194.0770,177.2300) -- (194.0770,177.3900) -- (193.8110,177.3900) -- cycle(194.6110,177.2300) -- (194.8770,177.2300) -- (194.8770,177.3900) -- (194.6110,177.3900) -- cycle(195.4110,177.2300) -- (195.6770,177.2300) -- (195.6770,177.3900) -- (195.4110,177.3900) -- cycle(196.2110,177.2300) -- (196.4770,177.2300) -- (196.4770,177.3900) -- (196.2110,177.3900) -- cycle(197.0110,177.2300) -- (197.2770,177.2300) -- (197.2770,177.3900) -- (197.0110,177.3900) -- cycle(197.8110,177.2300) -- (198.0770,177.2300) -- (198.0770,177.3900) -- (197.8110,177.3900) -- cycle(198.6110,177.2300) -- (198.8770,177.2300) -- (198.8770,177.3900) -- (198.6110,177.3900) -- cycle(199.4110,177.2300) -- (199.6770,177.2300) -- (199.6770,177.3900) -- (199.4110,177.3900) -- cycle(200.2110,177.2300) -- (200.4770,177.2300) -- (200.4770,177.3900) -- (200.2110,177.3900) -- cycle(201.0110,177.2300) -- (201.2770,177.2300) -- (201.2770,177.3900) -- (201.0110,177.3900) -- cycle(201.8110,177.2300) -- (202.0770,177.2300) -- (202.0770,177.3900) -- (201.8110,177.3900) -- cycle(202.6110,177.2300) -- (202.8770,177.2300) -- (202.8770,177.3900) -- (202.6110,177.3900) -- cycle(203.4110,177.2300) -- (203.6770,177.2300) -- (203.6770,177.3900) -- (203.4110,177.3900) -- cycle(204.2110,177.2300) -- (204.4770,177.2300) -- (204.4770,177.3900) -- (204.2110,177.3900) -- cycle(205.0110,177.2300) -- (205.2770,177.2300) -- (205.2770,177.3900) -- (205.0110,177.3900) -- cycle(205.8110,177.2300) -- (206.0770,177.2300) -- (206.0770,177.3900) -- (205.8110,177.3900) -- cycle(206.6110,177.2300) -- (206.8770,177.2300) -- (206.8770,177.3900) -- (206.6110,177.3900) -- cycle(207.4110,177.2300) -- (207.6770,177.2300) -- (207.6770,177.3900) -- (207.4110,177.3900) -- cycle(208.2110,177.2300) -- (208.4770,177.2300) -- (208.4770,177.3900) -- (208.2110,177.3900) -- cycle(209.0110,177.2300) -- (209.2770,177.2300) -- (209.2770,177.3900) -- (209.0110,177.3900) -- cycle(209.8110,177.2300) -- (210.0770,177.2300) -- (210.0770,177.3900) -- (209.8110,177.3900) -- cycle(210.6110,177.2300) -- (210.8770,177.2300) -- (210.8770,177.3900) -- (210.6110,177.3900) -- cycle(211.4110,177.2300) -- (211.6770,177.2300) -- (211.6770,177.3900) -- (211.4110,177.3900) -- cycle(212.2110,177.2300) -- (212.4770,177.2300) -- (212.4770,177.3900) -- (212.2110,177.3900) -- cycle(213.0110,177.2300) -- (213.2770,177.2300) -- (213.2770,177.3900) -- (213.0110,177.3900) -- cycle(213.8110,177.2300) -- (214.0770,177.2300) -- (214.0770,177.3900) -- (213.8110,177.3900) -- cycle(214.6110,177.2300) -- (214.8770,177.2300) -- (214.8770,177.3900) -- (214.6110,177.3900) -- cycle(215.4110,177.2300) -- (215.6770,177.2300) -- (215.6770,177.3900) -- (215.4110,177.3900) -- cycle(216.2110,177.2300) -- (216.4770,177.2300) -- (216.4770,177.3900) -- (216.2110,177.3900) -- cycle(217.0110,177.2300) -- (217.2770,177.2300) -- (217.2770,177.3900) -- (217.0110,177.3900) -- cycle(217.8110,177.2300) -- (218.0770,177.2300) -- (218.0770,177.3900) -- (217.8110,177.3900) -- cycle(218.6110,177.2300) -- (218.8770,177.2300) -- (218.8770,177.3900) -- (218.6110,177.3900) -- cycle(219.4110,177.2300) -- (219.6770,177.2300) -- (219.6770,177.3900) -- (219.4110,177.3900) -- cycle(220.2110,177.2300) -- (220.4770,177.2300) -- (220.4770,177.3900) -- (220.2110,177.3900) -- cycle(221.0110,177.2300) -- (221.2770,177.2300) -- (221.2770,177.3900) -- (221.0110,177.3900) -- cycle(221.8110,177.2300) -- (222.0770,177.2300) -- (222.0770,177.3900) -- (221.8110,177.3900) -- cycle(222.6110,177.2300) -- (222.8770,177.2300) -- (222.8770,177.3900) -- (222.6110,177.3900) -- cycle(223.4110,177.2300) -- (223.6770,177.2300) -- (223.6770,177.3900) -- (223.4110,177.3900) -- cycle(224.2110,177.2300) -- (224.4770,177.2300) -- (224.4770,177.3900) -- (224.2110,177.3900) -- cycle(225.0110,177.2300) -- (225.2770,177.2300) -- (225.2770,177.3900) -- (225.0110,177.3900) -- cycle(225.8110,177.2300) -- (226.0770,177.2300) -- (226.0770,177.3900) -- (225.8110,177.3900) -- cycle(226.6110,177.2300) -- (226.8770,177.2300) -- (226.8770,177.3900) -- (226.6110,177.3900) -- cycle(227.4110,177.2300) -- (227.6770,177.2300) -- (227.6770,177.3900) -- (227.4110,177.3900) -- cycle(228.2110,177.2300) -- (228.4770,177.2300) -- (228.4770,177.3900) -- (228.2110,177.3900) -- cycle(229.0110,177.2300) -- (229.2770,177.2300) -- (229.2770,177.3900) -- (229.0110,177.3900) -- cycle(229.8110,177.2300) -- (230.0770,177.2300) -- (230.0770,177.3900) -- (229.8110,177.3900) -- cycle(230.6110,177.2300) -- (230.8770,177.2300) -- (230.8770,177.3900) -- (230.6110,177.3900) -- cycle(231.4110,177.2300) -- (231.6770,177.2300) -- (231.6770,177.3900) -- (231.4110,177.3900) -- cycle(232.2110,177.2300) -- (232.4770,177.2300) -- (232.4770,177.3900) -- (232.2110,177.3900) -- cycle(233.0110,177.2300) -- (233.2770,177.2300) -- (233.2770,177.3900) -- (233.0110,177.3900) -- cycle(233.8110,177.2300) -- (234.0770,177.2300) -- (234.0770,177.3900) -- (233.8110,177.3900) -- cycle(234.6110,177.2300) -- (234.8770,177.2300) -- (234.8770,177.3900) -- (234.6110,177.3900) -- cycle(235.4110,177.2300) -- (235.6770,177.2300) -- (235.6770,177.3900) -- (235.4110,177.3900) -- cycle(236.2110,177.2300) -- (236.4770,177.2300) -- (236.4770,177.3900) -- (236.2110,177.3900) -- cycle(237.0110,177.2300) -- (237.2770,177.2300) -- (237.2770,177.3900) -- (237.0110,177.3900) -- cycle(237.8110,177.2300) -- (238.0770,177.2300) -- (238.0770,177.3900) -- (237.8110,177.3900) -- cycle(238.6110,177.2300) -- (238.8770,177.2300) -- (238.8770,177.3900) -- (238.6110,177.3900) -- cycle(239.4110,177.2300) -- (239.6770,177.2300) -- (239.6770,177.3900) -- (239.4110,177.3900) -- cycle(240.2110,177.2300) -- (240.4770,177.2300) -- (240.4770,177.3900) -- (240.2110,177.3900) -- cycle(241.0110,177.2300) -- (241.2770,177.2300) -- (241.2770,177.3900) -- (241.0110,177.3900) -- cycle(241.8110,177.2300) -- (242.0770,177.2300) -- (242.0770,177.3900) -- (241.8110,177.3900) -- cycle(242.6110,177.2300) -- (242.8770,177.2300) -- (242.8770,177.3900) -- (242.6110,177.3900) -- cycle(243.4110,177.2300) -- (243.6770,177.2300) -- (243.6770,177.3900) -- (243.4110,177.3900) -- cycle(244.2110,177.2300) -- (244.4770,177.2300) -- (244.4770,177.3900) -- (244.2110,177.3900) -- cycle(245.0110,177.2300) -- (245.2770,177.2300) -- (245.2770,177.3900) -- (245.0110,177.3900) -- cycle(245.8110,177.2300) -- (246.0770,177.2300) -- (246.0770,177.3900) -- (245.8110,177.3900) -- cycle(246.6110,177.2300) -- (246.8770,177.2300) -- (246.8770,177.3900) -- (246.6110,177.3900) -- cycle(247.4110,177.2300) -- (247.6770,177.2300) -- (247.6770,177.3900) -- (247.4110,177.3900) -- cycle(248.2110,177.2300) -- (248.4770,177.2300) -- (248.4770,177.3900) -- (248.2110,177.3900) -- cycle(249.0110,177.2300) -- (249.2770,177.2300) -- (249.2770,177.3900) -- (249.0110,177.3900) -- cycle(249.8110,177.2300) -- (250.0770,177.2300) -- (250.0770,177.3900) -- (249.8110,177.3900) -- cycle(250.6110,177.2300) -- (250.8770,177.2300) -- (250.8770,177.3900) -- (250.6110,177.3900) -- cycle(251.4110,177.2300) -- (251.6190,177.2300) -- (251.6190,177.3900) -- (251.4110,177.3900) -- cycle(45.8111,177.2300) -- (46.0778,177.2300) -- (46.0778,177.3900) -- (45.8111,177.3900) -- cycle;



  \path[fill=c7e7e7e,line join=round,line width=0.256pt] (38.2439,200.2340) -- (38.5106,200.2340) -- (38.5106,200.3940) -- (38.2439,200.3940) -- cycle(39.0439,200.2340) -- (39.3106,200.2340) -- (39.3106,200.3940) -- (39.0439,200.3940) -- cycle(39.8439,200.2340) -- (40.1106,200.2340) -- (40.1106,200.3940) -- (39.8439,200.3940) -- cycle(40.6439,200.2340) -- (40.9106,200.2340) -- (40.9106,200.3940) -- (40.6439,200.3940) -- cycle(41.4439,200.2340) -- (41.7106,200.2340) -- (41.7106,200.3940) -- (41.4439,200.3940) -- cycle(42.2439,200.2340) -- (42.5106,200.2340) -- (42.5106,200.3940) -- (42.2439,200.3940) -- cycle(43.0439,200.2340) -- (43.3106,200.2340) -- (43.3106,200.3940) -- (43.0439,200.3940) -- cycle(43.8439,200.2340) -- (44.1106,200.2340) -- (44.1106,200.3940) -- (43.8439,200.3940) -- cycle(44.6439,200.2340) -- (44.9106,200.2340) -- (44.9106,200.3940) -- (44.6439,200.3940) -- cycle(45.4439,200.2340) -- (45.7106,200.2340) -- (45.7106,200.3940) -- (45.4439,200.3940) -- cycle(46.2439,200.2340) -- (46.5106,200.2340) -- (46.5106,200.3940) -- (46.2439,200.3940) -- cycle(47.0439,200.2340) -- (47.3106,200.2340) -- (47.3106,200.3940) -- (47.0439,200.3940) -- cycle(47.8439,200.2340) -- (48.1106,200.2340) -- (48.1106,200.3940) -- (47.8439,200.3940) -- cycle(48.6439,200.2340) -- (48.9106,200.2340) -- (48.9106,200.3940) -- (48.6439,200.3940) -- cycle(49.4439,200.2340) -- (49.7106,200.2340) -- (49.7106,200.3940) -- (49.4439,200.3940) -- cycle(50.2439,200.2340) -- (50.5106,200.2340) -- (50.5106,200.3940) -- (50.2439,200.3940) -- cycle(51.0439,200.2340) -- (51.3105,200.2340) -- (51.3105,200.3940) -- (51.0439,200.3940) -- cycle(51.8439,200.2340) -- (52.1106,200.2340) -- (52.1106,200.3940) -- (51.8439,200.3940) -- cycle(52.6439,200.2340) -- (52.9106,200.2340) -- (52.9106,200.3940) -- (52.6439,200.3940) -- cycle(53.4439,200.2340) -- (53.7106,200.2340) -- (53.7106,200.3940) -- (53.4439,200.3940) -- cycle(54.2439,200.2340) -- (54.5106,200.2340) -- (54.5106,200.3940) -- (54.2439,200.3940) -- cycle(55.0439,200.2340) -- (55.3105,200.2340) -- (55.3105,200.3940) -- (55.0439,200.3940) -- cycle(55.8439,200.2340) -- (56.1105,200.2340) -- (56.1105,200.3940) -- (55.8439,200.3940) -- cycle(56.6439,200.2340) -- (56.9106,200.2340) -- (56.9106,200.3940) -- (56.6439,200.3940) -- cycle(57.4439,200.2340) -- (57.7105,200.2340) -- (57.7105,200.3940) -- (57.4439,200.3940) -- cycle(58.2439,200.2340) -- (58.5106,200.2340) -- (58.5106,200.3940) -- (58.2439,200.3940) -- cycle(59.0439,200.2340) -- (59.3105,200.2340) -- (59.3105,200.3940) -- (59.0439,200.3940) -- cycle(59.8439,200.2340) -- (60.1105,200.2340) -- (60.1105,200.3940) -- (59.8439,200.3940) -- cycle(60.6439,200.2340) -- (60.9105,200.2340) -- (60.9105,200.3940) -- (60.6439,200.3940) -- cycle(61.4439,200.2340) -- (61.7105,200.2340) -- (61.7105,200.3940) -- (61.4439,200.3940) -- cycle(62.2439,200.2340) -- (62.5105,200.2340) -- (62.5105,200.3940) -- (62.2439,200.3940) -- cycle(63.0439,200.2340) -- (63.3105,200.2340) -- (63.3105,200.3940) -- (63.0439,200.3940) -- cycle(63.8439,200.2340) -- (64.1105,200.2340) -- (64.1105,200.3940) -- (63.8439,200.3940) -- cycle(64.6439,200.2340) -- (64.9105,200.2340) -- (64.9105,200.3940) -- (64.6439,200.3940) -- cycle(65.4438,200.2340) -- (65.7105,200.2340) -- (65.7105,200.3940) -- (65.4438,200.3940) -- cycle(66.2439,200.2340) -- (66.5105,200.2340) -- (66.5105,200.3940) -- (66.2439,200.3940) -- cycle(67.0439,200.2340) -- (67.3105,200.2340) -- (67.3105,200.3940) -- (67.0439,200.3940) -- cycle(67.8438,200.2340) -- (68.1105,200.2340) -- (68.1105,200.3940) -- (67.8438,200.3940) -- cycle(68.6439,200.2340) -- (68.9105,200.2340) -- (68.9105,200.3940) -- (68.6439,200.3940) -- cycle(69.4438,200.2340) -- (69.7105,200.2340) -- (69.7105,200.3940) -- (69.4438,200.3940) -- cycle(70.2439,200.2340) -- (70.5105,200.2340) -- (70.5105,200.3940) -- (70.2439,200.3940) -- cycle(71.0438,200.2340) -- (71.3105,200.2340) -- (71.3105,200.3940) -- (71.0438,200.3940) -- cycle(71.8438,200.2340) -- (72.1105,200.2340) -- (72.1105,200.3940) -- (71.8438,200.3940) -- cycle(72.6439,200.2340) -- (72.9105,200.2340) -- (72.9105,200.3940) -- (72.6439,200.3940) -- cycle(73.4438,200.2340) -- (73.7105,200.2340) -- (73.7105,200.3940) -- (73.4438,200.3940) -- cycle(74.2438,200.2340) -- (74.5105,200.2340) -- (74.5105,200.3940) -- (74.2438,200.3940) -- cycle(75.0438,200.2340) -- (75.3105,200.2340) -- (75.3105,200.3940) -- (75.0438,200.3940) -- cycle(75.8438,200.2340) -- (76.1105,200.2340) -- (76.1105,200.3940) -- (75.8438,200.3940) -- cycle(76.6439,200.2340) -- (76.9105,200.2340) -- (76.9105,200.3940) -- (76.6439,200.3940) -- cycle(77.4438,200.2340) -- (77.7105,200.2340) -- (77.7105,200.3940) -- (77.4438,200.3940) -- cycle(78.2438,200.2340) -- (78.5105,200.2340) -- (78.5105,200.3940) -- (78.2438,200.3940) -- cycle(79.0438,200.2340) -- (79.3105,200.2340) -- (79.3105,200.3940) -- (79.0438,200.3940) -- cycle(79.8438,200.2340) -- (80.1105,200.2340) -- (80.1105,200.3940) -- (79.8438,200.3940) -- cycle(80.6438,200.2340) -- (80.9105,200.2340) -- (80.9105,200.3940) -- (80.6438,200.3940) -- cycle(81.4438,200.2340) -- (81.7105,200.2340) -- (81.7105,200.3940) -- (81.4438,200.3940) -- cycle(82.2438,200.2340) -- (82.5105,200.2340) -- (82.5105,200.3940) -- (82.2438,200.3940) -- cycle(83.0438,200.2340) -- (83.3105,200.2340) -- (83.3105,200.3940) -- (83.0438,200.3940) -- cycle(83.8438,200.2340) -- (84.1105,200.2340) -- (84.1105,200.3940) -- (83.8438,200.3940) -- cycle(84.6438,200.2340) -- (84.9105,200.2340) -- (84.9105,200.3940) -- (84.6438,200.3940) -- cycle(85.4438,200.2340) -- (85.7104,200.2340) -- (85.7104,200.3940) -- (85.4438,200.3940) -- cycle(86.2438,200.2340) -- (86.5105,200.2340) -- (86.5105,200.3940) -- (86.2438,200.3940) -- cycle(87.0438,200.2340) -- (87.3105,200.2340) -- (87.3105,200.3940) -- (87.0438,200.3940) -- cycle(87.8438,200.2340) -- (88.1105,200.2340) -- (88.1105,200.3940) -- (87.8438,200.3940) -- cycle(88.6438,200.2340) -- (88.9105,200.2340) -- (88.9105,200.3940) -- (88.6438,200.3940) -- cycle(89.4438,200.2340) -- (89.7104,200.2340) -- (89.7104,200.3940) -- (89.4438,200.3940) -- cycle(90.2438,200.2340) -- (90.5105,200.2340) -- (90.5105,200.3940) -- (90.2438,200.3940) -- cycle(91.0438,200.2340) -- (91.3105,200.2340) -- (91.3105,200.3940) -- (91.0438,200.3940) -- cycle(91.8438,200.2340) -- (92.1104,200.2340) -- (92.1104,200.3940) -- (91.8438,200.3940) -- cycle(92.6438,200.2340) -- (92.9105,200.2340) -- (92.9105,200.3940) -- (92.6438,200.3940) -- cycle(93.4438,200.2340) -- (93.7104,200.2340) -- (93.7104,200.3940) -- (93.4438,200.3940) -- cycle(94.2438,200.2340) -- (94.5105,200.2340) -- (94.5105,200.3940) -- (94.2438,200.3940) -- cycle(95.0438,200.2340) -- (95.3104,200.2340) -- (95.3104,200.3940) -- (95.0438,200.3940) -- cycle(95.8438,200.2340) -- (96.1104,200.2340) -- (96.1104,200.3940) -- (95.8438,200.3940) -- cycle(96.6438,200.2340) -- (96.9105,200.2340) -- (96.9105,200.3940) -- (96.6438,200.3940) -- cycle(97.4438,200.2340) -- (97.7104,200.2340) -- (97.7104,200.3940) -- (97.4438,200.3940) -- cycle(98.2438,200.2340) -- (98.5104,200.2340) -- (98.5104,200.3940) -- (98.2438,200.3940) -- cycle(99.0438,200.2340) -- (99.3104,200.2340) -- (99.3104,200.3940) -- (99.0438,200.3940) -- cycle(99.8438,200.2340) -- (100.1100,200.2340) -- (100.1100,200.3940) -- (99.8438,200.3940) -- cycle(100.6440,200.2340) -- (100.9100,200.2340) -- (100.9100,200.3940) -- (100.6440,200.3940) -- cycle(101.4440,200.2340) -- (101.7100,200.2340) -- (101.7100,200.3940) -- (101.4440,200.3940) -- cycle(102.2440,200.2340) -- (102.5100,200.2340) -- (102.5100,200.3940) -- (102.2440,200.3940) -- cycle(103.0440,200.2340) -- (103.3100,200.2340) -- (103.3100,200.3940) -- (103.0440,200.3940) -- cycle(103.8440,200.2340) -- (104.1100,200.2340) -- (104.1100,200.3940) -- (103.8440,200.3940) -- cycle(104.6440,200.2340) -- (104.9100,200.2340) -- (104.9100,200.3940) -- (104.6440,200.3940) -- cycle(105.4440,200.2340) -- (105.7100,200.2340) -- (105.7100,200.3940) -- (105.4440,200.3940) -- cycle(106.2440,200.2340) -- (106.5100,200.2340) -- (106.5100,200.3940) -- (106.2440,200.3940) -- cycle(107.0440,200.2340) -- (107.3100,200.2340) -- (107.3100,200.3940) -- (107.0440,200.3940) -- cycle(107.8440,200.2340) -- (108.1100,200.2340) -- (108.1100,200.3940) -- (107.8440,200.3940) -- cycle(108.6440,200.2340) -- (108.9100,200.2340) -- (108.9100,200.3940) -- (108.6440,200.3940) -- cycle(109.4440,200.2340) -- (109.7100,200.2340) -- (109.7100,200.3940) -- (109.4440,200.3940) -- cycle(110.2440,200.2340) -- (110.5100,200.2340) -- (110.5100,200.3940) -- (110.2440,200.3940) -- cycle(111.0440,200.2340) -- (111.3100,200.2340) -- (111.3100,200.3940) -- (111.0440,200.3940) -- cycle(111.8440,200.2340) -- (112.1100,200.2340) -- (112.1100,200.3940) -- (111.8440,200.3940) -- cycle(112.6440,200.2340) -- (112.9100,200.2340) -- (112.9100,200.3940) -- (112.6440,200.3940) -- cycle(113.4440,200.2340) -- (113.7100,200.2340) -- (113.7100,200.3940) -- (113.4440,200.3940) -- cycle(114.2440,200.2340) -- (114.5100,200.2340) -- (114.5100,200.3940) -- (114.2440,200.3940) -- cycle(115.0440,200.2340) -- (115.3100,200.2340) -- (115.3100,200.3940) -- (115.0440,200.3940) -- cycle(115.8440,200.2340) -- (116.1100,200.2340) -- (116.1100,200.3940) -- (115.8440,200.3940) -- cycle(116.6440,200.2340) -- (116.9100,200.2340) -- (116.9100,200.3940) -- (116.6440,200.3940) -- cycle(117.4440,200.2340) -- (117.7100,200.2340) -- (117.7100,200.3940) -- (117.4440,200.3940) -- cycle(118.2440,200.2340) -- (118.5100,200.2340) -- (118.5100,200.3940) -- (118.2440,200.3940) -- cycle(119.0440,200.2340) -- (119.3100,200.2340) -- (119.3100,200.3940) -- (119.0440,200.3940) -- cycle(119.8440,200.2340) -- (120.1100,200.2340) -- (120.1100,200.3940) -- (119.8440,200.3940) -- cycle(120.6440,200.2340) -- (120.9100,200.2340) -- (120.9100,200.3940) -- (120.6440,200.3940) -- cycle(121.4440,200.2340) -- (121.7100,200.2340) -- (121.7100,200.3940) -- (121.4440,200.3940) -- cycle(122.2440,200.2340) -- (122.5100,200.2340) -- (122.5100,200.3940) -- (122.2440,200.3940) -- cycle(123.0440,200.2340) -- (123.3100,200.2340) -- (123.3100,200.3940) -- (123.0440,200.3940) -- cycle(123.8440,200.2340) -- (124.1100,200.2340) -- (124.1100,200.3940) -- (123.8440,200.3940) -- cycle(124.6440,200.2340) -- (124.9100,200.2340) -- (124.9100,200.3940) -- (124.6440,200.3940) -- cycle(125.4440,200.2340) -- (125.7100,200.2340) -- (125.7100,200.3940) -- (125.4440,200.3940) -- cycle(126.2440,200.2340) -- (126.5100,200.2340) -- (126.5100,200.3940) -- (126.2440,200.3940) -- cycle(127.0440,200.2340) -- (127.3100,200.2340) -- (127.3100,200.3940) -- (127.0440,200.3940) -- cycle(127.8440,200.2340) -- (128.1100,200.2340) -- (128.1100,200.3940) -- (127.8440,200.3940) -- cycle(128.6440,200.2340) -- (128.9100,200.2340) -- (128.9100,200.3940) -- (128.6440,200.3940) -- cycle(129.4440,200.2340) -- (129.7100,200.2340) -- (129.7100,200.3940) -- (129.4440,200.3940) -- cycle(130.2440,200.2340) -- (130.5100,200.2340) -- (130.5100,200.3940) -- (130.2440,200.3940) -- cycle(131.0440,200.2340) -- (131.3100,200.2340) -- (131.3100,200.3940) -- (131.0440,200.3940) -- cycle(131.8440,200.2340) -- (132.1100,200.2340) -- (132.1100,200.3940) -- (131.8440,200.3940) -- cycle(132.6440,200.2340) -- (132.9100,200.2340) -- (132.9100,200.3940) -- (132.6440,200.3940) -- cycle(133.4440,200.2340) -- (133.7100,200.2340) -- (133.7100,200.3940) -- (133.4440,200.3940) -- cycle(134.2440,200.2340) -- (134.5100,200.2340) -- (134.5100,200.3940) -- (134.2440,200.3940) -- cycle(135.0440,200.2340) -- (135.3100,200.2340) -- (135.3100,200.3940) -- (135.0440,200.3940) -- cycle(135.8440,200.2340) -- (136.1100,200.2340) -- (136.1100,200.3940) -- (135.8440,200.3940) -- cycle(136.6440,200.2340) -- (136.9100,200.2340) -- (136.9100,200.3940) -- (136.6440,200.3940) -- cycle(137.4440,200.2340) -- (137.7100,200.2340) -- (137.7100,200.3940) -- (137.4440,200.3940) -- cycle(138.2440,200.2340) -- (138.5100,200.2340) -- (138.5100,200.3940) -- (138.2440,200.3940) -- cycle(139.0440,200.2340) -- (139.3100,200.2340) -- (139.3100,200.3940) -- (139.0440,200.3940) -- cycle(139.8440,200.2340) -- (140.1100,200.2340) -- (140.1100,200.3940) -- (139.8440,200.3940) -- cycle(140.6440,200.2340) -- (140.9100,200.2340) -- (140.9100,200.3940) -- (140.6440,200.3940) -- cycle(141.4440,200.2340) -- (141.7100,200.2340) -- (141.7100,200.3940) -- (141.4440,200.3940) -- cycle(142.2440,200.2340) -- (142.5100,200.2340) -- (142.5100,200.3940) -- (142.2440,200.3940) -- cycle(143.0440,200.2340) -- (143.3100,200.2340) -- (143.3100,200.3940) -- (143.0440,200.3940) -- cycle(143.8440,200.2340) -- (144.1100,200.2340) -- (144.1100,200.3940) -- (143.8440,200.3940) -- cycle(144.6440,200.2340) -- (144.9100,200.2340) -- (144.9100,200.3940) -- (144.6440,200.3940) -- cycle(145.4440,200.2340) -- (145.7100,200.2340) -- (145.7100,200.3940) -- (145.4440,200.3940) -- cycle(146.2440,200.2340) -- (146.5100,200.2340) -- (146.5100,200.3940) -- (146.2440,200.3940) -- cycle(147.0440,200.2340) -- (147.3100,200.2340) -- (147.3100,200.3940) -- (147.0440,200.3940) -- cycle(147.8440,200.2340) -- (148.1100,200.2340) -- (148.1100,200.3940) -- (147.8440,200.3940) -- cycle(148.6440,200.2340) -- (148.9100,200.2340) -- (148.9100,200.3940) -- (148.6440,200.3940) -- cycle(149.4440,200.2340) -- (149.7100,200.2340) -- (149.7100,200.3940) -- (149.4440,200.3940) -- cycle(150.2440,200.2340) -- (150.5100,200.2340) -- (150.5100,200.3940) -- (150.2440,200.3940) -- cycle(151.0440,200.2340) -- (151.3100,200.2340) -- (151.3100,200.3940) -- (151.0440,200.3940) -- cycle(151.8440,200.2340) -- (152.1100,200.2340) -- (152.1100,200.3940) -- (151.8440,200.3940) -- cycle(152.6440,200.2340) -- (152.9100,200.2340) -- (152.9100,200.3940) -- (152.6440,200.3940) -- cycle(153.4440,200.2340) -- (153.7100,200.2340) -- (153.7100,200.3940) -- (153.4440,200.3940) -- cycle(154.2440,200.2340) -- (154.5100,200.2340) -- (154.5100,200.3940) -- (154.2440,200.3940) -- cycle(155.0440,200.2340) -- (155.3100,200.2340) -- (155.3100,200.3940) -- (155.0440,200.3940) -- cycle(155.8440,200.2340) -- (156.1100,200.2340) -- (156.1100,200.3940) -- (155.8440,200.3940) -- cycle(156.6440,200.2340) -- (156.9100,200.2340) -- (156.9100,200.3940) -- (156.6440,200.3940) -- cycle(157.4440,200.2340) -- (157.7100,200.2340) -- (157.7100,200.3940) -- (157.4440,200.3940) -- cycle(158.2440,200.2340) -- (158.5100,200.2340) -- (158.5100,200.3940) -- (158.2440,200.3940) -- cycle(159.0440,200.2340) -- (159.3100,200.2340) -- (159.3100,200.3940) -- (159.0440,200.3940) -- cycle(159.8440,200.2340) -- (160.1100,200.2340) -- (160.1100,200.3940) -- (159.8440,200.3940) -- cycle(160.6440,200.2340) -- (160.9100,200.2340) -- (160.9100,200.3940) -- (160.6440,200.3940) -- cycle(161.4440,200.2340) -- (161.7100,200.2340) -- (161.7100,200.3940) -- (161.4440,200.3940) -- cycle(162.2440,200.2340) -- (162.5100,200.2340) -- (162.5100,200.3940) -- (162.2440,200.3940) -- cycle(163.0440,200.2340) -- (163.3100,200.2340) -- (163.3100,200.3940) -- (163.0440,200.3940) -- cycle(163.8440,200.2340) -- (164.1100,200.2340) -- (164.1100,200.3940) -- (163.8440,200.3940) -- cycle(164.6440,200.2340) -- (164.9100,200.2340) -- (164.9100,200.3940) -- (164.6440,200.3940) -- cycle(165.4440,200.2340) -- (165.7100,200.2340) -- (165.7100,200.3940) -- (165.4440,200.3940) -- cycle(166.2440,200.2340) -- (166.5100,200.2340) -- (166.5100,200.3940) -- (166.2440,200.3940) -- cycle(167.0440,200.2340) -- (167.3100,200.2340) -- (167.3100,200.3940) -- (167.0440,200.3940) -- cycle(167.8440,200.2340) -- (168.1100,200.2340) -- (168.1100,200.3940) -- (167.8440,200.3940) -- cycle(168.6440,200.2340) -- (168.9100,200.2340) -- (168.9100,200.3940) -- (168.6440,200.3940) -- cycle(169.4440,200.2340) -- (169.7100,200.2340) -- (169.7100,200.3940) -- (169.4440,200.3940) -- cycle(170.2440,200.2340) -- (170.5100,200.2340) -- (170.5100,200.3940) -- (170.2440,200.3940) -- cycle(171.0440,200.2340) -- (171.3100,200.2340) -- (171.3100,200.3940) -- (171.0440,200.3940) -- cycle(171.8440,200.2340) -- (172.1100,200.2340) -- (172.1100,200.3940) -- (171.8440,200.3940) -- cycle(172.6440,200.2340) -- (172.9100,200.2340) -- (172.9100,200.3940) -- (172.6440,200.3940) -- cycle(173.4440,200.2340) -- (173.7100,200.2340) -- (173.7100,200.3940) -- (173.4440,200.3940) -- cycle(174.2440,200.2340) -- (174.5100,200.2340) -- (174.5100,200.3940) -- (174.2440,200.3940) -- cycle(175.0440,200.2340) -- (175.3100,200.2340) -- (175.3100,200.3940) -- (175.0440,200.3940) -- cycle(175.8440,200.2340) -- (176.1100,200.2340) -- (176.1100,200.3940) -- (175.8440,200.3940) -- cycle(176.6440,200.2340) -- (176.9100,200.2340) -- (176.9100,200.3940) -- (176.6440,200.3940) -- cycle(177.4440,200.2340) -- (177.7100,200.2340) -- (177.7100,200.3940) -- (177.4440,200.3940) -- cycle(178.2440,200.2340) -- (178.5100,200.2340) -- (178.5100,200.3940) -- (178.2440,200.3940) -- cycle(179.0440,200.2340) -- (179.3100,200.2340) -- (179.3100,200.3940) -- (179.0440,200.3940) -- cycle(179.8440,200.2340) -- (180.1100,200.2340) -- (180.1100,200.3940) -- (179.8440,200.3940) -- cycle(180.6440,200.2340) -- (180.9100,200.2340) -- (180.9100,200.3940) -- (180.6440,200.3940) -- cycle(181.4440,200.2340) -- (181.7100,200.2340) -- (181.7100,200.3940) -- (181.4440,200.3940) -- cycle(182.2440,200.2340) -- (182.5100,200.2340) -- (182.5100,200.3940) -- (182.2440,200.3940) -- cycle(183.0440,200.2340) -- (183.3100,200.2340) -- (183.3100,200.3940) -- (183.0440,200.3940) -- cycle(183.8440,200.2340) -- (184.1100,200.2340) -- (184.1100,200.3940) -- (183.8440,200.3940) -- cycle(184.6440,200.2340) -- (184.9100,200.2340) -- (184.9100,200.3940) -- (184.6440,200.3940) -- cycle(185.4440,200.2340) -- (185.7100,200.2340) -- (185.7100,200.3940) -- (185.4440,200.3940) -- cycle(186.2440,200.2340) -- (186.5100,200.2340) -- (186.5100,200.3940) -- (186.2440,200.3940) -- cycle(187.0440,200.2340) -- (187.3100,200.2340) -- (187.3100,200.3940) -- (187.0440,200.3940) -- cycle(187.8440,200.2340) -- (188.1100,200.2340) -- (188.1100,200.3940) -- (187.8440,200.3940) -- cycle(188.6440,200.2340) -- (188.9100,200.2340) -- (188.9100,200.3940) -- (188.6440,200.3940) -- cycle(189.4440,200.2340) -- (189.7100,200.2340) -- (189.7100,200.3940) -- (189.4440,200.3940) -- cycle(190.2440,200.2340) -- (190.5100,200.2340) -- (190.5100,200.3940) -- (190.2440,200.3940) -- cycle(191.0440,200.2340) -- (191.3100,200.2340) -- (191.3100,200.3940) -- (191.0440,200.3940) -- cycle(191.8440,200.2340) -- (192.1100,200.2340) -- (192.1100,200.3940) -- (191.8440,200.3940) -- cycle(192.6440,200.2340) -- (192.9100,200.2340) -- (192.9100,200.3940) -- (192.6440,200.3940) -- cycle(193.4440,200.2340) -- (193.7100,200.2340) -- (193.7100,200.3940) -- (193.4440,200.3940) -- cycle(194.2440,200.2340) -- (194.5100,200.2340) -- (194.5100,200.3940) -- (194.2440,200.3940) -- cycle(195.0440,200.2340) -- (195.3100,200.2340) -- (195.3100,200.3940) -- (195.0440,200.3940) -- cycle(195.8440,200.2340) -- (196.1100,200.2340) -- (196.1100,200.3940) -- (195.8440,200.3940) -- cycle(196.6440,200.2340) -- (196.9100,200.2340) -- (196.9100,200.3940) -- (196.6440,200.3940) -- cycle(197.4440,200.2340) -- (197.7100,200.2340) -- (197.7100,200.3940) -- (197.4440,200.3940) -- cycle(198.2440,200.2340) -- (198.5100,200.2340) -- (198.5100,200.3940) -- (198.2440,200.3940) -- cycle(199.0440,200.2340) -- (199.3100,200.2340) -- (199.3100,200.3940) -- (199.0440,200.3940) -- cycle(199.8440,200.2340) -- (200.1100,200.2340) -- (200.1100,200.3940) -- (199.8440,200.3940) -- cycle(200.6440,200.2340) -- (200.9100,200.2340) -- (200.9100,200.3940) -- (200.6440,200.3940) -- cycle(201.4440,200.2340) -- (201.7100,200.2340) -- (201.7100,200.3940) -- (201.4440,200.3940) -- cycle(202.2430,200.2340) -- (202.5100,200.2340) -- (202.5100,200.3940) -- (202.2430,200.3940) -- cycle(203.0440,200.2340) -- (203.3100,200.2340) -- (203.3100,200.3940) -- (203.0440,200.3940) -- cycle(203.8440,200.2340) -- (204.1100,200.2340) -- (204.1100,200.3940) -- (203.8440,200.3940) -- cycle(204.6440,200.2340) -- (204.9100,200.2340) -- (204.9100,200.3940) -- (204.6440,200.3940) -- cycle(205.4430,200.2340) -- (205.7100,200.2340) -- (205.7100,200.3940) -- (205.4430,200.3940) -- cycle(206.2430,200.2340) -- (206.5100,200.2340) -- (206.5100,200.3940) -- (206.2430,200.3940) -- cycle(207.0440,200.2340) -- (207.3100,200.2340) -- (207.3100,200.3940) -- (207.0440,200.3940) -- cycle(207.8440,200.2340) -- (208.1100,200.2340) -- (208.1100,200.3940) -- (207.8440,200.3940) -- cycle(208.6430,200.2340) -- (208.9100,200.2340) -- (208.9100,200.3940) -- (208.6430,200.3940) -- cycle(209.4430,200.2340) -- (209.7100,200.2340) -- (209.7100,200.3940) -- (209.4430,200.3940) -- cycle(210.2430,200.2340) -- (210.5100,200.2340) -- (210.5100,200.3940) -- (210.2430,200.3940) -- cycle(211.0440,200.2340) -- (211.3100,200.2340) -- (211.3100,200.3940) -- (211.0440,200.3940) -- cycle(211.8430,200.2340) -- (212.1100,200.2340) -- (212.1100,200.3940) -- (211.8430,200.3940) -- cycle(212.6430,200.2340) -- (212.9100,200.2340) -- (212.9100,200.3940) -- (212.6430,200.3940) -- cycle(213.4430,200.2340) -- (213.7100,200.2340) -- (213.7100,200.3940) -- (213.4430,200.3940) -- cycle(214.2430,200.2340) -- (214.5100,200.2340) -- (214.5100,200.3940) -- (214.2430,200.3940) -- cycle(215.0430,200.2340) -- (215.3100,200.2340) -- (215.3100,200.3940) -- (215.0430,200.3940) -- cycle(215.8430,200.2340) -- (216.1100,200.2340) -- (216.1100,200.3940) -- (215.8430,200.3940) -- cycle(216.6430,200.2340) -- (216.9100,200.2340) -- (216.9100,200.3940) -- (216.6430,200.3940) -- cycle(217.4430,200.2340) -- (217.7100,200.2340) -- (217.7100,200.3940) -- (217.4430,200.3940) -- cycle(218.2430,200.2340) -- (218.5100,200.2340) -- (218.5100,200.3940) -- (218.2430,200.3940) -- cycle(219.0430,200.2340) -- (219.3100,200.2340) -- (219.3100,200.3940) -- (219.0430,200.3940) -- cycle(219.8430,200.2340) -- (220.1100,200.2340) -- (220.1100,200.3940) -- (219.8430,200.3940) -- cycle(220.6430,200.2340) -- (220.9100,200.2340) -- (220.9100,200.3940) -- (220.6430,200.3940) -- cycle(221.4430,200.2340) -- (221.7100,200.2340) -- (221.7100,200.3940) -- (221.4430,200.3940) -- cycle(222.2430,200.2340) -- (222.5100,200.2340) -- (222.5100,200.3940) -- (222.2430,200.3940) -- cycle(223.0430,200.2340) -- (223.3100,200.2340) -- (223.3100,200.3940) -- (223.0430,200.3940) -- cycle(223.8430,200.2340) -- (224.1100,200.2340) -- (224.1100,200.3940) -- (223.8430,200.3940) -- cycle(224.6430,200.2340) -- (224.9100,200.2340) -- (224.9100,200.3940) -- (224.6430,200.3940) -- cycle(225.4430,200.2340) -- (225.7100,200.2340) -- (225.7100,200.3940) -- (225.4430,200.3940) -- cycle(226.2430,200.2340) -- (226.5100,200.2340) -- (226.5100,200.3940) -- (226.2430,200.3940) -- cycle(227.0430,200.2340) -- (227.3100,200.2340) -- (227.3100,200.3940) -- (227.0430,200.3940) -- cycle(227.8430,200.2340) -- (228.1100,200.2340) -- (228.1100,200.3940) -- (227.8430,200.3940) -- cycle(228.6430,200.2340) -- (228.9100,200.2340) -- (228.9100,200.3940) -- (228.6430,200.3940) -- cycle(229.4430,200.2340) -- (229.7100,200.2340) -- (229.7100,200.3940) -- (229.4430,200.3940) -- cycle(230.2430,200.2340) -- (230.5100,200.2340) -- (230.5100,200.3940) -- (230.2430,200.3940) -- cycle(231.0430,200.2340) -- (231.3100,200.2340) -- (231.3100,200.3940) -- (231.0430,200.3940) -- cycle(231.8430,200.2340) -- (232.1100,200.2340) -- (232.1100,200.3940) -- (231.8430,200.3940) -- cycle(232.6430,200.2340) -- (232.9100,200.2340) -- (232.9100,200.3940) -- (232.6430,200.3940) -- cycle(233.4430,200.2340) -- (233.7100,200.2340) -- (233.7100,200.3940) -- (233.4430,200.3940) -- cycle(234.2430,200.2340) -- (234.5100,200.2340) -- (234.5100,200.3940) -- (234.2430,200.3940) -- cycle(235.0430,200.2340) -- (235.3100,200.2340) -- (235.3100,200.3940) -- (235.0430,200.3940) -- cycle(235.8430,200.2340) -- (236.1100,200.2340) -- (236.1100,200.3940) -- (235.8430,200.3940) -- cycle(236.6430,200.2340) -- (236.9100,200.2340) -- (236.9100,200.3940) -- (236.6430,200.3940) -- cycle(237.4430,200.2340) -- (237.7100,200.2340) -- (237.7100,200.3940) -- (237.4430,200.3940) -- cycle(238.2430,200.2340) -- (238.5100,200.2340) -- (238.5100,200.3940) -- (238.2430,200.3940) -- cycle(239.0430,200.2340) -- (239.3100,200.2340) -- (239.3100,200.3940) -- (239.0430,200.3940) -- cycle(239.8430,200.2340) -- (240.1100,200.2340) -- (240.1100,200.3940) -- (239.8430,200.3940) -- cycle(240.6430,200.2340) -- (240.9100,200.2340) -- (240.9100,200.3940) -- (240.6430,200.3940) -- cycle(241.4430,200.2340) -- (241.7100,200.2340) -- (241.7100,200.3940) -- (241.4430,200.3940) -- cycle(242.2430,200.2340) -- (242.5100,200.2340) -- (242.5100,200.3940) -- (242.2430,200.3940) -- cycle(243.0430,200.2340) -- (243.2520,200.2340) -- (243.2520,200.3940) -- (243.0430,200.3940) -- cycle(37.4439,200.2340) -- (37.7106,200.2340) -- (37.7106,200.3940) -- (37.4439,200.3940) -- cycle;



  \path[fill=c7e7e7e,line join=round,line width=0.256pt] (29.8598,223.2660) -- (30.1264,223.2660) -- (30.1264,223.4260) -- (29.8598,223.4260) -- cycle(30.6598,223.2660) -- (30.9265,223.2660) -- (30.9265,223.4260) -- (30.6598,223.4260) -- cycle(31.4598,223.2660) -- (31.7264,223.2660) -- (31.7264,223.4260) -- (31.4598,223.4260) -- cycle(32.2598,223.2660) -- (32.5264,223.2660) -- (32.5264,223.4260) -- (32.2598,223.4260) -- cycle(33.0598,223.2660) -- (33.3264,223.2660) -- (33.3264,223.4260) -- (33.0598,223.4260) -- cycle(33.8598,223.2660) -- (34.1264,223.2660) -- (34.1264,223.4260) -- (33.8598,223.4260) -- cycle(34.6598,223.2660) -- (34.9264,223.2660) -- (34.9264,223.4260) -- (34.6598,223.4260) -- cycle(35.4598,223.2660) -- (35.7264,223.2660) -- (35.7264,223.4260) -- (35.4598,223.4260) -- cycle(36.2598,223.2660) -- (36.5264,223.2660) -- (36.5264,223.4260) -- (36.2598,223.4260) -- cycle(37.0598,223.2660) -- (37.3264,223.2660) -- (37.3264,223.4260) -- (37.0598,223.4260) -- cycle(37.8597,223.2660) -- (38.1264,223.2660) -- (38.1264,223.4260) -- (37.8597,223.4260) -- cycle(38.6598,223.2660) -- (38.9264,223.2660) -- (38.9264,223.4260) -- (38.6598,223.4260) -- cycle(39.4597,223.2660) -- (39.7264,223.2660) -- (39.7264,223.4260) -- (39.4597,223.4260) -- cycle(40.2598,223.2660) -- (40.5264,223.2660) -- (40.5264,223.4260) -- (40.2598,223.4260) -- cycle(41.0598,223.2660) -- (41.3264,223.2660) -- (41.3264,223.4260) -- (41.0598,223.4260) -- cycle(41.8597,223.2660) -- (42.1264,223.2660) -- (42.1264,223.4260) -- (41.8597,223.4260) -- cycle(42.6597,223.2660) -- (42.9264,223.2660) -- (42.9264,223.4260) -- (42.6597,223.4260) -- cycle(43.4597,223.2660) -- (43.7264,223.2660) -- (43.7264,223.4260) -- (43.4597,223.4260) -- cycle(44.2597,223.2660) -- (44.5264,223.2660) -- (44.5264,223.4260) -- (44.2597,223.4260) -- cycle(45.0598,223.2660) -- (45.3264,223.2660) -- (45.3264,223.4260) -- (45.0598,223.4260) -- cycle(45.8597,223.2660) -- (46.1264,223.2660) -- (46.1264,223.4260) -- (45.8597,223.4260) -- cycle(46.6597,223.2660) -- (46.9264,223.2660) -- (46.9264,223.4260) -- (46.6597,223.4260) -- cycle(47.4597,223.2660) -- (47.7264,223.2660) -- (47.7264,223.4260) -- (47.4597,223.4260) -- cycle(48.2597,223.2660) -- (48.5264,223.2660) -- (48.5264,223.4260) -- (48.2597,223.4260) -- cycle(49.0597,223.2660) -- (49.3264,223.2660) -- (49.3264,223.4260) -- (49.0597,223.4260) -- cycle(49.8597,223.2660) -- (50.1264,223.2660) -- (50.1264,223.4260) -- (49.8597,223.4260) -- cycle(50.6597,223.2660) -- (50.9264,223.2660) -- (50.9264,223.4260) -- (50.6597,223.4260) -- cycle(51.4597,223.2660) -- (51.7264,223.2660) -- (51.7264,223.4260) -- (51.4597,223.4260) -- cycle(52.2597,223.2660) -- (52.5264,223.2660) -- (52.5264,223.4260) -- (52.2597,223.4260) -- cycle(53.0597,223.2660) -- (53.3264,223.2660) -- (53.3264,223.4260) -- (53.0597,223.4260) -- cycle(53.8597,223.2660) -- (54.1264,223.2660) -- (54.1264,223.4260) -- (53.8597,223.4260) -- cycle(54.6597,223.2660) -- (54.9264,223.2660) -- (54.9264,223.4260) -- (54.6597,223.4260) -- cycle(55.4597,223.2660) -- (55.7264,223.2660) -- (55.7264,223.4260) -- (55.4597,223.4260) -- cycle(56.2597,223.2660) -- (56.5264,223.2660) -- (56.5264,223.4260) -- (56.2597,223.4260) -- cycle(57.0597,223.2660) -- (57.3264,223.2660) -- (57.3264,223.4260) -- (57.0597,223.4260) -- cycle(57.8597,223.2660) -- (58.1264,223.2660) -- (58.1264,223.4260) -- (57.8597,223.4260) -- cycle(58.6597,223.2660) -- (58.9264,223.2660) -- (58.9264,223.4260) -- (58.6597,223.4260) -- cycle(59.4597,223.2660) -- (59.7264,223.2660) -- (59.7264,223.4260) -- (59.4597,223.4260) -- cycle(60.2597,223.2660) -- (60.5264,223.2660) -- (60.5264,223.4260) -- (60.2597,223.4260) -- cycle(61.0597,223.2660) -- (61.3264,223.2660) -- (61.3264,223.4260) -- (61.0597,223.4260) -- cycle(61.8597,223.2660) -- (62.1263,223.2660) -- (62.1263,223.4260) -- (61.8597,223.4260) -- cycle(62.6597,223.2660) -- (62.9264,223.2660) -- (62.9264,223.4260) -- (62.6597,223.4260) -- cycle(63.4597,223.2660) -- (63.7263,223.2660) -- (63.7263,223.4260) -- (63.4597,223.4260) -- cycle(64.2597,223.2660) -- (64.5264,223.2660) -- (64.5264,223.4260) -- (64.2597,223.4260) -- cycle(65.0597,223.2660) -- (65.3264,223.2660) -- (65.3264,223.4260) -- (65.0597,223.4260) -- cycle(65.8597,223.2660) -- (66.1263,223.2660) -- (66.1263,223.4260) -- (65.8597,223.4260) -- cycle(66.6597,223.2660) -- (66.9264,223.2660) -- (66.9264,223.4260) -- (66.6597,223.4260) -- cycle(67.4597,223.2660) -- (67.7263,223.2660) -- (67.7263,223.4260) -- (67.4597,223.4260) -- cycle(68.2597,223.2660) -- (68.5263,223.2660) -- (68.5263,223.4260) -- (68.2597,223.4260) -- cycle(69.0597,223.2660) -- (69.3264,223.2660) -- (69.3264,223.4260) -- (69.0597,223.4260) -- cycle(69.8597,223.2660) -- (70.1263,223.2660) -- (70.1263,223.4260) -- (69.8597,223.4260) -- cycle(70.6597,223.2660) -- (70.9263,223.2660) -- (70.9263,223.4260) -- (70.6597,223.4260) -- cycle(71.4597,223.2660) -- (71.7263,223.2660) -- (71.7263,223.4260) -- (71.4597,223.4260) -- cycle(72.2596,223.2660) -- (72.5263,223.2660) -- (72.5263,223.4260) -- (72.2596,223.4260) -- cycle(73.0597,223.2660) -- (73.3264,223.2660) -- (73.3264,223.4260) -- (73.0597,223.4260) -- cycle(73.8597,223.2660) -- (74.1263,223.2660) -- (74.1263,223.4260) -- (73.8597,223.4260) -- cycle(74.6597,223.2660) -- (74.9263,223.2660) -- (74.9263,223.4260) -- (74.6597,223.4260) -- cycle(75.4597,223.2660) -- (75.7263,223.2660) -- (75.7263,223.4260) -- (75.4597,223.4260) -- cycle(76.2596,223.2660) -- (76.5263,223.2660) -- (76.5263,223.4260) -- (76.2596,223.4260) -- cycle(77.0597,223.2660) -- (77.3263,223.2660) -- (77.3263,223.4260) -- (77.0597,223.4260) -- cycle(77.8597,223.2660) -- (78.1263,223.2660) -- (78.1263,223.4260) -- (77.8597,223.4260) -- cycle(78.6596,223.2660) -- (78.9263,223.2660) -- (78.9263,223.4260) -- (78.6596,223.4260) -- cycle(79.4597,223.2660) -- (79.7263,223.2660) -- (79.7263,223.4260) -- (79.4597,223.4260) -- cycle(80.2596,223.2660) -- (80.5263,223.2660) -- (80.5263,223.4260) -- (80.2596,223.4260) -- cycle(81.0597,223.2660) -- (81.3263,223.2660) -- (81.3263,223.4260) -- (81.0597,223.4260) -- cycle(81.8596,223.2660) -- (82.1263,223.2660) -- (82.1263,223.4260) -- (81.8596,223.4260) -- cycle(82.6596,223.2660) -- (82.9263,223.2660) -- (82.9263,223.4260) -- (82.6596,223.4260) -- cycle(83.4597,223.2660) -- (83.7263,223.2660) -- (83.7263,223.4260) -- (83.4597,223.4260) -- cycle(84.2596,223.2660) -- (84.5263,223.2660) -- (84.5263,223.4260) -- (84.2596,223.4260) -- cycle(85.0596,223.2660) -- (85.3263,223.2660) -- (85.3263,223.4260) -- (85.0596,223.4260) -- cycle(85.8596,223.2660) -- (86.1263,223.2660) -- (86.1263,223.4260) -- (85.8596,223.4260) -- cycle(86.6596,223.2660) -- (86.9263,223.2660) -- (86.9263,223.4260) -- (86.6596,223.4260) -- cycle(87.4597,223.2660) -- (87.7263,223.2660) -- (87.7263,223.4260) -- (87.4597,223.4260) -- cycle(88.2596,223.2660) -- (88.5263,223.2660) -- (88.5263,223.4260) -- (88.2596,223.4260) -- cycle(89.0596,223.2660) -- (89.3263,223.2660) -- (89.3263,223.4260) -- (89.0596,223.4260) -- cycle(89.8596,223.2660) -- (90.1263,223.2660) -- (90.1263,223.4260) -- (89.8596,223.4260) -- cycle(90.6596,223.2660) -- (90.9263,223.2660) -- (90.9263,223.4260) -- (90.6596,223.4260) -- cycle(91.4596,223.2660) -- (91.7263,223.2660) -- (91.7263,223.4260) -- (91.4596,223.4260) -- cycle(92.2596,223.2660) -- (92.5263,223.2660) -- (92.5263,223.4260) -- (92.2596,223.4260) -- cycle(93.0596,223.2660) -- (93.3263,223.2660) -- (93.3263,223.4260) -- (93.0596,223.4260) -- cycle(93.8596,223.2660) -- (94.1263,223.2660) -- (94.1263,223.4260) -- (93.8596,223.4260) -- cycle(94.6596,223.2660) -- (94.9263,223.2660) -- (94.9263,223.4260) -- (94.6596,223.4260) -- cycle(95.4596,223.2660) -- (95.7263,223.2660) -- (95.7263,223.4260) -- (95.4596,223.4260) -- cycle(96.2596,223.2660) -- (96.5262,223.2660) -- (96.5262,223.4260) -- (96.2596,223.4260) -- cycle(97.0596,223.2660) -- (97.3263,223.2660) -- (97.3263,223.4260) -- (97.0596,223.4260) -- cycle(97.8596,223.2660) -- (98.1263,223.2660) -- (98.1263,223.4260) -- (97.8596,223.4260) -- cycle(98.6596,223.2660) -- (98.9263,223.2660) -- (98.9263,223.4260) -- (98.6596,223.4260) -- cycle(99.4596,223.2660) -- (99.7263,223.2660) -- (99.7263,223.4260) -- (99.4596,223.4260) -- cycle(100.2600,223.2660) -- (100.5260,223.2660) -- (100.5260,223.4260) -- (100.2600,223.4260) -- cycle(101.0600,223.2660) -- (101.3260,223.2660) -- (101.3260,223.4260) -- (101.0600,223.4260) -- cycle(101.8600,223.2660) -- (102.1260,223.2660) -- (102.1260,223.4260) -- (101.8600,223.4260) -- cycle(102.6600,223.2660) -- (102.9260,223.2660) -- (102.9260,223.4260) -- (102.6600,223.4260) -- cycle(103.4600,223.2660) -- (103.7260,223.2660) -- (103.7260,223.4260) -- (103.4600,223.4260) -- cycle(104.2600,223.2660) -- (104.5260,223.2660) -- (104.5260,223.4260) -- (104.2600,223.4260) -- cycle(105.0600,223.2660) -- (105.3260,223.2660) -- (105.3260,223.4260) -- (105.0600,223.4260) -- cycle(105.8600,223.2660) -- (106.1260,223.2660) -- (106.1260,223.4260) -- (105.8600,223.4260) -- cycle(106.6600,223.2660) -- (106.9260,223.2660) -- (106.9260,223.4260) -- (106.6600,223.4260) -- cycle(107.4600,223.2660) -- (107.7260,223.2660) -- (107.7260,223.4260) -- (107.4600,223.4260) -- cycle(108.2600,223.2660) -- (108.5260,223.2660) -- (108.5260,223.4260) -- (108.2600,223.4260) -- cycle(109.0600,223.2660) -- (109.3260,223.2660) -- (109.3260,223.4260) -- (109.0600,223.4260) -- cycle(109.8600,223.2660) -- (110.1260,223.2660) -- (110.1260,223.4260) -- (109.8600,223.4260) -- cycle(110.6600,223.2660) -- (110.9260,223.2660) -- (110.9260,223.4260) -- (110.6600,223.4260) -- cycle(111.4600,223.2660) -- (111.7260,223.2660) -- (111.7260,223.4260) -- (111.4600,223.4260) -- cycle(112.2600,223.2660) -- (112.5260,223.2660) -- (112.5260,223.4260) -- (112.2600,223.4260) -- cycle(113.0600,223.2660) -- (113.3260,223.2660) -- (113.3260,223.4260) -- (113.0600,223.4260) -- cycle(113.8600,223.2660) -- (114.1260,223.2660) -- (114.1260,223.4260) -- (113.8600,223.4260) -- cycle(114.6600,223.2660) -- (114.9260,223.2660) -- (114.9260,223.4260) -- (114.6600,223.4260) -- cycle(115.4600,223.2660) -- (115.7260,223.2660) -- (115.7260,223.4260) -- (115.4600,223.4260) -- cycle(116.2600,223.2660) -- (116.5260,223.2660) -- (116.5260,223.4260) -- (116.2600,223.4260) -- cycle(117.0600,223.2660) -- (117.3260,223.2660) -- (117.3260,223.4260) -- (117.0600,223.4260) -- cycle(117.8600,223.2660) -- (118.1260,223.2660) -- (118.1260,223.4260) -- (117.8600,223.4260) -- cycle(118.6600,223.2660) -- (118.9260,223.2660) -- (118.9260,223.4260) -- (118.6600,223.4260) -- cycle(119.4600,223.2660) -- (119.7260,223.2660) -- (119.7260,223.4260) -- (119.4600,223.4260) -- cycle(120.2600,223.2660) -- (120.5260,223.2660) -- (120.5260,223.4260) -- (120.2600,223.4260) -- cycle(121.0600,223.2660) -- (121.3260,223.2660) -- (121.3260,223.4260) -- (121.0600,223.4260) -- cycle(121.8600,223.2660) -- (122.1260,223.2660) -- (122.1260,223.4260) -- (121.8600,223.4260) -- cycle(122.6600,223.2660) -- (122.9260,223.2660) -- (122.9260,223.4260) -- (122.6600,223.4260) -- cycle(123.4600,223.2660) -- (123.7260,223.2660) -- (123.7260,223.4260) -- (123.4600,223.4260) -- cycle(124.2600,223.2660) -- (124.5260,223.2660) -- (124.5260,223.4260) -- (124.2600,223.4260) -- cycle(125.0600,223.2660) -- (125.3260,223.2660) -- (125.3260,223.4260) -- (125.0600,223.4260) -- cycle(125.8600,223.2660) -- (126.1260,223.2660) -- (126.1260,223.4260) -- (125.8600,223.4260) -- cycle(126.6600,223.2660) -- (126.9260,223.2660) -- (126.9260,223.4260) -- (126.6600,223.4260) -- cycle(127.4600,223.2660) -- (127.7260,223.2660) -- (127.7260,223.4260) -- (127.4600,223.4260) -- cycle(128.2600,223.2660) -- (128.5260,223.2660) -- (128.5260,223.4260) -- (128.2600,223.4260) -- cycle(129.0600,223.2660) -- (129.3260,223.2660) -- (129.3260,223.4260) -- (129.0600,223.4260) -- cycle(129.8590,223.2660) -- (130.1260,223.2660) -- (130.1260,223.4260) -- (129.8590,223.4260) -- cycle(130.6600,223.2660) -- (130.9260,223.2660) -- (130.9260,223.4260) -- (130.6600,223.4260) -- cycle(131.4600,223.2660) -- (131.7260,223.2660) -- (131.7260,223.4260) -- (131.4600,223.4260) -- cycle(132.2600,223.2660) -- (132.5260,223.2660) -- (132.5260,223.4260) -- (132.2600,223.4260) -- cycle(133.0600,223.2660) -- (133.3260,223.2660) -- (133.3260,223.4260) -- (133.0600,223.4260) -- cycle(133.8590,223.2660) -- (134.1260,223.2660) -- (134.1260,223.4260) -- (133.8590,223.4260) -- cycle(134.6600,223.2660) -- (134.9260,223.2660) -- (134.9260,223.4260) -- (134.6600,223.4260) -- cycle(135.4600,223.2660) -- (135.7260,223.2660) -- (135.7260,223.4260) -- (135.4600,223.4260) -- cycle(136.2590,223.2660) -- (136.5260,223.2660) -- (136.5260,223.4260) -- (136.2590,223.4260) -- cycle(137.0600,223.2660) -- (137.3260,223.2660) -- (137.3260,223.4260) -- (137.0600,223.4260) -- cycle(137.8590,223.2660) -- (138.1260,223.2660) -- (138.1260,223.4260) -- (137.8590,223.4260) -- cycle(138.6600,223.2660) -- (138.9260,223.2660) -- (138.9260,223.4260) -- (138.6600,223.4260) -- cycle(139.4590,223.2660) -- (139.7260,223.2660) -- (139.7260,223.4260) -- (139.4590,223.4260) -- cycle(140.2590,223.2660) -- (140.5260,223.2660) -- (140.5260,223.4260) -- (140.2590,223.4260) -- cycle(141.0600,223.2660) -- (141.3260,223.2660) -- (141.3260,223.4260) -- (141.0600,223.4260) -- cycle(141.8590,223.2660) -- (142.1260,223.2660) -- (142.1260,223.4260) -- (141.8590,223.4260) -- cycle(142.6590,223.2660) -- (142.9260,223.2660) -- (142.9260,223.4260) -- (142.6590,223.4260) -- cycle(143.4590,223.2660) -- (143.7260,223.2660) -- (143.7260,223.4260) -- (143.4590,223.4260) -- cycle(144.2590,223.2660) -- (144.5260,223.2660) -- (144.5260,223.4260) -- (144.2590,223.4260) -- cycle(145.0600,223.2660) -- (145.3260,223.2660) -- (145.3260,223.4260) -- (145.0600,223.4260) -- cycle(145.8590,223.2660) -- (146.1260,223.2660) -- (146.1260,223.4260) -- (145.8590,223.4260) -- cycle(146.6590,223.2660) -- (146.9260,223.2660) -- (146.9260,223.4260) -- (146.6590,223.4260) -- cycle(147.4590,223.2660) -- (147.7260,223.2660) -- (147.7260,223.4260) -- (147.4590,223.4260) -- cycle(148.2590,223.2660) -- (148.5260,223.2660) -- (148.5260,223.4260) -- (148.2590,223.4260) -- cycle(149.0590,223.2660) -- (149.3260,223.2660) -- (149.3260,223.4260) -- (149.0590,223.4260) -- cycle(149.8590,223.2660) -- (150.1260,223.2660) -- (150.1260,223.4260) -- (149.8590,223.4260) -- cycle(150.6590,223.2660) -- (150.9260,223.2660) -- (150.9260,223.4260) -- (150.6590,223.4260) -- cycle(151.4590,223.2660) -- (151.7260,223.2660) -- (151.7260,223.4260) -- (151.4590,223.4260) -- cycle(152.2590,223.2660) -- (152.5260,223.2660) -- (152.5260,223.4260) -- (152.2590,223.4260) -- cycle(153.0590,223.2660) -- (153.3260,223.2660) -- (153.3260,223.4260) -- (153.0590,223.4260) -- cycle(153.8590,223.2660) -- (154.1260,223.2660) -- (154.1260,223.4260) -- (153.8590,223.4260) -- cycle(154.6590,223.2660) -- (154.9260,223.2660) -- (154.9260,223.4260) -- (154.6590,223.4260) -- cycle(155.4590,223.2660) -- (155.7260,223.2660) -- (155.7260,223.4260) -- (155.4590,223.4260) -- cycle(156.2590,223.2660) -- (156.5260,223.2660) -- (156.5260,223.4260) -- (156.2590,223.4260) -- cycle(157.0590,223.2660) -- (157.3260,223.2660) -- (157.3260,223.4260) -- (157.0590,223.4260) -- cycle(157.8590,223.2660) -- (158.1260,223.2660) -- (158.1260,223.4260) -- (157.8590,223.4260) -- cycle(158.6590,223.2660) -- (158.9260,223.2660) -- (158.9260,223.4260) -- (158.6590,223.4260) -- cycle(159.4590,223.2660) -- (159.7260,223.2660) -- (159.7260,223.4260) -- (159.4590,223.4260) -- cycle(160.2590,223.2660) -- (160.5260,223.2660) -- (160.5260,223.4260) -- (160.2590,223.4260) -- cycle(161.0590,223.2660) -- (161.3260,223.2660) -- (161.3260,223.4260) -- (161.0590,223.4260) -- cycle(161.8590,223.2660) -- (162.1260,223.2660) -- (162.1260,223.4260) -- (161.8590,223.4260) -- cycle(162.6590,223.2660) -- (162.9260,223.2660) -- (162.9260,223.4260) -- (162.6590,223.4260) -- cycle(163.4590,223.2660) -- (163.7260,223.2660) -- (163.7260,223.4260) -- (163.4590,223.4260) -- cycle(164.2590,223.2660) -- (164.5260,223.2660) -- (164.5260,223.4260) -- (164.2590,223.4260) -- cycle(165.0590,223.2660) -- (165.3260,223.2660) -- (165.3260,223.4260) -- (165.0590,223.4260) -- cycle(165.8590,223.2660) -- (166.1260,223.2660) -- (166.1260,223.4260) -- (165.8590,223.4260) -- cycle(166.6590,223.2660) -- (166.9260,223.2660) -- (166.9260,223.4260) -- (166.6590,223.4260) -- cycle(167.4590,223.2660) -- (167.7260,223.2660) -- (167.7260,223.4260) -- (167.4590,223.4260) -- cycle(168.2590,223.2660) -- (168.5260,223.2660) -- (168.5260,223.4260) -- (168.2590,223.4260) -- cycle(169.0590,223.2660) -- (169.3260,223.2660) -- (169.3260,223.4260) -- (169.0590,223.4260) -- cycle(169.8590,223.2660) -- (170.1260,223.2660) -- (170.1260,223.4260) -- (169.8590,223.4260) -- cycle(170.6590,223.2660) -- (170.9260,223.2660) -- (170.9260,223.4260) -- (170.6590,223.4260) -- cycle(171.4590,223.2660) -- (171.7260,223.2660) -- (171.7260,223.4260) -- (171.4590,223.4260) -- cycle(172.2590,223.2660) -- (172.5260,223.2660) -- (172.5260,223.4260) -- (172.2590,223.4260) -- cycle(173.0590,223.2660) -- (173.3260,223.2660) -- (173.3260,223.4260) -- (173.0590,223.4260) -- cycle(173.8590,223.2660) -- (174.1260,223.2660) -- (174.1260,223.4260) -- (173.8590,223.4260) -- cycle(174.6590,223.2660) -- (174.9260,223.2660) -- (174.9260,223.4260) -- (174.6590,223.4260) -- cycle(175.4590,223.2660) -- (175.7260,223.2660) -- (175.7260,223.4260) -- (175.4590,223.4260) -- cycle(176.2590,223.2660) -- (176.5260,223.2660) -- (176.5260,223.4260) -- (176.2590,223.4260) -- cycle(177.0590,223.2660) -- (177.3260,223.2660) -- (177.3260,223.4260) -- (177.0590,223.4260) -- cycle(177.8590,223.2660) -- (178.1260,223.2660) -- (178.1260,223.4260) -- (177.8590,223.4260) -- cycle(178.6590,223.2660) -- (178.9260,223.2660) -- (178.9260,223.4260) -- (178.6590,223.4260) -- cycle(179.4590,223.2660) -- (179.7260,223.2660) -- (179.7260,223.4260) -- (179.4590,223.4260) -- cycle(180.2590,223.2660) -- (180.5260,223.2660) -- (180.5260,223.4260) -- (180.2590,223.4260) -- cycle(181.0590,223.2660) -- (181.3260,223.2660) -- (181.3260,223.4260) -- (181.0590,223.4260) -- cycle(181.8590,223.2660) -- (182.1260,223.2660) -- (182.1260,223.4260) -- (181.8590,223.4260) -- cycle(182.6590,223.2660) -- (182.9260,223.2660) -- (182.9260,223.4260) -- (182.6590,223.4260) -- cycle(183.4590,223.2660) -- (183.7260,223.2660) -- (183.7260,223.4260) -- (183.4590,223.4260) -- cycle(184.2590,223.2660) -- (184.5260,223.2660) -- (184.5260,223.4260) -- (184.2590,223.4260) -- cycle(185.0590,223.2660) -- (185.3260,223.2660) -- (185.3260,223.4260) -- (185.0590,223.4260) -- cycle(185.8590,223.2660) -- (186.1260,223.2660) -- (186.1260,223.4260) -- (185.8590,223.4260) -- cycle(186.6590,223.2660) -- (186.9260,223.2660) -- (186.9260,223.4260) -- (186.6590,223.4260) -- cycle(187.4590,223.2660) -- (187.7260,223.2660) -- (187.7260,223.4260) -- (187.4590,223.4260) -- cycle(188.2590,223.2660) -- (188.5260,223.2660) -- (188.5260,223.4260) -- (188.2590,223.4260) -- cycle(189.0590,223.2660) -- (189.3260,223.2660) -- (189.3260,223.4260) -- (189.0590,223.4260) -- cycle(189.8590,223.2660) -- (190.1260,223.2660) -- (190.1260,223.4260) -- (189.8590,223.4260) -- cycle(190.6590,223.2660) -- (190.9260,223.2660) -- (190.9260,223.4260) -- (190.6590,223.4260) -- cycle(191.4590,223.2660) -- (191.7260,223.2660) -- (191.7260,223.4260) -- (191.4590,223.4260) -- cycle(192.2590,223.2660) -- (192.5260,223.2660) -- (192.5260,223.4260) -- (192.2590,223.4260) -- cycle(193.0590,223.2660) -- (193.3260,223.2660) -- (193.3260,223.4260) -- (193.0590,223.4260) -- cycle(193.8590,223.2660) -- (194.1260,223.2660) -- (194.1260,223.4260) -- (193.8590,223.4260) -- cycle(194.6590,223.2660) -- (194.9260,223.2660) -- (194.9260,223.4260) -- (194.6590,223.4260) -- cycle(195.4590,223.2660) -- (195.7260,223.2660) -- (195.7260,223.4260) -- (195.4590,223.4260) -- cycle(196.2590,223.2660) -- (196.5260,223.2660) -- (196.5260,223.4260) -- (196.2590,223.4260) -- cycle(197.0590,223.2660) -- (197.3260,223.2660) -- (197.3260,223.4260) -- (197.0590,223.4260) -- cycle(197.8590,223.2660) -- (198.1260,223.2660) -- (198.1260,223.4260) -- (197.8590,223.4260) -- cycle(198.6590,223.2660) -- (198.9260,223.2660) -- (198.9260,223.4260) -- (198.6590,223.4260) -- cycle(199.4590,223.2660) -- (199.7260,223.2660) -- (199.7260,223.4260) -- (199.4590,223.4260) -- cycle(200.2590,223.2660) -- (200.5260,223.2660) -- (200.5260,223.4260) -- (200.2590,223.4260) -- cycle(201.0590,223.2660) -- (201.3260,223.2660) -- (201.3260,223.4260) -- (201.0590,223.4260) -- cycle(201.8590,223.2660) -- (202.1260,223.2660) -- (202.1260,223.4260) -- (201.8590,223.4260) -- cycle(202.6590,223.2660) -- (202.9260,223.2660) -- (202.9260,223.4260) -- (202.6590,223.4260) -- cycle(203.4590,223.2660) -- (203.7260,223.2660) -- (203.7260,223.4260) -- (203.4590,223.4260) -- cycle(204.2590,223.2660) -- (204.5260,223.2660) -- (204.5260,223.4260) -- (204.2590,223.4260) -- cycle(205.0590,223.2660) -- (205.3260,223.2660) -- (205.3260,223.4260) -- (205.0590,223.4260) -- cycle(205.8590,223.2660) -- (206.1260,223.2660) -- (206.1260,223.4260) -- (205.8590,223.4260) -- cycle(206.6590,223.2660) -- (206.9260,223.2660) -- (206.9260,223.4260) -- (206.6590,223.4260) -- cycle(207.4590,223.2660) -- (207.7260,223.2660) -- (207.7260,223.4260) -- (207.4590,223.4260) -- cycle(208.2590,223.2660) -- (208.5260,223.2660) -- (208.5260,223.4260) -- (208.2590,223.4260) -- cycle(209.0590,223.2660) -- (209.3260,223.2660) -- (209.3260,223.4260) -- (209.0590,223.4260) -- cycle(209.8590,223.2660) -- (210.1260,223.2660) -- (210.1260,223.4260) -- (209.8590,223.4260) -- cycle(210.6590,223.2660) -- (210.9260,223.2660) -- (210.9260,223.4260) -- (210.6590,223.4260) -- cycle(211.4590,223.2660) -- (211.7260,223.2660) -- (211.7260,223.4260) -- (211.4590,223.4260) -- cycle(212.2590,223.2660) -- (212.5260,223.2660) -- (212.5260,223.4260) -- (212.2590,223.4260) -- cycle(213.0590,223.2660) -- (213.3260,223.2660) -- (213.3260,223.4260) -- (213.0590,223.4260) -- cycle(213.8590,223.2660) -- (214.1260,223.2660) -- (214.1260,223.4260) -- (213.8590,223.4260) -- cycle(214.6590,223.2660) -- (214.9260,223.2660) -- (214.9260,223.4260) -- (214.6590,223.4260) -- cycle(215.4590,223.2660) -- (215.7260,223.2660) -- (215.7260,223.4260) -- (215.4590,223.4260) -- cycle(216.2590,223.2660) -- (216.5260,223.2660) -- (216.5260,223.4260) -- (216.2590,223.4260) -- cycle(217.0590,223.2660) -- (217.3260,223.2660) -- (217.3260,223.4260) -- (217.0590,223.4260) -- cycle(217.8590,223.2660) -- (218.1260,223.2660) -- (218.1260,223.4260) -- (217.8590,223.4260) -- cycle(218.6590,223.2660) -- (218.9260,223.2660) -- (218.9260,223.4260) -- (218.6590,223.4260) -- cycle(219.4590,223.2660) -- (219.7260,223.2660) -- (219.7260,223.4260) -- (219.4590,223.4260) -- cycle(220.2590,223.2660) -- (220.5260,223.2660) -- (220.5260,223.4260) -- (220.2590,223.4260) -- cycle(221.0590,223.2660) -- (221.3260,223.2660) -- (221.3260,223.4260) -- (221.0590,223.4260) -- cycle(221.8590,223.2660) -- (222.1260,223.2660) -- (222.1260,223.4260) -- (221.8590,223.4260) -- cycle(222.6590,223.2660) -- (222.9260,223.2660) -- (222.9260,223.4260) -- (222.6590,223.4260) -- cycle(223.4590,223.2660) -- (223.7260,223.2660) -- (223.7260,223.4260) -- (223.4590,223.4260) -- cycle(224.2590,223.2660) -- (224.5260,223.2660) -- (224.5260,223.4260) -- (224.2590,223.4260) -- cycle(225.0590,223.2660) -- (225.3260,223.2660) -- (225.3260,223.4260) -- (225.0590,223.4260) -- cycle(225.8590,223.2660) -- (226.1260,223.2660) -- (226.1260,223.4260) -- (225.8590,223.4260) -- cycle(226.6590,223.2660) -- (226.9260,223.2660) -- (226.9260,223.4260) -- (226.6590,223.4260) -- cycle(227.4590,223.2660) -- (227.7260,223.2660) -- (227.7260,223.4260) -- (227.4590,223.4260) -- cycle(228.2590,223.2660) -- (228.5260,223.2660) -- (228.5260,223.4260) -- (228.2590,223.4260) -- cycle(229.0590,223.2660) -- (229.3260,223.2660) -- (229.3260,223.4260) -- (229.0590,223.4260) -- cycle(229.8590,223.2660) -- (230.1260,223.2660) -- (230.1260,223.4260) -- (229.8590,223.4260) -- cycle(230.6590,223.2660) -- (230.9260,223.2660) -- (230.9260,223.4260) -- (230.6590,223.4260) -- cycle(231.4590,223.2660) -- (231.7260,223.2660) -- (231.7260,223.4260) -- (231.4590,223.4260) -- cycle(232.2590,223.2660) -- (232.5260,223.2660) -- (232.5260,223.4260) -- (232.2590,223.4260) -- cycle(233.0590,223.2660) -- (233.3260,223.2660) -- (233.3260,223.4260) -- (233.0590,223.4260) -- cycle(233.8590,223.2660) -- (234.1260,223.2660) -- (234.1260,223.4260) -- (233.8590,223.4260) -- cycle(234.6590,223.2660) -- (234.8680,223.2660) -- (234.8680,223.4260) -- (234.6590,223.4260) -- cycle(29.0598,223.2660) -- (29.3264,223.2660) -- (29.3264,223.4260) -- (29.0598,223.4260) -- cycle;



  \path[fill=c7e7e7e,line join=round,line width=0.256pt] (21.4868,246.2790) -- (21.7534,246.2790) -- (21.7534,246.4390) -- (21.4868,246.4390) -- cycle(22.2868,246.2790) -- (22.5534,246.2790) -- (22.5534,246.4390) -- (22.2868,246.4390) -- cycle(23.0868,246.2790) -- (23.3534,246.2790) -- (23.3534,246.4390) -- (23.0868,246.4390) -- cycle(23.8868,246.2790) -- (24.1534,246.2790) -- (24.1534,246.4390) -- (23.8868,246.4390) -- cycle(24.6868,246.2790) -- (24.9534,246.2790) -- (24.9534,246.4390) -- (24.6868,246.4390) -- cycle(25.4868,246.2790) -- (25.7534,246.2790) -- (25.7534,246.4390) -- (25.4868,246.4390) -- cycle(26.2867,246.2790) -- (26.5534,246.2790) -- (26.5534,246.4390) -- (26.2867,246.4390) -- cycle(27.0868,246.2790) -- (27.3534,246.2790) -- (27.3534,246.4390) -- (27.0868,246.4390) -- cycle(27.8867,246.2790) -- (28.1534,246.2790) -- (28.1534,246.4390) -- (27.8867,246.4390) -- cycle(28.6868,246.2790) -- (28.9534,246.2790) -- (28.9534,246.4390) -- (28.6868,246.4390) -- cycle(29.4868,246.2790) -- (29.7534,246.2790) -- (29.7534,246.4390) -- (29.4868,246.4390) -- cycle(30.2867,246.2790) -- (30.5534,246.2790) -- (30.5534,246.4390) -- (30.2867,246.4390) -- cycle(31.0867,246.2790) -- (31.3534,246.2790) -- (31.3534,246.4390) -- (31.0867,246.4390) -- cycle(31.8867,246.2790) -- (32.1534,246.2790) -- (32.1534,246.4390) -- (31.8867,246.4390) -- cycle(32.6867,246.2790) -- (32.9534,246.2790) -- (32.9534,246.4390) -- (32.6867,246.4390) -- cycle(33.4868,246.2790) -- (33.7534,246.2790) -- (33.7534,246.4390) -- (33.4868,246.4390) -- cycle(34.2867,246.2790) -- (34.5534,246.2790) -- (34.5534,246.4390) -- (34.2867,246.4390) -- cycle(35.0867,246.2790) -- (35.3534,246.2790) -- (35.3534,246.4390) -- (35.0867,246.4390) -- cycle(35.8867,246.2790) -- (36.1534,246.2790) -- (36.1534,246.4390) -- (35.8867,246.4390) -- cycle(36.6867,246.2790) -- (36.9534,246.2790) -- (36.9534,246.4390) -- (36.6867,246.4390) -- cycle(37.4867,246.2790) -- (37.7534,246.2790) -- (37.7534,246.4390) -- (37.4867,246.4390) -- cycle(38.2867,246.2790) -- (38.5534,246.2790) -- (38.5534,246.4390) -- (38.2867,246.4390) -- cycle(39.0867,246.2790) -- (39.3534,246.2790) -- (39.3534,246.4390) -- (39.0867,246.4390) -- cycle(39.8867,246.2790) -- (40.1534,246.2790) -- (40.1534,246.4390) -- (39.8867,246.4390) -- cycle(40.6867,246.2790) -- (40.9534,246.2790) -- (40.9534,246.4390) -- (40.6867,246.4390) -- cycle(41.4867,246.2790) -- (41.7534,246.2790) -- (41.7534,246.4390) -- (41.4867,246.4390) -- cycle(42.2867,246.2790) -- (42.5534,246.2790) -- (42.5534,246.4390) -- (42.2867,246.4390) -- cycle(43.0867,246.2790) -- (43.3534,246.2790) -- (43.3534,246.4390) -- (43.0867,246.4390) -- cycle(43.8867,246.2790) -- (44.1534,246.2790) -- (44.1534,246.4390) -- (43.8867,246.4390) -- cycle(44.6867,246.2790) -- (44.9534,246.2790) -- (44.9534,246.4390) -- (44.6867,246.4390) -- cycle(45.4867,246.2790) -- (45.7534,246.2790) -- (45.7534,246.4390) -- (45.4867,246.4390) -- cycle(46.2867,246.2790) -- (46.5534,246.2790) -- (46.5534,246.4390) -- (46.2867,246.4390) -- cycle(47.0867,246.2790) -- (47.3534,246.2790) -- (47.3534,246.4390) -- (47.0867,246.4390) -- cycle(47.8867,246.2790) -- (48.1534,246.2790) -- (48.1534,246.4390) -- (47.8867,246.4390) -- cycle(48.6867,246.2790) -- (48.9534,246.2790) -- (48.9534,246.4390) -- (48.6867,246.4390) -- cycle(49.4867,246.2790) -- (49.7534,246.2790) -- (49.7534,246.4390) -- (49.4867,246.4390) -- cycle(50.2867,246.2790) -- (50.5533,246.2790) -- (50.5533,246.4390) -- (50.2867,246.4390) -- cycle(51.0867,246.2790) -- (51.3534,246.2790) -- (51.3534,246.4390) -- (51.0867,246.4390) -- cycle(51.8867,246.2790) -- (52.1534,246.2790) -- (52.1534,246.4390) -- (51.8867,246.4390) -- cycle(52.6867,246.2790) -- (52.9534,246.2790) -- (52.9534,246.4390) -- (52.6867,246.4390) -- cycle(53.4867,246.2790) -- (53.7534,246.2790) -- (53.7534,246.4390) -- (53.4867,246.4390) -- cycle(54.2867,246.2790) -- (54.5533,246.2790) -- (54.5533,246.4390) -- (54.2867,246.4390) -- cycle(55.0867,246.2790) -- (55.3533,246.2790) -- (55.3533,246.4390) -- (55.0867,246.4390) -- cycle(55.8867,246.2790) -- (56.1534,246.2790) -- (56.1534,246.4390) -- (55.8867,246.4390) -- cycle(56.6867,246.2790) -- (56.9533,246.2790) -- (56.9533,246.4390) -- (56.6867,246.4390) -- cycle(57.4867,246.2790) -- (57.7534,246.2790) -- (57.7534,246.4390) -- (57.4867,246.4390) -- cycle(58.2867,246.2790) -- (58.5533,246.2790) -- (58.5533,246.4390) -- (58.2867,246.4390) -- cycle(59.0867,246.2790) -- (59.3533,246.2790) -- (59.3533,246.4390) -- (59.0867,246.4390) -- cycle(59.8867,246.2790) -- (60.1533,246.2790) -- (60.1533,246.4390) -- (59.8867,246.4390) -- cycle(60.6867,246.2790) -- (60.9533,246.2790) -- (60.9533,246.4390) -- (60.6867,246.4390) -- cycle(61.4867,246.2790) -- (61.7533,246.2790) -- (61.7533,246.4390) -- (61.4867,246.4390) -- cycle(62.2867,246.2790) -- (62.5533,246.2790) -- (62.5533,246.4390) -- (62.2867,246.4390) -- cycle(63.0867,246.2790) -- (63.3533,246.2790) -- (63.3533,246.4390) -- (63.0867,246.4390) -- cycle(63.8867,246.2790) -- (64.1533,246.2790) -- (64.1533,246.4390) -- (63.8867,246.4390) -- cycle(64.6866,246.2790) -- (64.9533,246.2790) -- (64.9533,246.4390) -- (64.6866,246.4390) -- cycle(65.4867,246.2790) -- (65.7533,246.2790) -- (65.7533,246.4390) -- (65.4867,246.4390) -- cycle(66.2867,246.2790) -- (66.5533,246.2790) -- (66.5533,246.4390) -- (66.2867,246.4390) -- cycle(67.0866,246.2790) -- (67.3533,246.2790) -- (67.3533,246.4390) -- (67.0866,246.4390) -- cycle(67.8867,246.2790) -- (68.1533,246.2790) -- (68.1533,246.4390) -- (67.8867,246.4390) -- cycle(68.6866,246.2790) -- (68.9533,246.2790) -- (68.9533,246.4390) -- (68.6866,246.4390) -- cycle(69.4867,246.2790) -- (69.7533,246.2790) -- (69.7533,246.4390) -- (69.4867,246.4390) -- cycle(70.2866,246.2790) -- (70.5533,246.2790) -- (70.5533,246.4390) -- (70.2866,246.4390) -- cycle(71.0866,246.2790) -- (71.3533,246.2790) -- (71.3533,246.4390) -- (71.0866,246.4390) -- cycle(71.8867,246.2790) -- (72.1533,246.2790) -- (72.1533,246.4390) -- (71.8867,246.4390) -- cycle(72.6866,246.2790) -- (72.9533,246.2790) -- (72.9533,246.4390) -- (72.6866,246.4390) -- cycle(73.4866,246.2790) -- (73.7533,246.2790) -- (73.7533,246.4390) -- (73.4866,246.4390) -- cycle(74.2866,246.2790) -- (74.5533,246.2790) -- (74.5533,246.4390) -- (74.2866,246.4390) -- cycle(75.0866,246.2790) -- (75.3533,246.2790) -- (75.3533,246.4390) -- (75.0866,246.4390) -- cycle(75.8867,246.2790) -- (76.1533,246.2790) -- (76.1533,246.4390) -- (75.8867,246.4390) -- cycle(76.6866,246.2790) -- (76.9533,246.2790) -- (76.9533,246.4390) -- (76.6866,246.4390) -- cycle(77.4866,246.2790) -- (77.7533,246.2790) -- (77.7533,246.4390) -- (77.4866,246.4390) -- cycle(78.2866,246.2790) -- (78.5533,246.2790) -- (78.5533,246.4390) -- (78.2866,246.4390) -- cycle(79.0866,246.2790) -- (79.3533,246.2790) -- (79.3533,246.4390) -- (79.0866,246.4390) -- cycle(79.8866,246.2790) -- (80.1533,246.2790) -- (80.1533,246.4390) -- (79.8866,246.4390) -- cycle(80.6866,246.2790) -- (80.9533,246.2790) -- (80.9533,246.4390) -- (80.6866,246.4390) -- cycle(81.4866,246.2790) -- (81.7533,246.2790) -- (81.7533,246.4390) -- (81.4866,246.4390) -- cycle(82.2866,246.2790) -- (82.5533,246.2790) -- (82.5533,246.4390) -- (82.2866,246.4390) -- cycle(83.0866,246.2790) -- (83.3533,246.2790) -- (83.3533,246.4390) -- (83.0866,246.4390) -- cycle(83.8866,246.2790) -- (84.1533,246.2790) -- (84.1533,246.4390) -- (83.8866,246.4390) -- cycle(84.6866,246.2790) -- (84.9532,246.2790) -- (84.9532,246.4390) -- (84.6866,246.4390) -- cycle(85.4866,246.2790) -- (85.7533,246.2790) -- (85.7533,246.4390) -- (85.4866,246.4390) -- cycle(86.2866,246.2790) -- (86.5533,246.2790) -- (86.5533,246.4390) -- (86.2866,246.4390) -- cycle(87.0866,246.2790) -- (87.3533,246.2790) -- (87.3533,246.4390) -- (87.0866,246.4390) -- cycle(87.8866,246.2790) -- (88.1533,246.2790) -- (88.1533,246.4390) -- (87.8866,246.4390) -- cycle(88.6866,246.2790) -- (88.9532,246.2790) -- (88.9532,246.4390) -- (88.6866,246.4390) -- cycle(89.4866,246.2790) -- (89.7533,246.2790) -- (89.7533,246.4390) -- (89.4866,246.4390) -- cycle(90.2866,246.2790) -- (90.5533,246.2790) -- (90.5533,246.4390) -- (90.2866,246.4390) -- cycle(91.0866,246.2790) -- (91.3532,246.2790) -- (91.3532,246.4390) -- (91.0866,246.4390) -- cycle(91.8866,246.2790) -- (92.1533,246.2790) -- (92.1533,246.4390) -- (91.8866,246.4390) -- cycle(92.6866,246.2790) -- (92.9532,246.2790) -- (92.9532,246.4390) -- (92.6866,246.4390) -- cycle(93.4866,246.2790) -- (93.7533,246.2790) -- (93.7533,246.4390) -- (93.4866,246.4390) -- cycle(94.2866,246.2790) -- (94.5532,246.2790) -- (94.5532,246.4390) -- (94.2866,246.4390) -- cycle(95.0866,246.2790) -- (95.3532,246.2790) -- (95.3532,246.4390) -- (95.0866,246.4390) -- cycle(95.8866,246.2790) -- (96.1533,246.2790) -- (96.1533,246.4390) -- (95.8866,246.4390) -- cycle(96.6866,246.2790) -- (96.9532,246.2790) -- (96.9532,246.4390) -- (96.6866,246.4390) -- cycle(97.4866,246.2790) -- (97.7532,246.2790) -- (97.7532,246.4390) -- (97.4866,246.4390) -- cycle(98.2866,246.2790) -- (98.5532,246.2790) -- (98.5532,246.4390) -- (98.2866,246.4390) -- cycle(99.0865,246.2790) -- (99.3532,246.2790) -- (99.3532,246.4390) -- (99.0865,246.4390) -- cycle(99.8866,246.2790) -- (100.1530,246.2790) -- (100.1530,246.4390) -- (99.8866,246.4390) -- cycle(100.6870,246.2790) -- (100.9530,246.2790) -- (100.9530,246.4390) -- (100.6870,246.4390) -- cycle(101.4870,246.2790) -- (101.7530,246.2790) -- (101.7530,246.4390) -- (101.4870,246.4390) -- cycle(102.2870,246.2790) -- (102.5530,246.2790) -- (102.5530,246.4390) -- (102.2870,246.4390) -- cycle(103.0870,246.2790) -- (103.3530,246.2790) -- (103.3530,246.4390) -- (103.0870,246.4390) -- cycle(103.8870,246.2790) -- (104.1530,246.2790) -- (104.1530,246.4390) -- (103.8870,246.4390) -- cycle(104.6870,246.2790) -- (104.9530,246.2790) -- (104.9530,246.4390) -- (104.6870,246.4390) -- cycle(105.4870,246.2790) -- (105.7530,246.2790) -- (105.7530,246.4390) -- (105.4870,246.4390) -- cycle(106.2870,246.2790) -- (106.5530,246.2790) -- (106.5530,246.4390) -- (106.2870,246.4390) -- cycle(107.0870,246.2790) -- (107.3530,246.2790) -- (107.3530,246.4390) -- (107.0870,246.4390) -- cycle(107.8870,246.2790) -- (108.1530,246.2790) -- (108.1530,246.4390) -- (107.8870,246.4390) -- cycle(108.6870,246.2790) -- (108.9530,246.2790) -- (108.9530,246.4390) -- (108.6870,246.4390) -- cycle(109.4870,246.2790) -- (109.7530,246.2790) -- (109.7530,246.4390) -- (109.4870,246.4390) -- cycle(110.2870,246.2790) -- (110.5530,246.2790) -- (110.5530,246.4390) -- (110.2870,246.4390) -- cycle(111.0870,246.2790) -- (111.3530,246.2790) -- (111.3530,246.4390) -- (111.0870,246.4390) -- cycle(111.8870,246.2790) -- (112.1530,246.2790) -- (112.1530,246.4390) -- (111.8870,246.4390) -- cycle(112.6870,246.2790) -- (112.9530,246.2790) -- (112.9530,246.4390) -- (112.6870,246.4390) -- cycle(113.4870,246.2790) -- (113.7530,246.2790) -- (113.7530,246.4390) -- (113.4870,246.4390) -- cycle(114.2870,246.2790) -- (114.5530,246.2790) -- (114.5530,246.4390) -- (114.2870,246.4390) -- cycle(115.0870,246.2790) -- (115.3530,246.2790) -- (115.3530,246.4390) -- (115.0870,246.4390) -- cycle(115.8870,246.2790) -- (116.1530,246.2790) -- (116.1530,246.4390) -- (115.8870,246.4390) -- cycle(116.6870,246.2790) -- (116.9530,246.2790) -- (116.9530,246.4390) -- (116.6870,246.4390) -- cycle(117.4870,246.2790) -- (117.7530,246.2790) -- (117.7530,246.4390) -- (117.4870,246.4390) -- cycle(118.2860,246.2790) -- (118.5530,246.2790) -- (118.5530,246.4390) -- (118.2860,246.4390) -- cycle(119.0870,246.2790) -- (119.3530,246.2790) -- (119.3530,246.4390) -- (119.0870,246.4390) -- cycle(119.8870,246.2790) -- (120.1530,246.2790) -- (120.1530,246.4390) -- (119.8870,246.4390) -- cycle(120.6870,246.2790) -- (120.9530,246.2790) -- (120.9530,246.4390) -- (120.6870,246.4390) -- cycle(121.4870,246.2790) -- (121.7530,246.2790) -- (121.7530,246.4390) -- (121.4870,246.4390) -- cycle(122.2860,246.2790) -- (122.5530,246.2790) -- (122.5530,246.4390) -- (122.2860,246.4390) -- cycle(123.0870,246.2790) -- (123.3530,246.2790) -- (123.3530,246.4390) -- (123.0870,246.4390) -- cycle(123.8870,246.2790) -- (124.1530,246.2790) -- (124.1530,246.4390) -- (123.8870,246.4390) -- cycle(124.6860,246.2790) -- (124.9530,246.2790) -- (124.9530,246.4390) -- (124.6860,246.4390) -- cycle(125.4870,246.2790) -- (125.7530,246.2790) -- (125.7530,246.4390) -- (125.4870,246.4390) -- cycle(126.2860,246.2790) -- (126.5530,246.2790) -- (126.5530,246.4390) -- (126.2860,246.4390) -- cycle(127.0870,246.2790) -- (127.3530,246.2790) -- (127.3530,246.4390) -- (127.0870,246.4390) -- cycle(127.8860,246.2790) -- (128.1530,246.2790) -- (128.1530,246.4390) -- (127.8860,246.4390) -- cycle(128.6860,246.2790) -- (128.9530,246.2790) -- (128.9530,246.4390) -- (128.6860,246.4390) -- cycle(129.4870,246.2790) -- (129.7530,246.2790) -- (129.7530,246.4390) -- (129.4870,246.4390) -- cycle(130.2860,246.2790) -- (130.5530,246.2790) -- (130.5530,246.4390) -- (130.2860,246.4390) -- cycle(131.0860,246.2790) -- (131.3530,246.2790) -- (131.3530,246.4390) -- (131.0860,246.4390) -- cycle(131.8860,246.2790) -- (132.1530,246.2790) -- (132.1530,246.4390) -- (131.8860,246.4390) -- cycle(132.6860,246.2790) -- (132.9530,246.2790) -- (132.9530,246.4390) -- (132.6860,246.4390) -- cycle(133.4870,246.2790) -- (133.7530,246.2790) -- (133.7530,246.4390) -- (133.4870,246.4390) -- cycle(134.2860,246.2790) -- (134.5530,246.2790) -- (134.5530,246.4390) -- (134.2860,246.4390) -- cycle(135.0860,246.2790) -- (135.3530,246.2790) -- (135.3530,246.4390) -- (135.0860,246.4390) -- cycle(135.8860,246.2790) -- (136.1530,246.2790) -- (136.1530,246.4390) -- (135.8860,246.4390) -- cycle(136.6860,246.2790) -- (136.9530,246.2790) -- (136.9530,246.4390) -- (136.6860,246.4390) -- cycle(137.4860,246.2790) -- (137.7530,246.2790) -- (137.7530,246.4390) -- (137.4860,246.4390) -- cycle(138.2860,246.2790) -- (138.5530,246.2790) -- (138.5530,246.4390) -- (138.2860,246.4390) -- cycle(139.0860,246.2790) -- (139.3530,246.2790) -- (139.3530,246.4390) -- (139.0860,246.4390) -- cycle(139.8860,246.2790) -- (140.1530,246.2790) -- (140.1530,246.4390) -- (139.8860,246.4390) -- cycle(140.6860,246.2790) -- (140.9530,246.2790) -- (140.9530,246.4390) -- (140.6860,246.4390) -- cycle(141.4860,246.2790) -- (141.7530,246.2790) -- (141.7530,246.4390) -- (141.4860,246.4390) -- cycle(142.2860,246.2790) -- (142.5530,246.2790) -- (142.5530,246.4390) -- (142.2860,246.4390) -- cycle(143.0860,246.2790) -- (143.3530,246.2790) -- (143.3530,246.4390) -- (143.0860,246.4390) -- cycle(143.8860,246.2790) -- (144.1530,246.2790) -- (144.1530,246.4390) -- (143.8860,246.4390) -- cycle(144.6860,246.2790) -- (144.9530,246.2790) -- (144.9530,246.4390) -- (144.6860,246.4390) -- cycle(145.4860,246.2790) -- (145.7530,246.2790) -- (145.7530,246.4390) -- (145.4860,246.4390) -- cycle(146.2860,246.2790) -- (146.5530,246.2790) -- (146.5530,246.4390) -- (146.2860,246.4390) -- cycle(147.0860,246.2790) -- (147.3530,246.2790) -- (147.3530,246.4390) -- (147.0860,246.4390) -- cycle(147.8860,246.2790) -- (148.1530,246.2790) -- (148.1530,246.4390) -- (147.8860,246.4390) -- cycle(148.6860,246.2790) -- (148.9530,246.2790) -- (148.9530,246.4390) -- (148.6860,246.4390) -- cycle(149.4860,246.2790) -- (149.7530,246.2790) -- (149.7530,246.4390) -- (149.4860,246.4390) -- cycle(150.2860,246.2790) -- (150.5530,246.2790) -- (150.5530,246.4390) -- (150.2860,246.4390) -- cycle(151.0860,246.2790) -- (151.3530,246.2790) -- (151.3530,246.4390) -- (151.0860,246.4390) -- cycle(151.8860,246.2790) -- (152.1530,246.2790) -- (152.1530,246.4390) -- (151.8860,246.4390) -- cycle(152.6860,246.2790) -- (152.9530,246.2790) -- (152.9530,246.4390) -- (152.6860,246.4390) -- cycle(153.4860,246.2790) -- (153.7530,246.2790) -- (153.7530,246.4390) -- (153.4860,246.4390) -- cycle(154.2860,246.2790) -- (154.5530,246.2790) -- (154.5530,246.4390) -- (154.2860,246.4390) -- cycle(155.0860,246.2790) -- (155.3530,246.2790) -- (155.3530,246.4390) -- (155.0860,246.4390) -- cycle(155.8860,246.2790) -- (156.1530,246.2790) -- (156.1530,246.4390) -- (155.8860,246.4390) -- cycle(156.6860,246.2790) -- (156.9530,246.2790) -- (156.9530,246.4390) -- (156.6860,246.4390) -- cycle(157.4860,246.2790) -- (157.7530,246.2790) -- (157.7530,246.4390) -- (157.4860,246.4390) -- cycle(158.2860,246.2790) -- (158.5530,246.2790) -- (158.5530,246.4390) -- (158.2860,246.4390) -- cycle(159.0860,246.2790) -- (159.3530,246.2790) -- (159.3530,246.4390) -- (159.0860,246.4390) -- cycle(159.8860,246.2790) -- (160.1530,246.2790) -- (160.1530,246.4390) -- (159.8860,246.4390) -- cycle(160.6860,246.2790) -- (160.9530,246.2790) -- (160.9530,246.4390) -- (160.6860,246.4390) -- cycle(161.4860,246.2790) -- (161.7530,246.2790) -- (161.7530,246.4390) -- (161.4860,246.4390) -- cycle(162.2860,246.2790) -- (162.5530,246.2790) -- (162.5530,246.4390) -- (162.2860,246.4390) -- cycle(163.0860,246.2790) -- (163.3530,246.2790) -- (163.3530,246.4390) -- (163.0860,246.4390) -- cycle(163.8860,246.2790) -- (164.1530,246.2790) -- (164.1530,246.4390) -- (163.8860,246.4390) -- cycle(164.6860,246.2790) -- (164.9530,246.2790) -- (164.9530,246.4390) -- (164.6860,246.4390) -- cycle(165.4860,246.2790) -- (165.7530,246.2790) -- (165.7530,246.4390) -- (165.4860,246.4390) -- cycle(166.2860,246.2790) -- (166.5530,246.2790) -- (166.5530,246.4390) -- (166.2860,246.4390) -- cycle(167.0860,246.2790) -- (167.3530,246.2790) -- (167.3530,246.4390) -- (167.0860,246.4390) -- cycle(167.8860,246.2790) -- (168.1530,246.2790) -- (168.1530,246.4390) -- (167.8860,246.4390) -- cycle(168.6860,246.2790) -- (168.9530,246.2790) -- (168.9530,246.4390) -- (168.6860,246.4390) -- cycle(169.4860,246.2790) -- (169.7530,246.2790) -- (169.7530,246.4390) -- (169.4860,246.4390) -- cycle(170.2860,246.2790) -- (170.5530,246.2790) -- (170.5530,246.4390) -- (170.2860,246.4390) -- cycle(171.0860,246.2790) -- (171.3530,246.2790) -- (171.3530,246.4390) -- (171.0860,246.4390) -- cycle(171.8860,246.2790) -- (172.1530,246.2790) -- (172.1530,246.4390) -- (171.8860,246.4390) -- cycle(172.6860,246.2790) -- (172.9530,246.2790) -- (172.9530,246.4390) -- (172.6860,246.4390) -- cycle(173.4860,246.2790) -- (173.7530,246.2790) -- (173.7530,246.4390) -- (173.4860,246.4390) -- cycle(174.2860,246.2790) -- (174.5530,246.2790) -- (174.5530,246.4390) -- (174.2860,246.4390) -- cycle(175.0860,246.2790) -- (175.3530,246.2790) -- (175.3530,246.4390) -- (175.0860,246.4390) -- cycle(175.8860,246.2790) -- (176.1530,246.2790) -- (176.1530,246.4390) -- (175.8860,246.4390) -- cycle(176.6860,246.2790) -- (176.9530,246.2790) -- (176.9530,246.4390) -- (176.6860,246.4390) -- cycle(177.4860,246.2790) -- (177.7530,246.2790) -- (177.7530,246.4390) -- (177.4860,246.4390) -- cycle(178.2860,246.2790) -- (178.5530,246.2790) -- (178.5530,246.4390) -- (178.2860,246.4390) -- cycle(179.0860,246.2790) -- (179.3530,246.2790) -- (179.3530,246.4390) -- (179.0860,246.4390) -- cycle(179.8860,246.2790) -- (180.1530,246.2790) -- (180.1530,246.4390) -- (179.8860,246.4390) -- cycle(180.6860,246.2790) -- (180.9530,246.2790) -- (180.9530,246.4390) -- (180.6860,246.4390) -- cycle(181.4860,246.2790) -- (181.7530,246.2790) -- (181.7530,246.4390) -- (181.4860,246.4390) -- cycle(182.2860,246.2790) -- (182.5530,246.2790) -- (182.5530,246.4390) -- (182.2860,246.4390) -- cycle(183.0860,246.2790) -- (183.3530,246.2790) -- (183.3530,246.4390) -- (183.0860,246.4390) -- cycle(183.8860,246.2790) -- (184.1530,246.2790) -- (184.1530,246.4390) -- (183.8860,246.4390) -- cycle(184.6860,246.2790) -- (184.9530,246.2790) -- (184.9530,246.4390) -- (184.6860,246.4390) -- cycle(185.4860,246.2790) -- (185.7530,246.2790) -- (185.7530,246.4390) -- (185.4860,246.4390) -- cycle(186.2860,246.2790) -- (186.5530,246.2790) -- (186.5530,246.4390) -- (186.2860,246.4390) -- cycle(187.0860,246.2790) -- (187.3530,246.2790) -- (187.3530,246.4390) -- (187.0860,246.4390) -- cycle(187.8860,246.2790) -- (188.1530,246.2790) -- (188.1530,246.4390) -- (187.8860,246.4390) -- cycle(188.6860,246.2790) -- (188.9530,246.2790) -- (188.9530,246.4390) -- (188.6860,246.4390) -- cycle(189.4860,246.2790) -- (189.7530,246.2790) -- (189.7530,246.4390) -- (189.4860,246.4390) -- cycle(190.2860,246.2790) -- (190.5530,246.2790) -- (190.5530,246.4390) -- (190.2860,246.4390) -- cycle(191.0860,246.2790) -- (191.3530,246.2790) -- (191.3530,246.4390) -- (191.0860,246.4390) -- cycle(191.8860,246.2790) -- (192.1530,246.2790) -- (192.1530,246.4390) -- (191.8860,246.4390) -- cycle(192.6860,246.2790) -- (192.9530,246.2790) -- (192.9530,246.4390) -- (192.6860,246.4390) -- cycle(193.4860,246.2790) -- (193.7530,246.2790) -- (193.7530,246.4390) -- (193.4860,246.4390) -- cycle(194.2860,246.2790) -- (194.5530,246.2790) -- (194.5530,246.4390) -- (194.2860,246.4390) -- cycle(195.0860,246.2790) -- (195.3530,246.2790) -- (195.3530,246.4390) -- (195.0860,246.4390) -- cycle(195.8860,246.2790) -- (196.1530,246.2790) -- (196.1530,246.4390) -- (195.8860,246.4390) -- cycle(196.6860,246.2790) -- (196.9530,246.2790) -- (196.9530,246.4390) -- (196.6860,246.4390) -- cycle(197.4860,246.2790) -- (197.7530,246.2790) -- (197.7530,246.4390) -- (197.4860,246.4390) -- cycle(198.2860,246.2790) -- (198.5530,246.2790) -- (198.5530,246.4390) -- (198.2860,246.4390) -- cycle(199.0860,246.2790) -- (199.3530,246.2790) -- (199.3530,246.4390) -- (199.0860,246.4390) -- cycle(199.8860,246.2790) -- (200.1530,246.2790) -- (200.1530,246.4390) -- (199.8860,246.4390) -- cycle(200.6860,246.2790) -- (200.9530,246.2790) -- (200.9530,246.4390) -- (200.6860,246.4390) -- cycle(201.4860,246.2790) -- (201.7530,246.2790) -- (201.7530,246.4390) -- (201.4860,246.4390) -- cycle(202.2860,246.2790) -- (202.5530,246.2790) -- (202.5530,246.4390) -- (202.2860,246.4390) -- cycle(203.0860,246.2790) -- (203.3530,246.2790) -- (203.3530,246.4390) -- (203.0860,246.4390) -- cycle(203.8860,246.2790) -- (204.1530,246.2790) -- (204.1530,246.4390) -- (203.8860,246.4390) -- cycle(204.6860,246.2790) -- (204.9530,246.2790) -- (204.9530,246.4390) -- (204.6860,246.4390) -- cycle(205.4860,246.2790) -- (205.7530,246.2790) -- (205.7530,246.4390) -- (205.4860,246.4390) -- cycle(206.2860,246.2790) -- (206.5530,246.2790) -- (206.5530,246.4390) -- (206.2860,246.4390) -- cycle(207.0860,246.2790) -- (207.3530,246.2790) -- (207.3530,246.4390) -- (207.0860,246.4390) -- cycle(207.8860,246.2790) -- (208.1530,246.2790) -- (208.1530,246.4390) -- (207.8860,246.4390) -- cycle(208.6860,246.2790) -- (208.9530,246.2790) -- (208.9530,246.4390) -- (208.6860,246.4390) -- cycle(209.4860,246.2790) -- (209.7530,246.2790) -- (209.7530,246.4390) -- (209.4860,246.4390) -- cycle(210.2860,246.2790) -- (210.5530,246.2790) -- (210.5530,246.4390) -- (210.2860,246.4390) -- cycle(211.0860,246.2790) -- (211.3530,246.2790) -- (211.3530,246.4390) -- (211.0860,246.4390) -- cycle(211.8860,246.2790) -- (212.1530,246.2790) -- (212.1530,246.4390) -- (211.8860,246.4390) -- cycle(212.6860,246.2790) -- (212.9530,246.2790) -- (212.9530,246.4390) -- (212.6860,246.4390) -- cycle(213.4860,246.2790) -- (213.7530,246.2790) -- (213.7530,246.4390) -- (213.4860,246.4390) -- cycle(214.2860,246.2790) -- (214.5530,246.2790) -- (214.5530,246.4390) -- (214.2860,246.4390) -- cycle(215.0860,246.2790) -- (215.3530,246.2790) -- (215.3530,246.4390) -- (215.0860,246.4390) -- cycle(215.8860,246.2790) -- (216.1530,246.2790) -- (216.1530,246.4390) -- (215.8860,246.4390) -- cycle(216.6860,246.2790) -- (216.9530,246.2790) -- (216.9530,246.4390) -- (216.6860,246.4390) -- cycle(217.4860,246.2790) -- (217.7530,246.2790) -- (217.7530,246.4390) -- (217.4860,246.4390) -- cycle(218.2860,246.2790) -- (218.5530,246.2790) -- (218.5530,246.4390) -- (218.2860,246.4390) -- cycle(219.0860,246.2790) -- (219.3530,246.2790) -- (219.3530,246.4390) -- (219.0860,246.4390) -- cycle(219.8860,246.2790) -- (220.1530,246.2790) -- (220.1530,246.4390) -- (219.8860,246.4390) -- cycle(220.6860,246.2790) -- (220.9530,246.2790) -- (220.9530,246.4390) -- (220.6860,246.4390) -- cycle(221.4860,246.2790) -- (221.7530,246.2790) -- (221.7530,246.4390) -- (221.4860,246.4390) -- cycle(222.2860,246.2790) -- (222.5530,246.2790) -- (222.5530,246.4390) -- (222.2860,246.4390) -- cycle(223.0860,246.2790) -- (223.3530,246.2790) -- (223.3530,246.4390) -- (223.0860,246.4390) -- cycle(223.8860,246.2790) -- (224.1530,246.2790) -- (224.1530,246.4390) -- (223.8860,246.4390) -- cycle(224.6860,246.2790) -- (224.9530,246.2790) -- (224.9530,246.4390) -- (224.6860,246.4390) -- cycle(225.4860,246.2790) -- (225.7530,246.2790) -- (225.7530,246.4390) -- (225.4860,246.4390) -- cycle(226.2860,246.2790) -- (226.4950,246.2790) -- (226.4950,246.4390) -- (226.2860,246.4390) -- cycle(20.6868,246.2790) -- (20.9534,246.2790) -- (20.9534,246.4390) -- (20.6868,246.4390) -- cycle;



  \path[fill=c7e7e7e,line join=round,line width=0.256pt] (13.1108,269.3060) -- (13.3775,269.3060) -- (13.3775,269.4670) -- (13.1108,269.4670) -- cycle(13.9108,269.3060) -- (14.1775,269.3060) -- (14.1775,269.4670) -- (13.9108,269.4670) -- cycle(14.7108,269.3060) -- (14.9775,269.3060) -- (14.9775,269.4670) -- (14.7108,269.4670) -- cycle(15.5108,269.3060) -- (15.7775,269.3060) -- (15.7775,269.4670) -- (15.5108,269.4670) -- cycle(16.3108,269.3060) -- (16.5775,269.3060) -- (16.5775,269.4670) -- (16.3108,269.4670) -- cycle(17.1108,269.3060) -- (17.3775,269.3060) -- (17.3775,269.4670) -- (17.1108,269.4670) -- cycle(17.9108,269.3060) -- (18.1775,269.3060) -- (18.1775,269.4670) -- (17.9108,269.4670) -- cycle(18.7108,269.3060) -- (18.9775,269.3060) -- (18.9775,269.4670) -- (18.7108,269.4670) -- cycle(19.5108,269.3060) -- (19.7775,269.3060) -- (19.7775,269.4670) -- (19.5108,269.4670) -- cycle(20.3108,269.3060) -- (20.5775,269.3060) -- (20.5775,269.4670) -- (20.3108,269.4670) -- cycle(21.1108,269.3060) -- (21.3775,269.3060) -- (21.3775,269.4670) -- (21.1108,269.4670) -- cycle(21.9108,269.3060) -- (22.1775,269.3060) -- (22.1775,269.4670) -- (21.9108,269.4670) -- cycle(22.7108,269.3060) -- (22.9775,269.3060) -- (22.9775,269.4670) -- (22.7108,269.4670) -- cycle(23.5108,269.3060) -- (23.7775,269.3060) -- (23.7775,269.4670) -- (23.5108,269.4670) -- cycle(24.3108,269.3060) -- (24.5775,269.3060) -- (24.5775,269.4670) -- (24.3108,269.4670) -- cycle(25.1108,269.3060) -- (25.3775,269.3060) -- (25.3775,269.4670) -- (25.1108,269.4670) -- cycle(25.9108,269.3060) -- (26.1775,269.3060) -- (26.1775,269.4670) -- (25.9108,269.4670) -- cycle(26.7108,269.3060) -- (26.9775,269.3060) -- (26.9775,269.4670) -- (26.7108,269.4670) -- cycle(27.5108,269.3060) -- (27.7775,269.3060) -- (27.7775,269.4670) -- (27.5108,269.4670) -- cycle(28.3108,269.3060) -- (28.5775,269.3060) -- (28.5775,269.4670) -- (28.3108,269.4670) -- cycle(29.1108,269.3060) -- (29.3774,269.3060) -- (29.3774,269.4670) -- (29.1108,269.4670) -- cycle(29.9108,269.3060) -- (30.1775,269.3060) -- (30.1775,269.4670) -- (29.9108,269.4670) -- cycle(30.7108,269.3060) -- (30.9774,269.3060) -- (30.9774,269.4670) -- (30.7108,269.4670) -- cycle(31.5108,269.3060) -- (31.7775,269.3060) -- (31.7775,269.4670) -- (31.5108,269.4670) -- cycle(32.3108,269.3060) -- (32.5775,269.3060) -- (32.5775,269.4670) -- (32.3108,269.4670) -- cycle(33.1108,269.3060) -- (33.3774,269.3060) -- (33.3774,269.4670) -- (33.1108,269.4670) -- cycle(33.9108,269.3060) -- (34.1774,269.3060) -- (34.1774,269.4670) -- (33.9108,269.4670) -- cycle(34.7108,269.3060) -- (34.9774,269.3060) -- (34.9774,269.4670) -- (34.7108,269.4670) -- cycle(35.5108,269.3060) -- (35.7774,269.3060) -- (35.7774,269.4670) -- (35.5108,269.4670) -- cycle(36.3108,269.3060) -- (36.5775,269.3060) -- (36.5775,269.4670) -- (36.3108,269.4670) -- cycle(37.1108,269.3060) -- (37.3774,269.3060) -- (37.3774,269.4670) -- (37.1108,269.4670) -- cycle(37.9108,269.3060) -- (38.1774,269.3060) -- (38.1774,269.4670) -- (37.9108,269.4670) -- cycle(38.7108,269.3060) -- (38.9774,269.3060) -- (38.9774,269.4670) -- (38.7108,269.4670) -- cycle(39.5108,269.3060) -- (39.7774,269.3060) -- (39.7774,269.4670) -- (39.5108,269.4670) -- cycle(40.3108,269.3060) -- (40.5774,269.3060) -- (40.5774,269.4670) -- (40.3108,269.4670) -- cycle(41.1108,269.3060) -- (41.3774,269.3060) -- (41.3774,269.4670) -- (41.1108,269.4670) -- cycle(41.9108,269.3060) -- (42.1774,269.3060) -- (42.1774,269.4670) -- (41.9108,269.4670) -- cycle(42.7108,269.3060) -- (42.9774,269.3060) -- (42.9774,269.4670) -- (42.7108,269.4670) -- cycle(43.5107,269.3060) -- (43.7774,269.3060) -- (43.7774,269.4670) -- (43.5107,269.4670) -- cycle(44.3108,269.3060) -- (44.5774,269.3060) -- (44.5774,269.4670) -- (44.3108,269.4670) -- cycle(45.1107,269.3060) -- (45.3774,269.3060) -- (45.3774,269.4670) -- (45.1107,269.4670) -- cycle(45.9108,269.3060) -- (46.1774,269.3060) -- (46.1774,269.4670) -- (45.9108,269.4670) -- cycle(46.7108,269.3060) -- (46.9774,269.3060) -- (46.9774,269.4670) -- (46.7108,269.4670) -- cycle(47.5107,269.3060) -- (47.7774,269.3060) -- (47.7774,269.4670) -- (47.5107,269.4670) -- cycle(48.3107,269.3060) -- (48.5774,269.3060) -- (48.5774,269.4670) -- (48.3107,269.4670) -- cycle(49.1107,269.3060) -- (49.3774,269.3060) -- (49.3774,269.4670) -- (49.1107,269.4670) -- cycle(49.9107,269.3060) -- (50.1774,269.3060) -- (50.1774,269.4670) -- (49.9107,269.4670) -- cycle(50.7108,269.3060) -- (50.9774,269.3060) -- (50.9774,269.4670) -- (50.7108,269.4670) -- cycle(51.5107,269.3060) -- (51.7774,269.3060) -- (51.7774,269.4670) -- (51.5107,269.4670) -- cycle(52.3107,269.3060) -- (52.5774,269.3060) -- (52.5774,269.4670) -- (52.3107,269.4670) -- cycle(53.1107,269.3060) -- (53.3774,269.3060) -- (53.3774,269.4670) -- (53.1107,269.4670) -- cycle(53.9107,269.3060) -- (54.1774,269.3060) -- (54.1774,269.4670) -- (53.9107,269.4670) -- cycle(54.7107,269.3060) -- (54.9774,269.3060) -- (54.9774,269.4670) -- (54.7107,269.4670) -- cycle(55.5107,269.3060) -- (55.7774,269.3060) -- (55.7774,269.4670) -- (55.5107,269.4670) -- cycle(56.3107,269.3060) -- (56.5774,269.3060) -- (56.5774,269.4670) -- (56.3107,269.4670) -- cycle(57.1107,269.3060) -- (57.3774,269.3060) -- (57.3774,269.4670) -- (57.1107,269.4670) -- cycle(57.9107,269.3060) -- (58.1774,269.3060) -- (58.1774,269.4670) -- (57.9107,269.4670) -- cycle(58.7107,269.3060) -- (58.9774,269.3060) -- (58.9774,269.4670) -- (58.7107,269.4670) -- cycle(59.5107,269.3060) -- (59.7774,269.3060) -- (59.7774,269.4670) -- (59.5107,269.4670) -- cycle(60.3107,269.3060) -- (60.5774,269.3060) -- (60.5774,269.4670) -- (60.3107,269.4670) -- cycle(61.1107,269.3060) -- (61.3774,269.3060) -- (61.3774,269.4670) -- (61.1107,269.4670) -- cycle(61.9107,269.3060) -- (62.1774,269.3060) -- (62.1774,269.4670) -- (61.9107,269.4670) -- cycle(62.7107,269.3060) -- (62.9774,269.3060) -- (62.9774,269.4670) -- (62.7107,269.4670) -- cycle(63.5107,269.3060) -- (63.7774,269.3060) -- (63.7774,269.4670) -- (63.5107,269.4670) -- cycle(64.3107,269.3060) -- (64.5774,269.3060) -- (64.5774,269.4670) -- (64.3107,269.4670) -- cycle(65.1107,269.3060) -- (65.3774,269.3060) -- (65.3774,269.4670) -- (65.1107,269.4670) -- cycle(65.9107,269.3060) -- (66.1774,269.3060) -- (66.1774,269.4670) -- (65.9107,269.4670) -- cycle(66.7107,269.3060) -- (66.9774,269.3060) -- (66.9774,269.4670) -- (66.7107,269.4670) -- cycle(67.5107,269.3060) -- (67.7773,269.3060) -- (67.7773,269.4670) -- (67.5107,269.4670) -- cycle(68.3107,269.3060) -- (68.5774,269.3060) -- (68.5774,269.4670) -- (68.3107,269.4670) -- cycle(69.1107,269.3060) -- (69.3774,269.3060) -- (69.3774,269.4670) -- (69.1107,269.4670) -- cycle(69.9107,269.3060) -- (70.1773,269.3060) -- (70.1773,269.4670) -- (69.9107,269.4670) -- cycle(70.7107,269.3060) -- (70.9774,269.3060) -- (70.9774,269.4670) -- (70.7107,269.4670) -- cycle(71.5107,269.3060) -- (71.7773,269.3060) -- (71.7773,269.4670) -- (71.5107,269.4670) -- cycle(72.3107,269.3060) -- (72.5774,269.3060) -- (72.5774,269.4670) -- (72.3107,269.4670) -- cycle(73.1107,269.3060) -- (73.3773,269.3060) -- (73.3773,269.4670) -- (73.1107,269.4670) -- cycle(73.9107,269.3060) -- (74.1773,269.3060) -- (74.1773,269.4670) -- (73.9107,269.4670) -- cycle(74.7107,269.3060) -- (74.9774,269.3060) -- (74.9774,269.4670) -- (74.7107,269.4670) -- cycle(75.5107,269.3060) -- (75.7773,269.3060) -- (75.7773,269.4670) -- (75.5107,269.4670) -- cycle(76.3107,269.3060) -- (76.5773,269.3060) -- (76.5773,269.4670) -- (76.3107,269.4670) -- cycle(77.1107,269.3060) -- (77.3773,269.3060) -- (77.3773,269.4670) -- (77.1107,269.4670) -- cycle(77.9106,269.3060) -- (78.1773,269.3060) -- (78.1773,269.4670) -- (77.9106,269.4670) -- cycle(78.7107,269.3060) -- (78.9774,269.3060) -- (78.9774,269.4670) -- (78.7107,269.4670) -- cycle(79.5107,269.3060) -- (79.7773,269.3060) -- (79.7773,269.4670) -- (79.5107,269.4670) -- cycle(80.3107,269.3060) -- (80.5773,269.3060) -- (80.5773,269.4670) -- (80.3107,269.4670) -- cycle(81.1107,269.3060) -- (81.3773,269.3060) -- (81.3773,269.4670) -- (81.1107,269.4670) -- cycle(81.9106,269.3060) -- (82.1773,269.3060) -- (82.1773,269.4670) -- (81.9106,269.4670) -- cycle(82.7107,269.3060) -- (82.9773,269.3060) -- (82.9773,269.4670) -- (82.7107,269.4670) -- cycle(83.5107,269.3060) -- (83.7773,269.3060) -- (83.7773,269.4670) -- (83.5107,269.4670) -- cycle(84.3106,269.3060) -- (84.5773,269.3060) -- (84.5773,269.4670) -- (84.3106,269.4670) -- cycle(85.1107,269.3060) -- (85.3773,269.3060) -- (85.3773,269.4670) -- (85.1107,269.4670) -- cycle(85.9106,269.3060) -- (86.1773,269.3060) -- (86.1773,269.4670) -- (85.9106,269.4670) -- cycle(86.7107,269.3060) -- (86.9773,269.3060) -- (86.9773,269.4670) -- (86.7107,269.4670) -- cycle(87.5106,269.3060) -- (87.7773,269.3060) -- (87.7773,269.4670) -- (87.5106,269.4670) -- cycle(88.3106,269.3060) -- (88.5773,269.3060) -- (88.5773,269.4670) -- (88.3106,269.4670) -- cycle(89.1107,269.3060) -- (89.3773,269.3060) -- (89.3773,269.4670) -- (89.1107,269.4670) -- cycle(89.9106,269.3060) -- (90.1773,269.3060) -- (90.1773,269.4670) -- (89.9106,269.4670) -- cycle(90.7106,269.3060) -- (90.9773,269.3060) -- (90.9773,269.4670) -- (90.7106,269.4670) -- cycle(91.5106,269.3060) -- (91.7773,269.3060) -- (91.7773,269.4670) -- (91.5106,269.4670) -- cycle(92.3106,269.3060) -- (92.5773,269.3060) -- (92.5773,269.4670) -- (92.3106,269.4670) -- cycle(93.1107,269.3060) -- (93.3773,269.3060) -- (93.3773,269.4670) -- (93.1107,269.4670) -- cycle(93.9106,269.3060) -- (94.1773,269.3060) -- (94.1773,269.4670) -- (93.9106,269.4670) -- cycle(94.7106,269.3060) -- (94.9773,269.3060) -- (94.9773,269.4670) -- (94.7106,269.4670) -- cycle(95.5106,269.3060) -- (95.7773,269.3060) -- (95.7773,269.4670) -- (95.5106,269.4670) -- cycle(96.3106,269.3060) -- (96.5773,269.3060) -- (96.5773,269.4670) -- (96.3106,269.4670) -- cycle(97.1106,269.3060) -- (97.3773,269.3060) -- (97.3773,269.4670) -- (97.1106,269.4670) -- cycle(97.9106,269.3060) -- (98.1773,269.3060) -- (98.1773,269.4670) -- (97.9106,269.4670) -- cycle(98.7106,269.3060) -- (98.9773,269.3060) -- (98.9773,269.4670) -- (98.7106,269.4670) -- cycle(99.5106,269.3060) -- (99.7773,269.3060) -- (99.7773,269.4670) -- (99.5106,269.4670) -- cycle(100.3110,269.3060) -- (100.5770,269.3060) -- (100.5770,269.4670) -- (100.3110,269.4670) -- cycle(101.1110,269.3060) -- (101.3770,269.3060) -- (101.3770,269.4670) -- (101.1110,269.4670) -- cycle(101.9110,269.3060) -- (102.1770,269.3060) -- (102.1770,269.4670) -- (101.9110,269.4670) -- cycle(102.7110,269.3060) -- (102.9770,269.3060) -- (102.9770,269.4670) -- (102.7110,269.4670) -- cycle(103.5110,269.3060) -- (103.7770,269.3060) -- (103.7770,269.4670) -- (103.5110,269.4670) -- cycle(104.3110,269.3060) -- (104.5770,269.3060) -- (104.5770,269.4670) -- (104.3110,269.4670) -- cycle(105.1110,269.3060) -- (105.3770,269.3060) -- (105.3770,269.4670) -- (105.1110,269.4670) -- cycle(105.9110,269.3060) -- (106.1770,269.3060) -- (106.1770,269.4670) -- (105.9110,269.4670) -- cycle(106.7110,269.3060) -- (106.9770,269.3060) -- (106.9770,269.4670) -- (106.7110,269.4670) -- cycle(107.5110,269.3060) -- (107.7770,269.3060) -- (107.7770,269.4670) -- (107.5110,269.4670) -- cycle(108.3110,269.3060) -- (108.5770,269.3060) -- (108.5770,269.4670) -- (108.3110,269.4670) -- cycle(109.1110,269.3060) -- (109.3770,269.3060) -- (109.3770,269.4670) -- (109.1110,269.4670) -- cycle(109.9110,269.3060) -- (110.1770,269.3060) -- (110.1770,269.4670) -- (109.9110,269.4670) -- cycle(110.7110,269.3060) -- (110.9770,269.3060) -- (110.9770,269.4670) -- (110.7110,269.4670) -- cycle(111.5110,269.3060) -- (111.7770,269.3060) -- (111.7770,269.4670) -- (111.5110,269.4670) -- cycle(112.3110,269.3060) -- (112.5770,269.3060) -- (112.5770,269.4670) -- (112.3110,269.4670) -- cycle(113.1110,269.3060) -- (113.3770,269.3060) -- (113.3770,269.4670) -- (113.1110,269.4670) -- cycle(113.9110,269.3060) -- (114.1770,269.3060) -- (114.1770,269.4670) -- (113.9110,269.4670) -- cycle(114.7110,269.3060) -- (114.9770,269.3060) -- (114.9770,269.4670) -- (114.7110,269.4670) -- cycle(115.5110,269.3060) -- (115.7770,269.3060) -- (115.7770,269.4670) -- (115.5110,269.4670) -- cycle(116.3110,269.3060) -- (116.5770,269.3060) -- (116.5770,269.4670) -- (116.3110,269.4670) -- cycle(117.1110,269.3060) -- (117.3770,269.3060) -- (117.3770,269.4670) -- (117.1110,269.4670) -- cycle(117.9110,269.3060) -- (118.1770,269.3060) -- (118.1770,269.4670) -- (117.9110,269.4670) -- cycle(118.7110,269.3060) -- (118.9770,269.3060) -- (118.9770,269.4670) -- (118.7110,269.4670) -- cycle(119.5110,269.3060) -- (119.7770,269.3060) -- (119.7770,269.4670) -- (119.5110,269.4670) -- cycle(120.3110,269.3060) -- (120.5770,269.3060) -- (120.5770,269.4670) -- (120.3110,269.4670) -- cycle(121.1110,269.3060) -- (121.3770,269.3060) -- (121.3770,269.4670) -- (121.1110,269.4670) -- cycle(121.9110,269.3060) -- (122.1770,269.3060) -- (122.1770,269.4670) -- (121.9110,269.4670) -- cycle(122.7110,269.3060) -- (122.9770,269.3060) -- (122.9770,269.4670) -- (122.7110,269.4670) -- cycle(123.5110,269.3060) -- (123.7770,269.3060) -- (123.7770,269.4670) -- (123.5110,269.4670) -- cycle(124.3110,269.3060) -- (124.5770,269.3060) -- (124.5770,269.4670) -- (124.3110,269.4670) -- cycle(125.1110,269.3060) -- (125.3770,269.3060) -- (125.3770,269.4670) -- (125.1110,269.4670) -- cycle(125.9110,269.3060) -- (126.1770,269.3060) -- (126.1770,269.4670) -- (125.9110,269.4670) -- cycle(126.7110,269.3060) -- (126.9770,269.3060) -- (126.9770,269.4670) -- (126.7110,269.4670) -- cycle(127.5110,269.3060) -- (127.7770,269.3060) -- (127.7770,269.4670) -- (127.5110,269.4670) -- cycle(128.3110,269.3060) -- (128.5770,269.3060) -- (128.5770,269.4670) -- (128.3110,269.4670) -- cycle(129.1110,269.3060) -- (129.3770,269.3060) -- (129.3770,269.4670) -- (129.1110,269.4670) -- cycle(129.9110,269.3060) -- (130.1770,269.3060) -- (130.1770,269.4670) -- (129.9110,269.4670) -- cycle(130.7110,269.3060) -- (130.9770,269.3060) -- (130.9770,269.4670) -- (130.7110,269.4670) -- cycle(131.5110,269.3060) -- (131.7770,269.3060) -- (131.7770,269.4670) -- (131.5110,269.4670) -- cycle(132.3110,269.3060) -- (132.5770,269.3060) -- (132.5770,269.4670) -- (132.3110,269.4670) -- cycle(133.1110,269.3060) -- (133.3770,269.3060) -- (133.3770,269.4670) -- (133.1110,269.4670) -- cycle(133.9110,269.3060) -- (134.1770,269.3060) -- (134.1770,269.4670) -- (133.9110,269.4670) -- cycle(134.7110,269.3060) -- (134.9770,269.3060) -- (134.9770,269.4670) -- (134.7110,269.4670) -- cycle(135.5100,269.3060) -- (135.7770,269.3060) -- (135.7770,269.4670) -- (135.5100,269.4670) -- cycle(136.3110,269.3060) -- (136.5770,269.3060) -- (136.5770,269.4670) -- (136.3110,269.4670) -- cycle(137.1110,269.3060) -- (137.3770,269.3060) -- (137.3770,269.4670) -- (137.1110,269.4670) -- cycle(137.9110,269.3060) -- (138.1770,269.3060) -- (138.1770,269.4670) -- (137.9110,269.4670) -- cycle(138.7110,269.3060) -- (138.9770,269.3060) -- (138.9770,269.4670) -- (138.7110,269.4670) -- cycle(139.5100,269.3060) -- (139.7770,269.3060) -- (139.7770,269.4670) -- (139.5100,269.4670) -- cycle(140.3110,269.3060) -- (140.5770,269.3060) -- (140.5770,269.4670) -- (140.3110,269.4670) -- cycle(141.1110,269.3060) -- (141.3770,269.3060) -- (141.3770,269.4670) -- (141.1110,269.4670) -- cycle(141.9100,269.3060) -- (142.1770,269.3060) -- (142.1770,269.4670) -- (141.9100,269.4670) -- cycle(142.7110,269.3060) -- (142.9770,269.3060) -- (142.9770,269.4670) -- (142.7110,269.4670) -- cycle(143.5100,269.3060) -- (143.7770,269.3060) -- (143.7770,269.4670) -- (143.5100,269.4670) -- cycle(144.3110,269.3060) -- (144.5770,269.3060) -- (144.5770,269.4670) -- (144.3110,269.4670) -- cycle(145.1100,269.3060) -- (145.3770,269.3060) -- (145.3770,269.4670) -- (145.1100,269.4670) -- cycle(145.9100,269.3060) -- (146.1770,269.3060) -- (146.1770,269.4670) -- (145.9100,269.4670) -- cycle(146.7110,269.3060) -- (146.9770,269.3060) -- (146.9770,269.4670) -- (146.7110,269.4670) -- cycle(147.5100,269.3060) -- (147.7770,269.3060) -- (147.7770,269.4670) -- (147.5100,269.4670) -- cycle(148.3100,269.3060) -- (148.5770,269.3060) -- (148.5770,269.4670) -- (148.3100,269.4670) -- cycle(149.1100,269.3060) -- (149.3770,269.3060) -- (149.3770,269.4670) -- (149.1100,269.4670) -- cycle(149.9100,269.3060) -- (150.1770,269.3060) -- (150.1770,269.4670) -- (149.9100,269.4670) -- cycle(150.7110,269.3060) -- (150.9770,269.3060) -- (150.9770,269.4670) -- (150.7110,269.4670) -- cycle(151.5100,269.3060) -- (151.7770,269.3060) -- (151.7770,269.4670) -- (151.5100,269.4670) -- cycle(152.3100,269.3060) -- (152.5770,269.3060) -- (152.5770,269.4670) -- (152.3100,269.4670) -- cycle(153.1100,269.3060) -- (153.3770,269.3060) -- (153.3770,269.4670) -- (153.1100,269.4670) -- cycle(153.9100,269.3060) -- (154.1770,269.3060) -- (154.1770,269.4670) -- (153.9100,269.4670) -- cycle(154.7100,269.3060) -- (154.9770,269.3060) -- (154.9770,269.4670) -- (154.7100,269.4670) -- cycle(155.5100,269.3060) -- (155.7770,269.3060) -- (155.7770,269.4670) -- (155.5100,269.4670) -- cycle(156.3100,269.3060) -- (156.5770,269.3060) -- (156.5770,269.4670) -- (156.3100,269.4670) -- cycle(157.1100,269.3060) -- (157.3770,269.3060) -- (157.3770,269.4670) -- (157.1100,269.4670) -- cycle(157.9100,269.3060) -- (158.1770,269.3060) -- (158.1770,269.4670) -- (157.9100,269.4670) -- cycle(158.7100,269.3060) -- (158.9770,269.3060) -- (158.9770,269.4670) -- (158.7100,269.4670) -- cycle(159.5100,269.3060) -- (159.7770,269.3060) -- (159.7770,269.4670) -- (159.5100,269.4670) -- cycle(160.3100,269.3060) -- (160.5770,269.3060) -- (160.5770,269.4670) -- (160.3100,269.4670) -- cycle(161.1100,269.3060) -- (161.3770,269.3060) -- (161.3770,269.4670) -- (161.1100,269.4670) -- cycle(161.9100,269.3060) -- (162.1770,269.3060) -- (162.1770,269.4670) -- (161.9100,269.4670) -- cycle(162.7100,269.3060) -- (162.9770,269.3060) -- (162.9770,269.4670) -- (162.7100,269.4670) -- cycle(163.5100,269.3060) -- (163.7770,269.3060) -- (163.7770,269.4670) -- (163.5100,269.4670) -- cycle(164.3100,269.3060) -- (164.5770,269.3060) -- (164.5770,269.4670) -- (164.3100,269.4670) -- cycle(165.1100,269.3060) -- (165.3770,269.3060) -- (165.3770,269.4670) -- (165.1100,269.4670) -- cycle(165.9100,269.3060) -- (166.1770,269.3060) -- (166.1770,269.4670) -- (165.9100,269.4670) -- cycle(166.7100,269.3070) -- (166.9770,269.3070) -- (166.9770,269.4670) -- (166.7100,269.4670) -- cycle(167.5100,269.3070) -- (167.7770,269.3070) -- (167.7770,269.4670) -- (167.5100,269.4670) -- cycle(168.3100,269.3070) -- (168.5770,269.3070) -- (168.5770,269.4670) -- (168.3100,269.4670) -- cycle(169.1100,269.3070) -- (169.3770,269.3070) -- (169.3770,269.4670) -- (169.1100,269.4670) -- cycle(169.9100,269.3070) -- (170.1770,269.3070) -- (170.1770,269.4670) -- (169.9100,269.4670) -- cycle(170.7100,269.3070) -- (170.9770,269.3070) -- (170.9770,269.4670) -- (170.7100,269.4670) -- cycle(171.5100,269.3070) -- (171.7770,269.3070) -- (171.7770,269.4670) -- (171.5100,269.4670) -- cycle(172.3100,269.3070) -- (172.5770,269.3070) -- (172.5770,269.4670) -- (172.3100,269.4670) -- cycle(173.1100,269.3070) -- (173.3770,269.3070) -- (173.3770,269.4670) -- (173.1100,269.4670) -- cycle(173.9100,269.3070) -- (174.1770,269.3070) -- (174.1770,269.4670) -- (173.9100,269.4670) -- cycle(174.7100,269.3070) -- (174.9770,269.3070) -- (174.9770,269.4670) -- (174.7100,269.4670) -- cycle(175.5100,269.3070) -- (175.7770,269.3070) -- (175.7770,269.4670) -- (175.5100,269.4670) -- cycle(176.3100,269.3070) -- (176.5770,269.3070) -- (176.5770,269.4670) -- (176.3100,269.4670) -- cycle(177.1100,269.3070) -- (177.3770,269.3070) -- (177.3770,269.4670) -- (177.1100,269.4670) -- cycle(177.9100,269.3070) -- (178.1770,269.3070) -- (178.1770,269.4670) -- (177.9100,269.4670) -- cycle(178.7100,269.3070) -- (178.9770,269.3070) -- (178.9770,269.4670) -- (178.7100,269.4670) -- cycle(179.5100,269.3070) -- (179.7770,269.3070) -- (179.7770,269.4670) -- (179.5100,269.4670) -- cycle(180.3100,269.3070) -- (180.5770,269.3070) -- (180.5770,269.4670) -- (180.3100,269.4670) -- cycle(181.1100,269.3070) -- (181.3770,269.3070) -- (181.3770,269.4670) -- (181.1100,269.4670) -- cycle(181.9100,269.3070) -- (182.1770,269.3070) -- (182.1770,269.4670) -- (181.9100,269.4670) -- cycle(182.7100,269.3070) -- (182.9770,269.3070) -- (182.9770,269.4670) -- (182.7100,269.4670) -- cycle(183.5100,269.3070) -- (183.7770,269.3070) -- (183.7770,269.4670) -- (183.5100,269.4670) -- cycle(184.3100,269.3070) -- (184.5770,269.3070) -- (184.5770,269.4670) -- (184.3100,269.4670) -- cycle(185.1100,269.3070) -- (185.3770,269.3070) -- (185.3770,269.4670) -- (185.1100,269.4670) -- cycle(185.9100,269.3070) -- (186.1770,269.3070) -- (186.1770,269.4670) -- (185.9100,269.4670) -- cycle(186.7100,269.3070) -- (186.9770,269.3070) -- (186.9770,269.4670) -- (186.7100,269.4670) -- cycle(187.5100,269.3070) -- (187.7770,269.3070) -- (187.7770,269.4670) -- (187.5100,269.4670) -- cycle(188.3100,269.3070) -- (188.5770,269.3070) -- (188.5770,269.4670) -- (188.3100,269.4670) -- cycle(189.1100,269.3070) -- (189.3770,269.3070) -- (189.3770,269.4670) -- (189.1100,269.4670) -- cycle(189.9100,269.3070) -- (190.1770,269.3070) -- (190.1770,269.4670) -- (189.9100,269.4670) -- cycle(190.7100,269.3070) -- (190.9770,269.3070) -- (190.9770,269.4670) -- (190.7100,269.4670) -- cycle(191.5100,269.3070) -- (191.7770,269.3070) -- (191.7770,269.4670) -- (191.5100,269.4670) -- cycle(192.3100,269.3070) -- (192.5770,269.3070) -- (192.5770,269.4670) -- (192.3100,269.4670) -- cycle(193.1100,269.3070) -- (193.3770,269.3070) -- (193.3770,269.4670) -- (193.1100,269.4670) -- cycle(193.9100,269.3070) -- (194.1770,269.3070) -- (194.1770,269.4670) -- (193.9100,269.4670) -- cycle(194.7100,269.3070) -- (194.9770,269.3070) -- (194.9770,269.4670) -- (194.7100,269.4670) -- cycle(195.5100,269.3070) -- (195.7770,269.3070) -- (195.7770,269.4670) -- (195.5100,269.4670) -- cycle(196.3100,269.3070) -- (196.5770,269.3070) -- (196.5770,269.4670) -- (196.3100,269.4670) -- cycle(197.1100,269.3070) -- (197.3770,269.3070) -- (197.3770,269.4670) -- (197.1100,269.4670) -- cycle(197.9100,269.3070) -- (198.1770,269.3070) -- (198.1770,269.4670) -- (197.9100,269.4670) -- cycle(198.7100,269.3070) -- (198.9770,269.3070) -- (198.9770,269.4670) -- (198.7100,269.4670) -- cycle(199.5100,269.3070) -- (199.7770,269.3070) -- (199.7770,269.4670) -- (199.5100,269.4670) -- cycle(200.3100,269.3070) -- (200.5770,269.3070) -- (200.5770,269.4670) -- (200.3100,269.4670) -- cycle(201.1100,269.3070) -- (201.3770,269.3070) -- (201.3770,269.4670) -- (201.1100,269.4670) -- cycle(201.9100,269.3070) -- (202.1770,269.3070) -- (202.1770,269.4670) -- (201.9100,269.4670) -- cycle(202.7100,269.3070) -- (202.9770,269.3070) -- (202.9770,269.4670) -- (202.7100,269.4670) -- cycle(203.5100,269.3070) -- (203.7770,269.3070) -- (203.7770,269.4670) -- (203.5100,269.4670) -- cycle(204.3100,269.3070) -- (204.5770,269.3070) -- (204.5770,269.4670) -- (204.3100,269.4670) -- cycle(205.1100,269.3070) -- (205.3770,269.3070) -- (205.3770,269.4670) -- (205.1100,269.4670) -- cycle(205.9100,269.3070) -- (206.1770,269.3070) -- (206.1770,269.4670) -- (205.9100,269.4670) -- cycle(206.7100,269.3070) -- (206.9770,269.3070) -- (206.9770,269.4670) -- (206.7100,269.4670) -- cycle(207.5100,269.3070) -- (207.7770,269.3070) -- (207.7770,269.4670) -- (207.5100,269.4670) -- cycle(208.3100,269.3070) -- (208.5770,269.3070) -- (208.5770,269.4670) -- (208.3100,269.4670) -- cycle(209.1100,269.3070) -- (209.3770,269.3070) -- (209.3770,269.4670) -- (209.1100,269.4670) -- cycle(209.9100,269.3070) -- (210.1770,269.3070) -- (210.1770,269.4670) -- (209.9100,269.4670) -- cycle(210.7100,269.3070) -- (210.9770,269.3070) -- (210.9770,269.4670) -- (210.7100,269.4670) -- cycle(211.5100,269.3070) -- (211.7770,269.3070) -- (211.7770,269.4670) -- (211.5100,269.4670) -- cycle(212.3100,269.3070) -- (212.5770,269.3070) -- (212.5770,269.4670) -- (212.3100,269.4670) -- cycle(213.1100,269.3070) -- (213.3770,269.3070) -- (213.3770,269.4670) -- (213.1100,269.4670) -- cycle(213.9100,269.3070) -- (214.1770,269.3070) -- (214.1770,269.4670) -- (213.9100,269.4670) -- cycle(214.7100,269.3070) -- (214.9770,269.3070) -- (214.9770,269.4670) -- (214.7100,269.4670) -- cycle(215.5100,269.3070) -- (215.7770,269.3070) -- (215.7770,269.4670) -- (215.5100,269.4670) -- cycle(216.3100,269.3070) -- (216.5770,269.3070) -- (216.5770,269.4670) -- (216.3100,269.4670) -- cycle(217.1100,269.3070) -- (217.3770,269.3070) -- (217.3770,269.4670) -- (217.1100,269.4670) -- cycle(217.9100,269.3070) -- (218.1190,269.3070) -- (218.1190,269.4670) -- (217.9100,269.4670) -- cycle(12.3109,269.3060) -- (12.5775,269.3060) -- (12.5775,269.4670) -- (12.3109,269.4670) -- cycle;



\end{scope}
\path[draw=blue,line join=round,line width=0.512pt] (120.1363,145.3700) .. controls (110.3323,172.3050) and (80.5504,194.1400) .. (53.6156,194.1400) .. controls (26.6809,194.1400) and (12.7932,172.3050) .. (22.5967,145.3700);



\path[draw=blue,line join=round,line width=0.512pt] (111.1293,57.6350) .. controls (118.2003,38.2078) and (139.6813,22.4589) .. (159.1083,22.4589) .. controls (178.5363,22.4589) and (188.5523,38.2078) .. (181.4823,57.6350);



\path[draw=blue,line join=round,line width=0.512pt] (126.3643,145.3310) .. controls (116.5603,172.2650) and (86.7785,194.1000) .. (59.8438,194.1000) .. controls (32.9090,194.1000) and (19.0214,172.2650) .. (28.8248,145.3310);



\path[draw=blue,line join=round,line width=0.512pt] (132.4263,145.3740) .. controls (122.6233,172.3090) and (92.8408,194.1440) .. (65.9060,194.1440) .. controls (38.9713,194.1440) and (25.0836,172.3090) .. (34.8871,145.3740);



\path[draw=blue,line join=round,line width=0.512pt] (149.6273,145.3660) .. controls (139.8243,172.3000) and (110.0423,194.1350) .. (83.1073,194.1350) .. controls (56.1726,194.1350) and (42.2849,172.3000) .. (52.0884,145.3660);



\path[draw=blue,line join=round,line width=0.512pt] (78.6900,146.0240) .. controls (68.8866,172.9590) and (82.7742,194.7930) .. (109.7093,194.7930) .. controls (136.6433,194.7930) and (166.4263,172.9590) .. (176.2293,146.0240) -- (199.4753,82.0996);



\path[draw=blue,line join=round,line width=0.512pt] (192.3144,165.3447) .. controls (178.5312,182.7397) and (156.8197,194.9260) .. (136.6983,194.9260) .. controls (109.7633,194.9260) and (95.8757,173.0910) .. (105.6793,146.1560);



\path[draw=blue,line join=round,line width=0.512pt] (60.9041,57.6673) .. controls (70.0559,32.5231) and (97.8582,12.1397) .. (123.0023,12.1397) .. controls (148.1463,12.1397) and (161.1113,32.5231) .. (151.9593,57.6673);



\path[draw=blue,line join=round,line width=0.512pt] (164.3813,57.8627) .. controls (173.5333,32.7186) and (160.5683,12.3352) .. (135.4483,12.3212) -- (128.9963,12.6252) .. controls (101.0163,17.4767) and (95.2134,38.2692) .. (90.9964,44.7409) .. controls (86.6372,51.4311) and (84.0698,57.6784) .. (84.0698,57.6784) -- (84.0090,57.8741) -- (83.9193,58.2043);



\path[draw=blue,line join=round,line width=0.512pt] (132.1408,-0.0621) .. controls (128.3119,5.2514) and (121.7993,10.2271) .. (110.7303,12.1463) .. controls (82.8116,16.9872) and (67.7168,35.3583) .. (63.4334,41.9112);



\path[draw=blue,line join=round,line width=0.512pt] (67.0702,57.8055) .. controls (76.2219,32.6614) and (104.0243,12.2780) .. (129.1683,12.2780) .. controls (154.3123,12.2780) and (167.2773,32.6614) .. (158.1253,57.8055);



\path[draw=blue,line join=round,line width=0.512pt] (54.8647,57.1820) .. controls (54.8647,57.1820) and (59.2381,46.7836) .. (63.6380,41.5836);



\path[draw=blue,line join=round,line width=0.512pt] (22.5839,145.3980) -- (54.8805,57.1444);



\path[draw=blue,line join=round,line width=0.512pt] (28.7831,145.4080) -- (61.0797,57.1548);



\path[draw=blue,line join=round,line width=0.512pt] (34.8951,145.3700) -- (67.1067,57.6985);



\path[draw=blue,line join=round,line width=0.512pt] (137.9553,57.8127) .. controls (145.0263,38.3855) and (166.5073,22.6366) .. (185.9343,22.6366) .. controls (205.3613,22.6366) and (215.3783,38.3855) .. (208.3073,57.8127);



\path[draw=blue,line join=round,line width=0.512pt] (52.0637,145.3971) -- (83.9688,58.1425);



\path[draw=blue,line join=round,line width=0.512pt] (78.6815,146.0230) -- (111.1593,57.5890);



\path[draw=blue,line join=round,line width=0.512pt] (105.6812,146.1901) -- (137.9673,57.7878);



\path[draw=blue,line join=round,line width=0.512pt] (120.1373,145.3990) -- (151.9493,57.6496);



\path[draw=blue,line join=round,line width=0.512pt] (126.3643,145.3390) -- (158.1423,57.7755);



\path[draw=blue,line join=round,line width=0.512pt] (132.4412,145.3850) -- (164.3742,57.8535);



\path[draw=blue,line join=round,line width=0.512pt] (149.6163,145.3760) -- (181.4853,57.6289);



\begin{scope}[cm={{1.33334,0.0,0.0,1.33334,(226.63568,-18.04678)}}]
  \begin{scope}[draw=blue,even odd rule,line cap=rect,line join=bevel,line width=0.800pt]
  \end{scope}
  \begin{scope}[scale=1.012,draw=blue,even odd rule,line cap=rect,line join=bevel,line width=0.800pt]
  \end{scope}
  \path[fill=cffffff,line cap=butt,line join=round,line width=0.437pt,miter limit=4.00,rounded corners=0.0000cm] (-4.1656,113.5623) rectangle (69.1486,176.1082);



  \begin{scope}[cm={{0.72986,0.0,0.0,0.90534,(-13.69026,97.79668)}},draw=blue,even odd rule,line cap=round,line join=round,line width=0.400pt]
    \path[draw] (29.5000,86.5000) -- (29.5000,83.5000);



  \end{scope}
  \begin{scope}[scale=1.012,draw=blue,even odd rule,line cap=rect,line join=bevel,line width=0.800pt]
  \end{scope}
  \begin{scope}[cm={{1.0125,0.0,0.0,1.0125,(24.3,99.225)}},draw=blue,even odd rule,line cap=rect,line join=bevel,line width=0.800pt]
  \end{scope}
  \begin{scope}[cm={{1.0125,0.0,0.0,1.0125,(24.3,99.225)}},draw=blue,even odd rule,line cap=rect,line join=bevel,line width=0.800pt]
  \end{scope}
  \begin{scope}[cm={{1.0125,0.0,0.0,1.0125,(24.3,99.225)}},draw=blue,even odd rule,line cap=rect,line join=bevel,line width=0.800pt]
  \end{scope}
  \begin{scope}[cm={{1.0125,0.0,0.0,1.0125,(24.3,99.225)}},draw=blue,even odd rule,line cap=rect,line join=bevel,line width=0.800pt]
  \end{scope}
  \begin{scope}[cm={{1.0125,0.0,0.0,1.0125,(24.3,99.225)}},draw=blue,even odd rule,line cap=rect,line join=bevel,line width=0.800pt]
  \end{scope}
  \begin{scope}[cm={{1.0125,0.0,0.0,1.0125,(1.8,184.725)}},draw=blue,even odd rule,line cap=rect,line join=bevel,line width=0.800pt]
    \path[fill=blue] (0.0000,0.0000) node[above right] (text32) {-2};



  \end{scope}
  \begin{scope}[cm={{1.0125,0.0,0.0,1.0125,(24.3,99.225)}},draw=blue,even odd rule,line cap=rect,line join=bevel,line width=0.800pt]
  \end{scope}
  \begin{scope}[scale=1.012,draw=blue,even odd rule,line cap=rect,line join=bevel,line width=0.800pt]
  \end{scope}
  \begin{scope}[cm={{0.72986,0.0,0.0,0.90534,(-13.69026,97.79668)}},draw=blue,even odd rule,line cap=round,line join=round,line width=0.400pt]
    \path[draw] (63.5000,86.5000) -- (63.5000,83.5000);



  \end{scope}
  \begin{scope}[scale=1.012,draw=blue,even odd rule,line cap=rect,line join=bevel,line width=0.800pt]
  \end{scope}
  \begin{scope}[cm={{1.0125,0.0,0.0,1.0125,(60.75,99.225)}},draw=blue,even odd rule,line cap=rect,line join=bevel,line width=0.800pt]
  \end{scope}
  \begin{scope}[cm={{1.0125,0.0,0.0,1.0125,(60.75,99.225)}},draw=blue,even odd rule,line cap=rect,line join=bevel,line width=0.800pt]
  \end{scope}
  \begin{scope}[cm={{1.0125,0.0,0.0,1.0125,(60.75,99.225)}},draw=blue,even odd rule,line cap=rect,line join=bevel,line width=0.800pt]
  \end{scope}
  \begin{scope}[cm={{1.0125,0.0,0.0,1.0125,(60.75,99.225)}},draw=blue,even odd rule,line cap=rect,line join=bevel,line width=0.800pt]
  \end{scope}
  \begin{scope}[cm={{1.0125,0.0,0.0,1.0125,(60.75,99.225)}},draw=blue,even odd rule,line cap=rect,line join=bevel,line width=0.800pt]
  \end{scope}
  \begin{scope}[cm={{1.0125,0.0,0.0,1.0125,(29.25,184.725)}},draw=blue,even odd rule,line cap=rect,line join=bevel,line width=0.800pt]
    \path[fill=blue] (0.0000,0.0000) node[above right] (text60) {0};



  \end{scope}
  \begin{scope}[cm={{1.0125,0.0,0.0,1.0125,(60.75,99.225)}},draw=blue,even odd rule,line cap=rect,line join=bevel,line width=0.800pt]
  \end{scope}
  \begin{scope}[scale=1.012,draw=blue,even odd rule,line cap=rect,line join=bevel,line width=0.800pt]
  \end{scope}
  \begin{scope}[cm={{0.72986,0.0,0.0,0.90534,(-13.69026,97.79668)}},draw=blue,even odd rule,line cap=round,line join=round,line width=0.400pt]
    \path[draw] (97.5000,86.5000) -- (97.5000,83.5000);



  \end{scope}
  \begin{scope}[scale=1.012,draw=blue,even odd rule,line cap=rect,line join=bevel,line width=0.800pt]
  \end{scope}
  \begin{scope}[cm={{1.0125,0.0,0.0,1.0125,(95.175,99.225)}},draw=blue,even odd rule,line cap=rect,line join=bevel,line width=0.800pt]
  \end{scope}
  \begin{scope}[cm={{1.0125,0.0,0.0,1.0125,(95.175,99.225)}},draw=blue,even odd rule,line cap=rect,line join=bevel,line width=0.800pt]
  \end{scope}
  \begin{scope}[cm={{1.0125,0.0,0.0,1.0125,(95.175,99.225)}},draw=blue,even odd rule,line cap=rect,line join=bevel,line width=0.800pt]
  \end{scope}
  \begin{scope}[cm={{1.0125,0.0,0.0,1.0125,(95.175,99.225)}},draw=blue,even odd rule,line cap=rect,line join=bevel,line width=0.800pt]
  \end{scope}
  \begin{scope}[cm={{1.0125,0.0,0.0,1.0125,(95.175,99.225)}},draw=blue,even odd rule,line cap=rect,line join=bevel,line width=0.800pt]
  \end{scope}
  \begin{scope}[cm={{1.0125,0.0,0.0,1.0125,(80.175,189.225)}},draw=blue,even odd rule,line cap=rect,line join=bevel,line width=0.800pt]
    \path[fill=blue] (-25.1852,-4.4444) node[above right] (text88) {2};



  \end{scope}
  \begin{scope}[cm={{1.0125,0.0,0.0,1.0125,(95.175,99.225)}},draw=blue,even odd rule,line cap=rect,line join=bevel,line width=0.800pt]
  \end{scope}
  \begin{scope}[scale=1.012,draw=blue,even odd rule,line cap=rect,line join=bevel,line width=0.800pt]
  \end{scope}
  \begin{scope}[cm={{0.72986,0.0,0.0,0.90534,(-13.69026,97.79668)}},draw=ca0a0a4,dash pattern=on 0.40pt off 0.80pt,even odd rule,line cap=round,line join=round,line width=0.400pt]
    \path[draw] (12.5000,86.5000) -- (113.5000,86.5000);



  \end{scope}
  \begin{scope}[cm={{0.72986,0.0,0.0,0.90534,(-13.69026,97.79668)}},draw=blue,even odd rule,line cap=round,line join=round,line width=0.400pt]
    \path[draw] (113.5000,86.5000) -- (113.5000,86.5000) -- (110.5000,86.5000);



  \end{scope}
  \begin{scope}[scale=1.012,draw=blue,even odd rule,line cap=rect,line join=bevel,line width=0.800pt]
  \end{scope}
  \begin{scope}[cm={{1.0125,0.0,0.0,1.0125,(118.462,91.125)}},draw=blue,even odd rule,line cap=rect,line join=bevel,line width=0.800pt]
  \end{scope}
  \begin{scope}[cm={{1.0125,0.0,0.0,1.0125,(118.462,91.125)}},draw=blue,even odd rule,line cap=rect,line join=bevel,line width=0.800pt]
  \end{scope}
  \begin{scope}[cm={{1.0125,0.0,0.0,1.0125,(118.462,91.125)}},draw=blue,even odd rule,line cap=rect,line join=bevel,line width=0.800pt]
  \end{scope}
  \begin{scope}[cm={{1.0125,0.0,0.0,1.0125,(118.462,91.125)}},draw=blue,even odd rule,line cap=rect,line join=bevel,line width=0.800pt]
  \end{scope}
  \begin{scope}[cm={{1.0125,0.0,0.0,1.0125,(118.462,91.125)}},draw=blue,even odd rule,line cap=rect,line join=bevel,line width=0.800pt]
  \end{scope}
  \begin{scope}[cm={{1.0125,0.0,0.0,1.0125,(-10.538,179.625)}},draw=blue,even odd rule,line cap=rect,line join=bevel,line width=0.800pt]
    \path[fill=blue] (0.0000,0.0000) node[above right] (text116) {0};



  \end{scope}
  \begin{scope}[cm={{1.0125,0.0,0.0,1.0125,(118.462,91.125)}},draw=blue,even odd rule,line cap=rect,line join=bevel,line width=0.800pt]
  \end{scope}
  \begin{scope}[scale=1.012,draw=blue,even odd rule,line cap=rect,line join=bevel,line width=0.800pt]
  \end{scope}
  \begin{scope}[cm={{0.72986,0.0,0.0,0.90534,(-84.81511,97.79668)}},draw=blue,even odd rule,line cap=round,line join=round,line width=0.400pt]
    \path[draw] (113.5000,67.5000) -- (113.5000,67.5000) -- (110.5000,67.5000);



  \end{scope}
  \begin{scope}[scale=1.012,draw=blue,even odd rule,line cap=rect,line join=bevel,line width=0.800pt]
  \end{scope}
  \begin{scope}[cm={{1.0125,0.0,0.0,1.0125,(118.462,72.9)}},draw=blue,even odd rule,line cap=rect,line join=bevel,line width=0.800pt]
  \end{scope}
  \begin{scope}[cm={{1.0125,0.0,0.0,1.0125,(118.462,72.9)}},draw=blue,even odd rule,line cap=rect,line join=bevel,line width=0.800pt]
  \end{scope}
  \begin{scope}[cm={{1.0125,0.0,0.0,1.0125,(118.462,72.9)}},draw=blue,even odd rule,line cap=rect,line join=bevel,line width=0.800pt]
  \end{scope}
  \begin{scope}[cm={{1.0125,0.0,0.0,1.0125,(118.462,72.9)}},draw=blue,even odd rule,line cap=rect,line join=bevel,line width=0.800pt]
  \end{scope}
  \begin{scope}[cm={{1.0125,0.0,0.0,1.0125,(118.462,72.9)}},draw=blue,even odd rule,line cap=rect,line join=bevel,line width=0.800pt]
  \end{scope}
  \begin{scope}[cm={{1.0125,0.0,0.0,1.0125,(-10.538,161.4)}},draw=blue,even odd rule,line cap=rect,line join=bevel,line width=0.800pt]
    \path[fill=blue] (0.0000,0.0000) node[above right] (text144) {2};



  \end{scope}
  \begin{scope}[cm={{1.0125,0.0,0.0,1.0125,(118.462,72.9)}},draw=blue,even odd rule,line cap=rect,line join=bevel,line width=0.800pt]
  \end{scope}
  \begin{scope}[scale=1.012,draw=blue,even odd rule,line cap=rect,line join=bevel,line width=0.800pt]
  \end{scope}
  \begin{scope}[cm={{0.72986,0.0,0.0,0.90534,(-84.81511,97.79668)}},draw=blue,even odd rule,line cap=round,line join=round,line width=0.400pt]
    \path[draw] (113.5000,48.5000) -- (113.5000,48.5000) -- (110.5000,48.5000);



  \end{scope}
  \begin{scope}[scale=1.012,draw=blue,even odd rule,line cap=rect,line join=bevel,line width=0.800pt]
  \end{scope}
  \begin{scope}[cm={{1.0125,0.0,0.0,1.0125,(118.969,53.6625)}},draw=blue,even odd rule,line cap=rect,line join=bevel,line width=0.800pt]
  \end{scope}
  \begin{scope}[cm={{1.0125,0.0,0.0,1.0125,(118.969,53.6625)}},draw=blue,even odd rule,line cap=rect,line join=bevel,line width=0.800pt]
  \end{scope}
  \begin{scope}[cm={{1.0125,0.0,0.0,1.0125,(118.969,53.6625)}},draw=blue,even odd rule,line cap=rect,line join=bevel,line width=0.800pt]
  \end{scope}
  \begin{scope}[cm={{1.0125,0.0,0.0,1.0125,(118.969,53.6625)}},draw=blue,even odd rule,line cap=rect,line join=bevel,line width=0.800pt]
  \end{scope}
  \begin{scope}[cm={{1.0125,0.0,0.0,1.0125,(118.969,53.6625)}},draw=blue,even odd rule,line cap=rect,line join=bevel,line width=0.800pt]
  \end{scope}
  \begin{scope}[cm={{1.0125,0.0,0.0,1.0125,(-23.91053,149.56373)}},draw=blue,even odd rule,line cap=rect,line join=bevel,line width=0.800pt]
    \path[fill=blue] (13.3333,-5.9259) node[above right] (text172) {4};



    \path[fill=blue,even odd rule,line width=0.800pt] (4.4380,-27.8667) node[above right] (text172-3) {4000};



  \end{scope}
  \begin{scope}[cm={{1.0125,0.0,0.0,1.0125,(118.969,53.6625)}},draw=blue,even odd rule,line cap=rect,line join=bevel,line width=0.800pt]
  \end{scope}
  \begin{scope}[scale=1.012,draw=blue,even odd rule,line cap=rect,line join=bevel,line width=0.800pt]
  \end{scope}
  \begin{scope}[scale=1.012,draw=blue,even odd rule,line cap=rect,line join=bevel,line width=0.800pt]
  \end{scope}
  \begin{scope}[cm={{0.0,-1.0125,1.0125,0.0,(134.662,91.125)}},draw=blue,even odd rule,line cap=rect,line join=bevel,line width=0.800pt]
  \end{scope}
  \begin{scope}[cm={{0.0,-1.0125,1.0125,0.0,(134.662,91.125)}},draw=blue,even odd rule,line cap=rect,line join=bevel,line width=0.800pt]
  \end{scope}
  \begin{scope}[cm={{0.0,-1.0125,1.0125,0.0,(134.662,91.125)}},draw=blue,even odd rule,line cap=rect,line join=bevel,line width=0.800pt]
  \end{scope}
  \begin{scope}[cm={{0.0,-1.0125,1.0125,0.0,(134.662,91.125)}},draw=blue,even odd rule,line cap=rect,line join=bevel,line width=0.800pt]
  \end{scope}
  \begin{scope}[cm={{0.0,-1.0125,1.0125,0.0,(134.662,91.125)}},draw=blue,even odd rule,line cap=rect,line join=bevel,line width=0.800pt]
  \end{scope}
  \begin{scope}[cm={{0.0,-1.0125,1.0125,0.0,(-22.80695,173.125)}},draw=blue,even odd rule,line cap=rect,line join=bevel,line width=0.800pt]
    \path[fill=blue] (0.0000,0.0000) node[above right] (text196) {\rotatebox{90}{Spectrum ($\text{10}^\text{2}$\hspace*{.2ex}dB)}};



  \end{scope}
  \begin{scope}[cm={{0.0,-1.0125,1.0125,0.0,(134.662,91.125)}},draw=blue,even odd rule,line cap=rect,line join=bevel,line width=0.800pt]
  \end{scope}
  \begin{scope}[cm={{1.0125,0.0,0.0,1.0125,(25.8187,114.412)}},draw=blue,even odd rule,line cap=rect,line join=bevel,line width=0.800pt]
  \end{scope}
  \begin{scope}[cm={{1.0125,0.0,0.0,1.0125,(25.8187,114.412)}},draw=blue,even odd rule,line cap=rect,line join=bevel,line width=0.800pt]
  \end{scope}
  \begin{scope}[cm={{1.0125,0.0,0.0,1.0125,(25.8187,114.412)}},draw=blue,even odd rule,line cap=rect,line join=bevel,line width=0.800pt]
  \end{scope}
  \begin{scope}[cm={{1.0125,0.0,0.0,1.0125,(25.8187,114.412)}},draw=blue,even odd rule,line cap=rect,line join=bevel,line width=0.800pt]
  \end{scope}
  \begin{scope}[cm={{1.0125,0.0,0.0,1.0125,(25.8187,114.412)}},draw=blue,even odd rule,line cap=rect,line join=bevel,line width=0.800pt]
  \end{scope}
  \begin{scope}[cm={{1.0125,0.0,0.0,1.0125,(4.8187,193.912)}},draw=blue,even odd rule,line cap=rect,line join=bevel,line width=0.800pt]
    \path[fill=blue] (0.0000,0.0000) node[above right] (text212) {Frequency ($\text{10}^{\text{-2}}$\hspace*{.2ex}Hz)};



  \end{scope}
  \begin{scope}[cm={{1.0125,0.0,0.0,1.0125,(25.8187,114.412)}},draw=blue,even odd rule,line cap=rect,line join=bevel,line width=0.800pt]
  \end{scope}
  \begin{scope}[scale=1.012,draw=blue,even odd rule,line cap=rect,line join=bevel,line width=0.800pt]
  \end{scope}
  \begin{scope}[scale=1.012,draw=blue,even odd rule,line cap=rect,line join=bevel,line width=0.800pt]
  \end{scope}
  \begin{scope}[scale=1.012,draw=blue,even odd rule,line cap=rect,line join=bevel,line width=0.800pt]
  \end{scope}
  \begin{scope}[cm={{0.72986,0.0,0.0,0.90534,(-13.69026,97.79668)}},draw=blue,even odd rule,line cap=round,line join=round,line width=0.400pt]
    \path[draw] (12.5000,86.2000) -- (12.5000,86.2000) -- (13.6000,86.2000) -- (14.6000,86.4000) -- (15.7000,85.0000) -- (16.7000,86.4000) -- (17.8000,85.4000) -- (18.8000,86.1000) -- (19.9000,85.9000) -- (20.9000,86.2000) -- (22.0000,86.0000) -- (23.1000,86.0000) -- (24.1000,85.4000) -- (25.2000,86.0000) -- (26.2000,86.2000) -- (27.3000,86.2000) -- (28.3000,86.1000) -- (29.4000,85.3000) -- (30.5000,85.5000) -- (31.5000,86.2000) -- (32.6000,86.1000) -- (33.6000,86.4000) -- (34.7000,85.4000) -- (35.7000,84.1000) -- (36.8000,86.1000) -- (37.9000,85.9000) -- (38.9000,86.3000) -- (40.0000,86.0000) -- (41.0000,86.1000) -- (42.1000,84.0000) -- (43.1000,81.0000) -- (44.2000,84.4000) -- (45.2000,84.9000) -- (46.3000,84.6000) -- (47.4000,85.0000) -- (48.4000,83.2000) -- (49.5000,43.3000) -- (50.5000,84.9000) -- (51.6000,83.8000) -- (52.6000,79.7000) -- (53.7000,84.3000) -- (54.8000,83.0000) -- (55.8000,83.4000) -- (56.9000,51.8000) -- (57.9000,86.1000) -- (59.0000,77.0000) -- (60.0000,81.0000) -- (61.1000,79.0000) -- (62.1000,85.1000) -- (62.1000,33.7178);



    \path[draw] (64.3000,33.7516) -- (64.3000,85.1000) -- (65.3000,79.0000) -- (66.4000,81.0000) -- (67.4000,77.0000) -- (68.5000,86.1000) -- (69.5000,51.8000) -- (70.6000,83.4000) -- (71.6000,83.0000) -- (72.7000,84.3000) -- (73.8000,79.7000) -- (74.8000,83.8000) -- (75.9000,84.9000) -- (76.9000,43.3000) -- (78.0000,83.2000) -- (79.0000,85.0000) -- (80.1000,84.6000) -- (81.2000,84.9000) -- (82.2000,84.4000) -- (83.3000,81.0000) -- (84.3000,84.0000) -- (85.4000,86.1000) -- (86.4000,86.0000) -- (87.5000,86.3000) -- (88.6000,85.9000) -- (89.6000,86.1000) -- (90.7000,84.1000) -- (91.7000,85.4000) -- (92.8000,86.4000) -- (93.8000,86.1000) -- (94.9000,86.2000) -- (95.9000,85.5000) -- (97.0000,85.3000) -- (98.1000,86.1000) -- (99.1000,86.2000) -- (100.2000,86.2000) -- (101.2000,86.0000) -- (102.3000,85.4000) -- (103.3000,86.0000) -- (104.4000,86.0000) -- (105.5000,86.2000) -- (106.5000,85.9000) -- (107.6000,86.1000) -- (108.6000,85.4000) -- (109.7000,86.4000) -- (110.7000,85.0000) -- (111.8000,86.4000) -- (112.8000,86.2000) -- (113.9000,86.2000) -- (113.9000,86.2000);



  \end{scope}
  \begin{scope}[scale=1.012,draw=blue,even odd rule,line cap=rect,line join=bevel,line width=0.800pt]
  \end{scope}
  \begin{scope}[scale=1.012,draw=blue,even odd rule,line cap=rect,line join=bevel,line width=0.800pt]
  \end{scope}
  \begin{scope}[scale=1.012,draw=blue,even odd rule,line cap=rect,line join=bevel,line width=0.800pt]
  \end{scope}
  \begin{scope}[draw=blue,even odd rule,line cap=rect,line join=bevel,line width=0.800pt]
  \end{scope}
  \begin{scope}[cm={{1.01288,0.0,0.0,1.0125,(8.01061,0.31012)}},draw=ca0a0a4,dash pattern=on 0.40pt off 0.80pt,even odd rule,line cap=round,line join=round,line width=0.400pt]
  \end{scope}
  \begin{scope}[cm={{1.0125,0.0,0.0,1.0125,(0.00644,0.30976)}},draw=ca0a0a4,dash pattern=on 0.40pt off 0.80pt,even odd rule,line cap=round,line join=round,line width=0.400pt]
  \end{scope}
  \begin{scope}[cm={{0.72986,0.0,0.0,0.90534,(-13.58349,57.33816)}},draw=ca0a0a4,dash pattern=on 0.40pt off 0.80pt,even odd rule,line cap=round,line join=round,line width=0.400pt]
    \path[color=blue,fill=blue,nonzero rule,line cap=round,line join=round,line width=0.502pt,miter limit=4.00] (60.9979,75.6531) .. controls (60.7904,75.7177) and (60.6555,75.8770) .. (60.6966,76.0088) .. controls (60.7376,76.1406) and (60.9392,76.1951) .. (61.1467,76.1304) -- (65.6553,74.7253) .. controls (65.8628,74.6606) and (65.9977,74.5013) .. (65.9566,74.3695) .. controls (65.9156,74.2377) and (65.7140,74.1832) .. (65.5065,74.2479) -- cycle;



    \begin{scope}[cm={{1.50288,-0.46838,0.29754,0.95471,(-111.368,107.84373)}},draw=blue,even odd rule,line cap=rect,line join=miter,line width=0.400pt]
      \path[color=blue,fill=blue,nonzero rule,line cap=rect,line join=miter,line width=0.400pt,miter limit=4.00] (110.3148,22.2500) .. controls (110.1767,22.2500) and (110.0648,22.3619) .. (110.0648,22.5000) .. controls (110.0648,22.6381) and (110.1767,22.7500) .. (110.3148,22.7500) -- (113.3148,22.7500) .. controls (113.4529,22.7500) and (113.5648,22.6381) .. (113.5648,22.5000) .. controls (113.5648,22.3619) and (113.4529,22.2500) .. (113.3148,22.2500) -- cycle;



    \end{scope}
    \path[color=blue,fill=blue,nonzero rule,line cap=round,line join=round,line width=0.502pt,miter limit=4.00] (10.1034,75.7781) .. controls (9.8959,75.8427) and (9.7610,76.0020) .. (9.8020,76.1338) .. controls (9.8431,76.2656) and (10.0447,76.3201) .. (10.2522,76.2554) -- (14.7608,74.8503) .. controls (14.9683,74.7856) and (15.1032,74.6263) .. (15.0621,74.4945) .. controls (15.0210,74.3627) and (14.8195,74.3082) .. (14.6120,74.3729) -- cycle;



    \begin{scope}[cm={{1.50288,-0.46838,0.29754,0.95471,(-162.26251,107.96873)}},draw=blue,even odd rule,line cap=rect,line join=miter,line width=0.400pt]
      \path[color=blue,fill=blue,nonzero rule,line cap=rect,line join=miter,line width=0.400pt,miter limit=4.00] (110.3148,22.2500) .. controls (110.1767,22.2500) and (110.0648,22.3619) .. (110.0648,22.5000) .. controls (110.0648,22.6381) and (110.1767,22.7500) .. (110.3148,22.7500) -- (113.3148,22.7500) .. controls (113.4529,22.7500) and (113.5648,22.6381) .. (113.5648,22.5000) .. controls (113.5648,22.3619) and (113.4529,22.2500) .. (113.3148,22.2500) -- cycle;



    \end{scope}
  \end{scope}
  \begin{scope}[cm={{0.72986,0.0,0.0,0.90534,(-84.81047,98.07366)}},draw=blue,even odd rule,line cap=round,line join=round,line width=0.400pt]
    \path[color=blue,fill=blue,nonzero rule,line cap=butt,line join=miter,line width=0.400pt,miter limit=4.00] (110.3148,22.2500) .. controls (110.1767,22.2500) and (110.0648,22.3619) .. (110.0648,22.5000) .. controls (110.0648,22.6381) and (110.1767,22.7500) .. (110.3148,22.7500) -- (113.3148,22.7500) .. controls (113.4529,22.7500) and (113.5648,22.6381) .. (113.5648,22.5000) .. controls (113.5648,22.3619) and (113.4529,22.2500) .. (113.3148,22.2500) -- cycle;



  \end{scope}
  \begin{scope}[cm={{0.72986,0.0,0.0,0.90534,(-13.57921,98.26368)}},draw=blue,even odd rule,line cap=round,line join=round,line width=0.400pt]
    \path[color=blue,fill=blue,nonzero rule,line cap=butt,line join=miter,line width=0.400pt,miter limit=4.00] (113.5824,29.1349) -- (113.5648,16.5000) .. controls (113.5648,16.3619) and (113.4529,16.2500) .. (113.3148,16.2500) -- (12.5137,16.2500) -- (12.5000,16.2500) .. controls (12.4862,16.2502) and (12.4725,16.2515) .. (12.4590,16.2539) .. controls (12.4451,16.2560) and (12.4313,16.2593) .. (12.4180,16.2637) .. controls (12.3781,16.2780) and (12.3424,16.3022) .. (12.3145,16.3340) .. controls (12.2733,16.3796) and (12.2504,16.4387) .. (12.2500,16.5000) -- (12.2487,29.1320) -- (12.9297,29.1333) -- (12.8979,16.8982) -- (112.9167,16.8982) -- (112.9137,29.1371);



  \end{scope}
  \begin{scope}[cm={{0.73013,0.0,0.0,0.90534,(-7.91583,98.07398)}},draw=blue,even odd rule,line cap=round,line join=round,line width=0.400pt]
    \path[color=blue,fill=blue,nonzero rule,line cap=butt,line join=miter,line width=0.400pt,miter limit=4.00] (56.0430,29.3507) .. controls (56.0178,27.6026) and (55.9492,24.1035) .. (55.9492,24.1035) -- (55.9492,24.0957) .. controls (55.9462,23.9637) and (55.8410,23.8567) .. (55.7090,23.8516) .. controls (55.5694,23.8461) and (55.4524,23.9561) .. (55.4492,24.0957) -- (55.4492,24.1035) -- (55.3661,29.3497);



  \end{scope}
  \begin{scope}[cm={{1.09689,-0.42404,0.21716,0.86433,(-58.45742,153.66011)}},draw=blue,even odd rule,line cap=round,line join=round,line width=0.400pt]
    \path[color=blue,fill=blue,nonzero rule,line cap=round,line join=round,line width=0.400pt,miter limit=4.00] (110.3148,22.2500) .. controls (110.1767,22.2500) and (110.0648,22.3619) .. (110.0648,22.5000) .. controls (110.0648,22.6381) and (110.1767,22.7500) .. (110.3148,22.7500) -- (113.3148,22.7500) .. controls (113.4529,22.7500) and (113.5648,22.6381) .. (113.5648,22.5000) .. controls (113.5648,22.3619) and (113.4529,22.2500) .. (113.3148,22.2500) -- cycle;



  \end{scope}
  \begin{scope}[cm={{1.09689,-0.42404,0.21716,0.86433,(-58.42517,155.25665)}},draw=blue,even odd rule,line cap=rect,line join=miter,line width=0.400pt]
    \path[color=blue,fill=blue,nonzero rule,line cap=rect,line join=miter,line width=0.400pt,miter limit=4.00] (110.3148,22.2500) .. controls (110.1767,22.2500) and (110.0648,22.3619) .. (110.0648,22.5000) .. controls (110.0648,22.6381) and (110.1767,22.7500) .. (110.3148,22.7500) -- (113.3148,22.7500) .. controls (113.4529,22.7500) and (113.5648,22.6381) .. (113.5648,22.5000) .. controls (113.5648,22.3619) and (113.4529,22.2500) .. (113.3148,22.2500) -- cycle;



  \end{scope}
  \begin{scope}[cm={{-0.72986,0.0,0.0,-0.90534,(78.24495,191.04635)}},draw=blue,even odd rule,line cap=round,line join=round,line width=0.400pt]
    \path[color=blue,fill=blue,nonzero rule,line cap=butt,line join=miter,line width=0.400pt,miter limit=4.00] (113.5824,69.1349) -- (113.5648,16.5000) .. controls (113.5648,16.3619) and (113.4529,16.2500) .. (113.3148,16.2500) -- (12.4393,16.2500) -- (12.4256,16.2500) .. controls (12.4119,16.2502) and (12.3981,16.2515) .. (12.3846,16.2539) .. controls (12.3707,16.2560) and (12.3570,16.2593) .. (12.3436,16.2637) .. controls (12.3037,16.2780) and (12.2681,16.3022) .. (12.2401,16.3340) .. controls (12.1989,16.3796) and (12.1760,16.4387) .. (12.1756,16.5000) -- (12.1743,69.1320) -- (12.8558,69.1333) -- (12.8239,16.8978) -- (112.9165,16.8978) -- (112.9135,69.1371);



  \end{scope}
  \begin{scope}[cm={{-0.72986,0.0,0.0,-0.90534,(44.26819,215.19002)}},draw=blue,even odd rule,line cap=round,line join=round,line width=0.400pt]
  \end{scope}
  \path[draw=blue,line cap=butt,line join=miter,line width=0.480pt,miter limit=3.10] (-59.8947,181.4231) -- (-59.8947,211.5052) -- (-4.7120,211.5052);



  \path[draw=blue,dash pattern=on 0.78pt off 0.78pt,line cap=butt,line join=miter,line width=0.195pt,miter limit=4.00] (-59.8459,199.8557) -- (-4.5057,199.8557);



  \begin{scope}[cm={{0.59309,0.0,0.0,0.73569,(-125.42512,161.7328)}},draw=blue,even odd rule,line cap=round,line join=round,line width=0.400pt]
    \path[draw] (113.5000,48.5000) -- (113.5000,48.5000) -- (110.5000,48.5000);



  \end{scope}
  \begin{scope}[cm={{0.59309,0.0,0.0,0.73569,(-125.45425,155.31348)}},draw=blue,even odd rule,line cap=round,line join=round,line width=0.400pt]
    \path[draw] (113.5000,48.5000) -- (113.5000,48.5000) -- (110.5000,48.5000);



  \end{scope}
  \begin{scope}[cm={{0.59309,0.0,0.0,0.73569,(-125.38661,173.01344)}},draw=blue,even odd rule,line cap=round,line join=round,line width=0.400pt]
    \path[draw] (113.5000,48.5000) -- (113.5000,48.5000) -- (110.5000,48.5000);



  \end{scope}
  \begin{scope}[cm={{0.59309,0.0,0.0,0.73569,(-125.41574,166.59411)}},draw=blue,even odd rule,line cap=round,line join=round,line width=0.400pt]
    \path[draw] (113.5000,48.5000) -- (113.5000,48.5000) -- (110.5000,48.5000);



  \end{scope}
  \begin{scope}[cm={{1.0125,0.0,0.0,1.0125,(-16.39995,214.32288)}},draw=blue,even odd rule,line cap=rect,line join=bevel,line width=0.800pt]
    \path[fill=blue] (14.8148,-0.0000) node[above right] (text212-0) {$t$};



    \path[fill=blue,even odd rule] (-58.3353,-18.0869) node[above right] (text212-0-9) {plan};



    \path[fill=blue,even odd rule] (-69.1312,-7.6393) node[above right] (text212-0-9-9) {schedule};



  \end{scope}
  \path[draw=blue,dash pattern=on 0.78pt off 0.78pt,line cap=butt,line join=miter,line width=0.195pt,miter limit=4.00] (-48.1864,211.6225) -- (-48.1864,187.9448);



  \path[draw=blue,dash pattern=on 0.78pt off 0.78pt,line cap=butt,line join=miter,line width=0.195pt,miter limit=4.00] (-36.6188,211.6649) -- (-36.6188,187.9982);



  \path[draw=blue,dash pattern=on 0.78pt off 0.78pt,line cap=butt,line join=miter,line width=0.195pt,miter limit=4.00] (-24.7810,211.6124) -- (-24.7810,187.7436);



  \path[fill=blue,line cap=butt,line join=miter,line width=0.600pt] (-7.0639,210.2684) -- (-2.7369,211.5065) -- (-7.0457,212.7303) -- (-5.8404,211.5174) -- cycle;



  \path[fill=blue,line cap=butt,line join=miter,line width=0.600pt] (-61.1641,183.5851) -- (-59.9260,179.2581) -- (-58.7022,183.5669) -- (-59.9151,182.3616) -- cycle;



  \path[fill=cffffff,line cap=butt,line join=round,line width=0.459pt,miter limit=4.00,rounded corners=0.0000cm] (-4.5353,28.4336) rectangle (69.1630,91.5565);



\end{scope}
\path[fill=cffffff,line join=round,line width=0.160pt] (152.8873,136.6550) -- (150.4903,138.3440) -- (148.5183,141.5750) -- (149.6933,142.8750) -- (154.4343,139.8540) -- (156.5733,137.3370) .. controls (158.5763,134.7590) and (158.1783,133.2570) .. (158.1783,133.2570) -- (158.1863,132.2400) -- (148.0543,109.7100) -- (148.7153,107.7650) -- (149.3023,104.9640) .. controls (149.3243,104.6390) and (148.7703,103.7500) .. (148.6313,103.3910) .. controls (148.4923,103.0310) and (146.7163,102.2060) .. (145.8413,102.8140) .. controls (144.9663,103.4220) and (144.4223,103.8430) .. (144.4223,103.8430) .. controls (144.4223,103.8430) and (143.0423,105.3660) .. (142.3143,106.8230) -- (131.0593,108.9150) .. controls (131.0593,108.9150) and (130.1063,109.2030) .. (129.7543,109.4740) -- (125.3393,113.4030) -- (124.6853,117.4220) -- (129.1743,114.7150) .. controls (129.1743,114.7150) and (130.1173,114.2930) .. (130.3913,114.4010) -- (136.8803,116.8390) -- (136.7613,118.8990) -- (138.8453,118.7340) -- (139.1913,119.1630) -- (138.6953,119.9810) -- (138.6433,120.0960) .. controls (138.5543,120.1300) and (138.5723,120.3620) .. (138.5723,120.3620) .. controls (138.5723,120.3620) and (138.5633,121.0460) .. (138.9003,121.8140) .. controls (138.9003,121.8140) and (138.7043,122.4650) .. (139.0893,122.9150) -- (139.4353,123.4030) -- (140.5363,123.7170) .. controls (140.5363,123.7170) and (141.4433,123.7870) .. (142.4463,123.8020) .. controls (142.4463,123.8020) and (142.0943,122.8210) .. (142.4243,122.4430) -- (143.2583,121.5780) -- (143.5733,121.8770) -- cycle;



\begin{scope}[shift={(0.22227,-34.242)}]
  \path[fill=cffffff,line join=round,line width=0.160pt] (153.9360,170.4430) -- (154.1700,170.9620) -- (150.8760,173.6250) -- (149.6610,175.8210) -- (153.6070,172.8400) -- (155.6950,170.2690) .. controls (157.5940,167.8250) and (156.7810,166.7600) .. (156.7810,166.7600) -- (156.6860,166.5500) -- (146.7280,143.9250) -- (147.4520,141.6930) -- (147.8090,139.7450) .. controls (147.8300,139.4370) and (147.6500,138.6450) .. (147.5180,138.3040) .. controls (147.3860,137.9630) and (146.7620,137.6850) .. (145.9320,138.2620) .. controls (145.1030,138.8390) and (144.4160,139.8080) .. (144.4160,139.8080) .. controls (144.4160,139.8080) and (143.4420,140.9950) .. (142.7510,142.3770) -- (131.3690,144.0330) .. controls (131.3690,144.0330) and (131.0100,144.0620) .. (130.6760,144.3190) -- (125.8370,147.9960) -- (125.5290,150.0450) -- (129.0120,148.0790) .. controls (129.0120,148.0790) and (129.7990,147.7280) .. (130.0590,147.8300) -- (137.5740,150.6490) -- (137.2940,152.2960) -- (140.0800,151.8150) -- (140.2500,153.2340) -- (139.7160,154.7630) -- (139.7050,154.7990) .. controls (139.6610,154.7070) and (138.9050,154.6060) .. (138.9050,154.6060) .. controls (138.9050,154.6060) and (139.2350,154.9590) .. (139.5550,155.6870) .. controls (139.5550,155.6870) and (139.4660,156.4750) .. (139.8300,156.9020) -- (140.1070,156.8990) -- (140.3960,156.5800) .. controls (140.3960,156.5800) and (140.7980,157.6650) .. (141.7490,157.6790) .. controls (141.7490,157.6790) and (141.4920,157.5280) .. (141.0070,156.5710) -- (140.6930,156.3190) .. controls (141.1700,155.4770) and (141.4400,155.0200) .. (141.4400,155.0200) -- (141.3020,155.9040) -- (143.3300,154.7340) -- (144.1440,155.3980) -- cycle;



  \path[draw=blue,even odd rule,line join=round,line width=0.256pt] (154.1860,170.9400) -- (153.9450,170.4360) -- (144.1410,155.4030) -- (143.3520,154.7040) -- (141.2910,155.8960) -- (141.6720,153.7540) -- (143.6340,151.1200) -- (143.3590,154.7040);



  \path[draw=blue,line join=round,line width=0.256pt] (140.2550,147.9460) -- (137.6340,150.3750) -- (137.2930,152.2970) -- (140.0790,151.7850) -- (140.2420,147.9530) .. controls (140.2420,147.9530) and (140.5860,146.7570) .. (141.2720,145.2720);



  \path[draw=blue,line join=round,line width=0.256pt] (142.7720,142.3620) .. controls (142.7720,142.3620) and (139.3480,147.2660) .. (140.2480,153.2140) -- (139.7350,154.7100) .. controls (139.7350,154.7100) and (139.2370,156.0470) .. (139.8390,156.8930) -- (140.0960,156.8810) -- (140.3870,156.5660);



  \path[draw=blue,line join=round,line width=0.256pt] (141.4490,155.0020) .. controls (141.4490,155.0020) and (141.1790,155.4580) .. (140.7130,156.3010);



  \path[draw=blue,line join=round,line width=0.256pt] (138.9140,154.5870) .. controls (138.9140,154.5870) and (139.6700,154.6880) .. (139.7050,154.7970);



  \path[draw=blue,line join=round,line width=0.256pt] (138.9000,154.5980) .. controls (138.9000,154.5980) and (139.0840,154.7030) .. (139.5370,155.7210);



  \path[draw=blue,line join=round,line width=0.256pt] (140.3140,156.2830) -- (140.5530,156.2030) -- (141.0250,156.5420) .. controls (141.0250,156.5420) and (141.4960,157.5180) .. (141.7240,157.6250) .. controls (141.9510,157.7330) and (140.7720,157.8120) .. (140.3140,156.2830) -- cycle;



  \path[draw=blue,line join=round,line width=0.256pt] (156.6470,168.7510) .. controls (156.6470,168.7510) and (156.4290,169.5750) .. (153.6380,172.8140) -- (149.6550,175.8230) -- (150.9010,173.6330) .. controls (150.9640,173.6120) and (154.1560,170.9870) .. (155.9020,169.5950) .. controls (157.6490,168.2030) and (156.7010,166.5250) .. (156.7010,166.5250) -- (146.7430,143.8990) -- (147.4670,141.6680) -- (147.8240,139.7200) .. controls (147.8450,139.4110) and (147.6650,138.6190) .. (147.5330,138.2780) .. controls (147.4010,137.9380) and (146.7770,137.6600) .. (145.9470,138.2370) .. controls (145.1180,138.8130) and (144.4310,139.7830) .. (144.4310,139.7830) .. controls (144.4310,139.7830) and (143.4570,140.9700) .. (142.7660,142.3510) -- (131.3840,144.0080) .. controls (131.3840,144.0080) and (131.0250,144.0360) .. (130.6910,144.2930) -- (125.8530,147.9710) -- (125.5440,150.0200) -- (129.0270,148.0540) .. controls (129.0270,148.0540) and (129.8150,147.7020) .. (130.0740,147.8040) -- (137.5890,150.6240);



  \path[draw=blue,line join=round,line width=0.256pt] (144.4090,139.8110) -- (144.5610,139.8830) -- (144.7460,139.6540) .. controls (144.7460,139.6540) and (144.9040,139.4160) .. (145.1420,139.9120) -- (144.9620,140.2360) .. controls (144.9620,140.2360) and (146.2130,140.5820) .. (147.7900,139.8000);



  \path[draw=blue,line join=round,line width=0.256pt] (144.1080,140.4590) .. controls (144.1080,140.4590) and (144.6900,140.4770) .. (144.8620,140.7410) -- (143.7960,142.3040) .. controls (143.7960,142.3040) and (143.2830,142.2680) .. (143.0390,142.0060) -- cycle;



  \path[draw=blue,line join=round,line width=0.256pt] (144.1300,140.4560) -- (143.8390,141.7070) -- (143.0630,142.0060);



  \path[draw=blue,line join=round,line width=0.256pt] (143.8480,141.7110) -- (144.1190,141.8120);



  \path[draw=blue,line join=round,line width=0.256pt] (142.0640,143.6150) -- (142.5020,143.6190) .. controls (142.5020,143.6190) and (143.3980,144.1050) .. (143.5220,144.8230) .. controls (143.6470,145.5410) and (143.3140,146.7850) .. (143.3140,146.7850) .. controls (143.3140,146.7850) and (141.9740,151.5030) .. (141.9030,151.6880) .. controls (141.9030,151.6880) and (141.7350,152.2910) .. (141.0500,152.8530) .. controls (141.0500,152.8530) and (140.4400,153.2920) .. (140.2440,152.7870) -- (140.1550,152.4360);



  \path[draw=blue,line join=round,line width=0.256pt] (130.6250,148.0030) -- (130.4930,146.9970) -- (134.6290,148.1780) -- (134.8480,149.5970);



  \path[draw=blue,line join=round,line width=0.256pt] (148.1890,161.5910) -- (148.8110,160.3220) -- (154.1440,168.3190) -- (153.3080,169.4630);



  \path[draw=blue,line join=round,line width=0.256pt] (147.1850,145.0060) .. controls (147.1850,145.0060) and (147.0210,144.8870) .. (146.9350,144.9550) .. controls (146.8490,145.0230) and (146.2560,145.2530) .. (145.1540,147.7720) .. controls (144.0510,150.2900) and (143.6620,151.0850) .. (143.6620,151.0850) -- (143.6370,151.1280);



  \path[draw=blue,line join=round,line width=0.256pt] (147.2220,144.9990) .. controls (147.2220,144.9990) and (147.0450,144.9420) .. (146.9360,144.9830) .. controls (146.8480,145.0160) and (146.2570,145.2810) .. (145.1540,147.8000) .. controls (144.0520,150.3180) and (143.6630,151.1130) .. (143.6630,151.1130) -- (143.6380,151.1560);



  \path[draw=blue,line join=round,line width=0.256pt] (131.1400,144.0090) -- (131.1250,144.0670);



  \path[draw=blue,line join=round,line width=0.256pt] (130.2540,147.8750) .. controls (130.2540,147.8750) and (130.0220,145.8200) .. (130.2170,145.3400) .. controls (130.4120,144.8590) and (130.6640,144.5390) .. (130.6640,144.5390) -- (130.9400,144.2320) .. controls (130.9400,144.2320) and (131.0180,144.1320) .. (131.1870,144.0660);



  \path[draw=blue,line join=round,line width=0.256pt] (156.6330,166.5170) .. controls (156.6330,166.5170) and (155.7490,167.2260) .. (155.2760,167.9840) .. controls (154.8040,168.7420) and (154.0290,170.4020) .. (154.0290,170.4020) -- (153.9880,170.4900);



  \path[draw=blue,line join=round,line width=0.256pt] (130.7400,144.9350) -- (136.8400,144.9360);



  \path[draw=blue,line join=round,line width=0.256pt] (155.2510,167.3650) -- (148.9180,154.8790);



\end{scope}
\path (153.0253,107.4016) node[above right] (text192) {\textlabel{ii}{sth:ii}};



\path[fill=cffffff,line join=round,line width=0.160pt] (68.7969,141.6320) -- (67.3024,139.0340) -- (66.8297,135.2590) -- (68.4370,134.6400) -- (71.5245,139.4960) -- (72.4437,142.7130) .. controls (73.2128,145.9210) and (72.2320,147.0750) .. (72.2320,147.0750) -- (71.8209,147.9830) -- (53.3088,163.3790) -- (53.1129,165.4110) -- (52.4981,168.1700) .. controls (52.3846,168.4690) and (51.5130,169.0060) .. (51.2385,169.2620) .. controls (50.9640,169.5180) and (49.0026,169.4390) .. (48.4537,168.4980) .. controls (47.9049,167.5570) and (47.5808,166.9330) .. (47.5808,166.9330) .. controls (47.5808,166.9330) and (46.9466,164.9480) .. (46.8808,163.3190) -- (37.4642,156.3090) .. controls (37.4642,156.3090) and (36.7115,155.6170) .. (36.5013,155.2150) -- (34.0846,149.7010) -- (35.1396,145.8280) -- (38.1265,150.2890) .. controls (38.1265,150.2890) and (38.8144,151.0950) .. (39.1082,151.1250) -- (46.0360,151.9270) -- (46.7735,150.0410) -- (48.6090,151.1410) -- (49.1016,150.9180) -- (48.9849,149.9640) -- (48.9844,149.8370) .. controls (48.9170,149.7660) and (49.0294,149.5690) .. (49.0294,149.5690) .. controls (49.0294,149.5690) and (49.3015,148.9570) .. (49.9250,148.4280) .. controls (49.9250,148.4280) and (50.0140,147.7590) .. (50.5500,147.5350) -- (51.0666,147.2600) -- (52.2016,147.4840) .. controls (52.2016,147.4840) and (53.0581,147.8370) .. (53.9801,148.2830) .. controls (53.9801,148.2830) and (53.2552,148.9940) .. (53.4017,149.4810) -- (53.8075,150.6320) -- (54.2177,150.5100) -- cycle;



\begin{scope}[shift={(0.22227,-34.242)}]
  \path[fill=cffffff,line join=round,line width=0.160pt] (69.5489,176.8600) -- (69.9756,176.5060) -- (68.0626,172.6300) -- (67.8560,170.1220) -- (70.2329,174.5780) -- (71.0838,177.8190) .. controls (71.8132,180.8620) and (70.6334,181.4370) .. (70.6334,181.4370) -- (70.4597,181.5800) -- (52.0671,197.1400) -- (51.8115,199.4560) -- (51.3366,201.3510) .. controls (51.2289,201.6350) and (50.7386,202.2570) .. (50.4783,202.4990) .. controls (50.2180,202.7420) and (49.5337,202.7030) .. (49.0132,201.8110) .. controls (48.4928,200.9180) and (48.2645,199.7420) .. (48.2645,199.7420) .. controls (48.2645,199.7420) and (47.8622,198.2410) .. (47.7998,196.6970) -- (38.0870,190.0150) .. controls (38.0870,190.0150) and (37.7708,189.8260) .. (37.5714,189.4440) -- (34.6650,183.9600) -- (35.2253,181.9970) -- (37.5978,185.3390) .. controls (37.5978,185.3390) and (38.1723,186.0120) .. (38.4509,186.0400) -- (46.4722,186.9720) -- (46.8932,185.3800) -- (49.2397,187.0820) -- (49.9785,185.8990) -- (50.1196,184.2950) -- (50.1241,184.2580) .. controls (50.0457,184.3190) and (49.3137,184.0630) .. (49.3137,184.0630) .. controls (49.3137,184.0630) and (49.7606,183.9010) .. (50.3518,183.3990) .. controls (50.3518,183.3990) and (50.5943,182.6580) .. (51.1026,182.4450) -- (51.3542,182.5750) -- (51.4866,182.9900) .. controls (51.4866,182.9900) and (52.3000,182.2100) .. (53.1742,182.6330) .. controls (53.1742,182.6330) and (52.8777,182.6490) .. (52.0412,183.2780) -- (51.6504,183.3580) .. controls (51.7401,184.3250) and (51.7990,184.8550) .. (51.7990,184.8550) -- (52.0363,184.0060) -- (53.4071,185.9740) -- (54.4231,185.7560) -- cycle;



  \path[draw=blue,even odd rule,line join=round,line width=0.256pt] (69.9818,176.5330) -- (69.5547,176.8700) -- (54.4233,185.7500) -- (53.4148,186.0110) -- (52.0234,184.0080) -- (51.4907,186.0870) -- (52.1992,189.3270) -- (53.4213,186.0140);



  \path[draw=blue,line join=round,line width=0.256pt] (47.8098,190.6020) -- (46.4145,187.2430) -- (46.8931,185.3780) -- (49.2265,187.1090) -- (47.8005,190.5910) .. controls (47.8005,190.5910) and (47.6231,191.8110) .. (47.6388,193.4460);



  \path[draw=blue,line join=round,line width=0.256pt] (47.8128,196.7190) .. controls (47.8128,196.7190) and (46.7022,190.7920) .. (49.9683,185.9150) -- (50.1152,184.3510) .. controls (50.1152,184.3510) and (50.2095,182.9340) .. (51.1066,182.4570) -- (51.3372,182.5850) -- (51.4730,182.9990);



  \path[draw=blue,line join=round,line width=0.256pt] (51.7997,184.8750) .. controls (51.7997,184.8750) and (51.7408,184.3460) .. (51.6621,183.3840);



  \path[draw=blue,line join=round,line width=0.256pt] (49.3145,184.0840) .. controls (49.3145,184.0840) and (50.0464,184.3400) .. (50.1230,184.2600);



  \path[draw=blue,line join=round,line width=0.256pt] (49.3064,184.0680) .. controls (49.3064,184.0680) and (49.5178,184.0590) .. (50.3494,183.3610);



  \path[draw=blue,line join=round,line width=0.256pt] (51.2902,183.2170) -- (51.4755,183.3970) -- (52.0458,183.3110) .. controls (52.0458,183.3110) and (52.8772,182.6600) .. (53.1292,182.6680) .. controls (53.3810,182.6770) and (52.3369,182.0670) .. (51.2902,183.2170) -- cycle;



  \path[draw=blue,line join=round,line width=0.256pt] (71.3289,179.6050) .. controls (71.3289,179.6050) and (71.4681,178.7730) .. (70.2507,174.6150) -- (67.8510,170.1170) -- (68.0882,172.6350) .. controls (68.1368,172.6820) and (69.9736,176.4770) .. (70.9960,178.5140) .. controls (72.0183,180.5500) and (70.4631,181.6090) .. (70.4631,181.6090) -- (52.0706,197.1690) -- (51.8149,199.4850) -- (51.3401,201.3810) .. controls (51.2324,201.6650) and (50.7420,202.2860) .. (50.4818,202.5290) .. controls (50.2215,202.7720) and (49.5371,202.7330) .. (49.0167,201.8410) .. controls (48.4962,200.9480) and (48.2679,199.7720) .. (48.2679,199.7720) .. controls (48.2679,199.7720) and (47.8657,198.2710) .. (47.8032,196.7260) -- (38.0905,190.0450) .. controls (38.0905,190.0450) and (37.7742,189.8550) .. (37.5748,189.4740) -- (34.6684,183.9900) -- (35.2288,182.0270) -- (37.6012,185.3690) .. controls (37.6012,185.3690) and (38.1757,186.0420) .. (38.4543,186.0700) -- (46.4756,187.0020);



  \path[draw=blue,line join=round,line width=0.256pt] (48.2588,199.7370) -- (48.4268,199.7420) -- (48.5023,200.0310) .. controls (48.5023,200.0310) and (48.5486,200.3140) .. (48.9695,199.9820) -- (48.9386,199.6120) .. controls (48.9386,199.6120) and (50.2237,199.8770) .. (51.3420,201.2930);



  \path[draw=blue,line join=round,line width=0.256pt] (48.2506,199.0230) .. controls (48.2506,199.0230) and (48.7894,199.2740) .. (49.0545,199.1170) -- (48.7239,197.2390) .. controls (48.7239,197.2390) and (48.2410,197.0370) .. (47.9098,197.1580) -- cycle;



  \path[draw=blue,line join=round,line width=0.256pt] (48.2688,199.0360) -- (48.5176,197.7900) -- (47.9324,197.1690);



  \path[draw=blue,line join=round,line width=0.256pt] (48.5281,197.7910) -- (48.8168,197.8250);



  \path[draw=blue,line join=round,line width=0.256pt] (47.6810,195.2810) -- (48.0825,195.4780) .. controls (48.0826,195.4780) and (49.1005,195.4560) .. (49.5093,194.8740) .. controls (49.9182,194.2930) and (50.1253,193.0340) .. (50.1253,193.0340) .. controls (50.1253,193.0340) and (50.8412,188.2260) .. (50.8520,188.0300) .. controls (50.8520,188.0300) and (50.9466,187.4170) .. (50.5519,186.6030) .. controls (50.5520,186.6030) and (50.1759,185.9340) .. (49.7894,186.2940) -- (49.5634,186.5640);



  \path[draw=blue,line join=round,line width=0.256pt] (39.0393,186.1450) -- (38.5051,186.9790) -- (42.7678,187.8220) -- (43.5504,186.6600);



  \path[draw=blue,line join=round,line width=0.256pt] (60.6623,182.1010) -- (60.7092,183.5140) -- (68.8658,178.8440) -- (68.5720,177.4440);



  \path[draw=blue,line join=round,line width=0.256pt] (52.9291,196.3880) .. controls (52.9291,196.3880) and (52.7305,196.4190) .. (52.6799,196.3190) .. controls (52.6293,196.2190) and (52.1827,195.7430) .. (52.2108,192.9990) .. controls (52.2389,190.2550) and (52.2109,189.3700) .. (52.2109,189.3700) -- (52.2053,189.3210);



  \path[draw=blue,line join=round,line width=0.256pt] (52.9604,196.4110) .. controls (52.9604,196.4110) and (52.7747,196.3810) .. (52.6923,196.2940) .. controls (52.6255,196.2240) and (52.1951,195.7180) .. (52.2232,192.9750) .. controls (52.2513,190.2310) and (52.2233,189.3460) .. (52.2233,189.3460) -- (52.2177,189.2960);



  \path[draw=blue,line join=round,line width=0.256pt] (37.8678,189.9310) -- (37.8782,189.8730);



  \path[draw=blue,line join=round,line width=0.256pt] (38.6479,186.0890) .. controls (38.6479,186.0890) and (37.5913,187.8090) .. (37.5721,188.3260) .. controls (37.5528,188.8430) and (37.6509,189.2420) .. (37.6509,189.2420) -- (37.7770,189.6430) .. controls (37.7770,189.6430) and (37.8072,189.7670) .. (37.9341,189.9020);



  \path[draw=blue,line join=round,line width=0.256pt] (70.3978,181.5850) .. controls (70.3978,181.5850) and (69.8816,180.5500) .. (69.7620,179.6600) .. controls (69.6424,178.7700) and (69.6170,176.9400) .. (69.6170,176.9400) -- (69.6154,176.8420);



  \path[draw=blue,line join=round,line width=0.256pt] (37.8838,188.9260) -- (43.4541,191.7160);



  \path[draw=blue,line join=round,line width=0.256pt] (69.4845,180.1980) -- (58.5695,188.4030);



\end{scope}
\path (59.2223,166.7580) node[above right] (text246) {\textlabel{iii}{sth:iii}};



\path[fill=cffffff,line join=round,line width=0.160pt] (136.4133,15.9220) -- (136.8433,15.9115) -- (138.1613,13.5194) -- (141.6433,11.6314) -- (140.0003,16.3625) -- (138.7983,17.7340) .. controls (136.7013,19.5573) and (135.7383,19.9433) .. (135.7383,19.9433) -- (134.0113,20.1909) -- (102.6663,17.6793) -- (100.5263,18.5184) -- (98.0460,19.0954) .. controls (97.6771,19.1676) and (95.6481,19.0644) .. (95.1668,19.0324) .. controls (94.6856,19.0002) and (93.8157,17.3783) .. (94.1285,16.6725) .. controls (94.4413,15.9665) and (94.7315,15.7652) .. (94.7315,15.7652) .. controls (94.7315,15.7652) and (95.4198,14.3585) .. (96.7874,13.6046) -- (93.9240,6.6190) .. controls (93.9240,6.6190) and (93.6416,5.7435) .. (93.7980,5.4548) -- (96.4412,0.0111) -- (99.1788,-0.4293) -- (101.6643,-0.8070) .. controls (101.6643,-0.8070) and (100.7553,3.2723) .. (101.0043,3.4408) -- (106.6973,7.1549) -- (108.9633,6.9660) -- (109.5923,9.3731) -- (110.0493,9.6406) -- (110.6783,9.3574) -- (110.7253,9.2001) .. controls (110.5913,9.1850) and (110.6863,8.5548) .. (110.6863,8.5548) .. controls (110.6863,8.5548) and (112.2263,8.5058) .. (112.9283,8.6023) .. controls (112.9283,8.6023) and (113.2923,8.5907) .. (113.9903,8.7751) -- (115.2223,8.8686) -- (115.3983,9.4203) .. controls (115.3983,9.4203) and (116.8473,10.0103) .. (117.3173,10.6947) .. controls (117.3173,10.6947) and (116.6623,11.2506) .. (115.2563,11.0769) -- (114.9523,11.5148) -- (115.5413,11.8423) -- cycle;



\begin{scope}[shift={(0.22227,-34.242)}]
  \path[fill=cffffff,line join=round,line width=0.160pt] (136.2700,50.8716) -- (136.9800,50.9409) -- (138.5940,48.2163) -- (140.6030,46.9987) -- (138.9130,50.2256) -- (136.8640,52.1079) .. controls (134.8750,53.8380) and (133.2660,53.4677) .. (133.2660,53.4677) -- (132.9780,53.4389) -- (102.1120,50.5805) -- (99.8395,51.4685) -- (97.7313,52.0545) .. controls (97.3812,52.1230) and (96.3766,52.1382) .. (95.9200,52.1078) .. controls (95.4633,52.0774) and (94.8573,51.6980) .. (95.1541,51.0282) .. controls (95.4510,50.3584) and (96.2701,49.7180) .. (96.2701,49.7180) .. controls (96.2701,49.7180) and (97.2118,48.8418) .. (98.5094,48.1265) -- (95.2907,40.0344) .. controls (95.2907,40.0344) and (95.1614,39.7832) .. (95.3098,39.5092) -- (97.4065,35.5480) -- (99.6542,34.9772) -- (98.9388,37.7094) .. controls (98.9388,37.7094) and (98.8855,38.3107) .. (99.1214,38.4705) -- (105.8060,43.1272) -- (107.5980,42.6466) -- (108.2980,44.6406) -- (110.0280,44.5087) -- (111.5690,43.8745) -- (111.6050,43.8606) .. controls (111.4780,43.8462) and (111.0180,43.3457) .. (111.0180,43.3457) .. controls (111.0180,43.3457) and (111.5790,43.5105) .. (112.5720,43.6018) .. controls (112.5720,43.6018) and (113.4500,43.4026) .. (114.1120,43.5777) -- (114.2340,43.7680) -- (113.9930,44.0216) .. controls (113.9930,44.0216) and (115.4390,44.1072) .. (115.8850,44.7566) .. controls (115.8850,44.7566) and (115.5930,44.6070) .. (114.2590,44.4421) -- (113.8230,44.2709) .. controls (113.0580,44.7455) and (112.6480,45.0107) .. (112.6480,45.0107) -- (113.6150,44.7611) -- (113.1690,46.3562) -- (114.3110,46.7976) -- cycle;



  \path[draw=blue,even odd rule,line join=round,line width=0.256pt] (136.9620,50.9560) -- (136.2660,50.8791) -- (114.3160,46.7950) -- (113.1440,46.3764) -- (113.6010,44.7553) -- (111.2770,45.3921) -- (109.0950,47.1983) -- (113.1470,46.3815);



  \path[draw=blue,line join=round,line width=0.256pt] (103.8700,45.4389) -- (105.5140,43.2165) -- (107.5990,42.6457) -- (108.2620,44.6450) -- (103.8710,45.4289) .. controls (103.8710,45.4289) and (102.6340,45.8742) .. (101.2130,46.6048);



  \path[draw=blue,line join=round,line width=0.256pt] (98.5015,48.1432) .. controls (98.5015,48.1432) and (102.6670,44.9365) .. (110.0040,44.5103) -- (111.5150,43.8967) .. controls (111.5150,43.8967) and (112.8470,43.3208) .. (114.1050,43.5847) -- (114.2080,43.7636) -- (113.9720,44.0180);



  \path[draw=blue,line join=round,line width=0.256pt] (112.6300,45.0201) .. controls (112.6300,45.0201) and (113.0400,44.7549) .. (113.8110,44.2881);



  \path[draw=blue,line join=round,line width=0.256pt] (111.0000,43.3552) .. controls (111.0000,43.3552) and (111.4600,43.8556) .. (111.6020,43.8607);



  \path[draw=blue,line join=round,line width=0.256pt] (111.0070,43.3438) .. controls (111.0070,43.3438) and (111.2130,43.4516) .. (112.6030,43.5835);



  \path[draw=blue,line join=round,line width=0.256pt] (113.6090,44.0179) -- (113.6250,44.1955) -- (114.2330,44.4593) .. controls (114.2330,44.4593) and (115.5830,44.6115) .. (115.8110,44.7485) .. controls (116.0400,44.8855) and (115.5980,44.0635) .. (113.6090,44.0179) -- cycle;



  \path[draw=blue,line join=round,line width=0.256pt] (135.5240,53.0266) .. controls (135.5240,53.0266) and (136.3860,52.7323) .. (138.8970,50.2512) -- (140.6020,46.9939) -- (138.6130,48.2318) .. controls (138.6170,48.2785) and (137.0040,50.9270) .. (136.1720,52.3680) .. controls (135.3400,53.8089) and (132.9550,53.4538) .. (132.9550,53.4538) -- (102.0890,50.5952) -- (99.8164,51.4833) -- (97.7082,52.0693) .. controls (97.3581,52.1378) and (96.3535,52.1530) .. (95.8969,52.1225) .. controls (95.4402,52.0921) and (94.8342,51.7128) .. (95.1310,51.0430) .. controls (95.4279,50.3733) and (96.2469,49.7328) .. (96.2469,49.7328) .. controls (96.2469,49.7328) and (97.1887,48.8567) .. (98.4863,48.1413) -- (95.2677,40.0491) .. controls (95.2677,40.0491) and (95.1383,39.7980) .. (95.2867,39.5240) -- (97.3834,35.5628) -- (99.6311,34.9920) -- (98.9156,37.7243) .. controls (98.9156,37.7243) and (98.8624,38.3255) .. (99.0983,38.4853) -- (105.7830,43.1421);



  \path[draw=blue,line join=round,line width=0.256pt] (96.2692,49.7124) -- (96.4214,49.8037) -- (96.2388,49.9712) .. controls (96.2388,49.9712) and (96.0335,50.1208) .. (96.7185,50.1970) -- (97.0151,50.0169) .. controls (97.0151,50.0169) and (97.9844,50.8141) .. (97.7866,52.0318);



  \path[draw=blue,line join=round,line width=0.256pt] (96.8882,49.3928) .. controls (96.8882,49.3928) and (97.1724,49.7885) .. (97.5580,49.8596) -- (98.8975,48.8549) .. controls (98.8975,48.8549) and (98.6235,48.5100) .. (98.2074,48.3882) -- cycle;



  \path[draw=blue,line join=round,line width=0.256pt] (96.8942,49.4079) -- (98.2203,48.9894) -- (98.2185,48.4052);



  \path[draw=blue,line join=round,line width=0.256pt] (98.2299,48.9951) -- (98.4696,49.1631);



  \path[draw=blue,line join=round,line width=0.256pt] (99.6408,47.4381) -- (99.8437,47.7375) .. controls (99.8437,47.7375) and (100.8150,48.2664) .. (101.7080,48.2259) .. controls (102.6010,48.1853) and (103.9000,47.7387) .. (103.9000,47.7387) .. controls (103.9000,47.7387) and (108.7910,45.9933) .. (108.9740,45.9121) .. controls (108.9740,45.9121) and (109.6010,45.6914) .. (109.9460,45.1232) .. controls (109.9460,45.1233) and (110.1820,44.6284) .. (109.5040,44.5828) -- (109.0550,44.5828);



  \path[draw=blue,line join=round,line width=0.256pt] (99.5795,38.8281) -- (98.3475,38.9140) -- (101.5950,41.5422) -- (103.3470,41.4430);



  \path[draw=blue,line join=round,line width=0.256pt] (123.3560,48.4843) -- (122.1590,49.1337) -- (133.8890,51.3864) -- (134.8430,50.6127);



  \path[draw=blue,line join=round,line width=0.256pt] (103.5780,50.7043) .. controls (103.5780,50.7043) and (103.3650,50.6130) .. (103.4060,50.5420) .. controls (103.4460,50.4709) and (103.4460,50.0244) .. (105.8820,48.8269) .. controls (108.3170,47.6295) and (109.0680,47.2236) .. (109.0680,47.2236) -- (109.1060,47.1989);



  \path[draw=blue,line join=round,line width=0.256pt] (103.5870,50.7312) .. controls (103.5870,50.7312) and (103.4400,50.6197) .. (103.4390,50.5379) .. controls (103.4380,50.4716) and (103.4790,50.0203) .. (105.9150,48.8229) .. controls (108.3500,47.6255) and (109.1010,47.2196) .. (109.1010,47.2196) -- (109.1390,47.1949);



  \path[draw=blue,line join=round,line width=0.256pt] (95.1590,39.8813) -- (95.2198,39.8609);



  \path[draw=blue,line join=round,line width=0.256pt] (99.2624,38.5962) .. controls (99.2624,38.5962) and (96.7635,38.7973) .. (96.2917,39.0155) .. controls (95.8198,39.2337) and (95.5611,39.4619) .. (95.5611,39.4619) -- (95.3277,39.7055) .. controls (95.3277,39.7055) and (95.2465,39.7766) .. (95.2465,39.9035);



  \path[draw=blue,line join=round,line width=0.256pt] (132.9150,53.4086) .. controls (132.9150,53.4086) and (133.3410,52.6779) .. (134.0110,52.2213) .. controls (134.6810,51.7647) and (136.2640,50.9428) .. (136.2640,50.9428) -- (136.3480,50.8990);



  \path[draw=blue,line join=round,line width=0.256pt] (96.0571,39.4453) -- (98.8173,43.6260);



  \path[draw=blue,line join=round,line width=0.256pt] (133.2790,52.3125) -- (115.8650,50.1612);



\end{scope}
\path (81.0092,13.4726) node[above right] (text300) {\textlabel{i}{sth:i}};



\begin{scope}[cm={{1.33334,0.0,0.0,1.33334,(160.9027,-28.96512)}},draw=blue,even odd rule,line cap=rect,line join=bevel,line width=0.800pt]
  \begin{scope}[draw=blue,line cap=rect,line join=bevel,line width=0.800pt]
  \end{scope}
  \begin{scope}[scale=1.011,draw=blue,line cap=rect,line join=bevel,line width=0.800pt]
  \end{scope}
  \begin{scope}[cm={{0.73698,0.0,0.0,0.91475,(5.33482,22.44371)}},draw=blue,line cap=round,line join=round,line width=0.400pt]
    \path[draw] (58.5000,78.5000) -- (53.5000,78.5000);



  \end{scope}
  \begin{scope}[scale=1.011,draw=blue,line cap=rect,line join=bevel,line width=0.800pt]
  \end{scope}
  \begin{scope}[cm={{1.01111,0.0,0.0,1.01111,(36.9056,82.9111)}},draw=blue,line cap=rect,line join=bevel,line width=0.800pt]
  \end{scope}
  \begin{scope}[cm={{1.01111,0.0,0.0,1.01111,(36.9056,82.9111)}},draw=blue,line cap=rect,line join=bevel,line width=0.800pt]
  \end{scope}
  \begin{scope}[cm={{1.01111,0.0,0.0,1.01111,(36.9056,82.9111)}},draw=blue,line cap=rect,line join=bevel,line width=0.800pt]
  \end{scope}
  \begin{scope}[cm={{1.01111,0.0,0.0,1.01111,(36.9056,82.9111)}},draw=blue,line cap=rect,line join=bevel,line width=0.800pt]
  \end{scope}
  \begin{scope}[cm={{1.01111,0.0,0.0,1.01111,(36.9056,82.9111)}},draw=blue,line cap=rect,line join=bevel,line width=0.800pt]
  \end{scope}
  \begin{scope}[cm={{1.01111,0.0,0.0,1.01111,(35.01131,96.44526)}},draw=blue,line cap=rect,line join=bevel,line width=0.800pt]
    \path[fill=blue] (0.0000,0.0000) node[above right] (text3633) {20};



  \end{scope}
  \begin{scope}[cm={{1.01111,0.0,0.0,1.01111,(36.9056,82.9111)}},draw=blue,line cap=rect,line join=bevel,line width=0.800pt]
  \end{scope}
  \begin{scope}[scale=1.011,draw=blue,line cap=rect,line join=bevel,line width=0.800pt]
  \end{scope}
  \begin{scope}[cm={{0.73698,0.0,0.0,0.91475,(5.33482,22.44371)}},draw=blue,line cap=round,line join=round,line width=0.400pt]
    \path[draw] (58.5000,49.5000) -- (53.5000,49.5000);



  \end{scope}
  \begin{scope}[scale=1.011,draw=blue,line cap=rect,line join=bevel,line width=0.800pt]
  \end{scope}
  \begin{scope}[cm={{1.01111,0.0,0.0,1.01111,(37.4111,54.6)}},draw=blue,line cap=rect,line join=bevel,line width=0.800pt]
  \end{scope}
  \begin{scope}[cm={{1.01111,0.0,0.0,1.01111,(37.4111,54.6)}},draw=blue,line cap=rect,line join=bevel,line width=0.800pt]
  \end{scope}
  \begin{scope}[cm={{1.01111,0.0,0.0,1.01111,(37.4111,54.6)}},draw=blue,line cap=rect,line join=bevel,line width=0.800pt]
  \end{scope}
  \begin{scope}[cm={{1.01111,0.0,0.0,1.01111,(37.4111,54.6)}},draw=blue,line cap=rect,line join=bevel,line width=0.800pt]
  \end{scope}
  \begin{scope}[cm={{1.01111,0.0,0.0,1.01111,(37.4111,54.6)}},draw=blue,line cap=rect,line join=bevel,line width=0.800pt]
  \end{scope}
  \begin{scope}[cm={{1.01111,0.0,0.0,1.01111,(35.31869,71.13416)}},draw=blue,line cap=rect,line join=bevel,line width=0.800pt]
    \path[fill=blue] (0.0000,0.0000) node[above right] (text3657) {25};



  \end{scope}
  \begin{scope}[cm={{1.01111,0.0,0.0,1.01111,(37.4111,54.6)}},draw=blue,line cap=rect,line join=bevel,line width=0.800pt]
  \end{scope}
  \begin{scope}[scale=1.011,draw=blue,line cap=rect,line join=bevel,line width=0.800pt]
  \end{scope}
  \begin{scope}[cm={{0.73698,0.0,0.0,0.91475,(5.33482,22.44371)}},draw=blue,line cap=round,line join=round,line width=0.400pt]
    \path[draw] (58.5000,20.5000) -- (53.5000,20.5000);



  \end{scope}
  \begin{scope}[scale=1.011,draw=blue,line cap=rect,line join=bevel,line width=0.800pt]
  \end{scope}
  \begin{scope}[cm={{1.01111,0.0,0.0,1.01111,(36.9056,25.2778)}},draw=blue,line cap=rect,line join=bevel,line width=0.800pt]
  \end{scope}
  \begin{scope}[cm={{1.01111,0.0,0.0,1.01111,(36.9056,25.2778)}},draw=blue,line cap=rect,line join=bevel,line width=0.800pt]
  \end{scope}
  \begin{scope}[cm={{1.01111,0.0,0.0,1.01111,(36.9056,25.2778)}},draw=blue,line cap=rect,line join=bevel,line width=0.800pt]
  \end{scope}
  \begin{scope}[cm={{1.01111,0.0,0.0,1.01111,(36.9056,25.2778)}},draw=blue,line cap=rect,line join=bevel,line width=0.800pt]
  \end{scope}
  \begin{scope}[cm={{1.01111,0.0,0.0,1.01111,(36.9056,25.2778)}},draw=blue,line cap=rect,line join=bevel,line width=0.800pt]
  \end{scope}
  \begin{scope}[cm={{1.01111,0.0,0.0,1.01111,(28.77684,46.31195)}},draw=blue,line cap=rect,line join=bevel,line width=0.800pt]
    \path[fill=blue] (6.1660,-2.5032) node[above right] (text3681) {30};



  \end{scope}
  \begin{scope}[cm={{1.01111,0.0,0.0,1.01111,(36.9056,25.2778)}},draw=blue,line cap=rect,line join=bevel,line width=0.800pt]
  \end{scope}
  \begin{scope}[scale=1.011,draw=blue,line cap=rect,line join=bevel,line width=0.800pt]
  \end{scope}
  \begin{scope}[cm={{0.73698,0.0,0.0,0.91475,(1.65074,18.78975)}},draw=blue,line cap=round,line join=round,line width=0.400pt]
    \path[draw] (58.5000,84.5000) -- (58.5000,88.5000);



  \end{scope}
  \begin{scope}[scale=1.011,draw=blue,line cap=rect,line join=bevel,line width=0.800pt]
  \end{scope}
  \begin{scope}[cm={{1.01111,0.0,0.0,1.01111,(54.0944,101.111)}},draw=blue,line cap=rect,line join=bevel,line width=0.800pt]
  \end{scope}
  \begin{scope}[cm={{1.01111,0.0,0.0,1.01111,(54.0944,101.111)}},draw=blue,line cap=rect,line join=bevel,line width=0.800pt]
  \end{scope}
  \begin{scope}[cm={{1.01111,0.0,0.0,1.01111,(54.0944,101.111)}},draw=blue,line cap=rect,line join=bevel,line width=0.800pt]
  \end{scope}
  \begin{scope}[cm={{1.01111,0.0,0.0,1.01111,(54.0944,101.111)}},draw=blue,line cap=rect,line join=bevel,line width=0.800pt]
  \end{scope}
  \begin{scope}[cm={{1.01111,0.0,0.0,1.01111,(54.0944,101.111)}},draw=blue,line cap=rect,line join=bevel,line width=0.800pt]
  \end{scope}
  \begin{scope}[cm={{1.01111,0.0,0.0,1.01111,(42.09441,108.611)}},draw=blue,line cap=rect,line join=bevel,line width=0.800pt]
    \path[fill=blue] (0.0000,0.0000) node[above right] (text3705) {0};



  \end{scope}
  \begin{scope}[cm={{1.01111,0.0,0.0,1.01111,(54.0944,101.111)}},draw=blue,line cap=rect,line join=bevel,line width=0.800pt]
  \end{scope}
  \begin{scope}[scale=1.011,draw=blue,line cap=rect,line join=bevel,line width=0.800pt]
  \end{scope}
  \begin{scope}[cm={{0.73698,0.0,0.0,0.91475,(1.65074,18.78975)}},draw=blue,line cap=round,line join=round,line width=0.400pt]
    \path[draw] (89.5000,84.5000) -- (89.5000,88.5000);



  \end{scope}
  \begin{scope}[scale=1.011,draw=blue,line cap=rect,line join=bevel,line width=0.800pt]
  \end{scope}
  \begin{scope}[cm={{1.01111,0.0,0.0,1.01111,(80.3833,101.111)}},draw=blue,line cap=rect,line join=bevel,line width=0.800pt]
  \end{scope}
  \begin{scope}[cm={{1.01111,0.0,0.0,1.01111,(80.3833,101.111)}},draw=blue,line cap=rect,line join=bevel,line width=0.800pt]
  \end{scope}
  \begin{scope}[cm={{1.01111,0.0,0.0,1.01111,(80.3833,101.111)}},draw=blue,line cap=rect,line join=bevel,line width=0.800pt]
  \end{scope}
  \begin{scope}[cm={{1.01111,0.0,0.0,1.01111,(80.3833,101.111)}},draw=blue,line cap=rect,line join=bevel,line width=0.800pt]
  \end{scope}
  \begin{scope}[cm={{1.01111,0.0,0.0,1.01111,(80.3833,101.111)}},draw=blue,line cap=rect,line join=bevel,line width=0.800pt]
  \end{scope}
  \begin{scope}[cm={{1.01111,0.0,0.0,1.01111,(59.38331,108.611)}},draw=blue,line cap=rect,line join=bevel,line width=0.800pt]
    \path[fill=blue] (0.0000,0.0000) node[above right] (text3729) {100};



  \end{scope}
  \begin{scope}[cm={{1.01111,0.0,0.0,1.01111,(80.3833,101.111)}},draw=blue,line cap=rect,line join=bevel,line width=0.800pt]
  \end{scope}
  \begin{scope}[scale=1.011,draw=blue,line cap=rect,line join=bevel,line width=0.800pt]
  \end{scope}
  \begin{scope}[cm={{0.73698,0.0,0.0,0.91475,(1.65074,18.78975)}},draw=blue,line cap=round,line join=round,line width=0.400pt]
    \path[draw] (121.5000,84.5000) -- (121.5000,88.5000);



  \end{scope}
  \begin{scope}[scale=1.011,draw=blue,line cap=rect,line join=bevel,line width=0.800pt]
  \end{scope}
  \begin{scope}[cm={{1.01111,0.0,0.0,1.01111,(111.728,101.111)}},draw=blue,line cap=rect,line join=bevel,line width=0.800pt]
  \end{scope}
  \begin{scope}[cm={{1.01111,0.0,0.0,1.01111,(111.728,101.111)}},draw=blue,line cap=rect,line join=bevel,line width=0.800pt]
  \end{scope}
  \begin{scope}[cm={{1.01111,0.0,0.0,1.01111,(111.728,101.111)}},draw=blue,line cap=rect,line join=bevel,line width=0.800pt]
  \end{scope}
  \begin{scope}[cm={{1.01111,0.0,0.0,1.01111,(111.728,101.111)}},draw=blue,line cap=rect,line join=bevel,line width=0.800pt]
  \end{scope}
  \begin{scope}[cm={{1.01111,0.0,0.0,1.01111,(111.728,101.111)}},draw=blue,line cap=rect,line join=bevel,line width=0.800pt]
  \end{scope}
  \begin{scope}[cm={{1.01111,0.0,0.0,1.01111,(83.22801,107.111)}},draw=blue,line cap=rect,line join=bevel,line width=0.800pt]
    \path[fill=blue] (0.0000,1.4835) node[above right] (text3753) {200};



  \end{scope}
  \begin{scope}[cm={{1.01111,0.0,0.0,1.01111,(111.728,101.111)}},draw=blue,line cap=rect,line join=bevel,line width=0.800pt]
  \end{scope}
  \begin{scope}[scale=1.011,draw=blue,line cap=rect,line join=bevel,line width=0.800pt]
  \end{scope}
  \begin{scope}[cm={{0.73698,0.0,0.0,0.91475,(1.65074,18.78975)}},draw=blue,line cap=round,line join=round,line width=0.400pt]
    \path[draw] (152.5000,84.5000) -- (152.5000,88.5000);



  \end{scope}
  \begin{scope}[scale=1.011,draw=blue,line cap=rect,line join=bevel,line width=0.800pt]
  \end{scope}
  \begin{scope}[cm={{1.01111,0.0,0.0,1.01111,(144.083,101.111)}},draw=blue,line cap=rect,line join=bevel,line width=0.800pt]
  \end{scope}
  \begin{scope}[cm={{1.01111,0.0,0.0,1.01111,(144.083,101.111)}},draw=blue,line cap=rect,line join=bevel,line width=0.800pt]
  \end{scope}
  \begin{scope}[cm={{1.01111,0.0,0.0,1.01111,(144.083,101.111)}},draw=blue,line cap=rect,line join=bevel,line width=0.800pt]
  \end{scope}
  \begin{scope}[cm={{1.01111,0.0,0.0,1.01111,(144.083,101.111)}},draw=blue,line cap=rect,line join=bevel,line width=0.800pt]
  \end{scope}
  \begin{scope}[cm={{1.01111,0.0,0.0,1.01111,(144.083,101.111)}},draw=blue,line cap=rect,line join=bevel,line width=0.800pt]
  \end{scope}
  \begin{scope}[cm={{1.01111,0.0,0.0,1.01111,(105.083,108.611)}},draw=blue,line cap=rect,line join=bevel,line width=0.800pt]
    \path[fill=blue] (0.0000,0.0000) node[above right] (text3777) {300};



  \end{scope}
  \begin{scope}[cm={{1.01111,0.0,0.0,1.01111,(144.083,101.111)}},draw=blue,line cap=rect,line join=bevel,line width=0.800pt]
  \end{scope}
  \begin{scope}[scale=1.011,draw=blue,line cap=rect,line join=bevel,line width=0.800pt]
  \end{scope}
  \begin{scope}[cm={{0.73698,0.0,0.0,0.91475,(1.65074,22.44371)}},draw=blue,line cap=round,line join=round,line width=0.400pt]
    \path[draw] (58.5000,15.5000) -- (58.5000,84.5000) -- (158.5000,84.5000) -- (158.5000,15.5000) -- (58.5000,15.5000);



  \end{scope}
  \begin{scope}[scale=1.011,draw=blue,line cap=rect,line join=bevel,line width=0.800pt]
  \end{scope}
  \begin{scope}[cm={{0.0,-1.01111,1.01111,0.0,(32.3556,76.8444)}},draw=blue,line cap=rect,line join=bevel,line width=0.800pt]
  \end{scope}
  \begin{scope}[cm={{0.0,-1.01111,1.01111,0.0,(32.3556,76.8444)}},draw=blue,line cap=rect,line join=bevel,line width=0.800pt]
  \end{scope}
  \begin{scope}[cm={{0.0,-1.01111,1.01111,0.0,(32.3556,76.8444)}},draw=blue,line cap=rect,line join=bevel,line width=0.800pt]
  \end{scope}
  \begin{scope}[cm={{0.0,-1.01111,1.01111,0.0,(32.3556,76.8444)}},draw=blue,line cap=rect,line join=bevel,line width=0.800pt]
  \end{scope}
  \begin{scope}[cm={{0.0,-1.01111,1.01111,0.0,(32.3556,76.8444)}},draw=blue,line cap=rect,line join=bevel,line width=0.800pt]
  \end{scope}
  \begin{scope}[cm={{0.0,-1.01111,1.01111,0.0,(27.8556,81.3444)}},draw=blue,line cap=rect,line join=bevel,line width=0.800pt]
    \path[fill=blue] (0.0000,0.0000) node[above right] (text3801) {\rotatebox{90}{Power (W)}};



  \end{scope}
  \begin{scope}[cm={{0.0,-1.01111,1.01111,0.0,(32.3556,76.8444)}},draw=blue,line cap=rect,line join=bevel,line width=0.800pt]
  \end{scope}
  \begin{scope}[cm={{1.01111,0.0,0.0,1.01111,(80.3833,115.267)}},draw=blue,line cap=rect,line join=bevel,line width=0.800pt]
  \end{scope}
  \begin{scope}[cm={{1.01111,0.0,0.0,1.01111,(80.3833,115.267)}},draw=blue,line cap=rect,line join=bevel,line width=0.800pt]
  \end{scope}
  \begin{scope}[cm={{1.01111,0.0,0.0,1.01111,(80.3833,115.267)}},draw=blue,line cap=rect,line join=bevel,line width=0.800pt]
  \end{scope}
  \begin{scope}[cm={{1.01111,0.0,0.0,1.01111,(80.3833,115.267)}},draw=blue,line cap=rect,line join=bevel,line width=0.800pt]
  \end{scope}
  \begin{scope}[cm={{1.01111,0.0,0.0,1.01111,(80.3833,115.267)}},draw=blue,line cap=rect,line join=bevel,line width=0.800pt]
  \end{scope}
  \begin{scope}[cm={{1.01111,0.0,0.0,1.01111,(65.38331,118.26699)}},draw=blue,line cap=rect,line join=bevel,line width=0.800pt]
    \path[fill=blue] (0.0000,0.0000) node[above right] (text3817) {Time (sec)};



  \end{scope}
  \begin{scope}[cm={{1.01111,0.0,0.0,1.01111,(80.3833,115.267)}},draw=blue,line cap=rect,line join=bevel,line width=0.800pt]
  \end{scope}
  \begin{scope}[scale=1.011,draw=blue,line cap=rect,line join=bevel,line width=0.800pt]
  \end{scope}
  \begin{scope}[scale=1.011,draw=blue,line cap=rect,line join=bevel,line width=0.800pt]
  \end{scope}
  \begin{scope}[scale=1.011,draw=blue,line cap=rect,line join=bevel,line width=0.800pt]
  \end{scope}
  \begin{scope}[cm={{0.73698,0.0,0.0,0.91475,(1.65074,22.44371)}},draw=blue,line cap=round,line join=round,line width=0.400pt]
    \path[draw] (58.0000,32.3000) -- (58.0000,32.3000) -- (59.0000,15.9000) -- (60.0000,61.6000) -- (61.1000,62.7000) -- (62.1000,79.7000) -- (63.1000,77.1000) -- (64.1000,76.7000) -- (65.1000,70.8000) -- (66.2000,61.4000) -- (67.2000,52.7000) -- (68.2000,37.6000) -- (69.2000,39.2000) -- (70.2000,40.7000) -- (71.2000,54.5000) -- (72.3000,52.9000) -- (73.3000,42.0000) -- (74.3000,30.3000) -- (75.3000,29.3000) -- (76.3000,53.7000) -- (77.4000,48.6000) -- (78.4000,56.1000) -- (79.4000,62.2000) -- (80.4000,55.7000) -- (81.4000,54.2000) -- (82.5000,50.4000) -- (83.5000,41.8000) -- (84.5000,47.7000) -- (85.5000,47.9000) -- (86.5000,53.4000) -- (87.6000,57.3000) -- (88.6000,43.7000) -- (89.6000,35.2000) -- (90.6000,29.6000) -- (91.6000,53.1000) -- (92.7000,60.7000) -- (93.7000,64.9000) -- (94.7000,75.7000) -- (95.7000,75.1000) -- (96.7000,68.1000) -- (97.7000,53.9000) -- (98.8000,46.5000) -- (99.8000,44.8000) -- (100.8000,44.5000) -- (101.8000,46.9000) -- (102.8000,51.6000) -- (103.9000,37.7000) -- (104.9000,32.2000) -- (105.9000,26.7000) -- (106.9000,49.2000) -- (107.9000,61.2000) -- (109.0000,66.6000) -- (110.0000,70.8000) -- (111.0000,58.1000) -- (112.0000,57.7000) -- (113.0000,57.0000) -- (114.1000,60.0000) -- (115.1000,56.3000) -- (116.1000,59.6000) -- (117.1000,68.0000) -- (118.1000,72.0000) -- (119.2000,65.1000) -- (120.2000,40.3000) -- (121.2000,31.5000) -- (122.2000,26.9000) -- (123.2000,48.6000) -- (124.2000,50.9000) -- (125.3000,59.4000) -- (126.3000,64.4000) -- (127.3000,57.3000) -- (128.3000,53.3000) -- (129.3000,48.8000) -- (130.4000,48.6000) -- (131.4000,47.6000) -- (132.4000,52.7000) -- (133.4000,57.9000) -- (134.4000,57.2000) -- (135.5000,55.6000) -- (136.5000,39.3000) -- (137.5000,27.6000) -- (138.5000,30.2000) -- (139.5000,45.0000) -- (140.6000,37.7000) -- (141.6000,43.9000) -- (142.6000,57.7000) -- (143.6000,56.1000) -- (144.6000,58.5000) -- (145.7000,56.2000) -- (146.7000,46.6000) -- (147.7000,40.7000) -- (148.7000,40.4000) -- (149.7000,42.9000) -- (150.7000,53.2000) -- (151.8000,45.4000) -- (152.8000,32.4000) -- (153.8000,23.1000) -- (154.8000,35.8000) -- (155.8000,56.6000) -- (156.9000,64.4000) -- (157.9000,67.0000) -- (158.9000,62.0000);



  \end{scope}
  \begin{scope}[scale=1.011,draw=blue,line cap=rect,line join=bevel,line width=0.800pt]
  \end{scope}
  \begin{scope}[scale=1.011,draw=blue,line cap=rect,line join=bevel,line width=0.800pt]
  \end{scope}
  \begin{scope}[cm={{0.73698,0.0,0.0,0.91475,(1.65074,22.44371)}},draw=blue,line cap=round,line join=round,line width=0.400pt]
    \path[draw] (58.5000,15.5000) -- (58.5000,84.5000) -- (158.5000,84.5000) -- (158.5000,15.5000) -- (58.5000,15.5000);



  \end{scope}
  \begin{scope}[scale=1.011,draw=blue,line cap=rect,line join=bevel,line width=0.800pt]
  \end{scope}
  \begin{scope}[draw=blue,line cap=rect,line join=bevel,line width=0.800pt]
  \end{scope}
  \path[fill=blue] (0.0000,0.0000) node[above right] (text6269) {};



\end{scope}
\path (160.6642,229.8670) node[above right] (text300-3) {\ref{sth:i}};



\path (175.7367,229.8670) node[above right] (text300-1) {\ref{sth:ii}};



\path (190.0484,229.8670) node[above right] (text300-34) {\ref{sth:iii}};



\path[draw=blue,line cap=butt,line join=miter,line width=0.582pt,miter limit=4.00] (146.7121,236.6610) -- (185.5734,236.6610) -- (185.5734,245.2473) -- (221.0595,245.2473);



\path[draw=blue,line cap=butt,line join=miter,line width=0.582pt,miter limit=4.00] (146.7965,251.6540) -- (170.4664,251.6540) -- (170.4664,251.7141) -- (170.4664,260.4350) -- (220.8794,260.4350);




\end{tikzpicture}

}
 %  \node[inner sep=0pt] () at (-2,-5.67)
 %      {\scriptsize 
\definecolor{ca0a0a4}{RGB}{255,255,255}


\def \globalscale {0.950000}
\begin{tikzpicture}[y=0.80pt, x=0.80pt, yscale=-\globalscale, xscale=0.82, inner sep=0pt, outer sep=0pt]
\begin{scope}[draw=black,line join=bevel,line cap=rect,even odd rule,line width=0.800pt]
  \begin{scope}[cm={{1.0,0.0,0.0,1.0,(0.0,0.0)}},draw=black,line join=bevel,line cap=rect,line width=0.800pt]
  \end{scope}
  \begin{scope}[cm={{1.01111,0.0,0.0,1.01111,(0.0,0.0)}},draw=black,line join=bevel,line cap=rect,line width=0.800pt]
  \end{scope}
  \begin{scope}[cm={{1.01111,0.0,0.0,1.01111,(0.0,0.0)}},draw=ca0a0a4,dash pattern=on 0.40pt off 0.80pt,line join=round,line cap=round,line width=0.400pt]
    \path[draw] (43.5000,80.5000) -- (164.5000,80.5000);



  \end{scope}
  \begin{scope}[cm={{1.01111,0.0,0.0,1.01111,(0.0,0.0)}},draw=black,line join=round,line cap=round,line width=0.400pt]
    \path[draw] (43.5000,80.5000) -- (46.5000,80.5000);



  \end{scope}
  \begin{scope}[cm={{1.01111,0.0,0.0,1.01111,(0.0,0.0)}},draw=black,line join=bevel,line cap=rect,line width=0.800pt]
  \end{scope}
  \begin{scope}[cm={{1.01111,0.0,0.0,1.01111,(26.7944,85.9444)}},draw=black,line join=bevel,line cap=rect,line width=0.800pt]
  \end{scope}
  \begin{scope}[cm={{1.01111,0.0,0.0,1.01111,(26.7944,85.9444)}},draw=black,line join=bevel,line cap=rect,line width=0.800pt]
  \end{scope}
  \begin{scope}[cm={{1.01111,0.0,0.0,1.01111,(26.7944,85.9444)}},draw=black,line join=bevel,line cap=rect,line width=0.800pt]
  \end{scope}
  \begin{scope}[cm={{1.01111,0.0,0.0,1.01111,(26.7944,85.9444)}},draw=black,line join=bevel,line cap=rect,line width=0.800pt]
  \end{scope}
  \begin{scope}[cm={{1.01111,0.0,0.0,1.01111,(26.7944,85.9444)}},draw=black,line join=bevel,line cap=rect,line width=0.800pt]
  \end{scope}
  \begin{scope}[cm={{1.01111,0.0,0.0,1.01111,(26.7944,85.9444)}},draw=black,line join=bevel,line cap=rect,line width=0.800pt]
    \path[fill=black] (0.0000,0.0000) node[above right] () {20};



  \end{scope}
  \begin{scope}[cm={{1.01111,0.0,0.0,1.01111,(26.7944,85.9444)}},draw=black,line join=bevel,line cap=rect,line width=0.800pt]
  \end{scope}
  \begin{scope}[cm={{1.01111,0.0,0.0,1.01111,(0.0,0.0)}},draw=black,line join=bevel,line cap=rect,line width=0.800pt]
  \end{scope}
  \begin{scope}[cm={{1.01111,0.0,0.0,1.01111,(0.0,0.0)}},draw=ca0a0a4,dash pattern=on 0.40pt off 0.80pt,line join=round,line cap=round,line width=0.400pt]
    \path[draw] (43.5000,51.5000) -- (164.5000,51.5000);



  \end{scope}
  \begin{scope}[cm={{1.01111,0.0,0.0,1.01111,(0.0,0.0)}},draw=black,line join=round,line cap=round,line width=0.400pt]
    \path[draw] (43.5000,51.5000) -- (46.5000,51.5000);



  \end{scope}
  \begin{scope}[cm={{1.01111,0.0,0.0,1.01111,(0.0,0.0)}},draw=black,line join=bevel,line cap=rect,line width=0.800pt]
  \end{scope}
  \begin{scope}[cm={{1.01111,0.0,0.0,1.01111,(27.3,55.6111)}},draw=black,line join=bevel,line cap=rect,line width=0.800pt]
  \end{scope}
  \begin{scope}[cm={{1.01111,0.0,0.0,1.01111,(27.3,55.6111)}},draw=black,line join=bevel,line cap=rect,line width=0.800pt]
  \end{scope}
  \begin{scope}[cm={{1.01111,0.0,0.0,1.01111,(27.3,55.6111)}},draw=black,line join=bevel,line cap=rect,line width=0.800pt]
  \end{scope}
  \begin{scope}[cm={{1.01111,0.0,0.0,1.01111,(27.3,55.6111)}},draw=black,line join=bevel,line cap=rect,line width=0.800pt]
  \end{scope}
  \begin{scope}[cm={{1.01111,0.0,0.0,1.01111,(27.3,55.6111)}},draw=black,line join=bevel,line cap=rect,line width=0.800pt]
  \end{scope}
  \begin{scope}[cm={{1.01111,0.0,0.0,1.01111,(27.3,55.6111)}},draw=black,line join=bevel,line cap=rect,line width=0.800pt]
    \path[fill=black] (0.0000,0.0000) node[above right] () {25};



  \end{scope}
  \begin{scope}[cm={{1.01111,0.0,0.0,1.01111,(27.3,55.6111)}},draw=black,line join=bevel,line cap=rect,line width=0.800pt]
  \end{scope}
  \begin{scope}[cm={{1.01111,0.0,0.0,1.01111,(0.0,0.0)}},draw=black,line join=bevel,line cap=rect,line width=0.800pt]
  \end{scope}
  \begin{scope}[cm={{1.01111,0.0,0.0,1.01111,(0.0,0.0)}},draw=ca0a0a4,dash pattern=on 0.40pt off 0.80pt,line join=round,line cap=round,line width=0.400pt]
    \path[draw] (43.5000,22.5000) -- (164.5000,22.5000);



  \end{scope}
  \begin{scope}[cm={{1.01111,0.0,0.0,1.01111,(0.0,0.0)}},draw=black,line join=round,line cap=round,line width=0.400pt]
    \path[draw] (43.5000,22.5000) -- (46.5000,22.5000);



  \end{scope}
  \begin{scope}[cm={{1.01111,0.0,0.0,1.01111,(0.0,0.0)}},draw=black,line join=bevel,line cap=rect,line width=0.800pt]
  \end{scope}
  \begin{scope}[cm={{1.01111,0.0,0.0,1.01111,(26.7944,26.2889)}},draw=black,line join=bevel,line cap=rect,line width=0.800pt]
  \end{scope}
  \begin{scope}[cm={{1.01111,0.0,0.0,1.01111,(26.7944,26.2889)}},draw=black,line join=bevel,line cap=rect,line width=0.800pt]
  \end{scope}
  \begin{scope}[cm={{1.01111,0.0,0.0,1.01111,(26.7944,26.2889)}},draw=black,line join=bevel,line cap=rect,line width=0.800pt]
  \end{scope}
  \begin{scope}[cm={{1.01111,0.0,0.0,1.01111,(26.7944,26.2889)}},draw=black,line join=bevel,line cap=rect,line width=0.800pt]
  \end{scope}
  \begin{scope}[cm={{1.01111,0.0,0.0,1.01111,(26.7944,26.2889)}},draw=black,line join=bevel,line cap=rect,line width=0.800pt]
  \end{scope}
  \begin{scope}[cm={{1.01111,0.0,0.0,1.01111,(26.7944,26.2889)}},draw=black,line join=bevel,line cap=rect,line width=0.800pt]
    \path[fill=black] (0.0000,0.0000) node[above right] () {30};



  \end{scope}
  \begin{scope}[cm={{1.01111,0.0,0.0,1.01111,(26.7944,26.2889)}},draw=black,line join=bevel,line cap=rect,line width=0.800pt]
  \end{scope}
  \begin{scope}[cm={{1.01111,0.0,0.0,1.01111,(0.0,0.0)}},draw=black,line join=bevel,line cap=rect,line width=0.800pt]
  \end{scope}
  \begin{scope}[cm={{1.01111,0.0,0.0,1.01111,(0.0,0.0)}},draw=ca0a0a4,dash pattern=on 0.40pt off 0.80pt,line join=round,line cap=round,line width=0.400pt]
    \path[draw] (43.5000,86.5000) -- (43.5000,16.5000);



  \end{scope}
  \begin{scope}[cm={{1.01111,0.0,0.0,1.01111,(0.0,0.0)}},draw=black,line join=round,line cap=round,line width=0.400pt]
    \path[draw] (43.5000,86.5000) -- (43.5000,83.5000);



  \end{scope}
  \begin{scope}[cm={{1.01111,0.0,0.0,1.01111,(0.0,0.0)}},draw=black,line join=bevel,line cap=rect,line width=0.800pt]
  \end{scope}
  \begin{scope}[cm={{1.01111,0.0,0.0,1.01111,(36.9056,99.0889)}},draw=black,line join=bevel,line cap=rect,line width=0.800pt]
  \end{scope}
  \begin{scope}[cm={{1.01111,0.0,0.0,1.01111,(36.9056,99.0889)}},draw=black,line join=bevel,line cap=rect,line width=0.800pt]
  \end{scope}
  \begin{scope}[cm={{1.01111,0.0,0.0,1.01111,(36.9056,99.0889)}},draw=black,line join=bevel,line cap=rect,line width=0.800pt]
  \end{scope}
  \begin{scope}[cm={{1.01111,0.0,0.0,1.01111,(36.9056,99.0889)}},draw=black,line join=bevel,line cap=rect,line width=0.800pt]
  \end{scope}
  \begin{scope}[cm={{1.01111,0.0,0.0,1.01111,(36.9056,99.0889)}},draw=black,line join=bevel,line cap=rect,line width=0.800pt]
  \end{scope}
  \begin{scope}[cm={{1.01111,0.0,0.0,1.01111,(36.9056,99.0889)}},draw=black,line join=bevel,line cap=rect,line width=0.800pt]
    \path[fill=black] (0.0000,0.0000) node[above right] () {0};



  \end{scope}
  \begin{scope}[cm={{1.01111,0.0,0.0,1.01111,(36.9056,99.0889)}},draw=black,line join=bevel,line cap=rect,line width=0.800pt]
  \end{scope}
  \begin{scope}[cm={{1.01111,0.0,0.0,1.01111,(0.0,0.0)}},draw=black,line join=bevel,line cap=rect,line width=0.800pt]
  \end{scope}
  \begin{scope}[cm={{1.01111,0.0,0.0,1.01111,(0.0,0.0)}},draw=ca0a0a4,dash pattern=on 0.40pt off 0.80pt,line join=round,line cap=round,line width=0.400pt]
    \path[draw] (81.5000,86.5000) -- (81.5000,16.5000);



  \end{scope}
  \begin{scope}[cm={{1.01111,0.0,0.0,1.01111,(0.0,0.0)}},draw=black,line join=round,line cap=round,line width=0.400pt]
    \path[draw] (81.5000,86.5000) -- (81.5000,83.5000);



  \end{scope}
  \begin{scope}[cm={{1.01111,0.0,0.0,1.01111,(0.0,0.0)}},draw=black,line join=bevel,line cap=rect,line width=0.800pt]
  \end{scope}
  \begin{scope}[cm={{1.01111,0.0,0.0,1.01111,(69.2611,99.0889)}},draw=black,line join=bevel,line cap=rect,line width=0.800pt]
  \end{scope}
  \begin{scope}[cm={{1.01111,0.0,0.0,1.01111,(69.2611,99.0889)}},draw=black,line join=bevel,line cap=rect,line width=0.800pt]
  \end{scope}
  \begin{scope}[cm={{1.01111,0.0,0.0,1.01111,(69.2611,99.0889)}},draw=black,line join=bevel,line cap=rect,line width=0.800pt]
  \end{scope}
  \begin{scope}[cm={{1.01111,0.0,0.0,1.01111,(69.2611,99.0889)}},draw=black,line join=bevel,line cap=rect,line width=0.800pt]
  \end{scope}
  \begin{scope}[cm={{1.01111,0.0,0.0,1.01111,(69.2611,99.0889)}},draw=black,line join=bevel,line cap=rect,line width=0.800pt]
  \end{scope}
  \begin{scope}[cm={{1.01111,0.0,0.0,1.01111,(69.2611,99.0889)}},draw=black,line join=bevel,line cap=rect,line width=0.800pt]
    \path[fill=black] (0.0000,0.0000) node[above right] () {100};



  \end{scope}
  \begin{scope}[cm={{1.01111,0.0,0.0,1.01111,(69.2611,99.0889)}},draw=black,line join=bevel,line cap=rect,line width=0.800pt]
  \end{scope}
  \begin{scope}[cm={{1.01111,0.0,0.0,1.01111,(0.0,0.0)}},draw=black,line join=bevel,line cap=rect,line width=0.800pt]
  \end{scope}
  \begin{scope}[cm={{1.01111,0.0,0.0,1.01111,(0.0,0.0)}},draw=ca0a0a4,dash pattern=on 0.40pt off 0.80pt,line join=round,line cap=round,line width=0.400pt]
    \path[draw] (119.5000,86.5000) -- (119.5000,16.5000);



  \end{scope}
  \begin{scope}[cm={{1.01111,0.0,0.0,1.01111,(0.0,0.0)}},draw=black,line join=round,line cap=round,line width=0.400pt]
    \path[draw] (119.5000,86.5000) -- (119.5000,83.5000);



  \end{scope}
  \begin{scope}[cm={{1.01111,0.0,0.0,1.01111,(0.0,0.0)}},draw=black,line join=bevel,line cap=rect,line width=0.800pt]
  \end{scope}
  \begin{scope}[cm={{1.01111,0.0,0.0,1.01111,(107.683,99.0889)}},draw=black,line join=bevel,line cap=rect,line width=0.800pt]
  \end{scope}
  \begin{scope}[cm={{1.01111,0.0,0.0,1.01111,(107.683,99.0889)}},draw=black,line join=bevel,line cap=rect,line width=0.800pt]
  \end{scope}
  \begin{scope}[cm={{1.01111,0.0,0.0,1.01111,(107.683,99.0889)}},draw=black,line join=bevel,line cap=rect,line width=0.800pt]
  \end{scope}
  \begin{scope}[cm={{1.01111,0.0,0.0,1.01111,(107.683,99.0889)}},draw=black,line join=bevel,line cap=rect,line width=0.800pt]
  \end{scope}
  \begin{scope}[cm={{1.01111,0.0,0.0,1.01111,(107.683,99.0889)}},draw=black,line join=bevel,line cap=rect,line width=0.800pt]
  \end{scope}
  \begin{scope}[cm={{1.01111,0.0,0.0,1.01111,(107.683,99.0889)}},draw=black,line join=bevel,line cap=rect,line width=0.800pt]
    \path[fill=black] (0.0000,0.0000) node[above right] () {200};



  \end{scope}
  \begin{scope}[cm={{1.01111,0.0,0.0,1.01111,(107.683,99.0889)}},draw=black,line join=bevel,line cap=rect,line width=0.800pt]
  \end{scope}
  \begin{scope}[cm={{1.01111,0.0,0.0,1.01111,(0.0,0.0)}},draw=black,line join=bevel,line cap=rect,line width=0.800pt]
  \end{scope}
  \begin{scope}[cm={{1.01111,0.0,0.0,1.01111,(0.0,0.0)}},draw=ca0a0a4,dash pattern=on 0.40pt off 0.80pt,line join=round,line cap=round,line width=0.400pt]
    \path[draw] (157.5000,86.5000) -- (157.5000,22.5000) -- (157.5000,22.5000) -- (157.5000,16.5000);



  \end{scope}
  \begin{scope}[cm={{1.01111,0.0,0.0,1.01111,(0.0,0.0)}},draw=black,line join=round,line cap=round,line width=0.400pt]
    \path[draw] (157.5000,86.5000) -- (157.5000,83.5000);



  \end{scope}
  \begin{scope}[cm={{1.01111,0.0,0.0,1.01111,(0.0,0.0)}},draw=black,line join=bevel,line cap=rect,line width=0.800pt]
  \end{scope}
  \begin{scope}[cm={{1.01111,0.0,0.0,1.01111,(146.106,99.0889)}},draw=black,line join=bevel,line cap=rect,line width=0.800pt]
  \end{scope}
  \begin{scope}[cm={{1.01111,0.0,0.0,1.01111,(146.106,99.0889)}},draw=black,line join=bevel,line cap=rect,line width=0.800pt]
  \end{scope}
  \begin{scope}[cm={{1.01111,0.0,0.0,1.01111,(146.106,99.0889)}},draw=black,line join=bevel,line cap=rect,line width=0.800pt]
  \end{scope}
  \begin{scope}[cm={{1.01111,0.0,0.0,1.01111,(146.106,99.0889)}},draw=black,line join=bevel,line cap=rect,line width=0.800pt]
  \end{scope}
  \begin{scope}[cm={{1.01111,0.0,0.0,1.01111,(146.106,99.0889)}},draw=black,line join=bevel,line cap=rect,line width=0.800pt]
  \end{scope}
  \begin{scope}[cm={{1.01111,0.0,0.0,1.01111,(146.106,99.0889)}},draw=black,line join=bevel,line cap=rect,line width=0.800pt]
    \path[fill=black] (0.0000,0.0000) node[above right] () {300};



  \end{scope}
  \begin{scope}[cm={{1.01111,0.0,0.0,1.01111,(146.106,99.0889)}},draw=black,line join=bevel,line cap=rect,line width=0.800pt]
  \end{scope}
  \begin{scope}[cm={{1.01111,0.0,0.0,1.01111,(0.0,0.0)}},draw=black,line join=bevel,line cap=rect,line width=0.800pt]
  \end{scope}
  \begin{scope}[cm={{1.01111,0.0,0.0,1.01111,(0.0,0.0)}},draw=black,line join=round,line cap=round,line width=0.400pt]
    \path[draw] (43.5000,16.5000) -- (43.5000,86.5000) -- (164.5000,86.5000) -- (164.5000,16.5000) -- (43.5000,16.5000);



  \end{scope}
  \begin{scope}[cm={{1.01111,0.0,0.0,1.01111,(0.0,0.0)}},draw=black,line join=bevel,line cap=rect,line width=0.800pt]
  \end{scope}
  \begin{scope}[cm={{0.0,-1.01111,1.01111,0.0,(24.2667,78.8667)}},draw=black,line join=bevel,line cap=rect,line width=0.800pt]
  \end{scope}
  \begin{scope}[cm={{0.0,-1.01111,1.01111,0.0,(24.2667,78.8667)}},draw=black,line join=bevel,line cap=rect,line width=0.800pt]
  \end{scope}
  \begin{scope}[cm={{0.0,-1.01111,1.01111,0.0,(24.2667,78.8667)}},draw=black,line join=bevel,line cap=rect,line width=0.800pt]
  \end{scope}
  \begin{scope}[cm={{0.0,-1.01111,1.01111,0.0,(24.2667,78.8667)}},draw=black,line join=bevel,line cap=rect,line width=0.800pt]
  \end{scope}
  \begin{scope}[cm={{0.0,-1.01111,1.01111,0.0,(24.2667,78.8667)}},draw=black,line join=bevel,line cap=rect,line width=0.800pt]
  \end{scope}
  \begin{scope}[cm={{0.0,-1.01111,1.01111,0.0,(12.2667,78.8667)}},draw=black,line join=bevel,line cap=rect,line width=0.800pt]
    \path[fill=black] (0.0000,0.0000) node[above right] () {\rotatebox{90}{Power (W)}};



  \end{scope}
  \begin{scope}[cm={{0.0,-1.01111,1.01111,0.0,(24.2667,78.8667)}},draw=black,line join=bevel,line cap=rect,line width=0.800pt]
  \end{scope}
  \begin{scope}[cm={{1.01111,0.0,0.0,1.01111,(77.35,114.256)}},draw=black,line join=bevel,line cap=rect,line width=0.800pt]
  \end{scope}
  \begin{scope}[cm={{1.01111,0.0,0.0,1.01111,(77.35,114.256)}},draw=black,line join=bevel,line cap=rect,line width=0.800pt]
  \end{scope}
  \begin{scope}[cm={{1.01111,0.0,0.0,1.01111,(77.35,114.256)}},draw=black,line join=bevel,line cap=rect,line width=0.800pt]
  \end{scope}
  \begin{scope}[cm={{1.01111,0.0,0.0,1.01111,(77.35,114.256)}},draw=black,line join=bevel,line cap=rect,line width=0.800pt]
  \end{scope}
  \begin{scope}[cm={{1.01111,0.0,0.0,1.01111,(77.35,114.256)}},draw=black,line join=bevel,line cap=rect,line width=0.800pt]
  \end{scope}
  \begin{scope}[cm={{1.01111,0.0,0.0,1.01111,(77.35,114.256)}},draw=black,line join=bevel,line cap=rect,line width=0.800pt]
    \path[fill=black] (0.0000,0.0000) node[above right] () {Time (sec)};



  \end{scope}
  \begin{scope}[cm={{1.01111,0.0,0.0,1.01111,(77.35,114.256)}},draw=black,line join=bevel,line cap=rect,line width=0.800pt]
  \end{scope}
  \begin{scope}[cm={{1.01111,0.0,0.0,1.01111,(0.0,0.0)}},draw=black,line join=bevel,line cap=rect,line width=0.800pt]
  \end{scope}
  \begin{scope}[cm={{1.01111,0.0,0.0,1.01111,(0.0,0.0)}},draw=black,line join=bevel,line cap=rect,line width=0.800pt]
  \end{scope}
  \begin{scope}[cm={{1.01111,0.0,0.0,1.01111,(0.0,0.0)}},draw=black,line join=bevel,line cap=rect,line width=0.800pt]
  \end{scope}
  \begin{scope}[cm={{1.01111,0.0,0.0,1.01111,(0.0,0.0)}},draw=black,line join=round,line cap=round,line width=0.400pt]
    \path[draw] (43.5000,33.7000) -- (43.5000,33.7000) -- (44.7000,16.9000) -- (46.0000,63.6000) -- (47.2000,64.7000) -- (48.4000,82.0000) -- (49.6000,79.3000) -- (50.9000,78.9000) -- (52.1000,72.9000) -- (53.3000,63.3000) -- (54.5000,54.4000) -- (55.8000,39.1000) -- (57.0000,40.7000) -- (58.2000,42.3000) -- (59.4000,56.3000) -- (60.7000,54.7000) -- (61.9000,43.6000) -- (63.1000,31.7000) -- (64.3000,30.6000) -- (65.6000,55.5000) -- (66.8000,50.3000) -- (68.0000,57.9000) -- (69.3000,64.2000) -- (70.5000,57.5000) -- (71.7000,56.0000) -- (72.9000,52.1000) -- (74.2000,43.3000) -- (75.4000,49.3000) -- (76.6000,49.6000) -- (77.8000,55.2000) -- (79.1000,59.2000) -- (80.3000,45.3000) -- (81.5000,36.6000) -- (82.7000,30.9000) -- (84.0000,54.9000) -- (85.2000,62.6000) -- (86.4000,66.9000) -- (87.6000,77.9000) -- (88.9000,77.3000) -- (90.1000,70.1000) -- (91.3000,55.7000) -- (92.5000,48.1000) -- (93.8000,46.5000) -- (95.0000,46.0000) -- (96.2000,48.6000) -- (97.5000,53.4000) -- (98.7000,39.2000) -- (99.9000,33.6000) -- (101.1000,27.9000) -- (102.4000,50.9000) -- (103.6000,63.1000) -- (104.8000,68.7000) -- (106.0000,72.9000) -- (107.3000,60.0000) -- (108.5000,59.5000) -- (109.7000,58.9000) -- (110.9000,61.9000) -- (112.2000,58.1000) -- (113.4000,61.5000) -- (114.6000,70.0000) -- (115.9000,74.2000) -- (117.1000,67.1000) -- (118.3000,41.8000) -- (119.5000,32.8000) -- (120.8000,28.1000) -- (122.0000,50.3000) -- (123.2000,52.6000) -- (124.4000,61.3000) -- (125.7000,66.4000) -- (126.9000,59.1000) -- (128.1000,55.0000) -- (129.3000,50.4000) -- (130.6000,50.3000) -- (131.8000,49.2000) -- (133.0000,54.4000) -- (134.2000,59.8000) -- (135.5000,59.0000) -- (136.7000,57.5000) -- (137.9000,40.8000) -- (139.1000,28.9000) -- (140.4000,31.5000) -- (141.6000,46.6000) -- (142.8000,39.2000) -- (144.1000,45.5000) -- (145.3000,59.5000) -- (146.5000,58.0000) -- (147.7000,60.4000) -- (149.0000,58.0000) -- (150.2000,48.2000) -- (151.4000,42.2000) -- (152.6000,41.9000) -- (153.9000,44.4000) -- (155.1000,55.0000) -- (156.3000,47.0000) -- (157.5000,33.7000) -- (158.8000,24.3000) -- (160.0000,37.2000) -- (161.2000,58.4000) -- (162.4000,66.4000) -- (163.7000,69.0000) -- (164.9000,64.0000);



  \end{scope}
  \begin{scope}[cm={{1.01111,0.0,0.0,1.01111,(0.0,0.0)}},draw=black,line join=bevel,line cap=rect,line width=0.800pt]
  \end{scope}
  \begin{scope}[cm={{1.01111,0.0,0.0,1.01111,(0.0,0.0)}},draw=black,line join=bevel,line cap=rect,line width=0.800pt]
  \end{scope}
  \begin{scope}[cm={{1.01111,0.0,0.0,1.01111,(0.0,0.0)}},draw=black,line join=round,line cap=round,line width=0.400pt]
    \path[draw] (43.5000,16.5000) -- (43.5000,86.5000) -- (164.5000,86.5000) -- (164.5000,16.5000) -- (43.5000,16.5000);



  \end{scope}
  \begin{scope}[cm={{1.01111,0.0,0.0,1.01111,(0.0,0.0)}},draw=black,line join=bevel,line cap=rect,line width=0.800pt]
  \end{scope}
  \begin{scope}[cm={{1.0,0.0,0.0,1.0,(0.0,0.0)}},draw=black,line join=bevel,line cap=rect,line width=0.800pt]
  \end{scope}
\end{scope}

\end{tikzpicture}

};
 %  \node[inner sep=0pt] () at (2.1,-5.67)
 %      {\scriptsize 
\definecolor{ca0a0a4}{RGB}{160,160,164}


\def \globalscale {0.950000}
\begin{tikzpicture}[y=0.80pt, x=0.80pt, yscale=-\globalscale, xscale=\globalscale, inner sep=0pt, outer sep=0pt]
\begin{scope}[draw=black,line join=bevel,line cap=rect,even odd rule,line width=0.800pt]
\end{scope}
\begin{scope}[scale=1.012,draw=black,line join=bevel,line cap=rect,even odd rule,line width=0.800pt]
\end{scope}
\begin{scope}[scale=1.012,draw=black,line join=round,line cap=round,even odd rule,line width=0.400pt]
  \path[draw] (29.5000,86.5000) -- (29.5000,83.5000);



\end{scope}
\begin{scope}[scale=1.012,draw=black,line join=bevel,line cap=rect,even odd rule,line width=0.800pt]
\end{scope}
\begin{scope}[cm={{1.0125,0.0,0.0,1.0125,(24.3,99.225)}},draw=black,line join=bevel,line cap=rect,even odd rule,line width=0.800pt]
\end{scope}
\begin{scope}[cm={{1.0125,0.0,0.0,1.0125,(24.3,99.225)}},draw=black,line join=bevel,line cap=rect,even odd rule,line width=0.800pt]
\end{scope}
\begin{scope}[cm={{1.0125,0.0,0.0,1.0125,(24.3,99.225)}},draw=black,line join=bevel,line cap=rect,even odd rule,line width=0.800pt]
\end{scope}
\begin{scope}[cm={{1.0125,0.0,0.0,1.0125,(24.3,99.225)}},draw=black,line join=bevel,line cap=rect,even odd rule,line width=0.800pt]
\end{scope}
\begin{scope}[cm={{1.0125,0.0,0.0,1.0125,(24.3,99.225)}},draw=black,line join=bevel,line cap=rect,even odd rule,line width=0.800pt]
\end{scope}
\begin{scope}[cm={{1.0125,0.0,0.0,1.0125,(24.3,99.225)}},draw=black,line join=bevel,line cap=rect,even odd rule,line width=0.800pt]
  \path[fill=black] (0.0000,0.0000) node[above right] (text32) {-2};



\end{scope}
\begin{scope}[cm={{1.0125,0.0,0.0,1.0125,(24.3,99.225)}},draw=black,line join=bevel,line cap=rect,even odd rule,line width=0.800pt]
\end{scope}
\begin{scope}[scale=1.012,draw=black,line join=bevel,line cap=rect,even odd rule,line width=0.800pt]
\end{scope}
\begin{scope}[scale=1.012,draw=black,line join=round,line cap=round,even odd rule,line width=0.400pt]
  \path[draw] (63.5000,86.5000) -- (63.5000,83.5000);



\end{scope}
\begin{scope}[scale=1.012,draw=black,line join=bevel,line cap=rect,even odd rule,line width=0.800pt]
\end{scope}
\begin{scope}[cm={{1.0125,0.0,0.0,1.0125,(60.75,99.225)}},draw=black,line join=bevel,line cap=rect,even odd rule,line width=0.800pt]
\end{scope}
\begin{scope}[cm={{1.0125,0.0,0.0,1.0125,(60.75,99.225)}},draw=black,line join=bevel,line cap=rect,even odd rule,line width=0.800pt]
\end{scope}
\begin{scope}[cm={{1.0125,0.0,0.0,1.0125,(60.75,99.225)}},draw=black,line join=bevel,line cap=rect,even odd rule,line width=0.800pt]
\end{scope}
\begin{scope}[cm={{1.0125,0.0,0.0,1.0125,(60.75,99.225)}},draw=black,line join=bevel,line cap=rect,even odd rule,line width=0.800pt]
\end{scope}
\begin{scope}[cm={{1.0125,0.0,0.0,1.0125,(60.75,99.225)}},draw=black,line join=bevel,line cap=rect,even odd rule,line width=0.800pt]
\end{scope}
\begin{scope}[cm={{1.0125,0.0,0.0,1.0125,(60.75,99.225)}},draw=black,line join=bevel,line cap=rect,even odd rule,line width=0.800pt]
  \path[fill=black] (0.0000,0.0000) node[above right] (text60) {0};



\end{scope}
\begin{scope}[cm={{1.0125,0.0,0.0,1.0125,(60.75,99.225)}},draw=black,line join=bevel,line cap=rect,even odd rule,line width=0.800pt]
\end{scope}
\begin{scope}[scale=1.012,draw=black,line join=bevel,line cap=rect,even odd rule,line width=0.800pt]
\end{scope}
\begin{scope}[scale=1.012,draw=black,line join=round,line cap=round,even odd rule,line width=0.400pt]
  \path[draw] (97.5000,86.5000) -- (97.5000,83.5000);



\end{scope}
\begin{scope}[scale=1.012,draw=black,line join=bevel,line cap=rect,even odd rule,line width=0.800pt]
\end{scope}
\begin{scope}[cm={{1.0125,0.0,0.0,1.0125,(95.175,99.225)}},draw=black,line join=bevel,line cap=rect,even odd rule,line width=0.800pt]
\end{scope}
\begin{scope}[cm={{1.0125,0.0,0.0,1.0125,(95.175,99.225)}},draw=black,line join=bevel,line cap=rect,even odd rule,line width=0.800pt]
\end{scope}
\begin{scope}[cm={{1.0125,0.0,0.0,1.0125,(95.175,99.225)}},draw=black,line join=bevel,line cap=rect,even odd rule,line width=0.800pt]
\end{scope}
\begin{scope}[cm={{1.0125,0.0,0.0,1.0125,(95.175,99.225)}},draw=black,line join=bevel,line cap=rect,even odd rule,line width=0.800pt]
\end{scope}
\begin{scope}[cm={{1.0125,0.0,0.0,1.0125,(95.175,99.225)}},draw=black,line join=bevel,line cap=rect,even odd rule,line width=0.800pt]
\end{scope}
\begin{scope}[cm={{1.0125,0.0,0.0,1.0125,(95.175,99.225)}},draw=black,line join=bevel,line cap=rect,even odd rule,line width=0.800pt]
  \path[fill=black] (0.0000,0.0000) node[above right] (text88) {2};



\end{scope}
\begin{scope}[cm={{1.0125,0.0,0.0,1.0125,(95.175,99.225)}},draw=black,line join=bevel,line cap=rect,even odd rule,line width=0.800pt]
\end{scope}
\begin{scope}[scale=1.012,draw=black,line join=bevel,line cap=rect,even odd rule,line width=0.800pt]
\end{scope}
\begin{scope}[scale=1.012,draw=ca0a0a4,dash pattern=on 0.40pt off 0.80pt,line join=round,line cap=round,even odd rule,line width=0.400pt]
  \path[draw] (12.5000,86.5000) -- (113.5000,86.5000);



\end{scope}
\begin{scope}[scale=1.012,draw=black,line join=round,line cap=round,even odd rule,line width=0.400pt]
  \path[draw] (113.5000,86.5000) -- (113.5000,86.5000) -- (110.5000,86.5000);



\end{scope}
\begin{scope}[scale=1.012,draw=black,line join=bevel,line cap=rect,even odd rule,line width=0.800pt]
\end{scope}
\begin{scope}[cm={{1.0125,0.0,0.0,1.0125,(118.462,91.125)}},draw=black,line join=bevel,line cap=rect,even odd rule,line width=0.800pt]
\end{scope}
\begin{scope}[cm={{1.0125,0.0,0.0,1.0125,(118.462,91.125)}},draw=black,line join=bevel,line cap=rect,even odd rule,line width=0.800pt]
\end{scope}
\begin{scope}[cm={{1.0125,0.0,0.0,1.0125,(118.462,91.125)}},draw=black,line join=bevel,line cap=rect,even odd rule,line width=0.800pt]
\end{scope}
\begin{scope}[cm={{1.0125,0.0,0.0,1.0125,(118.462,91.125)}},draw=black,line join=bevel,line cap=rect,even odd rule,line width=0.800pt]
\end{scope}
\begin{scope}[cm={{1.0125,0.0,0.0,1.0125,(118.462,91.125)}},draw=black,line join=bevel,line cap=rect,even odd rule,line width=0.800pt]
\end{scope}
\begin{scope}[cm={{1.0125,0.0,0.0,1.0125,(118.462,91.125)}},draw=black,line join=bevel,line cap=rect,even odd rule,line width=0.800pt]
  \path[fill=black] (0.0000,0.0000) node[above right] (text116) {0};



\end{scope}
\begin{scope}[cm={{1.0125,0.0,0.0,1.0125,(118.462,91.125)}},draw=black,line join=bevel,line cap=rect,even odd rule,line width=0.800pt]
\end{scope}
\begin{scope}[scale=1.012,draw=black,line join=bevel,line cap=rect,even odd rule,line width=0.800pt]
\end{scope}
\begin{scope}[scale=1.012,draw=black,line join=round,line cap=round,even odd rule,line width=0.400pt]
  \path[draw] (113.5000,67.5000) -- (113.5000,67.5000) -- (110.5000,67.5000);



\end{scope}
\begin{scope}[scale=1.012,draw=black,line join=bevel,line cap=rect,even odd rule,line width=0.800pt]
\end{scope}
\begin{scope}[cm={{1.0125,0.0,0.0,1.0125,(118.462,72.9)}},draw=black,line join=bevel,line cap=rect,even odd rule,line width=0.800pt]
\end{scope}
\begin{scope}[cm={{1.0125,0.0,0.0,1.0125,(118.462,72.9)}},draw=black,line join=bevel,line cap=rect,even odd rule,line width=0.800pt]
\end{scope}
\begin{scope}[cm={{1.0125,0.0,0.0,1.0125,(118.462,72.9)}},draw=black,line join=bevel,line cap=rect,even odd rule,line width=0.800pt]
\end{scope}
\begin{scope}[cm={{1.0125,0.0,0.0,1.0125,(118.462,72.9)}},draw=black,line join=bevel,line cap=rect,even odd rule,line width=0.800pt]
\end{scope}
\begin{scope}[cm={{1.0125,0.0,0.0,1.0125,(118.462,72.9)}},draw=black,line join=bevel,line cap=rect,even odd rule,line width=0.800pt]
\end{scope}
\begin{scope}[cm={{1.0125,0.0,0.0,1.0125,(118.462,72.9)}},draw=black,line join=bevel,line cap=rect,even odd rule,line width=0.800pt]
  \path[fill=black] (0.0000,0.0000) node[above right] (text144) {2};



\end{scope}
\begin{scope}[cm={{1.0125,0.0,0.0,1.0125,(118.462,72.9)}},draw=black,line join=bevel,line cap=rect,even odd rule,line width=0.800pt]
\end{scope}
\begin{scope}[scale=1.012,draw=black,line join=bevel,line cap=rect,even odd rule,line width=0.800pt]
\end{scope}
\begin{scope}[scale=1.012,draw=black,line join=round,line cap=round,even odd rule,line width=0.400pt]
  \path[draw] (113.5000,48.5000) -- (113.5000,48.5000) -- (110.5000,48.5000);



\end{scope}
\begin{scope}[scale=1.012,draw=black,line join=bevel,line cap=rect,even odd rule,line width=0.800pt]
\end{scope}
\begin{scope}[cm={{1.0125,0.0,0.0,1.0125,(118.969,53.6625)}},draw=black,line join=bevel,line cap=rect,even odd rule,line width=0.800pt]
\end{scope}
\begin{scope}[cm={{1.0125,0.0,0.0,1.0125,(118.969,53.6625)}},draw=black,line join=bevel,line cap=rect,even odd rule,line width=0.800pt]
\end{scope}
\begin{scope}[cm={{1.0125,0.0,0.0,1.0125,(118.969,53.6625)}},draw=black,line join=bevel,line cap=rect,even odd rule,line width=0.800pt]
\end{scope}
\begin{scope}[cm={{1.0125,0.0,0.0,1.0125,(118.969,53.6625)}},draw=black,line join=bevel,line cap=rect,even odd rule,line width=0.800pt]
\end{scope}
\begin{scope}[cm={{1.0125,0.0,0.0,1.0125,(118.969,53.6625)}},draw=black,line join=bevel,line cap=rect,even odd rule,line width=0.800pt]
\end{scope}
\begin{scope}[cm={{1.0125,0.0,0.0,1.0125,(118.969,53.6625)}},draw=black,line join=bevel,line cap=rect,even odd rule,line width=0.800pt]
  \path[fill=black] (0.0000,0.0000) node[above right] (text172) {4};



  \path[fill=black,even odd rule,line width=0.800pt] (-0.0065,-27.8667) node[above right] (text172-3) {4000};



\end{scope}
\begin{scope}[cm={{1.0125,0.0,0.0,1.0125,(118.969,53.6625)}},draw=black,line join=bevel,line cap=rect,even odd rule,line width=0.800pt]
\end{scope}
\begin{scope}[scale=1.012,draw=black,line join=bevel,line cap=rect,even odd rule,line width=0.800pt]
\end{scope}
\begin{scope}[scale=1.012,draw=black,line join=bevel,line cap=rect,even odd rule,line width=0.800pt]
\end{scope}
\begin{scope}[cm={{0.0,-1.0125,1.0125,0.0,(134.662,91.125)}},draw=black,line join=bevel,line cap=rect,even odd rule,line width=0.800pt]
\end{scope}
\begin{scope}[cm={{0.0,-1.0125,1.0125,0.0,(134.662,91.125)}},draw=black,line join=bevel,line cap=rect,even odd rule,line width=0.800pt]
\end{scope}
\begin{scope}[cm={{0.0,-1.0125,1.0125,0.0,(134.662,91.125)}},draw=black,line join=bevel,line cap=rect,even odd rule,line width=0.800pt]
\end{scope}
\begin{scope}[cm={{0.0,-1.0125,1.0125,0.0,(134.662,91.125)}},draw=black,line join=bevel,line cap=rect,even odd rule,line width=0.800pt]
\end{scope}
\begin{scope}[cm={{0.0,-1.0125,1.0125,0.0,(134.662,91.125)}},draw=black,line join=bevel,line cap=rect,even odd rule,line width=0.800pt]
\end{scope}
\begin{scope}[cm={{0.0,-1.0125,1.0125,0.0,(148.162,100.625)}},draw=black,line join=bevel,line cap=rect,even odd rule,line width=0.800pt]
  \path[fill=black] (0.0000,0.0000) node[above right] (text196) {\rotatebox{90}{Spectrum ($10^2$\hspace*{.2ex}dB)}};



\end{scope}
\begin{scope}[cm={{0.0,-1.0125,1.0125,0.0,(134.662,91.125)}},draw=black,line join=bevel,line cap=rect,even odd rule,line width=0.800pt]
\end{scope}
\begin{scope}[cm={{1.0125,0.0,0.0,1.0125,(25.8187,114.412)}},draw=black,line join=bevel,line cap=rect,even odd rule,line width=0.800pt]
\end{scope}
\begin{scope}[cm={{1.0125,0.0,0.0,1.0125,(25.8187,114.412)}},draw=black,line join=bevel,line cap=rect,even odd rule,line width=0.800pt]
\end{scope}
\begin{scope}[cm={{1.0125,0.0,0.0,1.0125,(25.8187,114.412)}},draw=black,line join=bevel,line cap=rect,even odd rule,line width=0.800pt]
\end{scope}
\begin{scope}[cm={{1.0125,0.0,0.0,1.0125,(25.8187,114.412)}},draw=black,line join=bevel,line cap=rect,even odd rule,line width=0.800pt]
\end{scope}
\begin{scope}[cm={{1.0125,0.0,0.0,1.0125,(25.8187,114.412)}},draw=black,line join=bevel,line cap=rect,even odd rule,line width=0.800pt]
\end{scope}
\begin{scope}[cm={{1.0125,0.0,0.0,1.0125,(17.8187,114.412)}},draw=black,line join=bevel,line cap=rect,even odd rule,line width=0.800pt]
  \path[fill=black] (0.0000,0.0000) node[above right] (text212) {Frequency ($10^{-2}$\hspace*{.2ex}Hz)};



\end{scope}
\begin{scope}[cm={{1.0125,0.0,0.0,1.0125,(25.8187,114.412)}},draw=black,line join=bevel,line cap=rect,even odd rule,line width=0.800pt]
\end{scope}
\begin{scope}[scale=1.012,draw=black,line join=bevel,line cap=rect,even odd rule,line width=0.800pt]
\end{scope}
\begin{scope}[scale=1.012,draw=black,line join=bevel,line cap=rect,even odd rule,line width=0.800pt]
\end{scope}
\begin{scope}[scale=1.012,draw=black,line join=bevel,line cap=rect,even odd rule,line width=0.800pt]
\end{scope}
\begin{scope}[scale=1.012,draw=black,line join=round,line cap=round,even odd rule,line width=0.400pt]
  \path[draw] (12.5000,86.2000) -- (12.5000,86.2000) -- (13.6000,86.2000) -- (14.6000,86.4000) -- (15.7000,85.0000) -- (16.7000,86.4000) -- (17.8000,85.4000) -- (18.8000,86.1000) -- (19.9000,85.9000) -- (20.9000,86.2000) -- (22.0000,86.0000) -- (23.1000,86.0000) -- (24.1000,85.4000) -- (25.2000,86.0000) -- (26.2000,86.2000) -- (27.3000,86.2000) -- (28.3000,86.1000) -- (29.4000,85.3000) -- (30.5000,85.5000) -- (31.5000,86.2000) -- (32.6000,86.1000) -- (33.6000,86.4000) -- (34.7000,85.4000) -- (35.7000,84.1000) -- (36.8000,86.1000) -- (37.9000,85.9000) -- (38.9000,86.3000) -- (40.0000,86.0000) -- (41.0000,86.1000) -- (42.1000,84.0000) -- (43.1000,81.0000) -- (44.2000,84.4000) -- (45.2000,84.9000) -- (46.3000,84.6000) -- (47.4000,85.0000) -- (48.4000,83.2000) -- (49.5000,43.3000) -- (50.5000,84.9000) -- (51.6000,83.8000) -- (52.6000,79.7000) -- (53.7000,84.3000) -- (54.8000,83.0000) -- (55.8000,83.4000) -- (56.9000,51.8000) -- (57.9000,86.1000) -- (59.0000,77.0000) -- (60.0000,81.0000) -- (61.1000,79.0000) -- (62.1000,85.1000) -- (62.1000,33.7178);



  \path[draw] (64.3000,33.7516) -- (64.3000,85.1000) -- (65.3000,79.0000) -- (66.4000,81.0000) -- (67.4000,77.0000) -- (68.5000,86.1000) -- (69.5000,51.8000) -- (70.6000,83.4000) -- (71.6000,83.0000) -- (72.7000,84.3000) -- (73.8000,79.7000) -- (74.8000,83.8000) -- (75.9000,84.9000) -- (76.9000,43.3000) -- (78.0000,83.2000) -- (79.0000,85.0000) -- (80.1000,84.6000) -- (81.2000,84.9000) -- (82.2000,84.4000) -- (83.3000,81.0000) -- (84.3000,84.0000) -- (85.4000,86.1000) -- (86.4000,86.0000) -- (87.5000,86.3000) -- (88.6000,85.9000) -- (89.6000,86.1000) -- (90.7000,84.1000) -- (91.7000,85.4000) -- (92.8000,86.4000) -- (93.8000,86.1000) -- (94.9000,86.2000) -- (95.9000,85.5000) -- (97.0000,85.3000) -- (98.1000,86.1000) -- (99.1000,86.2000) -- (100.2000,86.2000) -- (101.2000,86.0000) -- (102.3000,85.4000) -- (103.3000,86.0000) -- (104.4000,86.0000) -- (105.5000,86.2000) -- (106.5000,85.9000) -- (107.6000,86.1000) -- (108.6000,85.4000) -- (109.7000,86.4000) -- (110.7000,85.0000) -- (111.8000,86.4000) -- (112.8000,86.2000) -- (113.9000,86.2000) -- (113.9000,86.2000);



\end{scope}
\begin{scope}[scale=1.012,draw=black,line join=bevel,line cap=rect,even odd rule,line width=0.800pt]
\end{scope}
\begin{scope}[scale=1.012,draw=black,line join=bevel,line cap=rect,even odd rule,line width=0.800pt]
\end{scope}
\begin{scope}[scale=1.012,draw=black,line join=bevel,line cap=rect,even odd rule,line width=0.800pt]
\end{scope}
\begin{scope}[draw=black,line join=bevel,line cap=rect,even odd rule,line width=0.800pt]
\end{scope}
\begin{scope}[cm={{1.01288,0.0,0.0,1.0125,(8.01061,0.31012)}},draw=ca0a0a4,dash pattern=on 0.40pt off 0.80pt,line join=round,line cap=round,even odd rule,line width=0.400pt]
\end{scope}
\begin{scope}[cm={{1.0125,0.0,0.0,1.0125,(0.00644,0.30976)}},draw=ca0a0a4,dash pattern=on 0.40pt off 0.80pt,line join=round,line cap=round,even odd rule,line width=0.400pt]
\end{scope}
\begin{scope}[cm={{1.0125,0.0,0.0,1.0125,(0.14812,-45.24757)}},draw=ca0a0a4,dash pattern=on 0.40pt off 0.80pt,line join=round,line cap=round,even odd rule,line width=0.400pt]
  \path[color=black,fill=black,line join=round,line cap=round,miter limit=4.00,nonzero rule,line width=0.502pt] (60.9979,75.6531) .. controls (60.7904,75.7177) and (60.6555,75.8770) .. (60.6966,76.0088) .. controls (60.7376,76.1406) and (60.9392,76.1951) .. (61.1467,76.1304) -- (65.6553,74.7253) .. controls (65.8628,74.6606) and (65.9977,74.5013) .. (65.9566,74.3695) .. controls (65.9156,74.2377) and (65.7140,74.1832) .. (65.5065,74.2479) -- cycle;



  \begin{scope}[cm={{1.50288,-0.46838,0.29754,0.95471,(-111.368,107.84373)}},draw=black,line join=miter,line cap=rect,even odd rule,line width=0.400pt]
    \path[color=black,fill=black,line join=miter,line cap=rect,miter limit=4.00,nonzero rule,line width=0.400pt] (110.3148,22.2500) .. controls (110.1767,22.2500) and (110.0648,22.3619) .. (110.0648,22.5000) .. controls (110.0648,22.6381) and (110.1767,22.7500) .. (110.3148,22.7500) -- (113.3148,22.7500) .. controls (113.4529,22.7500) and (113.5648,22.6381) .. (113.5648,22.5000) .. controls (113.5648,22.3619) and (113.4529,22.2500) .. (113.3148,22.2500) -- cycle;



  \end{scope}
  \path[color=black,fill=black,line join=round,line cap=round,miter limit=4.00,nonzero rule,line width=0.502pt] (10.1034,75.7781) .. controls (9.8959,75.8427) and (9.7610,76.0020) .. (9.8020,76.1338) .. controls (9.8431,76.2656) and (10.0447,76.3201) .. (10.2522,76.2554) -- (14.7608,74.8503) .. controls (14.9683,74.7856) and (15.1032,74.6263) .. (15.0621,74.4945) .. controls (15.0210,74.3627) and (14.8195,74.3082) .. (14.6120,74.3729) -- cycle;



  \begin{scope}[cm={{1.50288,-0.46838,0.29754,0.95471,(-162.26251,107.96873)}},draw=black,line join=miter,line cap=rect,even odd rule,line width=0.400pt]
    \path[color=black,fill=black,line join=miter,line cap=rect,miter limit=4.00,nonzero rule,line width=0.400pt] (110.3148,22.2500) .. controls (110.1767,22.2500) and (110.0648,22.3619) .. (110.0648,22.5000) .. controls (110.0648,22.6381) and (110.1767,22.7500) .. (110.3148,22.7500) -- (113.3148,22.7500) .. controls (113.4529,22.7500) and (113.5648,22.6381) .. (113.5648,22.5000) .. controls (113.5648,22.3619) and (113.4529,22.2500) .. (113.3148,22.2500) -- cycle;



  \end{scope}
\end{scope}
\begin{scope}[cm={{1.0125,0.0,0.0,1.0125,(0.00644,0.30976)}},draw=black,line join=round,line cap=round,even odd rule,line width=0.400pt]
  \path[color=black,fill=black,line join=miter,line cap=butt,miter limit=4.00,nonzero rule,line width=0.400pt] (110.3148,22.2500) .. controls (110.1767,22.2500) and (110.0648,22.3619) .. (110.0648,22.5000) .. controls (110.0648,22.6381) and (110.1767,22.7500) .. (110.3148,22.7500) -- (113.3148,22.7500) .. controls (113.4529,22.7500) and (113.5648,22.6381) .. (113.5648,22.5000) .. controls (113.5648,22.3619) and (113.4529,22.2500) .. (113.3148,22.2500) -- cycle;



\end{scope}
\begin{scope}[cm={{1.0125,0.0,0.0,1.0125,(0.15405,0.52227)}},draw=black,line join=round,line cap=round,even odd rule,line width=0.400pt]
  \path[color=black,fill=black,line join=miter,line cap=butt,miter limit=4.00,nonzero rule,line width=0.400pt] (113.5824,29.1349) -- (113.5648,16.5000) .. controls (113.5648,16.3619) and (113.4529,16.2500) .. (113.3148,16.2500) -- (12.5137,16.2500) -- (12.5000,16.2500) .. controls (12.4862,16.2502) and (12.4725,16.2515) .. (12.4590,16.2539) .. controls (12.4451,16.2560) and (12.4313,16.2593) .. (12.4180,16.2637) .. controls (12.3781,16.2780) and (12.3424,16.3022) .. (12.3145,16.3340) .. controls (12.2733,16.3796) and (12.2504,16.4387) .. (12.2500,16.5000) -- (12.2487,29.1320) -- (12.9297,29.1333) -- (12.8979,16.8982) -- (112.9167,16.8982) -- (112.9137,29.1371);



\end{scope}
\begin{scope}[cm={{1.01288,0.0,0.0,1.0125,(8.01061,0.31012)}},draw=black,line join=round,line cap=round,even odd rule,line width=0.400pt]
  \path[color=black,fill=black,line join=miter,line cap=butt,miter limit=4.00,nonzero rule,line width=0.400pt] (56.0430,29.3507) .. controls (56.0178,27.6026) and (55.9492,24.1035) .. (55.9492,24.1035) -- (55.9492,24.0957) .. controls (55.9462,23.9637) and (55.8410,23.8567) .. (55.7090,23.8516) .. controls (55.5694,23.8461) and (55.4524,23.9561) .. (55.4492,24.0957) -- (55.4492,24.1035) -- (55.3661,29.3497);



\end{scope}
\begin{scope}[cm={{1.52166,-0.47424,0.30126,0.96664,(-62.10352,62.47593)}},draw=black,line join=round,line cap=round,even odd rule,line width=0.400pt]
  \path[color=black,fill=black,line join=round,line cap=round,miter limit=4.00,nonzero rule,line width=0.400pt] (110.3148,22.2500) .. controls (110.1767,22.2500) and (110.0648,22.3619) .. (110.0648,22.5000) .. controls (110.0648,22.6381) and (110.1767,22.7500) .. (110.3148,22.7500) -- (113.3148,22.7500) .. controls (113.4529,22.7500) and (113.5648,22.6381) .. (113.5648,22.5000) .. controls (113.5648,22.3619) and (113.4529,22.2500) .. (113.3148,22.2500) -- cycle;



\end{scope}
\begin{scope}[cm={{1.52166,-0.47424,0.30126,0.96664,(-62.05879,64.26145)}},draw=black,line join=miter,line cap=rect,even odd rule,line width=0.400pt]
  \path[color=black,fill=black,line join=miter,line cap=rect,miter limit=4.00,nonzero rule,line width=0.400pt] (110.3148,22.2500) .. controls (110.1767,22.2500) and (110.0648,22.3619) .. (110.0648,22.5000) .. controls (110.0648,22.6381) and (110.1767,22.7500) .. (110.3148,22.7500) -- (113.3148,22.7500) .. controls (113.4529,22.7500) and (113.5648,22.6381) .. (113.5648,22.5000) .. controls (113.5648,22.3619) and (113.4529,22.2500) .. (113.3148,22.2500) -- cycle;



\end{scope}
\begin{scope}[cm={{-1.0125,0.0,0.0,-1.0125,(127.53769,104.28755)}},draw=black,line join=round,line cap=round,even odd rule,line width=0.400pt]
  \path[color=black,fill=black,line join=miter,line cap=butt,miter limit=4.00,nonzero rule,line width=0.400pt] (113.5824,69.1349) -- (113.5648,16.5000) .. controls (113.5648,16.3619) and (113.4529,16.2500) .. (113.3148,16.2500) -- (12.4393,16.2500) -- (12.4256,16.2500) .. controls (12.4119,16.2502) and (12.3981,16.2515) .. (12.3846,16.2539) .. controls (12.3707,16.2560) and (12.3570,16.2593) .. (12.3436,16.2637) .. controls (12.3037,16.2780) and (12.2681,16.3022) .. (12.2401,16.3340) .. controls (12.1989,16.3796) and (12.1760,16.4387) .. (12.1756,16.5000) -- (12.1743,69.1320) -- (12.8558,69.1333) -- (12.8239,16.8978) -- (112.9165,16.8978) -- (112.9135,69.1371);



\end{scope}

\end{tikzpicture}

};
 %\end{tikzpicture}
  \vspace*{.6ex}
  \caption{%Illustrative example of planning-scheduling. 
  %The figure shows 
  An initial plan~(in~\ref{sth:i}) {\color{black}is} re-planned online, changing the %resolution, 
  detection rate or other computational aspects~(in~\ref{sth:ii}) and %changing 
  the number of fly-bys or other motion aspects~(in~\ref{sth:iii}). 
  {\color{black}On the right} %shows 
  {\color{black}are} the energy data of a %physical 
  fixed-wing aerial robot flying a static coverage plan similar to the one illustrated\hspace*{.5ex}here\hspace*{.5ex}{\color{black}and %; %below is 
  }%. Bottom-right shows 
  the spectrum analysis, %of the %energy 
  %data, 
  revealing the periodicity %of the data 
  exploited in the %overall 
  energy model.
  %Follow the collected energy data of a physical aerial robot flying the static coverage.
  }
  \label{fig:il-abs}
  \vspace*{-3.8ex}
\end{figure}

There are numerous planning approaches applied to a variety of robots. An instance is an algorithm selecting an energy-optimized trajectory~\cite{mei2004energy} by, e.g., maximizing the operational time~\cite{wahab2015energy}. Many approaches apply to a small number of robots~\cite{kim2005energy} and focus exclusively on planning the trajectory~\cite{kim2008minimum}, despite compelling evidence of the energy influence of onboard {\color{black}computations}~\cite{mei2005case,ondruska2015scheduled,sudhakar2020balancing,brateman2006energy}. In view of the availability of powerful heterogeneous computing hardware~\cite{rizvi2017general}, the use of onboard computations is further expected to increase in the foreseeable future~\cite{%abramov2012real,
%satria2016real,
jaramillo2019visual}. In this context, planning-scheduling energy awareness is a recent research direction% energy awareness for mobile robots are scarce
~\cite{brateman2006energy,sudhakar2020balancing,lahijanian2018resource,ondruska2015scheduled}. Early studies (2000--2010) varied hardware-dependent aspects, e.g., frequency {\color{black}and} voltage, along with motion aspects, e.g., motor and travel velocities~\cite{mei2005case,brateman2006energy,zhang2007low,sadrpour2013mission} whereas the literature from the past decade derives energy-aware plans-schedules in broader terms. These include simultaneous considerations for planning-scheduling in perception~\cite{ondruska2015scheduled}, localization~\cite{lahijanian2018resource}, navigation~\cite{ho2019qos}, and anytime planning~\cite{sudhakar2020balancing}.
These studies are focused on ground-based robots~\cite{mei2005case,sadrpour2013mission,lahijanian2018resource,ondruska2015scheduled}, yet, aerial robots are particularly affected by energy considerations, as it would be generally required to land to recharge the battery. 
{\color{black}In terms of aerial coverage, past work considers criteria including the completeness of the coverage and resolution~\cite{difranco2015energy}, and deals with aspects such as the quality of the cover~\cite{difranco2016coverage}, but neglects the energy expenditure of computations and favors rotary-wing aerial robots rather than aerial robots broadly.} 
Such a state of practice has prompted us to propose the planning-scheduling approach for autonomous aerial robots, combining the past body of knowledge but addressing aerial robots' peculiarities such as the atmospheric, battery, and turning radius constraints. Numerical simulations and experimental data of %both 
static and dynamic plans and schedules show improved power savings and fault tolerance with the %aerial 
robot %s
remedying in-flight failures. 
 
Our focus is on fixed wings, i.e., airborne robots where wings provide lift, propellers provide forward thrust, and control surfaces perform maneuvering. Here, motion and computations energies are within an order of magnitude from each other~\cite{seewald2020mechanical,zamanakos2020energy}. %Indeed 
{\color{black}T}here are other classes where planning-scheduling energy awareness leads to irrelevant savings, i.e., when the motion energy contribution far outreaches the computations or vice-versa. The {\color{black}motion outreaching computation energy} frequently happens with rotary-wing aerial robots (e.g., quadrotors or quadcopters, hexacopters, etc.){\color{black}, the opposite occurs with} lighter-than-air aerial robots (e.g., blimps). It is %a 
common %theme 
{\color{black}in} %wider 
planning-scheduling literature, focusing on %energy-
efficient ground-based robots such as Pioneer~3DX~\cite{ho2019qos,mei2005case}, ARC~Q14~\cite{ondruska2015scheduled,lahijanian2018resource}, and Pack-Bot~UGV~\cite{sadrpour2013mission}.
%\begin{ctb}
%  We extend the literature on planning-schedu- ling to CPP with aerial robots, model the overall and computations energies and battery evolution, and derive a variable coverage motion for constrained systems, e.g., fixed wings.
%\end{ctb}

To guarantee energy awareness, our approach uses optimal control {\color{black} and heuristics} where both the paths and schedules variations are trajectories, varying between given bounds (i.e., physical constraints of the robot and computing hardware, quality of service, desired quality of the coverage, etc.). Past planning-scheduling studies also employ optimization techniques~\cite{brateman2006energy,zhang2007low,ondruska2015scheduled,lahijanian2018resource}; some use a greedy approach~\cite{mei2005case,sudhakar2020balancing,sadrpour2013mission}; whereas others use reinforcement learning-based approaches~\cite{ho2019qos,ho2018towards}. {\color{black}Hybrid approaches~\cite{ondruska2015scheduled} are also available, where the techniques are mixed.} Both the path{\color{black}s} and schedules variations trajectories are derived for future time instants employing computations and overall energies and battery models. The energy model for the computations uses regressional analysis from our earlier study on heterogeneous computing hardware% energy modeling
~\cite{seewald2019coarse,seewald2019component}, whereas the battery uses an equivalent circuit model (ECM) from the literature~\cite{%he2011evaluation,
hinz2019comparison,mousavi2014various}. The overall model wraps these two aspects in a cohesive model that uses dynamics modeling to predict the energy behavior of future plans and schedules. In Fig.~\ref{fig:il-abs}, collected energy data ({\color{black}top-right}) and spectrum analysis ({\color{black}below}) of a fixed wing %aerial robot 
flying CPP motivate the overall energy model: the evolution is periodic--CPP often involves repetitive motions to cover the space~\cite{choset2001coverage,galceran2013survey}--an observation exploited in Sec.~\ref{sec:energy-model}.% for the computations and motion energies and battery models.

The %remaining sections of the letter are 
{\color{black}remainder is} then organized as follows. Sec.~\ref{sec:prob} provides basic constructs%, such as the concepts of the stages, path functions, triggering and final points, and plan, as well as 
{ \color{black}and} %the problem formulation.
Sec.~\ref{sec:algo} describes %in detail 
the methodology of planning-scheduling. Sec.~\ref{sec:experimental} presents the results, %and showcases the performances, 
and Sec.~\ref{sec:conclusion} concludes and provides future perspectives. %Appendices~\ref{app:proof-eqv}--\ref{app:proof-bat} provide supplementary material.


%%%%%%%%%%%%%%%%%%%%%%%%%%%%%%%%
\section{Problem Formulation}  %
\label{sec:prob}               %
                               %
%Before defining the problem of energy-aware planning-scheduling in Sec.~\ref{sec:pbfor}, Sec.~\ref{sec:prelim} provides necessary preliminaries. 
%For CPP and, e.g., pattern detections in precision agriculture, 
{\color{black}W}e assume %that %aerial 
{\color{black}the }robot contains a \emph{plan} composed of \emph{stages}. At each, %stage, %the aerial robot 
{\color{black}it }travels a path and runs a schedule on the computing hardware. Both are %to be 
altered in Sec.~\ref{sec:algo} within given boundaries with \emph{path}- and \emph{computation}-specific \emph{parameters}.%with information about the path and the schedule.

\begin{figure}[t]
  \footnotesize
  \begin{minipage}[t]{0.65\columnwidth}
    \centering
    
\definecolor{cDEDEDE}{RGB}{222,222,222}
\definecolor{c989898}{RGB}{152,152,152}
\definecolor{cFFFFFF}{RGB}{255,255,255}
\definecolor{c2B2B2B}{RGB}{43,43,43}
\definecolor{c9B9B9B}{RGB}{155,155,155}
\definecolor{c0000FF}{RGB}{0,0,255}


\def \globalscale {.870000}
\begin{tikzpicture}[y=0.80pt, x=0.80pt, yscale=-\globalscale, xscale=\globalscale, inner sep=0pt, outer sep=0pt]
  \path[fill=cDEDEDE,line join=round,even odd rule,line width=0.160pt] (42.8768,176.5960) -- (17.9813,176.5960) .. controls (17.9813,137.6550) and (49.5496,106.0870) .. (88.4909,106.0870) -- (88.4909,130.7340) .. controls (63.2867,130.8920) and (42.9005,151.3600) .. (42.8768,176.5960) -- cycle;



  \path[fill=cDEDEDE,line join=round,even odd rule,line width=0.160pt] (17.3956,255.7850) -- (42.8792,255.7850) -- (42.8799,176.0840) -- (17.3962,176.0840) -- (17.3956,255.7850) -- cycle;



  \path[fill=cDEDEDE,line join=round,even odd rule,line width=0.160pt] (61.3636,176.3530) -- (36.4682,176.3530) .. controls (36.4682,137.4120) and (68.0365,105.8440) .. (106.9780,105.8440) -- (106.9780,130.4910) .. controls (81.7736,130.6490) and (61.3874,151.1170) .. (61.3636,176.3530) -- cycle;



  \path[fill=cDEDEDE,line join=round,even odd rule,line width=0.160pt] (106.9930,130.4730) -- (106.9930,105.5780) .. controls (145.9340,105.5780) and (177.5030,137.1460) .. (177.5030,176.0870) -- (152.8550,176.0870) .. controls (152.6980,150.8830) and (132.2290,130.4970) .. (106.9930,130.4730) -- cycle;



  \path[fill=cDEDEDE,line join=round,even odd rule,line width=0.160pt] (35.8695,255.7970) -- (61.3532,255.7970) -- (61.3535,176.0940) -- (35.8698,176.0940) -- (35.8695,255.7970) -- cycle;



  \path[fill=cDEDEDE,line join=round,even odd rule,line width=0.160pt] (152.7850,255.7810) -- (178.2690,255.7810) -- (178.2870,175.9450) -- (152.8030,175.9450) -- (152.7850,255.7810) -- cycle;



  \path[draw=c989898,line join=round,line width=0.512pt] (88.6908,177.1150) ellipse (1.9899cm and 1.9899cm);



  \path[cm={{1.0,0.0,0.0,1.0,(191.0,218.0)}}] (0.0000,0.0000) node[above right] () {qk};



    \path[fill=cFFFFFF,line join=round,line width=0.160pt,rounded corners=0.0000cm] (50.0136,231.2480) rectangle (68.4310,249.6655);



    \path[cm={{1.0,0.0,0.0,1.0,(47.0,245.0)}}] (0.0000,0.0000) node[above right] () {VX};



  \path[draw=c2B2B2B,line join=round,line width=0.512pt] (107.0350,176.5420) ellipse (1.9899cm and 1.9899cm);



  \path[draw=c2B2B2B,line join=round,line width=0.512pt] (36.1073,83.2720) -- (36.1072,255.5570);



  \path[draw=c2B2B2B,line join=round,line width=0.512pt] (177.6410,83.2929) -- (177.6410,255.5790);



  \path[draw=c2B2B2B,line join=round,line width=0.512pt] (109.2040,178.6380) -- (105.4740,174.9080);



  \path[draw=c2B2B2B,line join=round,line width=0.512pt] (105.4750,178.6370) -- (109.2050,174.9080);



  \path[fill=black,line join=round,line width=0.256pt] (177.0200,218.0430) -- (177.0240,212.7090) -- (178.3040,212.7100) -- (178.3000,218.0440) -- (177.0200,218.0430) -- cycle(177.0280,207.3760) -- (177.0320,202.0430) -- (178.3120,202.0440) -- (178.3080,207.3770) -- (177.0280,207.3760) -- cycle(177.0370,196.7090) -- (177.0410,191.3760) -- (178.3210,191.3770) -- (178.3170,196.7100) -- (177.0370,196.7090) -- cycle(177.0450,186.0430) -- (177.0490,180.7090) -- (178.3290,180.7100) -- (178.3250,186.0440) -- (177.0450,186.0430) -- cycle(177.0530,175.3760) -- (177.0540,173.5390) -- (177.0650,173.4140) -- (177.1010,173.2930) -- (177.1610,173.1830) -- (177.2420,173.0860) -- (177.3380,173.0060) -- (177.4490,172.9460) -- (177.5690,172.9100) -- (177.6940,172.8990) -- (177.0000,172.8690) -- (176.7740,170.8340) -- (176.6700,170.1480) -- (177.9390,169.9840) -- (178.0430,170.6700) -- (178.2750,172.7540) -- (177.6940,174.1790) -- (178.3340,173.5400) -- (178.3330,175.3770) -- (177.0530,175.3760) -- cycle(175.7470,164.9240) -- (175.4680,163.5210) -- (174.5110,159.7710) -- (175.7600,159.4900) -- (176.7160,163.2400) -- (177.0090,164.7060) -- (175.7470,164.9240) -- cycle(172.9600,154.7070) -- (172.4530,153.1270) -- (171.1010,149.7570) -- (172.3060,149.3250) -- (173.6580,152.6960) -- (174.1910,154.3560) -- (172.9600,154.7070) -- cycle(168.9430,144.9370) -- (167.0500,141.1410) -- (166.5030,140.2620) -- (167.6220,139.6400) -- (168.1690,140.5190) -- (170.1120,144.4160) -- (168.9430,144.9370) -- cycle(163.6880,135.7320) -- (163.2420,135.0140) -- (160.5200,131.5260) -- (161.5710,130.7960) -- (164.2930,134.2840) -- (164.8070,135.1110) -- (163.6880,135.7320) -- cycle(157.0940,127.5450) -- (153.3350,123.7620) -- (154.2980,122.9180) -- (158.0570,126.7020) -- (157.0940,127.5450) -- cycle(149.2200,120.5030) -- (146.3770,118.3070) -- (144.9400,117.4880) -- (145.6520,116.4240) -- (147.0880,117.2430) -- (150.0690,119.5450) -- (149.2200,120.5030) -- cycle(140.3070,114.8450) -- (138.6530,113.9020) -- (135.5470,112.6770) -- (136.1020,111.5240) -- (139.2080,112.7490) -- (141.0190,113.7820) -- (140.3070,114.8450) -- cycle(130.5860,110.7200) -- (129.7310,110.3830) -- (125.5270,109.3750) -- (125.9120,108.1550) -- (130.1160,109.1630) -- (131.1400,109.5670) -- (130.5860,110.7200) -- cycle(120.3410,108.1320) -- (119.5170,107.9340) -- (115.1410,107.4840) -- (115.3560,106.2230) -- (119.7320,106.6730) -- (120.7260,106.9110) -- (120.3410,108.1320) -- cycle(109.8360,106.9390) -- (107.9220,106.7420) -- (107.9740,105.4630) -- (110.0510,105.6770) -- (109.8360,106.9390) -- cycle(177.0120,228.7090) -- (177.0160,223.3760) -- (178.2960,223.3770) -- (178.2920,228.7100) -- (177.0120,228.7090) -- cycle;



  \path[fill=black,line join=round,line width=0.256pt] (35.3466,245.1920) -- (35.3619,239.8590) -- (36.6419,239.8620) -- (36.6266,245.1950) -- (35.3466,245.1920) -- cycle(35.3772,234.5250) -- (35.3925,229.1920) -- (36.6725,229.1960) -- (36.6572,234.5290) -- (35.3772,234.5250) -- cycle(35.4077,223.8590) -- (35.4230,218.5250) -- (36.7030,218.5290) -- (36.6877,223.8620) -- (35.4077,223.8590) -- cycle(35.4383,213.1920) -- (35.4536,207.8590) -- (36.7336,207.8620) -- (36.7183,213.1960) -- (35.4383,213.1920) -- cycle(35.4688,202.5250) -- (35.4841,197.1920) -- (36.7641,197.1960) -- (36.7488,202.5290) -- (35.4688,202.5250) -- cycle(35.4994,191.8590) -- (35.5146,186.5250) -- (36.7946,186.5290) -- (36.7794,191.8620) -- (35.4994,191.8590) -- cycle(35.5299,181.1920) -- (35.5431,176.5870) -- (35.5541,176.4620) -- (35.5905,176.3420) -- (35.6502,176.2320) -- (35.7312,176.1360) -- (35.8273,176.0550) -- (35.9379,175.9960) -- (36.0581,175.9600) -- (36.1831,175.9490) -- (36.2239,175.2220) -- (36.2239,176.5020) -- (36.1831,177.2290) -- (36.8231,176.5910) -- (36.8099,181.1960) -- (35.5299,181.1920) -- cycle(36.1788,170.4950) -- (36.2252,170.1340) -- (37.0086,165.6870) -- (37.1246,165.1870) -- (38.3794,165.4400) -- (38.2633,165.9400) -- (37.4910,170.3250) -- (37.4517,170.6300) -- (36.1788,170.4950) -- cycle(38.3654,159.9630) -- (39.8875,154.8510) -- (41.1257,155.1760) -- (39.6036,160.2870) -- (38.3654,159.9630) -- cycle(41.7635,149.8140) -- (42.3659,148.2190) -- (43.9453,144.8980) -- (45.1243,145.3960) -- (43.5449,148.7170) -- (42.9774,150.2200) -- (41.7635,149.8140) -- cycle(46.4153,140.1130) -- (49.1568,135.5380) -- (50.2867,136.1400) -- (47.5451,140.7140) -- (46.4153,140.1130) -- cycle(52.3792,131.2090) -- (54.2589,128.7360) -- (55.8672,127.0890) -- (56.8387,127.9230) -- (55.2304,129.5690) -- (53.4413,131.9240) -- (52.3792,131.2090) -- cycle(59.5942,123.2750) -- (60.1272,122.7290) -- (63.8018,119.8730) -- (64.6566,120.8260) -- (60.9820,123.6820) -- (60.5657,124.1080) -- (59.5942,123.2750) -- cycle(68.1836,116.6940) -- (72.8231,114.0630) -- (73.5353,115.1270) -- (68.8957,117.7570) -- (68.1836,116.6940) -- cycle(77.7085,111.7330) -- (82.6852,109.8160) -- (83.2336,110.9720) -- (78.2568,112.8890) -- (77.7085,111.7330) -- cycle(87.8844,108.3070) -- (93.0857,107.1270) -- (93.4584,108.3520) -- (88.2571,109.5310) -- (87.8844,108.3070) -- cycle(98.4341,106.3060) -- (103.7470,105.8420) -- (103.9440,107.1070) -- (98.6309,107.5710) -- (98.4341,106.3060) -- cycle(35.3161,255.8580) -- (35.3314,250.5250) -- (36.6114,250.5290) -- (36.5961,255.8620) -- (35.3161,255.8580) -- cycle;



    \path[fill=cFFFFFF,line join=round,line width=0.160pt,rounded corners=0.0000cm] (169.2360,90.1478) rectangle (187.6535,108.5653);



    \path[cm={{1.0,0.0,0.0,1.0,(167.0,104.0)}}] (0.0000,0.0000) node[above right] () {VX};



  \path[fill=cFFFFFF,line join=round,line width=0.160pt,rounded corners=0.0000cm] (147.3710,118.1840) rectangle (165.7885,136.6015);



  \path[cm={{1.0,0.0,0.0,1.0,(148.0,131.0)}}] (0.0000,0.0000) node[above right] () {VX};



  \path[cm={{1.0,0.0,0.0,1.0,(3.0,155.0)}}] (0.0000,0.0000) node[above right] () {p};



  \path[draw=c989898,line join=round,line width=0.512pt] (53.9034,178.5010) -- (50.1740,174.7710);



  \path[draw=c989898,line join=round,line width=0.512pt] (50.1759,178.4980) -- (53.9057,174.7690);



  \path[draw=c989898,line join=round,line width=0.512pt] (18.1387,83.0726) -- (18.1375,255.3590);



  \path[draw=black,line join=round,line width=1.024pt] (18.0216,177.2890) .. controls (18.0216,142.2570) and (43.5702,113.1920) .. (77.0515,107.7090);



  \path[draw=black,line join=round,line width=1.024pt] (18.0693,176.9450) -- (18.1304,177.6070);



  \path[draw=black,line join=round,line width=1.024pt] (94.6706,35.8121) .. controls (105.2670,38.6713) and (107.7740,49.5496) .. (107.7740,49.5495) .. controls (107.7740,49.5495) and (127.8500,98.4552) .. (75.2699,108.1860);



  \path[draw=black,fill=cFFFFFF,line join=round,line width=0.512pt] (18.7671,166.8360) -- (30.5910,148.4490) -- (22.8677,149.5770) -- (17.0346,145.0440) -- (18.7671,166.8360) -- cycle;



  \path[draw=black,line join=round,line width=1.024pt] (18.0603,255.8890) -- (18.0608,176.9270);



    \path[fill=cFFFFFF,line join=round,line width=0.160pt,rounded corners=0.0000cm] (30.3745,87.7230) rectangle (48.7920,106.1405);



    \path[cm={{1.0,0.0,0.0,1.0,(27.0,102.0)}}] (0.0000,0.0000) node[above right] () {VX};



  \path[draw=black,line join=round,line width=0.512pt] (19.0252,165.8830) -- (22.8096,149.6730);



  \path[draw=black,fill=cFFFFFF,line join=round,line width=0.512pt] (61.7543,111.6660) -- (83.5859,112.8000) -- (77.8654,107.5910) -- (79.8106,99.3419) -- (61.7543,111.6660) -- cycle;



  \path[draw=black,line join=round,line width=0.512pt] (62.2361,111.5330) -- (77.7207,107.5900);



  \path[draw=black,line join=round,line width=1.024pt] (177.6620,255.7830) -- (177.6630,199.4550);



  \path[draw=black,fill=c9B9B9B,line join=round,line width=0.512pt] (177.6450,197.0920) -- (170.3720,217.7070) -- (177.6280,214.8310) -- (184.3480,217.9000) -- (177.6450,197.0920) -- cycle;



  \path[draw=black,line join=round,line width=0.512pt] (177.6130,198.0790) -- (177.6620,214.7250);



  \path[draw=black,fill=cFFFFFF,line join=round,line width=0.512pt] (114.8590,60.9003) -- (109.7300,39.6783) -- (105.0370,45.1384) -- (97.9220,47.1230) -- (114.8590,60.9003) -- cycle;



  \path[draw=black,line join=round,line width=0.512pt] (106.6750,47.4733) -- (113.8910,13.7000);



  \path[draw=black,line join=round,line width=0.512pt] (106.7820,47.6598) -- (133.1800,72.9958);



  \path[cm={{1.0,0.0,0.0,1.0,(125.0,14.0)}}] (0.0000,0.0000) node[above right] () {P};



  \path[cm={{1.0,0.0,0.0,1.0,(144.0,71.0)}}] (0.0000,0.0000) node[above right] () {p};



  \path[cm={{1.0,0.0,0.0,1.0,(73.0,82.0)}}] (0.0000,0.0000) node[above right] () {pd};



  \path[draw=black,line join=round,line width=0.512pt] (11.1051,22.5112) -- (11.1051,56.1472);



  \path[draw=black,line join=round,line width=0.512pt] (44.5105,55.8748) -- (10.8746,55.8748);



  \path[cm={{1.0,0.0,0.0,1.0,(0.0,71.0)}}] (0.0000,0.0000) node[above right] () {Ow};



  \path[draw=black,line join=round,line width=0.512pt] (12.1657,55.7150) -- (105.7700,47.3245);



    \path[fill=cFFFFFF,line join=round,line width=0.160pt,rounded corners=0.0000cm] (54.3956,35.3830) rectangle (83.1904,53.7767);



    \path[cm={{1.0,0.0,0.0,1.0,(59.0,52.0)}}] (0.0000,0.0000) node[above right] () {p};



  \path[draw=black,line join=round,line width=0.512pt] (40.7325,11.2791) -- (66.1100,11.2791);



  \path[cm={{1.0,0.0,0.0,1.0,(47.0,7.0)}}] (0.0000,0.0000) node[above right] () {w};



  \path[draw=black,line join=round,line width=0.512pt] (105.0200,45.1395) -- (114.7210,60.7614);



  \path[fill=black,line join=round,line width=0.160pt] (42.2791,52.3580) -- (42.2882,59.3251) -- (48.6655,55.8338) -- (42.2791,52.3580) -- cycle;



  \path[fill=black,line join=round,line width=0.160pt] (7.6356,24.4325) -- (14.6026,24.4274) -- (11.1151,18.0482) -- (7.6356,24.4325) -- cycle;



  \path[fill=black,line join=round,line width=0.160pt] (132.4720,67.1442) -- (127.4190,71.9410) -- (134.3390,74.1713) -- (132.4720,67.1442) -- cycle;



  \path[fill=black,line join=round,line width=0.160pt] (110.0570,15.7764) -- (116.9120,17.0209) -- (114.6260,10.1196) -- (110.0570,15.7764) -- cycle;



  \path[fill=black,line join=round,line width=0.160pt] (91.8888,85.3087) -- (97.4148,81.0659) -- (90.7657,78.1249) -- (91.8888,85.3087) -- cycle;



  \path[fill=black,line join=round,line width=0.160pt] (65.0964,7.4797) -- (65.1056,14.4467) -- (71.4829,10.9554) -- (65.0964,7.4797) -- cycle;



  \path[fill=black,line join=round,line width=0.160pt] (100.0910,45.0280) -- (101.2270,51.9019) -- (106.9560,47.4254) -- (100.0910,45.0280) -- cycle;



  \path[draw=black,line join=round,line width=0.512pt] (92.9862,82.1757) -- (106.8080,47.8196);



    \path[fill=cFFFFFF,line join=round,line width=0.160pt,rounded corners=0.0000cm] (9.6299,87.8350) rectangle (28.0474,106.2525);



    \path[cm={{1.0,0.0,0.0,1.0,(6.0,102.0)}}] (0.0000,0.0000) node[above right] () {VX};



  \path[fill=cDEDEDE,line join=round,even odd rule,line width=0.160pt,rounded corners=0.0000cm] (182.2300,10.8166) rectangle (203.3489,31.9356);



  \path[cm={{1.0,0.0,0.0,1.0,(214.0,27.0)}}] (0.0000,0.0000) node[above right] () {P};



  \path[draw=black,line join=round,line width=1.024pt] (182.2300,53.4302) -- (203.3500,53.4304);



  \path[cm={{1.0,0.0,0.0,1.0,(213.0,56.0)}}] (0.0000,0.0000) node[above right] () {P};



\path[fill=c0000FF,line join=round,line width=0.160pt] (193.3120,200.5350) .. controls (190.4790,200.5350) and (188.1820,198.2380) .. (188.1820,195.4050) .. controls (188.1820,192.5720) and (190.4790,190.2760) .. (193.3120,190.2760);




\end{tikzpicture}


  \end{minipage}
  \begin{minipage}[t]{0.012\columnwidth}
  \end{minipage}\hfill
  \begin{minipage}[t]{0.328\columnwidth}
    \vspace*{-28.2ex}
    \caption{Definitions~\ref{def:stage}--\hyperref[def:plan]{4} on a slice of the plan $\Gamma$. $\mathbf{p}_{\Gamma_i}$ are triggering points in which proximity happens {\color{black}the} change of stages $\Gamma_i$. Each contains a path function $\varphi_i$ and parameters to alter the path and schedule $c_{i,1},\dots$.}
    \label{fig:traj1}
  \end{minipage}
  \vspace*{-4ex}
\end{figure}

\subsection{Preliminaries}
\label{sec:prelim}

%Here, $\Gamma_i$ is a \emph{stage}, $\varepsilon_i$ is a stage-dependent value detailed 
%along with $\mathbf{p}_{\Gamma_i}$ in and 
%after Definition~\ref{def:trigs}.

\begin{defn}[Stage]\label{def:stage}
  Given a generic point $\mathbf{p}\in\mathbb{R}^2$ w.r.t. a reference frame $\mathcal{O}_W$ of the aerial robot flying at a given altitude $h\in\mathbb{R}_{>0}$, the $i$th \emph{stage} $\Gamma_i$ %at time instant $t$ %of a plan $\Gamma$ 
  is
  \begin{equation*}\begin{split}
    \Gamma_i:=\{{\color{black}\varphi_i(\mathbf{p}%(t)
    ,c_i^\rho)},c_i^\sigma\mid
    \,&\forall j\in\,[\rho]_{>0},\,c_{i,j}\,\,\,\,\,\,\,\in\mathcal{C}_{i,j},\,\\
      &\forall k\in[\sigma]_{>0},\,c_{i,\rho+k}\in\mathcal{S}_{i,k}\,\},
  \end{split}\end{equation*}
  where $c_i^\rho${\color{black}$:=\{c_{i,1},c_{i,2},\dots,c_{i,\rho}\}$} and $c_i^\sigma${\color{black}$:=\{c_{i,\rho+1},c_{i,\rho+2},$ $\dots,c_{i,\rho+\sigma}\}$} are $\rho$ \emph{path} and $\sigma$ \emph{computation parameters}{\color{black}, e.g., $c_i^\rho:=\{c_{i,1}\}$ is a value that changes the distance of the coverage lines and $c^\sigma_i:=\{c_{i,2}\}$ the detection rate with $\rho$ and $\sigma$ being one (see Sec.~\ref{sec:experimental})}. $\mathcal{C}_{i,j}:=[\underline{c}_{i,j},\overline{c}_{i,j}]\subseteq\mathbb{R}$ is the $j$th path parameter %$c_{i,j}$ 
  constraint set, %and 
  $\mathcal{S}_{i,k}:=[\underline{c}_{i,\rho+k},\overline{c}_{i,\rho+k}]\subseteq\mathbb{Z}_{\geq 0}$ %is 
  the $k$th computation parameter constraint set. {\color{black}Indices $j,k$ serve~to~differentiate path and computation parameters constraints and indicate that each parameter can have a different constraint set.} 
\end{defn}


For a set $\mathbb{X}$, $\mathbb{X}_{\geq 0}$ indicates its members are positive, $\mathbb{X}_{> 0}$ strictly positive, and $|\mathbb{X}|$ its cardinality. $\mathbb{Z},\mathbb{R}$ are %the sets of 
integers and reals. % respectively.
Bold letters indicate vectors. 
The notation $[x]$ denotes positive naturals up to $x$, i.e., $\{0,1,\dots,x\}$, {\color{black}$[x]_{>0}$ strictly positive naturals, i.e., $\{1,2,\dots,x\}$,} $x'$ the transpose of $x$, and $[\underline{x},\overline{x}]$ the upper/lower bounds of %a parameter 
$x$, i.e.,
%\begin{equation}
  $\underline{x}\leq x\leq\overline{x}$.
%\end{equation}

The function $\varphi_i$ is a \emph{path function}{\color{black}--a }%specifying the path. These are 
stage-dependent mathematical function %s 
the %aerial 
robot tracks as it travels %the motion for 
the coverage. 

\begin{defn}[Path functions]
  \label{def:paths}
  $\varphi_i:\mathbb{R}^2\times\mathbb{R}^\rho\rightarrow\mathbb{R},\,\forall i\in\{1,2,\dots\}
  $ are \emph{path functions}, forming the path. They are a function of {\color{black}$\mathbf{p}%(t)
  $} and path parameters $c_i^\rho%(t)
  $ and are continuous.% and twice differentiable.
\end{defn}

The change of stages happens in the proximity of given points termed \emph{triggering points}, whereas the plan is complete at the occurrence of the \emph{final point}.

\begin{defn}[Triggering and final points]
  \label{def:trigs}
  The \emph{triggering point} $\mathbf{p}_{\Gamma_{i}}$ describes the transition between stages. \emph{Final point} is the last triggering point $\mathbf{p}_{\Gamma_{l}}$ relative to the last stage $\Gamma_l$.
\end{defn}

The plan merges the concepts from Definitions~\ref{def:stage}--\hyperref[def:trigs]{3}. %via a .

\begin{defn}[Plan]\label{def:plan}
  The \emph{plan} is a finite state machine (FSM) $\Gamma$, where the state-transition function $s:\bigcup_i{\Gamma_i}\times\mathbb{R}^2\rightarrow\bigcup_i{\Gamma_i}$ maps a stage and a point to the next stage
  \begin{equation*}{\color{black}s(\Gamma_i,\mathbf{p}%(t)
    )}:=\begin{cases}
    \Gamma_{i+j} & {\color{black}\text{if }\norm{\mathbf{p}%(t)
    -\mathbf{p}_{\Gamma_i}}<\varepsilon_i,\,\exists j\in\mathbb{Z},}\\
    \Gamma_i & \text{otherwise}.
  \end{cases}\end{equation*}
\end{defn}

The stage-dependent value $\varepsilon_i\in\mathbb{R}_{\geq 0}$ in Definition~\ref{def:plan} expresses the radius of a non-existent circle over $\mathbf{p}_{\Gamma_i}$.

Fig.~\ref{fig:traj1} illustrates the concepts in Definitions~\ref{def:stage}--\hyperref[def:plan]{4}. $\varphi_0,\dots,\varphi_5$ are path functions. $\varphi_0$ and $\varphi_4$ are circles, while $\varphi_1$, $\varphi_3$, and $\varphi_5$ are lines. They are relative to different stages $\Gamma_1,\dots$ but $\Gamma_0$ (the starting stage) and are changed in the proximity of $\mathbf{p}_{\Gamma_0},\dots$. %The constraint set $\mathcal{C}_{4,1}$ forms the area where 
It is possible to alter the paths $\varphi_1,\dots,\varphi_4$ with the parameters $c_{1,1},\dots,c_{4,1}$--%illustrated by 
the gray area.% in the figure. %The area relative to $\Gamma_5$ is bounded by $\underline{c}_{5,1}\overline{c}_{5,1}$.

A convenient way of defining $\Gamma$ is specifying a set of stages, a shift, and a final point. The set is termed \emph{primitive stages} and iterated with the shift up to %reaching 
the final point.

\begin{defn}[Primitive stages]
  \label{def:primitive}
  Given the number of \emph{primitive stages} $n\in\mathbb{Z}_{>0}$, a \emph{shift} $\mathbf{d}\in\mathbb{R}^2$, and a final point $\mathbf{p}_{\Gamma_l}$, the stages $\Gamma_1,\Gamma_2,\dots,\Gamma_n$ %with the path functions $\varphi_1,\dots\varphi_n$ and the parameters $c_1,c_2,\dots,c_n$ 
  are \emph{primitive} if they form the remainder of the plan with $\mathbf{d}$ up to $\mathbf{p}_{\Gamma_l}$. 
\end{defn}
\noindent In this case, the path functions have a constant distance $e_j$ per each value in $[n]_{>0}$, i.e., 
\begin{equation}\small\label{eq:primitive}
  \varphi_{(i-1)n+j}(\mathbf{p}+(i-1)\mathbf{d},c_1^\rho)-\varphi_{in+j}(\mathbf{p}+i\mathbf{d},c_1^\rho)=e_j,
\end{equation}
holds $\forall i\in[l/n-1]_{>0},j\in[n]_{>0}$ assuming the total number of stages is known and is $l\in\mathbb{Z}_{>0}$. $e_j\in\mathbb{R}$ given a shift $\mathbf{d}$, initial point $\mathbf{p}$, and initial value of path parameters $c_1^\rho$.

%\begin{figure}[h!]
%  \footnotesize
%  \begin{minipage}[l]{0.75\columnwidth}
%    \vspace*{-2.7ex}
%    \begin{tikzpicture}[shorten >=.3pt,node distance=12.5ex,on grid,auto,initial text=\hspace*{-.8ex}\footnotesize{$\Gamma_0$}\hspace*{-.5ex}]
%      \scriptsize
%      \node[state,initial] (q_i) {$\Gamma_1$}; 
%      \node[state] (q_2) [right=of q_i] {$\Gamma_2$}; 
%      \node        [right=of q_2] (q_dots0) {$\cdots$};
%      \node[state] (q_0) [right=of q_dots0] {$\Gamma_n$};
%      \node[state,accepting] (q_f) [right=of q_0] {$\Gamma_f$};
%      \path[->]
%      (q_i) edge node {$\mathbf{p}_{\Gamma_{1}}$} (q_2)
%      (q_2) edge node {$\mathbf{p}_{\Gamma_{2}}$} (q_dots0)
%      (q_dots0) edge node{$\mathbf{p}_{\Gamma_{n-1}}$} (q_0)
%      (q_0) edge [bend right=60] node [above] {$\mathbf{p}_{\Gamma_n}$} (q_i)
%      (q_i) edge [bend left=-60] node [above] {$\mathbf{p}_{\Gamma_l}$} (q_f)
%      (q_2) edge [bend left=-40] node [above] {$\mathbf{p}_{\Gamma_l}$} (q_f)
%      (q_0) edge node {$\mathbf{p}_{\Gamma_{l}}$} (q_f)    
%      (q_i) edge [loop above] node {$\mathbf{p}(t_1)$} (q_i)
%      (q_2) edge [loop above] node {$\mathbf{p}(t_2)$} (q_2)
%      (q_0) edge [loop above] node {$\mathbf{p}(t_3)$} (q_0)
%      (q_f) edge [loop above] node {$\mathbf{p}(t_4)$} (q_f)
%      ; %end path 
%      %\draw [decorate,decoration={brace,amplitude=10pt,mirror,raise=47pt},yshift=0pt]
%      %(q_i.south west) -- (q_f.south west) node [black,midway,yshift=-19ex]{$\Gamma$};
%    \end{tikzpicture}
%  \end{minipage}\hfill
%  \begin{minipage}[c]{0.18\columnwidth}
%    \centering
%    \caption{Plan $\Gamma$ with $n$ primitive stages in Def.~\ref{def:primitive}.}
%    \label{fig:state-machine-loop}
%  \end{minipage}
%  \vspace*{-4ex}
%\end{figure}
%\begin{figure}[h!]
%  \center
%  \vspace*{-2.2ex}
%  \begin{tikzpicture}[shorten >=.5pt,node distance=12.5ex,on grid,auto,initial text=\hspace*{-3ex}\footnotesize{$\Gamma_0$}]
%    \scriptsize
%    \node[state,initial] (q_i) {$\Gamma_1$}; 
%    \node[state] (q_2) [right=of q_i] {$\Gamma_2$}; 
%    \node        [right=of q_2] (q_dots0) {$\cdots$};
%    \node[state] (q_0) [right=of q_dots0] {$\Gamma_n$};
%    \node[state,accepting] (q_f) [right=of q_0] {$\Gamma_f$};
%    \path[->]
%    (q_i) edge node {$\mathbf{p}_{\Gamma_{1}}$} (q_2)
%    (q_2) edge node {$\mathbf{p}_{\Gamma_{2}}$} (q_dots0)
%    (q_dots0) edge node{$\mathbf{p}_{\Gamma_{n-1}}$} (q_0)
%    (q_0) edge [bend right=60] node [above] {$\mathbf{p}_{\Gamma_n}$} (q_i)
%    (q_i) edge [bend left=-60] node [above] {$\mathbf{p}_{\Gamma_l}$} (q_f)
%    (q_2) edge [bend left=-40] node [above] {$\mathbf{p}_{\Gamma_l}$} (q_f)
%    (q_0) edge node {$\mathbf{p}_{\Gamma_{l}}$} (q_f)    
%    (q_i) edge [loop above] node {$\mathbf{p}(t_1)$} (q_i)
%    (q_2) edge [loop above] node {$\mathbf{p}(t_2)$} (q_2)
%    (q_0) edge [loop above] node {$\mathbf{p}(t_3)$} (q_0)
%    (q_f) edge [loop above] node {$\mathbf{p}(t_4)$} (q_f)
%    ; %end path 
    %\draw [decorate,decoration={brace,amplitude=10pt,mirror,raise=47pt},yshift=0pt]
    %(q_i.south west) -- (q_f.south west) node [black,midway,yshift=-19ex]{$\Gamma$};
%  \end{tikzpicture}
%  \vspace*{-2.3ex}
%  \caption[Definition of a plan with a loop]{Definition of a plan $\Gamma$ with periodic patterns. Stages $\Gamma_1,\Gamma_2,\dots,\Gamma_n$ containing primitive paths $\varphi_1,\varphi_2,\dots,\varphi_n$ are iterated with a shift $\mathbf{d}$.}
%  \vspace*{-1ex}
%  \label{fig:state-machine-loop}
%\end{figure}

%Fig.~\ref{fig:state-machine-loop} illustrates the concept. %s in Definition~\ref{def:primitive}. 
%A plan composed of $n$ stages $\Gamma_1,\dots,\Gamma_n$ (%containing 
%{\color{black}with }primitive paths $\varphi_1,\dots,\varphi_n$) is reiterated with the shift $\mathbf{d}$. $t_1<\dots<t_4$ are time instants $\in\mathbb{R}_{> 0}$. $\Gamma_f$ is the accepting stage, indicating the plan is complete, $\Gamma_0$ the initial stage where the aerial robot awaits the starting command.

\vspace*{-.7ex}
\subsection{Energy-aware planning-scheduling problem}
\label{sec:pbfor}

The problem %of planning-scheduling 
is {\color{black}split into }%composed of 
%two%sub-problems.
{\color{black}the derivation }%. One is 
%to form a %static 
{\color{black} of} a {\color{black}coverage} plan %that visits every point in space 
{\color{black}and its} {\color{black}energy-aware} %other to
re-plan{\color{black}ing}~and~-schedul{\color{black}ing~}%the plan 
in-flight.~The %in an energy-aware way. %{\color{black}, whereas }
{\color{black} 
re-planning-scheduling increases a unitless performance metric--%, which is %quality of the plan and schedule against battery state of charge (SoC), i.e., the 
the weighted average of parameters divided by the remaining battery state of charge (SoC) at the end of flight, both in percent (e.g., $\underline{c}_{i,j}$, $\overline{c}_{i,j}$ correspond to 0 and 100). The objective is~%then 
high~average~parameters configuration and battery usage with successful %area 
coverage.
%The performance metric the re-planning-scheduling %in the letter 
%improves is %then
%the quality of the plan-schedule against battery state of charge (SoC), e.g., weighted average value of parameters divided by remaining %battery 
%SoC.% at the end of the flight.
}

\begin{pb}[Coverage and re-planning-scheduling problem]
  \label{pb:cov-pb}
  Consider a finite set of vertices of a polygon $v:=\{v_1,v_2,\dots\}$ where each %vertex $v_i:=(x_{v_i},y_{v_i})%,\forall i\in[|v|]_{>0}$
  %,\,o_{j,k}:=(x_{o_{j,k}},y_{o_{j,k}}),\,\forall j\in|o|,k\in|o_j|$ 
  is a point w.r.t. $\mathcal{O}_W$. 
  Let $\underline{r}\in\mathbb{R}_{\geq 0}$, the vehicle's turning radius, and $\mathbf{p}(t_0)$, the starting point at the time instant $t_0$, be given. 
  The \emph{coverage problem} is the problem of finding a plan $\Gamma$ to cover the polygon, whereas the \emph{re-planning-scheduling problem} is finding the {\color{black}energy-aware} trajectory of parameters $c_i$ in time{\color{black}.}%, optimizing battery SoC}.
\end{pb}    

%, i.e, $|\mathbb{X}|:=\sum_{x\in\mathbb{X}}1$
%.

%\begin{pb}[Re-planning-scheduling problem]
%  \label{pb}
%  Consider an initial plan $\Gamma$ in Definition~\ref{def:plan}. The \emph{re-planning-scheduling problem} is the problem of finding the optimal configuration of path and computations parameters $c_i(t),\,\forall i\in\{1,2,\dots\}$ under energy constraints and uncertainty at each time step $t$.
%\end{pb}

Here, $c_i$ denotes a row vector with both the path and computation parameters in sequence, i.e., $c_i:=[\begin{matrix}\,c_i^\rho & c_i^\sigma\,\end{matrix}]'$. %The notation .


%The algorithm inputs a user-specified initial plan that consist of different stages. At each stage the plan contains some parameters that allow to alter the path and computations along an energy budget. The alterations are bounded. There is one path constraint set which bounds the path and multiple computation constraint sets, one per each computation parameter, that bound computations. In Fig.~\ref{fig:il-abs}, there are two parameters. One relative to the path (\ref{sth:ii}~has a shorter distance between the lines than~\ref{sth:iii}), and the other one to the computations (\ref{sth:i}~processes more images per second than~\ref{sth:ii}). 

%The algorithm outputs the control (the parameters) using model predictive control (MPC)~\cite{rawlings2017model} where it checks the satisfaction of the battery constraints. The control is data-driven. Energy sensor data estimates some coefficients of an energy model used to predict future energy consumption in presence of uncertainty. The energy budget is the battery capacity and other battery parameters. These are fixed values that are not replanned by the algorithm. Our goal is to complete the plan with the highest possible parameters configuration as the UAV flies and its batteries drain. 

%\subsection{Plan definition}
%\label{sec:prelim}

%Let us adopt the following mathematical notation. Given an integer $a$, $[a]$ is the set $\{0,1,\dots,a\}$, $[a]^+$ the set $[a]/\{0\}$. Bold lower-case letters indicates vectors. $c_{i,j}$ the $j$-th parameter of the $i$-th parameters set $c_i$. $\underline{c}_{i,j},\overline{c}_{i,j}$ are the lower and upper bounds of the parameter $c_{i,j}$.

%Let us assume that the path at stage $i$ can be altered with $\rho$ path parameters $c_i^\rho:=\{c_{i,1},c_{i,2},\dots,c_{i,\rho}\}$, and the computations with $\sigma$ computation parameters $c_i^\sigma:=\{c_{i,\rho+1},c_{i,\rho+2},\dots,c_{i,\rho+\sigma}\}$. We then express the path as a continuous twice differentiable function $\varphi_i:\mathbb{R}^2\times\mathbb{R}^\rho\rightarrow\mathbb{R}$ of a point and the path parameters. The function returns a metric of the distance between the point and the nominal trajectory. We express the computations as the value of the computation parameters. We discuss the concrete meaning of the value of path parameters in Subsection~\ref{sec:model}, and computation parameters in Subsection~\ref{sec:computations-model}.

%\begin{defn}[Stage, plan, triggering, and final point]\label{def:mission}
%  The $i$-th \emph{stage} $\Gamma_i$ at time instant $k$ of a plan $\Gamma$ is defined
%  \begin{equation*}\begin{split}
%    \Gamma_i:=\{\varphi_i(\mathbf{p}_k,c_i^\rho),c_i^\sigma\mid
%    \,&\exists\,\,\mathbf{p}_k,\,\varphi_i(\mathbf{p}_k,c_i^\rho)\in\mathcal{C}_i,\,\\
%      &\,\forall j\in[\sigma]^+,\,c_{i,\rho+j}\in\mathcal{S}_{i,j}\,\},
%\end{split}\end{equation*}
%where $\mathcal{C}_i:=[\underline{c}_i,\overline{c}_i]\subseteq\mathbb{R}$ is the path constraint set, and $\mathcal{S}_{i,j}:=[\underline{c}_{i,\rho+j},\overline{c}_{i,\rho+j}]\subseteq\mathbb{Z}_{\geq 0}$ the $j$-th computation constraint set. $\mathbf{p}_k$ is a point of a UAV flying at an altitude $h\in\mathbb{R}_{>0}$
%\footnote{The altitude might change at different flying phases and under different atmospheric conditions}
%w.r.t. some inertial navigation frame $\mathcal{O}_W$.   
  
%In Fig.~\ref{fig:traj1}, $\varphi_1,\dots,\varphi_6$ are paths. $\varphi_1$ and $\varphi_5$ are circles, while $\varphi_2$, $\varphi_4$, and $\varphi_6$ are lines. They are all relative to different stages $\Gamma_1,\Gamma_2,\dots$. The constraints set $\mathcal{C}_1,\mathcal{C}_2,\dots$ forms the area where the paths $\varphi_1,\varphi_2,\dots$ can be altered with the parameters $c_{i,1},\dots,c_{i,\rho}$ (gray area in the figure). This area is bounded by $\underline{c}_i,\overline{c}_i$, and can be different per each stage (in Fig.~\ref{fig:traj1}, the area relative to $\Gamma_4$ is bounded by $\underline{c}_4,\overline{c}_4$).

%The \emph{plan} is a finite state machine (FSM) $\Gamma$ where the state-transition function $s:\bigcup_i{\Gamma_i}\times\mathbb{R}^2\rightarrow\bigcup_i{\Gamma_i}$ maps a stage and a point to the next stage
%\begin{equation*}s(\Gamma_i,\mathbf{p}_k):=\begin{cases}
%  \Gamma_{i+1} & \text{if }\mathbf{p}_k=\mathbf{p}_{\Gamma_i}\\
%  \Gamma_i & \text{otherwise}
%\end{cases}.\end{equation*}
%The point $\mathbf{p}_{\Gamma_{i}}$ that allows the transition between $\Gamma_i$ and $\Gamma_{i+1}$ is called \emph{triggering point}. In Fig.~\ref{fig:traj1}, $\mathbf{p}_{\Gamma_1}$ allows the transition between $\Gamma_1$ and $\Gamma_2$, $\mathbf{p}_{\Gamma_4}$ between $\Gamma_4$ and $\Gamma_5$, and $\mathbf{p}_{\Gamma_5}$ between $\Gamma_5$ and $\Gamma_6$. The last triggering point $\mathbf{p}_{\Gamma_{l}}$ relative to the last stage $\Gamma_l$ is called \emph{final point}.
%\end{defn}

%\begin{figure}[h]
%  \center
%  \begin{tikzpicture}[shorten >=.5pt,node distance=10.5ex,on grid,auto]
%    \footnotesize
%    \node[state,initial] (q_i) {$\Gamma_1$}; 
%    \node        [right=of q_i] (q_dots0) {$\cdots$};
%    \node[state] (q_0) [right=of q_dots0] {$\Gamma_i$};
%    \node        (q_dots1) [right=of q_0] {$\cdots$};
%    \node[state,accepting] (q_f) [right=of q_dots1] {$\Gamma_f$};
%    \path[->]
%    (q_i) edge node {$\mathbf{p}_{\Gamma_{1}}$} (q_dots0)
%    (q_dots0) edge node{$\mathbf{p}_{\Gamma_{i-1}}$} (q_0)
%    (q_0) edge node {$\mathbf{p}_{\Gamma_i}$} (q_dots1)
%    (q_dots1) edge node {$\mathbf{p}_{\Gamma_{l}}$} (q_f)    
%    (q_i) edge [loop above] node {$\mathbf{p}_{k_1}$} (q_i)
%    (q_0) edge [loop above] node {$\mathbf{p}_{k_2}$} (q_0)
%    (q_f) edge [loop above] node {$\mathbf{p}_{k_3}$} (q_f)
%    ; %end path 
%    \draw [decorate,decoration={brace,amplitude=10pt,mirror,raise=10pt},yshift=0pt]
%    (q_i.south west) -- (q_f.south west) node [black,midway,yshift=-9ex]{$\Gamma$};
%  \end{tikzpicture}
%  \caption{The plan defined as a FSM}
%  \label{fig:state-machine}
%\end{figure}

%A slice of the plan in Fig.~\ref{fig:state-machine} shows the transition between the stages with the FSM. The triggering point $\mathbf{p}_{\Gamma_{i-1}}$ allows the transition to the stage $\Gamma_i$. The UAV remains in the stage with any generic point $\mathbf{p}_{k_2}$. It eventually enters the stage $\Gamma_{i+1}$ with the triggering point $\mathbf{p}_{\Gamma_i}$ and so on, until it reaches the final point. The stage $\Gamma_f$ is the accepting stage (it indicates that the UAV has completed the plan).

%\begin{figure}[h]
%  \center
%  \begin{tikzpicture}[shorten >=1pt,node distance=23ex,on grid,auto] 
%    \node        (q_dots0) {$\cdots$};
%    \node[state] (q_0) [right=of q_dots0] {$\Gamma_i$};
%    \node        (q_dots1) [right=of q_0] {$\cdots$};   
%    \path[->]
%    (q_dots0) edge node{$\mathbf{p}_{\Gamma_{i-1}}(c_1^\rho,\dots,c_{i-1}^\rho)$} (q_0)
%    (q_0) edge node {$\mathbf{p}_{\Gamma_i}(c_1^\rho,\dots,c_{i}^\rho)$} (q_dots1)    
%    (q_0) edge [loop above] node {$\mathbf{p}_{k}$} (q_0)
%    ; %end path
%  \end{tikzpicture}
%  \caption{Detail of the stage $\Gamma_i$ in the FSM}
%  \label{fig:state-machine2}
%\end{figure}

%Generally, one can express the triggering points in function of the $i$-th trajectory parameters $c_{i}^{\rho}$, or any previous trajectory parameters, propagating the information therein if necessary (see Fig.~\ref{fig:state-machine2}).

%We refer the reader to an example in Appendix~\ref{app:plan-example} for a detailed implementation. In the example, the first trajectory parameter is propagated to all the following trajectories and triggering points. 

%We store the initial plan in the plan specification, the format is described in Appendix~\ref{app:plan-spec}. 

%\subsection{Problem formulation}

%In order to simplify the problem formulation, we consider some primitive paths. All the other paths are built from these paths with a shift $\mathbf{d}:=(x_d,y_d)$.

%Given $n\in\mathbb{Z}_{>0}$ ($n<l,l/n\in\mathbb{Z}$) primitive paths $\varphi_1,\dots\varphi_n$, a generic starting point $\mathbf{p}$ and the current levels of the path parameters $c_1^\rho$, all the other paths $\varphi_{n+1},\dots,\varphi_l$ are built
%\begin{equation}\label{eq:primitive}\begin{split}
%  &\varphi_{(i-1)n+j}(\mathbf{p}+(i-1)\mathbf{d},c_1^\rho)-\\ &\,\,\,\varphi_{in+j}(\mathbf{p}+i\mathbf{d},c_1^\rho)=e_j,
%\end{split}\end{equation}
%$\forall i\in[l/n-1]^+,j\in[n]^+$, where $e_j\in\mathbb{R}$ is the $j$-th constant difference.

%\begin{defn}[Period]\label{def:period}
%  The period $T\in\mathbb{R}_{> 0}$ is the time between $\varphi_{(i-1)n+j}$ and $\varphi_{in+j}$ in Eq.~(\ref{eq:primitive}).
%\end{defn} 

%The algorithm measures the time between the paths and assumes the initial period is one. The periods might be different for different $j$s due to atmospheric interferences.

%One can define the plan using primitive paths or define all the stages explicitly and find $n$ searching the value which satisfies the Eq.~(\ref{eq:primitive}). If there is no such value, (e.g., when the plan is composed of only one stage), the period $T$ from Definition~\ref{def:period} can be determined empirically from energy data (such as these shown in Fig.~\ref{fig:il-abs}).

%\begin{pb}[UAV planning problem]\label{pb}
%  Consider an initial plan $\Gamma$ from Definition~\ref{def:mission}. We are interested in the planning of the parameters $c_i,\,\forall i\in[l]^+$ and energy constraints and in the guidance of the UAV to the path resulting from such plan.
%\end{pb}


%%%%%%%%%%%%%%%%%%%%%%%%%%
\section{Energy Models}  %
\label{sec:energy-model} %
                         %
The solution to the problem requires energy models, predicting the impact of changes to path and computation parameters on the battery. %at future time instants. To this end, 
Sec.~\ref{sec:mod-mot}--\hyperref[sec:mod-bat]{C} {\color{black}thus} provide models for the overall and computations energies {\color{black}and }%as well as 
battery evolution.

\subsection{Overall energy model% for the motion
}
\label{sec:mod-mot}

The collected energy data and corresponding spectrum analysis in Fig.~\ref{fig:il-abs} show the energy of a static coverage plan. It is relative to one flight of a series of flights for CPP %exhibiting periodic behavior 
in a precision agriculture use case~\cite{seewald2020mechanical}. %$\Gamma$ with four primitive path functions iterated with a shift. 
Assuming the primitive paths have approximately the same length and the aerial robot has a fixed ground speed, the data exhibits periodic behavior with a constant set of frequencies, independent of the shift. The hypothesis is further backed by the power spectrum analysis, indicating that to model the energy, three frequencies are adequate.

%We refer to the instantaneous energy consumption evolution simply as the energy signal. We model the energy using energy coefficients $\mathbf{q}\in\mathbb{R}^m$ that characterize such energy signal. The coefficients are derived from Fourier analysis (the size of the energy coefficients vector $m$ is related to the order of a Fourier series) and estimated using a state estimator. %(Kalman filter); well, potentially here with simulated data one could use other estimators (least square filter would work I bet) 

%We prove a relation between the energy signal and the energy coefficients in Lem.~\ref{lem:eqv}. We show after the main results how this approach allows us variability in terms of non-periodic signals.

%After having illustrated the energy model, we enhance it with the energy contribution of the path in Subsection~\ref{sec:model}, and of the computations in Subsection~\ref{sec:computations-model}. 

An intuitive way of modeling the energy data is a Fourier series of a given order $r\in\mathbb{Z}_{\geq 0}$ and period $T\in\mathbb{R}_{>0}$
\begin{equation}\label{eq:fourier}
  h(t)=a_0/T+(2/T)\sum_{j=1}^{r}{\left(a_j\cos{\omega jt}+b_j\sin{\omega jt}\right)},
\end{equation}
where $h:\mathbb{R}_{\geq 0}\rightarrow\mathbb{R}$ maps time to the instantaneous energy% consumption
, $\omega:=2\pi/T$ is the angular frequency, and $a,b\in\mathbb{R}$ %the %series 
coefficients.

%The model in 
Equation~(\ref{eq:fourier}) does not account for the variation of parameters, where, e.g., two schedules result in different instantaneous energies.
%The model in Equation~(\ref{eq:fourier}) does not quantify the contribution of path and computations parameters $c_i$, where, e.g., different schedules result in different instantaneous energy. 
For this latter purpose, we use the dynamics %another model
\begin{subequations}\label{eq:state-perf}\vspace*{-3ex}
  \begin{align}
  \dot{\mathbf{q}}(t)&=A\mathbf{q}(t)+B\mathbf{u}(t),\label{eq:state-perf-q}\\
  y(t)&=C\mathbf{q}(t),\label{eq:state-perf-y}
\end{align}
\end{subequations}
where $y(t)\in\mathbb{R}$ is the instantaneous energy consumption. The state $\mathbf{q}\in\mathbb{R}^m$ with $m:=2r+1$ contains energy coefficients
\begin{equation}
  \mathbf{q}(t)=\begin{bmatrix}
    \alpha_0(t) & \alpha_1(t) & \beta_1(t) & \cdots & \alpha_r(t) & \beta_r(t)
  \end{bmatrix}'.
\end{equation}

The state transition matrix
\begin{equation}\label{eq:mat_A}{\small
  A=\begin{bmatrix}
    0            & 0^{1\times 2}& \dots & 0^{1\times 2} \\
    0^{2\times 1}& A_1          & \dots & 0^{2\times 2} \\
    \vdots       & \vdots       & \ddots& \vdots        \\
    0^{2\times 1}& 0^{2\times 2}& \dots & A_r 
  \end{bmatrix},\,\,A_j:=\begin{bmatrix}0 & \omega j \\ -\omega j & 0\end{bmatrix}},
\end{equation}
where $A\in\mathbb{R}^{m\times m}$ contains $r$ sub-matrices $A_j$ and $0^{i\times j}$ is a zero matrix of $i$ rows and $j$ columns. In matrix $A$, the top left entry is zero, the diagonal entries are $A_1,\dots,A_r$, the remaining entries are zeros.

The output matrix\vspace*{-3ex}
\begin{equation}\label{eq:mat_C}
  C=(1/T)\Big[1 \,\,\, \overbrace{\begin{matrix}1 & 0 &\cdots & 1 & 0\end{matrix}}^{2r}\Big],
\end{equation}
where $C\in\mathbb{R}^m$ (the first value in the first column is one, the pattern one--zero is then repeated $2r$ times).


%\begin{equation}\label{eq:state-details}\begin{split}
%  A&=\left[\begin{array}{cccc}
%    0&                     &       & \makebox(-5,-5){*}  \\
%     & A_1                 &       &  \\
%     & \makebox(-25,-15){*}& \ddots&  \\
%     &                     &       & A_r 
%  \end{array}\right],\,A_j:=\begin{bmatrix}0 & \omega j \\ -\omega j & 0\end{bmatrix},\\
%  C&=(1/T)\left[\begin{array}{cccccc}
%    1 & 1 & 0 &\cdots & 1 & 0
%  \end{array}\right],
%\end{split}\end{equation}
%where $\mathbf{q}\in\mathbb{R}^m$ with $m=2r+1$, $A\in\mathbb{R}^{m\times m}$ is the state transition matrix, and $C\in\mathbb{R}^m$ is the output matrix. In matrix $A$, the top left entry is zero, the diagonal entries are $A_1,\dots,A_r$, the remaining entries are zeros (*).

%The linear model in Eq.~(\ref{eq:state-perf}) allows us to include the control in the model of Eq.~(\ref{eq:fourier}).

To define the nominal control and the output matrix, we exploit the effect of variation of path and computation parameters on the energy. 
%We assume a variation in computations parameters $c_i^\sigma$ affects the instantaneous energy, i.e., different schedules of power-demanding computational tasks have different power drains from the computing hardware.
%\begin{lem}[Parameters, energy relation]\label{lem:new}
  Given $c_i(t)$ parameters at two following time instants $t\in\{t_j,t_{j+1}\}\subset\mathbb{R}_{\geq 0}$ s.t. $t_j<t_{j+1}$ for an arbitrary stage $\Gamma_i$, a change in parameters $c_i(t_j)\neq c_i(t_{j+1})$ results in different overall and instantaneous energies for path and computation parameters respectively.
%\end{lem}

%Apps.~\ref{app:proof-eqv}--\ref{app:proof-new} contain the proofs of Lemmas~\ref{lem:eqv}--\hyperref[lem:new]{2}.

%Using the same notation from Lem.~\ref{lem:new}, 
%
%\begin{equation}\label{eq:nom-cont}
%  \mathbf{u}(t_{j+1}):=\hat{\mathbf{u}}(t_{j+1})-\hat{\mathbf{u}}(t_j),
%\end{equation}
%for all time instants. 
The nominal control and input matrix in Eq.~(\ref{eq:state-perf}) simply includes the change in energy for all time instants, i.e.,
\begin{equation}\label{eq:mat_B}{\small
  \mathbf{u}(t_{j+1})\hspace*{-.5ex}:=\hspace*{-.5ex}\hat{\mathbf{u}}(t_{j+1})\hspace*{-.5ex}-\hspace*{-.5ex}\hat{\mathbf{u}}(t_j),\,\,\,B=\begin{bmatrix}
      0^{1\times\rho} & 1      & \cdots & 1      \\
      0^{1\times\rho} & 0      & \cdots & 0      \\ 
      \vdots          & \vdots & \ddots & \vdots \\
      0^{1\times\rho} & 0      & \cdots & 0   
  \end{bmatrix}},
\end{equation}
shifts the base frequency $\alpha_0$ assuming the energy of the computations does not alter the other frequencies. $B\in\mathbb{R}^{m\times n}$ with $n:=\rho+\sigma$ contains zeros but in the first row where the first $\rho$ columns are zeros and the remaining $\sigma$ are ones. Different combinations of $\mathbf{u}$ with matrix $B$ in Eq.~(\ref{eq:mat_B}) are possible %, as we discuss briefly in 
{\color{black}(see }Sec.~\ref{sec:conclusion}{\color{black})}.
The dynamics in Eq.~(\ref{eq:state-perf}--\ref{eq:mat_B}) additionally allows us to use state estimation techniques, such as the Kalman filter in Sec.~\ref{sec:repla-algo}, to refine the states $\mathbf{q}$ and model the energy of the aerial robot flying under diverse %atmospheric 
conditions.

{\color{black}Matrices $A$ and $C$ are constructed such that t}he %Under favorable conditions, %the 
models in Eq.~(\ref{eq:fourier}--\ref{eq:state-perf}) are equal
%\begin{lem}[Signal, output equality]\label{lem:eqv}Given 
when $\mathbf{u}$ is a zero vector %, matrices $A,C$ described by Eqs.~(\ref{eq:mat_A}--\ref{eq:mat_C}), 
and an initial guess $\mathbf{q}(t_0)=\mathbf{q}_0$ at {\color{black}the} initial time instant $t_0$
  {\color{black}\begin{equation}
  {\color{black}\mathbf{q}_0=\begin{bmatrix}a_0 & a_1/2 & b_1/2 & \cdots & a_r/2 & b_r/2\end{bmatrix}',}
  \end{equation}}
  %$h$ in Eq.~(\ref{eq:fourier}) is equal to $y$ in Eq.~(\ref{eq:state-perf}).
%\end{lem}
%\begin{proof}
%The equality of the signal and output is achieved by a proper choice of the items of matrices $A,C$ and the initial guess $\mathbf{q}_0$. We refer the reader to Appendix~\ref{app:proof-eqv} for a formal proof, where we justify the choices of the items of the matrices and of the initial guess. 
%\end{proof}
%Appendix~\ref{app:proof-eqv} contains a proof of Lem.~\ref{lem:eqv}.
i.e., $h,y$ are %both 
harmonic signals with the same frequencies{\color{black}. For further details see the first author's Ph.D. thesis~\cite{seewaldphdthesis}}.

$\hat{\mathbf{u}}$ in Eq.~(\ref{eq:mat_B}) is then a scale transformation
\begin{equation}
  \hat{\mathbf{u}}(t):=\mathrm{diag}(\nu_i)c_i(t)+\tau_i,
\end{equation}
where $\mathrm{diag}(x)$ is a diagonal matrix with items of a set $x$ on the diagonal and zeros elsewhere. $\nu_i:=\begin{bmatrix}\nu_{i,1}&\cdots&\nu_{i,n}\end{bmatrix}'$ and $\tau_i:=\begin{bmatrix}\tau_{i,1}&\cdots&\tau_{i,n}\end{bmatrix}'$ are scaling factors{\color{black},} %that 
transform{\color{black}ing} parameters %domain 
(see Definition~\ref{def:stage}) to time and power domains.

%For ease of notation, 
{\color{black}W}e assume that the coverage time evolves linearly{ \color{black}and that the }path parameters{ \color{black}contribute to it equally.} $c_i^\rho$ can be {\color{black}then} transformed into a time measure with scaling factors\begin{subequations}%\small
  \label{eq:scale-traj}\begin{align}
  \nu_{i,j}&=\left((\overline{t}-\underline{t})/(\overline{c}_{i,j}-\underline{c}_{i,j})\right)/\rho,\\
  \tau_{i,j}&=\left(\underline{c}_{i,j}(\underline{t}-\overline{t})/(\overline{c}_{i,j}-\underline{c}_{i,j})+\underline{t}\right)/\rho,
\end{align}\end{subequations} 
$\forall j\in[\rho]_{>0}$ where $\overline{t},\underline{t}$ are time measures needed to complete the coverage with configurations $\underline{c}_i^\rho,\overline{c}_i^\rho$ ($\underline{\Gamma},\overline{\Gamma}$).

Similarly to Eq.~(\ref{eq:scale-traj}), computation parameters $c_i^\sigma$ can be transformed into an instantaneous energy measure with %scaling factors
\begin{subequations}%\small
  \label{eq:scale-comp}\begin{align}
  \nu_{i,j}&=(g(\overline{c}_{i,j})-g(\underline{c}_{i,j}))/(\overline{c}_{i,j}-\underline{c}_{i,j}),\\
  \tau_{i,j}&=\underline{c}_{i,j}(g(\underline{c}_{i,j})-g(\overline{c}_{i,j}))/(\overline{c}_{i,j}-\underline{c}_{i,j})+g(\underline{c}_{i,j}),
\end{align}\end{subequations}
$\forall j\in[\rho+1,n]$. The function $g$ is detailed in Sec.~\ref{sec:mod-com} and quantifies the power of the computing hardware.

\vspace*{.4ex}
\subsection{Energy model for the computations}
\label{sec:mod-com}

Models for heterogeneous computing hardware in the literature often rely on analytical expressions~\cite{marowka2017energy,%goraczko2008energy%calore2015energy,
yang2017designing} or different techniques, such as regressional analysis~\cite{bailey2014adaptive,ma2012holistic,seewald2019coarse}, aiding the selection of hardware- or software-specific parameters. This section presents an energy model based on our early studies~\cite{seewald2019component,seewald2019coarse}, which relies on regressional analysis to quantify the computations energy of any configuration of computations $c_i^\sigma$ within the bounds (see Definition~\ref{def:stage}).

The model compromises a %n automatic 
modeling and profiling tool~\cite{seewald2019coarse} named \powprof{} distributed %~\cite{powprofiler} 
under the open-source MIT license. It is segmented into two layers. In the \emph{measurement layer}, the tool measures a discrete set of computation parameters and infers the energy of the remaining in the \emph{predictive layer} via a piecewise linear regression.

We assume there is at least one measuring device, i.e., shunt or internal power resistor, multimeter, or amperemeter, quantifying the power drain of a %specific 
component, e.g., CPU, GPU, memory, etc., or of the entire computing hardware.

\begin{defn}[Measurement layer]\label{def:meas}
  Given a measuring device, computation parameters, and initial and final time instants, the \emph{measurement layer} is the function $\gamma:\mathbb{Z}_{>0}\times\mathbb{Z}^\sigma\times\mathcal{T}\rightarrow\mathbb{R}$ that returns an energy measure.
\end{defn}

The notation $\mathcal{T}$ encloses all the time intervals from initial $t_0$ to final $t_f$, i.e., $\mathcal{T}:=[t_0,t_f]$.

\begin{defn}[Predictive layer]\label{def:pred}
  Given a measuring device and computation parameters, the \emph{predictive layer} is the function $g:\mathbb{Z}_{>0}\times\mathbb{Z}^\sigma\rightarrow\mathbb{R}$ that returns an energy measure.
\end{defn}

The energy measures in Definitions~\ref{def:meas}--\hyperref[def:pred]{2} can be either average expressed in watts or overall expressed in joules. Additionally, the \powprof{} tool supports the battery SoC detailed in Sec.~\ref{sec:mod-bat}. The function $g$ in Definition~\ref{def:pred} is contained in the %computations scaling 
factors in Eq.~(\ref{eq:scale-comp}), assuming the computations energy behaves linearly between $\underline{c}_i^\sigma$ and $\overline{c}_i^\sigma$, otherwise
\begin{equation}%\small
  \label{eq:piece-wise-reg}\begin{split}
  g(c_i^\sigma)=(&\gamma(\lceil c_i^\sigma\rceil,\mathcal{T}_1)-\gamma(\lfloor c_i^\sigma\rfloor,\mathcal{T}_2))\\(&c_i^\sigma-\lfloor c_i^\sigma\rfloor)/(\lceil c_i^\sigma\rceil-\lfloor c_i^\sigma\rfloor)+\gamma(\lfloor c_i^\sigma\rfloor,\mathcal{T}_2),
\end{split}\end{equation}
where notation $\lceil c_i^\sigma\rceil,\lfloor c_i^\sigma\rfloor$ indicates two adjacent measurement layers, and $\mathcal{T}_1,\mathcal{T}_2$ %are 
the corresponding two time intervals. {\color{black}M}easuring device in $\gamma$ and $g$ is {\color{black}not %explicitly 
stated in Eq.~(\ref{eq:piece-wise-reg})}.

\subsection{Battery model}
\label{sec:mod-bat}

The battery model predicts the battery SoC in the function of a given load at future time 
instants. There are multiple models in the literature~\cite{rao2003battery} with varying complexity {\color{black}and} accuracy %, and ease of implementation 
ranging from accurate but costly physical models~\cite{%moura2017battery,
marcicki2013design}, to abstract models~\cite{%xing2014state,
%he2011evaluation,
hinz2019comparison,mousavi2014various} %that have
{\color{black}with} compelling trade-offs in terms of the latter two. 
We model a Li-ion battery %of an aerial robot 
in-flight with an abstract ``Rint'' ECM in the literature~\cite{%he2011evaluation,
hinz2019comparison,mousavi2014various}.

The battery SoC changes according to~\cite{hasan2018exogenous%,zhang2018online
}, i.e.,
\begin{equation}\label{eq:batdyn}
  \dot{b}(y(t))=-k_bI(y(t))/Q_c,
\end{equation}
where $I(y(t))\hspace*{-.5ex}\in\hspace*{-.5ex}\mathbb{R}$ is the internal current measured~in~am- peres, $y(t)\hspace*{-.5ex}\in\hspace*{-.5ex}\mathbb{R}_{\geq 0}$ the power drain, and $Q_c\hspace*{-.5ex}\in\hspace*{-.5ex}\mathbb{R}$ the battery constant nominal capacity measured in amperes per hour. $k_b$ is a battery coefficient added to~\cite{hasan2018exogenous%,zhang2018online
} and derived experimentally. The ``Rint'' circuit models the battery as a perfect voltage source connected with a resistor $R_r\in\mathbb{R}$ measured in ohm, representing the battery resistance. The voltage on the extremes of ECM respects $V_e=V-R_rI$, where $V,V_e\in\mathbb{R}$ are the internal and external battery voltages measured in volts. The former can be retrieved from the battery data sheet~\cite{hinz2019comparison} and depends on the SoC~\cite{hasan2018exogenous}.

If the voltage %of the power drain 
is stable, Kirchhoff's circuit laws lead to $V_sI_l=V_eI$, where $I_l$ is the current required by the load %measured 
in amperes. Combining $V_e,V_sI_l$ results in the %quadratic 
expression $R_rI^2-$ $VI+V_sI_l=0$. Solving the expression utilizing the negative solution (when $I_l$ is zero, $I$ should also be zero) %leads to% 
results in \vspace*{-1ex}
\begin{equation}\label{eq:batdyn2}
I(y(t))=(V-\sqrt{V^2-4R_ry(t)})/(2R_r).
\end{equation} 
%as in Lem.~\ref{lem:bat}.
%
%\begin{lem}[Battery SoC]\label{lem:bat}
%  Given the internal battery voltage $V\in\mathbb{R}$ measured in volts, resistance $R_r$ in ohms, constant nominal capacity $Q_c$ in amperes per hour, and 
%thus given a battery coefficient $k_b$, the \emph{battery SoC} evolves
%\begin{equation*}\label{eq:batdyn}
%  \dot{b}(t)=-k_b\left(V-
%  \sqrt{
%    V^2-
%    4R_ry(t)}
%  \right)/(2R_rQ_c).
%\end{equation*}
%\end{lem}

%App.~\ref{app:proof-bat} contains the proof of Lem.~\ref{lem:bat}. 

Eq.~(\ref{eq:state-perf}) states that the output $y$ evolves in $\mathbb{R}$, %conflicting with the the findings in this section. 
yet, aerial robots usually use a battery.
We thus use instead
%\begin{defn}[Output constraint]\label{def:const}
\begin{equation}\label{eq:output-const}
  {%\small
  \mathcal{Y}(t):=\{y\mid y\in[0,b\,Q_cV]\subseteq{\mathbb{R}_{\geq 0}}\},} 
\end{equation}
%the \emph{output constraint}, 
where $b\,Q_cV$, the maximum instantaneous energy %consumption
measured in watts, is derived from Eq{\color{black}}.~(\ref{eq:batdyn}--\ref{eq:batdyn2}), i.e., the computation parameters in Algorithm~\ref{alg2} and Eq.~(\ref{eq:ocp-output-mpc}) later in Sec.~\ref{sec:repla-algo} will have an energy constraint.
%\end{defn}

%Let us consider for practical reasons the discretized version of the system in Eq.~\ref{eq:state-perf}. 

%Let us suppose that at time instant $k$ the plan reached the $i$-th stage $\Gamma_i$ and the control
%\begin{equation}\label{eq:state-control2}
%  \mathbf{u}_k=\begin{bmatrix}c_k^\rho & c_k^\sigma\end{bmatrix}^T,
%\end{equation}
%where $\mathbf{u}_k\in\mathbb{R}^n$ with $n=\rho+\sigma$ differs from the nominal control $\mathbf{u}$ in Eq.~(\ref{eq:state-perf}). We include the control in the nominal control exploiting the following observation. 

%We observe that a change in path parameters affects the energy indirectly. It alters the time when the UAV reaches the final point $\mathbf{p}_{\Gamma_l}$. We use this information later in the algorithm to check that the battery discharge time is greater and replan the path parameters accordingly. A change in computation parameters affects the energy directly. It alters the instantaneous energy consumption as more computations require more power (and vice versa). We replan the computation parameters to maximize the instantaneous energy consumption against the maximum battery discharge rate.

%The nominal control and the input matrix
%\begin{equation}\label{eq:state-control}\begin{split}
%  \mathbf{u}&=(\hat{\mathbf{u}}_k-\hat{\mathbf{u}}_{k-1}),\,\,\,\hat{\mathbf{u}}_k:=\mathrm{diag}(\nu_i)\mathbf{u}_k+\tau_i,\\
%  B&=\begin{bmatrix} 0 & \cdots & 0 & 1 & \cdots & 1 \\ 
%    & & & \makebox(-15,-3){*} & &\end{bmatrix}.
%\end{split}\end{equation}
%The matrix $B\in\mathbb{R}^{m\times n}$ contains zeros (*) except the first row where the first $\rho$ columns are still zeros and the remaining $\sigma$ are ones. $\hat{\mathbf{u}}_k$ is a scale transformation with $\nu_i=\begin{bmatrix}\nu_i^\rho & \nu_i^\sigma\end{bmatrix}^T$ and $\tau_i=\begin{bmatrix}\tau_i^\rho & \tau_i^\sigma\end{bmatrix}^T$ scaling factors quantifying the contribution to the plan of a given parameter in terms of time for the first $\rho$ parameters, and power for the remaining $\sigma$ (we use the same notation for the path and computation scaling factors as for the parameters). The nominal control $\mathbf{u}$ is then the difference of these contributions of two consecutive controls $\mathbf{u}_{k-1},\mathbf{u}_k$ applied to the system. $B\mathbf{u}$ merely includes the difference in power into the model in Eq.~(\ref{eq:state-perf}).
 
%Indeed an alteration of the path affects the overall flying time (and consequently the energy). An alteration the the computations affects the instantaneous energy consumption.

%We clarify how we derive the factors $\nu_i,\tau_i$ in the next two subsections.

%\subsection{Path parameters energy contribution}
%\label{sec:model}

%Eq.~(\ref{eq:state-control}) accounts for the energy due to the change of parameters $\mathbf{u}_k-\mathbf{u}_{k-1}$. For instance, when the trajectory $\varphi_1$ is a circle (see Fig.~\ref{fig:tee1}), a decrement in the trajectory parameter $c_{1,1}$--the radius of the circle--adds a negative contribution. It thus simulates the lowering of instantaneous energy consumption ($\nu_{1,1}c_{1,1}>\nu_{1,1}c_{1,1}^-$) for a given $\nu_{1,1}$, that is then summed to the first coefficient $\alpha_0$ in Eq.~(\ref{eq:state-details}), shifting the modeled energy.

%The set
%\begin{equation}\label{eq:area}
%  \mathcal{P}_i:=\{\mathbf{p}_k\mid\varphi_i(\mathbf{p}_k,c_{i}^\rho)\in\mathcal{C}_i\},
%\end{equation}
%delimits the area where the $i$-th path $\varphi_i$ is free to evolve using the path parameters $c_i^\rho$ (the gray area in Fig.~\ref{fig:tee1}). $\varphi_i$ is a function of the two coordinates and the path parameters, and is equal to zero when a point $\mathbf{p}_k$ is on the path. Physically, this means the UAV is flying exactly over the nominal trajectory. The path parameters allows to change the path. They are a way to alter the nominal trajectory in the initial plan and thus alter the energy by changing the flying time in the example in Fig.~\ref{fig:il-abs}.
%In fact, the algorithm uses the set from Eq.~(\ref{eq:area}) to find the path parameters such that the plan consisting of flying $\varphi_i$ has the highest energy, while still respecting the constraints. In Fig.~\ref{fig:tee1}, the parameter radius of the circle $c_{1,1}$ is replanned as, e.g., averse atmospheric conditions do not allow to terminate the plan.

%We derive the new position $\mathbf{p}_{k+1}$ computing the vector field $\nabla\varphi_i:=\begin{bmatrix}\partial\varphi_i/\partial x & \partial\varphi_i/\partial y\end{bmatrix}^T$, and the direction to follow in the form of velocity vector~\cite{de2017guidance}
%\begin{equation}\label{eq:pd}
%  \dot{\mathbf{p}}_d(\mathbf{p}_k):=E\nabla\varphi_i-k_e\varphi_i\nabla\varphi_i,\,\,\,E=\begin{bmatrix}
%    0&1\\-1&0
%  \end{bmatrix},
%\end{equation}
%where $E$ specifies the rotation (it influence the tracking direction), and $k_e\in\mathbb{R}_{\geq 0}$ the gain to adjusts the speed of convergence. The direction the velocity vector $\dot{\mathbf{p}}_d$ is pointing at is generally different from the course heading $\dot{\mathbf{p}}$ due to the atmospheric interferences (wind $w\in\mathbb{R}$ in the top of Fig.~\ref{fig:tee1}).

%The scaling factors for the path parameters from Eq.~(\ref{eq:state-control}) are derived empirically. For the example in Fig.~\ref{fig:tee1}, we can obtain the scaling factor $\nu_{1,1}$ measuring the time needed to compute the path with the lowest configuration $\underline{c}_1$, $\underline{t}$ and the highest $\overline{t}$. The variation of the control hence results in an approximate measure of the plans' time variation with factors
%\begin{equation}\label{eq:scale-traj}\begin{split}
%  \nu_{i,j}&=\left((\overline{t}-\underline{t})/(\overline{c}_{i,j}-\underline{c}_{i,j})\right)/\rho,\\
%  \tau_{i,j}&=\left(\underline{c}_{i,j}(\underline{t}-\overline{t})/(\overline{c}_{i,j}-\underline{c}_{i,j})+\underline{t}\right)/\rho,
%\end{split}\end{equation} 
%Whenever the trajectory parameters are not equally distributed, one can define $(y_{\overline{c}_i}-y_{\underline{c}_i})$ as a the highest (and lowest) levels of specific trajectory parameters. 
%$\forall j\in[\rho]^+$. Moreover, let the factors be zero when the parameters set $c_i^\rho=\{\emptyset\}$.

%\subsection{Computation parameters energy contribution}
%\label{sec:computations-model}

%Let us recall from Definition~\ref{def:mission} that the $i$-th stage $\Gamma_i$ of the plan $\Gamma$ contains the computation parameters which characterize the computations. We estimate the energy cost of these computations using \powprof{}, the open-source modeling tool adapted from earlier work on computational energy analysis~\cite{seewald2019coarse, seewald2019component}, and energy estimation of a fixed-wing UAV~\cite{seewald2020mechanical}. 

%For this purpose, we assume the UAV carries an embedded board that runs the computations. Our tool measures the instantaneous energy consumption of a subset of possible computation parameters within the computation constraint sets and builds an energy model: a linear interpolation, one per each computation. 

%The computations are implemented by software components, e.g., Robot Operating System (ROS) nodes in a ROS-based system~\cite{quigley2009ros}. The user implements these nodes such that they change the computational load according to node-specific ROS parameters--the computation parameters. In a generic software component system, the user maps the computational load to the arguments~\cite{seewald2019component}. In both cases, with ROS~\cite{zamanakos2020energy} or with generic software components system~\cite{seewald2019component}, the tool performs automatic modeling. For instance, if the computation is an object detector, a computation parameter $c_{1,2}$ might correspond to frames-per-second (fps) rate. The tool then measures power according to the detection frequency.

%We note that while the path can differ for each stage, the tasks remain the same. However, the user can inhibit or enable a computation varying its computation constraint set.

%Let us define $g:\mathbb{Z}_{\geq 0}\rightarrow\mathbb{R}_{\geq 0}$ as the instantaneous computational energy consumption value obtained using the tool.

%The scaling factors add the computational energy component to the model in Eq.~(\ref{eq:state-perf}). They are derived similarly to Eq.~(\ref{eq:scale-traj})
%\begin{equation*}\begin{split}
%  \nu_{i,j}&=(g(\overline{c}_{i,j})-g(\underline{c}_{i,j}))/(\overline{c}_{i,j}-\underline{c}_{i,j}),\\
%  \tau_{i,j}&=\underline{c}_{i,j}(g(\underline{c}_{i,j})-g(\overline{c}_{i,j}))/(\overline{c}_{i,j}-\underline{c}_{i,j})+g(\underline{c}_{i,j}),
%\end{split}\end{equation*}
%$\forall j\in[\rho+1,\rho+\sigma]$. Moreover, let the factors be zero when the parameters set $c_i^\sigma=\{\emptyset\}$.


%%%%%%%%%%%%%%%%%%%%%%%%%%%%%%%%
\section{Planning-Scheduling}  %
\label{sec:algo}               %
                               %
\begin{figure}[t]
  \footnotesize
  \begin{minipage}[c]{0.23\columnwidth}
    \vspace*{-6.6ex}
    \caption{Change of the path parameter $c_{i,1}$, the radius of the circle (i.e., the alteration of the plan in Fig.~\ref{fig:il-abs}).}
    \label{fig:tee1}
  \end{minipage}\hfill
  \begin{minipage}[c]{0.05\columnwidth}
  \end{minipage}\hfill
  \begin{minipage}[c]{0.71\columnwidth}
    \centering
    
\definecolor{c2B2B2B}{RGB}{43,43,43}
\definecolor{cDEDEDE}{RGB}{222,222,222}
\definecolor{c989898}{RGB}{152,152,152}
\definecolor{cFFFFFF}{RGB}{255,255,255}
\definecolor{c4D4D4D}{RGB}{77,77,77}
\definecolor{c9B9B9B}{RGB}{155,155,155}


\def \globalscale {1.000000}
\begin{tikzpicture}[y=0.80pt, x=0.80pt, yscale=-\globalscale, xscale=\globalscale, inner sep=0pt, outer sep=0pt]
\path[fill=c2B2B2B,line join=round,line width=0.256pt] (101.0350,53.6179) -- (96.2027,55.8748) -- (95.7468,54.6788) -- (100.5790,52.4218) -- (101.0350,53.6179) -- cycle(111.1420,50.5445) -- (106.0290,52.0626) -- (105.7610,50.8111) -- (110.8730,49.2930) -- (111.1420,50.5445) -- cycle;



\path[draw=c2B2B2B,line join=round,line width=1.024pt] (128.3820,48.6757) .. controls (127.0480,48.5917) and (125.7060,48.5494) .. (124.3560,48.5494) .. controls (119.7720,48.5494) and (115.2670,49.0390) .. (110.8680,49.9835);



  \path[fill=cDEDEDE,line join=round,even odd rule,line width=0.160pt] (201.2220,113.9290) .. controls (207.0530,111.9750) and (213.0250,110.6150) .. (219.4860,110.5010) -- (219.4860,142.5600) .. controls (204.1730,143.1220) and (191.8220,155.2870) .. (190.9640,170.5230) -- (190.9770,183.1500) -- (190.9640,183.4360) -- (190.9640,183.7260) .. controls (190.9770,184.4540) and (191.0820,185.2110) .. (191.0620,185.9270) -- (191.0670,185.9340) -- (191.0670,187.7060) -- (190.9640,227.8200) -- (190.9640,235.0380) -- (190.9450,235.0380) -- (190.9450,235.1810) -- (190.9340,235.9910) -- (190.9920,239.6820) -- (173.2000,239.6860) .. controls (173.4790,219.6570) and (172.4790,191.1610) .. (173.3370,172.5260) .. controls (173.7560,163.4250) and (175.7760,156.4540) .. (176.3670,154.3980) .. controls (178.5870,146.6790) and (177.0030,123.9390) .. (200.7770,114.0980) -- (201.2220,113.9290) -- cycle;



  \path[draw=c989898,line join=round,line width=0.512pt] (220.5790,172.1070) ellipse (1.7421cm and 1.7421cm);



  \path[draw=black,line join=round,line width=0.512pt] (222.7480,174.1440) -- (218.4670,169.8620);



  \path[draw=c2B2B2B,line join=round,line width=0.512pt] (283.0110,95.4499) -- (283.0100,239.6230);



  \path[draw=black,line join=round,line width=1.024pt] (212.5660,111.2220) .. controls (214.2910,110.4830) and (220.9040,110.5350) .. (220.9040,110.5350) .. controls (254.9950,110.5350) and (282.6310,138.1720) .. (282.6310,172.2630);



  \path[draw=black,line join=round,line width=0.512pt] (218.4680,174.1410) -- (222.7490,169.8600);



    \path[fill=cFFFFFF,line join=round,line width=0.160pt] (289.4020,105.0980) -- (276.9530,105.0980) -- (276.9530,121.0670) -- (289.4080,121.0670) -- (289.4020,105.0980) -- cycle;



    \path[cm={{1.0,0.0,0.0,1.0,(273.0,118.0)}}] (0.0000,0.0000) node[above right] () {$\varphi_{12}$};



  \path[draw=black,line join=round,line width=1.024pt] (282.6300,239.6340) -- (282.6300,172.0640);



  \path[draw=c2B2B2B,line join=round,line width=0.512pt] (221.1170,172.3260) ellipse (1.3451cm and 1.3451cm);



  \path[draw=c4D4D4D,line join=round,line width=0.512pt] (173.4260,95.3725) -- (173.4260,239.5460);



  \path[draw=black,line join=round,line width=0.512pt] (265.2780,189.1390) -- (220.7780,172.0720);



  \path[draw=black,line join=round,line width=1.024pt] (212.0450,110.8620) .. controls (176.5020,117.4390) and (178.8960,145.6300) .. (176.3720,154.4050) .. controls (173.7630,163.4760) and (173.3650,171.9460) .. (173.3650,171.9460) -- (173.3790,172.2110) -- (173.4210,172.6590);



  \path[draw=black,fill=c9B9B9B,line join=round,line width=0.512pt] (173.8510,165.3770) -- (181.8430,152.9480) -- (176.6230,153.7110) -- (172.6800,150.6460) -- (173.8510,165.3770) -- cycle;



  \path[fill=black,line join=round,line width=0.256pt] (172.7330,229.2390) -- (172.7330,223.9060) -- (174.0130,223.9060) -- (174.0130,229.2390) -- (172.7330,229.2390) -- cycle(172.7330,218.5730) -- (172.7330,213.2390) -- (174.0130,213.2390) -- (174.0130,218.5730) -- (172.7330,218.5730) -- cycle(172.7330,207.9060) -- (172.7330,202.5730) -- (174.0130,202.5730) -- (174.0130,207.9060) -- (172.7330,207.9060) -- cycle(172.7330,197.2390) -- (172.7330,191.9060) -- (174.0130,191.9060) -- (174.0130,197.2390) -- (172.7330,197.2390) -- cycle(172.7330,186.5730) -- (172.7330,181.2390) -- (174.0130,181.2390) -- (174.0130,186.5730) -- (172.7330,186.5730) -- cycle(172.7330,175.9060) -- (172.7330,172.3360) -- (174.0130,172.3360) -- (174.0130,175.9060) -- (172.7330,175.9060) -- cycle(172.7330,239.9060) -- (172.7330,234.5730) -- (174.0130,234.5730) -- (174.0130,239.9060) -- (172.7330,239.9060) -- cycle;



    \path[fill=cFFFFFF,line join=round,line width=0.160pt,rounded corners=0.0000cm] (167.3680,105.0980) rectangle (179.8176,121.0676);



    \path[cm={{1.0,0.0,0.0,1.0,(164.0,118.0)}}] (0.0000,0.0000) node[above right] () {$\varphi_{14}^-$};



    \path[fill=cFFFFFF,line join=round,line width=0.160pt] (177.5750,234.3400) -- (165.1260,234.3400) -- (165.0920,246.4150) -- (177.5710,246.3840) -- (177.5750,234.3400) -- cycle;



    \path[cm={{1.0,0.0,0.0,1.0,(164.0,245.0)}}] (0.0000,0.0000) node[above right] () {$\overline{c}_{14}^-$};



    \path[fill=cFFFFFF,line join=round,line width=0.160pt,rounded corners=0.0000cm] (230.3230,173.2320) rectangle (251.5725,189.2016);



    \path[cm={{1.0,0.0,0.0,1.0,(232.0,185.0)}}] (0.0000,0.0000) node[above right] () {$c_{1,1}^-$};



  \path[draw=black,fill=cFFFFFF,line join=round,line width=0.512pt] (203.3560,114.4210) -- (218.0050,112.4770) -- (213.5600,109.7230) -- (213.8340,104.0000) -- (203.3560,114.4210) -- cycle;



    \path[fill=cFFFFFF,line join=round,line width=0.160pt] (199.4770,144.8830) -- (183.5080,144.8830) -- (183.5080,160.8530) -- (199.4770,160.8530) -- (199.4770,144.8830) -- cycle;



    \path[cm={{1.0,0.0,0.0,1.0,(185.0,158.0)}}] (0.0000,0.0000) node[above right] () {$\mathbf{p}_{k_6}$};

    \path[cm={{1.0,0.0,0.0,1.0,(161.0,170.0)}}] (0.0000,0.0000) node[above right] () {\ref{sth:iii}};

    \path[fill=cFFFFFF,line join=round,line width=0.160pt,rounded corners=0.0000cm] (192.4050,96.8819) rectangle (204.8546,109.3314);


  \path[draw=black,line join=round,line width=0.512pt] (203.7170,114.2570) -- (213.5160,109.7120);



  \path[draw=black,line join=round,line width=0.512pt] (174.0250,164.7330) -- (176.5840,153.7760);



    \path[fill=cFFFFFF,line join=round,line width=0.160pt] (219.2260,208.5700) -- (203.2560,208.5700) -- (203.2560,224.5400) -- (219.2260,224.5390) -- (219.2260,208.5700) -- cycle;



    \path[cm={{1.0,0.0,0.0,1.0,(202.0,222.0)}}] (0.0000,0.0000) node[above right] () {$\varphi_{13}^-$};



  \path[draw=c989898,line join=round,line width=0.512pt] (158.8570,95.3893) -- (158.8570,239.5620);



  \path[cm={{1.0,0.0,0.0,1.0,(181.0,261.0)}}] (0.0000,0.0000) node[above right] () {\footnotesize (b) replanned path};



  \path[fill=black,line join=round,line width=0.160pt] (262.5930,185.4440) -- (260.7610,189.7900) -- (265.6570,189.2950) -- (262.5930,185.4440) -- cycle;



\path[draw=c2B2B2B,line join=round,line width=0.512pt] (137.9100,95.3391) -- (137.9100,239.5120);



  \path[fill=cFFFFFF,line join=round,line width=0.160pt] (144.6860,105.0630) -- (131.2360,105.0630) -- (131.2360,121.0330) -- (144.6960,121.0670) -- (144.6860,105.0630) -- cycle;



  \path[cm={{1.0,0.0,0.0,1.0,(128.0,118.0)}}] (0.0000,0.0000) node[above right] () {$\varphi_{12}$};



\path[fill=cDEDEDE,line join=round,line width=0.160pt] (14.4062,169.9360) -- (14.4311,169.9360) .. controls (15.3584,137.0320) and (42.0351,110.5710) .. (75.0243,109.9900) -- (75.0243,142.0480) .. controls (59.7109,142.6110) and (47.3600,154.7750) .. (46.5019,170.0120) -- (46.5148,182.6390) -- (46.5019,182.9250) -- (46.5019,183.2150) .. controls (46.5154,183.9430) and (46.6203,184.7000) .. (46.6001,185.4160) -- (46.6055,185.4230) -- (46.6055,187.1950) -- (46.5019,227.3090) -- (46.5019,234.5270) -- (46.4833,234.5270) -- (46.4830,234.6700) -- (46.4723,235.4800) -- (46.5303,239.1710) -- (14.2304,239.1780) .. controls (14.2304,237.5050) and (14.2092,240.8130) .. (14.1991,237.1110) -- (14.1991,235.5690) -- (14.1991,235.2560) -- (14.1991,234.8370) -- (14.1991,234.5320) -- (14.1991,171.8980) -- (14.4062,169.9360) -- cycle;



\path[draw=c2B2B2B,line join=round,line width=0.512pt] (75.8516,172.0930) ellipse (1.7421cm and 1.7421cm);



\path[draw=black,line join=round,line width=0.512pt] (78.0152,174.1270) -- (73.7395,169.8460);



\path[draw=black,line join=round,line width=0.512pt] (73.7417,174.1270) -- (78.0230,169.8460);



\path[draw=c2B2B2B,line join=round,line width=0.512pt] (14.1483,95.3578) -- (14.1477,239.5310);



\path[draw=black,line join=round,line width=0.512pt] (75.9234,172.0190) -- (14.0743,172.0190);



\path[fill=black,line join=round,line width=0.160pt] (19.0789,174.3350) -- (19.0727,169.6190) -- (14.7563,171.9820) -- (19.0789,174.3350) -- cycle;



  \path[fill=cFFFFFF,line join=round,line width=0.160pt,rounded corners=0.0000cm] (8.0901,105.0980) rectangle (20.5396,121.0676);



  \path[cm={{1.0,0.0,0.0,1.0,(5.0,118.0)}}] (0.0000,0.0000) node[above right] () {$\varphi_{14}$};



\path[draw=black,line join=round,line width=1.024pt] (67.5256,111.0380) .. controls (69.2504,110.2990) and (75.8630,110.3510) .. (75.8630,110.3510) .. controls (109.9540,110.3510) and (137.5900,137.9880) .. (137.5900,172.0790);



  \path[fill=cFFFFFF,line join=round,line width=0.160pt,rounded corners=0.0000cm] (115.2640,126.2090) rectangle (131.2335,138.6585);



  \path[cm={{1.0,0.0,0.0,1.0,(114.0,137.0)}}] (0.0000,0.0000) node[above right] () {$\varphi_{13}$};



  \path[fill=cFFFFFF,line join=round,line width=0.160pt] (52.6996,234.3410) -- (40.2501,234.3410) -- (40.2318,246.3700) -- (52.7105,246.3700) -- (52.6996,234.3410) -- cycle;



  \path[cm={{1.0,0.0,0.0,1.0,(44.0,245.0)}}] (0.0000,0.0000) node[above right] () {$\underline{c}_{14}$};



  \path[fill=cFFFFFF,line join=round,line width=0.160pt] (64.6687,169.3940) -- (46.9391,169.3930) -- (46.9391,181.8430) -- (64.6687,181.8430) -- (64.6687,169.3940) -- cycle;



  \path[cm={{1.0,0.0,0.0,1.0,(47.0,181.0)}}] (0.0000,0.0000) node[above right] () {$c_{1,1}$};



\path[fill=black,line join=round,line width=0.256pt] (13.8207,228.7410) -- (13.8435,223.4080) -- (15.1234,223.4140) -- (15.1007,228.7470) -- (13.8207,228.7410) -- cycle(13.8662,218.0750) -- (13.8890,212.7420) -- (15.1690,212.7470) -- (15.1462,218.0800) -- (13.8662,218.0750) -- cycle(13.9117,207.4080) -- (13.9345,202.0750) -- (15.2145,202.0800) -- (15.1917,207.4140) -- (13.9117,207.4080) -- cycle(13.9572,196.7420) -- (13.9800,191.4080) -- (15.2600,191.4140) -- (15.2372,196.7470) -- (13.9572,196.7420) -- cycle(14.0028,186.0750) -- (14.0255,180.7420) -- (15.3055,180.7470) -- (15.2828,186.0810) -- (14.0028,186.0750) -- cycle(14.0483,175.4090) -- (14.0711,170.0750) -- (15.3510,170.0810) -- (15.3283,175.4140) -- (14.0483,175.4090) -- cycle(14.0938,164.7420) -- (14.1004,163.1940) -- (14.1116,163.0700) -- (14.1481,162.9500) -- (14.2078,162.8390) -- (14.2888,162.7440) -- (14.3849,162.6630) -- (14.4954,162.6040) -- (14.6155,162.5680) -- (14.7404,162.5570) -- (14.1975,162.5580) -- (14.4982,161.0400) -- (14.8987,159.3600) -- (16.1499,159.6300) -- (15.7494,161.3100) -- (15.4579,162.7820) -- (14.7404,163.8370) -- (15.3804,163.2000) -- (15.3738,164.7470) -- (14.0938,164.7420) -- cycle(16.4171,154.1910) -- (17.1786,151.9470) -- (18.2993,149.1670) -- (19.5002,149.6100) -- (18.3795,152.3900) -- (17.6397,154.5700) -- (16.4171,154.1910) -- cycle(20.5366,144.2860) -- (20.9681,143.3850) -- (23.1393,139.5830) -- (24.2741,140.1750) -- (22.1029,143.9770) -- (21.7089,144.7990) -- (20.5366,144.2860) -- cycle(26.1128,135.1000) -- (26.8488,134.0250) -- (29.4547,130.8780) -- (30.4785,131.6470) -- (27.8725,134.7930) -- (27.1990,135.7770) -- (26.1128,135.1000) -- cycle(33.1525,126.9590) -- (35.2417,124.9010) -- (37.1822,123.3740) -- (38.0303,124.3330) -- (36.0897,125.8600) -- (34.0975,127.8220) -- (33.1525,126.9590) -- cycle(41.5104,120.1440) -- (46.0624,117.3650) -- (46.7947,118.4150) -- (42.2427,121.1940) -- (41.5104,120.1440) -- cycle(50.9592,115.0720) -- (53.4451,113.9480) -- (56.0106,113.1560) -- (56.4648,114.3530) -- (53.8993,115.1450) -- (51.5589,116.2030) -- (50.9592,115.0720) -- cycle(61.1063,111.5820) -- (61.2085,111.5500) -- (66.4314,110.6160) -- (66.7338,111.8590) -- (61.5109,112.7940) -- (61.5606,112.7790) -- (61.1063,111.5820) -- cycle(13.7752,239.4080) -- (13.7979,234.0750) -- (15.0779,234.0800) -- (15.0552,239.4130) -- (13.7752,239.4080) -- cycle;



\path[cm={{1.0,0.0,0.0,1.0,(43.0,110.0)}}] (0.0000,0.0000) node[above right] () {$\mathbf{p}_{k_5}$};

\path[cm={{1.0,0.0,0.0,1.0,(72.0,105.0)}}] (0.0000,0.0000) node[above right] () {\ref{sth:i}};

  \path[fill=cFFFFFF,line join=round,line width=0.160pt] (18.9147,234.3410) -- (6.4652,234.3410) -- (6.4316,246.4160) -- (18.9102,246.3850) -- (18.9147,234.3410) -- cycle;



  \path[cm={{1.0,0.0,0.0,1.0,(6.0,245.0)}}] (0.0000,0.0000) node[above right] () {$\overline{c}_{14}$};



\path[draw=black,fill=c9B9B9B,line join=round,line width=0.512pt] (58.9895,114.4960) -- (73.6370,112.5410) -- (69.1896,109.7900) -- (69.4587,104.0670) -- (58.9895,114.4960) -- cycle;



\path[draw=black,line join=round,line width=0.512pt] (59.2931,114.3470) -- (69.0932,109.8070);



\path[cm={{1.0,0.0,0.0,1.0,(37.0,261.0)}}] (0.0000,0.0000) node[above right] () {\footnotesize (a) initial path};



\path[draw=black,line join=round,line width=1.024pt] (137.5900,239.4500) -- (137.5900,171.8800);



\path[fill=cDEDEDE,line join=round,even odd rule,line width=0.160pt] (237.0260,15.0828) -- (265.6100,15.0831) -- (265.6100,33.6677) -- (237.0260,33.6675) -- (237.0260,15.0828) -- cycle;



\path[cm={{1.0,0.0,0.0,1.0,(274.0,29.0)}}] (0.0000,0.0000) node[above right] () {$\mathcal{P}_i$};



\path[draw=black,line join=round,line width=1.024pt] (237.0260,52.5827) -- (265.6110,52.5829);



\path[cm={{1.0,0.0,0.0,1.0,(274.0,56.0)}}] (0.0000,0.0000) node[above right] () {$\mathcal{P}$};



\path[draw=black,fill=c9B9B9B,line join=round,line width=0.512pt] (103.0020,54.8369) -- (121.9910,51.9150) -- (117.5520,47.3939) -- (116.3400,41.0080) -- (103.0020,54.8369) -- cycle;



\path[draw=black,line join=round,line width=0.512pt] (115.1520,48.6445) -- (111.3670,18.4898);



\path[draw=black,line join=round,line width=0.512pt] (115.2340,48.4465) -- (94.3696,72.9709);



\path[cm={{1.0,0.0,0.0,1.0,(102.0,15.0)}}] (0.0000,0.0000) node[above right] () {$\nabla\varPhi$};



\path[cm={{1.0,0.0,0.0,1.0,(96.0,86.0)}}] (0.0000,0.0000) node[above right] () {$\dot{\mathbf{p}}$};



\path[cm={{1.0,0.0,0.0,1.0,(79.0,45.0)}}] (0.0000,0.0000) node[above right] () {$\dot{\mathbf{p}}_d$};



\path[draw=black,line join=round,line width=0.512pt] (117.5520,47.3796) -- (103.1340,54.7265);



\path[fill=black,line join=round,line width=0.160pt] (99.4724,72.0401) -- (94.9925,67.8546) -- (93.3980,74.0506) -- (99.4724,72.0401) -- cycle;



\path[fill=black,line join=round,line width=0.160pt] (108.7830,21.3255) -- (114.8390,20.3742) -- (110.9400,15.3018) -- (108.7830,21.3255) -- cycle;



\path[fill=black,line join=round,line width=0.160pt] (79.6380,52.9767) -- (85.2585,55.4260) -- (84.6922,49.0532) -- (79.6380,52.9767) -- cycle;



\path[draw=black,line join=round,line width=0.512pt] (82.5181,52.4879) -- (114.8620,48.5042);



\path[draw=black,line join=round,line width=0.512pt] (40.3847,53.3572) -- (62.7169,53.3572);



\path[cm={{1.0,0.0,0.0,1.0,(47.0,50.0)}}] (0.0000,0.0000) node[above right] () {$w$};



\path[fill=black,line join=round,line width=0.160pt] (61.8249,50.0137) -- (61.8330,56.1447) -- (67.4450,53.0723) -- (61.8249,50.0137) -- cycle;




\end{tikzpicture}


  \end{minipage}
  \vspace*{-4.2ex}
\end{figure}

This section solves the problem %described 
in Sec.~\ref{sec:pbfor}. It provides a plan %$\Gamma$ 
and re-plans-schedules such plan energy-wise. %respectively
%in Sec.~\ref{sec:cov-algo}--\hyperref[sec:repla-algo]{B}.

\vspace*{-1.4ex}
\subsection{Coverage}
\label{sec:cov-algo}

There are various approaches in the literature to solve CPP problems (e.g., Sec.~\ref{sec:pbfor}). Those that ensure %the 
completeness %of the cover 
are NP-hard~\cite{arkin2000approximation} and use cellular decomposition, dividing the free space into sub-regions to be easily covered~\cite{choset2001coverage,galceran2013survey}.

An intuitive way to solve the problem is with a back-and-forth motion, sweeping the space delimited by $v$ we term $\mathcal{Q}^v$. Although abundant in both mobile ground-based~\cite{choset2001coverage%,%choset2005principles,
%lavalle2006planning
} and aerial~\cite{araujo2013multiple,%artemenko2016energy,
cabreira2018energy,difranco2015energy} robotics literature, the motion, called \emph{boustrophedon motion}~\cite{choset2001coverage}, is unsuitable for aerial robots broadly, especially for fixed-wing aerial robots. These robots have reduced maneuverability~\cite{dille2013efficient,mannadiar2010optimal,%xu2011optimal,
xu2014efficient} and are generally unable to fly quick turns~\cite{wang2017curvature}.

To address fixed wings and aerial robots generally, this section details a different motion with a wide turning radius. It is similar to another motion in the literature, the \emph{Zamboni motion}~\cite{araujo2013multiple}, but additionally allows variable CPP %at the very core of this work
{\color{black} by dynamically altering the distance between the survey lines with the path parameters}. 
{\color{black} Although cover variability is already considered in the literature~\cite{difranco2015energy}, it is limited to boustrophedon motion for rotary wings.}
The novel motion is termed \emph{Zamboni-like motion} and is composed of four primitive paths% (see Definition~\ref{def:primitive})
: two lines $\varphi_1,\varphi_2$ and two circles $\varphi_3,\varphi_4$.% (see Fig.~\ref{fig:zambo}). 

We assume the vertices $v_1,v_2,\dots$ are ordered from the top-left-most vertex %in 
clockwise% order
, the aerial robot can overfly the edges formed by the vertices, and ${}^{v_x}|_{v_y}$ indicates the edge formed by vertices $v_x,v_y$. 
Algorithm~\ref{alg2} details the procedure to generate the plan $\Gamma$ that covers $\mathcal{Q}^v$ %per each 
{\color{black}at }discretized time step{\color{black}s}, i.e., $\mathcal{T}:=\{t_0,t_0+h,\dots,t_f\}$ for a given step $h\in\mathbb{R}_{>0}$. The algorithm assumes that the line parallel to ${}^{v_1}|_{v_{|v|}}$ is always connected. %as it swipes $\mathcal{Q}^{v}$. %Nonetheless, a 
{\color{black}C}omplex covering is possible by, e.g., dividing $\mathcal{Q}^{v}$ into cells %to be easily covered 
and %subsequently 
covering each cell~\cite{choset2001coverage}.% with the novel motion.

%\begin{enumerate*}[label={(\alph*)},font={\textit}]
%  \item place a line $\varphi_i$ parallel to ${}^{v_1}|_{v_{|v|}}$ that intersects edges ${}^{v_{|v|}}|_{v_{|v-1|}},{}^{v_1}|_{v_2}$ (i.e., first and last),
%  \item place a circle $\varphi_{i+1}$ whose left most point lays at $\mathbf{p}_{\Gamma_i}$, the intersection of $\varphi_i$ and the edge ${}^{v_{|v|}}|_{v_{|v-1|}}$. $\varphi_{i+1}$ intersects $v$ again at $\mathbf{p}_{\Gamma_{i+1}}$,
%  \item place a line $\varphi_{i+2}$ parallel to $\varphi_i$ that intersects $v$ at $\mathbf{p}_{\Gamma_{i+1}},\mathbf{p}_{\Gamma_{i+2}}$,
%  \item finally, place a circle $\varphi_{i+3}$ whose right most point lays at $\mathbf{p}_{\Gamma_{i+2}}$ and intersects $v$ again at $\mathbf{p}_{\Gamma_{i+3}}$.
%\end{enumerate*}

%For ease of notation, the procedure assume that the polygon $v$ is constructed in such a way that the lines $\varphi_{k+1},\varphi_{k+3}$ are never disconnected. %These are
%\begin{subequations}\label{eq:line-gene}\begin{align}
%  \varphi_{k+1}&:=x-x_{\Gamma_1}-\lfloor i/4\rfloor x_\mathbf{d},\label{eq:v1}\\
%  \varphi_{k+3}&:=x-x_{\Gamma_2}-\lfloor i/4\rfloor x_\mathbf{d}\label{eq:v3},
%\end{align}
%\end{subequations}
%where $(x_\mathbf{p},y_\mathbf{p})=:\mathbf{p}$ for any point $\mathbf{p}$, and $\lfloor\,\cdot\,\rfloor$ is the integer division (it has a different meaning with parameters set $c_i$, see Eq.~\ref{eq:piece-wise-reg}). The circles are
%\begin{subequations}\label{eq:circ-gene}\begin{align}
%  \begin{split}\varphi_{i+1}:=(&x-x_{\Gamma_1}-r_1-\lfloor i/4\rfloor x_\mathbf{d})^2+\\ (&y-y_{\Gamma_1}-\lfloor i/4\rfloor y_\mathbf{d})^2-r_1^2,\end{split}\label{eq:v2}\\
%  \begin{split}\varphi_{i+3}:=(&x-x_{\Gamma_2}+r_2-\lfloor i/4\rfloor x_\mathbf{d})^2+\\ (&y-y_{\Gamma_3}-\lfloor i/4\rfloor y_\mathbf{d})^2-r_2^2,\label{eq:second-circ-gene}\end{split}
%\end{align}
%\end{subequations}
%where $r_1,r_2\in\mathbb{R}_{>0}$ s.t. $r_1>r_2>\underline{r}$ are radiuses. $r_1=r_2+x_\mathbf{d}/2$, whereas $r_2$ is in Eq.~(\ref{eq:r2}).

%Similarly, the triggering points are
%\begin{subequations}\label{eq:trigs-gene}\begin{align}
%  \mathbf{p}_{\Gamma_i}&\hspace*{-.35ex}:=(x_{\Gamma_1}\hspace*{-.35ex}+\hspace*{-.35ex}\lfloor i/4\rfloor x_\mathbf{d},y_{\Gamma_1}\hspace*{-.35ex}+\hspace*{-.35ex}\lfloor i/4\rfloor y_\mathbf{d}),\label{eq:t1}\\
%  \mathbf{p}_{\Gamma_{i+1}}&\hspace*{-.35ex}:=(x_{\Gamma_1}\hspace*{-.35ex}+\hspace*{-.35ex}2r_1+\lfloor i/4\rfloor x_\mathbf{d},y_{\Gamma_1}\hspace*{-.35ex}+\hspace*{-.35ex}\lfloor i/4\rfloor y_\mathbf{d}),\label{eq:t2}\\
%  \mathbf{p}_{\Gamma_{i+2}}&\hspace*{-.35ex}:=(x_{\Gamma_1}\hspace*{-.35ex}+\hspace*{-.35ex}2r_1+\lfloor i/4\rfloor x_\mathbf{d},y_{\Gamma_3}\hspace*{-.35ex}+\hspace*{-.35ex}\lfloor i/4\rfloor y_\mathbf{d}),\label{eq:t3}\\
%  \mathbf{p}_{\Gamma_{i+3}}&\hspace*{-.35ex}:=(x_{\Gamma_1}\hspace*{-.35ex}+\hspace*{-.35ex}2(r_1\hspace*{-.35ex}-\hspace*{-.35ex}r_2)+\lfloor i/4\rfloor x_\mathbf{d},y_{\Gamma_3}\hspace*{-.35ex}+\hspace*{-.35ex}\lfloor i/4\rfloor y_\mathbf{d})\label{eq:last-trig-gene}.
%\end{align}
%\end{subequations}

\begin{algorithm}[t]
  \begin{algorithmic}[1]
    \small
    %\STATE $j\gets-1$

    \FORALL{$t\in\mathcal{T}$}
      \color{black}\STATE \textbf{if} $\mathbf{p}%(t)
      =\mathbf{p}_{\Gamma_l}${ in Definition~\ref{def:trigs}} \textbf{then return }$\Gamma$\vspace*{.3ex}\color{black}

%      \IF{$\mathbf{p}(i)=\mathbf{p}_{\Gamma_l}${ in Definition~\ref{def:trigs}}}
        
%        \RETURN{$\Gamma$}\vspace*{.3ex}
%      \ENDIF

      %\vspace*{.8ex}
      \color{black}\IF{$\mathbf{p}%(t)
      =\mathbf{p}_{\Gamma_i}$}\color{black}
        \STATE $i\gets i+1$\vspace*{.3ex}
        \IF{$i\notin[n]_{>0}$}\label{alg2:cond}
          \STATE $i\gets 1$\vspace*{0ex}
          \STATE $\varphi_{|\Gamma|+1}\gets$ line in Definition~\ref{def:paths} %in Eq.~(\ref{eq:v1}) 
          parallel to ${}^{v_1}|_{v_{|v|}}$ that\vspace*{.5ex} \hspace*{1em}intersects $\mathbf{p}_{|\Gamma|}$\vspace*{.3ex}

          \STATE $\mathbf{p}_{|\Gamma|+1}\gets$ other intersection %in Eq.~(\ref{eq:t1}) 
          of $\varphi_{|\Gamma|+1}$ and $v$\vspace*{.3ex}

          \STATE $\varphi_{|\Gamma|+2}\gets$ circle %in Eq.~(\ref{eq:v2}) 
          whose left most point lays on $\mathbf{p}_{|\Gamma|+1}$\vspace*{.3ex}\label{alg2:circ1}
          
          \STATE $\mathbf{p}_{|\Gamma|+2}\gets$ other inter. %in Eq~(\ref{eq:t2}) 
          of $\varphi_{|\Gamma|+2}$ and $v$\vspace*{.3ex}

          \STATE $\varphi_{|\Gamma|+3}\gets$ line %in Eq.~(\ref{eq:v3}) 
          par. to $\varphi_{|\Gamma|+1}$ that inter. $\mathbf{p}_{|\Gamma|+2}$\vspace*{.3ex}

          \STATE $\mathbf{p}_{|\Gamma|+3}\gets$ other inter. %in Eq~(\ref{eq:t3})
          of $\varphi_{|\Gamma|+3}$ and $v$\vspace*{.3ex}

          \STATE $\varphi_{|\Gamma|+4}\gets$ circle in Eq.\hspace*{.7ex}(\ref{eq:second-circ-gene}) 
          whose right most point\vspace*{.3ex} \hspace*{1em}lays on $\mathbf{p}_{|\Gamma|+3}$\vspace*{.3ex}\label{alg2:circ2}

          \STATE $\mathbf{p}_{|\Gamma|+4}\gets$ other inter. %in Eq~(\ref{eq:last-trig-gene}) 
          of $\varphi_{|\Gamma|+4}$ and $v$\vspace*{.3ex}\label{alg2:trig4}

          \vspace*{.8ex}
          \STATE $\Gamma\gets\Gamma\cup\{\Gamma_{|\Gamma|+1},\dots,\Gamma_{|\Gamma|+4}\}${ in Definitions~\ref{def:stage}--\hyperref[def:plan]{4}}\label{alg2:last}

        \ENDIF
      \ENDIF
    \ENDFOR
  \end{algorithmic}
  \caption{Zamboni-like motion for CPP}\label{alg2}
\end{algorithm}

To implement the variable CPP, the radius $r_2$ of the second circle $\varphi_{|\Gamma|+4}$ on Line~\ref{alg2:circ2}
\vspace*{-1ex}
\begin{equation}\label{eq:r2}
  {\small r_2(c_{i,1}):=\sqrt{\smash[b]{r^2+c_{i,1}}},}
  \vspace*{-.4ex}
\end{equation}
is expressed as a function of a path parameter $c_{i,1}\in(\underline{r}^2-r^2,0]$, relative to the last circle in each set of primitive stages. $r\in\mathbb{R}_{>0}$ is a given ideal turning radius along with the minimum radius (see Sec.~\ref{pb:cov-pb}). The center also changes
\begin{equation}\label{eq:second-circ-gene}
  \varphi_{|\Gamma|+4}:=(x-x_{\mathbf{p}_{|\Gamma|+3}}+r_2)^2+(y-y_{\mathbf{p}_{|\Gamma|+3}})^2-r_2^2,
\end{equation}
where $(x_\mathbf{p},y_\mathbf{p})=:\mathbf{p}$ for any point $\mathbf{p}$. Fig.~\ref{fig:tee1} illustrates the concept of $c_{i,1}$ altering the CPP. The radius of the first circle on Line~\ref{alg2:circ1} is then $r_1:=r+x_\mathbf{d}/2$ (i.e., the radiuses of the two circles ensure that the primitive paths are shifted of $\mathbf{d}$).% in Definition~\ref{def:primitive}).

Algorithm~\ref{alg2} initializes $i$ to minus one and builds the first four primitive functions $\varphi_1,\dots,\varphi_4$. The remaining $\Gamma$ is built with the shift $\mathbf{d}$ up to the final point $\mathbf{p}_{\Gamma_l}$. The initial point is $\mathbf{p}_{\Gamma_1}$, placed s.t. the line $\varphi_1$ is at the same distance from an eventual previous line, e.g., $x_{\mathbf{p}_{\Gamma_1}}=x_{v_1}+x_{\mathbf{d}}/2$ in Fig.~\ref{fig:zambo}.

\begin{figure}[t]
  \vspace*{-1ex}
  \footnotesize
  \begin{minipage}[l]{0.7\columnwidth}
    \centering
    


\def \globalscale {.800000}
\begin{tikzpicture}[y=0.80pt, x=0.80pt, yscale=-\globalscale, xscale=.96*\globalscale, inner sep=0pt, outer sep=0pt]
\path[draw=black,line join=round,line width=0.512pt] (20.8642,24.8115) -- (118.7080,14.9592) -- (118.7080,74.3650) -- (20.8642,84.2173) -- (20.8642,24.8115) -- cycle;



\path[draw=black,line join=round,line width=0.384pt] (23.2872,24.4960) -- (23.2868,79.6844);



\path[draw=black,line join=round,line width=0.384pt] (27.4231,24.1801) .. controls (27.4231,12.2018) and (37.1342,2.4907) .. (49.1125,2.4907) .. controls (61.0909,2.4907) and (70.8020,12.2018) .. (70.8020,24.1801);



\path[draw=black,line join=round,line width=0.384pt] (70.8020,79.1472) .. controls (70.8020,92.2682) and (60.1654,102.9050) .. (47.0447,102.9050) .. controls (33.9239,102.9050) and (23.2873,92.2682) .. (23.2873,79.1472);



\path[draw=black,line join=round,line width=0.384pt] (27.4062,24.1425) -- (27.4062,78.7861);



\path[draw=black,line join=round,line width=0.384pt] (70.8019,24.1163) -- (70.8020,70.1775);



\path[draw=black,line join=round,line width=0.256pt] (70.8020,70.1107) -- (70.8020,79.2413);



\path[draw=black,line join=round,line width=0.384pt] (35.6779,23.2120) -- (35.6779,77.9830);



\path[draw=black,line join=round,line width=0.384pt] (39.8135,22.9055) -- (39.8136,77.5574);



\path[draw=black,line join=round,line width=0.384pt] (43.9493,22.5103) -- (43.9494,77.1626);



\path[draw=black,line join=round,line width=0.384pt] (48.0851,22.1465) -- (48.0851,81.3569);



\path[draw=black,line join=round,line width=0.384pt] (74.9207,23.6118) -- (74.9209,78.7368);



\path[draw=black,line join=round,line width=0.384pt] (31.5420,23.7911) .. controls (31.5420,11.8128) and (41.2531,2.1017) .. (53.2314,2.1017) .. controls (65.2098,2.1017) and (74.9209,11.8128) .. (74.9209,23.7911);



\path[draw=black,line join=round,line width=0.384pt] (31.5423,23.7650) -- (31.5420,78.4794);



\path[draw=black,line join=round,line width=0.384pt] (23.2872,24.4960) -- (23.2868,79.6844);



\path[draw=black,line join=round,line width=0.384pt] (79.0565,23.3177) -- (79.0567,78.4427);



\path[draw=black,line join=round,line width=0.384pt] (35.6778,23.4077) .. controls (35.6778,11.4293) and (45.3889,1.7182) .. (57.3672,1.7182) .. controls (69.3455,1.7182) and (79.0566,11.4293) .. (79.0566,23.4077);



\path[draw=black,line join=round,line width=0.384pt] (83.1923,22.8332) -- (83.1924,77.9583);



\path[draw=black,line join=round,line width=0.384pt] (39.8135,22.9506) .. controls (39.8135,10.9723) and (49.5246,1.2612) .. (61.5029,1.2612) .. controls (73.4812,1.2612) and (83.1924,10.9723) .. (83.1924,22.9506);



\path[draw=black,line join=round,line width=0.384pt] (79.0567,78.3999) .. controls (79.0567,91.5208) and (68.4202,102.1570) .. (55.2994,102.1570) .. controls (42.1786,102.1570) and (31.5421,91.5208) .. (31.5421,78.3999);



\path[draw=black,line join=round,line width=0.384pt] (83.1924,77.9265) .. controls (83.1924,91.0475) and (72.5559,101.6840) .. (59.4351,101.6840) .. controls (46.3143,101.6840) and (35.6778,91.0475) .. (35.6778,77.9265);



\path[draw=black,line join=round,line width=0.384pt] (85.7727,85.9816) .. controls (82.3601,94.9191) and (73.7067,101.2660) .. (63.5708,101.2660) .. controls (50.4501,101.2660) and (39.8135,90.6295) .. (39.8135,77.5085);



\path[draw=black,line join=round,line width=0.384pt] (87.3280,22.3917) -- (87.3282,77.5167);



\path[draw=black,line join=round,line width=0.384pt] (91.4639,77.1154) .. controls (91.4639,90.2362) and (80.8274,100.8730) .. (67.7066,100.8730) .. controls (54.5858,100.8730) and (43.9493,90.2362) .. (43.9493,77.1154);



\path[draw=black,line join=round,line width=0.384pt] (91.4639,21.9514) -- (91.4640,77.0764);



\path[draw=black,line join=round,line width=0.512pt] (172.6530,23.1022) -- (272.3070,13.0677) -- (272.3070,73.5725) -- (172.6530,83.6069) -- (172.6530,23.1022) -- cycle;



\path[draw=black,fill=black,line join=round,line width=0.512pt] (172.6450,21.8970) .. controls (173.2860,21.8970) and (173.8060,22.4172) .. (173.8060,23.0587) .. controls (173.8060,23.7003) and (173.2860,24.2205) .. (172.6450,24.2205) .. controls (172.0030,24.2205) and (171.4830,23.7003) .. (171.4830,23.0587) .. controls (171.4830,22.4172) and (172.0030,21.8970) .. (172.6450,21.8970) -- cycle;



\path[cm={{1.0,0.0,0.0,1.0,(154.0,23.0)}}] (0.0000,0.0000) node[above right] () {$v_1$};



\path[cm={{1.0,0.0,0.0,1.0,(153.0,88.0)}}] (0.0000,0.0000) node[above right] () {$v_4$};



\path[cm={{1.0,0.0,0.0,1.0,(278.0,11.0)}}] (0.0000,0.0000) node[above right] () {$v_2$};



\path[cm={{1.0,0.0,0.0,1.0,(280.0,77.0)}}] (0.0000,0.0000) node[above right] () {$v_3$};



\path[draw=black,line join=round,line width=0.384pt] (175.1210,22.7809) -- (175.1200,78.9904);



\path[draw=black,line join=round,line width=0.384pt] (183.1970,22.0352) .. controls (183.1970,10.9023) and (192.2230,1.8766) .. (203.3560,1.8766) .. controls (214.4890,1.8766) and (223.5150,10.9023) .. (223.5150,22.0352);



\path[draw=black,line join=round,line width=0.384pt] (223.5150,78.4434) .. controls (223.5150,91.8070) and (212.6810,102.6400) .. (199.3180,102.6400) .. controls (185.9540,102.6400) and (175.1210,91.8070) .. (175.1210,78.4434);



\path[draw=black,line join=round,line width=0.384pt] (223.5140,21.9733) -- (223.5150,69.3075);



\path[draw=black,line join=round,line width=0.384pt] (223.5150,69.2395) -- (223.5150,78.5389);



\path[draw=black,line join=round,line width=0.384pt] (183.1980,21.4732) -- (183.1980,77.7401);



\path[draw=black,line join=round,line width=0.384pt] (191.2740,21.2319) -- (191.2740,77.1984);



\path[draw=black,line join=round,line width=0.384pt] (175.1210,22.7809) -- (175.1200,78.9904);



\path[draw=black,line join=round,line width=0.384pt] (231.5910,21.0884) -- (231.5910,77.6542);



\path[draw=black,line join=round,line width=0.384pt] (231.5910,77.6230) .. controls (231.5910,90.9867) and (220.7580,101.8200) .. (207.3940,101.8200) .. controls (194.0310,101.8200) and (183.1980,90.9867) .. (183.1980,77.6230);



\path[draw=black,line join=round,line width=0.384pt] (191.2740,21.2576) .. controls (191.2740,10.1247) and (200.3000,1.0990) .. (211.4330,1.0990) .. controls (222.5660,1.0990) and (231.5910,10.1247) .. (231.5910,21.2576);



\path[draw=black,line join=round,line width=0.384pt] (239.6680,76.8666) .. controls (239.6680,90.2302) and (228.8350,101.0630) .. (215.4710,101.0630) .. controls (202.1070,101.0630) and (191.2740,90.2302) .. (191.2740,76.8666);



\path[draw=black,line join=round,line width=0.384pt] (239.6680,20.3473) -- (239.6680,76.9131);



\path[draw=black,line join=round,line width=0.384pt] (199.3510,20.3985) .. controls (199.3510,9.2656) and (208.3760,0.2399) .. (219.5090,0.2399) .. controls (230.6420,0.2399) and (239.6680,9.2656) .. (239.6680,20.3985);



\path[draw=black,line join=round,line width=0.384pt] (199.3510,20.5179) -- (199.3510,80.9456);



\path[draw=black,line join=round,line width=0.384pt] (74.9230,78.7915) .. controls (74.9230,91.9124) and (64.2864,102.5490) .. (51.1657,102.5490) .. controls (38.0449,102.5490) and (27.4083,91.9124) .. (27.4083,78.7915);



\path[draw=black,line join=round,line width=0.384pt] (87.3488,77.5084) .. controls (87.3488,90.6294) and (76.7122,101.2660) .. (63.5914,101.2660);



\path[draw=black,line join=round,line width=0.384pt] (43.9493,22.5457) .. controls (43.9493,10.5673) and (53.6604,0.8563) .. (65.6387,0.8563) .. controls (77.6171,0.8563) and (87.3282,10.5673) .. (87.3282,22.5457);



\path[draw=black,line join=round,line width=0.384pt] (48.0851,22.1941) .. controls (48.0851,10.2158) and (57.7962,0.5047) .. (69.7745,0.5047) .. controls (81.7528,0.5047) and (91.4639,10.2158) .. (91.4639,22.1941);



\path[draw=black,fill=black,line join=round,line width=0.512pt] (172.5900,82.3602) .. controls (173.2320,82.3602) and (173.7520,82.8804) .. (173.7520,83.5220) .. controls (173.7520,84.1636) and (173.2320,84.6837) .. (172.5900,84.6837) .. controls (171.9480,84.6837) and (171.4280,84.1636) .. (171.4280,83.5220) .. controls (171.4280,82.8804) and (171.9480,82.3602) .. (172.5900,82.3602) -- cycle;



\path[draw=black,fill=black,line join=round,line width=0.512pt] (272.1850,72.3452) .. controls (272.8270,72.3452) and (273.3470,72.8654) .. (273.3470,73.5070) .. controls (273.3470,74.1486) and (272.8270,74.6687) .. (272.1850,74.6687) .. controls (271.5430,74.6687) and (271.0230,74.1486) .. (271.0230,73.5070) .. controls (271.0230,72.8654) and (271.5430,72.3452) .. (272.1850,72.3452) -- cycle;



\path[draw=black,fill=black,line join=round,line width=0.512pt] (272.3250,12.1252) .. controls (272.9670,12.1252) and (273.4870,12.6453) .. (273.4870,13.2869) .. controls (273.4870,13.9285) and (272.9670,14.4487) .. (272.3250,14.4487) .. controls (271.6830,14.4487) and (271.1630,13.9285) .. (271.1630,13.2869) .. controls (271.1630,12.6453) and (271.6830,12.1252) .. (272.3250,12.1252) -- cycle;



\path[draw=black,fill=black,line join=round,line width=0.512pt] (20.8596,23.6827) .. controls (21.5013,23.6827) and (22.0214,24.2028) .. (22.0214,24.8444) .. controls (22.0214,25.4860) and (21.5013,26.0061) .. (20.8596,26.0061) .. controls (20.2181,26.0061) and (19.6980,25.4860) .. (19.6980,24.8444) .. controls (19.6980,24.2028) and (20.2181,23.6827) .. (20.8596,23.6827) -- cycle;



\path[draw=black,fill=black,line join=round,line width=0.512pt] (20.8363,82.8461) .. controls (21.4779,82.8461) and (21.9980,83.3662) .. (21.9980,84.0078) .. controls (21.9980,84.6494) and (21.4779,85.1696) .. (20.8363,85.1696) .. controls (20.1947,85.1696) and (19.6746,84.6494) .. (19.6746,84.0078) .. controls (19.6746,83.3662) and (20.1947,82.8461) .. (20.8363,82.8461) -- cycle;



\path[draw=black,fill=black,line join=round,line width=0.512pt] (118.6260,13.9161) .. controls (119.2680,13.9161) and (119.7880,14.4362) .. (119.7880,15.0778) .. controls (119.7880,15.7194) and (119.2680,16.2396) .. (118.6260,16.2396) .. controls (117.9850,16.2396) and (117.4640,15.7194) .. (117.4640,15.0778) .. controls (117.4640,14.4362) and (117.9850,13.9161) .. (118.6260,13.9161) -- cycle;



\path[draw=black,fill=black,line join=round,line width=0.512pt] (118.5960,73.2495) .. controls (119.2380,73.2495) and (119.7580,73.7697) .. (119.7580,74.4111) .. controls (119.7580,75.0527) and (119.2380,75.5729) .. (118.5960,75.5729) .. controls (117.9550,75.5729) and (117.4350,75.0527) .. (117.4350,74.4111) .. controls (117.4350,73.7697) and (117.9550,73.2495) .. (118.5960,73.2495) -- cycle;



\path[cm={{1.0,0.0,0.0,1.0,(1.0,24.0)}}] (0.0000,0.0000) node[above right] () {$v_1$};



\path[cm={{1.0,0.0,0.0,1.0,(0.0,88.0)}}] (0.0000,0.0000) node[above right] () {$v_4$};



\path[cm={{1.0,0.0,0.0,1.0,(125.0,11.0)}}] (0.0000,0.0000) node[above right] () {$v_2$};



\path[cm={{1.0,0.0,0.0,1.0,(127.0,78.0)}}] (0.0000,0.0000) node[above right] () {$v_3$};



\path[cm={{1.0,0.0,0.0,1.0,(142.0,54.0)}}] (0.0000,0.0000) node[above right] () {$\rightarrow$};



\path[cm={{1.0,0.0,0.0,1.0,(97.0,69.0)}}] (0.0000,0.0000) node[above right] () {$\overline{\Gamma}$};



\path[cm={{1.0,0.0,0.0,1.0,(248.0,68.0)}}] (0.0000,0.0000) node[above right] () {$\underline{\Gamma}$};




\end{tikzpicture}


  \end{minipage}\hfill
  \begin{minipage}[l]{0.26\columnwidth}
    \caption{Zamboni-li- ke motion: $\overline{\Gamma}$ with four primitive paths (Lines~\ref{alg2:circ1}--\ref{alg2:trig4} in Algorithm~\ref{alg2}) can be re-planned up to $\underline{\Gamma}$ {\color{black}with the radius} $r_2$.% in Eq.~(\ref{eq:r2}).
    }
    \label{fig:zambo}
  \end{minipage}
  \vspace*{-5ex}
\end{figure}

\subsection{Re-planning-scheduling}
\label{sec:repla-algo}

Past literature on planning-scheduling often relies on %optimal control and 
optimization {\color{black} as well as heuristics-based} approaches~\cite{brateman2006energy,zhang2007low,ondruska2015scheduled,lahijanian2018resource}. We similarly derive an optimal control problem {\color{black}and a greedy approach} returning the trajectory of parameters $c_i(\mathcal{T})$ with $\mathcal{T}:=[t_0+h,t_f]$ (see Definition~\ref{def:meas}). Since the final time %instant 
and the %exact 
value of the state $\mathbf{q}$ are not known, we use %a technique in the literature named 
output model predictive control (MPC) that derives the configuration for a finite horizon on an estimated state $\hat{\mathbf{q}}$, i.e., $t_f:=t_0+N$ for a given $N\in\mathbb{R}_{>0}$. %~\cite{rawlings2017model}. 
{\color{black} We utilize MPC to derive the trajectory of the computation parameters and the greedy approach with heuristics remaining coverage time for the path parameters.}

\begin{algorithm}[t]
  \begin{algorithmic}[1]
    \small
    \FORALL{$t\in\mathcal{T}$}
      \makeatletter
      \setcounter{ALC@line}{15}
      \makeatother
      \color{black}\STATE $\mathbf{q}(\mathcal{K}\setminus\{t+N\}),c_i^\sigma(\mathcal{K})\gets${ \vspace*{.3ex}solve NLP }$\argmax_{\mathbf{q}(k),c_i(k)}$ \vspace*{.7ex}\hspace*{1em}${l_f(\mathbf{q}(t+\hspace*{-.4ex}N),t+\hspace*{-.4ex}N)}+\hspace*{-.5ex}{\sum_{k\in\mathcal{K}}{l_d(\mathbf{q}(k),c_i(k),k)}}${\hspace*{.4ex}in\hspace*{.4ex}Eq.\hspace*{.4ex}(\ref{eq:ocp-output-mpc}) \hspace*{1em}on }$\mathcal{K}=\{t,t+h,\dots,t+N\}$\vspace*{.3ex}\label{alg:mpc}
      
      \vspace*{.8ex}
      \color{black}\STATE $k\gets t$\vspace*{.3ex}\label{alg:bat1}
      \WHILE{$b_d(y(k))>0$}\vspace*{.3ex}
        \IF{$k+h\notin\mathcal{K}$}
          \STATE $\mathbf{q}(k+h)\gets${ solve model in Eq.~(\ref{eq:state-perf-q})}\vspace*{.3ex}\label{alg:evol}
        \ENDIF
        \STATE $b_d(y(k+h))\gets${ solve model in Eq.~(\ref{eq:batdyn})}\vspace*{.3ex}
        \STATE $k\gets k+h$\vspace*{.3ex}
      \ENDWHILE
      \STATE $t_b\gets k-t$\vspace*{.3ex}\label{alg:bat2}
      \vspace*{-2.4ex}
      %\STATE $t_s\gets(\mathrm{diag}(\nu_i^\rho)\vspace*{.3ex}c_i^\rho(t)+\tau_i^\rho)[\overbrace{\begin{matrix}1&1&\cdots&1\end{matrix}}^{\rho}]$\vspace*{.3ex}\label{alg:traj1}
      %\STATE $t_r\gets(t_s/\overline{t})(\overline{t}-t)$\vspace*{.3ex}
      \color{black}\STATE $t_r\gets (\mathrm{diag}(\nu_i^\rho)\vspace*{.3ex}c_i^\rho(t-h)+\tau_i^\rho)[\overbrace{\begin{matrix}1&1&\cdots&1\end{matrix}}^{\rho}]-t$\vspace*{.3ex}\label{alg:traj1}
      \IF{$t_r>t_b$}
        \color{black}\STATE $c_i^{\rho}(t)\gets${ find }$c_i^{\rho}${ with }$t_r\in[0,t_b]${, otherwise take }$\underline{c}_i^\rho$\vspace*{.3ex}\label{alg:traj2}
      \color{black}\ENDIF
      \vspace*{.8ex}
      \color{black}\STATE $\hat{\mathbf{q}}(t+h)\gets${ estimate }$\mathbf{q}${ in Eq.~(\ref{eq:state-perf-q}) with energy sensor }$\Upsilon(t)$\vspace*{-1.8ex}\label{alg:klm1}
      \color{black}\STATE $\hat{y}(t+h)\gets${ derive }$y${ from Eq.~(\ref{eq:state-perf-y}) with est. state }$\hat{\mathbf{q}}(t+h)$%\vspace*{.3ex}
      \label{alg:klm2}
    \ENDFOR
  \end{algorithmic}
  \caption{Coverage re-planning-scheduling}\label{alg}
\end{algorithm}

An optimal control problem (OCP) that selects %the %highest configuration of 
{\color{black} $c_i^\sigma$} %and respects the constraints
\begin{subequations}\small\label{eq:ocp-output-mpc}\begin{align}
  \max_{\mathbf{q}(t),c_i(t)}&{l_f(\mathbf{q}(t_f),t_f)+\int_{t_0}^{t_f}{l(\mathbf{q}(t),c_i(t),t)\,dt}},\label{eq:ocp-costs}\\
  \text{s.t. }\dot{\mathbf{q}}&=f(\mathbf{q}(t),c_i(t),t),\label{eq:dyn-evol}\\
  \mathbf{q}(t)&\hspace*{-.2ex}\in\hspace*{-.2ex}\mathbb{R}^m,\,y(t)\hspace*{-.2ex}\in\hspace*{-.2ex}\mathcal{Y}(t)\text{, i.e., constraint in Eq.~(\ref{eq:output-const})},\label{eq:batt-const-mpc}\\
  c_{i,j}(t)&\hspace*{-.2ex}\in\hspace*{-.2ex}\mathcal{C}_{i,j},\,c_{i,\rho+k}(t)\hspace*{-.2ex}\in\hspace*{-.2ex}\mathcal{S}_{i,k}\,\forall j\hspace*{-.2ex}\in\hspace*{-.2ex}[\rho]_{>0},\,k\hspace*{-.2ex}\in\hspace*{-.2ex}[\sigma]_{>0},\label{eq:state-cont-const-mpc}\\
  \mathbf{q}(t_0)&=\hat{\mathbf{q}}_0\,\,\,\text{given (last estimated state)},\text{ and}\label{eq:ocp-outp-mpc-state-est}\\
  b(t_0)&=b_0\,\,\,\text{given}\label{eq:ocp-outp-bat},
\end{align}\end{subequations}
where $\mathbf{q}(t)$ and $c_i(t)$ are the state and parameters trajectories and $l:\mathbb{R}^m\times\mathcal{C}_i\times\mathcal{S}_i\times\mathbb{R}_{\geq 0}\rightarrow\mathbb{R}$ is a given initial cost function with the quadratic expression
\vspace*{-.5ex}
\begin{equation}\label{eq:insta-cost-mpc}
  l(\mathbf{q}(t),c_i(t),t)=\mathbf{q}'(t)Q\mathbf{q}(t)+c_i'(t)Rc_i(t),
  \vspace*{-.5ex}
\end{equation}
where $Q\in\mathbb{R}^{m\times m},R\in\mathbb{R}^{n\times n}$ are given positive semidefinite matrices. %, resulting in the convexity of the cost $l$~\citep{nocedal2006numerical}. A useful property to guarantee a solution~\citep{beck2014introduction}.
The final cost function $l_f:\mathbb{R}^m\times\mathbb{R}_{> 0}\rightarrow\mathbb{R}$ is also a quadratic expression %but with no control %~\cite{rawlings2017model}
\vspace*{-.5ex}
\begin{equation}\label{eq:final-cost-mpc}
  l_f(\mathbf{q}(T),T)=\mathbf{q}'(T)Q_f\mathbf{q}(T), 
  \vspace*{-.5ex}
\end{equation}
where $Q_f\in\mathbb{R}^{m\times m}$ is a given positive semidefinite matrix.

Eq.~(\ref{eq:dyn-evol}) is the model in Eq.~(\ref{eq:state-perf}). {\color{black}It} requires a value of the period $T$, which is %simply 
the time needed to fly the four primitive paths in the Zamboni-like motion, i.e., the time %elapsed
between two positive evaluations of the condition on Line~\ref{alg2:cond}.

Eq.~(\ref{eq:state-cont-const-mpc}) are the parameters constraints sets in Definition~\ref{def:stage}. Eq.~(\ref{eq:batt-const-mpc}) are the state and output constraints %in Eq.~(\ref{eq:output-const}) 
that evolve the battery model in Eq.~(\ref{eq:batdyn}). Eq.~(\ref{eq:ocp-outp-mpc-state-est}) is the state guess estimated via state estimation % (e.g., linear Kalman filter)
(%the very 
first estimate is given). Eq.~(\ref{eq:ocp-outp-bat}) is the initial battery SoC from, e.g., flight controller.

Line~\ref{alg:mpc} in Algorithm~\ref{alg} contains a transcribed version of the OCP in Eq.~(\ref{eq:ocp-output-mpc}) into a nonlinear program (NLP) that can be %easily 
solved with available NLP solvers%~\cite{rawlings2017model}
. Its solution leads to both trajectories of {\color{black} computation} parameters and states for future $N$ instants. Here, the sets $\mathcal{K},\mathcal{T}$ have possibly different steps $h$ (not to be confused with the altitude){\color{black}: 
the set $\mathcal{K}$ is used for the numerical simulation, whereas $\mathcal{T}$ is for re-planning, meaning that $h$ tunes the precision and the frequency of re-planning for $\mathcal{K}$ and $\mathcal{T}$ respectively.}
The functions $l_d,b_d$ are the discretized versions of Eq.~(\ref{eq:insta-cost-mpc})~and~(\ref{eq:batdyn}).%, with, e.g., Runge-Kutta or Euler methods%~\cite{iserles2009first}.

Lines~\ref{alg:bat1}--\ref{alg:bat2} estimate the time needed to completely drain the battery, exploiting the SoC already predicted previously on Line~\ref{alg:mpc}. The {\color{black}path parameters and thus the} coverage is then re{\color{black}-}planned %accordingly 
on Lines~\ref{alg:traj1}--\ref{alg:traj2} using {\color{black} the heuristics with the} %Lem.~\ref{lem:new} and 
scaling factors from Eq.~(\ref{eq:scale-traj}) with $c_{i}^{\rho}(t_0)$ given. 
{\color{black} Concretely, these lines implement the greedy approach by decreasing the path parameters of a given value $\delta_i$ or similarly increasing the parameters when $t_r\leq t_b$ within the bounds (this latter analogous case is not shown explicitly in Algorithm~\ref{alg2} but implemented in Sec.~\ref{sec:experimental})}.
Lines~\ref{alg:klm1}--\ref{alg:klm2} estimate the %energy model's 
state with %current 
energy sensor reading $\Upsilon$, {\color{black}using}, e.g., %linear 
Kalman filter%~\cite{kalman1960new}
.

Algorithm~\ref{alg} implements Eq.~(\ref{eq:ocp-output-mpc}) for the purpose of energy-aware re-planning-scheduling of $\Gamma$ from Algorithm~\ref{alg2}, i.e, Lines~\ref{alg:mpc}--\ref{alg:klm2} continue after Line~\ref{alg2:last} in Algorithm~\ref{alg2}.

%The main purpose of the algorithm is to output a valid control sequence $\mathbf{u}:=\{\mathbf{u}_0,\mathbf{u}_1,\dots\}$ at each time step given an initial plan $\Gamma$ and to guide the UAV on the path resulting from such sequence--to solve Problem~\ref{pb}.

%A valid control sequence has to respect the energy constraints. We consider some realistic constraints to the energy of a flying UAV in the following subsection.

%\subsection{Output and control constraint sets}

%We stated earlier the output $y$--the instantaneous energy consumption--evolves in $\mathbb{R}$. This is generally untrue. Physical UAVs are bounded by strict energy budgets due to battery limitations.

%Let us hence consider the state of charge (SoC) $b$ of a UAV battery with a simplistic difference equation~\cite{seewald2020mechanical}
%\begin{equation}\label{eq:bat}
%  b_k=b_{k-1}-k_b\left(V-
%  \sqrt{
%    V^2-
%    4R_ry_k}
%  \right)/(2R_rQ_c),
%\end{equation}
%where $k_b$ is the battery coefficient determined experimentally,  $V\in\mathbb{R}$ is the internal battery 
% and $\tilde{V}\in\mathbb{R}$ the stabilized 
%voltage measured in volts, $R_r\in\mathbb{R}$ the resistance measured in ohms, and $Q_c\in\mathbb{R}$ the constant nominal capacity measured in amperes per hour. 

%\begin{defn}[Output, control constrain sets]\label{def:const}
%The output constrain set is then the set
%\begin{equation*}
%  \mathcal{Y}_k:=\{y_k\mid y_k\in[0,b_kQ_cV]\subseteq{\mathbb{R}_{\geq 0}}\},
%\end{equation*}
%and $b_kQ_cV$ is the maximum instantaneous energy consumption.

%The control constraint set is the path constraint set for the path parameters and computation constraint sets for the computation parameters (Definition~\ref{def:mission})
%\begin{equation*}
%  \mathcal{U}_k:=\begin{cases}
%    \mathcal{C}_i & \text{for } c_{i,j} \text{ with } j\leq\rho\\
%    \mathcal{S}_{i,j-\rho} & \text{for } c_{i,j} \text{ with } \rho<j\leq\sigma
%  \end{cases}.
%\end{equation*}
%\end{defn}

%\subsection{Deployment algorithm}

%The algorithm first initializes the position, energy coefficients, and control (line~\ref{alg:init}). It updates the position at line~\ref{alg:pos}, using the expression from Eq.~(\ref{eq:pd}) and the velocity $v\in\mathbb{R}_{\geq 0}$. The expression depends on the path $\varphi_i$ from stage $\Gamma_i$. The algorithm iterates all the stages in the plan $\Gamma$ (line~\ref{alg:stages-loop}), and enters the next stage $\Gamma_{i+1}$ when the UAV reaches the triggering point $\mathbf{p}_{\Gamma_i}$ (FSM in Definition~\ref{def:mission}).

%The energy coefficients are updated at line~\ref{alg:evolution}, using the expression from Eq.~(\ref{eq:state-perf}). A priori state estimate $\hat{\mathbf{q}}_k$ is refined using a state estimator--such as Kalman filter (KF)~\cite{simon2006optimal}--and the data from an energy sensor (line~\ref{alg:kalman_end}).
%At line~\ref{alg:mpc}, the algorithm uses MPC to select the control $\mathbf{u}_k$ for a given horizon $N\in\mathbb{Z}_{>0}$ from the cost function (higher the horizon, higher the complexity and the robustness of the control to the output constraints)
%\begin{equation*}\label{eq:cost}\begin{split}
%  V_f(\mathbf{q}_k)&=l(\mathbf{q}_k,\mathbf{u}_k)=(1/2)\mathbf{q}_k^T\mathrm{diag}(C)\mathbf{q}_k,
%\end{split}\end{equation*}
%where $\mathrm{diag}(C)$ is a diagonal matrix with the items of $C$ from Eq.~(\ref{eq:state-details}).

%We note that at every step of the sum on line~\ref{alg:mpc}, the algorithm evolves the state to check if the output satisfies the output constraint set, and if the control satisfies the control constraint set. In particular, it performs a subroutine
%\begin{algorithmic}
%  \WHILE{$\overline{c}_i\notin\mathcal{U}_k,\, y_k\notin\mathcal{Y}_k$}\label{alg2:while}
%    \STATE $\overline{c}_i\gets\overline{c}_i-\delta$
%  \ENDWHILE
%  \STATE $\mathbf{u}_k\gets\overline{c}_i$
%\end{algorithmic} 
%where $\delta\in\mathbb{R}^\rho\times\mathbb{Z}_{\geq 0}^\sigma$ are reduction steps. Both conditions of the loop have to be respected. The maximum values of path and computation parameters are reduced by the steps if they don't meet the constraints.

%We finally note that one can express the tradeoffs between parameters (e.g., a decrement in the distance between the lines in a survey scenario is related to a decrement in the number of detections per second) enriching the control constraint set with the constraints
%\begin{equation*}
%  R_i\mathbf{u}_k-r_i\geq 0,
%\end{equation*}
%where $r_i\in\mathbb{R}^n$ and $R_i\in\mathbb{R}^{n\times n}$ expresses the relation between the parameters (if $R_i$ is the identity matrix, there is no relation between the parameters).

%\begin{figure*}
%  \centering
%  \vspace*{-4ex}
%  \begin{subfigure}{.4\textwidth}
%    \centering
%    \footnotesize
%    
\definecolor{cebebeb}{RGB}{235,235,235}
\definecolor{ca0a0a4}{RGB}{160,160,164}
\definecolor{cffffff}{RGB}{255,255,255}
\definecolor{cd9d9d9}{RGB}{217,217,217}


\def \globalscale {1.100000}
\begin{tikzpicture}[y=0.80pt, x=0.80pt, yscale=-\globalscale, xscale=.915*\globalscale, inner sep=0pt, outer sep=0pt]
\begin{scope}[shift={(-44.46032,-13.27762)},draw=black,line join=bevel,line cap=rect,even odd rule,line width=0.800pt]
  \begin{scope}[draw=black,line join=bevel,line cap=rect,line width=0.800pt]
  \end{scope}
  \begin{scope}[scale=1.006,draw=black,line join=bevel,line cap=rect,line width=0.800pt]
  \end{scope}
  \begin{scope}[scale=1.006,draw=black,line join=bevel,line cap=rect,line width=0.800pt]
  \end{scope}
  \begin{scope}[cm={{1.00588,0.0,0.0,1.00588,(18.1059,106.624)}},draw=black,line join=bevel,line cap=rect,line width=0.800pt]
  \end{scope}
  \begin{scope}[cm={{1.00588,0.0,0.0,1.00588,(18.1059,106.624)}},draw=black,line join=bevel,line cap=rect,line width=0.800pt]
  \end{scope}
  \begin{scope}[cm={{1.00588,0.0,0.0,1.00588,(18.1059,106.624)}},draw=black,line join=bevel,line cap=rect,line width=0.800pt]
  \end{scope}
  \begin{scope}[cm={{1.00588,0.0,0.0,1.00588,(18.1059,106.624)}},draw=black,line join=bevel,line cap=rect,line width=0.800pt]
  \end{scope}
  \begin{scope}[cm={{1.00588,0.0,0.0,1.00588,(18.1059,106.624)}},draw=black,line join=bevel,line cap=rect,line width=0.800pt]
  \end{scope}
  \begin{scope}[cm={{1.00588,0.0,0.0,1.00588,(18.1059,106.624)}},draw=black,line join=bevel,line cap=rect,line width=0.800pt]
  \end{scope}
  \begin{scope}[scale=1.006,draw=black,line join=bevel,line cap=rect,line width=0.800pt]
  \end{scope}
  \begin{scope}[scale=1.006,draw=black,line join=bevel,line cap=rect,line width=0.800pt]
  \end{scope}
  \begin{scope}[cm={{1.00588,0.0,0.0,1.00588,(34.2,84.4941)}},draw=black,line join=bevel,line cap=rect,line width=0.800pt]
  \end{scope}
  \begin{scope}[cm={{1.00588,0.0,0.0,1.00588,(34.2,84.4941)}},draw=black,line join=bevel,line cap=rect,line width=0.800pt]
  \end{scope}
  \begin{scope}[cm={{1.00588,0.0,0.0,1.00588,(34.2,84.4941)}},draw=black,line join=bevel,line cap=rect,line width=0.800pt]
  \end{scope}
  \begin{scope}[cm={{1.00588,0.0,0.0,1.00588,(34.2,84.4941)}},draw=black,line join=bevel,line cap=rect,line width=0.800pt]
  \end{scope}
  \begin{scope}[cm={{1.00588,0.0,0.0,1.00588,(34.2,84.4941)}},draw=black,line join=bevel,line cap=rect,line width=0.800pt]
  \end{scope}
  \begin{scope}[cm={{1.00588,0.0,0.0,1.00588,(34.2,84.4941)}},draw=black,line join=bevel,line cap=rect,line width=0.800pt]
  \end{scope}
  \begin{scope}[scale=1.006,draw=black,line join=bevel,line cap=rect,line width=0.800pt]
  \end{scope}
  \begin{scope}[scale=1.006,draw=black,line join=bevel,line cap=rect,line width=0.800pt]
  \end{scope}
  \begin{scope}[cm={{1.00588,0.0,0.0,1.00588,(22.1294,62.3647)}},draw=black,line join=bevel,line cap=rect,line width=0.800pt]
  \end{scope}
  \begin{scope}[cm={{1.00588,0.0,0.0,1.00588,(22.1294,62.3647)}},draw=black,line join=bevel,line cap=rect,line width=0.800pt]
  \end{scope}
  \begin{scope}[cm={{1.00588,0.0,0.0,1.00588,(22.1294,62.3647)}},draw=black,line join=bevel,line cap=rect,line width=0.800pt]
  \end{scope}
  \begin{scope}[cm={{1.00588,0.0,0.0,1.00588,(22.1294,62.3647)}},draw=black,line join=bevel,line cap=rect,line width=0.800pt]
  \end{scope}
  \begin{scope}[cm={{1.00588,0.0,0.0,1.00588,(22.1294,62.3647)}},draw=black,line join=bevel,line cap=rect,line width=0.800pt]
  \end{scope}
  \begin{scope}[cm={{1.00588,0.0,0.0,1.00588,(22.1294,62.3647)}},draw=black,line join=bevel,line cap=rect,line width=0.800pt]
  \end{scope}
  \begin{scope}[scale=1.006,draw=black,line join=bevel,line cap=rect,line width=0.800pt]
  \end{scope}
  \begin{scope}[scale=1.006,draw=black,line join=bevel,line cap=rect,line width=0.800pt]
  \end{scope}
  \begin{scope}[cm={{1.00588,0.0,0.0,1.00588,(22.1294,40.2353)}},draw=black,line join=bevel,line cap=rect,line width=0.800pt]
  \end{scope}
  \begin{scope}[cm={{1.00588,0.0,0.0,1.00588,(22.1294,40.2353)}},draw=black,line join=bevel,line cap=rect,line width=0.800pt]
  \end{scope}
  \begin{scope}[cm={{1.00588,0.0,0.0,1.00588,(22.1294,40.2353)}},draw=black,line join=bevel,line cap=rect,line width=0.800pt]
  \end{scope}
  \begin{scope}[cm={{1.00588,0.0,0.0,1.00588,(22.1294,40.2353)}},draw=black,line join=bevel,line cap=rect,line width=0.800pt]
  \end{scope}
  \begin{scope}[cm={{1.00588,0.0,0.0,1.00588,(22.1294,40.2353)}},draw=black,line join=bevel,line cap=rect,line width=0.800pt]
  \end{scope}
  \begin{scope}[cm={{1.00588,0.0,0.0,1.00588,(22.1294,40.2353)}},draw=black,line join=bevel,line cap=rect,line width=0.800pt]
  \end{scope}
  \begin{scope}[scale=1.006,draw=black,line join=bevel,line cap=rect,line width=0.800pt]
  \end{scope}
  \begin{scope}[scale=1.006,draw=black,line join=bevel,line cap=rect,line width=0.800pt]
  \end{scope}
  \begin{scope}[cm={{1.00588,0.0,0.0,1.00588,(22.1294,18.1059)}},draw=black,line join=bevel,line cap=rect,line width=0.800pt]
  \end{scope}
  \begin{scope}[cm={{1.00588,0.0,0.0,1.00588,(22.1294,18.1059)}},draw=black,line join=bevel,line cap=rect,line width=0.800pt]
  \end{scope}
  \begin{scope}[cm={{1.00588,0.0,0.0,1.00588,(22.1294,18.1059)}},draw=black,line join=bevel,line cap=rect,line width=0.800pt]
  \end{scope}
  \begin{scope}[cm={{1.00588,0.0,0.0,1.00588,(22.1294,18.1059)}},draw=black,line join=bevel,line cap=rect,line width=0.800pt]
  \end{scope}
  \begin{scope}[cm={{1.00588,0.0,0.0,1.00588,(22.1294,18.1059)}},draw=black,line join=bevel,line cap=rect,line width=0.800pt]
  \end{scope}
  \begin{scope}[cm={{1.00588,0.0,0.0,1.00588,(22.1294,18.1059)}},draw=black,line join=bevel,line cap=rect,line width=0.800pt]
  \end{scope}
  \begin{scope}[scale=1.006,draw=black,line join=bevel,line cap=rect,line width=0.800pt]
  \end{scope}
  \begin{scope}[scale=1.006,draw=black,line join=bevel,line cap=rect,line width=0.800pt]
  \end{scope}
  \begin{scope}[cm={{1.00588,0.0,0.0,1.00588,(45.2647,118.694)}},draw=black,line join=bevel,line cap=rect,line width=0.800pt]
  \end{scope}
  \begin{scope}[cm={{1.00588,0.0,0.0,1.00588,(45.2647,118.694)}},draw=black,line join=bevel,line cap=rect,line width=0.800pt]
  \end{scope}
  \begin{scope}[cm={{1.00588,0.0,0.0,1.00588,(45.2647,118.694)}},draw=black,line join=bevel,line cap=rect,line width=0.800pt]
  \end{scope}
  \begin{scope}[cm={{1.00588,0.0,0.0,1.00588,(45.2647,118.694)}},draw=black,line join=bevel,line cap=rect,line width=0.800pt]
  \end{scope}
  \begin{scope}[cm={{1.00588,0.0,0.0,1.00588,(45.2647,118.694)}},draw=black,line join=bevel,line cap=rect,line width=0.800pt]
  \end{scope}
  \begin{scope}[cm={{1.00588,0.0,0.0,1.00588,(45.2647,118.694)}},draw=black,line join=bevel,line cap=rect,line width=0.800pt]
  \end{scope}
  \begin{scope}[scale=1.006,draw=black,line join=bevel,line cap=rect,line width=0.800pt]
  \end{scope}
  \begin{scope}[scale=1.006,draw=black,line join=bevel,line cap=rect,line width=0.800pt]
  \end{scope}
  \begin{scope}[cm={{1.00588,0.0,0.0,1.00588,(69.4059,118.694)}},draw=black,line join=bevel,line cap=rect,line width=0.800pt]
  \end{scope}
  \begin{scope}[cm={{1.00588,0.0,0.0,1.00588,(69.4059,118.694)}},draw=black,line join=bevel,line cap=rect,line width=0.800pt]
  \end{scope}
  \begin{scope}[cm={{1.00588,0.0,0.0,1.00588,(69.4059,118.694)}},draw=black,line join=bevel,line cap=rect,line width=0.800pt]
  \end{scope}
  \begin{scope}[cm={{1.00588,0.0,0.0,1.00588,(69.4059,118.694)}},draw=black,line join=bevel,line cap=rect,line width=0.800pt]
  \end{scope}
  \begin{scope}[cm={{1.00588,0.0,0.0,1.00588,(69.4059,118.694)}},draw=black,line join=bevel,line cap=rect,line width=0.800pt]
  \end{scope}
  \begin{scope}[cm={{1.00588,0.0,0.0,1.00588,(69.4059,118.694)}},draw=black,line join=bevel,line cap=rect,line width=0.800pt]
  \end{scope}
  \begin{scope}[scale=1.006,draw=black,line join=bevel,line cap=rect,line width=0.800pt]
  \end{scope}
  \begin{scope}[scale=1.006,draw=black,line join=bevel,line cap=rect,line width=0.800pt]
  \end{scope}
  \begin{scope}[cm={{1.00588,0.0,0.0,1.00588,(94.5529,118.694)}},draw=black,line join=bevel,line cap=rect,line width=0.800pt]
  \end{scope}
  \begin{scope}[cm={{1.00588,0.0,0.0,1.00588,(94.5529,118.694)}},draw=black,line join=bevel,line cap=rect,line width=0.800pt]
  \end{scope}
  \begin{scope}[cm={{1.00588,0.0,0.0,1.00588,(94.5529,118.694)}},draw=black,line join=bevel,line cap=rect,line width=0.800pt]
  \end{scope}
  \begin{scope}[cm={{1.00588,0.0,0.0,1.00588,(94.5529,118.694)}},draw=black,line join=bevel,line cap=rect,line width=0.800pt]
  \end{scope}
  \begin{scope}[cm={{1.00588,0.0,0.0,1.00588,(94.5529,118.694)}},draw=black,line join=bevel,line cap=rect,line width=0.800pt]
  \end{scope}
  \begin{scope}[cm={{1.00588,0.0,0.0,1.00588,(94.5529,118.694)}},draw=black,line join=bevel,line cap=rect,line width=0.800pt]
  \end{scope}
  \begin{scope}[scale=1.006,draw=black,line join=bevel,line cap=rect,line width=0.800pt]
  \end{scope}
  \begin{scope}[scale=1.006,draw=black,line join=bevel,line cap=rect,line width=0.800pt]
  \end{scope}
  \begin{scope}[cm={{1.00588,0.0,0.0,1.00588,(118.694,118.694)}},draw=black,line join=bevel,line cap=rect,line width=0.800pt]
  \end{scope}
  \begin{scope}[cm={{1.00588,0.0,0.0,1.00588,(118.694,118.694)}},draw=black,line join=bevel,line cap=rect,line width=0.800pt]
  \end{scope}
  \begin{scope}[cm={{1.00588,0.0,0.0,1.00588,(118.694,118.694)}},draw=black,line join=bevel,line cap=rect,line width=0.800pt]
  \end{scope}
  \begin{scope}[cm={{1.00588,0.0,0.0,1.00588,(118.694,118.694)}},draw=black,line join=bevel,line cap=rect,line width=0.800pt]
  \end{scope}
  \begin{scope}[cm={{1.00588,0.0,0.0,1.00588,(118.694,118.694)}},draw=black,line join=bevel,line cap=rect,line width=0.800pt]
  \end{scope}
  \begin{scope}[cm={{1.00588,0.0,0.0,1.00588,(118.694,118.694)}},draw=black,line join=bevel,line cap=rect,line width=0.800pt]
  \end{scope}
  \begin{scope}[scale=1.006,draw=black,line join=bevel,line cap=rect,line width=0.800pt]
  \end{scope}
  \begin{scope}[scale=1.006,draw=black,line join=bevel,line cap=rect,line width=0.800pt]
  \end{scope}
  \begin{scope}[cm={{1.00588,0.0,0.0,1.00588,(143.841,118.694)}},draw=black,line join=bevel,line cap=rect,line width=0.800pt]
  \end{scope}
  \begin{scope}[cm={{1.00588,0.0,0.0,1.00588,(143.841,118.694)}},draw=black,line join=bevel,line cap=rect,line width=0.800pt]
  \end{scope}
  \begin{scope}[cm={{1.00588,0.0,0.0,1.00588,(143.841,118.694)}},draw=black,line join=bevel,line cap=rect,line width=0.800pt]
  \end{scope}
  \begin{scope}[cm={{1.00588,0.0,0.0,1.00588,(143.841,118.694)}},draw=black,line join=bevel,line cap=rect,line width=0.800pt]
  \end{scope}
  \begin{scope}[cm={{1.00588,0.0,0.0,1.00588,(143.841,118.694)}},draw=black,line join=bevel,line cap=rect,line width=0.800pt]
  \end{scope}
  \begin{scope}[cm={{1.00588,0.0,0.0,1.00588,(143.841,118.694)}},draw=black,line join=bevel,line cap=rect,line width=0.800pt]
  \end{scope}
  \begin{scope}[scale=1.006,draw=black,line join=bevel,line cap=rect,line width=0.800pt]
  \end{scope}
  \begin{scope}[scale=1.006,draw=black,line join=bevel,line cap=rect,line width=0.800pt]
  \end{scope}
  \begin{scope}[scale=1.006,draw=black,line join=bevel,line cap=rect,line width=0.800pt]
  \end{scope}
  \begin{scope}[scale=1.006,draw=black,line join=bevel,line cap=rect,line width=0.800pt]
  \end{scope}
  \begin{scope}[scale=1.006,draw=black,line join=bevel,line cap=rect,line width=0.800pt]
  \end{scope}
  \begin{scope}[scale=1.006,draw=black,line join=bevel,line cap=rect,line width=0.800pt]
  \end{scope}
  \begin{scope}[cm={{1.00588,0.0,0.0,1.00588,(126.741,30.1765)}},draw=black,line join=bevel,line cap=rect,line width=0.800pt]
  \end{scope}
  \begin{scope}[cm={{1.00588,0.0,0.0,1.00588,(126.741,30.1765)}},draw=black,line join=bevel,line cap=rect,line width=0.800pt]
  \end{scope}
  \begin{scope}[cm={{1.00588,0.0,0.0,1.00588,(126.741,30.1765)}},draw=black,line join=bevel,line cap=rect,line width=0.800pt]
  \end{scope}
  \begin{scope}[cm={{1.00588,0.0,0.0,1.00588,(126.741,30.1765)}},draw=black,line join=bevel,line cap=rect,line width=0.800pt]
  \end{scope}
  \begin{scope}[cm={{1.00588,0.0,0.0,1.00588,(126.741,30.1765)}},draw=black,line join=bevel,line cap=rect,line width=0.800pt]
  \end{scope}
  \begin{scope}[cm={{1.00588,0.0,0.0,1.00588,(126.741,30.1765)}},draw=black,line join=bevel,line cap=rect,line width=0.800pt]
  \end{scope}
  \begin{scope}[cm={{0.0,-1.00588,1.00588,0.0,(14.0824,184.076)}},draw=black,line join=bevel,line cap=rect,line width=0.800pt]
  \end{scope}
  \begin{scope}[cm={{0.0,-1.00588,1.00588,0.0,(14.0824,184.076)}},draw=black,line join=bevel,line cap=rect,line width=0.800pt]
  \end{scope}
  \begin{scope}[cm={{0.0,-1.00588,1.00588,0.0,(14.0824,184.076)}},draw=black,line join=bevel,line cap=rect,line width=0.800pt]
  \end{scope}
  \begin{scope}[cm={{0.0,-1.00588,1.00588,0.0,(14.0824,184.076)}},draw=black,line join=bevel,line cap=rect,line width=0.800pt]
  \end{scope}
  \begin{scope}[cm={{0.0,-1.00588,1.00588,0.0,(14.0824,184.076)}},draw=black,line join=bevel,line cap=rect,line width=0.800pt]
  \end{scope}
  \begin{scope}[cm={{0.0,-1.00588,1.00588,0.0,(14.0824,184.076)}},draw=black,line join=bevel,line cap=rect,line width=0.800pt]
  \end{scope}
  \begin{scope}[cm={{1.00588,0.0,0.0,1.00588,(50.2941,27.1588)}},draw=black,line join=bevel,line cap=rect,line width=0.800pt]
  \end{scope}
  \begin{scope}[cm={{1.00588,0.0,0.0,1.00588,(50.2941,27.1588)}},draw=black,line join=bevel,line cap=rect,line width=0.800pt]
  \end{scope}
  \begin{scope}[cm={{1.00588,0.0,0.0,1.00588,(50.2941,27.1588)}},draw=black,line join=bevel,line cap=rect,line width=0.800pt]
  \end{scope}
  \begin{scope}[cm={{1.00588,0.0,0.0,1.00588,(50.2941,27.1588)}},draw=black,line join=bevel,line cap=rect,line width=0.800pt]
  \end{scope}
  \begin{scope}[cm={{1.00588,0.0,0.0,1.00588,(50.2941,27.1588)}},draw=black,line join=bevel,line cap=rect,line width=0.800pt]
  \end{scope}
  \begin{scope}[cm={{1.00588,0.0,0.0,1.00588,(50.2941,27.1588)}},draw=black,line join=bevel,line cap=rect,line width=0.800pt]
  \end{scope}
  \begin{scope}[scale=1.006,draw=black,line join=bevel,line cap=rect,line width=0.800pt]
  \end{scope}
  \begin{scope}[scale=1.006,draw=black,line join=bevel,line cap=rect,line width=0.800pt]
  \end{scope}
  \begin{scope}[scale=1.006,draw=black,line join=bevel,line cap=rect,line width=0.800pt]
  \end{scope}
  \begin{scope}[scale=1.006,draw=black,line join=bevel,line cap=rect,line width=0.800pt]
  \end{scope}
  \begin{scope}[scale=1.006,draw=black,line join=bevel,line cap=rect,line width=0.800pt]
  \end{scope}
  \begin{scope}[scale=1.006,draw=black,line join=bevel,line cap=rect,line width=0.800pt]
  \end{scope}
  \begin{scope}[scale=1.006,draw=black,line join=bevel,line cap=rect,line width=0.800pt]
  \end{scope}
  \begin{scope}[scale=1.006,draw=black,line join=bevel,line cap=rect,line width=0.800pt]
  \end{scope}
  \begin{scope}[cm={{1.00588,0.0,0.0,1.00588,(143.841,102.6)}},draw=black,line join=bevel,line cap=rect,line width=0.800pt]
  \end{scope}
  \begin{scope}[cm={{1.00588,0.0,0.0,1.00588,(143.841,102.6)}},draw=black,line join=bevel,line cap=rect,line width=0.800pt]
  \end{scope}
  \begin{scope}[cm={{1.00588,0.0,0.0,1.00588,(143.841,102.6)}},draw=black,line join=bevel,line cap=rect,line width=0.800pt]
  \end{scope}
  \begin{scope}[cm={{1.00588,0.0,0.0,1.00588,(143.841,102.6)}},draw=black,line join=bevel,line cap=rect,line width=0.800pt]
  \end{scope}
  \begin{scope}[cm={{1.00588,0.0,0.0,1.00588,(143.841,102.6)}},draw=black,line join=bevel,line cap=rect,line width=0.800pt]
  \end{scope}
  \begin{scope}[cm={{1.00588,0.0,0.0,1.00588,(143.841,102.6)}},draw=black,line join=bevel,line cap=rect,line width=0.800pt]
  \end{scope}
  \begin{scope}[scale=1.006,draw=black,line join=bevel,line cap=rect,line width=0.800pt]
  \end{scope}
  \begin{scope}[scale=1.006,draw=black,line join=bevel,line cap=rect,line width=0.800pt]
  \end{scope}
  \begin{scope}[cm={{1.00588,0.0,0.0,1.00588,(143.841,80.4706)}},draw=black,line join=bevel,line cap=rect,line width=0.800pt]
  \end{scope}
  \begin{scope}[cm={{1.00588,0.0,0.0,1.00588,(143.841,80.4706)}},draw=black,line join=bevel,line cap=rect,line width=0.800pt]
  \end{scope}
  \begin{scope}[cm={{1.00588,0.0,0.0,1.00588,(143.841,80.4706)}},draw=black,line join=bevel,line cap=rect,line width=0.800pt]
  \end{scope}
  \begin{scope}[cm={{1.00588,0.0,0.0,1.00588,(143.841,80.4706)}},draw=black,line join=bevel,line cap=rect,line width=0.800pt]
  \end{scope}
  \begin{scope}[cm={{1.00588,0.0,0.0,1.00588,(143.841,80.4706)}},draw=black,line join=bevel,line cap=rect,line width=0.800pt]
  \end{scope}
  \begin{scope}[cm={{1.00588,0.0,0.0,1.00588,(143.841,80.4706)}},draw=black,line join=bevel,line cap=rect,line width=0.800pt]
  \end{scope}
  \begin{scope}[scale=1.006,draw=black,line join=bevel,line cap=rect,line width=0.800pt]
  \end{scope}
  \begin{scope}[scale=1.006,draw=black,line join=bevel,line cap=rect,line width=0.800pt]
  \end{scope}
  \begin{scope}[cm={{1.00588,0.0,0.0,1.00588,(143.841,58.3412)}},draw=black,line join=bevel,line cap=rect,line width=0.800pt]
  \end{scope}
  \begin{scope}[cm={{1.00588,0.0,0.0,1.00588,(143.841,58.3412)}},draw=black,line join=bevel,line cap=rect,line width=0.800pt]
  \end{scope}
  \begin{scope}[cm={{1.00588,0.0,0.0,1.00588,(143.841,58.3412)}},draw=black,line join=bevel,line cap=rect,line width=0.800pt]
  \end{scope}
  \begin{scope}[cm={{1.00588,0.0,0.0,1.00588,(143.841,58.3412)}},draw=black,line join=bevel,line cap=rect,line width=0.800pt]
  \end{scope}
  \begin{scope}[cm={{1.00588,0.0,0.0,1.00588,(143.841,58.3412)}},draw=black,line join=bevel,line cap=rect,line width=0.800pt]
  \end{scope}
  \begin{scope}[cm={{1.00588,0.0,0.0,1.00588,(143.841,58.3412)}},draw=black,line join=bevel,line cap=rect,line width=0.800pt]
  \end{scope}
  \begin{scope}[scale=1.006,draw=black,line join=bevel,line cap=rect,line width=0.800pt]
  \end{scope}
  \begin{scope}[scale=1.006,draw=black,line join=bevel,line cap=rect,line width=0.800pt]
  \end{scope}
  \begin{scope}[cm={{1.00588,0.0,0.0,1.00588,(143.841,36.2118)}},draw=black,line join=bevel,line cap=rect,line width=0.800pt]
  \end{scope}
  \begin{scope}[cm={{1.00588,0.0,0.0,1.00588,(143.841,36.2118)}},draw=black,line join=bevel,line cap=rect,line width=0.800pt]
  \end{scope}
  \begin{scope}[cm={{1.00588,0.0,0.0,1.00588,(143.841,36.2118)}},draw=black,line join=bevel,line cap=rect,line width=0.800pt]
  \end{scope}
  \begin{scope}[cm={{1.00588,0.0,0.0,1.00588,(143.841,36.2118)}},draw=black,line join=bevel,line cap=rect,line width=0.800pt]
  \end{scope}
  \begin{scope}[cm={{1.00588,0.0,0.0,1.00588,(143.841,36.2118)}},draw=black,line join=bevel,line cap=rect,line width=0.800pt]
  \end{scope}
  \begin{scope}[cm={{1.00588,0.0,0.0,1.00588,(143.841,36.2118)}},draw=black,line join=bevel,line cap=rect,line width=0.800pt]
  \end{scope}
  \begin{scope}[scale=1.006,draw=black,line join=bevel,line cap=rect,line width=0.800pt]
  \end{scope}
  \begin{scope}[scale=1.006,draw=black,line join=bevel,line cap=rect,line width=0.800pt]
  \end{scope}
  \begin{scope}[cm={{1.00588,0.0,0.0,1.00588,(143.841,14.0824)}},draw=black,line join=bevel,line cap=rect,line width=0.800pt]
  \end{scope}
  \begin{scope}[cm={{1.00588,0.0,0.0,1.00588,(143.841,14.0824)}},draw=black,line join=bevel,line cap=rect,line width=0.800pt]
  \end{scope}
  \begin{scope}[cm={{1.00588,0.0,0.0,1.00588,(143.841,14.0824)}},draw=black,line join=bevel,line cap=rect,line width=0.800pt]
  \end{scope}
  \begin{scope}[cm={{1.00588,0.0,0.0,1.00588,(143.841,14.0824)}},draw=black,line join=bevel,line cap=rect,line width=0.800pt]
  \end{scope}
  \begin{scope}[cm={{1.00588,0.0,0.0,1.00588,(143.841,14.0824)}},draw=black,line join=bevel,line cap=rect,line width=0.800pt]
  \end{scope}
  \begin{scope}[cm={{1.00588,0.0,0.0,1.00588,(143.841,14.0824)}},draw=black,line join=bevel,line cap=rect,line width=0.800pt]
  \end{scope}
  \begin{scope}[scale=1.006,draw=black,line join=bevel,line cap=rect,line width=0.800pt]
  \end{scope}
  \begin{scope}[scale=1.006,draw=black,line join=bevel,line cap=rect,line width=0.800pt]
  \end{scope}
  \begin{scope}[cm={{1.00588,0.0,0.0,1.00588,(148.871,118.694)}},draw=black,line join=bevel,line cap=rect,line width=0.800pt]
  \end{scope}
  \begin{scope}[cm={{1.00588,0.0,0.0,1.00588,(148.871,118.694)}},draw=black,line join=bevel,line cap=rect,line width=0.800pt]
  \end{scope}
  \begin{scope}[cm={{1.00588,0.0,0.0,1.00588,(148.871,118.694)}},draw=black,line join=bevel,line cap=rect,line width=0.800pt]
  \end{scope}
  \begin{scope}[cm={{1.00588,0.0,0.0,1.00588,(148.871,118.694)}},draw=black,line join=bevel,line cap=rect,line width=0.800pt]
  \end{scope}
  \begin{scope}[cm={{1.00588,0.0,0.0,1.00588,(148.871,118.694)}},draw=black,line join=bevel,line cap=rect,line width=0.800pt]
  \end{scope}
  \begin{scope}[cm={{1.00588,0.0,0.0,1.00588,(148.871,118.694)}},draw=black,line join=bevel,line cap=rect,line width=0.800pt]
  \end{scope}
  \begin{scope}[scale=1.006,draw=black,line join=bevel,line cap=rect,line width=0.800pt]
  \end{scope}
  \begin{scope}[scale=1.006,draw=black,line join=bevel,line cap=rect,line width=0.800pt]
  \end{scope}
  \begin{scope}[cm={{1.00588,0.0,0.0,1.00588,(174.018,118.694)}},draw=black,line join=bevel,line cap=rect,line width=0.800pt]
  \end{scope}
  \begin{scope}[cm={{1.00588,0.0,0.0,1.00588,(174.018,118.694)}},draw=black,line join=bevel,line cap=rect,line width=0.800pt]
  \end{scope}
  \begin{scope}[cm={{1.00588,0.0,0.0,1.00588,(174.018,118.694)}},draw=black,line join=bevel,line cap=rect,line width=0.800pt]
  \end{scope}
  \begin{scope}[cm={{1.00588,0.0,0.0,1.00588,(174.018,118.694)}},draw=black,line join=bevel,line cap=rect,line width=0.800pt]
  \end{scope}
  \begin{scope}[cm={{1.00588,0.0,0.0,1.00588,(174.018,118.694)}},draw=black,line join=bevel,line cap=rect,line width=0.800pt]
  \end{scope}
  \begin{scope}[cm={{1.00588,0.0,0.0,1.00588,(174.018,118.694)}},draw=black,line join=bevel,line cap=rect,line width=0.800pt]
  \end{scope}
  \begin{scope}[scale=1.006,draw=black,line join=bevel,line cap=rect,line width=0.800pt]
  \end{scope}
  \begin{scope}[scale=1.006,draw=black,line join=bevel,line cap=rect,line width=0.800pt]
  \end{scope}
  \begin{scope}[cm={{1.00588,0.0,0.0,1.00588,(198.159,118.694)}},draw=black,line join=bevel,line cap=rect,line width=0.800pt]
  \end{scope}
  \begin{scope}[cm={{1.00588,0.0,0.0,1.00588,(198.159,118.694)}},draw=black,line join=bevel,line cap=rect,line width=0.800pt]
  \end{scope}
  \begin{scope}[cm={{1.00588,0.0,0.0,1.00588,(198.159,118.694)}},draw=black,line join=bevel,line cap=rect,line width=0.800pt]
  \end{scope}
  \begin{scope}[cm={{1.00588,0.0,0.0,1.00588,(198.159,118.694)}},draw=black,line join=bevel,line cap=rect,line width=0.800pt]
  \end{scope}
  \begin{scope}[cm={{1.00588,0.0,0.0,1.00588,(198.159,118.694)}},draw=black,line join=bevel,line cap=rect,line width=0.800pt]
  \end{scope}
  \begin{scope}[cm={{1.00588,0.0,0.0,1.00588,(198.159,118.694)}},draw=black,line join=bevel,line cap=rect,line width=0.800pt]
  \end{scope}
  \begin{scope}[scale=1.006,draw=black,line join=bevel,line cap=rect,line width=0.800pt]
  \end{scope}
  \begin{scope}[scale=1.006,draw=black,line join=bevel,line cap=rect,line width=0.800pt]
  \end{scope}
  \begin{scope}[cm={{1.00588,0.0,0.0,1.00588,(223.306,118.694)}},draw=black,line join=bevel,line cap=rect,line width=0.800pt]
  \end{scope}
  \begin{scope}[cm={{1.00588,0.0,0.0,1.00588,(223.306,118.694)}},draw=black,line join=bevel,line cap=rect,line width=0.800pt]
  \end{scope}
  \begin{scope}[cm={{1.00588,0.0,0.0,1.00588,(223.306,118.694)}},draw=black,line join=bevel,line cap=rect,line width=0.800pt]
  \end{scope}
  \begin{scope}[cm={{1.00588,0.0,0.0,1.00588,(223.306,118.694)}},draw=black,line join=bevel,line cap=rect,line width=0.800pt]
  \end{scope}
  \begin{scope}[cm={{1.00588,0.0,0.0,1.00588,(223.306,118.694)}},draw=black,line join=bevel,line cap=rect,line width=0.800pt]
  \end{scope}
  \begin{scope}[cm={{1.00588,0.0,0.0,1.00588,(223.306,118.694)}},draw=black,line join=bevel,line cap=rect,line width=0.800pt]
  \end{scope}
  \begin{scope}[scale=1.006,draw=black,line join=bevel,line cap=rect,line width=0.800pt]
  \end{scope}
  \begin{scope}[scale=1.006,draw=black,line join=bevel,line cap=rect,line width=0.800pt]
  \end{scope}
  \begin{scope}[cm={{1.00588,0.0,0.0,1.00588,(247.447,118.694)}},draw=black,line join=bevel,line cap=rect,line width=0.800pt]
  \end{scope}
  \begin{scope}[cm={{1.00588,0.0,0.0,1.00588,(247.447,118.694)}},draw=black,line join=bevel,line cap=rect,line width=0.800pt]
  \end{scope}
  \begin{scope}[cm={{1.00588,0.0,0.0,1.00588,(247.447,118.694)}},draw=black,line join=bevel,line cap=rect,line width=0.800pt]
  \end{scope}
  \begin{scope}[cm={{1.00588,0.0,0.0,1.00588,(247.447,118.694)}},draw=black,line join=bevel,line cap=rect,line width=0.800pt]
  \end{scope}
  \begin{scope}[cm={{1.00588,0.0,0.0,1.00588,(247.447,118.694)}},draw=black,line join=bevel,line cap=rect,line width=0.800pt]
  \end{scope}
  \begin{scope}[cm={{1.00588,0.0,0.0,1.00588,(247.447,118.694)}},draw=black,line join=bevel,line cap=rect,line width=0.800pt]
  \end{scope}
  \begin{scope}[scale=1.006,draw=black,line join=bevel,line cap=rect,line width=0.800pt]
  \end{scope}
  \begin{scope}[scale=1.006,draw=black,line join=bevel,line cap=rect,line width=0.800pt]
  \end{scope}
  \begin{scope}[scale=1.006,draw=black,line join=bevel,line cap=rect,line width=0.800pt]
  \end{scope}
  \begin{scope}[scale=1.006,draw=black,line join=bevel,line cap=rect,line width=0.800pt]
  \end{scope}
  \begin{scope}[scale=1.006,draw=black,line join=bevel,line cap=rect,line width=0.800pt]
  \end{scope}
  \begin{scope}[scale=1.006,draw=black,line join=bevel,line cap=rect,line width=0.800pt]
  \end{scope}
  \begin{scope}[cm={{1.00588,0.0,0.0,1.00588,(227.329,30.1765)}},draw=black,line join=bevel,line cap=rect,line width=0.800pt]
  \end{scope}
  \begin{scope}[cm={{1.00588,0.0,0.0,1.00588,(227.329,30.1765)}},draw=black,line join=bevel,line cap=rect,line width=0.800pt]
  \end{scope}
  \begin{scope}[cm={{1.00588,0.0,0.0,1.00588,(227.329,30.1765)}},draw=black,line join=bevel,line cap=rect,line width=0.800pt]
  \end{scope}
  \begin{scope}[cm={{1.00588,0.0,0.0,1.00588,(227.329,30.1765)}},draw=black,line join=bevel,line cap=rect,line width=0.800pt]
  \end{scope}
  \begin{scope}[cm={{1.00588,0.0,0.0,1.00588,(227.329,30.1765)}},draw=black,line join=bevel,line cap=rect,line width=0.800pt]
  \end{scope}
  \begin{scope}[cm={{1.00588,0.0,0.0,1.00588,(227.329,30.1765)}},draw=black,line join=bevel,line cap=rect,line width=0.800pt]
  \end{scope}
  \begin{scope}[scale=1.006,draw=black,line join=bevel,line cap=rect,line width=0.800pt]
  \end{scope}
  \begin{scope}[scale=1.006,draw=black,line join=bevel,line cap=rect,line width=0.800pt]
  \end{scope}
  \begin{scope}[scale=1.006,draw=black,line join=bevel,line cap=rect,line width=0.800pt]
  \end{scope}
  \begin{scope}[scale=1.006,draw=black,line join=bevel,line cap=rect,line width=0.800pt]
  \end{scope}
  \begin{scope}[scale=1.006,draw=black,line join=bevel,line cap=rect,line width=0.800pt]
  \end{scope}
  \begin{scope}[scale=1.006,draw=black,line join=bevel,line cap=rect,line width=0.800pt]
  \end{scope}
  \begin{scope}[cm={{1.00588,0.0,0.0,1.00588,(18.1059,216.265)}},draw=black,line join=bevel,line cap=rect,line width=0.800pt]
  \end{scope}
  \begin{scope}[cm={{1.00588,0.0,0.0,1.00588,(18.1059,216.265)}},draw=black,line join=bevel,line cap=rect,line width=0.800pt]
  \end{scope}
  \begin{scope}[cm={{1.00588,0.0,0.0,1.00588,(18.1059,216.265)}},draw=black,line join=bevel,line cap=rect,line width=0.800pt]
  \end{scope}
  \begin{scope}[cm={{1.00588,0.0,0.0,1.00588,(18.1059,216.265)}},draw=black,line join=bevel,line cap=rect,line width=0.800pt]
  \end{scope}
  \begin{scope}[cm={{1.00588,0.0,0.0,1.00588,(18.1059,216.265)}},draw=black,line join=bevel,line cap=rect,line width=0.800pt]
  \end{scope}
  \begin{scope}[cm={{1.00588,0.0,0.0,1.00588,(18.1059,216.265)}},draw=black,line join=bevel,line cap=rect,line width=0.800pt]
  \end{scope}
  \begin{scope}[scale=1.006,draw=black,line join=bevel,line cap=rect,line width=0.800pt]
  \end{scope}
  \begin{scope}[scale=1.006,draw=black,line join=bevel,line cap=rect,line width=0.800pt]
  \end{scope}
  \begin{scope}[cm={{1.00588,0.0,0.0,1.00588,(31.1824,187.094)}},draw=black,line join=bevel,line cap=rect,line width=0.800pt]
  \end{scope}
  \begin{scope}[cm={{1.00588,0.0,0.0,1.00588,(31.1824,187.094)}},draw=black,line join=bevel,line cap=rect,line width=0.800pt]
  \end{scope}
  \begin{scope}[cm={{1.00588,0.0,0.0,1.00588,(31.1824,187.094)}},draw=black,line join=bevel,line cap=rect,line width=0.800pt]
  \end{scope}
  \begin{scope}[cm={{1.00588,0.0,0.0,1.00588,(31.1824,187.094)}},draw=black,line join=bevel,line cap=rect,line width=0.800pt]
  \end{scope}
  \begin{scope}[cm={{1.00588,0.0,0.0,1.00588,(31.1824,187.094)}},draw=black,line join=bevel,line cap=rect,line width=0.800pt]
  \end{scope}
  \begin{scope}[cm={{1.00588,0.0,0.0,1.00588,(31.1824,187.094)}},draw=black,line join=bevel,line cap=rect,line width=0.800pt]
  \end{scope}
  \begin{scope}[scale=1.006,draw=black,line join=bevel,line cap=rect,line width=0.800pt]
  \end{scope}
  \begin{scope}[scale=1.006,draw=black,line join=bevel,line cap=rect,line width=0.800pt]
  \end{scope}
  \begin{scope}[cm={{1.00588,0.0,0.0,1.00588,(19.1118,157.924)}},draw=black,line join=bevel,line cap=rect,line width=0.800pt]
  \end{scope}
  \begin{scope}[cm={{1.00588,0.0,0.0,1.00588,(19.1118,157.924)}},draw=black,line join=bevel,line cap=rect,line width=0.800pt]
  \end{scope}
  \begin{scope}[cm={{1.00588,0.0,0.0,1.00588,(19.1118,157.924)}},draw=black,line join=bevel,line cap=rect,line width=0.800pt]
  \end{scope}
  \begin{scope}[cm={{1.00588,0.0,0.0,1.00588,(19.1118,157.924)}},draw=black,line join=bevel,line cap=rect,line width=0.800pt]
  \end{scope}
  \begin{scope}[cm={{1.00588,0.0,0.0,1.00588,(19.1118,157.924)}},draw=black,line join=bevel,line cap=rect,line width=0.800pt]
  \end{scope}
  \begin{scope}[cm={{1.00588,0.0,0.0,1.00588,(19.1118,157.924)}},draw=black,line join=bevel,line cap=rect,line width=0.800pt]
  \end{scope}
  \begin{scope}[scale=1.006,draw=black,line join=bevel,line cap=rect,line width=0.800pt]
  \end{scope}
  \begin{scope}[scale=1.006,draw=black,line join=bevel,line cap=rect,line width=0.800pt]
  \end{scope}
  \begin{scope}[cm={{1.00588,0.0,0.0,1.00588,(19.1118,128.753)}},draw=black,line join=bevel,line cap=rect,line width=0.800pt]
  \end{scope}
  \begin{scope}[cm={{1.00588,0.0,0.0,1.00588,(19.1118,128.753)}},draw=black,line join=bevel,line cap=rect,line width=0.800pt]
  \end{scope}
  \begin{scope}[cm={{1.00588,0.0,0.0,1.00588,(19.1118,128.753)}},draw=black,line join=bevel,line cap=rect,line width=0.800pt]
  \end{scope}
  \begin{scope}[cm={{1.00588,0.0,0.0,1.00588,(19.1118,128.753)}},draw=black,line join=bevel,line cap=rect,line width=0.800pt]
  \end{scope}
  \begin{scope}[cm={{1.00588,0.0,0.0,1.00588,(19.1118,128.753)}},draw=black,line join=bevel,line cap=rect,line width=0.800pt]
  \end{scope}
  \begin{scope}[cm={{1.00588,0.0,0.0,1.00588,(19.1118,128.753)}},draw=black,line join=bevel,line cap=rect,line width=0.800pt]
  \end{scope}
  \begin{scope}[scale=1.006,draw=black,line join=bevel,line cap=rect,line width=0.800pt]
  \end{scope}
  \begin{scope}[scale=1.006,draw=black,line join=bevel,line cap=rect,line width=0.800pt]
  \end{scope}
  \begin{scope}[cm={{1.00588,0.0,0.0,1.00588,(45.2647,228.335)}},draw=black,line join=bevel,line cap=rect,line width=0.800pt]
  \end{scope}
  \begin{scope}[cm={{1.00588,0.0,0.0,1.00588,(45.2647,228.335)}},draw=black,line join=bevel,line cap=rect,line width=0.800pt]
  \end{scope}
  \begin{scope}[cm={{1.00588,0.0,0.0,1.00588,(45.2647,228.335)}},draw=black,line join=bevel,line cap=rect,line width=0.800pt]
  \end{scope}
  \begin{scope}[cm={{1.00588,0.0,0.0,1.00588,(45.2647,228.335)}},draw=black,line join=bevel,line cap=rect,line width=0.800pt]
  \end{scope}
  \begin{scope}[cm={{1.00588,0.0,0.0,1.00588,(45.2647,228.335)}},draw=black,line join=bevel,line cap=rect,line width=0.800pt]
  \end{scope}
  \begin{scope}[cm={{1.00588,0.0,0.0,1.00588,(45.2647,228.335)}},draw=black,line join=bevel,line cap=rect,line width=0.800pt]
  \end{scope}
  \begin{scope}[scale=1.006,draw=black,line join=bevel,line cap=rect,line width=0.800pt]
  \end{scope}
  \begin{scope}[scale=1.006,draw=black,line join=bevel,line cap=rect,line width=0.800pt]
  \end{scope}
  \begin{scope}[cm={{1.00588,0.0,0.0,1.00588,(78.4588,228.335)}},draw=black,line join=bevel,line cap=rect,line width=0.800pt]
  \end{scope}
  \begin{scope}[cm={{1.00588,0.0,0.0,1.00588,(78.4588,228.335)}},draw=black,line join=bevel,line cap=rect,line width=0.800pt]
  \end{scope}
  \begin{scope}[cm={{1.00588,0.0,0.0,1.00588,(78.4588,228.335)}},draw=black,line join=bevel,line cap=rect,line width=0.800pt]
  \end{scope}
  \begin{scope}[cm={{1.00588,0.0,0.0,1.00588,(78.4588,228.335)}},draw=black,line join=bevel,line cap=rect,line width=0.800pt]
  \end{scope}
  \begin{scope}[cm={{1.00588,0.0,0.0,1.00588,(78.4588,228.335)}},draw=black,line join=bevel,line cap=rect,line width=0.800pt]
  \end{scope}
  \begin{scope}[cm={{1.00588,0.0,0.0,1.00588,(78.4588,228.335)}},draw=black,line join=bevel,line cap=rect,line width=0.800pt]
  \end{scope}
  \begin{scope}[scale=1.006,draw=black,line join=bevel,line cap=rect,line width=0.800pt]
  \end{scope}
  \begin{scope}[scale=1.006,draw=black,line join=bevel,line cap=rect,line width=0.800pt]
  \end{scope}
  \begin{scope}[cm={{1.00588,0.0,0.0,1.00588,(112.659,228.335)}},draw=black,line join=bevel,line cap=rect,line width=0.800pt]
  \end{scope}
  \begin{scope}[cm={{1.00588,0.0,0.0,1.00588,(112.659,228.335)}},draw=black,line join=bevel,line cap=rect,line width=0.800pt]
  \end{scope}
  \begin{scope}[cm={{1.00588,0.0,0.0,1.00588,(112.659,228.335)}},draw=black,line join=bevel,line cap=rect,line width=0.800pt]
  \end{scope}
  \begin{scope}[cm={{1.00588,0.0,0.0,1.00588,(112.659,228.335)}},draw=black,line join=bevel,line cap=rect,line width=0.800pt]
  \end{scope}
  \begin{scope}[cm={{1.00588,0.0,0.0,1.00588,(112.659,228.335)}},draw=black,line join=bevel,line cap=rect,line width=0.800pt]
  \end{scope}
  \begin{scope}[cm={{1.00588,0.0,0.0,1.00588,(112.659,228.335)}},draw=black,line join=bevel,line cap=rect,line width=0.800pt]
  \end{scope}
  \begin{scope}[scale=1.006,draw=black,line join=bevel,line cap=rect,line width=0.800pt]
  \end{scope}
  \begin{scope}[scale=1.006,draw=black,line join=bevel,line cap=rect,line width=0.800pt]
  \end{scope}
  \begin{scope}[cm={{1.00588,0.0,0.0,1.00588,(146.859,228.335)}},draw=black,line join=bevel,line cap=rect,line width=0.800pt]
  \end{scope}
  \begin{scope}[cm={{1.00588,0.0,0.0,1.00588,(146.859,228.335)}},draw=black,line join=bevel,line cap=rect,line width=0.800pt]
  \end{scope}
  \begin{scope}[cm={{1.00588,0.0,0.0,1.00588,(146.859,228.335)}},draw=black,line join=bevel,line cap=rect,line width=0.800pt]
  \end{scope}
  \begin{scope}[cm={{1.00588,0.0,0.0,1.00588,(146.859,228.335)}},draw=black,line join=bevel,line cap=rect,line width=0.800pt]
  \end{scope}
  \begin{scope}[cm={{1.00588,0.0,0.0,1.00588,(146.859,228.335)}},draw=black,line join=bevel,line cap=rect,line width=0.800pt]
  \end{scope}
  \begin{scope}[cm={{1.00588,0.0,0.0,1.00588,(146.859,228.335)}},draw=black,line join=bevel,line cap=rect,line width=0.800pt]
  \end{scope}
  \begin{scope}[scale=1.006,draw=black,line join=bevel,line cap=rect,line width=0.800pt]
  \end{scope}
  \begin{scope}[scale=1.006,draw=black,line join=bevel,line cap=rect,line width=0.800pt]
  \end{scope}
  \begin{scope}[cm={{1.00588,0.0,0.0,1.00588,(180.053,228.335)}},draw=black,line join=bevel,line cap=rect,line width=0.800pt]
  \end{scope}
  \begin{scope}[cm={{1.00588,0.0,0.0,1.00588,(180.053,228.335)}},draw=black,line join=bevel,line cap=rect,line width=0.800pt]
  \end{scope}
  \begin{scope}[cm={{1.00588,0.0,0.0,1.00588,(180.053,228.335)}},draw=black,line join=bevel,line cap=rect,line width=0.800pt]
  \end{scope}
  \begin{scope}[cm={{1.00588,0.0,0.0,1.00588,(180.053,228.335)}},draw=black,line join=bevel,line cap=rect,line width=0.800pt]
  \end{scope}
  \begin{scope}[cm={{1.00588,0.0,0.0,1.00588,(180.053,228.335)}},draw=black,line join=bevel,line cap=rect,line width=0.800pt]
  \end{scope}
  \begin{scope}[cm={{1.00588,0.0,0.0,1.00588,(180.053,228.335)}},draw=black,line join=bevel,line cap=rect,line width=0.800pt]
  \end{scope}
  \begin{scope}[scale=1.006,draw=black,line join=bevel,line cap=rect,line width=0.800pt]
  \end{scope}
  \begin{scope}[scale=1.006,draw=black,line join=bevel,line cap=rect,line width=0.800pt]
  \end{scope}
  \begin{scope}[scale=1.006,draw=black,line join=bevel,line cap=rect,line width=0.800pt]
  \end{scope}
  \begin{scope}[scale=1.006,draw=black,line join=bevel,line cap=rect,line width=0.800pt]
  \end{scope}
  \begin{scope}[scale=1.006,draw=black,line join=bevel,line cap=rect,line width=0.800pt]
  \end{scope}
  \begin{scope}[scale=1.006,draw=black,line join=bevel,line cap=rect,line width=0.800pt]
  \end{scope}
  \begin{scope}[cm={{1.00588,0.0,0.0,1.00588,(161.947,127.747)}},draw=black,line join=bevel,line cap=rect,line width=0.800pt]
  \end{scope}
  \begin{scope}[cm={{1.00588,0.0,0.0,1.00588,(161.947,127.747)}},draw=black,line join=bevel,line cap=rect,line width=0.800pt]
  \end{scope}
  \begin{scope}[cm={{1.00588,0.0,0.0,1.00588,(161.947,127.747)}},draw=black,line join=bevel,line cap=rect,line width=0.800pt]
  \end{scope}
  \begin{scope}[cm={{1.00588,0.0,0.0,1.00588,(161.947,127.747)}},draw=black,line join=bevel,line cap=rect,line width=0.800pt]
  \end{scope}
  \begin{scope}[cm={{1.00588,0.0,0.0,1.00588,(161.947,127.747)}},draw=black,line join=bevel,line cap=rect,line width=0.800pt]
  \end{scope}
  \begin{scope}[cm={{1.00588,0.0,0.0,1.00588,(161.947,127.747)}},draw=black,line join=bevel,line cap=rect,line width=0.800pt]
  \end{scope}
  \begin{scope}[scale=1.006,draw=black,line join=bevel,line cap=rect,line width=0.800pt]
  \end{scope}
  \begin{scope}[scale=1.006,draw=black,line join=bevel,line cap=rect,line width=0.800pt]
  \end{scope}
  \begin{scope}[scale=1.006,draw=black,line join=bevel,line cap=rect,line width=0.800pt]
  \end{scope}
  \begin{scope}[scale=1.006,draw=black,line join=bevel,line cap=rect,line width=0.800pt]
  \end{scope}
  \begin{scope}[scale=1.006,draw=black,line join=bevel,line cap=rect,line width=0.800pt]
  \end{scope}
  \begin{scope}[scale=1.006,draw=black,line join=bevel,line cap=rect,line width=0.800pt]
  \end{scope}
  \begin{scope}[cm={{1.00588,0.0,0.0,1.00588,(181.059,152.894)}},draw=black,line join=bevel,line cap=rect,line width=0.800pt]
  \end{scope}
  \begin{scope}[cm={{1.00588,0.0,0.0,1.00588,(181.059,152.894)}},draw=black,line join=bevel,line cap=rect,line width=0.800pt]
  \end{scope}
  \begin{scope}[cm={{1.00588,0.0,0.0,1.00588,(181.059,152.894)}},draw=black,line join=bevel,line cap=rect,line width=0.800pt]
  \end{scope}
  \begin{scope}[cm={{1.00588,0.0,0.0,1.00588,(181.059,152.894)}},draw=black,line join=bevel,line cap=rect,line width=0.800pt]
  \end{scope}
  \begin{scope}[cm={{1.00588,0.0,0.0,1.00588,(181.059,152.894)}},draw=black,line join=bevel,line cap=rect,line width=0.800pt]
  \end{scope}
  \begin{scope}[cm={{1.00588,0.0,0.0,1.00588,(181.059,152.894)}},draw=black,line join=bevel,line cap=rect,line width=0.800pt]
  \end{scope}
  \begin{scope}[scale=1.006,draw=black,line join=bevel,line cap=rect,line width=0.800pt]
  \end{scope}
  \begin{scope}[scale=1.006,draw=black,line join=bevel,line cap=rect,line width=0.800pt]
  \end{scope}
  \begin{scope}[cm={{1.00588,0.0,0.0,1.00588,(181.059,133.782)}},draw=black,line join=bevel,line cap=rect,line width=0.800pt]
  \end{scope}
  \begin{scope}[cm={{1.00588,0.0,0.0,1.00588,(181.059,133.782)}},draw=black,line join=bevel,line cap=rect,line width=0.800pt]
  \end{scope}
  \begin{scope}[cm={{1.00588,0.0,0.0,1.00588,(181.059,133.782)}},draw=black,line join=bevel,line cap=rect,line width=0.800pt]
  \end{scope}
  \begin{scope}[cm={{1.00588,0.0,0.0,1.00588,(181.059,133.782)}},draw=black,line join=bevel,line cap=rect,line width=0.800pt]
  \end{scope}
  \begin{scope}[cm={{1.00588,0.0,0.0,1.00588,(181.059,133.782)}},draw=black,line join=bevel,line cap=rect,line width=0.800pt]
  \end{scope}
  \begin{scope}[cm={{1.00588,0.0,0.0,1.00588,(181.059,133.782)}},draw=black,line join=bevel,line cap=rect,line width=0.800pt]
  \end{scope}
  \begin{scope}[scale=1.006,draw=black,line join=bevel,line cap=rect,line width=0.800pt]
  \end{scope}
  \begin{scope}[scale=1.006,draw=black,line join=bevel,line cap=rect,line width=0.800pt]
  \end{scope}
  \begin{scope}[cm={{1.00588,0.0,0.0,1.00588,(181.059,114.671)}},draw=black,line join=bevel,line cap=rect,line width=0.800pt]
  \end{scope}
  \begin{scope}[cm={{1.00588,0.0,0.0,1.00588,(181.059,114.671)}},draw=black,line join=bevel,line cap=rect,line width=0.800pt]
  \end{scope}
  \begin{scope}[cm={{1.00588,0.0,0.0,1.00588,(181.059,114.671)}},draw=black,line join=bevel,line cap=rect,line width=0.800pt]
  \end{scope}
  \begin{scope}[cm={{1.00588,0.0,0.0,1.00588,(181.059,114.671)}},draw=black,line join=bevel,line cap=rect,line width=0.800pt]
  \end{scope}
  \begin{scope}[cm={{1.00588,0.0,0.0,1.00588,(181.059,114.671)}},draw=black,line join=bevel,line cap=rect,line width=0.800pt]
  \end{scope}
  \begin{scope}[cm={{1.00588,0.0,0.0,1.00588,(181.059,114.671)}},draw=black,line join=bevel,line cap=rect,line width=0.800pt]
  \end{scope}
  \begin{scope}[scale=1.006,draw=black,line join=bevel,line cap=rect,line width=0.800pt]
  \end{scope}
  \begin{scope}[scale=1.006,draw=black,line join=bevel,line cap=rect,line width=0.800pt]
  \end{scope}
  \begin{scope}[cm={{1.00588,0.0,0.0,1.00588,(186.088,173.012)}},draw=black,line join=bevel,line cap=rect,line width=0.800pt]
  \end{scope}
  \begin{scope}[cm={{1.00588,0.0,0.0,1.00588,(186.088,173.012)}},draw=black,line join=bevel,line cap=rect,line width=0.800pt]
  \end{scope}
  \begin{scope}[cm={{1.00588,0.0,0.0,1.00588,(186.088,173.012)}},draw=black,line join=bevel,line cap=rect,line width=0.800pt]
  \end{scope}
  \begin{scope}[cm={{1.00588,0.0,0.0,1.00588,(186.088,173.012)}},draw=black,line join=bevel,line cap=rect,line width=0.800pt]
  \end{scope}
  \begin{scope}[cm={{1.00588,0.0,0.0,1.00588,(186.088,173.012)}},draw=black,line join=bevel,line cap=rect,line width=0.800pt]
  \end{scope}
  \begin{scope}[cm={{1.00588,0.0,0.0,1.00588,(186.088,173.012)}},draw=black,line join=bevel,line cap=rect,line width=0.800pt]
  \end{scope}
  \begin{scope}[scale=1.006,draw=black,line join=bevel,line cap=rect,line width=0.800pt]
  \end{scope}
  \begin{scope}[scale=1.006,draw=black,line join=bevel,line cap=rect,line width=0.800pt]
  \end{scope}
  \begin{scope}[cm={{1.00588,0.0,0.0,1.00588,(204.194,173.012)}},draw=black,line join=bevel,line cap=rect,line width=0.800pt]
  \end{scope}
  \begin{scope}[cm={{1.00588,0.0,0.0,1.00588,(204.194,173.012)}},draw=black,line join=bevel,line cap=rect,line width=0.800pt]
  \end{scope}
  \begin{scope}[cm={{1.00588,0.0,0.0,1.00588,(204.194,173.012)}},draw=black,line join=bevel,line cap=rect,line width=0.800pt]
  \end{scope}
  \begin{scope}[cm={{1.00588,0.0,0.0,1.00588,(204.194,173.012)}},draw=black,line join=bevel,line cap=rect,line width=0.800pt]
  \end{scope}
  \begin{scope}[cm={{1.00588,0.0,0.0,1.00588,(204.194,173.012)}},draw=black,line join=bevel,line cap=rect,line width=0.800pt]
  \end{scope}
  \begin{scope}[cm={{1.00588,0.0,0.0,1.00588,(204.194,173.012)}},draw=black,line join=bevel,line cap=rect,line width=0.800pt]
  \end{scope}
  \begin{scope}[scale=1.006,draw=black,line join=bevel,line cap=rect,line width=0.800pt]
  \end{scope}
  \begin{scope}[scale=1.006,draw=black,line join=bevel,line cap=rect,line width=0.800pt]
  \end{scope}
  \begin{scope}[cm={{1.00588,0.0,0.0,1.00588,(222.3,173.012)}},draw=black,line join=bevel,line cap=rect,line width=0.800pt]
  \end{scope}
  \begin{scope}[cm={{1.00588,0.0,0.0,1.00588,(222.3,173.012)}},draw=black,line join=bevel,line cap=rect,line width=0.800pt]
  \end{scope}
  \begin{scope}[cm={{1.00588,0.0,0.0,1.00588,(222.3,173.012)}},draw=black,line join=bevel,line cap=rect,line width=0.800pt]
  \end{scope}
  \begin{scope}[cm={{1.00588,0.0,0.0,1.00588,(222.3,173.012)}},draw=black,line join=bevel,line cap=rect,line width=0.800pt]
  \end{scope}
  \begin{scope}[cm={{1.00588,0.0,0.0,1.00588,(222.3,173.012)}},draw=black,line join=bevel,line cap=rect,line width=0.800pt]
  \end{scope}
  \begin{scope}[cm={{1.00588,0.0,0.0,1.00588,(222.3,173.012)}},draw=black,line join=bevel,line cap=rect,line width=0.800pt]
  \end{scope}
  \begin{scope}[scale=1.006,draw=black,line join=bevel,line cap=rect,line width=0.800pt]
  \end{scope}
  \begin{scope}[scale=1.006,draw=black,line join=bevel,line cap=rect,line width=0.800pt]
  \end{scope}
  \begin{scope}[cm={{1.00588,0.0,0.0,1.00588,(241.412,173.012)}},draw=black,line join=bevel,line cap=rect,line width=0.800pt]
  \end{scope}
  \begin{scope}[cm={{1.00588,0.0,0.0,1.00588,(241.412,173.012)}},draw=black,line join=bevel,line cap=rect,line width=0.800pt]
  \end{scope}
  \begin{scope}[cm={{1.00588,0.0,0.0,1.00588,(241.412,173.012)}},draw=black,line join=bevel,line cap=rect,line width=0.800pt]
  \end{scope}
  \begin{scope}[cm={{1.00588,0.0,0.0,1.00588,(241.412,173.012)}},draw=black,line join=bevel,line cap=rect,line width=0.800pt]
  \end{scope}
  \begin{scope}[cm={{1.00588,0.0,0.0,1.00588,(241.412,173.012)}},draw=black,line join=bevel,line cap=rect,line width=0.800pt]
  \end{scope}
  \begin{scope}[cm={{1.00588,0.0,0.0,1.00588,(241.412,173.012)}},draw=black,line join=bevel,line cap=rect,line width=0.800pt]
  \end{scope}
  \begin{scope}[scale=1.006,draw=black,line join=bevel,line cap=rect,line width=0.800pt]
  \end{scope}
  \begin{scope}[scale=1.006,draw=black,line join=bevel,line cap=rect,line width=0.800pt]
  \end{scope}
  \begin{scope}[cm={{1.00588,0.0,0.0,1.00588,(249.459,156.918)}},draw=black,line join=bevel,line cap=rect,line width=0.800pt]
  \end{scope}
  \begin{scope}[cm={{1.00588,0.0,0.0,1.00588,(249.459,156.918)}},draw=black,line join=bevel,line cap=rect,line width=0.800pt]
  \end{scope}
  \begin{scope}[cm={{1.00588,0.0,0.0,1.00588,(249.459,156.918)}},draw=black,line join=bevel,line cap=rect,line width=0.800pt]
  \end{scope}
  \begin{scope}[cm={{1.00588,0.0,0.0,1.00588,(249.459,156.918)}},draw=black,line join=bevel,line cap=rect,line width=0.800pt]
  \end{scope}
  \begin{scope}[cm={{1.00588,0.0,0.0,1.00588,(249.459,156.918)}},draw=black,line join=bevel,line cap=rect,line width=0.800pt]
  \end{scope}
  \begin{scope}[cm={{1.00588,0.0,0.0,1.00588,(249.459,156.918)}},draw=black,line join=bevel,line cap=rect,line width=0.800pt]
  \end{scope}
  \begin{scope}[scale=1.006,draw=black,line join=bevel,line cap=rect,line width=0.800pt]
  \end{scope}
  \begin{scope}[scale=1.006,draw=black,line join=bevel,line cap=rect,line width=0.800pt]
  \end{scope}
  \begin{scope}[cm={{1.00588,0.0,0.0,1.00588,(249.459,137.806)}},draw=black,line join=bevel,line cap=rect,line width=0.800pt]
  \end{scope}
  \begin{scope}[cm={{1.00588,0.0,0.0,1.00588,(249.459,137.806)}},draw=black,line join=bevel,line cap=rect,line width=0.800pt]
  \end{scope}
  \begin{scope}[cm={{1.00588,0.0,0.0,1.00588,(249.459,137.806)}},draw=black,line join=bevel,line cap=rect,line width=0.800pt]
  \end{scope}
  \begin{scope}[cm={{1.00588,0.0,0.0,1.00588,(249.459,137.806)}},draw=black,line join=bevel,line cap=rect,line width=0.800pt]
  \end{scope}
  \begin{scope}[cm={{1.00588,0.0,0.0,1.00588,(249.459,137.806)}},draw=black,line join=bevel,line cap=rect,line width=0.800pt]
  \end{scope}
  \begin{scope}[cm={{1.00588,0.0,0.0,1.00588,(249.459,137.806)}},draw=black,line join=bevel,line cap=rect,line width=0.800pt]
  \end{scope}
  \begin{scope}[scale=1.006,draw=black,line join=bevel,line cap=rect,line width=0.800pt]
  \end{scope}
  \begin{scope}[scale=1.006,draw=black,line join=bevel,line cap=rect,line width=0.800pt]
  \end{scope}
  \begin{scope}[cm={{1.00588,0.0,0.0,1.00588,(249.459,118.694)}},draw=black,line join=bevel,line cap=rect,line width=0.800pt]
  \end{scope}
  \begin{scope}[cm={{1.00588,0.0,0.0,1.00588,(249.459,118.694)}},draw=black,line join=bevel,line cap=rect,line width=0.800pt]
  \end{scope}
  \begin{scope}[cm={{1.00588,0.0,0.0,1.00588,(249.459,118.694)}},draw=black,line join=bevel,line cap=rect,line width=0.800pt]
  \end{scope}
  \begin{scope}[cm={{1.00588,0.0,0.0,1.00588,(249.459,118.694)}},draw=black,line join=bevel,line cap=rect,line width=0.800pt]
  \end{scope}
  \begin{scope}[cm={{1.00588,0.0,0.0,1.00588,(249.459,118.694)}},draw=black,line join=bevel,line cap=rect,line width=0.800pt]
  \end{scope}
  \begin{scope}[cm={{1.00588,0.0,0.0,1.00588,(249.459,118.694)}},draw=black,line join=bevel,line cap=rect,line width=0.800pt]
  \end{scope}
  \begin{scope}[scale=1.006,draw=black,line join=bevel,line cap=rect,line width=0.800pt]
  \end{scope}
  \begin{scope}[scale=1.006,draw=black,line join=bevel,line cap=rect,line width=0.800pt]
  \end{scope}
  \begin{scope}[cm={{0.0,-1.00588,1.00588,0.0,(276.618,139.315)}},draw=black,line join=bevel,line cap=rect,line width=0.800pt]
  \end{scope}
  \begin{scope}[cm={{0.0,-1.00588,1.00588,0.0,(276.618,139.315)}},draw=black,line join=bevel,line cap=rect,line width=0.800pt]
  \end{scope}
  \begin{scope}[cm={{0.0,-1.00588,1.00588,0.0,(276.618,139.315)}},draw=black,line join=bevel,line cap=rect,line width=0.800pt]
  \end{scope}
  \begin{scope}[cm={{0.0,-1.00588,1.00588,0.0,(276.618,139.315)}},draw=black,line join=bevel,line cap=rect,line width=0.800pt]
  \end{scope}
  \begin{scope}[cm={{0.0,-1.00588,1.00588,0.0,(276.618,139.315)}},draw=black,line join=bevel,line cap=rect,line width=0.800pt]
  \end{scope}
  \begin{scope}[cm={{0.0,-1.00588,1.00588,0.0,(276.618,139.315)}},draw=black,line join=bevel,line cap=rect,line width=0.800pt]
  \end{scope}
  \begin{scope}[scale=1.006,draw=black,line join=bevel,line cap=rect,line width=0.800pt]
  \end{scope}
  \begin{scope}[scale=1.006,draw=black,line join=bevel,line cap=rect,line width=0.800pt]
  \end{scope}
  \begin{scope}[scale=1.006,draw=black,line join=bevel,line cap=rect,line width=0.800pt]
  \end{scope}
  \begin{scope}[scale=1.006,draw=black,line join=bevel,line cap=rect,line width=0.800pt]
  \end{scope}
  \begin{scope}[scale=1.006,draw=black,line join=bevel,line cap=rect,line width=0.800pt]
  \end{scope}
  \begin{scope}[scale=1.006,draw=black,line join=bevel,line cap=rect,line width=0.800pt]
  \end{scope}
  \begin{scope}[cm={{1.00588,0.0,0.0,1.00588,(181.059,201.176)}},draw=black,line join=bevel,line cap=rect,line width=0.800pt]
  \end{scope}
  \begin{scope}[cm={{1.00588,0.0,0.0,1.00588,(181.059,201.176)}},draw=black,line join=bevel,line cap=rect,line width=0.800pt]
  \end{scope}
  \begin{scope}[cm={{1.00588,0.0,0.0,1.00588,(181.059,201.176)}},draw=black,line join=bevel,line cap=rect,line width=0.800pt]
  \end{scope}
  \begin{scope}[cm={{1.00588,0.0,0.0,1.00588,(181.059,201.176)}},draw=black,line join=bevel,line cap=rect,line width=0.800pt]
  \end{scope}
  \begin{scope}[cm={{1.00588,0.0,0.0,1.00588,(181.059,201.176)}},draw=black,line join=bevel,line cap=rect,line width=0.800pt]
  \end{scope}
  \begin{scope}[cm={{1.00588,0.0,0.0,1.00588,(181.059,201.176)}},draw=black,line join=bevel,line cap=rect,line width=0.800pt]
  \end{scope}
  \begin{scope}[scale=1.006,draw=black,line join=bevel,line cap=rect,line width=0.800pt]
  \end{scope}
  \begin{scope}[scale=1.006,draw=black,line join=bevel,line cap=rect,line width=0.800pt]
  \end{scope}
  \begin{scope}[cm={{1.00588,0.0,0.0,1.00588,(181.059,181.059)}},draw=black,line join=bevel,line cap=rect,line width=0.800pt]
  \end{scope}
  \begin{scope}[cm={{1.00588,0.0,0.0,1.00588,(181.059,181.059)}},draw=black,line join=bevel,line cap=rect,line width=0.800pt]
  \end{scope}
  \begin{scope}[cm={{1.00588,0.0,0.0,1.00588,(181.059,181.059)}},draw=black,line join=bevel,line cap=rect,line width=0.800pt]
  \end{scope}
  \begin{scope}[cm={{1.00588,0.0,0.0,1.00588,(181.059,181.059)}},draw=black,line join=bevel,line cap=rect,line width=0.800pt]
  \end{scope}
  \begin{scope}[cm={{1.00588,0.0,0.0,1.00588,(181.059,181.059)}},draw=black,line join=bevel,line cap=rect,line width=0.800pt]
  \end{scope}
  \begin{scope}[cm={{1.00588,0.0,0.0,1.00588,(181.059,181.059)}},draw=black,line join=bevel,line cap=rect,line width=0.800pt]
  \end{scope}
  \begin{scope}[scale=1.006,draw=black,line join=bevel,line cap=rect,line width=0.800pt]
  \end{scope}
  \begin{scope}[scale=1.006,draw=black,line join=bevel,line cap=rect,line width=0.800pt]
  \end{scope}
  \begin{scope}[cm={{1.00588,0.0,0.0,1.00588,(181.059,161.947)}},draw=black,line join=bevel,line cap=rect,line width=0.800pt]
  \end{scope}
  \begin{scope}[cm={{1.00588,0.0,0.0,1.00588,(181.059,161.947)}},draw=black,line join=bevel,line cap=rect,line width=0.800pt]
  \end{scope}
  \begin{scope}[cm={{1.00588,0.0,0.0,1.00588,(181.059,161.947)}},draw=black,line join=bevel,line cap=rect,line width=0.800pt]
  \end{scope}
  \begin{scope}[cm={{1.00588,0.0,0.0,1.00588,(181.059,161.947)}},draw=black,line join=bevel,line cap=rect,line width=0.800pt]
  \end{scope}
  \begin{scope}[cm={{1.00588,0.0,0.0,1.00588,(181.059,161.947)}},draw=black,line join=bevel,line cap=rect,line width=0.800pt]
  \end{scope}
  \begin{scope}[cm={{1.00588,0.0,0.0,1.00588,(181.059,161.947)}},draw=black,line join=bevel,line cap=rect,line width=0.800pt]
  \end{scope}
  \begin{scope}[scale=1.006,draw=black,line join=bevel,line cap=rect,line width=0.800pt]
  \end{scope}
  \begin{scope}[scale=1.006,draw=black,line join=bevel,line cap=rect,line width=0.800pt]
  \end{scope}
  \begin{scope}[cm={{1.00588,0.0,0.0,1.00588,(183.071,220.288)}},draw=black,line join=bevel,line cap=rect,line width=0.800pt]
  \end{scope}
  \begin{scope}[cm={{1.00588,0.0,0.0,1.00588,(183.071,220.288)}},draw=black,line join=bevel,line cap=rect,line width=0.800pt]
  \end{scope}
  \begin{scope}[cm={{1.00588,0.0,0.0,1.00588,(183.071,220.288)}},draw=black,line join=bevel,line cap=rect,line width=0.800pt]
  \end{scope}
  \begin{scope}[cm={{1.00588,0.0,0.0,1.00588,(183.071,220.288)}},draw=black,line join=bevel,line cap=rect,line width=0.800pt]
  \end{scope}
  \begin{scope}[cm={{1.00588,0.0,0.0,1.00588,(183.071,220.288)}},draw=black,line join=bevel,line cap=rect,line width=0.800pt]
  \end{scope}
  \begin{scope}[cm={{1.00588,0.0,0.0,1.00588,(183.071,220.288)}},draw=black,line join=bevel,line cap=rect,line width=0.800pt]
  \end{scope}
  \begin{scope}[scale=1.006,draw=black,line join=bevel,line cap=rect,line width=0.800pt]
  \end{scope}
  \begin{scope}[scale=1.006,draw=black,line join=bevel,line cap=rect,line width=0.800pt]
  \end{scope}
  \begin{scope}[cm={{1.00588,0.0,0.0,1.00588,(201.176,220.288)}},draw=black,line join=bevel,line cap=rect,line width=0.800pt]
  \end{scope}
  \begin{scope}[cm={{1.00588,0.0,0.0,1.00588,(201.176,220.288)}},draw=black,line join=bevel,line cap=rect,line width=0.800pt]
  \end{scope}
  \begin{scope}[cm={{1.00588,0.0,0.0,1.00588,(201.176,220.288)}},draw=black,line join=bevel,line cap=rect,line width=0.800pt]
  \end{scope}
  \begin{scope}[cm={{1.00588,0.0,0.0,1.00588,(201.176,220.288)}},draw=black,line join=bevel,line cap=rect,line width=0.800pt]
  \end{scope}
  \begin{scope}[cm={{1.00588,0.0,0.0,1.00588,(201.176,220.288)}},draw=black,line join=bevel,line cap=rect,line width=0.800pt]
  \end{scope}
  \begin{scope}[cm={{1.00588,0.0,0.0,1.00588,(201.176,220.288)}},draw=black,line join=bevel,line cap=rect,line width=0.800pt]
  \end{scope}
  \begin{scope}[scale=1.006,draw=black,line join=bevel,line cap=rect,line width=0.800pt]
  \end{scope}
  \begin{scope}[scale=1.006,draw=black,line join=bevel,line cap=rect,line width=0.800pt]
  \end{scope}
  \begin{scope}[cm={{1.00588,0.0,0.0,1.00588,(219.785,220.288)}},draw=black,line join=bevel,line cap=rect,line width=0.800pt]
  \end{scope}
  \begin{scope}[cm={{1.00588,0.0,0.0,1.00588,(219.785,220.288)}},draw=black,line join=bevel,line cap=rect,line width=0.800pt]
  \end{scope}
  \begin{scope}[cm={{1.00588,0.0,0.0,1.00588,(219.785,220.288)}},draw=black,line join=bevel,line cap=rect,line width=0.800pt]
  \end{scope}
  \begin{scope}[cm={{1.00588,0.0,0.0,1.00588,(219.785,220.288)}},draw=black,line join=bevel,line cap=rect,line width=0.800pt]
  \end{scope}
  \begin{scope}[cm={{1.00588,0.0,0.0,1.00588,(219.785,220.288)}},draw=black,line join=bevel,line cap=rect,line width=0.800pt]
  \end{scope}
  \begin{scope}[cm={{1.00588,0.0,0.0,1.00588,(219.785,220.288)}},draw=black,line join=bevel,line cap=rect,line width=0.800pt]
  \end{scope}
  \begin{scope}[scale=1.006,draw=black,line join=bevel,line cap=rect,line width=0.800pt]
  \end{scope}
  \begin{scope}[scale=1.006,draw=black,line join=bevel,line cap=rect,line width=0.800pt]
  \end{scope}
  \begin{scope}[cm={{1.00588,0.0,0.0,1.00588,(238.394,220.288)}},draw=black,line join=bevel,line cap=rect,line width=0.800pt]
  \end{scope}
  \begin{scope}[cm={{1.00588,0.0,0.0,1.00588,(238.394,220.288)}},draw=black,line join=bevel,line cap=rect,line width=0.800pt]
  \end{scope}
  \begin{scope}[cm={{1.00588,0.0,0.0,1.00588,(238.394,220.288)}},draw=black,line join=bevel,line cap=rect,line width=0.800pt]
  \end{scope}
  \begin{scope}[cm={{1.00588,0.0,0.0,1.00588,(238.394,220.288)}},draw=black,line join=bevel,line cap=rect,line width=0.800pt]
  \end{scope}
  \begin{scope}[cm={{1.00588,0.0,0.0,1.00588,(238.394,220.288)}},draw=black,line join=bevel,line cap=rect,line width=0.800pt]
  \end{scope}
  \begin{scope}[cm={{1.00588,0.0,0.0,1.00588,(238.394,220.288)}},draw=black,line join=bevel,line cap=rect,line width=0.800pt]
  \end{scope}
  \begin{scope}[scale=1.006,draw=black,line join=bevel,line cap=rect,line width=0.800pt]
  \end{scope}
  \begin{scope}[scale=1.006,draw=black,line join=bevel,line cap=rect,line width=0.800pt]
  \end{scope}
  \begin{scope}[cm={{1.00588,0.0,0.0,1.00588,(253.482,205.2)}},draw=black,line join=bevel,line cap=rect,line width=0.800pt]
  \end{scope}
  \begin{scope}[cm={{1.00588,0.0,0.0,1.00588,(253.482,205.2)}},draw=black,line join=bevel,line cap=rect,line width=0.800pt]
  \end{scope}
  \begin{scope}[cm={{1.00588,0.0,0.0,1.00588,(253.482,205.2)}},draw=black,line join=bevel,line cap=rect,line width=0.800pt]
  \end{scope}
  \begin{scope}[cm={{1.00588,0.0,0.0,1.00588,(253.482,205.2)}},draw=black,line join=bevel,line cap=rect,line width=0.800pt]
  \end{scope}
  \begin{scope}[cm={{1.00588,0.0,0.0,1.00588,(253.482,205.2)}},draw=black,line join=bevel,line cap=rect,line width=0.800pt]
  \end{scope}
  \begin{scope}[cm={{1.00588,0.0,0.0,1.00588,(253.482,205.2)}},draw=black,line join=bevel,line cap=rect,line width=0.800pt]
  \end{scope}
  \begin{scope}[scale=1.006,draw=black,line join=bevel,line cap=rect,line width=0.800pt]
  \end{scope}
  \begin{scope}[scale=1.006,draw=black,line join=bevel,line cap=rect,line width=0.800pt]
  \end{scope}
  \begin{scope}[cm={{1.00588,0.0,0.0,1.00588,(253.482,185.082)}},draw=black,line join=bevel,line cap=rect,line width=0.800pt]
  \end{scope}
  \begin{scope}[cm={{1.00588,0.0,0.0,1.00588,(253.482,185.082)}},draw=black,line join=bevel,line cap=rect,line width=0.800pt]
  \end{scope}
  \begin{scope}[cm={{1.00588,0.0,0.0,1.00588,(253.482,185.082)}},draw=black,line join=bevel,line cap=rect,line width=0.800pt]
  \end{scope}
  \begin{scope}[cm={{1.00588,0.0,0.0,1.00588,(253.482,185.082)}},draw=black,line join=bevel,line cap=rect,line width=0.800pt]
  \end{scope}
  \begin{scope}[cm={{1.00588,0.0,0.0,1.00588,(253.482,185.082)}},draw=black,line join=bevel,line cap=rect,line width=0.800pt]
  \end{scope}
  \begin{scope}[cm={{1.00588,0.0,0.0,1.00588,(253.482,185.082)}},draw=black,line join=bevel,line cap=rect,line width=0.800pt]
  \end{scope}
  \begin{scope}[scale=1.006,draw=black,line join=bevel,line cap=rect,line width=0.800pt]
  \end{scope}
  \begin{scope}[scale=1.006,draw=black,line join=bevel,line cap=rect,line width=0.800pt]
  \end{scope}
  \begin{scope}[cm={{1.00588,0.0,0.0,1.00588,(253.482,165.971)}},draw=black,line join=bevel,line cap=rect,line width=0.800pt]
  \end{scope}
  \begin{scope}[cm={{1.00588,0.0,0.0,1.00588,(253.482,165.971)}},draw=black,line join=bevel,line cap=rect,line width=0.800pt]
  \end{scope}
  \begin{scope}[cm={{1.00588,0.0,0.0,1.00588,(253.482,165.971)}},draw=black,line join=bevel,line cap=rect,line width=0.800pt]
  \end{scope}
  \begin{scope}[cm={{1.00588,0.0,0.0,1.00588,(253.482,165.971)}},draw=black,line join=bevel,line cap=rect,line width=0.800pt]
  \end{scope}
  \begin{scope}[cm={{1.00588,0.0,0.0,1.00588,(253.482,165.971)}},draw=black,line join=bevel,line cap=rect,line width=0.800pt]
  \end{scope}
  \begin{scope}[cm={{1.00588,0.0,0.0,1.00588,(253.482,165.971)}},draw=black,line join=bevel,line cap=rect,line width=0.800pt]
  \end{scope}
  \begin{scope}[scale=1.006,draw=black,line join=bevel,line cap=rect,line width=0.800pt]
  \end{scope}
  \begin{scope}[scale=1.006,draw=black,line join=bevel,line cap=rect,line width=0.800pt]
  \end{scope}
  \begin{scope}[cm={{0.0,-1.00588,1.00588,0.0,(276.618,186.591)}},draw=black,line join=bevel,line cap=rect,line width=0.800pt]
  \end{scope}
  \begin{scope}[cm={{0.0,-1.00588,1.00588,0.0,(276.618,186.591)}},draw=black,line join=bevel,line cap=rect,line width=0.800pt]
  \end{scope}
  \begin{scope}[cm={{0.0,-1.00588,1.00588,0.0,(276.618,186.591)}},draw=black,line join=bevel,line cap=rect,line width=0.800pt]
  \end{scope}
  \begin{scope}[cm={{0.0,-1.00588,1.00588,0.0,(276.618,186.591)}},draw=black,line join=bevel,line cap=rect,line width=0.800pt]
  \end{scope}
  \begin{scope}[cm={{0.0,-1.00588,1.00588,0.0,(276.618,186.591)}},draw=black,line join=bevel,line cap=rect,line width=0.800pt]
  \end{scope}
  \begin{scope}[cm={{0.0,-1.00588,1.00588,0.0,(276.618,186.591)}},draw=black,line join=bevel,line cap=rect,line width=0.800pt]
  \end{scope}
  \begin{scope}[scale=1.006,draw=black,line join=bevel,line cap=rect,line width=0.800pt]
  \end{scope}
  \begin{scope}[scale=1.006,draw=black,line join=bevel,line cap=rect,line width=0.800pt]
  \end{scope}
  \begin{scope}[scale=1.006,draw=black,line join=bevel,line cap=rect,line width=0.800pt]
  \end{scope}
  \begin{scope}[scale=1.006,draw=black,line join=bevel,line cap=rect,line width=0.800pt]
  \end{scope}
  \begin{scope}[scale=1.006,draw=black,line join=bevel,line cap=rect,line width=0.800pt]
  \end{scope}
  \begin{scope}[scale=1.006,draw=black,line join=bevel,line cap=rect,line width=0.800pt]
  \end{scope}
  \begin{scope}[cm={{1.00588,0.0,0.0,1.00588,(18.1059,325.906)}},draw=black,line join=bevel,line cap=rect,line width=0.800pt]
  \end{scope}
  \begin{scope}[cm={{1.00588,0.0,0.0,1.00588,(18.1059,325.906)}},draw=black,line join=bevel,line cap=rect,line width=0.800pt]
  \end{scope}
  \begin{scope}[cm={{1.00588,0.0,0.0,1.00588,(18.1059,325.906)}},draw=black,line join=bevel,line cap=rect,line width=0.800pt]
  \end{scope}
  \begin{scope}[cm={{1.00588,0.0,0.0,1.00588,(18.1059,325.906)}},draw=black,line join=bevel,line cap=rect,line width=0.800pt]
  \end{scope}
  \begin{scope}[cm={{1.00588,0.0,0.0,1.00588,(18.1059,325.906)}},draw=black,line join=bevel,line cap=rect,line width=0.800pt]
  \end{scope}
  \begin{scope}[cm={{1.00588,0.0,0.0,1.00588,(18.1059,325.906)}},draw=black,line join=bevel,line cap=rect,line width=0.800pt]
  \end{scope}
  \begin{scope}[scale=1.006,draw=black,line join=bevel,line cap=rect,line width=0.800pt]
  \end{scope}
  \begin{scope}[scale=1.006,draw=black,line join=bevel,line cap=rect,line width=0.800pt]
  \end{scope}
  \begin{scope}[cm={{1.00588,0.0,0.0,1.00588,(34.2,296.735)}},draw=black,line join=bevel,line cap=rect,line width=0.800pt]
  \end{scope}
  \begin{scope}[cm={{1.00588,0.0,0.0,1.00588,(34.2,296.735)}},draw=black,line join=bevel,line cap=rect,line width=0.800pt]
  \end{scope}
  \begin{scope}[cm={{1.00588,0.0,0.0,1.00588,(34.2,296.735)}},draw=black,line join=bevel,line cap=rect,line width=0.800pt]
  \end{scope}
  \begin{scope}[cm={{1.00588,0.0,0.0,1.00588,(34.2,296.735)}},draw=black,line join=bevel,line cap=rect,line width=0.800pt]
  \end{scope}
  \begin{scope}[cm={{1.00588,0.0,0.0,1.00588,(34.2,296.735)}},draw=black,line join=bevel,line cap=rect,line width=0.800pt]
  \end{scope}
  \begin{scope}[cm={{1.00588,0.0,0.0,1.00588,(34.2,296.735)}},draw=black,line join=bevel,line cap=rect,line width=0.800pt]
  \end{scope}
  \begin{scope}[scale=1.006,draw=black,line join=bevel,line cap=rect,line width=0.800pt]
  \end{scope}
  \begin{scope}[scale=1.006,draw=black,line join=bevel,line cap=rect,line width=0.800pt]
  \end{scope}
  \begin{scope}[cm={{1.00588,0.0,0.0,1.00588,(22.1294,266.559)}},draw=black,line join=bevel,line cap=rect,line width=0.800pt]
  \end{scope}
  \begin{scope}[cm={{1.00588,0.0,0.0,1.00588,(22.1294,266.559)}},draw=black,line join=bevel,line cap=rect,line width=0.800pt]
  \end{scope}
  \begin{scope}[cm={{1.00588,0.0,0.0,1.00588,(22.1294,266.559)}},draw=black,line join=bevel,line cap=rect,line width=0.800pt]
  \end{scope}
  \begin{scope}[cm={{1.00588,0.0,0.0,1.00588,(22.1294,266.559)}},draw=black,line join=bevel,line cap=rect,line width=0.800pt]
  \end{scope}
  \begin{scope}[cm={{1.00588,0.0,0.0,1.00588,(22.1294,266.559)}},draw=black,line join=bevel,line cap=rect,line width=0.800pt]
  \end{scope}
  \begin{scope}[cm={{1.00588,0.0,0.0,1.00588,(22.1294,266.559)}},draw=black,line join=bevel,line cap=rect,line width=0.800pt]
  \end{scope}
  \begin{scope}[scale=1.006,draw=black,line join=bevel,line cap=rect,line width=0.800pt]
  \end{scope}
  \begin{scope}[scale=1.006,draw=black,line join=bevel,line cap=rect,line width=0.800pt]
  \end{scope}
  \begin{scope}[cm={{1.00588,0.0,0.0,1.00588,(22.1294,237.388)}},draw=black,line join=bevel,line cap=rect,line width=0.800pt]
  \end{scope}
  \begin{scope}[cm={{1.00588,0.0,0.0,1.00588,(22.1294,237.388)}},draw=black,line join=bevel,line cap=rect,line width=0.800pt]
  \end{scope}
  \begin{scope}[cm={{1.00588,0.0,0.0,1.00588,(22.1294,237.388)}},draw=black,line join=bevel,line cap=rect,line width=0.800pt]
  \end{scope}
  \begin{scope}[cm={{1.00588,0.0,0.0,1.00588,(22.1294,237.388)}},draw=black,line join=bevel,line cap=rect,line width=0.800pt]
  \end{scope}
  \begin{scope}[cm={{1.00588,0.0,0.0,1.00588,(22.1294,237.388)}},draw=black,line join=bevel,line cap=rect,line width=0.800pt]
  \end{scope}
  \begin{scope}[cm={{1.00588,0.0,0.0,1.00588,(22.1294,237.388)}},draw=black,line join=bevel,line cap=rect,line width=0.800pt]
  \end{scope}
  \begin{scope}[scale=1.006,draw=black,line join=bevel,line cap=rect,line width=0.800pt]
  \end{scope}
  \begin{scope}[scale=1.006,draw=black,line join=bevel,line cap=rect,line width=0.800pt]
  \end{scope}
  \begin{scope}[cm={{1.00588,0.0,0.0,1.00588,(34.2,336.971)}},draw=black,line join=bevel,line cap=rect,line width=0.800pt]
  \end{scope}
  \begin{scope}[cm={{1.00588,0.0,0.0,1.00588,(34.2,336.971)}},draw=black,line join=bevel,line cap=rect,line width=0.800pt]
  \end{scope}
  \begin{scope}[cm={{1.00588,0.0,0.0,1.00588,(34.2,336.971)}},draw=black,line join=bevel,line cap=rect,line width=0.800pt]
  \end{scope}
  \begin{scope}[cm={{1.00588,0.0,0.0,1.00588,(34.2,336.971)}},draw=black,line join=bevel,line cap=rect,line width=0.800pt]
  \end{scope}
  \begin{scope}[cm={{1.00588,0.0,0.0,1.00588,(34.2,336.971)}},draw=black,line join=bevel,line cap=rect,line width=0.800pt]
  \end{scope}
  \begin{scope}[cm={{1.00588,0.0,0.0,1.00588,(34.2,336.971)}},draw=black,line join=bevel,line cap=rect,line width=0.800pt]
  \end{scope}
  \begin{scope}[scale=1.006,draw=black,line join=bevel,line cap=rect,line width=0.800pt]
  \end{scope}
  \begin{scope}[scale=1.006,draw=black,line join=bevel,line cap=rect,line width=0.800pt]
  \end{scope}
  \begin{scope}[cm={{1.00588,0.0,0.0,1.00588,(70.4118,336.971)}},draw=black,line join=bevel,line cap=rect,line width=0.800pt]
  \end{scope}
  \begin{scope}[cm={{1.00588,0.0,0.0,1.00588,(70.4118,336.971)}},draw=black,line join=bevel,line cap=rect,line width=0.800pt]
  \end{scope}
  \begin{scope}[cm={{1.00588,0.0,0.0,1.00588,(70.4118,336.971)}},draw=black,line join=bevel,line cap=rect,line width=0.800pt]
  \end{scope}
  \begin{scope}[cm={{1.00588,0.0,0.0,1.00588,(70.4118,336.971)}},draw=black,line join=bevel,line cap=rect,line width=0.800pt]
  \end{scope}
  \begin{scope}[cm={{1.00588,0.0,0.0,1.00588,(70.4118,336.971)}},draw=black,line join=bevel,line cap=rect,line width=0.800pt]
  \end{scope}
  \begin{scope}[cm={{1.00588,0.0,0.0,1.00588,(70.4118,336.971)}},draw=black,line join=bevel,line cap=rect,line width=0.800pt]
  \end{scope}
  \begin{scope}[scale=1.006,draw=black,line join=bevel,line cap=rect,line width=0.800pt]
  \end{scope}
  \begin{scope}[scale=1.006,draw=black,line join=bevel,line cap=rect,line width=0.800pt]
  \end{scope}
  \begin{scope}[cm={{1.00588,0.0,0.0,1.00588,(105.115,336.971)}},draw=black,line join=bevel,line cap=rect,line width=0.800pt]
  \end{scope}
  \begin{scope}[cm={{1.00588,0.0,0.0,1.00588,(105.115,336.971)}},draw=black,line join=bevel,line cap=rect,line width=0.800pt]
  \end{scope}
  \begin{scope}[cm={{1.00588,0.0,0.0,1.00588,(105.115,336.971)}},draw=black,line join=bevel,line cap=rect,line width=0.800pt]
  \end{scope}
  \begin{scope}[cm={{1.00588,0.0,0.0,1.00588,(105.115,336.971)}},draw=black,line join=bevel,line cap=rect,line width=0.800pt]
  \end{scope}
  \begin{scope}[cm={{1.00588,0.0,0.0,1.00588,(105.115,336.971)}},draw=black,line join=bevel,line cap=rect,line width=0.800pt]
  \end{scope}
  \begin{scope}[cm={{1.00588,0.0,0.0,1.00588,(105.115,336.971)}},draw=black,line join=bevel,line cap=rect,line width=0.800pt]
  \end{scope}
  \begin{scope}[scale=1.006,draw=black,line join=bevel,line cap=rect,line width=0.800pt]
  \end{scope}
  \begin{scope}[scale=1.006,draw=black,line join=bevel,line cap=rect,line width=0.800pt]
  \end{scope}
  \begin{scope}[cm={{1.00588,0.0,0.0,1.00588,(136.297,336.971)}},draw=black,line join=bevel,line cap=rect,line width=0.800pt]
  \end{scope}
  \begin{scope}[cm={{1.00588,0.0,0.0,1.00588,(136.297,336.971)}},draw=black,line join=bevel,line cap=rect,line width=0.800pt]
  \end{scope}
  \begin{scope}[cm={{1.00588,0.0,0.0,1.00588,(136.297,336.971)}},draw=black,line join=bevel,line cap=rect,line width=0.800pt]
  \end{scope}
  \begin{scope}[cm={{1.00588,0.0,0.0,1.00588,(136.297,336.971)}},draw=black,line join=bevel,line cap=rect,line width=0.800pt]
  \end{scope}
  \begin{scope}[cm={{1.00588,0.0,0.0,1.00588,(136.297,336.971)}},draw=black,line join=bevel,line cap=rect,line width=0.800pt]
  \end{scope}
  \begin{scope}[cm={{1.00588,0.0,0.0,1.00588,(136.297,336.971)}},draw=black,line join=bevel,line cap=rect,line width=0.800pt]
  \end{scope}
  \begin{scope}[scale=1.006,draw=black,line join=bevel,line cap=rect,line width=0.800pt]
  \end{scope}
  \begin{scope}[scale=1.006,draw=black,line join=bevel,line cap=rect,line width=0.800pt]
  \end{scope}
  \begin{scope}[scale=1.006,draw=black,line join=bevel,line cap=rect,line width=0.800pt]
  \end{scope}
  \begin{scope}[scale=1.006,draw=black,line join=bevel,line cap=rect,line width=0.800pt]
  \end{scope}
  \begin{scope}[scale=1.006,draw=black,line join=bevel,line cap=rect,line width=0.800pt]
  \end{scope}
  \begin{scope}[scale=1.006,draw=black,line join=bevel,line cap=rect,line width=0.800pt]
  \end{scope}
  \begin{scope}[cm={{1.00588,0.0,0.0,1.00588,(161.947,237.388)}},draw=black,line join=bevel,line cap=rect,line width=0.800pt]
  \end{scope}
  \begin{scope}[cm={{1.00588,0.0,0.0,1.00588,(161.947,237.388)}},draw=black,line join=bevel,line cap=rect,line width=0.800pt]
  \end{scope}
  \begin{scope}[cm={{1.00588,0.0,0.0,1.00588,(161.947,237.388)}},draw=black,line join=bevel,line cap=rect,line width=0.800pt]
  \end{scope}
  \begin{scope}[cm={{1.00588,0.0,0.0,1.00588,(161.947,237.388)}},draw=black,line join=bevel,line cap=rect,line width=0.800pt]
  \end{scope}
  \begin{scope}[cm={{1.00588,0.0,0.0,1.00588,(161.947,237.388)}},draw=black,line join=bevel,line cap=rect,line width=0.800pt]
  \end{scope}
  \begin{scope}[cm={{1.00588,0.0,0.0,1.00588,(161.947,237.388)}},draw=black,line join=bevel,line cap=rect,line width=0.800pt]
  \end{scope}
  \begin{scope}[scale=1.006,draw=black,line join=bevel,line cap=rect,line width=0.800pt]
  \end{scope}
  \begin{scope}[scale=1.006,draw=black,line join=bevel,line cap=rect,line width=0.800pt]
  \end{scope}
  \begin{scope}[scale=1.006,draw=black,line join=bevel,line cap=rect,line width=0.800pt]
  \end{scope}
  \begin{scope}[scale=1.006,draw=black,line join=bevel,line cap=rect,line width=0.800pt]
  \end{scope}
  \begin{scope}[scale=1.006,draw=black,line join=bevel,line cap=rect,line width=0.800pt]
  \end{scope}
  \begin{scope}[scale=1.006,draw=black,line join=bevel,line cap=rect,line width=0.800pt]
  \end{scope}
  \begin{scope}[cm={{1.00588,0.0,0.0,1.00588,(181.059,268.571)}},draw=black,line join=bevel,line cap=rect,line width=0.800pt]
  \end{scope}
  \begin{scope}[cm={{1.00588,0.0,0.0,1.00588,(181.059,268.571)}},draw=black,line join=bevel,line cap=rect,line width=0.800pt]
  \end{scope}
  \begin{scope}[cm={{1.00588,0.0,0.0,1.00588,(181.059,268.571)}},draw=black,line join=bevel,line cap=rect,line width=0.800pt]
  \end{scope}
  \begin{scope}[cm={{1.00588,0.0,0.0,1.00588,(181.059,268.571)}},draw=black,line join=bevel,line cap=rect,line width=0.800pt]
  \end{scope}
  \begin{scope}[cm={{1.00588,0.0,0.0,1.00588,(181.059,268.571)}},draw=black,line join=bevel,line cap=rect,line width=0.800pt]
  \end{scope}
  \begin{scope}[cm={{1.00588,0.0,0.0,1.00588,(181.059,268.571)}},draw=black,line join=bevel,line cap=rect,line width=0.800pt]
  \end{scope}
  \begin{scope}[scale=1.006,draw=black,line join=bevel,line cap=rect,line width=0.800pt]
  \end{scope}
  \begin{scope}[scale=1.006,draw=black,line join=bevel,line cap=rect,line width=0.800pt]
  \end{scope}
  \begin{scope}[cm={{1.00588,0.0,0.0,1.00588,(181.059,249.459)}},draw=black,line join=bevel,line cap=rect,line width=0.800pt]
  \end{scope}
  \begin{scope}[cm={{1.00588,0.0,0.0,1.00588,(181.059,249.459)}},draw=black,line join=bevel,line cap=rect,line width=0.800pt]
  \end{scope}
  \begin{scope}[cm={{1.00588,0.0,0.0,1.00588,(181.059,249.459)}},draw=black,line join=bevel,line cap=rect,line width=0.800pt]
  \end{scope}
  \begin{scope}[cm={{1.00588,0.0,0.0,1.00588,(181.059,249.459)}},draw=black,line join=bevel,line cap=rect,line width=0.800pt]
  \end{scope}
  \begin{scope}[cm={{1.00588,0.0,0.0,1.00588,(181.059,249.459)}},draw=black,line join=bevel,line cap=rect,line width=0.800pt]
  \end{scope}
  \begin{scope}[cm={{1.00588,0.0,0.0,1.00588,(181.059,249.459)}},draw=black,line join=bevel,line cap=rect,line width=0.800pt]
  \end{scope}
  \begin{scope}[scale=1.006,draw=black,line join=bevel,line cap=rect,line width=0.800pt]
  \end{scope}
  \begin{scope}[scale=1.006,draw=black,line join=bevel,line cap=rect,line width=0.800pt]
  \end{scope}
  \begin{scope}[cm={{1.00588,0.0,0.0,1.00588,(181.059,230.347)}},draw=black,line join=bevel,line cap=rect,line width=0.800pt]
  \end{scope}
  \begin{scope}[cm={{1.00588,0.0,0.0,1.00588,(181.059,230.347)}},draw=black,line join=bevel,line cap=rect,line width=0.800pt]
  \end{scope}
  \begin{scope}[cm={{1.00588,0.0,0.0,1.00588,(181.059,230.347)}},draw=black,line join=bevel,line cap=rect,line width=0.800pt]
  \end{scope}
  \begin{scope}[cm={{1.00588,0.0,0.0,1.00588,(181.059,230.347)}},draw=black,line join=bevel,line cap=rect,line width=0.800pt]
  \end{scope}
  \begin{scope}[cm={{1.00588,0.0,0.0,1.00588,(181.059,230.347)}},draw=black,line join=bevel,line cap=rect,line width=0.800pt]
  \end{scope}
  \begin{scope}[cm={{1.00588,0.0,0.0,1.00588,(181.059,230.347)}},draw=black,line join=bevel,line cap=rect,line width=0.800pt]
  \end{scope}
  \begin{scope}[scale=1.006,draw=black,line join=bevel,line cap=rect,line width=0.800pt]
  \end{scope}
  \begin{scope}[scale=1.006,draw=black,line join=bevel,line cap=rect,line width=0.800pt]
  \end{scope}
  \begin{scope}[cm={{1.00588,0.0,0.0,1.00588,(186.088,284.665)}},draw=black,line join=bevel,line cap=rect,line width=0.800pt]
  \end{scope}
  \begin{scope}[cm={{1.00588,0.0,0.0,1.00588,(186.088,284.665)}},draw=black,line join=bevel,line cap=rect,line width=0.800pt]
  \end{scope}
  \begin{scope}[cm={{1.00588,0.0,0.0,1.00588,(186.088,284.665)}},draw=black,line join=bevel,line cap=rect,line width=0.800pt]
  \end{scope}
  \begin{scope}[cm={{1.00588,0.0,0.0,1.00588,(186.088,284.665)}},draw=black,line join=bevel,line cap=rect,line width=0.800pt]
  \end{scope}
  \begin{scope}[cm={{1.00588,0.0,0.0,1.00588,(186.088,284.665)}},draw=black,line join=bevel,line cap=rect,line width=0.800pt]
  \end{scope}
  \begin{scope}[cm={{1.00588,0.0,0.0,1.00588,(186.088,284.665)}},draw=black,line join=bevel,line cap=rect,line width=0.800pt]
  \end{scope}
  \begin{scope}[scale=1.006,draw=black,line join=bevel,line cap=rect,line width=0.800pt]
  \end{scope}
  \begin{scope}[scale=1.006,draw=black,line join=bevel,line cap=rect,line width=0.800pt]
  \end{scope}
  \begin{scope}[cm={{1.00588,0.0,0.0,1.00588,(203.188,284.665)}},draw=black,line join=bevel,line cap=rect,line width=0.800pt]
  \end{scope}
  \begin{scope}[cm={{1.00588,0.0,0.0,1.00588,(203.188,284.665)}},draw=black,line join=bevel,line cap=rect,line width=0.800pt]
  \end{scope}
  \begin{scope}[cm={{1.00588,0.0,0.0,1.00588,(203.188,284.665)}},draw=black,line join=bevel,line cap=rect,line width=0.800pt]
  \end{scope}
  \begin{scope}[cm={{1.00588,0.0,0.0,1.00588,(203.188,284.665)}},draw=black,line join=bevel,line cap=rect,line width=0.800pt]
  \end{scope}
  \begin{scope}[cm={{1.00588,0.0,0.0,1.00588,(203.188,284.665)}},draw=black,line join=bevel,line cap=rect,line width=0.800pt]
  \end{scope}
  \begin{scope}[cm={{1.00588,0.0,0.0,1.00588,(203.188,284.665)}},draw=black,line join=bevel,line cap=rect,line width=0.800pt]
  \end{scope}
  \begin{scope}[scale=1.006,draw=black,line join=bevel,line cap=rect,line width=0.800pt]
  \end{scope}
  \begin{scope}[scale=1.006,draw=black,line join=bevel,line cap=rect,line width=0.800pt]
  \end{scope}
  \begin{scope}[cm={{1.00588,0.0,0.0,1.00588,(219.282,284.665)}},draw=black,line join=bevel,line cap=rect,line width=0.800pt]
  \end{scope}
  \begin{scope}[cm={{1.00588,0.0,0.0,1.00588,(219.282,284.665)}},draw=black,line join=bevel,line cap=rect,line width=0.800pt]
  \end{scope}
  \begin{scope}[cm={{1.00588,0.0,0.0,1.00588,(219.282,284.665)}},draw=black,line join=bevel,line cap=rect,line width=0.800pt]
  \end{scope}
  \begin{scope}[cm={{1.00588,0.0,0.0,1.00588,(219.282,284.665)}},draw=black,line join=bevel,line cap=rect,line width=0.800pt]
  \end{scope}
  \begin{scope}[cm={{1.00588,0.0,0.0,1.00588,(219.282,284.665)}},draw=black,line join=bevel,line cap=rect,line width=0.800pt]
  \end{scope}
  \begin{scope}[cm={{1.00588,0.0,0.0,1.00588,(219.282,284.665)}},draw=black,line join=bevel,line cap=rect,line width=0.800pt]
  \end{scope}
  \begin{scope}[scale=1.006,draw=black,line join=bevel,line cap=rect,line width=0.800pt]
  \end{scope}
  \begin{scope}[scale=1.006,draw=black,line join=bevel,line cap=rect,line width=0.800pt]
  \end{scope}
  \begin{scope}[cm={{1.00588,0.0,0.0,1.00588,(236.382,284.665)}},draw=black,line join=bevel,line cap=rect,line width=0.800pt]
  \end{scope}
  \begin{scope}[cm={{1.00588,0.0,0.0,1.00588,(236.382,284.665)}},draw=black,line join=bevel,line cap=rect,line width=0.800pt]
  \end{scope}
  \begin{scope}[cm={{1.00588,0.0,0.0,1.00588,(236.382,284.665)}},draw=black,line join=bevel,line cap=rect,line width=0.800pt]
  \end{scope}
  \begin{scope}[cm={{1.00588,0.0,0.0,1.00588,(236.382,284.665)}},draw=black,line join=bevel,line cap=rect,line width=0.800pt]
  \end{scope}
  \begin{scope}[cm={{1.00588,0.0,0.0,1.00588,(236.382,284.665)}},draw=black,line join=bevel,line cap=rect,line width=0.800pt]
  \end{scope}
  \begin{scope}[cm={{1.00588,0.0,0.0,1.00588,(236.382,284.665)}},draw=black,line join=bevel,line cap=rect,line width=0.800pt]
  \end{scope}
  \begin{scope}[scale=1.006,draw=black,line join=bevel,line cap=rect,line width=0.800pt]
  \end{scope}
  \begin{scope}[scale=1.006,draw=black,line join=bevel,line cap=rect,line width=0.800pt]
  \end{scope}
  \begin{scope}[cm={{1.00588,0.0,0.0,1.00588,(249.459,272.594)}},draw=black,line join=bevel,line cap=rect,line width=0.800pt]
  \end{scope}
  \begin{scope}[cm={{1.00588,0.0,0.0,1.00588,(249.459,272.594)}},draw=black,line join=bevel,line cap=rect,line width=0.800pt]
  \end{scope}
  \begin{scope}[cm={{1.00588,0.0,0.0,1.00588,(249.459,272.594)}},draw=black,line join=bevel,line cap=rect,line width=0.800pt]
  \end{scope}
  \begin{scope}[cm={{1.00588,0.0,0.0,1.00588,(249.459,272.594)}},draw=black,line join=bevel,line cap=rect,line width=0.800pt]
  \end{scope}
  \begin{scope}[cm={{1.00588,0.0,0.0,1.00588,(249.459,272.594)}},draw=black,line join=bevel,line cap=rect,line width=0.800pt]
  \end{scope}
  \begin{scope}[cm={{1.00588,0.0,0.0,1.00588,(249.459,272.594)}},draw=black,line join=bevel,line cap=rect,line width=0.800pt]
  \end{scope}
  \begin{scope}[scale=1.006,draw=black,line join=bevel,line cap=rect,line width=0.800pt]
  \end{scope}
  \begin{scope}[scale=1.006,draw=black,line join=bevel,line cap=rect,line width=0.800pt]
  \end{scope}
  \begin{scope}[cm={{1.00588,0.0,0.0,1.00588,(249.459,253.482)}},draw=black,line join=bevel,line cap=rect,line width=0.800pt]
  \end{scope}
  \begin{scope}[cm={{1.00588,0.0,0.0,1.00588,(249.459,253.482)}},draw=black,line join=bevel,line cap=rect,line width=0.800pt]
  \end{scope}
  \begin{scope}[cm={{1.00588,0.0,0.0,1.00588,(249.459,253.482)}},draw=black,line join=bevel,line cap=rect,line width=0.800pt]
  \end{scope}
  \begin{scope}[cm={{1.00588,0.0,0.0,1.00588,(249.459,253.482)}},draw=black,line join=bevel,line cap=rect,line width=0.800pt]
  \end{scope}
  \begin{scope}[cm={{1.00588,0.0,0.0,1.00588,(249.459,253.482)}},draw=black,line join=bevel,line cap=rect,line width=0.800pt]
  \end{scope}
  \begin{scope}[cm={{1.00588,0.0,0.0,1.00588,(249.459,253.482)}},draw=black,line join=bevel,line cap=rect,line width=0.800pt]
  \end{scope}
  \begin{scope}[scale=1.006,draw=black,line join=bevel,line cap=rect,line width=0.800pt]
  \end{scope}
  \begin{scope}[scale=1.006,draw=black,line join=bevel,line cap=rect,line width=0.800pt]
  \end{scope}
  \begin{scope}[cm={{1.00588,0.0,0.0,1.00588,(249.459,234.371)}},draw=black,line join=bevel,line cap=rect,line width=0.800pt]
  \end{scope}
  \begin{scope}[cm={{1.00588,0.0,0.0,1.00588,(249.459,234.371)}},draw=black,line join=bevel,line cap=rect,line width=0.800pt]
  \end{scope}
  \begin{scope}[cm={{1.00588,0.0,0.0,1.00588,(249.459,234.371)}},draw=black,line join=bevel,line cap=rect,line width=0.800pt]
  \end{scope}
  \begin{scope}[cm={{1.00588,0.0,0.0,1.00588,(249.459,234.371)}},draw=black,line join=bevel,line cap=rect,line width=0.800pt]
  \end{scope}
  \begin{scope}[cm={{1.00588,0.0,0.0,1.00588,(249.459,234.371)}},draw=black,line join=bevel,line cap=rect,line width=0.800pt]
  \end{scope}
  \begin{scope}[cm={{1.00588,0.0,0.0,1.00588,(249.459,234.371)}},draw=black,line join=bevel,line cap=rect,line width=0.800pt]
  \end{scope}
  \begin{scope}[scale=1.006,draw=black,line join=bevel,line cap=rect,line width=0.800pt]
  \end{scope}
  \begin{scope}[scale=1.006,draw=black,line join=bevel,line cap=rect,line width=0.800pt]
  \end{scope}
  \begin{scope}[cm={{0.0,-1.00588,1.00588,0.0,(276.618,254.991)}},draw=black,line join=bevel,line cap=rect,line width=0.800pt]
  \end{scope}
  \begin{scope}[cm={{0.0,-1.00588,1.00588,0.0,(276.618,254.991)}},draw=black,line join=bevel,line cap=rect,line width=0.800pt]
  \end{scope}
  \begin{scope}[cm={{0.0,-1.00588,1.00588,0.0,(276.618,254.991)}},draw=black,line join=bevel,line cap=rect,line width=0.800pt]
  \end{scope}
  \begin{scope}[cm={{0.0,-1.00588,1.00588,0.0,(276.618,254.991)}},draw=black,line join=bevel,line cap=rect,line width=0.800pt]
  \end{scope}
  \begin{scope}[cm={{0.0,-1.00588,1.00588,0.0,(276.618,254.991)}},draw=black,line join=bevel,line cap=rect,line width=0.800pt]
  \end{scope}
  \begin{scope}[cm={{0.0,-1.00588,1.00588,0.0,(276.618,254.991)}},draw=black,line join=bevel,line cap=rect,line width=0.800pt]
  \end{scope}
  \begin{scope}[scale=1.006,draw=black,line join=bevel,line cap=rect,line width=0.800pt]
  \end{scope}
  \begin{scope}[scale=1.006,draw=black,line join=bevel,line cap=rect,line width=0.800pt]
  \end{scope}
  \begin{scope}[scale=1.006,draw=black,line join=bevel,line cap=rect,line width=0.800pt]
  \end{scope}
  \begin{scope}[scale=1.006,draw=black,line join=bevel,line cap=rect,line width=0.800pt]
  \end{scope}
  \begin{scope}[scale=1.006,draw=black,line join=bevel,line cap=rect,line width=0.800pt]
  \end{scope}
  \begin{scope}[scale=1.006,draw=black,line join=bevel,line cap=rect,line width=0.800pt]
  \end{scope}
  \begin{scope}[cm={{1.00588,0.0,0.0,1.00588,(181.059,317.859)}},draw=black,line join=bevel,line cap=rect,line width=0.800pt]
  \end{scope}
  \begin{scope}[cm={{1.00588,0.0,0.0,1.00588,(181.059,317.859)}},draw=black,line join=bevel,line cap=rect,line width=0.800pt]
  \end{scope}
  \begin{scope}[cm={{1.00588,0.0,0.0,1.00588,(181.059,317.859)}},draw=black,line join=bevel,line cap=rect,line width=0.800pt]
  \end{scope}
  \begin{scope}[cm={{1.00588,0.0,0.0,1.00588,(181.059,317.859)}},draw=black,line join=bevel,line cap=rect,line width=0.800pt]
  \end{scope}
  \begin{scope}[cm={{1.00588,0.0,0.0,1.00588,(181.059,317.859)}},draw=black,line join=bevel,line cap=rect,line width=0.800pt]
  \end{scope}
  \begin{scope}[cm={{1.00588,0.0,0.0,1.00588,(181.059,317.859)}},draw=black,line join=bevel,line cap=rect,line width=0.800pt]
  \end{scope}
  \begin{scope}[scale=1.006,draw=black,line join=bevel,line cap=rect,line width=0.800pt]
  \end{scope}
  \begin{scope}[scale=1.006,draw=black,line join=bevel,line cap=rect,line width=0.800pt]
  \end{scope}
  \begin{scope}[cm={{1.00588,0.0,0.0,1.00588,(181.059,297.741)}},draw=black,line join=bevel,line cap=rect,line width=0.800pt]
  \end{scope}
  \begin{scope}[cm={{1.00588,0.0,0.0,1.00588,(181.059,297.741)}},draw=black,line join=bevel,line cap=rect,line width=0.800pt]
  \end{scope}
  \begin{scope}[cm={{1.00588,0.0,0.0,1.00588,(181.059,297.741)}},draw=black,line join=bevel,line cap=rect,line width=0.800pt]
  \end{scope}
  \begin{scope}[cm={{1.00588,0.0,0.0,1.00588,(181.059,297.741)}},draw=black,line join=bevel,line cap=rect,line width=0.800pt]
  \end{scope}
  \begin{scope}[cm={{1.00588,0.0,0.0,1.00588,(181.059,297.741)}},draw=black,line join=bevel,line cap=rect,line width=0.800pt]
  \end{scope}
  \begin{scope}[cm={{1.00588,0.0,0.0,1.00588,(181.059,297.741)}},draw=black,line join=bevel,line cap=rect,line width=0.800pt]
  \end{scope}
  \begin{scope}[scale=1.006,draw=black,line join=bevel,line cap=rect,line width=0.800pt]
  \end{scope}
  \begin{scope}[scale=1.006,draw=black,line join=bevel,line cap=rect,line width=0.800pt]
  \end{scope}
  \begin{scope}[cm={{1.00588,0.0,0.0,1.00588,(181.059,277.624)}},draw=black,line join=bevel,line cap=rect,line width=0.800pt]
  \end{scope}
  \begin{scope}[cm={{1.00588,0.0,0.0,1.00588,(181.059,277.624)}},draw=black,line join=bevel,line cap=rect,line width=0.800pt]
  \end{scope}
  \begin{scope}[cm={{1.00588,0.0,0.0,1.00588,(181.059,277.624)}},draw=black,line join=bevel,line cap=rect,line width=0.800pt]
  \end{scope}
  \begin{scope}[cm={{1.00588,0.0,0.0,1.00588,(181.059,277.624)}},draw=black,line join=bevel,line cap=rect,line width=0.800pt]
  \end{scope}
  \begin{scope}[cm={{1.00588,0.0,0.0,1.00588,(181.059,277.624)}},draw=black,line join=bevel,line cap=rect,line width=0.800pt]
  \end{scope}
  \begin{scope}[cm={{1.00588,0.0,0.0,1.00588,(181.059,277.624)}},draw=black,line join=bevel,line cap=rect,line width=0.800pt]
  \end{scope}
  \begin{scope}[scale=1.006,draw=black,line join=bevel,line cap=rect,line width=0.800pt]
  \end{scope}
  \begin{scope}[scale=1.006,draw=black,line join=bevel,line cap=rect,line width=0.800pt]
  \end{scope}
  \begin{scope}[cm={{1.00588,0.0,0.0,1.00588,(183.071,336.971)}},draw=black,line join=bevel,line cap=rect,line width=0.800pt]
  \end{scope}
  \begin{scope}[cm={{1.00588,0.0,0.0,1.00588,(183.071,336.971)}},draw=black,line join=bevel,line cap=rect,line width=0.800pt]
  \end{scope}
  \begin{scope}[cm={{1.00588,0.0,0.0,1.00588,(183.071,336.971)}},draw=black,line join=bevel,line cap=rect,line width=0.800pt]
  \end{scope}
  \begin{scope}[cm={{1.00588,0.0,0.0,1.00588,(183.071,336.971)}},draw=black,line join=bevel,line cap=rect,line width=0.800pt]
  \end{scope}
  \begin{scope}[cm={{1.00588,0.0,0.0,1.00588,(183.071,336.971)}},draw=black,line join=bevel,line cap=rect,line width=0.800pt]
  \end{scope}
  \begin{scope}[cm={{1.00588,0.0,0.0,1.00588,(183.071,336.971)}},draw=black,line join=bevel,line cap=rect,line width=0.800pt]
  \end{scope}
  \begin{scope}[scale=1.006,draw=black,line join=bevel,line cap=rect,line width=0.800pt]
  \end{scope}
  \begin{scope}[scale=1.006,draw=black,line join=bevel,line cap=rect,line width=0.800pt]
  \end{scope}
  \begin{scope}[cm={{1.00588,0.0,0.0,1.00588,(200.674,336.971)}},draw=black,line join=bevel,line cap=rect,line width=0.800pt]
  \end{scope}
  \begin{scope}[cm={{1.00588,0.0,0.0,1.00588,(200.674,336.971)}},draw=black,line join=bevel,line cap=rect,line width=0.800pt]
  \end{scope}
  \begin{scope}[cm={{1.00588,0.0,0.0,1.00588,(200.674,336.971)}},draw=black,line join=bevel,line cap=rect,line width=0.800pt]
  \end{scope}
  \begin{scope}[cm={{1.00588,0.0,0.0,1.00588,(200.674,336.971)}},draw=black,line join=bevel,line cap=rect,line width=0.800pt]
  \end{scope}
  \begin{scope}[cm={{1.00588,0.0,0.0,1.00588,(200.674,336.971)}},draw=black,line join=bevel,line cap=rect,line width=0.800pt]
  \end{scope}
  \begin{scope}[cm={{1.00588,0.0,0.0,1.00588,(200.674,336.971)}},draw=black,line join=bevel,line cap=rect,line width=0.800pt]
  \end{scope}
  \begin{scope}[scale=1.006,draw=black,line join=bevel,line cap=rect,line width=0.800pt]
  \end{scope}
  \begin{scope}[scale=1.006,draw=black,line join=bevel,line cap=rect,line width=0.800pt]
  \end{scope}
  \begin{scope}[cm={{1.00588,0.0,0.0,1.00588,(216.265,336.971)}},draw=black,line join=bevel,line cap=rect,line width=0.800pt]
  \end{scope}
  \begin{scope}[cm={{1.00588,0.0,0.0,1.00588,(216.265,336.971)}},draw=black,line join=bevel,line cap=rect,line width=0.800pt]
  \end{scope}
  \begin{scope}[cm={{1.00588,0.0,0.0,1.00588,(216.265,336.971)}},draw=black,line join=bevel,line cap=rect,line width=0.800pt]
  \end{scope}
  \begin{scope}[cm={{1.00588,0.0,0.0,1.00588,(216.265,336.971)}},draw=black,line join=bevel,line cap=rect,line width=0.800pt]
  \end{scope}
  \begin{scope}[cm={{1.00588,0.0,0.0,1.00588,(216.265,336.971)}},draw=black,line join=bevel,line cap=rect,line width=0.800pt]
  \end{scope}
  \begin{scope}[cm={{1.00588,0.0,0.0,1.00588,(216.265,336.971)}},draw=black,line join=bevel,line cap=rect,line width=0.800pt]
  \end{scope}
  \begin{scope}[scale=1.006,draw=black,line join=bevel,line cap=rect,line width=0.800pt]
  \end{scope}
  \begin{scope}[scale=1.006,draw=black,line join=bevel,line cap=rect,line width=0.800pt]
  \end{scope}
  \begin{scope}[cm={{1.00588,0.0,0.0,1.00588,(233.365,336.971)}},draw=black,line join=bevel,line cap=rect,line width=0.800pt]
  \end{scope}
  \begin{scope}[cm={{1.00588,0.0,0.0,1.00588,(233.365,336.971)}},draw=black,line join=bevel,line cap=rect,line width=0.800pt]
  \end{scope}
  \begin{scope}[cm={{1.00588,0.0,0.0,1.00588,(233.365,336.971)}},draw=black,line join=bevel,line cap=rect,line width=0.800pt]
  \end{scope}
  \begin{scope}[cm={{1.00588,0.0,0.0,1.00588,(233.365,336.971)}},draw=black,line join=bevel,line cap=rect,line width=0.800pt]
  \end{scope}
  \begin{scope}[cm={{1.00588,0.0,0.0,1.00588,(233.365,336.971)}},draw=black,line join=bevel,line cap=rect,line width=0.800pt]
  \end{scope}
  \begin{scope}[cm={{1.00588,0.0,0.0,1.00588,(233.365,336.971)}},draw=black,line join=bevel,line cap=rect,line width=0.800pt]
  \end{scope}
  \begin{scope}[scale=1.006,draw=black,line join=bevel,line cap=rect,line width=0.800pt]
  \end{scope}
  \begin{scope}[scale=1.006,draw=black,line join=bevel,line cap=rect,line width=0.800pt]
  \end{scope}
  \begin{scope}[cm={{1.00588,0.0,0.0,1.00588,(253.482,321.882)}},draw=black,line join=bevel,line cap=rect,line width=0.800pt]
  \end{scope}
  \begin{scope}[cm={{1.00588,0.0,0.0,1.00588,(253.482,321.882)}},draw=black,line join=bevel,line cap=rect,line width=0.800pt]
  \end{scope}
  \begin{scope}[cm={{1.00588,0.0,0.0,1.00588,(253.482,321.882)}},draw=black,line join=bevel,line cap=rect,line width=0.800pt]
  \end{scope}
  \begin{scope}[cm={{1.00588,0.0,0.0,1.00588,(253.482,321.882)}},draw=black,line join=bevel,line cap=rect,line width=0.800pt]
  \end{scope}
  \begin{scope}[cm={{1.00588,0.0,0.0,1.00588,(253.482,321.882)}},draw=black,line join=bevel,line cap=rect,line width=0.800pt]
  \end{scope}
  \begin{scope}[cm={{1.00588,0.0,0.0,1.00588,(253.482,321.882)}},draw=black,line join=bevel,line cap=rect,line width=0.800pt]
  \end{scope}
  \begin{scope}[scale=1.006,draw=black,line join=bevel,line cap=rect,line width=0.800pt]
  \end{scope}
  \begin{scope}[scale=1.006,draw=black,line join=bevel,line cap=rect,line width=0.800pt]
  \end{scope}
  \begin{scope}[cm={{1.00588,0.0,0.0,1.00588,(253.482,301.765)}},draw=black,line join=bevel,line cap=rect,line width=0.800pt]
  \end{scope}
  \begin{scope}[cm={{1.00588,0.0,0.0,1.00588,(253.482,301.765)}},draw=black,line join=bevel,line cap=rect,line width=0.800pt]
  \end{scope}
  \begin{scope}[cm={{1.00588,0.0,0.0,1.00588,(253.482,301.765)}},draw=black,line join=bevel,line cap=rect,line width=0.800pt]
  \end{scope}
  \begin{scope}[cm={{1.00588,0.0,0.0,1.00588,(253.482,301.765)}},draw=black,line join=bevel,line cap=rect,line width=0.800pt]
  \end{scope}
  \begin{scope}[cm={{1.00588,0.0,0.0,1.00588,(253.482,301.765)}},draw=black,line join=bevel,line cap=rect,line width=0.800pt]
  \end{scope}
  \begin{scope}[cm={{1.00588,0.0,0.0,1.00588,(253.482,301.765)}},draw=black,line join=bevel,line cap=rect,line width=0.800pt]
  \end{scope}
  \begin{scope}[scale=1.006,draw=black,line join=bevel,line cap=rect,line width=0.800pt]
  \end{scope}
  \begin{scope}[scale=1.006,draw=black,line join=bevel,line cap=rect,line width=0.800pt]
  \end{scope}
  \begin{scope}[cm={{1.00588,0.0,0.0,1.00588,(253.482,281.647)}},draw=black,line join=bevel,line cap=rect,line width=0.800pt]
  \end{scope}
  \begin{scope}[cm={{1.00588,0.0,0.0,1.00588,(253.482,281.647)}},draw=black,line join=bevel,line cap=rect,line width=0.800pt]
  \end{scope}
  \begin{scope}[cm={{1.00588,0.0,0.0,1.00588,(253.482,281.647)}},draw=black,line join=bevel,line cap=rect,line width=0.800pt]
  \end{scope}
  \begin{scope}[cm={{1.00588,0.0,0.0,1.00588,(253.482,281.647)}},draw=black,line join=bevel,line cap=rect,line width=0.800pt]
  \end{scope}
  \begin{scope}[cm={{1.00588,0.0,0.0,1.00588,(253.482,281.647)}},draw=black,line join=bevel,line cap=rect,line width=0.800pt]
  \end{scope}
  \begin{scope}[cm={{1.00588,0.0,0.0,1.00588,(253.482,281.647)}},draw=black,line join=bevel,line cap=rect,line width=0.800pt]
  \end{scope}
  \begin{scope}[scale=1.006,draw=black,line join=bevel,line cap=rect,line width=0.800pt]
  \end{scope}
  \begin{scope}[scale=1.006,draw=black,line join=bevel,line cap=rect,line width=0.800pt]
  \end{scope}
  \begin{scope}[cm={{0.0,-1.00588,1.00588,0.0,(276.618,303.274)}},draw=black,line join=bevel,line cap=rect,line width=0.800pt]
  \end{scope}
  \begin{scope}[cm={{0.0,-1.00588,1.00588,0.0,(276.618,303.274)}},draw=black,line join=bevel,line cap=rect,line width=0.800pt]
  \end{scope}
  \begin{scope}[cm={{0.0,-1.00588,1.00588,0.0,(276.618,303.274)}},draw=black,line join=bevel,line cap=rect,line width=0.800pt]
  \end{scope}
  \begin{scope}[cm={{0.0,-1.00588,1.00588,0.0,(276.618,303.274)}},draw=black,line join=bevel,line cap=rect,line width=0.800pt]
  \end{scope}
  \begin{scope}[cm={{0.0,-1.00588,1.00588,0.0,(276.618,303.274)}},draw=black,line join=bevel,line cap=rect,line width=0.800pt]
  \end{scope}
  \begin{scope}[cm={{0.0,-1.00588,1.00588,0.0,(276.618,303.274)}},draw=black,line join=bevel,line cap=rect,line width=0.800pt]
  \end{scope}
  \begin{scope}[scale=1.006,draw=black,line join=bevel,line cap=rect,line width=0.800pt]
  \end{scope}
  \begin{scope}[scale=1.006,draw=black,line join=bevel,line cap=rect,line width=0.800pt]
  \end{scope}
  \begin{scope}[scale=1.006,draw=black,line join=bevel,line cap=rect,line width=0.800pt]
  \end{scope}
  \begin{scope}[scale=1.006,draw=black,line join=bevel,line cap=rect,line width=0.800pt]
  \end{scope}
  \begin{scope}[draw=black,line join=bevel,line cap=rect,line width=0.800pt]
  \end{scope}
  \path[fill=cebebeb,line join=round,even odd rule,line width=0.861pt,rounded corners=0.0000cm] (17.7758,108.3847) rectangle (308.3198,224.7745);



  \path[fill=cebebeb,line join=round,even odd rule,line width=0.612pt,rounded corners=0.0000cm] (146.6215,4.6986) rectangle (308.3218,110.3459);



  \begin{scope}[cm={{1.00588,0.0,0.0,1.00588,(0.01175,0.51144)}},draw=ca0a0a4,dash pattern=on 0.40pt off 0.80pt,line join=round,line cap=round,line width=0.400pt]
    \path[draw] (44.5000,102.5000) -- (142.5000,102.5000);



  \end{scope}
  \begin{scope}[cm={{1.00588,0.0,0.0,1.00588,(0.01175,0.51144)}},draw=black,line join=round,line cap=round,line width=0.480pt]
    \path[draw] (44.5000,102.5000) -- (48.5000,102.5000);



    \path[draw] (142.5000,102.5000) -- (139.5000,102.5000);



  \end{scope}
  \begin{scope}[cm={{1.00588,0.0,0.0,1.00588,(23.47712,107.13543)}},draw=black,line join=bevel,line cap=rect,line width=0.800pt]
    \path[fill=black] (0.0000,0.0000) node[above right] (text34-9) {-100};



  \end{scope}
  \begin{scope}[cm={{1.00588,0.0,0.0,1.00588,(0.01175,0.51144)}},draw=ca0a0a4,dash pattern=on 0.40pt off 0.80pt,line join=round,line cap=round,line width=0.400pt]
    \path[draw] (44.5000,79.5000) -- (142.5000,79.5000);



  \end{scope}
  \begin{scope}[cm={{1.00588,0.0,0.0,1.00588,(0.01175,0.51144)}},draw=black,line join=round,line cap=round,line width=0.480pt]
    \path[draw] (44.5000,79.5000) -- (48.5000,79.5000);



    \path[draw] (142.5000,79.5000) -- (139.5000,79.5000);



  \end{scope}
  \begin{scope}[cm={{1.00588,0.0,0.0,1.00588,(34.21175,85.00554)}},draw=black,line join=bevel,line cap=rect,line width=0.800pt]
    \path[fill=black] (0.0000,0.0000) node[above right] (text64-6) {0};



  \end{scope}
  \begin{scope}[cm={{1.00588,0.0,0.0,1.00588,(0.01175,0.51144)}},draw=ca0a0a4,dash pattern=on 0.40pt off 0.80pt,line join=round,line cap=round,line width=0.400pt]
    \path[draw] (44.5000,57.5000) -- (142.5000,57.5000);



  \end{scope}
  \begin{scope}[cm={{1.00588,0.0,0.0,1.00588,(0.01175,0.51144)}},draw=black,line join=round,line cap=round,line width=0.480pt]
    \path[draw] (44.5000,57.5000) -- (48.5000,57.5000);



    \path[draw] (142.5000,57.5000) -- (139.5000,57.5000);



  \end{scope}
  \begin{scope}[cm={{1.00588,0.0,0.0,1.00588,(26.16471,62.87615)}},draw=black,line join=bevel,line cap=rect,line width=0.800pt]
    \path[fill=black] (0.0000,0.0000) node[above right] (text94-7) {100};



  \end{scope}
  \begin{scope}[cm={{1.00588,0.0,0.0,1.00588,(0.01175,0.51144)}},draw=ca0a0a4,dash pattern=on 0.40pt off 0.80pt,line join=round,line cap=round,line width=0.400pt]
    \path[draw] (44.5000,35.5000) -- (142.5000,35.5000);



  \end{scope}
  \begin{scope}[cm={{1.00588,0.0,0.0,1.00588,(0.01175,0.51144)}},draw=black,line join=round,line cap=round,line width=0.480pt]
    \path[draw] (44.5000,35.5000) -- (48.5000,35.5000);



    \path[draw] (142.5000,35.5000) -- (139.5000,35.5000);



  \end{scope}
  \begin{scope}[cm={{1.00588,0.0,0.0,1.00588,(26.16471,40.74674)}},draw=black,line join=bevel,line cap=rect,line width=0.800pt]
    \path[fill=black] (0.0000,0.0000) node[above right] (text124-9) {200};



  \end{scope}
  \begin{scope}[cm={{1.00588,0.0,0.0,1.00588,(0.01175,0.51144)}},draw=ca0a0a4,dash pattern=on 0.40pt off 0.80pt,line join=round,line cap=round,line width=0.400pt]
    \path[draw] (44.5000,13.5000) -- (142.5000,13.5000);



  \end{scope}
  \begin{scope}[cm={{1.00588,0.0,0.0,1.00588,(0.01175,0.51144)}},draw=black,line join=round,line cap=round,line width=0.480pt]
    \path[draw] (44.5000,13.5000) -- (48.5000,13.5000);



    \path[draw] (142.5000,13.5000) -- (139.5000,13.5000);



  \end{scope}
  \begin{scope}[cm={{1.00588,0.0,0.0,1.00588,(26.16471,18.61735)}},draw=black,line join=bevel,line cap=rect,line width=0.800pt]
    \path[fill=black] (0.0000,0.0000) node[above right] (text154) {300};



  \end{scope}
  \begin{scope}[cm={{1.00588,0.0,0.0,1.00588,(0.01175,0.51144)}},draw=ca0a0a4,dash pattern=on 0.40pt off 0.80pt,line join=round,line cap=round,line width=0.400pt]
    \path[draw] (44.5000,102.5000) -- (44.5000,13.5000);



  \end{scope}
  \begin{scope}[cm={{1.00588,0.0,0.0,1.00588,(0.01175,0.51144)}},draw=black,line join=round,line cap=round,line width=0.480pt]
    \path[draw] (44.5000,102.5000) -- (44.5000,99.5000);



    \path[draw] (44.5000,13.5000) -- (44.5000,16.5000);



  \end{scope}
  \begin{scope}[cm={{1.00588,0.0,0.0,1.00588,(0.01175,0.51144)}},draw=ca0a0a4,dash pattern=on 0.40pt off 0.80pt,line join=round,line cap=round,line width=0.400pt]
    \path[draw] (69.5000,102.5000) -- (69.5000,27.5000);



    \path[draw] (69.5000,19.5000) -- (69.5000,13.5000);



  \end{scope}
  \begin{scope}[cm={{1.00588,0.0,0.0,1.00588,(0.01175,0.51144)}},draw=black,line join=round,line cap=round,line width=0.480pt]
    \path[draw] (69.5000,102.5000) -- (69.5000,99.5000);



    \path[draw] (69.5000,13.5000) -- (69.5000,16.5000);



  \end{scope}
  \begin{scope}[cm={{1.00588,0.0,0.0,1.00588,(0.01175,0.51144)}},draw=ca0a0a4,dash pattern=on 0.40pt off 0.80pt,line join=round,line cap=round,line width=0.400pt]
    \path[draw] (93.5000,102.5000) -- (93.5000,27.5000);



    \path[draw] (93.5000,19.5000) -- (93.5000,13.5000);



  \end{scope}
  \begin{scope}[cm={{1.00588,0.0,0.0,1.00588,(0.01175,0.51144)}},draw=black,line join=round,line cap=round,line width=0.480pt]
    \path[draw] (93.5000,102.5000) -- (93.5000,99.5000);



    \path[draw] (93.5000,13.5000) -- (93.5000,16.5000);



  \end{scope}
  \begin{scope}[cm={{1.00588,0.0,0.0,1.00588,(0.01175,0.51144)}},draw=ca0a0a4,dash pattern=on 0.40pt off 0.80pt,line join=round,line cap=round,line width=0.400pt]
    \path[draw] (118.5000,102.5000) -- (118.5000,13.5000);



  \end{scope}
  \begin{scope}[cm={{1.00588,0.0,0.0,1.00588,(0.01175,0.51144)}},draw=black,line join=round,line cap=round,line width=0.480pt]
    \path[draw] (118.5000,102.5000) -- (118.5000,99.5000);



    \path[draw] (118.5000,13.5000) -- (118.5000,16.5000);



  \end{scope}
  \begin{scope}[cm={{1.00588,0.0,0.0,1.00588,(0.01175,0.51144)}},draw=ca0a0a4,dash pattern=on 0.40pt off 0.80pt,line join=round,line cap=round,line width=0.400pt]
    \path[draw] (142.5000,102.5000) -- (142.5000,13.5000);



  \end{scope}
  \begin{scope}[cm={{1.00588,0.0,0.0,1.00588,(0.01175,0.51144)}},draw=black,line join=round,line cap=round,line width=0.480pt]
    \path[draw] (142.5000,102.5000) -- (142.5000,99.5000);



    \path[draw] (142.5000,13.5000) -- (142.5000,16.5000);



  \end{scope}
  \begin{scope}[cm={{1.00588,0.0,0.0,1.00588,(0.01175,0.51144)}},draw=black,line join=round,line cap=round,line width=0.480pt]
    \path[draw] (44.5000,13.5000) -- (44.5000,102.5000) -- (142.5000,102.5000) -- (142.5000,13.5000) -- (44.5000,13.5000);



  \end{scope}
  \begin{scope}[cm={{1.00588,0.0,0.0,1.00588,(0.01175,0.51144)}},fill=cffffff]
    \path[fill,rounded corners=0.0000cm] (123.0000,18.0000) rectangle (134.0000,34.0000);



  \end{scope}
  \begin{scope}[cm={{1.00588,0.0,0.0,1.00588,(0.01175,0.51144)}},draw=black,line join=round,line cap=round,line width=0.800pt]
    \path[draw] (122.5000,34.5000) -- (122.5000,18.5000) -- (133.5000,18.5000) -- (133.5000,34.5000) -- (122.5000,34.5000);



  \end{scope}
  \begin{scope}[cm={{1.00588,0.0,0.0,1.00588,(127.40468,30.47064)}},draw=black,line join=bevel,line cap=rect,line width=0.800pt]
    \path[fill=black] (0.0000,0.0000) node[above right] (text328) {\label{fig:trajs-I-static}I};



  \end{scope}
  \begin{scope}[cm={{0.0,-1.00588,1.00588,0.0,(15.59415,192.08743)}},draw=black,line join=bevel,line cap=rect,line width=0.800pt]
    \path[fill=black] (0.0000,0.0000) node[above right] (text344) {\rotatebox{90}{y (m)}};



  \end{scope}
  \begin{scope}[cm={{1.00588,0.0,0.0,1.00588,(53.80585,28.67025)}},draw=black,line join=bevel,line cap=rect,line width=0.800pt]
    \path[fill=black] (0.0000,0.0000) node[above right] (text360) {$\mathbf{p}(t)$};



  \end{scope}
  \begin{scope}[cm={{1.00588,0.0,0.0,1.00588,(0.01175,0.51144)}},draw=black,line join=round,line cap=round,line width=0.480pt]
    \path[draw,even odd rule] (74.5000,23.5000) -- (101.5000,23.5000);



  \end{scope}
  \begin{scope}[cm={{1.00588,0.0,0.0,1.00588,(0.01175,0.51144)}},draw=black,line join=round,line cap=round,line width=0.480pt]
    \path[draw] (57.5000,32.8000) -- (57.5000,32.8000) -- (58.3000,35.4000) -- (57.8000,37.2000) -- (56.3000,38.9000) -- (54.9000,41.1000) -- (53.9000,43.5000) -- (53.5000,46.1000) -- (53.4000,48.6000) -- (53.4000,51.1000) -- (53.4000,53.6000) -- (53.4000,56.1000) -- (53.4000,58.6000) -- (53.4000,61.1000) -- (53.4000,63.6000) -- (53.4000,66.1000) -- (53.4000,68.7000) -- (53.4000,71.2000) -- (53.4000,73.7000) -- (53.4000,76.2000) -- (53.4000,78.7000) -- (53.9000,81.3000) -- (54.7000,83.7000) -- (56.0000,86.0000) -- (57.6000,88.1000) -- (59.5000,89.9000) -- (61.7000,91.4000) -- (64.1000,92.6000) -- (66.6000,93.4000) -- (69.2000,93.9000) -- (71.8000,94.1000) -- (74.4000,93.9000) -- (77.0000,93.4000) -- (79.5000,92.6000) -- (81.8000,91.4000) -- (84.0000,90.0000) -- (85.9000,88.2000) -- (87.5000,86.1000) -- (88.6000,83.8000) -- (89.1000,81.3000) -- (89.2000,78.8000) -- (89.3000,76.2000) -- (89.3000,73.7000) -- (89.3000,71.1000) -- (89.3000,68.6000) -- (89.3000,66.1000) -- (89.3000,63.6000) -- (89.3000,61.0000) -- (89.3000,58.5000) -- (89.3000,56.0000) -- (89.3000,53.5000) -- (89.6000,51.0000) -- (90.0000,48.5000) -- (89.9000,46.0000) -- (89.4000,43.6000) -- (88.5000,41.3000) -- (87.2000,39.1000) -- (85.5000,37.2000) -- (83.5000,35.5000) -- (81.3000,34.1000) -- (78.9000,33.1000) -- (76.3000,32.5000) -- (73.6000,32.2000) -- (71.0000,32.3000) -- (68.3000,32.8000) -- (65.8000,33.7000) -- (63.5000,34.9000) -- (61.4000,36.5000) -- (59.6000,38.3000) -- (58.2000,40.4000) -- (57.1000,42.7000) -- (56.5000,45.2000) -- (56.3000,47.7000) -- (56.3000,50.1000) -- (56.3000,52.6000) -- (56.3000,55.1000) -- (56.3000,57.6000) -- (56.3000,60.2000) -- (56.3000,62.7000) -- (56.3000,65.2000) -- (56.3000,67.7000) -- (56.3000,70.2000) -- (56.3000,72.7000) -- (56.3000,75.3000) -- (56.3000,77.8000) -- (56.6000,80.3000) -- (57.3000,82.8000) -- (58.5000,85.1000) -- (60.1000,87.2000) -- (61.9000,89.1000) -- (64.1000,90.7000) -- (66.4000,91.9000) -- (68.9000,92.8000) -- (71.5000,93.4000) -- (74.1000,93.6000) -- (76.7000,93.5000) -- (79.3000,93.0000) -- (81.8000,92.2000) -- (84.1000,91.0000) -- (86.3000,89.5000) -- (88.2000,87.7000) -- (89.8000,85.7000) -- (90.9000,83.4000) -- (91.4000,80.9000) -- (91.6000,78.3000) -- (91.6000,75.8000) -- (91.6000,73.2000) -- (91.6000,70.7000) -- (91.6000,68.2000) -- (91.6000,65.6000) -- (91.6000,63.1000) -- (91.6000,60.6000) -- (91.6000,58.1000) -- (91.6000,55.6000) -- (91.6000,53.0000) -- (91.9000,50.5000) -- (92.3000,48.0000) -- (92.2000,45.6000) -- (91.7000,43.2000) -- (90.7000,40.9000) -- (89.4000,38.7000) -- (87.7000,36.8000) -- (85.6000,35.2000) -- (83.3000,33.9000) -- (80.9000,32.9000) -- (78.3000,32.4000) -- (75.6000,32.2000) -- (72.9000,32.4000) -- (70.3000,33.0000) -- (67.9000,33.9000) -- (65.6000,35.2000) -- (63.6000,36.9000) -- (61.9000,38.8000) -- (60.6000,41.0000) -- (59.7000,43.4000) -- (59.3000,45.9000) -- (59.2000,48.3000) -- (59.1000,50.8000) -- (59.1000,53.3000) -- (59.1000,55.8000) -- (59.1000,58.3000) -- (59.1000,60.9000) -- (59.1000,63.4000) -- (59.1000,65.9000) -- (59.1000,68.4000) -- (59.1000,70.9000) -- (59.1000,73.4000) -- (59.1000,76.0000) -- (59.2000,78.5000) -- (59.6000,81.0000) -- (60.6000,83.5000) -- (61.9000,85.7000) -- (63.6000,87.7000) -- (65.6000,89.4000) -- (67.8000,90.9000) -- (70.2000,92.0000) -- (72.7000,92.8000) -- (75.3000,93.2000) -- (78.0000,93.2000) -- (80.6000,92.9000) -- (83.1000,92.3000) -- (85.5000,91.3000) -- (87.8000,89.9000) -- (89.8000,88.3000) -- (91.6000,86.3000) -- (92.9000,84.1000) -- (93.6000,81.7000) -- (93.9000,79.2000) -- (94.0000,76.6000) -- (94.0000,74.0000) -- (94.1000,71.5000) -- (94.1000,69.0000) -- (94.1000,66.4000) -- (94.1000,63.9000) -- (94.1000,61.4000) -- (94.1000,58.9000) -- (94.1000,56.4000) -- (94.1000,53.9000) -- (94.2000,51.3000) -- (94.6000,48.8000) -- (94.7000,46.4000) -- (94.4000,43.9000) -- (93.6000,41.6000) -- (92.4000,39.3000) -- (90.8000,37.3000) -- (88.9000,35.6000) -- (86.7000,34.2000) -- (84.3000,33.1000) -- (81.7000,32.4000) -- (79.1000,32.0000) -- (76.4000,32.1000) -- (73.8000,32.5000) -- (71.3000,33.4000) -- (68.9000,34.6000) -- (66.8000,36.1000) -- (65.0000,37.9000) -- (63.6000,40.1000) -- (62.6000,42.4000) -- (62.0000,44.8000) -- (61.8000,47.3000) -- (61.8000,49.8000) -- (61.7000,52.3000) -- (61.7000,54.8000) -- (61.7000,57.3000) -- (61.7000,59.8000) -- (61.7000,62.4000) -- (61.7000,64.9000) -- (61.7000,67.4000) -- (61.7000,69.9000) -- (61.7000,72.4000) -- (61.7000,75.0000) -- (61.7000,77.5000) -- (62.0000,80.0000) -- (62.8000,82.5000) -- (64.0000,84.8000) -- (65.6000,86.9000) -- (67.5000,88.7000) -- (69.7000,90.2000) -- (72.1000,91.4000) -- (74.6000,92.3000) -- (77.1000,92.8000) -- (79.8000,93.0000) -- (82.4000,92.8000) -- (84.9000,92.2000) -- (87.4000,91.3000) -- (89.7000,90.0000) -- (91.8000,88.5000) -- (93.6000,86.6000) -- (95.1000,84.4000) -- (95.9000,82.1000) -- (96.3000,79.6000) -- (96.4000,77.0000) -- (96.5000,74.4000) -- (96.5000,71.9000) -- (96.5000,69.3000) -- (96.5000,66.8000) -- (96.5000,64.3000) -- (96.5000,61.8000) -- (96.5000,59.3000) -- (96.5000,56.7000) -- (96.5000,54.2000) -- (96.5000,51.7000) -- (97.0000,49.2000) -- (97.2000,46.7000) -- (96.9000,44.3000) -- (96.3000,41.9000) -- (95.1000,39.6000) -- (93.6000,37.6000) -- (91.8000,35.8000) -- (89.7000,34.3000) -- (87.3000,33.1000) -- (84.8000,32.3000) -- (82.1000,31.9000) -- (79.5000,31.8000) -- (76.8000,32.2000) -- (74.3000,32.9000) -- (71.9000,34.1000) -- (69.7000,35.5000) -- (67.9000,37.3000) -- (66.4000,39.4000) -- (65.2000,41.7000) -- (64.6000,44.1000) -- (64.3000,46.6000) -- (64.3000,49.1000) -- (64.2000,51.6000) -- (64.2000,54.1000) -- (64.2000,56.6000) -- (64.2000,59.1000) -- (64.2000,61.6000) -- (64.2000,64.1000) -- (64.2000,66.6000) -- (64.2000,69.2000) -- (64.2000,71.7000) -- (64.2000,74.2000) -- (64.2000,76.7000) -- (64.4000,79.3000) -- (65.1000,81.8000) -- (66.2000,84.1000) -- (67.8000,86.2000) -- (69.6000,88.1000) -- (71.7000,89.7000) -- (74.0000,91.0000) -- (76.5000,91.9000) -- (79.1000,92.5000) -- (81.7000,92.7000) -- (84.3000,92.6000) -- (86.9000,92.1000) -- (89.4000,91.3000) -- (91.7000,90.1000) -- (93.9000,88.6000) -- (95.7000,86.7000) -- (97.3000,84.6000) -- (98.2000,82.3000) -- (98.7000,79.8000) -- (98.9000,77.3000) -- (98.9000,74.7000) -- (98.9000,72.1000) -- (99.0000,69.6000) -- (99.0000,67.1000) -- (99.0000,64.6000) -- (99.0000,62.0000) -- (99.0000,59.5000) -- (99.0000,57.0000) -- (99.0000,54.5000) -- (99.0000,52.0000) -- (99.3000,49.4000) -- (99.6000,47.0000) -- (99.5000,44.5000) -- (98.9000,42.1000) -- (97.8000,39.8000) -- (96.4000,37.7000) -- (94.6000,35.9000) -- (92.5000,34.3000) -- (90.2000,33.1000) -- (87.7000,32.2000) -- (85.1000,31.7000) -- (82.4000,31.6000) -- (79.7000,31.9000) -- (77.2000,32.6000) -- (74.8000,33.6000) -- (72.6000,35.1000) -- (70.6000,36.8000) -- (69.1000,38.8000) -- (67.9000,41.1000) -- (67.2000,43.5000) -- (66.9000,46.0000) -- (66.8000,48.5000) -- (66.8000,51.0000) -- (66.7000,53.5000) -- (66.7000,56.0000) -- (66.7000,58.5000) -- (66.7000,61.0000) -- (66.7000,63.5000) -- (66.7000,66.0000) -- (66.7000,68.6000) -- (66.7000,71.1000) -- (66.7000,73.6000) -- (66.7000,76.1000) -- (66.9000,78.7000) -- (67.5000,81.2000) -- (68.6000,83.5000) -- (70.0000,85.7000) -- (71.8000,87.6000) -- (73.9000,89.2000) -- (76.2000,90.6000) -- (78.7000,91.6000) -- (81.2000,92.2000) -- (83.8000,92.5000) -- (86.5000,92.4000) -- (89.0000,92.0000) -- (91.5000,91.2000) -- (93.9000,90.0000) -- (96.1000,88.5000) -- (98.0000,86.7000) -- (99.5000,84.7000) -- (100.6000,82.4000) -- (101.1000,79.9000) -- (101.3000,77.3000) -- (101.4000,74.8000) -- (101.4000,72.2000) -- (101.4000,69.7000) -- (101.4000,67.2000) -- (101.4000,64.6000) -- (101.4000,62.1000) -- (101.4000,59.6000) -- (101.4000,57.1000) -- (101.4000,54.6000) -- (101.4000,52.0000) -- (101.7000,49.5000) -- (102.1000,47.0000) -- (102.0000,44.6000) -- (101.4000,42.2000) -- (100.4000,39.9000) -- (99.0000,37.7000) -- (97.3000,35.9000) -- (95.2000,34.2000) -- (92.9000,33.0000) -- (90.4000,32.1000) -- (87.8000,31.5000) -- (85.2000,31.4000) -- (82.5000,31.6000) -- (79.9000,32.3000) -- (77.5000,33.3000) -- (75.3000,34.6000) -- (73.3000,36.3000) -- (71.7000,38.3000) -- (70.5000,40.6000) -- (69.7000,43.0000) -- (69.3000,45.5000) -- (69.2000,47.9000) -- (69.2000,50.4000) -- (69.2000,52.9000) -- (69.2000,55.4000) -- (69.2000,57.9000) -- (69.2000,60.5000) -- (69.2000,63.0000) -- (69.2000,65.5000) -- (69.2000,68.0000) -- (69.2000,70.5000) -- (69.2000,73.1000) -- (69.2000,75.6000) -- (69.3000,78.1000) -- (69.8000,80.6000) -- (70.8000,83.0000) -- (72.3000,85.2000) -- (74.0000,87.2000) -- (76.1000,88.8000) -- (78.3000,90.2000) -- (80.8000,91.2000) -- (83.3000,91.9000) -- (85.9000,92.3000) -- (88.6000,92.2000) -- (91.2000,91.8000) -- (93.7000,91.1000) -- (96.0000,90.0000) -- (98.2000,88.5000) -- (100.2000,86.8000) -- (101.8000,84.7000) -- (102.9000,82.5000) -- (103.5000,80.0000) -- (103.7000,77.4000) -- (103.8000,74.9000) -- (103.8000,72.3000) -- (103.8000,69.8000) -- (103.8000,67.3000) -- (103.8000,64.7000) -- (103.9000,62.2000) -- (103.8000,59.7000) -- (103.8000,57.2000) -- (103.8000,54.7000) -- (103.8000,52.1000) -- (104.1000,49.6000) -- (104.5000,47.1000) -- (104.4000,44.7000) -- (103.9000,42.3000) -- (103.0000,39.9000) -- (101.7000,37.8000) -- (100.0000,35.9000) -- (98.0000,34.2000) -- (95.7000,32.9000) -- (93.2000,31.9000) -- (90.6000,31.4000) -- (88.0000,31.2000) -- (85.3000,31.3000) -- (82.7000,31.9000) -- (80.2000,32.9000) -- (78.0000,34.2000) -- (76.0000,35.9000) -- (74.3000,37.8000) -- (73.0000,40.0000) -- (72.2000,42.4000) -- (71.8000,44.9000) -- (71.7000,47.4000) -- (71.6000,49.9000) -- (71.6000,52.4000) -- (71.6000,54.9000) -- (71.6000,57.4000) -- (71.6000,59.9000) -- (71.6000,62.4000) -- (71.6000,65.0000) -- (71.6000,67.5000) -- (71.6000,70.0000) -- (71.6000,72.5000) -- (71.6000,75.0000) -- (71.7000,77.6000) -- (72.2000,80.1000) -- (73.1000,82.5000) -- (74.5000,84.7000) -- (76.2000,86.7000) -- (78.2000,88.4000) -- (80.5000,89.8000) -- (82.9000,90.9000) -- (85.4000,91.6000) -- (88.0000,92.0000) -- (90.7000,92.0000) -- (93.3000,91.7000) -- (95.8000,91.0000) -- (98.2000,89.9000) -- (100.4000,88.5000) -- (102.4000,86.8000) -- (104.1000,84.8000) -- (105.3000,82.6000) -- (105.9000,80.1000) -- (106.2000,77.6000) -- (106.2000,75.0000) -- (106.3000,72.4000) -- (106.3000,69.9000) -- (106.3000,67.4000) -- (106.3000,64.9000) -- (106.3000,62.3000) -- (106.3000,59.8000) -- (106.3000,57.3000) -- (106.3000,54.8000) -- (106.3000,52.3000) -- (106.5000,49.7000) -- (106.9000,47.2000) -- (106.9000,44.8000) -- (106.5000,42.4000) -- (105.6000,40.0000) -- (104.3000,37.8000) -- (102.7000,35.9000) -- (100.7000,34.2000) -- (98.4000,32.8000) -- (96.0000,31.8000) -- (93.4000,31.2000) -- (90.8000,30.9000) -- (88.1000,31.1000) -- (85.5000,31.6000) -- (83.0000,32.5000) -- (80.7000,33.8000) -- (78.7000,35.4000) -- (77.0000,37.4000) -- (75.6000,39.5000) -- (74.8000,41.9000) -- (74.3000,44.4000) -- (74.2000,46.9000) -- (74.1000,49.3000) -- (74.1000,51.8000) -- (74.1000,54.3000) -- (74.1000,56.9000) -- (74.1000,59.4000) -- (74.1000,61.9000) -- (74.1000,64.4000) -- (74.1000,66.9000) -- (74.1000,69.4000) -- (74.1000,72.0000) -- (74.1000,74.5000) -- (74.1000,77.0000) -- (74.5000,79.6000) -- (75.4000,82.0000) -- (76.8000,84.2000) -- (78.5000,86.2000) -- (80.4000,88.0000) -- (82.6000,89.4000) -- (85.0000,90.6000) -- (87.6000,91.3000) -- (90.2000,91.8000) -- (92.8000,91.8000) -- (95.4000,91.5000) -- (97.9000,90.9000) -- (100.3000,89.8000) -- (102.6000,88.5000) -- (104.6000,86.8000) -- (106.3000,84.8000) -- (107.6000,82.6000) -- (108.3000,80.2000) -- (108.6000,77.7000) -- (108.7000,75.1000) -- (108.7000,72.5000) -- (108.7000,70.0000) -- (108.7000,67.5000) -- (108.7000,64.9000) -- (108.7000,62.4000) -- (108.7000,59.9000) -- (108.7000,57.4000) -- (108.7000,54.9000) -- (108.7000,52.3000) -- (108.9000,49.8000) -- (109.3000,47.3000) -- (109.4000,44.9000) -- (109.0000,42.4000) -- (108.2000,40.1000) -- (106.9000,37.9000) -- (105.3000,35.9000) -- (103.4000,34.2000) -- (101.2000,32.8000) -- (98.8000,31.7000) -- (96.2000,31.0000) -- (93.5000,30.7000) -- (90.9000,30.8000) -- (88.2000,31.3000) -- (85.7000,32.2000) -- (83.4000,33.4000) -- (81.4000,35.0000) -- (79.6000,36.9000) -- (78.2000,39.0000) -- (77.3000,41.4000) -- (76.8000,43.8000) -- (76.6000,46.3000) -- (76.5000,48.8000) -- (76.5000,51.3000) -- (76.5000,53.8000) -- (76.5000,56.3000) -- (76.5000,58.8000) -- (76.5000,61.4000) -- (76.5000,63.9000) -- (76.5000,66.4000) -- (76.5000,68.9000) -- (76.5000,71.4000) -- (76.5000,73.9000) -- (76.5000,76.5000) -- (76.9000,79.0000) -- (77.7000,81.5000) -- (79.0000,83.7000) -- (80.7000,85.8000) -- (82.6000,87.6000) -- (84.8000,89.0000) -- (87.2000,90.2000) -- (89.7000,91.0000) -- (92.3000,91.5000) -- (94.9000,91.6000) -- (97.5000,91.4000) -- (100.1000,90.7000) -- (102.5000,89.8000) -- (104.8000,88.4000) -- (106.8000,86.8000) -- (108.6000,84.9000) -- (109.9000,82.7000) -- (110.7000,80.3000) -- (111.0000,77.8000) -- (111.1000,75.2000) -- (111.2000,72.6000) -- (111.2000,70.1000) -- (111.2000,67.6000) -- (111.2000,65.0000) -- (111.2000,62.5000) -- (111.2000,60.0000) -- (111.2000,57.5000) -- (111.2000,55.0000) -- (111.2000,52.4000) -- (111.3000,49.9000) -- (111.7000,47.4000) -- (111.9000,45.0000) -- (111.5000,42.5000) -- (110.8000,40.2000) -- (109.6000,37.9000) -- (108.0000,35.9000) -- (106.1000,34.1000) -- (103.9000,32.7000) -- (101.5000,31.6000) -- (99.0000,30.9000) -- (96.3000,30.5000) -- (93.6000,30.6000) -- (91.0000,31.0000) -- (88.5000,31.8000) -- (86.2000,33.0000) -- (84.1000,34.6000) -- (82.3000,36.4000) -- (80.8000,38.5000) -- (79.8000,40.8000) -- (79.3000,43.3000) -- (79.1000,45.8000) -- (79.0000,48.3000) -- (79.0000,50.8000) -- (79.0000,53.3000) -- (79.0000,55.8000) -- (79.0000,58.3000) -- (79.0000,60.8000) -- (79.0000,63.3000) -- (79.0000,65.9000) -- (79.0000,68.4000) -- (79.0000,70.9000) -- (79.0000,73.4000) -- (79.0000,75.9000) -- (79.3000,78.5000) -- (80.1000,81.0000) -- (81.3000,83.2000) -- (82.9000,85.3000) -- (84.8000,87.1000) -- (87.0000,88.6000) -- (89.3000,89.9000) -- (91.8000,90.7000) -- (94.4000,91.3000) -- (97.0000,91.4000) -- (99.6000,91.2000) -- (102.2000,90.6000) -- (104.6000,89.7000) -- (106.9000,88.4000) -- (109.0000,86.8000) -- (110.8000,84.9000) -- (112.2000,82.8000) -- (113.1000,80.4000) -- (113.4000,77.9000) -- (113.6000,75.3000) -- (113.6000,72.7000) -- (113.6000,70.2000) -- (113.6000,67.6000) -- (113.6000,65.1000) -- (113.6000,62.6000) -- (113.6000,60.1000) -- (113.6000,57.6000) -- (113.6000,55.1000) -- (113.6000,52.5000) -- (113.7000,50.0000) -- (114.1000,47.5000) -- (114.3000,45.0000) -- (114.0000,42.6000) -- (113.3000,40.2000) -- (112.2000,38.0000) -- (110.7000,35.9000) -- (108.8000,34.1000) -- (106.6000,32.6000) -- (104.3000,31.5000) -- (101.7000,30.7000) -- (99.1000,30.3000) -- (96.4000,30.3000) -- (93.8000,30.7000) -- (91.2000,31.5000) -- (88.9000,32.6000) -- (86.7000,34.1000) -- (84.9000,36.0000) -- (83.4000,38.0000) -- (82.4000,40.3000) -- (81.8000,42.8000) -- (81.6000,45.3000) -- (81.5000,47.8000) -- (81.4000,50.3000) -- (81.4000,52.8000) -- (81.4000,55.3000) -- (81.4000,57.8000) -- (81.4000,60.3000) -- (81.4000,62.8000) -- (81.4000,65.4000) -- (81.4000,67.9000) -- (81.4000,70.4000) -- (81.4000,72.9000) -- (81.4000,75.4000) -- (81.7000,78.0000) -- (82.4000,80.5000) -- (83.6000,82.8000) -- (85.2000,84.9000) -- (87.0000,86.7000) -- (89.2000,88.3000) -- (91.5000,89.5000) -- (94.0000,90.4000) -- (96.6000,91.0000) -- (99.2000,91.2000) -- (101.8000,91.0000) -- (104.4000,90.5000) -- (106.8000,89.6000) -- (109.2000,88.3000) -- (111.3000,86.8000) -- (113.1000,84.9000) -- (114.6000,82.8000) -- (115.5000,80.4000) -- (115.9000,77.9000) -- (116.0000,75.3000) -- (116.1000,72.8000) -- (116.1000,70.2000) -- (116.1000,67.7000) -- (116.1000,65.2000) -- (116.1000,62.7000) -- (116.1000,60.1000) -- (116.1000,57.6000) -- (116.1000,55.1000) -- (116.1000,52.6000) -- (116.1000,50.1000) -- (116.5000,47.5000) -- (116.8000,45.1000) -- (116.5000,42.6000) -- (115.9000,40.2000) -- (114.8000,38.0000) -- (113.3000,35.9000) -- (111.5000,34.1000) -- (109.3000,32.6000) -- (107.0000,31.4000) -- (104.5000,30.6000) -- (101.8000,30.1000) -- (99.2000,30.1000) -- (96.5000,30.4000) -- (94.0000,31.2000) -- (91.6000,32.3000) -- (89.4000,33.7000) -- (87.5000,35.5000) -- (86.0000,37.6000) -- (84.9000,39.9000) -- (84.3000,42.3000) -- (84.0000,44.8000) -- (83.9000,47.3000) -- (83.9000,49.8000) -- (83.9000,52.3000) -- (83.9000,54.8000) -- (83.9000,57.3000) -- (83.9000,59.8000) -- (83.9000,62.3000) -- (83.9000,64.8000) -- (83.9000,67.4000) -- (83.9000,69.9000) -- (83.9000,72.4000) -- (83.9000,74.9000) -- (84.0000,77.5000) -- (84.7000,80.0000) -- (85.9000,82.3000) -- (87.4000,84.4000) -- (89.2000,86.3000) -- (91.3000,87.9000) -- (93.7000,89.2000) -- (96.1000,90.1000) -- (98.7000,90.7000) -- (101.3000,91.0000) -- (103.9000,90.8000) -- (106.5000,90.3000) -- (109.0000,89.5000) -- (111.3000,88.3000) -- (113.5000,86.8000) -- (115.3000,84.9000) -- (116.9000,82.8000) -- (117.8000,80.5000) -- (118.3000,78.0000) -- (118.4000,75.4000) -- (118.5000,72.9000) -- (118.5000,70.3000) -- (118.5000,67.8000) -- (118.5000,65.3000) -- (118.5000,62.7000) -- (118.5000,60.2000) -- (118.5000,57.7000) -- (118.5000,55.2000) -- (118.5000,52.7000) -- (118.5000,50.2000) -- (118.9000,47.6000) -- (119.2000,45.1000) -- (119.0000,42.7000) -- (118.4000,40.3000) -- (117.4000,38.0000) -- (115.9000,35.9000) -- (114.1000,34.1000) -- (112.1000,32.5000) -- (109.7000,31.3000) -- (107.2000,30.4000) -- (104.6000,29.9000) -- (101.9000,29.9000) -- (99.3000,30.2000) -- (96.7000,30.8000) -- (94.3000,31.9000) -- (92.1000,33.3000) -- (90.2000,35.1000) -- (88.6000,37.1000) -- (87.5000,39.4000) -- (86.8000,41.8000) -- (86.5000,44.3000) -- (86.4000,46.8000) -- (86.3000,49.3000) -- (86.3000,51.8000) -- (86.3000,54.3000) -- (86.3000,56.8000) -- (86.3000,59.3000) -- (86.3000,61.8000) -- (86.3000,64.3000) -- (86.3000,66.8000) -- (86.3000,69.4000) -- (86.3000,71.9000) -- (86.3000,74.4000) -- (86.5000,76.9000) -- (87.1000,79.5000) -- (88.2000,81.8000) -- (89.7000,84.0000) -- (91.5000,85.9000) -- (93.6000,87.5000) -- (95.8000,88.8000) -- (98.3000,89.8000) -- (100.9000,90.5000) -- (103.5000,90.7000) -- (106.1000,90.6000) -- (108.7000,90.2000) -- (111.2000,89.4000) -- (113.5000,88.2000) -- (115.7000,86.7000) -- (117.6000,84.9000) -- (119.2000,82.9000) -- (120.2000,80.5000) -- (120.7000,78.1000) -- (120.9000,75.5000) -- (120.9000,72.9000) -- (121.0000,70.4000) -- (121.0000,67.8000) -- (121.0000,65.3000) -- (121.0000,62.8000) -- (121.0000,60.3000) -- (121.0000,57.8000) -- (121.0000,55.2000) -- (121.0000,52.7000) -- (121.0000,50.2000) -- (121.3000,47.7000) -- (121.6000,45.2000) -- (121.5000,42.8000) -- (121.0000,40.4000) -- (120.0000,38.0000) -- (118.6000,35.9000) -- (116.8000,34.0000) -- (114.8000,32.4000) -- (112.4000,31.2000) -- (110.0000,30.3000) -- (107.4000,29.7000) -- (104.7000,29.6000) -- (102.0000,29.9000) -- (99.4000,30.5000) -- (97.0000,31.5000) -- (94.8000,32.9000) -- (92.8000,34.6000) -- (91.2000,36.6000) -- (90.0000,38.8000) -- (89.2000,41.2000) -- (88.9000,43.7000) -- (88.8000,46.2000) -- (88.7000,48.7000) -- (88.7000,51.2000) -- (88.7000,53.7000) -- (88.7000,56.2000) -- (88.7000,58.7000) -- (88.7000,61.2000) -- (88.7000,63.8000) -- (88.7000,66.3000) -- (88.7000,68.8000) -- (88.7000,71.3000) -- (88.7000,73.8000) -- (88.8000,76.4000) -- (89.4000,78.9000) -- (90.4000,81.3000) -- (91.8000,83.5000) -- (93.6000,85.4000) -- (95.6000,87.1000) -- (97.9000,88.4000) -- (100.3000,89.5000) -- (102.9000,90.2000) -- (105.5000,90.5000) -- (108.1000,90.5000) -- (110.7000,90.1000) -- (113.2000,89.3000) -- (115.6000,88.2000) -- (117.8000,86.8000) -- (119.8000,85.0000) -- (121.4000,83.0000) -- (122.5000,80.7000) -- (123.1000,78.3000) -- (123.3000,75.7000) -- (123.4000,73.1000) -- (123.4000,70.6000) -- (123.4000,68.0000) -- (123.4000,65.5000) -- (123.4000,63.0000) -- (123.4000,60.5000) -- (123.4000,58.0000) -- (123.4000,55.4000) -- (123.4000,52.9000) -- (123.4000,50.4000) -- (123.7000,47.9000) -- (124.1000,45.4000) -- (124.0000,42.9000) -- (123.5000,40.5000) -- (122.6000,38.2000) -- (121.3000,36.0000) -- (119.6000,34.1000) -- (117.6000,32.5000) -- (115.3000,31.1000) -- (112.8000,30.2000) -- (110.2000,29.6000) -- (107.6000,29.4000) -- (104.9000,29.6000) -- (102.3000,30.1000) -- (99.9000,31.1000) -- (97.6000,32.4000) -- (95.6000,34.1000) -- (93.9000,36.0000) -- (92.6000,38.2000) -- (91.8000,40.6000) -- (91.4000,43.1000) -- (91.2000,45.6000) -- (91.2000,48.0000) -- (91.2000,50.5000) -- (91.2000,53.0000) -- (91.2000,55.6000) -- (91.2000,58.1000) -- (91.2000,60.6000) -- (91.2000,63.1000) -- (91.2000,65.6000) -- (91.2000,68.1000) -- (91.2000,70.7000) -- (91.2000,73.2000) -- (91.2000,75.7000) -- (91.7000,78.3000) -- (92.6000,80.7000) -- (94.0000,82.9000) -- (95.7000,84.9000) -- (97.7000,86.6000) -- (99.9000,88.0000) -- (102.4000,89.1000) -- (104.9000,89.9000) -- (107.5000,90.3000) -- (110.1000,90.3000) -- (112.7000,90.0000) -- (115.3000,89.3000) -- (117.7000,88.2000) -- (119.9000,86.8000) -- (121.9000,85.1000) -- (123.6000,83.1000) -- (124.8000,80.9000) -- (125.5000,78.5000) -- (125.7000,75.9000) -- (125.8000,73.4000) -- (125.9000,70.8000) -- (125.9000,68.2000) -- (125.9000,65.7000) -- (125.9000,63.2000) -- (125.9000,60.7000) -- (125.9000,58.2000) -- (125.9000,55.7000) -- (125.9000,53.1000) -- (125.9000,50.6000) -- (126.0000,48.1000) -- (126.5000,45.5000);



  \end{scope}
  \begin{scope}[cm={{1.00588,0.0,0.0,1.00588,(0.01175,0.51144)}},draw=black,line join=round,line cap=round,line width=0.480pt]
    \path[draw] (44.5000,13.5000) -- (44.5000,102.5000) -- (142.5000,102.5000) -- (142.5000,13.5000) -- (44.5000,13.5000);



  \end{scope}
  \begin{scope}[cm={{1.00588,0.0,0.0,1.00588,(0.01175,0.51144)}},draw=ca0a0a4,dash pattern=on 0.40pt off 0.80pt,line join=round,line cap=round,line width=0.400pt]
    \path[draw] (148.5000,102.5000) -- (246.5000,102.5000);



  \end{scope}
  \begin{scope}[cm={{1.00588,0.0,0.0,1.00588,(0.01175,0.51144)}},draw=black,line join=round,line cap=round,line width=0.480pt]
    \path[draw] (148.5000,102.5000) -- (151.5000,102.5000);



    \path[draw] (246.5000,102.5000) -- (243.5000,102.5000);



  \end{scope}
  \begin{scope}[cm={{1.00588,0.0,0.0,1.00588,(0.01175,0.51144)}},draw=ca0a0a4,dash pattern=on 0.40pt off 0.80pt,line join=round,line cap=round,line width=0.400pt]
    \path[draw] (148.5000,79.5000) -- (246.5000,79.5000);



  \end{scope}
  \begin{scope}[cm={{1.00588,0.0,0.0,1.00588,(0.01175,0.51144)}},draw=black,line join=round,line cap=round,line width=0.480pt]
    \path[draw] (148.5000,79.5000) -- (151.5000,79.5000);



    \path[draw] (246.5000,79.5000) -- (243.5000,79.5000);



  \end{scope}
  \begin{scope}[cm={{1.00588,0.0,0.0,1.00588,(0.01175,0.51144)}},draw=ca0a0a4,dash pattern=on 0.40pt off 0.80pt,line join=round,line cap=round,line width=0.400pt]
    \path[draw] (148.5000,57.5000) -- (246.5000,57.5000);



  \end{scope}
  \begin{scope}[cm={{1.00588,0.0,0.0,1.00588,(0.01175,0.51144)}},draw=black,line join=round,line cap=round,line width=0.480pt]
    \path[draw] (148.5000,57.5000) -- (151.5000,57.5000);



    \path[draw] (246.5000,57.5000) -- (243.5000,57.5000);



  \end{scope}
  \begin{scope}[cm={{1.00588,0.0,0.0,1.00588,(0.01175,0.51144)}},draw=ca0a0a4,dash pattern=on 0.40pt off 0.80pt,line join=round,line cap=round,line width=0.400pt]
    \path[draw] (148.5000,35.5000) -- (246.5000,35.5000);



  \end{scope}
  \begin{scope}[cm={{1.00588,0.0,0.0,1.00588,(0.01175,0.51144)}},draw=black,line join=round,line cap=round,line width=0.480pt]
    \path[draw] (148.5000,35.5000) -- (151.5000,35.5000);



    \path[draw] (246.5000,35.5000) -- (243.5000,35.5000);



  \end{scope}
  \begin{scope}[cm={{1.00588,0.0,0.0,1.00588,(0.01175,0.51144)}},draw=ca0a0a4,dash pattern=on 0.40pt off 0.80pt,line join=round,line cap=round,line width=0.400pt]
    \path[draw] (148.5000,13.5000) -- (246.5000,13.5000);



  \end{scope}
  \begin{scope}[cm={{1.00588,0.0,0.0,1.00588,(54.32927,0.51144)}},draw=ca0a0a4,dash pattern=on 0.40pt off 0.80pt,line join=round,line cap=round,line width=0.400pt]
    \path[draw] (118.5000,102.5000) -- (118.5000,13.5000);



  \end{scope}
  \begin{scope}[cm={{1.00588,0.0,0.0,1.00588,(0.01175,0.51144)}},draw=black,line join=round,line cap=round,line width=0.480pt]
    \path[draw] (148.5000,13.5000) -- (151.5000,13.5000);



    \path[draw] (246.5000,13.5000) -- (243.5000,13.5000);



  \end{scope}
  \begin{scope}[cm={{1.00588,0.0,0.0,1.00588,(0.01175,0.51144)}},draw=ca0a0a4,dash pattern=on 0.40pt off 0.80pt,line join=round,line cap=round,line width=0.400pt]
    \path[draw] (148.5000,102.5000) -- (148.5000,13.5000);



  \end{scope}
  \begin{scope}[cm={{1.00588,0.0,0.0,1.00588,(0.01175,0.51144)}},draw=black,line join=round,line cap=round,line width=0.480pt]
    \path[draw] (148.5000,102.5000) -- (148.5000,99.5000);



    \path[draw] (148.5000,13.5000) -- (148.5000,16.5000);



  \end{scope}
  \begin{scope}[cm={{1.00588,0.0,0.0,1.00588,(0.01175,0.51144)}},draw=black,line join=round,line cap=round,line width=0.480pt]
    \path[draw] (172.5000,102.5000) -- (172.5000,99.5000);



    \path[draw] (172.5000,13.5000) -- (172.5000,16.5000);



  \end{scope}
  \begin{scope}[cm={{1.00588,0.0,0.0,1.00588,(79.47627,0.51144)}},draw=ca0a0a4,dash pattern=on 0.40pt off 0.80pt,line join=round,line cap=round,line width=0.400pt]
    \path[draw] (118.5000,102.5000) -- (118.5000,13.5000);



  \end{scope}
  \begin{scope}[cm={{1.00588,0.0,0.0,1.00588,(0.01175,0.51144)}},draw=black,line join=round,line cap=round,line width=0.480pt]
    \path[draw] (197.5000,102.5000) -- (197.5000,99.5000);



    \path[draw] (197.5000,13.5000) -- (197.5000,16.5000);



  \end{scope}
  \begin{scope}[cm={{1.00588,0.0,0.0,1.00588,(103.61739,0.51144)}},draw=ca0a0a4,dash pattern=on 0.40pt off 0.80pt,line join=round,line cap=round,line width=0.400pt]
    \path[draw] (118.5000,102.5000) -- (118.5000,13.5000);



  \end{scope}
  \begin{scope}[cm={{1.00588,0.0,0.0,1.00588,(0.01175,0.51144)}},draw=black,line join=round,line cap=round,line width=0.480pt]
    \path[draw] (221.5000,102.5000) -- (221.5000,99.5000);



    \path[draw] (221.5000,13.5000) -- (221.5000,16.5000);



  \end{scope}
  \begin{scope}[cm={{1.00588,0.0,0.0,1.00588,(0.01175,0.51144)}},draw=ca0a0a4,dash pattern=on 0.40pt off 0.80pt,line join=round,line cap=round,line width=0.400pt]
    \path[draw] (246.5000,102.5000) -- (246.5000,27.5000);



    \path[draw] (246.5000,19.5000) -- (246.5000,13.5000);



  \end{scope}
  \begin{scope}[cm={{1.00588,0.0,0.0,1.00588,(0.01175,0.51144)}},draw=black,line join=round,line cap=round,line width=0.480pt]
    \path[draw] (246.5000,102.5000) -- (246.5000,99.5000);



    \path[draw] (246.5000,13.5000) -- (246.5000,16.5000);



  \end{scope}
  \begin{scope}[cm={{1.00588,0.0,0.0,1.00588,(0.01175,0.51144)}},draw=black,line join=round,line cap=round,line width=0.480pt]
    \path[draw] (148.5000,13.5000) -- (148.5000,102.5000) -- (246.5000,102.5000) -- (246.5000,13.5000) -- (148.5000,13.5000);



  \end{scope}
  \begin{scope}[cm={{1.00588,0.0,0.0,1.00588,(0.01175,0.51144)}},fill=cd9d9d9]
    \path[fill=cd9d9d9,rounded corners=0.0000cm] (222.0000,18.0000) rectangle (238.0000,34.0000);



  \end{scope}
  \begin{scope}[cm={{1.00588,0.0,0.0,1.00588,(0.01175,0.51144)}},draw=black,fill=cebebeb,line join=round,line cap=round,line width=0.800pt]
    \path[draw,fill=cebebeb] (222.5000,34.5000) -- (222.5000,18.5000) -- (238.5000,18.5000) -- (238.5000,34.5000) -- (222.5000,34.5000);



  \end{scope}
  \begin{scope}[cm={{1.00588,0.0,0.0,1.00588,(228.52218,30.47064)}},draw=black,line join=bevel,line cap=rect,line width=0.800pt]
    \path[fill=black] (0.0000,0.0000) node[above right] (text692) {\label{fig:trajs-II-static}II};



  \end{scope}
  \begin{scope}[cm={{1.00588,0.0,0.0,1.00588,(0.01175,0.51144)}},draw=black,line join=round,line cap=round,line width=0.480pt]
    \path[draw] (160.6000,31.9000) -- (160.6000,31.9000) -- (160.7000,33.1000) -- (160.7000,34.2000) -- (160.8000,35.2000) -- (160.9000,36.2000) -- (160.9000,37.2000) -- (160.6000,38.2000) -- (159.9000,39.1000) -- (159.3000,40.0000) -- (158.7000,40.9000) -- (158.2000,41.8000) -- (157.7000,42.8000) -- (157.4000,43.8000) -- (157.2000,44.8000) -- (157.1000,45.8000) -- (157.0000,46.9000) -- (157.0000,48.0000) -- (157.0000,49.0000) -- (157.0000,50.1000) -- (157.0000,51.1000) -- (157.0000,52.2000) -- (157.0000,53.3000) -- (157.0000,54.3000) -- (157.0000,55.4000) -- (157.0000,56.4000) -- (157.0000,57.5000) -- (157.0000,58.5000) -- (157.0000,59.6000) -- (157.0000,60.6000) -- (157.0000,61.7000) -- (157.0000,62.8000) -- (157.0000,63.8000) -- (157.0000,64.9000) -- (157.0000,65.9000) -- (157.0000,67.0000) -- (157.0000,68.0000) -- (157.0000,69.1000) -- (157.0000,70.1000) -- (157.0000,71.2000) -- (157.0000,72.2000) -- (157.0000,73.3000) -- (157.0000,74.4000) -- (157.0000,75.4000) -- (157.0000,76.5000) -- (157.0000,77.5000) -- (157.0000,78.6000) -- (157.1000,79.6000) -- (157.3000,80.7000) -- (157.6000,81.7000) -- (157.9000,82.7000) -- (158.3000,83.7000) -- (158.8000,84.7000) -- (159.4000,85.6000) -- (160.0000,86.5000) -- (160.7000,87.4000) -- (161.4000,88.3000) -- (162.2000,89.0000) -- (163.1000,89.8000) -- (164.0000,90.5000) -- (165.0000,91.1000) -- (166.0000,91.7000) -- (167.1000,92.2000) -- (168.2000,92.7000) -- (169.3000,93.1000) -- (170.5000,93.4000) -- (171.7000,93.6000) -- (172.9000,93.8000) -- (174.1000,93.9000) -- (175.3000,94.0000) -- (176.6000,93.9000) -- (177.8000,93.8000) -- (179.0000,93.6000) -- (180.2000,93.4000) -- (181.4000,93.0000) -- (182.5000,92.6000) -- (183.7000,92.1000) -- (184.7000,91.6000) -- (185.8000,91.0000) -- (186.7000,90.3000) -- (187.7000,89.6000) -- (188.5000,88.8000) -- (189.4000,88.0000) -- (190.1000,87.2000) -- (190.8000,86.3000) -- (191.4000,85.3000) -- (191.9000,84.4000) -- (192.3000,83.3000) -- (192.6000,82.3000) -- (192.8000,81.3000) -- (193.0000,80.3000) -- (193.1000,79.2000) -- (193.2000,78.2000) -- (193.3000,77.2000) -- (193.3000,76.1000) -- (193.4000,75.1000) -- (193.4000,74.0000) -- (193.4000,73.0000) -- (193.4000,71.9000) -- (193.4000,70.9000) -- (193.4000,69.8000) -- (193.4000,68.8000) -- (193.5000,67.7000) -- (193.5000,66.6000) -- (193.5000,65.6000) -- (193.5000,64.5000) -- (193.5000,63.5000) -- (193.5000,62.4000) -- (193.5000,61.4000) -- (193.5000,60.3000) -- (193.5000,59.3000) -- (193.5000,58.2000) -- (193.5000,57.2000) -- (193.5000,56.1000) -- (193.5000,55.0000) -- (193.5000,54.0000) -- (193.5000,52.9000) -- (193.5000,51.9000) -- (193.5000,50.8000) -- (193.7000,49.8000) -- (193.9000,48.7000) -- (194.0000,47.7000) -- (194.0000,46.6000) -- (193.9000,45.6000) -- (193.8000,44.5000) -- (193.6000,43.5000) -- (193.2000,42.5000) -- (192.8000,41.5000) -- (192.4000,40.5000) -- (191.8000,39.6000) -- (191.2000,38.7000) -- (190.4000,37.8000) -- (189.6000,37.0000) -- (188.8000,36.3000) -- (187.8000,35.6000) -- (186.8000,34.9000) -- (185.8000,34.4000) -- (184.7000,33.9000) -- (183.5000,33.5000) -- (182.3000,33.2000) -- (181.1000,33.0000) -- (179.9000,32.8000) -- (178.7000,32.8000) -- (177.4000,32.8000) -- (176.2000,32.9000) -- (175.0000,33.1000) -- (173.8000,33.4000) -- (172.7000,33.8000) -- (171.6000,34.2000) -- (170.5000,34.8000) -- (169.5000,35.3000) -- (168.5000,36.0000) -- (167.6000,36.7000) -- (166.8000,37.5000) -- (166.0000,38.3000) -- (165.4000,39.2000) -- (164.7000,40.1000) -- (164.2000,41.1000) -- (163.8000,42.1000) -- (163.4000,43.1000) -- (163.2000,44.1000) -- (163.1000,45.2000) -- (163.0000,46.2000) -- (163.0000,47.3000) -- (163.0000,48.4000) -- (163.0000,49.4000) -- (163.0000,50.5000) -- (163.0000,51.5000) -- (163.0000,52.6000) -- (163.0000,53.6000) -- (163.0000,54.7000) -- (162.9000,55.7000) -- (162.9000,56.8000) -- (162.9000,57.9000) -- (162.9000,58.9000) -- (162.9000,60.0000) -- (162.9000,61.0000) -- (162.9000,62.1000) -- (162.9000,63.1000) -- (162.9000,64.2000) -- (162.9000,65.2000) -- (162.9000,66.3000) -- (162.9000,67.4000) -- (162.9000,68.4000) -- (162.9000,69.5000) -- (162.9000,70.5000) -- (162.9000,71.6000) -- (162.9000,72.6000) -- (162.9000,73.7000) -- (162.9000,74.7000) -- (162.9000,75.8000) -- (162.9000,76.9000) -- (163.0000,77.9000) -- (163.1000,79.0000) -- (163.3000,80.0000) -- (163.5000,81.0000) -- (163.9000,82.1000) -- (164.3000,83.1000) -- (164.7000,84.0000) -- (165.3000,85.0000) -- (165.9000,85.9000) -- (166.6000,86.8000) -- (167.3000,87.6000) -- (168.1000,88.4000) -- (168.9000,89.2000) -- (169.9000,89.9000) -- (170.8000,90.5000) -- (171.8000,91.1000) -- (172.9000,91.7000) -- (174.0000,92.1000) -- (175.1000,92.5000) -- (176.3000,92.9000) -- (177.4000,93.2000) -- (178.6000,93.4000) -- (179.9000,93.5000) -- (181.1000,93.5000) -- (182.3000,93.5000) -- (183.6000,93.4000) -- (184.8000,93.3000) -- (186.0000,93.0000) -- (187.2000,92.7000) -- (188.3000,92.3000) -- (189.5000,91.9000) -- (190.6000,91.4000) -- (191.6000,90.8000) -- (192.6000,90.1000) -- (193.6000,89.4000) -- (194.5000,88.7000) -- (195.3000,87.9000) -- (196.1000,87.1000) -- (196.8000,86.2000) -- (197.4000,85.2000) -- (197.9000,84.3000) -- (198.4000,83.3000) -- (198.7000,82.3000) -- (198.9000,81.2000) -- (199.1000,80.2000) -- (199.3000,79.2000) -- (199.4000,78.1000) -- (199.4000,77.1000) -- (199.5000,76.1000) -- (199.5000,75.0000) -- (199.6000,74.0000) -- (199.6000,72.9000) -- (199.6000,71.9000) -- (199.6000,70.8000) -- (199.6000,69.8000) -- (199.6000,68.7000) -- (199.6000,67.7000) -- (199.6000,66.6000) -- (199.6000,65.5000) -- (199.7000,64.5000) -- (199.7000,63.4000) -- (199.7000,62.4000) -- (199.7000,61.3000) -- (199.7000,60.3000) -- (199.7000,59.2000) -- (199.7000,58.2000) -- (199.7000,57.1000) -- (199.7000,56.0000) -- (199.7000,55.0000) -- (199.7000,53.9000) -- (199.7000,52.9000) -- (199.7000,51.8000) -- (199.7000,50.8000) -- (199.8000,49.7000) -- (200.0000,48.7000) -- (200.1000,47.6000) -- (200.2000,46.6000) -- (200.1000,45.5000) -- (200.0000,44.5000) -- (199.8000,43.4000) -- (199.6000,42.4000) -- (199.2000,41.4000) -- (198.8000,40.4000) -- (198.3000,39.5000) -- (197.6000,38.5000) -- (197.0000,37.7000) -- (196.2000,36.8000) -- (195.4000,36.1000) -- (194.5000,35.3000) -- (193.5000,34.7000) -- (192.5000,34.1000) -- (191.4000,33.5000) -- (190.2000,33.1000) -- (189.1000,32.7000) -- (187.9000,32.5000) -- (186.7000,32.3000) -- (185.4000,32.2000) -- (184.2000,32.2000) -- (183.0000,32.2000) -- (181.8000,32.4000) -- (180.6000,32.7000) -- (179.4000,33.0000) -- (178.3000,33.4000) -- (177.2000,33.9000) -- (176.1000,34.4000) -- (175.2000,35.1000) -- (174.2000,35.8000) -- (173.4000,36.5000) -- (172.6000,37.3000) -- (171.8000,38.2000) -- (171.2000,39.0000) -- (170.6000,40.0000) -- (170.1000,41.0000) -- (169.7000,42.0000) -- (169.4000,43.0000) -- (169.2000,44.0000) -- (169.2000,45.1000) -- (169.1000,46.1000) -- (169.1000,47.2000) -- (169.1000,48.3000) -- (169.1000,49.3000) -- (169.1000,50.4000) -- (169.1000,51.4000) -- (169.1000,52.5000) -- (169.1000,53.6000) -- (169.1000,54.6000) -- (169.1000,55.7000) -- (169.1000,56.7000) -- (169.1000,57.8000) -- (169.1000,58.8000) -- (169.1000,59.9000) -- (169.1000,60.9000) -- (169.1000,62.0000) -- (169.1000,63.1000) -- (169.1000,64.1000) -- (169.1000,65.2000) -- (169.1000,66.2000) -- (169.1000,67.3000) -- (169.1000,68.3000) -- (169.1000,69.4000) -- (169.1000,70.4000) -- (169.1000,71.5000) -- (169.1000,72.6000) -- (169.1000,73.6000) -- (169.1000,74.7000) -- (169.1000,75.7000) -- (169.1000,76.8000) -- (169.1000,77.8000) -- (169.3000,78.9000) -- (169.5000,79.9000) -- (169.8000,80.9000) -- (170.1000,82.0000) -- (170.6000,82.9000) -- (171.1000,83.9000) -- (171.7000,84.8000) -- (172.3000,85.7000) -- (173.0000,86.6000) -- (173.8000,87.4000) -- (174.6000,88.2000) -- (175.5000,88.9000) -- (176.4000,89.6000) -- (177.4000,90.2000) -- (178.4000,90.8000) -- (179.5000,91.3000) -- (180.6000,91.8000) -- (181.7000,92.2000) -- (182.9000,92.5000) -- (184.1000,92.7000) -- (185.3000,92.9000) -- (186.5000,93.0000) -- (187.8000,93.0000) -- (189.0000,93.0000) -- (190.2000,92.8000) -- (191.4000,92.6000) -- (192.6000,92.4000) -- (193.8000,92.0000) -- (195.0000,91.6000) -- (196.1000,91.1000) -- (197.2000,90.6000) -- (198.2000,90.0000) -- (199.2000,89.3000) -- (200.1000,88.6000) -- (201.0000,87.8000) -- (201.8000,87.0000) -- (202.5000,86.2000) -- (203.2000,85.3000) -- (203.8000,84.3000) -- (204.3000,83.3000) -- (204.7000,82.3000) -- (205.0000,81.3000) -- (205.2000,80.3000) -- (205.3000,79.2000) -- (205.5000,78.2000) -- (205.6000,77.2000) -- (205.6000,76.1000) -- (205.7000,75.1000) -- (205.7000,74.0000) -- (205.7000,73.0000) -- (205.8000,71.9000) -- (205.8000,70.9000) -- (205.8000,69.8000) -- (205.8000,68.8000) -- (205.8000,67.7000) -- (205.8000,66.7000) -- (205.8000,65.6000) -- (205.8000,64.6000) -- (205.8000,63.5000) -- (205.8000,62.5000) -- (205.8000,61.4000) -- (205.8000,60.3000) -- (205.8000,59.3000) -- (205.8000,58.2000) -- (205.8000,57.2000) -- (205.8000,56.1000) -- (205.8000,55.1000) -- (205.8000,54.0000) -- (205.8000,53.0000) -- (205.8000,51.9000) -- (205.8000,50.8000) -- (205.9000,49.8000) -- (206.0000,48.7000) -- (206.2000,47.7000) -- (206.3000,46.6000) -- (206.3000,45.6000) -- (206.3000,44.5000) -- (206.1000,43.5000) -- (205.9000,42.5000) -- (205.6000,41.4000) -- (205.2000,40.4000) -- (204.7000,39.5000) -- (204.2000,38.5000) -- (203.5000,37.6000) -- (202.8000,36.8000) -- (202.0000,36.0000) -- (201.2000,35.2000) -- (200.2000,34.5000) -- (199.2000,33.9000) -- (198.2000,33.3000) -- (197.1000,32.8000) -- (195.9000,32.4000) -- (194.8000,32.1000) -- (193.5000,31.8000) -- (192.3000,31.7000) -- (191.1000,31.6000) -- (189.9000,31.6000) -- (188.6000,31.7000) -- (187.4000,31.9000) -- (186.3000,32.2000) -- (185.1000,32.6000) -- (184.0000,33.0000) -- (182.9000,33.5000) -- (181.9000,34.1000) -- (180.9000,34.8000) -- (180.0000,35.5000) -- (179.2000,36.2000) -- (178.4000,37.1000) -- (177.7000,37.9000) -- (177.1000,38.9000) -- (176.6000,39.8000) -- (176.1000,40.8000) -- (175.7000,41.8000) -- (175.5000,42.8000) -- (175.3000,43.9000) -- (175.3000,44.9000) -- (175.3000,46.0000) -- (175.2000,47.1000) -- (175.2000,48.1000) -- (175.2000,49.2000) -- (175.2000,50.2000) -- (175.2000,51.3000) -- (175.2000,52.4000) -- (175.2000,53.4000) -- (175.2000,54.5000) -- (175.2000,55.5000) -- (175.2000,56.6000) -- (175.2000,57.6000) -- (175.2000,58.7000) -- (175.2000,59.7000) -- (175.2000,60.8000) -- (175.2000,61.9000) -- (175.2000,62.9000) -- (175.2000,64.0000) -- (175.2000,65.0000) -- (175.2000,66.1000) -- (175.2000,67.1000) -- (175.2000,68.2000) -- (175.2000,69.2000) -- (175.2000,70.3000) -- (175.2000,71.4000) -- (175.2000,72.4000) -- (175.2000,73.5000) -- (175.2000,74.5000) -- (175.2000,75.6000) -- (175.2000,76.6000) -- (175.3000,77.7000) -- (175.5000,78.7000) -- (175.7000,79.8000) -- (176.1000,80.8000) -- (176.5000,81.8000) -- (176.9000,82.8000) -- (177.4000,83.7000) -- (178.0000,84.6000) -- (178.7000,85.5000) -- (179.4000,86.4000) -- (180.2000,87.2000) -- (181.1000,87.9000) -- (182.0000,88.7000) -- (182.9000,89.3000) -- (183.9000,89.9000) -- (185.0000,90.5000) -- (186.0000,91.0000) -- (187.2000,91.4000) -- (188.3000,91.7000) -- (189.5000,92.0000) -- (190.7000,92.2000) -- (191.9000,92.4000) -- (193.1000,92.4000) -- (194.4000,92.4000) -- (195.6000,92.4000) -- (196.8000,92.2000) -- (198.0000,92.0000) -- (199.2000,91.7000) -- (200.4000,91.3000) -- (201.5000,90.9000) -- (202.6000,90.4000) -- (203.7000,89.8000) -- (204.7000,89.2000) -- (205.7000,88.5000) -- (206.6000,87.8000) -- (207.4000,87.0000) -- (208.2000,86.1000) -- (208.9000,85.3000) -- (209.6000,84.3000) -- (210.1000,83.4000) -- (210.6000,82.4000) -- (210.9000,81.4000) -- (211.2000,80.3000) -- (211.4000,79.3000) -- (211.5000,78.3000) -- (211.6000,77.3000) -- (211.7000,76.2000) -- (211.8000,75.2000) -- (211.8000,74.1000) -- (211.9000,73.1000) -- (211.9000,72.0000) -- (211.9000,71.0000) -- (211.9000,69.9000) -- (211.9000,68.9000) -- (211.9000,67.8000) -- (211.9000,66.8000) -- (211.9000,65.7000) -- (211.9000,64.7000) -- (211.9000,63.6000) -- (211.9000,62.5000) -- (211.9000,61.5000) -- (211.9000,60.4000) -- (211.9000,59.4000) -- (211.9000,58.3000) -- (211.9000,57.3000) -- (211.9000,56.2000) -- (211.9000,55.2000) -- (211.9000,54.1000) -- (211.9000,53.0000) -- (211.9000,52.0000) -- (211.9000,50.9000) -- (212.0000,49.9000) -- (212.1000,48.8000) -- (212.2000,47.8000) -- (212.4000,46.7000) -- (212.5000,45.7000) -- (212.4000,44.6000) -- (212.4000,43.6000) -- (212.2000,42.5000) -- (211.9000,41.5000) -- (211.6000,40.5000) -- (211.2000,39.5000) -- (210.7000,38.6000) -- (210.1000,37.6000) -- (209.4000,36.7000) -- (208.7000,35.9000) -- (207.8000,35.1000) -- (206.9000,34.4000) -- (206.0000,33.7000) -- (205.0000,33.1000) -- (203.9000,32.5000) -- (202.8000,32.1000) -- (201.6000,31.7000) -- (200.4000,31.4000) -- (199.2000,31.2000) -- (198.0000,31.1000) -- (196.8000,31.1000) -- (195.5000,31.1000) -- (194.3000,31.3000) -- (193.1000,31.5000) -- (191.9000,31.8000) -- (190.8000,32.2000) -- (189.7000,32.7000) -- (188.7000,33.2000) -- (187.7000,33.8000) -- (186.7000,34.5000) -- (185.8000,35.2000) -- (185.0000,36.0000) -- (184.3000,36.9000) -- (183.6000,37.7000) -- (183.0000,38.7000) -- (182.5000,39.6000) -- (182.1000,40.6000) -- (181.7000,41.6000) -- (181.6000,42.7000) -- (181.5000,43.7000) -- (181.4000,44.8000) -- (181.4000,45.9000) -- (181.4000,46.9000) -- (181.4000,48.0000) -- (181.4000,49.0000) -- (181.4000,50.1000) -- (181.4000,51.2000) -- (181.4000,52.2000) -- (181.4000,53.3000) -- (181.4000,54.3000) -- (181.4000,55.4000) -- (181.4000,56.4000) -- (181.4000,57.5000) -- (181.4000,58.5000) -- (181.4000,59.6000) -- (181.4000,60.7000) -- (181.4000,61.7000) -- (181.4000,62.8000) -- (181.4000,63.8000) -- (181.4000,64.9000) -- (181.4000,65.9000) -- (181.4000,67.0000) -- (181.4000,68.0000) -- (181.4000,69.1000) -- (181.4000,70.2000) -- (181.4000,71.2000) -- (181.4000,72.3000) -- (181.4000,73.3000) -- (181.4000,74.4000) -- (181.4000,75.4000) -- (181.4000,76.5000) -- (181.5000,77.5000) -- (181.7000,78.6000) -- (182.0000,79.6000) -- (182.3000,80.6000) -- (182.8000,81.6000) -- (183.3000,82.6000) -- (183.8000,83.5000) -- (184.4000,84.4000) -- (185.1000,85.3000) -- (185.9000,86.1000) -- (186.7000,86.9000) -- (187.6000,87.7000) -- (188.5000,88.4000) -- (189.4000,89.0000) -- (190.5000,89.6000) -- (191.5000,90.1000) -- (192.6000,90.6000) -- (193.8000,91.0000) -- (194.9000,91.3000) -- (196.1000,91.6000) -- (197.3000,91.7000) -- (198.5000,91.9000) -- (199.8000,91.9000) -- (201.0000,91.9000) -- (202.2000,91.8000) -- (203.5000,91.6000) -- (204.7000,91.3000) -- (205.8000,91.0000) -- (207.0000,90.6000) -- (208.1000,90.1000) -- (209.2000,89.6000) -- (210.2000,89.0000) -- (211.2000,88.4000) -- (212.2000,87.7000) -- (213.1000,86.9000) -- (213.9000,86.1000) -- (214.6000,85.2000) -- (215.3000,84.3000) -- (215.9000,83.4000) -- (216.5000,82.4000) -- (216.9000,81.4000) -- (217.2000,80.4000) -- (217.4000,79.4000) -- (217.6000,78.4000) -- (217.7000,77.3000) -- (217.8000,76.3000) -- (217.9000,75.3000) -- (217.9000,74.2000) -- (218.0000,73.2000) -- (218.0000,72.1000) -- (218.0000,71.1000) -- (218.1000,70.0000) -- (218.1000,69.0000) -- (218.1000,67.9000) -- (218.1000,66.9000) -- (218.1000,65.8000) -- (218.1000,64.7000) -- (218.1000,63.7000) -- (218.1000,62.6000) -- (218.1000,61.6000) -- (218.1000,60.5000) -- (218.1000,59.5000) -- (218.1000,58.4000) -- (218.1000,57.4000) -- (218.1000,56.3000) -- (218.1000,55.2000) -- (218.1000,54.2000) -- (218.1000,53.1000) -- (218.1000,52.1000) -- (218.1000,51.0000) -- (218.1000,50.0000) -- (218.1000,48.9000) -- (218.3000,47.9000) -- (218.4000,46.8000) -- (218.6000,45.8000) -- (218.6000,44.7000) -- (218.6000,43.7000) -- (218.4000,42.6000) -- (218.2000,41.6000) -- (217.9000,40.6000) -- (217.6000,39.6000) -- (217.1000,38.6000) -- (216.6000,37.6000) -- (216.0000,36.7000) -- (215.3000,35.9000) -- (214.5000,35.0000) -- (213.6000,34.3000) -- (212.7000,33.5000) -- (211.7000,32.9000) -- (210.7000,32.3000) -- (209.6000,31.8000) -- (208.5000,31.4000) -- (207.3000,31.0000) -- (206.1000,30.8000) -- (204.9000,30.6000) -- (203.6000,30.5000) -- (202.4000,30.5000) -- (201.2000,30.6000) -- (200.0000,30.8000) -- (198.8000,31.1000) -- (197.6000,31.4000) -- (196.5000,31.8000) -- (195.4000,32.3000) -- (194.4000,32.9000) -- (193.4000,33.5000) -- (192.5000,34.2000) -- (191.6000,35.0000) -- (190.9000,35.8000) -- (190.1000,36.6000) -- (189.5000,37.5000) -- (188.9000,38.5000) -- (188.5000,39.5000) -- (188.1000,40.5000) -- (187.8000,41.5000) -- (187.7000,42.6000) -- (187.6000,43.6000) -- (187.5000,44.7000) -- (187.5000,45.7000) -- (187.5000,46.8000) -- (187.5000,47.9000) -- (187.5000,48.9000) -- (187.5000,50.0000) -- (187.5000,51.0000) -- (187.5000,52.1000) -- (187.5000,53.1000) -- (187.5000,54.2000) -- (187.5000,55.2000) -- (187.5000,56.3000) -- (187.5000,57.4000) -- (187.5000,58.4000) -- (187.5000,59.5000) -- (187.5000,60.5000) -- (187.5000,61.6000) -- (187.5000,62.6000) -- (187.5000,63.7000) -- (187.5000,64.7000) -- (187.5000,65.8000) -- (187.5000,66.9000) -- (187.5000,67.9000) -- (187.5000,69.0000) -- (187.5000,70.0000) -- (187.5000,71.1000) -- (187.5000,72.1000) -- (187.5000,73.2000) -- (187.5000,74.2000) -- (187.5000,75.3000) -- (187.6000,76.4000) -- (187.7000,77.4000) -- (188.0000,78.4000) -- (188.3000,79.5000) -- (188.7000,80.5000) -- (189.1000,81.5000) -- (189.6000,82.4000) -- (190.2000,83.3000) -- (190.8000,84.2000) -- (191.6000,85.1000) -- (192.3000,85.9000) -- (193.2000,86.7000) -- (194.1000,87.4000) -- (195.0000,88.1000) -- (196.0000,88.7000) -- (197.0000,89.2000) -- (198.1000,89.7000) -- (199.2000,90.2000) -- (200.4000,90.6000) -- (201.5000,90.9000) -- (202.7000,91.1000) -- (204.0000,91.2000) -- (205.2000,91.3000) -- (206.4000,91.3000) -- (207.6000,91.3000) -- (208.9000,91.1000) -- (210.1000,90.9000) -- (211.3000,90.6000) -- (212.5000,90.3000) -- (213.6000,89.9000) -- (214.7000,89.4000) -- (215.8000,88.8000) -- (216.8000,88.2000) -- (217.8000,87.5000) -- (218.7000,86.8000) -- (219.5000,86.0000) -- (220.3000,85.2000) -- (221.1000,84.3000) -- (221.7000,83.4000) -- (222.3000,82.5000) -- (222.8000,81.5000) -- (223.1000,80.5000) -- (223.4000,79.4000) -- (223.6000,78.4000) -- (223.8000,77.4000) -- (223.9000,76.4000) -- (224.0000,75.3000) -- (224.1000,74.3000) -- (224.1000,73.2000) -- (224.1000,72.2000) -- (224.2000,71.1000) -- (224.2000,70.1000) -- (224.2000,69.0000) -- (224.2000,68.0000) -- (224.2000,66.9000) -- (224.2000,65.9000) -- (224.2000,64.8000) -- (224.2000,63.8000) -- (224.2000,62.7000) -- (224.2000,61.7000) -- (224.2000,60.6000) -- (224.2000,59.5000) -- (224.2000,58.5000) -- (224.2000,57.4000) -- (224.2000,56.4000) -- (224.2000,55.3000) -- (224.2000,54.3000) -- (224.2000,53.2000) -- (224.2000,52.2000) -- (224.2000,51.1000) -- (224.2000,50.0000) -- (224.2000,49.0000) -- (224.3000,47.9000) -- (224.5000,46.9000) -- (224.6000,45.8000) -- (224.7000,44.8000) -- (224.7000,43.7000) -- (224.7000,42.7000) -- (224.5000,41.6000) -- (224.3000,40.6000) -- (223.9000,39.6000) -- (223.5000,38.6000) -- (223.1000,37.6000) -- (222.5000,36.7000) -- (221.8000,35.8000) -- (221.1000,35.0000) -- (220.3000,34.2000) -- (219.4000,33.4000) -- (218.5000,32.7000) -- (217.5000,32.1000) -- (216.4000,31.5000) -- (215.3000,31.1000) -- (214.1000,30.7000) -- (213.0000,30.4000) -- (211.7000,30.1000) -- (210.5000,30.0000) -- (209.3000,29.9000) -- (208.1000,30.0000) -- (206.8000,30.1000) -- (205.6000,30.3000) -- (204.5000,30.6000) -- (203.3000,31.0000) -- (202.2000,31.5000) -- (201.2000,32.0000) -- (200.1000,32.6000) -- (199.2000,33.2000) -- (198.3000,34.0000) -- (197.5000,34.7000) -- (196.7000,35.6000) -- (196.0000,36.4000) -- (195.4000,37.4000) -- (194.9000,38.3000) -- (194.4000,39.3000) -- (194.1000,40.3000) -- (193.9000,41.4000) -- (193.8000,42.4000) -- (193.7000,43.5000) -- (193.7000,44.5000) -- (193.7000,45.6000) -- (193.7000,46.7000) -- (193.6000,47.7000) -- (193.6000,48.8000) -- (193.6000,49.8000) -- (193.6000,50.9000) -- (193.6000,51.9000) -- (193.6000,53.0000) -- (193.6000,54.1000) -- (193.6000,55.1000) -- (193.6000,56.2000) -- (193.6000,57.2000) -- (193.6000,58.3000) -- (193.6000,59.3000) -- (193.6000,60.4000) -- (193.6000,61.4000) -- (193.6000,62.5000) -- (193.6000,63.6000) -- (193.6000,64.6000) -- (193.6000,65.7000) -- (193.6000,66.7000) -- (193.6000,67.8000) -- (193.6000,68.8000) -- (193.6000,69.9000) -- (193.6000,70.9000) -- (193.6000,72.0000) -- (193.6000,73.1000) -- (193.6000,74.1000) -- (193.7000,75.2000) -- (193.8000,76.2000) -- (194.0000,77.3000) -- (194.2000,78.3000) -- (194.6000,79.3000) -- (195.0000,80.3000) -- (195.4000,81.3000) -- (196.0000,82.2000) -- (196.6000,83.1000) -- (197.3000,84.0000) -- (198.0000,84.9000) -- (198.8000,85.7000) -- (199.7000,86.4000) -- (200.6000,87.1000) -- (201.5000,87.8000) -- (202.5000,88.4000) -- (203.6000,88.9000) -- (204.7000,89.4000) -- (205.8000,89.8000) -- (207.0000,90.1000) -- (208.2000,90.4000) -- (209.4000,90.6000) -- (210.6000,90.7000) -- (211.8000,90.8000) -- (213.0000,90.8000) -- (214.3000,90.7000) -- (215.5000,90.5000) -- (216.7000,90.3000) -- (217.9000,90.0000) -- (219.0000,89.6000) -- (220.2000,89.1000) -- (221.3000,88.6000) -- (222.3000,88.0000) -- (223.3000,87.4000) -- (224.3000,86.7000) -- (225.2000,85.9000) -- (226.0000,85.2000) -- (226.8000,84.3000) -- (227.5000,83.4000) -- (228.1000,82.5000) -- (228.6000,81.5000) -- (229.1000,80.5000) -- (229.4000,79.5000) -- (229.7000,78.5000) -- (229.9000,77.5000) -- (230.0000,76.4000) -- (230.1000,75.4000) -- (230.2000,74.4000) -- (230.2000,73.3000) -- (230.3000,72.3000) -- (230.3000,71.2000) -- (230.3000,70.2000) -- (230.3000,69.1000) -- (230.3000,68.1000) -- (230.4000,67.0000) -- (230.4000,66.0000) -- (230.4000,64.9000) -- (230.4000,63.9000) -- (230.4000,62.8000) -- (230.4000,61.7000) -- (230.4000,60.7000) -- (230.4000,59.6000) -- (230.4000,58.6000) -- (230.4000,57.5000) -- (230.4000,56.5000) -- (230.4000,55.4000) -- (230.4000,54.4000) -- (230.4000,53.3000) -- (230.4000,52.2000) -- (230.4000,51.2000) -- (230.4000,50.1000) -- (230.4000,49.1000) -- (230.4000,48.0000) -- (230.5000,47.0000) -- (230.7000,45.9000) -- (230.8000,44.8000);



  \end{scope}
  \begin{scope}[cm={{1.00588,0.0,0.0,1.00588,(0.01175,0.51144)}},draw=black,line join=round,line cap=round,line width=0.480pt]
    \path[draw] (148.5000,13.5000) -- (148.5000,102.5000) -- (246.5000,102.5000) -- (246.5000,13.5000) -- (148.5000,13.5000);



  \end{scope}
  \begin{scope}[cm={{1.00588,0.0,0.0,1.00588,(0.01175,117.51142)}},draw=ca0a0a4,dash pattern=on 0.40pt off 0.80pt,line join=round,line cap=round,line width=0.400pt]
    \path[draw] (44.5000,210.5000) -- (179.5000,210.5000);



  \end{scope}
  \begin{scope}[cm={{1.00588,0.0,0.0,1.00588,(0.01175,117.51142)}},draw=black,line join=round,line cap=round,line width=0.480pt]
    \path[draw] (44.5000,210.5000) -- (48.5000,210.5000);



    \path[draw] (179.5000,210.5000) -- (175.5000,210.5000);



  \end{scope}
  \begin{scope}[cm={{1.00588,0.0,0.0,1.00588,(23.47712,330.77643)}},draw=black,line join=bevel,line cap=rect,line width=0.800pt]
    \path[fill=black] (0.0000,0.0000) node[above right] (text738) {-100};



  \end{scope}
  \begin{scope}[cm={{1.00588,0.0,0.0,1.00588,(0.01175,117.51142)}},draw=ca0a0a4,dash pattern=on 0.40pt off 0.80pt,line join=round,line cap=round,line width=0.400pt]
    \path[draw] (44.5000,181.5000) -- (179.5000,181.5000);



  \end{scope}
  \begin{scope}[cm={{1.00588,0.0,0.0,1.00588,(0.01175,117.51142)}},draw=black,line join=round,line cap=round,line width=0.480pt]
    \path[draw] (44.5000,181.5000) -- (48.5000,181.5000);



    \path[draw] (179.5000,181.5000) -- (175.5000,181.5000);



  \end{scope}
  \begin{scope}[cm={{1.00588,0.0,0.0,1.00588,(32.19999,304.60543)}},draw=black,line join=bevel,line cap=rect,line width=0.800pt]
    \path[fill=black] (0.0000,0.0000) node[above right] (text768) {0};



  \end{scope}
  \begin{scope}[cm={{1.00588,0.0,0.0,1.00588,(0.01175,117.51142)}},draw=ca0a0a4,dash pattern=on 0.40pt off 0.80pt,line join=round,line cap=round,line width=0.400pt]
    \path[draw] (44.5000,152.5000) -- (179.5000,152.5000);



  \end{scope}
  \begin{scope}[cm={{1.00588,0.0,0.0,1.00588,(0.01175,117.51142)}},draw=black,line join=round,line cap=round,line width=0.480pt]
    \path[draw] (44.5000,152.5000) -- (48.5000,152.5000);



    \path[draw] (179.5000,152.5000) -- (175.5000,152.5000);



  \end{scope}
  \begin{scope}[cm={{1.00588,0.0,0.0,1.00588,(24.15295,275.43543)}},draw=black,line join=bevel,line cap=rect,line width=0.800pt]
    \path[fill=black] (0.0000,0.0000) node[above right] (text798) {100};



  \end{scope}
  \begin{scope}[cm={{1.00588,0.0,0.0,1.00588,(0.01175,117.51142)}},draw=ca0a0a4,dash pattern=on 0.40pt off 0.80pt,line join=round,line cap=round,line width=0.400pt]
    \path[draw] (44.5000,123.5000) -- (179.5000,123.5000);



  \end{scope}
  \begin{scope}[cm={{1.00588,0.0,0.0,1.00588,(0.01175,117.51142)}},draw=black,line join=round,line cap=round,line width=0.480pt]
    \path[draw] (44.5000,123.5000) -- (48.5000,123.5000);



    \path[draw] (179.5000,123.5000) -- (175.5000,123.5000);



  \end{scope}
  \begin{scope}[cm={{1.00588,0.0,0.0,1.00588,(24.15295,246.26443)}},draw=black,line join=bevel,line cap=rect,line width=0.800pt]
    \path[fill=black] (0.0000,0.0000) node[above right] (text828) {200};



  \end{scope}
  \begin{scope}[cm={{1.00588,0.0,0.0,1.00588,(0.01175,117.51142)}},draw=ca0a0a4,dash pattern=on 0.40pt off 0.80pt,line join=round,line cap=round,line width=0.400pt]
    \path[draw] (44.5000,210.5000) -- (44.5000,108.5000);



  \end{scope}
  \begin{scope}[cm={{1.00588,0.0,0.0,1.00588,(0.01175,117.51142)}},draw=black,line join=round,line cap=round,line width=0.480pt]
    \path[draw] (44.5000,210.5000) -- (44.5000,208.5000);



    \path[draw] (44.5000,108.5000) -- (44.5000,111.5000);



  \end{scope}
  \begin{scope}[cm={{1.00588,0.0,0.0,1.00588,(0.01175,117.51142)}},draw=ca0a0a4,dash pattern=on 0.40pt off 0.80pt,line join=round,line cap=round,line width=0.400pt]
    \path[draw] (78.5000,210.5000) -- (78.5000,108.5000);



  \end{scope}
  \begin{scope}[cm={{1.00588,0.0,0.0,1.00588,(0.01175,117.51142)}},draw=black,line join=round,line cap=round,line width=0.480pt]
    \path[draw] (78.5000,210.5000) -- (78.5000,208.5000);



    \path[draw] (78.5000,108.5000) -- (78.5000,111.5000);



  \end{scope}
  \begin{scope}[cm={{1.00588,0.0,0.0,1.00588,(0.01175,117.51142)}},draw=ca0a0a4,dash pattern=on 0.40pt off 0.80pt,line join=round,line cap=round,line width=0.400pt]
    \path[draw] (112.5000,210.5000) -- (112.5000,108.5000);



  \end{scope}
  \begin{scope}[cm={{1.00588,0.0,0.0,1.00588,(0.01175,117.51142)}},draw=black,line join=round,line cap=round,line width=0.480pt]
    \path[draw] (112.5000,210.5000) -- (112.5000,208.5000);



    \path[draw] (112.5000,108.5000) -- (112.5000,111.5000);



  \end{scope}
  \begin{scope}[cm={{1.00588,0.0,0.0,1.00588,(0.01175,117.51142)}},draw=ca0a0a4,dash pattern=on 0.40pt off 0.80pt,line join=round,line cap=round,line width=0.400pt]
    \path[draw] (145.5000,210.5000) -- (145.5000,108.5000);



  \end{scope}
  \begin{scope}[cm={{1.00588,0.0,0.0,1.00588,(0.01175,117.51142)}},draw=black,line join=round,line cap=round,line width=0.480pt]
    \path[draw] (145.5000,210.5000) -- (145.5000,208.5000);



    \path[draw] (145.5000,108.5000) -- (145.5000,111.5000);



  \end{scope}
  \begin{scope}[cm={{1.00588,0.0,0.0,1.00588,(0.01175,117.51142)}},draw=ca0a0a4,dash pattern=on 0.40pt off 0.80pt,line join=round,line cap=round,line width=0.400pt]
    \path[draw] (179.5000,210.5000) -- (179.5000,108.5000);



  \end{scope}
  \begin{scope}[cm={{1.00588,0.0,0.0,1.00588,(0.01175,117.51142)}},draw=black,line join=round,line cap=round,line width=0.480pt]
    \path[draw] (179.5000,210.5000) -- (179.5000,208.5000);



    \path[draw] (179.5000,108.5000) -- (179.5000,111.5000);



  \end{scope}
  \begin{scope}[cm={{1.00588,0.0,0.0,1.00588,(0.01175,117.51142)}},draw=black,line join=round,line cap=round,line width=0.480pt]
    \path[draw] (44.5000,108.5000) -- (44.5000,210.5000) -- (179.5000,210.5000) -- (179.5000,108.5000) -- (44.5000,108.5000);



  \end{scope}
  \begin{scope}[cm={{1.00588,0.0,0.0,1.00588,(0.01175,117.51142)}},fill=cffffff]
    \path[fill,rounded corners=0.0000cm] (158.0000,115.0000) rectangle (169.0000,131.0000);



  \end{scope}
  \begin{scope}[cm={{1.00588,0.0,0.0,1.00588,(0.01175,117.51142)}},draw=black,line join=round,line cap=round,line width=0.800pt]
    \path[draw] (157.5000,131.5000) -- (157.5000,115.5000) -- (168.5000,115.5000) -- (168.5000,131.5000) -- (157.5000,131.5000);



  \end{scope}
  \begin{scope}[cm={{1.00588,0.0,0.0,1.00588,(161.95875,245.25843)}},draw=black,line join=bevel,line cap=rect,line width=0.800pt]
    \path[fill=black] (0.6481,-0.2160) node[above right] (text998) {\label{fig:trajs-dyn-i}i};



  \end{scope}
  \begin{scope}[cm={{1.00588,0.0,0.0,1.00588,(0.01175,117.51142)}},draw=black,line join=round,line cap=round,line width=0.480pt]
    \path[draw] (61.9000,118.7000) -- (61.9000,118.7000) -- (62.6000,120.9000) -- (63.3000,122.6000) -- (63.5000,124.1000) -- (62.7000,125.3000) -- (61.5000,126.5000) -- (60.3000,127.9000) -- (59.2000,129.5000) -- (58.3000,131.2000) -- (57.5000,133.0000) -- (57.0000,134.8000) -- (56.8000,136.7000) -- (56.7000,138.6000) -- (56.6000,140.4000) -- (56.6000,142.2000) -- (56.6000,144.1000) -- (56.6000,145.9000) -- (56.6000,147.7000) -- (56.6000,149.5000) -- (56.6000,151.4000) -- (56.6000,153.2000) -- (56.6000,155.1000) -- (56.6000,156.9000) -- (56.6000,158.8000) -- (56.6000,160.6000) -- (56.6000,162.5000) -- (56.6000,164.3000) -- (56.6000,166.2000) -- (56.6000,168.0000) -- (56.6000,169.9000) -- (56.6000,171.7000) -- (56.6000,173.6000) -- (56.6000,175.4000) -- (56.6000,177.2000) -- (56.6000,179.1000) -- (56.7000,181.0000) -- (57.1000,182.8000) -- (57.6000,184.7000) -- (58.3000,186.4000) -- (59.2000,188.2000) -- (60.2000,189.8000) -- (61.4000,191.3000) -- (62.7000,192.8000) -- (64.2000,194.1000) -- (65.7000,195.3000) -- (67.4000,196.4000) -- (69.1000,197.4000) -- (71.0000,198.2000) -- (72.9000,198.9000) -- (74.8000,199.5000) -- (76.8000,199.9000) -- (78.8000,200.2000) -- (80.8000,200.4000) -- (82.8000,200.4000) -- (84.8000,200.2000) -- (86.7000,200.0000) -- (88.7000,199.6000) -- (90.6000,199.0000) -- (92.5000,198.4000) -- (94.3000,197.5000) -- (96.1000,196.6000) -- (97.8000,195.5000) -- (99.3000,194.3000) -- (100.8000,193.0000) -- (102.1000,191.6000) -- (103.3000,190.0000) -- (104.2000,188.4000) -- (104.9000,186.7000) -- (105.3000,184.9000) -- (105.6000,183.0000) -- (105.7000,181.1000) -- (105.8000,179.2000) -- (105.8000,177.4000) -- (105.8000,175.5000) -- (105.8000,173.6000) -- (105.8000,171.7000) -- (105.8000,169.9000) -- (105.9000,168.0000) -- (105.9000,166.2000) -- (105.9000,164.3000) -- (105.9000,162.5000) -- (105.9000,160.6000) -- (105.9000,158.8000) -- (105.9000,157.0000) -- (105.9000,155.1000) -- (105.9000,153.3000) -- (105.9000,151.4000) -- (105.9000,149.6000) -- (105.9000,147.7000) -- (105.8000,145.9000) -- (106.1000,144.0000) -- (106.5000,142.1000) -- (106.8000,140.3000) -- (106.8000,138.6000) -- (106.7000,136.8000) -- (106.4000,135.0000) -- (105.9000,133.2000) -- (105.2000,131.5000) -- (104.3000,129.8000) -- (103.3000,128.3000) -- (102.1000,126.8000) -- (100.8000,125.4000) -- (99.3000,124.1000) -- (97.7000,122.9000) -- (96.0000,121.9000) -- (94.3000,121.0000) -- (92.4000,120.3000) -- (90.5000,119.7000) -- (88.5000,119.2000) -- (86.5000,118.9000) -- (84.5000,118.8000) -- (82.4000,118.8000) -- (80.4000,119.0000) -- (78.4000,119.3000) -- (76.4000,119.8000) -- (74.5000,120.4000) -- (72.6000,121.2000) -- (70.9000,122.1000) -- (69.2000,123.1000) -- (67.6000,124.3000) -- (66.2000,125.6000) -- (64.9000,127.0000) -- (63.8000,128.5000) -- (62.8000,130.1000) -- (62.0000,131.8000) -- (61.3000,133.6000) -- (60.9000,135.4000) -- (60.7000,137.2000) -- (60.6000,139.1000) -- (60.6000,140.9000) -- (60.6000,142.7000) -- (60.6000,144.5000) -- (60.6000,146.4000) -- (60.6000,148.2000) -- (60.6000,150.0000) -- (60.6000,151.9000) -- (60.6000,153.7000) -- (60.5000,155.6000) -- (60.5000,157.4000) -- (60.5000,159.3000) -- (60.5000,161.1000) -- (60.5000,163.0000) -- (60.5000,164.8000) -- (60.5000,166.7000) -- (60.5000,168.5000) -- (60.6000,170.4000) -- (60.6000,172.2000) -- (60.6000,174.1000) -- (60.6000,175.9000) -- (60.5000,177.7000) -- (60.6000,179.6000) -- (60.8000,181.5000) -- (61.2000,183.3000) -- (61.9000,185.1000) -- (62.7000,186.9000) -- (63.6000,188.6000) -- (64.7000,190.1000) -- (66.0000,191.6000) -- (67.4000,193.0000) -- (68.9000,194.3000) -- (70.6000,195.4000) -- (72.3000,196.4000) -- (74.1000,197.3000) -- (76.0000,198.1000) -- (77.9000,198.7000) -- (79.8000,199.2000) -- (81.8000,199.5000) -- (83.8000,199.7000) -- (85.8000,199.7000) -- (87.8000,199.6000) -- (89.8000,199.4000) -- (91.8000,199.0000) -- (93.7000,198.5000) -- (95.6000,197.8000) -- (97.4000,197.0000) -- (99.2000,196.1000) -- (100.8000,195.0000) -- (102.4000,193.8000) -- (103.9000,192.5000) -- (105.2000,191.1000) -- (106.4000,189.5000) -- (107.3000,187.9000) -- (108.0000,186.2000) -- (108.5000,184.4000) -- (108.7000,182.5000) -- (108.9000,180.6000) -- (109.0000,178.7000) -- (109.0000,176.9000) -- (109.0000,175.0000) -- (109.0000,173.1000) -- (109.0000,171.2000) -- (109.1000,169.4000) -- (109.1000,167.5000) -- (109.1000,165.7000) -- (109.1000,163.8000) -- (109.1000,162.0000) -- (109.1000,160.1000) -- (109.1000,158.3000) -- (109.1000,156.5000) -- (109.1000,154.6000) -- (109.1000,152.8000) -- (109.1000,150.9000) -- (109.1000,149.1000) -- (109.1000,147.2000) -- (109.1000,145.4000) -- (109.3000,143.5000) -- (109.7000,141.6000) -- (110.0000,139.8000) -- (110.0000,138.1000) -- (109.8000,136.3000) -- (109.5000,134.5000) -- (109.0000,132.7000) -- (108.3000,131.0000) -- (107.4000,129.4000) -- (106.3000,127.8000) -- (105.1000,126.3000) -- (103.7000,125.0000) -- (102.3000,123.7000) -- (100.6000,122.6000) -- (98.9000,121.6000) -- (97.1000,120.7000) -- (95.2000,120.0000) -- (93.3000,119.5000) -- (91.3000,119.1000) -- (89.3000,118.8000) -- (87.3000,118.7000) -- (85.2000,118.8000) -- (83.2000,119.0000) -- (81.2000,119.4000) -- (79.3000,119.9000) -- (77.4000,120.6000) -- (75.6000,121.5000) -- (73.9000,122.6000) -- (72.4000,123.8000) -- (70.9000,125.1000) -- (69.6000,126.5000) -- (68.5000,128.0000) -- (67.5000,129.6000) -- (66.7000,131.3000) -- (66.0000,133.0000) -- (65.7000,134.9000) -- (65.5000,136.7000) -- (65.4000,138.6000) -- (65.4000,140.4000) -- (65.4000,142.2000) -- (65.4000,144.0000) -- (65.3000,145.8000) -- (65.3000,147.7000) -- (65.3000,149.5000) -- (65.3000,151.4000) -- (65.3000,153.2000) -- (65.3000,155.1000) -- (65.3000,156.9000) -- (65.3000,158.8000) -- (65.3000,160.6000) -- (65.3000,162.4000) -- (65.3000,164.3000) -- (65.3000,166.1000) -- (65.3000,168.0000) -- (65.3000,169.8000) -- (65.3000,171.7000) -- (65.3000,173.5000) -- (65.3000,175.4000) -- (65.3000,177.2000) -- (65.4000,179.1000) -- (65.6000,181.0000) -- (66.0000,182.8000) -- (66.6000,184.6000) -- (67.4000,186.4000) -- (68.3000,188.1000) -- (69.4000,189.7000) -- (70.6000,191.2000) -- (71.9000,192.6000) -- (73.4000,193.9000) -- (75.0000,195.1000) -- (76.7000,196.2000) -- (78.5000,197.1000) -- (80.3000,198.0000) -- (82.2000,198.7000) -- (84.1000,199.2000) -- (86.1000,199.6000) -- (88.1000,199.9000) -- (90.1000,200.1000) -- (92.1000,200.2000) -- (94.1000,200.1000) -- (96.1000,199.8000) -- (98.0000,199.5000) -- (100.0000,199.0000) -- (101.9000,198.4000) -- (103.8000,197.7000) -- (105.6000,196.9000) -- (107.3000,195.9000) -- (109.0000,194.8000) -- (110.6000,193.6000) -- (112.0000,192.3000) -- (113.4000,190.9000) -- (114.6000,189.4000) -- (115.4000,187.7000) -- (115.9000,185.9000) -- (116.1000,184.1000) -- (116.2000,182.2000) -- (116.2000,180.3000) -- (116.3000,178.4000) -- (116.3000,176.5000) -- (116.3000,174.6000) -- (116.3000,172.8000) -- (116.3000,170.9000) -- (116.3000,169.1000) -- (116.3000,167.2000) -- (116.3000,165.4000) -- (116.3000,163.5000) -- (116.3000,161.7000) -- (116.3000,159.8000) -- (116.3000,158.0000) -- (116.3000,156.1000) -- (116.3000,154.3000) -- (116.3000,152.4000) -- (116.3000,150.6000) -- (116.3000,148.8000) -- (116.3000,146.9000) -- (116.3000,145.1000) -- (116.3000,143.2000) -- (116.5000,141.3000) -- (116.7000,139.5000) -- (116.7000,137.7000) -- (116.6000,135.9000) -- (116.3000,134.1000) -- (115.8000,132.3000) -- (115.1000,130.6000) -- (114.2000,129.0000) -- (113.2000,127.4000) -- (112.0000,125.9000) -- (110.6000,124.6000) -- (109.1000,123.3000) -- (107.5000,122.2000) -- (105.8000,121.2000) -- (104.0000,120.3000) -- (102.1000,119.6000) -- (100.2000,119.0000) -- (98.2000,118.6000) -- (96.2000,118.4000) -- (94.1000,118.3000) -- (92.1000,118.4000) -- (90.1000,118.6000) -- (88.1000,119.0000) -- (86.1000,119.5000) -- (84.2000,120.2000) -- (82.4000,121.1000) -- (80.7000,122.0000) -- (79.1000,123.2000) -- (77.6000,124.4000) -- (76.2000,125.8000) -- (75.0000,127.3000) -- (74.0000,128.8000) -- (73.1000,130.5000) -- (72.4000,132.2000) -- (72.0000,134.0000) -- (71.8000,135.9000) -- (71.7000,137.7000) -- (71.6000,139.5000) -- (71.6000,141.4000) -- (71.6000,143.2000) -- (71.6000,145.0000) -- (71.6000,146.8000) -- (71.6000,148.7000) -- (71.6000,150.5000) -- (71.6000,152.4000) -- (71.6000,154.2000) -- (71.6000,156.1000) -- (71.6000,157.9000) -- (71.6000,159.8000) -- (71.6000,161.6000) -- (71.6000,163.5000) -- (71.6000,165.3000) -- (71.6000,167.2000) -- (71.6000,169.0000) -- (71.6000,170.8000) -- (71.6000,172.7000) -- (71.6000,174.5000) -- (71.6000,176.4000) -- (71.6000,178.2000) -- (71.8000,180.1000) -- (72.2000,182.0000) -- (72.7000,183.8000) -- (73.5000,185.6000) -- (74.4000,187.3000) -- (75.5000,188.9000) -- (76.7000,190.4000) -- (78.0000,191.8000) -- (79.5000,193.1000) -- (81.1000,194.3000) -- (82.8000,195.4000) -- (84.6000,196.3000) -- (86.4000,197.1000) -- (88.3000,197.8000) -- (90.3000,198.3000) -- (92.2000,198.7000) -- (94.2000,199.0000) -- (96.2000,199.1000) -- (98.2000,199.1000) -- (100.2000,198.9000) -- (102.2000,198.6000) -- (104.2000,198.2000) -- (106.1000,197.6000) -- (107.9000,196.9000) -- (109.7000,196.0000) -- (111.5000,195.1000) -- (113.1000,194.0000) -- (114.7000,192.7000) -- (116.1000,191.4000) -- (117.4000,189.9000) -- (118.5000,188.4000) -- (119.4000,186.7000) -- (120.0000,185.0000) -- (120.4000,183.2000) -- (120.6000,181.3000) -- (120.7000,179.4000) -- (120.8000,177.5000) -- (120.8000,175.6000) -- (120.8000,173.8000) -- (120.8000,171.9000) -- (120.8000,170.0000) -- (120.9000,168.2000) -- (120.9000,166.3000) -- (120.9000,164.5000) -- (120.9000,162.6000) -- (120.9000,160.8000) -- (120.9000,158.9000) -- (120.9000,157.1000) -- (120.9000,155.2000) -- (120.9000,153.4000) -- (120.9000,151.5000) -- (120.9000,149.7000) -- (120.9000,147.8000) -- (120.9000,146.0000) -- (120.9000,144.1000) -- (121.2000,142.3000) -- (121.7000,140.4000) -- (121.9000,138.6000) -- (121.9000,136.9000) -- (121.8000,135.1000) -- (121.4000,133.3000) -- (120.9000,131.6000) -- (120.1000,129.9000) -- (119.2000,128.2000) -- (118.1000,126.7000) -- (116.9000,125.2000) -- (115.5000,123.9000) -- (114.0000,122.7000) -- (112.3000,121.6000) -- (110.6000,120.6000) -- (108.7000,119.8000) -- (106.8000,119.2000) -- (104.9000,118.7000) -- (102.8000,118.4000) -- (100.8000,118.2000) -- (98.8000,118.2000) -- (96.7000,118.4000) -- (94.7000,118.7000) -- (92.8000,119.2000) -- (90.9000,119.8000) -- (89.0000,120.6000) -- (87.3000,121.6000) -- (85.6000,122.6000) -- (84.1000,123.9000) -- (82.7000,125.2000) -- (81.5000,126.7000) -- (80.4000,128.2000) -- (79.5000,129.9000) -- (78.7000,131.6000) -- (78.3000,133.4000) -- (78.0000,135.2000) -- (77.9000,137.1000) -- (77.9000,138.9000) -- (77.8000,140.7000) -- (77.8000,142.5000) -- (77.8000,144.4000) -- (77.8000,146.2000) -- (77.8000,148.0000) -- (77.8000,149.9000) -- (77.8000,151.7000) -- (77.8000,153.6000) -- (77.8000,155.4000) -- (77.8000,157.3000) -- (77.8000,159.1000) -- (77.8000,161.0000) -- (77.8000,162.8000) -- (77.8000,164.6000) -- (77.8000,166.5000) -- (77.8000,168.3000) -- (77.8000,170.2000) -- (77.8000,172.0000) -- (77.8000,173.9000) -- (77.8000,175.7000) -- (77.8000,177.6000) -- (78.0000,179.5000) -- (78.4000,181.3000) -- (78.9000,183.1000) -- (79.7000,184.9000) -- (80.6000,186.6000) -- (81.7000,188.2000) -- (82.9000,189.7000) -- (84.2000,191.1000) -- (85.7000,192.5000) -- (87.3000,193.6000) -- (89.0000,194.7000) -- (90.8000,195.6000) -- (92.6000,196.4000) -- (94.5000,197.1000) -- (96.5000,197.6000) -- (98.5000,198.0000) -- (100.5000,198.2000) -- (102.5000,198.3000) -- (104.5000,198.3000) -- (106.5000,198.1000) -- (108.4000,197.7000) -- (110.4000,197.3000) -- (112.3000,196.7000) -- (114.1000,195.9000) -- (115.9000,195.0000) -- (117.6000,194.0000) -- (119.2000,192.9000) -- (120.8000,191.6000) -- (122.1000,190.2000) -- (123.4000,188.7000) -- (124.5000,187.1000) -- (125.3000,185.5000) -- (125.8000,183.7000) -- (126.1000,181.8000) -- (126.3000,180.0000) -- (126.4000,178.1000) -- (126.5000,176.2000) -- (126.5000,174.3000) -- (126.5000,172.4000) -- (126.5000,170.6000) -- (126.5000,168.7000) -- (126.5000,166.9000) -- (126.5000,165.0000) -- (126.5000,163.2000) -- (126.6000,161.3000) -- (126.6000,159.5000) -- (126.6000,157.6000) -- (126.6000,155.8000) -- (126.6000,153.9000) -- (126.6000,152.1000) -- (126.5000,150.2000) -- (126.5000,148.4000) -- (126.5000,146.5000) -- (126.5000,144.7000) -- (126.6000,142.8000) -- (127.1000,141.0000) -- (127.5000,139.2000) -- (127.6000,137.4000) -- (127.6000,135.6000) -- (127.3000,133.8000) -- (126.9000,132.0000) -- (126.3000,130.3000) -- (125.5000,128.6000) -- (124.5000,127.0000) -- (123.3000,125.5000) -- (122.0000,124.1000) -- (120.5000,122.8000) -- (118.9000,121.7000) -- (117.2000,120.7000) -- (115.4000,119.8000) -- (113.5000,119.1000) -- (111.6000,118.5000) -- (109.6000,118.1000) -- (107.6000,117.9000) -- (105.6000,117.8000) -- (103.5000,117.9000) -- (101.5000,118.2000) -- (99.5000,118.6000) -- (97.6000,119.2000) -- (95.7000,119.9000) -- (93.9000,120.8000) -- (92.2000,121.8000) -- (90.7000,123.0000) -- (89.2000,124.3000) -- (87.9000,125.7000) -- (86.8000,127.2000) -- (85.8000,128.8000) -- (85.0000,130.5000) -- (84.5000,132.3000) -- (84.1000,134.1000) -- (84.0000,136.0000) -- (83.9000,137.8000) -- (83.9000,139.6000) -- (83.9000,141.4000) -- (83.9000,143.3000) -- (83.8000,145.1000) -- (83.8000,146.9000) -- (83.8000,148.8000) -- (83.8000,150.6000) -- (83.8000,152.5000) -- (83.8000,154.3000) -- (83.8000,156.2000) -- (83.8000,158.0000) -- (83.8000,159.9000) -- (83.8000,161.7000) -- (83.8000,163.5000) -- (83.8000,165.4000) -- (83.8000,167.2000) -- (83.8000,169.1000) -- (83.8000,170.9000) -- (83.8000,172.8000) -- (83.8000,174.6000) -- (83.8000,176.5000) -- (83.9000,178.3000) -- (84.2000,180.2000) -- (84.8000,182.1000) -- (85.5000,183.8000) -- (86.3000,185.6000) -- (87.3000,187.2000) -- (88.5000,188.7000) -- (89.9000,190.2000) -- (91.3000,191.5000) -- (92.9000,192.7000) -- (94.5000,193.8000) -- (96.3000,194.8000) -- (98.1000,195.6000) -- (100.0000,196.3000) -- (102.0000,196.9000) -- (103.9000,197.3000) -- (105.9000,197.6000) -- (107.9000,197.7000) -- (109.9000,197.7000) -- (111.9000,197.5000) -- (113.9000,197.2000) -- (115.8000,196.8000) -- (117.7000,196.2000) -- (119.6000,195.5000) -- (121.4000,194.6000) -- (123.1000,193.6000) -- (124.8000,192.5000) -- (126.3000,191.3000) -- (127.7000,189.9000) -- (129.0000,188.4000) -- (130.1000,186.9000) -- (130.9000,185.2000) -- (131.5000,183.4000) -- (131.9000,181.6000) -- (132.1000,179.7000) -- (132.2000,177.8000) -- (132.3000,176.0000) -- (132.3000,174.1000) -- (132.3000,172.2000) -- (132.3000,170.3000) -- (132.3000,168.5000) -- (132.3000,166.6000) -- (132.3000,164.8000) -- (132.3000,162.9000) -- (132.3000,161.1000) -- (132.3000,159.2000) -- (132.3000,157.4000) -- (132.3000,155.5000) -- (132.3000,153.7000) -- (132.3000,151.8000) -- (132.3000,150.0000) -- (132.3000,148.1000) -- (132.3000,146.3000) -- (132.3000,144.4000) -- (132.4000,142.6000) -- (132.9000,140.7000) -- (133.2000,138.9000) -- (133.4000,137.1000) -- (133.4000,135.3000) -- (133.2000,133.5000) -- (132.8000,131.8000) -- (132.2000,130.0000) -- (131.3000,128.4000) -- (130.3000,126.8000) -- (129.1000,125.3000) -- (127.8000,123.9000) -- (126.3000,122.7000) -- (124.7000,121.6000) -- (122.9000,120.6000) -- (121.1000,119.8000) -- (119.2000,119.2000) -- (117.2000,118.7000) -- (115.2000,118.4000) -- (113.1000,118.3000) -- (111.1000,118.3000) -- (109.1000,118.6000) -- (107.1000,119.0000) -- (105.2000,119.6000) -- (103.3000,120.3000) -- (101.5000,121.2000) -- (99.9000,122.3000) -- (98.3000,123.5000) -- (96.9000,124.8000) -- (95.7000,126.3000) -- (94.6000,127.9000) -- (93.8000,129.5000) -- (93.1000,131.3000) -- (92.7000,133.1000) -- (92.6000,134.9000) -- (92.5000,136.8000) -- (92.4000,138.6000) -- (92.4000,140.4000) -- (92.4000,142.2000) -- (92.4000,144.1000) -- (92.4000,145.9000) -- (92.4000,147.7000) -- (92.4000,149.6000) -- (92.4000,151.4000) -- (92.4000,153.3000) -- (92.4000,155.1000) -- (92.4000,157.0000) -- (92.4000,158.8000) -- (92.4000,160.7000) -- (92.4000,162.5000) -- (92.4000,164.4000) -- (92.4000,166.2000) -- (92.4000,168.1000) -- (92.4000,169.9000) -- (92.4000,171.8000) -- (92.4000,173.6000) -- (92.4000,175.4000) -- (92.4000,177.3000) -- (92.7000,179.2000) -- (93.2000,181.0000) -- (93.8000,182.8000) -- (94.7000,184.5000) -- (95.7000,186.2000) -- (96.9000,187.8000) -- (98.2000,189.2000) -- (99.6000,190.6000) -- (101.1000,191.8000) -- (102.8000,192.9000) -- (104.5000,193.9000) -- (106.4000,194.8000) -- (108.2000,195.5000) -- (110.2000,196.0000) -- (112.1000,196.5000) -- (114.1000,196.8000) -- (116.1000,196.9000) -- (118.1000,196.9000) -- (120.1000,196.8000) -- (122.1000,196.5000) -- (124.1000,196.1000) -- (126.0000,195.5000) -- (127.8000,194.8000) -- (129.7000,194.0000) -- (131.4000,193.0000) -- (133.0000,191.9000) -- (134.6000,190.6000) -- (136.0000,189.3000) -- (137.3000,187.8000) -- (138.4000,186.3000) -- (139.3000,184.6000) -- (139.9000,182.8000) -- (140.3000,181.0000) -- (140.5000,179.2000) -- (140.6000,177.3000) -- (140.7000,175.4000) -- (140.7000,173.5000) -- (140.7000,171.6000) -- (140.7000,169.8000) -- (140.8000,167.9000) -- (140.8000,166.0000) -- (140.8000,164.2000) -- (140.8000,162.3000) -- (140.8000,160.5000) -- (140.8000,158.6000) -- (140.8000,156.8000) -- (140.8000,154.9000) -- (140.8000,153.1000) -- (140.8000,151.3000) -- (140.8000,149.4000) -- (140.8000,147.6000) -- (140.8000,145.7000) -- (140.8000,143.9000) -- (140.8000,142.0000) -- (141.3000,140.1000) -- (141.7000,138.3000) -- (142.0000,136.5000) -- (142.0000,134.8000) -- (141.8000,133.0000) -- (141.4000,131.2000) -- (140.8000,129.5000) -- (140.0000,127.8000) -- (139.0000,126.2000) -- (137.9000,124.7000) -- (136.5000,123.3000) -- (135.1000,122.0000) -- (133.5000,120.9000) -- (131.7000,119.9000) -- (129.9000,119.1000) -- (128.0000,118.4000) -- (126.1000,117.9000) -- (124.0000,117.6000) -- (122.0000,117.4000) -- (120.0000,117.4000) -- (117.9000,117.6000) -- (115.9000,118.0000) -- (114.0000,118.5000) -- (112.1000,119.2000) -- (110.3000,120.1000) -- (108.6000,121.1000) -- (107.0000,122.2000) -- (105.6000,123.5000) -- (104.3000,124.9000) -- (103.2000,126.5000) -- (102.2000,128.1000) -- (101.5000,129.8000) -- (101.0000,131.6000) -- (100.8000,133.5000) -- (100.6000,135.3000) -- (100.6000,137.1000) -- (100.6000,138.9000) -- (100.6000,140.8000) -- (100.5000,142.6000) -- (100.5000,144.4000) -- (100.5000,146.3000) -- (100.5000,148.1000) -- (100.5000,149.9000) -- (100.5000,151.8000) -- (100.5000,153.6000) -- (100.5000,155.5000) -- (100.5000,157.3000) -- (100.5000,159.2000) -- (100.5000,161.0000) -- (100.5000,162.9000) -- (100.5000,164.7000) -- (100.5000,166.6000) -- (100.5000,168.4000) -- (100.5000,170.3000) -- (100.5000,172.1000) -- (100.5000,174.0000) -- (100.5000,175.8000) -- (100.7000,177.7000) -- (101.1000,179.6000) -- (101.7000,181.4000) -- (102.5000,183.1000) -- (103.4000,184.8000) -- (104.5000,186.4000) -- (105.8000,187.9000) -- (107.1000,189.3000) -- (108.6000,190.6000) -- (110.2000,191.8000) -- (111.9000,192.8000) -- (113.7000,193.7000) -- (115.6000,194.5000) -- (117.5000,195.2000) -- (119.4000,195.7000) -- (121.4000,196.0000) -- (123.4000,196.2000) -- (125.4000,196.3000) -- (127.4000,196.2000) -- (129.4000,196.0000) -- (131.4000,195.7000) -- (133.3000,195.2000) -- (135.2000,194.6000) -- (137.1000,193.8000) -- (138.8000,192.9000) -- (140.5000,191.9000) -- (142.1000,190.7000) -- (143.6000,189.4000) -- (145.0000,188.0000) -- (146.2000,186.5000) -- (147.3000,184.9000) -- (148.0000,183.2000) -- (148.5000,181.4000) -- (148.8000,179.6000) -- (149.0000,177.7000) -- (149.1000,175.8000) -- (149.1000,173.9000) -- (149.2000,172.1000) -- (149.2000,170.2000) -- (149.2000,168.3000) -- (149.2000,166.5000) -- (149.2000,164.6000) -- (149.2000,162.8000) -- (149.2000,160.9000) -- (149.2000,159.1000) -- (149.2000,157.2000) -- (149.2000,155.4000) -- (149.2000,153.5000) -- (149.2000,151.7000) -- (149.2000,149.8000) -- (149.2000,148.0000) -- (149.2000,146.1000) -- (149.2000,144.3000) -- (149.2000,142.4000) -- (149.4000,140.6000) -- (149.9000,138.7000) -- (150.3000,136.9000) -- (150.4000,135.1000) -- (150.4000,133.4000) -- (150.1000,131.6000) -- (149.7000,129.8000) -- (149.0000,128.1000) -- (148.1000,126.5000) -- (147.1000,124.9000) -- (145.9000,123.4000) -- (144.5000,122.1000) -- (142.9000,120.8000) -- (141.3000,119.8000) -- (139.5000,118.8000) -- (137.7000,118.0000) -- (135.8000,117.4000) -- (133.8000,117.0000) -- (131.8000,116.7000) -- (129.7000,116.6000) -- (127.7000,116.7000) -- (125.7000,117.0000) -- (123.7000,117.4000) -- (121.7000,118.0000) -- (119.9000,118.7000) -- (118.1000,119.6000) -- (116.4000,120.7000) -- (114.9000,121.9000) -- (113.5000,123.2000) -- (112.3000,124.7000) -- (111.2000,126.3000) -- (110.3000,127.9000) -- (109.7000,129.7000) -- (109.3000,131.5000) -- (109.1000,133.3000) -- (109.0000,135.2000) -- (109.0000,137.0000) -- (109.0000,138.8000) -- (108.9000,140.6000) -- (108.9000,142.5000) -- (108.9000,144.3000) -- (108.9000,146.1000) -- (108.9000,148.0000) -- (108.9000,149.8000) -- (108.9000,151.7000) -- (108.9000,153.5000) -- (108.9000,155.4000) -- (108.9000,157.2000) -- (108.9000,159.1000) -- (108.9000,160.9000) -- (108.9000,162.8000) -- (108.9000,164.6000) -- (108.9000,166.5000) -- (108.9000,168.3000) -- (108.9000,170.1000) -- (108.9000,172.0000) -- (108.9000,173.8000) -- (109.0000,175.7000) -- (109.2000,177.6000) -- (109.7000,179.4000) -- (110.4000,181.2000) -- (111.2000,183.0000) -- (112.2000,184.6000) -- (113.3000,186.2000) -- (114.6000,187.7000) -- (116.0000,189.0000) -- (117.5000,190.3000) -- (119.2000,191.4000) -- (120.9000,192.4000) -- (122.7000,193.3000) -- (124.6000,194.0000) -- (126.5000,194.6000) -- (128.5000,195.1000) -- (130.5000,195.4000) -- (132.5000,195.6000) -- (134.5000,195.6000) -- (136.5000,195.5000) -- (138.5000,195.2000) -- (140.4000,194.8000) -- (142.3000,194.3000) -- (144.2000,193.6000) -- (146.0000,192.8000) -- (147.8000,191.9000) -- (149.5000,190.8000) -- (151.0000,189.6000) -- (152.5000,188.3000) -- (153.8000,186.9000) -- (155.0000,185.3000) -- (156.0000,183.7000) -- (156.6000,182.0000) -- (157.1000,180.2000) -- (157.3000,178.3000) -- (157.5000,176.4000) -- (157.6000,174.5000) -- (157.6000,172.7000) -- (157.6000,170.8000) -- (157.6000,168.9000) -- (157.6000,167.0000) -- (157.6000,165.2000) -- (157.6000,163.3000) -- (157.7000,161.5000) -- (157.7000,159.6000) -- (157.7000,157.8000) -- (157.7000,155.9000) -- (157.7000,154.1000) -- (157.7000,152.2000) -- (157.7000,150.4000) -- (157.7000,148.6000) -- (157.7000,146.7000) -- (157.7000,144.9000) -- (157.6000,143.0000) -- (157.6000,141.2000) -- (158.0000,139.3000) -- (158.5000,137.4000) -- (158.8000,135.6000);



  \end{scope}
  \begin{scope}[cm={{1.00588,0.0,0.0,1.00588,(0.01175,-137.48856)}},draw=black,line join=round,line cap=round,line width=0.480pt]
    \path[shift={(0,253.5093)},draw] (44.5000,108.5000) -- (44.5000,210.5000) -- (179.5000,210.5000) -- (179.5000,108.5000) -- (44.5000,108.5000);



  \end{scope}
  \begin{scope}[cm={{1.00588,0.0,0.0,1.00588,(0.01175,129.51142)}},draw=ca0a0a4,dash pattern=on 0.40pt off 0.80pt,line join=round,line cap=round,line width=0.400pt]
    \path[shift={(0,-5.96493)},draw] (184.5000,151.5000) -- (246.5000,151.5000);



  \end{scope}
  \begin{scope}[cm={{1.00588,0.0,0.0,1.00588,(0.01175,123.51142)}},draw=black,line join=round,line cap=round,line width=0.480pt]
    \path[draw] (184.5000,151.5000) -- (186.5000,151.5000);



    \path[draw] (246.5000,151.5000) -- (245.5000,151.5000);



  \end{scope}
  \begin{scope}[cm={{1.00588,0.0,0.0,1.00588,(0.01175,123.51142)}},draw=ca0a0a4,dash pattern=on 0.40pt off 0.80pt,line join=round,line cap=round,line width=0.400pt]
    \path[draw] (184.5000,132.5000) -- (246.5000,132.5000);



  \end{scope}
  \begin{scope}[cm={{1.00588,0.0,0.0,1.00588,(0.01175,123.51142)}},draw=black,line join=round,line cap=round,line width=0.480pt]
    \path[draw] (184.5000,132.5000) -- (186.5000,132.5000);



    \path[draw] (246.5000,132.5000) -- (245.5000,132.5000);



  \end{scope}
  \begin{scope}[cm={{1.00588,0.0,0.0,1.00588,(0.01175,123.51142)}},draw=ca0a0a4,dash pattern=on 0.40pt off 0.80pt,line join=round,line cap=round,line width=0.400pt]
    \path[draw] (184.5000,113.5000) -- (246.5000,113.5000);



  \end{scope}
  \begin{scope}[cm={{1.00588,0.0,0.0,1.00588,(0.01175,129.51142)}},draw=black,line join=round,line cap=round,line width=0.480pt]
    \path[shift={(0,-5.96493)},draw] (184.5000,113.5000) -- (186.5000,113.5000);



    \path[shift={(0,-5.96493)},draw] (246.5000,113.5000) -- (245.5000,113.5000);



  \end{scope}
  \begin{scope}[cm={{1.00588,0.0,0.0,1.00588,(0.01175,123.51142)}},draw=ca0a0a4,dash pattern=on 0.40pt off 0.80pt,line join=round,line cap=round,line width=0.400pt]
    \path[draw] (184.5000,156.5000) -- (184.5000,108.5000);



  \end{scope}
  \begin{scope}[cm={{1.00588,0.0,0.0,1.00588,(0.01175,123.51142)}},draw=black,line join=round,line cap=round,line width=0.480pt]
    \path[draw] (184.5000,156.5000) -- (184.5000,155.5000);



    \path[draw] (184.5000,108.5000) -- (184.5000,109.5000);



  \end{scope}
  \begin{scope}[cm={{1.00588,0.0,0.0,1.00588,(0.01175,123.51142)}},draw=ca0a0a4,dash pattern=on 0.40pt off 0.80pt,line join=round,line cap=round,line width=0.400pt]
    \path[draw] (203.5000,156.5000) -- (203.5000,108.5000);



  \end{scope}
  \begin{scope}[cm={{1.00588,0.0,0.0,1.00588,(0.01175,123.51142)}},draw=black,line join=round,line cap=round,line width=0.480pt]
    \path[draw] (203.5000,156.5000) -- (203.5000,155.5000);



    \path[draw] (203.5000,108.5000) -- (203.5000,109.5000);



  \end{scope}
  \begin{scope}[cm={{1.00588,0.0,0.0,1.00588,(0.01175,123.51142)}},draw=ca0a0a4,dash pattern=on 0.40pt off 0.80pt,line join=round,line cap=round,line width=0.400pt]
    \path[draw] (221.5000,156.5000) -- (221.5000,108.5000);



  \end{scope}
  \begin{scope}[cm={{1.00588,0.0,0.0,1.00588,(0.01175,123.51142)}},draw=black,line join=round,line cap=round,line width=0.480pt]
    \path[draw] (221.5000,156.5000) -- (221.5000,155.5000);



    \path[draw] (221.5000,108.5000) -- (221.5000,109.5000);



  \end{scope}
  \begin{scope}[cm={{1.00588,0.0,0.0,1.00588,(0.01175,123.51142)}},draw=ca0a0a4,dash pattern=on 0.40pt off 0.80pt,line join=round,line cap=round,line width=0.400pt]
    \path[draw] (239.5000,156.5000) -- (239.5000,108.5000);



  \end{scope}
  \begin{scope}[cm={{1.00588,0.0,0.0,1.00588,(0.01175,123.51142)}},draw=black,line join=round,line cap=round,line width=0.480pt]
    \path[draw] (239.5000,156.5000) -- (239.5000,155.5000);



    \path[draw] (239.5000,108.5000) -- (239.5000,109.5000);



  \end{scope}
  \begin{scope}[cm={{1.00588,0.0,0.0,1.00588,(0.01175,129.51142)}},draw=black,line join=round,line cap=round,line width=0.480pt]
    \path[shift={(0,-5.96493)},draw] (246.5000,151.5000) -- (245.5000,151.5000);



  \end{scope}
  \begin{scope}[cm={{1.00588,0.0,0.0,1.00588,(255.22933,277.42943)}},draw=black,line join=bevel,line cap=rect,line width=0.800pt]
    \path[fill=black] (0.0000,0.0000) node[above right] (text1220) {\scriptsize 2};



  \end{scope}
  \begin{scope}[cm={{1.00588,0.0,0.0,1.00588,(0.01175,129.51142)}},draw=black,line join=round,line cap=round,line width=0.480pt]
    \path[shift={(0,-5.96493)},draw] (246.5000,132.5000) -- (245.5000,132.5000);



  \end{scope}
  \begin{scope}[cm={{1.00588,0.0,0.0,1.00588,(255.19715,259.81743)}},draw=black,line join=bevel,line cap=rect,line width=0.800pt]
    \path[fill=black] (0.0000,0.0000) node[above right] (text1244) {\scriptsize 6};



  \end{scope}
  \begin{scope}[cm={{1.00588,0.0,0.0,1.00588,(0.01175,123.51142)}},draw=black,line join=round,line cap=round,line width=0.480pt]
    \path[draw] (246.5000,113.5000) -- (245.5000,113.5000);



  \end{scope}
  \begin{scope}[cm={{1.00588,0.0,0.0,1.00588,(251.57752,242.20543)}},draw=black,line join=bevel,line cap=rect,line width=0.800pt]
    \path[fill=black] (2.9825,0.0000) node[above right] (text1268) {\scriptsize 10};



  \end{scope}
  \begin{scope}[cm={{1.00588,0.0,0.0,1.00588,(0.01175,123.51142)}},draw=black,line join=round,line cap=round,line width=0.480pt]
    \path[draw] (184.5000,108.5000) -- (184.5000,156.5000) -- (246.5000,156.5000) -- (246.5000,108.5000) -- (184.5000,108.5000);



  \end{scope}
  \begin{scope}[cm={{0.0,-1.00588,1.00588,0.0,(268.66194,262.82643)}},draw=black,line join=bevel,line cap=rect,line width=0.800pt]
    \path[fill=black] (0.0000,0.0000) node[above right] (text1292) {\rotatebox{-90}{\scriptsize $c_{i,2}$}};



  \end{scope}
  \begin{scope}[cm={{1.00588,0.0,0.0,1.00588,(0.01175,123.51142)}},draw=black,line join=round,line cap=round,line width=0.480pt]
    \path[draw] (184.8000,113.6000) -- (184.8000,113.6000) -- (184.9000,113.6000) -- (185.0000,113.6000) -- (185.0000,113.6000) -- (185.1000,113.6000) -- (185.1000,113.6000) -- (185.2000,113.6000) -- (185.3000,113.6000) -- (185.3000,113.6000) -- (185.4000,113.6000) -- (185.4000,113.6000) -- (185.5000,113.6000) -- (185.6000,113.6000) -- (185.6000,113.6000) -- (185.7000,113.6000) -- (185.8000,113.6000) -- (185.8000,113.6000) -- (185.9000,113.6000) -- (185.9000,113.6000) -- (186.0000,113.6000) -- (186.1000,113.6000) -- (186.1000,113.6000) -- (186.2000,113.6000) -- (186.2000,113.6000) -- (186.3000,113.6000) -- (186.4000,113.6000) -- (186.4000,113.6000) -- (186.5000,113.6000) -- (186.6000,113.6000) -- (186.6000,113.6000) -- (186.7000,113.6000) -- (186.7000,113.6000) -- (186.8000,113.6000) -- (186.9000,113.6000) -- (186.9000,113.6000) -- (187.0000,113.6000) -- (187.0000,113.6000) -- (187.1000,113.6000) -- (187.2000,113.6000) -- (187.2000,113.6000) -- (187.3000,113.6000) -- (187.4000,113.6000) -- (187.4000,113.6000) -- (187.5000,113.6000) -- (187.5000,113.6000) -- (187.6000,113.6000) -- (187.7000,113.6000) -- (187.7000,113.6000) -- (187.8000,113.6000) -- (187.8000,113.6000) -- (187.9000,113.6000) -- (188.0000,113.6000) -- (188.0000,113.6000) -- (188.1000,113.6000) -- (188.2000,113.6000) -- (188.2000,113.6000) -- (188.3000,113.6000) -- (188.3000,113.6000) -- (188.4000,113.6000) -- (188.5000,113.6000) -- (188.5000,113.6000) -- (188.6000,113.6000) -- (188.6000,113.6000) -- (188.7000,113.6000) -- (188.8000,113.6000) -- (188.8000,113.6000) -- (188.9000,113.6000) -- (189.0000,113.6000) -- (189.0000,113.6000) -- (189.1000,113.6000) -- (189.1000,113.6000) -- (189.2000,113.6000) -- (189.3000,113.6000) -- (189.3000,113.6000) -- (189.4000,113.6000) -- (189.4000,113.6000) -- (189.5000,113.6000) -- (189.6000,113.6000) -- (189.6000,113.6000) -- (189.7000,113.6000) -- (189.8000,113.6000) -- (189.8000,113.6000) -- (189.9000,113.6000) -- (189.9000,113.6000) -- (190.0000,113.6000) -- (190.1000,113.6000) -- (190.1000,113.6000) -- (190.2000,113.6000) -- (190.2000,113.6000) -- (190.3000,113.6000) -- (190.4000,113.6000) -- (190.4000,113.6000) -- (190.5000,113.6000) -- (190.6000,113.6000) -- (190.6000,113.6000) -- (190.7000,113.6000) -- (190.7000,113.6000) -- (190.8000,113.6000) -- (190.9000,113.6000) -- (190.9000,113.6000) -- (191.0000,113.6000) -- (191.0000,113.6000) -- (191.1000,113.6000) -- (191.2000,113.6000) -- (191.2000,113.6000) -- (191.3000,113.6000) -- (191.4000,113.6000) -- (191.4000,113.6000) -- (191.5000,113.6000) -- (191.5000,113.6000) -- (191.6000,113.6000) -- (191.7000,113.6000) -- (191.7000,113.6000) -- (191.8000,113.6000) -- (191.8000,113.6000) -- (191.9000,113.6000) -- (192.0000,113.6000) -- (192.0000,113.6000) -- (192.1000,113.6000) -- (192.2000,113.6000) -- (192.2000,113.6000) -- (192.3000,113.6000) -- (192.3000,113.6000) -- (192.4000,113.6000) -- (192.5000,113.6000) -- (192.5000,113.6000) -- (192.6000,113.6000) -- (192.6000,113.6000) -- (192.7000,113.6000) -- (192.8000,113.6000) -- (192.8000,113.6000) -- (192.9000,113.6000) -- (193.0000,113.6000) -- (193.0000,113.6000) -- (193.1000,113.6000) -- (193.1000,113.6000) -- (193.2000,113.6000) -- (193.3000,113.6000) -- (193.3000,113.6000) -- (193.4000,113.6000) -- (193.4000,113.6000) -- (193.5000,113.6000) -- (193.6000,113.6000) -- (193.6000,113.6000) -- (193.7000,113.6000) -- (193.8000,113.6000) -- (193.8000,113.6000) -- (193.9000,113.6000) -- (193.9000,113.6000) -- (194.0000,113.6000) -- (194.1000,113.6000) -- (194.1000,113.6000) -- (194.2000,113.6000) -- (194.2000,113.6000) -- (194.3000,113.6000) -- (194.4000,113.6000) -- (194.4000,113.6000) -- (194.5000,113.6000) -- (194.6000,113.6000) -- (194.6000,113.6000) -- (194.7000,113.6000) -- (194.7000,113.6000) -- (194.8000,113.6000) -- (194.9000,113.6000) -- (194.9000,113.6000) -- (195.0000,113.6000) -- (195.0000,113.6000) -- (195.1000,113.6000) -- (195.2000,113.6000) -- (195.2000,113.6000) -- (195.3000,113.6000) -- (195.4000,113.6000) -- (195.4000,113.6000) -- (195.5000,113.6000) -- (195.5000,113.6000) -- (195.6000,113.6000) -- (195.7000,113.6000) -- (195.7000,113.6000) -- (195.8000,113.6000) -- (195.8000,113.6000) -- (195.9000,113.6000) -- (196.0000,113.6000) -- (196.0000,113.6000) -- (196.1000,113.6000) -- (196.2000,113.6000) -- (196.2000,113.6000) -- (196.3000,113.6000) -- (196.3000,113.6000) -- (196.4000,113.6000) -- (196.5000,113.6000) -- (196.5000,113.6000) -- (196.6000,113.6000) -- (196.6000,113.6000) -- (196.7000,113.6000) -- (196.8000,113.6000) -- (196.8000,113.6000) -- (196.9000,113.6000) -- (196.9000,113.6000) -- (197.0000,113.6000) -- (197.1000,113.6000) -- (197.1000,113.6000) -- (197.2000,113.6000) -- (197.3000,113.6000) -- (197.3000,113.6000) -- (197.4000,113.6000) -- (197.4000,113.6000) -- (197.5000,113.6000) -- (197.6000,113.6000) -- (197.6000,113.6000) -- (197.7000,113.6000) -- (197.7000,113.6000) -- (197.8000,113.6000) -- (197.9000,113.6000) -- (197.9000,113.6000) -- (198.0000,113.6000) -- (198.1000,113.6000) -- (198.1000,113.6000) -- (198.2000,113.6000) -- (198.2000,113.6000) -- (198.3000,113.6000) -- (198.4000,113.6000) -- (198.4000,113.6000) -- (198.5000,113.6000) -- (198.5000,113.6000) -- (198.6000,113.6000) -- (198.7000,113.6000) -- (198.7000,113.6000) -- (198.8000,113.6000) -- (198.9000,113.6000) -- (198.9000,113.6000) -- (199.0000,113.6000) -- (199.0000,113.6000) -- (199.1000,113.6000) -- (199.2000,113.6000) -- (199.2000,113.6000) -- (199.3000,113.6000) -- (199.3000,113.6000) -- (199.4000,113.6000) -- (199.5000,113.6000) -- (199.5000,113.6000) -- (199.6000,113.6000) -- (199.7000,113.6000) -- (199.7000,113.6000) -- (199.8000,113.6000) -- (199.8000,113.6000) -- (199.9000,113.6000) -- (200.0000,113.6000) -- (200.0000,113.6000) -- (200.1000,113.6000) -- (200.1000,113.6000) -- (200.2000,113.6000) -- (200.3000,113.6000) -- (200.3000,113.6000) -- (200.4000,113.6000) -- (200.5000,113.6000) -- (200.5000,113.6000) -- (200.6000,113.6000) -- (200.6000,113.6000) -- (200.7000,113.6000) -- (200.8000,113.6000) -- (200.8000,113.6000) -- (200.9000,113.6000) -- (200.9000,113.6000) -- (201.0000,113.6000) -- (201.1000,113.6000) -- (201.1000,113.6000) -- (201.2000,113.6000) -- (201.3000,113.6000) -- (201.3000,113.6000) -- (201.4000,113.6000) -- (201.4000,113.6000) -- (201.5000,113.6000) -- (201.6000,113.6000) -- (201.6000,113.6000) -- (201.7000,113.6000) -- (201.7000,113.6000) -- (201.8000,113.6000) -- (201.9000,113.6000) -- (201.9000,113.6000) -- (202.0000,113.6000) -- (202.1000,113.6000) -- (202.1000,113.6000) -- (202.2000,113.6000) -- (202.2000,113.6000) -- (202.3000,113.6000) -- (202.4000,113.6000) -- (202.4000,113.6000) -- (202.5000,113.6000) -- (202.5000,113.6000) -- (202.6000,113.6000) -- (202.7000,113.6000) -- (202.7000,113.6000) -- (202.8000,113.6000) -- (202.9000,113.6000) -- (202.9000,113.6000) -- (203.0000,113.6000) -- (203.0000,113.6000) -- (203.1000,113.6000) -- (203.2000,113.6000) -- (203.2000,113.8000) -- (203.3000,121.5000) -- (203.3000,123.4000) -- (203.4000,123.1000) -- (203.5000,123.1000) -- (203.5000,123.1000) -- (203.6000,123.1000) -- (203.7000,123.1000) -- (203.7000,123.1000) -- (203.8000,123.1000) -- (203.8000,123.1000) -- (203.9000,123.1000) -- (204.0000,123.1000) -- (204.0000,123.1000) -- (204.1000,123.1000) -- (204.1000,123.1000) -- (204.2000,123.1000) -- (204.3000,123.1000) -- (204.3000,123.1000) -- (204.4000,123.1000) -- (204.5000,123.1000) -- (204.5000,123.1000) -- (204.6000,123.1000) -- (204.6000,123.1000) -- (204.7000,123.1000) -- (204.8000,123.1000) -- (204.8000,123.1000) -- (204.9000,122.8000) -- (204.9000,128.6000) -- (205.0000,132.8000) -- (205.1000,134.0000) -- (205.1000,141.8000) -- (205.2000,142.2000) -- (205.3000,142.1000) -- (205.3000,142.1000) -- (205.4000,142.1000) -- (205.4000,142.1000) -- (205.5000,141.8000) -- (205.6000,148.4000) -- (205.6000,152.0000) -- (205.7000,151.6000) -- (205.7000,151.6000) -- (205.8000,151.6000) -- (205.9000,151.6000) -- (205.9000,151.6000) -- (206.0000,151.6000) -- (206.1000,151.6000) -- (206.1000,151.6000) -- (206.2000,151.6000) -- (206.2000,151.9000) -- (206.3000,149.1000) -- (206.4000,142.0000) -- (206.4000,142.1000) -- (206.5000,142.1000) -- (206.5000,142.1000) -- (206.6000,142.1000) -- (206.7000,142.2000) -- (206.7000,142.2000) -- (206.8000,134.6000) -- (206.9000,132.4000) -- (206.9000,132.6000) -- (207.0000,132.6000) -- (207.0000,132.8000) -- (207.1000,140.5000) -- (207.2000,142.4000) -- (207.2000,142.1000) -- (207.3000,142.1000) -- (207.3000,142.1000) -- (207.4000,142.1000) -- (207.5000,142.1000) -- (207.5000,142.1000) -- (207.6000,142.1000) -- (207.7000,142.1000) -- (207.7000,142.1000) -- (207.8000,142.1000) -- (207.8000,142.1000) -- (207.9000,142.1000) -- (208.0000,142.1000) -- (208.0000,142.1000) -- (208.1000,142.1000) -- (208.1000,142.1000) -- (208.2000,142.1000) -- (208.3000,142.1000) -- (208.3000,142.1000) -- (208.4000,142.5000) -- (208.5000,137.1000) -- (208.5000,132.4000) -- (208.6000,131.6000) -- (208.6000,123.7000) -- (208.7000,123.1000) -- (208.8000,123.1000) -- (208.8000,123.3000) -- (208.9000,121.9000) -- (208.9000,114.0000) -- (209.0000,113.6000) -- (209.1000,113.6000) -- (209.1000,113.6000) -- (209.2000,113.6000) -- (209.3000,113.6000) -- (209.3000,113.6000) -- (209.4000,113.6000) -- (209.4000,113.4000) -- (209.5000,115.5000) -- (209.6000,123.1000) -- (209.6000,123.2000) -- (209.7000,123.1000) -- (209.7000,122.9000) -- (209.8000,125.3000) -- (209.9000,132.7000) -- (209.9000,132.5000) -- (210.0000,139.4000) -- (210.1000,142.5000) -- (210.1000,142.1000) -- (210.2000,142.1000) -- (210.2000,142.0000) -- (210.3000,149.3000) -- (210.4000,151.9000) -- (210.4000,151.6000) -- (210.5000,151.6000) -- (210.5000,151.6000) -- (210.6000,151.6000) -- (210.7000,151.6000) -- (210.7000,151.6000) -- (210.8000,151.6000) -- (210.9000,151.6000) -- (210.9000,151.6000) -- (211.0000,151.6000) -- (211.0000,151.6000) -- (211.1000,151.6000) -- (211.2000,151.6000) -- (211.2000,151.6000) -- (211.3000,151.6000) -- (211.3000,151.6000) -- (211.4000,151.6000) -- (211.5000,151.6000) -- (211.5000,151.6000) -- (211.6000,151.6000) -- (211.7000,151.6000) -- (211.7000,151.6000) -- (211.8000,151.6000) -- (211.8000,151.6000) -- (211.9000,151.6000) -- (212.0000,151.6000) -- (212.0000,151.6000) -- (212.1000,151.6000) -- (212.1000,151.6000) -- (212.2000,151.6000) -- (212.3000,151.6000) -- (212.3000,151.6000) -- (212.4000,151.6000) -- (212.5000,151.6000) -- (212.5000,151.6000) -- (212.6000,151.6000) -- (212.6000,151.6000) -- (212.7000,151.6000) -- (212.8000,151.6000) -- (212.8000,151.6000) -- (212.9000,151.6000) -- (212.9000,151.6000) -- (213.0000,151.6000) -- (213.1000,151.6000) -- (213.1000,151.6000) -- (213.2000,151.6000) -- (213.3000,151.6000) -- (213.3000,151.6000) -- (213.4000,151.6000) -- (213.4000,151.6000) -- (213.5000,151.6000) -- (213.6000,151.6000) -- (213.6000,151.6000) -- (213.7000,151.6000) -- (213.7000,151.6000) -- (213.8000,151.6000) -- (213.9000,151.6000) -- (213.9000,151.6000) -- (214.0000,151.6000) -- (214.1000,151.6000) -- (214.1000,151.6000) -- (214.2000,151.6000) -- (214.2000,151.6000) -- (214.3000,151.6000) -- (214.4000,151.6000) -- (214.4000,151.6000) -- (214.5000,151.6000) -- (214.5000,151.6000) -- (214.6000,151.6000) -- (214.7000,151.6000) -- (214.7000,151.6000) -- (214.8000,151.6000) -- (214.9000,151.6000) -- (214.9000,151.6000) -- (215.0000,151.6000) -- (215.0000,151.6000) -- (215.1000,151.6000) -- (215.2000,151.6000) -- (215.2000,151.6000) -- (215.3000,151.6000) -- (215.3000,151.6000) -- (215.4000,151.6000) -- (215.5000,151.6000) -- (215.5000,151.6000) -- (215.6000,151.6000) -- (215.7000,151.6000) -- (215.7000,151.6000) -- (215.8000,151.6000) -- (215.8000,151.6000) -- (215.9000,151.6000) -- (216.0000,151.6000) -- (216.0000,151.6000) -- (216.1000,151.6000) -- (216.1000,151.6000) -- (216.2000,151.6000) -- (216.3000,151.6000) -- (216.3000,151.6000) -- (216.4000,151.6000) -- (216.5000,151.6000) -- (216.5000,151.6000) -- (216.6000,151.6000) -- (216.6000,151.6000) -- (216.7000,151.6000) -- (216.8000,151.6000) -- (216.8000,151.6000) -- (216.9000,151.6000) -- (216.9000,151.6000) -- (217.0000,151.6000) -- (217.1000,151.6000) -- (217.1000,151.6000) -- (217.2000,151.6000) -- (217.3000,151.6000) -- (217.3000,151.6000) -- (217.4000,151.6000) -- (217.4000,151.6000) -- (217.5000,151.6000) -- (217.6000,151.6000) -- (217.6000,151.6000) -- (217.7000,151.6000) -- (217.7000,151.6000) -- (217.8000,151.6000) -- (217.9000,151.6000) -- (217.9000,151.6000) -- (218.0000,151.6000) -- (218.1000,151.6000) -- (218.1000,151.6000) -- (218.2000,151.6000) -- (218.2000,151.6000) -- (218.3000,151.6000) -- (218.4000,151.6000) -- (218.4000,151.6000) -- (218.5000,151.6000) -- (218.5000,151.6000) -- (218.6000,151.6000) -- (218.7000,151.6000) -- (218.7000,151.6000) -- (218.8000,151.6000) -- (218.9000,151.6000) -- (218.9000,151.6000) -- (219.0000,151.6000) -- (219.0000,151.6000) -- (219.1000,151.6000) -- (219.2000,151.6000) -- (219.2000,151.6000) -- (219.3000,151.6000) -- (219.3000,151.6000) -- (219.4000,151.6000) -- (219.5000,151.6000) -- (219.5000,151.6000) -- (219.6000,151.6000) -- (219.7000,151.6000) -- (219.7000,151.6000) -- (219.8000,151.6000) -- (219.8000,151.6000) -- (219.9000,151.6000) -- (220.0000,151.6000) -- (220.0000,151.6000) -- (220.1000,151.6000) -- (220.1000,151.6000) -- (220.2000,151.6000) -- (220.3000,151.6000) -- (220.3000,151.6000) -- (220.4000,151.6000) -- (220.5000,151.6000) -- (220.5000,151.6000) -- (220.6000,151.6000) -- (220.6000,151.6000) -- (220.7000,151.6000) -- (220.8000,151.6000) -- (220.8000,151.6000) -- (220.9000,151.6000) -- (220.9000,151.6000) -- (221.0000,151.6000) -- (221.1000,151.6000) -- (221.1000,151.6000) -- (221.2000,151.6000) -- (221.3000,151.6000) -- (221.3000,151.6000) -- (221.4000,151.6000) -- (221.4000,151.6000) -- (221.5000,151.6000) -- (221.6000,151.6000) -- (221.6000,151.6000) -- (221.7000,151.6000) -- (221.7000,151.6000) -- (221.8000,151.6000) -- (221.9000,151.6000) -- (221.9000,151.6000) -- (222.0000,151.6000) -- (222.1000,151.6000) -- (222.1000,151.6000) -- (222.2000,151.6000) -- (222.2000,151.6000) -- (222.3000,151.6000) -- (222.4000,151.6000) -- (222.4000,151.6000) -- (222.5000,151.6000) -- (222.5000,151.6000) -- (222.6000,151.6000) -- (222.7000,151.6000) -- (222.7000,151.6000) -- (222.8000,151.6000) -- (222.9000,151.6000) -- (222.9000,151.6000) -- (223.0000,151.6000) -- (223.0000,151.6000) -- (223.1000,151.6000) -- (223.2000,151.6000) -- (223.2000,151.6000) -- (223.3000,151.6000) -- (223.3000,151.6000) -- (223.4000,151.6000) -- (223.5000,151.6000) -- (223.5000,151.6000) -- (223.6000,151.6000) -- (223.7000,151.6000) -- (223.7000,151.6000) -- (223.8000,151.6000) -- (223.8000,151.6000) -- (223.9000,151.6000) -- (224.0000,151.6000) -- (224.0000,151.6000) -- (224.1000,151.6000) -- (224.1000,151.6000) -- (224.2000,151.6000) -- (224.3000,151.6000) -- (224.3000,151.6000) -- (224.4000,151.6000) -- (224.5000,151.6000) -- (224.5000,151.6000) -- (224.6000,151.6000) -- (224.6000,151.6000) -- (224.7000,151.6000) -- (224.8000,151.6000) -- (224.8000,151.6000) -- (224.9000,151.6000) -- (224.9000,151.6000) -- (225.0000,151.6000) -- (225.1000,151.6000) -- (225.1000,151.6000) -- (225.2000,151.6000) -- (225.3000,151.6000) -- (225.3000,151.6000) -- (225.4000,151.6000) -- (225.4000,151.6000) -- (225.5000,151.6000) -- (225.6000,151.6000) -- (225.6000,151.6000) -- (225.7000,151.6000) -- (225.7000,151.6000) -- (225.8000,151.6000) -- (225.9000,151.6000) -- (225.9000,151.6000) -- (226.0000,151.6000) -- (226.0000,151.6000) -- (226.1000,151.6000) -- (226.2000,151.6000) -- (226.2000,151.6000) -- (226.3000,151.6000) -- (226.4000,151.6000) -- (226.4000,151.6000) -- (226.5000,151.6000) -- (226.5000,151.6000) -- (226.6000,151.6000) -- (226.7000,151.6000) -- (226.7000,151.6000) -- (226.8000,151.6000) -- (226.8000,151.6000) -- (226.9000,151.6000) -- (227.0000,151.6000) -- (227.0000,151.6000) -- (227.1000,151.6000) -- (227.2000,151.6000) -- (227.2000,151.6000) -- (227.3000,151.6000) -- (227.3000,151.6000) -- (227.4000,151.6000) -- (227.5000,151.6000) -- (227.5000,151.6000) -- (227.6000,151.6000) -- (227.7000,151.6000) -- (227.7000,151.6000) -- (227.8000,151.6000) -- (227.8000,151.6000) -- (227.9000,151.6000) -- (228.0000,151.6000) -- (228.0000,151.6000) -- (228.1000,151.6000) -- (228.1000,151.6000) -- (228.2000,151.6000) -- (228.3000,151.6000) -- (228.3000,151.6000) -- (228.4000,151.6000) -- (228.5000,151.6000) -- (228.5000,151.6000) -- (228.6000,151.6000) -- (228.6000,151.6000) -- (228.7000,151.6000) -- (228.8000,151.6000) -- (228.8000,151.6000) -- (228.9000,151.6000) -- (228.9000,151.6000) -- (229.0000,151.6000) -- (229.1000,151.6000) -- (229.1000,151.6000) -- (229.2000,151.6000) -- (229.2000,151.6000) -- (229.3000,151.6000) -- (229.4000,151.6000) -- (229.4000,151.6000) -- (229.5000,151.6000) -- (229.6000,151.6000) -- (229.6000,151.6000) -- (229.7000,151.6000) -- (229.7000,151.6000) -- (229.8000,151.6000) -- (229.9000,151.6000) -- (229.9000,151.6000) -- (230.0000,151.6000) -- (230.0000,151.6000) -- (230.1000,151.6000) -- (230.2000,151.6000) -- (230.2000,151.6000) -- (230.3000,151.6000) -- (230.4000,151.6000) -- (230.4000,151.6000) -- (230.5000,151.6000) -- (230.5000,151.6000) -- (230.6000,151.6000) -- (230.7000,151.6000) -- (230.7000,151.6000) -- (230.8000,151.6000) -- (230.8000,151.6000) -- (230.9000,151.6000) -- (231.0000,151.6000) -- (231.0000,151.6000) -- (231.1000,151.6000) -- (231.2000,151.6000) -- (231.2000,151.6000) -- (231.3000,151.6000) -- (231.3000,151.6000) -- (231.4000,151.6000) -- (231.5000,151.6000) -- (231.5000,151.6000) -- (231.6000,151.6000) -- (231.6000,151.6000) -- (231.7000,151.6000) -- (231.8000,151.6000) -- (231.8000,151.6000) -- (231.9000,151.6000) -- (232.0000,151.6000) -- (232.0000,151.6000) -- (232.1000,151.6000) -- (232.1000,151.6000) -- (232.2000,151.6000) -- (232.3000,151.6000) -- (232.3000,151.6000) -- (232.4000,151.6000) -- (232.4000,151.6000) -- (232.5000,151.6000) -- (232.6000,151.6000) -- (232.6000,151.6000) -- (232.7000,151.6000) -- (232.8000,151.6000) -- (232.8000,151.6000) -- (232.9000,151.6000) -- (232.9000,151.6000) -- (233.0000,151.6000) -- (233.1000,151.6000) -- (233.1000,151.6000) -- (233.2000,151.6000) -- (233.2000,151.6000) -- (233.3000,151.6000) -- (233.4000,151.6000) -- (233.4000,151.6000) -- (233.5000,151.6000) -- (233.6000,151.6000) -- (233.6000,151.6000) -- (233.7000,151.6000) -- (233.7000,151.6000) -- (233.8000,151.6000) -- (233.9000,151.6000) -- (233.9000,151.6000) -- (234.0000,151.6000) -- (234.0000,151.6000) -- (234.1000,151.6000) -- (234.2000,151.6000) -- (234.2000,151.6000) -- (234.3000,151.6000) -- (234.4000,151.6000) -- (234.4000,151.6000) -- (234.5000,151.6000) -- (234.5000,151.6000) -- (234.6000,151.6000) -- (234.7000,151.6000) -- (234.7000,151.6000) -- (234.8000,151.6000) -- (234.8000,151.6000) -- (234.9000,151.6000) -- (235.0000,151.6000) -- (235.0000,151.6000) -- (235.1000,151.6000) -- (235.2000,151.6000) -- (235.2000,151.6000) -- (235.3000,151.6000) -- (235.3000,151.6000) -- (235.4000,151.6000) -- (235.5000,151.6000) -- (235.5000,151.6000) -- (235.6000,151.6000) -- (235.6000,151.6000) -- (235.7000,151.6000) -- (235.8000,151.6000) -- (235.8000,151.6000) -- (235.9000,151.6000) -- (236.0000,151.6000) -- (236.0000,151.6000) -- (236.1000,151.6000) -- (236.1000,151.6000) -- (236.2000,151.6000) -- (236.3000,151.6000) -- (236.3000,151.6000) -- (236.4000,151.6000) -- (236.4000,151.6000) -- (236.5000,151.6000) -- (236.6000,151.6000) -- (236.6000,151.6000) -- (236.7000,151.6000) -- (236.8000,151.6000) -- (236.8000,151.6000) -- (236.9000,151.6000) -- (236.9000,151.6000) -- (237.0000,151.6000) -- (237.1000,151.6000) -- (237.1000,151.6000) -- (237.2000,151.6000) -- (237.2000,151.6000) -- (237.3000,151.6000) -- (237.4000,151.6000) -- (237.4000,151.6000) -- (237.5000,151.6000) -- (237.6000,151.6000) -- (237.6000,151.6000) -- (237.7000,151.6000) -- (237.7000,151.6000) -- (237.8000,151.6000) -- (237.9000,151.6000) -- (237.9000,151.6000) -- (238.0000,151.6000) -- (238.0000,151.6000) -- (238.1000,151.6000) -- (238.2000,151.6000) -- (238.2000,151.6000) -- (238.3000,151.6000) -- (238.4000,151.6000) -- (238.4000,151.6000) -- (238.5000,151.6000) -- (238.5000,151.6000) -- (238.6000,151.6000) -- (238.7000,151.6000) -- (238.7000,151.6000) -- (238.8000,151.6000) -- (238.8000,151.6000) -- (238.9000,151.6000) -- (239.0000,151.6000) -- (239.0000,151.6000) -- (239.1000,151.6000) -- (239.2000,151.6000) -- (239.2000,151.6000) -- (239.3000,151.6000) -- (239.3000,151.6000) -- (239.4000,151.6000) -- (239.5000,151.6000) -- (239.5000,151.6000) -- (239.6000,151.6000) -- (239.6000,151.6000) -- (239.7000,151.6000) -- (239.8000,151.6000) -- (239.8000,151.6000) -- (239.9000,151.6000) -- (240.0000,151.6000) -- (240.0000,151.6000) -- (240.1000,151.6000) -- (240.1000,151.6000) -- (240.2000,151.6000) -- (240.3000,151.6000) -- (240.3000,151.6000) -- (240.4000,151.6000) -- (240.4000,151.6000) -- (240.5000,151.6000) -- (240.6000,151.6000) -- (240.6000,151.6000) -- (240.7000,151.6000) -- (240.8000,151.6000) -- (240.8000,151.6000) -- (240.9000,151.6000) -- (240.9000,151.6000) -- (241.0000,151.6000) -- (241.1000,151.6000) -- (241.1000,151.6000) -- (241.2000,151.6000) -- (241.2000,151.6000) -- (241.3000,151.6000) -- (241.4000,151.6000) -- (241.4000,151.6000) -- (241.5000,151.6000) -- (241.6000,151.6000) -- (241.6000,151.6000) -- (241.7000,151.6000) -- (241.7000,151.6000) -- (241.8000,151.6000) -- (241.9000,151.6000) -- (241.9000,151.6000) -- (242.0000,151.6000) -- (242.0000,151.6000) -- (242.1000,151.6000) -- (242.2000,151.6000) -- (242.2000,151.6000) -- (242.3000,151.6000) -- (242.4000,151.6000) -- (242.4000,151.6000) -- (242.5000,151.6000) -- (242.5000,151.6000) -- (242.6000,151.6000) -- (242.7000,151.6000) -- (242.7000,151.6000) -- (242.8000,151.6000) -- (242.8000,151.6000) -- (242.9000,151.6000) -- (243.0000,151.6000) -- (243.0000,151.6000) -- (243.1000,151.6000) -- (243.2000,151.6000) -- (243.2000,151.6000) -- (243.3000,151.6000) -- (243.3000,151.6000) -- (243.4000,151.6000) -- (243.5000,151.6000) -- (243.5000,151.6000) -- (243.6000,151.6000) -- (243.6000,151.6000) -- (243.7000,151.6000) -- (243.8000,151.6000) -- (243.8000,151.6000) -- (243.9000,151.6000) -- (244.0000,151.6000) -- (244.0000,151.6000) -- (244.1000,151.6000) -- (244.1000,151.6000) -- (244.2000,151.6000) -- (244.3000,151.6000) -- (244.3000,151.6000) -- (244.4000,151.6000) -- (244.4000,151.6000) -- (244.5000,151.6000) -- (244.6000,151.6000) -- (244.6000,151.6000) -- (244.7000,151.6000) -- (244.8000,151.6000) -- (244.8000,151.6000) -- (244.9000,151.6000) -- (244.9000,151.6000) -- (245.0000,151.6000) -- (245.1000,151.6000) -- (245.1000,151.6000) -- (245.2000,151.6000) -- (245.2000,151.6000) -- (245.3000,151.6000) -- (245.4000,151.6000) -- (245.4000,151.6000) -- (245.5000,151.6000) -- (245.6000,151.6000) -- (245.6000,151.6000) -- (245.7000,151.6000) -- (245.7000,151.6000) -- (245.8000,151.6000) -- (245.9000,151.6000) -- (245.9000,151.6000) -- (246.0000,151.6000) -- (246.0000,151.6000) -- (246.1000,151.6000) -- (246.2000,151.6000) -- (246.2000,151.6000) -- (246.3000,151.6000);



  \end{scope}
  \begin{scope}[cm={{1.00588,0.0,0.0,1.00588,(0.01175,123.51142)}},draw=black,line join=round,line cap=round,line width=0.480pt]
    \path[draw] (184.5000,108.5000) -- (184.5000,156.5000) -- (246.5000,156.5000) -- (246.5000,108.5000) -- (184.5000,108.5000);



  \end{scope}
  \begin{scope}[cm={{1.00588,0.0,0.0,1.00588,(0.01175,123.51142)}},draw=ca0a0a4,dash pattern=on 0.40pt off 0.80pt,line join=round,line cap=round,line width=0.400pt]
    \path[draw] (184.5000,200.5000) -- (246.5000,200.5000);



  \end{scope}
  \begin{scope}[cm={{1.00588,0.0,0.0,1.00588,(0.01175,123.51142)}},draw=black,line join=round,line cap=round,line width=0.480pt]
    \path[draw] (184.5000,200.5000) -- (186.5000,200.5000);



    \path[draw] (246.5000,200.5000) -- (245.5000,200.5000);



  \end{scope}
  \begin{scope}[cm={{1.00588,0.0,0.0,1.00588,(0.01175,123.51142)}},draw=ca0a0a4,dash pattern=on 0.40pt off 0.80pt,line join=round,line cap=round,line width=0.400pt]
    \path[draw] (184.5000,180.5000) -- (246.5000,180.5000);



  \end{scope}
  \begin{scope}[cm={{1.00588,0.0,0.0,1.00588,(0.01175,123.51142)}},draw=black,line join=round,line cap=round,line width=0.480pt]
    \path[draw] (184.5000,180.5000) -- (186.5000,180.5000);



    \path[draw] (246.5000,180.5000) -- (245.5000,180.5000);



  \end{scope}
  \begin{scope}[cm={{1.00588,0.0,0.0,1.00588,(0.01175,123.51142)}},draw=ca0a0a4,dash pattern=on 0.40pt off 0.80pt,line join=round,line cap=round,line width=0.400pt]
    \path[draw] (184.5000,160.5000) -- (246.5000,160.5000);



  \end{scope}
  \begin{scope}[cm={{1.00588,0.0,0.0,1.00588,(0.01175,123.51142)}},draw=black,line join=round,line cap=round,line width=0.480pt]
    \path[draw] (184.5000,160.5000) -- (186.5000,160.5000);



    \path[draw] (246.5000,160.5000) -- (245.5000,160.5000);



  \end{scope}
  \begin{scope}[cm={{1.00588,0.0,0.0,1.00588,(0.01175,123.51142)}},draw=ca0a0a4,dash pattern=on 0.40pt off 0.80pt,line join=round,line cap=round,line width=0.400pt]
    \path[draw] (184.5000,204.5000) -- (184.5000,156.5000);



  \end{scope}
  \begin{scope}[cm={{1.00588,0.0,0.0,1.00588,(0.01175,123.51142)}},draw=black,line join=round,line cap=round,line width=0.480pt]
    \path[draw] (184.5000,204.5000) -- (184.5000,203.5000);



    \path[draw] (184.5000,156.5000) -- (184.5000,157.5000);



  \end{scope}
  \begin{scope}[cm={{1.00588,0.0,0.0,1.00588,(184.1935,340.85267)}},draw=black,line join=bevel,line cap=rect,line width=0.800pt]
    \path[fill=black] (0.0000,0.0000) node[above right] (text1416) {\scriptsize 0};



  \end{scope}
  \begin{scope}[cm={{1.00588,0.0,0.0,1.00588,(0.01175,123.51142)}},draw=ca0a0a4,dash pattern=on 0.40pt off 0.80pt,line join=round,line cap=round,line width=0.400pt]
    \path[draw] (203.5000,204.5000) -- (203.5000,156.5000);



  \end{scope}
  \begin{scope}[cm={{1.00588,0.0,0.0,1.00588,(0.01175,123.51142)}},draw=black,line join=round,line cap=round,line width=0.480pt]
    \path[draw] (203.5000,204.5000) -- (203.5000,203.5000);



    \path[draw] (203.5000,156.5000) -- (203.5000,157.5000);



  \end{scope}
  \begin{scope}[cm={{1.00588,0.0,0.0,1.00588,(202.2985,340.96533)}},draw=black,line join=bevel,line cap=rect,line width=0.800pt]
    \path[fill=black] (0.0000,0.0000) node[above right] (text1446) {\scriptsize 2};



  \end{scope}
  \begin{scope}[cm={{1.00588,0.0,0.0,1.00588,(0.01175,123.51142)}},draw=ca0a0a4,dash pattern=on 0.40pt off 0.80pt,line join=round,line cap=round,line width=0.400pt]
    \path[draw] (221.5000,204.5000) -- (221.5000,156.5000);



  \end{scope}
  \begin{scope}[cm={{1.00588,0.0,0.0,1.00588,(0.01175,123.51142)}},draw=black,line join=round,line cap=round,line width=0.480pt]
    \path[draw] (221.5000,204.5000) -- (221.5000,203.5000);



    \path[draw] (221.5000,156.5000) -- (221.5000,157.5000);



  \end{scope}
  \begin{scope}[cm={{1.00588,0.0,0.0,1.00588,(220.9075,340.96533)}},draw=black,line join=bevel,line cap=rect,line width=0.800pt]
    \path[fill=black] (0.0000,0.0000) node[above right] (text1476) {\scriptsize 4};



  \end{scope}
  \begin{scope}[cm={{1.00588,0.0,0.0,1.00588,(0.01175,123.51142)}},draw=ca0a0a4,dash pattern=on 0.40pt off 0.80pt,line join=round,line cap=round,line width=0.400pt]
    \path[draw] (239.5000,204.5000) -- (239.5000,156.5000);



  \end{scope}
  \begin{scope}[cm={{1.00588,0.0,0.0,1.00588,(0.01175,123.51142)}},draw=black,line join=round,line cap=round,line width=0.480pt]
    \path[draw] (239.5000,204.5000) -- (239.5000,203.5000);



    \path[draw] (239.5000,156.5000) -- (239.5000,157.5000);



  \end{scope}
  \begin{scope}[cm={{1.00588,0.0,0.0,1.00588,(239.5165,340.85267)}},draw=black,line join=bevel,line cap=rect,line width=0.800pt]
    \path[fill=black] (0.0000,0.0000) node[above right] (text1506) {\scriptsize 6};



  \end{scope}
  \begin{scope}[cm={{1.00588,0.0,0.0,1.00588,(0.01175,123.51142)}},draw=black,line join=round,line cap=round,line width=0.480pt]
    \path[draw] (246.5000,200.5000) -- (245.5000,200.5000);



  \end{scope}
  \begin{scope}[cm={{1.00588,0.0,0.0,1.00588,(255.15691,327.21143)}},draw=black,line join=bevel,line cap=rect,line width=0.800pt]
    \path[fill=black] (0.0000,0.0000) node[above right] (text1530) {\scriptsize -1};



  \end{scope}
  \begin{scope}[cm={{1.00588,0.0,0.0,1.00588,(0.01175,123.51142)}},draw=black,line join=round,line cap=round,line width=0.480pt]
    \path[draw] (246.5000,180.5000) -- (245.5000,180.5000);



  \end{scope}
  \begin{scope}[cm={{1.00588,0.0,0.0,1.00588,(255.15691,307.09343)}},draw=black,line join=bevel,line cap=rect,line width=0.800pt]
    \path[fill=black] (0.0000,0.0000) node[above right] (text1554) {\scriptsize -.5};



  \end{scope}
  \begin{scope}[cm={{1.00588,0.0,0.0,1.00588,(0.01175,123.51142)}},draw=black,line join=round,line cap=round,line width=0.480pt]
    \path[draw] (246.5000,160.5000) -- (245.5000,160.5000);



  \end{scope}
  \begin{scope}[cm={{1.00588,0.0,0.0,1.00588,(255.27762,289.48243)}},draw=black,line join=bevel,line cap=rect,line width=0.800pt]
    \path[fill=black] (0.0000,0.0000) node[above right] (text1578) {\scriptsize 0};



  \end{scope}
  \begin{scope}[cm={{1.00588,0.0,0.0,1.00588,(0.01175,123.51142)}},draw=black,line join=round,line cap=round,line width=0.480pt]
    \path[draw] (184.5000,156.5000) -- (184.5000,204.5000) -- (246.5000,204.5000) -- (246.5000,156.5000) -- (184.5000,156.5000);



  \end{scope}
  \begin{scope}[cm={{0.0,-1.00588,1.00588,0.0,(268.66194,310.10243)}},draw=black,line join=bevel,line cap=rect,line width=0.800pt]
    \path[fill=black] (0.0000,0.0000) node[above right] (text1602) {\rotatebox{-90}{\scriptsize k$c_{i,1}$}};



  \end{scope}
  \begin{scope}[cm={{1.00588,0.0,0.0,1.00588,(0.01175,123.51142)}},draw=black,line join=round,line cap=round,line width=0.480pt]
    \path[draw] (184.8000,160.5000) -- (184.8000,160.5000) -- (184.9000,160.5000) -- (185.0000,160.5000) -- (185.0000,160.5000) -- (185.1000,160.5000) -- (185.1000,160.5000) -- (185.2000,160.5000) -- (185.3000,160.5000) -- (185.3000,160.5000) -- (185.4000,160.5000) -- (185.4000,160.5000) -- (185.5000,160.5000) -- (185.6000,160.5000) -- (185.6000,160.5000) -- (185.7000,160.5000) -- (185.8000,160.5000) -- (185.8000,160.5000) -- (185.9000,160.5000) -- (185.9000,160.5000) -- (186.0000,160.5000) -- (186.1000,160.5000) -- (186.1000,160.5000) -- (186.2000,160.5000) -- (186.2000,160.5000) -- (186.3000,160.5000) -- (186.4000,160.5000) -- (186.4000,160.5000) -- (186.5000,160.5000) -- (186.6000,160.5000) -- (186.6000,160.5000) -- (186.7000,160.5000) -- (186.7000,160.5000) -- (186.8000,160.5000) -- (186.9000,160.5000) -- (186.9000,160.5000) -- (187.0000,160.5000) -- (187.0000,160.5000) -- (187.1000,160.5000) -- (187.2000,160.5000) -- (187.2000,160.5000) -- (187.3000,160.5000) -- (187.4000,160.5000) -- (187.4000,160.5000) -- (187.5000,160.5000) -- (187.5000,160.5000) -- (187.6000,160.5000) -- (187.7000,160.5000) -- (187.7000,160.5000) -- (187.8000,160.5000) -- (187.8000,160.5000) -- (187.9000,160.5000) -- (188.0000,160.5000) -- (188.0000,160.5000) -- (188.1000,160.5000) -- (188.2000,160.5000) -- (188.2000,160.5000) -- (188.3000,160.5000) -- (188.3000,160.5000) -- (188.4000,160.5000) -- (188.5000,160.5000) -- (188.5000,160.5000) -- (188.6000,160.5000) -- (188.6000,160.5000) -- (188.7000,160.5000) -- (188.8000,160.5000) -- (188.8000,160.5000) -- (188.9000,160.5000) -- (189.0000,160.5000) -- (189.0000,160.5000) -- (189.1000,160.5000) -- (189.1000,160.5000) -- (189.2000,160.5000) -- (189.3000,160.5000) -- (189.3000,160.5000) -- (189.4000,160.5000) -- (189.4000,160.5000) -- (189.5000,160.5000) -- (189.6000,160.5000) -- (189.6000,160.5000) -- (189.7000,160.5000) -- (189.8000,160.5000) -- (189.8000,160.5000) -- (189.9000,160.5000) -- (189.9000,160.5000) -- (190.0000,160.5000) -- (190.1000,160.5000) -- (190.1000,160.5000) -- (190.2000,160.5000) -- (190.2000,160.5000) -- (190.3000,160.5000) -- (190.4000,160.5000) -- (190.4000,160.5000) -- (190.5000,160.5000) -- (190.6000,160.5000) -- (190.6000,160.5000) -- (190.7000,160.5000) -- (190.7000,160.5000) -- (190.8000,160.5000) -- (190.9000,160.5000) -- (190.9000,160.5000) -- (191.0000,160.5000) -- (191.0000,160.5000) -- (191.1000,160.5000) -- (191.2000,160.5000) -- (191.2000,160.5000) -- (191.3000,160.5000) -- (191.4000,160.5000) -- (191.4000,160.5000) -- (191.5000,160.5000) -- (191.5000,160.5000) -- (191.6000,160.5000) -- (191.7000,160.5000) -- (191.7000,160.5000) -- (191.8000,160.5000) -- (191.8000,160.5000) -- (191.9000,160.5000) -- (192.0000,160.5000) -- (192.0000,160.5000) -- (192.1000,160.5000) -- (192.2000,160.5000) -- (192.2000,160.5000) -- (192.3000,160.5000) -- (192.3000,160.5000) -- (192.4000,160.5000) -- (192.5000,160.5000) -- (192.5000,160.5000) -- (192.6000,160.5000) -- (192.6000,160.5000) -- (192.7000,160.5000) -- (192.8000,160.5000) -- (192.8000,160.5000) -- (192.9000,160.5000) -- (193.0000,160.5000) -- (193.0000,160.5000) -- (193.1000,160.5000) -- (193.1000,160.5000) -- (193.2000,160.5000) -- (193.3000,160.5000) -- (193.3000,160.5000) -- (193.4000,160.5000) -- (193.4000,160.5000) -- (193.5000,160.5000) -- (193.6000,160.5000) -- (193.6000,160.5000) -- (193.7000,160.5000) -- (193.8000,160.5000) -- (193.8000,160.5000) -- (193.9000,160.5000) -- (193.9000,160.5000) -- (194.0000,160.5000) -- (194.1000,160.5000) -- (194.1000,160.5000) -- (194.2000,160.5000) -- (194.2000,160.5000) -- (194.3000,160.5000) -- (194.4000,160.5000) -- (194.4000,160.5000) -- (194.5000,160.5000) -- (194.6000,160.5000) -- (194.6000,160.5000) -- (194.7000,160.5000) -- (194.7000,160.5000) -- (194.8000,160.5000) -- (194.9000,160.5000) -- (194.9000,160.5000) -- (195.0000,160.5000) -- (195.0000,160.5000) -- (195.1000,160.5000) -- (195.2000,160.5000) -- (195.2000,160.5000) -- (195.3000,160.5000) -- (195.4000,160.5000) -- (195.4000,160.5000) -- (195.5000,160.5000) -- (195.5000,160.5000) -- (195.6000,160.5000) -- (195.7000,160.5000) -- (195.7000,160.5000) -- (195.8000,160.5000) -- (195.8000,160.5000) -- (195.9000,160.5000) -- (196.0000,160.5000) -- (196.0000,160.5000) -- (196.1000,160.5000) -- (196.2000,160.5000) -- (196.2000,160.5000) -- (196.3000,160.5000) -- (196.3000,160.5000) -- (196.4000,160.5000) -- (196.5000,160.5000) -- (196.5000,160.5000) -- (196.6000,160.5000) -- (196.6000,160.5000) -- (196.7000,160.5000) -- (196.8000,160.5000) -- (196.8000,160.5000) -- (196.9000,160.5000) -- (196.9000,160.5000) -- (197.0000,160.5000) -- (197.1000,160.5000) -- (197.1000,160.5000) -- (197.2000,160.5000) -- (197.3000,160.5000) -- (197.3000,160.5000) -- (197.4000,160.5000) -- (197.4000,160.5000) -- (197.5000,160.5000) -- (197.6000,160.5000) -- (197.6000,160.5000) -- (197.7000,160.5000) -- (197.7000,160.5000) -- (197.8000,160.5000) -- (197.9000,160.5000) -- (197.9000,160.5000) -- (198.0000,160.5000) -- (198.1000,160.5000) -- (198.1000,160.5000) -- (198.2000,160.5000) -- (198.2000,160.5000) -- (198.3000,160.5000) -- (198.4000,160.5000) -- (198.4000,160.5000) -- (198.5000,160.5000) -- (198.5000,160.5000) -- (198.6000,160.5000) -- (198.7000,160.5000) -- (198.7000,160.5000) -- (198.8000,160.5000) -- (198.9000,159.9000) -- (198.9000,165.0000) -- (199.0000,180.4000) -- (199.0000,180.3000) -- (199.1000,180.2000) -- (199.2000,180.2000) -- (199.2000,180.2000) -- (199.3000,180.2000) -- (199.3000,180.2000) -- (199.4000,180.2000) -- (199.5000,180.2000) -- (199.5000,180.2000) -- (199.6000,180.2000) -- (199.7000,180.2000) -- (199.7000,180.2000) -- (199.8000,180.2000) -- (199.8000,180.2000) -- (199.9000,180.2000) -- (200.0000,180.2000) -- (200.0000,180.2000) -- (200.1000,180.2000) -- (200.1000,180.2000) -- (200.2000,180.2000) -- (200.3000,180.2000) -- (200.3000,180.2000) -- (200.4000,180.2000) -- (200.5000,180.2000) -- (200.5000,180.2000) -- (200.6000,180.2000) -- (200.6000,180.2000) -- (200.7000,180.2000) -- (200.8000,180.2000) -- (200.8000,180.2000) -- (200.9000,180.2000) -- (200.9000,180.2000) -- (201.0000,180.2000) -- (201.1000,180.2000) -- (201.1000,180.2000) -- (201.2000,180.2000) -- (201.3000,180.2000) -- (201.3000,180.2000) -- (201.4000,180.2000) -- (201.4000,180.2000) -- (201.5000,180.2000) -- (201.6000,180.2000) -- (201.6000,180.2000) -- (201.7000,180.2000) -- (201.7000,180.2000) -- (201.8000,180.2000) -- (201.9000,180.2000) -- (201.9000,180.2000) -- (202.0000,180.2000) -- (202.1000,180.2000) -- (202.1000,180.2000) -- (202.2000,180.2000) -- (202.2000,180.2000) -- (202.3000,180.2000) -- (202.4000,180.2000) -- (202.4000,180.2000) -- (202.5000,180.2000) -- (202.5000,180.2000) -- (202.6000,180.2000) -- (202.7000,180.2000) -- (202.7000,180.2000) -- (202.8000,180.2000) -- (202.9000,180.2000) -- (202.9000,180.2000) -- (203.0000,180.2000) -- (203.0000,180.2000) -- (203.1000,180.2000) -- (203.2000,180.2000) -- (203.2000,180.2000) -- (203.3000,180.2000) -- (203.3000,180.2000) -- (203.4000,180.2000) -- (203.5000,180.2000) -- (203.5000,180.2000) -- (203.6000,180.2000) -- (203.7000,180.2000) -- (203.7000,180.2000) -- (203.8000,180.2000) -- (203.8000,180.2000) -- (203.9000,180.2000) -- (204.0000,180.2000) -- (204.0000,180.2000) -- (204.1000,180.2000) -- (204.1000,180.2000) -- (204.2000,180.2000) -- (204.3000,180.2000) -- (204.3000,180.2000) -- (204.4000,180.2000) -- (204.5000,180.2000) -- (204.5000,180.2000) -- (204.6000,180.2000) -- (204.6000,180.2000) -- (204.7000,180.2000) -- (204.8000,180.2000) -- (204.8000,180.2000) -- (204.9000,180.2000) -- (204.9000,180.2000) -- (205.0000,180.2000) -- (205.1000,180.2000) -- (205.1000,180.2000) -- (205.2000,180.2000) -- (205.3000,180.2000) -- (205.3000,180.2000) -- (205.4000,180.2000) -- (205.4000,180.2000) -- (205.5000,180.2000) -- (205.6000,180.2000) -- (205.6000,180.2000) -- (205.7000,180.2000) -- (205.7000,180.2000) -- (205.8000,180.2000) -- (205.9000,180.2000) -- (205.9000,180.2000) -- (206.0000,180.2000) -- (206.1000,180.2000) -- (206.1000,180.2000) -- (206.2000,180.2000) -- (206.2000,180.2000) -- (206.3000,180.2000) -- (206.4000,180.2000) -- (206.4000,180.2000) -- (206.5000,180.2000) -- (206.5000,180.2000) -- (206.6000,180.2000) -- (206.7000,180.2000) -- (206.7000,180.2000) -- (206.8000,180.2000) -- (206.9000,180.2000) -- (206.9000,180.2000) -- (207.0000,180.2000) -- (207.0000,180.2000) -- (207.1000,180.2000) -- (207.2000,180.2000) -- (207.2000,180.2000) -- (207.3000,180.2000) -- (207.3000,180.2000) -- (207.4000,180.2000) -- (207.5000,180.2000) -- (207.5000,180.2000) -- (207.6000,180.2000) -- (207.7000,180.2000) -- (207.7000,180.2000) -- (207.8000,180.2000) -- (207.8000,180.2000) -- (207.9000,180.2000) -- (208.0000,180.2000) -- (208.0000,180.2000) -- (208.1000,180.2000) -- (208.1000,180.2000) -- (208.2000,180.2000) -- (208.3000,180.2000) -- (208.3000,180.2000) -- (208.4000,180.2000) -- (208.5000,180.2000) -- (208.5000,180.2000) -- (208.6000,180.2000) -- (208.6000,180.2000) -- (208.7000,180.2000) -- (208.8000,180.2000) -- (208.8000,180.2000) -- (208.9000,180.2000) -- (208.9000,180.2000) -- (209.0000,180.2000) -- (209.1000,180.2000) -- (209.1000,180.2000) -- (209.2000,180.2000) -- (209.3000,180.2000) -- (209.3000,180.2000) -- (209.4000,180.2000) -- (209.4000,180.2000) -- (209.5000,180.2000) -- (209.6000,180.2000) -- (209.6000,180.2000) -- (209.7000,180.2000) -- (209.7000,180.2000) -- (209.8000,180.2000) -- (209.9000,180.2000) -- (209.9000,180.2000) -- (210.0000,180.2000) -- (210.1000,180.2000) -- (210.1000,180.2000) -- (210.2000,180.2000) -- (210.2000,180.2000) -- (210.3000,180.2000) -- (210.4000,180.2000) -- (210.4000,180.2000) -- (210.5000,180.2000) -- (210.5000,180.2000) -- (210.6000,180.2000) -- (210.7000,180.2000) -- (210.7000,180.2000) -- (210.8000,180.2000) -- (210.9000,180.2000) -- (210.9000,180.2000) -- (211.0000,180.2000) -- (211.0000,180.2000) -- (211.1000,180.2000) -- (211.2000,180.2000) -- (211.2000,180.2000) -- (211.3000,180.2000) -- (211.3000,180.2000) -- (211.4000,180.2000) -- (211.5000,180.2000) -- (211.5000,180.2000) -- (211.6000,180.2000) -- (211.7000,180.2000) -- (211.7000,180.2000) -- (211.8000,180.2000) -- (211.8000,180.2000) -- (211.9000,180.2000) -- (212.0000,180.2000) -- (212.0000,180.2000) -- (212.1000,180.2000) -- (212.1000,180.2000) -- (212.2000,180.2000) -- (212.3000,180.2000) -- (212.3000,180.2000) -- (212.4000,180.2000) -- (212.5000,180.2000) -- (212.5000,180.2000) -- (212.6000,180.2000) -- (212.6000,180.2000) -- (212.7000,180.2000) -- (212.8000,180.2000) -- (212.8000,180.2000) -- (212.9000,180.2000) -- (212.9000,180.2000) -- (213.0000,180.2000) -- (213.1000,180.2000) -- (213.1000,180.2000) -- (213.2000,180.2000) -- (213.3000,180.2000) -- (213.3000,180.2000) -- (213.4000,180.2000) -- (213.4000,180.2000) -- (213.5000,180.2000) -- (213.6000,180.2000) -- (213.6000,180.2000) -- (213.7000,180.2000) -- (213.7000,180.2000) -- (213.8000,180.2000) -- (213.9000,180.2000) -- (213.9000,180.2000) -- (214.0000,180.2000) -- (214.1000,180.2000) -- (214.1000,180.2000) -- (214.2000,180.2000) -- (214.2000,180.2000) -- (214.3000,180.2000) -- (214.4000,180.2000) -- (214.4000,180.2000) -- (214.5000,180.2000) -- (214.5000,180.2000) -- (214.6000,180.2000) -- (214.7000,180.2000) -- (214.7000,180.2000) -- (214.8000,180.2000) -- (214.9000,180.2000) -- (214.9000,180.2000) -- (215.0000,180.2000) -- (215.0000,180.2000) -- (215.1000,180.2000) -- (215.2000,180.2000) -- (215.2000,180.2000) -- (215.3000,180.2000) -- (215.3000,180.2000) -- (215.4000,180.2000) -- (215.5000,180.2000) -- (215.5000,180.2000) -- (215.6000,180.2000) -- (215.7000,180.2000) -- (215.7000,180.2000) -- (215.8000,180.2000) -- (215.8000,180.2000) -- (215.9000,180.2000) -- (216.0000,180.2000) -- (216.0000,180.2000) -- (216.1000,180.2000) -- (216.1000,180.2000) -- (216.2000,180.2000) -- (216.3000,180.2000) -- (216.3000,180.2000) -- (216.4000,180.2000) -- (216.5000,180.2000) -- (216.5000,180.2000) -- (216.6000,180.2000) -- (216.6000,180.2000) -- (216.7000,180.2000) -- (216.8000,180.2000) -- (216.8000,180.2000) -- (216.9000,180.2000) -- (216.9000,180.2000) -- (217.0000,180.2000) -- (217.1000,180.2000) -- (217.1000,180.2000) -- (217.2000,180.2000) -- (217.3000,180.2000) -- (217.3000,180.2000) -- (217.4000,180.2000) -- (217.4000,180.2000) -- (217.5000,180.2000) -- (217.6000,180.2000) -- (217.6000,180.2000) -- (217.7000,180.2000) -- (217.7000,180.2000) -- (217.8000,180.2000) -- (217.9000,180.2000) -- (217.9000,180.2000) -- (218.0000,180.2000) -- (218.1000,180.2000) -- (218.1000,180.2000) -- (218.2000,180.2000) -- (218.2000,180.2000) -- (218.3000,180.2000) -- (218.4000,180.2000) -- (218.4000,180.2000) -- (218.5000,180.2000) -- (218.5000,180.2000) -- (218.6000,180.2000) -- (218.7000,180.2000) -- (218.7000,180.2000) -- (218.8000,180.2000) -- (218.9000,180.2000) -- (218.9000,180.2000) -- (219.0000,180.2000) -- (219.0000,180.2000) -- (219.1000,180.2000) -- (219.2000,180.2000) -- (219.2000,180.2000) -- (219.3000,180.2000) -- (219.3000,180.2000) -- (219.4000,180.2000) -- (219.5000,180.2000) -- (219.5000,180.2000) -- (219.6000,180.2000) -- (219.7000,180.2000) -- (219.7000,180.2000) -- (219.8000,180.2000) -- (219.8000,180.2000) -- (219.9000,180.2000) -- (220.0000,180.2000) -- (220.0000,180.2000) -- (220.1000,180.2000) -- (220.1000,180.2000) -- (220.2000,180.2000) -- (220.3000,180.2000) -- (220.3000,180.2000) -- (220.4000,180.2000) -- (220.5000,180.2000) -- (220.5000,180.2000) -- (220.6000,180.2000) -- (220.6000,180.2000) -- (220.7000,180.2000) -- (220.8000,180.2000) -- (220.8000,180.2000) -- (220.9000,180.2000) -- (220.9000,180.2000) -- (221.0000,180.2000) -- (221.1000,180.2000) -- (221.1000,180.2000) -- (221.2000,180.2000) -- (221.3000,180.2000) -- (221.3000,180.2000) -- (221.4000,180.2000) -- (221.4000,180.2000) -- (221.5000,180.2000) -- (221.6000,180.2000) -- (221.6000,180.2000) -- (221.7000,180.2000) -- (221.7000,180.2000) -- (221.8000,180.2000) -- (221.9000,180.2000) -- (221.9000,180.2000) -- (222.0000,180.2000) -- (222.1000,180.2000) -- (222.1000,180.2000) -- (222.2000,180.2000) -- (222.2000,180.2000) -- (222.3000,180.2000) -- (222.4000,180.2000) -- (222.4000,180.2000) -- (222.5000,180.2000) -- (222.5000,180.2000) -- (222.6000,180.2000) -- (222.7000,180.2000) -- (222.7000,180.2000) -- (222.8000,180.2000) -- (222.9000,180.2000) -- (222.9000,180.2000) -- (223.0000,180.2000) -- (223.0000,180.2000) -- (223.1000,180.2000) -- (223.2000,180.2000) -- (223.2000,180.2000) -- (223.3000,180.2000) -- (223.3000,180.2000) -- (223.4000,180.2000) -- (223.5000,180.2000) -- (223.5000,180.2000) -- (223.6000,180.2000) -- (223.7000,180.2000) -- (223.7000,180.2000) -- (223.8000,180.2000) -- (223.8000,180.2000) -- (223.9000,180.2000) -- (224.0000,180.2000) -- (224.0000,180.2000) -- (224.1000,180.2000) -- (224.1000,180.2000) -- (224.2000,180.2000) -- (224.3000,180.2000) -- (224.3000,180.2000) -- (224.4000,180.2000) -- (224.5000,180.2000) -- (224.5000,180.2000) -- (224.6000,180.2000) -- (224.6000,180.2000) -- (224.7000,180.2000) -- (224.8000,180.2000) -- (224.8000,180.2000) -- (224.9000,180.2000) -- (224.9000,180.2000) -- (225.0000,180.2000) -- (225.1000,180.2000) -- (225.1000,180.2000) -- (225.2000,180.2000) -- (225.3000,180.2000) -- (225.3000,180.2000) -- (225.4000,180.2000) -- (225.4000,180.2000) -- (225.5000,180.2000) -- (225.6000,180.2000) -- (225.6000,180.2000) -- (225.7000,180.2000) -- (225.7000,180.2000) -- (225.8000,180.2000) -- (225.9000,179.4000) -- (225.9000,190.4000) -- (226.0000,200.9000) -- (226.0000,200.0000) -- (226.1000,200.0000) -- (226.2000,200.0000) -- (226.2000,200.0000) -- (226.3000,200.0000) -- (226.4000,200.0000) -- (226.4000,200.0000) -- (226.5000,200.0000) -- (226.5000,200.0000) -- (226.6000,200.0000) -- (226.7000,200.0000) -- (226.7000,200.0000) -- (226.8000,200.0000) -- (226.8000,200.0000) -- (226.9000,200.0000) -- (227.0000,200.0000) -- (227.0000,200.0000) -- (227.1000,200.0000) -- (227.2000,200.0000) -- (227.2000,200.0000) -- (227.3000,200.0000) -- (227.3000,200.0000) -- (227.4000,200.0000) -- (227.5000,200.0000) -- (227.5000,200.0000) -- (227.6000,200.0000) -- (227.7000,200.0000) -- (227.7000,200.0000) -- (227.8000,200.0000) -- (227.8000,200.0000) -- (227.9000,200.0000) -- (228.0000,200.0000) -- (228.0000,200.0000) -- (228.1000,200.0000) -- (228.1000,200.0000) -- (228.2000,200.0000) -- (228.3000,200.0000) -- (228.3000,200.0000) -- (228.4000,200.0000) -- (228.5000,200.0000) -- (228.5000,200.0000) -- (228.6000,200.0000) -- (228.6000,200.0000) -- (228.7000,200.0000) -- (228.8000,200.0000) -- (228.8000,200.0000) -- (228.9000,200.0000) -- (228.9000,200.0000) -- (229.0000,200.0000) -- (229.1000,200.0000) -- (229.1000,200.0000) -- (229.2000,200.0000) -- (229.2000,200.0000) -- (229.3000,200.0000) -- (229.4000,200.0000) -- (229.4000,200.0000) -- (229.5000,200.0000) -- (229.6000,200.0000) -- (229.6000,200.0000) -- (229.7000,200.0000) -- (229.7000,200.0000) -- (229.8000,200.0000) -- (229.9000,200.0000) -- (229.9000,200.0000) -- (230.0000,200.0000) -- (230.0000,200.0000) -- (230.1000,200.0000) -- (230.2000,200.0000) -- (230.2000,200.0000) -- (230.3000,200.0000) -- (230.4000,200.0000) -- (230.4000,200.0000) -- (230.5000,200.0000) -- (230.5000,200.0000) -- (230.6000,200.0000) -- (230.7000,200.0000) -- (230.7000,200.0000) -- (230.8000,200.0000) -- (230.8000,200.0000) -- (230.9000,200.0000) -- (231.0000,200.0000) -- (231.0000,200.0000) -- (231.1000,200.0000) -- (231.2000,200.0000) -- (231.2000,200.0000) -- (231.3000,200.0000) -- (231.3000,200.0000) -- (231.4000,200.0000) -- (231.5000,200.0000) -- (231.5000,200.0000) -- (231.6000,200.0000) -- (231.6000,200.0000) -- (231.7000,200.0000) -- (231.8000,200.0000) -- (231.8000,200.0000) -- (231.9000,200.0000) -- (232.0000,200.0000) -- (232.0000,200.0000) -- (232.1000,200.0000) -- (232.1000,200.0000) -- (232.2000,200.0000) -- (232.3000,200.0000) -- (232.3000,200.0000) -- (232.4000,200.0000) -- (232.4000,200.0000) -- (232.5000,200.0000) -- (232.6000,200.0000) -- (232.6000,200.0000) -- (232.7000,200.0000) -- (232.8000,200.0000) -- (232.8000,200.0000) -- (232.9000,200.0000) -- (232.9000,200.0000) -- (233.0000,200.0000) -- (233.1000,200.0000) -- (233.1000,200.0000) -- (233.2000,200.0000) -- (233.2000,200.0000) -- (233.3000,200.0000) -- (233.4000,200.0000) -- (233.4000,200.0000) -- (233.5000,200.0000) -- (233.6000,200.0000) -- (233.6000,200.0000) -- (233.7000,200.0000) -- (233.7000,200.0000) -- (233.8000,200.0000) -- (233.9000,200.0000) -- (233.9000,200.0000) -- (234.0000,200.0000) -- (234.0000,200.0000) -- (234.1000,200.0000) -- (234.2000,200.0000) -- (234.2000,200.0000) -- (234.3000,200.0000) -- (234.4000,200.0000) -- (234.4000,200.0000) -- (234.5000,200.0000) -- (234.5000,200.0000) -- (234.6000,200.0000) -- (234.7000,200.0000) -- (234.7000,200.0000) -- (234.8000,200.0000) -- (234.8000,200.0000) -- (234.9000,200.0000) -- (235.0000,200.0000) -- (235.0000,200.0000) -- (235.1000,200.0000) -- (235.2000,200.0000) -- (235.2000,200.0000) -- (235.3000,200.0000) -- (235.3000,200.0000) -- (235.4000,200.0000) -- (235.5000,200.0000) -- (235.5000,200.0000) -- (235.6000,200.0000) -- (235.6000,200.0000) -- (235.7000,200.0000) -- (235.8000,200.0000) -- (235.8000,200.0000) -- (235.9000,200.0000) -- (236.0000,200.0000) -- (236.0000,200.0000) -- (236.1000,200.0000) -- (236.1000,200.0000) -- (236.2000,200.0000) -- (236.3000,200.0000) -- (236.3000,200.0000) -- (236.4000,200.0000) -- (236.4000,200.0000) -- (236.5000,200.0000) -- (236.6000,200.0000) -- (236.6000,200.0000) -- (236.7000,200.0000) -- (236.8000,200.0000) -- (236.8000,200.0000) -- (236.9000,200.0000) -- (236.9000,200.0000) -- (237.0000,200.0000) -- (237.1000,200.0000) -- (237.1000,200.0000) -- (237.2000,200.0000) -- (237.2000,200.0000) -- (237.3000,200.0000) -- (237.4000,200.0000) -- (237.4000,200.0000) -- (237.5000,200.0000) -- (237.6000,200.0000) -- (237.6000,200.0000) -- (237.7000,200.0000) -- (237.7000,200.0000) -- (237.8000,200.0000) -- (237.9000,200.0000) -- (237.9000,200.0000) -- (238.0000,200.0000) -- (238.0000,200.0000) -- (238.1000,200.0000) -- (238.2000,200.0000) -- (238.2000,200.0000) -- (238.3000,200.0000) -- (238.4000,200.0000) -- (238.4000,200.0000) -- (238.5000,200.0000) -- (238.5000,200.0000) -- (238.6000,200.0000) -- (238.7000,200.0000) -- (238.7000,200.0000) -- (238.8000,200.0000) -- (238.8000,200.0000) -- (238.9000,200.0000) -- (239.0000,200.0000) -- (239.0000,200.0000) -- (239.1000,200.0000) -- (239.2000,200.0000) -- (239.2000,200.0000) -- (239.3000,200.0000) -- (239.3000,200.0000) -- (239.4000,200.0000) -- (239.5000,200.0000) -- (239.5000,200.0000) -- (239.6000,200.0000) -- (239.6000,200.0000) -- (239.7000,200.0000) -- (239.8000,200.0000) -- (239.8000,200.0000) -- (239.9000,200.0000) -- (240.0000,200.0000) -- (240.0000,200.0000) -- (240.1000,200.0000) -- (240.1000,200.0000) -- (240.2000,200.0000) -- (240.3000,200.0000) -- (240.3000,200.0000) -- (240.4000,200.0000) -- (240.4000,200.0000) -- (240.5000,200.0000) -- (240.6000,200.0000) -- (240.6000,200.0000) -- (240.7000,200.0000) -- (240.8000,200.0000) -- (240.8000,200.0000) -- (240.9000,200.0000) -- (240.9000,200.0000) -- (241.0000,200.0000) -- (241.1000,200.0000) -- (241.1000,200.0000) -- (241.2000,200.0000) -- (241.2000,200.0000) -- (241.3000,200.0000) -- (241.4000,200.0000) -- (241.4000,200.0000) -- (241.5000,200.0000) -- (241.6000,200.0000) -- (241.6000,200.0000) -- (241.7000,200.0000) -- (241.7000,200.0000) -- (241.8000,200.0000) -- (241.9000,200.0000) -- (241.9000,200.0000) -- (242.0000,200.0000) -- (242.0000,200.0000) -- (242.1000,200.0000) -- (242.2000,200.0000) -- (242.2000,200.0000) -- (242.3000,200.0000) -- (242.4000,200.0000) -- (242.4000,200.0000) -- (242.5000,200.0000) -- (242.5000,200.0000) -- (242.6000,200.0000) -- (242.7000,200.0000) -- (242.7000,200.0000) -- (242.8000,200.0000) -- (242.8000,200.0000) -- (242.9000,200.0000) -- (243.0000,200.0000) -- (243.0000,200.0000) -- (243.1000,200.0000) -- (243.2000,200.0000) -- (243.2000,200.0000) -- (243.3000,200.0000) -- (243.3000,200.0000) -- (243.4000,200.0000) -- (243.5000,200.0000) -- (243.5000,200.0000) -- (243.6000,200.0000) -- (243.6000,200.0000) -- (243.7000,200.0000) -- (243.8000,200.0000) -- (243.8000,200.0000) -- (243.9000,200.0000) -- (244.0000,200.0000) -- (244.0000,200.0000) -- (244.1000,200.0000) -- (244.1000,200.0000) -- (244.2000,200.0000) -- (244.3000,200.0000) -- (244.3000,200.0000) -- (244.4000,200.0000) -- (244.4000,200.0000) -- (244.5000,200.0000) -- (244.6000,200.0000) -- (244.6000,200.0000) -- (244.7000,200.0000) -- (244.8000,200.0000) -- (244.8000,200.0000) -- (244.9000,200.0000) -- (244.9000,200.0000) -- (245.0000,200.0000) -- (245.1000,200.0000) -- (245.1000,200.0000) -- (245.2000,200.0000) -- (245.2000,200.0000) -- (245.3000,200.0000) -- (245.4000,200.0000) -- (245.4000,200.0000) -- (245.5000,200.0000) -- (245.6000,200.0000) -- (245.6000,200.0000) -- (245.7000,200.0000) -- (245.7000,200.0000) -- (245.8000,200.0000) -- (245.9000,200.0000) -- (245.9000,200.0000) -- (246.0000,200.0000) -- (246.0000,200.0000) -- (246.1000,200.0000) -- (246.2000,200.0000) -- (246.2000,200.0000) -- (246.3000,200.0000);



  \end{scope}
  \begin{scope}[cm={{1.00588,0.0,0.0,1.00588,(0.01175,123.51142)}},draw=black,line join=round,line cap=round,line width=0.480pt]
    \path[draw] (184.5000,156.5000) -- (184.5000,204.5000) -- (246.5000,204.5000) -- (246.5000,156.5000) -- (184.5000,156.5000);



  \end{scope}
  \begin{scope}[cm={{1.00588,0.0,0.0,1.00588,(0.01175,-105.61832)}},draw=ca0a0a4,dash pattern=on 0.40pt off 0.80pt,line join=round,line cap=round,line width=0.400pt]
    \path[draw] (44.5000,319.5000) -- (179.5000,319.5000);



  \end{scope}
  \begin{scope}[cm={{1.00588,0.0,0.0,1.00588,(0.01175,-105.61832)}},draw=black,line join=round,line cap=round,line width=0.480pt]
    \path[draw] (44.5000,319.5000) -- (48.5000,319.5000);



    \path[draw] (179.5000,319.5000) -- (175.5000,319.5000);



  \end{scope}
  \begin{scope}[cm={{1.00588,0.0,0.0,1.00588,(23.47712,217.28767)}},draw=black,line join=bevel,line cap=rect,line width=0.800pt]
    \path[fill=black] (0.0000,0.0000) node[above right] (text1648) {-100};



  \end{scope}
  \begin{scope}[cm={{1.00588,0.0,0.0,1.00588,(0.01175,-105.61832)}},draw=ca0a0a4,dash pattern=on 0.40pt off 0.80pt,line join=round,line cap=round,line width=0.400pt]
    \path[draw] (44.5000,290.5000) -- (179.5000,290.5000);



  \end{scope}
  \begin{scope}[cm={{1.00588,0.0,0.0,1.00588,(0.01175,-105.61832)}},draw=black,line join=round,line cap=round,line width=0.480pt]
    \path[draw] (44.5000,290.5000) -- (48.5000,290.5000);



    \path[draw] (179.5000,290.5000) -- (175.5000,290.5000);



  \end{scope}
  \begin{scope}[cm={{1.00588,0.0,0.0,1.00588,(34.21175,191.11667)}},draw=black,line join=bevel,line cap=rect,line width=0.800pt]
    \path[fill=black] (0.0000,0.0000) node[above right] (text1678) {0};



  \end{scope}
  \begin{scope}[cm={{1.00588,0.0,0.0,1.00588,(0.01175,-105.61832)}},draw=ca0a0a4,dash pattern=on 0.40pt off 0.80pt,line join=round,line cap=round,line width=0.400pt]
    \path[draw] (44.5000,261.5000) -- (179.5000,261.5000);



  \end{scope}
  \begin{scope}[cm={{1.00588,0.0,0.0,1.00588,(0.01175,-105.61832)}},draw=black,line join=round,line cap=round,line width=0.480pt]
    \path[draw] (44.5000,261.5000) -- (48.5000,261.5000);



    \path[draw] (179.5000,261.5000) -- (175.5000,261.5000);



  \end{scope}
  \begin{scope}[cm={{1.00588,0.0,0.0,1.00588,(26.16471,160.94067)}},draw=black,line join=bevel,line cap=rect,line width=0.800pt]
    \path[fill=black] (0.0000,0.0000) node[above right] (text1708) {100};



  \end{scope}
  \begin{scope}[cm={{1.00588,0.0,0.0,1.00588,(0.01175,-105.61832)}},draw=ca0a0a4,dash pattern=on 0.40pt off 0.80pt,line join=round,line cap=round,line width=0.400pt]
    \path[draw] (44.5000,232.5000) -- (179.5000,232.5000);



  \end{scope}
  \begin{scope}[cm={{1.00588,0.0,0.0,1.00588,(0.01175,-105.61832)}},draw=black,line join=round,line cap=round,line width=0.480pt]
    \path[draw] (44.5000,232.5000) -- (48.5000,232.5000);



    \path[draw] (179.5000,232.5000) -- (175.5000,232.5000);



  \end{scope}
  \begin{scope}[cm={{1.00588,0.0,0.0,1.00588,(26.16471,131.76967)}},draw=black,line join=bevel,line cap=rect,line width=0.800pt]
    \path[fill=black] (0.0000,0.0000) node[above right] (text1738) {200};



  \end{scope}
  \begin{scope}[cm={{1.00588,0.0,0.0,1.00588,(0.01175,-105.61832)}},draw=ca0a0a4,dash pattern=on 0.40pt off 0.80pt,line join=round,line cap=round,line width=0.400pt]
    \path[draw] (44.5000,319.5000) -- (44.5000,217.5000);



  \end{scope}
  \begin{scope}[cm={{1.00588,0.0,0.0,1.00588,(0.01175,-105.61832)}},draw=black,line join=round,line cap=round,line width=0.480pt]
    \path[draw] (44.5000,319.5000) -- (44.5000,316.5000);



    \path[draw] (44.5000,217.5000) -- (44.5000,220.5000);



  \end{scope}
  \begin{scope}[cm={{1.00588,0.0,0.0,1.00588,(34.21175,340.85267)}},draw=black,line join=bevel,line cap=rect,line width=0.800pt]
    \path[fill=black] (0.0000,0.0000) node[above right] (text1768) {-150};



  \end{scope}
  \begin{scope}[cm={{1.00588,0.0,0.0,1.00588,(0.01175,-105.61832)}},draw=ca0a0a4,dash pattern=on 0.40pt off 0.80pt,line join=round,line cap=round,line width=0.400pt]
    \path[draw] (78.5000,319.5000) -- (78.5000,217.5000);



  \end{scope}
  \begin{scope}[cm={{1.00588,0.0,0.0,1.00588,(0.01175,-105.61832)}},draw=black,line join=round,line cap=round,line width=0.480pt]
    \path[draw] (78.5000,319.5000) -- (78.5000,316.5000);



    \path[draw] (78.5000,217.5000) -- (78.5000,220.5000);



  \end{scope}
  \begin{scope}[cm={{1.00588,0.0,0.0,1.00588,(70.42355,340.85267)}},draw=black,line join=bevel,line cap=rect,line width=0.800pt]
    \path[fill=black] (0.0000,0.0000) node[above right] (text1798) {-50};



  \end{scope}
  \begin{scope}[cm={{1.00588,0.0,0.0,1.00588,(0.01175,-105.61832)}},draw=ca0a0a4,dash pattern=on 0.40pt off 0.80pt,line join=round,line cap=round,line width=0.400pt]
    \path[draw] (112.5000,319.5000) -- (112.5000,217.5000);



  \end{scope}
  \begin{scope}[cm={{1.00588,0.0,0.0,1.00588,(0.01175,-105.61832)}},draw=black,line join=round,line cap=round,line width=0.480pt]
    \path[draw] (112.5000,319.5000) -- (112.5000,316.5000);



    \path[draw] (112.5000,217.5000) -- (112.5000,220.5000);



  \end{scope}
  \begin{scope}[cm={{1.00588,0.0,0.0,1.00588,(105.12675,340.85267)}},draw=black,line join=bevel,line cap=rect,line width=0.800pt]
    \path[fill=black] (0.0000,0.0000) node[above right] (text1828) {50};



  \end{scope}
  \begin{scope}[cm={{1.00588,0.0,0.0,1.00588,(0.01175,-105.61832)}},draw=ca0a0a4,dash pattern=on 0.40pt off 0.80pt,line join=round,line cap=round,line width=0.400pt]
    \path[draw] (145.5000,319.5000) -- (145.5000,217.5000);



  \end{scope}
  \begin{scope}[cm={{1.00588,0.0,0.0,1.00588,(0.01175,-105.61832)}},draw=black,line join=round,line cap=round,line width=0.480pt]
    \path[draw] (145.5000,319.5000) -- (145.5000,316.5000);



    \path[draw] (145.5000,217.5000) -- (145.5000,220.5000);



  \end{scope}
  \begin{scope}[cm={{1.00588,0.0,0.0,1.00588,(136.30875,340.85267)}},draw=black,line join=bevel,line cap=rect,line width=0.800pt]
    \path[fill=black] (0.0000,0.0000) node[above right] (text1858) {150};



  \end{scope}
  \begin{scope}[cm={{1.00588,0.0,0.0,1.00588,(0.01175,-105.61832)}},draw=black,line join=round,line cap=round,line width=0.480pt]
    \path[draw] (44.5000,217.5000) -- (44.5000,319.5000) -- (179.5000,319.5000) -- (179.5000,217.5000) -- (44.5000,217.5000);



  \end{scope}
  \begin{scope}[cm={{1.00588,0.0,0.0,1.00588,(0.01175,-105.61832)}},fill=cffffff]
    \path[fill,rounded corners=0.0000cm] (156.0000,224.0000) rectangle (172.0000,240.0000);



  \end{scope}
  \begin{scope}[cm={{1.00588,0.0,0.0,1.00588,(0.01175,-105.61832)}},draw=black,fill=cebebeb,line join=round,line cap=round,line width=0.800pt]
    \path[draw,fill=cebebeb] (156.5000,240.5000) -- (156.5000,224.5000) -- (172.5000,224.5000) -- (172.5000,240.5000) -- (156.5000,240.5000);



  \end{scope}
  \begin{scope}[cm={{1.00588,0.0,0.0,1.00588,(163.0453,131.55236)}},draw=black,line join=bevel,line cap=rect,line width=0.800pt]
    \path[fill=black] (0.0000,0.0000) node[above right] (text1898) {\label{fig:trajs-dyn-ii}ii};



  \end{scope}
  \begin{scope}[cm={{1.00588,0.0,0.0,1.00588,(0.01175,-105.61832)}},draw=black,line join=round,line cap=round,line width=0.480pt]
    \path[draw] (61.6000,227.9000) -- (61.6000,227.9000) -- (61.8000,231.0000) -- (61.9000,233.8000) -- (60.9000,236.4000) -- (59.0000,238.8000) -- (57.6000,241.4000) -- (56.8000,244.1000) -- (56.6000,247.0000) -- (56.6000,249.9000) -- (56.5000,252.8000) -- (56.5000,255.7000) -- (56.5000,258.6000) -- (56.5000,261.5000) -- (56.5000,264.4000) -- (56.5000,267.3000) -- (56.5000,270.2000) -- (56.5000,273.1000) -- (56.5000,276.0000) -- (56.5000,278.9000) -- (56.5000,281.8000) -- (56.5000,284.6000) -- (56.5000,287.5000) -- (56.8000,290.4000) -- (57.5000,293.3000) -- (58.7000,296.0000) -- (60.2000,298.6000) -- (62.2000,301.0000) -- (64.5000,303.1000) -- (67.1000,305.0000) -- (70.1000,306.6000) -- (73.2000,307.7000) -- (76.6000,308.6000) -- (80.1000,309.0000) -- (83.6000,308.9000) -- (87.0000,308.5000) -- (90.4000,307.6000) -- (93.6000,306.4000) -- (96.6000,304.8000) -- (99.2000,302.8000) -- (101.6000,300.6000) -- (103.4000,298.1000) -- (104.8000,295.4000) -- (105.6000,292.6000) -- (106.1000,289.8000) -- (106.3000,286.9000) -- (106.5000,284.1000) -- (106.5000,281.2000) -- (106.6000,278.3000) -- (106.6000,275.4000) -- (106.6000,272.5000) -- (106.6000,269.6000) -- (106.6000,266.8000) -- (106.6000,263.9000) -- (106.6000,261.0000) -- (106.6000,258.1000) -- (106.6000,255.2000) -- (106.7000,252.3000) -- (107.2000,249.4000) -- (107.4000,246.5000) -- (107.1000,243.7000) -- (106.2000,240.9000) -- (104.9000,238.2000) -- (103.2000,235.7000) -- (100.9000,233.5000) -- (98.2000,231.6000) -- (95.2000,230.1000) -- (91.9000,229.0000) -- (88.5000,228.4000) -- (85.0000,228.3000) -- (81.5000,228.7000) -- (78.2000,229.6000) -- (75.1000,230.9000) -- (72.3000,232.6000) -- (69.9000,234.7000) -- (67.8000,237.1000) -- (66.3000,239.7000) -- (65.3000,242.4000) -- (64.9000,245.3000) -- (64.8000,248.2000) -- (64.8000,251.1000) -- (64.8000,254.0000) -- (64.8000,256.9000) -- (64.8000,259.8000) -- (64.8000,262.7000) -- (64.8000,265.6000) -- (64.8000,268.5000) -- (64.8000,271.4000) -- (64.8000,274.2000) -- (64.8000,277.1000) -- (64.8000,280.0000) -- (64.8000,282.9000) -- (64.7000,285.8000) -- (64.9000,288.7000) -- (65.4000,291.6000) -- (66.4000,294.3000) -- (67.8000,297.0000) -- (69.6000,299.4000) -- (71.8000,301.7000) -- (74.4000,303.7000) -- (77.2000,305.4000) -- (80.3000,306.7000) -- (83.5000,307.7000) -- (87.0000,308.3000) -- (90.5000,308.4000) -- (94.0000,308.2000) -- (97.4000,307.5000) -- (100.7000,306.4000) -- (103.8000,305.0000) -- (106.6000,303.2000) -- (109.1000,301.1000) -- (111.2000,298.7000) -- (112.8000,296.1000) -- (113.8000,293.3000) -- (114.4000,290.5000) -- (114.8000,287.7000) -- (114.9000,284.8000) -- (115.0000,282.0000) -- (115.1000,279.1000) -- (115.1000,276.2000) -- (115.1000,273.3000) -- (115.1000,270.4000) -- (115.1000,267.5000) -- (115.1000,264.6000) -- (115.1000,261.7000) -- (115.1000,258.9000) -- (115.1000,256.0000) -- (115.1000,253.1000) -- (115.4000,250.2000) -- (115.8000,247.3000) -- (115.7000,244.4000) -- (115.1000,241.6000) -- (114.1000,238.8000) -- (112.7000,236.2000) -- (110.8000,233.8000) -- (108.6000,231.5000) -- (105.9000,229.6000) -- (102.9000,228.0000) -- (99.7000,226.8000) -- (96.3000,226.0000) -- (92.8000,225.7000) -- (89.3000,225.8000) -- (85.9000,226.4000) -- (82.6000,227.4000) -- (79.5000,228.7000) -- (76.7000,230.5000) -- (74.3000,232.5000) -- (72.2000,234.8000) -- (70.4000,237.3000) -- (69.1000,240.0000) -- (68.4000,242.8000) -- (68.1000,245.7000) -- (68.1000,248.6000) -- (68.1000,251.5000) -- (68.1000,254.4000) -- (68.1000,257.3000) -- (68.1000,260.2000) -- (68.1000,263.1000) -- (68.1000,266.0000) -- (68.1000,268.9000) -- (68.1000,271.8000) -- (68.1000,274.6000) -- (68.1000,277.5000) -- (68.1000,280.4000) -- (68.1000,283.3000) -- (68.0000,286.2000) -- (68.3000,289.1000) -- (68.9000,291.9000) -- (70.0000,294.7000) -- (71.5000,297.3000) -- (73.4000,299.7000) -- (75.7000,301.9000) -- (78.3000,303.8000) -- (81.2000,305.4000) -- (84.3000,306.7000) -- (87.6000,307.6000) -- (91.1000,308.1000) -- (94.6000,308.2000) -- (98.1000,307.8000) -- (101.5000,307.1000) -- (104.7000,305.9000) -- (107.7000,304.4000) -- (110.5000,302.5000) -- (112.9000,300.3000) -- (114.9000,297.9000) -- (116.4000,295.3000) -- (117.4000,292.5000) -- (117.9000,289.7000) -- (118.2000,286.9000) -- (118.3000,284.0000) -- (118.4000,281.1000) -- (118.4000,278.3000) -- (118.5000,275.4000) -- (118.5000,272.5000) -- (118.5000,269.6000) -- (118.5000,266.7000) -- (118.5000,263.8000) -- (118.5000,260.9000) -- (118.5000,258.0000) -- (118.5000,255.1000) -- (118.5000,252.2000) -- (118.9000,249.4000) -- (119.1000,246.5000) -- (118.9000,243.6000) -- (118.3000,240.8000) -- (117.2000,238.0000) -- (115.6000,235.4000) -- (113.6000,233.0000) -- (111.3000,230.9000) -- (108.5000,229.0000) -- (105.5000,227.5000) -- (102.2000,226.4000) -- (98.8000,225.8000) -- (95.3000,225.6000) -- (91.8000,225.8000) -- (88.4000,226.5000) -- (85.2000,227.5000) -- (82.2000,229.0000) -- (79.4000,230.8000) -- (77.1000,232.9000) -- (75.0000,235.3000) -- (73.4000,237.8000) -- (72.3000,240.6000) -- (71.7000,243.4000) -- (71.5000,246.3000) -- (71.5000,249.2000) -- (71.5000,252.1000) -- (71.5000,255.0000) -- (71.5000,257.9000) -- (71.5000,260.8000) -- (71.5000,263.7000) -- (71.5000,266.6000) -- (71.5000,269.5000) -- (71.5000,272.4000) -- (71.5000,275.3000) -- (71.5000,278.1000) -- (71.5000,281.0000) -- (71.5000,283.9000) -- (71.5000,286.8000) -- (71.9000,289.7000) -- (72.7000,292.5000) -- (73.9000,295.2000) -- (75.5000,297.8000) -- (77.5000,300.1000) -- (79.9000,302.2000) -- (82.6000,304.1000) -- (85.6000,305.6000) -- (88.8000,306.7000) -- (92.2000,307.5000) -- (95.6000,307.8000) -- (99.1000,307.8000) -- (102.6000,307.3000) -- (106.0000,306.4000) -- (109.2000,305.1000) -- (112.1000,303.4000) -- (114.7000,301.4000) -- (117.0000,299.2000) -- (118.9000,296.7000) -- (120.2000,294.0000) -- (120.9000,291.2000) -- (121.3000,288.3000) -- (121.6000,285.5000) -- (121.7000,282.6000) -- (121.8000,279.8000) -- (121.8000,276.9000) -- (121.8000,274.0000) -- (121.8000,271.1000) -- (121.8000,268.2000) -- (121.8000,265.3000) -- (121.8000,262.4000) -- (121.8000,259.5000) -- (121.8000,256.6000) -- (121.8000,253.7000) -- (122.0000,250.9000) -- (122.4000,248.0000) -- (122.4000,245.1000) -- (122.1000,242.2000) -- (121.2000,239.4000) -- (120.0000,236.7000) -- (118.3000,234.2000) -- (116.1000,231.9000) -- (113.6000,229.9000) -- (110.8000,228.1000) -- (107.6000,226.8000) -- (104.3000,225.9000) -- (100.8000,225.4000) -- (97.3000,225.3000) -- (93.8000,225.7000) -- (90.5000,226.6000) -- (87.3000,227.8000) -- (84.4000,229.4000) -- (81.8000,231.3000) -- (79.6000,233.5000) -- (77.7000,236.0000) -- (76.3000,238.6000) -- (75.3000,241.4000) -- (75.0000,244.3000) -- (74.9000,247.2000) -- (74.9000,250.1000) -- (74.9000,253.0000) -- (74.9000,255.9000) -- (74.9000,258.8000) -- (74.9000,261.6000) -- (74.9000,264.5000) -- (74.9000,267.4000) -- (74.9000,270.3000) -- (74.9000,273.2000) -- (74.9000,276.1000) -- (74.9000,279.0000) -- (74.9000,281.9000) -- (74.9000,284.8000) -- (75.0000,287.7000) -- (75.5000,290.5000) -- (76.5000,293.3000) -- (77.8000,296.0000) -- (79.6000,298.4000) -- (81.8000,300.7000) -- (84.3000,302.7000) -- (87.1000,304.4000) -- (90.2000,305.8000) -- (93.5000,306.8000) -- (96.9000,307.4000) -- (100.4000,307.6000) -- (103.9000,307.3000) -- (107.3000,306.7000) -- (110.6000,305.6000) -- (113.7000,304.2000) -- (116.5000,302.4000) -- (119.0000,300.3000) -- (121.1000,297.9000) -- (122.8000,295.3000) -- (123.9000,292.6000) -- (124.5000,289.7000) -- (124.8000,286.9000) -- (125.0000,284.1000) -- (125.1000,281.2000) -- (125.1000,278.3000) -- (125.1000,275.4000) -- (125.2000,272.5000) -- (125.2000,269.6000) -- (125.2000,266.8000) -- (125.2000,263.9000) -- (125.2000,261.0000) -- (125.2000,258.1000) -- (125.2000,255.2000) -- (125.2000,252.3000) -- (125.5000,249.4000) -- (125.8000,246.5000) -- (125.7000,243.6000) -- (125.1000,240.8000) -- (124.1000,238.1000) -- (122.7000,235.4000) -- (120.8000,233.0000) -- (118.5000,230.8000) -- (115.9000,228.8000) -- (112.9000,227.3000) -- (109.6000,226.1000) -- (106.2000,225.3000) -- (102.7000,225.0000) -- (99.3000,225.2000) -- (95.8000,225.7000) -- (92.6000,226.7000) -- (89.5000,228.1000) -- (86.7000,229.8000) -- (84.2000,231.9000) -- (82.2000,234.2000) -- (80.5000,236.7000) -- (79.2000,239.4000) -- (78.5000,242.3000) -- (78.3000,245.2000) -- (78.2000,248.1000) -- (78.2000,251.0000) -- (78.2000,253.9000) -- (78.2000,256.7000) -- (78.2000,259.6000) -- (78.2000,262.5000) -- (78.2000,265.4000) -- (78.2000,268.3000) -- (78.2000,271.2000) -- (78.2000,274.1000) -- (78.2000,277.0000) -- (78.2000,279.9000) -- (78.2000,282.8000) -- (78.2000,285.7000) -- (78.5000,288.5000) -- (79.2000,291.4000) -- (80.3000,294.1000) -- (81.9000,296.7000) -- (83.8000,299.1000) -- (86.1000,301.3000) -- (88.8000,303.1000) -- (91.7000,304.7000) -- (94.9000,305.9000) -- (98.2000,306.8000) -- (101.7000,307.2000) -- (105.2000,307.2000) -- (108.7000,306.8000) -- (112.0000,306.0000) -- (115.3000,304.8000) -- (118.3000,303.2000) -- (120.9000,301.2000) -- (123.3000,299.0000) -- (125.2000,296.6000) -- (126.7000,293.9000) -- (127.5000,291.1000) -- (128.0000,288.3000) -- (128.2000,285.4000) -- (128.4000,282.6000) -- (128.5000,279.7000) -- (128.5000,276.8000) -- (128.5000,273.9000) -- (128.5000,271.0000) -- (128.5000,268.2000) -- (128.5000,265.3000) -- (128.5000,262.4000) -- (128.5000,259.5000) -- (128.5000,256.6000) -- (128.5000,253.7000) -- (128.6000,250.8000) -- (129.0000,247.9000) -- (129.2000,245.0000) -- (128.9000,242.2000) -- (128.1000,239.4000) -- (126.9000,236.7000) -- (125.3000,234.1000) -- (123.3000,231.7000) -- (120.8000,229.6000) -- (118.0000,227.9000) -- (114.9000,226.4000) -- (111.6000,225.4000) -- (108.2000,224.9000) -- (104.6000,224.7000) -- (101.2000,225.1000) -- (97.8000,225.8000) -- (94.6000,227.0000) -- (91.7000,228.5000) -- (89.0000,230.4000) -- (86.7000,232.5000) -- (84.8000,234.9000) -- (83.2000,237.5000) -- (82.2000,240.3000) -- (81.7000,243.2000) -- (81.6000,246.1000) -- (81.6000,249.0000) -- (81.6000,251.9000) -- (81.6000,254.8000) -- (81.6000,257.7000) -- (81.6000,260.6000) -- (81.6000,263.4000) -- (81.6000,266.3000) -- (81.6000,269.2000) -- (81.6000,272.1000) -- (81.6000,275.0000) -- (81.6000,277.9000) -- (81.6000,280.8000) -- (81.6000,283.7000) -- (81.6000,286.6000) -- (82.1000,289.4000) -- (83.0000,292.2000) -- (84.3000,294.9000) -- (86.0000,297.4000) -- (88.1000,299.7000) -- (90.6000,301.8000) -- (93.3000,303.5000) -- (96.4000,305.0000) -- (99.6000,306.0000) -- (103.0000,306.7000) -- (106.5000,307.0000) -- (110.0000,306.8000) -- (113.4000,306.2000) -- (116.8000,305.2000) -- (119.9000,303.8000) -- (122.8000,302.1000) -- (125.3000,300.1000) -- (127.5000,297.7000) -- (129.3000,295.2000) -- (130.4000,292.5000) -- (131.1000,289.6000) -- (131.5000,286.8000) -- (131.7000,284.0000) -- (131.8000,281.1000) -- (131.8000,278.2000) -- (131.9000,275.3000) -- (131.9000,272.4000) -- (131.9000,269.6000) -- (131.9000,266.7000) -- (131.9000,263.8000) -- (131.9000,260.9000) -- (131.9000,258.0000) -- (131.9000,255.1000) -- (131.9000,252.2000) -- (132.1000,249.3000) -- (132.5000,246.5000) -- (132.5000,243.6000) -- (132.0000,240.7000) -- (131.1000,237.9000) -- (129.7000,235.3000) -- (127.9000,232.8000) -- (125.7000,230.5000) -- (123.1000,228.6000) -- (120.1000,226.9000) -- (116.9000,225.7000) -- (113.5000,224.8000) -- (110.1000,224.5000) -- (106.6000,224.5000) -- (103.1000,225.0000) -- (99.8000,225.9000) -- (96.7000,227.3000) -- (93.9000,228.9000) -- (91.4000,230.9000) -- (89.2000,233.2000) -- (87.4000,235.7000) -- (86.1000,238.4000) -- (85.3000,241.2000) -- (85.0000,244.1000) -- (85.0000,247.0000) -- (84.9000,249.9000) -- (84.9000,252.8000) -- (84.9000,255.7000) -- (84.9000,258.6000) -- (84.9000,261.4000) -- (84.9000,264.3000) -- (84.9000,267.2000) -- (84.9000,270.1000) -- (84.9000,273.0000) -- (84.9000,275.9000) -- (84.9000,278.8000) -- (84.9000,281.7000) -- (84.9000,284.6000) -- (85.1000,287.5000) -- (85.8000,290.3000) -- (86.8000,293.1000) -- (88.3000,295.7000) -- (90.2000,298.1000) -- (92.4000,300.3000) -- (95.0000,302.3000) -- (97.9000,303.9000) -- (101.0000,305.2000) -- (104.3000,306.1000) -- (107.8000,306.6000) -- (111.3000,306.7000) -- (114.8000,306.3000) -- (118.2000,305.6000) -- (121.4000,304.4000) -- (124.5000,302.9000) -- (127.2000,301.0000) -- (129.6000,298.9000) -- (131.6000,296.5000) -- (133.2000,293.8000) -- (134.1000,291.0000) -- (134.7000,288.2000) -- (134.9000,285.4000) -- (135.1000,282.5000) -- (135.2000,279.7000) -- (135.2000,276.8000) -- (135.2000,273.9000) -- (135.2000,271.0000) -- (135.2000,268.1000) -- (135.2000,265.2000) -- (135.2000,262.3000) -- (135.2000,259.4000) -- (135.2000,256.5000) -- (135.2000,253.6000) -- (135.3000,250.8000) -- (135.7000,247.9000) -- (135.9000,245.0000) -- (135.7000,242.1000) -- (135.0000,239.3000) -- (133.9000,236.6000) -- (132.4000,234.0000) -- (130.4000,231.6000) -- (128.0000,229.4000) -- (125.3000,227.6000) -- (122.2000,226.1000) -- (118.9000,225.0000) -- (115.5000,224.3000) -- (112.0000,224.1000) -- (108.5000,224.4000) -- (105.1000,225.0000) -- (101.9000,226.1000) -- (98.9000,227.6000) -- (96.2000,229.4000) -- (93.8000,231.5000) -- (91.8000,233.9000) -- (90.2000,236.4000) -- (89.1000,239.2000) -- (88.5000,242.0000) -- (88.3000,244.9000) -- (88.3000,247.8000) -- (88.3000,250.7000) -- (88.3000,253.6000) -- (88.3000,256.5000) -- (88.3000,259.4000) -- (88.3000,262.3000) -- (88.3000,265.2000) -- (88.3000,268.1000) -- (88.3000,271.0000) -- (88.3000,273.9000) -- (88.3000,276.8000) -- (88.3000,279.7000) -- (88.3000,282.5000) -- (88.3000,285.4000) -- (88.7000,288.3000) -- (89.5000,291.1000) -- (90.7000,293.8000) -- (92.4000,296.4000) -- (94.4000,298.7000) -- (96.8000,300.8000) -- (99.5000,302.7000) -- (102.5000,304.1000) -- (105.7000,305.3000) -- (109.0000,306.0000) -- (112.5000,306.4000) -- (116.0000,306.3000) -- (119.5000,305.8000) -- (122.8000,304.9000) -- (126.0000,303.6000) -- (129.0000,301.9000) -- (131.6000,299.9000) -- (133.8000,297.6000) -- (135.7000,295.1000) -- (137.0000,292.4000) -- (137.7000,289.6000) -- (138.1000,286.8000) -- (138.4000,284.0000) -- (138.5000,281.1000) -- (138.5000,278.2000) -- (138.6000,275.3000) -- (138.6000,272.4000) -- (138.6000,269.5000) -- (138.6000,266.7000) -- (138.6000,263.8000) -- (138.6000,260.9000) -- (138.6000,258.0000) -- (138.6000,255.1000) -- (138.6000,252.2000) -- (138.8000,249.3000) -- (139.2000,246.4000) -- (139.2000,243.5000) -- (138.8000,240.7000) -- (138.0000,237.9000) -- (136.7000,235.2000) -- (135.0000,232.7000) -- (132.9000,230.4000) -- (130.3000,228.4000) -- (127.4000,226.6000) -- (124.3000,225.3000) -- (121.0000,224.4000) -- (117.5000,223.9000) -- (114.0000,223.9000) -- (110.5000,224.3000) -- (107.2000,225.1000) -- (104.0000,226.4000) -- (101.1000,228.0000) -- (98.6000,229.9000) -- (96.3000,232.1000) -- (94.5000,234.6000) -- (93.0000,237.2000) -- (92.1000,240.0000) -- (91.8000,242.9000) -- (91.7000,245.8000) -- (91.7000,248.7000) -- (91.7000,251.6000) -- (91.6000,254.5000) -- (91.6000,257.4000) -- (91.6000,260.3000) -- (91.6000,263.2000) -- (91.6000,266.1000) -- (91.6000,269.0000) -- (91.6000,271.8000) -- (91.6000,274.7000) -- (91.6000,277.6000) -- (91.6000,280.5000) -- (91.6000,283.4000) -- (91.8000,286.3000) -- (92.3000,289.2000) -- (93.3000,291.9000) -- (94.7000,294.6000) -- (96.5000,297.1000) -- (98.7000,299.3000) -- (101.2000,301.3000) -- (104.0000,303.0000) -- (107.1000,304.4000) -- (110.3000,305.3000) -- (113.8000,305.9000) -- (117.3000,306.1000) -- (120.8000,305.8000) -- (124.2000,305.2000) -- (127.5000,304.1000) -- (130.6000,302.6000) -- (133.4000,300.8000) -- (135.9000,298.7000) -- (138.0000,296.4000) -- (139.6000,293.8000) -- (140.7000,291.0000) -- (141.3000,288.2000) -- (141.6000,285.4000) -- (141.8000,282.5000) -- (141.9000,279.7000) -- (141.9000,276.8000) -- (141.9000,273.9000) -- (141.9000,271.0000) -- (142.0000,268.1000) -- (142.0000,265.2000) -- (142.0000,262.3000) -- (142.0000,259.4000) -- (142.0000,256.5000) -- (142.0000,253.6000) -- (142.0000,250.8000) -- (142.3000,247.9000) -- (142.6000,245.0000) -- (142.5000,242.1000) -- (141.9000,239.3000) -- (140.9000,236.5000) -- (139.4000,233.9000) -- (137.5000,231.5000) -- (135.2000,229.2000) -- (132.6000,227.3000) -- (129.6000,225.8000) -- (126.3000,224.6000) -- (122.9000,223.9000) -- (119.4000,223.6000) -- (115.9000,223.7000) -- (112.5000,224.3000) -- (109.3000,225.3000) -- (106.2000,226.7000) -- (103.4000,228.4000) -- (101.0000,230.5000) -- (98.9000,232.8000) -- (97.2000,235.3000) -- (96.0000,238.0000) -- (95.3000,240.9000) -- (95.1000,243.8000) -- (95.0000,246.7000) -- (95.0000,249.6000) -- (95.0000,252.5000) -- (95.0000,255.4000) -- (95.0000,258.3000) -- (95.0000,261.1000) -- (95.0000,264.0000) -- (95.0000,266.9000) -- (95.0000,269.8000) -- (95.0000,272.7000) -- (95.0000,275.6000) -- (95.0000,278.5000) -- (95.0000,281.4000) -- (95.0000,284.3000) -- (95.3000,287.1000) -- (96.0000,290.0000) -- (97.2000,292.7000) -- (98.7000,295.3000) -- (100.7000,297.7000) -- (103.0000,299.9000) -- (105.6000,301.7000) -- (108.6000,303.3000) -- (111.7000,304.5000) -- (115.1000,305.3000) -- (118.5000,305.8000) -- (122.0000,305.8000) -- (125.5000,305.3000) -- (128.9000,304.5000) -- (132.1000,303.3000) -- (135.1000,301.7000) -- (137.8000,299.7000) -- (140.1000,297.5000) -- (142.1000,295.1000) -- (143.5000,292.4000) -- (144.3000,289.6000) -- (144.8000,286.8000) -- (145.0000,283.9000) -- (145.2000,281.1000) -- (145.2000,278.2000) -- (145.3000,275.3000) -- (145.3000,272.4000) -- (145.3000,269.5000) -- (145.3000,266.7000) -- (145.3000,263.8000) -- (145.3000,260.9000) -- (145.3000,258.0000) -- (145.3000,255.1000) -- (145.3000,252.2000) -- (145.4000,249.3000) -- (145.8000,246.4000) -- (146.0000,243.6000) -- (145.7000,240.7000) -- (144.9000,237.9000) -- (143.7000,235.2000) -- (142.1000,232.6000) -- (140.0000,230.3000) -- (137.6000,228.2000) -- (134.8000,226.4000) -- (131.7000,225.0000) -- (128.4000,224.0000) -- (124.9000,223.4000) -- (121.4000,223.3000) -- (117.9000,223.6000) -- (114.6000,224.3000) -- (111.4000,225.5000) -- (108.4000,227.0000) -- (105.8000,228.9000) -- (103.5000,231.1000) -- (101.5000,233.5000) -- (100.0000,236.1000) -- (99.0000,238.9000) -- (98.5000,241.7000) -- (98.4000,244.6000) -- (98.4000,247.5000) -- (98.4000,250.4000) -- (98.4000,253.3000) -- (98.4000,256.2000) -- (98.4000,259.1000) -- (98.4000,262.0000) -- (98.4000,264.9000) -- (98.4000,267.8000) -- (98.4000,270.7000) -- (98.4000,273.6000) -- (98.4000,276.5000) -- (98.4000,279.3000) -- (98.4000,282.2000) -- (98.4000,285.1000) -- (98.9000,288.0000) -- (99.8000,290.8000) -- (101.1000,293.5000) -- (102.8000,296.0000) -- (104.9000,298.3000) -- (107.4000,300.3000) -- (110.1000,302.1000) -- (113.1000,303.5000) -- (116.4000,304.6000) -- (119.8000,305.3000) -- (123.3000,305.5000) -- (126.8000,305.3000) -- (130.2000,304.8000) -- (133.6000,303.8000) -- (136.7000,302.4000) -- (139.6000,300.7000) -- (142.1000,298.6000) -- (144.3000,296.3000) -- (146.1000,293.7000) -- (147.2000,291.0000) -- (147.9000,288.2000) -- (148.3000,285.4000) -- (148.5000,282.5000) -- (148.6000,279.7000) -- (148.6000,276.8000) -- (148.6000,273.9000) -- (148.7000,271.0000) -- (148.7000,268.1000) -- (148.7000,265.2000) -- (148.7000,262.3000) -- (148.7000,259.4000) -- (148.7000,256.5000) -- (148.7000,253.6000) -- (148.7000,250.8000) -- (148.9000,247.9000) -- (149.3000,245.0000) -- (149.3000,242.1000) -- (148.8000,239.3000) -- (147.8000,236.5000) -- (146.5000,233.8000) -- (144.7000,231.3000) -- (142.5000,229.1000) -- (139.9000,227.1000) -- (136.9000,225.5000) -- (133.7000,224.2000) -- (130.3000,223.4000) -- (126.9000,223.0000) -- (123.4000,223.1000) -- (119.9000,223.6000) -- (116.6000,224.5000) -- (113.5000,225.8000) -- (110.7000,227.5000) -- (108.2000,229.5000) -- (106.0000,231.7000) -- (104.2000,234.2000) -- (102.9000,236.9000) -- (102.1000,239.7000) -- (101.8000,242.6000) -- (101.7000,245.5000) -- (101.7000,248.4000) -- (101.7000,251.3000) -- (101.7000,254.2000) -- (101.7000,257.1000) -- (101.7000,260.0000) -- (101.7000,262.9000) -- (101.7000,265.8000) -- (101.7000,268.6000) -- (101.7000,271.5000) -- (101.7000,274.4000) -- (101.7000,277.3000) -- (101.7000,280.2000) -- (101.7000,283.1000) -- (101.9000,286.0000) -- (102.5000,288.8000) -- (103.6000,291.6000) -- (105.1000,294.2000) -- (107.0000,296.6000) -- (109.2000,298.9000) -- (111.8000,300.8000) -- (114.7000,302.4000) -- (117.8000,303.7000) -- (121.1000,304.6000) -- (124.5000,305.1000) -- (128.0000,305.2000) -- (131.5000,304.9000) -- (134.9000,304.1000) -- (138.2000,303.0000) -- (141.2000,301.4000) -- (144.0000,299.6000) -- (146.4000,297.4000) -- (148.4000,295.0000) -- (150.0000,292.4000) -- (150.9000,289.6000) -- (151.4000,286.8000) -- (151.7000,283.9000) -- (151.9000,281.1000) -- (152.0000,278.2000) -- (152.0000,275.3000) -- (152.0000,272.4000) -- (152.0000,269.5000) -- (152.0000,266.7000) -- (152.0000,263.8000) -- (152.0000,260.9000) -- (152.0000,258.0000) -- (152.0000,255.1000) -- (152.0000,252.2000) -- (152.1000,249.3000) -- (152.4000,246.4000) -- (152.7000,243.6000) -- (152.5000,240.7000) -- (151.8000,237.8000) -- (150.7000,235.1000) -- (149.2000,232.5000) -- (147.2000,230.1000) -- (144.8000,228.0000) -- (142.1000,226.1000) -- (139.0000,224.6000) -- (135.8000,223.5000) -- (132.3000,222.9000) -- (128.8000,222.7000) -- (125.3000,222.9000) -- (121.9000,223.6000) -- (118.7000,224.7000) -- (115.7000,226.1000) -- (113.0000,227.9000) -- (110.6000,230.0000) -- (108.6000,232.4000) -- (107.0000,235.0000) -- (105.8000,237.7000) -- (105.3000,240.6000) -- (105.1000,243.5000) -- (105.1000,246.4000) -- (105.1000,249.3000) -- (105.1000,252.2000) -- (105.1000,255.1000) -- (105.1000,257.9000) -- (105.1000,260.8000) -- (105.1000,263.7000) -- (105.1000,266.6000) -- (105.1000,269.5000) -- (105.1000,272.4000) -- (105.1000,275.3000) -- (105.1000,278.2000) -- (105.1000,281.1000) -- (105.1000,284.0000) -- (105.5000,286.8000) -- (106.3000,289.7000) -- (107.5000,292.4000) -- (109.1000,294.9000) -- (111.2000,297.3000) -- (113.6000,299.4000) -- (116.3000,301.2000) -- (119.2000,302.7000) -- (122.4000,303.8000) -- (125.8000,304.6000) -- (129.3000,304.9000) -- (132.8000,304.8000) -- (136.3000,304.3000) -- (139.6000,303.4000) -- (142.8000,302.1000) -- (145.7000,300.4000) -- (148.3000,298.5000) -- (150.6000,296.2000) -- (152.5000,293.7000) -- (153.8000,291.0000) -- (154.5000,288.2000) -- (154.9000,285.3000) -- (155.1000,282.5000) -- (155.3000,279.7000) -- (155.3000,276.8000) -- (155.4000,273.9000) -- (155.4000,271.0000) -- (155.4000,268.1000) -- (155.4000,265.2000) -- (155.4000,262.3000) -- (155.4000,259.4000) -- (155.4000,256.5000) -- (155.4000,253.6000) -- (155.4000,250.8000) -- (155.5000,247.9000) -- (155.9000,244.9000);



  \end{scope}
  \begin{scope}[cm={{1.00588,0.0,0.0,1.00588,(0.01175,-105.61832)}},draw=black,line join=round,line cap=round,line width=0.480pt]
    \path[draw] (44.5000,217.5000) -- (44.5000,319.5000) -- (179.5000,319.5000) -- (179.5000,217.5000) -- (44.5000,217.5000);



  \end{scope}
  \begin{scope}[cm={{1.00588,0.0,0.0,1.00588,(0.01175,-112.68806)}},draw=ca0a0a4,dash pattern=on 0.40pt off 0.80pt,line join=round,line cap=round,line width=0.400pt]
    \path[draw] (184.5000,267.5000) -- (246.5000,267.5000);



  \end{scope}
  \begin{scope}[cm={{1.00588,0.0,0.0,1.00588,(0.01175,-112.68806)}},draw=black,line join=round,line cap=round,line width=0.480pt]
    \path[draw] (184.5000,267.5000) -- (186.5000,267.5000);



    \path[draw] (246.5000,267.5000) -- (245.5000,267.5000);



  \end{scope}
  \begin{scope}[cm={{1.00588,0.0,0.0,1.00588,(0.01175,-112.68806)}},draw=ca0a0a4,dash pattern=on 0.40pt off 0.80pt,line join=round,line cap=round,line width=0.400pt]
    \path[draw] (184.5000,248.5000) -- (246.5000,248.5000);



  \end{scope}
  \begin{scope}[cm={{1.00588,0.0,0.0,1.00588,(0.01175,-112.68806)}},draw=black,line join=round,line cap=round,line width=0.480pt]
    \path[draw] (184.5000,248.5000) -- (186.5000,248.5000);



    \path[draw] (246.5000,248.5000) -- (245.5000,248.5000);



  \end{scope}
  \begin{scope}[cm={{1.00588,0.0,0.0,1.00588,(0.01175,-112.68806)}},draw=ca0a0a4,dash pattern=on 0.40pt off 0.80pt,line join=round,line cap=round,line width=0.400pt]
    \path[draw] (184.5000,229.5000) -- (246.5000,229.5000);



  \end{scope}
  \begin{scope}[cm={{1.00588,0.0,0.0,1.00588,(0.01175,-112.68806)}},draw=black,line join=round,line cap=round,line width=0.480pt]
    \path[draw] (184.5000,229.5000) -- (186.5000,229.5000);



    \path[draw] (246.5000,229.5000) -- (245.5000,229.5000);



  \end{scope}
  \begin{scope}[cm={{1.00588,0.0,0.0,1.00588,(0.01175,-112.68806)}},draw=ca0a0a4,dash pattern=on 0.40pt off 0.80pt,line join=round,line cap=round,line width=0.400pt]
    \path[draw] (184.5000,272.5000) -- (184.5000,224.5000);



  \end{scope}
  \begin{scope}[cm={{1.00588,0.0,0.0,1.00588,(0.01175,-112.68806)}},draw=black,line join=round,line cap=round,line width=0.480pt]
    \path[draw] (184.5000,272.5000) -- (184.5000,270.5000);



    \path[draw] (184.5000,224.5000) -- (184.5000,225.5000);



  \end{scope}
  \begin{scope}[cm={{1.00588,0.0,0.0,1.00588,(0.01175,-112.68806)}},draw=ca0a0a4,dash pattern=on 0.40pt off 0.80pt,line join=round,line cap=round,line width=0.400pt]
    \path[draw] (201.5000,272.5000) -- (201.5000,224.5000);



  \end{scope}
  \begin{scope}[cm={{1.00588,0.0,0.0,1.00588,(0.01175,-112.68806)}},draw=black,line join=round,line cap=round,line width=0.480pt]
    \path[draw] (201.5000,272.5000) -- (201.5000,270.5000);



    \path[draw] (201.5000,224.5000) -- (201.5000,225.5000);



  \end{scope}
  \begin{scope}[cm={{1.00588,0.0,0.0,1.00588,(0.01175,-112.68806)}},draw=ca0a0a4,dash pattern=on 0.40pt off 0.80pt,line join=round,line cap=round,line width=0.400pt]
    \path[draw] (218.5000,272.5000) -- (218.5000,224.5000);



  \end{scope}
  \begin{scope}[cm={{1.00588,0.0,0.0,1.00588,(0.01175,-112.68806)}},draw=black,line join=round,line cap=round,line width=0.480pt]
    \path[draw] (218.5000,272.5000) -- (218.5000,270.5000);



    \path[draw] (218.5000,224.5000) -- (218.5000,225.5000);



  \end{scope}
  \begin{scope}[cm={{1.00588,0.0,0.0,1.00588,(0.01175,-112.68806)}},draw=ca0a0a4,dash pattern=on 0.40pt off 0.80pt,line join=round,line cap=round,line width=0.400pt]
    \path[draw] (235.5000,272.5000) -- (235.5000,224.5000);



  \end{scope}
  \begin{scope}[cm={{1.00588,0.0,0.0,1.00588,(0.01175,-112.68806)}},draw=black,line join=round,line cap=round,line width=0.480pt]
    \path[draw] (235.5000,272.5000) -- (235.5000,270.5000);



    \path[draw] (235.5000,224.5000) -- (235.5000,225.5000);



  \end{scope}
  \begin{scope}[cm={{1.00588,0.0,0.0,1.00588,(0.01175,-112.68806)}},draw=black,line join=round,line cap=round,line width=0.480pt]
    \path[draw] (246.5000,267.5000) -- (245.5000,267.5000);



  \end{scope}
  \begin{scope}[cm={{1.00588,0.0,0.0,1.00588,(255.22933,156.90591)}},draw=black,line join=bevel,line cap=rect,line width=0.800pt]
    \path[fill=black] (0.0000,0.0000) node[above right] (text2120) {\scriptsize 2};



  \end{scope}
  \begin{scope}[cm={{1.00588,0.0,0.0,1.00588,(0.01175,-112.68806)}},draw=black,line join=round,line cap=round,line width=0.480pt]
    \path[draw] (246.5000,248.5000) -- (245.5000,248.5000);



  \end{scope}
  \begin{scope}[cm={{1.00588,0.0,0.0,1.00588,(255.19715,139.29391)}},draw=black,line join=bevel,line cap=rect,line width=0.800pt]
    \path[fill=black] (0.0000,0.0000) node[above right] (text2144) {\scriptsize 6};



  \end{scope}
  \begin{scope}[cm={{1.00588,0.0,0.0,1.00588,(0.01175,-112.68806)}},draw=black,line join=round,line cap=round,line width=0.480pt]
    \path[draw] (246.5000,229.5000) -- (245.5000,229.5000);



  \end{scope}
  \begin{scope}[cm={{1.00588,0.0,0.0,1.00588,(254.57752,121.68291)}},draw=black,line join=bevel,line cap=rect,line width=0.800pt]
    \path[fill=black] (0.0000,0.0000) node[above right] (text2168) {\scriptsize 10};



  \end{scope}
  \begin{scope}[cm={{1.00588,0.0,0.0,1.00588,(0.01175,-112.68806)}},draw=black,line join=round,line cap=round,line width=0.480pt]
    \path[draw] (184.5000,224.5000) -- (184.5000,272.5000) -- (246.5000,272.5000) -- (246.5000,224.5000) -- (184.5000,224.5000);



  \end{scope}
  \begin{scope}[cm={{0.0,-1.00588,1.00588,0.0,(268.62975,142.30291)}},draw=black,line join=bevel,line cap=rect,line width=0.800pt]
    \path[fill=black] (0.0000,0.0000) node[above right] (text2192) {\rotatebox{-90}{\scriptsize $c_{i,2}$}};



  \end{scope}
  \begin{scope}[cm={{1.00588,0.0,0.0,1.00588,(0.01175,-112.68806)}},draw=black,line join=round,line cap=round,line width=0.480pt]
    \path[draw] (184.8000,267.2000) -- (184.8000,267.2000) -- (184.9000,267.2000) -- (185.0000,267.2000) -- (185.0000,267.2000) -- (185.1000,267.2000) -- (185.1000,267.2000) -- (185.2000,267.2000) -- (185.3000,267.2000) -- (185.3000,267.2000) -- (185.4000,267.2000) -- (185.4000,267.2000) -- (185.5000,267.2000) -- (185.6000,267.2000) -- (185.6000,267.2000) -- (185.7000,267.2000) -- (185.8000,267.2000) -- (185.8000,267.2000) -- (185.9000,267.2000) -- (185.9000,267.2000) -- (186.0000,267.2000) -- (186.1000,267.2000) -- (186.1000,267.2000) -- (186.2000,267.2000) -- (186.2000,267.2000) -- (186.3000,267.2000) -- (186.4000,267.2000) -- (186.4000,267.2000) -- (186.5000,267.2000) -- (186.6000,267.2000) -- (186.6000,267.2000) -- (186.7000,267.2000) -- (186.7000,267.2000) -- (186.8000,267.2000) -- (186.9000,267.2000) -- (186.9000,267.2000) -- (187.0000,267.2000) -- (187.0000,267.2000) -- (187.1000,267.2000) -- (187.2000,267.2000) -- (187.2000,267.2000) -- (187.3000,267.2000) -- (187.4000,267.2000) -- (187.4000,267.2000) -- (187.5000,267.2000) -- (187.5000,267.2000) -- (187.6000,267.2000) -- (187.7000,267.2000) -- (187.7000,267.2000) -- (187.8000,267.2000) -- (187.8000,267.2000) -- (187.9000,267.2000) -- (188.0000,267.2000) -- (188.0000,267.2000) -- (188.1000,267.2000) -- (188.2000,267.2000) -- (188.2000,267.2000) -- (188.3000,267.2000) -- (188.3000,267.2000) -- (188.4000,267.2000) -- (188.5000,267.2000) -- (188.5000,267.2000) -- (188.6000,267.2000) -- (188.6000,267.2000) -- (188.7000,267.2000) -- (188.8000,267.2000) -- (188.8000,267.2000) -- (188.9000,267.2000) -- (189.0000,267.2000) -- (189.0000,267.2000) -- (189.1000,267.2000) -- (189.1000,267.2000) -- (189.2000,267.2000) -- (189.3000,267.2000) -- (189.3000,267.2000) -- (189.4000,267.2000) -- (189.4000,267.2000) -- (189.5000,267.2000) -- (189.6000,267.2000) -- (189.6000,267.2000) -- (189.7000,267.2000) -- (189.8000,267.2000) -- (189.8000,267.2000) -- (189.9000,267.2000) -- (189.9000,267.2000) -- (190.0000,267.2000) -- (190.1000,267.2000) -- (190.1000,267.2000) -- (190.2000,267.2000) -- (190.2000,267.2000) -- (190.3000,267.2000) -- (190.4000,267.2000) -- (190.4000,267.2000) -- (190.5000,267.2000) -- (190.5000,267.2000) -- (190.6000,267.2000) -- (190.7000,267.2000) -- (190.7000,267.2000) -- (190.8000,267.2000) -- (190.9000,267.2000) -- (190.9000,267.2000) -- (191.0000,267.2000) -- (191.0000,267.2000) -- (191.1000,267.2000) -- (191.2000,267.2000) -- (191.2000,267.2000) -- (191.3000,267.2000) -- (191.3000,267.2000) -- (191.4000,267.2000) -- (191.5000,267.2000) -- (191.5000,267.2000) -- (191.6000,267.2000) -- (191.7000,267.2000) -- (191.7000,267.2000) -- (191.8000,267.2000) -- (191.8000,267.2000) -- (191.9000,267.2000) -- (192.0000,267.2000) -- (192.0000,267.2000) -- (192.1000,267.2000) -- (192.1000,267.2000) -- (192.2000,267.2000) -- (192.3000,267.2000) -- (192.3000,267.2000) -- (192.4000,267.2000) -- (192.5000,267.2000) -- (192.5000,267.2000) -- (192.6000,267.3000) -- (192.6000,268.7000) -- (192.7000,244.7000) -- (192.8000,227.7000) -- (192.8000,229.2000) -- (192.9000,229.2000) -- (192.9000,229.2000) -- (193.0000,229.2000) -- (193.1000,229.2000) -- (193.1000,229.2000) -- (193.2000,229.2000) -- (193.3000,229.2000) -- (193.3000,229.2000) -- (193.4000,229.2000) -- (193.4000,229.2000) -- (193.5000,229.2000) -- (193.6000,229.2000) -- (193.6000,229.2000) -- (193.7000,229.2000) -- (193.7000,229.2000) -- (193.8000,229.2000) -- (193.9000,229.2000) -- (193.9000,229.2000) -- (194.0000,229.2000) -- (194.1000,229.2000) -- (194.1000,229.2000) -- (194.2000,229.2000) -- (194.2000,229.2000) -- (194.3000,229.2000) -- (194.4000,229.2000) -- (194.4000,229.2000) -- (194.5000,229.2000) -- (194.5000,229.2000) -- (194.6000,229.2000) -- (194.7000,229.2000) -- (194.7000,229.2000) -- (194.8000,229.2000) -- (194.9000,229.2000) -- (194.9000,229.2000) -- (195.0000,229.2000) -- (195.0000,229.2000) -- (195.1000,229.2000) -- (195.2000,229.2000) -- (195.2000,229.2000) -- (195.3000,229.2000) -- (195.3000,229.2000) -- (195.4000,229.2000) -- (195.5000,229.2000) -- (195.5000,229.2000) -- (195.6000,229.2000) -- (195.7000,229.2000) -- (195.7000,229.2000) -- (195.8000,229.2000) -- (195.8000,229.2000) -- (195.9000,229.2000) -- (196.0000,229.2000) -- (196.0000,229.2000) -- (196.1000,229.2000) -- (196.1000,229.2000) -- (196.2000,229.2000) -- (196.3000,229.2000) -- (196.3000,229.2000) -- (196.4000,229.2000) -- (196.5000,229.2000) -- (196.5000,229.2000) -- (196.6000,229.2000) -- (196.6000,229.2000) -- (196.7000,229.2000) -- (196.8000,229.2000) -- (196.8000,229.2000) -- (196.9000,229.2000) -- (196.9000,229.2000) -- (197.0000,229.2000) -- (197.1000,229.2000) -- (197.1000,229.2000) -- (197.2000,229.2000) -- (197.3000,229.2000) -- (197.3000,229.2000) -- (197.4000,229.2000) -- (197.4000,229.2000) -- (197.5000,229.2000) -- (197.6000,229.2000) -- (197.6000,229.2000) -- (197.7000,229.2000) -- (197.7000,229.2000) -- (197.8000,229.2000) -- (197.9000,229.2000) -- (197.9000,229.2000) -- (198.0000,229.2000) -- (198.1000,229.2000) -- (198.1000,229.2000) -- (198.2000,229.2000) -- (198.2000,229.2000) -- (198.3000,229.2000) -- (198.4000,229.2000) -- (198.4000,229.2000) -- (198.5000,229.2000) -- (198.5000,229.2000) -- (198.6000,229.2000) -- (198.7000,229.2000) -- (198.7000,229.2000) -- (198.8000,229.2000) -- (198.9000,229.2000) -- (198.9000,229.2000) -- (199.0000,229.2000) -- (199.0000,229.2000) -- (199.1000,229.2000) -- (199.2000,229.2000) -- (199.2000,229.2000) -- (199.3000,229.2000) -- (199.3000,229.2000) -- (199.4000,229.2000) -- (199.5000,229.2000) -- (199.5000,229.2000) -- (199.6000,229.2000) -- (199.7000,229.2000) -- (199.7000,229.2000) -- (199.8000,229.2000) -- (199.8000,229.2000) -- (199.9000,229.2000) -- (200.0000,229.2000) -- (200.0000,229.2000) -- (200.1000,229.2000) -- (200.1000,229.2000) -- (200.2000,229.2000) -- (200.3000,229.2000) -- (200.3000,229.2000) -- (200.4000,229.2000) -- (200.5000,229.2000) -- (200.5000,229.2000) -- (200.6000,229.2000) -- (200.6000,229.2000) -- (200.7000,229.2000) -- (200.8000,229.2000) -- (200.8000,229.2000) -- (200.9000,229.2000) -- (200.9000,229.2000) -- (201.0000,229.2000) -- (201.1000,229.2000) -- (201.1000,229.2000) -- (201.2000,229.2000) -- (201.3000,229.2000) -- (201.3000,229.2000) -- (201.4000,229.2000) -- (201.4000,229.2000) -- (201.5000,229.2000) -- (201.6000,229.2000) -- (201.6000,229.2000) -- (201.7000,229.2000) -- (201.7000,229.2000) -- (201.8000,229.2000) -- (201.9000,229.2000) -- (201.9000,229.2000) -- (202.0000,229.2000) -- (202.0000,229.2000) -- (202.1000,229.2000) -- (202.2000,229.2000) -- (202.2000,229.2000) -- (202.3000,229.2000) -- (202.4000,229.2000) -- (202.4000,229.2000) -- (202.5000,229.2000) -- (202.5000,229.2000) -- (202.6000,229.2000) -- (202.7000,229.2000) -- (202.7000,229.2000) -- (202.8000,229.2000) -- (202.8000,229.2000) -- (202.9000,229.2000) -- (203.0000,229.2000) -- (203.0000,229.2000) -- (203.1000,229.2000) -- (203.2000,229.2000) -- (203.2000,229.2000) -- (203.3000,229.2000) -- (203.3000,229.2000) -- (203.4000,229.2000) -- (203.5000,229.2000) -- (203.5000,229.2000) -- (203.6000,229.2000) -- (203.6000,229.2000) -- (203.7000,229.2000) -- (203.8000,229.2000) -- (203.8000,229.2000) -- (203.9000,229.2000) -- (204.0000,229.2000) -- (204.0000,229.2000) -- (204.1000,229.2000) -- (204.1000,229.2000) -- (204.2000,229.2000) -- (204.3000,229.2000) -- (204.3000,229.2000) -- (204.4000,229.2000) -- (204.4000,229.2000) -- (204.5000,229.2000) -- (204.6000,229.2000) -- (204.6000,229.2000) -- (204.7000,229.2000) -- (204.8000,229.2000) -- (204.8000,229.2000) -- (204.9000,229.2000) -- (204.9000,229.2000) -- (205.0000,229.2000) -- (205.1000,229.2000) -- (205.1000,229.2000) -- (205.2000,229.2000) -- (205.2000,229.2000) -- (205.3000,229.2000) -- (205.4000,229.2000) -- (205.4000,229.2000) -- (205.5000,229.2000) -- (205.6000,229.2000) -- (205.6000,229.2000) -- (205.7000,229.2000) -- (205.7000,229.2000) -- (205.8000,229.2000) -- (205.9000,229.2000) -- (205.9000,229.2000) -- (206.0000,229.2000) -- (206.0000,229.2000) -- (206.1000,229.2000) -- (206.2000,229.2000) -- (206.2000,229.2000) -- (206.3000,229.2000) -- (206.4000,229.2000) -- (206.4000,229.2000) -- (206.5000,229.2000) -- (206.5000,229.2000) -- (206.6000,229.2000) -- (206.7000,229.2000) -- (206.7000,229.2000) -- (206.8000,229.2000) -- (206.8000,229.2000) -- (206.9000,229.2000) -- (207.0000,229.2000) -- (207.0000,229.2000) -- (207.1000,229.2000) -- (207.2000,229.2000) -- (207.2000,229.2000) -- (207.3000,229.2000) -- (207.3000,229.2000) -- (207.4000,229.2000) -- (207.5000,229.2000) -- (207.5000,229.2000) -- (207.6000,229.2000) -- (207.6000,229.2000) -- (207.7000,229.2000) -- (207.8000,229.2000) -- (207.8000,229.2000) -- (207.9000,229.2000) -- (208.0000,229.2000) -- (208.0000,229.2000) -- (208.1000,229.2000) -- (208.1000,229.2000) -- (208.2000,229.2000) -- (208.3000,229.2000) -- (208.3000,229.2000) -- (208.4000,229.2000) -- (208.4000,229.2000) -- (208.5000,229.2000) -- (208.6000,229.2000) -- (208.6000,229.2000) -- (208.7000,229.2000) -- (208.8000,229.2000) -- (208.8000,229.2000) -- (208.9000,229.2000) -- (208.9000,229.2000) -- (209.0000,229.2000) -- (209.1000,229.2000) -- (209.1000,229.2000) -- (209.2000,229.2000) -- (209.2000,229.2000) -- (209.3000,229.2000) -- (209.4000,229.2000) -- (209.4000,229.2000) -- (209.5000,229.2000) -- (209.6000,229.2000) -- (209.6000,229.2000) -- (209.7000,229.2000) -- (209.7000,229.2000) -- (209.8000,229.2000) -- (209.9000,229.2000) -- (209.9000,229.2000) -- (210.0000,229.2000) -- (210.0000,229.2000) -- (210.1000,229.2000) -- (210.2000,229.2000) -- (210.2000,229.2000) -- (210.3000,229.2000) -- (210.4000,229.2000) -- (210.4000,229.2000) -- (210.5000,229.2000) -- (210.5000,229.2000) -- (210.6000,229.2000) -- (210.7000,229.2000) -- (210.7000,229.2000) -- (210.8000,229.2000) -- (210.8000,229.2000) -- (210.9000,229.2000) -- (211.0000,229.2000) -- (211.0000,229.2000) -- (211.1000,229.2000) -- (211.2000,229.2000) -- (211.2000,229.2000) -- (211.3000,229.2000) -- (211.3000,229.2000) -- (211.4000,229.2000) -- (211.5000,229.2000) -- (211.5000,229.2000) -- (211.6000,229.2000) -- (211.6000,229.2000) -- (211.7000,229.2000) -- (211.8000,229.2000) -- (211.8000,229.2000) -- (211.9000,229.2000) -- (212.0000,229.2000) -- (212.0000,229.2000) -- (212.1000,229.2000) -- (212.1000,229.2000) -- (212.2000,229.2000) -- (212.3000,229.2000) -- (212.3000,229.2000) -- (212.4000,229.2000) -- (212.4000,229.2000) -- (212.5000,229.2000) -- (212.6000,229.2000) -- (212.6000,229.2000) -- (212.7000,229.2000) -- (212.8000,229.2000) -- (212.8000,229.2000) -- (212.9000,229.2000) -- (212.9000,229.2000) -- (213.0000,229.2000) -- (213.1000,229.2000) -- (213.1000,229.2000) -- (213.2000,229.2000) -- (213.2000,229.2000) -- (213.3000,229.2000) -- (213.4000,229.2000) -- (213.4000,229.2000) -- (213.5000,229.2000) -- (213.6000,229.2000) -- (213.6000,229.2000) -- (213.7000,229.2000) -- (213.7000,229.2000) -- (213.8000,229.2000) -- (213.9000,229.2000) -- (213.9000,229.2000) -- (214.0000,229.2000) -- (214.0000,229.2000) -- (214.1000,229.2000) -- (214.2000,229.2000) -- (214.2000,229.2000) -- (214.3000,229.2000) -- (214.4000,229.2000) -- (214.4000,229.2000) -- (214.5000,229.2000) -- (214.5000,229.2000) -- (214.6000,229.2000) -- (214.7000,229.2000) -- (214.7000,229.2000) -- (214.8000,229.2000) -- (214.8000,229.2000) -- (214.9000,229.2000) -- (215.0000,229.2000) -- (215.0000,229.2000) -- (215.1000,229.2000) -- (215.1000,229.2000) -- (215.2000,229.2000) -- (215.3000,229.2000) -- (215.3000,229.2000) -- (215.4000,229.2000) -- (215.5000,229.2000) -- (215.5000,229.2000) -- (215.6000,229.2000) -- (215.6000,229.2000) -- (215.7000,229.2000) -- (215.8000,229.2000) -- (215.8000,229.2000) -- (215.9000,229.2000) -- (215.9000,228.9000) -- (216.0000,233.3000) -- (216.1000,239.1000) -- (216.1000,238.8000) -- (216.2000,238.7000) -- (216.3000,238.7000) -- (216.3000,238.7000) -- (216.4000,238.7000) -- (216.4000,238.7000) -- (216.5000,238.7000) -- (216.6000,238.7000) -- (216.6000,238.7000) -- (216.7000,238.7000) -- (216.7000,238.7000) -- (216.8000,238.7000) -- (216.9000,238.7000) -- (216.9000,238.7000) -- (217.0000,238.7000) -- (217.1000,238.7000) -- (217.1000,238.7000) -- (217.2000,238.7000) -- (217.2000,238.7000) -- (217.3000,238.7000) -- (217.4000,238.7000) -- (217.4000,238.7000) -- (217.5000,238.7000) -- (217.5000,238.7000) -- (217.6000,239.0000) -- (217.7000,236.4000) -- (217.7000,229.1000) -- (217.8000,229.2000) -- (217.9000,229.2000) -- (217.9000,229.2000) -- (218.0000,229.2000) -- (218.0000,229.2000) -- (218.1000,229.2000) -- (218.2000,229.2000) -- (218.2000,229.2000) -- (218.3000,229.2000) -- (218.3000,229.2000) -- (218.4000,229.2000) -- (218.5000,229.2000) -- (218.5000,229.2000) -- (218.6000,229.2000) -- (218.7000,229.2000) -- (218.7000,229.2000) -- (218.8000,229.2000) -- (218.8000,229.2000) -- (218.9000,229.2000) -- (219.0000,229.2000) -- (219.0000,229.2000) -- (219.1000,229.2000) -- (219.1000,229.2000) -- (219.2000,229.2000) -- (219.3000,229.2000) -- (219.3000,229.2000) -- (219.4000,229.2000) -- (219.5000,229.2000) -- (219.5000,229.2000) -- (219.6000,229.2000) -- (219.6000,229.2000) -- (219.7000,229.2000) -- (219.8000,229.2000) -- (219.8000,229.2000) -- (219.9000,229.2000) -- (219.9000,229.2000) -- (220.0000,229.2000) -- (220.1000,229.2000) -- (220.1000,228.9000) -- (220.2000,233.5000) -- (220.3000,239.1000) -- (220.3000,238.8000) -- (220.4000,238.7000) -- (220.4000,239.1000) -- (220.5000,247.0000) -- (220.6000,248.4000) -- (220.6000,248.2000) -- (220.7000,248.2000) -- (220.7000,248.2000) -- (220.8000,248.2000) -- (220.9000,248.2000) -- (220.9000,248.2000) -- (221.0000,248.2000) -- (221.1000,248.2000) -- (221.1000,248.2000) -- (221.2000,248.2000) -- (221.2000,248.2000) -- (221.3000,248.2000) -- (221.4000,248.2000) -- (221.4000,248.2000) -- (221.5000,248.2000) -- (221.5000,248.2000) -- (221.6000,248.2000) -- (221.7000,248.2000) -- (221.7000,248.2000) -- (221.8000,248.2000) -- (221.9000,248.2000) -- (221.9000,248.2000) -- (222.0000,248.2000) -- (222.0000,248.3000) -- (222.1000,248.5000) -- (222.2000,241.5000) -- (222.2000,238.4000) -- (222.3000,238.7000) -- (222.3000,239.0000) -- (222.4000,236.7000) -- (222.5000,229.2000) -- (222.5000,229.2000) -- (222.6000,229.2000) -- (222.7000,229.2000) -- (222.7000,229.2000) -- (222.8000,229.2000) -- (222.8000,229.2000) -- (222.9000,229.2000) -- (223.0000,229.2000) -- (223.0000,229.2000) -- (223.1000,229.2000) -- (223.1000,229.2000) -- (223.2000,229.2000) -- (223.3000,229.2000) -- (223.3000,229.2000) -- (223.4000,229.2000) -- (223.5000,229.2000) -- (223.5000,229.2000) -- (223.6000,229.2000) -- (223.6000,229.2000) -- (223.7000,229.2000) -- (223.8000,229.2000) -- (223.8000,229.2000) -- (223.9000,229.2000) -- (223.9000,229.2000) -- (224.0000,229.2000) -- (224.1000,229.2000) -- (224.1000,229.2000) -- (224.2000,229.2000) -- (224.3000,229.2000) -- (224.3000,229.2000) -- (224.4000,229.2000) -- (224.4000,229.8000) -- (224.5000,237.7000) -- (224.6000,238.8000) -- (224.6000,239.2000) -- (224.7000,247.1000) -- (224.7000,248.4000) -- (224.8000,248.2000) -- (224.9000,247.9000) -- (224.9000,252.1000) -- (225.0000,258.1000) -- (225.1000,257.8000) -- (225.1000,257.7000) -- (225.2000,257.7000) -- (225.2000,257.7000) -- (225.3000,257.7000) -- (225.4000,257.7000) -- (225.4000,257.7000) -- (225.5000,257.7000) -- (225.5000,257.7000) -- (225.6000,257.7000) -- (225.7000,257.7000) -- (225.7000,257.7000) -- (225.8000,257.7000) -- (225.9000,257.7000) -- (225.9000,257.7000) -- (226.0000,257.7000) -- (226.0000,257.7000) -- (226.1000,257.7000) -- (226.2000,257.7000) -- (226.2000,257.7000) -- (226.3000,257.7000) -- (226.3000,257.7000) -- (226.4000,257.7000) -- (226.5000,257.7000) -- (226.5000,258.0000) -- (226.6000,255.6000) -- (226.6000,248.2000) -- (226.7000,248.2000) -- (226.8000,248.3000) -- (226.8000,248.5000) -- (226.9000,241.8000) -- (227.0000,238.4000) -- (227.0000,239.1000) -- (227.1000,232.5000) -- (227.1000,228.9000) -- (227.2000,229.2000) -- (227.3000,229.2000) -- (227.3000,229.2000) -- (227.4000,229.2000) -- (227.4000,229.2000) -- (227.5000,229.2000) -- (227.6000,229.2000) -- (227.6000,229.2000) -- (227.7000,229.2000) -- (227.8000,229.2000) -- (227.8000,229.2000) -- (227.9000,229.2000) -- (227.9000,229.2000) -- (228.0000,229.2000) -- (228.1000,229.2000) -- (228.1000,229.2000) -- (228.2000,229.2000) -- (228.2000,229.2000) -- (228.3000,229.2000) -- (228.4000,229.2000) -- (228.4000,229.2000) -- (228.5000,229.2000) -- (228.6000,229.2000) -- (228.6000,229.2000) -- (228.7000,229.2000) -- (228.7000,229.2000) -- (228.8000,229.8000) -- (228.9000,237.3000) -- (228.9000,243.1000) -- (229.0000,248.6000) -- (229.0000,247.9000) -- (229.1000,252.3000) -- (229.2000,258.1000) -- (229.2000,257.8000) -- (229.3000,257.7000) -- (229.4000,258.0000) -- (229.4000,265.8000) -- (229.5000,267.4000) -- (229.5000,267.2000) -- (229.6000,267.2000) -- (229.7000,267.2000) -- (229.7000,267.2000) -- (229.8000,267.2000) -- (229.8000,267.2000) -- (229.9000,267.2000) -- (230.0000,267.2000) -- (230.0000,267.2000) -- (230.1000,267.2000) -- (230.2000,267.2000) -- (230.2000,267.2000) -- (230.3000,267.2000) -- (230.3000,267.2000) -- (230.4000,267.2000) -- (230.5000,267.2000) -- (230.5000,267.2000) -- (230.6000,267.2000) -- (230.6000,267.2000) -- (230.7000,267.2000) -- (230.8000,267.2000) -- (230.8000,267.2000) -- (230.9000,267.2000) -- (231.0000,267.3000) -- (231.0000,267.5000) -- (231.1000,260.7000) -- (231.1000,257.4000) -- (231.2000,257.7000) -- (231.3000,258.0000) -- (231.3000,255.9000) -- (231.4000,248.3000) -- (231.4000,248.4000) -- (231.5000,246.6000) -- (231.6000,238.9000) -- (231.6000,238.9000) -- (231.7000,237.2000) -- (231.8000,229.4000) -- (231.8000,229.2000) -- (231.9000,229.2000) -- (231.9000,229.2000) -- (232.0000,229.2000) -- (232.1000,229.2000) -- (232.1000,229.2000) -- (232.2000,229.2000) -- (232.2000,229.2000) -- (232.3000,229.2000) -- (232.4000,229.2000) -- (232.4000,229.2000) -- (232.5000,229.2000) -- (232.6000,229.2000) -- (232.6000,229.2000) -- (232.7000,229.2000) -- (232.7000,229.2000) -- (232.8000,229.2000) -- (232.9000,229.2000) -- (232.9000,229.2000) -- (233.0000,229.2000) -- (233.0000,229.2000) -- (233.1000,229.2000) -- (233.2000,229.8000) -- (233.2000,236.6000) -- (233.3000,251.6000) -- (233.4000,268.4000) -- (233.4000,267.3000) -- (233.5000,267.2000) -- (233.5000,267.2000) -- (233.6000,267.2000) -- (233.7000,267.2000) -- (233.7000,267.2000) -- (233.8000,267.2000) -- (233.8000,267.2000) -- (233.9000,267.2000) -- (234.0000,267.2000) -- (234.0000,267.2000) -- (234.1000,267.2000) -- (234.2000,267.2000) -- (234.2000,267.2000) -- (234.3000,267.2000) -- (234.3000,267.2000) -- (234.4000,267.2000) -- (234.5000,267.2000) -- (234.5000,267.2000) -- (234.6000,267.2000) -- (234.6000,267.2000) -- (234.7000,267.2000) -- (234.8000,267.2000) -- (234.8000,267.2000) -- (234.9000,267.2000) -- (235.0000,267.2000) -- (235.0000,267.2000) -- (235.1000,267.2000) -- (235.1000,267.2000) -- (235.2000,267.2000) -- (235.3000,267.2000) -- (235.3000,267.2000) -- (235.4000,267.2000) -- (235.4000,267.2000) -- (235.5000,267.2000) -- (235.6000,267.2000) -- (235.6000,267.2000) -- (235.7000,267.3000) -- (235.8000,267.6000) -- (235.8000,261.0000) -- (235.9000,257.4000) -- (235.9000,258.1000) -- (236.0000,251.7000) -- (236.1000,247.9000) -- (236.1000,248.6000) -- (236.2000,242.4000) -- (236.2000,238.4000) -- (236.3000,239.1000) -- (236.4000,233.1000) -- (236.4000,228.9000) -- (236.5000,229.2000) -- (236.6000,229.2000) -- (236.6000,229.2000) -- (236.7000,229.2000) -- (236.7000,229.2000) -- (236.8000,229.2000) -- (236.9000,229.2000) -- (236.9000,229.2000) -- (237.0000,229.2000) -- (237.0000,229.2000) -- (237.1000,229.1000) -- (237.2000,230.0000) -- (237.2000,238.0000) -- (237.3000,238.9000) -- (237.4000,238.7000) -- (237.4000,238.7000) -- (237.5000,238.7000) -- (237.5000,238.7000) -- (237.6000,237.6000) -- (237.7000,251.4000) -- (237.7000,268.4000) -- (237.8000,267.3000) -- (237.8000,267.2000) -- (237.9000,267.2000) -- (238.0000,267.2000) -- (238.0000,267.2000) -- (238.1000,267.2000) -- (238.1000,267.2000) -- (238.2000,267.2000) -- (238.3000,267.2000) -- (238.3000,267.2000) -- (238.4000,267.2000) -- (238.5000,267.2000) -- (238.5000,267.2000) -- (238.6000,267.2000) -- (238.6000,267.2000) -- (238.7000,267.2000) -- (238.8000,267.2000) -- (238.8000,267.2000) -- (238.9000,267.2000) -- (238.9000,267.2000) -- (239.0000,267.2000) -- (239.1000,267.2000) -- (239.1000,267.2000) -- (239.2000,267.2000) -- (239.3000,267.2000) -- (239.3000,267.2000) -- (239.4000,267.2000) -- (239.4000,267.2000) -- (239.5000,267.2000) -- (239.6000,267.2000) -- (239.6000,267.2000) -- (239.7000,267.2000) -- (239.7000,267.2000) -- (239.8000,267.2000) -- (239.9000,267.2000) -- (239.9000,267.2000) -- (240.0000,267.2000) -- (240.1000,267.2000) -- (240.1000,267.2000) -- (240.2000,267.2000) -- (240.2000,267.2000) -- (240.3000,267.2000) -- (240.4000,267.5000) -- (240.4000,265.7000) -- (240.5000,258.0000) -- (240.5000,257.9000) -- (240.6000,256.4000) -- (240.7000,248.5000) -- (240.7000,248.4000) -- (240.8000,247.0000) -- (240.9000,239.1000) -- (240.9000,238.7000) -- (241.0000,238.7000) -- (241.0000,238.7000) -- (241.1000,238.7000) -- (241.2000,238.7000) -- (241.2000,238.7000) -- (241.3000,238.7000) -- (241.3000,238.7000) -- (241.4000,238.7000) -- (241.5000,238.6000) -- (241.5000,239.5000) -- (241.6000,247.5000) -- (241.7000,248.4000) -- (241.7000,248.2000) -- (241.8000,248.2000) -- (241.8000,248.1000) -- (241.9000,249.3000) -- (242.0000,265.2000) -- (242.0000,267.6000) -- (242.1000,267.2000) -- (242.1000,267.2000) -- (242.2000,267.2000) -- (242.3000,267.2000) -- (242.3000,267.2000) -- (242.4000,267.2000) -- (242.5000,267.2000) -- (242.5000,267.2000) -- (242.6000,267.2000) -- (242.6000,267.2000) -- (242.7000,267.2000) -- (242.8000,267.2000) -- (242.8000,267.2000) -- (242.9000,267.2000) -- (242.9000,267.2000) -- (243.0000,267.2000) -- (243.1000,267.2000) -- (243.1000,267.2000) -- (243.2000,267.2000) -- (243.3000,267.2000) -- (243.3000,267.2000) -- (243.4000,267.2000) -- (243.4000,267.2000) -- (243.5000,267.2000) -- (243.6000,267.2000) -- (243.6000,267.2000) -- (243.7000,267.2000) -- (243.7000,267.2000) -- (243.8000,267.2000) -- (243.9000,267.2000) -- (243.9000,267.2000) -- (244.0000,267.2000) -- (244.1000,267.2000) -- (244.1000,267.2000) -- (244.2000,267.2000) -- (244.2000,267.2000) -- (244.3000,267.2000) -- (244.4000,267.2000) -- (244.4000,267.2000) -- (244.5000,267.2000) -- (244.5000,267.2000) -- (244.6000,267.2000) -- (244.7000,267.2000) -- (244.7000,267.2000) -- (244.8000,267.2000) -- (244.9000,267.2000) -- (244.9000,267.2000) -- (245.0000,267.3000) -- (245.0000,267.6000) -- (245.1000,261.6000) -- (245.2000,257.4000) -- (245.2000,258.1000) -- (245.3000,252.3000) -- (245.3000,247.9000) -- (245.4000,248.2000) -- (245.5000,248.2000) -- (245.5000,248.2000) -- (245.6000,248.2000) -- (245.7000,248.2000) -- (245.7000,248.2000) -- (245.8000,248.2000) -- (245.8000,248.2000) -- (245.9000,248.2000) -- (246.0000,247.9000) -- (246.0000,252.9000) -- (246.1000,258.1000) -- (246.1000,257.4000) -- (246.2000,262.2000) -- (246.3000,267.6000);



  \end{scope}
  \begin{scope}[cm={{1.00588,0.0,0.0,1.00588,(0.01175,-112.68806)}},draw=black,line join=round,line cap=round,line width=0.480pt]
    \path[draw] (184.5000,224.5000) -- (184.5000,272.5000) -- (246.5000,272.5000) -- (246.5000,224.5000) -- (184.5000,224.5000);



  \end{scope}
  \begin{scope}[cm={{1.00588,0.0,0.0,1.00588,(0.01175,-112.68806)}},draw=ca0a0a4,dash pattern=on 0.40pt off 0.80pt,line join=round,line cap=round,line width=0.400pt]
    \path[draw] (184.5000,315.5000) -- (246.5000,315.5000);



  \end{scope}
  \begin{scope}[cm={{1.00588,0.0,0.0,1.00588,(0.01175,-112.68806)}},draw=black,line join=round,line cap=round,line width=0.480pt]
    \path[draw] (184.5000,315.5000) -- (186.5000,315.5000);



    \path[draw] (246.5000,315.5000) -- (245.5000,315.5000);



  \end{scope}
  \begin{scope}[cm={{1.00588,0.0,0.0,1.00588,(0.01175,-112.68806)}},draw=ca0a0a4,dash pattern=on 0.40pt off 0.80pt,line join=round,line cap=round,line width=0.400pt]
    \path[draw] (184.5000,295.5000) -- (246.5000,295.5000);



  \end{scope}
  \begin{scope}[cm={{1.00588,0.0,0.0,1.00588,(0.01175,-112.68806)}},draw=black,line join=round,line cap=round,line width=0.480pt]
    \path[draw] (184.5000,295.5000) -- (186.5000,295.5000);



    \path[draw] (246.5000,295.5000) -- (245.5000,295.5000);



  \end{scope}
  \begin{scope}[cm={{1.00588,0.0,0.0,1.00588,(0.01175,-112.68806)}},draw=ca0a0a4,dash pattern=on 0.40pt off 0.80pt,line join=round,line cap=round,line width=0.400pt]
    \path[draw] (184.5000,276.5000) -- (246.5000,276.5000);



  \end{scope}
  \begin{scope}[cm={{1.00588,0.0,0.0,1.00588,(0.01175,-112.68806)}},draw=black,line join=round,line cap=round,line width=0.480pt]
    \path[draw] (184.5000,276.5000) -- (186.5000,276.5000);



    \path[draw] (246.5000,276.5000) -- (245.5000,276.5000);



  \end{scope}
  \begin{scope}[cm={{1.00588,0.0,0.0,1.00588,(0.01175,-112.68806)}},draw=ca0a0a4,dash pattern=on 0.40pt off 0.80pt,line join=round,line cap=round,line width=0.400pt]
    \path[draw] (184.5000,319.5000) -- (184.5000,272.5000);



  \end{scope}
  \begin{scope}[cm={{1.00588,0.0,0.0,1.00588,(0.01175,-112.68806)}},draw=black,line join=round,line cap=round,line width=0.480pt]
    \path[draw] (184.5000,319.5000) -- (184.5000,318.5000);



    \path[draw] (184.5000,272.5000) -- (184.5000,273.5000);



  \end{scope}
  \begin{scope}[cm={{1.00588,0.0,0.0,1.00588,(184.58275,219.78291)}},draw=black,line join=bevel,line cap=rect,line width=0.800pt]
    \path[fill=black] (0.0000,0.0000) node[above right] (text2316) {\scriptsize 0};



  \end{scope}
  \begin{scope}[cm={{1.00588,0.0,0.0,1.00588,(0.01175,-112.68806)}},draw=ca0a0a4,dash pattern=on 0.40pt off 0.80pt,line join=round,line cap=round,line width=0.400pt]
    \path[draw] (201.5000,319.5000) -- (201.5000,272.5000);



  \end{scope}
  \begin{scope}[cm={{1.00588,0.0,0.0,1.00588,(0.01175,-112.68806)}},draw=black,line join=round,line cap=round,line width=0.480pt]
    \path[draw] (201.5000,319.5000) -- (201.5000,318.5000);



    \path[draw] (201.5000,272.5000) -- (201.5000,273.5000);



  \end{scope}
  \begin{scope}[cm={{1.00588,0.0,0.0,1.00588,(200.68575,219.78291)}},draw=black,line join=bevel,line cap=rect,line width=0.800pt]
    \path[fill=black] (0.0000,0.0000) node[above right] (text2346) {\scriptsize 3};



  \end{scope}
  \begin{scope}[cm={{1.00588,0.0,0.0,1.00588,(0.01175,-112.68806)}},draw=ca0a0a4,dash pattern=on 0.40pt off 0.80pt,line join=round,line cap=round,line width=0.400pt]
    \path[draw] (218.5000,319.5000) -- (218.5000,272.5000);



  \end{scope}
  \begin{scope}[cm={{1.00588,0.0,0.0,1.00588,(0.01175,-112.68806)}},draw=black,line join=round,line cap=round,line width=0.480pt]
    \path[draw] (218.5000,319.5000) -- (218.5000,318.5000);



    \path[draw] (218.5000,272.5000) -- (218.5000,273.5000);



  \end{scope}
  \begin{scope}[cm={{1.00588,0.0,0.0,1.00588,(217.77675,219.78291)}},draw=black,line join=bevel,line cap=rect,line width=0.800pt]
    \path[fill=black] (0.0000,0.0000) node[above right] (text2376) {\scriptsize 6};



  \end{scope}
  \begin{scope}[cm={{1.00588,0.0,0.0,1.00588,(0.01175,-112.68806)}},draw=ca0a0a4,dash pattern=on 0.40pt off 0.80pt,line join=round,line cap=round,line width=0.400pt]
    \path[draw] (235.5000,319.5000) -- (235.5000,272.5000);



  \end{scope}
  \begin{scope}[cm={{1.00588,0.0,0.0,1.00588,(0.01175,-112.68806)}},draw=black,line join=round,line cap=round,line width=0.480pt]
    \path[draw] (235.5000,319.5000) -- (235.5000,318.5000);



    \path[draw] (235.5000,272.5000) -- (235.5000,273.5000);



  \end{scope}
  \begin{scope}[cm={{1.00588,0.0,0.0,1.00588,(234.87675,219.78291)}},draw=black,line join=bevel,line cap=rect,line width=0.800pt]
    \path[fill=black] (0.0000,0.0000) node[above right] (text2406) {\scriptsize 9};



  \end{scope}
  \begin{scope}[cm={{1.00588,0.0,0.0,1.00588,(0.01175,-112.68806)}},draw=black,line join=round,line cap=round,line width=0.480pt]
    \path[draw] (246.5000,315.5000) -- (245.5000,315.5000);



  \end{scope}
  \begin{scope}[cm={{1.00588,0.0,0.0,1.00588,(255.15691,207.69391)}},draw=black,line join=bevel,line cap=rect,line width=0.800pt]
    \path[fill=black] (0.0000,0.0000) node[above right] (text2430) {\scriptsize -1};



  \end{scope}
  \begin{scope}[cm={{1.00588,0.0,0.0,1.00588,(0.01175,-112.68806)}},draw=black,line join=round,line cap=round,line width=0.480pt]
    \path[draw] (246.5000,295.5000) -- (245.5000,295.5000);



  \end{scope}
  \begin{scope}[cm={{1.00588,0.0,0.0,1.00588,(255.15691,187.57691)}},draw=black,line join=bevel,line cap=rect,line width=0.800pt]
    \path[fill=black] (0.0000,0.0000) node[above right] (text2454) {\scriptsize -.5};



  \end{scope}
  \begin{scope}[cm={{1.00588,0.0,0.0,1.00588,(0.01175,-112.68806)}},draw=black,line join=round,line cap=round,line width=0.480pt]
    \path[draw] (246.5000,276.5000) -- (245.5000,276.5000);



  \end{scope}
  \begin{scope}[cm={{1.00588,0.0,0.0,1.00588,(255.27762,168.95891)}},draw=black,line join=bevel,line cap=rect,line width=0.800pt]
    \path[fill=black] (0.0000,0.0000) node[above right] (text2478) {\scriptsize 0};



  \end{scope}
  \begin{scope}[cm={{1.00588,0.0,0.0,1.00588,(0.01175,-112.68806)}},draw=black,line join=round,line cap=round,line width=0.480pt]
    \path[draw] (184.5000,272.5000) -- (184.5000,319.5000) -- (246.5000,319.5000) -- (246.5000,272.5000) -- (184.5000,272.5000);



  \end{scope}
  \begin{scope}[cm={{0.0,-1.00588,1.00588,0.0,(268.66194,190.58591)}},draw=black,line join=bevel,line cap=rect,line width=0.800pt]
    \path[fill=black] (0.0000,0.0000) node[above right] (text2502) {\rotatebox{-90}{\scriptsize k$c_{i,1}$}};



  \end{scope}
  \begin{scope}[cm={{1.00588,0.0,0.0,1.00588,(0.01175,-112.68806)}},draw=black,line join=round,line cap=round,line width=0.480pt]
    \path[draw] (184.8000,315.6000) -- (184.8000,315.6000) -- (184.9000,315.6000) -- (185.0000,315.6000) -- (185.0000,315.6000) -- (185.1000,315.6000) -- (185.1000,315.6000) -- (185.2000,315.6000) -- (185.3000,315.6000) -- (185.3000,315.6000) -- (185.4000,315.6000) -- (185.4000,315.6000) -- (185.5000,315.6000) -- (185.6000,315.6000) -- (185.6000,315.6000) -- (185.7000,315.6000) -- (185.8000,315.6000) -- (185.8000,315.6000) -- (185.9000,315.6000) -- (185.9000,315.6000) -- (186.0000,315.6000) -- (186.1000,315.6000) -- (186.1000,315.6000) -- (186.2000,315.6000) -- (186.2000,315.6000) -- (186.3000,315.6000) -- (186.4000,315.6000) -- (186.4000,315.6000) -- (186.5000,315.6000) -- (186.6000,315.6000) -- (186.6000,315.6000) -- (186.7000,315.6000) -- (186.7000,315.6000) -- (186.8000,315.6000) -- (186.9000,315.6000) -- (186.9000,315.6000) -- (187.0000,315.6000) -- (187.0000,315.6000) -- (187.1000,315.6000) -- (187.2000,315.6000) -- (187.2000,315.6000) -- (187.3000,315.6000) -- (187.4000,315.6000) -- (187.4000,315.6000) -- (187.5000,315.6000) -- (187.5000,315.6000) -- (187.6000,315.6000) -- (187.7000,315.6000) -- (187.7000,315.6000) -- (187.8000,315.6000) -- (187.8000,315.6000) -- (187.9000,315.6000) -- (188.0000,315.6000) -- (188.0000,315.6000) -- (188.1000,315.6000) -- (188.2000,315.6000) -- (188.2000,315.6000) -- (188.3000,315.6000) -- (188.3000,315.6000) -- (188.4000,315.6000) -- (188.5000,315.6000) -- (188.5000,315.6000) -- (188.6000,315.6000) -- (188.6000,315.6000) -- (188.7000,315.6000) -- (188.8000,315.6000) -- (188.8000,315.6000) -- (188.9000,315.6000) -- (189.0000,315.6000) -- (189.0000,315.6000) -- (189.1000,315.6000) -- (189.1000,315.6000) -- (189.2000,315.6000) -- (189.3000,315.6000) -- (189.3000,315.6000) -- (189.4000,315.6000) -- (189.4000,315.6000) -- (189.5000,315.6000) -- (189.6000,315.6000) -- (189.6000,315.6000) -- (189.7000,315.6000) -- (189.8000,315.6000) -- (189.8000,315.6000) -- (189.9000,315.6000) -- (189.9000,315.6000) -- (190.0000,315.6000) -- (190.1000,315.6000) -- (190.1000,315.6000) -- (190.2000,315.6000) -- (190.2000,315.6000) -- (190.3000,315.6000) -- (190.4000,315.6000) -- (190.4000,315.6000) -- (190.5000,315.6000) -- (190.5000,315.6000) -- (190.6000,315.6000) -- (190.7000,315.6000) -- (190.7000,315.6000) -- (190.8000,315.6000) -- (190.9000,315.6000) -- (190.9000,315.6000) -- (191.0000,315.6000) -- (191.0000,315.6000) -- (191.1000,315.6000) -- (191.2000,315.6000) -- (191.2000,315.6000) -- (191.3000,315.6000) -- (191.3000,315.6000) -- (191.4000,315.6000) -- (191.5000,315.6000) -- (191.5000,315.6000) -- (191.6000,315.6000) -- (191.7000,315.6000) -- (191.7000,315.6000) -- (191.8000,315.6000) -- (191.8000,315.6000) -- (191.9000,315.6000) -- (192.0000,315.6000) -- (192.0000,315.6000) -- (192.1000,315.6000) -- (192.1000,315.6000) -- (192.2000,315.6000) -- (192.3000,315.6000) -- (192.3000,315.6000) -- (192.4000,315.6000) -- (192.5000,315.6000) -- (192.5000,315.6000) -- (192.6000,315.6000) -- (192.6000,317.2000) -- (192.7000,292.4000) -- (192.8000,274.5000) -- (192.8000,276.1000) -- (192.9000,276.1000) -- (192.9000,276.1000) -- (193.0000,276.1000) -- (193.1000,276.1000) -- (193.1000,276.1000) -- (193.2000,276.1000) -- (193.3000,276.1000) -- (193.3000,276.1000) -- (193.4000,276.1000) -- (193.4000,276.1000) -- (193.5000,276.1000) -- (193.6000,276.1000) -- (193.6000,276.1000) -- (193.7000,276.1000) -- (193.7000,276.1000) -- (193.8000,276.1000) -- (193.9000,276.1000) -- (193.9000,276.1000) -- (194.0000,276.1000) -- (194.1000,276.1000) -- (194.1000,276.1000) -- (194.2000,276.1000) -- (194.2000,276.1000) -- (194.3000,276.1000) -- (194.4000,276.1000) -- (194.4000,276.1000) -- (194.5000,276.1000) -- (194.5000,276.1000) -- (194.6000,276.1000) -- (194.7000,276.1000) -- (194.7000,276.1000) -- (194.8000,276.1000) -- (194.9000,276.1000) -- (194.9000,276.1000) -- (195.0000,276.1000) -- (195.0000,276.1000) -- (195.1000,276.1000) -- (195.2000,276.1000) -- (195.2000,276.1000) -- (195.3000,276.1000) -- (195.3000,276.1000) -- (195.4000,276.1000) -- (195.5000,276.1000) -- (195.5000,276.1000) -- (195.6000,276.1000) -- (195.7000,276.1000) -- (195.7000,276.1000) -- (195.8000,276.1000) -- (195.8000,276.1000) -- (195.9000,276.1000) -- (196.0000,276.1000) -- (196.0000,276.1000) -- (196.1000,276.1000) -- (196.1000,276.1000) -- (196.2000,276.1000) -- (196.3000,276.1000) -- (196.3000,276.1000) -- (196.4000,276.1000) -- (196.5000,276.1000) -- (196.5000,276.1000) -- (196.6000,276.1000) -- (196.6000,276.1000) -- (196.7000,276.1000) -- (196.8000,276.1000) -- (196.8000,276.1000) -- (196.9000,276.1000) -- (196.9000,276.1000) -- (197.0000,276.1000) -- (197.1000,276.1000) -- (197.1000,276.1000) -- (197.2000,276.1000) -- (197.3000,276.1000) -- (197.3000,276.1000) -- (197.4000,276.1000) -- (197.4000,276.1000) -- (197.5000,276.1000) -- (197.6000,276.1000) -- (197.6000,276.1000) -- (197.7000,276.1000) -- (197.7000,276.1000) -- (197.8000,276.1000) -- (197.9000,276.1000) -- (197.9000,276.1000) -- (198.0000,276.1000) -- (198.1000,276.1000) -- (198.1000,276.1000) -- (198.2000,276.1000) -- (198.2000,276.1000) -- (198.3000,276.1000) -- (198.4000,276.1000) -- (198.4000,276.1000) -- (198.5000,276.1000) -- (198.5000,276.1000) -- (198.6000,276.1000) -- (198.7000,276.1000) -- (198.7000,276.1000) -- (198.8000,276.1000) -- (198.9000,276.1000) -- (198.9000,276.1000) -- (199.0000,276.1000) -- (199.0000,276.1000) -- (199.1000,276.1000) -- (199.2000,276.1000) -- (199.2000,276.1000) -- (199.3000,276.1000) -- (199.3000,276.1000) -- (199.4000,276.1000) -- (199.5000,276.1000) -- (199.5000,276.1000) -- (199.6000,276.1000) -- (199.7000,276.1000) -- (199.7000,276.1000) -- (199.8000,276.1000) -- (199.8000,276.1000) -- (199.9000,276.1000) -- (200.0000,276.1000) -- (200.0000,276.1000) -- (200.1000,276.1000) -- (200.1000,276.1000) -- (200.2000,276.1000) -- (200.3000,276.1000) -- (200.3000,276.1000) -- (200.4000,276.1000) -- (200.5000,276.1000) -- (200.5000,276.1000) -- (200.6000,276.1000) -- (200.6000,276.1000) -- (200.7000,276.1000) -- (200.8000,276.1000) -- (200.8000,276.1000) -- (200.9000,276.1000) -- (200.9000,276.1000) -- (201.0000,276.1000) -- (201.1000,276.1000) -- (201.1000,276.1000) -- (201.2000,276.1000) -- (201.3000,276.1000) -- (201.3000,276.1000) -- (201.4000,276.1000) -- (201.4000,276.1000) -- (201.5000,276.1000) -- (201.6000,276.1000) -- (201.6000,276.1000) -- (201.7000,276.1000) -- (201.7000,276.1000) -- (201.8000,276.1000) -- (201.9000,276.1000) -- (201.9000,276.1000) -- (202.0000,276.1000) -- (202.0000,276.1000) -- (202.1000,276.1000) -- (202.2000,276.1000) -- (202.2000,276.1000) -- (202.3000,276.1000) -- (202.4000,276.1000) -- (202.4000,276.1000) -- (202.5000,276.1000) -- (202.5000,276.1000) -- (202.6000,276.1000) -- (202.7000,276.1000) -- (202.7000,276.1000) -- (202.8000,276.1000) -- (202.8000,276.1000) -- (202.9000,276.1000) -- (203.0000,276.1000) -- (203.0000,276.1000) -- (203.1000,276.1000) -- (203.2000,276.1000) -- (203.2000,276.1000) -- (203.3000,276.1000) -- (203.3000,276.1000) -- (203.4000,276.1000) -- (203.5000,276.1000) -- (203.5000,276.1000) -- (203.6000,276.1000) -- (203.6000,276.1000) -- (203.7000,276.1000) -- (203.8000,276.1000) -- (203.8000,276.1000) -- (203.9000,276.1000) -- (204.0000,276.1000) -- (204.0000,276.1000) -- (204.1000,276.1000) -- (204.1000,276.1000) -- (204.2000,276.1000) -- (204.3000,276.1000) -- (204.3000,276.1000) -- (204.4000,276.1000) -- (204.4000,276.1000) -- (204.5000,276.1000) -- (204.6000,276.1000) -- (204.6000,276.1000) -- (204.7000,276.1000) -- (204.8000,276.1000) -- (204.8000,276.1000) -- (204.9000,276.1000) -- (204.9000,276.1000) -- (205.0000,276.1000) -- (205.1000,276.1000) -- (205.1000,276.1000) -- (205.2000,276.1000) -- (205.2000,276.1000) -- (205.3000,276.1000) -- (205.4000,276.1000) -- (205.4000,276.1000) -- (205.5000,276.1000) -- (205.6000,276.1000) -- (205.6000,276.1000) -- (205.7000,276.1000) -- (205.7000,276.1000) -- (205.8000,276.1000) -- (205.9000,276.1000) -- (205.9000,276.1000) -- (206.0000,276.1000) -- (206.0000,276.1000) -- (206.1000,276.1000) -- (206.2000,276.1000) -- (206.2000,276.1000) -- (206.3000,276.1000) -- (206.4000,276.1000) -- (206.4000,276.1000) -- (206.5000,276.1000) -- (206.5000,276.1000) -- (206.6000,276.1000) -- (206.7000,276.1000) -- (206.7000,276.1000) -- (206.8000,276.1000) -- (206.8000,276.1000) -- (206.9000,276.1000) -- (207.0000,276.1000) -- (207.0000,276.1000) -- (207.1000,276.1000) -- (207.2000,276.1000) -- (207.2000,276.1000) -- (207.3000,276.1000) -- (207.3000,276.1000) -- (207.4000,276.1000) -- (207.5000,276.1000) -- (207.5000,276.1000) -- (207.6000,276.1000) -- (207.6000,276.1000) -- (207.7000,276.1000) -- (207.8000,276.1000) -- (207.8000,276.1000) -- (207.9000,276.1000) -- (208.0000,276.1000) -- (208.0000,276.1000) -- (208.1000,276.1000) -- (208.1000,276.1000) -- (208.2000,276.1000) -- (208.3000,276.1000) -- (208.3000,276.1000) -- (208.4000,276.1000) -- (208.4000,276.1000) -- (208.5000,276.1000) -- (208.6000,276.1000) -- (208.6000,276.1000) -- (208.7000,276.1000) -- (208.8000,276.1000) -- (208.8000,276.1000) -- (208.9000,276.1000) -- (208.9000,276.1000) -- (209.0000,276.1000) -- (209.1000,276.1000) -- (209.1000,276.1000) -- (209.2000,276.1000) -- (209.2000,276.1000) -- (209.3000,276.1000) -- (209.4000,276.1000) -- (209.4000,276.1000) -- (209.5000,276.1000) -- (209.6000,276.1000) -- (209.6000,276.1000) -- (209.7000,276.1000) -- (209.7000,276.1000) -- (209.8000,276.1000) -- (209.9000,276.1000) -- (209.9000,276.1000) -- (210.0000,276.1000) -- (210.0000,276.1000) -- (210.1000,276.1000) -- (210.2000,276.1000) -- (210.2000,276.1000) -- (210.3000,276.1000) -- (210.4000,276.1000) -- (210.4000,276.1000) -- (210.5000,276.1000) -- (210.5000,276.1000) -- (210.6000,276.1000) -- (210.7000,276.1000) -- (210.7000,276.1000) -- (210.8000,276.1000) -- (210.8000,276.1000) -- (210.9000,276.1000) -- (211.0000,276.1000) -- (211.0000,276.1000) -- (211.1000,276.1000) -- (211.2000,276.1000) -- (211.2000,276.1000) -- (211.3000,276.1000) -- (211.3000,276.1000) -- (211.4000,276.1000) -- (211.5000,276.1000) -- (211.5000,276.1000) -- (211.6000,276.1000) -- (211.6000,276.1000) -- (211.7000,276.1000) -- (211.8000,276.1000) -- (211.8000,276.1000) -- (211.9000,276.1000) -- (212.0000,276.1000) -- (212.0000,276.1000) -- (212.1000,276.1000) -- (212.1000,276.1000) -- (212.2000,276.1000) -- (212.3000,276.1000) -- (212.3000,276.1000) -- (212.4000,276.1000) -- (212.4000,276.1000) -- (212.5000,276.1000) -- (212.6000,276.1000) -- (212.6000,276.1000) -- (212.7000,276.1000) -- (212.8000,276.1000) -- (212.8000,276.1000) -- (212.9000,276.1000) -- (212.9000,276.1000) -- (213.0000,276.1000) -- (213.1000,276.1000) -- (213.1000,276.1000) -- (213.2000,276.1000) -- (213.2000,276.1000) -- (213.3000,276.1000) -- (213.4000,276.1000) -- (213.4000,276.1000) -- (213.5000,276.1000) -- (213.6000,276.1000) -- (213.6000,276.1000) -- (213.7000,276.1000) -- (213.7000,276.1000) -- (213.8000,276.1000) -- (213.9000,276.1000) -- (213.9000,276.1000) -- (214.0000,276.1000) -- (214.0000,276.1000) -- (214.1000,276.1000) -- (214.2000,276.1000) -- (214.2000,276.1000) -- (214.3000,276.1000) -- (214.4000,276.1000) -- (214.4000,276.1000) -- (214.5000,276.1000) -- (214.5000,276.1000) -- (214.6000,276.1000) -- (214.7000,276.1000) -- (214.7000,276.1000) -- (214.8000,276.1000) -- (214.8000,276.1000) -- (214.9000,276.1000) -- (215.0000,276.1000) -- (215.0000,276.1000) -- (215.1000,276.1000) -- (215.1000,276.1000) -- (215.2000,276.1000) -- (215.3000,276.1000) -- (215.3000,276.1000) -- (215.4000,276.1000) -- (215.5000,276.1000) -- (215.5000,276.1000) -- (215.6000,276.1000) -- (215.6000,276.1000) -- (215.7000,276.1000) -- (215.8000,276.1000) -- (215.8000,276.1000) -- (215.9000,276.1000) -- (215.9000,276.1000) -- (216.0000,276.1000) -- (216.1000,276.1000) -- (216.1000,276.1000) -- (216.2000,276.1000) -- (216.3000,276.1000) -- (216.3000,276.1000) -- (216.4000,276.1000) -- (216.4000,276.1000) -- (216.5000,276.1000) -- (216.6000,276.1000) -- (216.6000,276.1000) -- (216.7000,276.1000) -- (216.7000,276.1000) -- (216.8000,276.1000) -- (216.9000,276.1000) -- (216.9000,276.1000) -- (217.0000,276.1000) -- (217.1000,276.1000) -- (217.1000,276.1000) -- (217.2000,276.1000) -- (217.2000,276.1000) -- (217.3000,276.1000) -- (217.4000,276.1000) -- (217.4000,276.1000) -- (217.5000,276.1000) -- (217.5000,276.1000) -- (217.6000,276.1000) -- (217.7000,276.1000) -- (217.7000,276.1000) -- (217.8000,276.1000) -- (217.9000,276.1000) -- (217.9000,276.1000) -- (218.0000,276.1000) -- (218.0000,276.1000) -- (218.1000,276.1000) -- (218.2000,276.1000) -- (218.2000,276.1000) -- (218.3000,276.1000) -- (218.3000,276.1000) -- (218.4000,276.1000) -- (218.5000,276.1000) -- (218.5000,276.1000) -- (218.6000,276.1000) -- (218.7000,276.1000) -- (218.7000,276.1000) -- (218.8000,276.1000) -- (218.8000,276.1000) -- (218.9000,276.1000) -- (219.0000,276.1000) -- (219.0000,276.1000) -- (219.1000,276.1000) -- (219.1000,276.1000) -- (219.2000,276.1000) -- (219.3000,276.1000) -- (219.3000,276.1000) -- (219.4000,276.1000) -- (219.5000,276.1000) -- (219.5000,276.1000) -- (219.6000,276.1000) -- (219.6000,276.1000) -- (219.7000,276.1000) -- (219.8000,276.1000) -- (219.8000,276.1000) -- (219.9000,276.1000) -- (219.9000,276.1000) -- (220.0000,276.1000) -- (220.1000,276.1000) -- (220.1000,276.1000) -- (220.2000,276.1000) -- (220.3000,276.1000) -- (220.3000,276.1000) -- (220.4000,276.1000) -- (220.4000,276.1000) -- (220.5000,276.1000) -- (220.6000,276.1000) -- (220.6000,276.1000) -- (220.7000,276.1000) -- (220.7000,276.1000) -- (220.8000,276.1000) -- (220.9000,276.1000) -- (220.9000,276.1000) -- (221.0000,276.1000) -- (221.1000,276.1000) -- (221.1000,276.1000) -- (221.2000,276.1000) -- (221.2000,276.1000) -- (221.3000,276.1000) -- (221.4000,276.1000) -- (221.4000,276.1000) -- (221.5000,276.1000) -- (221.5000,276.1000) -- (221.6000,276.1000) -- (221.7000,276.1000) -- (221.7000,276.1000) -- (221.8000,276.1000) -- (221.9000,276.1000) -- (221.9000,276.1000) -- (222.0000,276.1000) -- (222.0000,276.1000) -- (222.1000,276.1000) -- (222.2000,276.1000) -- (222.2000,276.1000) -- (222.3000,276.1000) -- (222.3000,276.1000) -- (222.4000,276.1000) -- (222.5000,276.1000) -- (222.5000,276.1000) -- (222.6000,276.1000) -- (222.7000,276.1000) -- (222.7000,276.1000) -- (222.8000,276.1000) -- (222.8000,276.1000) -- (222.9000,276.1000) -- (223.0000,276.1000) -- (223.0000,276.1000) -- (223.1000,276.1000) -- (223.1000,276.1000) -- (223.2000,276.1000) -- (223.3000,276.1000) -- (223.3000,276.1000) -- (223.4000,276.1000) -- (223.5000,276.1000) -- (223.5000,276.1000) -- (223.6000,276.1000) -- (223.6000,276.1000) -- (223.7000,276.1000) -- (223.8000,276.1000) -- (223.8000,276.1000) -- (223.9000,276.1000) -- (223.9000,276.1000) -- (224.0000,276.1000) -- (224.1000,276.1000) -- (224.1000,276.1000) -- (224.2000,276.1000) -- (224.3000,276.1000) -- (224.3000,276.1000) -- (224.4000,276.1000) -- (224.4000,276.1000) -- (224.5000,276.1000) -- (224.6000,276.1000) -- (224.6000,276.1000) -- (224.7000,276.1000) -- (224.7000,276.1000) -- (224.8000,276.1000) -- (224.9000,276.1000) -- (224.9000,276.1000) -- (225.0000,276.1000) -- (225.1000,276.1000) -- (225.1000,276.1000) -- (225.2000,276.1000) -- (225.2000,276.1000) -- (225.3000,276.1000) -- (225.4000,276.1000) -- (225.4000,276.1000) -- (225.5000,276.1000) -- (225.5000,276.1000) -- (225.6000,276.1000) -- (225.7000,276.1000) -- (225.7000,276.1000) -- (225.8000,276.1000) -- (225.9000,276.1000) -- (225.9000,276.1000) -- (226.0000,276.1000) -- (226.0000,276.1000) -- (226.1000,276.1000) -- (226.2000,276.1000) -- (226.2000,276.1000) -- (226.3000,276.1000) -- (226.3000,276.1000) -- (226.4000,276.1000) -- (226.5000,276.1000) -- (226.5000,276.1000) -- (226.6000,276.1000) -- (226.6000,276.1000) -- (226.7000,276.1000) -- (226.8000,276.1000) -- (226.8000,276.1000) -- (226.9000,276.1000) -- (227.0000,276.1000) -- (227.0000,276.1000) -- (227.1000,276.1000) -- (227.1000,276.1000) -- (227.2000,276.1000) -- (227.3000,276.1000) -- (227.3000,276.1000) -- (227.4000,276.1000) -- (227.4000,276.1000) -- (227.5000,276.1000) -- (227.6000,276.1000) -- (227.6000,276.1000) -- (227.7000,276.1000) -- (227.8000,276.1000) -- (227.8000,276.1000) -- (227.9000,276.1000) -- (227.9000,276.1000) -- (228.0000,276.1000) -- (228.1000,276.1000) -- (228.1000,276.1000) -- (228.2000,276.1000) -- (228.2000,276.1000) -- (228.3000,276.1000) -- (228.4000,276.1000) -- (228.4000,276.1000) -- (228.5000,276.1000) -- (228.6000,276.1000) -- (228.6000,276.1000) -- (228.7000,276.1000) -- (228.7000,276.1000) -- (228.8000,276.1000) -- (228.9000,276.1000) -- (228.9000,276.1000) -- (229.0000,276.1000) -- (229.0000,276.1000) -- (229.1000,276.1000) -- (229.2000,276.1000) -- (229.2000,276.1000) -- (229.3000,276.1000) -- (229.4000,276.1000) -- (229.4000,276.1000) -- (229.5000,276.1000) -- (229.5000,276.1000) -- (229.6000,276.1000) -- (229.7000,276.1000) -- (229.7000,276.1000) -- (229.8000,276.1000) -- (229.8000,276.1000) -- (229.9000,276.1000) -- (230.0000,276.1000) -- (230.0000,276.1000) -- (230.1000,276.1000) -- (230.2000,276.1000) -- (230.2000,276.1000) -- (230.3000,276.1000) -- (230.3000,276.1000) -- (230.4000,276.1000) -- (230.5000,276.1000) -- (230.5000,276.1000) -- (230.6000,276.1000) -- (230.6000,276.1000) -- (230.7000,276.1000) -- (230.8000,276.1000) -- (230.8000,276.1000) -- (230.9000,276.1000) -- (231.0000,276.1000) -- (231.0000,276.1000) -- (231.1000,276.1000) -- (231.1000,276.1000) -- (231.2000,276.1000) -- (231.3000,276.1000) -- (231.3000,276.1000) -- (231.4000,276.1000) -- (231.4000,276.1000) -- (231.5000,276.1000) -- (231.6000,276.1000) -- (231.6000,276.1000) -- (231.7000,276.1000) -- (231.8000,276.1000) -- (231.8000,276.1000) -- (231.9000,276.1000) -- (231.9000,276.1000) -- (232.0000,276.1000) -- (232.1000,276.1000) -- (232.1000,276.1000) -- (232.2000,276.1000) -- (232.2000,276.1000) -- (232.3000,276.1000) -- (232.4000,276.1000) -- (232.4000,276.1000) -- (232.5000,276.1000) -- (232.6000,276.1000) -- (232.6000,276.1000) -- (232.7000,276.1000) -- (232.7000,276.1000) -- (232.8000,276.1000) -- (232.9000,276.1000) -- (232.9000,276.1000) -- (233.0000,276.1000) -- (233.0000,276.1000) -- (233.1000,276.1000) -- (233.2000,276.1000) -- (233.2000,276.1000) -- (233.3000,276.1000) -- (233.4000,276.1000) -- (233.4000,276.1000) -- (233.5000,276.1000) -- (233.5000,276.1000) -- (233.6000,276.1000) -- (233.7000,276.1000) -- (233.7000,276.1000) -- (233.8000,276.1000) -- (233.8000,276.1000) -- (233.9000,276.1000) -- (234.0000,276.1000) -- (234.0000,276.1000) -- (234.1000,276.1000) -- (234.2000,276.1000) -- (234.2000,276.1000) -- (234.3000,276.1000) -- (234.3000,276.1000) -- (234.4000,276.1000) -- (234.5000,276.1000) -- (234.5000,276.1000) -- (234.6000,276.1000) -- (234.6000,276.1000) -- (234.7000,276.1000) -- (234.8000,276.1000) -- (234.8000,276.1000) -- (234.9000,276.1000) -- (235.0000,276.1000) -- (235.0000,276.1000) -- (235.1000,276.1000) -- (235.1000,276.1000) -- (235.2000,276.1000) -- (235.3000,276.1000) -- (235.3000,276.1000) -- (235.4000,276.1000) -- (235.4000,276.1000) -- (235.5000,276.1000) -- (235.6000,276.1000) -- (235.6000,276.1000) -- (235.7000,276.1000) -- (235.8000,276.1000) -- (235.8000,276.1000) -- (235.9000,276.1000) -- (235.9000,276.1000) -- (236.0000,276.1000) -- (236.1000,276.1000) -- (236.1000,276.1000) -- (236.2000,276.1000) -- (236.2000,276.1000) -- (236.3000,276.1000) -- (236.4000,276.1000) -- (236.4000,276.1000) -- (236.5000,276.1000) -- (236.6000,276.1000) -- (236.6000,276.1000) -- (236.7000,276.1000) -- (236.7000,276.1000) -- (236.8000,276.1000) -- (236.9000,276.1000) -- (236.9000,276.1000) -- (237.0000,276.1000) -- (237.0000,276.1000) -- (237.1000,276.1000) -- (237.2000,276.1000) -- (237.2000,276.1000) -- (237.3000,276.1000) -- (237.4000,276.1000) -- (237.4000,276.1000) -- (237.5000,276.1000) -- (237.5000,276.1000) -- (237.6000,276.1000) -- (237.7000,276.1000) -- (237.7000,276.1000) -- (237.8000,276.1000) -- (237.8000,276.1000) -- (237.9000,276.1000) -- (238.0000,276.1000) -- (238.0000,276.1000) -- (238.1000,276.1000) -- (238.1000,276.1000) -- (238.2000,276.1000) -- (238.3000,276.1000) -- (238.3000,276.1000) -- (238.4000,276.1000) -- (238.5000,276.1000) -- (238.5000,276.1000) -- (238.6000,276.1000) -- (238.6000,276.1000) -- (238.7000,276.1000) -- (238.8000,276.1000) -- (238.8000,276.1000) -- (238.9000,276.1000) -- (238.9000,276.1000) -- (239.0000,276.1000) -- (239.1000,276.1000) -- (239.1000,276.1000) -- (239.2000,276.1000) -- (239.3000,276.1000) -- (239.3000,276.1000) -- (239.4000,276.1000) -- (239.4000,276.1000) -- (239.5000,276.1000) -- (239.6000,276.1000) -- (239.6000,276.1000) -- (239.7000,276.1000) -- (239.7000,276.1000) -- (239.8000,276.1000) -- (239.9000,276.1000) -- (239.9000,276.1000) -- (240.0000,276.1000) -- (240.1000,276.1000) -- (240.1000,276.1000) -- (240.2000,276.1000) -- (240.2000,276.1000) -- (240.3000,276.1000) -- (240.4000,276.1000) -- (240.4000,276.1000) -- (240.5000,276.1000) -- (240.5000,276.1000) -- (240.6000,276.1000) -- (240.7000,276.1000) -- (240.7000,276.1000) -- (240.8000,276.1000) -- (240.9000,276.1000) -- (240.9000,276.1000) -- (241.0000,276.1000) -- (241.0000,276.1000) -- (241.1000,276.1000) -- (241.2000,276.1000) -- (241.2000,276.1000) -- (241.3000,276.1000) -- (241.3000,276.1000) -- (241.4000,276.1000) -- (241.5000,276.1000) -- (241.5000,276.1000) -- (241.6000,276.1000) -- (241.7000,276.1000) -- (241.7000,276.1000) -- (241.8000,276.1000) -- (241.8000,276.1000) -- (241.9000,276.1000) -- (242.0000,276.1000) -- (242.0000,276.1000) -- (242.1000,276.1000) -- (242.1000,276.1000) -- (242.2000,276.1000) -- (242.3000,276.1000) -- (242.3000,276.1000) -- (242.4000,276.1000) -- (242.5000,276.1000) -- (242.5000,276.1000) -- (242.6000,276.1000) -- (242.6000,276.1000) -- (242.7000,276.1000) -- (242.8000,276.1000) -- (242.8000,276.1000) -- (242.9000,276.1000) -- (242.9000,276.1000) -- (243.0000,276.1000) -- (243.1000,276.1000) -- (243.1000,276.1000) -- (243.2000,276.1000) -- (243.3000,276.1000) -- (243.3000,276.1000) -- (243.4000,276.1000) -- (243.4000,276.1000) -- (243.5000,276.1000) -- (243.6000,276.1000) -- (243.6000,276.1000) -- (243.7000,276.1000) -- (243.7000,276.1000) -- (243.8000,276.1000) -- (243.9000,276.1000) -- (243.9000,276.1000) -- (244.0000,276.1000) -- (244.1000,276.1000) -- (244.1000,276.1000) -- (244.2000,276.1000) -- (244.2000,276.1000) -- (244.3000,276.1000) -- (244.4000,276.1000) -- (244.4000,276.1000) -- (244.5000,276.1000) -- (244.5000,276.1000) -- (244.6000,276.1000) -- (244.7000,276.1000) -- (244.7000,276.1000) -- (244.8000,276.1000) -- (244.9000,276.1000) -- (244.9000,276.1000) -- (245.0000,276.1000) -- (245.0000,276.1000) -- (245.1000,276.1000) -- (245.2000,276.1000) -- (245.2000,276.1000) -- (245.3000,276.1000) -- (245.3000,276.1000) -- (245.4000,276.1000) -- (245.5000,276.1000) -- (245.5000,276.1000) -- (245.6000,276.1000) -- (245.7000,276.1000) -- (245.7000,276.1000) -- (245.8000,276.1000) -- (245.8000,276.1000) -- (245.9000,276.1000) -- (246.0000,276.1000) -- (246.0000,276.1000) -- (246.1000,276.1000) -- (246.1000,276.1000) -- (246.2000,276.1000) -- (246.3000,276.1000);



  \end{scope}
  \begin{scope}[cm={{1.00588,0.0,0.0,1.00588,(98.0737,353.5865)}},draw=black,line join=bevel,line cap=rect,line width=0.800pt]
    \path[fill=black] (0.0000,0.0000) node[above right] (text344-6) {x (m)};



    \begin{scope}[shift={(99.64232,0)},draw=black,line join=bevel,line cap=rect,line width=0.800pt]
      \path[fill=black] (0.0000,0.0000) node[above right] (text1412-7) {\scriptsize Time (min)};



    \end{scope}
  \end{scope}
\end{scope}

\end{tikzpicture}


%    \caption{Path, parameters ($c_{i,1},c_{i,2}$) evolutions}
%    \label{fig:trajs}
%  \end{subfigure}\hspace*{3ex}
%  \begin{subfigure}{.54\textwidth}
%    \centering
%    \footnotesize
%    
\definecolor{cebebeb}{RGB}{235,235,235}
\definecolor{ca0a0a4}{RGB}{160,160,164}
\definecolor{cd9d9d9}{RGB}{217,217,217}
\definecolor{cffffff}{RGB}{255,255,255}
\definecolor{ce10000}{RGB}{225,0,0}
\definecolor{cff0000}{RGB}{255,0,0}
\definecolor{c00ff00}{RGB}{0,255,0}


\def \globalscale {1.100000}
\begin{tikzpicture}[y=0.80pt, x=0.80pt, yscale=-\globalscale, xscale=.945*\globalscale, inner sep=0pt, outer sep=0pt]
\begin{scope}[shift={(275.47253,-4.27761)},draw=black,line join=bevel,line cap=rect,even odd rule,line width=0.800pt]
  \path[fill=cebebeb,dash pattern=on 7.51pt off 0.63pt,line join=round,miter limit=4.00,even odd rule,line width=0.626pt,rounded corners=0.0000cm] (-300.5533,-4.7126) rectangle (-73.4257,215.1424);



  \begin{scope}[cm={{1.00588,0.0,0.0,1.00588,(-334.50007,11.83512)}},draw=ca0a0a4,dash pattern=on 0.40pt off 0.80pt,line join=round,line cap=round,line width=0.400pt]
    \path[draw] (56.5000,232.5000) -- (251.5000,232.5000);



  \end{scope}
  \begin{scope}[draw=black,line join=bevel,line cap=rect,line width=0.800pt]
  \end{scope}
  \begin{scope}[scale=1.006,draw=black,line join=bevel,line cap=rect,line width=0.800pt]
  \end{scope}
  \begin{scope}[scale=1.006,draw=black,line join=bevel,line cap=rect,line width=0.800pt]
  \end{scope}
  \begin{scope}[cm={{1.00588,0.0,0.0,1.00588,(39.2294,93.5471)}},draw=black,line join=bevel,line cap=rect,line width=0.800pt]
  \end{scope}
  \begin{scope}[cm={{1.00588,0.0,0.0,1.00588,(39.2294,93.5471)}},draw=black,line join=bevel,line cap=rect,line width=0.800pt]
  \end{scope}
  \begin{scope}[cm={{1.00588,0.0,0.0,1.00588,(39.2294,93.5471)}},draw=black,line join=bevel,line cap=rect,line width=0.800pt]
  \end{scope}
  \begin{scope}[cm={{1.00588,0.0,0.0,1.00588,(39.2294,93.5471)}},draw=black,line join=bevel,line cap=rect,line width=0.800pt]
  \end{scope}
  \begin{scope}[cm={{1.00588,0.0,0.0,1.00588,(39.2294,93.5471)}},draw=black,line join=bevel,line cap=rect,line width=0.800pt]
  \end{scope}
  \begin{scope}[cm={{1.00588,0.0,0.0,1.00588,(-71.3301,182.37827)}},draw=black,line join=bevel,line cap=rect,line width=0.800pt]
    \path[fill=black] (0.0000,0.0000) node[above right] (text34) {32};



  \end{scope}
  \begin{scope}[cm={{1.00588,0.0,0.0,1.00588,(39.2294,93.5471)}},draw=black,line join=bevel,line cap=rect,line width=0.800pt]
  \end{scope}
  \begin{scope}[scale=1.006,draw=black,line join=bevel,line cap=rect,line width=0.800pt]
  \end{scope}
  \begin{scope}[scale=1.006,draw=black,line join=bevel,line cap=rect,line width=0.800pt]
  \end{scope}
  \begin{scope}[cm={{1.00588,0.0,0.0,1.00588,(39.2294,68.4)}},draw=black,line join=bevel,line cap=rect,line width=0.800pt]
  \end{scope}
  \begin{scope}[cm={{1.00588,0.0,0.0,1.00588,(39.2294,68.4)}},draw=black,line join=bevel,line cap=rect,line width=0.800pt]
  \end{scope}
  \begin{scope}[cm={{1.00588,0.0,0.0,1.00588,(39.2294,68.4)}},draw=black,line join=bevel,line cap=rect,line width=0.800pt]
  \end{scope}
  \begin{scope}[cm={{1.00588,0.0,0.0,1.00588,(39.2294,68.4)}},draw=black,line join=bevel,line cap=rect,line width=0.800pt]
  \end{scope}
  \begin{scope}[cm={{1.00588,0.0,0.0,1.00588,(39.2294,68.4)}},draw=black,line join=bevel,line cap=rect,line width=0.800pt]
  \end{scope}
  \begin{scope}[cm={{1.00588,0.0,0.0,1.00588,(-71.27377,157.23118)}},draw=black,line join=bevel,line cap=rect,line width=0.800pt]
    \path[fill=black] (0.0000,0.0000) node[above right] (text64) {36};



  \end{scope}
  \begin{scope}[cm={{1.00588,0.0,0.0,1.00588,(39.2294,68.4)}},draw=black,line join=bevel,line cap=rect,line width=0.800pt]
  \end{scope}
  \begin{scope}[scale=1.006,draw=black,line join=bevel,line cap=rect,line width=0.800pt]
  \end{scope}
  \begin{scope}[scale=1.006,draw=black,line join=bevel,line cap=rect,line width=0.800pt]
  \end{scope}
  \begin{scope}[cm={{1.00588,0.0,0.0,1.00588,(39.2294,43.2529)}},draw=black,line join=bevel,line cap=rect,line width=0.800pt]
  \end{scope}
  \begin{scope}[cm={{1.00588,0.0,0.0,1.00588,(39.2294,43.2529)}},draw=black,line join=bevel,line cap=rect,line width=0.800pt]
  \end{scope}
  \begin{scope}[cm={{1.00588,0.0,0.0,1.00588,(39.2294,43.2529)}},draw=black,line join=bevel,line cap=rect,line width=0.800pt]
  \end{scope}
  \begin{scope}[cm={{1.00588,0.0,0.0,1.00588,(39.2294,43.2529)}},draw=black,line join=bevel,line cap=rect,line width=0.800pt]
  \end{scope}
  \begin{scope}[cm={{1.00588,0.0,0.0,1.00588,(39.2294,43.2529)}},draw=black,line join=bevel,line cap=rect,line width=0.800pt]
  \end{scope}
  \begin{scope}[cm={{1.00588,0.0,0.0,1.00588,(-71.33815,132.08407)}},draw=black,line join=bevel,line cap=rect,line width=0.800pt]
    \path[fill=black] (0.0000,0.0000) node[above right] (text94) {40};



  \end{scope}
  \begin{scope}[cm={{1.00588,0.0,0.0,1.00588,(39.2294,43.2529)}},draw=black,line join=bevel,line cap=rect,line width=0.800pt]
  \end{scope}
  \begin{scope}[scale=1.006,draw=black,line join=bevel,line cap=rect,line width=0.800pt]
  \end{scope}
  \begin{scope}[scale=1.006,draw=black,line join=bevel,line cap=rect,line width=0.800pt]
  \end{scope}
  \begin{scope}[cm={{1.00588,0.0,0.0,1.00588,(53.3118,110.647)}},draw=black,line join=bevel,line cap=rect,line width=0.800pt]
  \end{scope}
  \begin{scope}[cm={{1.00588,0.0,0.0,1.00588,(53.3118,110.647)}},draw=black,line join=bevel,line cap=rect,line width=0.800pt]
  \end{scope}
  \begin{scope}[cm={{1.00588,0.0,0.0,1.00588,(53.3118,110.647)}},draw=black,line join=bevel,line cap=rect,line width=0.800pt]
  \end{scope}
  \begin{scope}[cm={{1.00588,0.0,0.0,1.00588,(53.3118,110.647)}},draw=black,line join=bevel,line cap=rect,line width=0.800pt]
  \end{scope}
  \begin{scope}[cm={{1.00588,0.0,0.0,1.00588,(53.3118,110.647)}},draw=black,line join=bevel,line cap=rect,line width=0.800pt]
  \end{scope}
  \begin{scope}[cm={{1.00588,0.0,0.0,1.00588,(-59.37572,331.271)}},draw=black,line join=bevel,line cap=rect,line width=0.800pt]
    \path[fill=black] (0.0000,0.0000) node[above right] (text124) {\scriptsize 0};



  \end{scope}
  \begin{scope}[cm={{1.00588,0.0,0.0,1.00588,(53.3118,110.647)}},draw=black,line join=bevel,line cap=rect,line width=0.800pt]
  \end{scope}
  \begin{scope}[scale=1.006,draw=black,line join=bevel,line cap=rect,line width=0.800pt]
  \end{scope}
  \begin{scope}[scale=1.006,draw=black,line join=bevel,line cap=rect,line width=0.800pt]
  \end{scope}
  \begin{scope}[cm={{1.00588,0.0,0.0,1.00588,(79.4647,110.647)}},draw=black,line join=bevel,line cap=rect,line width=0.800pt]
  \end{scope}
  \begin{scope}[cm={{1.00588,0.0,0.0,1.00588,(79.4647,110.647)}},draw=black,line join=bevel,line cap=rect,line width=0.800pt]
  \end{scope}
  \begin{scope}[cm={{1.00588,0.0,0.0,1.00588,(79.4647,110.647)}},draw=black,line join=bevel,line cap=rect,line width=0.800pt]
  \end{scope}
  \begin{scope}[cm={{1.00588,0.0,0.0,1.00588,(79.4647,110.647)}},draw=black,line join=bevel,line cap=rect,line width=0.800pt]
  \end{scope}
  \begin{scope}[cm={{1.00588,0.0,0.0,1.00588,(79.4647,110.647)}},draw=black,line join=bevel,line cap=rect,line width=0.800pt]
  \end{scope}
  \begin{scope}[cm={{1.00588,0.0,0.0,1.00588,(-33.22283,331.38366)}},draw=black,line join=bevel,line cap=rect,line width=0.800pt]
    \path[fill=black] (0.0000,0.0000) node[above right] (text156) {\scriptsize 1};



  \end{scope}
  \begin{scope}[cm={{1.00588,0.0,0.0,1.00588,(79.4647,110.647)}},draw=black,line join=bevel,line cap=rect,line width=0.800pt]
  \end{scope}
  \begin{scope}[scale=1.006,draw=black,line join=bevel,line cap=rect,line width=0.800pt]
  \end{scope}
  \begin{scope}[scale=1.006,draw=black,line join=bevel,line cap=rect,line width=0.800pt]
  \end{scope}
  \begin{scope}[cm={{1.00588,0.0,0.0,1.00588,(105.618,110.647)}},draw=black,line join=bevel,line cap=rect,line width=0.800pt]
  \end{scope}
  \begin{scope}[cm={{1.00588,0.0,0.0,1.00588,(105.618,110.647)}},draw=black,line join=bevel,line cap=rect,line width=0.800pt]
  \end{scope}
  \begin{scope}[cm={{1.00588,0.0,0.0,1.00588,(105.618,110.647)}},draw=black,line join=bevel,line cap=rect,line width=0.800pt]
  \end{scope}
  \begin{scope}[cm={{1.00588,0.0,0.0,1.00588,(105.618,110.647)}},draw=black,line join=bevel,line cap=rect,line width=0.800pt]
  \end{scope}
  \begin{scope}[cm={{1.00588,0.0,0.0,1.00588,(105.618,110.647)}},draw=black,line join=bevel,line cap=rect,line width=0.800pt]
  \end{scope}
  \begin{scope}[cm={{1.00588,0.0,0.0,1.00588,(-7.06952,331.38366)}},draw=black,line join=bevel,line cap=rect,line width=0.800pt]
    \path[fill=black] (0.0000,0.0000) node[above right] (text188) {\scriptsize 2};



  \end{scope}
  \begin{scope}[cm={{1.00588,0.0,0.0,1.00588,(105.618,110.647)}},draw=black,line join=bevel,line cap=rect,line width=0.800pt]
  \end{scope}
  \begin{scope}[scale=1.006,draw=black,line join=bevel,line cap=rect,line width=0.800pt]
  \end{scope}
  \begin{scope}[scale=1.006,draw=black,line join=bevel,line cap=rect,line width=0.800pt]
  \end{scope}
  \begin{scope}[cm={{1.00588,0.0,0.0,1.00588,(132.274,110.647)}},draw=black,line join=bevel,line cap=rect,line width=0.800pt]
  \end{scope}
  \begin{scope}[cm={{1.00588,0.0,0.0,1.00588,(132.274,110.647)}},draw=black,line join=bevel,line cap=rect,line width=0.800pt]
  \end{scope}
  \begin{scope}[cm={{1.00588,0.0,0.0,1.00588,(132.274,110.647)}},draw=black,line join=bevel,line cap=rect,line width=0.800pt]
  \end{scope}
  \begin{scope}[cm={{1.00588,0.0,0.0,1.00588,(132.274,110.647)}},draw=black,line join=bevel,line cap=rect,line width=0.800pt]
  \end{scope}
  \begin{scope}[cm={{1.00588,0.0,0.0,1.00588,(132.274,110.647)}},draw=black,line join=bevel,line cap=rect,line width=0.800pt]
  \end{scope}
  \begin{scope}[cm={{1.00588,0.0,0.0,1.00588,(19.58648,331.271)}},draw=black,line join=bevel,line cap=rect,line width=0.800pt]
    \path[fill=black] (0.0000,0.0000) node[above right] (text218) {\scriptsize 3};



  \end{scope}
  \begin{scope}[cm={{1.00588,0.0,0.0,1.00588,(132.274,110.647)}},draw=black,line join=bevel,line cap=rect,line width=0.800pt]
  \end{scope}
  \begin{scope}[scale=1.006,draw=black,line join=bevel,line cap=rect,line width=0.800pt]
  \end{scope}
  \begin{scope}[scale=1.006,draw=black,line join=bevel,line cap=rect,line width=0.800pt]
  \end{scope}
  \begin{scope}[scale=1.006,draw=black,line join=bevel,line cap=rect,line width=0.800pt]
  \end{scope}
  \begin{scope}[scale=1.006,draw=black,line join=bevel,line cap=rect,line width=0.800pt]
  \end{scope}
  \begin{scope}[scale=1.006,draw=black,line join=bevel,line cap=rect,line width=0.800pt]
  \end{scope}
  \begin{scope}[scale=1.006,draw=black,line join=bevel,line cap=rect,line width=0.800pt]
  \end{scope}
  \begin{scope}[cm={{1.00588,0.0,0.0,1.00588,(128.753,29.1706)}},draw=black,line join=bevel,line cap=rect,line width=0.800pt]
  \end{scope}
  \begin{scope}[cm={{1.00588,0.0,0.0,1.00588,(128.753,29.1706)}},draw=black,line join=bevel,line cap=rect,line width=0.800pt]
  \end{scope}
  \begin{scope}[cm={{1.00588,0.0,0.0,1.00588,(128.753,29.1706)}},draw=black,line join=bevel,line cap=rect,line width=0.800pt]
  \end{scope}
  \begin{scope}[cm={{1.00588,0.0,0.0,1.00588,(128.753,29.1706)}},draw=black,line join=bevel,line cap=rect,line width=0.800pt]
  \end{scope}
  \begin{scope}[cm={{1.00588,0.0,0.0,1.00588,(128.753,29.1706)}},draw=black,line join=bevel,line cap=rect,line width=0.800pt]
  \end{scope}
  \begin{scope}[cm={{1.00588,0.0,0.0,1.00588,(128.753,29.1706)}},draw=black,line join=bevel,line cap=rect,line width=0.800pt]
  \end{scope}
  \begin{scope}[cm={{0.0,-1.00588,1.00588,0.0,(29.1706,189.106)}},draw=black,line join=bevel,line cap=rect,line width=0.800pt]
  \end{scope}
  \begin{scope}[cm={{0.0,-1.00588,1.00588,0.0,(29.1706,189.106)}},draw=black,line join=bevel,line cap=rect,line width=0.800pt]
  \end{scope}
  \begin{scope}[cm={{0.0,-1.00588,1.00588,0.0,(29.1706,189.106)}},draw=black,line join=bevel,line cap=rect,line width=0.800pt]
  \end{scope}
  \begin{scope}[cm={{0.0,-1.00588,1.00588,0.0,(29.1706,189.106)}},draw=black,line join=bevel,line cap=rect,line width=0.800pt]
  \end{scope}
  \begin{scope}[cm={{0.0,-1.00588,1.00588,0.0,(29.1706,189.106)}},draw=black,line join=bevel,line cap=rect,line width=0.800pt]
  \end{scope}
  \begin{scope}[cm={{0.0,-1.00588,1.00588,0.0,(281.91864,214.19272)}},draw=black,line join=bevel,line cap=rect,line width=0.800pt]
    \path[fill=black] (35.7896,-587.5454) node[above right] (text274) {\rotatebox{90}{Power (W)}};



  \end{scope}
  \begin{scope}[cm={{0.0,-1.00588,1.00588,0.0,(29.1706,189.106)}},draw=black,line join=bevel,line cap=rect,line width=0.800pt]
  \end{scope}
  \begin{scope}[cm={{1.00588,0.0,0.0,1.00588,(62.3647,28.1647)}},draw=black,line join=bevel,line cap=rect,line width=0.800pt]
  \end{scope}
  \begin{scope}[cm={{1.00588,0.0,0.0,1.00588,(62.3647,28.1647)}},draw=black,line join=bevel,line cap=rect,line width=0.800pt]
  \end{scope}
  \begin{scope}[cm={{1.00588,0.0,0.0,1.00588,(62.3647,28.1647)}},draw=black,line join=bevel,line cap=rect,line width=0.800pt]
  \end{scope}
  \begin{scope}[cm={{1.00588,0.0,0.0,1.00588,(62.3647,28.1647)}},draw=black,line join=bevel,line cap=rect,line width=0.800pt]
  \end{scope}
  \begin{scope}[cm={{1.00588,0.0,0.0,1.00588,(62.3647,28.1647)}},draw=black,line join=bevel,line cap=rect,line width=0.800pt]
  \end{scope}
  \begin{scope}[cm={{1.00588,0.0,0.0,1.00588,(62.3647,28.1647)}},draw=black,line join=bevel,line cap=rect,line width=0.800pt]
  \end{scope}
  \begin{scope}[scale=1.006,draw=black,line join=bevel,line cap=rect,line width=0.800pt]
  \end{scope}
  \begin{scope}[scale=1.006,draw=black,line join=bevel,line cap=rect,line width=0.800pt]
  \end{scope}
  \begin{scope}[scale=1.006,draw=black,line join=bevel,line cap=rect,line width=0.800pt]
  \end{scope}
  \begin{scope}[scale=1.006,draw=black,line join=bevel,line cap=rect,line width=0.800pt]
  \end{scope}
  \begin{scope}[scale=1.006,draw=black,line join=bevel,line cap=rect,line width=0.800pt]
  \end{scope}
  \begin{scope}[scale=1.006,draw=black,line join=bevel,line cap=rect,line width=0.800pt]
  \end{scope}
  \begin{scope}[cm={{1.00588,0.0,0.0,1.00588,(60.3529,36.2118)}},draw=black,line join=bevel,line cap=rect,line width=0.800pt]
  \end{scope}
  \begin{scope}[cm={{1.00588,0.0,0.0,1.00588,(60.3529,36.2118)}},draw=black,line join=bevel,line cap=rect,line width=0.800pt]
  \end{scope}
  \begin{scope}[cm={{1.00588,0.0,0.0,1.00588,(60.3529,36.2118)}},draw=black,line join=bevel,line cap=rect,line width=0.800pt]
  \end{scope}
  \begin{scope}[cm={{1.00588,0.0,0.0,1.00588,(60.3529,36.2118)}},draw=black,line join=bevel,line cap=rect,line width=0.800pt]
  \end{scope}
  \begin{scope}[cm={{1.00588,0.0,0.0,1.00588,(60.3529,36.2118)}},draw=black,line join=bevel,line cap=rect,line width=0.800pt]
  \end{scope}
  \begin{scope}[cm={{1.00588,0.0,0.0,1.00588,(60.3529,36.2118)}},draw=black,line join=bevel,line cap=rect,line width=0.800pt]
  \end{scope}
  \begin{scope}[scale=1.006,draw=black,line join=bevel,line cap=rect,line width=0.800pt]
  \end{scope}
  \begin{scope}[scale=1.006,draw=black,line join=bevel,line cap=rect,line width=0.800pt]
  \end{scope}
  \begin{scope}[scale=1.006,draw=black,line join=bevel,line cap=rect,line width=0.800pt]
  \end{scope}
  \begin{scope}[scale=1.006,draw=black,line join=bevel,line cap=rect,line width=0.800pt]
  \end{scope}
  \begin{scope}[scale=1.006,draw=black,line join=bevel,line cap=rect,line width=0.800pt]
  \end{scope}
  \begin{scope}[scale=1.006,draw=black,line join=bevel,line cap=rect,line width=0.800pt]
  \end{scope}
  \begin{scope}[scale=1.006,draw=black,line join=bevel,line cap=rect,line width=0.800pt]
  \end{scope}
  \begin{scope}[scale=1.006,draw=black,line join=round,line cap=round,line width=0.480pt]
    \path[fill=cd9d9d9,line join=round,even odd rule,line width=0.423pt,rounded corners=0.0000cm] (-56.7670,121.3308) rectangle (-55.0960,184.3255);



    \path[draw=cd9d9d9,line join=miter,line cap=butt,miter limit=4.00,line width=0.596pt] (-56.7670,184.3255) -- (-57.8163,95.7681);



    \path[shift={(-113.33363,88.94668)},draw] (56.5000,13.5000) -- (56.5000,95.5000) -- (142.5000,95.5000) -- (142.5000,13.5000) -- (56.5000,13.5000);



    \begin{scope}[cm={{1.0018,0.0,0.0,0.97485,(-209.02548,-60.42265)}},draw=ca0a0a4,dash pattern=on 0.40pt off 0.80pt,line join=round,line cap=round,line width=0.400pt]
      \path[shift={(110.39113,-5.49717)},draw] (70.5000,164.5000) -- (70.5000,88.5000);



    \end{scope}
    \begin{scope}[cm={{1.0018,0.0,0.0,0.97485,(-98.43569,-64.76728)}},draw=ca0a0a4,dash pattern=on 0.40pt off 0.80pt,line join=round,line cap=round,line width=0.400pt]
      \path[draw] (98.5000,164.5000) -- (98.5000,88.5000);



    \end{scope}
    \path[fill=cffffff,line join=round,even odd rule,line width=0.366pt,rounded corners=0.0000cm] (-276.0985,21.1427) rectangle (-274.4182,68.1237);



    \path[draw=cffffff,line join=miter,line cap=butt,miter limit=4.00,line width=0.596pt] (-275.6926,67.7575) -- (-168.4330,95.4081);



    \path[draw=cffffff,line join=miter,line cap=butt,line width=0.596pt] (-167.5446,19.4576) -- (-274.5156,21.5404);



    \begin{scope}[cm={{0.7438,0.0,0.0,0.77563,(-75.95485,181.46752)}},draw=ca0a0a4,dash pattern=on 0.40pt off 0.80pt,line join=round,line cap=round,line width=0.400pt]
      \path[draw] (25.5000,26.5000) -- (108.5000,26.5000);



      \path[draw] (137.5000,26.5000) -- (142.5000,26.5000);



    \end{scope}
    \begin{scope}[cm={{0.7438,0.0,0.0,0.77563,(-75.95485,181.46752)}},draw=black,line join=round,line cap=round,line width=0.480pt]
      \path[draw] (25.5000,26.5000) -- (28.5000,26.5000);



      \path[draw] (142.5000,26.5000) -- (139.5000,26.5000);



    \end{scope}
    \begin{scope}[cm={{0.95389,0.0,0.0,0.95389,(-70.56111,203.49392)}},draw=black,fill=ce10000,line join=bevel,line cap=rect,line width=0.800pt]
      \path[fill=ce10000] (0.0000,0.0000) node[above right] (text36) {\scriptsize 30};



    \end{scope}
    \begin{scope}[cm={{0.7438,0.0,0.0,0.77563,(-75.95485,181.46752)}},draw=ca0a0a4,dash pattern=on 0.40pt off 0.80pt,line join=round,line cap=round,line width=0.400pt]
      \path[draw] (25.5000,11.5000) -- (142.5000,11.5000);



    \end{scope}
    \begin{scope}[cm={{0.7438,0.0,0.0,0.77563,(-75.95485,181.46752)}},draw=black,line join=round,line cap=round,line width=0.480pt]
      \path[draw] (25.5000,11.5000) -- (28.5000,11.5000);



      \path[draw] (142.5000,11.5000) -- (139.5000,11.5000);



    \end{scope}
    \begin{scope}[cm={{0.95389,0.0,0.0,0.95389,(-70.56111,192.73926)}},draw=black,fill=ce10000,line join=bevel,line cap=rect,line width=0.800pt]
      \path[fill=ce10000] (0.0000,0.0000) node[above right] (text66) {\scriptsize 40};



    \end{scope}
    \begin{scope}[cm={{0.7438,0.0,0.0,0.77563,(-75.95485,181.46752)}},draw=ca0a0a4,dash pattern=on 0.40pt off 0.80pt,line join=round,line cap=round,line width=0.400pt]
      \path[draw] (25.5000,32.5000) -- (25.5000,8.5000);



    \end{scope}
    \begin{scope}[cm={{0.7438,0.0,0.0,0.77563,(-75.95485,181.46752)}},draw=black,line join=round,line cap=round,line width=0.480pt]
      \path[draw] (25.5000,32.5000) -- (25.5000,28.5000);



      \path[draw] (25.5000,8.5000) -- (25.5000,11.5000);



    \end{scope}
    \begin{scope}[cm={{0.7438,0.0,0.0,0.77563,(-75.95485,181.46752)}},draw=ca0a0a4,dash pattern=on 0.40pt off 0.80pt,line join=round,line cap=round,line width=0.400pt]
      \path[draw] (60.5000,32.5000) -- (60.5000,8.5000);



    \end{scope}
    \begin{scope}[cm={{0.7438,0.0,0.0,0.77563,(-75.95485,181.46752)}},draw=black,line join=round,line cap=round,line width=0.480pt]
      \path[draw] (60.5000,32.5000) -- (60.5000,28.5000);



      \path[draw] (60.5000,8.5000) -- (60.5000,11.5000);



    \end{scope}
    \begin{scope}[cm={{0.7438,0.0,0.0,0.77563,(-75.95485,181.46752)}},draw=ca0a0a4,dash pattern=on 0.40pt off 0.80pt,line join=round,line cap=round,line width=0.400pt]
      \path[draw] (95.5000,32.5000) -- (95.5000,8.5000);



    \end{scope}
    \begin{scope}[cm={{0.7438,0.0,0.0,0.77563,(-75.95485,181.46752)}},draw=black,line join=round,line cap=round,line width=0.480pt]
      \path[draw] (95.5000,32.5000) -- (95.5000,28.5000);



      \path[draw] (95.5000,8.5000) -- (95.5000,11.5000);



    \end{scope}
    \begin{scope}[cm={{0.7438,0.0,0.0,0.77563,(-75.95485,181.46752)}},draw=ca0a0a4,dash pattern=on 0.40pt off 0.80pt,line join=round,line cap=round,line width=0.400pt]
      \path[draw] (130.5000,20.5000) -- (130.5000,8.5000);



    \end{scope}
    \begin{scope}[cm={{0.7438,0.0,0.0,0.77563,(-75.95485,181.46752)}},draw=black,line join=round,line cap=round,line width=0.480pt]
      \path[draw] (130.5000,32.5000) -- (130.5000,28.5000);



      \path[draw] (130.5000,8.5000) -- (130.5000,11.5000);



    \end{scope}
    \begin{scope}[cm={{0.7438,0.0,0.0,0.77563,(-75.95485,181.46752)}},draw=black,line join=round,line cap=round,line width=0.480pt]
      \path[draw] (25.5000,8.5000) -- (25.5000,32.5000) -- (142.5000,32.5000) -- (142.5000,8.5000) -- (25.5000,8.5000);



    \end{scope}
    \begin{scope}[cm={{0.95389,0.0,0.0,0.95389,(5.3584,204.65381)}},draw=black,line join=bevel,line cap=rect,line width=0.800pt]
      \path[fill=black] (0.0000,0.0000) node[above right] (text194) {\scriptsize $\alpha_0$};



    \end{scope}
    \begin{scope}[cm={{0.7438,0.0,0.0,0.77563,(-73.80457,178.25993)}},draw=black,line join=round,line cap=round,line width=0.480pt]
      \path[draw,even odd rule] (123.5000,28.5000) -- (132.5000,28.5000);



    \end{scope}
    \begin{scope}[cm={{0.7438,0.0,0.0,0.77563,(-75.95485,181.46752)}},draw=black,line join=round,line cap=round,line width=0.480pt]
      \path[draw] (25.8000,32.0000) -- (26.2000,14.3000) -- (26.7000,17.2000) -- (27.1000,18.1000) -- (27.5000,14.6000) -- (27.9000,12.6000) -- (28.3000,14.9000) -- (28.8000,17.3000) -- (29.2000,10.3000) -- (29.6000,14.5000) -- (30.0000,19.1000) -- (30.5000,15.7000) -- (30.9000,12.7000) -- (31.3000,15.0000) -- (31.7000,18.7000) -- (32.2000,19.3000) -- (32.6000,16.6000) -- (33.0000,13.7000) -- (33.4000,12.8000) -- (33.8000,14.3000) -- (34.3000,16.9000) -- (34.7000,18.9000) -- (35.1000,19.4000) -- (35.5000,18.6000) -- (36.0000,17.2000) -- (36.4000,15.9000) -- (36.8000,15.5000) -- (37.2000,15.9000) -- (37.6000,17.1000) -- (38.1000,18.6000) -- (38.5000,20.0000) -- (38.9000,21.0000) -- (39.3000,21.3000) -- (39.8000,21.0000) -- (40.2000,20.3000) -- (40.6000,19.3000) -- (41.0000,18.3000) -- (41.5000,17.4000) -- (41.9000,16.9000) -- (42.3000,16.8000) -- (42.7000,17.1000) -- (43.1000,17.6000) -- (43.6000,18.3000) -- (44.0000,19.0000) -- (44.4000,19.7000) -- (44.8000,20.1000) -- (45.3000,20.4000) -- (45.7000,20.5000) -- (46.1000,20.4000) -- (46.5000,20.1000) -- (47.0000,19.8000) -- (47.4000,19.3000) -- (47.8000,18.5000) -- (48.2000,17.8000) -- (48.6000,17.2000) -- (49.1000,16.7000) -- (49.5000,16.5000) -- (49.9000,16.4000) -- (50.3000,16.4000) -- (50.8000,16.5000) -- (51.2000,16.7000) -- (51.6000,17.0000) -- (52.0000,17.2000) -- (52.4000,17.4000) -- (52.9000,17.5000) -- (53.3000,17.5000) -- (53.7000,17.3000) -- (54.1000,17.0000) -- (54.6000,16.6000) -- (55.0000,16.1000) -- (55.4000,15.6000) -- (55.8000,15.0000) -- (56.3000,14.4000) -- (56.7000,13.9000) -- (57.1000,13.5000) -- (57.5000,13.2000) -- (57.9000,13.2000) -- (58.4000,13.3000) -- (58.8000,13.5000) -- (59.2000,13.9000) -- (59.6000,14.3000) -- (60.1000,14.7000) -- (60.5000,15.0000) -- (60.9000,15.4000) -- (61.3000,15.7000) -- (61.8000,16.0000) -- (62.2000,16.2000) -- (62.6000,16.3000) -- (63.0000,16.2000) -- (63.4000,16.2000) -- (63.9000,16.1000) -- (64.3000,16.0000) -- (64.7000,15.9000) -- (65.1000,15.9000) -- (65.6000,15.9000) -- (66.0000,16.0000) -- (66.4000,16.2000) -- (66.8000,16.4000) -- (67.3000,16.7000) -- (67.7000,16.9000) -- (68.1000,17.2000) -- (68.5000,17.3000) -- (68.9000,17.4000) -- (69.4000,17.5000) -- (69.8000,17.7000) -- (70.2000,17.7000) -- (70.6000,17.7000) -- (71.1000,17.6000) -- (71.5000,17.5000) -- (71.9000,17.3000) -- (72.3000,17.1000) -- (72.7000,17.0000) -- (73.2000,16.9000) -- (73.6000,16.8000) -- (74.0000,16.9000) -- (74.4000,17.0000) -- (74.9000,16.9000) -- (75.3000,16.8000) -- (75.7000,16.8000) -- (76.1000,16.8000) -- (76.6000,16.9000) -- (77.0000,17.1000) -- (77.4000,17.2000) -- (77.8000,17.3000) -- (78.2000,17.4000) -- (78.7000,17.4000) -- (79.1000,17.4000) -- (79.5000,17.4000) -- (79.9000,17.4000) -- (80.4000,17.3000) -- (80.8000,17.2000) -- (81.2000,17.1000) -- (81.6000,17.1000) -- (82.1000,17.0000) -- (82.5000,17.0000) -- (82.9000,16.9000) -- (83.3000,16.7000) -- (83.7000,16.6000) -- (84.2000,16.4000) -- (84.6000,16.4000) -- (85.0000,16.3000) -- (85.4000,16.3000) -- (85.9000,16.3000) -- (86.3000,16.3000) -- (86.7000,16.2000) -- (87.1000,16.2000) -- (87.6000,16.1000) -- (88.0000,16.1000) -- (88.4000,16.2000) -- (88.8000,16.3000) -- (89.2000,16.3000) -- (89.7000,16.3000) -- (90.1000,16.3000) -- (90.5000,16.3000) -- (90.9000,16.3000) -- (91.4000,16.3000) -- (91.8000,16.3000) -- (92.2000,16.3000) -- (92.6000,16.3000) -- (93.0000,16.2000) -- (93.5000,16.2000) -- (93.9000,16.2000) -- (94.3000,16.2000) -- (94.7000,16.2000) -- (95.2000,16.1000) -- (95.6000,16.1000) -- (96.0000,16.1000) -- (96.4000,16.2000) -- (96.9000,16.4000) -- (97.3000,16.5000) -- (97.7000,16.5000) -- (98.1000,16.6000) -- (98.5000,16.6000) -- (99.0000,16.6000) -- (99.4000,16.6000) -- (99.8000,16.5000) -- (100.2000,16.5000) -- (100.7000,16.6000) -- (101.1000,16.7000) -- (101.5000,16.6000) -- (101.9000,16.5000) -- (102.3000,16.3000) -- (102.8000,16.3000) -- (103.2000,16.3000) -- (103.6000,16.3000) -- (104.0000,16.3000) -- (104.5000,16.3000) -- (104.9000,16.3000) -- (105.3000,16.3000) -- (105.7000,16.2000) -- (106.2000,16.2000) -- (106.6000,16.2000) -- (107.0000,16.2000) -- (107.4000,16.2000) -- (107.8000,16.2000) -- (108.3000,16.3000) -- (108.7000,16.3000) -- (109.1000,16.4000) -- (109.5000,16.5000) -- (110.0000,16.5000) -- (110.4000,16.5000) -- (110.8000,16.5000) -- (111.2000,16.5000) -- (111.7000,16.5000) -- (112.1000,16.6000) -- (112.5000,16.6000) -- (112.9000,16.6000) -- (113.3000,16.5000) -- (113.8000,16.5000) -- (114.2000,16.4000) -- (114.6000,16.3000) -- (115.0000,16.3000) -- (115.5000,16.4000) -- (115.9000,16.5000) -- (116.3000,16.5000) -- (116.7000,16.5000) -- (117.2000,16.5000) -- (117.6000,16.5000) -- (118.0000,16.5000) -- (118.4000,16.5000) -- (118.8000,16.6000) -- (119.3000,16.6000) -- (119.7000,16.6000) -- (120.1000,16.5000) -- (120.5000,16.5000) -- (121.0000,16.5000) -- (121.4000,16.5000) -- (121.8000,16.5000) -- (122.2000,16.3000) -- (122.6000,16.3000) -- (123.1000,16.3000) -- (123.5000,16.4000) -- (123.9000,16.4000) -- (124.3000,16.4000) -- (124.8000,16.4000) -- (125.2000,16.4000) -- (125.6000,16.3000) -- (126.0000,16.3000) -- (126.5000,16.2000) -- (126.9000,16.2000) -- (127.3000,16.3000) -- (127.7000,16.5000) -- (128.1000,16.5000) -- (128.6000,16.4000) -- (129.0000,16.3000) -- (129.4000,16.3000) -- (129.8000,16.4000) -- (130.3000,16.5000) -- (130.7000,16.5000) -- (131.1000,16.6000) -- (131.5000,16.6000) -- (131.9000,16.6000) -- (132.4000,16.6000) -- (132.8000,16.5000) -- (133.2000,16.5000) -- (133.6000,16.4000) -- (134.1000,16.4000) -- (134.5000,16.3000) -- (134.9000,16.3000) -- (135.3000,16.3000) -- (135.8000,16.4000) -- (136.2000,16.4000) -- (136.6000,16.4000) -- (137.0000,16.4000) -- (137.4000,16.4000) -- (137.9000,16.4000) -- (138.3000,16.4000) -- (138.7000,16.5000) -- (139.1000,16.5000) -- (139.6000,16.5000) -- (140.0000,16.5000) -- (140.4000,16.5000) -- (140.8000,16.4000) -- (141.3000,16.4000) -- (141.7000,16.3000) -- (142.1000,16.4000) -- (142.3000,16.4000);



    \end{scope}
    \begin{scope}[cm={{0.7438,0.0,0.0,0.77563,(-75.95485,181.46752)}},draw=black,line join=round,line cap=round,line width=0.480pt]
      \path[draw] (25.5000,8.5000) -- (25.5000,32.5000) -- (142.5000,32.5000) -- (142.5000,8.5000) -- (25.5000,8.5000);



    \end{scope}
    \begin{scope}[cm={{0.7438,0.0,0.0,0.77563,(-75.95485,181.46752)}},draw=ca0a0a4,dash pattern=on 0.40pt off 0.80pt,line join=round,line cap=round,line width=0.400pt]
      \path[draw] (25.5000,46.5000) -- (108.5000,46.5000);



      \path[draw] (137.5000,46.5000) -- (142.5000,46.5000);



    \end{scope}
    \begin{scope}[cm={{0.7438,0.0,0.0,0.77563,(-75.95485,181.46752)}},draw=black,line join=round,line cap=round,line width=0.480pt]
      \path[draw] (25.5000,46.5000) -- (28.5000,46.5000);



      \path[draw] (142.5000,46.5000) -- (139.5000,46.5000);



    \end{scope}
    \begin{scope}[cm={{0.95389,0.0,0.0,0.95389,(-73.10978,219.77968)}},draw=black,fill=ce10000,line join=bevel,line cap=rect,line width=0.800pt]
      \path[fill=ce10000] (0.0000,0.0000) node[above right] (text250) {\scriptsize -10};



    \end{scope}
    \begin{scope}[cm={{0.7438,0.0,0.0,0.77563,(-75.95485,181.46752)}},draw=ca0a0a4,dash pattern=on 0.40pt off 0.80pt,line join=round,line cap=round,line width=0.400pt]
      \path[draw] (25.5000,34.5000) -- (142.5000,34.5000);



    \end{scope}
    \begin{scope}[cm={{0.7438,0.0,0.0,0.77563,(-75.95485,181.46752)}},draw=black,line join=round,line cap=round,line width=0.480pt]
      \path[draw] (25.5000,34.5000) -- (28.5000,34.5000);



      \path[draw] (142.5000,34.5000) -- (139.5000,34.5000);



    \end{scope}
    \begin{scope}[cm={{0.95389,0.0,0.0,0.95389,(-70.56111,210.96852)}},draw=black,fill=ce10000,line join=bevel,line cap=rect,line width=0.800pt]
      \path[fill=ce10000] (0.0000,0.0000) node[above right] (text280) {\scriptsize 10};



    \end{scope}
    \begin{scope}[cm={{0.7438,0.0,0.0,0.77563,(-75.95485,181.46752)}},draw=ca0a0a4,dash pattern=on 0.40pt off 0.80pt,line join=round,line cap=round,line width=0.400pt]
      \path[draw] (25.5000,56.5000) -- (25.5000,32.5000);



    \end{scope}
    \begin{scope}[cm={{0.7438,0.0,0.0,0.77563,(-75.95485,181.46752)}},draw=black,line join=round,line cap=round,line width=0.480pt]
      \path[draw] (25.5000,56.5000) -- (25.5000,51.5000);



      \path[draw] (25.5000,32.5000) -- (25.5000,36.5000);



    \end{scope}
    \begin{scope}[cm={{0.7438,0.0,0.0,0.77563,(-75.95485,181.46752)}},draw=ca0a0a4,dash pattern=on 0.40pt off 0.80pt,line join=round,line cap=round,line width=0.400pt]
      \path[draw] (60.5000,56.5000) -- (60.5000,32.5000);



    \end{scope}
    \begin{scope}[cm={{0.7438,0.0,0.0,0.77563,(-75.95485,181.46752)}},draw=black,line join=round,line cap=round,line width=0.480pt]
      \path[draw] (60.5000,56.5000) -- (60.5000,51.5000);



      \path[draw] (60.5000,32.5000) -- (60.5000,36.5000);



    \end{scope}
    \begin{scope}[cm={{0.7438,0.0,0.0,0.77563,(-75.95485,181.46752)}},draw=ca0a0a4,dash pattern=on 0.40pt off 0.80pt,line join=round,line cap=round,line width=0.400pt]
      \path[draw] (95.5000,56.5000) -- (95.5000,32.5000);



    \end{scope}
    \begin{scope}[cm={{0.7438,0.0,0.0,0.77563,(-75.95485,181.46752)}},draw=black,line join=round,line cap=round,line width=0.480pt]
      \path[draw] (95.5000,56.5000) -- (95.5000,51.5000);



      \path[draw] (95.5000,32.5000) -- (95.5000,36.5000);



    \end{scope}
    \begin{scope}[cm={{0.7438,0.0,0.0,0.77563,(-75.95485,181.46752)}},draw=ca0a0a4,dash pattern=on 0.40pt off 0.80pt,line join=round,line cap=round,line width=0.400pt]
      \path[draw] (130.5000,44.5000) -- (130.5000,32.5000);



    \end{scope}
    \begin{scope}[cm={{0.7438,0.0,0.0,0.77563,(-75.95485,181.46752)}},draw=black,line join=round,line cap=round,line width=0.480pt]
      \path[draw] (130.5000,56.5000) -- (130.5000,51.5000);



      \path[draw] (130.5000,32.5000) -- (130.5000,36.5000);



    \end{scope}
    \begin{scope}[cm={{0.7438,0.0,0.0,0.77563,(-75.95485,181.46752)}},draw=black,line join=round,line cap=round,line width=0.480pt]
      \path[draw] (25.5000,32.5000) -- (25.5000,56.5000) -- (142.5000,56.5000) -- (142.5000,32.5000) -- (25.5000,32.5000);



    \end{scope}
    \begin{scope}[cm={{0.95389,0.0,0.0,0.95389,(5.3584,223.44629)}},draw=black,line join=bevel,line cap=rect,line width=0.800pt]
      \path[fill=black] (0.0000,0.0000) node[above right] (text408) {\scriptsize $\alpha_1$};



    \end{scope}
    \begin{scope}[cm={{0.7438,0.0,0.0,0.77563,(-73.80457,178.25993)}},draw=black,line join=round,line cap=round,line width=0.480pt]
      \path[draw,even odd rule] (123.5000,52.5000) -- (132.5000,52.5000);



    \end{scope}
    \begin{scope}[cm={{0.7438,0.0,0.0,0.77563,(-75.95485,181.46752)}},draw=black,line join=round,line cap=round,line width=0.480pt]
      \path[draw] (25.8000,36.4000) -- (25.8000,36.4000) -- (26.2000,41.9000) -- (26.7000,39.4000) -- (27.1000,40.5000) -- (27.5000,40.1000) -- (27.9000,39.9000) -- (28.3000,40.6000) -- (28.8000,43.1000) -- (29.2000,39.4000) -- (29.6000,39.6000) -- (30.0000,41.8000) -- (30.5000,40.8000) -- (30.9000,39.0000) -- (31.3000,39.5000) -- (31.7000,41.2000) -- (32.2000,42.0000) -- (32.6000,41.2000) -- (33.0000,39.7000) -- (33.4000,38.9000) -- (33.8000,39.1000) -- (34.3000,40.2000) -- (34.7000,41.3000) -- (35.1000,41.9000) -- (35.5000,41.9000) -- (36.0000,41.5000) -- (36.4000,40.9000) -- (36.8000,40.4000) -- (37.2000,40.4000) -- (37.6000,40.7000) -- (38.1000,41.2000) -- (38.5000,41.8000) -- (38.9000,42.2000) -- (39.3000,42.3000) -- (39.8000,42.2000) -- (40.2000,41.8000) -- (40.6000,41.3000) -- (41.0000,40.6000) -- (41.5000,39.9000) -- (41.9000,39.3000) -- (42.3000,38.9000) -- (42.7000,38.7000) -- (43.1000,38.7000) -- (43.6000,38.8000) -- (44.0000,39.0000) -- (44.4000,39.2000) -- (44.8000,39.4000) -- (45.3000,39.6000) -- (45.7000,39.8000) -- (46.1000,39.8000) -- (46.5000,39.8000) -- (47.0000,39.8000) -- (47.4000,39.7000) -- (47.8000,39.5000) -- (48.2000,39.3000) -- (48.6000,39.1000) -- (49.1000,39.0000) -- (49.5000,38.9000) -- (49.9000,38.9000) -- (50.3000,39.0000) -- (50.8000,39.2000) -- (51.2000,39.4000) -- (51.6000,39.7000) -- (52.0000,39.9000) -- (52.4000,40.2000) -- (52.9000,40.5000) -- (53.3000,40.7000) -- (53.7000,40.8000) -- (54.1000,40.9000) -- (54.6000,40.9000) -- (55.0000,40.9000) -- (55.4000,40.8000) -- (55.8000,40.6000) -- (56.3000,40.5000) -- (56.7000,40.3000) -- (57.1000,40.1000) -- (57.5000,40.0000) -- (57.9000,40.0000) -- (58.4000,40.0000) -- (58.8000,40.1000) -- (59.2000,40.3000) -- (59.6000,40.5000) -- (60.1000,40.7000) -- (60.5000,40.9000) -- (60.9000,41.1000) -- (61.3000,41.4000) -- (61.8000,41.6000) -- (62.2000,41.7000) -- (62.6000,41.8000) -- (63.0000,41.9000) -- (63.4000,41.9000) -- (63.9000,41.8000) -- (64.3000,41.8000) -- (64.7000,41.7000) -- (65.1000,41.6000) -- (65.6000,41.6000) -- (66.0000,41.5000) -- (66.4000,41.4000) -- (66.8000,41.4000) -- (67.3000,41.4000) -- (67.7000,41.3000) -- (68.1000,41.3000) -- (68.5000,41.2000) -- (68.9000,41.1000) -- (69.4000,41.0000) -- (69.8000,40.9000) -- (70.2000,40.8000) -- (70.6000,40.7000) -- (71.1000,40.5000) -- (71.5000,40.3000) -- (71.9000,40.1000) -- (72.3000,39.9000) -- (72.7000,39.7000) -- (73.2000,39.5000) -- (73.6000,39.4000) -- (74.0000,39.3000) -- (74.4000,39.2000) -- (74.9000,39.0000) -- (75.3000,38.9000) -- (75.7000,38.9000) -- (76.1000,38.8000) -- (76.6000,38.8000) -- (77.0000,38.9000) -- (77.4000,38.9000) -- (77.8000,39.0000) -- (78.2000,39.0000) -- (78.7000,39.1000) -- (79.1000,39.2000) -- (79.5000,39.3000) -- (79.9000,39.3000) -- (80.4000,39.4000) -- (80.8000,39.5000) -- (81.2000,39.6000) -- (81.6000,39.7000) -- (82.1000,39.8000) -- (82.5000,39.9000) -- (82.9000,40.0000) -- (83.3000,40.1000) -- (83.7000,40.1000) -- (84.2000,40.2000) -- (84.6000,40.3000) -- (85.0000,40.4000) -- (85.4000,40.5000) -- (85.9000,40.7000) -- (86.3000,40.8000) -- (86.7000,40.9000) -- (87.1000,41.0000) -- (87.6000,41.1000) -- (88.0000,41.1000) -- (88.4000,41.3000) -- (88.8000,41.4000) -- (89.2000,41.5000) -- (89.7000,41.6000) -- (90.1000,41.6000) -- (90.5000,41.7000) -- (90.9000,41.7000) -- (91.4000,41.7000) -- (91.8000,41.7000) -- (92.2000,41.7000) -- (92.6000,41.7000) -- (93.0000,41.6000) -- (93.5000,41.6000) -- (93.9000,41.5000) -- (94.3000,41.5000) -- (94.7000,41.4000) -- (95.2000,41.3000) -- (95.6000,41.2000) -- (96.0000,41.1000) -- (96.4000,41.0000) -- (96.9000,41.0000) -- (97.3000,40.9000) -- (97.7000,40.8000) -- (98.1000,40.7000) -- (98.5000,40.6000) -- (99.0000,40.5000) -- (99.4000,40.4000) -- (99.8000,40.3000) -- (100.2000,40.1000) -- (100.7000,40.1000) -- (101.1000,40.0000) -- (101.5000,39.9000) -- (101.9000,39.7000) -- (102.3000,39.6000) -- (102.8000,39.5000) -- (103.2000,39.4000) -- (103.6000,39.3000) -- (104.0000,39.2000) -- (104.5000,39.2000) -- (104.9000,39.1000) -- (105.3000,39.1000) -- (105.7000,39.1000) -- (106.2000,39.1000) -- (106.6000,39.1000) -- (107.0000,39.1000) -- (107.4000,39.2000) -- (107.8000,39.2000) -- (108.3000,39.3000) -- (108.7000,39.4000) -- (109.1000,39.5000) -- (109.5000,39.7000) -- (110.0000,39.8000) -- (110.4000,39.9000) -- (110.8000,40.0000) -- (111.2000,40.1000) -- (111.7000,40.3000) -- (112.1000,40.4000) -- (112.5000,40.6000) -- (112.9000,40.7000) -- (113.3000,40.8000) -- (113.8000,40.9000) -- (114.2000,41.0000) -- (114.6000,41.1000) -- (115.0000,41.2000) -- (115.5000,41.3000) -- (115.9000,41.4000) -- (116.3000,41.5000) -- (116.7000,41.6000) -- (117.2000,41.6000) -- (117.6000,41.7000) -- (118.0000,41.7000) -- (118.4000,41.8000) -- (118.8000,41.8000) -- (119.3000,41.8000) -- (119.7000,41.8000) -- (120.1000,41.7000) -- (120.5000,41.7000) -- (121.0000,41.6000) -- (121.4000,41.6000) -- (121.8000,41.5000) -- (122.2000,41.3000) -- (122.6000,41.2000) -- (123.1000,41.1000) -- (123.5000,41.0000) -- (123.9000,40.9000) -- (124.3000,40.8000) -- (124.8000,40.7000) -- (125.2000,40.5000) -- (125.6000,40.4000) -- (126.0000,40.2000) -- (126.5000,40.1000) -- (126.9000,39.9000) -- (127.3000,39.8000) -- (127.7000,39.8000) -- (128.1000,39.7000) -- (128.6000,39.5000) -- (129.0000,39.4000) -- (129.4000,39.4000) -- (129.8000,39.3000) -- (130.3000,39.3000) -- (130.7000,39.2000) -- (131.1000,39.2000) -- (131.5000,39.2000) -- (131.9000,39.2000) -- (132.4000,39.2000) -- (132.8000,39.2000) -- (133.2000,39.2000) -- (133.6000,39.2000) -- (134.1000,39.3000) -- (134.5000,39.3000) -- (134.9000,39.4000) -- (135.3000,39.5000) -- (135.8000,39.6000) -- (136.2000,39.7000) -- (136.6000,39.8000) -- (137.0000,39.9000) -- (137.4000,40.0000) -- (137.9000,40.1000) -- (138.3000,40.3000) -- (138.7000,40.4000) -- (139.1000,40.6000) -- (139.6000,40.7000) -- (140.0000,40.8000) -- (140.4000,40.9000) -- (140.8000,41.0000) -- (141.3000,41.1000) -- (141.7000,41.2000) -- (142.1000,41.3000) -- (142.3000,41.4000);



    \end{scope}
    \begin{scope}[cm={{0.7438,0.0,0.0,0.77563,(-75.95485,181.46752)}},draw=black,line join=round,line cap=round,line width=0.480pt]
      \path[draw] (25.5000,32.5000) -- (25.5000,56.5000) -- (142.5000,56.5000) -- (142.5000,32.5000) -- (25.5000,32.5000);



    \end{scope}
    \begin{scope}[cm={{0.7438,0.0,0.0,0.77563,(-75.95485,181.46752)}},draw=ca0a0a4,dash pattern=on 0.40pt off 0.80pt,line join=round,line cap=round,line width=0.400pt]
      \path[draw] (25.5000,70.5000) -- (108.5000,70.5000);



      \path[draw] (137.5000,70.5000) -- (142.5000,70.5000);



    \end{scope}
    \begin{scope}[cm={{0.7438,0.0,0.0,0.77563,(-75.95485,181.46752)}},draw=black,line join=round,line cap=round,line width=0.480pt]
      \path[draw] (25.5000,70.5000) -- (28.5000,70.5000);



      \path[draw] (142.5000,70.5000) -- (139.5000,70.5000);



    \end{scope}
    \begin{scope}[cm={{0.95389,0.0,0.0,0.95389,(-73.10978,238.00895)}},draw=black,fill=ce10000,line join=bevel,line cap=rect,line width=0.800pt]
      \path[fill=ce10000] (0.0000,0.0000) node[above right] (text464) {\scriptsize -10};



    \end{scope}
    \begin{scope}[cm={{0.7438,0.0,0.0,0.77563,(-75.95485,181.46752)}},draw=ca0a0a4,dash pattern=on 0.40pt off 0.80pt,line join=round,line cap=round,line width=0.400pt]
      \path[draw] (25.5000,58.5000) -- (142.5000,58.5000);



    \end{scope}
    \begin{scope}[cm={{0.7438,0.0,0.0,0.77563,(-75.95485,181.46752)}},draw=black,line join=round,line cap=round,line width=0.480pt]
      \path[draw] (25.5000,58.5000) -- (28.5000,58.5000);



      \path[draw] (142.5000,58.5000) -- (139.5000,58.5000);



    \end{scope}
    \begin{scope}[cm={{0.95389,0.0,0.0,0.95389,(-70.56111,229.00738)}},draw=black,fill=ce10000,line join=bevel,line cap=rect,line width=0.800pt]
      \path[fill=ce10000] (0.0000,0.0000) node[above right] (text494) {\scriptsize 10};



    \end{scope}
    \begin{scope}[cm={{0.7438,0.0,0.0,0.77563,(-75.95485,181.46752)}},draw=ca0a0a4,dash pattern=on 0.40pt off 0.80pt,line join=round,line cap=round,line width=0.400pt]
      \path[draw] (25.5000,80.5000) -- (25.5000,56.5000);



    \end{scope}
    \begin{scope}[cm={{0.7438,0.0,0.0,0.77563,(-75.95485,181.46752)}},draw=black,line join=round,line cap=round,line width=0.480pt]
      \path[draw] (25.5000,80.5000) -- (25.5000,75.5000);



      \path[draw] (25.5000,56.5000) -- (25.5000,60.5000);



    \end{scope}
    \begin{scope}[cm={{0.7438,0.0,0.0,0.77563,(-75.95485,181.46752)}},draw=ca0a0a4,dash pattern=on 0.40pt off 0.80pt,line join=round,line cap=round,line width=0.400pt]
      \path[draw] (60.5000,80.5000) -- (60.5000,56.5000);



    \end{scope}
    \begin{scope}[cm={{0.7438,0.0,0.0,0.77563,(-75.95485,181.46752)}},draw=black,line join=round,line cap=round,line width=0.480pt]
      \path[draw] (60.5000,80.5000) -- (60.5000,75.5000);



      \path[draw] (60.5000,56.5000) -- (60.5000,60.5000);



    \end{scope}
    \begin{scope}[cm={{0.7438,0.0,0.0,0.77563,(-75.95485,181.46752)}},draw=ca0a0a4,dash pattern=on 0.40pt off 0.80pt,line join=round,line cap=round,line width=0.400pt]
      \path[draw] (95.5000,80.5000) -- (95.5000,56.5000);



    \end{scope}
    \begin{scope}[cm={{0.7438,0.0,0.0,0.77563,(-75.95485,181.46752)}},draw=black,line join=round,line cap=round,line width=0.480pt]
      \path[draw] (95.5000,80.5000) -- (95.5000,75.5000);



      \path[draw] (95.5000,56.5000) -- (95.5000,60.5000);



    \end{scope}
    \begin{scope}[cm={{0.7438,0.0,0.0,0.77563,(-75.95485,181.46752)}},draw=ca0a0a4,dash pattern=on 0.40pt off 0.80pt,line join=round,line cap=round,line width=0.400pt]
      \path[draw] (130.5000,68.5000) -- (130.5000,56.5000);



    \end{scope}
    \begin{scope}[cm={{0.7438,0.0,0.0,0.77563,(-75.95485,181.46752)}},draw=black,line join=round,line cap=round,line width=0.480pt]
      \path[draw] (130.5000,80.5000) -- (130.5000,75.5000);



      \path[draw] (130.5000,56.5000) -- (130.5000,60.5000);



    \end{scope}
    \begin{scope}[cm={{0.7438,0.0,0.0,0.77563,(-75.95485,181.46752)}},draw=black,line join=round,line cap=round,line width=0.480pt]
      \path[draw] (25.5000,56.5000) -- (25.5000,80.5000) -- (142.5000,80.5000) -- (142.5000,56.5000) -- (25.5000,56.5000);



    \end{scope}
    \begin{scope}[cm={{0.95389,0.0,0.0,0.95389,(5.61786,241.86597)}},draw=black,line join=bevel,line cap=rect,line width=0.800pt]
      \path[fill=black] (0.0000,0.0000) node[above right] (text622) {\scriptsize $\beta_1$};



    \end{scope}
    \begin{scope}[cm={{0.7438,0.0,0.0,0.77563,(-73.80457,178.25993)}},draw=black,line join=round,line cap=round,line width=0.480pt]
      \path[draw,even odd rule] (123.5000,76.5000) -- (132.5000,76.5000);



    \end{scope}
    \begin{scope}[cm={{0.7438,0.0,0.0,0.77563,(-75.95485,181.46752)}},draw=black,line join=round,line cap=round,line width=0.480pt]
      \path[draw] (25.8000,67.9000) -- (25.8000,67.9000) -- (26.2000,64.4000) -- (26.7000,63.4000) -- (27.1000,65.9000) -- (27.5000,64.8000) -- (27.9000,64.0000) -- (28.3000,65.9000) -- (28.8000,65.0000) -- (29.2000,61.7000) -- (29.6000,64.2000) -- (30.0000,65.6000) -- (30.5000,63.7000) -- (30.9000,62.9000) -- (31.3000,64.5000) -- (31.7000,65.9000) -- (32.2000,65.6000) -- (32.6000,64.1000) -- (33.0000,62.9000) -- (33.4000,63.0000) -- (33.8000,64.1000) -- (34.3000,65.3000) -- (34.7000,66.0000) -- (35.1000,65.8000) -- (35.5000,65.1000) -- (36.0000,64.3000) -- (36.4000,63.6000) -- (36.8000,63.4000) -- (37.2000,63.6000) -- (37.6000,64.0000) -- (38.1000,64.5000) -- (38.5000,64.8000) -- (38.9000,64.8000) -- (39.3000,64.6000) -- (39.8000,64.1000) -- (40.2000,63.5000) -- (40.6000,63.0000) -- (41.0000,62.5000) -- (41.5000,62.2000) -- (41.9000,62.1000) -- (42.3000,62.3000) -- (42.7000,62.7000) -- (43.1000,63.2000) -- (43.6000,63.7000) -- (44.0000,64.1000) -- (44.4000,64.6000) -- (44.8000,64.9000) -- (45.3000,65.1000) -- (45.7000,65.3000) -- (46.1000,65.3000) -- (46.5000,65.3000) -- (47.0000,65.2000) -- (47.4000,65.1000) -- (47.8000,64.9000) -- (48.2000,64.8000) -- (48.6000,64.7000) -- (49.1000,64.8000) -- (49.5000,64.8000) -- (49.9000,65.0000) -- (50.3000,65.2000) -- (50.8000,65.4000) -- (51.2000,65.6000) -- (51.6000,65.8000) -- (52.0000,66.0000) -- (52.4000,66.1000) -- (52.9000,66.2000) -- (53.3000,66.1000) -- (53.7000,66.0000) -- (54.1000,65.9000) -- (54.6000,65.7000) -- (55.0000,65.4000) -- (55.4000,65.2000) -- (55.8000,64.9000) -- (56.3000,64.7000) -- (56.7000,64.5000) -- (57.1000,64.4000) -- (57.5000,64.3000) -- (57.9000,64.4000) -- (58.4000,64.5000) -- (58.8000,64.6000) -- (59.2000,64.8000) -- (59.6000,64.9000) -- (60.1000,65.0000) -- (60.5000,65.1000) -- (60.9000,65.2000) -- (61.3000,65.2000) -- (61.8000,65.2000) -- (62.2000,65.1000) -- (62.6000,65.0000) -- (63.0000,64.8000) -- (63.4000,64.6000) -- (63.9000,64.3000) -- (64.3000,64.1000) -- (64.7000,63.9000) -- (65.1000,63.8000) -- (65.6000,63.7000) -- (66.0000,63.6000) -- (66.4000,63.5000) -- (66.8000,63.4000) -- (67.3000,63.4000) -- (67.7000,63.4000) -- (68.1000,63.4000) -- (68.5000,63.3000) -- (68.9000,63.3000) -- (69.4000,63.3000) -- (69.8000,63.3000) -- (70.2000,63.2000) -- (70.6000,63.2000) -- (71.1000,63.1000) -- (71.5000,63.1000) -- (71.9000,63.0000) -- (72.3000,63.0000) -- (72.7000,63.0000) -- (73.2000,63.1000) -- (73.6000,63.2000) -- (74.0000,63.3000) -- (74.4000,63.4000) -- (74.9000,63.5000) -- (75.3000,63.6000) -- (75.7000,63.8000) -- (76.1000,64.0000) -- (76.6000,64.2000) -- (77.0000,64.4000) -- (77.4000,64.6000) -- (77.8000,64.8000) -- (78.2000,64.9000) -- (78.7000,65.1000) -- (79.1000,65.2000) -- (79.5000,65.3000) -- (79.9000,65.4000) -- (80.4000,65.5000) -- (80.8000,65.6000) -- (81.2000,65.6000) -- (81.6000,65.7000) -- (82.1000,65.7000) -- (82.5000,65.8000) -- (82.9000,65.8000) -- (83.3000,65.8000) -- (83.7000,65.8000) -- (84.2000,65.7000) -- (84.6000,65.7000) -- (85.0000,65.7000) -- (85.4000,65.7000) -- (85.9000,65.7000) -- (86.3000,65.6000) -- (86.7000,65.6000) -- (87.1000,65.5000) -- (87.6000,65.5000) -- (88.0000,65.4000) -- (88.4000,65.3000) -- (88.8000,65.3000) -- (89.2000,65.2000) -- (89.7000,65.1000) -- (90.1000,65.0000) -- (90.5000,64.9000) -- (90.9000,64.7000) -- (91.4000,64.6000) -- (91.8000,64.5000) -- (92.2000,64.4000) -- (92.6000,64.2000) -- (93.0000,64.1000) -- (93.5000,64.0000) -- (93.9000,63.8000) -- (94.3000,63.7000) -- (94.7000,63.6000) -- (95.2000,63.5000) -- (95.6000,63.4000) -- (96.0000,63.4000) -- (96.4000,63.4000) -- (96.9000,63.4000) -- (97.3000,63.4000) -- (97.7000,63.3000) -- (98.1000,63.3000) -- (98.5000,63.3000) -- (99.0000,63.3000) -- (99.4000,63.3000) -- (99.8000,63.3000) -- (100.2000,63.3000) -- (100.7000,63.3000) -- (101.1000,63.4000) -- (101.5000,63.4000) -- (101.9000,63.4000) -- (102.3000,63.5000) -- (102.8000,63.6000) -- (103.2000,63.7000) -- (103.6000,63.8000) -- (104.0000,63.9000) -- (104.5000,64.0000) -- (104.9000,64.1000) -- (105.3000,64.3000) -- (105.7000,64.4000) -- (106.2000,64.5000) -- (106.6000,64.6000) -- (107.0000,64.8000) -- (107.4000,64.9000) -- (107.8000,65.0000) -- (108.3000,65.2000) -- (108.7000,65.3000) -- (109.1000,65.4000) -- (109.5000,65.5000) -- (110.0000,65.6000) -- (110.4000,65.7000) -- (110.8000,65.7000) -- (111.2000,65.7000) -- (111.7000,65.8000) -- (112.1000,65.8000) -- (112.5000,65.8000) -- (112.9000,65.8000) -- (113.3000,65.7000) -- (113.8000,65.7000) -- (114.2000,65.6000) -- (114.6000,65.5000) -- (115.0000,65.4000) -- (115.5000,65.4000) -- (115.9000,65.3000) -- (116.3000,65.2000) -- (116.7000,65.1000) -- (117.2000,65.0000) -- (117.6000,64.9000) -- (118.0000,64.7000) -- (118.4000,64.6000) -- (118.8000,64.5000) -- (119.3000,64.4000) -- (119.7000,64.2000) -- (120.1000,64.1000) -- (120.5000,64.0000) -- (121.0000,63.8000) -- (121.4000,63.7000) -- (121.8000,63.6000) -- (122.2000,63.4000) -- (122.6000,63.3000) -- (123.1000,63.3000) -- (123.5000,63.2000) -- (123.9000,63.2000) -- (124.3000,63.2000) -- (124.8000,63.1000) -- (125.2000,63.1000) -- (125.6000,63.1000) -- (126.0000,63.1000) -- (126.5000,63.1000) -- (126.9000,63.1000) -- (127.3000,63.2000) -- (127.7000,63.4000) -- (128.1000,63.4000) -- (128.6000,63.5000) -- (129.0000,63.5000) -- (129.4000,63.6000) -- (129.8000,63.8000) -- (130.3000,63.9000) -- (130.7000,64.1000) -- (131.1000,64.2000) -- (131.5000,64.3000) -- (131.9000,64.4000) -- (132.4000,64.6000) -- (132.8000,64.7000) -- (133.2000,64.8000) -- (133.6000,64.9000) -- (134.1000,65.0000) -- (134.5000,65.1000) -- (134.9000,65.2000) -- (135.3000,65.3000) -- (135.8000,65.4000) -- (136.2000,65.5000) -- (136.6000,65.6000) -- (137.0000,65.6000) -- (137.4000,65.6000) -- (137.9000,65.7000) -- (138.3000,65.7000) -- (138.7000,65.8000) -- (139.1000,65.8000) -- (139.6000,65.8000) -- (140.0000,65.7000) -- (140.4000,65.7000) -- (140.8000,65.6000) -- (141.3000,65.5000) -- (141.7000,65.4000) -- (142.1000,65.4000) -- (142.3000,65.3000);



    \end{scope}
    \begin{scope}[cm={{0.7438,0.0,0.0,0.77563,(-75.95485,181.46752)}},draw=black,line join=round,line cap=round,line width=0.480pt]
      \path[draw] (25.5000,56.5000) -- (25.5000,80.5000) -- (142.5000,80.5000) -- (142.5000,56.5000) -- (25.5000,56.5000);



    \end{scope}
    \begin{scope}[cm={{0.7438,0.0,0.0,0.77563,(-75.95485,181.46752)}},draw=ca0a0a4,dash pattern=on 0.40pt off 0.80pt,line join=round,line cap=round,line width=0.400pt]
      \path[draw] (25.5000,94.5000) -- (108.5000,94.5000);



      \path[draw] (137.5000,94.5000) -- (142.5000,94.5000);



    \end{scope}
    \begin{scope}[cm={{0.7438,0.0,0.0,0.77563,(-75.95485,181.46752)}},draw=black,line join=round,line cap=round,line width=0.480pt]
      \path[draw] (25.5000,94.5000) -- (28.5000,94.5000);



      \path[draw] (142.5000,94.5000) -- (139.5000,94.5000);



    \end{scope}
    \begin{scope}[cm={{0.95389,0.0,0.0,0.95389,(-73.10978,256.42862)}},draw=black,fill=ce10000,line join=bevel,line cap=rect,line width=0.800pt]
      \path[fill=ce10000] (0.0000,0.0000) node[above right] (text678) {\scriptsize -10};



    \end{scope}
    \begin{scope}[cm={{0.7438,0.0,0.0,0.77563,(-75.95485,181.46752)}},draw=ca0a0a4,dash pattern=on 0.40pt off 0.80pt,line join=round,line cap=round,line width=0.400pt]
      \path[draw] (25.5000,82.5000) -- (142.5000,82.5000);



    \end{scope}
    \begin{scope}[cm={{0.7438,0.0,0.0,0.77563,(-75.95485,181.46752)}},draw=black,line join=round,line cap=round,line width=0.480pt]
      \path[draw] (25.5000,82.5000) -- (28.5000,82.5000);



      \path[draw] (142.5000,82.5000) -- (139.5000,82.5000);



    \end{scope}
    \begin{scope}[cm={{0.95389,0.0,0.0,0.95389,(-70.56111,247.42706)}},draw=black,fill=ce10000,line join=bevel,line cap=rect,line width=0.800pt]
      \path[fill=ce10000] (0.0000,0.0000) node[above right] (text708) {\scriptsize 10};



    \end{scope}
    \begin{scope}[cm={{0.7438,0.0,0.0,0.77563,(-75.95485,181.46752)}},draw=ca0a0a4,dash pattern=on 0.40pt off 0.80pt,line join=round,line cap=round,line width=0.400pt]
      \path[draw] (25.5000,104.5000) -- (25.5000,80.5000);



    \end{scope}
    \begin{scope}[cm={{0.7438,0.0,0.0,0.77563,(-75.95485,181.46752)}},draw=black,line join=round,line cap=round,line width=0.480pt]
      \path[draw] (25.5000,104.5000) -- (25.5000,99.5000);



      \path[draw] (25.5000,80.5000) -- (25.5000,84.5000);



    \end{scope}
    \begin{scope}[cm={{0.7438,0.0,0.0,0.77563,(-75.95485,181.46752)}},draw=ca0a0a4,dash pattern=on 0.40pt off 0.80pt,line join=round,line cap=round,line width=0.400pt]
      \path[draw] (60.5000,104.5000) -- (60.5000,80.5000);



    \end{scope}
    \begin{scope}[cm={{0.7438,0.0,0.0,0.77563,(-75.95485,181.46752)}},draw=black,line join=round,line cap=round,line width=0.480pt]
      \path[draw] (60.5000,104.5000) -- (60.5000,99.5000);



      \path[draw] (60.5000,80.5000) -- (60.5000,84.5000);



    \end{scope}
    \begin{scope}[cm={{0.7438,0.0,0.0,0.77563,(-75.95485,181.46752)}},draw=ca0a0a4,dash pattern=on 0.40pt off 0.80pt,line join=round,line cap=round,line width=0.400pt]
      \path[draw] (95.5000,104.5000) -- (95.5000,80.5000);



    \end{scope}
    \begin{scope}[cm={{0.7438,0.0,0.0,0.77563,(-75.95485,181.46752)}},draw=black,line join=round,line cap=round,line width=0.480pt]
      \path[draw] (95.5000,104.5000) -- (95.5000,99.5000);



      \path[draw] (95.5000,80.5000) -- (95.5000,84.5000);



    \end{scope}
    \begin{scope}[cm={{0.7438,0.0,0.0,0.77563,(-75.95485,181.46752)}},draw=ca0a0a4,dash pattern=on 0.40pt off 0.80pt,line join=round,line cap=round,line width=0.400pt]
      \path[draw] (130.5000,92.5000) -- (130.5000,80.5000);



    \end{scope}
    \begin{scope}[cm={{0.7438,0.0,0.0,0.77563,(-75.95485,181.46752)}},draw=black,line join=round,line cap=round,line width=0.480pt]
      \path[draw] (130.5000,104.5000) -- (130.5000,99.5000);



      \path[draw] (130.5000,80.5000) -- (130.5000,84.5000);



    \end{scope}
    \begin{scope}[cm={{0.7438,0.0,0.0,0.77563,(-75.95485,181.46752)}},draw=black,line join=round,line cap=round,line width=0.480pt]
      \path[draw] (25.5000,80.5000) -- (25.5000,104.5000) -- (142.5000,104.5000) -- (142.5000,80.5000) -- (25.5000,80.5000);



    \end{scope}
    \begin{scope}[cm={{0.95389,0.0,0.0,0.95389,(5.3584,260.28562)}},draw=black,line join=bevel,line cap=rect,line width=0.800pt]
      \path[fill=black] (0.0000,0.0000) node[above right] (text836) {\scriptsize $\alpha_2$};



    \end{scope}
    \begin{scope}[cm={{0.7438,0.0,0.0,0.77563,(-73.80457,178.25993)}},draw=black,line join=round,line cap=round,line width=0.480pt]
      \path[draw,even odd rule] (123.5000,100.5000) -- (132.5000,100.5000);



    \end{scope}
    \begin{scope}[cm={{0.7438,0.0,0.0,0.77563,(-75.95485,181.46752)}},draw=black,line join=round,line cap=round,line width=0.480pt]
      \path[draw] (25.8000,86.8000) -- (25.8000,86.8000) -- (26.2000,88.5000) -- (26.7000,88.7000) -- (27.1000,88.0000) -- (27.5000,88.0000) -- (27.9000,90.0000) -- (28.3000,85.4000) -- (28.8000,92.2000) -- (29.2000,86.5000) -- (29.6000,86.2000) -- (30.0000,92.2000) -- (30.5000,91.0000) -- (30.9000,85.7000) -- (31.3000,84.6000) -- (31.7000,88.3000) -- (32.2000,91.6000) -- (32.6000,91.4000) -- (33.0000,88.6000) -- (33.4000,86.0000) -- (33.8000,85.5000) -- (34.3000,87.1000) -- (34.7000,89.4000) -- (35.1000,91.2000) -- (35.5000,91.5000) -- (36.0000,90.5000) -- (36.4000,88.8000) -- (36.8000,87.2000) -- (37.2000,86.2000) -- (37.6000,86.2000) -- (38.1000,86.9000) -- (38.5000,88.0000) -- (38.9000,89.2000) -- (39.3000,90.2000) -- (39.8000,90.6000) -- (40.2000,90.5000) -- (40.6000,89.9000) -- (41.0000,89.0000) -- (41.5000,88.0000) -- (41.9000,87.1000) -- (42.3000,86.4000) -- (42.7000,86.1000) -- (43.1000,86.2000) -- (43.6000,86.6000) -- (44.0000,87.3000) -- (44.4000,88.1000) -- (44.8000,88.9000) -- (45.3000,89.6000) -- (45.7000,90.1000) -- (46.1000,90.5000) -- (46.5000,90.6000) -- (47.0000,90.6000) -- (47.4000,90.3000) -- (47.8000,89.7000) -- (48.2000,89.0000) -- (48.6000,88.3000) -- (49.1000,87.6000) -- (49.5000,86.9000) -- (49.9000,86.4000) -- (50.3000,86.0000) -- (50.8000,85.9000) -- (51.2000,86.0000) -- (51.6000,86.2000) -- (52.0000,86.6000) -- (52.4000,87.2000) -- (52.9000,87.8000) -- (53.3000,88.3000) -- (53.7000,88.9000) -- (54.1000,89.3000) -- (54.6000,89.6000) -- (55.0000,89.8000) -- (55.4000,89.8000) -- (55.8000,89.6000) -- (56.3000,89.4000) -- (56.7000,89.0000) -- (57.1000,88.6000) -- (57.5000,88.2000) -- (57.9000,87.9000) -- (58.4000,87.7000) -- (58.8000,87.6000) -- (59.2000,87.6000) -- (59.6000,87.7000) -- (60.1000,87.9000) -- (60.5000,88.2000) -- (60.9000,88.5000) -- (61.3000,88.8000) -- (61.8000,89.1000) -- (62.2000,89.4000) -- (62.6000,89.5000) -- (63.0000,89.6000) -- (63.4000,89.6000) -- (63.9000,89.4000) -- (64.3000,89.2000) -- (64.7000,89.0000) -- (65.1000,88.7000) -- (65.6000,88.5000) -- (66.0000,88.2000) -- (66.4000,88.1000) -- (66.8000,87.9000) -- (67.3000,87.9000) -- (67.7000,87.9000) -- (68.1000,87.9000) -- (68.5000,88.0000) -- (68.9000,88.1000) -- (69.4000,88.2000) -- (69.8000,88.4000) -- (70.2000,88.5000) -- (70.6000,88.6000) -- (71.1000,88.7000) -- (71.5000,88.7000) -- (71.9000,88.7000) -- (72.3000,88.6000) -- (72.7000,88.6000) -- (73.2000,88.5000) -- (73.6000,88.5000) -- (74.0000,88.5000) -- (74.4000,88.5000) -- (74.9000,88.4000) -- (75.3000,88.4000) -- (75.7000,88.3000) -- (76.1000,88.3000) -- (76.6000,88.4000) -- (77.0000,88.4000) -- (77.4000,88.4000) -- (77.8000,88.5000) -- (78.2000,88.5000) -- (78.7000,88.6000) -- (79.1000,88.6000) -- (79.5000,88.6000) -- (79.9000,88.6000) -- (80.4000,88.6000) -- (80.8000,88.5000) -- (81.2000,88.5000) -- (81.6000,88.4000) -- (82.1000,88.3000) -- (82.5000,88.3000) -- (82.9000,88.2000) -- (83.3000,88.1000) -- (83.7000,88.0000) -- (84.2000,88.0000) -- (84.6000,87.9000) -- (85.0000,87.9000) -- (85.4000,87.9000) -- (85.9000,87.9000) -- (86.3000,88.0000) -- (86.7000,88.0000) -- (87.1000,88.1000) -- (87.6000,88.2000) -- (88.0000,88.3000) -- (88.4000,88.4000) -- (88.8000,88.5000) -- (89.2000,88.7000) -- (89.7000,88.8000) -- (90.1000,88.9000) -- (90.5000,89.0000) -- (90.9000,89.0000) -- (91.4000,89.1000) -- (91.8000,89.1000) -- (92.2000,89.0000) -- (92.6000,89.0000) -- (93.0000,88.9000) -- (93.5000,88.8000) -- (93.9000,88.7000) -- (94.3000,88.6000) -- (94.7000,88.5000) -- (95.2000,88.3000) -- (95.6000,88.2000) -- (96.0000,88.1000) -- (96.4000,88.0000) -- (96.9000,88.0000) -- (97.3000,88.0000) -- (97.7000,88.0000) -- (98.1000,88.0000) -- (98.5000,88.0000) -- (99.0000,88.1000) -- (99.4000,88.1000) -- (99.8000,88.2000) -- (100.2000,88.2000) -- (100.7000,88.3000) -- (101.1000,88.5000) -- (101.5000,88.5000) -- (101.9000,88.5000) -- (102.3000,88.5000) -- (102.8000,88.6000) -- (103.2000,88.6000) -- (103.6000,88.6000) -- (104.0000,88.7000) -- (104.5000,88.7000) -- (104.9000,88.7000) -- (105.3000,88.7000) -- (105.7000,88.7000) -- (106.2000,88.6000) -- (106.6000,88.6000) -- (107.0000,88.6000) -- (107.4000,88.5000) -- (107.8000,88.5000) -- (108.3000,88.4000) -- (108.7000,88.4000) -- (109.1000,88.4000) -- (109.5000,88.4000) -- (110.0000,88.4000) -- (110.4000,88.4000) -- (110.8000,88.3000) -- (111.2000,88.3000) -- (111.7000,88.3000) -- (112.1000,88.3000) -- (112.5000,88.3000) -- (112.9000,88.3000) -- (113.3000,88.3000) -- (113.8000,88.3000) -- (114.2000,88.3000) -- (114.6000,88.3000) -- (115.0000,88.3000) -- (115.5000,88.3000) -- (115.9000,88.4000) -- (116.3000,88.4000) -- (116.7000,88.4000) -- (117.2000,88.5000) -- (117.6000,88.5000) -- (118.0000,88.6000) -- (118.4000,88.6000) -- (118.8000,88.6000) -- (119.3000,88.7000) -- (119.7000,88.7000) -- (120.1000,88.7000) -- (120.5000,88.6000) -- (121.0000,88.6000) -- (121.4000,88.6000) -- (121.8000,88.5000) -- (122.2000,88.4000) -- (122.6000,88.4000) -- (123.1000,88.3000) -- (123.5000,88.3000) -- (123.9000,88.3000) -- (124.3000,88.3000) -- (124.8000,88.2000) -- (125.2000,88.2000) -- (125.6000,88.2000) -- (126.0000,88.1000) -- (126.5000,88.1000) -- (126.9000,88.2000) -- (127.3000,88.2000) -- (127.7000,88.3000) -- (128.1000,88.4000) -- (128.6000,88.4000) -- (129.0000,88.4000) -- (129.4000,88.5000) -- (129.8000,88.6000) -- (130.3000,88.6000) -- (130.7000,88.7000) -- (131.1000,88.8000) -- (131.5000,88.8000) -- (131.9000,88.8000) -- (132.4000,88.8000) -- (132.8000,88.8000) -- (133.2000,88.7000) -- (133.6000,88.7000) -- (134.1000,88.6000) -- (134.5000,88.5000) -- (134.9000,88.4000) -- (135.3000,88.4000) -- (135.8000,88.3000) -- (136.2000,88.3000) -- (136.6000,88.2000) -- (137.0000,88.2000) -- (137.4000,88.1000) -- (137.9000,88.1000) -- (138.3000,88.1000) -- (138.7000,88.1000) -- (139.1000,88.2000) -- (139.6000,88.2000) -- (140.0000,88.3000) -- (140.4000,88.3000) -- (140.8000,88.3000) -- (141.3000,88.4000) -- (141.7000,88.4000) -- (142.1000,88.5000) -- (142.3000,88.5000);



    \end{scope}
    \begin{scope}[cm={{0.7438,0.0,0.0,0.77563,(-75.95485,181.46752)}},draw=black,line join=round,line cap=round,line width=0.480pt]
      \path[draw] (25.5000,80.5000) -- (25.5000,104.5000) -- (142.5000,104.5000) -- (142.5000,80.5000) -- (25.5000,80.5000);



    \end{scope}
    \begin{scope}[cm={{0.7438,0.0,0.0,0.77563,(-75.95485,181.46752)}},draw=ca0a0a4,dash pattern=on 0.40pt off 0.80pt,line join=round,line cap=round,line width=0.400pt]
      \path[draw] (25.5000,118.5000) -- (108.5000,118.5000);



      \path[draw] (137.5000,118.5000) -- (142.5000,118.5000);



    \end{scope}
    \begin{scope}[cm={{0.7438,0.0,0.0,0.77563,(-75.95485,181.46752)}},draw=black,line join=round,line cap=round,line width=0.480pt]
      \path[draw] (25.5000,118.5000) -- (28.5000,118.5000);



      \path[draw] (142.5000,118.5000) -- (139.5000,118.5000);



    \end{scope}
    \begin{scope}[cm={{0.95389,0.0,0.0,0.95389,(-73.10978,274.8483)}},draw=black,fill=ce10000,line join=bevel,line cap=rect,line width=0.800pt]
      \path[fill=ce10000] (0.0000,0.0000) node[above right] (text892) {\scriptsize -10};



    \end{scope}
    \begin{scope}[cm={{0.7438,0.0,0.0,0.77563,(-75.95485,181.46752)}},draw=ca0a0a4,dash pattern=on 0.40pt off 0.80pt,line join=round,line cap=round,line width=0.400pt]
      \path[draw] (25.5000,106.5000) -- (142.5000,106.5000);



    \end{scope}
    \begin{scope}[cm={{0.7438,0.0,0.0,0.77563,(-75.95485,181.46752)}},draw=black,line join=round,line cap=round,line width=0.480pt]
      \path[draw] (25.5000,106.5000) -- (28.5000,106.5000);



      \path[draw] (142.5000,106.5000) -- (139.5000,106.5000);



    \end{scope}
    \begin{scope}[cm={{0.95389,0.0,0.0,0.95389,(-70.56111,265.84674)}},draw=black,fill=ce10000,line join=bevel,line cap=rect,line width=0.800pt]
      \path[fill=ce10000] (0.0000,0.0000) node[above right] (text922) {\scriptsize 10};



    \end{scope}
    \begin{scope}[cm={{0.7438,0.0,0.0,0.77563,(-75.95485,181.46752)}},draw=ca0a0a4,dash pattern=on 0.40pt off 0.80pt,line join=round,line cap=round,line width=0.400pt]
      \path[draw] (25.5000,128.5000) -- (25.5000,104.5000);



    \end{scope}
    \begin{scope}[cm={{0.7438,0.0,0.0,0.77563,(-75.95485,181.46752)}},draw=black,line join=round,line cap=round,line width=0.480pt]
      \path[draw] (25.5000,128.5000) -- (25.5000,123.5000);



      \path[draw] (25.5000,104.5000) -- (25.5000,108.5000);



    \end{scope}
    \begin{scope}[cm={{0.7438,0.0,0.0,0.77563,(-75.95485,181.46752)}},draw=ca0a0a4,dash pattern=on 0.40pt off 0.80pt,line join=round,line cap=round,line width=0.400pt]
      \path[draw] (60.5000,128.5000) -- (60.5000,104.5000);



    \end{scope}
    \begin{scope}[cm={{0.7438,0.0,0.0,0.77563,(-75.95485,181.46752)}},draw=black,line join=round,line cap=round,line width=0.480pt]
      \path[draw] (60.5000,128.5000) -- (60.5000,123.5000);



      \path[draw] (60.5000,104.5000) -- (60.5000,108.5000);



    \end{scope}
    \begin{scope}[cm={{0.7438,0.0,0.0,0.77563,(-75.95485,181.46752)}},draw=ca0a0a4,dash pattern=on 0.40pt off 0.80pt,line join=round,line cap=round,line width=0.400pt]
      \path[draw] (95.5000,128.5000) -- (95.5000,104.5000);



    \end{scope}
    \begin{scope}[cm={{0.7438,0.0,0.0,0.77563,(-75.95485,181.46752)}},draw=black,line join=round,line cap=round,line width=0.480pt]
      \path[draw] (95.5000,128.5000) -- (95.5000,123.5000);



      \path[draw] (95.5000,104.5000) -- (95.5000,108.5000);



    \end{scope}
    \begin{scope}[cm={{0.7438,0.0,0.0,0.77563,(-75.95485,181.46752)}},draw=ca0a0a4,dash pattern=on 0.40pt off 0.80pt,line join=round,line cap=round,line width=0.400pt]
      \path[draw] (130.5000,116.5000) -- (130.5000,104.5000);



    \end{scope}
    \begin{scope}[cm={{0.7438,0.0,0.0,0.77563,(-75.95485,181.46752)}},draw=black,line join=round,line cap=round,line width=0.480pt]
      \path[draw] (130.5000,128.5000) -- (130.5000,123.5000);



      \path[draw] (130.5000,104.5000) -- (130.5000,108.5000);



    \end{scope}
    \begin{scope}[cm={{0.7438,0.0,0.0,0.77563,(-75.95485,181.46752)}},draw=black,line join=round,line cap=round,line width=0.480pt]
      \path[draw] (25.5000,104.5000) -- (25.5000,128.5000) -- (142.5000,128.5000) -- (142.5000,104.5000) -- (25.5000,104.5000);



    \end{scope}
    \begin{scope}[cm={{0.95389,0.0,0.0,0.95389,(5.61786,279.2645)}},draw=black,line join=bevel,line cap=rect,line width=0.800pt]
      \path[fill=black] (0.0000,0.0000) node[above right] (text1050) {\scriptsize $\beta_2$};



    \end{scope}
    \begin{scope}[cm={{0.7438,0.0,0.0,0.77563,(-73.80457,178.25993)}},draw=black,line join=round,line cap=round,line width=0.480pt]
      \path[draw,even odd rule] (123.5000,124.5000) -- (132.5000,124.5000);



    \end{scope}
    \begin{scope}[cm={{0.7438,0.0,0.0,0.77563,(-75.95485,181.46752)}},draw=black,line join=round,line cap=round,line width=0.480pt]
      \path[draw] (25.8000,114.2000) -- (25.8000,114.2000) -- (26.2000,112.5000) -- (26.7000,112.2000) -- (27.1000,112.0000) -- (27.5000,115.1000) -- (27.9000,109.2000) -- (28.3000,114.1000) -- (28.8000,113.9000) -- (29.2000,108.7000) -- (29.6000,114.6000) -- (30.0000,115.3000) -- (30.5000,109.5000) -- (30.9000,108.5000) -- (31.3000,113.0000) -- (31.7000,116.4000) -- (32.2000,115.2000) -- (32.6000,111.5000) -- (33.0000,109.2000) -- (33.4000,109.7000) -- (33.8000,112.1000) -- (34.3000,114.6000) -- (34.7000,115.4000) -- (35.1000,114.5000) -- (35.5000,112.6000) -- (36.0000,110.6000) -- (36.4000,109.5000) -- (36.8000,109.5000) -- (37.2000,110.5000) -- (37.6000,112.0000) -- (38.1000,113.4000) -- (38.5000,114.4000) -- (38.9000,114.7000) -- (39.3000,114.3000) -- (39.8000,113.4000) -- (40.2000,112.3000) -- (40.6000,111.3000) -- (41.0000,110.6000) -- (41.5000,110.2000) -- (41.9000,110.3000) -- (42.3000,110.8000) -- (42.7000,111.6000) -- (43.1000,112.5000) -- (43.6000,113.4000) -- (44.0000,114.1000) -- (44.4000,114.5000) -- (44.8000,114.7000) -- (45.3000,114.6000) -- (45.7000,114.3000) -- (46.1000,113.7000) -- (46.5000,113.0000) -- (47.0000,112.4000) -- (47.4000,111.7000) -- (47.8000,111.0000) -- (48.2000,110.5000) -- (48.6000,110.2000) -- (49.1000,110.2000) -- (49.5000,110.4000) -- (49.9000,110.8000) -- (50.3000,111.4000) -- (50.8000,112.0000) -- (51.2000,112.7000) -- (51.6000,113.4000) -- (52.0000,114.0000) -- (52.4000,114.5000) -- (52.9000,114.8000) -- (53.3000,114.8000) -- (53.7000,114.7000) -- (54.1000,114.5000) -- (54.6000,114.0000) -- (55.0000,113.5000) -- (55.4000,113.0000) -- (55.8000,112.5000) -- (56.3000,112.0000) -- (56.7000,111.6000) -- (57.1000,111.4000) -- (57.5000,111.3000) -- (57.9000,111.3000) -- (58.4000,111.5000) -- (58.8000,111.8000) -- (59.2000,112.2000) -- (59.6000,112.5000) -- (60.1000,112.8000) -- (60.5000,113.0000) -- (60.9000,113.2000) -- (61.3000,113.2000) -- (61.8000,113.2000) -- (62.2000,113.1000) -- (62.6000,112.8000) -- (63.0000,112.6000) -- (63.4000,112.2000) -- (63.9000,111.9000) -- (64.3000,111.7000) -- (64.7000,111.5000) -- (65.1000,111.4000) -- (65.6000,111.3000) -- (66.0000,111.4000) -- (66.4000,111.5000) -- (66.8000,111.7000) -- (67.3000,112.0000) -- (67.7000,112.2000) -- (68.1000,112.4000) -- (68.5000,112.6000) -- (68.9000,112.7000) -- (69.4000,112.8000) -- (69.8000,112.9000) -- (70.2000,112.9000) -- (70.6000,112.9000) -- (71.1000,112.8000) -- (71.5000,112.7000) -- (71.9000,112.6000) -- (72.3000,112.5000) -- (72.7000,112.4000) -- (73.2000,112.3000) -- (73.6000,112.3000) -- (74.0000,112.3000) -- (74.4000,112.3000) -- (74.9000,112.3000) -- (75.3000,112.3000) -- (75.7000,112.3000) -- (76.1000,112.3000) -- (76.6000,112.4000) -- (77.0000,112.5000) -- (77.4000,112.5000) -- (77.8000,112.5000) -- (78.2000,112.5000) -- (78.7000,112.5000) -- (79.1000,112.5000) -- (79.5000,112.5000) -- (79.9000,112.4000) -- (80.4000,112.4000) -- (80.8000,112.3000) -- (81.2000,112.3000) -- (81.6000,112.3000) -- (82.1000,112.2000) -- (82.5000,112.3000) -- (82.9000,112.3000) -- (83.3000,112.3000) -- (83.7000,112.3000) -- (84.2000,112.3000) -- (84.6000,112.4000) -- (85.0000,112.5000) -- (85.4000,112.6000) -- (85.9000,112.7000) -- (86.3000,112.8000) -- (86.7000,112.8000) -- (87.1000,112.9000) -- (87.6000,112.9000) -- (88.0000,113.0000) -- (88.4000,113.0000) -- (88.8000,113.0000) -- (89.2000,113.0000) -- (89.7000,113.0000) -- (90.1000,112.9000) -- (90.5000,112.8000) -- (90.9000,112.7000) -- (91.4000,112.6000) -- (91.8000,112.4000) -- (92.2000,112.3000) -- (92.6000,112.2000) -- (93.0000,112.0000) -- (93.5000,111.9000) -- (93.9000,111.9000) -- (94.3000,111.8000) -- (94.7000,111.8000) -- (95.2000,111.8000) -- (95.6000,111.8000) -- (96.0000,111.9000) -- (96.4000,112.0000) -- (96.9000,112.2000) -- (97.3000,112.3000) -- (97.7000,112.4000) -- (98.1000,112.5000) -- (98.5000,112.6000) -- (99.0000,112.6000) -- (99.4000,112.7000) -- (99.8000,112.7000) -- (100.2000,112.8000) -- (100.7000,112.8000) -- (101.1000,112.9000) -- (101.5000,112.9000) -- (101.9000,112.8000) -- (102.3000,112.7000) -- (102.8000,112.7000) -- (103.2000,112.6000) -- (103.6000,112.6000) -- (104.0000,112.6000) -- (104.5000,112.5000) -- (104.9000,112.5000) -- (105.3000,112.4000) -- (105.7000,112.3000) -- (106.2000,112.3000) -- (106.6000,112.3000) -- (107.0000,112.2000) -- (107.4000,112.2000) -- (107.8000,112.2000) -- (108.3000,112.2000) -- (108.7000,112.2000) -- (109.1000,112.3000) -- (109.5000,112.3000) -- (110.0000,112.3000) -- (110.4000,112.3000) -- (110.8000,112.3000) -- (111.2000,112.3000) -- (111.7000,112.4000) -- (112.1000,112.4000) -- (112.5000,112.4000) -- (112.9000,112.5000) -- (113.3000,112.5000) -- (113.8000,112.5000) -- (114.2000,112.5000) -- (114.6000,112.5000) -- (115.0000,112.5000) -- (115.5000,112.6000) -- (115.9000,112.6000) -- (116.3000,112.6000) -- (116.7000,112.6000) -- (117.2000,112.6000) -- (117.6000,112.6000) -- (118.0000,112.6000) -- (118.4000,112.6000) -- (118.8000,112.6000) -- (119.3000,112.5000) -- (119.7000,112.5000) -- (120.1000,112.4000) -- (120.5000,112.4000) -- (121.0000,112.3000) -- (121.4000,112.3000) -- (121.8000,112.2000) -- (122.2000,112.2000) -- (122.6000,112.2000) -- (123.1000,112.2000) -- (123.5000,112.2000) -- (123.9000,112.3000) -- (124.3000,112.3000) -- (124.8000,112.3000) -- (125.2000,112.4000) -- (125.6000,112.4000) -- (126.0000,112.4000) -- (126.5000,112.5000) -- (126.9000,112.5000) -- (127.3000,112.6000) -- (127.7000,112.7000) -- (128.1000,112.7000) -- (128.6000,112.7000) -- (129.0000,112.7000) -- (129.4000,112.7000) -- (129.8000,112.7000) -- (130.3000,112.7000) -- (130.7000,112.7000) -- (131.1000,112.6000) -- (131.5000,112.6000) -- (131.9000,112.5000) -- (132.4000,112.4000) -- (132.8000,112.3000) -- (133.2000,112.2000) -- (133.6000,112.2000) -- (134.1000,112.1000) -- (134.5000,112.1000) -- (134.9000,112.1000) -- (135.3000,112.1000) -- (135.8000,112.1000) -- (136.2000,112.2000) -- (136.6000,112.2000) -- (137.0000,112.2000) -- (137.4000,112.3000) -- (137.9000,112.4000) -- (138.3000,112.4000) -- (138.7000,112.5000) -- (139.1000,112.6000) -- (139.6000,112.7000) -- (140.0000,112.7000) -- (140.4000,112.7000) -- (140.8000,112.7000) -- (141.3000,112.7000) -- (141.7000,112.7000) -- (142.1000,112.7000) -- (142.3000,112.7000);



    \end{scope}
    \begin{scope}[cm={{0.7438,0.0,0.0,0.77563,(-75.95485,181.46752)}},draw=black,line join=round,line cap=round,line width=0.480pt]
      \path[draw] (25.5000,104.5000) -- (25.5000,128.5000) -- (142.5000,128.5000) -- (142.5000,104.5000) -- (25.5000,104.5000);



    \end{scope}
    \begin{scope}[cm={{0.7438,0.0,0.0,0.77563,(-75.95485,181.46752)}},draw=ca0a0a4,dash pattern=on 0.40pt off 0.80pt,line join=round,line cap=round,line width=0.400pt]
      \path[draw] (25.5000,144.5000) -- (108.5000,144.5000);



      \path[draw] (137.5000,144.5000) -- (142.5000,144.5000);



    \end{scope}
    \begin{scope}[cm={{0.7438,0.0,0.0,0.77563,(-75.95485,181.46752)}},draw=black,line join=round,line cap=round,line width=0.480pt]
      \path[draw] (25.5000,144.5000) -- (28.5000,144.5000);



      \path[draw] (142.5000,144.5000) -- (139.5000,144.5000);



    \end{scope}
    \begin{scope}[cm={{0.95389,0.0,0.0,0.95389,(-73.10978,294.63842)}},draw=black,fill=ce10000,line join=bevel,line cap=rect,line width=0.800pt]
      \path[fill=ce10000] (0.0000,0.0000) node[above right] (text1106) {\scriptsize -20};



    \end{scope}
    \begin{scope}[cm={{0.7438,0.0,0.0,0.77563,(-75.95485,181.46752)}},draw=ca0a0a4,dash pattern=on 0.40pt off 0.80pt,line join=round,line cap=round,line width=0.400pt]
      \path[draw] (25.5000,129.5000) -- (142.5000,129.5000);



    \end{scope}
    \begin{scope}[cm={{0.7438,0.0,0.0,0.77563,(-75.95485,181.46752)}},draw=black,line join=round,line cap=round,line width=0.480pt]
      \path[draw] (25.5000,129.5000) -- (28.5000,129.5000);



      \path[draw] (142.5000,129.5000) -- (139.5000,129.5000);



    \end{scope}
    \begin{scope}[cm={{0.95389,0.0,0.0,0.95389,(-70.56111,283.31253)}},draw=black,fill=ce10000,line join=bevel,line cap=rect,line width=0.800pt]
      \path[fill=ce10000] (0.0000,0.0000) node[above right] (text1136) {\scriptsize 20};



    \end{scope}
    \begin{scope}[cm={{0.7438,0.0,0.0,0.77563,(-75.95485,181.46752)}},draw=ca0a0a4,dash pattern=on 0.40pt off 0.80pt,line join=round,line cap=round,line width=0.400pt]
      \path[draw] (25.5000,152.5000) -- (25.5000,128.5000);



    \end{scope}
    \begin{scope}[cm={{0.7438,0.0,0.0,0.77563,(-75.95485,181.46752)}},draw=black,line join=round,line cap=round,line width=0.480pt]
      \path[draw] (25.5000,152.5000) -- (25.5000,147.5000);



      \path[draw] (25.5000,128.5000) -- (25.5000,132.5000);



    \end{scope}
    \begin{scope}[cm={{0.7438,0.0,0.0,0.77563,(-75.95485,181.46752)}},draw=ca0a0a4,dash pattern=on 0.40pt off 0.80pt,line join=round,line cap=round,line width=0.400pt]
      \path[draw] (60.5000,152.5000) -- (60.5000,128.5000);



    \end{scope}
    \begin{scope}[cm={{0.7438,0.0,0.0,0.77563,(-75.95485,181.46752)}},draw=black,line join=round,line cap=round,line width=0.480pt]
      \path[draw] (60.5000,152.5000) -- (60.5000,147.5000);



      \path[draw] (60.5000,128.5000) -- (60.5000,132.5000);



    \end{scope}
    \begin{scope}[cm={{0.7438,0.0,0.0,0.77563,(-75.95485,181.46752)}},draw=ca0a0a4,dash pattern=on 0.40pt off 0.80pt,line join=round,line cap=round,line width=0.400pt]
      \path[draw] (95.5000,152.5000) -- (95.5000,128.5000);



    \end{scope}
    \begin{scope}[cm={{0.7438,0.0,0.0,0.77563,(-75.95485,181.46752)}},draw=black,line join=round,line cap=round,line width=0.480pt]
      \path[draw] (95.5000,152.5000) -- (95.5000,147.5000);



      \path[draw] (95.5000,128.5000) -- (95.5000,132.5000);



    \end{scope}
    \begin{scope}[cm={{0.7438,0.0,0.0,0.77563,(-75.95485,181.46752)}},draw=ca0a0a4,dash pattern=on 0.40pt off 0.80pt,line join=round,line cap=round,line width=0.400pt]
      \path[draw] (130.5000,140.5000) -- (130.5000,128.5000);



    \end{scope}
    \begin{scope}[cm={{1.0018,0.0,0.0,0.97485,(-98.43569,-63.97927)}},draw=ca0a0a4,dash pattern=on 0.40pt off 0.80pt,line join=round,line cap=round,line width=0.400pt]
      \path[draw] (41.5000,101.5000) -- (127.5000,101.5000);



    \end{scope}
    \begin{scope}[cm={{0.95389,0.0,0.0,0.95389,(-114.73226,308.36388)}},draw=black,line join=bevel,line cap=rect,line width=0.800pt]
      \begin{scope}[cm={{1.05023,0.0,0.0,1.00798,(-98.60108,-308.22806)}},draw=ca0a0a4,dash pattern=on 0.40pt off 0.80pt,line join=round,line cap=round,line width=0.400pt]
        \path[draw] (70.5000,84.5000) -- (70.5000,30.5000);



        \path[draw] (70.5000,14.5000) -- (70.5000,8.5000);



      \end{scope}
      \begin{scope}[cm={{1.05023,0.0,0.0,1.00798,(-98.60107,-309.29135)}},draw=ca0a0a4,dash pattern=on 0.40pt off 0.80pt,line join=round,line cap=round,line width=0.400pt]
        \path[draw] (98.5000,84.5000) -- (98.5000,30.5000);



        \path[draw] (98.5000,14.5000) -- (98.5000,8.5000);



      \end{scope}
      \path[fill=ce10000] (43.6345,6.5984) node[above right] (text1320) {\scriptsize -20};



      \begin{scope}[cm={{1.05023,0.0,0.0,1.00798,(17.08433,-309.29135)}},draw=black,line join=round,line cap=round,line width=0.480pt]
        \path[draw=cd9d9d9,line join=miter,line cap=rect,miter limit=4.00,line width=2.127pt] (41.5000,8.5000) -- (41.5000,84.5000) -- (127.5000,84.5000) -- (127.5000,8.5000) -- (41.5000,8.5000);



      \end{scope}
      \begin{scope}[cm={{1.05023,0.0,0.0,1.00798,(17.08433,-309.29135)}},draw=ca0a0a4,dash pattern=on 0.40pt off 0.80pt,line join=round,line cap=round,line width=0.400pt]
        \path[draw] (41.5000,74.5000) -- (127.5000,74.5000);



      \end{scope}
      \begin{scope}[cm={{1.05023,0.0,0.0,1.00798,(17.08433,-309.29135)}},draw=black,line join=round,line cap=round,line width=0.480pt]
        \path[draw] (41.5000,74.5000) -- (44.5000,74.5000);



        \path[draw] (127.5000,74.5000) -- (124.5000,74.5000);



      \end{scope}
      \begin{scope}[cm={{1.05023,0.0,0.0,1.05023,(45.92258,-230.55872)}},draw=black,line join=bevel,line cap=rect,line width=0.800pt]
        \path[fill=black] (0.0000,0.0000) node[above right] (text34-5) {\scriptsize 32};



      \end{scope}
      \begin{scope}[cm={{1.05023,0.0,0.0,1.00798,(17.08433,-309.29135)}},draw=ca0a0a4,dash pattern=on 0.40pt off 0.80pt,line join=round,line cap=round,line width=0.400pt]
        \path[draw] (41.5000,48.5000) -- (127.5000,48.5000);



      \end{scope}
      \begin{scope}[cm={{1.05023,0.0,0.0,1.00798,(17.08433,-309.29135)}},draw=black,line join=round,line cap=round,line width=0.480pt]
        \path[draw] (41.5000,48.5000) -- (44.5000,48.5000);



        \path[draw] (127.5000,48.5000) -- (124.5000,48.5000);



      \end{scope}
      \begin{scope}[cm={{1.05023,0.0,0.0,1.05023,(45.98139,-258.91487)}},draw=black,line join=bevel,line cap=rect,line width=0.800pt]
        \path[fill=black] (0.0000,0.0000) node[above right] (text64-7) {\scriptsize 36};



      \end{scope}
      \begin{scope}[cm={{1.05023,0.0,0.0,1.00798,(-90.78449,-310.11745)}},draw=ca0a0a4,dash pattern=on 0.40pt off 0.80pt,line join=round,line cap=round,line width=0.400pt]
        \path[shift={(-7.44276,0)},draw] (41.5000,22.5000) -- (46.5000,22.5000);



        \path[shift={(-7.44276,0)},draw] (103.5000,22.5000) -- (127.5000,22.5000);



      \end{scope}
      \begin{scope}[cm={{1.05023,0.0,0.0,1.00798,(17.08433,-309.29135)}},draw=black,line join=round,line cap=round,line width=0.480pt]
        \path[draw] (41.5000,22.5000) -- (44.5000,22.5000);



        \path[draw] (127.5000,22.5000) -- (124.5000,22.5000);



      \end{scope}
      \begin{scope}[cm={{1.05023,0.0,0.0,1.05023,(45.91417,-286.22079)}},draw=black,line join=bevel,line cap=rect,line width=0.800pt]
        \path[fill=black] (0.0000,0.0000) node[above right] (text96) {\scriptsize 40};



      \end{scope}
      \begin{scope}[cm={{1.05023,0.0,0.0,1.00798,(17.08433,-309.29135)}},draw=ca0a0a4,dash pattern=on 0.40pt off 0.80pt,line join=round,line cap=round,line width=0.400pt]
        \path[draw] (41.5000,84.5000) -- (41.5000,8.5000);



      \end{scope}
      \begin{scope}[cm={{1.05023,0.0,0.0,1.00798,(17.08433,-309.29135)}},draw=black,line join=round,line cap=round,line width=0.480pt]
        \path[draw] (41.5000,84.5000) -- (41.5000,80.5000);



        \path[draw] (41.5000,8.5000) -- (41.5000,11.5000);



      \end{scope}
      \begin{scope}[cm={{1.05023,0.0,0.0,1.00798,(17.08433,-309.29135)}},draw=black,line join=round,line cap=round,line width=0.480pt]
        \path[draw] (70.5000,84.5000) -- (70.5000,80.5000);



        \path[draw] (70.5000,8.5000) -- (70.5000,11.5000);



      \end{scope}
      \begin{scope}[cm={{1.05023,0.0,0.0,1.00798,(17.08433,-309.29135)}},draw=black,line join=round,line cap=round,line width=0.480pt]
        \path[draw] (98.5000,84.5000) -- (98.5000,80.5000);



        \path[draw] (98.5000,8.5000) -- (98.5000,11.5000);



      \end{scope}
      \begin{scope}[cm={{1.05023,0.0,0.0,1.00798,(17.08433,-309.29135)}},draw=ca0a0a4,dash pattern=on 0.40pt off 0.80pt,line join=round,line cap=round,line width=0.400pt]
        \path[draw] (127.5000,84.5000) -- (127.5000,8.5000);



      \end{scope}
      \begin{scope}[cm={{1.05023,0.0,0.0,1.00798,(17.08433,-309.29135)}},draw=black,line join=round,line cap=round,line width=0.480pt]
        \path[draw] (127.5000,84.5000) -- (127.5000,80.5000);



        \path[draw] (127.5000,8.5000) -- (127.5000,11.5000);



      \end{scope}
      \begin{scope}[cm={{1.05023,0.0,0.0,1.00798,(-44.63034,-287.23447)}},draw=black,line join=bevel,line cap=rect,line width=0.800pt]
        \path[fill=black] (0.0000,0.0000) node[above right] (text276) {\scriptsize $\Upsilon(i)$};



      \end{scope}
      \begin{scope}[cm={{1.05023,0.0,0.0,1.00798,(-98.60107,-309.29135)}},draw=black,line join=round,line cap=round,line width=0.480pt]
        \path[draw,even odd rule] (71.5000,18.5000) -- (98.5000,18.5000);



      \end{scope}
      \begin{scope}[cm={{1.05023,0.0,0.0,1.00798,(17.08433,-309.29135)}},draw=black,line join=round,line cap=round,line width=0.480pt]
        \path[draw] (41.6000,17.0000) -- (41.6000,17.0000) -- (41.7000,17.8000) -- (41.9000,18.6000) -- (42.0000,19.4000) -- (42.2000,20.2000) -- (42.3000,20.9000) -- (42.5000,21.6000) -- (42.6000,22.4000) -- (42.7000,23.1000) -- (42.9000,23.8000) -- (43.0000,24.5000) -- (43.2000,25.2000) -- (43.3000,25.8000) -- (43.5000,26.5000) -- (43.6000,27.1000) -- (43.7000,27.7000) -- (43.9000,28.4000) -- (44.0000,29.0000) -- (44.2000,29.6000) -- (44.3000,30.1000) -- (44.5000,30.7000) -- (44.6000,31.3000) -- (44.7000,31.8000) -- (44.9000,32.4000) -- (45.0000,32.9000) -- (45.2000,33.4000) -- (45.3000,34.0000) -- (45.5000,34.5000) -- (45.6000,35.0000) -- (45.7000,35.5000) -- (45.9000,35.9000) -- (46.0000,36.4000) -- (46.2000,36.9000) -- (46.3000,37.3000) -- (46.5000,37.8000) -- (46.6000,38.2000) -- (46.7000,38.7000) -- (46.9000,39.1000) -- (47.0000,39.5000) -- (47.2000,39.9000) -- (47.3000,40.3000) -- (47.5000,40.7000) -- (47.6000,41.1000) -- (47.7000,41.5000) -- (47.9000,41.9000) -- (48.0000,42.2000) -- (48.2000,42.6000) -- (48.3000,43.0000) -- (48.5000,43.3000) -- (48.6000,43.6000) -- (48.7000,44.0000) -- (48.9000,44.3000) -- (49.0000,44.6000) -- (49.2000,45.0000) -- (49.3000,45.3000) -- (49.5000,45.6000) -- (49.6000,45.9000) -- (49.7000,46.2000) -- (49.9000,46.5000) -- (50.0000,46.8000) -- (50.2000,47.0000) -- (50.3000,47.3000) -- (50.5000,47.6000) -- (50.6000,47.8000) -- (50.7000,48.1000) -- (50.9000,48.4000) -- (51.0000,48.6000) -- (51.2000,48.8000) -- (51.3000,49.1000) -- (51.5000,49.3000) -- (51.6000,49.5000) -- (51.7000,49.8000) -- (51.9000,50.0000) -- (52.0000,50.2000) -- (52.2000,50.4000) -- (52.3000,50.6000) -- (52.5000,50.8000) -- (52.6000,51.0000) -- (52.7000,51.2000) -- (52.9000,51.4000) -- (53.0000,51.6000) -- (53.2000,51.7000) -- (53.3000,51.9000) -- (53.5000,52.1000) -- (53.6000,52.3000) -- (53.7000,52.4000) -- (53.9000,52.6000) -- (54.0000,52.7000) -- (54.2000,52.9000) -- (54.3000,53.0000) -- (54.5000,53.1000) -- (54.6000,53.3000) -- (54.7000,53.4000) -- (54.9000,53.5000) -- (55.0000,53.7000) -- (55.2000,53.8000) -- (55.3000,53.9000) -- (55.5000,54.0000) -- (55.6000,54.1000) -- (55.7000,54.2000) -- (55.9000,54.3000) -- (56.0000,54.4000) -- (56.2000,54.5000) -- (56.3000,54.6000) -- (56.5000,54.6000) -- (56.6000,54.7000) -- (56.7000,54.8000) -- (56.9000,54.9000) -- (57.0000,54.9000) -- (57.2000,55.0000) -- (57.3000,55.0000) -- (57.5000,55.1000) -- (57.6000,55.1000) -- (57.7000,55.2000) -- (57.9000,55.2000) -- (58.0000,55.2000) -- (58.2000,55.3000) -- (58.3000,55.3000) -- (58.5000,55.3000) -- (58.6000,55.3000) -- (58.7000,55.3000) -- (58.9000,55.3000) -- (59.0000,55.3000) -- (59.2000,55.3000) -- (59.3000,55.3000) -- (59.5000,55.3000) -- (59.6000,55.2000) -- (59.7000,55.2000) -- (59.9000,55.2000) -- (60.0000,55.1000) -- (60.2000,55.1000) -- (60.3000,55.0000) -- (60.5000,54.9000) -- (60.6000,54.9000) -- (60.7000,54.8000) -- (60.9000,54.7000) -- (61.0000,54.6000) -- (61.2000,54.5000) -- (61.3000,54.4000) -- (61.5000,54.2000) -- (61.6000,54.1000) -- (61.7000,54.0000) -- (61.9000,53.8000) -- (62.0000,53.7000) -- (62.2000,53.5000) -- (62.3000,53.3000) -- (62.5000,53.1000) -- (62.6000,53.0000) -- (62.7000,52.8000) -- (62.9000,52.6000) -- (63.0000,52.4000) -- (63.2000,52.2000) -- (63.3000,52.0000) -- (63.5000,51.7000) -- (63.6000,51.5000) -- (63.7000,51.3000) -- (63.9000,51.1000) -- (64.0000,50.8000) -- (64.2000,50.6000) -- (64.3000,50.4000) -- (64.5000,50.1000) -- (64.6000,49.9000) -- (64.7000,49.7000) -- (64.9000,49.4000) -- (65.0000,49.2000) -- (65.2000,49.0000) -- (65.3000,48.7000) -- (65.5000,48.5000) -- (65.6000,48.2000) -- (65.7000,48.0000) -- (65.9000,47.8000) -- (66.0000,47.5000) -- (66.2000,47.3000) -- (66.3000,47.1000) -- (66.5000,46.8000) -- (66.6000,46.6000) -- (66.7000,46.4000) -- (66.9000,46.1000) -- (67.0000,45.9000) -- (67.2000,45.7000) -- (67.3000,45.4000) -- (67.5000,45.2000) -- (67.6000,45.0000) -- (67.7000,44.8000) -- (67.9000,44.5000) -- (68.0000,44.3000) -- (68.2000,44.1000) -- (68.3000,43.9000) -- (68.5000,43.7000) -- (68.6000,43.5000) -- (68.7000,43.2000) -- (68.9000,43.0000) -- (69.0000,42.8000) -- (69.2000,42.6000) -- (69.3000,42.4000) -- (69.5000,42.2000) -- (69.6000,42.0000) -- (69.7000,41.8000) -- (69.9000,41.6000) -- (70.0000,41.4000) -- (70.2000,41.2000) -- (70.3000,41.0000) -- (70.5000,40.9000) -- (70.6000,40.7000) -- (70.7000,40.5000) -- (70.9000,40.3000) -- (71.0000,40.1000) -- (71.2000,40.0000) -- (71.3000,39.8000) -- (71.5000,39.6000) -- (71.6000,39.4000) -- (71.7000,39.3000) -- (71.9000,39.1000) -- (72.0000,38.9000) -- (72.2000,38.8000) -- (72.3000,38.6000) -- (72.5000,38.5000) -- (72.6000,38.3000) -- (72.7000,38.1000) -- (72.9000,38.0000) -- (73.0000,37.9000) -- (73.2000,37.7000) -- (73.3000,37.6000) -- (73.5000,37.4000) -- (73.6000,37.3000) -- (73.7000,37.1000) -- (73.9000,37.0000) -- (74.0000,36.9000) -- (74.2000,36.8000) -- (74.3000,36.6000) -- (74.5000,36.5000) -- (74.6000,36.4000) -- (74.7000,36.2000) -- (74.9000,36.1000) -- (75.0000,36.0000) -- (75.2000,35.9000) -- (75.3000,35.8000) -- (75.5000,35.7000) -- (75.6000,35.6000) -- (75.7000,35.5000) -- (75.9000,35.4000) -- (76.0000,35.3000) -- (76.2000,35.2000) -- (76.3000,35.1000) -- (76.5000,35.0000) -- (76.6000,34.9000) -- (76.7000,34.8000) -- (76.9000,34.7000) -- (77.0000,34.6000) -- (77.2000,34.5000) -- (77.3000,34.4000) -- (77.5000,34.4000) -- (77.6000,34.3000) -- (77.7000,34.2000) -- (77.9000,34.1000) -- (78.0000,34.1000) -- (78.2000,34.0000) -- (78.3000,33.9000) -- (78.5000,33.9000) -- (78.6000,33.8000) -- (78.7000,33.7000) -- (78.9000,33.7000) -- (79.0000,33.6000) -- (79.2000,33.6000) -- (79.3000,33.5000) -- (79.5000,33.5000) -- (79.6000,33.4000) -- (79.7000,33.4000) -- (79.9000,33.3000) -- (80.0000,33.3000) -- (80.2000,33.2000) -- (80.3000,33.2000) -- (80.5000,33.1000) -- (80.6000,33.1000) -- (80.7000,33.1000) -- (80.9000,33.0000) -- (81.0000,33.0000) -- (81.2000,33.0000) -- (81.3000,32.9000) -- (81.5000,32.9000) -- (81.6000,32.9000) -- (81.7000,32.9000) -- (81.9000,32.8000) -- (82.0000,32.8000) -- (82.2000,32.8000) -- (82.3000,32.8000) -- (82.5000,32.8000) -- (82.6000,32.7000) -- (82.7000,32.7000) -- (82.9000,32.7000) -- (83.0000,32.7000) -- (83.2000,32.7000) -- (83.3000,32.7000) -- (83.5000,32.7000) -- (83.6000,32.7000) -- (83.7000,32.7000) -- (83.9000,32.6000) -- (84.0000,32.6000) -- (84.2000,32.6000) -- (84.3000,32.6000) -- (84.5000,32.6000) -- (84.6000,32.6000) -- (84.7000,32.6000) -- (84.9000,32.7000) -- (85.0000,32.7000) -- (85.2000,32.7000) -- (85.3000,32.7000) -- (85.4000,32.7000) -- (85.6000,32.7000) -- (85.7000,32.7000) -- (85.9000,32.7000) -- (86.0000,32.7000) -- (86.2000,32.7000) -- (86.3000,32.8000) -- (86.4000,32.8000) -- (86.6000,32.8000) -- (86.7000,32.8000) -- (86.9000,32.8000) -- (87.0000,32.8000) -- (87.2000,32.9000) -- (87.3000,32.9000) -- (87.4000,32.9000) -- (87.6000,32.9000) -- (87.7000,33.0000) -- (87.9000,33.0000) -- (88.0000,33.0000) -- (88.2000,33.0000) -- (88.3000,33.1000) -- (88.4000,33.1000) -- (88.6000,33.1000) -- (88.7000,33.1000) -- (88.9000,33.2000) -- (89.0000,33.2000) -- (89.2000,33.2000) -- (89.3000,33.3000) -- (89.4000,33.3000) -- (89.6000,33.3000) -- (89.7000,33.4000) -- (89.9000,33.4000) -- (90.0000,33.4000) -- (90.2000,33.5000) -- (90.3000,33.5000) -- (90.4000,33.6000) -- (90.6000,33.6000) -- (90.7000,33.6000) -- (90.9000,33.7000) -- (91.0000,33.7000) -- (91.2000,33.8000) -- (91.3000,33.8000) -- (91.4000,33.8000) -- (91.6000,33.9000) -- (91.7000,33.9000) -- (91.9000,34.0000) -- (92.0000,34.0000) -- (92.2000,34.1000) -- (92.3000,34.1000) -- (92.4000,34.1000) -- (92.6000,34.2000) -- (92.7000,34.2000) -- (92.9000,34.3000) -- (93.0000,34.3000) -- (93.2000,34.4000) -- (93.3000,34.4000) -- (93.4000,34.5000) -- (93.6000,34.5000) -- (93.7000,34.6000) -- (93.9000,34.6000) -- (94.0000,34.7000) -- (94.2000,34.7000) -- (94.3000,34.8000) -- (94.4000,34.8000) -- (94.6000,34.9000) -- (94.7000,34.9000) -- (94.9000,35.0000) -- (95.0000,35.0000) -- (95.2000,35.1000) -- (95.3000,35.1000) -- (95.4000,35.2000) -- (95.6000,35.2000) -- (95.7000,35.3000) -- (95.9000,35.3000) -- (96.0000,35.4000) -- (96.2000,35.4000) -- (96.3000,35.5000) -- (96.4000,35.5000) -- (96.6000,35.6000) -- (96.7000,35.6000) -- (96.9000,35.7000) -- (97.0000,35.7000) -- (97.2000,35.8000) -- (97.3000,35.9000) -- (97.4000,35.9000) -- (97.6000,36.0000) -- (97.7000,36.0000) -- (97.9000,36.1000) -- (98.0000,36.1000) -- (98.2000,36.2000) -- (98.3000,36.2000) -- (98.4000,36.3000) -- (98.6000,36.4000) -- (98.7000,36.4000) -- (98.9000,36.5000) -- (99.0000,36.6000) -- (99.2000,36.6000) -- (99.3000,36.7000) -- (99.4000,36.8000) -- (99.6000,36.8000) -- (99.7000,36.9000) -- (99.9000,37.0000) -- (100.0000,37.0000) -- (100.2000,37.1000) -- (100.3000,37.2000) -- (100.4000,37.2000) -- (100.6000,37.3000) -- (100.7000,37.4000) -- (100.9000,37.4000) -- (101.0000,37.5000) -- (101.2000,37.6000) -- (101.3000,37.6000) -- (101.4000,37.7000) -- (101.6000,37.8000) -- (101.7000,37.9000) -- (101.9000,37.9000) -- (102.0000,38.0000) -- (102.2000,38.1000) -- (102.3000,38.2000) -- (102.4000,38.2000) -- (102.6000,38.3000) -- (102.7000,38.4000) -- (102.9000,38.4000) -- (103.0000,38.5000) -- (103.2000,38.6000) -- (103.3000,38.7000) -- (103.4000,38.7000) -- (103.6000,38.8000) -- (103.7000,38.9000) -- (103.9000,38.9000) -- (104.0000,39.0000) -- (104.2000,39.1000) -- (104.3000,39.2000) -- (104.4000,39.2000) -- (104.6000,39.3000) -- (104.7000,39.4000) -- (104.9000,39.4000) -- (105.0000,39.5000) -- (105.2000,39.6000) -- (105.3000,39.7000) -- (105.4000,39.7000) -- (105.6000,39.8000) -- (105.7000,39.9000) -- (105.9000,39.9000) -- (106.0000,40.0000) -- (106.2000,40.1000) -- (106.3000,40.1000) -- (106.4000,40.2000) -- (106.6000,40.3000) -- (106.7000,40.3000) -- (106.9000,40.4000) -- (107.0000,40.5000) -- (107.2000,40.5000) -- (107.3000,40.6000) -- (107.4000,40.6000) -- (107.6000,40.7000) -- (107.7000,40.8000) -- (107.9000,40.8000) -- (108.0000,40.9000) -- (108.2000,41.0000) -- (108.3000,41.0000) -- (108.4000,41.1000) -- (108.6000,41.1000) -- (108.7000,41.2000) -- (108.9000,41.3000) -- (109.0000,41.3000) -- (109.2000,41.4000) -- (109.3000,41.4000) -- (109.4000,41.5000) -- (109.6000,41.5000) -- (109.7000,41.6000) -- (109.9000,41.6000) -- (110.0000,41.7000) -- (110.2000,41.7000) -- (110.3000,41.8000) -- (110.4000,41.9000) -- (110.6000,41.9000) -- (110.7000,42.0000) -- (110.9000,42.0000) -- (111.0000,42.1000) -- (111.2000,42.1000) -- (111.3000,42.1000) -- (111.4000,42.2000) -- (111.6000,42.2000) -- (111.7000,42.3000) -- (111.9000,42.3000) -- (112.0000,42.4000) -- (112.2000,42.4000) -- (112.3000,42.5000) -- (112.4000,42.5000) -- (112.6000,42.6000) -- (112.7000,42.6000) -- (112.9000,42.6000) -- (113.0000,42.7000) -- (113.2000,42.7000) -- (113.3000,42.8000) -- (113.4000,42.8000) -- (113.6000,42.8000) -- (113.7000,42.9000) -- (113.9000,42.9000) -- (114.0000,43.0000) -- (114.2000,43.0000) -- (114.3000,43.0000) -- (114.4000,43.1000) -- (114.6000,43.1000) -- (114.7000,43.1000) -- (114.9000,43.2000) -- (115.0000,43.2000) -- (115.2000,43.2000) -- (115.3000,43.3000) -- (115.4000,43.3000) -- (115.6000,43.3000) -- (115.7000,43.4000) -- (115.9000,43.4000) -- (116.0000,43.4000) -- (116.2000,43.4000) -- (116.3000,43.5000) -- (116.4000,43.5000) -- (116.6000,43.5000) -- (116.7000,43.6000) -- (116.9000,43.6000) -- (117.0000,43.6000) -- (117.2000,43.6000) -- (117.3000,43.7000) -- (117.4000,43.7000) -- (117.6000,43.7000) -- (117.7000,43.7000) -- (117.9000,43.7000) -- (118.0000,43.8000) -- (118.2000,43.8000) -- (118.3000,43.8000) -- (118.4000,43.8000) -- (118.6000,43.9000) -- (118.7000,43.9000) -- (118.9000,43.9000) -- (119.0000,43.9000) -- (119.2000,43.9000) -- (119.3000,44.0000) -- (119.4000,44.0000) -- (119.6000,44.0000) -- (119.7000,44.0000) -- (119.9000,44.0000) -- (120.0000,44.0000) -- (120.2000,44.0000) -- (120.3000,44.1000) -- (120.4000,44.1000) -- (120.6000,44.1000) -- (120.7000,44.1000) -- (120.9000,44.1000) -- (121.0000,44.1000) -- (121.2000,44.1000) -- (121.3000,44.2000) -- (121.4000,44.2000) -- (121.6000,44.2000) -- (121.7000,44.2000) -- (121.9000,44.2000) -- (122.0000,44.2000) -- (122.2000,44.2000) -- (122.3000,44.2000) -- (122.4000,44.2000) -- (122.6000,44.2000) -- (122.7000,44.3000) -- (122.9000,44.3000) -- (123.0000,44.3000) -- (123.2000,44.3000) -- (123.3000,44.3000) -- (123.4000,44.3000) -- (123.6000,44.3000) -- (123.7000,44.3000) -- (123.9000,44.3000) -- (124.0000,44.3000) -- (124.2000,44.3000) -- (124.3000,44.3000) -- (124.4000,44.3000) -- (124.6000,44.3000) -- (124.7000,44.3000) -- (124.9000,44.3000) -- (125.0000,44.3000) -- (125.2000,44.3000) -- (125.3000,44.3000) -- (125.4000,44.4000) -- (125.6000,44.4000) -- (125.7000,44.4000) -- (125.9000,44.4000) -- (126.0000,44.4000) -- (126.2000,44.4000) -- (126.3000,44.4000) -- (126.4000,44.4000) -- (126.6000,44.4000) -- (126.7000,44.4000) -- (126.9000,44.4000) -- (127.0000,44.4000) -- (127.2000,44.4000) -- (127.3000,44.4000);



      \end{scope}
      \begin{scope}[cm={{1.05023,0.0,0.0,1.00798,(-42.24248,-276.60724)}},draw=black,line join=bevel,line cap=rect,line width=0.800pt]
        \path[fill=black] (0.0000,0.0000) node[above right] (text312) {\scriptsize $\hat{y}(i)$};



      \end{scope}
      \begin{scope}[cm={{1.05023,0.0,0.0,1.00798,(-98.60107,-309.29135)}},draw=cff0000,line join=round,line cap=round,line width=0.480pt]
        \path[draw,even odd rule] (71.5000,26.5000) -- (98.5000,26.5000);



      \end{scope}
      \begin{scope}[cm={{1.05023,0.0,0.0,1.00798,(17.08433,-309.29135)}},draw=cff0000,line join=round,line cap=round,line width=0.480pt]
        \path[draw] (41.6000,21.8000) -- (41.6000,21.8000) -- (41.7000,17.5000) -- (41.9000,18.2000) -- (42.0000,19.1000) -- (42.2000,20.0000) -- (42.3000,20.9000) -- (42.5000,21.7000) -- (42.6000,22.5000) -- (42.7000,23.3000) -- (42.9000,24.1000) -- (43.0000,24.8000) -- (43.2000,25.5000) -- (43.3000,26.2000) -- (43.5000,26.9000) -- (43.6000,27.6000) -- (43.7000,28.2000) -- (43.9000,28.8000) -- (44.0000,29.4000) -- (44.2000,30.0000) -- (44.3000,30.5000) -- (44.5000,31.1000) -- (44.6000,31.6000) -- (44.7000,32.1000) -- (44.9000,32.7000) -- (45.0000,33.2000) -- (45.2000,33.7000) -- (45.3000,34.2000) -- (45.5000,34.6000) -- (45.6000,35.1000) -- (45.7000,35.6000) -- (45.9000,36.1000) -- (46.0000,36.5000) -- (46.2000,37.0000) -- (46.3000,37.5000) -- (46.5000,37.9000) -- (46.6000,38.4000) -- (46.7000,38.8000) -- (46.9000,39.3000) -- (47.0000,39.7000) -- (47.2000,40.1000) -- (47.3000,40.6000) -- (47.5000,41.0000) -- (47.6000,41.4000) -- (47.7000,41.8000) -- (47.9000,42.2000) -- (48.0000,42.5000) -- (48.2000,42.9000) -- (48.3000,43.3000) -- (48.5000,43.6000) -- (48.6000,43.9000) -- (48.7000,44.3000) -- (48.9000,44.6000) -- (49.0000,44.9000) -- (49.2000,45.2000) -- (49.3000,45.5000) -- (49.5000,45.8000) -- (49.6000,46.0000) -- (49.7000,46.3000) -- (49.9000,46.6000) -- (50.0000,46.8000) -- (50.2000,47.1000) -- (50.3000,47.4000) -- (50.5000,47.6000) -- (50.6000,47.9000) -- (50.7000,48.1000) -- (50.9000,48.4000) -- (51.0000,48.7000) -- (51.2000,48.9000) -- (51.3000,49.2000) -- (51.5000,49.4000) -- (51.6000,49.7000) -- (51.7000,50.0000) -- (51.9000,50.2000) -- (52.0000,50.5000) -- (52.2000,50.8000) -- (52.3000,51.0000) -- (52.5000,51.3000) -- (52.6000,51.5000) -- (52.7000,51.7000) -- (52.9000,51.9000) -- (53.0000,52.1000) -- (53.2000,52.3000) -- (53.3000,52.5000) -- (53.5000,52.7000) -- (53.6000,52.8000) -- (53.7000,52.9000) -- (53.9000,53.1000) -- (54.0000,53.2000) -- (54.2000,53.3000) -- (54.3000,53.3000) -- (54.5000,53.4000) -- (54.6000,53.5000) -- (54.7000,53.5000) -- (54.9000,53.6000) -- (55.0000,53.6000) -- (55.2000,53.6000) -- (55.3000,53.7000) -- (55.5000,53.7000) -- (55.6000,53.7000) -- (55.7000,53.8000) -- (55.9000,53.8000) -- (56.0000,53.8000) -- (56.2000,53.9000) -- (56.3000,54.0000) -- (56.5000,54.0000) -- (56.6000,54.1000) -- (56.7000,54.2000) -- (56.9000,54.3000) -- (57.0000,54.4000) -- (57.2000,54.5000) -- (57.3000,54.6000) -- (57.5000,54.7000) -- (57.6000,54.8000) -- (57.7000,54.9000) -- (57.9000,55.0000) -- (58.0000,55.1000) -- (58.2000,55.2000) -- (58.3000,55.3000) -- (58.5000,55.3000) -- (58.6000,55.4000) -- (58.7000,55.4000) -- (58.9000,55.5000) -- (59.0000,55.5000) -- (59.2000,55.5000) -- (59.3000,55.5000) -- (59.5000,55.5000) -- (59.6000,55.5000) -- (59.7000,55.5000) -- (59.9000,55.4000) -- (60.0000,55.3000) -- (60.2000,55.3000) -- (60.3000,55.2000) -- (60.5000,55.1000) -- (60.6000,55.0000) -- (60.7000,54.9000) -- (60.9000,54.8000) -- (61.0000,54.6000) -- (61.2000,54.5000) -- (61.3000,54.4000) -- (61.5000,54.2000) -- (61.6000,54.0000) -- (61.7000,53.9000) -- (61.9000,53.7000) -- (62.0000,53.5000) -- (62.2000,53.3000) -- (62.3000,53.1000) -- (62.5000,52.9000) -- (62.6000,52.7000) -- (62.7000,52.5000) -- (62.9000,52.3000) -- (63.0000,52.1000) -- (63.2000,51.9000) -- (63.3000,51.7000) -- (63.5000,51.5000) -- (63.6000,51.3000) -- (63.7000,51.1000) -- (63.9000,50.9000) -- (64.0000,50.7000) -- (64.2000,50.5000) -- (64.3000,50.3000) -- (64.5000,50.0000) -- (64.6000,49.8000) -- (64.7000,49.6000) -- (64.9000,49.4000) -- (65.0000,49.2000) -- (65.2000,49.0000) -- (65.3000,48.8000) -- (65.5000,48.6000) -- (65.6000,48.3000) -- (65.7000,48.1000) -- (65.9000,47.9000) -- (66.0000,47.7000) -- (66.2000,47.5000) -- (66.3000,47.3000) -- (66.5000,47.0000) -- (66.6000,46.8000) -- (66.7000,46.6000) -- (66.9000,46.4000) -- (67.0000,46.2000) -- (67.2000,45.9000) -- (67.3000,45.7000) -- (67.5000,45.5000) -- (67.6000,45.3000) -- (67.7000,45.0000) -- (67.9000,44.8000) -- (68.0000,44.6000) -- (68.2000,44.3000) -- (68.3000,44.1000) -- (68.5000,43.9000) -- (68.6000,43.6000) -- (68.7000,43.4000) -- (68.9000,43.2000) -- (69.0000,43.0000) -- (69.2000,42.7000) -- (69.3000,42.5000) -- (69.5000,42.3000) -- (69.6000,42.1000) -- (69.7000,41.8000) -- (69.9000,41.6000) -- (70.0000,41.4000) -- (70.2000,41.2000) -- (70.3000,41.0000) -- (70.5000,40.8000) -- (70.6000,40.6000) -- (70.7000,40.4000) -- (70.9000,40.2000) -- (71.0000,40.0000) -- (71.2000,39.8000) -- (71.3000,39.6000) -- (71.5000,39.4000) -- (71.6000,39.2000) -- (71.7000,39.0000) -- (71.9000,38.9000) -- (72.0000,38.7000) -- (72.2000,38.5000) -- (72.3000,38.3000) -- (72.5000,38.2000) -- (72.6000,38.0000) -- (72.7000,37.9000) -- (72.9000,37.7000) -- (73.0000,37.6000) -- (73.2000,37.4000) -- (73.3000,37.3000) -- (73.5000,37.2000) -- (73.6000,37.0000) -- (73.7000,36.9000) -- (73.9000,36.8000) -- (74.0000,36.7000) -- (74.2000,36.5000) -- (74.3000,36.4000) -- (74.5000,36.3000) -- (74.6000,36.2000) -- (74.7000,36.1000) -- (74.9000,36.0000) -- (75.0000,35.9000) -- (75.2000,35.8000) -- (75.3000,35.7000) -- (75.5000,35.6000) -- (75.6000,35.5000) -- (75.7000,35.4000) -- (75.9000,35.3000) -- (76.0000,35.3000) -- (76.2000,35.2000) -- (76.3000,35.1000) -- (76.5000,35.0000) -- (76.6000,34.9000) -- (76.7000,34.9000) -- (76.9000,34.8000) -- (77.0000,34.7000) -- (77.2000,34.6000) -- (77.3000,34.6000) -- (77.5000,34.5000) -- (77.6000,34.4000) -- (77.7000,34.4000) -- (77.9000,34.3000) -- (78.0000,34.3000) -- (78.2000,34.2000) -- (78.3000,34.1000) -- (78.5000,34.1000) -- (78.6000,34.0000) -- (78.7000,34.0000) -- (78.9000,33.9000) -- (79.0000,33.9000) -- (79.2000,33.8000) -- (79.3000,33.7000) -- (79.5000,33.7000) -- (79.6000,33.6000) -- (79.7000,33.6000) -- (79.9000,33.5000) -- (80.0000,33.5000) -- (80.2000,33.5000) -- (80.3000,33.4000) -- (80.5000,33.4000) -- (80.6000,33.3000) -- (80.7000,33.3000) -- (80.9000,33.2000) -- (81.0000,33.2000) -- (81.2000,33.2000) -- (81.3000,33.1000) -- (81.5000,33.1000) -- (81.6000,33.0000) -- (81.7000,33.0000) -- (81.9000,33.0000) -- (82.0000,32.9000) -- (82.2000,32.9000) -- (82.3000,32.9000) -- (82.5000,32.9000) -- (82.6000,32.8000) -- (82.7000,32.8000) -- (82.9000,32.8000) -- (83.0000,32.8000) -- (83.2000,32.7000) -- (83.3000,32.7000) -- (83.5000,32.7000) -- (83.6000,32.7000) -- (83.7000,32.7000) -- (83.9000,32.7000) -- (84.0000,32.6000) -- (84.2000,32.6000) -- (84.3000,32.6000) -- (84.5000,32.6000) -- (84.6000,32.6000) -- (84.7000,32.6000) -- (84.9000,32.6000) -- (85.0000,32.6000) -- (85.2000,32.6000) -- (85.3000,32.6000) -- (85.4000,32.6000) -- (85.6000,32.6000) -- (85.7000,32.6000) -- (85.9000,32.6000) -- (86.0000,32.6000) -- (86.2000,32.6000) -- (86.3000,32.6000) -- (86.4000,32.7000) -- (86.6000,32.7000) -- (86.7000,32.7000) -- (86.9000,32.7000) -- (87.0000,32.7000) -- (87.2000,32.7000) -- (87.3000,32.8000) -- (87.4000,32.8000) -- (87.6000,32.8000) -- (87.7000,32.8000) -- (87.9000,32.8000) -- (88.0000,32.9000) -- (88.2000,32.9000) -- (88.3000,32.9000) -- (88.4000,33.0000) -- (88.6000,33.0000) -- (88.7000,33.0000) -- (88.9000,33.1000) -- (89.0000,33.1000) -- (89.2000,33.1000) -- (89.3000,33.2000) -- (89.4000,33.2000) -- (89.6000,33.2000) -- (89.7000,33.3000) -- (89.9000,33.3000) -- (90.0000,33.4000) -- (90.2000,33.4000) -- (90.3000,33.5000) -- (90.4000,33.5000) -- (90.6000,33.5000) -- (90.7000,33.6000) -- (90.9000,33.6000) -- (91.0000,33.7000) -- (91.2000,33.7000) -- (91.3000,33.8000) -- (91.4000,33.8000) -- (91.6000,33.9000) -- (91.7000,33.9000) -- (91.9000,34.0000) -- (92.0000,34.0000) -- (92.2000,34.1000) -- (92.3000,34.1000) -- (92.4000,34.2000) -- (92.6000,34.2000) -- (92.7000,34.3000) -- (92.9000,34.3000) -- (93.0000,34.4000) -- (93.2000,34.4000) -- (93.3000,34.5000) -- (93.4000,34.5000) -- (93.6000,34.6000) -- (93.7000,34.6000) -- (93.9000,34.7000) -- (94.0000,34.7000) -- (94.2000,34.8000) -- (94.3000,34.8000) -- (94.4000,34.9000) -- (94.6000,34.9000) -- (94.7000,35.0000) -- (94.9000,35.1000) -- (95.0000,35.1000) -- (95.2000,35.2000) -- (95.3000,35.2000) -- (95.4000,35.3000) -- (95.6000,35.3000) -- (95.7000,35.4000) -- (95.9000,35.4000) -- (96.0000,35.5000) -- (96.2000,35.5000) -- (96.3000,35.6000) -- (96.4000,35.6000) -- (96.6000,35.7000) -- (96.7000,35.7000) -- (96.9000,35.8000) -- (97.0000,35.9000) -- (97.2000,35.9000) -- (97.3000,36.0000) -- (97.4000,36.0000) -- (97.6000,36.1000) -- (97.7000,36.1000) -- (97.9000,36.2000) -- (98.0000,36.2000) -- (98.2000,36.3000) -- (98.3000,36.3000) -- (98.4000,36.4000) -- (98.6000,36.5000) -- (98.7000,36.5000) -- (98.9000,36.6000) -- (99.0000,36.6000) -- (99.2000,36.7000) -- (99.3000,36.8000) -- (99.4000,36.8000) -- (99.6000,36.9000) -- (99.7000,37.0000) -- (99.9000,37.0000) -- (100.0000,37.1000) -- (100.2000,37.1000) -- (100.3000,37.2000) -- (100.4000,37.3000) -- (100.6000,37.3000) -- (100.7000,37.4000) -- (100.9000,37.5000) -- (101.0000,37.5000) -- (101.2000,37.6000) -- (101.3000,37.7000) -- (101.4000,37.7000) -- (101.6000,37.8000) -- (101.7000,37.9000) -- (101.9000,37.9000) -- (102.0000,38.0000) -- (102.2000,38.1000) -- (102.3000,38.1000) -- (102.4000,38.2000) -- (102.6000,38.3000) -- (102.7000,38.4000) -- (102.9000,38.4000) -- (103.0000,38.5000) -- (103.2000,38.6000) -- (103.3000,38.6000) -- (103.4000,38.7000) -- (103.6000,38.8000) -- (103.7000,38.8000) -- (103.9000,38.9000) -- (104.0000,39.0000) -- (104.2000,39.0000) -- (104.3000,39.1000) -- (104.4000,39.2000) -- (104.6000,39.2000) -- (104.7000,39.3000) -- (104.9000,39.4000) -- (105.0000,39.4000) -- (105.2000,39.5000) -- (105.3000,39.6000) -- (105.4000,39.6000) -- (105.6000,39.7000) -- (105.7000,39.8000) -- (105.9000,39.8000) -- (106.0000,39.9000) -- (106.2000,40.0000) -- (106.3000,40.0000) -- (106.4000,40.1000) -- (106.6000,40.2000) -- (106.7000,40.2000) -- (106.9000,40.3000) -- (107.0000,40.4000) -- (107.2000,40.4000) -- (107.3000,40.5000) -- (107.4000,40.5000) -- (107.6000,40.6000) -- (107.7000,40.7000) -- (107.9000,40.7000) -- (108.0000,40.8000) -- (108.2000,40.9000) -- (108.3000,40.9000) -- (108.4000,41.0000) -- (108.6000,41.0000) -- (108.7000,41.1000) -- (108.9000,41.2000) -- (109.0000,41.2000) -- (109.2000,41.3000) -- (109.3000,41.3000) -- (109.4000,41.4000) -- (109.6000,41.4000) -- (109.7000,41.5000) -- (109.9000,41.5000) -- (110.0000,41.6000) -- (110.2000,41.7000) -- (110.3000,41.7000) -- (110.4000,41.8000) -- (110.6000,41.8000) -- (110.7000,41.9000) -- (110.9000,41.9000) -- (111.0000,42.0000) -- (111.2000,42.0000) -- (111.3000,42.1000) -- (111.4000,42.1000) -- (111.6000,42.2000) -- (111.7000,42.2000) -- (111.9000,42.3000) -- (112.0000,42.3000) -- (112.2000,42.4000) -- (112.3000,42.4000) -- (112.4000,42.5000) -- (112.6000,42.5000) -- (112.7000,42.6000) -- (112.9000,42.6000) -- (113.0000,42.6000) -- (113.2000,42.7000) -- (113.3000,42.7000) -- (113.4000,42.8000) -- (113.6000,42.8000) -- (113.7000,42.9000) -- (113.9000,42.9000) -- (114.0000,42.9000) -- (114.2000,43.0000) -- (114.3000,43.0000) -- (114.4000,43.1000) -- (114.6000,43.1000) -- (114.7000,43.1000) -- (114.9000,43.2000) -- (115.0000,43.2000) -- (115.2000,43.2000) -- (115.3000,43.3000) -- (115.4000,43.3000) -- (115.6000,43.3000) -- (115.7000,43.4000) -- (115.9000,43.4000) -- (116.0000,43.4000) -- (116.2000,43.5000) -- (116.3000,43.5000) -- (116.4000,43.5000) -- (116.6000,43.6000) -- (116.7000,43.6000) -- (116.9000,43.6000) -- (117.0000,43.6000) -- (117.2000,43.7000) -- (117.3000,43.7000) -- (117.4000,43.7000) -- (117.6000,43.7000) -- (117.7000,43.8000) -- (117.9000,43.8000) -- (118.0000,43.8000) -- (118.2000,43.8000) -- (118.3000,43.9000) -- (118.4000,43.9000) -- (118.6000,43.9000) -- (118.7000,43.9000) -- (118.9000,44.0000) -- (119.0000,44.0000) -- (119.2000,44.0000) -- (119.3000,44.0000) -- (119.4000,44.0000) -- (119.6000,44.1000) -- (119.7000,44.1000) -- (119.9000,44.1000) -- (120.0000,44.1000) -- (120.2000,44.1000) -- (120.3000,44.1000) -- (120.4000,44.1000) -- (120.6000,44.2000) -- (120.7000,44.2000) -- (120.9000,44.2000) -- (121.0000,44.2000) -- (121.2000,44.2000) -- (121.3000,44.2000) -- (121.4000,44.2000) -- (121.6000,44.3000) -- (121.7000,44.3000) -- (121.9000,44.3000) -- (122.0000,44.3000) -- (122.2000,44.3000) -- (122.3000,44.3000) -- (122.4000,44.3000) -- (122.6000,44.3000) -- (122.7000,44.3000) -- (122.9000,44.3000) -- (123.0000,44.3000) -- (123.2000,44.4000) -- (123.3000,44.4000) -- (123.4000,44.4000) -- (123.6000,44.4000) -- (123.7000,44.4000) -- (123.9000,44.4000) -- (124.0000,44.4000) -- (124.2000,44.4000) -- (124.3000,44.4000) -- (124.4000,44.4000) -- (124.6000,44.4000) -- (124.7000,44.4000) -- (124.9000,44.4000) -- (125.0000,44.4000) -- (125.2000,44.4000) -- (125.3000,44.4000) -- (125.4000,44.4000) -- (125.6000,44.4000) -- (125.7000,44.4000) -- (125.9000,44.4000) -- (126.0000,44.4000) -- (126.2000,44.4000) -- (126.3000,44.4000) -- (126.4000,44.4000) -- (126.6000,44.4000) -- (126.7000,44.4000) -- (126.9000,44.4000) -- (127.0000,44.4000) -- (127.2000,44.4000) -- (127.3000,44.4000);



      \end{scope}
      \begin{scope}[cm={{1.05023,0.0,0.0,1.00798,(17.08433,-310.85467)}},draw=black,line join=round,line cap=round,line width=0.480pt]
        \path[draw=black] (41.5000,10.0509) -- (41.5000,86.0509) -- (127.5000,86.0509) -- (127.5000,10.0509) -- (41.5000,10.0509);



      \end{scope}
      \begin{scope}[cm={{1.05023,0.0,0.0,1.00798,(17.08433,-309.29135)}},fill=cffffff]
        \path[fill,rounded corners=0.0000cm] (81.0000,50.0000) rectangle (121.0000,78.0000);



      \end{scope}
      \begin{scope}[cm={{1.05023,0.0,0.0,1.00798,(17.08433,-309.29135)}},draw=ca0a0a4,dash pattern=on 0.40pt off 0.80pt,line join=round,line cap=round,line width=0.400pt]
        \path[draw] (81.5000,55.5000) -- (121.5000,55.5000);



      \end{scope}
      \begin{scope}[cm={{1.05023,0.0,0.0,1.00798,(17.08433,-309.29135)}},draw=black,line join=round,line cap=round,line width=0.480pt]
        \path[draw] (81.5000,55.5000) -- (82.6777,55.5000);



        \path[draw] (121.5000,55.5000) -- (120.1220,55.5000);



      \end{scope}
      \begin{scope}[cm={{1.05023,0.0,0.0,1.05023,(70.48235,-249.9858)}},draw=black,line join=bevel,line cap=rect,line width=0.800pt]
        \path[fill=black] (0.0000,0.0000) node[above right] (text672) {\scriptsize $T=$ 46};



      \end{scope}
      \begin{scope}[cm={{1.05023,0.0,0.0,1.00798,(17.08433,-309.29135)}},draw=ca0a0a4,dash pattern=on 0.40pt off 0.80pt,line join=round,line cap=round,line width=0.400pt]
        \path[draw] (97.5000,78.5000) -- (97.5000,50.5000);



      \end{scope}
      \begin{scope}[cm={{1.05023,0.0,0.0,1.00798,(17.08433,-309.29135)}},draw=black,line join=round,line cap=round,line width=0.480pt]
        \path[draw] (97.5000,50.5000) -- (97.5000,50.5000) -- (97.5000,51.6564);



      \end{scope}
      \begin{scope}[cm={{1.05023,0.0,0.0,1.05023,(114.23036,-264.55082)}},draw=black,line join=bevel,line cap=rect,line width=0.800pt]
        \path[fill=black] (1.4886,2.9771) node[above right] (text700) {\scriptsize 83};



      \end{scope}
      \begin{scope}[cm={{1.05023,0.0,0.0,1.00798,(17.08433,-309.29135)}},draw=black,line join=round,line cap=round,line width=0.480pt]
        \path[draw] (81.5000,50.5000) -- (81.5000,78.5000) -- (121.5000,78.5000) -- (121.5000,50.5000) -- (81.5000,50.5000);



      \end{scope}
      \begin{scope}[cm={{1.05023,0.0,0.0,1.00798,(17.08433,-309.29135)}},draw=black,line join=round,line cap=round,line width=0.480pt]
        \path[draw] (81.2000,77.5000) -- (81.2000,77.5000) -- (81.2000,77.5000) -- (81.2000,77.5000) -- (81.2000,77.5000) -- (81.2000,77.5000) -- (81.2000,77.5000) -- (81.2000,77.5000) -- (81.2000,77.5000) -- (81.2000,77.5000) -- (81.2000,77.5000) -- (81.2000,77.5000) -- (81.2000,77.5000) -- (81.2000,77.5000) -- (81.2000,77.5000) -- (81.2000,77.5000) -- (81.2000,77.4000) -- (81.2000,77.4000) -- (81.2000,77.4000) -- (81.2000,77.4000) -- (81.2000,77.4000) -- (81.2000,77.4000) -- (81.2000,77.4000) -- (81.2000,77.4000) -- (81.2000,77.4000) -- (81.2000,77.4000) -- (81.3000,77.4000) -- (81.3000,77.4000) -- (81.3000,77.4000) -- (81.3000,77.4000) -- (81.3000,77.4000) -- (81.3000,77.4000) -- (81.3000,77.4000) -- (81.3000,77.4000) -- (81.3000,77.4000) -- (81.3000,77.4000) -- (81.3000,77.4000) -- (81.3000,77.3000) -- (81.3000,77.3000) -- (81.3000,77.3000) -- (81.3000,77.3000) -- (81.3000,77.3000) -- (81.3000,77.3000) -- (81.3000,77.3000) -- (81.3000,77.3000) -- (81.3000,77.3000) -- (81.3000,77.3000) -- (81.3000,77.3000) -- (81.3000,77.3000) -- (81.3000,77.3000) -- (81.3000,77.3000) -- (81.3000,77.3000) -- (81.3000,77.3000) -- (81.3000,77.3000) -- (81.3000,77.3000) -- (81.3000,77.3000) -- (81.3000,77.3000) -- (81.3000,77.2000) -- (81.3000,77.2000) -- (81.3000,77.2000) -- (81.3000,77.2000) -- (81.3000,77.2000) -- (81.3000,77.2000) -- (81.3000,77.2000) -- (81.3000,77.2000) -- (81.3000,77.2000) -- (81.3000,77.2000) -- (81.3000,77.2000) -- (81.3000,77.2000) -- (81.3000,77.2000) -- (81.3000,77.2000) -- (81.3000,77.2000) -- (81.3000,77.2000) -- (81.3000,77.2000) -- (81.3000,77.2000) -- (81.3000,77.2000) -- (81.4000,77.2000) -- (81.4000,77.2000) -- (81.4000,77.1000) -- (81.4000,77.1000) -- (81.4000,77.1000) -- (81.4000,77.1000) -- (81.4000,77.1000) -- (81.4000,77.1000) -- (81.4000,77.1000) -- (81.4000,77.1000) -- (81.4000,77.1000) -- (81.4000,77.1000) -- (81.4000,77.1000) -- (81.4000,77.1000) -- (81.4000,77.1000) -- (81.4000,77.1000) -- (81.4000,77.1000) -- (81.4000,77.1000) -- (81.4000,77.1000) -- (81.4000,77.1000) -- (81.4000,77.1000) -- (81.4000,77.1000) -- (81.4000,77.1000) -- (81.4000,77.0000) -- (81.4000,77.0000) -- (81.4000,77.0000) -- (81.4000,77.0000) -- (81.4000,77.0000) -- (81.4000,77.0000) -- (81.4000,77.0000) -- (81.4000,77.0000) -- (81.4000,77.0000) -- (81.4000,77.0000) -- (81.4000,77.0000) -- (81.4000,77.0000) -- (81.4000,77.0000) -- (81.4000,77.0000) -- (81.4000,77.0000) -- (81.4000,77.0000) -- (81.4000,77.0000) -- (81.4000,77.0000) -- (81.4000,77.0000) -- (81.4000,77.0000) -- (81.4000,77.0000) -- (81.4000,76.9000) -- (81.4000,76.9000) -- (81.4000,76.9000) -- (81.4000,76.9000) -- (81.4000,76.9000) -- (81.5000,76.9000) -- (81.5000,76.9000) -- (81.5000,76.9000) -- (81.5000,76.9000) -- (81.5000,76.9000) -- (81.5000,76.9000) -- (81.5000,76.9000) -- (81.5000,76.9000) -- (81.5000,76.9000) -- (81.5000,76.9000) -- (81.5000,76.9000) -- (81.5000,76.9000) -- (81.5000,76.9000) -- (81.5000,76.9000) -- (81.5000,76.9000) -- (81.5000,76.9000) -- (81.5000,76.8000) -- (81.5000,76.8000) -- (81.5000,76.8000) -- (81.5000,76.8000) -- (81.5000,76.8000) -- (81.5000,76.8000) -- (81.5000,76.8000) -- (81.5000,76.8000) -- (81.5000,76.8000) -- (81.5000,76.8000) -- (81.5000,76.8000) -- (81.5000,76.8000) -- (81.5000,76.8000) -- (81.5000,76.8000) -- (81.5000,76.8000) -- (81.5000,76.8000) -- (81.5000,76.8000) -- (81.5000,76.8000) -- (81.5000,76.8000) -- (81.5000,76.8000) -- (81.5000,76.7000) -- (81.5000,76.7000) -- (81.5000,76.7000) -- (81.5000,76.7000) -- (81.5000,76.7000) -- (81.5000,76.7000) -- (81.5000,76.7000) -- (81.5000,76.7000) -- (81.5000,76.7000) -- (81.5000,76.7000) -- (81.5000,76.7000) -- (81.5000,76.7000) -- (81.5000,76.7000) -- (81.5000,76.7000) -- (81.6000,76.7000) -- (81.6000,76.7000) -- (81.6000,76.7000) -- (81.6000,76.7000) -- (81.6000,76.7000) -- (81.6000,76.7000) -- (81.6000,76.7000) -- (81.6000,76.6000) -- (81.6000,76.6000) -- (81.6000,76.6000) -- (81.6000,76.6000) -- (81.6000,76.6000) -- (81.6000,76.6000) -- (81.6000,76.6000) -- (81.6000,76.6000) -- (81.6000,76.6000) -- (81.6000,76.6000) -- (81.6000,76.6000) -- (81.6000,76.6000) -- (81.6000,76.6000) -- (81.6000,76.6000) -- (81.6000,76.6000) -- (81.6000,76.6000) -- (81.6000,76.6000) -- (81.6000,76.6000) -- (81.6000,76.6000) -- (81.6000,76.6000) -- (81.6000,76.6000) -- (81.6000,76.5000) -- (81.6000,76.5000) -- (81.6000,76.5000) -- (81.6000,76.5000) -- (81.6000,76.5000) -- (81.6000,76.5000) -- (81.6000,76.5000) -- (81.6000,76.5000) -- (81.6000,76.5000) -- (81.6000,76.5000) -- (81.6000,76.5000) -- (81.6000,76.5000) -- (81.6000,76.5000) -- (81.6000,76.5000) -- (81.6000,76.5000) -- (81.6000,76.5000) -- (81.6000,76.5000) -- (81.6000,76.5000) -- (81.6000,76.5000) -- (81.6000,76.5000) -- (81.6000,76.5000) -- (81.7000,76.4000) -- (81.7000,76.4000) -- (81.7000,76.4000) -- (81.7000,76.4000) -- (81.7000,76.4000) -- (81.7000,76.4000) -- (81.7000,76.4000) -- (81.7000,76.4000) -- (81.7000,76.4000) -- (81.7000,76.4000) -- (81.7000,76.4000) -- (81.7000,76.4000) -- (81.7000,76.4000) -- (81.7000,76.4000) -- (81.7000,76.4000) -- (81.7000,76.4000) -- (81.7000,76.4000) -- (81.7000,76.4000) -- (81.7000,76.4000) -- (81.7000,76.4000) -- (81.7000,76.4000) -- (81.7000,76.3000) -- (81.7000,76.3000) -- (81.7000,76.3000) -- (81.7000,76.3000) -- (81.7000,76.3000) -- (81.7000,76.3000) -- (81.7000,76.3000) -- (81.7000,76.3000) -- (81.7000,76.3000) -- (81.7000,76.3000) -- (81.7000,76.3000) -- (81.7000,76.3000) -- (81.7000,76.3000) -- (81.7000,76.3000) -- (81.7000,76.3000) -- (81.7000,76.3000) -- (81.7000,76.3000) -- (81.7000,76.3000) -- (81.7000,76.3000) -- (81.7000,76.3000) -- (81.7000,76.2000) -- (81.7000,76.2000) -- (81.7000,76.2000) -- (81.7000,76.2000) -- (81.7000,76.2000) -- (81.7000,76.2000) -- (81.7000,76.2000) -- (81.7000,76.2000) -- (81.7000,76.2000) -- (81.8000,76.2000) -- (81.8000,76.2000) -- (81.8000,76.2000) -- (81.8000,76.2000) -- (81.8000,76.2000) -- (81.8000,76.2000) -- (81.8000,76.2000) -- (81.8000,76.2000) -- (81.8000,76.2000) -- (81.8000,76.2000) -- (81.8000,76.2000) -- (81.8000,76.2000) -- (81.8000,76.1000) -- (81.8000,76.1000) -- (81.8000,76.1000) -- (81.8000,76.1000) -- (81.8000,76.1000) -- (81.8000,76.1000) -- (81.8000,76.1000) -- (81.8000,76.1000) -- (81.8000,76.1000) -- (81.8000,76.1000) -- (81.8000,76.1000) -- (81.8000,76.1000) -- (81.8000,76.1000) -- (81.8000,76.1000) -- (81.8000,76.1000) -- (81.8000,76.1000) -- (81.8000,76.1000) -- (81.8000,76.1000) -- (81.8000,76.1000) -- (81.8000,76.1000) -- (81.8000,76.1000) -- (81.8000,76.0000) -- (81.8000,76.0000) -- (81.8000,76.0000) -- (81.8000,76.0000) -- (81.8000,76.0000) -- (81.8000,76.0000) -- (81.8000,76.0000) -- (81.8000,76.0000) -- (81.8000,76.0000) -- (81.8000,76.0000) -- (81.8000,76.0000) -- (81.8000,76.0000) -- (81.8000,76.0000) -- (81.8000,76.0000) -- (81.8000,76.0000) -- (81.8000,76.0000) -- (81.9000,76.0000) -- (81.9000,76.0000) -- (81.9000,76.0000) -- (81.9000,76.0000) -- (81.9000,76.0000) -- (81.9000,75.9000) -- (81.9000,75.9000) -- (81.9000,75.9000) -- (81.9000,75.9000) -- (81.9000,75.9000) -- (81.9000,75.9000) -- (81.9000,75.9000) -- (81.9000,75.9000) -- (81.9000,75.9000) -- (81.9000,75.9000) -- (81.9000,75.9000) -- (81.9000,75.9000) -- (81.9000,75.9000) -- (81.9000,75.9000) -- (81.9000,75.9000) -- (81.9000,75.9000) -- (81.9000,75.9000) -- (81.9000,75.9000) -- (81.9000,75.9000) -- (81.9000,75.9000) -- (81.9000,75.8000) -- (81.9000,75.8000) -- (81.9000,75.8000) -- (81.9000,75.8000) -- (81.9000,75.8000) -- (81.9000,75.8000) -- (81.9000,75.8000) -- (81.9000,75.8000) -- (81.9000,75.8000) -- (81.9000,75.8000) -- (81.9000,75.8000) -- (81.9000,75.8000) -- (81.9000,75.8000) -- (81.9000,75.8000) -- (81.9000,75.8000) -- (81.9000,75.8000) -- (81.9000,75.8000) -- (81.9000,75.8000) -- (81.9000,75.8000) -- (81.9000,75.8000) -- (81.9000,75.8000) -- (81.9000,75.7000) -- (81.9000,75.7000) -- (81.9000,75.7000) -- (81.9000,75.7000) -- (82.0000,75.7000) -- (82.0000,75.7000) -- (82.0000,75.7000) -- (82.0000,75.7000) -- (82.0000,75.7000) -- (82.0000,75.7000) -- (82.0000,75.7000) -- (82.0000,75.7000) -- (82.0000,75.7000) -- (82.0000,75.7000) -- (82.0000,75.7000) -- (82.0000,75.7000) -- (82.0000,75.7000) -- (82.0000,75.7000) -- (82.0000,75.7000) -- (82.0000,75.7000) -- (82.0000,75.7000) -- (82.0000,75.6000) -- (82.0000,75.6000) -- (82.0000,75.6000) -- (82.0000,75.6000) -- (82.0000,75.6000) -- (82.0000,75.6000) -- (82.0000,75.6000) -- (82.0000,75.6000) -- (82.0000,75.6000) -- (82.0000,75.6000) -- (82.0000,75.6000) -- (82.0000,75.6000) -- (82.0000,75.6000) -- (82.0000,75.6000) -- (82.0000,75.6000) -- (82.0000,75.6000) -- (82.0000,75.6000) -- (82.0000,75.6000) -- (82.0000,75.6000) -- (82.0000,75.6000) -- (82.0000,75.6000) -- (82.0000,75.5000) -- (82.0000,75.5000) -- (82.0000,75.5000) -- (82.0000,75.5000) -- (82.0000,75.5000) -- (82.0000,75.5000) -- (82.0000,75.5000) -- (82.0000,75.5000) -- (82.0000,75.5000) -- (82.0000,75.5000) -- (82.0000,75.5000) -- (82.1000,75.5000) -- (82.1000,75.5000) -- (82.1000,75.5000) -- (82.1000,75.5000) -- (82.1000,75.5000) -- (82.1000,75.5000) -- (82.1000,75.5000) -- (82.1000,75.5000) -- (82.1000,75.5000) -- (82.1000,75.5000) -- (82.1000,75.4000) -- (82.1000,75.4000) -- (82.1000,75.4000) -- (82.1000,75.4000) -- (82.1000,75.4000) -- (82.1000,75.4000) -- (82.1000,75.4000) -- (82.1000,75.4000) -- (82.1000,75.4000) -- (82.1000,75.4000) -- (82.1000,75.4000) -- (82.1000,75.4000) -- (82.1000,75.4000) -- (82.1000,75.4000) -- (82.1000,75.4000) -- (82.1000,75.4000) -- (82.1000,75.4000) -- (82.1000,75.4000) -- (82.1000,75.4000) -- (82.1000,75.4000) -- (82.1000,75.3000) -- (82.1000,75.3000) -- (82.1000,75.3000) -- (82.1000,75.3000) -- (82.1000,75.3000) -- (82.1000,75.3000) -- (82.1000,75.3000) -- (82.1000,75.3000) -- (82.1000,75.3000) -- (82.1000,75.3000) -- (82.1000,75.3000) -- (82.1000,75.3000) -- (82.1000,75.3000) -- (82.1000,75.3000) -- (82.1000,75.3000) -- (82.1000,75.3000) -- (82.1000,75.3000) -- (82.1000,75.3000) -- (82.1000,75.3000) -- (82.1000,75.3000) -- (82.2000,75.3000) -- (82.2000,75.2000) -- (82.2000,75.2000) -- (82.2000,75.2000) -- (82.2000,75.2000) -- (82.2000,75.2000) -- (82.2000,75.2000) -- (82.2000,75.2000) -- (82.2000,75.2000) -- (82.2000,75.2000) -- (82.2000,75.2000) -- (82.2000,75.2000) -- (82.2000,75.2000) -- (82.2000,75.2000) -- (82.2000,75.2000) -- (82.2000,75.2000) -- (82.2000,75.2000) -- (82.2000,75.2000) -- (82.2000,75.2000) -- (82.2000,75.2000) -- (82.2000,75.2000) -- (82.2000,75.2000) -- (82.2000,75.1000) -- (82.2000,75.1000) -- (82.2000,75.1000) -- (82.2000,75.1000) -- (82.2000,75.1000) -- (82.2000,75.1000) -- (82.2000,75.1000) -- (82.2000,75.1000) -- (82.2000,75.1000) -- (82.2000,75.1000) -- (82.2000,75.1000) -- (82.2000,75.1000) -- (82.2000,75.1000) -- (82.2000,75.1000) -- (82.2000,75.1000) -- (82.2000,75.1000) -- (82.2000,75.1000) -- (82.2000,75.1000) -- (82.2000,75.1000) -- (82.2000,75.1000) -- (82.2000,75.1000) -- (82.2000,75.0000) -- (82.2000,75.0000) -- (82.2000,75.0000) -- (82.2000,75.0000) -- (82.2000,75.0000) -- (82.2000,75.0000) -- (82.3000,75.0000) -- (82.3000,75.0000) -- (82.3000,75.0000) -- (82.3000,75.0000) -- (82.3000,75.0000) -- (82.3000,75.0000) -- (82.3000,75.0000) -- (82.3000,75.0000) -- (82.3000,75.0000) -- (82.3000,75.0000) -- (82.3000,75.0000) -- (82.3000,75.0000) -- (82.3000,75.0000) -- (82.3000,75.0000) -- (82.3000,75.0000) -- (82.3000,74.9000) -- (82.3000,74.9000) -- (82.3000,74.9000) -- (82.3000,74.9000) -- (82.3000,74.9000) -- (82.3000,74.9000) -- (82.3000,74.9000) -- (82.3000,74.9000) -- (82.3000,74.9000) -- (82.3000,74.9000) -- (82.3000,74.9000) -- (82.3000,74.9000) -- (82.3000,74.9000) -- (82.3000,74.9000) -- (82.3000,74.9000) -- (82.3000,74.9000) -- (82.3000,74.9000) -- (82.3000,74.9000) -- (82.3000,74.9000) -- (82.3000,74.9000) -- (82.3000,74.8000) -- (82.3000,74.8000) -- (82.3000,74.8000) -- (82.3000,74.8000) -- (82.3000,74.8000) -- (82.3000,74.8000) -- (82.3000,74.8000) -- (82.3000,74.8000) -- (82.3000,74.8000) -- (82.3000,74.8000) -- (82.3000,74.8000) -- (82.3000,74.8000) -- (82.3000,74.8000) -- (82.3000,74.8000) -- (82.3000,74.8000) -- (82.4000,74.8000) -- (82.4000,74.8000) -- (82.4000,74.8000) -- (82.4000,74.8000) -- (82.4000,74.8000) -- (82.4000,74.8000) -- (82.4000,74.7000) -- (82.4000,74.7000) -- (82.4000,74.7000) -- (82.4000,74.7000) -- (82.4000,74.7000) -- (82.4000,74.7000) -- (82.4000,74.7000) -- (82.4000,74.7000) -- (82.4000,74.7000) -- (82.4000,74.7000) -- (82.4000,74.7000) -- (82.4000,74.7000) -- (82.4000,74.7000) -- (82.4000,74.7000) -- (82.4000,74.7000) -- (82.4000,74.7000) -- (82.4000,74.7000) -- (82.4000,74.7000) -- (82.4000,74.7000) -- (82.4000,74.7000) -- (82.4000,74.7000) -- (82.4000,74.6000) -- (82.4000,74.6000) -- (82.4000,74.6000) -- (82.4000,74.6000) -- (82.4000,74.6000) -- (82.4000,74.6000) -- (82.4000,74.6000) -- (82.4000,74.6000) -- (82.4000,74.6000) -- (82.4000,74.6000) -- (82.4000,74.6000) -- (82.4000,74.6000) -- (82.4000,74.6000) -- (82.4000,74.6000) -- (82.4000,74.6000) -- (82.4000,74.6000) -- (82.4000,74.6000) -- (82.4000,74.6000) -- (82.4000,74.6000) -- (82.4000,74.6000) -- (82.4000,74.6000) -- (82.4000,74.5000) -- (82.5000,74.5000) -- (82.5000,74.5000) -- (82.5000,74.5000) -- (82.5000,74.5000) -- (82.5000,74.5000) -- (82.5000,74.5000) -- (82.5000,74.5000) -- (82.5000,74.5000) -- (82.5000,74.5000) -- (82.5000,74.5000) -- (82.5000,74.5000) -- (82.5000,74.5000) -- (82.5000,74.5000) -- (82.5000,74.5000) -- (82.5000,74.5000) -- (82.5000,74.5000) -- (82.5000,74.5000) -- (82.5000,74.5000) -- (82.5000,74.5000) -- (82.5000,74.4000) -- (82.5000,74.4000) -- (82.5000,74.4000) -- (82.5000,74.4000) -- (82.5000,74.4000) -- (82.5000,74.4000) -- (82.5000,74.4000) -- (82.5000,74.4000) -- (82.5000,74.4000) -- (82.5000,74.4000) -- (82.5000,74.4000) -- (82.5000,74.4000) -- (82.5000,74.4000) -- (82.5000,74.4000) -- (82.5000,74.4000) -- (82.5000,74.4000) -- (82.5000,74.4000) -- (82.5000,74.4000) -- (82.5000,74.4000) -- (82.5000,74.4000) -- (82.5000,74.4000) -- (82.5000,74.3000) -- (82.5000,74.3000) -- (82.5000,74.3000) -- (82.5000,74.3000) -- (82.5000,74.3000) -- (82.5000,74.3000) -- (82.5000,74.3000) -- (82.5000,74.3000) -- (82.5000,74.3000) -- (82.5000,74.3000) -- (82.6000,74.3000) -- (82.6000,74.3000) -- (82.6000,74.3000) -- (82.6000,74.3000) -- (82.6000,74.3000) -- (82.6000,74.3000) -- (82.6000,74.3000) -- (82.6000,74.3000) -- (82.6000,74.3000) -- (82.6000,74.3000) -- (82.6000,74.3000) -- (82.6000,74.2000) -- (82.6000,74.2000) -- (82.6000,74.2000) -- (82.6000,74.2000) -- (82.6000,74.2000) -- (82.6000,74.2000) -- (82.6000,74.2000) -- (82.6000,74.2000) -- (82.6000,74.2000) -- (82.6000,74.2000) -- (82.6000,74.2000) -- (82.6000,74.2000) -- (82.6000,74.2000) -- (82.6000,74.2000) -- (82.6000,74.2000) -- (82.6000,74.2000) -- (82.6000,74.2000) -- (82.6000,74.2000) -- (82.6000,74.2000) -- (82.6000,74.2000) -- (82.6000,74.2000) -- (82.6000,74.1000) -- (82.6000,74.1000) -- (82.6000,74.1000) -- (82.6000,74.1000) -- (82.6000,74.1000) -- (82.6000,74.1000) -- (82.6000,74.1000) -- (82.6000,74.1000) -- (82.6000,74.1000) -- (82.6000,74.1000) -- (82.6000,74.1000) -- (82.6000,74.1000) -- (82.6000,74.1000) -- (82.6000,74.1000) -- (82.6000,74.1000) -- (82.6000,74.1000) -- (82.6000,74.1000) -- (82.7000,74.1000) -- (82.7000,74.1000) -- (82.7000,74.1000) -- (82.7000,74.1000) -- (82.7000,74.0000) -- (82.7000,74.0000) -- (82.7000,74.0000) -- (82.7000,74.0000) -- (82.7000,74.0000) -- (82.7000,74.0000) -- (82.7000,74.0000) -- (82.7000,74.0000) -- (82.7000,74.0000) -- (82.7000,74.0000) -- (82.7000,74.0000) -- (82.7000,74.0000) -- (82.7000,74.0000) -- (82.7000,74.0000) -- (82.7000,74.0000) -- (82.7000,74.0000) -- (82.7000,74.0000) -- (82.7000,74.0000) -- (82.7000,74.0000) -- (82.7000,74.0000) -- (82.7000,73.9000) -- (82.7000,73.9000) -- (82.7000,73.9000) -- (82.7000,73.9000) -- (82.7000,73.9000) -- (82.7000,73.9000) -- (82.7000,73.9000) -- (82.7000,73.9000) -- (82.7000,73.9000) -- (82.7000,73.9000) -- (82.7000,73.9000) -- (82.7000,73.9000) -- (82.7000,73.9000) -- (82.7000,73.9000) -- (82.7000,73.9000) -- (82.7000,73.9000) -- (82.7000,73.9000) -- (82.7000,73.9000) -- (82.7000,73.9000) -- (82.7000,73.9000) -- (82.7000,73.9000) -- (82.7000,73.8000) -- (82.7000,73.8000) -- (82.7000,73.8000) -- (82.7000,73.8000) -- (82.7000,73.8000) -- (82.8000,73.8000) -- (82.8000,73.8000) -- (82.8000,73.8000) -- (82.8000,73.8000) -- (82.8000,73.8000) -- (82.8000,73.8000) -- (82.8000,73.8000) -- (82.8000,73.8000) -- (82.8000,73.8000) -- (82.8000,73.8000) -- (82.8000,73.8000) -- (82.8000,73.8000) -- (82.8000,73.8000) -- (82.8000,73.8000) -- (82.8000,73.8000) -- (82.8000,73.8000) -- (82.8000,73.7000) -- (82.8000,73.7000) -- (82.8000,73.7000) -- (82.8000,73.7000) -- (82.8000,73.7000) -- (82.8000,73.7000) -- (82.8000,73.7000) -- (82.8000,73.7000) -- (82.8000,73.7000) -- (82.8000,73.7000) -- (82.8000,73.7000) -- (82.8000,73.7000) -- (82.8000,73.7000) -- (82.8000,73.7000) -- (82.8000,73.7000) -- (82.8000,73.7000) -- (82.8000,73.7000) -- (82.8000,73.7000) -- (82.8000,73.7000) -- (82.8000,73.7000) -- (82.8000,73.7000) -- (82.8000,73.6000) -- (82.8000,73.6000) -- (82.8000,73.6000) -- (82.8000,73.6000) -- (82.8000,73.6000) -- (82.8000,73.6000) -- (82.8000,73.6000) -- (82.8000,73.6000) -- (82.8000,73.6000) -- (82.8000,73.6000) -- (82.8000,73.6000) -- (82.8000,73.6000) -- (82.9000,73.6000) -- (82.9000,73.6000) -- (82.9000,73.6000) -- (82.9000,73.6000) -- (82.9000,73.6000) -- (82.9000,73.6000) -- (82.9000,73.6000) -- (82.9000,73.6000) -- (82.9000,73.6000) -- (82.9000,73.5000) -- (82.9000,73.5000) -- (82.9000,73.5000) -- (82.9000,73.5000) -- (82.9000,73.5000) -- (82.9000,73.5000) -- (82.9000,73.5000) -- (82.9000,73.5000) -- (82.9000,73.5000) -- (82.9000,73.5000) -- (82.9000,73.5000) -- (82.9000,73.5000) -- (82.9000,73.5000) -- (82.9000,73.5000) -- (82.9000,73.5000) -- (82.9000,73.5000) -- (82.9000,73.5000) -- (82.9000,73.5000) -- (82.9000,73.5000) -- (82.9000,73.5000) -- (82.9000,73.4000) -- (82.9000,73.4000) -- (82.9000,73.4000) -- (82.9000,73.4000) -- (82.9000,73.4000) -- (82.9000,73.4000) -- (82.9000,73.4000) -- (82.9000,73.4000) -- (82.9000,73.4000) -- (82.9000,73.4000) -- (82.9000,73.4000) -- (82.9000,73.4000) -- (82.9000,73.4000) -- (82.9000,73.4000) -- (82.9000,73.4000) -- (82.9000,73.4000) -- (82.9000,73.4000) -- (82.9000,73.4000) -- (82.9000,73.4000) -- (82.9000,73.4000) -- (82.9000,73.4000) -- (83.0000,73.3000) -- (83.0000,73.3000) -- (83.0000,73.3000) -- (83.0000,73.3000) -- (83.0000,73.3000) -- (83.0000,73.3000) -- (83.0000,73.3000) -- (83.0000,73.3000) -- (83.0000,73.3000) -- (83.0000,73.3000) -- (83.0000,73.3000) -- (83.0000,73.3000) -- (83.0000,73.3000) -- (83.0000,73.3000) -- (83.0000,73.3000) -- (83.0000,73.3000) -- (83.0000,73.3000) -- (83.0000,73.3000) -- (83.0000,73.3000) -- (83.0000,73.3000) -- (83.0000,73.3000) -- (83.0000,73.2000) -- (83.0000,73.2000) -- (83.0000,73.2000) -- (83.0000,73.2000) -- (83.0000,73.2000) -- (83.0000,73.2000) -- (83.0000,73.2000) -- (83.0000,73.2000) -- (83.0000,73.2000) -- (83.0000,73.2000) -- (83.0000,73.2000) -- (83.0000,73.2000) -- (83.0000,73.2000) -- (83.0000,73.2000) -- (83.0000,73.2000) -- (83.0000,73.2000) -- (83.0000,73.2000) -- (83.0000,73.2000) -- (83.0000,73.2000) -- (83.0000,73.2000) -- (83.0000,73.2000) -- (83.0000,73.1000) -- (83.0000,73.1000) -- (83.0000,73.1000) -- (83.0000,73.1000) -- (83.0000,73.1000) -- (83.0000,73.1000) -- (83.0000,73.1000) -- (83.1000,73.1000) -- (83.1000,73.1000) -- (83.1000,73.1000) -- (83.1000,73.1000) -- (83.1000,73.1000) -- (83.1000,73.1000) -- (83.1000,73.1000) -- (83.1000,73.1000) -- (83.1000,73.1000) -- (83.1000,73.1000) -- (83.1000,73.1000) -- (83.1000,73.1000) -- (83.1000,73.1000) -- (83.1000,73.1000) -- (83.1000,73.0000) -- (83.1000,73.0000) -- (83.1000,73.0000) -- (83.1000,73.0000) -- (83.1000,73.0000) -- (83.1000,73.0000) -- (83.1000,73.0000) -- (83.1000,73.0000) -- (83.1000,73.0000) -- (83.1000,73.0000) -- (83.1000,73.0000) -- (83.1000,73.0000) -- (83.1000,73.0000) -- (83.1000,73.0000) -- (83.1000,73.0000) -- (83.1000,73.0000) -- (83.1000,73.0000) -- (83.1000,73.0000) -- (83.1000,73.0000) -- (83.1000,73.0000) -- (83.1000,72.9000) -- (83.1000,72.9000) -- (83.1000,72.9000) -- (83.1000,72.9000) -- (83.1000,72.9000) -- (83.1000,72.9000) -- (83.1000,72.9000) -- (83.1000,72.9000) -- (83.1000,72.9000) -- (83.1000,72.9000) -- (83.1000,72.9000) -- (83.1000,72.9000) -- (83.1000,72.9000) -- (83.1000,72.9000) -- (83.1000,72.9000) -- (83.1000,72.9000) -- (83.2000,72.9000) -- (83.2000,72.9000) -- (83.2000,72.9000) -- (83.2000,72.9000) -- (83.2000,72.9000) -- (83.2000,72.8000) -- (83.2000,72.8000) -- (83.2000,72.8000) -- (83.2000,72.8000) -- (83.2000,72.8000) -- (83.2000,72.8000) -- (83.2000,72.8000) -- (83.2000,72.8000) -- (83.2000,72.8000) -- (83.2000,72.8000) -- (83.2000,72.8000) -- (83.2000,72.8000) -- (83.2000,72.8000) -- (83.2000,72.8000) -- (83.2000,72.8000) -- (83.2000,72.8000) -- (83.2000,72.8000) -- (83.2000,72.8000) -- (83.2000,72.8000) -- (83.2000,72.8000) -- (83.2000,72.8000) -- (83.2000,72.7000) -- (83.2000,72.7000) -- (83.2000,72.7000) -- (83.2000,72.7000) -- (83.2000,72.7000) -- (83.2000,72.7000) -- (83.2000,72.7000) -- (83.2000,72.7000) -- (83.2000,72.7000) -- (83.2000,72.7000) -- (83.2000,72.7000) -- (83.2000,72.7000) -- (83.2000,72.7000) -- (83.2000,72.7000) -- (83.2000,72.7000) -- (83.2000,72.7000) -- (83.2000,72.7000) -- (83.2000,72.7000) -- (83.2000,72.7000) -- (83.2000,72.7000) -- (83.2000,72.7000) -- (83.2000,72.6000) -- (83.2000,72.6000) -- (83.3000,72.6000) -- (83.3000,72.6000) -- (83.3000,72.6000) -- (83.3000,72.6000) -- (83.3000,72.6000) -- (83.3000,72.6000) -- (83.3000,72.6000) -- (83.3000,72.6000) -- (83.3000,72.6000) -- (83.3000,72.6000) -- (83.3000,72.6000) -- (83.3000,72.6000) -- (83.3000,72.6000) -- (83.3000,72.6000) -- (83.3000,72.6000) -- (83.3000,72.6000) -- (83.3000,72.6000) -- (83.3000,72.6000) -- (83.3000,72.5000) -- (83.3000,72.5000) -- (83.3000,72.5000) -- (83.3000,72.5000) -- (83.3000,72.5000) -- (83.3000,72.5000) -- (83.3000,72.5000) -- (83.3000,72.5000) -- (83.3000,72.5000) -- (83.3000,72.5000) -- (83.3000,72.5000) -- (83.3000,72.5000) -- (83.3000,72.5000) -- (83.3000,72.5000) -- (83.3000,72.5000) -- (83.3000,72.5000) -- (83.3000,72.5000) -- (83.3000,72.5000) -- (83.3000,72.5000) -- (83.3000,72.5000) -- (83.3000,72.5000) -- (83.3000,72.4000) -- (83.3000,72.4000) -- (83.3000,72.4000) -- (83.3000,72.4000) -- (83.3000,72.4000) -- (83.3000,72.4000) -- (83.3000,72.4000) -- (83.3000,72.4000) -- (83.3000,72.4000) -- (83.3000,72.4000) -- (83.3000,72.4000) -- (83.4000,72.4000) -- (83.4000,72.4000) -- (83.4000,72.4000) -- (83.4000,72.4000) -- (83.4000,72.4000) -- (83.4000,72.4000) -- (83.4000,72.4000) -- (83.4000,72.4000) -- (83.4000,72.4000) -- (83.4000,72.4000) -- (83.4000,72.3000) -- (83.4000,72.3000) -- (83.4000,72.3000) -- (83.4000,72.3000) -- (83.4000,72.3000) -- (83.4000,72.3000) -- (83.4000,72.3000) -- (83.4000,72.3000) -- (83.4000,72.3000) -- (83.4000,72.3000) -- (83.4000,72.3000) -- (83.4000,72.3000) -- (83.4000,72.3000) -- (83.4000,72.3000) -- (83.4000,72.3000) -- (83.4000,72.3000) -- (83.4000,72.3000) -- (83.4000,72.3000) -- (83.4000,72.3000) -- (83.4000,72.3000) -- (83.4000,72.3000) -- (83.4000,72.2000) -- (83.4000,72.2000) -- (83.4000,72.2000) -- (83.4000,72.2000) -- (83.4000,72.2000) -- (83.4000,72.2000) -- (83.4000,72.2000) -- (83.4000,72.2000) -- (83.4000,72.2000) -- (83.4000,72.2000) -- (83.4000,72.2000) -- (83.4000,72.2000) -- (83.4000,72.2000) -- (83.4000,72.2000) -- (83.4000,72.2000) -- (83.4000,72.2000) -- (83.4000,72.2000) -- (83.4000,72.2000) -- (83.5000,72.2000) -- (83.5000,72.2000) -- (83.5000,72.2000) -- (83.5000,72.1000) -- (83.5000,72.1000) -- (83.5000,72.1000) -- (83.5000,72.1000) -- (83.5000,72.1000) -- (83.5000,72.1000) -- (83.5000,72.1000) -- (83.5000,72.1000) -- (83.5000,72.1000) -- (83.5000,72.1000) -- (83.5000,72.1000) -- (83.5000,72.1000) -- (83.5000,72.1000) -- (83.5000,72.1000) -- (83.5000,72.1000) -- (83.5000,72.1000) -- (83.5000,72.1000) -- (83.5000,72.1000) -- (83.5000,72.1000) -- (83.5000,72.1000) -- (83.5000,72.0000) -- (83.5000,72.0000) -- (83.5000,72.0000) -- (83.5000,72.0000) -- (83.5000,72.0000) -- (83.5000,72.0000) -- (83.5000,72.0000) -- (83.5000,72.0000) -- (83.5000,72.0000) -- (83.5000,72.0000) -- (83.5000,72.0000) -- (83.5000,72.0000) -- (83.5000,72.0000) -- (83.5000,72.0000) -- (83.5000,72.0000) -- (83.5000,72.0000) -- (83.5000,72.0000) -- (83.5000,72.0000) -- (83.5000,72.0000) -- (83.5000,72.0000) -- (83.5000,72.0000) -- (83.5000,71.9000) -- (83.5000,71.9000) -- (83.5000,71.9000) -- (83.5000,71.9000) -- (83.5000,71.9000) -- (83.5000,71.9000) -- (83.6000,71.9000) -- (83.6000,71.9000) -- (83.6000,71.9000) -- (83.6000,71.9000) -- (83.6000,71.9000) -- (83.6000,71.9000) -- (83.6000,71.9000) -- (83.6000,71.9000) -- (83.6000,71.9000) -- (83.6000,71.9000) -- (83.6000,71.9000) -- (83.6000,71.9000) -- (83.6000,71.9000) -- (83.6000,71.9000) -- (83.6000,71.9000) -- (83.6000,71.8000) -- (83.6000,71.8000) -- (83.6000,71.8000) -- (83.6000,71.8000) -- (83.6000,71.8000) -- (83.6000,71.8000) -- (83.6000,71.8000) -- (83.6000,71.8000) -- (83.6000,71.8000) -- (83.6000,71.8000) -- (83.6000,71.8000) -- (83.6000,71.8000) -- (83.6000,71.8000) -- (83.6000,71.8000) -- (83.6000,71.8000) -- (83.6000,71.8000) -- (83.6000,71.8000) -- (83.6000,71.8000) -- (83.6000,71.8000) -- (83.6000,71.8000) -- (83.6000,71.8000) -- (83.6000,71.7000) -- (83.6000,71.7000) -- (83.6000,71.7000) -- (83.6000,71.7000) -- (83.6000,71.7000) -- (83.6000,71.7000) -- (83.6000,71.7000) -- (83.6000,71.7000) -- (83.6000,71.7000) -- (83.6000,71.7000) -- (83.6000,71.7000) -- (83.6000,71.7000) -- (83.6000,71.7000) -- (83.7000,71.7000) -- (83.7000,71.7000) -- (83.7000,71.7000) -- (83.7000,71.7000) -- (83.7000,71.7000) -- (83.7000,71.7000) -- (83.7000,71.7000) -- (83.7000,71.7000) -- (83.7000,71.6000) -- (83.7000,71.6000) -- (83.7000,71.6000) -- (83.7000,71.6000) -- (83.7000,71.6000) -- (83.7000,71.6000) -- (83.7000,71.6000) -- (83.7000,71.6000) -- (83.7000,71.6000) -- (83.7000,71.6000) -- (83.7000,71.6000) -- (83.7000,71.6000) -- (83.7000,71.6000) -- (83.7000,71.6000) -- (83.7000,71.6000) -- (83.7000,71.6000) -- (83.7000,71.6000) -- (83.7000,71.6000) -- (83.7000,71.6000) -- (83.7000,71.6000) -- (83.7000,71.5000) -- (83.7000,71.5000) -- (83.7000,71.5000) -- (83.7000,71.5000) -- (83.7000,71.5000) -- (83.7000,71.5000) -- (83.7000,71.5000) -- (83.7000,71.5000) -- (83.7000,71.5000) -- (83.7000,71.5000) -- (83.7000,71.5000) -- (83.7000,71.5000) -- (83.7000,71.5000) -- (83.7000,71.5000) -- (83.7000,71.5000) -- (83.7000,71.5000) -- (83.7000,71.5000) -- (83.7000,71.5000) -- (83.7000,71.5000) -- (83.7000,71.5000) -- (83.7000,71.5000) -- (83.7000,71.4000) -- (83.8000,71.4000) -- (83.8000,71.4000) -- (83.8000,71.4000) -- (83.8000,71.4000) -- (83.8000,71.4000) -- (83.8000,71.4000) -- (83.8000,71.4000) -- (83.8000,71.4000) -- (83.8000,71.4000) -- (83.8000,71.4000) -- (83.8000,71.4000) -- (83.8000,71.4000) -- (83.8000,71.4000) -- (83.8000,71.4000) -- (83.8000,71.4000) -- (83.8000,71.4000) -- (83.8000,71.4000) -- (83.8000,71.4000) -- (83.8000,71.4000) -- (83.8000,71.4000) -- (83.8000,71.3000) -- (83.8000,71.3000) -- (83.8000,71.3000) -- (83.8000,71.3000) -- (83.8000,71.3000) -- (83.8000,71.3000) -- (83.8000,71.3000) -- (83.8000,71.3000) -- (83.8000,71.3000) -- (83.8000,71.3000) -- (83.8000,71.3000) -- (83.8000,71.3000) -- (83.8000,71.3000) -- (83.8000,71.3000) -- (83.8000,71.3000) -- (83.8000,71.3000) -- (83.8000,71.3000) -- (83.8000,71.3000) -- (83.8000,71.3000) -- (83.8000,71.3000) -- (83.8000,71.3000) -- (83.8000,71.2000) -- (83.8000,71.2000) -- (83.8000,71.2000) -- (83.8000,71.2000) -- (83.8000,71.2000) -- (83.8000,71.2000) -- (83.8000,71.2000) -- (83.8000,71.2000) -- (83.9000,71.2000) -- (83.9000,71.2000) -- (83.9000,71.2000) -- (83.9000,71.2000) -- (83.9000,71.2000) -- (83.9000,71.2000) -- (83.9000,71.2000) -- (83.9000,71.2000) -- (83.9000,71.2000) -- (83.9000,71.2000) -- (83.9000,71.2000) -- (83.9000,71.2000) -- (83.9000,71.2000) -- (83.9000,71.1000) -- (83.9000,71.1000) -- (83.9000,71.1000) -- (83.9000,71.1000) -- (83.9000,71.1000) -- (83.9000,71.1000) -- (83.9000,71.1000) -- (83.9000,71.1000) -- (83.9000,71.1000) -- (83.9000,71.1000) -- (83.9000,71.1000) -- (83.9000,71.1000) -- (83.9000,71.1000) -- (83.9000,71.1000) -- (83.9000,71.1000) -- (83.9000,71.1000) -- (83.9000,71.1000) -- (83.9000,71.1000) -- (83.9000,71.1000) -- (83.9000,71.1000) -- (83.9000,71.0000) -- (83.9000,71.0000) -- (83.9000,71.0000) -- (83.9000,71.0000) -- (83.9000,71.0000) -- (83.9000,71.0000) -- (83.9000,71.0000) -- (83.9000,71.0000) -- (83.9000,71.0000) -- (83.9000,71.0000) -- (83.9000,71.0000) -- (83.9000,71.0000) -- (83.9000,71.0000) -- (83.9000,71.0000) -- (83.9000,71.0000) -- (83.9000,71.0000) -- (83.9000,71.0000) -- (84.0000,71.0000) -- (84.0000,71.0000) -- (84.0000,71.0000) -- (84.0000,71.0000) -- (84.0000,70.9000) -- (84.0000,70.9000) -- (84.0000,70.9000) -- (84.0000,70.9000) -- (84.0000,70.9000) -- (84.0000,70.9000) -- (84.0000,70.9000) -- (84.0000,70.9000) -- (84.0000,70.9000) -- (84.0000,70.9000) -- (84.0000,70.9000) -- (84.0000,70.9000) -- (84.0000,70.9000) -- (84.0000,70.9000) -- (84.0000,70.9000) -- (84.0000,70.9000) -- (84.0000,70.9000) -- (84.0000,70.9000) -- (84.0000,70.9000) -- (84.0000,70.9000) -- (84.0000,70.9000) -- (84.0000,70.8000) -- (84.0000,70.8000) -- (84.0000,70.8000) -- (84.0000,70.8000) -- (84.0000,70.8000) -- (84.0000,70.8000) -- (84.0000,70.8000) -- (84.0000,70.8000) -- (84.0000,70.8000) -- (84.0000,70.8000) -- (84.0000,70.8000) -- (84.0000,70.8000) -- (84.0000,70.8000) -- (84.0000,70.8000) -- (84.0000,70.8000) -- (84.0000,70.8000) -- (84.0000,70.8000) -- (84.0000,70.8000) -- (84.0000,70.8000) -- (84.0000,70.8000) -- (84.0000,70.8000) -- (84.0000,70.7000) -- (84.0000,70.7000) -- (84.0000,70.7000) -- (84.1000,70.7000) -- (84.1000,70.7000) -- (84.1000,70.7000) -- (84.1000,70.7000) -- (84.1000,70.7000) -- (84.1000,70.7000) -- (84.1000,70.7000) -- (84.1000,70.7000) -- (84.1000,70.7000) -- (84.1000,70.7000) -- (84.1000,70.7000) -- (84.1000,70.7000) -- (84.1000,70.7000) -- (84.1000,70.7000) -- (84.1000,70.7000) -- (84.1000,70.7000) -- (84.1000,70.7000) -- (84.1000,70.6000) -- (84.1000,70.6000) -- (84.1000,70.6000) -- (84.1000,70.6000) -- (84.1000,70.6000) -- (84.1000,70.6000) -- (84.1000,70.6000) -- (84.1000,70.6000) -- (84.1000,70.6000) -- (84.1000,70.6000) -- (84.1000,70.6000) -- (84.1000,70.6000) -- (84.1000,70.6000) -- (84.1000,70.6000) -- (84.1000,70.6000) -- (84.1000,70.6000) -- (84.1000,70.6000) -- (84.1000,70.6000) -- (84.1000,70.6000) -- (84.1000,70.6000) -- (84.1000,70.6000) -- (84.1000,70.5000) -- (84.1000,70.5000) -- (84.1000,70.5000) -- (84.1000,70.5000) -- (84.1000,70.5000) -- (84.1000,70.5000) -- (84.1000,70.5000) -- (84.1000,70.5000) -- (84.1000,70.5000) -- (84.1000,70.5000) -- (84.1000,70.5000) -- (84.1000,70.5000) -- (84.2000,70.5000) -- (84.2000,70.5000) -- (84.2000,70.5000) -- (84.2000,70.5000) -- (84.2000,70.5000) -- (84.2000,70.5000) -- (84.2000,70.5000) -- (84.2000,70.5000) -- (84.2000,70.5000) -- (84.2000,70.4000) -- (84.2000,70.4000) -- (84.2000,70.4000) -- (84.2000,70.4000) -- (84.2000,70.4000) -- (84.2000,70.4000) -- (84.2000,70.4000) -- (84.2000,70.4000) -- (84.2000,70.4000) -- (84.2000,70.4000) -- (84.2000,70.4000) -- (84.2000,70.4000) -- (84.2000,70.4000) -- (84.2000,70.4000) -- (84.2000,70.4000) -- (84.2000,70.4000) -- (84.2000,70.4000) -- (84.2000,70.4000) -- (84.2000,70.4000) -- (84.2000,70.4000) -- (84.2000,70.4000) -- (84.2000,70.3000) -- (84.2000,70.3000) -- (84.2000,70.3000) -- (84.2000,70.3000) -- (84.2000,70.3000) -- (84.2000,70.3000) -- (84.2000,70.3000) -- (84.2000,70.3000) -- (84.2000,70.3000) -- (84.2000,70.3000) -- (84.2000,70.3000) -- (84.2000,70.3000) -- (84.2000,70.3000) -- (84.2000,70.3000) -- (84.2000,70.3000) -- (84.2000,70.3000) -- (84.2000,70.3000) -- (84.2000,70.3000) -- (84.2000,70.3000) -- (84.3000,70.3000) -- (84.3000,70.3000) -- (84.3000,70.2000) -- (84.3000,70.2000) -- (84.3000,70.2000) -- (84.3000,70.2000) -- (84.3000,70.2000) -- (84.3000,70.2000) -- (84.3000,70.2000) -- (84.3000,70.2000) -- (84.3000,70.2000) -- (84.3000,70.2000) -- (84.3000,70.2000) -- (84.3000,70.2000) -- (84.3000,70.2000) -- (84.3000,70.2000) -- (84.3000,70.2000) -- (84.3000,70.2000) -- (84.3000,70.2000) -- (84.3000,70.2000) -- (84.3000,70.2000) -- (84.3000,70.2000) -- (84.3000,70.1000) -- (84.3000,70.1000) -- (84.3000,70.1000) -- (84.3000,70.1000) -- (84.3000,70.1000) -- (84.3000,70.1000) -- (84.3000,70.1000) -- (84.3000,70.1000) -- (84.3000,70.1000) -- (84.3000,70.1000) -- (84.3000,70.1000) -- (84.3000,70.1000) -- (84.3000,70.1000) -- (84.3000,70.1000) -- (84.3000,70.1000) -- (84.3000,70.1000) -- (84.3000,70.1000) -- (84.3000,70.1000) -- (84.3000,70.1000) -- (84.3000,70.1000) -- (84.3000,70.1000) -- (84.3000,70.0000) -- (84.3000,70.0000) -- (84.3000,70.0000) -- (84.3000,70.0000) -- (84.3000,70.0000) -- (84.3000,70.0000) -- (84.3000,70.0000) -- (84.4000,70.0000) -- (84.4000,70.0000) -- (84.4000,70.0000) -- (84.4000,70.0000) -- (84.4000,70.0000) -- (84.4000,70.0000) -- (84.4000,70.0000) -- (84.4000,70.0000) -- (84.4000,70.0000) -- (84.4000,70.0000) -- (84.4000,70.0000) -- (84.4000,70.0000) -- (84.4000,70.0000) -- (84.4000,70.0000) -- (84.4000,69.9000) -- (84.4000,69.9000) -- (84.4000,69.9000) -- (84.4000,69.9000) -- (84.4000,69.9000) -- (84.4000,69.9000) -- (84.4000,69.9000) -- (84.4000,69.9000) -- (84.4000,69.9000) -- (84.4000,69.9000) -- (84.4000,69.9000) -- (84.4000,69.9000) -- (84.4000,69.9000) -- (84.4000,69.9000) -- (84.4000,69.9000) -- (84.4000,69.9000) -- (84.4000,69.9000) -- (84.4000,69.9000) -- (84.4000,69.9000) -- (84.4000,69.9000) -- (84.4000,69.9000) -- (84.4000,69.8000) -- (84.4000,69.8000) -- (84.4000,69.8000) -- (84.4000,69.8000) -- (84.4000,69.8000) -- (84.4000,69.8000) -- (84.4000,69.8000) -- (84.4000,69.8000) -- (84.4000,69.8000) -- (84.4000,69.8000) -- (84.4000,69.8000) -- (84.4000,69.8000) -- (84.4000,69.8000) -- (84.4000,69.8000) -- (84.5000,69.8000) -- (84.5000,69.8000) -- (84.5000,69.8000) -- (84.5000,69.8000) -- (84.5000,69.8000) -- (84.5000,69.8000) -- (84.5000,69.8000) -- (84.5000,69.7000) -- (84.5000,69.7000) -- (84.5000,69.7000) -- (84.5000,69.7000) -- (84.5000,69.7000) -- (84.5000,69.7000) -- (84.5000,69.7000) -- (84.5000,69.7000) -- (84.5000,69.7000) -- (84.5000,69.7000) -- (84.5000,69.7000) -- (84.5000,69.7000) -- (84.5000,69.7000) -- (84.5000,69.7000) -- (84.5000,69.7000) -- (84.5000,69.7000) -- (84.5000,69.7000) -- (84.5000,69.7000) -- (84.5000,69.7000) -- (84.5000,69.7000) -- (84.5000,69.6000) -- (84.5000,69.6000) -- (84.5000,69.6000) -- (84.5000,69.6000) -- (84.5000,69.6000) -- (84.5000,69.6000) -- (84.5000,69.6000) -- (84.5000,69.6000) -- (84.5000,69.6000) -- (84.5000,69.6000) -- (84.5000,69.6000) -- (84.5000,69.6000) -- (84.5000,69.6000) -- (84.5000,69.6000) -- (84.5000,69.6000) -- (84.5000,69.6000) -- (84.5000,69.6000) -- (84.5000,69.6000) -- (84.5000,69.6000) -- (84.5000,69.6000) -- (84.5000,69.6000) -- (84.5000,69.5000) -- (84.5000,69.5000) -- (84.6000,69.5000) -- (84.6000,69.5000) -- (84.6000,69.5000) -- (84.6000,69.5000) -- (84.6000,69.5000) -- (84.6000,69.5000) -- (84.6000,69.5000) -- (84.6000,69.5000) -- (84.6000,69.5000) -- (84.6000,69.5000) -- (84.6000,69.5000) -- (84.6000,69.5000) -- (84.6000,69.5000) -- (84.6000,69.5000) -- (84.6000,69.5000) -- (84.6000,69.5000) -- (84.6000,69.5000) -- (84.6000,69.5000) -- (84.6000,69.5000) -- (84.6000,69.4000) -- (84.6000,69.4000) -- (84.6000,69.4000) -- (84.6000,69.4000) -- (84.6000,69.4000) -- (84.6000,69.4000) -- (84.6000,69.4000) -- (84.6000,69.4000) -- (84.6000,69.4000) -- (84.6000,69.4000) -- (84.6000,69.4000) -- (84.6000,69.4000) -- (84.6000,69.4000) -- (84.6000,69.4000) -- (84.6000,69.4000) -- (84.6000,69.4000) -- (84.6000,69.4000) -- (84.6000,69.4000) -- (84.6000,69.4000) -- (84.6000,69.4000) -- (84.6000,69.4000) -- (84.6000,69.3000) -- (84.6000,69.3000) -- (84.6000,69.3000) -- (84.6000,69.3000) -- (84.6000,69.3000) -- (84.6000,69.3000) -- (84.6000,69.3000) -- (84.6000,69.3000) -- (84.6000,69.3000) -- (84.7000,69.3000) -- (84.7000,69.3000) -- (84.7000,69.3000) -- (84.7000,69.3000) -- (84.7000,69.3000) -- (84.7000,69.3000) -- (84.7000,69.3000) -- (84.7000,69.3000) -- (84.7000,69.3000) -- (84.7000,69.3000) -- (84.7000,69.3000) -- (84.7000,69.2000) -- (84.7000,69.2000) -- (84.7000,69.2000) -- (84.7000,69.2000) -- (84.7000,69.2000) -- (84.7000,69.2000) -- (84.7000,69.2000) -- (84.7000,69.2000) -- (84.7000,69.2000) -- (84.7000,69.2000) -- (84.7000,69.2000) -- (84.7000,69.2000) -- (84.7000,69.2000) -- (84.7000,69.2000) -- (84.7000,69.2000) -- (84.7000,69.2000) -- (84.7000,69.2000) -- (84.7000,69.2000) -- (84.7000,69.2000) -- (84.7000,69.2000) -- (84.7000,69.2000) -- (84.7000,69.1000) -- (84.7000,69.1000) -- (84.7000,69.1000) -- (84.7000,69.1000) -- (84.7000,69.1000) -- (84.7000,69.1000) -- (84.7000,69.1000) -- (84.7000,69.1000) -- (84.7000,69.1000) -- (84.7000,69.1000) -- (84.7000,69.1000) -- (84.7000,69.1000) -- (84.7000,69.1000) -- (84.7000,69.1000) -- (84.7000,69.1000) -- (84.7000,69.1000) -- (84.7000,69.1000) -- (84.7000,69.1000) -- (84.8000,69.1000) -- (84.8000,69.1000) -- (84.8000,69.1000) -- (84.8000,69.0000) -- (84.8000,69.0000) -- (84.8000,69.0000) -- (84.8000,69.0000) -- (84.8000,69.0000) -- (84.8000,69.0000) -- (84.8000,69.0000) -- (84.8000,69.0000) -- (84.8000,69.0000) -- (84.8000,69.0000) -- (84.8000,69.0000) -- (84.8000,69.0000) -- (84.8000,69.0000) -- (84.8000,69.0000) -- (84.8000,69.0000) -- (84.8000,69.0000) -- (84.8000,69.0000) -- (84.8000,69.0000) -- (84.8000,69.0000) -- (84.8000,69.0000) -- (84.8000,69.0000) -- (84.8000,68.9000) -- (84.8000,68.9000) -- (84.8000,68.9000) -- (84.8000,68.9000) -- (84.8000,68.9000) -- (84.8000,68.9000) -- (84.8000,68.9000) -- (84.8000,68.9000) -- (84.8000,68.9000) -- (84.8000,68.9000) -- (84.8000,68.9000) -- (84.8000,68.9000) -- (84.8000,68.9000) -- (84.8000,68.9000) -- (84.8000,68.9000) -- (84.8000,68.9000) -- (84.8000,68.9000) -- (84.8000,68.9000) -- (84.8000,68.9000) -- (84.8000,68.9000) -- (84.8000,68.9000) -- (84.8000,68.8000) -- (84.8000,68.8000) -- (84.8000,68.8000) -- (84.8000,68.8000) -- (84.9000,68.8000) -- (84.9000,68.8000) -- (84.9000,68.8000) -- (84.9000,68.8000) -- (84.9000,68.8000) -- (84.9000,68.8000) -- (84.9000,68.8000) -- (84.9000,68.8000) -- (84.9000,68.8000) -- (84.9000,68.8000) -- (84.9000,68.8000) -- (84.9000,68.8000) -- (84.9000,68.8000) -- (84.9000,68.8000) -- (84.9000,68.8000) -- (84.9000,68.8000) -- (84.9000,68.7000) -- (84.9000,68.7000) -- (84.9000,68.7000) -- (84.9000,68.7000) -- (84.9000,68.7000) -- (84.9000,68.7000) -- (84.9000,68.7000) -- (84.9000,68.7000) -- (84.9000,68.7000) -- (84.9000,68.7000) -- (84.9000,68.7000) -- (84.9000,68.7000) -- (84.9000,68.7000) -- (84.9000,68.7000) -- (84.9000,68.7000) -- (84.9000,68.7000) -- (84.9000,68.7000) -- (84.9000,68.7000) -- (84.9000,68.7000) -- (84.9000,68.7000) -- (84.9000,68.7000) -- (84.9000,68.6000) -- (84.9000,68.6000) -- (84.9000,68.6000) -- (84.9000,68.6000) -- (84.9000,68.6000) -- (84.9000,68.6000) -- (84.9000,68.6000) -- (84.9000,68.6000) -- (84.9000,68.6000) -- (84.9000,68.6000) -- (84.9000,68.6000) -- (84.9000,68.6000) -- (84.9000,68.6000) -- (85.0000,68.6000) -- (85.0000,68.6000) -- (85.0000,68.6000) -- (85.0000,68.6000) -- (85.0000,68.6000) -- (85.0000,68.6000) -- (85.0000,68.6000) -- (85.0000,68.6000) -- (85.0000,68.5000) -- (85.0000,68.5000) -- (85.0000,68.5000) -- (85.0000,68.5000) -- (85.0000,68.5000) -- (85.0000,68.5000) -- (85.0000,68.5000) -- (85.0000,68.5000) -- (85.0000,68.5000) -- (85.0000,68.5000) -- (85.0000,68.5000) -- (85.0000,68.5000) -- (85.0000,68.5000) -- (85.0000,68.5000) -- (85.0000,68.5000) -- (85.0000,68.5000) -- (85.0000,68.5000) -- (85.0000,68.5000) -- (85.0000,68.5000) -- (85.0000,68.5000) -- (85.0000,68.5000) -- (85.0000,68.4000) -- (85.0000,68.4000) -- (85.0000,68.4000) -- (85.0000,68.4000) -- (85.0000,68.4000) -- (85.0000,68.4000) -- (85.0000,68.4000) -- (85.0000,68.4000) -- (85.0000,68.4000) -- (85.0000,68.4000) -- (85.0000,68.4000) -- (85.0000,68.4000) -- (85.0000,68.4000) -- (85.0000,68.4000) -- (85.0000,68.4000) -- (85.0000,68.4000) -- (85.0000,68.4000) -- (85.0000,68.4000) -- (85.0000,68.4000) -- (85.0000,68.4000) -- (85.1000,68.4000) -- (85.1000,68.3000) -- (85.1000,68.3000) -- (85.1000,68.3000) -- (85.1000,68.3000) -- (85.1000,68.3000) -- (85.1000,68.3000) -- (85.1000,68.3000) -- (85.1000,68.3000) -- (85.1000,68.3000) -- (85.1000,68.3000) -- (85.1000,68.3000) -- (85.1000,68.3000) -- (85.1000,68.3000) -- (85.1000,68.3000) -- (85.1000,68.3000) -- (85.1000,68.3000) -- (85.1000,68.3000) -- (85.1000,68.3000) -- (85.1000,68.3000) -- (85.1000,68.3000) -- (85.1000,68.2000) -- (85.1000,68.2000) -- (85.1000,68.2000) -- (85.1000,68.2000) -- (85.1000,68.2000) -- (85.1000,68.2000) -- (85.1000,68.2000) -- (85.1000,68.2000) -- (85.1000,68.2000) -- (85.1000,68.2000) -- (85.1000,68.2000) -- (85.1000,68.2000) -- (85.1000,68.2000) -- (85.1000,68.2000) -- (85.1000,68.2000) -- (85.1000,68.2000) -- (85.1000,68.2000) -- (85.1000,68.2000) -- (85.1000,68.2000) -- (85.1000,68.2000) -- (85.1000,68.2000) -- (85.1000,68.1000) -- (85.1000,68.1000) -- (85.1000,68.1000) -- (85.1000,68.1000) -- (85.1000,68.1000) -- (85.1000,68.1000) -- (85.1000,68.1000) -- (85.1000,68.1000) -- (85.2000,68.1000) -- (85.2000,68.1000) -- (85.2000,68.1000) -- (85.2000,68.1000) -- (85.2000,68.1000) -- (85.2000,68.1000) -- (85.2000,68.1000) -- (85.2000,68.1000) -- (85.2000,68.1000) -- (85.2000,68.1000) -- (85.2000,68.1000) -- (85.2000,68.1000) -- (85.2000,68.1000) -- (85.2000,68.0000) -- (85.2000,68.0000) -- (85.2000,68.0000) -- (85.2000,68.0000) -- (85.2000,68.0000) -- (85.2000,68.0000) -- (85.2000,68.0000) -- (85.2000,68.0000) -- (85.2000,68.0000) -- (85.2000,68.0000) -- (85.2000,68.0000) -- (85.2000,68.0000) -- (85.2000,68.0000) -- (85.2000,68.0000) -- (85.2000,68.0000) -- (85.2000,68.0000) -- (85.2000,68.0000) -- (85.2000,68.0000) -- (85.2000,68.0000) -- (85.2000,68.0000) -- (85.2000,68.0000) -- (85.2000,67.9000) -- (85.2000,67.9000) -- (85.2000,67.9000) -- (85.2000,67.9000) -- (85.2000,67.9000) -- (85.2000,67.9000) -- (85.2000,67.9000) -- (85.2000,67.9000) -- (85.2000,67.9000) -- (85.2000,67.9000) -- (85.2000,67.9000) -- (85.2000,67.9000) -- (85.2000,67.9000) -- (85.2000,67.9000) -- (85.2000,67.9000) -- (85.3000,67.9000) -- (85.3000,67.9000) -- (85.3000,67.9000) -- (85.3000,67.9000) -- (85.3000,67.9000) -- (85.3000,67.9000) -- (85.3000,67.8000) -- (85.3000,67.8000) -- (85.3000,67.8000) -- (85.3000,67.8000) -- (85.3000,67.8000) -- (85.3000,67.8000) -- (85.3000,67.8000) -- (85.3000,67.8000) -- (85.3000,67.8000) -- (85.3000,67.8000) -- (85.3000,67.8000) -- (85.3000,67.8000) -- (85.3000,67.8000) -- (85.3000,67.8000) -- (85.3000,67.8000) -- (85.3000,67.8000) -- (85.3000,67.8000) -- (85.3000,67.8000) -- (85.3000,67.8000) -- (85.3000,67.8000) -- (85.3000,67.7000) -- (85.3000,67.7000) -- (85.3000,67.7000) -- (85.3000,67.7000) -- (85.3000,67.7000) -- (85.3000,67.7000) -- (85.3000,67.7000) -- (85.3000,67.7000) -- (85.3000,67.7000) -- (85.3000,67.7000) -- (85.3000,67.7000) -- (85.3000,67.7000) -- (85.3000,67.7000) -- (85.3000,67.7000) -- (85.3000,67.7000) -- (85.3000,67.7000) -- (85.3000,67.7000) -- (85.3000,67.7000) -- (85.3000,67.7000) -- (85.3000,67.7000) -- (85.3000,67.7000) -- (85.3000,67.6000) -- (85.3000,67.6000) -- (85.3000,67.6000) -- (85.4000,67.6000) -- (85.4000,67.6000) -- (85.4000,67.6000) -- (85.4000,67.6000) -- (85.4000,67.6000) -- (85.4000,67.6000) -- (85.4000,67.6000) -- (85.4000,67.6000) -- (85.4000,67.6000) -- (85.4000,67.6000) -- (85.4000,67.6000) -- (85.4000,67.6000) -- (85.4000,67.6000) -- (85.4000,67.6000) -- (85.4000,67.6000) -- (85.4000,67.6000) -- (85.4000,67.6000) -- (85.4000,67.6000) -- (85.4000,67.5000) -- (85.4000,67.5000) -- (85.4000,67.5000) -- (85.4000,67.5000) -- (85.4000,67.5000) -- (85.4000,67.5000) -- (85.4000,67.5000) -- (85.4000,67.5000) -- (85.4000,67.5000) -- (85.4000,67.5000) -- (85.4000,67.5000) -- (85.4000,67.5000) -- (85.4000,67.5000) -- (85.4000,67.5000) -- (85.4000,67.5000) -- (85.4000,67.5000) -- (85.4000,67.5000) -- (85.4000,67.5000) -- (85.4000,67.5000) -- (85.4000,67.5000) -- (85.4000,67.5000) -- (85.4000,67.4000) -- (85.4000,67.4000) -- (85.4000,67.4000) -- (85.4000,67.4000) -- (85.4000,67.4000) -- (85.4000,67.4000) -- (85.4000,67.4000) -- (85.4000,67.4000) -- (85.4000,67.4000) -- (85.4000,67.4000) -- (85.5000,67.4000) -- (85.5000,67.4000) -- (85.5000,67.4000) -- (85.5000,67.4000) -- (85.5000,67.4000) -- (85.5000,67.4000) -- (85.5000,67.4000) -- (85.5000,67.4000) -- (85.5000,67.4000) -- (85.5000,67.4000) -- (85.5000,67.3000) -- (85.5000,67.3000) -- (85.5000,67.3000) -- (85.5000,67.3000) -- (85.5000,67.3000) -- (85.5000,67.3000) -- (85.5000,67.3000) -- (85.5000,67.3000) -- (85.5000,67.3000) -- (85.5000,67.3000) -- (85.5000,67.3000) -- (85.5000,67.3000) -- (85.5000,67.3000) -- (85.5000,67.3000) -- (85.5000,67.3000) -- (85.5000,67.3000) -- (85.5000,67.3000) -- (85.5000,67.3000) -- (85.5000,67.3000) -- (85.5000,67.3000) -- (85.5000,67.3000) -- (85.5000,67.2000) -- (85.5000,67.2000) -- (85.5000,67.2000) -- (85.5000,67.2000) -- (85.5000,67.2000) -- (85.5000,67.2000) -- (85.5000,67.2000) -- (85.5000,67.2000) -- (85.5000,67.2000) -- (85.5000,67.2000) -- (85.5000,67.2000) -- (85.5000,67.2000) -- (85.5000,67.2000) -- (85.5000,67.2000) -- (85.5000,67.2000) -- (85.5000,67.2000) -- (85.5000,67.2000) -- (85.5000,67.2000) -- (85.5000,67.2000) -- (85.6000,67.2000) -- (85.6000,67.2000) -- (85.6000,67.1000) -- (85.6000,67.1000) -- (85.6000,67.1000) -- (85.6000,67.1000) -- (85.6000,67.1000) -- (85.6000,67.1000) -- (85.6000,67.1000) -- (85.6000,67.1000) -- (85.6000,67.1000) -- (85.6000,67.1000) -- (85.6000,67.1000) -- (85.6000,67.1000) -- (85.6000,67.1000) -- (85.6000,67.1000) -- (85.6000,67.1000) -- (85.6000,67.1000) -- (85.6000,67.1000) -- (85.6000,67.1000) -- (85.6000,67.1000) -- (85.6000,67.1000) -- (85.6000,67.1000) -- (85.6000,67.0000) -- (85.6000,67.0000) -- (85.6000,67.0000) -- (85.6000,67.0000) -- (85.6000,67.0000) -- (85.6000,67.0000) -- (85.6000,67.0000) -- (85.6000,67.0000) -- (85.6000,67.0000) -- (85.6000,67.0000) -- (85.6000,67.0000) -- (85.6000,67.0000) -- (85.6000,67.0000) -- (85.6000,67.0000) -- (85.6000,67.0000) -- (85.6000,67.0000) -- (85.6000,67.0000) -- (85.6000,67.0000) -- (85.6000,67.0000) -- (85.6000,67.0000) -- (85.6000,67.0000) -- (85.6000,66.9000) -- (85.6000,66.9000) -- (85.6000,66.9000) -- (85.6000,66.9000) -- (85.6000,66.9000) -- (85.7000,66.9000) -- (85.7000,66.9000) -- (85.7000,66.9000) -- (85.7000,66.9000) -- (85.7000,66.9000) -- (85.7000,66.9000) -- (85.7000,66.9000) -- (85.7000,66.9000) -- (85.7000,66.9000) -- (85.7000,66.9000) -- (85.7000,66.9000) -- (85.7000,66.9000) -- (85.7000,66.9000) -- (85.7000,66.9000) -- (85.7000,66.9000) -- (85.7000,66.8000) -- (85.7000,66.8000) -- (85.7000,66.8000) -- (85.7000,66.8000) -- (85.7000,66.8000) -- (85.7000,66.8000) -- (85.7000,66.8000) -- (85.7000,66.8000) -- (85.7000,66.8000) -- (85.7000,66.8000) -- (85.7000,66.8000) -- (85.7000,66.8000) -- (85.7000,66.8000) -- (85.7000,66.8000) -- (85.7000,66.8000) -- (85.7000,66.8000) -- (85.7000,66.8000) -- (85.7000,66.8000) -- (85.7000,66.8000) -- (85.7000,66.8000) -- (85.7000,66.8000) -- (85.7000,66.7000) -- (85.7000,66.7000) -- (85.7000,66.7000) -- (85.7000,66.7000) -- (85.7000,66.7000) -- (85.7000,66.7000) -- (85.7000,66.7000) -- (85.7000,66.7000) -- (85.7000,66.7000) -- (85.7000,66.7000) -- (85.7000,66.7000) -- (85.7000,66.7000) -- (85.7000,66.7000) -- (85.7000,66.7000) -- (85.8000,66.7000) -- (85.8000,66.7000) -- (85.8000,66.7000) -- (85.8000,66.7000) -- (85.8000,66.7000) -- (85.8000,66.7000) -- (85.8000,66.7000) -- (85.8000,66.6000) -- (85.8000,66.6000) -- (85.8000,66.6000) -- (85.8000,66.6000) -- (85.8000,66.6000) -- (85.8000,66.6000) -- (85.8000,66.6000) -- (85.8000,66.6000) -- (85.8000,66.6000) -- (85.8000,66.6000) -- (85.8000,66.6000) -- (85.8000,66.6000) -- (85.8000,66.6000) -- (85.8000,66.6000) -- (85.8000,66.6000) -- (85.8000,66.6000) -- (85.8000,66.6000) -- (85.8000,66.6000) -- (85.8000,66.6000) -- (85.8000,66.6000) -- (85.8000,66.6000) -- (85.8000,66.5000) -- (85.8000,66.5000) -- (85.8000,66.5000) -- (85.8000,66.5000) -- (85.8000,66.5000) -- (85.8000,66.5000) -- (85.8000,66.5000) -- (85.8000,66.5000) -- (85.8000,66.5000) -- (85.8000,66.5000) -- (85.8000,66.5000) -- (85.8000,66.5000) -- (85.8000,66.5000) -- (85.8000,66.5000) -- (85.8000,66.5000) -- (85.8000,66.5000) -- (85.8000,66.5000) -- (85.8000,66.5000) -- (85.8000,66.5000) -- (85.8000,66.5000) -- (85.8000,66.5000) -- (85.9000,66.4000) -- (85.9000,66.4000) -- (85.9000,66.4000) -- (85.9000,66.4000) -- (85.9000,66.4000) -- (85.9000,66.4000) -- (85.9000,66.4000) -- (85.9000,66.4000) -- (85.9000,66.4000) -- (85.9000,66.4000) -- (85.9000,66.4000) -- (85.9000,66.4000) -- (85.9000,66.4000) -- (85.9000,66.4000) -- (85.9000,66.4000) -- (85.9000,66.4000) -- (85.9000,66.4000) -- (85.9000,66.4000) -- (85.9000,66.4000) -- (85.9000,66.4000) -- (85.9000,66.3000) -- (85.9000,66.3000) -- (85.9000,66.3000) -- (85.9000,66.3000) -- (85.9000,66.3000) -- (85.9000,66.3000) -- (85.9000,66.3000) -- (85.9000,66.3000) -- (85.9000,66.3000) -- (85.9000,66.3000) -- (85.9000,66.3000) -- (85.9000,66.3000) -- (85.9000,66.3000) -- (85.9000,66.3000) -- (85.9000,66.3000) -- (85.9000,66.3000) -- (85.9000,66.3000) -- (85.9000,66.3000) -- (85.9000,66.3000) -- (85.9000,66.3000) -- (85.9000,66.3000) -- (85.9000,66.2000) -- (85.9000,66.2000) -- (85.9000,66.2000) -- (85.9000,66.2000) -- (85.9000,66.2000) -- (85.9000,66.2000) -- (85.9000,66.2000) -- (85.9000,66.2000) -- (85.9000,66.2000) -- (86.0000,66.2000) -- (86.0000,66.2000) -- (86.0000,66.2000) -- (86.0000,66.2000) -- (86.0000,66.2000) -- (86.0000,66.2000) -- (86.0000,66.2000) -- (86.0000,66.2000) -- (86.0000,66.2000) -- (86.0000,66.2000) -- (86.0000,66.2000) -- (86.0000,66.2000) -- (86.0000,66.1000) -- (86.0000,66.1000) -- (86.0000,66.1000) -- (86.0000,66.1000) -- (86.0000,66.1000) -- (86.0000,66.1000) -- (86.0000,66.1000) -- (86.0000,66.1000) -- (86.0000,66.1000) -- (86.0000,66.1000) -- (86.0000,66.1000) -- (86.0000,66.1000) -- (86.0000,66.1000) -- (86.0000,66.1000) -- (86.0000,66.1000) -- (86.0000,66.1000) -- (86.0000,66.1000) -- (86.0000,66.1000) -- (86.0000,66.1000) -- (86.0000,66.1000) -- (86.0000,66.1000) -- (86.0000,66.0000) -- (86.0000,66.0000) -- (86.0000,66.0000) -- (86.0000,66.0000) -- (86.0000,66.0000) -- (86.0000,66.0000) -- (86.0000,66.0000) -- (86.0000,66.0000) -- (86.0000,66.0000) -- (86.0000,66.0000) -- (86.0000,66.0000) -- (86.0000,66.0000) -- (86.0000,66.0000) -- (86.0000,66.0000) -- (86.0000,66.0000) -- (86.0000,66.0000) -- (86.1000,66.0000) -- (86.1000,66.0000) -- (86.1000,66.0000) -- (86.1000,66.0000) -- (86.1000,66.0000) -- (86.1000,65.9000) -- (86.1000,65.9000) -- (86.1000,65.9000) -- (86.1000,65.9000) -- (86.1000,65.9000) -- (86.1000,65.9000) -- (86.1000,65.9000) -- (86.1000,65.9000) -- (86.1000,65.9000) -- (86.1000,65.9000) -- (86.1000,65.9000) -- (86.1000,65.9000) -- (86.1000,65.9000) -- (86.1000,65.9000) -- (86.1000,65.9000) -- (86.1000,65.9000) -- (86.1000,65.9000) -- (86.1000,65.9000) -- (86.1000,65.9000) -- (86.1000,65.9000) -- (86.1000,65.8000) -- (86.1000,65.8000) -- (86.1000,65.8000) -- (86.1000,65.8000) -- (86.1000,65.8000) -- (86.1000,65.8000) -- (86.1000,65.8000) -- (86.1000,65.8000) -- (86.1000,65.8000) -- (86.1000,65.8000) -- (86.1000,65.8000) -- (86.1000,65.8000) -- (86.1000,65.8000) -- (86.1000,65.8000) -- (86.1000,65.8000) -- (86.1000,65.8000) -- (86.1000,65.8000) -- (86.1000,65.8000) -- (86.1000,65.8000) -- (86.1000,65.8000) -- (86.1000,65.8000) -- (86.1000,65.7000) -- (86.1000,65.7000) -- (86.1000,65.7000) -- (86.1000,65.7000) -- (86.2000,65.7000) -- (86.2000,65.7000) -- (86.2000,65.7000) -- (86.2000,65.7000) -- (86.2000,65.7000) -- (86.2000,65.7000) -- (86.2000,65.7000) -- (86.2000,65.7000) -- (86.2000,65.7000) -- (86.2000,65.7000) -- (86.2000,65.7000) -- (86.2000,65.7000) -- (86.2000,65.7000) -- (86.2000,65.7000) -- (86.2000,65.7000) -- (86.2000,65.7000) -- (86.2000,65.7000) -- (86.2000,65.6000) -- (86.2000,65.6000) -- (86.2000,65.6000) -- (86.2000,65.6000) -- (86.2000,65.6000) -- (86.2000,65.6000) -- (86.2000,65.6000) -- (86.2000,65.6000) -- (86.2000,65.6000) -- (86.2000,65.6000) -- (86.2000,65.6000) -- (86.2000,65.6000) -- (86.2000,65.6000) -- (86.2000,65.6000) -- (86.2000,65.6000) -- (86.2000,65.6000) -- (86.2000,65.6000) -- (86.2000,65.6000) -- (86.2000,65.6000) -- (86.2000,65.6000) -- (86.2000,65.6000) -- (86.2000,65.5000) -- (86.2000,65.5000) -- (86.2000,65.5000) -- (86.2000,65.5000) -- (86.2000,65.5000) -- (86.2000,65.5000) -- (86.2000,65.5000) -- (86.2000,65.5000) -- (86.2000,65.5000) -- (86.2000,65.5000) -- (86.2000,65.5000) -- (86.2000,65.5000) -- (86.3000,65.5000) -- (86.3000,65.5000) -- (86.3000,65.5000) -- (86.3000,65.5000) -- (86.3000,65.5000) -- (86.3000,65.5000) -- (86.3000,65.5000) -- (86.3000,65.5000) -- (86.3000,65.4000) -- (86.3000,65.4000) -- (86.3000,65.4000) -- (86.3000,65.4000) -- (86.3000,65.4000) -- (86.3000,65.4000) -- (86.3000,65.4000) -- (86.3000,65.4000) -- (86.3000,65.4000) -- (86.3000,65.4000) -- (86.3000,65.4000) -- (86.3000,65.4000) -- (86.3000,65.4000) -- (86.3000,65.4000) -- (86.3000,65.4000) -- (86.3000,65.4000) -- (86.3000,65.4000) -- (86.3000,65.4000) -- (86.3000,65.4000) -- (86.3000,65.4000) -- (86.3000,65.4000) -- (86.3000,65.3000) -- (86.3000,65.3000) -- (86.3000,65.3000) -- (86.3000,65.3000) -- (86.3000,65.3000) -- (86.3000,65.3000) -- (86.3000,65.3000) -- (86.3000,65.3000) -- (86.3000,65.3000) -- (86.3000,65.3000) -- (86.3000,65.3000) -- (86.3000,65.3000) -- (86.3000,65.3000) -- (86.3000,65.3000) -- (86.3000,65.3000) -- (86.3000,65.3000) -- (86.3000,65.3000) -- (86.3000,65.3000) -- (86.3000,65.3000) -- (86.3000,65.3000) -- (86.4000,65.3000) -- (86.4000,65.2000) -- (86.4000,65.2000) -- (86.4000,65.2000) -- (86.4000,65.2000) -- (86.4000,65.2000) -- (86.4000,65.2000) -- (86.4000,65.2000) -- (86.4000,65.2000) -- (86.4000,65.2000) -- (86.4000,65.2000) -- (86.4000,65.2000) -- (86.4000,65.2000) -- (86.4000,65.2000) -- (86.4000,65.2000) -- (86.4000,65.2000) -- (86.4000,65.2000) -- (86.4000,65.2000) -- (86.4000,65.2000) -- (86.4000,65.2000) -- (86.4000,65.2000) -- (86.4000,65.2000) -- (86.4000,65.1000) -- (86.4000,65.1000) -- (86.4000,65.1000) -- (86.4000,65.1000) -- (86.4000,65.1000) -- (86.4000,65.1000) -- (86.4000,65.1000) -- (86.4000,65.1000) -- (86.4000,65.1000) -- (86.4000,65.1000) -- (86.4000,65.1000) -- (86.4000,65.1000) -- (86.4000,65.1000) -- (86.4000,65.1000) -- (86.4000,65.1000) -- (86.4000,65.1000) -- (86.4000,65.1000) -- (86.4000,65.1000) -- (86.4000,65.1000) -- (86.4000,65.1000) -- (86.4000,65.1000) -- (86.4000,65.0000) -- (86.4000,65.0000) -- (86.4000,65.0000) -- (86.4000,65.0000) -- (86.4000,65.0000) -- (86.4000,65.0000) -- (86.4000,65.0000) -- (86.5000,65.0000) -- (86.5000,65.0000) -- (86.5000,65.0000) -- (86.5000,65.0000) -- (86.5000,65.0000) -- (86.5000,65.0000) -- (86.5000,65.0000) -- (86.5000,65.0000) -- (86.5000,65.0000) -- (86.5000,65.0000) -- (86.5000,65.0000) -- (86.5000,65.0000) -- (86.5000,65.0000) -- (86.5000,64.9000) -- (86.5000,64.9000) -- (86.5000,64.9000) -- (86.5000,64.9000) -- (86.5000,64.9000) -- (86.5000,64.9000) -- (86.5000,64.9000) -- (86.5000,64.9000) -- (86.5000,64.9000) -- (86.5000,64.9000) -- (86.5000,64.9000) -- (86.5000,64.9000) -- (86.5000,64.9000) -- (86.5000,64.9000) -- (86.5000,64.9000) -- (86.5000,64.9000) -- (86.5000,64.9000) -- (86.5000,64.9000) -- (86.5000,64.9000) -- (86.5000,64.9000) -- (86.5000,64.9000) -- (86.5000,64.8000) -- (86.5000,64.8000) -- (86.5000,64.8000) -- (86.5000,64.8000) -- (86.5000,64.8000) -- (86.5000,64.8000) -- (86.5000,64.8000) -- (86.5000,64.8000) -- (86.5000,64.8000) -- (86.5000,64.8000) -- (86.5000,64.8000) -- (86.5000,64.8000) -- (86.5000,64.8000) -- (86.5000,64.8000) -- (86.5000,64.8000) -- (86.6000,64.8000) -- (86.6000,64.8000) -- (86.6000,64.8000) -- (86.6000,64.8000) -- (86.6000,64.8000) -- (86.6000,64.8000) -- (86.6000,64.7000) -- (86.6000,64.7000) -- (86.6000,64.7000) -- (86.6000,64.7000) -- (86.6000,64.7000) -- (86.6000,64.7000) -- (86.6000,64.7000) -- (86.6000,64.7000) -- (86.6000,64.7000) -- (86.6000,64.7000) -- (86.6000,64.7000) -- (86.6000,64.7000) -- (86.6000,64.7000) -- (86.6000,64.7000) -- (86.6000,64.7000) -- (86.6000,64.7000) -- (86.6000,64.7000) -- (86.6000,64.7000) -- (86.6000,64.7000) -- (86.6000,64.7000) -- (86.6000,64.7000) -- (86.6000,64.6000) -- (86.6000,64.6000) -- (86.6000,64.6000) -- (86.6000,64.6000) -- (86.6000,64.6000) -- (86.6000,64.6000) -- (86.6000,64.6000) -- (86.6000,64.6000) -- (86.6000,64.6000) -- (86.6000,64.6000) -- (86.6000,64.6000) -- (86.6000,64.6000) -- (86.6000,64.6000) -- (86.6000,64.6000) -- (86.6000,64.6000) -- (86.6000,64.6000) -- (86.6000,64.6000) -- (86.6000,64.6000) -- (86.6000,64.6000) -- (86.6000,64.6000) -- (86.6000,64.6000) -- (86.6000,64.5000) -- (86.6000,64.5000) -- (86.7000,64.5000) -- (86.7000,64.5000) -- (86.7000,64.5000) -- (86.7000,64.5000) -- (86.7000,64.5000) -- (86.7000,64.5000) -- (86.7000,64.5000) -- (86.7000,64.5000) -- (86.7000,64.5000) -- (86.7000,64.5000) -- (86.7000,64.5000) -- (86.7000,64.5000) -- (86.7000,64.5000) -- (86.7000,64.5000) -- (86.7000,64.5000) -- (86.7000,64.5000) -- (86.7000,64.5000) -- (86.7000,64.5000) -- (86.7000,64.4000) -- (86.7000,64.4000) -- (86.7000,64.4000) -- (86.7000,64.4000) -- (86.7000,64.4000) -- (86.7000,64.4000) -- (86.7000,64.4000) -- (86.7000,64.4000) -- (86.7000,64.4000) -- (86.7000,64.4000) -- (86.7000,64.4000) -- (86.7000,64.4000) -- (86.7000,64.4000) -- (86.7000,64.4000) -- (86.7000,64.4000) -- (86.7000,64.4000) -- (86.7000,64.4000) -- (86.7000,64.4000) -- (86.7000,64.4000) -- (86.7000,64.4000) -- (86.7000,64.4000) -- (86.7000,64.3000) -- (86.7000,64.3000) -- (86.7000,64.3000) -- (86.7000,64.3000) -- (86.7000,64.3000) -- (86.7000,64.3000) -- (86.7000,64.3000) -- (86.7000,64.3000) -- (86.7000,64.3000) -- (86.7000,64.3000) -- (86.8000,64.3000) -- (86.8000,64.3000) -- (86.8000,64.3000) -- (86.8000,64.3000) -- (86.8000,64.3000) -- (86.8000,64.3000) -- (86.8000,64.3000) -- (86.8000,64.3000) -- (86.8000,64.3000) -- (86.8000,64.3000) -- (86.8000,64.3000) -- (86.8000,64.2000) -- (86.8000,64.2000) -- (86.8000,64.2000) -- (86.8000,64.2000) -- (86.8000,64.2000) -- (86.8000,64.2000) -- (86.8000,64.2000) -- (86.8000,64.2000) -- (86.8000,64.2000) -- (86.8000,64.2000) -- (86.8000,64.2000) -- (86.8000,64.2000) -- (86.8000,64.2000) -- (86.8000,64.2000) -- (86.8000,64.2000) -- (86.8000,64.2000) -- (86.8000,64.2000) -- (86.8000,64.2000) -- (86.8000,64.2000) -- (86.8000,64.2000) -- (86.8000,64.2000) -- (86.8000,64.1000) -- (86.8000,64.1000) -- (86.8000,64.1000) -- (86.8000,64.1000) -- (86.8000,64.1000) -- (86.8000,64.1000) -- (86.8000,64.1000) -- (86.8000,64.1000) -- (86.8000,64.1000) -- (86.8000,64.1000) -- (86.8000,64.1000) -- (86.8000,64.1000) -- (86.8000,64.1000) -- (86.8000,64.1000) -- (86.8000,64.1000) -- (86.8000,64.1000) -- (86.8000,64.1000) -- (86.8000,64.1000) -- (86.9000,64.1000) -- (86.9000,64.1000) -- (86.9000,64.0000) -- (86.9000,64.0000) -- (86.9000,64.0000) -- (86.9000,64.0000) -- (86.9000,64.0000) -- (86.9000,64.0000) -- (86.9000,64.0000) -- (86.9000,64.0000) -- (86.9000,64.0000) -- (86.9000,64.0000) -- (86.9000,64.0000) -- (86.9000,64.0000) -- (86.9000,64.0000) -- (86.9000,64.0000) -- (86.9000,64.0000) -- (86.9000,64.0000) -- (86.9000,64.0000) -- (86.9000,64.0000) -- (86.9000,64.0000) -- (86.9000,64.0000) -- (86.9000,64.0000) -- (86.9000,63.9000) -- (86.9000,63.9000) -- (86.9000,63.9000) -- (86.9000,63.9000) -- (86.9000,63.9000) -- (86.9000,63.9000) -- (86.9000,63.9000) -- (86.9000,63.9000) -- (86.9000,63.9000) -- (86.9000,63.9000) -- (86.9000,63.9000) -- (86.9000,63.9000) -- (86.9000,63.9000) -- (86.9000,63.9000) -- (86.9000,63.9000) -- (86.9000,63.9000) -- (86.9000,63.9000) -- (86.9000,63.9000) -- (86.9000,63.9000) -- (86.9000,63.9000) -- (86.9000,63.9000) -- (86.9000,63.8000) -- (86.9000,63.8000) -- (86.9000,63.8000) -- (86.9000,63.8000) -- (86.9000,63.8000) -- (87.0000,63.8000) -- (87.0000,63.8000) -- (87.0000,63.8000) -- (87.0000,63.8000) -- (87.0000,63.8000) -- (87.0000,63.8000) -- (87.0000,63.8000) -- (87.0000,63.8000) -- (87.0000,63.8000) -- (87.0000,63.8000) -- (87.0000,63.8000) -- (87.0000,63.8000) -- (87.0000,63.8000) -- (87.0000,63.8000) -- (87.0000,63.8000) -- (87.0000,63.8000) -- (87.0000,63.7000) -- (87.0000,63.7000) -- (87.0000,63.7000) -- (87.0000,63.7000) -- (87.0000,63.7000) -- (87.0000,63.7000) -- (87.0000,63.7000) -- (87.0000,63.7000) -- (87.0000,63.7000) -- (87.0000,63.7000) -- (87.0000,63.7000) -- (87.0000,63.7000) -- (87.0000,63.7000) -- (87.0000,63.7000) -- (87.0000,63.7000) -- (87.0000,63.7000) -- (87.0000,63.7000) -- (87.0000,63.7000) -- (87.0000,63.7000) -- (87.0000,63.7000) -- (87.0000,63.7000) -- (87.0000,63.6000) -- (87.0000,63.6000) -- (87.0000,63.6000) -- (87.0000,63.6000) -- (87.0000,63.6000) -- (87.0000,63.6000) -- (87.0000,63.6000) -- (87.0000,63.6000) -- (87.0000,63.6000) -- (87.0000,63.6000) -- (87.0000,63.6000) -- (87.0000,63.6000) -- (87.0000,63.6000) -- (87.1000,63.6000) -- (87.1000,63.6000) -- (87.1000,63.6000) -- (87.1000,63.6000) -- (87.1000,63.6000) -- (87.1000,63.6000) -- (87.1000,63.6000) -- (87.1000,63.5000) -- (87.1000,63.5000) -- (87.1000,63.5000) -- (87.1000,63.5000) -- (87.1000,63.5000) -- (87.1000,63.5000) -- (87.1000,63.5000) -- (87.1000,63.5000) -- (87.1000,63.5000) -- (87.1000,63.5000) -- (87.1000,63.5000) -- (87.1000,63.5000) -- (87.1000,63.5000) -- (87.1000,63.5000) -- (87.1000,63.5000) -- (87.1000,63.5000) -- (87.1000,63.5000) -- (87.1000,63.5000) -- (87.1000,63.5000) -- (87.1000,63.5000) -- (87.1000,63.5000) -- (87.1000,63.4000) -- (87.1000,63.4000) -- (87.1000,63.4000) -- (87.1000,63.4000) -- (87.1000,63.4000) -- (87.1000,63.4000) -- (87.1000,63.4000) -- (87.1000,63.4000) -- (87.1000,63.4000) -- (87.1000,63.4000) -- (87.1000,63.4000) -- (87.1000,63.4000) -- (87.1000,63.4000) -- (87.1000,63.4000) -- (87.1000,63.4000) -- (87.1000,63.4000) -- (87.1000,63.4000) -- (87.1000,63.4000) -- (87.1000,63.4000) -- (87.1000,63.4000) -- (87.1000,63.4000) -- (87.2000,63.3000) -- (87.2000,63.3000) -- (87.2000,63.3000) -- (87.2000,63.3000) -- (87.2000,63.3000) -- (87.2000,63.3000) -- (87.2000,63.3000) -- (87.2000,63.3000) -- (87.2000,63.3000) -- (87.2000,63.3000) -- (87.2000,63.3000) -- (87.2000,63.3000) -- (87.2000,63.3000) -- (87.2000,63.3000) -- (87.2000,63.3000) -- (87.2000,63.3000) -- (87.2000,63.3000) -- (87.2000,63.3000) -- (87.2000,63.3000) -- (87.2000,63.3000) -- (87.2000,63.3000) -- (87.2000,63.2000) -- (87.2000,63.2000) -- (87.2000,63.2000) -- (87.2000,63.2000) -- (87.2000,63.2000) -- (87.2000,63.2000) -- (87.2000,63.2000) -- (87.2000,63.2000) -- (87.2000,63.2000) -- (87.2000,63.2000) -- (87.2000,63.2000) -- (87.2000,63.2000) -- (87.2000,63.2000) -- (87.2000,63.2000) -- (87.2000,63.2000) -- (87.2000,63.2000) -- (87.2000,63.2000) -- (87.2000,63.2000) -- (87.2000,63.2000) -- (87.2000,63.2000) -- (87.2000,63.2000) -- (87.2000,63.1000) -- (87.2000,63.1000) -- (87.2000,63.1000) -- (87.2000,63.1000) -- (87.2000,63.1000) -- (87.2000,63.1000) -- (87.2000,63.1000) -- (87.2000,63.1000) -- (87.3000,63.1000) -- (87.3000,63.1000) -- (87.3000,63.1000) -- (87.3000,63.1000) -- (87.3000,63.1000) -- (87.3000,63.1000) -- (87.3000,63.1000) -- (87.3000,63.1000) -- (87.3000,63.1000) -- (87.3000,63.1000) -- (87.3000,63.1000) -- (87.3000,63.1000) -- (87.3000,63.0000) -- (87.3000,63.0000) -- (87.3000,63.0000) -- (87.3000,63.0000) -- (87.3000,63.0000) -- (87.3000,63.0000) -- (87.3000,63.0000) -- (87.3000,63.0000) -- (87.3000,63.0000) -- (87.3000,63.0000) -- (87.3000,63.0000) -- (87.3000,63.0000) -- (87.3000,63.0000) -- (87.3000,63.0000) -- (87.3000,63.0000) -- (87.3000,63.0000) -- (87.3000,63.0000) -- (87.3000,63.0000) -- (87.3000,63.0000) -- (87.3000,63.0000) -- (87.3000,63.0000) -- (87.3000,62.9000) -- (87.3000,62.9000) -- (87.3000,62.9000) -- (87.3000,62.9000) -- (87.3000,62.9000) -- (87.3000,62.9000) -- (87.3000,62.9000) -- (87.3000,62.9000) -- (87.3000,62.9000) -- (87.3000,62.9000) -- (87.3000,62.9000) -- (87.3000,62.9000) -- (87.3000,62.9000) -- (87.3000,62.9000) -- (87.3000,62.9000) -- (87.3000,62.9000) -- (87.4000,62.9000) -- (87.4000,62.9000) -- (87.4000,62.9000) -- (87.4000,62.9000) -- (87.4000,62.9000) -- (87.4000,62.8000) -- (87.4000,62.8000) -- (87.4000,62.8000) -- (87.4000,62.8000) -- (87.4000,62.8000) -- (87.4000,62.8000) -- (87.4000,62.8000) -- (87.4000,62.8000) -- (87.4000,62.8000) -- (87.4000,62.8000) -- (87.4000,62.8000) -- (87.4000,62.8000) -- (87.4000,62.8000) -- (87.4000,62.8000) -- (87.4000,62.8000) -- (87.4000,62.8000) -- (87.4000,62.8000) -- (87.4000,62.8000) -- (87.4000,62.8000) -- (87.4000,62.8000) -- (87.4000,62.8000) -- (87.4000,62.7000) -- (87.4000,62.7000) -- (87.4000,62.7000) -- (87.4000,62.7000) -- (87.4000,62.7000) -- (87.4000,62.7000) -- (87.4000,62.7000) -- (87.4000,62.7000) -- (87.4000,62.7000) -- (87.4000,62.7000) -- (87.4000,62.7000) -- (87.4000,62.7000) -- (87.4000,62.7000) -- (87.4000,62.7000) -- (87.4000,62.7000) -- (87.4000,62.7000) -- (87.4000,62.7000) -- (87.4000,62.7000) -- (87.4000,62.7000) -- (87.4000,62.7000) -- (87.4000,62.7000) -- (87.4000,62.6000) -- (87.4000,62.6000) -- (87.4000,62.6000) -- (87.5000,62.6000) -- (87.5000,62.6000) -- (87.5000,62.6000) -- (87.5000,62.6000) -- (87.5000,62.6000) -- (87.5000,62.6000) -- (87.5000,62.6000) -- (87.5000,62.6000) -- (87.5000,62.6000) -- (87.5000,62.6000) -- (87.5000,62.6000) -- (87.5000,62.6000) -- (87.5000,62.6000) -- (87.5000,62.6000) -- (87.5000,62.6000) -- (87.5000,62.6000) -- (87.5000,62.6000) -- (87.5000,62.5000) -- (87.5000,62.5000) -- (87.5000,62.5000) -- (87.5000,62.5000) -- (87.5000,62.5000) -- (87.5000,62.5000) -- (87.5000,62.5000) -- (87.5000,62.5000) -- (87.5000,62.5000) -- (87.5000,62.5000) -- (87.5000,62.5000) -- (87.5000,62.5000) -- (87.5000,62.5000) -- (87.5000,62.5000) -- (87.5000,62.5000) -- (87.5000,62.5000) -- (87.5000,62.5000) -- (87.5000,62.5000) -- (87.5000,62.5000) -- (87.5000,62.5000) -- (87.5000,62.5000) -- (87.5000,62.4000) -- (87.5000,62.4000) -- (87.5000,62.4000) -- (87.5000,62.4000) -- (87.5000,62.4000) -- (87.5000,62.4000) -- (87.5000,62.4000) -- (87.5000,62.4000) -- (87.5000,62.4000) -- (87.5000,62.4000) -- (87.5000,62.4000) -- (87.6000,62.4000) -- (87.6000,62.4000) -- (87.6000,62.4000) -- (87.6000,62.4000) -- (87.6000,62.4000) -- (87.6000,62.4000) -- (87.6000,62.4000) -- (87.6000,62.4000) -- (87.6000,62.4000) -- (87.6000,62.4000) -- (87.6000,62.3000) -- (87.6000,62.3000) -- (87.6000,62.3000) -- (87.6000,62.3000) -- (87.6000,62.3000) -- (87.6000,62.3000) -- (87.6000,62.3000) -- (87.6000,62.3000) -- (87.6000,62.3000) -- (87.6000,62.3000) -- (87.6000,62.3000) -- (87.6000,62.3000) -- (87.6000,62.3000) -- (87.6000,62.3000) -- (87.6000,62.3000) -- (87.6000,62.3000) -- (87.6000,62.3000) -- (87.6000,62.3000) -- (87.6000,62.3000) -- (87.6000,62.3000) -- (87.6000,62.3000) -- (87.6000,62.2000) -- (87.6000,62.2000) -- (87.6000,62.2000) -- (87.6000,62.2000) -- (87.6000,62.2000) -- (87.6000,62.2000) -- (87.6000,62.2000) -- (87.6000,62.2000) -- (87.6000,62.2000) -- (87.6000,62.2000) -- (87.6000,62.2000) -- (87.6000,62.2000) -- (87.6000,62.2000) -- (87.6000,62.2000) -- (87.6000,62.2000) -- (87.6000,62.2000) -- (87.6000,62.2000) -- (87.6000,62.2000) -- (87.6000,62.2000) -- (87.7000,62.2000) -- (87.7000,62.1000) -- (87.7000,62.1000) -- (87.7000,62.1000) -- (87.7000,62.1000) -- (87.7000,62.1000) -- (87.7000,62.1000) -- (87.7000,62.1000) -- (87.7000,62.1000) -- (87.7000,62.1000) -- (87.7000,62.1000) -- (87.7000,62.1000) -- (87.7000,62.1000) -- (87.7000,62.1000) -- (87.7000,62.1000) -- (87.7000,62.1000) -- (87.7000,62.1000) -- (87.7000,62.1000) -- (87.7000,62.1000) -- (87.7000,62.1000) -- (87.7000,62.1000) -- (87.7000,62.1000) -- (87.7000,62.0000) -- (87.7000,62.0000) -- (87.7000,62.0000) -- (87.7000,62.0000) -- (87.7000,62.0000) -- (87.7000,62.0000) -- (87.7000,62.0000) -- (87.7000,62.0000) -- (87.7000,62.0000) -- (87.7000,62.0000) -- (87.7000,62.0000) -- (87.7000,62.0000) -- (87.7000,62.0000) -- (87.7000,62.0000) -- (87.7000,62.0000) -- (87.7000,62.0000) -- (87.7000,62.0000) -- (87.7000,62.0000) -- (87.7000,62.0000) -- (87.7000,62.0000) -- (87.7000,62.0000) -- (87.7000,61.9000) -- (87.7000,61.9000) -- (87.7000,61.9000) -- (87.7000,61.9000) -- (87.7000,61.9000) -- (87.7000,61.9000) -- (87.8000,61.9000) -- (87.8000,61.9000) -- (87.8000,61.9000) -- (87.8000,61.9000) -- (87.8000,61.9000) -- (87.8000,61.9000) -- (87.8000,61.9000) -- (87.8000,61.9000) -- (87.8000,61.9000) -- (87.8000,61.9000) -- (87.8000,61.9000) -- (87.8000,61.9000) -- (87.8000,61.9000) -- (87.8000,61.9000) -- (87.8000,61.9000) -- (87.8000,61.8000) -- (87.8000,61.8000) -- (87.8000,61.8000) -- (87.8000,61.8000) -- (87.8000,61.8000) -- (87.8000,61.8000) -- (87.8000,61.8000) -- (87.8000,61.8000) -- (87.8000,61.8000) -- (87.8000,61.8000) -- (87.8000,61.8000) -- (87.8000,61.8000) -- (87.8000,61.8000) -- (87.8000,61.8000) -- (87.8000,61.8000) -- (87.8000,61.8000) -- (87.8000,61.8000) -- (87.8000,61.8000) -- (87.8000,61.8000) -- (87.8000,61.8000) -- (87.8000,61.8000) -- (87.8000,61.7000) -- (87.8000,61.7000) -- (87.8000,61.7000) -- (87.8000,61.7000) -- (87.8000,61.7000) -- (87.8000,61.7000) -- (87.8000,61.7000) -- (87.8000,61.7000) -- (87.8000,61.7000) -- (87.8000,61.7000) -- (87.8000,61.7000) -- (87.8000,61.7000) -- (87.8000,61.7000) -- (87.8000,61.7000) -- (87.9000,61.7000) -- (87.9000,61.7000) -- (87.9000,61.7000) -- (87.9000,61.7000) -- (87.9000,61.7000) -- (87.9000,61.7000) -- (87.9000,61.6000) -- (87.9000,61.6000) -- (87.9000,61.6000) -- (87.9000,61.6000) -- (87.9000,61.6000) -- (87.9000,61.6000) -- (87.9000,61.6000) -- (87.9000,61.6000) -- (87.9000,61.6000) -- (87.9000,61.6000) -- (87.9000,61.6000) -- (87.9000,61.6000) -- (87.9000,61.6000) -- (87.9000,61.6000) -- (87.9000,61.6000) -- (87.9000,61.6000) -- (87.9000,61.6000) -- (87.9000,61.6000) -- (87.9000,61.6000) -- (87.9000,61.6000) -- (87.9000,61.6000) -- (87.9000,61.5000) -- (87.9000,61.5000) -- (87.9000,61.5000) -- (87.9000,61.5000) -- (87.9000,61.5000) -- (87.9000,61.5000) -- (87.9000,61.5000) -- (87.9000,61.5000) -- (87.9000,61.5000) -- (87.9000,61.5000) -- (87.9000,61.5000) -- (87.9000,61.5000) -- (87.9000,61.5000) -- (87.9000,61.5000) -- (87.9000,61.5000) -- (87.9000,61.5000) -- (87.9000,61.5000) -- (87.9000,61.5000) -- (87.9000,61.5000) -- (87.9000,61.5000) -- (87.9000,61.5000) -- (87.9000,61.4000) -- (88.0000,61.4000) -- (88.0000,61.4000) -- (88.0000,61.4000) -- (88.0000,61.4000) -- (88.0000,61.4000) -- (88.0000,61.4000) -- (88.0000,61.4000) -- (88.0000,61.4000) -- (88.0000,61.4000) -- (88.0000,61.4000) -- (88.0000,61.4000) -- (88.0000,61.4000) -- (88.0000,61.4000) -- (88.0000,61.4000) -- (88.0000,61.4000) -- (88.0000,61.4000) -- (88.0000,61.4000) -- (88.0000,61.4000) -- (88.0000,61.4000) -- (88.0000,61.4000) -- (88.0000,61.3000) -- (88.0000,61.3000) -- (88.0000,61.3000) -- (88.0000,61.3000) -- (88.0000,61.3000) -- (88.0000,61.3000) -- (88.0000,61.3000) -- (88.0000,61.3000) -- (88.0000,61.3000) -- (88.0000,61.3000) -- (88.0000,61.3000) -- (88.0000,61.3000) -- (88.0000,61.3000) -- (88.0000,61.3000) -- (88.0000,61.3000) -- (88.0000,61.3000) -- (88.0000,61.3000) -- (88.0000,61.3000) -- (88.0000,61.3000) -- (88.0000,61.3000) -- (88.0000,61.3000) -- (88.0000,61.2000) -- (88.0000,61.2000) -- (88.0000,61.2000) -- (88.0000,61.2000) -- (88.0000,61.2000) -- (88.0000,61.2000) -- (88.0000,61.2000) -- (88.0000,61.2000) -- (88.0000,61.2000) -- (88.1000,61.2000) -- (88.1000,61.2000) -- (88.1000,61.2000) -- (88.1000,61.2000) -- (88.1000,61.2000) -- (88.1000,61.2000) -- (88.1000,61.2000) -- (88.1000,61.2000) -- (88.1000,61.2000) -- (88.1000,61.2000) -- (88.1000,61.2000) -- (88.1000,61.1000) -- (88.1000,61.1000) -- (88.1000,61.1000) -- (88.1000,61.1000) -- (88.1000,61.1000) -- (88.1000,61.1000) -- (88.1000,61.1000) -- (88.1000,61.1000) -- (88.1000,61.1000) -- (88.1000,61.1000) -- (88.1000,61.1000) -- (88.1000,61.1000) -- (88.1000,61.1000) -- (88.1000,61.1000) -- (88.1000,61.1000) -- (88.1000,61.1000) -- (88.1000,61.1000) -- (88.1000,61.1000) -- (88.1000,61.1000) -- (88.1000,61.1000) -- (88.1000,61.1000) -- (88.1000,61.0000) -- (88.1000,61.0000) -- (88.1000,61.0000) -- (88.1000,61.0000) -- (88.1000,61.0000) -- (88.1000,61.0000) -- (88.1000,61.0000) -- (88.1000,61.0000) -- (88.1000,61.0000) -- (88.1000,61.0000) -- (88.1000,61.0000) -- (88.1000,61.0000) -- (88.1000,61.0000) -- (88.1000,61.0000) -- (88.1000,61.0000) -- (88.1000,61.0000) -- (88.1000,61.0000) -- (88.2000,61.0000) -- (88.2000,61.0000) -- (88.2000,61.0000) -- (88.2000,61.0000) -- (88.2000,60.9000) -- (88.2000,60.9000) -- (88.2000,60.9000) -- (88.2000,60.9000) -- (88.2000,60.9000) -- (88.2000,60.9000) -- (88.2000,60.9000) -- (88.2000,60.9000) -- (88.2000,60.9000) -- (88.2000,60.9000) -- (88.2000,60.9000) -- (88.2000,60.9000) -- (88.2000,60.9000) -- (88.2000,60.9000) -- (88.2000,60.9000) -- (88.2000,60.9000) -- (88.2000,60.9000) -- (88.2000,60.9000) -- (88.2000,60.9000) -- (88.2000,60.9000) -- (88.2000,60.9000) -- (88.2000,60.8000) -- (88.2000,60.8000) -- (88.2000,60.8000) -- (88.2000,60.8000) -- (88.2000,60.8000) -- (88.2000,60.8000) -- (88.2000,60.8000) -- (88.2000,60.8000) -- (88.2000,60.8000) -- (88.2000,60.8000) -- (88.2000,60.8000) -- (88.2000,60.8000) -- (88.2000,60.8000) -- (88.2000,60.8000) -- (88.2000,60.8000) -- (88.2000,60.8000) -- (88.2000,60.8000) -- (88.2000,60.8000) -- (88.2000,60.8000) -- (88.2000,60.8000) -- (88.2000,60.8000) -- (88.2000,60.7000) -- (88.2000,60.7000) -- (88.2000,60.7000) -- (88.2000,60.7000) -- (88.3000,60.7000) -- (88.3000,60.7000) -- (88.3000,60.7000) -- (88.3000,60.7000) -- (88.3000,60.7000) -- (88.3000,60.7000) -- (88.3000,60.7000) -- (88.3000,60.7000) -- (88.3000,60.7000) -- (88.3000,60.7000) -- (88.3000,60.7000) -- (88.3000,60.7000) -- (88.3000,60.7000) -- (88.3000,60.7000) -- (88.3000,60.7000) -- (88.3000,60.7000) -- (88.3000,60.6000) -- (88.3000,60.6000) -- (88.3000,60.6000) -- (88.3000,60.6000) -- (88.3000,60.6000) -- (88.3000,60.6000) -- (88.3000,60.6000) -- (88.3000,60.6000) -- (88.3000,60.6000) -- (88.3000,60.6000) -- (88.3000,60.6000) -- (88.3000,60.6000) -- (88.3000,60.6000) -- (88.3000,60.6000) -- (88.3000,60.6000) -- (88.3000,60.6000) -- (88.3000,60.6000) -- (88.3000,60.6000) -- (88.3000,60.6000) -- (88.3000,60.6000) -- (88.3000,60.6000) -- (95.7000,60.5000) -- (95.7000,60.5000) -- (95.7000,60.5000) -- (95.7000,60.5000) -- (95.7000,60.5000) -- (95.7000,60.5000) -- (95.7000,60.5000) -- (95.7000,60.5000) -- (95.7000,60.5000) -- (95.7000,60.5000) -- (95.7000,60.5000) -- (95.7000,60.5000) -- (95.7000,60.5000) -- (95.7000,60.5000) -- (95.7000,60.5000) -- (95.7000,60.5000) -- (95.7000,60.5000) -- (95.7000,60.5000) -- (95.7000,60.5000) -- (95.7000,60.5000) -- (95.7000,60.5000) -- (95.7000,60.4000) -- (95.7000,60.4000) -- (95.7000,60.4000) -- (95.7000,60.4000) -- (95.7000,60.4000) -- (95.7000,60.4000) -- (95.7000,60.4000) -- (95.7000,60.4000) -- (95.7000,60.4000) -- (95.7000,60.4000) -- (95.7000,60.4000) -- (95.7000,60.4000) -- (95.7000,60.4000) -- (95.7000,60.4000) -- (95.7000,60.4000) -- (95.7000,60.4000) -- (95.7000,60.4000) -- (95.7000,60.4000) -- (95.7000,60.4000) -- (95.7000,60.4000) -- (95.7000,60.4000) -- (95.7000,60.3000) -- (95.7000,60.3000) -- (95.7000,60.3000) -- (95.7000,60.3000) -- (95.7000,60.3000) -- (95.7000,60.3000) -- (95.8000,60.3000) -- (95.8000,60.3000) -- (95.8000,60.3000) -- (95.8000,60.3000) -- (95.8000,60.3000) -- (95.8000,60.3000) -- (95.8000,60.3000) -- (95.8000,60.3000) -- (95.8000,60.3000) -- (95.8000,60.3000) -- (95.8000,60.3000) -- (95.8000,60.3000) -- (95.8000,60.3000) -- (95.8000,60.3000) -- (95.8000,60.2000) -- (95.8000,60.2000) -- (95.8000,60.2000) -- (95.8000,60.2000) -- (95.8000,60.2000) -- (95.8000,60.2000) -- (95.8000,60.2000) -- (95.8000,60.2000) -- (95.8000,60.2000) -- (95.8000,60.2000) -- (95.8000,60.2000) -- (95.8000,60.2000) -- (95.8000,60.2000) -- (95.8000,60.2000) -- (95.8000,60.2000) -- (95.8000,60.2000) -- (95.8000,60.2000) -- (95.8000,60.2000) -- (95.8000,60.2000) -- (95.8000,60.2000) -- (95.8000,60.2000) -- (95.8000,60.1000) -- (95.8000,60.1000) -- (95.8000,60.1000) -- (95.8000,60.1000) -- (95.8000,60.1000) -- (95.8000,60.1000) -- (95.8000,60.1000) -- (95.8000,60.1000) -- (95.8000,60.1000) -- (95.8000,60.1000) -- (95.8000,60.1000) -- (95.8000,60.1000) -- (95.8000,60.1000) -- (95.8000,60.1000) -- (95.8000,60.1000) -- (95.9000,60.1000) -- (95.9000,60.1000) -- (95.9000,60.1000) -- (95.9000,60.1000) -- (95.9000,60.1000) -- (95.9000,60.1000) -- (95.9000,60.0000) -- (95.9000,60.0000) -- (95.9000,60.0000) -- (95.9000,60.0000) -- (95.9000,60.0000) -- (95.9000,60.0000) -- (95.9000,60.0000) -- (95.9000,60.0000) -- (95.9000,60.0000) -- (95.9000,60.0000) -- (95.9000,60.0000) -- (95.9000,60.0000) -- (95.9000,60.0000) -- (95.9000,60.0000) -- (95.9000,60.0000) -- (95.9000,60.0000) -- (95.9000,60.0000) -- (95.9000,60.0000) -- (95.9000,60.0000) -- (95.9000,60.0000) -- (95.9000,60.0000) -- (95.9000,59.9000) -- (95.9000,59.9000) -- (95.9000,59.9000) -- (95.9000,59.9000) -- (95.9000,59.9000) -- (95.9000,59.9000) -- (95.9000,59.9000) -- (95.9000,59.9000) -- (95.9000,59.9000) -- (95.9000,59.9000) -- (95.9000,59.9000) -- (95.9000,59.9000) -- (95.9000,59.9000) -- (95.9000,59.9000) -- (95.9000,59.9000) -- (95.9000,59.9000) -- (95.9000,59.9000) -- (95.9000,59.9000) -- (95.9000,59.9000) -- (95.9000,59.9000) -- (95.9000,59.9000) -- (95.9000,59.8000) -- (96.0000,59.8000) -- (96.0000,59.8000) -- (96.0000,59.8000) -- (96.0000,59.8000) -- (96.0000,59.8000) -- (96.0000,59.8000) -- (96.0000,59.8000) -- (96.0000,59.8000) -- (96.0000,59.8000) -- (96.0000,59.8000) -- (96.0000,59.8000) -- (96.0000,59.8000) -- (96.0000,59.8000) -- (96.0000,59.8000) -- (96.0000,59.8000) -- (96.0000,59.8000) -- (96.0000,59.8000) -- (96.0000,59.8000) -- (96.0000,59.8000) -- (96.0000,59.7000) -- (96.0000,59.7000) -- (96.0000,59.7000) -- (96.0000,59.7000) -- (96.0000,59.7000) -- (96.0000,59.7000) -- (96.0000,59.7000) -- (96.0000,59.7000) -- (96.0000,59.7000) -- (96.0000,59.7000) -- (96.0000,59.7000) -- (96.0000,59.7000) -- (96.0000,59.7000) -- (96.0000,59.7000) -- (96.0000,59.7000) -- (96.0000,59.7000) -- (96.0000,59.7000) -- (96.0000,59.7000) -- (96.0000,59.7000) -- (96.0000,59.7000) -- (96.0000,59.7000) -- (96.0000,59.6000) -- (96.0000,59.6000) -- (96.0000,59.6000) -- (96.0000,59.6000) -- (96.0000,59.6000) -- (96.0000,59.6000) -- (96.0000,59.6000) -- (96.0000,59.6000) -- (96.0000,59.6000) -- (96.0000,59.6000) -- (96.1000,59.6000) -- (96.1000,59.6000) -- (96.1000,59.6000) -- (96.1000,59.6000) -- (96.1000,59.6000) -- (96.1000,59.6000) -- (96.1000,59.6000) -- (96.1000,59.6000) -- (96.1000,59.6000) -- (96.1000,59.6000) -- (96.1000,59.6000) -- (96.1000,59.5000) -- (96.1000,59.5000) -- (96.1000,59.5000) -- (96.1000,59.5000) -- (96.1000,59.5000) -- (96.1000,59.5000) -- (96.1000,59.5000) -- (96.1000,59.5000) -- (96.1000,59.5000) -- (96.1000,59.5000) -- (96.1000,59.5000) -- (96.1000,59.5000) -- (96.1000,59.5000) -- (96.1000,59.5000) -- (96.1000,59.5000) -- (96.1000,59.5000) -- (96.1000,59.5000) -- (96.1000,59.5000) -- (96.1000,59.5000) -- (96.1000,59.5000) -- (96.1000,59.5000) -- (96.1000,59.4000) -- (96.1000,59.4000) -- (96.1000,59.4000) -- (96.1000,59.4000) -- (96.1000,59.4000) -- (96.1000,59.4000) -- (96.1000,59.4000) -- (96.1000,59.4000) -- (96.1000,59.4000) -- (96.1000,59.4000) -- (96.1000,59.4000) -- (96.1000,59.4000) -- (96.1000,59.4000) -- (96.1000,59.4000) -- (96.1000,59.4000) -- (96.1000,59.4000) -- (96.1000,59.4000) -- (96.2000,59.4000) -- (96.2000,59.4000) -- (96.2000,59.4000) -- (96.2000,59.4000) -- (96.2000,59.3000) -- (96.2000,59.3000) -- (96.2000,59.3000) -- (96.2000,59.3000) -- (96.2000,59.3000) -- (96.2000,59.3000) -- (96.2000,59.3000) -- (96.2000,59.3000) -- (96.2000,59.3000) -- (96.2000,59.3000) -- (96.2000,59.3000) -- (96.2000,59.3000) -- (96.2000,59.3000) -- (96.2000,59.3000) -- (96.2000,59.3000) -- (96.2000,59.3000) -- (96.2000,59.3000) -- (96.2000,59.3000) -- (96.2000,59.3000) -- (96.2000,59.3000) -- (96.2000,59.2000) -- (96.2000,59.2000) -- (96.2000,59.2000) -- (96.2000,59.2000) -- (96.2000,59.2000) -- (96.2000,59.2000) -- (96.2000,59.2000) -- (96.2000,59.2000) -- (96.2000,59.2000) -- (96.2000,59.2000) -- (96.2000,59.2000) -- (96.2000,59.2000) -- (96.2000,59.2000) -- (96.2000,59.2000) -- (96.2000,59.2000) -- (96.2000,59.2000) -- (96.2000,59.2000) -- (96.2000,59.2000) -- (96.2000,59.2000) -- (96.2000,59.2000) -- (96.2000,59.2000) -- (96.2000,59.1000) -- (96.2000,59.1000) -- (96.2000,59.1000) -- (96.2000,59.1000) -- (96.2000,59.1000) -- (96.3000,59.1000) -- (96.3000,59.1000) -- (96.3000,59.1000) -- (96.3000,59.1000) -- (96.3000,59.1000) -- (96.3000,59.1000) -- (96.3000,59.1000) -- (96.3000,59.1000) -- (96.3000,59.1000) -- (96.3000,59.1000) -- (96.3000,59.1000) -- (96.3000,59.1000) -- (96.3000,59.1000) -- (96.3000,59.1000) -- (96.3000,59.1000) -- (96.3000,59.1000) -- (96.3000,59.0000) -- (96.3000,59.0000) -- (96.3000,59.0000) -- (96.3000,59.0000) -- (96.3000,59.0000) -- (96.3000,59.0000) -- (96.3000,59.0000) -- (96.3000,59.0000) -- (96.3000,59.0000) -- (96.3000,59.0000) -- (96.3000,59.0000) -- (96.3000,59.0000) -- (96.3000,59.0000) -- (96.3000,59.0000) -- (96.3000,59.0000) -- (96.3000,59.0000) -- (96.3000,59.0000) -- (96.3000,59.0000) -- (96.3000,59.0000) -- (96.3000,59.0000) -- (96.3000,59.0000) -- (96.3000,58.9000) -- (96.3000,58.9000) -- (96.3000,58.9000) -- (96.3000,58.9000) -- (96.3000,58.9000) -- (96.3000,58.9000) -- (96.3000,58.9000) -- (96.3000,58.9000) -- (96.3000,58.9000) -- (96.3000,58.9000) -- (96.3000,58.9000) -- (96.3000,58.9000) -- (96.3000,58.9000) -- (96.4000,58.9000) -- (96.4000,58.9000) -- (96.4000,58.9000) -- (96.4000,58.9000) -- (96.4000,58.9000) -- (96.4000,58.9000) -- (96.4000,58.9000) -- (96.4000,58.9000) -- (96.4000,58.8000) -- (96.4000,58.8000) -- (96.4000,58.8000) -- (96.4000,58.8000) -- (96.4000,58.8000) -- (96.4000,58.8000) -- (96.4000,58.8000) -- (96.4000,58.8000) -- (96.4000,58.8000) -- (96.4000,58.8000) -- (96.4000,58.8000) -- (96.4000,58.8000) -- (96.4000,58.8000) -- (96.4000,58.8000) -- (96.4000,58.8000) -- (96.4000,58.8000) -- (96.4000,58.8000) -- (96.4000,58.8000) -- (96.4000,58.8000) -- (96.4000,58.8000) -- (96.4000,58.7000) -- (96.4000,58.7000) -- (96.4000,58.7000) -- (96.4000,58.7000) -- (96.4000,58.7000) -- (96.4000,58.7000) -- (96.4000,58.7000) -- (96.4000,58.7000) -- (96.4000,58.7000) -- (96.4000,58.7000) -- (96.4000,58.7000) -- (96.4000,58.7000) -- (96.4000,58.7000) -- (96.4000,58.7000) -- (96.4000,58.7000) -- (96.4000,58.7000) -- (96.4000,58.7000) -- (96.4000,58.7000) -- (96.4000,58.7000) -- (96.4000,58.7000) -- (96.4000,58.7000) -- (96.5000,58.6000) -- (96.5000,58.6000) -- (96.5000,58.6000) -- (96.5000,58.6000) -- (96.5000,58.6000) -- (96.5000,58.6000) -- (96.5000,58.6000) -- (96.5000,58.6000) -- (96.5000,58.6000) -- (96.5000,58.6000) -- (96.5000,58.6000) -- (96.5000,58.6000) -- (96.5000,58.6000) -- (96.5000,58.6000) -- (96.5000,58.6000) -- (96.5000,58.6000) -- (96.5000,58.6000) -- (96.5000,58.6000) -- (96.5000,58.6000) -- (96.5000,58.6000) -- (96.5000,58.6000) -- (96.5000,58.5000) -- (96.5000,58.5000) -- (96.5000,58.5000) -- (96.5000,58.5000) -- (96.5000,58.5000) -- (96.5000,58.5000) -- (96.5000,58.5000) -- (96.5000,58.5000) -- (96.5000,58.5000) -- (96.5000,58.5000) -- (96.5000,58.5000) -- (96.5000,58.5000) -- (96.5000,58.5000) -- (96.5000,58.5000) -- (96.5000,58.5000) -- (96.5000,58.5000) -- (96.5000,58.5000) -- (96.5000,58.5000) -- (96.5000,58.5000) -- (96.5000,58.5000) -- (96.5000,58.5000) -- (96.5000,58.4000) -- (96.5000,58.4000) -- (96.5000,58.4000) -- (96.5000,58.4000) -- (96.5000,58.4000) -- (96.5000,58.4000) -- (96.5000,58.4000) -- (96.5000,58.4000) -- (96.6000,58.4000) -- (96.6000,58.4000) -- (96.6000,58.4000) -- (96.6000,58.4000) -- (96.6000,58.4000) -- (96.6000,58.4000) -- (96.6000,58.4000) -- (96.6000,58.4000) -- (96.6000,58.4000) -- (96.6000,58.4000) -- (96.6000,58.4000) -- (96.6000,58.4000) -- (96.6000,58.3000) -- (96.6000,58.3000) -- (96.6000,58.3000) -- (96.6000,58.3000) -- (96.6000,58.3000) -- (96.6000,58.3000) -- (96.6000,58.3000) -- (96.6000,58.3000) -- (96.6000,58.3000) -- (96.6000,58.3000) -- (96.6000,58.3000) -- (96.6000,58.3000) -- (96.6000,58.3000) -- (96.6000,58.3000) -- (96.6000,58.3000) -- (96.6000,58.3000) -- (96.6000,58.3000) -- (96.6000,58.3000) -- (96.6000,58.3000) -- (96.6000,58.3000) -- (96.6000,58.3000) -- (96.6000,58.2000) -- (96.6000,58.2000) -- (96.6000,58.2000) -- (96.6000,58.2000) -- (96.6000,58.2000) -- (96.6000,58.2000) -- (96.6000,58.2000) -- (96.6000,58.2000) -- (96.6000,58.2000) -- (96.6000,58.2000) -- (96.6000,58.2000) -- (96.6000,58.2000) -- (96.6000,58.2000) -- (96.6000,58.2000) -- (96.6000,58.2000) -- (96.6000,58.2000) -- (96.7000,58.2000) -- (96.7000,58.2000) -- (96.7000,58.2000) -- (96.7000,58.2000) -- (96.7000,58.2000) -- (96.7000,58.1000) -- (96.7000,58.1000) -- (96.7000,58.1000) -- (96.7000,58.1000) -- (96.7000,58.1000) -- (96.7000,58.1000) -- (96.7000,58.1000) -- (96.7000,58.1000) -- (96.7000,58.1000) -- (96.7000,58.1000) -- (96.7000,58.1000) -- (96.7000,58.1000) -- (96.7000,58.1000) -- (96.7000,58.1000) -- (96.7000,58.1000) -- (96.7000,58.1000) -- (96.7000,58.1000) -- (96.7000,58.1000) -- (96.7000,58.1000) -- (96.7000,58.1000) -- (96.7000,58.1000) -- (96.7000,58.0000) -- (96.7000,58.0000) -- (96.7000,58.0000) -- (96.7000,58.0000) -- (96.7000,58.0000) -- (96.7000,58.0000) -- (96.7000,58.0000) -- (96.7000,58.0000) -- (96.7000,58.0000) -- (96.7000,58.0000) -- (96.7000,58.0000) -- (96.7000,58.0000) -- (96.7000,58.0000) -- (96.7000,58.0000) -- (96.7000,58.0000) -- (96.7000,58.0000) -- (96.7000,58.0000) -- (96.7000,58.0000) -- (96.7000,58.0000) -- (96.7000,58.0000) -- (96.7000,58.0000) -- (96.7000,57.9000) -- (96.7000,57.9000) -- (96.7000,57.9000) -- (96.8000,57.9000) -- (96.8000,57.9000) -- (96.8000,57.9000) -- (96.8000,57.9000) -- (96.8000,57.9000) -- (96.8000,57.9000) -- (96.8000,57.9000) -- (96.8000,57.9000) -- (96.8000,57.9000) -- (96.8000,57.9000) -- (96.8000,57.9000) -- (96.8000,57.9000) -- (96.8000,57.9000) -- (96.8000,57.9000) -- (96.8000,57.9000) -- (96.8000,57.9000) -- (96.8000,57.9000) -- (96.8000,57.8000) -- (96.8000,57.8000) -- (96.8000,57.8000) -- (96.8000,57.8000) -- (96.8000,57.8000) -- (96.8000,57.8000) -- (96.8000,57.8000) -- (96.8000,57.8000) -- (96.8000,57.8000) -- (96.8000,57.8000) -- (96.8000,57.8000) -- (96.8000,57.8000) -- (96.8000,57.8000) -- (96.8000,57.8000) -- (96.8000,57.8000) -- (96.8000,57.8000) -- (96.8000,57.8000) -- (96.8000,57.8000) -- (96.8000,57.8000) -- (96.8000,57.8000) -- (96.8000,57.8000) -- (96.8000,57.7000) -- (96.8000,57.7000) -- (96.8000,57.7000) -- (96.8000,57.7000) -- (96.8000,57.7000) -- (96.8000,57.7000) -- (96.8000,57.7000) -- (96.8000,57.7000) -- (96.8000,57.7000) -- (96.8000,57.7000) -- (96.8000,57.7000) -- (96.9000,57.7000) -- (96.9000,57.7000) -- (96.9000,57.7000) -- (96.9000,57.7000) -- (96.9000,57.7000) -- (96.9000,57.7000) -- (96.9000,57.7000) -- (96.9000,57.7000) -- (96.9000,57.7000) -- (96.9000,57.7000) -- (96.9000,57.6000) -- (96.9000,57.6000) -- (96.9000,57.6000) -- (96.9000,57.6000) -- (96.9000,57.6000) -- (96.9000,57.6000) -- (96.9000,57.6000) -- (96.9000,57.6000) -- (96.9000,57.6000) -- (96.9000,57.6000) -- (96.9000,57.6000) -- (96.9000,57.6000) -- (96.9000,57.6000) -- (96.9000,57.6000) -- (96.9000,57.6000) -- (96.9000,57.6000) -- (96.9000,57.6000) -- (96.9000,57.6000) -- (96.9000,57.6000) -- (96.9000,57.6000) -- (96.9000,57.6000) -- (96.9000,57.5000) -- (96.9000,57.5000) -- (96.9000,57.5000) -- (96.9000,57.5000) -- (96.9000,57.5000) -- (96.9000,57.5000) -- (96.9000,57.5000) -- (96.9000,57.5000) -- (96.9000,57.5000) -- (96.9000,57.5000) -- (96.9000,57.5000) -- (96.9000,57.5000) -- (96.9000,57.5000) -- (96.9000,57.5000) -- (96.9000,57.5000) -- (96.9000,57.5000) -- (96.9000,57.5000) -- (96.9000,57.5000) -- (96.9000,57.5000) -- (97.0000,57.5000) -- (97.0000,57.5000) -- (97.0000,57.4000) -- (97.0000,57.4000) -- (97.0000,57.4000) -- (97.0000,57.4000) -- (97.0000,57.4000) -- (97.0000,57.4000) -- (97.0000,57.4000) -- (97.0000,57.4000) -- (97.0000,57.4000) -- (97.0000,57.4000) -- (97.0000,57.4000) -- (97.0000,57.4000) -- (97.0000,57.4000) -- (97.0000,57.4000) -- (97.0000,57.4000) -- (97.0000,57.4000) -- (97.0000,57.4000) -- (97.0000,57.4000) -- (97.0000,57.4000) -- (97.0000,57.4000) -- (97.0000,57.3000) -- (97.0000,57.3000) -- (97.0000,57.3000) -- (97.0000,57.3000) -- (97.0000,57.3000) -- (97.0000,57.3000) -- (97.0000,57.3000) -- (97.0000,57.3000) -- (97.0000,57.3000) -- (97.0000,57.3000) -- (97.0000,57.3000) -- (97.0000,57.3000) -- (97.0000,57.3000) -- (97.0000,57.3000) -- (97.0000,57.3000) -- (97.0000,57.3000) -- (97.0000,57.3000) -- (97.0000,57.3000) -- (97.0000,57.3000) -- (97.0000,57.3000) -- (97.0000,57.3000) -- (97.0000,57.2000) -- (97.0000,57.2000) -- (97.0000,57.2000) -- (97.0000,57.2000) -- (97.0000,57.2000) -- (97.0000,57.2000) -- (97.1000,57.2000) -- (97.1000,57.2000) -- (97.1000,57.2000) -- (97.1000,57.2000) -- (97.1000,57.2000) -- (97.1000,57.2000) -- (97.1000,57.2000) -- (97.1000,57.2000) -- (97.1000,57.2000) -- (97.1000,57.2000) -- (97.1000,57.2000) -- (97.1000,57.2000) -- (97.1000,57.2000) -- (97.1000,57.2000) -- (97.1000,57.2000) -- (97.1000,57.1000) -- (97.1000,57.1000) -- (97.1000,57.1000) -- (97.1000,57.1000) -- (97.1000,57.1000) -- (97.1000,57.1000) -- (97.1000,57.1000) -- (97.1000,57.1000) -- (97.1000,57.1000) -- (97.1000,57.1000) -- (97.1000,57.1000) -- (97.1000,57.1000) -- (97.1000,57.1000) -- (97.1000,57.1000) -- (97.1000,57.1000) -- (97.1000,57.1000) -- (97.1000,57.1000) -- (97.1000,57.1000) -- (97.1000,57.1000) -- (97.1000,57.1000) -- (97.1000,57.1000) -- (97.1000,57.0000) -- (97.1000,57.0000) -- (97.1000,57.0000) -- (97.1000,57.0000) -- (97.1000,57.0000) -- (97.1000,57.0000) -- (97.1000,57.0000) -- (97.1000,57.0000) -- (97.1000,57.0000) -- (97.1000,57.0000) -- (97.1000,57.0000) -- (97.1000,57.0000) -- (97.1000,57.0000) -- (97.1000,57.0000) -- (97.2000,57.0000) -- (97.2000,57.0000) -- (97.2000,57.0000) -- (97.2000,57.0000) -- (97.2000,57.0000) -- (97.2000,57.0000) -- (97.2000,56.9000) -- (97.2000,56.9000) -- (97.2000,56.9000) -- (97.2000,56.9000) -- (97.2000,56.9000) -- (97.2000,56.9000) -- (97.2000,56.9000) -- (97.2000,56.9000) -- (97.2000,56.9000) -- (97.2000,56.9000) -- (97.2000,56.9000) -- (97.2000,56.9000) -- (97.2000,56.9000) -- (97.2000,56.9000) -- (97.2000,56.9000) -- (97.2000,56.9000) -- (97.2000,56.9000) -- (97.2000,56.9000) -- (97.2000,56.9000) -- (97.2000,56.9000) -- (97.2000,56.9000) -- (97.2000,56.8000) -- (97.2000,56.8000) -- (97.2000,56.8000) -- (97.2000,56.8000) -- (97.2000,56.8000) -- (97.2000,56.8000) -- (97.2000,56.8000) -- (97.2000,56.8000) -- (97.2000,56.8000) -- (97.2000,56.8000) -- (97.2000,56.8000) -- (97.2000,56.8000) -- (97.2000,56.8000) -- (97.2000,56.8000) -- (97.2000,56.8000) -- (97.2000,56.8000) -- (97.2000,56.8000) -- (97.2000,56.8000) -- (97.2000,56.8000) -- (97.2000,56.8000) -- (97.2000,56.8000) -- (97.2000,56.7000) -- (97.3000,56.7000) -- (97.3000,56.7000) -- (97.3000,56.7000) -- (97.3000,56.7000) -- (97.3000,56.7000) -- (97.3000,56.7000) -- (97.3000,56.7000) -- (97.3000,56.7000) -- (97.3000,56.7000) -- (97.3000,56.7000) -- (97.3000,56.7000) -- (97.3000,56.7000) -- (97.3000,56.7000) -- (97.3000,56.7000) -- (97.3000,56.7000) -- (97.3000,56.7000) -- (97.3000,56.7000) -- (97.3000,56.7000) -- (97.3000,56.7000) -- (97.3000,56.7000) -- (97.3000,56.6000) -- (97.3000,56.6000) -- (97.3000,56.6000) -- (97.3000,56.6000) -- (97.3000,56.6000) -- (97.3000,56.6000) -- (97.3000,56.6000) -- (97.3000,56.6000) -- (97.3000,56.6000) -- (97.3000,56.6000) -- (97.3000,56.6000) -- (97.3000,56.6000) -- (97.3000,56.6000) -- (97.3000,56.6000) -- (97.3000,56.6000) -- (97.3000,56.6000) -- (97.3000,56.6000) -- (97.3000,56.6000) -- (97.3000,56.6000) -- (97.3000,56.6000) -- (97.3000,56.6000) -- (97.3000,56.5000) -- (97.3000,56.5000) -- (97.3000,56.5000) -- (97.3000,56.5000) -- (97.3000,56.5000) -- (97.3000,56.5000) -- (97.3000,56.5000) -- (97.3000,56.5000) -- (97.3000,56.5000) -- (97.4000,56.5000) -- (97.4000,56.5000) -- (97.4000,56.5000) -- (97.4000,56.5000) -- (97.4000,56.5000) -- (97.4000,56.5000) -- (97.4000,56.5000) -- (97.4000,56.5000) -- (97.4000,56.5000) -- (97.4000,56.5000) -- (97.4000,56.5000) -- (97.4000,56.4000) -- (97.4000,56.4000) -- (97.4000,56.4000) -- (97.4000,56.4000) -- (97.4000,56.4000) -- (97.4000,56.4000) -- (97.4000,56.4000) -- (97.4000,56.4000) -- (97.4000,56.4000) -- (97.4000,56.4000) -- (97.4000,56.4000) -- (97.4000,56.4000) -- (97.4000,56.4000) -- (97.4000,56.4000) -- (97.4000,56.4000) -- (97.4000,56.4000) -- (97.4000,56.4000) -- (97.4000,56.4000) -- (97.4000,56.4000) -- (97.4000,56.4000) -- (97.4000,56.4000) -- (97.4000,56.3000) -- (97.4000,56.3000) -- (97.4000,56.3000) -- (97.4000,56.3000) -- (97.4000,56.3000) -- (97.4000,56.3000) -- (97.4000,56.3000) -- (97.4000,56.3000) -- (97.4000,56.3000) -- (97.4000,56.3000) -- (97.4000,56.3000) -- (97.4000,56.3000) -- (97.4000,56.3000) -- (97.4000,56.3000) -- (97.4000,56.3000) -- (97.4000,56.3000) -- (97.4000,56.3000) -- (97.5000,56.3000) -- (97.5000,56.3000) -- (97.5000,56.3000) -- (97.5000,56.3000) -- (97.5000,56.2000) -- (97.5000,56.2000) -- (97.5000,56.2000) -- (97.5000,56.2000) -- (97.5000,56.2000) -- (97.5000,56.2000) -- (97.5000,56.2000) -- (97.5000,56.2000) -- (97.5000,56.2000) -- (97.5000,56.2000) -- (97.5000,56.2000) -- (97.5000,56.2000) -- (97.5000,56.2000) -- (97.5000,56.2000) -- (97.5000,56.2000) -- (97.5000,56.2000) -- (97.5000,56.2000) -- (97.5000,56.2000) -- (97.5000,56.2000) -- (97.5000,56.2000) -- (97.5000,56.2000) -- (97.5000,56.1000) -- (97.5000,56.1000) -- (97.5000,56.1000) -- (97.5000,56.1000) -- (97.5000,56.1000) -- (97.5000,56.1000) -- (97.5000,56.1000) -- (97.5000,56.1000) -- (97.5000,56.1000) -- (97.5000,56.1000) -- (97.5000,56.1000) -- (97.5000,56.1000) -- (97.5000,56.1000) -- (97.5000,56.1000) -- (97.5000,56.1000) -- (97.5000,56.1000) -- (97.5000,56.1000) -- (97.5000,56.1000) -- (97.5000,56.1000) -- (97.5000,56.1000) -- (97.5000,56.1000) -- (97.5000,56.0000) -- (97.5000,56.0000) -- (97.5000,56.0000) -- (97.5000,56.0000) -- (97.6000,56.0000) -- (97.6000,56.0000) -- (97.6000,56.0000) -- (97.6000,56.0000) -- (97.6000,56.0000) -- (97.6000,56.0000) -- (97.6000,56.0000) -- (97.6000,56.0000) -- (97.6000,56.0000) -- (97.6000,56.0000) -- (97.6000,56.0000) -- (97.6000,56.0000) -- (97.6000,56.0000) -- (97.6000,56.0000) -- (97.6000,56.0000) -- (97.6000,56.0000) -- (97.6000,55.9000) -- (97.6000,55.9000) -- (97.6000,55.9000) -- (97.6000,55.9000) -- (97.6000,55.9000) -- (97.6000,55.9000) -- (97.6000,55.9000) -- (97.6000,55.9000) -- (97.6000,55.9000) -- (97.6000,55.9000) -- (97.6000,55.9000) -- (97.6000,55.9000) -- (97.6000,55.9000) -- (97.6000,55.9000) -- (97.6000,55.9000) -- (97.6000,55.9000) -- (97.6000,55.9000) -- (97.6000,55.9000) -- (97.6000,55.9000) -- (97.6000,55.9000) -- (97.6000,55.9000) -- (97.6000,55.8000) -- (97.6000,55.8000) -- (97.6000,55.8000) -- (97.6000,55.8000) -- (97.6000,55.8000) -- (97.6000,55.8000) -- (97.6000,55.8000) -- (97.6000,55.8000) -- (97.6000,55.8000) -- (97.6000,55.8000) -- (97.6000,55.8000) -- (97.6000,55.8000) -- (97.7000,55.8000) -- (97.7000,55.8000) -- (97.7000,55.8000) -- (97.7000,55.8000) -- (97.7000,55.8000) -- (97.7000,55.8000) -- (97.7000,55.8000) -- (97.7000,55.8000) -- (97.7000,55.8000) -- (97.7000,55.7000) -- (97.7000,55.7000) -- (97.7000,55.7000) -- (97.7000,55.7000) -- (97.7000,55.7000) -- (97.7000,55.7000) -- (97.7000,55.7000) -- (121.4000,55.9000);



      \end{scope}
      \begin{scope}[cm={{1.05023,0.0,0.0,1.00798,(17.08433,-309.29135)}},draw=black,line join=round,line cap=round,line width=0.480pt]
        \path[draw] (81.5000,50.5000) -- (81.5000,78.5000) -- (121.5000,78.5000) -- (121.5000,50.5000) -- (81.5000,50.5000);



      \end{scope}
    \end{scope}
    \begin{scope}[cm={{0.7438,0.0,0.0,0.77563,(-75.95485,181.46752)}},draw=black,line join=round,line cap=round,line width=0.480pt]
      \path[draw] (130.5000,152.5000) -- (130.5000,147.5000);



      \path[draw] (130.5000,128.5000) -- (130.5000,132.5000);



    \end{scope}
    \begin{scope}[cm={{0.7438,0.0,0.0,0.77563,(-75.95485,181.46752)}},draw=black,line join=round,line cap=round,line width=0.480pt]
      \path[draw] (25.5000,128.5000) -- (25.5000,152.5000) -- (142.5000,152.5000) -- (142.5000,128.5000) -- (25.5000,128.5000);



    \end{scope}
    \begin{scope}[cm={{0.95389,0.0,0.0,0.95389,(1.66933,299.78054)}},draw=black,line join=bevel,line cap=rect,line width=0.800pt]
      \path[fill=black] (3.8674,-1.6114) node[above right] (text1264) {\scriptsize $\alpha_3$};



    \end{scope}
    \begin{scope}[cm={{0.7438,0.0,0.0,0.77563,(-73.80457,178.25993)}},draw=black,line join=round,line cap=round,line width=0.480pt]
      \path[draw,even odd rule] (123.5000,148.5000) -- (132.5000,148.5000);



    \end{scope}
    \begin{scope}[cm={{0.7438,0.0,0.0,0.77563,(-75.95485,181.46752)}},draw=black,line join=round,line cap=round,line width=0.480pt]
      \path[draw] (25.8000,136.6000) -- (25.8000,136.6000) -- (26.2000,137.3000) -- (26.7000,137.7000) -- (27.1000,136.6000) -- (27.5000,137.5000) -- (27.9000,137.0000) -- (28.3000,139.3000) -- (28.8000,132.8000) -- (29.2000,140.7000) -- (29.6000,139.7000) -- (30.0000,133.1000) -- (30.5000,135.4000) -- (30.9000,140.8000) -- (31.3000,140.7000) -- (31.7000,136.2000) -- (32.2000,133.3000) -- (32.6000,134.6000) -- (33.0000,138.2000) -- (33.4000,140.7000) -- (33.8000,140.6000) -- (34.3000,138.4000) -- (34.7000,135.8000) -- (35.1000,134.3000) -- (35.5000,134.4000) -- (36.0000,136.0000) -- (36.4000,138.0000) -- (36.8000,139.6000) -- (37.2000,140.3000) -- (37.6000,140.0000) -- (38.1000,138.8000) -- (38.5000,137.3000) -- (38.9000,135.9000) -- (39.3000,135.0000) -- (39.8000,134.7000) -- (40.2000,135.0000) -- (40.6000,135.8000) -- (41.0000,136.8000) -- (41.5000,137.9000) -- (41.9000,138.8000) -- (42.3000,139.4000) -- (42.7000,139.6000) -- (43.1000,139.4000) -- (43.6000,139.0000) -- (44.0000,138.3000) -- (44.4000,137.4000) -- (44.8000,136.6000) -- (45.3000,136.0000) -- (45.7000,135.5000) -- (46.1000,135.2000) -- (46.5000,135.2000) -- (47.0000,135.5000) -- (47.4000,135.9000) -- (47.8000,136.4000) -- (48.2000,137.0000) -- (48.6000,137.6000) -- (49.1000,138.1000) -- (49.5000,138.5000) -- (49.9000,138.8000) -- (50.3000,138.9000) -- (50.8000,138.8000) -- (51.2000,138.5000) -- (51.6000,138.1000) -- (52.0000,137.6000) -- (52.4000,137.1000) -- (52.9000,136.5000) -- (53.3000,136.1000) -- (53.7000,135.7000) -- (54.1000,135.6000) -- (54.6000,135.6000) -- (55.0000,135.8000) -- (55.4000,136.1000) -- (55.8000,136.6000) -- (56.3000,137.1000) -- (56.7000,137.6000) -- (57.1000,138.0000) -- (57.5000,138.3000) -- (57.9000,138.6000) -- (58.4000,138.7000) -- (58.8000,138.6000) -- (59.2000,138.4000) -- (59.6000,138.1000) -- (60.1000,137.8000) -- (60.5000,137.3000) -- (60.9000,136.9000) -- (61.3000,136.6000) -- (61.8000,136.4000) -- (62.2000,136.3000) -- (62.6000,136.4000) -- (63.0000,136.5000) -- (63.4000,136.7000) -- (63.9000,137.0000) -- (64.3000,137.4000) -- (64.7000,137.7000) -- (65.1000,138.0000) -- (65.6000,138.2000) -- (66.0000,138.3000) -- (66.4000,138.2000) -- (66.8000,138.1000) -- (67.3000,137.9000) -- (67.7000,137.7000) -- (68.1000,137.4000) -- (68.5000,137.1000) -- (68.9000,136.8000) -- (69.4000,136.6000) -- (69.8000,136.5000) -- (70.2000,136.5000) -- (70.6000,136.6000) -- (71.1000,136.7000) -- (71.5000,136.9000) -- (71.9000,137.0000) -- (72.3000,137.2000) -- (72.7000,137.4000) -- (73.2000,137.6000) -- (73.6000,137.8000) -- (74.0000,137.9000) -- (74.4000,138.0000) -- (74.9000,138.0000) -- (75.3000,137.9000) -- (75.7000,137.8000) -- (76.1000,137.7000) -- (76.6000,137.5000) -- (77.0000,137.3000) -- (77.4000,137.2000) -- (77.8000,137.1000) -- (78.2000,136.9000) -- (78.7000,136.8000) -- (79.1000,136.8000) -- (79.5000,136.8000) -- (79.9000,136.8000) -- (80.4000,136.8000) -- (80.8000,136.9000) -- (81.2000,137.0000) -- (81.6000,137.1000) -- (82.1000,137.3000) -- (82.5000,137.4000) -- (82.9000,137.5000) -- (83.3000,137.6000) -- (83.7000,137.6000) -- (84.2000,137.6000) -- (84.6000,137.6000) -- (85.0000,137.6000) -- (85.4000,137.5000) -- (85.9000,137.4000) -- (86.3000,137.3000) -- (86.7000,137.2000) -- (87.1000,137.1000) -- (87.6000,137.0000) -- (88.0000,136.9000) -- (88.4000,136.9000) -- (88.8000,137.0000) -- (89.2000,137.0000) -- (89.7000,137.1000) -- (90.1000,137.2000) -- (90.5000,137.3000) -- (90.9000,137.4000) -- (91.4000,137.5000) -- (91.8000,137.5000) -- (92.2000,137.6000) -- (92.6000,137.6000) -- (93.0000,137.6000) -- (93.5000,137.5000) -- (93.9000,137.5000) -- (94.3000,137.4000) -- (94.7000,137.3000) -- (95.2000,137.2000) -- (95.6000,137.1000) -- (96.0000,137.1000) -- (96.4000,137.0000) -- (96.9000,137.1000) -- (97.3000,137.1000) -- (97.7000,137.1000) -- (98.1000,137.2000) -- (98.5000,137.3000) -- (99.0000,137.3000) -- (99.4000,137.4000) -- (99.8000,137.4000) -- (100.2000,137.5000) -- (100.7000,137.5000) -- (101.1000,137.6000) -- (101.5000,137.5000) -- (101.9000,137.4000) -- (102.3000,137.3000) -- (102.8000,137.3000) -- (103.2000,137.2000) -- (103.6000,137.1000) -- (104.0000,137.1000) -- (104.5000,137.0000) -- (104.9000,137.0000) -- (105.3000,137.0000) -- (105.7000,137.1000) -- (106.2000,137.1000) -- (106.6000,137.2000) -- (107.0000,137.3000) -- (107.4000,137.3000) -- (107.8000,137.4000) -- (108.3000,137.5000) -- (108.7000,137.6000) -- (109.1000,137.6000) -- (109.5000,137.7000) -- (110.0000,137.6000) -- (110.4000,137.6000) -- (110.8000,137.5000) -- (111.2000,137.4000) -- (111.7000,137.3000) -- (112.1000,137.2000) -- (112.5000,137.1000) -- (112.9000,137.0000) -- (113.3000,137.0000) -- (113.8000,136.9000) -- (114.2000,136.9000) -- (114.6000,136.9000) -- (115.0000,136.9000) -- (115.5000,137.0000) -- (115.9000,137.2000) -- (116.3000,137.3000) -- (116.7000,137.4000) -- (117.2000,137.5000) -- (117.6000,137.6000) -- (118.0000,137.7000) -- (118.4000,137.7000) -- (118.8000,137.7000) -- (119.3000,137.7000) -- (119.7000,137.6000) -- (120.1000,137.5000) -- (120.5000,137.4000) -- (121.0000,137.2000) -- (121.4000,137.1000) -- (121.8000,137.0000) -- (122.2000,136.8000) -- (122.6000,136.8000) -- (123.1000,136.8000) -- (123.5000,136.8000) -- (123.9000,136.9000) -- (124.3000,137.0000) -- (124.8000,137.2000) -- (125.2000,137.3000) -- (125.6000,137.4000) -- (126.0000,137.6000) -- (126.5000,137.7000) -- (126.9000,137.7000) -- (127.3000,137.8000) -- (127.7000,137.8000) -- (128.1000,137.8000) -- (128.6000,137.7000) -- (129.0000,137.5000) -- (129.4000,137.4000) -- (129.8000,137.3000) -- (130.3000,137.1000) -- (130.7000,137.0000) -- (131.1000,136.9000) -- (131.5000,136.9000) -- (131.9000,136.8000) -- (132.4000,136.8000) -- (132.8000,136.9000) -- (133.2000,137.0000) -- (133.6000,137.1000) -- (134.1000,137.2000) -- (134.5000,137.3000) -- (134.9000,137.4000) -- (135.3000,137.6000) -- (135.8000,137.7000) -- (136.2000,137.8000) -- (136.6000,137.8000) -- (137.0000,137.8000) -- (137.4000,137.7000) -- (137.9000,137.6000) -- (138.3000,137.5000) -- (138.7000,137.4000) -- (139.1000,137.3000) -- (139.6000,137.1000) -- (140.0000,137.0000) -- (140.4000,136.9000) -- (140.8000,136.8000) -- (141.3000,136.8000) -- (141.7000,136.8000) -- (142.1000,136.9000) -- (142.3000,136.9000);



    \end{scope}
    \begin{scope}[cm={{0.7438,0.0,0.0,0.77563,(-75.95485,181.46752)}},draw=black,line join=round,line cap=round,line width=0.480pt]
      \path[draw] (25.5000,128.5000) -- (25.5000,152.5000) -- (142.5000,152.5000) -- (142.5000,128.5000) -- (25.5000,128.5000);



    \end{scope}
    \begin{scope}[cm={{0.7438,0.0,0.0,0.77563,(-75.95485,181.46752)}},draw=ca0a0a4,dash pattern=on 0.40pt off 0.80pt,line join=round,line cap=round,line width=0.400pt]
      \path[draw] (25.5000,168.5000) -- (108.5000,168.5000);



      \path[draw] (137.5000,168.5000) -- (142.5000,168.5000);



    \end{scope}
    \begin{scope}[cm={{0.7438,0.0,0.0,0.77563,(-75.95485,181.46752)}},draw=black,line join=round,line cap=round,line width=0.480pt]
      \path[draw] (25.5000,168.5000) -- (28.5000,168.5000);



      \path[draw] (142.5000,168.5000) -- (139.5000,168.5000);



    \end{scope}
    \begin{scope}[cm={{0.7438,0.0,0.0,0.77563,(-75.95485,181.46752)}},draw=ca0a0a4,dash pattern=on 0.40pt off 0.80pt,line join=round,line cap=round,line width=0.400pt]
      \path[draw] (25.5000,153.5000) -- (142.5000,153.5000);



    \end{scope}
    \begin{scope}[cm={{0.7438,0.0,0.0,0.77563,(-75.95485,181.46752)}},draw=black,line join=round,line cap=round,line width=0.480pt]
      \path[draw] (25.5000,153.5000) -- (28.5000,153.5000);



      \path[draw] (142.5000,153.5000) -- (139.5000,153.5000);



    \end{scope}
    \begin{scope}[cm={{0.95389,0.0,0.0,0.95389,(-70.56111,303.22344)}},draw=black,fill=ce10000,line join=bevel,line cap=rect,line width=0.800pt]
      \path[fill=ce10000] (0.0000,0.0000) node[above right] (text1350) {\scriptsize 20};



    \end{scope}
    \begin{scope}[cm={{0.7438,0.0,0.0,0.77563,(-75.95485,181.46752)}},draw=ca0a0a4,dash pattern=on 0.40pt off 0.80pt,line join=round,line cap=round,line width=0.400pt]
      \path[draw] (25.5000,176.5000) -- (25.5000,152.5000);



    \end{scope}
    \begin{scope}[cm={{0.7438,0.0,0.0,0.77563,(-75.95485,181.46752)}},draw=black,line join=round,line cap=round,line width=0.480pt]
      \path[draw] (25.5000,176.5000) -- (25.5000,171.5000);



      \path[draw] (25.5000,152.5000) -- (25.5000,156.5000);



    \end{scope}
    \begin{scope}[cm={{0.7438,0.0,0.0,0.77563,(-75.95485,181.46752)}},draw=ca0a0a4,dash pattern=on 0.40pt off 0.80pt,line join=round,line cap=round,line width=0.400pt]
      \path[draw] (60.5000,176.5000) -- (60.5000,152.5000);



    \end{scope}
    \begin{scope}[cm={{0.7438,0.0,0.0,0.77563,(-75.95485,181.46752)}},draw=black,line join=round,line cap=round,line width=0.480pt]
      \path[draw] (60.5000,176.5000) -- (60.5000,171.5000);



      \path[draw] (60.5000,152.5000) -- (60.5000,156.5000);



    \end{scope}
    \begin{scope}[cm={{0.7438,0.0,0.0,0.77563,(-75.95485,181.46752)}},draw=ca0a0a4,dash pattern=on 0.40pt off 0.80pt,line join=round,line cap=round,line width=0.400pt]
      \path[draw] (95.5000,176.5000) -- (95.5000,152.5000);



    \end{scope}
    \begin{scope}[cm={{0.7438,0.0,0.0,0.77563,(-75.95485,181.46752)}},draw=black,line join=round,line cap=round,line width=0.480pt]
      \path[draw] (95.5000,176.5000) -- (95.5000,171.5000);



      \path[draw] (95.5000,152.5000) -- (95.5000,156.5000);



    \end{scope}
    \begin{scope}[cm={{0.7438,0.0,0.0,0.77563,(-75.95485,181.46752)}},draw=ca0a0a4,dash pattern=on 0.40pt off 0.80pt,line join=round,line cap=round,line width=0.400pt]
      \path[draw] (130.5000,164.5000) -- (130.5000,152.5000);



    \end{scope}
    \begin{scope}[cm={{0.7438,0.0,0.0,0.77563,(-75.95485,181.46752)}},draw=black,line join=round,line cap=round,line width=0.480pt]
      \path[draw] (130.5000,176.5000) -- (130.5000,171.5000);



      \path[draw] (130.5000,152.5000) -- (130.5000,156.5000);



    \end{scope}
    \begin{scope}[cm={{0.7438,0.0,0.0,0.77563,(-75.95485,181.46752)}},draw=black,line join=round,line cap=round,line width=0.480pt]
      \path[draw] (25.5000,152.5000) -- (25.5000,176.5000) -- (142.5000,176.5000) -- (142.5000,152.5000) -- (25.5000,152.5000);



    \end{scope}
    \begin{scope}[cm={{0.95389,0.0,0.0,0.95389,(5.61786,317.03589)}},draw=black,line join=bevel,line cap=rect,line width=0.800pt]
      \path[fill=black] (0.0000,0.0000) node[above right] (text1494) {\scriptsize $\beta_3$};



    \end{scope}
    \begin{scope}[cm={{0.7438,0.0,0.0,0.77563,(-73.80457,178.25993)}},draw=black,line join=round,line cap=round,line width=0.480pt]
      \path[draw,even odd rule] (123.5000,172.5000) -- (132.5000,172.5000);



    \end{scope}
    \begin{scope}[cm={{0.7438,0.0,0.0,0.77563,(-75.95485,181.46752)}},draw=black,line join=round,line cap=round,line width=0.480pt]
      \path[draw] (25.8000,162.1000) -- (25.8000,162.1000) -- (26.2000,161.2000) -- (26.7000,161.1000) -- (27.1000,162.2000) -- (27.5000,159.4000) -- (27.9000,164.7000) -- (28.3000,157.2000) -- (28.8000,163.3000) -- (29.2000,164.4000) -- (29.6000,156.6000) -- (30.0000,159.9000) -- (30.5000,165.6000) -- (30.9000,163.5000) -- (31.3000,158.2000) -- (31.7000,157.3000) -- (32.2000,161.0000) -- (32.6000,164.7000) -- (33.0000,165.0000) -- (33.4000,162.4000) -- (33.8000,159.2000) -- (34.3000,157.7000) -- (34.7000,158.4000) -- (35.1000,160.6000) -- (35.5000,162.9000) -- (36.0000,164.3000) -- (36.4000,164.2000) -- (36.8000,163.0000) -- (37.2000,161.2000) -- (37.6000,159.5000) -- (38.1000,158.5000) -- (38.5000,158.3000) -- (38.9000,159.0000) -- (39.3000,160.2000) -- (39.8000,161.6000) -- (40.2000,162.8000) -- (40.6000,163.6000) -- (41.0000,163.9000) -- (41.5000,163.6000) -- (41.9000,163.0000) -- (42.3000,162.1000) -- (42.7000,161.1000) -- (43.1000,160.2000) -- (43.6000,159.5000) -- (44.0000,159.1000) -- (44.4000,159.0000) -- (44.8000,159.2000) -- (45.3000,159.7000) -- (45.7000,160.4000) -- (46.1000,161.1000) -- (46.5000,161.8000) -- (47.0000,162.5000) -- (47.4000,163.0000) -- (47.8000,163.2000) -- (48.2000,163.3000) -- (48.6000,163.1000) -- (49.1000,162.8000) -- (49.5000,162.4000) -- (49.9000,161.8000) -- (50.3000,161.2000) -- (50.8000,160.7000) -- (51.2000,160.2000) -- (51.6000,159.9000) -- (52.0000,159.7000) -- (52.4000,159.7000) -- (52.9000,159.9000) -- (53.3000,160.3000) -- (53.7000,160.8000) -- (54.1000,161.3000) -- (54.6000,161.9000) -- (55.0000,162.4000) -- (55.4000,162.7000) -- (55.8000,163.0000) -- (56.3000,163.0000) -- (56.7000,162.8000) -- (57.1000,162.6000) -- (57.5000,162.2000) -- (57.9000,161.7000) -- (58.4000,161.2000) -- (58.8000,160.8000) -- (59.2000,160.4000) -- (59.6000,160.1000) -- (60.1000,160.0000) -- (60.5000,160.0000) -- (60.9000,160.1000) -- (61.3000,160.4000) -- (61.8000,160.7000) -- (62.2000,161.1000) -- (62.6000,161.5000) -- (63.0000,161.8000) -- (63.4000,162.1000) -- (63.9000,162.2000) -- (64.3000,162.2000) -- (64.7000,162.1000) -- (65.1000,161.9000) -- (65.6000,161.6000) -- (66.0000,161.3000) -- (66.4000,161.0000) -- (66.8000,160.7000) -- (67.3000,160.5000) -- (67.7000,160.4000) -- (68.1000,160.3000) -- (68.5000,160.4000) -- (68.9000,160.6000) -- (69.4000,160.8000) -- (69.8000,161.1000) -- (70.2000,161.4000) -- (70.6000,161.6000) -- (71.1000,161.8000) -- (71.5000,162.0000) -- (71.9000,162.0000) -- (72.3000,162.0000) -- (72.7000,162.0000) -- (73.2000,161.9000) -- (73.6000,161.8000) -- (74.0000,161.6000) -- (74.4000,161.5000) -- (74.9000,161.2000) -- (75.3000,161.0000) -- (75.7000,160.8000) -- (76.1000,160.7000) -- (76.6000,160.7000) -- (77.0000,160.7000) -- (77.4000,160.7000) -- (77.8000,160.8000) -- (78.2000,160.9000) -- (78.7000,161.0000) -- (79.1000,161.2000) -- (79.5000,161.3000) -- (79.9000,161.5000) -- (80.4000,161.6000) -- (80.8000,161.7000) -- (81.2000,161.8000) -- (81.6000,161.8000) -- (82.1000,161.8000) -- (82.5000,161.8000) -- (82.9000,161.7000) -- (83.3000,161.6000) -- (83.7000,161.5000) -- (84.2000,161.3000) -- (84.6000,161.2000) -- (85.0000,161.1000) -- (85.4000,161.0000) -- (85.9000,161.0000) -- (86.3000,161.0000) -- (86.7000,161.0000) -- (87.1000,161.0000) -- (87.6000,161.1000) -- (88.0000,161.1000) -- (88.4000,161.3000) -- (88.8000,161.4000) -- (89.2000,161.5000) -- (89.7000,161.6000) -- (90.1000,161.6000) -- (90.5000,161.6000) -- (90.9000,161.6000) -- (91.4000,161.6000) -- (91.8000,161.5000) -- (92.2000,161.4000) -- (92.6000,161.3000) -- (93.0000,161.2000) -- (93.5000,161.1000) -- (93.9000,161.1000) -- (94.3000,161.0000) -- (94.7000,161.0000) -- (95.2000,161.0000) -- (95.6000,161.0000) -- (96.0000,161.1000) -- (96.4000,161.2000) -- (96.9000,161.3000) -- (97.3000,161.4000) -- (97.7000,161.5000) -- (98.1000,161.5000) -- (98.5000,161.5000) -- (99.0000,161.5000) -- (99.4000,161.5000) -- (99.8000,161.4000) -- (100.2000,161.4000) -- (100.7000,161.3000) -- (101.1000,161.3000) -- (101.5000,161.2000) -- (101.9000,161.1000) -- (102.3000,161.1000) -- (102.8000,161.0000) -- (103.2000,161.1000) -- (103.6000,161.1000) -- (104.0000,161.2000) -- (104.5000,161.2000) -- (104.9000,161.3000) -- (105.3000,161.4000) -- (105.7000,161.4000) -- (106.2000,161.5000) -- (106.6000,161.5000) -- (107.0000,161.6000) -- (107.4000,161.6000) -- (107.8000,161.5000) -- (108.3000,161.5000) -- (108.7000,161.4000) -- (109.1000,161.4000) -- (109.5000,161.3000) -- (110.0000,161.2000) -- (110.4000,161.1000) -- (110.8000,161.0000) -- (111.2000,161.0000) -- (111.7000,160.9000) -- (112.1000,160.9000) -- (112.5000,161.0000) -- (112.9000,161.0000) -- (113.3000,161.1000) -- (113.8000,161.2000) -- (114.2000,161.3000) -- (114.6000,161.4000) -- (115.0000,161.5000) -- (115.5000,161.6000) -- (115.9000,161.7000) -- (116.3000,161.7000) -- (116.7000,161.7000) -- (117.2000,161.6000) -- (117.6000,161.6000) -- (118.0000,161.5000) -- (118.4000,161.3000) -- (118.8000,161.2000) -- (119.3000,161.1000) -- (119.7000,161.0000) -- (120.1000,160.9000) -- (120.5000,160.8000) -- (121.0000,160.8000) -- (121.4000,160.9000) -- (121.8000,160.9000) -- (122.2000,161.0000) -- (122.6000,161.2000) -- (123.1000,161.3000) -- (123.5000,161.5000) -- (123.9000,161.6000) -- (124.3000,161.7000) -- (124.8000,161.8000) -- (125.2000,161.8000) -- (125.6000,161.8000) -- (126.0000,161.7000) -- (126.5000,161.6000) -- (126.9000,161.5000) -- (127.3000,161.4000) -- (127.7000,161.3000) -- (128.1000,161.1000) -- (128.6000,160.9000) -- (129.0000,160.8000) -- (129.4000,160.8000) -- (129.8000,160.8000) -- (130.3000,160.8000) -- (130.7000,160.9000) -- (131.1000,161.0000) -- (131.5000,161.1000) -- (131.9000,161.3000) -- (132.4000,161.4000) -- (132.8000,161.5000) -- (133.2000,161.6000) -- (133.6000,161.7000) -- (134.1000,161.7000) -- (134.5000,161.8000) -- (134.9000,161.7000) -- (135.3000,161.7000) -- (135.8000,161.6000) -- (136.2000,161.5000) -- (136.6000,161.3000) -- (137.0000,161.2000) -- (137.4000,161.0000) -- (137.9000,160.9000) -- (138.3000,160.9000) -- (138.7000,160.8000) -- (139.1000,160.8000) -- (139.6000,160.9000) -- (140.0000,160.9000) -- (140.4000,161.0000) -- (140.8000,161.1000) -- (141.3000,161.3000) -- (141.7000,161.4000) -- (142.1000,161.6000) -- (142.3000,161.6000);



    \end{scope}
    \begin{scope}[cm={{1.0018,0.0,0.0,1.0018,(-58.81429,16.85298)}},draw=black,line join=bevel,line cap=rect,line width=0.800pt]
      \path[fill=black] (0.0000,0.0000) node[above right] (text456-4) {\scriptsize 0};



    \end{scope}
    \begin{scope}[cm={{1.0018,0.0,0.0,1.0018,(-30.76382,16.96518)}},draw=black,line join=bevel,line cap=rect,line width=0.800pt]
      \path[fill=black] (0.0000,0.0000) node[above right] (text486-5) {\scriptsize 2};



    \end{scope}
    \begin{scope}[cm={{1.0018,0.0,0.0,1.0018,(-1.21069,16.96518)}},draw=black,line join=bevel,line cap=rect,line width=0.800pt]
      \path[fill=black] (0.0000,0.0000) node[above right] (text516-4) {\scriptsize 4};



    \end{scope}
    \begin{scope}[cm={{1.0018,0.0,0.0,1.0018,(26.33903,16.85298)}},draw=black,line join=bevel,line cap=rect,line width=0.800pt]
      \path[fill=black] (0.0000,0.0000) node[above right] (text546-9) {\scriptsize 6};



    \end{scope}
    \begin{scope}[cm={{1.0018,0.0,0.0,1.0018,(-83.99314,4.97821)}},draw=black,line join=bevel,line cap=rect,line width=0.800pt]
      \path[fill=black] (0.0000,0.0000) node[above right] (text602-2) {\scriptsize Time (sec)};



    \end{scope}
    \path[draw=black,line join=miter,line cap=butt,line width=0.596pt] (-56.8336,102.4467) -- cycle;



    \path[draw=cd9d9d9,line join=miter,line cap=butt,miter limit=4.00,dash phase=3.579pt,line width=0.596pt] (-55.0867,184.0619) -- (30.3487,95.6113);



    \begin{scope}[cm={{1.0018,0.0,0.0,0.97485,(-208.78685,-65.87769)}},draw=ca0a0a4,dash pattern=on 0.40pt off 0.80pt,line join=round,line cap=round,line width=0.400pt]
      \path[draw] (41.5000,153.5000) -- (127.5000,153.5000);



    \end{scope}
    \begin{scope}[cm={{1.0018,0.0,0.0,0.97485,(-208.78685,-65.87769)}},draw=cffffff,line join=miter,line cap=rect,miter limit=10.00,line width=2.113pt]
      \path[draw=cffffff,line join=miter,line cap=rect,miter limit=10.00,line width=2.113pt] (41.5000,88.5000) -- (41.5000,164.5000) -- (127.5000,164.5000) -- (127.5000,88.5000) -- (41.5000,88.5000);



    \end{scope}
    \begin{scope}[cm={{1.0018,0.0,0.0,0.97485,(-208.78685,-65.87769)}},draw=black,line join=round,line cap=round,line width=0.480pt]
      \path[draw] (41.5000,153.5000) -- (44.5000,153.5000);



      \path[draw] (127.5000,153.5000) -- (124.5000,153.5000);



    \end{scope}
    \begin{scope}[cm={{1.0018,0.0,0.0,1.0018,(-180.11018,85.70628)}},draw=black,line join=bevel,line cap=rect,line width=0.800pt]
      \path[fill=black] (0.0000,0.0000) node[above right] (text366) {\scriptsize 30};



    \end{scope}
    \begin{scope}[cm={{1.0018,0.0,0.0,0.97485,(-208.78685,-65.87769)}},draw=ca0a0a4,dash pattern=on 0.40pt off 0.80pt,line join=round,line cap=round,line width=0.400pt]
      \path[draw] (41.5000,127.5000) -- (127.5000,127.5000);



    \end{scope}
    \begin{scope}[cm={{1.0018,0.0,0.0,0.97485,(-208.78685,-65.87769)}},draw=black,line join=round,line cap=round,line width=0.480pt]
      \path[draw] (41.5000,127.5000) -- (44.5000,127.5000);



      \path[draw] (127.5000,127.5000) -- (124.5000,127.5000);



    \end{scope}
    \begin{scope}[cm={{1.0018,0.0,0.0,1.0018,(-179.75755,60.92479)}},draw=black,line join=bevel,line cap=rect,line width=0.800pt]
      \path[fill=black] (0.0000,0.0000) node[above right] (text396) {\scriptsize 33};



    \end{scope}
    \begin{scope}[cm={{1.0018,0.0,0.0,0.97485,(-208.78685,-65.87769)}},draw=black,line join=round,line cap=round,line width=0.480pt]
      \path[draw] (41.5000,101.5000) -- (44.5000,101.5000);



      \path[draw] (127.5000,101.5000) -- (124.5000,101.5000);



    \end{scope}
    \begin{scope}[cm={{1.0018,0.0,0.0,1.0018,(-180.04607,36.36917)}},draw=black,line join=bevel,line cap=rect,line width=0.800pt]
      \path[fill=black] (0.0000,0.0000) node[above right] (text426) {\scriptsize 36};



    \end{scope}
    \begin{scope}[cm={{1.0018,0.0,0.0,0.97485,(-208.78685,-65.87769)}},draw=ca0a0a4,dash pattern=on 0.40pt off 0.80pt,line join=round,line cap=round,line width=0.400pt]
      \path[draw] (41.5000,164.5000) -- (41.5000,88.5000);



    \end{scope}
    \begin{scope}[cm={{1.0018,0.0,0.0,0.97485,(-208.78685,-65.87769)}},draw=black,line join=round,line cap=round,line width=0.480pt]
      \path[draw] (41.5000,164.5000) -- (41.5000,159.5000);



      \path[draw] (41.5000,88.5000) -- (41.5000,92.5000);



    \end{scope}
    \begin{scope}[cm={{1.0018,0.0,0.0,1.0018,(-169.71659,16.85298)}},draw=black,line join=bevel,line cap=rect,line width=0.800pt]
      \path[fill=black] (0.0000,0.0000) node[above right] (text456) {\scriptsize 0};



    \end{scope}
    \begin{scope}[cm={{1.0018,0.0,0.0,0.97485,(-208.78685,-65.87769)}},draw=black,line join=round,line cap=round,line width=0.480pt]
      \path[draw] (70.5000,164.5000) -- (70.5000,159.5000);



      \path[draw] (70.5000,88.5000) -- (70.5000,92.5000);



    \end{scope}
    \begin{scope}[cm={{1.0018,0.0,0.0,1.0018,(-141.66611,16.96518)}},draw=black,line join=bevel,line cap=rect,line width=0.800pt]
      \path[fill=black] (0.0000,0.0000) node[above right] (text486) {\scriptsize 2};



    \end{scope}
    \begin{scope}[cm={{1.0018,0.0,0.0,0.97485,(-208.78685,-65.87769)}},draw=black,line join=round,line cap=round,line width=0.480pt]
      \path[draw] (98.5000,164.5000) -- (98.5000,159.5000);



      \path[draw] (98.5000,88.5000) -- (98.5000,92.5000);



    \end{scope}
    \begin{scope}[cm={{1.0018,0.0,0.0,1.0018,(-112.11299,16.96518)}},draw=black,line join=bevel,line cap=rect,line width=0.800pt]
      \path[fill=black] (0.0000,0.0000) node[above right] (text516) {\scriptsize 4};



    \end{scope}
    \begin{scope}[cm={{1.0018,0.0,0.0,0.97485,(-208.78685,-65.87769)}},draw=ca0a0a4,dash pattern=on 0.40pt off 0.80pt,line join=round,line cap=round,line width=0.400pt]
      \path[draw] (127.5000,164.5000) -- (127.5000,88.5000);



    \end{scope}
    \begin{scope}[cm={{1.0018,0.0,0.0,0.97485,(-208.78685,-65.87769)}},draw=black,line join=round,line cap=round,line width=0.480pt]
      \path[draw] (127.5000,164.5000) -- (127.5000,159.5000);



      \path[draw] (127.5000,88.5000) -- (127.5000,92.5000);



    \end{scope}
    \begin{scope}[cm={{1.0018,0.0,0.0,1.0018,(-84.56327,16.85298)}},draw=black,line join=bevel,line cap=rect,line width=0.800pt]
      \path[fill=black] (0.0000,0.0000) node[above right] (text546) {\scriptsize 6};



    \end{scope}
    \begin{scope}[cm={{1.0018,0.0,0.0,0.97485,(-208.78685,-65.87769)}},draw=black,line join=round,line cap=round,line width=0.480pt]
      \path[draw] (41.6000,103.3000) -- (41.6000,103.3000) -- (41.7000,103.6000) -- (41.9000,104.0000) -- (42.0000,104.3000) -- (42.2000,104.6000) -- (42.3000,104.9000) -- (42.5000,105.1000) -- (42.6000,105.4000) -- (42.7000,105.7000) -- (42.9000,106.0000) -- (43.0000,106.3000) -- (43.2000,106.5000) -- (43.3000,106.8000) -- (43.5000,107.1000) -- (43.6000,107.3000) -- (43.7000,107.6000) -- (43.9000,107.9000) -- (44.0000,108.1000) -- (44.2000,108.4000) -- (44.3000,108.6000) -- (44.5000,108.8000) -- (44.6000,109.1000) -- (44.7000,109.3000) -- (44.9000,109.5000) -- (45.0000,109.8000) -- (45.2000,110.0000) -- (45.3000,110.2000) -- (45.5000,110.4000) -- (45.6000,110.7000) -- (45.7000,110.9000) -- (45.9000,111.1000) -- (46.0000,111.3000) -- (46.2000,111.5000) -- (46.3000,111.7000) -- (46.5000,111.9000) -- (46.6000,112.1000) -- (46.7000,112.3000) -- (46.9000,112.4000) -- (47.0000,112.6000) -- (47.2000,112.8000) -- (47.3000,113.0000) -- (47.5000,113.2000) -- (47.6000,113.3000) -- (47.7000,113.5000) -- (47.9000,113.7000) -- (48.0000,113.9000) -- (48.2000,114.0000) -- (48.3000,114.2000) -- (48.5000,114.3000) -- (48.6000,114.5000) -- (48.7000,114.7000) -- (48.9000,114.8000) -- (49.0000,115.0000) -- (49.2000,115.1000) -- (49.3000,115.2000) -- (49.5000,115.4000) -- (49.6000,115.5000) -- (49.7000,115.7000) -- (49.9000,115.8000) -- (50.0000,115.9000) -- (50.2000,116.1000) -- (50.3000,116.2000) -- (50.5000,116.3000) -- (50.6000,116.4000) -- (50.7000,116.6000) -- (50.9000,116.7000) -- (51.0000,116.8000) -- (51.2000,116.9000) -- (51.3000,117.0000) -- (51.5000,117.1000) -- (51.6000,117.2000) -- (51.7000,117.4000) -- (51.9000,117.5000) -- (52.0000,117.6000) -- (52.2000,117.7000) -- (52.3000,117.8000) -- (52.5000,117.9000) -- (52.6000,118.0000) -- (52.7000,118.0000) -- (52.9000,118.1000) -- (53.0000,118.2000) -- (53.2000,118.3000) -- (53.3000,118.4000) -- (53.5000,118.5000) -- (53.6000,118.6000) -- (53.7000,118.7000) -- (53.9000,118.7000) -- (54.0000,118.8000) -- (54.2000,118.9000) -- (54.3000,119.0000) -- (54.5000,119.0000) -- (54.6000,119.1000) -- (54.7000,119.2000) -- (54.9000,119.2000) -- (55.0000,119.3000) -- (55.2000,119.4000) -- (55.3000,119.4000) -- (55.5000,119.5000) -- (55.6000,119.6000) -- (55.7000,119.6000) -- (55.9000,119.7000) -- (56.0000,119.7000) -- (56.2000,119.8000) -- (56.3000,119.8000) -- (56.5000,119.9000) -- (56.6000,119.9000) -- (56.7000,120.0000) -- (56.9000,120.0000) -- (57.0000,120.1000) -- (57.2000,120.1000) -- (57.3000,120.2000) -- (57.5000,120.2000) -- (57.6000,120.2000) -- (57.7000,120.3000) -- (57.9000,120.3000) -- (58.0000,120.4000) -- (58.2000,120.4000) -- (58.3000,120.4000) -- (58.5000,120.5000) -- (58.6000,120.5000) -- (58.7000,120.5000) -- (58.9000,120.6000) -- (59.0000,120.6000) -- (59.2000,120.6000) -- (59.3000,120.6000) -- (59.5000,120.7000) -- (59.6000,120.7000) -- (59.7000,120.7000) -- (59.9000,120.7000) -- (60.0000,120.8000) -- (60.2000,120.8000) -- (60.3000,120.8000) -- (60.5000,120.8000) -- (60.6000,120.8000) -- (60.7000,120.9000) -- (60.9000,120.9000) -- (61.0000,120.9000) -- (61.2000,120.9000) -- (61.3000,120.9000) -- (61.5000,120.9000) -- (61.6000,120.9000) -- (61.7000,121.0000) -- (61.9000,121.0000) -- (62.0000,121.0000) -- (62.2000,121.0000) -- (62.3000,121.0000) -- (62.5000,121.0000) -- (62.6000,121.0000) -- (62.7000,121.0000) -- (62.9000,121.0000) -- (63.0000,121.0000) -- (63.2000,121.0000) -- (63.3000,121.0000) -- (63.5000,121.0000) -- (63.6000,121.0000) -- (63.7000,121.0000) -- (63.9000,121.0000) -- (64.0000,121.0000) -- (64.2000,121.0000) -- (64.3000,121.0000) -- (64.5000,121.0000) -- (64.6000,121.0000) -- (64.7000,121.0000) -- (64.9000,121.0000) -- (65.0000,121.0000) -- (65.2000,121.0000) -- (65.3000,121.0000) -- (65.5000,121.0000) -- (65.6000,121.0000) -- (65.7000,121.0000) -- (65.9000,121.0000) -- (66.0000,121.0000) -- (66.2000,121.0000) -- (66.3000,121.0000) -- (66.5000,120.9000) -- (66.6000,120.9000) -- (66.7000,120.9000) -- (66.9000,120.9000) -- (67.0000,120.9000) -- (67.2000,120.9000) -- (67.3000,120.9000) -- (67.5000,120.9000) -- (67.6000,120.9000) -- (67.7000,120.9000) -- (67.9000,120.9000) -- (68.0000,120.9000) -- (68.2000,120.8000) -- (68.3000,120.8000) -- (68.5000,120.8000) -- (68.6000,120.8000) -- (68.7000,120.8000) -- (68.9000,120.8000) -- (69.0000,120.8000) -- (69.2000,120.8000) -- (69.3000,120.8000) -- (69.5000,120.8000) -- (69.6000,120.8000) -- (69.7000,120.8000) -- (69.9000,120.8000) -- (70.0000,120.8000) -- (70.2000,120.8000) -- (70.3000,120.8000) -- (70.5000,120.8000) -- (70.6000,120.8000) -- (70.7000,120.8000) -- (70.9000,120.8000) -- (71.0000,120.8000) -- (71.2000,120.8000) -- (71.3000,120.8000) -- (71.5000,120.9000) -- (71.6000,120.9000) -- (71.7000,120.9000) -- (71.9000,120.9000) -- (72.0000,120.9000) -- (72.2000,120.9000) -- (72.3000,121.0000) -- (72.5000,121.0000) -- (72.6000,121.0000) -- (72.7000,121.0000) -- (72.9000,121.1000) -- (73.0000,121.1000) -- (73.2000,121.1000) -- (73.3000,121.2000) -- (73.5000,121.2000) -- (73.6000,121.2000) -- (73.7000,121.2000) -- (73.9000,121.3000) -- (74.0000,121.3000) -- (74.2000,121.3000) -- (74.3000,121.3000) -- (74.5000,121.4000) -- (74.6000,121.4000) -- (74.7000,121.4000) -- (74.9000,121.5000) -- (75.0000,121.5000) -- (75.2000,121.5000) -- (75.3000,121.5000) -- (75.5000,121.6000) -- (75.6000,121.6000) -- (75.7000,121.6000) -- (75.9000,121.6000) -- (76.0000,121.7000) -- (76.2000,121.7000) -- (76.3000,121.7000) -- (76.5000,121.7000) -- (76.6000,121.7000) -- (76.7000,121.7000) -- (76.9000,121.8000) -- (77.0000,121.8000) -- (77.2000,121.8000) -- (77.3000,121.8000) -- (77.5000,121.8000) -- (77.6000,121.8000) -- (77.7000,121.9000) -- (77.9000,121.9000) -- (78.0000,121.9000) -- (78.2000,121.9000) -- (78.3000,121.9000) -- (78.5000,121.9000) -- (78.6000,121.9000) -- (78.7000,121.9000) -- (78.9000,121.9000) -- (79.0000,121.9000) -- (79.2000,121.9000) -- (79.3000,121.9000) -- (79.5000,121.9000) -- (79.6000,121.9000) -- (79.7000,122.0000) -- (79.9000,122.0000) -- (80.0000,122.0000) -- (80.2000,122.0000) -- (80.3000,122.0000) -- (80.5000,121.9000) -- (80.6000,121.9000) -- (80.7000,121.9000) -- (80.9000,121.9000) -- (81.0000,121.9000) -- (81.2000,121.9000) -- (81.3000,121.9000) -- (81.5000,121.9000) -- (81.6000,121.9000) -- (81.7000,121.9000) -- (81.9000,121.9000) -- (82.0000,121.9000) -- (82.2000,121.9000) -- (82.3000,121.9000) -- (82.5000,121.9000) -- (82.6000,121.9000) -- (82.7000,121.8000) -- (82.9000,121.8000) -- (83.0000,121.8000) -- (83.2000,121.8000) -- (83.3000,121.8000) -- (83.5000,121.8000) -- (83.6000,121.8000) -- (83.7000,121.8000) -- (83.9000,121.7000) -- (84.0000,121.7000) -- (84.2000,121.7000) -- (84.3000,121.7000) -- (84.5000,121.7000) -- (84.6000,121.7000) -- (84.7000,121.6000) -- (84.9000,121.6000) -- (85.0000,121.6000) -- (85.2000,121.6000) -- (85.3000,121.6000) -- (85.5000,121.5000) -- (85.6000,121.5000) -- (85.7000,121.5000) -- (85.9000,121.5000) -- (86.0000,121.5000) -- (86.2000,121.5000) -- (86.3000,121.4000) -- (86.5000,121.4000) -- (86.6000,121.4000) -- (86.7000,121.4000) -- (86.9000,121.3000) -- (87.0000,121.3000) -- (87.2000,121.3000) -- (87.3000,121.3000) -- (87.5000,121.3000) -- (87.6000,121.2000) -- (87.7000,121.2000) -- (87.9000,121.2000) -- (88.0000,121.2000) -- (88.2000,121.1000) -- (88.3000,121.1000) -- (88.5000,121.1000) -- (88.6000,121.1000) -- (88.7000,121.0000) -- (88.9000,121.0000) -- (89.0000,121.0000) -- (89.2000,121.0000) -- (89.3000,120.9000) -- (89.5000,120.9000) -- (89.6000,120.9000) -- (89.7000,120.9000) -- (89.9000,120.8000) -- (90.0000,120.8000) -- (90.2000,120.8000) -- (90.3000,120.8000) -- (90.5000,120.7000) -- (90.6000,120.7000) -- (90.7000,120.7000) -- (90.9000,120.6000) -- (91.0000,120.6000) -- (91.2000,120.6000) -- (91.3000,120.6000) -- (91.5000,120.5000) -- (91.6000,120.5000) -- (91.7000,120.5000) -- (91.9000,120.5000) -- (92.0000,120.4000) -- (92.2000,120.4000) -- (92.3000,120.4000) -- (92.5000,120.4000) -- (92.6000,120.3000) -- (92.7000,120.3000) -- (92.9000,120.3000) -- (93.0000,120.2000) -- (93.2000,120.2000) -- (93.3000,120.2000) -- (93.4000,120.2000) -- (93.6000,120.1000) -- (93.7000,120.1000) -- (93.9000,120.1000) -- (94.0000,120.1000) -- (94.2000,120.0000) -- (94.3000,120.0000) -- (94.4000,120.0000) -- (94.6000,120.0000) -- (94.7000,119.9000) -- (94.9000,119.9000) -- (95.0000,119.9000) -- (95.2000,119.8000) -- (95.3000,119.8000) -- (95.4000,119.8000) -- (95.6000,119.8000) -- (95.7000,119.7000) -- (95.9000,119.7000) -- (96.0000,119.7000) -- (96.2000,119.7000) -- (96.3000,119.6000) -- (96.4000,119.6000) -- (96.6000,119.6000) -- (96.7000,119.6000) -- (96.9000,119.5000) -- (97.0000,119.5000) -- (97.2000,119.5000) -- (97.3000,119.5000) -- (97.4000,119.4000) -- (97.6000,119.4000) -- (97.7000,119.4000) -- (97.9000,119.3000) -- (98.0000,119.3000) -- (98.2000,119.3000) -- (98.3000,119.3000) -- (98.4000,119.2000) -- (98.6000,119.2000) -- (98.7000,119.2000) -- (98.9000,119.1000) -- (99.0000,119.1000) -- (99.2000,119.1000) -- (99.3000,119.1000) -- (99.4000,119.0000) -- (99.6000,119.0000) -- (99.7000,119.0000) -- (99.9000,119.0000) -- (100.0000,118.9000) -- (100.2000,118.9000) -- (100.3000,118.9000) -- (100.4000,118.9000) -- (100.6000,118.8000) -- (100.7000,118.8000) -- (100.9000,118.8000) -- (101.0000,118.8000) -- (101.2000,118.7000) -- (101.3000,118.7000) -- (101.4000,118.7000) -- (101.6000,118.7000) -- (101.7000,118.6000) -- (101.9000,118.6000) -- (102.0000,118.6000) -- (102.2000,118.6000) -- (102.3000,118.5000) -- (102.4000,118.5000) -- (102.6000,118.5000) -- (102.7000,118.5000) -- (102.9000,118.4000) -- (103.0000,118.4000) -- (103.2000,118.4000) -- (103.3000,118.4000) -- (103.4000,118.4000) -- (103.6000,118.3000) -- (103.7000,118.3000) -- (103.9000,118.3000) -- (104.0000,118.3000) -- (104.2000,118.3000) -- (104.3000,118.2000) -- (104.4000,118.2000) -- (104.6000,118.2000) -- (104.7000,118.2000) -- (104.9000,118.2000) -- (105.0000,118.1000) -- (105.2000,118.1000) -- (105.3000,118.1000) -- (105.4000,118.1000) -- (105.6000,118.1000) -- (105.7000,118.0000) -- (105.9000,118.0000) -- (106.0000,118.0000) -- (106.2000,118.0000) -- (106.3000,118.0000) -- (106.4000,118.0000) -- (106.6000,117.9000) -- (106.7000,117.9000) -- (106.9000,117.9000) -- (107.0000,117.9000) -- (107.2000,117.9000) -- (107.3000,117.9000) -- (107.4000,117.9000) -- (107.6000,117.8000) -- (107.7000,117.8000) -- (107.9000,117.8000) -- (108.0000,117.8000) -- (108.2000,117.8000) -- (108.3000,117.8000) -- (108.4000,117.8000) -- (108.6000,117.8000) -- (108.7000,117.7000) -- (108.9000,117.7000) -- (109.0000,117.7000) -- (109.2000,117.7000) -- (109.3000,117.7000) -- (109.4000,117.7000) -- (109.6000,117.7000) -- (109.7000,117.7000) -- (109.9000,117.7000) -- (110.0000,117.6000) -- (110.2000,117.6000) -- (110.3000,117.6000) -- (110.4000,117.6000) -- (110.6000,117.6000) -- (110.7000,117.6000) -- (110.9000,117.6000) -- (111.0000,117.6000) -- (111.2000,117.6000) -- (111.3000,117.6000) -- (111.4000,117.6000) -- (111.6000,117.6000) -- (111.7000,117.5000) -- (111.9000,117.5000) -- (112.0000,117.5000) -- (112.2000,117.5000) -- (112.3000,117.5000) -- (112.4000,117.5000) -- (112.6000,117.5000) -- (112.7000,117.5000) -- (112.9000,117.5000) -- (113.0000,117.5000) -- (113.2000,117.5000) -- (113.3000,117.5000) -- (113.4000,117.5000) -- (113.6000,117.5000) -- (113.7000,117.5000) -- (113.9000,117.5000) -- (114.0000,117.5000) -- (114.2000,117.5000) -- (114.3000,117.4000) -- (114.4000,117.4000) -- (114.6000,117.4000) -- (114.7000,117.4000) -- (114.9000,117.4000) -- (115.0000,117.4000) -- (115.2000,117.4000) -- (115.3000,117.4000) -- (115.4000,117.4000) -- (115.6000,117.4000) -- (115.7000,117.4000) -- (115.9000,117.4000) -- (116.0000,117.4000) -- (116.2000,117.4000) -- (116.3000,117.4000) -- (116.4000,117.4000) -- (116.6000,117.4000) -- (116.7000,117.4000) -- (116.9000,117.4000) -- (117.0000,117.4000) -- (117.2000,117.4000) -- (117.3000,117.4000) -- (117.4000,117.4000) -- (117.6000,117.4000) -- (117.7000,117.4000) -- (117.9000,117.4000) -- (118.0000,117.4000) -- (118.2000,117.4000) -- (118.3000,117.4000) -- (118.4000,117.4000) -- (118.6000,117.4000) -- (118.7000,117.4000) -- (118.9000,117.4000) -- (119.0000,117.4000) -- (119.2000,117.4000) -- (119.3000,117.4000) -- (119.4000,117.4000) -- (119.6000,117.4000) -- (119.7000,117.4000) -- (119.9000,117.4000) -- (120.0000,117.4000) -- (120.2000,117.4000) -- (120.3000,117.4000) -- (120.4000,117.4000) -- (120.6000,117.4000) -- (120.7000,117.4000) -- (120.9000,117.4000) -- (121.0000,117.4000) -- (121.2000,117.4000) -- (121.3000,117.4000) -- (121.4000,117.4000) -- (121.6000,117.4000) -- (121.7000,117.4000) -- (121.9000,117.4000) -- (122.0000,117.4000) -- (122.2000,117.4000) -- (122.3000,117.4000) -- (122.4000,117.4000) -- (122.6000,117.4000) -- (122.7000,117.4000) -- (122.9000,117.4000) -- (123.0000,117.4000) -- (123.2000,117.4000) -- (123.3000,117.4000) -- (123.4000,117.4000) -- (123.6000,117.4000) -- (123.7000,117.4000) -- (123.9000,117.4000) -- (124.0000,117.4000) -- (124.2000,117.4000) -- (124.3000,117.4000) -- (124.4000,117.5000) -- (124.6000,117.5000) -- (124.7000,117.5000) -- (124.9000,117.5000) -- (125.0000,117.5000) -- (125.2000,117.5000) -- (125.3000,117.5000) -- (125.4000,117.5000) -- (125.6000,117.5000) -- (125.7000,117.5000) -- (125.9000,117.5000) -- (126.0000,117.5000) -- (126.2000,117.5000) -- (126.3000,117.5000) -- (126.4000,117.5000) -- (126.6000,117.5000) -- (126.7000,117.5000) -- (126.9000,117.5000) -- (127.0000,117.5000) -- (127.2000,117.5000) -- (127.3000,117.5000);



    \end{scope}
    \begin{scope}[cm={{1.0018,0.0,0.0,0.97485,(-208.78685,-65.87769)}},draw=cff0000,line join=bevel,line cap=round,miter limit=4.00,line width=0.480pt]
      \path[draw,line join=round,line cap=round,miter limit=4.00,line width=0.480pt] (41.6000,109.8000) -- (41.6000,109.8000) -- (41.7000,103.2000) -- (41.9000,103.5000) -- (42.0000,103.9000) -- (42.2000,104.4000) -- (42.3000,104.8000) -- (42.5000,105.2000) -- (42.6000,105.6000) -- (42.7000,106.0000) -- (42.9000,106.4000) -- (43.0000,106.7000) -- (43.2000,107.1000) -- (43.3000,107.4000) -- (43.5000,107.7000) -- (43.6000,107.9000) -- (43.7000,108.2000) -- (43.9000,108.5000) -- (44.0000,108.7000) -- (44.2000,108.9000) -- (44.3000,109.1000) -- (44.5000,109.4000) -- (44.6000,109.6000) -- (44.7000,109.7000) -- (44.9000,109.9000) -- (45.0000,110.1000) -- (45.2000,110.3000) -- (45.3000,110.5000) -- (45.5000,110.7000) -- (45.6000,110.9000) -- (45.7000,111.1000) -- (45.9000,111.3000) -- (46.0000,111.5000) -- (46.2000,111.7000) -- (46.3000,111.9000) -- (46.5000,112.1000) -- (46.6000,112.3000) -- (46.7000,112.5000) -- (46.9000,112.7000) -- (47.0000,112.9000) -- (47.2000,113.1000) -- (47.3000,113.3000) -- (47.5000,113.5000) -- (47.6000,113.7000) -- (47.7000,113.9000) -- (47.9000,114.1000) -- (48.0000,114.3000) -- (48.2000,114.5000) -- (48.3000,114.6000) -- (48.5000,114.8000) -- (48.6000,114.9000) -- (48.7000,115.1000) -- (48.9000,115.2000) -- (49.0000,115.3000) -- (49.2000,115.4000) -- (49.3000,115.6000) -- (49.5000,115.7000) -- (49.6000,115.8000) -- (49.7000,115.9000) -- (49.9000,116.0000) -- (50.0000,116.1000) -- (50.2000,116.2000) -- (50.3000,116.3000) -- (50.5000,116.4000) -- (50.6000,116.5000) -- (50.7000,116.6000) -- (50.9000,116.8000) -- (51.0000,116.9000) -- (51.2000,117.0000) -- (51.3000,117.2000) -- (51.5000,117.3000) -- (51.6000,117.5000) -- (51.7000,117.7000) -- (51.9000,117.8000) -- (52.0000,118.0000) -- (52.2000,118.2000) -- (52.3000,118.3000) -- (52.5000,118.5000) -- (52.6000,118.6000) -- (52.7000,118.8000) -- (52.9000,118.9000) -- (53.0000,119.0000) -- (53.2000,119.1000) -- (53.3000,119.2000) -- (53.5000,119.3000) -- (53.6000,119.4000) -- (53.7000,119.4000) -- (53.9000,119.4000) -- (54.0000,119.5000) -- (54.2000,119.5000) -- (54.3000,119.5000) -- (54.5000,119.4000) -- (54.6000,119.4000) -- (54.7000,119.4000) -- (54.9000,119.3000) -- (55.0000,119.3000) -- (55.2000,119.2000) -- (55.3000,119.2000) -- (55.5000,119.1000) -- (55.6000,119.1000) -- (55.7000,119.1000) -- (55.9000,119.0000) -- (56.0000,119.0000) -- (56.2000,119.0000) -- (56.3000,119.1000) -- (56.5000,119.1000) -- (56.6000,119.2000) -- (56.7000,119.2000) -- (56.9000,119.3000) -- (57.0000,119.4000) -- (57.2000,119.5000) -- (57.3000,119.6000) -- (57.5000,119.7000) -- (57.6000,119.8000) -- (57.7000,119.9000) -- (57.9000,120.1000) -- (58.0000,120.2000) -- (58.2000,120.3000) -- (58.3000,120.4000) -- (58.5000,120.5000) -- (58.6000,120.6000) -- (58.7000,120.7000) -- (58.9000,120.8000) -- (59.0000,120.8000) -- (59.2000,120.9000) -- (59.3000,121.0000) -- (59.5000,121.0000) -- (59.6000,121.0000) -- (59.7000,121.1000) -- (59.9000,121.1000) -- (60.0000,121.1000) -- (60.2000,121.1000) -- (60.3000,121.1000) -- (60.5000,121.1000) -- (60.6000,121.1000) -- (60.7000,121.0000) -- (60.9000,121.0000) -- (61.0000,121.0000) -- (61.2000,121.0000) -- (61.3000,120.9000) -- (61.5000,120.9000) -- (61.6000,120.9000) -- (61.7000,120.8000) -- (61.9000,120.8000) -- (62.0000,120.8000) -- (62.2000,120.8000) -- (62.3000,120.7000) -- (62.5000,120.7000) -- (62.6000,120.7000) -- (62.7000,120.7000) -- (62.9000,120.7000) -- (63.0000,120.7000) -- (63.2000,120.7000) -- (63.3000,120.7000) -- (63.5000,120.7000) -- (63.6000,120.8000) -- (63.7000,120.8000) -- (63.9000,120.8000) -- (64.0000,120.8000) -- (64.2000,120.8000) -- (64.3000,120.9000) -- (64.5000,120.9000) -- (64.6000,120.9000) -- (64.7000,121.0000) -- (64.9000,121.0000) -- (65.0000,121.0000) -- (65.2000,121.0000) -- (65.3000,121.1000) -- (65.5000,121.1000) -- (65.6000,121.1000) -- (65.7000,121.1000) -- (65.9000,121.2000) -- (66.0000,121.2000) -- (66.2000,121.2000) -- (66.3000,121.2000) -- (66.5000,121.2000) -- (66.6000,121.2000) -- (66.7000,121.2000) -- (66.9000,121.2000) -- (67.0000,121.2000) -- (67.2000,121.2000) -- (67.3000,121.2000) -- (67.5000,121.2000) -- (67.6000,121.2000) -- (67.7000,121.2000) -- (67.9000,121.2000) -- (68.0000,121.2000) -- (68.2000,121.1000) -- (68.3000,121.1000) -- (68.5000,121.1000) -- (68.6000,121.1000) -- (68.7000,121.0000) -- (68.9000,121.0000) -- (69.0000,121.0000) -- (69.2000,120.9000) -- (69.3000,120.9000) -- (69.5000,120.9000) -- (69.6000,120.8000) -- (69.7000,120.8000) -- (69.9000,120.8000) -- (70.0000,120.7000) -- (70.2000,120.7000) -- (70.3000,120.7000) -- (70.5000,120.7000) -- (70.6000,120.6000) -- (70.7000,120.6000) -- (70.9000,120.6000) -- (71.0000,120.6000) -- (71.2000,120.6000) -- (71.3000,120.6000) -- (71.5000,120.6000) -- (71.6000,120.6000) -- (71.7000,120.6000) -- (71.9000,120.6000) -- (72.0000,120.6000) -- (72.2000,120.6000) -- (72.3000,120.6000) -- (72.5000,120.6000) -- (72.6000,120.6000) -- (72.7000,120.7000) -- (72.9000,120.7000) -- (73.0000,120.7000) -- (73.2000,120.8000) -- (73.3000,120.8000) -- (73.5000,120.8000) -- (73.6000,120.9000) -- (73.7000,120.9000) -- (73.9000,120.9000) -- (74.0000,121.0000) -- (74.2000,121.0000) -- (74.3000,121.1000) -- (74.5000,121.1000) -- (74.6000,121.2000) -- (74.7000,121.2000) -- (74.9000,121.3000) -- (75.0000,121.3000) -- (75.2000,121.3000) -- (75.3000,121.4000) -- (75.5000,121.4000) -- (75.6000,121.5000) -- (75.7000,121.5000) -- (75.9000,121.6000) -- (76.0000,121.6000) -- (76.2000,121.7000) -- (76.3000,121.7000) -- (76.5000,121.7000) -- (76.6000,121.8000) -- (76.7000,121.8000) -- (76.9000,121.9000) -- (77.0000,121.9000) -- (77.2000,121.9000) -- (77.3000,122.0000) -- (77.5000,122.0000) -- (77.6000,122.0000) -- (77.7000,122.1000) -- (77.9000,122.1000) -- (78.0000,122.1000) -- (78.2000,122.1000) -- (78.3000,122.1000) -- (78.5000,122.2000) -- (78.6000,122.2000) -- (78.7000,122.2000) -- (78.9000,122.2000) -- (79.0000,122.2000) -- (79.2000,122.2000) -- (79.3000,122.2000) -- (79.5000,122.2000) -- (79.6000,122.2000) -- (79.7000,122.2000) -- (79.9000,122.2000) -- (80.0000,122.2000) -- (80.2000,122.2000) -- (80.3000,122.2000) -- (80.5000,122.2000) -- (80.6000,122.2000) -- (80.7000,122.2000) -- (80.9000,122.2000) -- (81.0000,122.2000) -- (81.2000,122.2000) -- (81.3000,122.2000) -- (81.5000,122.1000) -- (81.6000,122.1000) -- (81.7000,122.1000) -- (81.9000,122.1000) -- (82.0000,122.1000) -- (82.2000,122.0000) -- (82.3000,122.0000) -- (82.5000,122.0000) -- (82.6000,122.0000) -- (82.7000,121.9000) -- (82.9000,121.9000) -- (83.0000,121.9000) -- (83.2000,121.9000) -- (83.3000,121.8000) -- (83.5000,121.8000) -- (83.6000,121.8000) -- (83.7000,121.8000) -- (83.9000,121.7000) -- (84.0000,121.7000) -- (84.2000,121.7000) -- (84.3000,121.7000) -- (84.5000,121.6000) -- (84.6000,121.6000) -- (84.7000,121.6000) -- (84.9000,121.5000) -- (85.0000,121.5000) -- (85.2000,121.5000) -- (85.3000,121.5000) -- (85.5000,121.4000) -- (85.6000,121.4000) -- (85.7000,121.4000) -- (85.9000,121.3000) -- (86.0000,121.3000) -- (86.2000,121.3000) -- (86.3000,121.3000) -- (86.5000,121.2000) -- (86.6000,121.2000) -- (86.7000,121.2000) -- (86.9000,121.2000) -- (87.0000,121.1000) -- (87.2000,121.1000) -- (87.3000,121.1000) -- (87.5000,121.1000) -- (87.6000,121.1000) -- (87.7000,121.0000) -- (87.9000,121.0000) -- (88.0000,121.0000) -- (88.2000,121.0000) -- (88.3000,120.9000) -- (88.5000,120.9000) -- (88.6000,120.9000) -- (88.7000,120.9000) -- (88.9000,120.9000) -- (89.0000,120.8000) -- (89.2000,120.8000) -- (89.3000,120.8000) -- (89.5000,120.8000) -- (89.6000,120.8000) -- (89.7000,120.7000) -- (89.9000,120.7000) -- (90.0000,120.7000) -- (90.2000,120.7000) -- (90.3000,120.7000) -- (90.5000,120.6000) -- (90.6000,120.6000) -- (90.7000,120.6000) -- (90.9000,120.6000) -- (91.0000,120.6000) -- (91.2000,120.5000) -- (91.3000,120.5000) -- (91.5000,120.5000) -- (91.6000,120.5000) -- (91.7000,120.5000) -- (91.9000,120.4000) -- (92.0000,120.4000) -- (92.2000,120.4000) -- (92.3000,120.4000) -- (92.5000,120.4000) -- (92.6000,120.4000) -- (92.7000,120.3000) -- (92.9000,120.3000) -- (93.0000,120.3000) -- (93.2000,120.3000) -- (93.3000,120.3000) -- (93.4000,120.2000) -- (93.6000,120.2000) -- (93.7000,120.2000) -- (93.9000,120.2000) -- (94.0000,120.2000) -- (94.2000,120.1000) -- (94.3000,120.1000) -- (94.4000,120.1000) -- (94.6000,120.1000) -- (94.7000,120.0000) -- (94.9000,120.0000) -- (95.0000,120.0000) -- (95.2000,120.0000) -- (95.3000,120.0000) -- (95.4000,119.9000) -- (95.6000,119.9000) -- (95.7000,119.9000) -- (95.9000,119.9000) -- (96.0000,119.8000) -- (96.2000,119.8000) -- (96.3000,119.8000) -- (96.4000,119.8000) -- (96.6000,119.7000) -- (96.7000,119.7000) -- (96.9000,119.7000) -- (97.0000,119.7000) -- (97.2000,119.6000) -- (97.3000,119.6000) -- (97.4000,119.6000) -- (97.6000,119.5000) -- (97.7000,119.5000) -- (97.9000,119.5000) -- (98.0000,119.5000) -- (98.2000,119.4000) -- (98.3000,119.4000) -- (98.4000,119.4000) -- (98.6000,119.3000) -- (98.7000,119.3000) -- (98.9000,119.3000) -- (99.0000,119.2000) -- (99.2000,119.2000) -- (99.3000,119.2000) -- (99.4000,119.1000) -- (99.6000,119.1000) -- (99.7000,119.1000) -- (99.9000,119.1000) -- (100.0000,119.0000) -- (100.2000,119.0000) -- (100.3000,119.0000) -- (100.4000,118.9000) -- (100.6000,118.9000) -- (100.7000,118.9000) -- (100.9000,118.8000) -- (101.0000,118.8000) -- (101.2000,118.8000) -- (101.3000,118.7000) -- (101.4000,118.7000) -- (101.6000,118.7000) -- (101.7000,118.6000) -- (101.9000,118.6000) -- (102.0000,118.6000) -- (102.2000,118.6000) -- (102.3000,118.5000) -- (102.4000,118.5000) -- (102.6000,118.5000) -- (102.7000,118.4000) -- (102.9000,118.4000) -- (103.0000,118.4000) -- (103.2000,118.4000) -- (103.3000,118.3000) -- (103.4000,118.3000) -- (103.6000,118.3000) -- (103.7000,118.3000) -- (103.9000,118.2000) -- (104.0000,118.2000) -- (104.2000,118.2000) -- (104.3000,118.2000) -- (104.4000,118.1000) -- (104.6000,118.1000) -- (104.7000,118.1000) -- (104.9000,118.1000) -- (105.0000,118.0000) -- (105.2000,118.0000) -- (105.3000,118.0000) -- (105.4000,118.0000) -- (105.6000,118.0000) -- (105.7000,117.9000) -- (105.9000,117.9000) -- (106.0000,117.9000) -- (106.2000,117.9000) -- (106.3000,117.9000) -- (106.4000,117.8000) -- (106.6000,117.8000) -- (106.7000,117.8000) -- (106.9000,117.8000) -- (107.0000,117.8000) -- (107.2000,117.8000) -- (107.3000,117.7000) -- (107.4000,117.7000) -- (107.6000,117.7000) -- (107.7000,117.7000) -- (107.9000,117.7000) -- (108.0000,117.7000) -- (108.2000,117.7000) -- (108.3000,117.6000) -- (108.4000,117.6000) -- (108.6000,117.6000) -- (108.7000,117.6000) -- (108.9000,117.6000) -- (109.0000,117.6000) -- (109.2000,117.6000) -- (109.3000,117.6000) -- (109.4000,117.6000) -- (109.6000,117.6000) -- (109.7000,117.5000) -- (109.9000,117.5000) -- (110.0000,117.5000) -- (110.2000,117.5000) -- (110.3000,117.5000) -- (110.4000,117.5000) -- (110.6000,117.5000) -- (110.7000,117.5000) -- (110.9000,117.5000) -- (111.0000,117.5000) -- (111.2000,117.5000) -- (111.3000,117.5000) -- (111.4000,117.5000) -- (111.6000,117.5000) -- (111.7000,117.5000) -- (111.9000,117.5000) -- (112.0000,117.5000) -- (112.2000,117.5000) -- (112.3000,117.4000) -- (112.4000,117.4000) -- (112.6000,117.4000) -- (112.7000,117.4000) -- (112.9000,117.4000) -- (113.0000,117.4000) -- (113.2000,117.4000) -- (113.3000,117.4000) -- (113.4000,117.4000) -- (113.6000,117.4000) -- (113.7000,117.4000) -- (113.9000,117.4000) -- (114.0000,117.4000) -- (114.2000,117.4000) -- (114.3000,117.4000) -- (114.4000,117.4000) -- (114.6000,117.4000) -- (114.7000,117.4000) -- (114.9000,117.4000) -- (115.0000,117.4000) -- (115.2000,117.4000) -- (115.3000,117.4000) -- (115.4000,117.4000) -- (115.6000,117.4000) -- (115.7000,117.4000) -- (115.9000,117.4000) -- (116.0000,117.4000) -- (116.2000,117.4000) -- (116.3000,117.4000) -- (116.4000,117.4000) -- (116.6000,117.4000) -- (116.7000,117.4000) -- (116.9000,117.5000) -- (117.0000,117.5000) -- (117.2000,117.5000) -- (117.3000,117.5000) -- (117.4000,117.5000) -- (117.6000,117.5000) -- (117.7000,117.5000) -- (117.9000,117.5000) -- (118.0000,117.5000) -- (118.2000,117.5000) -- (118.3000,117.5000) -- (118.4000,117.5000) -- (118.6000,117.5000) -- (118.7000,117.5000) -- (118.9000,117.5000) -- (119.0000,117.5000) -- (119.2000,117.5000) -- (119.3000,117.5000) -- (119.4000,117.5000) -- (119.6000,117.5000) -- (119.7000,117.5000) -- (119.9000,117.5000) -- (120.0000,117.5000) -- (120.2000,117.5000) -- (120.3000,117.5000) -- (120.4000,117.5000) -- (120.6000,117.5000) -- (120.7000,117.5000) -- (120.9000,117.5000) -- (121.0000,117.5000) -- (121.2000,117.5000) -- (121.3000,117.5000) -- (121.4000,117.5000) -- (121.6000,117.5000) -- (121.7000,117.5000) -- (121.9000,117.5000) -- (122.0000,117.5000) -- (122.2000,117.5000) -- (122.3000,117.5000) -- (122.4000,117.5000) -- (122.6000,117.5000) -- (122.7000,117.5000) -- (122.9000,117.5000) -- (123.0000,117.5000) -- (123.2000,117.5000) -- (123.3000,117.5000) -- (123.4000,117.5000) -- (123.6000,117.5000) -- (123.7000,117.5000) -- (123.9000,117.6000) -- (124.0000,117.6000) -- (124.2000,117.6000) -- (124.3000,117.6000) -- (124.4000,117.6000) -- (124.6000,117.6000) -- (124.7000,117.6000) -- (124.9000,117.6000) -- (125.0000,117.6000) -- (125.2000,117.6000) -- (125.3000,117.6000) -- (125.4000,117.6000) -- (125.6000,117.6000) -- (125.7000,117.6000) -- (125.9000,117.6000) -- (126.0000,117.6000) -- (126.2000,117.6000) -- (126.3000,117.6000) -- (126.4000,117.6000) -- (126.6000,117.6000) -- (126.7000,117.6000) -- (126.9000,117.6000) -- (127.0000,117.6000) -- (127.2000,117.6000) -- (127.3000,117.6000);



    \end{scope}
    \begin{scope}[cm={{1.0018,0.0,0.0,0.97485,(-208.78685,-65.87769)}},fill=cebebeb]
      \path[fill=cebebeb,rounded corners=0.0000cm] (81.0000,129.0000) rectangle (121.0000,157.0000);



    \end{scope}
    \begin{scope}[cm={{1.0018,0.0,0.0,0.97485,(-208.78685,-65.87769)}},draw=ca0a0a4,dash pattern=on 0.40pt off 0.80pt,line join=round,line cap=round,line width=0.400pt]
      \path[draw] (81.5000,135.5000) -- (121.5000,135.5000);



    \end{scope}
    \begin{scope}[cm={{1.0018,0.0,0.0,0.97485,(-208.78685,-65.87769)}},draw=black,line join=round,line cap=round,line width=0.480pt]
      \path[draw] (81.5000,135.5000) -- (82.6460,135.5000);



      \path[draw] (121.5000,135.5000) -- (120.1220,135.5000);



    \end{scope}
    \begin{scope}[cm={{1.0018,0.0,0.0,1.0018,(-159.61892,69.71391)}},draw=black,line join=bevel,line cap=rect,line width=0.800pt]
      \path[fill=black] (0.0000,0.0000) node[above right] (text766) {\scriptsize $T=$ 47};



    \end{scope}
    \begin{scope}[cm={{1.0018,0.0,0.0,0.97485,(-208.78685,-65.87769)}},draw=ca0a0a4,dash pattern=on 0.40pt off 0.80pt,line join=round,line cap=round,line width=0.400pt]
      \path[draw] (97.5000,157.5000) -- (97.5000,129.5000);



    \end{scope}
    \begin{scope}[cm={{1.0018,0.0,0.0,0.97485,(-208.78685,-65.87769)}},draw=black,line join=round,line cap=round,line width=0.480pt]
      \path[draw] (97.5000,129.5000) -- (97.5000,129.5000) -- (97.5000,130.6564);



    \end{scope}
    \begin{scope}[cm={{1.0018,0.0,0.0,1.0018,(-115.12985,58.02564)}},draw=black,line join=bevel,line cap=rect,line width=0.800pt]
      \path[fill=black] (0.0000,0.0000) node[above right] (text794) {\scriptsize 84};



    \end{scope}
    \begin{scope}[cm={{1.0018,0.0,0.0,0.97485,(-208.78685,-65.87769)}},draw=black,line join=round,line cap=round,line width=0.480pt]
      \path[draw] (81.5000,129.5000) -- (81.5000,157.5000) -- (121.5000,157.5000) -- (121.5000,129.5000) -- (81.5000,129.5000);



    \end{scope}
    \begin{scope}[cm={{1.0018,0.0,0.0,0.97485,(-208.78685,-65.87769)}},draw=black,line join=round,line cap=round,line width=0.480pt]
      \path[draw] (81.2000,156.5000) -- (81.2000,156.5000) -- (81.2000,156.5000) -- (81.2000,156.5000) -- (81.2000,156.5000) -- (81.2000,156.5000) -- (81.2000,156.5000) -- (81.2000,156.5000) -- (81.2000,156.5000) -- (81.2000,156.5000) -- (81.2000,156.5000) -- (81.2000,156.5000) -- (81.2000,156.5000) -- (81.2000,156.5000) -- (81.2000,156.5000) -- (81.2000,156.5000) -- (81.2000,156.5000) -- (81.2000,156.5000) -- (81.2000,156.5000) -- (81.2000,156.5000) -- (81.2000,156.4000) -- (81.2000,156.4000) -- (81.2000,156.4000) -- (81.2000,156.4000) -- (81.2000,156.4000) -- (81.2000,156.4000) -- (81.3000,156.4000) -- (81.3000,156.4000) -- (81.3000,156.4000) -- (81.3000,156.4000) -- (81.3000,156.4000) -- (81.3000,156.4000) -- (81.3000,156.4000) -- (81.3000,156.4000) -- (81.3000,156.4000) -- (81.3000,156.4000) -- (81.3000,156.4000) -- (81.3000,156.4000) -- (81.3000,156.4000) -- (81.3000,156.4000) -- (81.3000,156.4000) -- (81.3000,156.3000) -- (81.3000,156.3000) -- (81.3000,156.3000) -- (81.3000,156.3000) -- (81.3000,156.3000) -- (81.3000,156.3000) -- (81.3000,156.3000) -- (81.3000,156.3000) -- (81.3000,156.3000) -- (81.3000,156.3000) -- (81.3000,156.3000) -- (81.3000,156.3000) -- (81.3000,156.3000) -- (81.3000,156.3000) -- (81.3000,156.3000) -- (81.3000,156.3000) -- (81.3000,156.3000) -- (81.3000,156.3000) -- (81.3000,156.3000) -- (81.3000,156.3000) -- (81.3000,156.3000) -- (81.3000,156.3000) -- (81.3000,156.2000) -- (81.3000,156.2000) -- (81.3000,156.2000) -- (81.3000,156.2000) -- (81.3000,156.2000) -- (81.3000,156.2000) -- (81.3000,156.2000) -- (81.3000,156.2000) -- (81.3000,156.2000) -- (81.3000,156.2000) -- (81.3000,156.2000) -- (81.3000,156.2000) -- (81.3000,156.2000) -- (81.4000,156.2000) -- (81.4000,156.2000) -- (81.4000,156.2000) -- (81.4000,156.2000) -- (81.4000,156.2000) -- (81.4000,156.2000) -- (81.4000,156.2000) -- (81.4000,156.2000) -- (81.4000,156.1000) -- (81.4000,156.1000) -- (81.4000,156.1000) -- (81.4000,156.1000) -- (81.4000,156.1000) -- (81.4000,156.1000) -- (81.4000,156.1000) -- (81.4000,156.1000) -- (81.4000,156.1000) -- (81.4000,156.1000) -- (81.4000,156.1000) -- (81.4000,156.1000) -- (81.4000,156.1000) -- (81.4000,156.1000) -- (81.4000,156.1000) -- (81.4000,156.1000) -- (81.4000,156.1000) -- (81.4000,156.1000) -- (81.4000,156.1000) -- (81.4000,156.1000) -- (81.4000,156.1000) -- (81.4000,156.1000) -- (81.4000,156.0000) -- (81.4000,156.0000) -- (81.4000,156.0000) -- (81.4000,156.0000) -- (81.4000,156.0000) -- (81.4000,156.0000) -- (81.4000,156.0000) -- (81.4000,156.0000) -- (81.4000,156.0000) -- (81.4000,156.0000) -- (81.4000,156.0000) -- (81.4000,156.0000) -- (81.4000,156.0000) -- (81.4000,156.0000) -- (81.4000,156.0000) -- (81.4000,156.0000) -- (81.4000,156.0000) -- (81.4000,156.0000) -- (81.4000,156.0000) -- (81.5000,156.0000) -- (81.5000,156.0000) -- (81.5000,155.9000) -- (81.5000,155.9000) -- (81.5000,155.9000) -- (81.5000,155.9000) -- (81.5000,155.9000) -- (81.5000,155.9000) -- (81.5000,155.9000) -- (81.5000,155.9000) -- (81.5000,155.9000) -- (81.5000,155.9000) -- (81.5000,155.9000) -- (81.5000,155.9000) -- (81.5000,155.9000) -- (81.5000,155.9000) -- (81.5000,155.9000) -- (81.5000,155.9000) -- (81.5000,155.9000) -- (81.5000,155.9000) -- (81.5000,155.9000) -- (81.5000,155.9000) -- (81.5000,155.9000) -- (81.5000,155.9000) -- (81.5000,155.8000) -- (81.5000,155.8000) -- (81.5000,155.8000) -- (81.5000,155.8000) -- (81.5000,155.8000) -- (81.5000,155.8000) -- (81.5000,155.8000) -- (81.5000,155.8000) -- (81.5000,155.8000) -- (81.5000,155.8000) -- (81.5000,155.8000) -- (81.5000,155.8000) -- (81.5000,155.8000) -- (81.5000,155.8000) -- (81.5000,155.8000) -- (81.5000,155.8000) -- (81.5000,155.8000) -- (81.5000,155.8000) -- (81.5000,155.8000) -- (81.5000,155.8000) -- (81.5000,155.8000) -- (81.5000,155.7000) -- (81.5000,155.7000) -- (81.5000,155.7000) -- (81.5000,155.7000) -- (81.5000,155.7000) -- (81.6000,155.7000) -- (81.6000,155.7000) -- (81.6000,155.7000) -- (81.6000,155.7000) -- (81.6000,155.7000) -- (81.6000,155.7000) -- (81.6000,155.7000) -- (81.6000,155.7000) -- (81.6000,155.7000) -- (81.6000,155.7000) -- (81.6000,155.7000) -- (81.6000,155.7000) -- (81.6000,155.7000) -- (81.6000,155.7000) -- (81.6000,155.7000) -- (81.6000,155.7000) -- (81.6000,155.7000) -- (81.6000,155.6000) -- (81.6000,155.6000) -- (81.6000,155.6000) -- (81.6000,155.6000) -- (81.6000,155.6000) -- (81.6000,155.6000) -- (81.6000,155.6000) -- (81.6000,155.6000) -- (81.6000,155.6000) -- (81.6000,155.6000) -- (81.6000,155.6000) -- (81.6000,155.6000) -- (81.6000,155.6000) -- (81.6000,155.6000) -- (81.6000,155.6000) -- (81.6000,155.6000) -- (81.6000,155.6000) -- (81.6000,155.6000) -- (81.6000,155.6000) -- (81.6000,155.6000) -- (81.6000,155.6000) -- (81.6000,155.5000) -- (81.6000,155.5000) -- (81.6000,155.5000) -- (81.6000,155.5000) -- (81.6000,155.5000) -- (81.6000,155.5000) -- (81.6000,155.5000) -- (81.6000,155.5000) -- (81.6000,155.5000) -- (81.6000,155.5000) -- (81.6000,155.5000) -- (81.7000,155.5000) -- (81.7000,155.5000) -- (81.7000,155.5000) -- (81.7000,155.5000) -- (81.7000,155.5000) -- (81.7000,155.5000) -- (81.7000,155.5000) -- (81.7000,155.5000) -- (81.7000,155.5000) -- (81.7000,155.5000) -- (81.7000,155.5000) -- (81.7000,155.4000) -- (81.7000,155.4000) -- (81.7000,155.4000) -- (81.7000,155.4000) -- (81.7000,155.4000) -- (81.7000,155.4000) -- (81.7000,155.4000) -- (81.7000,155.4000) -- (81.7000,155.4000) -- (81.7000,155.4000) -- (81.7000,155.4000) -- (81.7000,155.4000) -- (81.7000,155.4000) -- (81.7000,155.4000) -- (81.7000,155.4000) -- (81.7000,155.4000) -- (81.7000,155.4000) -- (81.7000,155.4000) -- (81.7000,155.4000) -- (81.7000,155.4000) -- (81.7000,155.4000) -- (81.7000,155.3000) -- (81.7000,155.3000) -- (81.7000,155.3000) -- (81.7000,155.3000) -- (81.7000,155.3000) -- (81.7000,155.3000) -- (81.7000,155.3000) -- (81.7000,155.3000) -- (81.7000,155.3000) -- (81.7000,155.3000) -- (81.7000,155.3000) -- (81.7000,155.3000) -- (81.7000,155.3000) -- (81.7000,155.3000) -- (81.7000,155.3000) -- (81.7000,155.3000) -- (81.7000,155.3000) -- (81.7000,155.3000) -- (81.8000,155.3000) -- (81.8000,155.3000) -- (81.8000,155.3000) -- (81.8000,155.3000) -- (81.8000,155.2000) -- (81.8000,155.2000) -- (81.8000,155.2000) -- (81.8000,155.2000) -- (81.8000,155.2000) -- (81.8000,155.2000) -- (81.8000,155.2000) -- (81.8000,155.2000) -- (81.8000,155.2000) -- (81.8000,155.2000) -- (81.8000,155.2000) -- (81.8000,155.2000) -- (81.8000,155.2000) -- (81.8000,155.2000) -- (81.8000,155.2000) -- (81.8000,155.2000) -- (81.8000,155.2000) -- (81.8000,155.2000) -- (81.8000,155.2000) -- (81.8000,155.2000) -- (81.8000,155.2000) -- (81.8000,155.1000) -- (81.8000,155.1000) -- (81.8000,155.1000) -- (81.8000,155.1000) -- (81.8000,155.1000) -- (81.8000,155.1000) -- (81.8000,155.1000) -- (81.8000,155.1000) -- (81.8000,155.1000) -- (81.8000,155.1000) -- (81.8000,155.1000) -- (81.8000,155.1000) -- (81.8000,155.1000) -- (81.8000,155.1000) -- (81.8000,155.1000) -- (81.8000,155.1000) -- (81.8000,155.1000) -- (81.8000,155.1000) -- (81.8000,155.1000) -- (81.8000,155.1000) -- (81.8000,155.1000) -- (81.8000,155.1000) -- (81.8000,155.0000) -- (81.8000,155.0000) -- (81.9000,155.0000) -- (81.9000,155.0000) -- (81.9000,155.0000) -- (81.9000,155.0000) -- (81.9000,155.0000) -- (81.9000,155.0000) -- (81.9000,155.0000) -- (81.9000,155.0000) -- (81.9000,155.0000) -- (81.9000,155.0000) -- (81.9000,155.0000) -- (81.9000,155.0000) -- (81.9000,155.0000) -- (81.9000,155.0000) -- (81.9000,155.0000) -- (81.9000,155.0000) -- (81.9000,155.0000) -- (81.9000,155.0000) -- (81.9000,155.0000) -- (81.9000,154.9000) -- (81.9000,154.9000) -- (81.9000,154.9000) -- (81.9000,154.9000) -- (81.9000,154.9000) -- (81.9000,154.9000) -- (81.9000,154.9000) -- (81.9000,154.9000) -- (81.9000,154.9000) -- (81.9000,154.9000) -- (81.9000,154.9000) -- (81.9000,154.9000) -- (81.9000,154.9000) -- (81.9000,154.9000) -- (81.9000,154.9000) -- (81.9000,154.9000) -- (81.9000,154.9000) -- (81.9000,154.9000) -- (81.9000,154.9000) -- (81.9000,154.9000) -- (81.9000,154.9000) -- (81.9000,154.9000) -- (81.9000,154.8000) -- (81.9000,154.8000) -- (81.9000,154.8000) -- (81.9000,154.8000) -- (81.9000,154.8000) -- (81.9000,154.8000) -- (81.9000,154.8000) -- (81.9000,154.8000) -- (81.9000,154.8000) -- (82.0000,154.8000) -- (82.0000,154.8000) -- (82.0000,154.8000) -- (82.0000,154.8000) -- (82.0000,154.8000) -- (82.0000,154.8000) -- (82.0000,154.8000) -- (82.0000,154.8000) -- (82.0000,154.8000) -- (82.0000,154.8000) -- (82.0000,154.8000) -- (82.0000,154.8000) -- (82.0000,154.7000) -- (82.0000,154.7000) -- (82.0000,154.7000) -- (82.0000,154.7000) -- (82.0000,154.7000) -- (82.0000,154.7000) -- (82.0000,154.7000) -- (82.0000,154.7000) -- (82.0000,154.7000) -- (82.0000,154.7000) -- (82.0000,154.7000) -- (82.0000,154.7000) -- (82.0000,154.7000) -- (82.0000,154.7000) -- (82.0000,154.7000) -- (82.0000,154.7000) -- (82.0000,154.7000) -- (82.0000,154.7000) -- (82.0000,154.7000) -- (82.0000,154.7000) -- (82.0000,154.7000) -- (82.0000,154.7000) -- (82.0000,154.6000) -- (82.0000,154.6000) -- (82.0000,154.6000) -- (82.0000,154.6000) -- (82.0000,154.6000) -- (82.0000,154.6000) -- (82.0000,154.6000) -- (82.0000,154.6000) -- (82.0000,154.6000) -- (82.0000,154.6000) -- (82.0000,154.6000) -- (82.0000,154.6000) -- (82.0000,154.6000) -- (82.0000,154.6000) -- (82.0000,154.6000) -- (82.1000,154.6000) -- (82.1000,154.6000) -- (82.1000,154.6000) -- (82.1000,154.6000) -- (82.1000,154.6000) -- (82.1000,154.6000) -- (82.1000,154.5000) -- (82.1000,154.5000) -- (82.1000,154.5000) -- (82.1000,154.5000) -- (82.1000,154.5000) -- (82.1000,154.5000) -- (82.1000,154.5000) -- (82.1000,154.5000) -- (82.1000,154.5000) -- (82.1000,154.5000) -- (82.1000,154.5000) -- (82.1000,154.5000) -- (82.1000,154.5000) -- (82.1000,154.5000) -- (82.1000,154.5000) -- (82.1000,154.5000) -- (82.1000,154.5000) -- (82.1000,154.5000) -- (82.1000,154.5000) -- (82.1000,154.5000) -- (82.1000,154.5000) -- (82.1000,154.5000) -- (82.1000,154.4000) -- (82.1000,154.4000) -- (82.1000,154.4000) -- (82.1000,154.4000) -- (82.1000,154.4000) -- (82.1000,154.4000) -- (82.1000,154.4000) -- (82.1000,154.4000) -- (82.1000,154.4000) -- (82.1000,154.4000) -- (82.1000,154.4000) -- (82.1000,154.4000) -- (82.1000,154.4000) -- (82.1000,154.4000) -- (82.1000,154.4000) -- (82.1000,154.4000) -- (82.1000,154.4000) -- (82.1000,154.4000) -- (82.1000,154.4000) -- (82.1000,154.4000) -- (82.1000,154.4000) -- (82.1000,154.3000) -- (82.2000,154.3000) -- (82.2000,154.3000) -- (82.2000,154.3000) -- (82.2000,154.3000) -- (82.2000,154.3000) -- (82.2000,154.3000) -- (82.2000,154.3000) -- (82.2000,154.3000) -- (82.2000,154.3000) -- (82.2000,154.3000) -- (82.2000,154.3000) -- (82.2000,154.3000) -- (82.2000,154.3000) -- (82.2000,154.3000) -- (82.2000,154.3000) -- (82.2000,154.3000) -- (82.2000,154.3000) -- (82.2000,154.3000) -- (82.2000,154.3000) -- (82.2000,154.3000) -- (82.2000,154.3000) -- (82.2000,154.2000) -- (82.2000,154.2000) -- (82.2000,154.2000) -- (82.2000,154.2000) -- (82.2000,154.2000) -- (82.2000,154.2000) -- (82.2000,154.2000) -- (82.2000,154.2000) -- (82.2000,154.2000) -- (82.2000,154.2000) -- (82.2000,154.2000) -- (82.2000,154.2000) -- (82.2000,154.2000) -- (82.2000,154.2000) -- (82.2000,154.2000) -- (82.2000,154.2000) -- (82.2000,154.2000) -- (82.2000,154.2000) -- (82.2000,154.2000) -- (82.2000,154.2000) -- (82.2000,154.2000) -- (82.2000,154.1000) -- (82.2000,154.1000) -- (82.2000,154.1000) -- (82.2000,154.1000) -- (82.2000,154.1000) -- (82.2000,154.1000) -- (82.2000,154.1000) -- (82.3000,154.1000) -- (82.3000,154.1000) -- (82.3000,154.1000) -- (82.3000,154.1000) -- (82.3000,154.1000) -- (82.3000,154.1000) -- (82.3000,154.1000) -- (82.3000,154.1000) -- (82.3000,154.1000) -- (82.3000,154.1000) -- (82.3000,154.1000) -- (82.3000,154.1000) -- (82.3000,154.1000) -- (82.3000,154.1000) -- (82.3000,154.1000) -- (82.3000,154.0000) -- (82.3000,154.0000) -- (82.3000,154.0000) -- (82.3000,154.0000) -- (82.3000,154.0000) -- (82.3000,154.0000) -- (82.3000,154.0000) -- (82.3000,154.0000) -- (82.3000,154.0000) -- (82.3000,154.0000) -- (82.3000,154.0000) -- (82.3000,154.0000) -- (82.3000,154.0000) -- (82.3000,154.0000) -- (82.3000,154.0000) -- (82.3000,154.0000) -- (82.3000,154.0000) -- (82.3000,154.0000) -- (82.3000,154.0000) -- (82.3000,154.0000) -- (82.3000,154.0000) -- (82.3000,153.9000) -- (82.3000,153.9000) -- (82.3000,153.9000) -- (82.3000,153.9000) -- (82.3000,153.9000) -- (82.3000,153.9000) -- (82.3000,153.9000) -- (82.3000,153.9000) -- (82.3000,153.9000) -- (82.3000,153.9000) -- (82.3000,153.9000) -- (82.3000,153.9000) -- (82.3000,153.9000) -- (82.3000,153.9000) -- (82.4000,153.9000) -- (82.4000,153.9000) -- (82.4000,153.9000) -- (82.4000,153.9000) -- (82.4000,153.9000) -- (82.4000,153.9000) -- (82.4000,153.9000) -- (82.4000,153.9000) -- (82.4000,153.8000) -- (82.4000,153.8000) -- (82.4000,153.8000) -- (82.4000,153.8000) -- (82.4000,153.8000) -- (82.4000,153.8000) -- (82.4000,153.8000) -- (82.4000,153.8000) -- (82.4000,153.8000) -- (82.4000,153.8000) -- (82.4000,153.8000) -- (82.4000,153.8000) -- (82.4000,153.8000) -- (82.4000,153.8000) -- (82.4000,153.8000) -- (82.4000,153.8000) -- (82.4000,153.8000) -- (82.4000,153.8000) -- (82.4000,153.8000) -- (82.4000,153.8000) -- (82.4000,153.8000) -- (82.4000,153.7000) -- (82.4000,153.7000) -- (82.4000,153.7000) -- (82.4000,153.7000) -- (82.4000,153.7000) -- (82.4000,153.7000) -- (82.4000,153.7000) -- (82.4000,153.7000) -- (82.4000,153.7000) -- (82.4000,153.7000) -- (82.4000,153.7000) -- (82.4000,153.7000) -- (82.4000,153.7000) -- (82.4000,153.7000) -- (82.4000,153.7000) -- (82.4000,153.7000) -- (82.4000,153.7000) -- (82.4000,153.7000) -- (82.4000,153.7000) -- (82.4000,153.7000) -- (82.5000,153.7000) -- (82.5000,153.7000) -- (82.5000,153.6000) -- (82.5000,153.6000) -- (82.5000,153.6000) -- (82.5000,153.6000) -- (82.5000,153.6000) -- (82.5000,153.6000) -- (82.5000,153.6000) -- (82.5000,153.6000) -- (82.5000,153.6000) -- (82.5000,153.6000) -- (82.5000,153.6000) -- (82.5000,153.6000) -- (82.5000,153.6000) -- (82.5000,153.6000) -- (82.5000,153.6000) -- (82.5000,153.6000) -- (82.5000,153.6000) -- (82.5000,153.6000) -- (82.5000,153.6000) -- (82.5000,153.6000) -- (82.5000,153.6000) -- (82.5000,153.5000) -- (82.5000,153.5000) -- (82.5000,153.5000) -- (82.5000,153.5000) -- (82.5000,153.5000) -- (82.5000,153.5000) -- (82.5000,153.5000) -- (82.5000,153.5000) -- (82.5000,153.5000) -- (82.5000,153.5000) -- (82.5000,153.5000) -- (82.5000,153.5000) -- (82.5000,153.5000) -- (82.5000,153.5000) -- (82.5000,153.5000) -- (82.5000,153.5000) -- (82.5000,153.5000) -- (82.5000,153.5000) -- (82.5000,153.5000) -- (82.5000,153.5000) -- (82.5000,153.5000) -- (82.5000,153.5000) -- (82.5000,153.4000) -- (82.5000,153.4000) -- (82.5000,153.4000) -- (82.5000,153.4000) -- (82.5000,153.4000) -- (82.6000,153.4000) -- (82.6000,153.4000) -- (82.6000,153.4000) -- (82.6000,153.4000) -- (82.6000,153.4000) -- (82.6000,153.4000) -- (82.6000,153.4000) -- (82.6000,153.4000) -- (82.6000,153.4000) -- (82.6000,153.4000) -- (82.6000,153.4000) -- (82.6000,153.4000) -- (82.6000,153.4000) -- (82.6000,153.4000) -- (82.6000,153.4000) -- (82.6000,153.4000) -- (82.6000,153.3000) -- (82.6000,153.3000) -- (82.6000,153.3000) -- (82.6000,153.3000) -- (82.6000,153.3000) -- (82.6000,153.3000) -- (82.6000,153.3000) -- (82.6000,153.3000) -- (82.6000,153.3000) -- (82.6000,153.3000) -- (82.6000,153.3000) -- (82.6000,153.3000) -- (82.6000,153.3000) -- (82.6000,153.3000) -- (82.6000,153.3000) -- (82.6000,153.3000) -- (82.6000,153.3000) -- (82.6000,153.3000) -- (82.6000,153.3000) -- (82.6000,153.3000) -- (82.6000,153.3000) -- (82.6000,153.3000) -- (82.6000,153.2000) -- (82.6000,153.2000) -- (82.6000,153.2000) -- (82.6000,153.2000) -- (82.6000,153.2000) -- (82.6000,153.2000) -- (82.6000,153.2000) -- (82.6000,153.2000) -- (82.6000,153.2000) -- (82.6000,153.2000) -- (82.6000,153.2000) -- (82.7000,153.2000) -- (82.7000,153.2000) -- (82.7000,153.2000) -- (82.7000,153.2000) -- (82.7000,153.2000) -- (82.7000,153.2000) -- (82.7000,153.2000) -- (82.7000,153.2000) -- (82.7000,153.2000) -- (82.7000,153.2000) -- (82.7000,153.1000) -- (82.7000,153.1000) -- (82.7000,153.1000) -- (82.7000,153.1000) -- (82.7000,153.1000) -- (82.7000,153.1000) -- (82.7000,153.1000) -- (82.7000,153.1000) -- (82.7000,153.1000) -- (82.7000,153.1000) -- (82.7000,153.1000) -- (82.7000,153.1000) -- (82.7000,153.1000) -- (82.7000,153.1000) -- (82.7000,153.1000) -- (82.7000,153.1000) -- (82.7000,153.1000) -- (82.7000,153.1000) -- (82.7000,153.1000) -- (82.7000,153.1000) -- (82.7000,153.1000) -- (82.7000,153.1000) -- (82.7000,153.0000) -- (82.7000,153.0000) -- (82.7000,153.0000) -- (82.7000,153.0000) -- (82.7000,153.0000) -- (82.7000,153.0000) -- (82.7000,153.0000) -- (82.7000,153.0000) -- (82.7000,153.0000) -- (82.7000,153.0000) -- (82.7000,153.0000) -- (82.7000,153.0000) -- (82.7000,153.0000) -- (82.7000,153.0000) -- (82.7000,153.0000) -- (82.7000,153.0000) -- (82.7000,153.0000) -- (82.7000,153.0000) -- (82.8000,153.0000) -- (82.8000,153.0000) -- (82.8000,153.0000) -- (82.8000,152.9000) -- (82.8000,152.9000) -- (82.8000,152.9000) -- (82.8000,152.9000) -- (82.8000,152.9000) -- (82.8000,152.9000) -- (82.8000,152.9000) -- (82.8000,152.9000) -- (82.8000,152.9000) -- (82.8000,152.9000) -- (82.8000,152.9000) -- (82.8000,152.9000) -- (82.8000,152.9000) -- (82.8000,152.9000) -- (82.8000,152.9000) -- (82.8000,152.9000) -- (82.8000,152.9000) -- (82.8000,152.9000) -- (82.8000,152.9000) -- (82.8000,152.9000) -- (82.8000,152.9000) -- (82.8000,152.9000) -- (82.8000,152.8000) -- (82.8000,152.8000) -- (82.8000,152.8000) -- (82.8000,152.8000) -- (82.8000,152.8000) -- (82.8000,152.8000) -- (82.8000,152.8000) -- (82.8000,152.8000) -- (82.8000,152.8000) -- (82.8000,152.8000) -- (82.8000,152.8000) -- (82.8000,152.8000) -- (82.8000,152.8000) -- (82.8000,152.8000) -- (82.8000,152.8000) -- (82.8000,152.8000) -- (82.8000,152.8000) -- (82.8000,152.8000) -- (82.8000,152.8000) -- (82.8000,152.8000) -- (82.8000,152.8000) -- (82.8000,152.7000) -- (82.8000,152.7000) -- (82.8000,152.7000) -- (82.9000,152.7000) -- (82.9000,152.7000) -- (82.9000,152.7000) -- (82.9000,152.7000) -- (82.9000,152.7000) -- (82.9000,152.7000) -- (82.9000,152.7000) -- (82.9000,152.7000) -- (82.9000,152.7000) -- (82.9000,152.7000) -- (82.9000,152.7000) -- (82.9000,152.7000) -- (82.9000,152.7000) -- (82.9000,152.7000) -- (82.9000,152.7000) -- (82.9000,152.7000) -- (82.9000,152.7000) -- (82.9000,152.7000) -- (82.9000,152.7000) -- (82.9000,152.6000) -- (82.9000,152.6000) -- (82.9000,152.6000) -- (82.9000,152.6000) -- (82.9000,152.6000) -- (82.9000,152.6000) -- (82.9000,152.6000) -- (82.9000,152.6000) -- (82.9000,152.6000) -- (82.9000,152.6000) -- (82.9000,152.6000) -- (82.9000,152.6000) -- (82.9000,152.6000) -- (82.9000,152.6000) -- (82.9000,152.6000) -- (82.9000,152.6000) -- (82.9000,152.6000) -- (82.9000,152.6000) -- (82.9000,152.6000) -- (82.9000,152.6000) -- (82.9000,152.6000) -- (82.9000,152.5000) -- (82.9000,152.5000) -- (82.9000,152.5000) -- (82.9000,152.5000) -- (82.9000,152.5000) -- (82.9000,152.5000) -- (82.9000,152.5000) -- (82.9000,152.5000) -- (82.9000,152.5000) -- (82.9000,152.5000) -- (83.0000,152.5000) -- (83.0000,152.5000) -- (83.0000,152.5000) -- (83.0000,152.5000) -- (83.0000,152.5000) -- (83.0000,152.5000) -- (83.0000,152.5000) -- (83.0000,152.5000) -- (83.0000,152.5000) -- (83.0000,152.5000) -- (83.0000,152.5000) -- (83.0000,152.5000) -- (83.0000,152.4000) -- (83.0000,152.4000) -- (83.0000,152.4000) -- (83.0000,152.4000) -- (83.0000,152.4000) -- (83.0000,152.4000) -- (83.0000,152.4000) -- (83.0000,152.4000) -- (83.0000,152.4000) -- (83.0000,152.4000) -- (83.0000,152.4000) -- (83.0000,152.4000) -- (83.0000,152.4000) -- (83.0000,152.4000) -- (83.0000,152.4000) -- (83.0000,152.4000) -- (83.0000,152.4000) -- (83.0000,152.4000) -- (83.0000,152.4000) -- (83.0000,152.4000) -- (83.0000,152.4000) -- (83.0000,152.3000) -- (83.0000,152.3000) -- (83.0000,152.3000) -- (83.0000,152.3000) -- (83.0000,152.3000) -- (83.0000,152.3000) -- (83.0000,152.3000) -- (83.0000,152.3000) -- (83.0000,152.3000) -- (83.0000,152.3000) -- (83.0000,152.3000) -- (83.0000,152.3000) -- (83.0000,152.3000) -- (83.0000,152.3000) -- (83.0000,152.3000) -- (83.0000,152.3000) -- (83.1000,152.3000) -- (83.1000,152.3000) -- (83.1000,152.3000) -- (83.1000,152.3000) -- (83.1000,152.3000) -- (83.1000,152.3000) -- (83.1000,152.2000) -- (83.1000,152.2000) -- (83.1000,152.2000) -- (83.1000,152.2000) -- (83.1000,152.2000) -- (83.1000,152.2000) -- (83.1000,152.2000) -- (83.1000,152.2000) -- (83.1000,152.2000) -- (83.1000,152.2000) -- (83.1000,152.2000) -- (83.1000,152.2000) -- (83.1000,152.2000) -- (83.1000,152.2000) -- (83.1000,152.2000) -- (83.1000,152.2000) -- (83.1000,152.2000) -- (83.1000,152.2000) -- (83.1000,152.2000) -- (83.1000,152.2000) -- (83.1000,152.2000) -- (83.1000,152.2000) -- (83.1000,152.1000) -- (83.1000,152.1000) -- (83.1000,152.1000) -- (83.1000,152.1000) -- (83.1000,152.1000) -- (83.1000,152.1000) -- (83.1000,152.1000) -- (83.1000,152.1000) -- (83.1000,152.1000) -- (83.1000,152.1000) -- (83.1000,152.1000) -- (83.1000,152.1000) -- (83.1000,152.1000) -- (83.1000,152.1000) -- (83.1000,152.1000) -- (83.1000,152.1000) -- (83.1000,152.1000) -- (83.1000,152.1000) -- (83.1000,152.1000) -- (83.1000,152.1000) -- (83.1000,152.1000) -- (83.1000,152.0000) -- (83.2000,152.0000) -- (83.2000,152.0000) -- (83.2000,152.0000) -- (83.2000,152.0000) -- (83.2000,152.0000) -- (83.2000,152.0000) -- (83.2000,152.0000) -- (83.2000,152.0000) -- (83.2000,152.0000) -- (83.2000,152.0000) -- (83.2000,152.0000) -- (83.2000,152.0000) -- (83.2000,152.0000) -- (83.2000,152.0000) -- (83.2000,152.0000) -- (83.2000,152.0000) -- (83.2000,152.0000) -- (83.2000,152.0000) -- (83.2000,152.0000) -- (83.2000,152.0000) -- (83.2000,152.0000) -- (83.2000,151.9000) -- (83.2000,151.9000) -- (83.2000,151.9000) -- (83.2000,151.9000) -- (83.2000,151.9000) -- (83.2000,151.9000) -- (83.2000,151.9000) -- (83.2000,151.9000) -- (83.2000,151.9000) -- (83.2000,151.9000) -- (83.2000,151.9000) -- (83.2000,151.9000) -- (83.2000,151.9000) -- (83.2000,151.9000) -- (83.2000,151.9000) -- (83.2000,151.9000) -- (83.2000,151.9000) -- (83.2000,151.9000) -- (83.2000,151.9000) -- (83.2000,151.9000) -- (83.2000,151.9000) -- (83.2000,151.8000) -- (83.2000,151.8000) -- (83.2000,151.8000) -- (83.2000,151.8000) -- (83.2000,151.8000) -- (83.2000,151.8000) -- (83.2000,151.8000) -- (83.3000,151.8000) -- (83.3000,151.8000) -- (83.3000,151.8000) -- (83.3000,151.8000) -- (83.3000,151.8000) -- (83.3000,151.8000) -- (83.3000,151.8000) -- (83.3000,151.8000) -- (83.3000,151.8000) -- (83.3000,151.8000) -- (83.3000,151.8000) -- (83.3000,151.8000) -- (83.3000,151.8000) -- (83.3000,151.8000) -- (83.3000,151.8000) -- (83.3000,151.7000) -- (83.3000,151.7000) -- (83.3000,151.7000) -- (83.3000,151.7000) -- (83.3000,151.7000) -- (83.3000,151.7000) -- (83.3000,151.7000) -- (83.3000,151.7000) -- (83.3000,151.7000) -- (83.3000,151.7000) -- (83.3000,151.7000) -- (83.3000,151.7000) -- (83.3000,151.7000) -- (83.3000,151.7000) -- (83.3000,151.7000) -- (83.3000,151.7000) -- (83.3000,151.7000) -- (83.3000,151.7000) -- (83.3000,151.7000) -- (83.3000,151.7000) -- (83.3000,151.7000) -- (83.3000,151.6000) -- (83.3000,151.6000) -- (83.3000,151.6000) -- (83.3000,151.6000) -- (83.3000,151.6000) -- (83.3000,151.6000) -- (83.3000,151.6000) -- (83.3000,151.6000) -- (83.3000,151.6000) -- (83.3000,151.6000) -- (83.3000,151.6000) -- (83.3000,151.6000) -- (83.3000,151.6000) -- (83.3000,151.6000) -- (83.4000,151.6000) -- (83.4000,151.6000) -- (83.4000,151.6000) -- (83.4000,151.6000) -- (83.4000,151.6000) -- (83.4000,151.6000) -- (83.4000,151.6000) -- (83.4000,151.6000) -- (83.4000,151.5000) -- (83.4000,151.5000) -- (83.4000,151.5000) -- (83.4000,151.5000) -- (83.4000,151.5000) -- (83.4000,151.5000) -- (83.4000,151.5000) -- (83.4000,151.5000) -- (83.4000,151.5000) -- (83.4000,151.5000) -- (83.4000,151.5000) -- (83.4000,151.5000) -- (83.4000,151.5000) -- (83.4000,151.5000) -- (83.4000,151.5000) -- (83.4000,151.5000) -- (83.4000,151.5000) -- (83.4000,151.5000) -- (83.4000,151.5000) -- (83.4000,151.5000) -- (83.4000,151.5000) -- (83.4000,151.4000) -- (83.4000,151.4000) -- (83.4000,151.4000) -- (83.4000,151.4000) -- (83.4000,151.4000) -- (83.4000,151.4000) -- (83.4000,151.4000) -- (83.4000,151.4000) -- (83.4000,151.4000) -- (83.4000,151.4000) -- (83.4000,151.4000) -- (83.4000,151.4000) -- (83.4000,151.4000) -- (83.4000,151.4000) -- (83.4000,151.4000) -- (83.4000,151.4000) -- (83.4000,151.4000) -- (83.4000,151.4000) -- (83.4000,151.4000) -- (83.4000,151.4000) -- (83.5000,151.4000) -- (83.5000,151.4000) -- (83.5000,151.3000) -- (83.5000,151.3000) -- (83.5000,151.3000) -- (83.5000,151.3000) -- (83.5000,151.3000) -- (83.5000,151.3000) -- (83.5000,151.3000) -- (83.5000,151.3000) -- (83.5000,151.3000) -- (83.5000,151.3000) -- (83.5000,151.3000) -- (83.5000,151.3000) -- (83.5000,151.3000) -- (83.5000,151.3000) -- (83.5000,151.3000) -- (83.5000,151.3000) -- (83.5000,151.3000) -- (83.5000,151.3000) -- (83.5000,151.3000) -- (83.5000,151.3000) -- (83.5000,151.3000) -- (83.5000,151.2000) -- (83.5000,151.2000) -- (83.5000,151.2000) -- (83.5000,151.2000) -- (83.5000,151.2000) -- (83.5000,151.2000) -- (83.5000,151.2000) -- (83.5000,151.2000) -- (83.5000,151.2000) -- (83.5000,151.2000) -- (83.5000,151.2000) -- (83.5000,151.2000) -- (83.5000,151.2000) -- (83.5000,151.2000) -- (83.5000,151.2000) -- (83.5000,151.2000) -- (83.5000,151.2000) -- (83.5000,151.2000) -- (83.5000,151.2000) -- (83.5000,151.2000) -- (83.5000,151.2000) -- (83.5000,151.2000) -- (83.5000,151.1000) -- (83.5000,151.1000) -- (83.5000,151.1000) -- (83.5000,151.1000) -- (83.5000,151.1000) -- (83.6000,151.1000) -- (83.6000,151.1000) -- (83.6000,151.1000) -- (83.6000,151.1000) -- (83.6000,151.1000) -- (83.6000,151.1000) -- (83.6000,151.1000) -- (83.6000,151.1000) -- (83.6000,151.1000) -- (83.6000,151.1000) -- (83.6000,151.1000) -- (83.6000,151.1000) -- (83.6000,151.1000) -- (83.6000,151.1000) -- (83.6000,151.1000) -- (83.6000,151.1000) -- (83.6000,151.0000) -- (83.6000,151.0000) -- (83.6000,151.0000) -- (83.6000,151.0000) -- (83.6000,151.0000) -- (83.6000,151.0000) -- (83.6000,151.0000) -- (83.6000,151.0000) -- (83.6000,151.0000) -- (83.6000,151.0000) -- (83.6000,151.0000) -- (83.6000,151.0000) -- (83.6000,151.0000) -- (83.6000,151.0000) -- (83.6000,151.0000) -- (83.6000,151.0000) -- (83.6000,151.0000) -- (83.6000,151.0000) -- (83.6000,151.0000) -- (83.6000,151.0000) -- (83.6000,151.0000) -- (83.6000,151.0000) -- (83.6000,150.9000) -- (83.6000,150.9000) -- (83.6000,150.9000) -- (83.6000,150.9000) -- (83.6000,150.9000) -- (83.6000,150.9000) -- (83.6000,150.9000) -- (83.6000,150.9000) -- (83.6000,150.9000) -- (83.6000,150.9000) -- (83.6000,150.9000) -- (83.7000,150.9000) -- (83.7000,150.9000) -- (83.7000,150.9000) -- (83.7000,150.9000) -- (83.7000,150.9000) -- (83.7000,150.9000) -- (83.7000,150.9000) -- (83.7000,150.9000) -- (83.7000,150.9000) -- (83.7000,150.9000) -- (83.7000,150.8000) -- (83.7000,150.8000) -- (83.7000,150.8000) -- (83.7000,150.8000) -- (83.7000,150.8000) -- (83.7000,150.8000) -- (83.7000,150.8000) -- (83.7000,150.8000) -- (83.7000,150.8000) -- (83.7000,150.8000) -- (83.7000,150.8000) -- (83.7000,150.8000) -- (83.7000,150.8000) -- (83.7000,150.8000) -- (83.7000,150.8000) -- (83.7000,150.8000) -- (83.7000,150.8000) -- (83.7000,150.8000) -- (83.7000,150.8000) -- (83.7000,150.8000) -- (83.7000,150.8000) -- (83.7000,150.8000) -- (83.7000,150.7000) -- (83.7000,150.7000) -- (83.7000,150.7000) -- (83.7000,150.7000) -- (83.7000,150.7000) -- (83.7000,150.7000) -- (83.7000,150.7000) -- (83.7000,150.7000) -- (83.7000,150.7000) -- (83.7000,150.7000) -- (83.7000,150.7000) -- (83.7000,150.7000) -- (83.7000,150.7000) -- (83.7000,150.7000) -- (83.7000,150.7000) -- (83.7000,150.7000) -- (83.7000,150.7000) -- (83.7000,150.7000) -- (83.8000,150.7000) -- (83.8000,150.7000) -- (83.8000,150.7000) -- (83.8000,150.6000) -- (83.8000,150.6000) -- (83.8000,150.6000) -- (83.8000,150.6000) -- (83.8000,150.6000) -- (83.8000,150.6000) -- (83.8000,150.6000) -- (83.8000,150.6000) -- (83.8000,150.6000) -- (83.8000,150.6000) -- (83.8000,150.6000) -- (83.8000,150.6000) -- (83.8000,150.6000) -- (83.8000,150.6000) -- (83.8000,150.6000) -- (83.8000,150.6000) -- (83.8000,150.6000) -- (83.8000,150.6000) -- (83.8000,150.6000) -- (83.8000,150.6000) -- (83.8000,150.6000) -- (83.8000,150.6000) -- (83.8000,150.5000) -- (83.8000,150.5000) -- (83.8000,150.5000) -- (83.8000,150.5000) -- (83.8000,150.5000) -- (83.8000,150.5000) -- (83.8000,150.5000) -- (83.8000,150.5000) -- (83.8000,150.5000) -- (83.8000,150.5000) -- (83.8000,150.5000) -- (83.8000,150.5000) -- (83.8000,150.5000) -- (83.8000,150.5000) -- (83.8000,150.5000) -- (83.8000,150.5000) -- (83.8000,150.5000) -- (83.8000,150.5000) -- (83.8000,150.5000) -- (83.8000,150.5000) -- (83.8000,150.5000) -- (83.8000,150.4000) -- (83.8000,150.4000) -- (83.8000,150.4000) -- (83.9000,150.4000) -- (83.9000,150.4000) -- (83.9000,150.4000) -- (83.9000,150.4000) -- (83.9000,150.4000) -- (83.9000,150.4000) -- (83.9000,150.4000) -- (83.9000,150.4000) -- (83.9000,150.4000) -- (83.9000,150.4000) -- (83.9000,150.4000) -- (83.9000,150.4000) -- (83.9000,150.4000) -- (83.9000,150.4000) -- (83.9000,150.4000) -- (83.9000,150.4000) -- (83.9000,150.4000) -- (83.9000,150.4000) -- (83.9000,150.4000) -- (83.9000,150.3000) -- (83.9000,150.3000) -- (83.9000,150.3000) -- (83.9000,150.3000) -- (83.9000,150.3000) -- (83.9000,150.3000) -- (83.9000,150.3000) -- (83.9000,150.3000) -- (83.9000,150.3000) -- (83.9000,150.3000) -- (83.9000,150.3000) -- (83.9000,150.3000) -- (83.9000,150.3000) -- (83.9000,150.3000) -- (83.9000,150.3000) -- (83.9000,150.3000) -- (83.9000,150.3000) -- (83.9000,150.3000) -- (83.9000,150.3000) -- (83.9000,150.3000) -- (83.9000,150.3000) -- (83.9000,150.2000) -- (83.9000,150.2000) -- (83.9000,150.2000) -- (83.9000,150.2000) -- (83.9000,150.2000) -- (83.9000,150.2000) -- (83.9000,150.2000) -- (83.9000,150.2000) -- (83.9000,150.2000) -- (83.9000,150.2000) -- (84.0000,150.2000) -- (84.0000,150.2000) -- (84.0000,150.2000) -- (84.0000,150.2000) -- (84.0000,150.2000) -- (84.0000,150.2000) -- (84.0000,150.2000) -- (84.0000,150.2000) -- (84.0000,150.2000) -- (84.0000,150.2000) -- (84.0000,150.2000) -- (84.0000,150.2000) -- (84.0000,150.1000) -- (84.0000,150.1000) -- (84.0000,150.1000) -- (84.0000,150.1000) -- (84.0000,150.1000) -- (84.0000,150.1000) -- (84.0000,150.1000) -- (84.0000,150.1000) -- (84.0000,150.1000) -- (84.0000,150.1000) -- (84.0000,150.1000) -- (84.0000,150.1000) -- (84.0000,150.1000) -- (84.0000,150.1000) -- (84.0000,150.1000) -- (84.0000,150.1000) -- (84.0000,150.1000) -- (84.0000,150.1000) -- (84.0000,150.1000) -- (84.0000,150.1000) -- (84.0000,150.1000) -- (84.0000,150.0000) -- (84.0000,150.0000) -- (84.0000,150.0000) -- (84.0000,150.0000) -- (84.0000,150.0000) -- (84.0000,150.0000) -- (84.0000,150.0000) -- (84.0000,150.0000) -- (84.0000,150.0000) -- (84.0000,150.0000) -- (84.0000,150.0000) -- (84.0000,150.0000) -- (84.0000,150.0000) -- (84.0000,150.0000) -- (84.0000,150.0000) -- (84.0000,150.0000) -- (84.1000,150.0000) -- (84.1000,150.0000) -- (84.1000,150.0000) -- (84.1000,150.0000) -- (84.1000,150.0000) -- (84.1000,150.0000) -- (84.1000,149.9000) -- (84.1000,149.9000) -- (84.1000,149.9000) -- (84.1000,149.9000) -- (84.1000,149.9000) -- (84.1000,149.9000) -- (84.1000,149.9000) -- (84.1000,149.9000) -- (84.1000,149.9000) -- (84.1000,149.9000) -- (84.1000,149.9000) -- (84.1000,149.9000) -- (84.1000,149.9000) -- (84.1000,149.9000) -- (84.1000,149.9000) -- (84.1000,149.9000) -- (84.1000,149.9000) -- (84.1000,149.9000) -- (84.1000,149.9000) -- (84.1000,149.9000) -- (84.1000,149.9000) -- (84.1000,149.8000) -- (84.1000,149.8000) -- (84.1000,149.8000) -- (84.1000,149.8000) -- (84.1000,149.8000) -- (84.1000,149.8000) -- (84.1000,149.8000) -- (84.1000,149.8000) -- (84.1000,149.8000) -- (84.1000,149.8000) -- (84.1000,149.8000) -- (84.1000,149.8000) -- (84.1000,149.8000) -- (84.1000,149.8000) -- (84.1000,149.8000) -- (84.1000,149.8000) -- (84.1000,149.8000) -- (84.1000,149.8000) -- (84.1000,149.8000) -- (84.1000,149.8000) -- (84.1000,149.8000) -- (84.1000,149.8000) -- (84.1000,149.7000) -- (84.2000,149.7000) -- (84.2000,149.7000) -- (84.2000,149.7000) -- (84.2000,149.7000) -- (84.2000,149.7000) -- (84.2000,149.7000) -- (84.2000,149.7000) -- (84.2000,149.7000) -- (84.2000,149.7000) -- (84.2000,149.7000) -- (84.2000,149.7000) -- (84.2000,149.7000) -- (84.2000,149.7000) -- (84.2000,149.7000) -- (84.2000,149.7000) -- (84.2000,149.7000) -- (84.2000,149.7000) -- (84.2000,149.7000) -- (84.2000,149.7000) -- (84.2000,149.7000) -- (84.2000,149.6000) -- (84.2000,149.6000) -- (84.2000,149.6000) -- (84.2000,149.6000) -- (84.2000,149.6000) -- (84.2000,149.6000) -- (84.2000,149.6000) -- (84.2000,149.6000) -- (84.2000,149.6000) -- (84.2000,149.6000) -- (84.2000,149.6000) -- (84.2000,149.6000) -- (84.2000,149.6000) -- (84.2000,149.6000) -- (84.2000,149.6000) -- (84.2000,149.6000) -- (84.2000,149.6000) -- (84.2000,149.6000) -- (84.2000,149.6000) -- (84.2000,149.6000) -- (84.2000,149.6000) -- (84.2000,149.6000) -- (84.2000,149.5000) -- (84.2000,149.5000) -- (84.2000,149.5000) -- (84.2000,149.5000) -- (84.2000,149.5000) -- (84.2000,149.5000) -- (84.2000,149.5000) -- (84.3000,149.5000) -- (84.3000,149.5000) -- (84.3000,149.5000) -- (84.3000,149.5000) -- (84.3000,149.5000) -- (84.3000,149.5000) -- (84.3000,149.5000) -- (84.3000,149.5000) -- (84.3000,149.5000) -- (84.3000,149.5000) -- (84.3000,149.5000) -- (84.3000,149.5000) -- (84.3000,149.5000) -- (84.3000,149.5000) -- (84.3000,149.4000) -- (84.3000,149.4000) -- (84.3000,149.4000) -- (84.3000,149.4000) -- (84.3000,149.4000) -- (84.3000,149.4000) -- (84.3000,149.4000) -- (84.3000,149.4000) -- (84.3000,149.4000) -- (84.3000,149.4000) -- (84.3000,149.4000) -- (84.3000,149.4000) -- (84.3000,149.4000) -- (84.3000,149.4000) -- (84.3000,149.4000) -- (84.3000,149.4000) -- (84.3000,149.4000) -- (84.3000,149.4000) -- (84.3000,149.4000) -- (84.3000,149.4000) -- (84.3000,149.4000) -- (84.3000,149.4000) -- (84.3000,149.3000) -- (84.3000,149.3000) -- (84.3000,149.3000) -- (84.3000,149.3000) -- (84.3000,149.3000) -- (84.3000,149.3000) -- (84.3000,149.3000) -- (84.3000,149.3000) -- (84.3000,149.3000) -- (84.3000,149.3000) -- (84.3000,149.3000) -- (84.3000,149.3000) -- (84.3000,149.3000) -- (84.3000,149.3000) -- (84.4000,149.3000) -- (84.4000,149.3000) -- (84.4000,149.3000) -- (84.4000,149.3000) -- (84.4000,149.3000) -- (84.4000,149.3000) -- (84.4000,149.3000) -- (84.4000,149.2000) -- (84.4000,149.2000) -- (84.4000,149.2000) -- (84.4000,149.2000) -- (84.4000,149.2000) -- (84.4000,149.2000) -- (84.4000,149.2000) -- (84.4000,149.2000) -- (84.4000,149.2000) -- (84.4000,149.2000) -- (84.4000,149.2000) -- (84.4000,149.2000) -- (84.4000,149.2000) -- (84.4000,149.2000) -- (84.4000,149.2000) -- (84.4000,149.2000) -- (84.4000,149.2000) -- (84.4000,149.2000) -- (84.4000,149.2000) -- (84.4000,149.2000) -- (84.4000,149.2000) -- (84.4000,149.2000) -- (84.4000,149.1000) -- (84.4000,149.1000) -- (84.4000,149.1000) -- (84.4000,149.1000) -- (84.4000,149.1000) -- (84.4000,149.1000) -- (84.4000,149.1000) -- (84.4000,149.1000) -- (84.4000,149.1000) -- (84.4000,149.1000) -- (84.4000,149.1000) -- (84.4000,149.1000) -- (84.4000,149.1000) -- (84.4000,149.1000) -- (84.4000,149.1000) -- (84.4000,149.1000) -- (84.4000,149.1000) -- (84.4000,149.1000) -- (84.4000,149.1000) -- (84.4000,149.1000) -- (84.5000,149.1000) -- (84.5000,149.0000) -- (84.5000,149.0000) -- (84.5000,149.0000) -- (84.5000,149.0000) -- (84.5000,149.0000) -- (84.5000,149.0000) -- (84.5000,149.0000) -- (84.5000,149.0000) -- (84.5000,149.0000) -- (84.5000,149.0000) -- (84.5000,149.0000) -- (84.5000,149.0000) -- (84.5000,149.0000) -- (84.5000,149.0000) -- (84.5000,149.0000) -- (84.5000,149.0000) -- (84.5000,149.0000) -- (84.5000,149.0000) -- (84.5000,149.0000) -- (84.5000,149.0000) -- (84.5000,149.0000) -- (84.5000,149.0000) -- (84.5000,148.9000) -- (84.5000,148.9000) -- (84.5000,148.9000) -- (84.5000,148.9000) -- (84.5000,148.9000) -- (84.5000,148.9000) -- (84.5000,148.9000) -- (84.5000,148.9000) -- (84.5000,148.9000) -- (84.5000,148.9000) -- (84.5000,148.9000) -- (84.5000,148.9000) -- (84.5000,148.9000) -- (84.5000,148.9000) -- (84.5000,148.9000) -- (84.5000,148.9000) -- (84.5000,148.9000) -- (84.5000,148.9000) -- (84.5000,148.9000) -- (84.5000,148.9000) -- (84.5000,148.9000) -- (84.5000,148.8000) -- (84.5000,148.8000) -- (84.5000,148.8000) -- (84.5000,148.8000) -- (84.5000,148.8000) -- (84.5000,148.8000) -- (84.6000,148.8000) -- (84.6000,148.8000) -- (84.6000,148.8000) -- (84.6000,148.8000) -- (84.6000,148.8000) -- (84.6000,148.8000) -- (84.6000,148.8000) -- (84.6000,148.8000) -- (84.6000,148.8000) -- (84.6000,148.8000) -- (84.6000,148.8000) -- (84.6000,148.8000) -- (84.6000,148.8000) -- (84.6000,148.8000) -- (84.6000,148.8000) -- (84.6000,148.8000) -- (84.6000,148.7000) -- (84.6000,148.7000) -- (84.6000,148.7000) -- (84.6000,148.7000) -- (84.6000,148.7000) -- (84.6000,148.7000) -- (84.6000,148.7000) -- (84.6000,148.7000) -- (84.6000,148.7000) -- (84.6000,148.7000) -- (84.6000,148.7000) -- (84.6000,148.7000) -- (84.6000,148.7000) -- (84.6000,148.7000) -- (84.6000,148.7000) -- (84.6000,148.7000) -- (84.6000,148.7000) -- (84.6000,148.7000) -- (84.6000,148.7000) -- (84.6000,148.7000) -- (84.6000,148.7000) -- (84.6000,148.6000) -- (84.6000,148.6000) -- (84.6000,148.6000) -- (84.6000,148.6000) -- (84.6000,148.6000) -- (84.6000,148.6000) -- (84.6000,148.6000) -- (84.6000,148.6000) -- (84.6000,148.6000) -- (84.6000,148.6000) -- (84.6000,148.6000) -- (84.6000,148.6000) -- (84.7000,148.6000) -- (84.7000,148.6000) -- (84.7000,148.6000) -- (84.7000,148.6000) -- (84.7000,148.6000) -- (84.7000,148.6000) -- (84.7000,148.6000) -- (84.7000,148.6000) -- (84.7000,148.6000) -- (84.7000,148.6000) -- (84.7000,148.5000) -- (84.7000,148.5000) -- (84.7000,148.5000) -- (84.7000,148.5000) -- (84.7000,148.5000) -- (84.7000,148.5000) -- (84.7000,148.5000) -- (84.7000,148.5000) -- (84.7000,148.5000) -- (84.7000,148.5000) -- (84.7000,148.5000) -- (84.7000,148.5000) -- (84.7000,148.5000) -- (84.7000,148.5000) -- (84.7000,148.5000) -- (84.7000,148.5000) -- (84.7000,148.5000) -- (84.7000,148.5000) -- (84.7000,148.5000) -- (84.7000,148.5000) -- (84.7000,148.5000) -- (84.7000,148.4000) -- (84.7000,148.4000) -- (84.7000,148.4000) -- (84.7000,148.4000) -- (84.7000,148.4000) -- (84.7000,148.4000) -- (84.7000,148.4000) -- (84.7000,148.4000) -- (84.7000,148.4000) -- (84.7000,148.4000) -- (84.7000,148.4000) -- (84.7000,148.4000) -- (84.7000,148.4000) -- (84.7000,148.4000) -- (84.7000,148.4000) -- (84.7000,148.4000) -- (84.7000,148.4000) -- (84.7000,148.4000) -- (84.7000,148.4000) -- (84.8000,148.4000) -- (84.8000,148.4000) -- (84.8000,148.4000) -- (84.8000,148.3000) -- (84.8000,148.3000) -- (84.8000,148.3000) -- (84.8000,148.3000) -- (84.8000,148.3000) -- (84.8000,148.3000) -- (84.8000,148.3000) -- (84.8000,148.3000) -- (84.8000,148.3000) -- (84.8000,148.3000) -- (84.8000,148.3000) -- (84.8000,148.3000) -- (84.8000,148.3000) -- (84.8000,148.3000) -- (84.8000,148.3000) -- (84.8000,148.3000) -- (84.8000,148.3000) -- (84.8000,148.3000) -- (84.8000,148.3000) -- (84.8000,148.3000) -- (84.8000,148.3000) -- (84.8000,148.2000) -- (84.8000,148.2000) -- (84.8000,148.2000) -- (84.8000,148.2000) -- (84.8000,148.2000) -- (84.8000,148.2000) -- (84.8000,148.2000) -- (84.8000,148.2000) -- (84.8000,148.2000) -- (84.8000,148.2000) -- (84.8000,148.2000) -- (84.8000,148.2000) -- (84.8000,148.2000) -- (84.8000,148.2000) -- (84.8000,148.2000) -- (84.8000,148.2000) -- (84.8000,148.2000) -- (84.8000,148.2000) -- (84.8000,148.2000) -- (84.8000,148.2000) -- (84.8000,148.2000) -- (84.8000,148.2000) -- (84.8000,148.1000) -- (84.8000,148.1000) -- (84.8000,148.1000) -- (84.9000,148.1000) -- (84.9000,148.1000) -- (84.9000,148.1000) -- (84.9000,148.1000) -- (84.9000,148.1000) -- (84.9000,148.1000) -- (84.9000,148.1000) -- (84.9000,148.1000) -- (84.9000,148.1000) -- (84.9000,148.1000) -- (84.9000,148.1000) -- (84.9000,148.1000) -- (84.9000,148.1000) -- (84.9000,148.1000) -- (84.9000,148.1000) -- (84.9000,148.1000) -- (84.9000,148.1000) -- (84.9000,148.1000) -- (84.9000,148.0000) -- (84.9000,148.0000) -- (84.9000,148.0000) -- (84.9000,148.0000) -- (84.9000,148.0000) -- (84.9000,148.0000) -- (84.9000,148.0000) -- (84.9000,148.0000) -- (84.9000,148.0000) -- (84.9000,148.0000) -- (84.9000,148.0000) -- (84.9000,148.0000) -- (84.9000,148.0000) -- (84.9000,148.0000) -- (84.9000,148.0000) -- (84.9000,148.0000) -- (84.9000,148.0000) -- (84.9000,148.0000) -- (84.9000,148.0000) -- (84.9000,148.0000) -- (84.9000,148.0000) -- (84.9000,148.0000) -- (84.9000,147.9000) -- (84.9000,147.9000) -- (84.9000,147.9000) -- (84.9000,147.9000) -- (84.9000,147.9000) -- (84.9000,147.9000) -- (84.9000,147.9000) -- (84.9000,147.9000) -- (84.9000,147.9000) -- (84.9000,147.9000) -- (85.0000,147.9000) -- (85.0000,147.9000) -- (85.0000,147.9000) -- (85.0000,147.9000) -- (85.0000,147.9000) -- (85.0000,147.9000) -- (85.0000,147.9000) -- (85.0000,147.9000) -- (85.0000,147.9000) -- (85.0000,147.9000) -- (85.0000,147.9000) -- (85.0000,147.8000) -- (85.0000,147.8000) -- (85.0000,147.8000) -- (85.0000,147.8000) -- (85.0000,147.8000) -- (85.0000,147.8000) -- (85.0000,147.8000) -- (85.0000,147.8000) -- (85.0000,147.8000) -- (85.0000,147.8000) -- (85.0000,147.8000) -- (85.0000,147.8000) -- (85.0000,147.8000) -- (85.0000,147.8000) -- (85.0000,147.8000) -- (85.0000,147.8000) -- (85.0000,147.8000) -- (85.0000,147.8000) -- (85.0000,147.8000) -- (85.0000,147.8000) -- (85.0000,147.8000) -- (85.0000,147.8000) -- (85.0000,147.7000) -- (85.0000,147.7000) -- (85.0000,147.7000) -- (85.0000,147.7000) -- (85.0000,147.7000) -- (85.0000,147.7000) -- (85.0000,147.7000) -- (85.0000,147.7000) -- (85.0000,147.7000) -- (85.0000,147.7000) -- (85.0000,147.7000) -- (85.0000,147.7000) -- (85.0000,147.7000) -- (85.0000,147.7000) -- (85.0000,147.7000) -- (85.0000,147.7000) -- (85.1000,147.7000) -- (85.1000,147.7000) -- (85.1000,147.7000) -- (85.1000,147.7000) -- (85.1000,147.7000) -- (85.1000,147.6000) -- (85.1000,147.6000) -- (85.1000,147.6000) -- (85.1000,147.6000) -- (85.1000,147.6000) -- (85.1000,147.6000) -- (85.1000,147.6000) -- (85.1000,147.6000) -- (85.1000,147.6000) -- (85.1000,147.6000) -- (85.1000,147.6000) -- (85.1000,147.6000) -- (85.1000,147.6000) -- (85.1000,147.6000) -- (85.1000,147.6000) -- (85.1000,147.6000) -- (85.1000,147.6000) -- (85.1000,147.6000) -- (85.1000,147.6000) -- (85.1000,147.6000) -- (85.1000,147.6000) -- (85.1000,147.6000) -- (85.1000,147.5000) -- (85.1000,147.5000) -- (85.1000,147.5000) -- (85.1000,147.5000) -- (85.1000,147.5000) -- (85.1000,147.5000) -- (85.1000,147.5000) -- (85.1000,147.5000) -- (85.1000,147.5000) -- (85.1000,147.5000) -- (85.1000,147.5000) -- (85.1000,147.5000) -- (85.1000,147.5000) -- (85.1000,147.5000) -- (85.1000,147.5000) -- (85.1000,147.5000) -- (85.1000,147.5000) -- (85.1000,147.5000) -- (85.1000,147.5000) -- (85.1000,147.5000) -- (85.1000,147.5000) -- (85.1000,147.4000) -- (85.1000,147.4000) -- (85.2000,147.4000) -- (85.2000,147.4000) -- (85.2000,147.4000) -- (85.2000,147.4000) -- (85.2000,147.4000) -- (85.2000,147.4000) -- (85.2000,147.4000) -- (85.2000,147.4000) -- (85.2000,147.4000) -- (85.2000,147.4000) -- (85.2000,147.4000) -- (85.2000,147.4000) -- (85.2000,147.4000) -- (85.2000,147.4000) -- (85.2000,147.4000) -- (85.2000,147.4000) -- (85.2000,147.4000) -- (85.2000,147.4000) -- (85.2000,147.4000) -- (85.2000,147.4000) -- (85.2000,147.3000) -- (85.2000,147.3000) -- (85.2000,147.3000) -- (85.2000,147.3000) -- (85.2000,147.3000) -- (85.2000,147.3000) -- (85.2000,147.3000) -- (85.2000,147.3000) -- (85.2000,147.3000) -- (85.2000,147.3000) -- (85.2000,147.3000) -- (85.2000,147.3000) -- (85.2000,147.3000) -- (85.2000,147.3000) -- (85.2000,147.3000) -- (85.2000,147.3000) -- (85.2000,147.3000) -- (85.2000,147.3000) -- (85.2000,147.3000) -- (85.2000,147.3000) -- (85.2000,147.3000) -- (85.2000,147.2000) -- (85.2000,147.2000) -- (85.2000,147.2000) -- (85.2000,147.2000) -- (85.2000,147.2000) -- (85.2000,147.2000) -- (85.2000,147.2000) -- (85.2000,147.2000) -- (85.3000,147.2000) -- (85.3000,147.2000) -- (85.3000,147.2000) -- (85.3000,147.2000) -- (85.3000,147.2000) -- (85.3000,147.2000) -- (85.3000,147.2000) -- (85.3000,147.2000) -- (85.3000,147.2000) -- (85.3000,147.2000) -- (85.3000,147.2000) -- (85.3000,147.2000) -- (85.3000,147.2000) -- (85.3000,147.2000) -- (85.3000,147.1000) -- (85.3000,147.1000) -- (85.3000,147.1000) -- (85.3000,147.1000) -- (85.3000,147.1000) -- (85.3000,147.1000) -- (85.3000,147.1000) -- (85.3000,147.1000) -- (85.3000,147.1000) -- (85.3000,147.1000) -- (85.3000,147.1000) -- (85.3000,147.1000) -- (85.3000,147.1000) -- (85.3000,147.1000) -- (85.3000,147.1000) -- (85.3000,147.1000) -- (85.3000,147.1000) -- (85.3000,147.1000) -- (85.3000,147.1000) -- (85.3000,147.1000) -- (85.3000,147.1000) -- (85.3000,147.0000) -- (85.3000,147.0000) -- (85.3000,147.0000) -- (85.3000,147.0000) -- (85.3000,147.0000) -- (85.3000,147.0000) -- (85.3000,147.0000) -- (85.3000,147.0000) -- (85.3000,147.0000) -- (85.3000,147.0000) -- (85.3000,147.0000) -- (85.3000,147.0000) -- (85.3000,147.0000) -- (85.3000,147.0000) -- (85.3000,147.0000) -- (85.4000,147.0000) -- (85.4000,147.0000) -- (85.4000,147.0000) -- (85.4000,147.0000) -- (85.4000,147.0000) -- (85.4000,147.0000) -- (85.4000,147.0000) -- (85.4000,146.9000) -- (85.4000,146.9000) -- (85.4000,146.9000) -- (85.4000,146.9000) -- (85.4000,146.9000) -- (85.4000,146.9000) -- (85.4000,146.9000) -- (85.4000,146.9000) -- (85.4000,146.9000) -- (85.4000,146.9000) -- (85.4000,146.9000) -- (85.4000,146.9000) -- (85.4000,146.9000) -- (85.4000,146.9000) -- (85.4000,146.9000) -- (85.4000,146.9000) -- (85.4000,146.9000) -- (85.4000,146.9000) -- (85.4000,146.9000) -- (85.4000,146.9000) -- (85.4000,146.9000) -- (85.4000,146.8000) -- (85.4000,146.8000) -- (85.4000,146.8000) -- (85.4000,146.8000) -- (85.4000,146.8000) -- (85.4000,146.8000) -- (85.4000,146.8000) -- (85.4000,146.8000) -- (85.4000,146.8000) -- (85.4000,146.8000) -- (85.4000,146.8000) -- (85.4000,146.8000) -- (85.4000,146.8000) -- (85.4000,146.8000) -- (85.4000,146.8000) -- (85.4000,146.8000) -- (85.4000,146.8000) -- (85.4000,146.8000) -- (85.4000,146.8000) -- (85.4000,146.8000) -- (85.4000,146.8000) -- (85.5000,146.8000) -- (85.5000,146.7000) -- (85.5000,146.7000) -- (85.5000,146.7000) -- (85.5000,146.7000) -- (85.5000,146.7000) -- (85.5000,146.7000) -- (85.5000,146.7000) -- (85.5000,146.7000) -- (85.5000,146.7000) -- (85.5000,146.7000) -- (85.5000,146.7000) -- (85.5000,146.7000) -- (85.5000,146.7000) -- (85.5000,146.7000) -- (85.5000,146.7000) -- (85.5000,146.7000) -- (85.5000,146.7000) -- (85.5000,146.7000) -- (85.5000,146.7000) -- (85.5000,146.7000) -- (85.5000,146.7000) -- (85.5000,146.6000) -- (85.5000,146.6000) -- (85.5000,146.6000) -- (85.5000,146.6000) -- (85.5000,146.6000) -- (85.5000,146.6000) -- (85.5000,146.6000) -- (85.5000,146.6000) -- (85.5000,146.6000) -- (85.5000,146.6000) -- (85.5000,146.6000) -- (85.5000,146.6000) -- (85.5000,146.6000) -- (85.5000,146.6000) -- (85.5000,146.6000) -- (85.5000,146.6000) -- (85.5000,146.6000) -- (85.5000,146.6000) -- (85.5000,146.6000) -- (85.5000,146.6000) -- (85.5000,146.6000) -- (85.5000,146.6000) -- (85.5000,146.5000) -- (85.5000,146.5000) -- (85.5000,146.5000) -- (85.5000,146.5000) -- (85.5000,146.5000) -- (85.5000,146.5000) -- (85.6000,146.5000) -- (85.6000,146.5000) -- (85.6000,146.5000) -- (85.6000,146.5000) -- (85.6000,146.5000) -- (85.6000,146.5000) -- (85.6000,146.5000) -- (85.6000,146.5000) -- (85.6000,146.5000) -- (85.6000,146.5000) -- (85.6000,146.5000) -- (85.6000,146.5000) -- (85.6000,146.5000) -- (85.6000,146.5000) -- (85.6000,146.5000) -- (85.6000,146.4000) -- (85.6000,146.4000) -- (85.6000,146.4000) -- (85.6000,146.4000) -- (85.6000,146.4000) -- (85.6000,146.4000) -- (85.6000,146.4000) -- (85.6000,146.4000) -- (85.6000,146.4000) -- (85.6000,146.4000) -- (85.6000,146.4000) -- (85.6000,146.4000) -- (85.6000,146.4000) -- (85.6000,146.4000) -- (85.6000,146.4000) -- (85.6000,146.4000) -- (85.6000,146.4000) -- (85.6000,146.4000) -- (85.6000,146.4000) -- (85.6000,146.4000) -- (85.6000,146.4000) -- (85.6000,146.4000) -- (85.6000,146.3000) -- (85.6000,146.3000) -- (85.6000,146.3000) -- (85.6000,146.3000) -- (85.6000,146.3000) -- (85.6000,146.3000) -- (85.6000,146.3000) -- (85.6000,146.3000) -- (85.6000,146.3000) -- (85.6000,146.3000) -- (85.6000,146.3000) -- (85.6000,146.3000) -- (85.7000,146.3000) -- (85.7000,146.3000) -- (85.7000,146.3000) -- (85.7000,146.3000) -- (85.7000,146.3000) -- (85.7000,146.3000) -- (85.7000,146.3000) -- (85.7000,146.3000) -- (85.7000,146.3000) -- (85.7000,146.2000) -- (85.7000,146.2000) -- (85.7000,146.2000) -- (85.7000,146.2000) -- (85.7000,146.2000) -- (85.7000,146.2000) -- (85.7000,146.2000) -- (85.7000,146.2000) -- (85.7000,146.2000) -- (85.7000,146.2000) -- (85.7000,146.2000) -- (85.7000,146.2000) -- (85.7000,146.2000) -- (85.7000,146.2000) -- (85.7000,146.2000) -- (85.7000,146.2000) -- (85.7000,146.2000) -- (85.7000,146.2000) -- (85.7000,146.2000) -- (85.7000,146.2000) -- (85.7000,146.2000) -- (85.7000,146.2000) -- (85.7000,146.1000) -- (85.7000,146.1000) -- (85.7000,146.1000) -- (85.7000,146.1000) -- (85.7000,146.1000) -- (85.7000,146.1000) -- (85.7000,146.1000) -- (85.7000,146.1000) -- (85.7000,146.1000) -- (85.7000,146.1000) -- (85.7000,146.1000) -- (85.7000,146.1000) -- (85.7000,146.1000) -- (85.7000,146.1000) -- (85.7000,146.1000) -- (85.7000,146.1000) -- (85.7000,146.1000) -- (85.7000,146.1000) -- (85.7000,146.1000) -- (85.8000,146.1000) -- (85.8000,146.1000) -- (85.8000,146.0000) -- (85.8000,146.0000) -- (85.8000,146.0000) -- (85.8000,146.0000) -- (85.8000,146.0000) -- (85.8000,146.0000) -- (85.8000,146.0000) -- (85.8000,146.0000) -- (85.8000,146.0000) -- (85.8000,146.0000) -- (85.8000,146.0000) -- (85.8000,146.0000) -- (85.8000,146.0000) -- (85.8000,146.0000) -- (85.8000,146.0000) -- (85.8000,146.0000) -- (85.8000,146.0000) -- (85.8000,146.0000) -- (85.8000,146.0000) -- (85.8000,146.0000) -- (85.8000,146.0000) -- (85.8000,146.0000) -- (85.8000,145.9000) -- (85.8000,145.9000) -- (85.8000,145.9000) -- (85.8000,145.9000) -- (85.8000,145.9000) -- (85.8000,145.9000) -- (85.8000,145.9000) -- (85.8000,145.9000) -- (85.8000,145.9000) -- (85.8000,145.9000) -- (85.8000,145.9000) -- (85.8000,145.9000) -- (85.8000,145.9000) -- (85.8000,145.9000) -- (85.8000,145.9000) -- (85.8000,145.9000) -- (85.8000,145.9000) -- (85.8000,145.9000) -- (85.8000,145.9000) -- (85.8000,145.9000) -- (85.8000,145.9000) -- (85.8000,145.8000) -- (85.8000,145.8000) -- (85.8000,145.8000) -- (85.8000,145.8000) -- (85.9000,145.8000) -- (85.9000,145.8000) -- (85.9000,145.8000) -- (85.9000,145.8000) -- (85.9000,145.8000) -- (85.9000,145.8000) -- (85.9000,145.8000) -- (85.9000,145.8000) -- (85.9000,145.8000) -- (85.9000,145.8000) -- (85.9000,145.8000) -- (85.9000,145.8000) -- (85.9000,145.8000) -- (85.9000,145.8000) -- (85.9000,145.8000) -- (85.9000,145.8000) -- (85.9000,145.8000) -- (85.9000,145.8000) -- (85.9000,145.7000) -- (85.9000,145.7000) -- (85.9000,145.7000) -- (85.9000,145.7000) -- (85.9000,145.7000) -- (85.9000,145.7000) -- (85.9000,145.7000) -- (85.9000,145.7000) -- (85.9000,145.7000) -- (85.9000,145.7000) -- (85.9000,145.7000) -- (85.9000,145.7000) -- (85.9000,145.7000) -- (85.9000,145.7000) -- (85.9000,145.7000) -- (85.9000,145.7000) -- (85.9000,145.7000) -- (85.9000,145.7000) -- (85.9000,145.7000) -- (85.9000,145.7000) -- (85.9000,145.7000) -- (85.9000,145.6000) -- (85.9000,145.6000) -- (85.9000,145.6000) -- (85.9000,145.6000) -- (85.9000,145.6000) -- (85.9000,145.6000) -- (85.9000,145.6000) -- (85.9000,145.6000) -- (85.9000,145.6000) -- (85.9000,145.6000) -- (85.9000,145.6000) -- (86.0000,145.6000) -- (86.0000,145.6000) -- (86.0000,145.6000) -- (86.0000,145.6000) -- (86.0000,145.6000) -- (86.0000,145.6000) -- (86.0000,145.6000) -- (86.0000,145.6000) -- (86.0000,145.6000) -- (86.0000,145.6000) -- (86.0000,145.6000) -- (86.0000,145.5000) -- (86.0000,145.5000) -- (86.0000,145.5000) -- (86.0000,145.5000) -- (86.0000,145.5000) -- (86.0000,145.5000) -- (86.0000,145.5000) -- (86.0000,145.5000) -- (86.0000,145.5000) -- (86.0000,145.5000) -- (86.0000,145.5000) -- (86.0000,145.5000) -- (86.0000,145.5000) -- (86.0000,145.5000) -- (86.0000,145.5000) -- (86.0000,145.5000) -- (86.0000,145.5000) -- (86.0000,145.5000) -- (86.0000,145.5000) -- (86.0000,145.5000) -- (86.0000,145.5000) -- (86.0000,145.4000) -- (86.0000,145.4000) -- (86.0000,145.4000) -- (86.0000,145.4000) -- (86.0000,145.4000) -- (86.0000,145.4000) -- (86.0000,145.4000) -- (86.0000,145.4000) -- (86.0000,145.4000) -- (86.0000,145.4000) -- (86.0000,145.4000) -- (86.0000,145.4000) -- (86.0000,145.4000) -- (86.0000,145.4000) -- (86.0000,145.4000) -- (86.0000,145.4000) -- (86.0000,145.4000) -- (86.1000,145.4000) -- (86.1000,145.4000) -- (86.1000,145.4000) -- (86.1000,145.4000) -- (86.1000,145.4000) -- (86.1000,145.3000) -- (86.1000,145.3000) -- (86.1000,145.3000) -- (86.1000,145.3000) -- (86.1000,145.3000) -- (86.1000,145.3000) -- (86.1000,145.3000) -- (86.1000,145.3000) -- (86.1000,145.3000) -- (86.1000,145.3000) -- (86.1000,145.3000) -- (86.1000,145.3000) -- (86.1000,145.3000) -- (86.1000,145.3000) -- (86.1000,145.3000) -- (86.1000,145.3000) -- (86.1000,145.3000) -- (86.1000,145.3000) -- (86.1000,145.3000) -- (86.1000,145.3000) -- (86.1000,145.3000) -- (86.1000,145.2000) -- (86.1000,145.2000) -- (86.1000,145.2000) -- (86.1000,145.2000) -- (86.1000,145.2000) -- (86.1000,145.2000) -- (86.1000,145.2000) -- (86.1000,145.2000) -- (86.1000,145.2000) -- (86.1000,145.2000) -- (86.1000,145.2000) -- (86.1000,145.2000) -- (86.1000,145.2000) -- (86.1000,145.2000) -- (86.1000,145.2000) -- (86.1000,145.2000) -- (86.1000,145.2000) -- (86.1000,145.2000) -- (86.1000,145.2000) -- (86.1000,145.2000) -- (86.1000,145.2000) -- (86.1000,145.2000) -- (86.1000,145.1000) -- (86.1000,145.1000) -- (86.2000,145.1000) -- (86.2000,145.1000) -- (86.2000,145.1000) -- (86.2000,145.1000) -- (86.2000,145.1000) -- (86.2000,145.1000) -- (86.2000,145.1000) -- (86.2000,145.1000) -- (86.2000,145.1000) -- (86.2000,145.1000) -- (86.2000,145.1000) -- (86.2000,145.1000) -- (86.2000,145.1000) -- (86.2000,145.1000) -- (86.2000,145.1000) -- (86.2000,145.1000) -- (86.2000,145.1000) -- (86.2000,145.1000) -- (86.2000,145.1000) -- (86.2000,145.0000) -- (86.2000,145.0000) -- (86.2000,145.0000) -- (86.2000,145.0000) -- (86.2000,145.0000) -- (86.2000,145.0000) -- (86.2000,145.0000) -- (86.2000,145.0000) -- (86.2000,145.0000) -- (86.2000,145.0000) -- (86.2000,145.0000) -- (86.2000,145.0000) -- (86.2000,145.0000) -- (86.2000,145.0000) -- (86.2000,145.0000) -- (86.2000,145.0000) -- (86.2000,145.0000) -- (86.2000,145.0000) -- (86.2000,145.0000) -- (86.2000,145.0000) -- (86.2000,145.0000) -- (86.2000,145.0000) -- (86.2000,144.9000) -- (86.2000,144.9000) -- (86.2000,144.9000) -- (86.2000,144.9000) -- (86.2000,144.9000) -- (86.2000,144.9000) -- (86.2000,144.9000) -- (86.2000,144.9000) -- (86.2000,144.9000) -- (86.3000,144.9000) -- (86.3000,144.9000) -- (86.3000,144.9000) -- (86.3000,144.9000) -- (86.3000,144.9000) -- (86.3000,144.9000) -- (86.3000,144.9000) -- (86.3000,144.9000) -- (86.3000,144.9000) -- (86.3000,144.9000) -- (86.3000,144.9000) -- (86.3000,144.9000) -- (86.3000,144.8000) -- (86.3000,144.8000) -- (86.3000,144.8000) -- (86.3000,144.8000) -- (86.3000,144.8000) -- (86.3000,144.8000) -- (86.3000,144.8000) -- (86.3000,144.8000) -- (86.3000,144.8000) -- (86.3000,144.8000) -- (86.3000,144.8000) -- (86.3000,144.8000) -- (86.3000,144.8000) -- (86.3000,144.8000) -- (86.3000,144.8000) -- (86.3000,144.8000) -- (86.3000,144.8000) -- (86.3000,144.8000) -- (86.3000,144.8000) -- (86.3000,144.8000) -- (86.3000,144.8000) -- (86.3000,144.8000) -- (86.3000,144.7000) -- (86.3000,144.7000) -- (86.3000,144.7000) -- (86.3000,144.7000) -- (86.3000,144.7000) -- (86.3000,144.7000) -- (86.3000,144.7000) -- (86.3000,144.7000) -- (86.3000,144.7000) -- (86.3000,144.7000) -- (86.3000,144.7000) -- (86.3000,144.7000) -- (86.3000,144.7000) -- (86.3000,144.7000) -- (86.3000,144.7000) -- (86.4000,144.7000) -- (86.4000,144.7000) -- (86.4000,144.7000) -- (86.4000,144.7000) -- (86.4000,144.7000) -- (86.4000,144.7000) -- (86.4000,144.6000) -- (86.4000,144.6000) -- (86.4000,144.6000) -- (86.4000,144.6000) -- (86.4000,144.6000) -- (86.4000,144.6000) -- (86.4000,144.6000) -- (86.4000,144.6000) -- (86.4000,144.6000) -- (86.4000,144.6000) -- (86.4000,144.6000) -- (86.4000,144.6000) -- (86.4000,144.6000) -- (86.4000,144.6000) -- (86.4000,144.6000) -- (86.4000,144.6000) -- (86.4000,144.6000) -- (86.4000,144.6000) -- (86.4000,144.6000) -- (86.4000,144.6000) -- (86.4000,144.6000) -- (86.4000,144.6000) -- (86.4000,144.5000) -- (86.4000,144.5000) -- (86.4000,144.5000) -- (86.4000,144.5000) -- (86.4000,144.5000) -- (86.4000,144.5000) -- (86.4000,144.5000) -- (86.4000,144.5000) -- (86.4000,144.5000) -- (86.4000,144.5000) -- (86.4000,144.5000) -- (86.4000,144.5000) -- (86.4000,144.5000) -- (86.4000,144.5000) -- (86.4000,144.5000) -- (86.4000,144.5000) -- (86.4000,144.5000) -- (86.4000,144.5000) -- (86.4000,144.5000) -- (86.4000,144.5000) -- (86.4000,144.5000) -- (86.4000,144.4000) -- (86.5000,144.4000) -- (86.5000,144.4000) -- (86.5000,144.4000) -- (86.5000,144.4000) -- (86.5000,144.4000) -- (86.5000,144.4000) -- (86.5000,144.4000) -- (86.5000,144.4000) -- (86.5000,144.4000) -- (86.5000,144.4000) -- (86.5000,144.4000) -- (86.5000,144.4000) -- (86.5000,144.4000) -- (86.5000,144.4000) -- (86.5000,144.4000) -- (86.5000,144.4000) -- (86.5000,144.4000) -- (86.5000,144.4000) -- (86.5000,144.4000) -- (86.5000,144.4000) -- (86.5000,144.4000) -- (86.5000,144.3000) -- (86.5000,144.3000) -- (86.5000,144.3000) -- (86.5000,144.3000) -- (86.5000,144.3000) -- (86.5000,144.3000) -- (86.5000,144.3000) -- (86.5000,144.3000) -- (86.5000,144.3000) -- (86.5000,144.3000) -- (86.5000,144.3000) -- (86.5000,144.3000) -- (86.5000,144.3000) -- (86.5000,144.3000) -- (86.5000,144.3000) -- (86.5000,144.3000) -- (86.5000,144.3000) -- (86.5000,144.3000) -- (86.5000,144.3000) -- (86.5000,144.3000) -- (86.5000,144.3000) -- (86.5000,144.2000) -- (86.5000,144.2000) -- (86.5000,144.2000) -- (86.5000,144.2000) -- (86.5000,144.2000) -- (86.5000,144.2000) -- (86.5000,144.2000) -- (86.6000,144.2000) -- (86.6000,144.2000) -- (86.6000,144.2000) -- (86.6000,144.2000) -- (86.6000,144.2000) -- (86.6000,144.2000) -- (86.6000,144.2000) -- (86.6000,144.2000) -- (86.6000,144.2000) -- (86.6000,144.2000) -- (86.6000,144.2000) -- (86.6000,144.2000) -- (86.6000,144.2000) -- (86.6000,144.2000) -- (86.6000,144.2000) -- (86.6000,144.1000) -- (86.6000,144.1000) -- (86.6000,144.1000) -- (86.6000,144.1000) -- (86.6000,144.1000) -- (86.6000,144.1000) -- (86.6000,144.1000) -- (86.6000,144.1000) -- (86.6000,144.1000) -- (86.6000,144.1000) -- (86.6000,144.1000) -- (86.6000,144.1000) -- (86.6000,144.1000) -- (86.6000,144.1000) -- (86.6000,144.1000) -- (86.6000,144.1000) -- (86.6000,144.1000) -- (86.6000,144.1000) -- (86.6000,144.1000) -- (86.6000,144.1000) -- (86.6000,144.1000) -- (86.6000,144.0000) -- (86.6000,144.0000) -- (86.6000,144.0000) -- (86.6000,144.0000) -- (86.6000,144.0000) -- (86.6000,144.0000) -- (86.6000,144.0000) -- (86.6000,144.0000) -- (86.6000,144.0000) -- (86.6000,144.0000) -- (86.6000,144.0000) -- (86.6000,144.0000) -- (86.6000,144.0000) -- (86.6000,144.0000) -- (86.7000,144.0000) -- (86.7000,144.0000) -- (86.7000,144.0000) -- (86.7000,144.0000) -- (86.7000,144.0000) -- (86.7000,144.0000) -- (86.7000,144.0000) -- (86.7000,144.0000) -- (86.7000,143.9000) -- (86.7000,143.9000) -- (86.7000,143.9000) -- (86.7000,143.9000) -- (86.7000,143.9000) -- (86.7000,143.9000) -- (86.7000,143.9000) -- (86.7000,143.9000) -- (86.7000,143.9000) -- (86.7000,143.9000) -- (86.7000,143.9000) -- (86.7000,143.9000) -- (86.7000,143.9000) -- (86.7000,143.9000) -- (86.7000,143.9000) -- (86.7000,143.9000) -- (86.7000,143.9000) -- (86.7000,143.9000) -- (86.7000,143.9000) -- (86.7000,143.9000) -- (86.7000,143.9000) -- (86.7000,143.8000) -- (86.7000,143.8000) -- (86.7000,143.8000) -- (86.7000,143.8000) -- (86.7000,143.8000) -- (86.7000,143.8000) -- (86.7000,143.8000) -- (86.7000,143.8000) -- (86.7000,143.8000) -- (86.7000,143.8000) -- (86.7000,143.8000) -- (86.7000,143.8000) -- (86.7000,143.8000) -- (86.7000,143.8000) -- (86.7000,143.8000) -- (86.7000,143.8000) -- (86.7000,143.8000) -- (86.7000,143.8000) -- (86.7000,143.8000) -- (86.7000,143.8000) -- (86.8000,143.8000) -- (86.8000,143.8000) -- (86.8000,143.7000) -- (86.8000,143.7000) -- (86.8000,143.7000) -- (86.8000,143.7000) -- (86.8000,143.7000) -- (86.8000,143.7000) -- (86.8000,143.7000) -- (86.8000,143.7000) -- (86.8000,143.7000) -- (86.8000,143.7000) -- (86.8000,143.7000) -- (86.8000,143.7000) -- (86.8000,143.7000) -- (86.8000,143.7000) -- (86.8000,143.7000) -- (86.8000,143.7000) -- (86.8000,143.7000) -- (86.8000,143.7000) -- (86.8000,143.7000) -- (86.8000,143.7000) -- (86.8000,143.7000) -- (86.8000,143.6000) -- (86.8000,143.6000) -- (86.8000,143.6000) -- (86.8000,143.6000) -- (86.8000,143.6000) -- (86.8000,143.6000) -- (86.8000,143.6000) -- (86.8000,143.6000) -- (86.8000,143.6000) -- (86.8000,143.6000) -- (86.8000,143.6000) -- (86.8000,143.6000) -- (86.8000,143.6000) -- (86.8000,143.6000) -- (86.8000,143.6000) -- (86.8000,143.6000) -- (86.8000,143.6000) -- (86.8000,143.6000) -- (86.8000,143.6000) -- (86.8000,143.6000) -- (86.8000,143.6000) -- (86.8000,143.6000) -- (86.8000,143.5000) -- (86.8000,143.5000) -- (86.8000,143.5000) -- (86.8000,143.5000) -- (86.8000,143.5000) -- (86.9000,143.5000) -- (86.9000,143.5000) -- (86.9000,143.5000) -- (86.9000,143.5000) -- (86.9000,143.5000) -- (86.9000,143.5000) -- (86.9000,143.5000) -- (86.9000,143.5000) -- (86.9000,143.5000) -- (86.9000,143.5000) -- (86.9000,143.5000) -- (86.9000,143.5000) -- (86.9000,143.5000) -- (86.9000,143.5000) -- (86.9000,143.5000) -- (86.9000,143.5000) -- (86.9000,143.4000) -- (86.9000,143.4000) -- (86.9000,143.4000) -- (86.9000,143.4000) -- (86.9000,143.4000) -- (86.9000,143.4000) -- (86.9000,143.4000) -- (86.9000,143.4000) -- (86.9000,143.4000) -- (86.9000,143.4000) -- (86.9000,143.4000) -- (86.9000,143.4000) -- (86.9000,143.4000) -- (86.9000,143.4000) -- (86.9000,143.4000) -- (86.9000,143.4000) -- (86.9000,143.4000) -- (86.9000,143.4000) -- (86.9000,143.4000) -- (86.9000,143.4000) -- (86.9000,143.4000) -- (86.9000,143.4000) -- (86.9000,143.3000) -- (86.9000,143.3000) -- (86.9000,143.3000) -- (86.9000,143.3000) -- (86.9000,143.3000) -- (86.9000,143.3000) -- (86.9000,143.3000) -- (86.9000,143.3000) -- (86.9000,143.3000) -- (86.9000,143.3000) -- (86.9000,143.3000) -- (87.0000,143.3000) -- (87.0000,143.3000) -- (87.0000,143.3000) -- (87.0000,143.3000) -- (87.0000,143.3000) -- (87.0000,143.3000) -- (87.0000,143.3000) -- (87.0000,143.3000) -- (87.0000,143.3000) -- (87.0000,143.3000) -- (87.0000,143.2000) -- (87.0000,143.2000) -- (87.0000,143.2000) -- (87.0000,143.2000) -- (87.0000,143.2000) -- (87.0000,143.2000) -- (87.0000,143.2000) -- (87.0000,143.2000) -- (87.0000,143.2000) -- (87.0000,143.2000) -- (87.0000,143.2000) -- (87.0000,143.2000) -- (87.0000,143.2000) -- (87.0000,143.2000) -- (87.0000,143.2000) -- (87.0000,143.2000) -- (87.0000,143.2000) -- (87.0000,143.2000) -- (87.0000,143.2000) -- (87.0000,143.2000) -- (87.0000,143.2000) -- (87.0000,143.2000) -- (87.0000,143.1000) -- (87.0000,143.1000) -- (87.0000,143.1000) -- (87.0000,143.1000) -- (87.0000,143.1000) -- (87.0000,143.1000) -- (87.0000,143.1000) -- (87.0000,143.1000) -- (87.0000,143.1000) -- (87.0000,143.1000) -- (87.0000,143.1000) -- (87.0000,143.1000) -- (87.0000,143.1000) -- (87.0000,143.1000) -- (87.0000,143.1000) -- (87.0000,143.1000) -- (87.0000,143.1000) -- (87.0000,143.1000) -- (87.1000,143.1000) -- (87.1000,143.1000) -- (87.1000,143.1000) -- (87.1000,143.0000) -- (87.1000,143.0000) -- (87.1000,143.0000) -- (87.1000,143.0000) -- (87.1000,143.0000) -- (87.1000,143.0000) -- (87.1000,143.0000) -- (87.1000,143.0000) -- (87.1000,143.0000) -- (87.1000,143.0000) -- (87.1000,143.0000) -- (87.1000,143.0000) -- (87.1000,143.0000) -- (87.1000,143.0000) -- (87.1000,143.0000) -- (87.1000,143.0000) -- (87.1000,143.0000) -- (87.1000,143.0000) -- (87.1000,143.0000) -- (87.1000,143.0000) -- (87.1000,143.0000) -- (87.1000,143.0000) -- (87.1000,142.9000) -- (87.1000,142.9000) -- (87.1000,142.9000) -- (87.1000,142.9000) -- (87.1000,142.9000) -- (87.1000,142.9000) -- (87.1000,142.9000) -- (87.1000,142.9000) -- (87.1000,142.9000) -- (87.1000,142.9000) -- (87.1000,142.9000) -- (87.1000,142.9000) -- (87.1000,142.9000) -- (87.1000,142.9000) -- (87.1000,142.9000) -- (87.1000,142.9000) -- (87.1000,142.9000) -- (87.1000,142.9000) -- (87.1000,142.9000) -- (87.1000,142.9000) -- (87.1000,142.9000) -- (87.1000,142.9000) -- (87.1000,142.8000) -- (87.1000,142.8000) -- (87.2000,142.8000) -- (87.2000,142.8000) -- (87.2000,142.8000) -- (87.2000,142.8000) -- (87.2000,142.8000) -- (87.2000,142.8000) -- (87.2000,142.8000) -- (87.2000,142.8000) -- (87.2000,142.8000) -- (87.2000,142.8000) -- (87.2000,142.8000) -- (87.2000,142.8000) -- (87.2000,142.8000) -- (87.2000,142.8000) -- (87.2000,142.8000) -- (87.2000,142.8000) -- (87.2000,142.8000) -- (87.2000,142.8000) -- (87.2000,142.8000) -- (87.2000,142.7000) -- (87.2000,142.7000) -- (87.2000,142.7000) -- (87.2000,142.7000) -- (87.2000,142.7000) -- (87.2000,142.7000) -- (87.2000,142.7000) -- (87.2000,142.7000) -- (87.2000,142.7000) -- (87.2000,142.7000) -- (87.2000,142.7000) -- (87.2000,142.7000) -- (87.2000,142.7000) -- (87.2000,142.7000) -- (87.2000,142.7000) -- (87.2000,142.7000) -- (87.2000,142.7000) -- (87.2000,142.7000) -- (87.2000,142.7000) -- (87.2000,142.7000) -- (87.2000,142.7000) -- (87.2000,142.7000) -- (87.2000,142.6000) -- (87.2000,142.6000) -- (87.2000,142.6000) -- (87.2000,142.6000) -- (87.2000,142.6000) -- (87.2000,142.6000) -- (87.2000,142.6000) -- (87.2000,142.6000) -- (87.2000,142.6000) -- (87.3000,142.6000) -- (87.3000,142.6000) -- (87.3000,142.6000) -- (87.3000,142.6000) -- (87.3000,142.6000) -- (87.3000,142.6000) -- (87.3000,142.6000) -- (87.3000,142.6000) -- (87.3000,142.6000) -- (87.3000,142.6000) -- (87.3000,142.6000) -- (87.3000,142.6000) -- (87.3000,142.5000) -- (87.3000,142.5000) -- (87.3000,142.5000) -- (87.3000,142.5000) -- (87.3000,142.5000) -- (87.3000,142.5000) -- (87.3000,142.5000) -- (87.3000,142.5000) -- (87.3000,142.5000) -- (87.3000,142.5000) -- (87.3000,142.5000) -- (87.3000,142.5000) -- (87.3000,142.5000) -- (87.3000,142.5000) -- (87.3000,142.5000) -- (87.3000,142.5000) -- (87.3000,142.5000) -- (87.3000,142.5000) -- (87.3000,142.5000) -- (87.3000,142.5000) -- (87.3000,142.5000) -- (87.3000,142.5000) -- (87.3000,142.4000) -- (87.3000,142.4000) -- (87.3000,142.4000) -- (87.3000,142.4000) -- (87.3000,142.4000) -- (87.3000,142.4000) -- (87.3000,142.4000) -- (87.3000,142.4000) -- (87.3000,142.4000) -- (87.3000,142.4000) -- (87.3000,142.4000) -- (87.3000,142.4000) -- (87.3000,142.4000) -- (87.3000,142.4000) -- (87.3000,142.4000) -- (87.4000,142.4000) -- (87.4000,142.4000) -- (87.4000,142.4000) -- (87.4000,142.4000) -- (87.4000,142.4000) -- (87.4000,142.4000) -- (87.4000,142.3000) -- (87.4000,142.3000) -- (87.4000,142.3000) -- (87.4000,142.3000) -- (87.4000,142.3000) -- (87.4000,142.3000) -- (87.4000,142.3000) -- (87.4000,142.3000) -- (87.4000,142.3000) -- (87.4000,142.3000) -- (87.4000,142.3000) -- (87.4000,142.3000) -- (87.4000,142.3000) -- (87.4000,142.3000) -- (87.4000,142.3000) -- (87.4000,142.3000) -- (87.4000,142.3000) -- (87.4000,142.3000) -- (87.4000,142.3000) -- (87.4000,142.3000) -- (87.4000,142.3000) -- (87.4000,142.3000) -- (87.4000,142.2000) -- (87.4000,142.2000) -- (87.4000,142.2000) -- (87.4000,142.2000) -- (87.4000,142.2000) -- (87.4000,142.2000) -- (87.4000,142.2000) -- (87.4000,142.2000) -- (87.4000,142.2000) -- (87.4000,142.2000) -- (87.4000,142.2000) -- (87.4000,142.2000) -- (87.4000,142.2000) -- (87.4000,142.2000) -- (87.4000,142.2000) -- (87.4000,142.2000) -- (87.4000,142.2000) -- (87.4000,142.2000) -- (87.4000,142.2000) -- (87.4000,142.2000) -- (87.4000,142.2000) -- (87.4000,142.1000) -- (87.5000,142.1000) -- (87.5000,142.1000) -- (87.5000,142.1000) -- (87.5000,142.1000) -- (87.5000,142.1000) -- (87.5000,142.1000) -- (87.5000,142.1000) -- (87.5000,142.1000) -- (87.5000,142.1000) -- (87.5000,142.1000) -- (87.5000,142.1000) -- (87.5000,142.1000) -- (87.5000,142.1000) -- (87.5000,142.1000) -- (87.5000,142.1000) -- (87.5000,142.1000) -- (87.5000,142.1000) -- (87.5000,142.1000) -- (87.5000,142.1000) -- (87.5000,142.1000) -- (87.5000,142.1000) -- (87.5000,142.0000) -- (87.5000,142.0000) -- (87.5000,142.0000) -- (87.5000,142.0000) -- (87.5000,142.0000) -- (87.5000,142.0000) -- (87.5000,142.0000) -- (87.5000,142.0000) -- (87.5000,142.0000) -- (87.5000,142.0000) -- (87.5000,142.0000) -- (87.5000,142.0000) -- (87.5000,142.0000) -- (87.5000,142.0000) -- (87.5000,142.0000) -- (87.5000,142.0000) -- (87.5000,142.0000) -- (87.5000,142.0000) -- (87.5000,142.0000) -- (87.5000,142.0000) -- (87.5000,142.0000) -- (87.5000,141.9000) -- (87.5000,141.9000) -- (87.5000,141.9000) -- (87.5000,141.9000) -- (87.5000,141.9000) -- (87.5000,141.9000) -- (87.5000,141.9000) -- (87.6000,141.9000) -- (87.6000,141.9000) -- (87.6000,141.9000) -- (87.6000,141.9000) -- (87.6000,141.9000) -- (87.6000,141.9000) -- (87.6000,141.9000) -- (87.6000,141.9000) -- (87.6000,141.9000) -- (87.6000,141.9000) -- (87.6000,141.9000) -- (87.6000,141.9000) -- (87.6000,141.9000) -- (87.6000,141.9000) -- (87.6000,141.9000) -- (87.6000,141.8000) -- (87.6000,141.8000) -- (87.6000,141.8000) -- (87.6000,141.8000) -- (87.6000,141.8000) -- (87.6000,141.8000) -- (87.6000,141.8000) -- (87.6000,141.8000) -- (87.6000,141.8000) -- (87.6000,141.8000) -- (87.6000,141.8000) -- (87.6000,141.8000) -- (87.6000,141.8000) -- (87.6000,141.8000) -- (87.6000,141.8000) -- (87.6000,141.8000) -- (87.6000,141.8000) -- (87.6000,141.8000) -- (87.6000,141.8000) -- (87.6000,141.8000) -- (87.6000,141.8000) -- (87.6000,141.7000) -- (87.6000,141.7000) -- (87.6000,141.7000) -- (87.6000,141.7000) -- (87.6000,141.7000) -- (87.6000,141.7000) -- (87.6000,141.7000) -- (87.6000,141.7000) -- (87.6000,141.7000) -- (87.6000,141.7000) -- (87.6000,141.7000) -- (87.6000,141.7000) -- (87.6000,141.7000) -- (87.6000,141.7000) -- (87.7000,141.7000) -- (87.7000,141.7000) -- (87.7000,141.7000) -- (87.7000,141.7000) -- (87.7000,141.7000) -- (87.7000,141.7000) -- (87.7000,141.7000) -- (87.7000,141.7000) -- (87.7000,141.6000) -- (87.7000,141.6000) -- (87.7000,141.6000) -- (87.7000,141.6000) -- (87.7000,141.6000) -- (87.7000,141.6000) -- (87.7000,141.6000) -- (87.7000,141.6000) -- (87.7000,141.6000) -- (87.7000,141.6000) -- (87.7000,141.6000) -- (87.7000,141.6000) -- (87.7000,141.6000) -- (87.7000,141.6000) -- (87.7000,141.6000) -- (87.7000,141.6000) -- (87.7000,141.6000) -- (87.7000,141.6000) -- (87.7000,141.6000) -- (87.7000,141.6000) -- (87.7000,141.6000) -- (87.7000,141.5000) -- (87.7000,141.5000) -- (87.7000,141.5000) -- (87.7000,141.5000) -- (87.7000,141.5000) -- (87.7000,141.5000) -- (87.7000,141.5000) -- (87.7000,141.5000) -- (87.7000,141.5000) -- (87.7000,141.5000) -- (87.7000,141.5000) -- (87.7000,141.5000) -- (87.7000,141.5000) -- (87.7000,141.5000) -- (87.7000,141.5000) -- (87.7000,141.5000) -- (87.7000,141.5000) -- (87.7000,141.5000) -- (87.7000,141.5000) -- (87.7000,141.5000) -- (87.8000,141.5000) -- (87.8000,141.5000) -- (87.8000,141.4000) -- (87.8000,141.4000) -- (87.8000,141.4000) -- (87.8000,141.4000) -- (87.8000,141.4000) -- (87.8000,141.4000) -- (87.8000,141.4000) -- (87.8000,141.4000) -- (87.8000,141.4000) -- (87.8000,141.4000) -- (87.8000,141.4000) -- (87.8000,141.4000) -- (87.8000,141.4000) -- (87.8000,141.4000) -- (87.8000,141.4000) -- (87.8000,141.4000) -- (87.8000,141.4000) -- (87.8000,141.4000) -- (87.8000,141.4000) -- (87.8000,141.4000) -- (87.8000,141.4000) -- (87.8000,141.3000) -- (87.8000,141.3000) -- (87.8000,141.3000) -- (87.8000,141.3000) -- (87.8000,141.3000) -- (87.8000,141.3000) -- (87.8000,141.3000) -- (87.8000,141.3000) -- (87.8000,141.3000) -- (87.8000,141.3000) -- (87.8000,141.3000) -- (87.8000,141.3000) -- (87.8000,141.3000) -- (87.8000,141.3000) -- (87.8000,141.3000) -- (87.8000,141.3000) -- (87.8000,141.3000) -- (87.8000,141.3000) -- (87.8000,141.3000) -- (87.8000,141.3000) -- (87.8000,141.3000) -- (87.8000,141.3000) -- (87.8000,141.2000) -- (87.8000,141.2000) -- (87.8000,141.2000) -- (87.8000,141.2000) -- (87.8000,141.2000) -- (87.9000,141.2000) -- (87.9000,141.2000) -- (87.9000,141.2000) -- (87.9000,141.2000) -- (87.9000,141.2000) -- (87.9000,141.2000) -- (87.9000,141.2000) -- (87.9000,141.2000) -- (87.9000,141.2000) -- (87.9000,141.2000) -- (87.9000,141.2000) -- (87.9000,141.2000) -- (87.9000,141.2000) -- (87.9000,141.2000) -- (87.9000,141.2000) -- (87.9000,141.2000) -- (87.9000,141.1000) -- (87.9000,141.1000) -- (87.9000,141.1000) -- (87.9000,141.1000) -- (87.9000,141.1000) -- (87.9000,141.1000) -- (87.9000,141.1000) -- (87.9000,141.1000) -- (87.9000,141.1000) -- (87.9000,141.1000) -- (87.9000,141.1000) -- (87.9000,141.1000) -- (87.9000,141.1000) -- (87.9000,141.1000) -- (87.9000,141.1000) -- (87.9000,141.1000) -- (87.9000,141.1000) -- (87.9000,141.1000) -- (87.9000,141.1000) -- (87.9000,141.1000) -- (87.9000,141.1000) -- (87.9000,141.1000) -- (87.9000,141.0000) -- (87.9000,141.0000) -- (87.9000,141.0000) -- (87.9000,141.0000) -- (87.9000,141.0000) -- (87.9000,141.0000) -- (87.9000,141.0000) -- (87.9000,141.0000) -- (87.9000,141.0000) -- (87.9000,141.0000) -- (87.9000,141.0000) -- (88.0000,141.0000) -- (88.0000,141.0000) -- (88.0000,141.0000) -- (88.0000,141.0000) -- (88.0000,141.0000) -- (88.0000,141.0000) -- (88.0000,141.0000) -- (88.0000,141.0000) -- (88.0000,141.0000) -- (88.0000,141.0000) -- (88.0000,140.9000) -- (88.0000,140.9000) -- (88.0000,140.9000) -- (88.0000,140.9000) -- (88.0000,140.9000) -- (88.0000,140.9000) -- (88.0000,140.9000) -- (88.0000,140.9000) -- (88.0000,140.9000) -- (88.0000,140.9000) -- (88.0000,140.9000) -- (88.0000,140.9000) -- (88.0000,140.9000) -- (88.0000,140.9000) -- (88.0000,140.9000) -- (88.0000,140.9000) -- (88.0000,140.9000) -- (88.0000,140.9000) -- (88.0000,140.9000) -- (88.0000,140.9000) -- (88.0000,140.9000) -- (88.0000,140.9000) -- (88.0000,140.8000) -- (88.0000,140.8000) -- (88.0000,140.8000) -- (88.0000,140.8000) -- (88.0000,140.8000) -- (88.0000,140.8000) -- (88.0000,140.8000) -- (88.0000,140.8000) -- (88.0000,140.8000) -- (88.0000,140.8000) -- (88.0000,140.8000) -- (88.0000,140.8000) -- (88.0000,140.8000) -- (88.0000,140.8000) -- (88.0000,140.8000) -- (88.0000,140.8000) -- (88.0000,140.8000) -- (88.0000,140.8000) -- (88.1000,140.8000) -- (88.1000,140.8000) -- (88.1000,140.8000) -- (88.1000,140.7000) -- (88.1000,140.7000) -- (88.1000,140.7000) -- (88.1000,140.7000) -- (88.1000,140.7000) -- (88.1000,140.7000) -- (88.1000,140.7000) -- (88.1000,140.7000) -- (88.1000,140.7000) -- (88.1000,140.7000) -- (88.1000,140.7000) -- (88.1000,140.7000) -- (88.1000,140.7000) -- (88.1000,140.7000) -- (88.1000,140.7000) -- (88.1000,140.7000) -- (88.1000,140.7000) -- (88.1000,140.7000) -- (88.1000,140.7000) -- (88.1000,140.7000) -- (88.1000,140.7000) -- (88.1000,140.7000) -- (88.1000,140.6000) -- (88.1000,140.6000) -- (88.1000,140.6000) -- (88.1000,140.6000) -- (88.1000,140.6000) -- (88.1000,140.6000) -- (88.1000,140.6000) -- (88.1000,140.6000) -- (88.1000,140.6000) -- (88.1000,140.6000) -- (88.1000,140.6000) -- (88.1000,140.6000) -- (88.1000,140.6000) -- (88.1000,140.6000) -- (88.1000,140.6000) -- (88.1000,140.6000) -- (88.1000,140.6000) -- (88.1000,140.6000) -- (88.1000,140.6000) -- (88.1000,140.6000) -- (88.1000,140.6000) -- (88.1000,140.5000) -- (88.1000,140.5000) -- (88.1000,140.5000) -- (88.2000,140.5000) -- (88.2000,140.5000) -- (88.2000,140.5000) -- (88.2000,140.5000) -- (88.2000,140.5000) -- (88.2000,140.5000) -- (88.2000,140.5000) -- (88.2000,140.5000) -- (88.2000,140.5000) -- (88.2000,140.5000) -- (88.2000,140.5000) -- (88.2000,140.5000) -- (88.2000,140.5000) -- (88.2000,140.5000) -- (88.2000,140.5000) -- (88.2000,140.5000) -- (88.2000,140.5000) -- (88.2000,140.5000) -- (88.2000,140.5000) -- (88.2000,140.4000) -- (88.2000,140.4000) -- (88.2000,140.4000) -- (88.2000,140.4000) -- (88.2000,140.4000) -- (88.2000,140.4000) -- (88.2000,140.4000) -- (88.2000,140.4000) -- (88.2000,140.4000) -- (88.2000,140.4000) -- (88.2000,140.4000) -- (88.2000,140.4000) -- (88.2000,140.4000) -- (88.2000,140.4000) -- (88.2000,140.4000) -- (88.2000,140.4000) -- (88.2000,140.4000) -- (88.2000,140.4000) -- (88.2000,140.4000) -- (88.2000,140.4000) -- (88.2000,140.4000) -- (88.2000,140.3000) -- (88.2000,140.3000) -- (88.2000,140.3000) -- (88.2000,140.3000) -- (88.2000,140.3000) -- (88.2000,140.3000) -- (88.2000,140.3000) -- (88.2000,140.3000) -- (88.2000,140.3000) -- (88.2000,140.3000) -- (88.3000,140.3000) -- (88.3000,140.3000) -- (88.3000,140.3000) -- (88.3000,140.3000) -- (88.3000,140.3000) -- (88.3000,140.3000) -- (88.3000,140.3000) -- (88.3000,140.3000) -- (88.3000,140.3000) -- (88.3000,140.3000) -- (88.3000,140.3000) -- (88.3000,140.3000) -- (88.3000,140.2000) -- (88.3000,140.2000) -- (88.3000,140.2000) -- (88.3000,140.2000) -- (88.3000,140.2000) -- (88.3000,140.2000) -- (88.3000,140.2000) -- (88.3000,140.2000) -- (88.3000,140.2000) -- (88.3000,140.2000) -- (88.3000,140.2000) -- (88.3000,140.2000) -- (88.3000,140.2000) -- (88.3000,140.2000) -- (88.3000,140.2000) -- (88.3000,140.2000) -- (88.3000,140.2000) -- (88.3000,140.2000) -- (88.3000,140.2000) -- (88.3000,140.2000) -- (88.3000,140.2000) -- (88.3000,140.1000) -- (88.3000,140.1000) -- (88.3000,140.1000) -- (88.3000,140.1000) -- (88.3000,140.1000) -- (88.3000,140.1000) -- (88.3000,140.1000) -- (88.3000,140.1000) -- (88.3000,140.1000) -- (88.3000,140.1000) -- (88.3000,140.1000) -- (88.3000,140.1000) -- (88.3000,140.1000) -- (88.3000,140.1000) -- (88.3000,140.1000) -- (88.3000,140.1000) -- (88.4000,140.1000) -- (88.4000,140.1000) -- (88.4000,140.1000) -- (88.4000,140.1000) -- (88.4000,140.1000) -- (88.4000,140.1000) -- (88.4000,140.0000) -- (88.4000,140.0000) -- (88.4000,140.0000) -- (88.4000,140.0000) -- (88.4000,140.0000) -- (88.4000,140.0000) -- (88.4000,140.0000) -- (88.4000,140.0000) -- (88.4000,140.0000) -- (88.4000,140.0000) -- (88.4000,140.0000) -- (88.4000,140.0000) -- (88.4000,140.0000) -- (88.4000,140.0000) -- (88.4000,140.0000) -- (88.4000,140.0000) -- (88.4000,140.0000) -- (88.4000,140.0000) -- (88.4000,140.0000) -- (88.4000,140.0000) -- (88.4000,140.0000) -- (88.4000,139.9000) -- (88.4000,139.9000) -- (88.4000,139.9000) -- (88.4000,139.9000) -- (88.4000,139.9000) -- (88.4000,139.9000) -- (88.4000,139.9000) -- (88.4000,139.9000) -- (88.4000,139.9000) -- (88.4000,139.9000) -- (88.4000,139.9000) -- (88.4000,139.9000) -- (88.4000,139.9000) -- (88.4000,139.9000) -- (88.4000,139.9000) -- (88.4000,139.9000) -- (88.4000,139.9000) -- (88.4000,139.9000) -- (88.4000,139.9000) -- (88.4000,139.9000) -- (88.4000,139.9000) -- (88.4000,139.9000) -- (88.4000,139.8000) -- (88.5000,139.8000) -- (88.5000,139.8000) -- (88.5000,139.8000) -- (88.5000,139.8000) -- (88.5000,139.8000) -- (88.5000,139.8000) -- (88.5000,139.8000) -- (88.5000,139.8000) -- (88.5000,139.8000) -- (88.5000,139.8000) -- (88.5000,139.8000) -- (88.5000,139.8000) -- (88.5000,139.8000) -- (88.5000,139.8000) -- (88.5000,139.8000) -- (88.5000,139.8000) -- (88.5000,139.8000) -- (96.0000,139.8000) -- (96.0000,139.8000) -- (96.0000,139.8000) -- (96.0000,139.7000) -- (96.0000,139.7000) -- (96.0000,139.7000) -- (96.0000,139.7000) -- (96.0000,139.7000) -- (96.0000,139.7000) -- (96.0000,139.7000) -- (96.0000,139.7000) -- (96.0000,139.7000) -- (96.0000,139.7000) -- (96.0000,139.7000) -- (96.0000,139.7000) -- (96.0000,139.7000) -- (96.0000,139.7000) -- (96.0000,139.7000) -- (96.0000,139.7000) -- (96.0000,139.7000) -- (96.0000,139.7000) -- (96.0000,139.7000) -- (96.0000,139.7000) -- (96.0000,139.7000) -- (96.0000,139.7000) -- (96.0000,139.6000) -- (96.0000,139.6000) -- (96.0000,139.6000) -- (96.0000,139.6000) -- (96.0000,139.6000) -- (96.0000,139.6000) -- (96.0000,139.6000) -- (96.0000,139.6000) -- (96.0000,139.6000) -- (96.0000,139.6000) -- (96.0000,139.6000) -- (96.0000,139.6000) -- (96.0000,139.6000) -- (96.0000,139.6000) -- (96.1000,139.6000) -- (96.1000,139.6000) -- (96.1000,139.6000) -- (96.1000,139.6000) -- (96.1000,139.6000) -- (96.1000,139.6000) -- (96.1000,139.6000) -- (96.1000,139.5000) -- (96.1000,139.5000) -- (96.1000,139.5000) -- (96.1000,139.5000) -- (96.1000,139.5000) -- (96.1000,139.5000) -- (96.1000,139.5000) -- (96.1000,139.5000) -- (96.1000,139.5000) -- (96.1000,139.5000) -- (96.1000,139.5000) -- (96.1000,139.5000) -- (96.1000,139.5000) -- (96.1000,139.5000) -- (96.1000,139.5000) -- (96.1000,139.5000) -- (96.1000,139.5000) -- (96.1000,139.5000) -- (96.1000,139.5000) -- (96.1000,139.5000) -- (96.1000,139.5000) -- (96.1000,139.5000) -- (96.1000,139.4000) -- (96.1000,139.4000) -- (96.1000,139.4000) -- (96.1000,139.4000) -- (96.1000,139.4000) -- (96.1000,139.4000) -- (96.1000,139.4000) -- (96.1000,139.4000) -- (96.1000,139.4000) -- (96.1000,139.4000) -- (96.1000,139.4000) -- (96.1000,139.4000) -- (96.1000,139.4000) -- (96.1000,139.4000) -- (96.1000,139.4000) -- (96.1000,139.4000) -- (96.1000,139.4000) -- (96.1000,139.4000) -- (96.1000,139.4000) -- (96.1000,139.4000) -- (96.2000,139.4000) -- (96.2000,139.3000) -- (96.2000,139.3000) -- (96.2000,139.3000) -- (96.2000,139.3000) -- (96.2000,139.3000) -- (96.2000,139.3000) -- (96.2000,139.3000) -- (96.2000,139.3000) -- (96.2000,139.3000) -- (96.2000,139.3000) -- (96.2000,139.3000) -- (96.2000,139.3000) -- (96.2000,139.3000) -- (96.2000,139.3000) -- (96.2000,139.3000) -- (96.2000,139.3000) -- (96.2000,139.3000) -- (96.2000,139.3000) -- (96.2000,139.3000) -- (96.2000,139.3000) -- (96.2000,139.3000) -- (96.2000,139.3000) -- (96.2000,139.2000) -- (96.2000,139.2000) -- (96.2000,139.2000) -- (96.2000,139.2000) -- (96.2000,139.2000) -- (96.2000,139.2000) -- (96.2000,139.2000) -- (96.2000,139.2000) -- (96.2000,139.2000) -- (96.2000,139.2000) -- (96.2000,139.2000) -- (96.2000,139.2000) -- (96.2000,139.2000) -- (96.2000,139.2000) -- (96.2000,139.2000) -- (96.2000,139.2000) -- (96.2000,139.2000) -- (96.2000,139.2000) -- (96.2000,139.2000) -- (96.2000,139.2000) -- (96.2000,139.2000) -- (96.2000,139.1000) -- (96.2000,139.1000) -- (96.2000,139.1000) -- (96.2000,139.1000) -- (96.2000,139.1000) -- (96.2000,139.1000) -- (96.3000,139.1000) -- (96.3000,139.1000) -- (96.3000,139.1000) -- (96.3000,139.1000) -- (96.3000,139.1000) -- (96.3000,139.1000) -- (96.3000,139.1000) -- (96.3000,139.1000) -- (96.3000,139.1000) -- (96.3000,139.1000) -- (96.3000,139.1000) -- (96.3000,139.1000) -- (96.3000,139.1000) -- (96.3000,139.1000) -- (96.3000,139.1000) -- (96.3000,139.1000) -- (96.3000,139.0000) -- (96.3000,139.0000) -- (96.3000,139.0000) -- (96.3000,139.0000) -- (96.3000,139.0000) -- (96.3000,139.0000) -- (96.3000,139.0000) -- (96.3000,139.0000) -- (96.3000,139.0000) -- (96.3000,139.0000) -- (96.3000,139.0000) -- (96.3000,139.0000) -- (96.3000,139.0000) -- (96.3000,139.0000) -- (96.3000,139.0000) -- (96.3000,139.0000) -- (96.3000,139.0000) -- (96.3000,139.0000) -- (96.3000,139.0000) -- (96.3000,139.0000) -- (96.3000,139.0000) -- (96.3000,138.9000) -- (96.3000,138.9000) -- (96.3000,138.9000) -- (96.3000,138.9000) -- (96.3000,138.9000) -- (96.3000,138.9000) -- (96.3000,138.9000) -- (96.3000,138.9000) -- (96.3000,138.9000) -- (96.3000,138.9000) -- (96.3000,138.9000) -- (96.3000,138.9000) -- (96.3000,138.9000) -- (96.4000,138.9000) -- (96.4000,138.9000) -- (96.4000,138.9000) -- (96.4000,138.9000) -- (96.4000,138.9000) -- (96.4000,138.9000) -- (96.4000,138.9000) -- (96.4000,138.9000) -- (96.4000,138.9000) -- (96.4000,138.8000) -- (96.4000,138.8000) -- (96.4000,138.8000) -- (96.4000,138.8000) -- (96.4000,138.8000) -- (96.4000,138.8000) -- (96.4000,138.8000) -- (96.4000,138.8000) -- (96.4000,138.8000) -- (96.4000,138.8000) -- (96.4000,138.8000) -- (96.4000,138.8000) -- (96.4000,138.8000) -- (96.4000,138.8000) -- (96.4000,138.8000) -- (96.4000,138.8000) -- (96.4000,138.8000) -- (96.4000,138.8000) -- (96.4000,138.8000) -- (96.4000,138.8000) -- (96.4000,138.8000) -- (96.4000,138.7000) -- (96.4000,138.7000) -- (96.4000,138.7000) -- (96.4000,138.7000) -- (96.4000,138.7000) -- (96.4000,138.7000) -- (96.4000,138.7000) -- (96.4000,138.7000) -- (96.4000,138.7000) -- (96.4000,138.7000) -- (96.4000,138.7000) -- (96.4000,138.7000) -- (96.4000,138.7000) -- (96.4000,138.7000) -- (96.4000,138.7000) -- (96.4000,138.7000) -- (96.4000,138.7000) -- (96.4000,138.7000) -- (96.4000,138.7000) -- (96.5000,138.7000) -- (96.5000,138.7000) -- (96.5000,138.7000) -- (96.5000,138.6000) -- (96.5000,138.6000) -- (96.5000,138.6000) -- (96.5000,138.6000) -- (96.5000,138.6000) -- (96.5000,138.6000) -- (96.5000,138.6000) -- (96.5000,138.6000) -- (96.5000,138.6000) -- (96.5000,138.6000) -- (96.5000,138.6000) -- (96.5000,138.6000) -- (96.5000,138.6000) -- (96.5000,138.6000) -- (96.5000,138.6000) -- (96.5000,138.6000) -- (96.5000,138.6000) -- (96.5000,138.6000) -- (96.5000,138.6000) -- (96.5000,138.6000) -- (96.5000,138.6000) -- (96.5000,138.5000) -- (96.5000,138.5000) -- (96.5000,138.5000) -- (96.5000,138.5000) -- (96.5000,138.5000) -- (96.5000,138.5000) -- (96.5000,138.5000) -- (96.5000,138.5000) -- (96.5000,138.5000) -- (96.5000,138.5000) -- (96.5000,138.5000) -- (96.5000,138.5000) -- (96.5000,138.5000) -- (96.5000,138.5000) -- (96.5000,138.5000) -- (96.5000,138.5000) -- (96.5000,138.5000) -- (96.5000,138.5000) -- (96.5000,138.5000) -- (96.5000,138.5000) -- (96.5000,138.5000) -- (96.5000,138.5000) -- (96.5000,138.4000) -- (96.5000,138.4000) -- (96.5000,138.4000) -- (96.5000,138.4000) -- (96.6000,138.4000) -- (96.6000,138.4000) -- (96.6000,138.4000) -- (96.6000,138.4000) -- (96.6000,138.4000) -- (96.6000,138.4000) -- (96.6000,138.4000) -- (96.6000,138.4000) -- (96.6000,138.4000) -- (96.6000,138.4000) -- (96.6000,138.4000) -- (96.6000,138.4000) -- (96.6000,138.4000) -- (96.6000,138.4000) -- (96.6000,138.4000) -- (96.6000,138.4000) -- (96.6000,138.4000) -- (96.6000,138.3000) -- (96.6000,138.3000) -- (96.6000,138.3000) -- (96.6000,138.3000) -- (96.6000,138.3000) -- (96.6000,138.3000) -- (96.6000,138.3000) -- (96.6000,138.3000) -- (96.6000,138.3000) -- (96.6000,138.3000) -- (96.6000,138.3000) -- (96.6000,138.3000) -- (96.6000,138.3000) -- (96.6000,138.3000) -- (96.6000,138.3000) -- (96.6000,138.3000) -- (96.6000,138.3000) -- (96.6000,138.3000) -- (96.6000,138.3000) -- (96.6000,138.3000) -- (96.6000,138.3000) -- (96.6000,138.3000) -- (96.6000,138.2000) -- (96.6000,138.2000) -- (96.6000,138.2000) -- (96.6000,138.2000) -- (96.6000,138.2000) -- (96.6000,138.2000) -- (96.6000,138.2000) -- (96.6000,138.2000) -- (96.6000,138.2000) -- (96.6000,138.2000) -- (96.7000,138.2000) -- (96.7000,138.2000) -- (96.7000,138.2000) -- (96.7000,138.2000) -- (96.7000,138.2000) -- (96.7000,138.2000) -- (96.7000,138.2000) -- (96.7000,138.2000) -- (96.7000,138.2000) -- (96.7000,138.2000) -- (96.7000,138.2000) -- (96.7000,138.1000) -- (96.7000,138.1000) -- (96.7000,138.1000) -- (96.7000,138.1000) -- (96.7000,138.1000) -- (96.7000,138.1000) -- (96.7000,138.1000) -- (96.7000,138.1000) -- (96.7000,138.1000) -- (96.7000,138.1000) -- (96.7000,138.1000) -- (96.7000,138.1000) -- (96.7000,138.1000) -- (96.7000,138.1000) -- (96.7000,138.1000) -- (96.7000,138.1000) -- (96.7000,138.1000) -- (96.7000,138.1000) -- (96.7000,138.1000) -- (96.7000,138.1000) -- (96.7000,138.1000) -- (96.7000,138.1000) -- (96.7000,138.0000) -- (96.7000,138.0000) -- (96.7000,138.0000) -- (96.7000,138.0000) -- (96.7000,138.0000) -- (96.7000,138.0000) -- (96.7000,138.0000) -- (96.7000,138.0000) -- (96.7000,138.0000) -- (96.7000,138.0000) -- (96.7000,138.0000) -- (96.7000,138.0000) -- (96.7000,138.0000) -- (96.7000,138.0000) -- (96.7000,138.0000) -- (96.7000,138.0000) -- (96.7000,138.0000) -- (96.8000,138.0000) -- (96.8000,138.0000) -- (96.8000,138.0000) -- (96.8000,138.0000) -- (96.8000,137.9000) -- (96.8000,137.9000) -- (96.8000,137.9000) -- (96.8000,137.9000) -- (96.8000,137.9000) -- (96.8000,137.9000) -- (96.8000,137.9000) -- (96.8000,137.9000) -- (96.8000,137.9000) -- (96.8000,137.9000) -- (96.8000,137.9000) -- (96.8000,137.9000) -- (96.8000,137.9000) -- (96.8000,137.9000) -- (96.8000,137.9000) -- (96.8000,137.9000) -- (96.8000,137.9000) -- (96.8000,137.9000) -- (96.8000,137.9000) -- (96.8000,137.9000) -- (96.8000,137.9000) -- (96.8000,137.9000) -- (96.8000,137.8000) -- (96.8000,137.8000) -- (96.8000,137.8000) -- (96.8000,137.8000) -- (96.8000,137.8000) -- (96.8000,137.8000) -- (96.8000,137.8000) -- (96.8000,137.8000) -- (96.8000,137.8000) -- (96.8000,137.8000) -- (96.8000,137.8000) -- (96.8000,137.8000) -- (96.8000,137.8000) -- (96.8000,137.8000) -- (96.8000,137.8000) -- (96.8000,137.8000) -- (96.8000,137.8000) -- (96.8000,137.8000) -- (96.8000,137.8000) -- (96.8000,137.8000) -- (96.8000,137.8000) -- (96.8000,137.7000) -- (96.8000,137.7000) -- (96.9000,137.7000) -- (96.9000,137.7000) -- (96.9000,137.7000) -- (96.9000,137.7000) -- (96.9000,137.7000) -- (96.9000,137.7000) -- (96.9000,137.7000) -- (96.9000,137.7000) -- (96.9000,137.7000) -- (96.9000,137.7000) -- (96.9000,137.7000) -- (96.9000,137.7000) -- (96.9000,137.7000) -- (96.9000,137.7000) -- (96.9000,137.7000) -- (96.9000,137.7000) -- (96.9000,137.7000) -- (96.9000,137.7000) -- (96.9000,137.7000) -- (96.9000,137.7000) -- (96.9000,137.6000) -- (96.9000,137.6000) -- (96.9000,137.6000) -- (96.9000,137.6000) -- (96.9000,137.6000) -- (96.9000,137.6000) -- (96.9000,137.6000) -- (96.9000,137.6000) -- (96.9000,137.6000) -- (96.9000,137.6000) -- (96.9000,137.6000) -- (96.9000,137.6000) -- (96.9000,137.6000) -- (96.9000,137.6000) -- (96.9000,137.6000) -- (96.9000,137.6000) -- (96.9000,137.6000) -- (96.9000,137.6000) -- (96.9000,137.6000) -- (96.9000,137.6000) -- (96.9000,137.6000) -- (96.9000,137.5000) -- (96.9000,137.5000) -- (96.9000,137.5000) -- (96.9000,137.5000) -- (96.9000,137.5000) -- (96.9000,137.5000) -- (96.9000,137.5000) -- (96.9000,137.5000) -- (96.9000,137.5000) -- (97.0000,137.5000) -- (97.0000,137.5000) -- (97.0000,137.5000) -- (97.0000,137.5000) -- (97.0000,137.5000) -- (97.0000,137.5000) -- (97.0000,137.5000) -- (97.0000,137.5000) -- (97.0000,137.5000) -- (97.0000,137.5000) -- (97.0000,137.5000) -- (97.0000,137.5000) -- (97.0000,137.5000) -- (97.0000,137.4000) -- (97.0000,137.4000) -- (97.0000,137.4000) -- (97.0000,137.4000) -- (97.0000,137.4000) -- (97.0000,137.4000) -- (97.0000,137.4000) -- (97.0000,137.4000) -- (97.0000,137.4000) -- (97.0000,137.4000) -- (97.0000,137.4000) -- (97.0000,137.4000) -- (97.0000,137.4000) -- (97.0000,137.4000) -- (97.0000,137.4000) -- (97.0000,137.4000) -- (97.0000,137.4000) -- (97.0000,137.4000) -- (97.0000,137.4000) -- (97.0000,137.4000) -- (97.0000,137.4000) -- (97.0000,137.3000) -- (97.0000,137.3000) -- (97.0000,137.3000) -- (97.0000,137.3000) -- (97.0000,137.3000) -- (97.0000,137.3000) -- (97.0000,137.3000) -- (97.0000,137.3000) -- (97.0000,137.3000) -- (97.0000,137.3000) -- (97.0000,137.3000) -- (97.0000,137.3000) -- (97.0000,137.3000) -- (97.0000,137.3000) -- (97.0000,137.3000) -- (97.1000,137.3000) -- (97.1000,137.3000) -- (97.1000,137.3000) -- (97.1000,137.3000) -- (97.1000,137.3000) -- (97.1000,137.3000) -- (97.1000,137.3000) -- (97.1000,137.2000) -- (97.1000,137.2000) -- (97.1000,137.2000) -- (97.1000,137.2000) -- (97.1000,137.2000) -- (97.1000,137.2000) -- (97.1000,137.2000) -- (97.1000,137.2000) -- (97.1000,137.2000) -- (97.1000,137.2000) -- (97.1000,137.2000) -- (97.1000,137.2000) -- (97.1000,137.2000) -- (97.1000,137.2000) -- (97.1000,137.2000) -- (97.1000,137.2000) -- (97.1000,137.2000) -- (97.1000,137.2000) -- (97.1000,137.2000) -- (97.1000,137.2000) -- (97.1000,137.2000) -- (97.1000,137.1000) -- (97.1000,137.1000) -- (97.1000,137.1000) -- (97.1000,137.1000) -- (97.1000,137.1000) -- (97.1000,137.1000) -- (97.1000,137.1000) -- (97.1000,137.1000) -- (97.1000,137.1000) -- (97.1000,137.1000) -- (97.1000,137.1000) -- (97.1000,137.1000) -- (97.1000,137.1000) -- (97.1000,137.1000) -- (97.1000,137.1000) -- (97.1000,137.1000) -- (97.1000,137.1000) -- (97.1000,137.1000) -- (97.1000,137.1000) -- (97.1000,137.1000) -- (97.1000,137.1000) -- (97.1000,137.1000) -- (97.2000,137.0000) -- (97.2000,137.0000) -- (97.2000,137.0000) -- (97.2000,137.0000) -- (97.2000,137.0000) -- (97.2000,137.0000) -- (97.2000,137.0000) -- (97.2000,137.0000) -- (97.2000,137.0000) -- (97.2000,137.0000) -- (97.2000,137.0000) -- (97.2000,137.0000) -- (97.2000,137.0000) -- (97.2000,137.0000) -- (97.2000,137.0000) -- (97.2000,137.0000) -- (97.2000,137.0000) -- (97.2000,137.0000) -- (97.2000,137.0000) -- (97.2000,137.0000) -- (97.2000,137.0000) -- (97.2000,136.9000) -- (97.2000,136.9000) -- (97.2000,136.9000) -- (97.2000,136.9000) -- (97.2000,136.9000) -- (97.2000,136.9000) -- (97.2000,136.9000) -- (97.2000,136.9000) -- (97.2000,136.9000) -- (97.2000,136.9000) -- (97.2000,136.9000) -- (97.2000,136.9000) -- (97.2000,136.9000) -- (97.2000,136.9000) -- (97.2000,136.9000) -- (97.2000,136.9000) -- (97.2000,136.9000) -- (97.2000,136.9000) -- (97.2000,136.9000) -- (97.2000,136.9000) -- (97.2000,136.9000) -- (97.2000,136.9000) -- (97.2000,136.8000) -- (97.2000,136.8000) -- (97.2000,136.8000) -- (97.2000,136.8000) -- (97.2000,136.8000) -- (97.2000,136.8000) -- (97.3000,136.8000) -- (97.3000,136.8000) -- (97.3000,136.8000) -- (97.3000,136.8000) -- (97.3000,136.8000) -- (97.3000,136.8000) -- (97.3000,136.8000) -- (97.3000,136.8000) -- (97.3000,136.8000) -- (97.3000,136.8000) -- (97.3000,136.8000) -- (97.3000,136.8000) -- (97.3000,136.8000) -- (97.3000,136.8000) -- (97.3000,136.8000) -- (97.3000,136.7000) -- (97.3000,136.7000) -- (97.3000,136.7000) -- (97.3000,136.7000) -- (97.3000,136.7000) -- (97.3000,136.7000) -- (97.3000,136.7000) -- (97.3000,136.7000) -- (97.3000,136.7000) -- (97.3000,136.7000) -- (97.3000,136.7000) -- (97.3000,136.7000) -- (97.3000,136.7000) -- (97.3000,136.7000) -- (97.3000,136.7000) -- (97.3000,136.7000) -- (97.3000,136.7000) -- (97.3000,136.7000) -- (97.3000,136.7000) -- (97.3000,136.7000) -- (97.3000,136.7000) -- (97.3000,136.7000) -- (97.3000,136.6000) -- (97.3000,136.6000) -- (97.3000,136.6000) -- (97.3000,136.6000) -- (97.3000,136.6000) -- (97.3000,136.6000) -- (97.3000,136.6000) -- (97.3000,136.6000) -- (97.3000,136.6000) -- (97.3000,136.6000) -- (97.3000,136.6000) -- (97.3000,136.6000) -- (97.3000,136.6000) -- (97.4000,136.6000) -- (97.4000,136.6000) -- (97.4000,136.6000) -- (97.4000,136.6000) -- (97.4000,136.6000) -- (97.4000,136.6000) -- (97.4000,136.6000) -- (97.4000,136.6000) -- (97.4000,136.5000) -- (97.4000,136.5000) -- (97.4000,136.5000) -- (97.4000,136.5000) -- (97.4000,136.5000) -- (97.4000,136.5000) -- (97.4000,136.5000) -- (97.4000,136.5000) -- (97.4000,136.5000) -- (97.4000,136.5000) -- (97.4000,136.5000) -- (97.4000,136.5000) -- (97.4000,136.5000) -- (97.4000,136.5000) -- (97.4000,136.5000) -- (97.4000,136.5000) -- (97.4000,136.5000) -- (97.4000,136.5000) -- (97.4000,136.5000) -- (97.4000,136.5000) -- (97.4000,136.5000) -- (97.4000,136.5000) -- (97.4000,136.4000) -- (97.4000,136.4000) -- (97.4000,136.4000) -- (97.4000,136.4000) -- (97.4000,136.4000) -- (97.4000,136.4000) -- (97.4000,136.4000) -- (97.4000,136.4000) -- (97.4000,136.4000) -- (97.4000,136.4000) -- (97.4000,136.4000) -- (97.4000,136.4000) -- (97.4000,136.4000) -- (97.4000,136.4000) -- (97.4000,136.4000) -- (97.4000,136.4000) -- (97.4000,136.4000) -- (97.4000,136.4000) -- (97.4000,136.4000) -- (97.5000,136.4000) -- (97.5000,136.4000) -- (97.5000,136.3000) -- (97.5000,136.3000) -- (97.5000,136.3000) -- (97.5000,136.3000) -- (97.5000,136.3000) -- (97.5000,136.3000) -- (97.5000,136.3000) -- (97.5000,136.3000) -- (97.5000,136.3000) -- (97.5000,136.3000) -- (97.5000,136.3000) -- (97.5000,136.3000) -- (97.5000,136.3000) -- (97.5000,136.3000) -- (97.5000,136.3000) -- (97.5000,136.3000) -- (97.5000,136.3000) -- (97.5000,136.3000) -- (97.5000,136.3000) -- (97.5000,136.3000) -- (97.5000,136.3000) -- (97.5000,136.3000) -- (97.5000,136.2000) -- (97.5000,136.2000) -- (97.5000,136.2000) -- (97.5000,136.2000) -- (97.5000,136.2000) -- (97.5000,136.2000) -- (97.5000,136.2000) -- (97.5000,136.2000) -- (97.5000,136.2000) -- (97.5000,136.2000) -- (97.5000,136.2000) -- (97.5000,136.2000) -- (97.5000,136.2000) -- (97.5000,136.2000) -- (97.5000,136.2000) -- (97.5000,136.2000) -- (97.5000,136.2000) -- (97.5000,136.2000) -- (97.5000,136.2000) -- (97.5000,136.2000) -- (97.5000,136.2000) -- (97.5000,136.1000) -- (97.5000,136.1000) -- (97.5000,136.1000) -- (97.5000,136.1000) -- (97.5000,136.1000) -- (97.6000,136.1000) -- (97.6000,136.1000) -- (97.6000,136.1000) -- (97.6000,136.1000) -- (97.6000,136.1000) -- (97.6000,136.1000) -- (97.6000,136.1000) -- (97.6000,136.1000) -- (97.6000,136.1000) -- (97.6000,136.1000) -- (97.6000,136.1000) -- (97.6000,136.1000) -- (97.6000,136.1000) -- (97.6000,136.1000) -- (97.6000,136.1000) -- (97.6000,136.1000) -- (97.6000,136.1000) -- (97.6000,136.0000) -- (97.6000,136.0000) -- (97.6000,136.0000) -- (97.6000,136.0000) -- (97.6000,136.0000) -- (97.6000,136.0000) -- (97.6000,136.0000) -- (97.6000,136.0000) -- (97.6000,136.0000) -- (97.6000,136.0000) -- (97.6000,136.0000) -- (97.6000,136.0000) -- (97.6000,136.0000) -- (97.6000,136.0000) -- (97.6000,136.0000) -- (97.6000,136.0000) -- (97.6000,136.0000) -- (97.6000,136.0000) -- (97.6000,136.0000) -- (97.6000,136.0000) -- (97.6000,136.0000) -- (97.6000,135.9000) -- (97.6000,135.9000) -- (97.6000,135.9000) -- (97.6000,135.9000) -- (97.6000,135.9000) -- (97.6000,135.9000) -- (97.6000,135.9000) -- (97.6000,135.9000) -- (97.6000,135.9000) -- (97.6000,135.9000) -- (97.6000,135.9000) -- (97.7000,135.9000) -- (97.7000,135.9000) -- (97.7000,135.9000) -- (97.7000,135.9000) -- (97.7000,135.9000) -- (97.7000,135.9000) -- (97.7000,135.9000) -- (97.7000,135.9000) -- (97.7000,135.9000) -- (97.7000,135.9000) -- (97.7000,135.9000) -- (97.7000,135.8000) -- (97.7000,135.8000) -- (97.7000,135.8000) -- (97.7000,135.8000) -- (97.7000,135.8000) -- (97.7000,135.8000) -- (97.7000,135.8000) -- (97.7000,135.8000) -- (97.7000,135.8000) -- (97.7000,135.8000) -- (97.7000,135.8000) -- (97.7000,135.8000) -- (97.7000,135.8000) -- (97.7000,135.8000) -- (97.7000,135.8000) -- (97.7000,135.8000) -- (97.7000,135.8000) -- (97.7000,135.8000) -- (97.7000,135.8000) -- (97.7000,135.8000) -- (97.7000,135.8000) -- (97.7000,135.7000) -- (97.7000,135.7000) -- (97.7000,135.7000) -- (97.7000,135.7000) -- (97.7000,135.7000) -- (97.7000,135.7000) -- (97.7000,135.7000) -- (97.7000,135.7000) -- (97.7000,135.7000) -- (97.7000,135.7000) -- (97.7000,135.7000) -- (97.7000,135.7000) -- (97.7000,135.7000) -- (97.7000,135.7000) -- (97.7000,135.7000) -- (97.7000,135.7000) -- (97.7000,135.7000) -- (97.7000,135.7000) -- (97.8000,135.7000) -- (97.8000,135.7000) -- (97.8000,135.7000) -- (97.8000,135.7000) -- (97.8000,135.6000) -- (97.8000,135.6000) -- (97.8000,135.6000) -- (97.8000,135.6000) -- (97.8000,135.6000) -- (97.8000,135.6000) -- (97.8000,135.6000) -- (97.8000,135.6000) -- (97.8000,135.6000) -- (97.8000,135.6000) -- (97.8000,135.6000) -- (97.8000,135.6000) -- (97.8000,135.6000) -- (97.8000,135.6000) -- (97.8000,135.6000) -- (97.8000,135.6000) -- (97.8000,135.6000) -- (97.8000,135.6000) -- (97.8000,135.6000) -- (97.8000,135.6000) -- (97.8000,135.6000) -- (97.8000,135.5000) -- (97.8000,135.5000) -- (97.8000,135.5000) -- (97.8000,135.5000) -- (97.8000,135.5000) -- (97.8000,135.5000) -- (97.8000,135.5000) -- (97.8000,135.5000) -- (97.8000,135.5000) -- (97.8000,135.5000) -- (97.8000,135.5000) -- (97.8000,135.5000) -- (97.8000,135.5000) -- (97.8000,135.5000) -- (97.8000,135.5000) -- (97.8000,135.5000) -- (97.8000,135.5000) -- (97.8000,135.5000) -- (97.8000,135.5000) -- (97.8000,135.5000) -- (97.8000,135.5000) -- (97.8000,135.5000) -- (97.8000,135.4000) -- (97.8000,135.4000) -- (97.9000,135.4000) -- (97.9000,135.4000) -- (97.9000,135.4000) -- (97.9000,135.4000) -- (97.9000,135.4000) -- (97.9000,135.4000) -- (97.9000,135.4000) -- (97.9000,135.4000) -- (97.9000,135.4000) -- (97.9000,135.4000) -- (97.9000,135.4000) -- (97.9000,135.4000) -- (97.9000,135.4000) -- (97.9000,135.4000) -- (97.9000,135.4000) -- (97.9000,135.4000) -- (97.9000,135.4000) -- (97.9000,135.4000) -- (97.9000,135.4000) -- (97.9000,135.3000) -- (97.9000,135.3000) -- (97.9000,135.3000) -- (97.9000,135.3000) -- (97.9000,135.3000) -- (97.9000,135.3000) -- (97.9000,135.3000) -- (97.9000,135.3000) -- (97.9000,135.3000) -- (97.9000,135.3000) -- (97.9000,135.3000) -- (97.9000,135.3000) -- (97.9000,135.3000) -- (107.3000,135.3000) -- (107.3000,135.3000) -- (107.3000,135.3000) -- (107.4000,135.3000) -- (121.2657,135.3000);



    \end{scope}
    \begin{scope}[cm={{1.0018,0.0,0.0,0.97485,(-208.78685,-65.87769)}},draw=black,line join=round,line cap=round,line width=0.480pt]
      \path[draw] (81.5000,129.5000) -- (81.5000,157.5000) -- (121.5000,157.5000) -- (121.5000,129.5000) -- (81.5000,129.5000);



    \end{scope}
    \path[cm={{1.0018,0.0,0.0,0.97485,(-208.78685,-65.87769)}},draw=black] (41.5000,88.5000) -- (41.5000,164.5000) -- (127.5000,164.5000) -- (127.5000,88.5000) -- (41.5000,88.5000);



  \end{scope}
  \begin{scope}[cm={{1.00588,0.0,0.0,1.00588,(-444.19503,-9.0)}},draw=ca0a0a4,dash pattern=on 0.40pt off 0.80pt,line join=round,line cap=round,line width=0.400pt]
    \path[draw] (165.5000,88.5000) -- (251.5000,88.5000);



  \end{scope}
  \begin{scope}[cm={{1.00588,0.0,0.0,1.00588,(-444.19503,-9.0)}},draw=black,line join=round,line cap=round,line width=0.480pt]
    \path[draw] (165.5000,88.5000) -- (168.5000,88.5000);



    \path[draw] (251.5000,88.5000) -- (248.5000,88.5000);



  \end{scope}
  \begin{scope}[scale=1.006,draw=black,line join=bevel,line cap=rect,line width=0.800pt]
  \end{scope}
  \begin{scope}[cm={{1.00588,0.0,0.0,1.00588,(148.871,93.5471)}},draw=black,line join=bevel,line cap=rect,line width=0.800pt]
  \end{scope}
  \begin{scope}[cm={{1.00588,0.0,0.0,1.00588,(148.871,93.5471)}},draw=black,line join=bevel,line cap=rect,line width=0.800pt]
  \end{scope}
  \begin{scope}[cm={{1.00588,0.0,0.0,1.00588,(148.871,93.5471)}},draw=black,line join=bevel,line cap=rect,line width=0.800pt]
  \end{scope}
  \begin{scope}[cm={{1.00588,0.0,0.0,1.00588,(148.871,93.5471)}},draw=black,line join=bevel,line cap=rect,line width=0.800pt]
  \end{scope}
  \begin{scope}[cm={{1.00588,0.0,0.0,1.00588,(148.871,93.5471)}},draw=black,line join=bevel,line cap=rect,line width=0.800pt]
  \end{scope}
  \begin{scope}[cm={{1.00588,0.0,0.0,1.00588,(-293.82403,83.0471)}},draw=black,line join=bevel,line cap=rect,line width=0.800pt]
    \path[fill=black] (0.0000,0.0000) node[above right] (text380) {27};



  \end{scope}
  \begin{scope}[cm={{1.00588,0.0,0.0,1.00588,(148.871,93.5471)}},draw=black,line join=bevel,line cap=rect,line width=0.800pt]
  \end{scope}
  \begin{scope}[scale=1.006,draw=black,line join=bevel,line cap=rect,line width=0.800pt]
  \end{scope}
  \begin{scope}[cm={{1.00588,0.0,0.0,1.00588,(-444.19503,-9.0)}},draw=ca0a0a4,dash pattern=on 0.40pt off 0.80pt,line join=round,line cap=round,line width=0.400pt]
    \path[draw] (165.5000,63.5000) -- (251.5000,63.5000);



  \end{scope}
  \begin{scope}[cm={{1.00588,0.0,0.0,1.00588,(-444.19503,-9.0)}},draw=black,line join=round,line cap=round,line width=0.480pt]
    \path[draw] (165.5000,63.5000) -- (168.5000,63.5000);



    \path[draw] (251.5000,63.5000) -- (248.5000,63.5000);



  \end{scope}
  \begin{scope}[scale=1.006,draw=black,line join=bevel,line cap=rect,line width=0.800pt]
  \end{scope}
  \begin{scope}[cm={{1.00588,0.0,0.0,1.00588,(149.876,68.4)}},draw=black,line join=bevel,line cap=rect,line width=0.800pt]
  \end{scope}
  \begin{scope}[cm={{1.00588,0.0,0.0,1.00588,(149.876,68.4)}},draw=black,line join=bevel,line cap=rect,line width=0.800pt]
  \end{scope}
  \begin{scope}[cm={{1.00588,0.0,0.0,1.00588,(149.876,68.4)}},draw=black,line join=bevel,line cap=rect,line width=0.800pt]
  \end{scope}
  \begin{scope}[cm={{1.00588,0.0,0.0,1.00588,(149.876,68.4)}},draw=black,line join=bevel,line cap=rect,line width=0.800pt]
  \end{scope}
  \begin{scope}[cm={{1.00588,0.0,0.0,1.00588,(149.876,68.4)}},draw=black,line join=bevel,line cap=rect,line width=0.800pt]
  \end{scope}
  \begin{scope}[cm={{1.00588,0.0,0.0,1.00588,(-293.38144,57.9)}},draw=black,line join=bevel,line cap=rect,line width=0.800pt]
    \path[fill=black] (0.0000,0.0000) node[above right] (text410) {31};



  \end{scope}
  \begin{scope}[cm={{1.00588,0.0,0.0,1.00588,(149.876,68.4)}},draw=black,line join=bevel,line cap=rect,line width=0.800pt]
  \end{scope}
  \begin{scope}[scale=1.006,draw=black,line join=bevel,line cap=rect,line width=0.800pt]
  \end{scope}
  \begin{scope}[cm={{1.00588,0.0,0.0,1.00588,(-334.55411,-9.0)}},draw=ca0a0a4,dash pattern=on 0.40pt off 0.80pt,line join=round,line cap=round,line width=0.400pt]
    \path[draw] (108.5000,95.5000) -- (108.5000,36.5000);



    \path[draw] (108.5000,20.5000) -- (108.5000,13.5000);



  \end{scope}
  \begin{scope}[cm={{1.00588,0.0,0.0,1.00588,(-444.19503,-9.0)}},draw=ca0a0a4,dash pattern=on 0.40pt off 0.80pt,line join=round,line cap=round,line width=0.400pt]
    \path[draw] (165.5000,38.5000) -- (251.5000,38.5000);



  \end{scope}
  \begin{scope}[cm={{1.00588,0.0,0.0,1.00588,(-444.19503,-9.0)}},draw=black,line join=round,line cap=round,line width=0.480pt]
    \path[draw] (165.5000,38.5000) -- (168.5000,38.5000);



    \path[draw] (251.5000,38.5000) -- (248.5000,38.5000);



  \end{scope}
  \begin{scope}[scale=1.006,draw=black,line join=bevel,line cap=rect,line width=0.800pt]
  \end{scope}
  \begin{scope}[cm={{1.00588,0.0,0.0,1.00588,(149.876,43.2529)}},draw=black,line join=bevel,line cap=rect,line width=0.800pt]
  \end{scope}
  \begin{scope}[cm={{1.00588,0.0,0.0,1.00588,(149.876,43.2529)}},draw=black,line join=bevel,line cap=rect,line width=0.800pt]
  \end{scope}
  \begin{scope}[cm={{1.00588,0.0,0.0,1.00588,(149.876,43.2529)}},draw=black,line join=bevel,line cap=rect,line width=0.800pt]
  \end{scope}
  \begin{scope}[cm={{1.00588,0.0,0.0,1.00588,(149.876,43.2529)}},draw=black,line join=bevel,line cap=rect,line width=0.800pt]
  \end{scope}
  \begin{scope}[cm={{1.00588,0.0,0.0,1.00588,(149.876,43.2529)}},draw=black,line join=bevel,line cap=rect,line width=0.800pt]
  \end{scope}
  \begin{scope}[cm={{1.00588,0.0,0.0,1.00588,(-293.73551,32.7529)}},draw=black,line join=bevel,line cap=rect,line width=0.800pt]
    \path[fill=black] (0.0000,0.0000) node[above right] (text440) {35};



  \end{scope}
  \begin{scope}[cm={{1.00588,0.0,0.0,1.00588,(149.876,43.2529)}},draw=black,line join=bevel,line cap=rect,line width=0.800pt]
  \end{scope}
  \begin{scope}[scale=1.006,draw=black,line join=bevel,line cap=rect,line width=0.800pt]
  \end{scope}
  \begin{scope}[cm={{1.00588,0.0,0.0,1.00588,(-444.19503,-9.0)}},draw=ca0a0a4,dash pattern=on 0.40pt off 0.80pt,line join=round,line cap=round,line width=0.400pt]
    \path[draw] (165.5000,95.5000) -- (165.5000,13.5000);



  \end{scope}
  \begin{scope}[cm={{1.00588,0.0,0.0,1.00588,(-444.19503,-9.0)}},draw=black,line join=round,line cap=round,line width=0.480pt]
    \path[draw] (165.5000,95.5000) -- (165.5000,92.5000);



    \path[draw] (165.5000,13.5000) -- (165.5000,16.5000);



  \end{scope}
  \begin{scope}[scale=1.006,draw=black,line join=bevel,line cap=rect,line width=0.800pt]
  \end{scope}
  \begin{scope}[cm={{1.00588,0.0,0.0,1.00588,(162.953,110.647)}},draw=black,line join=bevel,line cap=rect,line width=0.800pt]
  \end{scope}
  \begin{scope}[cm={{1.00588,0.0,0.0,1.00588,(162.953,110.647)}},draw=black,line join=bevel,line cap=rect,line width=0.800pt]
  \end{scope}
  \begin{scope}[cm={{1.00588,0.0,0.0,1.00588,(162.953,110.647)}},draw=black,line join=bevel,line cap=rect,line width=0.800pt]
  \end{scope}
  \begin{scope}[cm={{1.00588,0.0,0.0,1.00588,(162.953,110.647)}},draw=black,line join=bevel,line cap=rect,line width=0.800pt]
  \end{scope}
  \begin{scope}[cm={{1.00588,0.0,0.0,1.00588,(162.953,110.647)}},draw=black,line join=bevel,line cap=rect,line width=0.800pt]
  \end{scope}
  \begin{scope}[cm={{1.00588,0.0,0.0,1.00588,(-280.18137,98.647)}},draw=black,line join=bevel,line cap=rect,line width=0.800pt]
    \path[fill=black] (0.0000,0.0000) node[above right] (text470) {0};



  \end{scope}
  \begin{scope}[cm={{1.00588,0.0,0.0,1.00588,(162.953,110.647)}},draw=black,line join=bevel,line cap=rect,line width=0.800pt]
  \end{scope}
  \begin{scope}[scale=1.006,draw=black,line join=bevel,line cap=rect,line width=0.800pt]
  \end{scope}
  \begin{scope}[cm={{1.00588,0.0,0.0,1.00588,(-223.64094,89.46968)}},draw=ca0a0a4,dash pattern=on 0.40pt off 0.80pt,line join=round,line cap=round,line width=0.400pt]
    \path[draw] (191.5000,95.5000) -- (191.5000,13.5000);



  \end{scope}
  \begin{scope}[cm={{1.00588,0.0,0.0,1.00588,(-334.55411,-9.0)}},draw=ca0a0a4,dash pattern=on 0.40pt off 0.80pt,line join=round,line cap=round,line width=0.400pt]
    \path[draw] (82.5000,95.5000) -- (82.5000,36.5000);



    \path[draw] (82.5000,20.5000) -- (82.5000,13.5000);



  \end{scope}
  \begin{scope}[cm={{1.00588,0.0,0.0,1.00588,(-444.19503,-9.0)}},draw=black,line join=round,line cap=round,line width=0.480pt]
    \path[draw] (191.5000,95.5000) -- (191.5000,92.5000);



    \path[draw] (191.5000,13.5000) -- (191.5000,16.5000);



  \end{scope}
  \begin{scope}[scale=1.006,draw=black,line join=bevel,line cap=rect,line width=0.800pt]
  \end{scope}
  \begin{scope}[cm={{1.00588,0.0,0.0,1.00588,(189.106,110.647)}},draw=black,line join=bevel,line cap=rect,line width=0.800pt]
  \end{scope}
  \begin{scope}[cm={{1.00588,0.0,0.0,1.00588,(189.106,110.647)}},draw=black,line join=bevel,line cap=rect,line width=0.800pt]
  \end{scope}
  \begin{scope}[cm={{1.00588,0.0,0.0,1.00588,(189.106,110.647)}},draw=black,line join=bevel,line cap=rect,line width=0.800pt]
  \end{scope}
  \begin{scope}[cm={{1.00588,0.0,0.0,1.00588,(189.106,110.647)}},draw=black,line join=bevel,line cap=rect,line width=0.800pt]
  \end{scope}
  \begin{scope}[cm={{1.00588,0.0,0.0,1.00588,(189.106,110.647)}},draw=black,line join=bevel,line cap=rect,line width=0.800pt]
  \end{scope}
  \begin{scope}[cm={{1.00588,0.0,0.0,1.00588,(-254.02837,98.647)}},draw=black,line join=bevel,line cap=rect,line width=0.800pt]
    \path[fill=black] (0.0000,0.0000) node[above right] (text500) {1};



  \end{scope}
  \begin{scope}[cm={{1.00588,0.0,0.0,1.00588,(189.106,110.647)}},draw=black,line join=bevel,line cap=rect,line width=0.800pt]
  \end{scope}
  \begin{scope}[scale=1.006,draw=black,line join=bevel,line cap=rect,line width=0.800pt]
  \end{scope}
  \begin{scope}[cm={{1.00588,0.0,0.0,1.00588,(-223.64094,89.46968)}},draw=ca0a0a4,dash pattern=on 0.40pt off 0.80pt,line join=round,line cap=round,line width=0.400pt]
    \path[draw] (217.5000,95.5000) -- (217.5000,13.5000);



  \end{scope}
  \begin{scope}[cm={{1.00588,0.0,0.0,1.00588,(-444.19503,-9.0)}},draw=black,line join=round,line cap=round,line width=0.480pt]
    \path[draw] (217.5000,95.5000) -- (217.5000,92.5000);



    \path[draw] (217.5000,13.5000) -- (217.5000,16.5000);



  \end{scope}
  \begin{scope}[scale=1.006,draw=black,line join=bevel,line cap=rect,line width=0.800pt]
  \end{scope}
  \begin{scope}[cm={{1.00588,0.0,0.0,1.00588,(215.259,110.647)}},draw=black,line join=bevel,line cap=rect,line width=0.800pt]
  \end{scope}
  \begin{scope}[cm={{1.00588,0.0,0.0,1.00588,(215.259,110.647)}},draw=black,line join=bevel,line cap=rect,line width=0.800pt]
  \end{scope}
  \begin{scope}[cm={{1.00588,0.0,0.0,1.00588,(215.259,110.647)}},draw=black,line join=bevel,line cap=rect,line width=0.800pt]
  \end{scope}
  \begin{scope}[cm={{1.00588,0.0,0.0,1.00588,(215.259,110.647)}},draw=black,line join=bevel,line cap=rect,line width=0.800pt]
  \end{scope}
  \begin{scope}[cm={{1.00588,0.0,0.0,1.00588,(215.259,110.647)}},draw=black,line join=bevel,line cap=rect,line width=0.800pt]
  \end{scope}
  \begin{scope}[cm={{1.00588,0.0,0.0,1.00588,(-227.87533,98.647)}},draw=black,line join=bevel,line cap=rect,line width=0.800pt]
    \path[fill=black] (0.0000,0.0000) node[above right] (text530) {2};



  \end{scope}
  \begin{scope}[cm={{1.00588,0.0,0.0,1.00588,(215.259,110.647)}},draw=black,line join=bevel,line cap=rect,line width=0.800pt]
  \end{scope}
  \begin{scope}[scale=1.006,draw=black,line join=bevel,line cap=rect,line width=0.800pt]
  \end{scope}
  \begin{scope}[cm={{1.00588,0.0,0.0,1.00588,(-444.19503,-9.0)}},draw=ca0a0a4,dash pattern=on 0.40pt off 0.80pt,line join=round,line cap=round,line width=0.400pt]
    \path[draw] (243.5000,95.5000) -- (243.5000,13.5000);



  \end{scope}
  \begin{scope}[cm={{1.00588,0.0,0.0,1.00588,(-444.19503,-9.0)}},draw=black,line join=round,line cap=round,line width=0.480pt]
    \path[draw] (243.5000,95.5000) -- (243.5000,92.5000);



    \path[draw] (243.5000,13.5000) -- (243.5000,16.5000);



  \end{scope}
  \begin{scope}[scale=1.006,draw=black,line join=bevel,line cap=rect,line width=0.800pt]
  \end{scope}
  \begin{scope}[cm={{1.00588,0.0,0.0,1.00588,(241.915,110.647)}},draw=black,line join=bevel,line cap=rect,line width=0.800pt]
  \end{scope}
  \begin{scope}[cm={{1.00588,0.0,0.0,1.00588,(241.915,110.647)}},draw=black,line join=bevel,line cap=rect,line width=0.800pt]
  \end{scope}
  \begin{scope}[cm={{1.00588,0.0,0.0,1.00588,(241.915,110.647)}},draw=black,line join=bevel,line cap=rect,line width=0.800pt]
  \end{scope}
  \begin{scope}[cm={{1.00588,0.0,0.0,1.00588,(241.915,110.647)}},draw=black,line join=bevel,line cap=rect,line width=0.800pt]
  \end{scope}
  \begin{scope}[cm={{1.00588,0.0,0.0,1.00588,(241.915,110.647)}},draw=black,line join=bevel,line cap=rect,line width=0.800pt]
  \end{scope}
  \begin{scope}[cm={{1.00588,0.0,0.0,1.00588,(-201.21933,98.647)}},draw=black,line join=bevel,line cap=rect,line width=0.800pt]
    \path[fill=black] (0.0000,0.0000) node[above right] (text560) {3};



  \end{scope}
  \begin{scope}[cm={{1.00588,0.0,0.0,1.00588,(241.915,110.647)}},draw=black,line join=bevel,line cap=rect,line width=0.800pt]
  \end{scope}
  \begin{scope}[scale=1.006,draw=black,line join=bevel,line cap=rect,line width=0.800pt]
  \end{scope}
  \begin{scope}[cm={{1.00588,0.0,0.0,1.00588,(-444.19503,-9.0)}},draw=black,line join=round,line cap=round,line width=0.480pt]
    \path[draw] (165.5000,13.5000) -- (165.5000,95.5000) -- (251.5000,95.5000) -- (251.5000,13.5000) -- (165.5000,13.5000);



  \end{scope}
  \begin{scope}[scale=1.006,draw=black,line join=bevel,line cap=rect,line width=0.800pt]
  \end{scope}
  \begin{scope}[scale=1.006,draw=black,line join=bevel,line cap=rect,line width=0.800pt]
  \end{scope}
  \begin{scope}[cm={{1.00588,0.0,0.0,1.00588,(-444.19503,-9.0)}},fill=cebebeb]
    \path[fill=cebebeb,rounded corners=0.0000cm] (229.0000,17.0000) rectangle (245.0000,33.0000);



  \end{scope}
  \begin{scope}[scale=1.006,draw=black,line join=bevel,line cap=rect,line width=0.800pt]
  \end{scope}
  \begin{scope}[scale=1.006,draw=black,line join=bevel,line cap=rect,line width=0.800pt]
  \end{scope}
  \begin{scope}[cm={{1.00588,0.0,0.0,1.00588,(-444.19503,-9.0)}},draw=black,line join=round,line cap=round,line width=0.800pt]
    \path[draw] (229.5000,33.5000) -- (229.5000,17.5000) -- (245.5000,17.5000) -- (245.5000,33.5000) -- (229.5000,33.5000);



  \end{scope}
  \begin{scope}[scale=1.006,draw=black,line join=bevel,line cap=rect,line width=0.800pt]
  \end{scope}
  \begin{scope}[cm={{1.00588,0.0,0.0,1.00588,(235.376,29.1706)}},draw=black,line join=bevel,line cap=rect,line width=0.800pt]
  \end{scope}
  \begin{scope}[cm={{1.00588,0.0,0.0,1.00588,(235.376,29.1706)}},draw=black,line join=bevel,line cap=rect,line width=0.800pt]
  \end{scope}
  \begin{scope}[cm={{1.00588,0.0,0.0,1.00588,(235.376,29.1706)}},draw=black,line join=bevel,line cap=rect,line width=0.800pt]
  \end{scope}
  \begin{scope}[cm={{1.00588,0.0,0.0,1.00588,(235.376,29.1706)}},draw=black,line join=bevel,line cap=rect,line width=0.800pt]
  \end{scope}
  \begin{scope}[cm={{1.00588,0.0,0.0,1.00588,(235.376,29.1706)}},draw=black,line join=bevel,line cap=rect,line width=0.800pt]
  \end{scope}
  \begin{scope}[cm={{1.00588,0.0,0.0,1.00588,(-208.53365,19.6192)}},draw=black,line join=bevel,line cap=rect,line width=0.800pt]
    \path[fill=black] (0.0000,0.0000) node[above right] (text600) {\label{fig:ener:static-II}II};



  \end{scope}
  \begin{scope}[cm={{1.00588,0.0,0.0,1.00588,(235.376,29.1706)}},draw=black,line join=bevel,line cap=rect,line width=0.800pt]
  \end{scope}
  \begin{scope}[scale=1.006,draw=black,line join=bevel,line cap=rect,line width=0.800pt]
  \end{scope}
  \begin{scope}[scale=1.006,draw=black,line join=bevel,line cap=rect,line width=0.800pt]
  \end{scope}
  \begin{scope}[scale=1.006,draw=black,line join=bevel,line cap=rect,line width=0.800pt]
  \end{scope}
  \begin{scope}[cm={{1.00588,0.0,0.0,1.00588,(-444.19503,-9.0)}},draw=black,line join=round,line cap=round,line width=0.480pt]
    \path[draw] (165.3000,36.9000) -- (165.3000,36.9000) -- (165.4000,41.9000) -- (165.5000,44.7000) -- (165.7000,46.2000) -- (165.8000,46.7000) -- (166.0000,46.7000) -- (166.1000,46.7000) -- (166.2000,47.2000) -- (166.4000,47.4000) -- (166.5000,47.2000) -- (166.6000,46.7000) -- (166.8000,46.1000) -- (166.9000,45.5000) -- (167.1000,44.9000) -- (167.2000,44.5000) -- (167.3000,44.2000) -- (167.5000,44.2000) -- (167.6000,44.2000) -- (167.8000,44.2000) -- (167.9000,44.3000) -- (168.0000,44.4000) -- (168.2000,44.4000) -- (168.3000,44.5000) -- (168.4000,44.5000) -- (168.6000,44.5000) -- (168.7000,44.5000) -- (168.9000,44.5000) -- (169.0000,44.5000) -- (169.1000,44.5000) -- (169.3000,44.5000) -- (169.4000,44.5000) -- (169.5000,44.5000) -- (169.7000,44.5000) -- (169.8000,44.5000) -- (170.0000,44.5000) -- (170.1000,44.5000) -- (170.2000,44.5000) -- (170.4000,44.5000) -- (170.5000,44.5000) -- (170.7000,44.5000) -- (170.8000,44.5000) -- (170.9000,44.5000) -- (171.1000,44.5000) -- (171.2000,44.5000) -- (171.3000,44.5000) -- (171.5000,44.5000) -- (171.6000,44.5000) -- (171.8000,44.6000) -- (171.9000,44.7000) -- (172.0000,44.9000) -- (172.2000,45.2000) -- (172.3000,45.5000) -- (172.5000,45.8000) -- (172.6000,46.3000) -- (172.7000,46.7000) -- (172.9000,47.3000) -- (173.0000,47.8000) -- (173.1000,48.5000) -- (173.3000,49.1000) -- (173.4000,49.9000) -- (173.6000,50.7000) -- (173.7000,51.5000) -- (173.8000,52.4000) -- (174.0000,53.4000) -- (174.1000,54.4000) -- (174.2000,55.4000) -- (174.4000,56.5000) -- (174.5000,57.7000) -- (174.7000,58.9000) -- (174.8000,60.2000) -- (174.9000,61.5000) -- (175.1000,62.8000) -- (175.2000,64.2000) -- (175.4000,65.6000) -- (175.5000,67.1000) -- (175.6000,68.5000) -- (175.8000,70.0000) -- (175.9000,71.5000) -- (176.0000,73.0000) -- (176.2000,74.5000) -- (176.3000,76.0000) -- (176.5000,77.4000) -- (176.6000,78.8000) -- (176.7000,80.2000) -- (176.9000,81.6000) -- (177.0000,83.1000) -- (177.1000,84.5000) -- (177.3000,85.7000) -- (177.4000,86.5000) -- (177.6000,87.0000) -- (177.7000,87.2000) -- (177.8000,87.2000) -- (178.0000,87.1000) -- (178.1000,86.9000) -- (178.3000,86.7000) -- (178.4000,86.6000) -- (178.5000,86.5000) -- (178.7000,86.4000) -- (178.8000,86.3000) -- (178.9000,86.3000) -- (179.1000,86.3000) -- (179.2000,86.3000) -- (179.4000,86.3000) -- (179.5000,86.3000) -- (179.6000,86.3000) -- (179.8000,86.3000) -- (179.9000,86.3000) -- (180.0000,86.3000) -- (180.2000,86.3000) -- (180.3000,86.3000) -- (180.5000,86.3000) -- (180.6000,86.3000) -- (180.7000,86.3000) -- (180.9000,86.3000) -- (181.0000,86.3000) -- (181.2000,86.3000) -- (181.3000,86.3000) -- (181.4000,86.2000) -- (181.6000,86.0000) -- (181.7000,85.8000) -- (181.8000,85.9000) -- (182.0000,86.1000) -- (182.1000,86.2000) -- (182.3000,86.0000) -- (182.4000,85.7000) -- (182.5000,85.0000) -- (182.7000,84.0000) -- (182.8000,82.8000) -- (183.0000,81.4000) -- (183.1000,79.9000) -- (183.2000,78.2000) -- (183.4000,76.4000) -- (183.5000,74.6000) -- (183.6000,72.8000) -- (183.8000,70.9000) -- (183.9000,69.1000) -- (184.1000,67.3000) -- (184.2000,65.5000) -- (184.3000,63.8000) -- (184.5000,62.1000) -- (184.6000,60.5000) -- (184.7000,59.0000) -- (184.9000,57.6000) -- (185.0000,56.2000) -- (185.2000,54.9000) -- (185.3000,53.7000) -- (185.4000,52.5000) -- (185.6000,51.5000) -- (185.7000,50.5000) -- (185.9000,49.6000) -- (186.0000,48.7000) -- (186.1000,48.0000) -- (186.3000,47.3000) -- (186.4000,46.7000) -- (186.5000,46.1000) -- (186.7000,45.6000) -- (186.8000,45.2000) -- (187.0000,44.9000) -- (187.1000,44.6000) -- (187.2000,44.3000) -- (187.4000,44.1000) -- (187.5000,44.1000) -- (187.6000,44.1000) -- (187.8000,44.2000) -- (187.9000,44.3000) -- (188.1000,44.4000) -- (188.2000,44.4000) -- (188.3000,44.5000) -- (188.5000,44.5000) -- (188.6000,44.5000) -- (188.8000,44.5000) -- (188.9000,44.5000) -- (189.0000,44.5000) -- (189.2000,44.5000) -- (189.3000,44.5000) -- (189.4000,44.5000) -- (189.6000,44.5000) -- (189.7000,44.5000) -- (189.9000,44.5000) -- (190.0000,44.5000) -- (190.1000,44.5000) -- (190.3000,44.5000) -- (190.4000,44.5000) -- (190.6000,44.5000) -- (190.7000,44.5000) -- (190.8000,44.5000) -- (191.0000,44.5000) -- (191.1000,44.5000) -- (191.2000,44.5000) -- (191.4000,44.5000) -- (191.5000,44.5000) -- (191.7000,44.5000) -- (191.8000,44.5000) -- (191.9000,44.6000) -- (192.1000,44.7000) -- (192.2000,44.9000) -- (192.3000,45.1000) -- (192.5000,45.4000) -- (192.6000,45.8000) -- (192.8000,46.2000) -- (192.9000,46.6000) -- (193.0000,47.2000) -- (193.2000,47.7000) -- (193.3000,48.3000) -- (193.5000,49.0000) -- (193.6000,49.7000) -- (193.7000,50.5000) -- (193.9000,51.3000) -- (194.0000,52.2000) -- (194.1000,53.1000) -- (194.3000,54.1000) -- (194.4000,55.1000) -- (194.6000,56.2000) -- (194.7000,57.3000) -- (194.8000,58.5000) -- (195.0000,59.8000) -- (195.1000,61.0000) -- (195.2000,62.4000) -- (195.4000,63.7000) -- (195.5000,65.1000) -- (195.7000,66.5000) -- (195.8000,68.0000) -- (195.9000,69.5000) -- (196.1000,71.0000) -- (196.2000,72.4000) -- (196.4000,73.9000) -- (196.5000,75.4000) -- (196.6000,76.8000) -- (196.8000,78.2000) -- (196.9000,79.6000) -- (197.0000,81.0000) -- (197.2000,82.5000) -- (197.3000,84.0000) -- (197.5000,85.3000) -- (197.6000,86.3000) -- (197.7000,86.9000) -- (197.9000,87.2000) -- (198.0000,87.3000) -- (198.1000,87.2000) -- (198.3000,87.0000) -- (198.4000,86.8000) -- (198.6000,86.6000) -- (198.7000,86.5000) -- (198.8000,86.4000) -- (199.0000,86.3000) -- (199.1000,86.3000) -- (199.3000,86.3000) -- (199.4000,86.3000) -- (199.5000,86.3000) -- (199.7000,86.3000) -- (199.8000,86.3000) -- (199.9000,86.3000) -- (200.1000,86.3000) -- (200.2000,86.3000) -- (200.4000,86.3000) -- (200.5000,86.3000) -- (200.6000,86.3000) -- (200.8000,86.3000) -- (200.9000,86.3000) -- (201.1000,86.3000) -- (201.2000,86.3000) -- (201.3000,86.3000) -- (201.5000,86.3000) -- (201.6000,86.3000) -- (201.7000,86.1000) -- (201.9000,85.9000) -- (202.0000,85.9000) -- (202.2000,86.0000) -- (202.3000,86.1000) -- (202.4000,86.1000) -- (202.6000,85.9000) -- (202.7000,85.3000) -- (202.8000,84.5000) -- (203.0000,83.5000) -- (203.1000,82.2000) -- (203.3000,80.7000) -- (203.4000,79.1000) -- (203.5000,77.3000) -- (203.7000,75.5000) -- (203.8000,73.7000) -- (204.0000,71.9000) -- (204.1000,70.0000) -- (204.2000,68.2000) -- (204.4000,66.4000) -- (204.5000,64.7000) -- (204.6000,63.0000) -- (204.8000,61.4000) -- (204.9000,59.8000) -- (205.1000,58.3000) -- (205.2000,56.9000) -- (205.3000,55.5000) -- (205.5000,54.3000) -- (205.6000,53.1000) -- (205.7000,52.0000) -- (205.9000,51.0000) -- (206.0000,50.0000) -- (206.2000,49.1000) -- (206.3000,48.4000) -- (206.4000,47.6000) -- (206.6000,47.0000) -- (206.7000,46.4000) -- (206.9000,45.9000) -- (207.0000,45.4000) -- (207.1000,45.1000) -- (207.3000,44.8000) -- (207.4000,44.5000) -- (207.5000,44.2000) -- (207.7000,44.1000) -- (207.8000,44.1000) -- (208.0000,44.1000) -- (208.1000,44.2000) -- (208.2000,44.3000) -- (208.4000,44.4000) -- (208.5000,44.4000) -- (208.6000,44.5000) -- (208.8000,44.5000) -- (208.9000,44.5000) -- (209.1000,44.5000) -- (209.2000,44.5000) -- (209.3000,44.5000) -- (209.5000,44.5000) -- (209.6000,44.5000) -- (209.8000,44.5000) -- (209.9000,44.5000) -- (210.0000,44.5000) -- (210.2000,44.5000) -- (210.3000,44.5000) -- (210.4000,44.5000) -- (210.6000,44.5000) -- (210.7000,44.5000) -- (210.9000,44.5000) -- (211.0000,44.5000) -- (211.1000,44.5000) -- (211.3000,44.5000) -- (211.4000,44.5000) -- (211.5000,44.5000) -- (211.7000,44.5000) -- (211.8000,44.5000) -- (212.0000,44.5000) -- (212.1000,44.6000) -- (212.2000,44.7000) -- (212.4000,44.8000) -- (212.5000,45.0000) -- (212.7000,45.3000) -- (212.8000,45.6000) -- (212.9000,45.9000) -- (213.1000,46.4000) -- (213.2000,46.9000) -- (213.3000,47.4000) -- (213.5000,48.0000) -- (213.6000,48.6000) -- (213.8000,49.3000) -- (213.9000,50.0000) -- (214.0000,50.8000) -- (214.2000,51.7000) -- (214.3000,52.6000) -- (214.5000,53.5000) -- (214.6000,54.5000) -- (214.7000,55.6000) -- (214.9000,56.7000) -- (215.0000,57.8000) -- (215.1000,59.0000) -- (215.3000,60.3000) -- (215.4000,61.6000) -- (215.6000,62.9000) -- (215.7000,64.3000) -- (215.8000,65.7000) -- (216.0000,67.1000) -- (216.1000,68.6000) -- (216.2000,70.1000) -- (216.4000,71.6000) -- (216.5000,73.0000) -- (216.7000,74.5000) -- (216.8000,76.0000) -- (216.9000,77.4000) -- (217.1000,78.8000) -- (217.2000,80.1000) -- (217.4000,81.6000) -- (217.5000,83.1000) -- (217.6000,84.6000) -- (217.8000,85.8000) -- (217.9000,86.6000) -- (218.0000,87.0000) -- (218.2000,87.2000) -- (218.3000,87.2000) -- (218.5000,87.1000) -- (218.6000,86.9000) -- (218.7000,86.7000) -- (218.9000,86.6000) -- (219.0000,86.5000) -- (219.1000,86.4000) -- (219.3000,86.3000) -- (219.4000,86.3000) -- (219.6000,86.3000) -- (219.7000,86.3000) -- (219.8000,86.3000) -- (220.0000,86.3000) -- (220.1000,86.3000) -- (220.3000,86.3000) -- (220.4000,86.3000) -- (220.5000,86.3000) -- (220.7000,86.3000) -- (220.8000,86.3000) -- (220.9000,86.3000) -- (221.1000,86.3000) -- (221.2000,86.3000) -- (221.4000,86.3000) -- (221.5000,86.3000) -- (221.6000,86.3000) -- (221.8000,86.3000) -- (221.9000,86.2000) -- (222.0000,86.0000) -- (222.2000,85.9000) -- (222.3000,85.9000) -- (222.5000,86.1000) -- (222.6000,86.2000) -- (222.7000,86.1000) -- (222.9000,85.7000) -- (223.0000,85.1000) -- (223.2000,84.1000) -- (223.3000,83.0000) -- (223.4000,81.6000) -- (223.6000,80.1000) -- (223.7000,78.4000) -- (223.8000,76.6000) -- (224.0000,74.8000) -- (224.1000,73.0000) -- (224.3000,71.2000) -- (224.4000,69.3000) -- (224.5000,67.5000) -- (224.7000,65.7000) -- (224.8000,64.0000) -- (225.0000,62.3000) -- (225.1000,60.7000) -- (225.2000,59.2000) -- (225.4000,57.7000) -- (225.5000,56.3000) -- (225.6000,55.0000) -- (225.8000,53.8000) -- (225.9000,52.7000) -- (226.1000,51.6000) -- (226.2000,50.6000) -- (226.3000,49.7000) -- (226.5000,48.8000) -- (226.6000,48.1000) -- (226.7000,47.4000) -- (226.9000,46.7000) -- (227.0000,46.2000) -- (227.2000,45.7000) -- (227.3000,45.3000) -- (227.4000,44.9000) -- (227.6000,44.7000) -- (227.7000,44.4000) -- (227.9000,44.2000) -- (228.0000,44.1000) -- (228.1000,44.1000) -- (228.3000,44.2000) -- (228.4000,44.3000) -- (228.5000,44.3000) -- (228.7000,44.4000) -- (228.8000,44.5000) -- (229.0000,44.5000) -- (229.1000,44.5000) -- (229.2000,44.5000) -- (229.4000,44.5000) -- (229.5000,44.5000) -- (229.6000,44.5000) -- (229.8000,44.5000) -- (229.9000,44.5000) -- (230.1000,44.5000) -- (230.2000,44.5000) -- (230.3000,44.5000) -- (230.5000,44.5000) -- (230.6000,44.5000) -- (230.8000,44.5000) -- (230.9000,44.5000) -- (231.0000,44.5000) -- (231.2000,44.5000) -- (231.3000,44.5000) -- (231.4000,44.5000) -- (231.6000,44.5000) -- (231.7000,44.5000) -- (231.9000,44.5000) -- (232.0000,44.5000) -- (232.1000,44.5000) -- (232.3000,44.5000) -- (232.4000,44.6000) -- (232.6000,44.7000) -- (232.7000,44.9000) -- (232.8000,45.1000) -- (233.0000,45.4000) -- (233.1000,45.7000) -- (233.2000,46.1000) -- (233.4000,46.6000) -- (233.5000,47.1000) -- (233.7000,47.6000) -- (233.8000,48.2000) -- (233.9000,48.9000) -- (234.1000,49.6000) -- (234.2000,50.3000) -- (234.3000,51.1000) -- (234.5000,52.0000) -- (234.6000,52.9000) -- (234.8000,53.9000) -- (234.9000,54.9000) -- (235.0000,56.0000) -- (235.2000,57.1000) -- (235.3000,58.3000) -- (235.5000,59.5000) -- (235.6000,60.8000) -- (235.7000,62.1000) -- (235.9000,63.4000) -- (236.0000,64.8000) -- (236.1000,66.3000) -- (236.3000,67.7000) -- (236.4000,69.2000) -- (236.6000,70.6000) -- (236.7000,72.1000) -- (236.8000,73.6000) -- (237.0000,75.1000) -- (237.1000,76.5000) -- (237.2000,77.9000) -- (237.4000,79.3000) -- (237.5000,80.7000) -- (237.7000,82.2000) -- (237.8000,83.7000) -- (237.9000,85.1000) -- (238.1000,86.1000) -- (238.2000,86.8000) -- (238.4000,87.2000) -- (238.5000,87.3000) -- (238.6000,87.2000) -- (238.8000,87.0000) -- (238.9000,86.9000) -- (239.0000,86.7000) -- (239.2000,86.5000) -- (239.3000,86.4000) -- (239.5000,86.3000) -- (239.6000,86.3000) -- (239.7000,86.3000) -- (239.9000,86.3000) -- (240.0000,86.3000) -- (240.2000,86.3000) -- (240.3000,86.3000) -- (240.4000,86.3000) -- (240.6000,86.3000) -- (240.7000,86.3000) -- (240.8000,86.3000) -- (241.0000,86.3000) -- (241.1000,86.3000) -- (241.3000,86.3000) -- (241.4000,86.3000) -- (241.5000,86.3000) -- (241.7000,86.3000) -- (241.8000,86.3000) -- (241.9000,86.3000) -- (242.1000,86.3000) -- (242.2000,86.2000) -- (242.4000,86.0000) -- (242.5000,85.9000) -- (242.6000,86.0000) -- (242.8000,86.1000) -- (242.9000,86.1000) -- (243.1000,85.9000) -- (243.2000,85.5000) -- (243.3000,84.7000) -- (243.5000,83.7000) -- (243.6000,82.5000) -- (243.7000,81.0000) -- (243.9000,79.4000) -- (244.0000,77.7000) -- (244.2000,75.9000) -- (244.3000,74.1000) -- (244.4000,72.3000) -- (244.6000,70.4000) -- (244.7000,68.6000) -- (244.8000,66.8000) -- (245.0000,65.1000) -- (245.1000,63.4000) -- (245.3000,61.7000) -- (245.4000,60.1000) -- (245.5000,58.6000) -- (245.7000,57.2000) -- (245.8000,55.8000) -- (246.0000,54.6000) -- (246.1000,53.3000) -- (246.2000,52.2000) -- (246.4000,51.2000) -- (246.5000,50.2000) -- (246.6000,49.3000) -- (246.8000,48.5000) -- (246.9000,47.8000) -- (247.1000,47.1000) -- (247.2000,46.5000) -- (247.3000,46.0000) -- (247.5000,45.5000) -- (247.6000,45.1000) -- (247.7000,44.8000) -- (247.9000,44.5000) -- (248.0000,44.3000) -- (248.2000,44.1000) -- (248.3000,44.1000) -- (248.4000,44.1000) -- (248.6000,44.2000) -- (248.7000,44.3000) -- (248.9000,44.4000) -- (249.0000,44.4000) -- (249.1000,44.5000) -- (249.3000,44.5000) -- (249.4000,44.5000) -- (249.5000,44.5000) -- (249.7000,44.5000) -- (249.8000,44.5000) -- (250.0000,44.5000) -- (250.1000,44.5000) -- (250.2000,44.5000) -- (250.4000,44.5000) -- (250.5000,44.5000) -- (250.7000,44.5000) -- (250.8000,44.5000) -- (250.9000,44.5000) -- (251.1000,44.5000) -- (251.2000,44.5000) -- (251.3000,44.5000) -- (251.5000,44.5000) -- (251.6000,44.5000) -- (251.8000,44.5000) -- (251.9000,44.5000) -- (251.9000,44.5000);



  \end{scope}
  \begin{scope}[scale=1.006,draw=black,line join=bevel,line cap=rect,line width=0.800pt]
  \end{scope}
  \begin{scope}[scale=1.006,draw=black,line join=bevel,line cap=rect,line width=0.800pt]
  \end{scope}
  \begin{scope}[scale=1.006,draw=black,line join=bevel,line cap=rect,line width=0.800pt]
  \end{scope}
  \begin{scope}[scale=1.006,draw=black,line join=bevel,line cap=rect,line width=0.800pt]
  \end{scope}
  \begin{scope}[cm={{1.00588,0.0,0.0,1.00588,(-444.19503,-9.0)}},draw=cff0000,line join=round,line cap=round,line width=0.480pt]
    \path[draw] (165.2000,51.8000) -- (165.2000,51.8000) -- (165.3000,51.0000) -- (165.4000,50.3000) -- (165.5000,49.6000) -- (165.5000,48.9000) -- (165.6000,48.3000) -- (165.7000,47.8000) -- (165.8000,47.2000) -- (165.9000,46.8000) -- (166.0000,46.4000) -- (166.1000,46.0000) -- (166.2000,45.7000) -- (166.2000,45.4000) -- (166.3000,45.2000) -- (166.4000,45.0000) -- (166.5000,44.8000) -- (166.6000,44.7000) -- (166.7000,44.7000) -- (166.8000,44.7000) -- (166.8000,44.7000) -- (166.9000,44.7000) -- (167.0000,44.8000) -- (167.1000,44.9000) -- (167.2000,45.1000) -- (167.3000,45.3000) -- (167.4000,45.4000) -- (167.5000,45.6000) -- (167.5000,45.9000) -- (167.6000,46.1000) -- (167.7000,46.4000) -- (167.8000,46.6000) -- (167.9000,46.9000) -- (168.0000,47.1000) -- (168.1000,47.4000) -- (168.2000,47.6000) -- (168.2000,47.8000) -- (168.3000,48.0000) -- (168.4000,48.2000) -- (168.5000,48.4000) -- (168.6000,48.6000) -- (168.7000,48.8000) -- (168.8000,48.9000) -- (168.8000,49.0000) -- (168.9000,49.1000) -- (169.0000,49.1000) -- (169.1000,49.1000) -- (169.2000,49.1000) -- (169.3000,49.1000) -- (169.4000,49.0000) -- (169.5000,48.9000) -- (169.5000,48.8000) -- (169.6000,48.6000) -- (169.7000,48.5000) -- (169.8000,48.3000) -- (169.9000,48.0000) -- (170.0000,47.8000) -- (170.1000,47.5000) -- (170.1000,47.2000) -- (170.2000,46.9000) -- (170.3000,46.6000) -- (170.4000,46.2000) -- (170.5000,45.9000) -- (170.6000,45.5000) -- (170.7000,45.1000) -- (170.8000,44.8000) -- (170.8000,44.4000) -- (170.9000,44.0000) -- (171.0000,43.7000) -- (171.1000,43.4000) -- (171.2000,43.0000) -- (171.3000,42.7000) -- (171.4000,42.5000) -- (171.4000,42.2000) -- (171.5000,42.0000) -- (171.6000,41.8000) -- (171.7000,41.6000) -- (171.8000,41.5000) -- (171.9000,41.4000) -- (172.0000,41.3000) -- (172.1000,41.3000) -- (172.1000,41.3000) -- (172.2000,41.4000) -- (172.3000,41.6000) -- (172.4000,41.7000) -- (172.5000,42.0000) -- (172.6000,42.3000) -- (172.7000,42.6000) -- (172.8000,43.0000) -- (172.8000,43.4000) -- (172.9000,43.9000) -- (173.0000,44.5000) -- (173.1000,45.1000) -- (173.2000,45.7000) -- (173.3000,46.4000) -- (173.4000,47.1000) -- (173.4000,47.9000) -- (173.5000,48.8000) -- (173.6000,49.6000) -- (173.7000,50.5000) -- (173.8000,51.5000) -- (173.9000,52.5000) -- (174.0000,53.5000) -- (174.1000,54.5000) -- (174.1000,55.6000) -- (174.2000,56.6000) -- (174.3000,57.7000) -- (174.4000,58.9000) -- (174.5000,60.0000) -- (174.6000,61.1000) -- (174.7000,62.2000) -- (174.7000,63.4000) -- (174.8000,64.5000) -- (174.9000,65.6000) -- (175.0000,66.8000) -- (175.1000,67.8000) -- (175.2000,68.9000) -- (175.3000,70.0000) -- (175.4000,71.0000) -- (175.4000,72.1000) -- (175.5000,73.0000) -- (175.6000,74.0000) -- (175.7000,74.9000) -- (175.8000,75.8000) -- (175.9000,76.7000) -- (176.0000,77.5000) -- (176.0000,78.3000) -- (176.1000,79.0000) -- (176.2000,79.7000) -- (176.3000,80.3000) -- (176.4000,80.9000) -- (176.5000,81.5000) -- (176.6000,82.0000) -- (176.7000,82.5000) -- (176.7000,83.0000) -- (176.8000,83.4000) -- (176.9000,83.7000) -- (177.0000,84.0000) -- (177.1000,84.3000) -- (177.2000,84.6000) -- (177.3000,84.8000) -- (177.4000,85.0000) -- (177.4000,85.1000) -- (177.5000,85.3000) -- (177.6000,85.3000) -- (177.7000,85.4000) -- (177.8000,85.5000) -- (177.9000,85.5000) -- (178.0000,85.5000) -- (178.0000,85.5000) -- (178.1000,85.5000) -- (178.2000,85.5000) -- (178.3000,85.4000) -- (178.4000,85.4000) -- (178.5000,85.4000) -- (178.6000,85.3000) -- (178.7000,85.3000) -- (178.7000,85.2000) -- (178.8000,85.2000) -- (178.9000,85.2000) -- (179.0000,85.1000) -- (179.1000,85.1000) -- (179.2000,85.1000) -- (179.3000,85.1000) -- (179.3000,85.1000) -- (179.4000,85.1000) -- (179.5000,85.1000) -- (179.6000,85.2000) -- (179.7000,85.2000) -- (179.8000,85.3000) -- (179.9000,85.4000) -- (180.0000,85.4000) -- (180.0000,85.5000) -- (180.1000,85.6000) -- (180.2000,85.7000) -- (180.3000,85.8000) -- (180.4000,85.9000) -- (180.5000,86.0000) -- (180.6000,86.1000) -- (180.6000,86.2000) -- (180.7000,86.3000) -- (180.8000,86.4000) -- (180.9000,86.4000) -- (181.0000,86.5000) -- (181.1000,86.5000) -- (181.2000,86.6000) -- (181.3000,86.6000) -- (181.3000,86.6000) -- (181.4000,86.5000) -- (181.5000,86.4000) -- (181.6000,86.4000) -- (181.7000,86.2000) -- (181.8000,86.1000) -- (181.9000,85.9000) -- (182.0000,85.6000) -- (182.0000,85.4000) -- (182.1000,85.1000) -- (182.2000,84.7000) -- (182.3000,84.3000) -- (182.4000,83.9000) -- (182.5000,83.5000) -- (182.6000,82.9000) -- (182.6000,82.4000) -- (182.7000,81.8000) -- (182.8000,81.2000) -- (182.9000,80.5000) -- (183.0000,79.8000) -- (183.1000,79.0000) -- (183.2000,78.2000) -- (183.3000,77.4000) -- (183.3000,76.5000) -- (183.4000,75.6000) -- (183.5000,74.7000) -- (183.6000,73.7000) -- (183.7000,72.7000) -- (183.8000,71.7000) -- (183.9000,70.7000) -- (183.9000,69.6000) -- (184.0000,68.6000) -- (184.1000,67.5000) -- (184.2000,66.4000) -- (184.3000,65.3000) -- (184.4000,64.2000) -- (184.5000,63.1000) -- (184.6000,62.1000) -- (184.6000,61.0000) -- (184.7000,59.9000) -- (184.8000,58.9000) -- (184.9000,57.9000) -- (185.0000,56.9000) -- (185.1000,55.9000) -- (185.2000,54.9000) -- (185.2000,54.0000) -- (185.3000,53.1000) -- (185.4000,52.3000) -- (185.5000,51.5000) -- (185.6000,50.7000) -- (185.7000,50.0000) -- (185.8000,49.3000) -- (185.9000,48.7000) -- (185.9000,48.1000) -- (186.0000,47.5000) -- (186.1000,47.0000) -- (186.2000,46.6000) -- (186.3000,46.2000) -- (186.4000,45.8000) -- (186.5000,45.5000) -- (186.5000,45.3000) -- (186.6000,45.0000) -- (186.7000,44.9000) -- (186.8000,44.8000) -- (186.9000,44.7000) -- (187.0000,44.6000) -- (187.1000,44.6000) -- (187.2000,44.7000) -- (187.2000,44.7000) -- (187.3000,44.8000) -- (187.4000,45.0000) -- (187.5000,45.1000) -- (187.6000,45.3000) -- (187.7000,45.5000) -- (187.8000,45.7000) -- (187.9000,46.0000) -- (187.9000,46.2000) -- (188.0000,46.4000) -- (188.1000,46.7000) -- (188.2000,47.0000) -- (188.3000,47.2000) -- (188.4000,47.4000) -- (188.5000,47.7000) -- (188.5000,47.9000) -- (188.6000,48.1000) -- (188.7000,48.3000) -- (188.8000,48.5000) -- (188.9000,48.7000) -- (189.0000,48.8000) -- (189.1000,48.9000) -- (189.2000,49.0000) -- (189.2000,49.1000) -- (189.3000,49.1000) -- (189.4000,49.1000) -- (189.5000,49.1000) -- (189.6000,49.1000) -- (189.7000,49.0000) -- (189.8000,48.9000) -- (189.8000,48.8000) -- (189.9000,48.6000) -- (190.0000,48.4000) -- (190.1000,48.2000) -- (190.2000,47.9000) -- (190.3000,47.7000) -- (190.4000,47.4000) -- (190.5000,47.1000) -- (190.5000,46.8000) -- (190.6000,46.4000) -- (190.7000,46.1000) -- (190.8000,45.7000) -- (190.9000,45.4000) -- (191.0000,45.0000) -- (191.1000,44.6000) -- (191.1000,44.3000) -- (191.2000,43.9000) -- (191.3000,43.6000) -- (191.4000,43.2000) -- (191.5000,42.9000) -- (191.6000,42.6000) -- (191.7000,42.3000) -- (191.8000,42.1000) -- (191.8000,41.9000) -- (191.9000,41.7000) -- (192.0000,41.5000) -- (192.1000,41.4000) -- (192.2000,41.3000) -- (192.3000,41.3000) -- (192.4000,41.3000) -- (192.5000,41.3000) -- (192.5000,41.4000) -- (192.6000,41.6000) -- (192.7000,41.8000) -- (192.8000,42.0000) -- (192.9000,42.4000) -- (193.0000,42.7000) -- (193.1000,43.1000) -- (193.1000,43.6000) -- (193.2000,44.1000) -- (193.3000,44.7000) -- (193.4000,45.3000) -- (193.5000,45.9000) -- (193.6000,46.7000) -- (193.7000,47.4000) -- (193.8000,48.2000) -- (193.8000,49.1000) -- (193.9000,50.0000) -- (194.0000,50.9000) -- (194.1000,51.8000) -- (194.2000,52.8000) -- (194.3000,53.9000) -- (194.4000,54.9000) -- (194.4000,56.0000) -- (194.5000,57.1000) -- (194.6000,58.2000) -- (194.7000,59.3000) -- (194.8000,60.4000) -- (194.9000,61.6000) -- (195.0000,62.7000) -- (195.1000,63.8000) -- (195.1000,65.0000) -- (195.2000,66.1000) -- (195.3000,67.2000) -- (195.4000,68.3000) -- (195.5000,69.4000) -- (195.6000,70.4000) -- (195.7000,71.5000) -- (195.7000,72.5000) -- (195.8000,73.4000) -- (195.9000,74.4000) -- (196.0000,75.3000) -- (196.1000,76.2000) -- (196.2000,77.0000) -- (196.3000,77.8000) -- (196.4000,78.6000) -- (196.4000,79.3000) -- (196.5000,80.0000) -- (196.6000,80.6000) -- (196.7000,81.2000) -- (196.8000,81.7000) -- (196.9000,82.3000) -- (197.0000,82.7000) -- (197.1000,83.2000) -- (197.1000,83.5000) -- (197.2000,83.9000) -- (197.3000,84.2000) -- (197.4000,84.5000) -- (197.5000,84.7000) -- (197.6000,84.9000) -- (197.7000,85.1000) -- (197.7000,85.2000) -- (197.8000,85.3000) -- (197.9000,85.4000) -- (198.0000,85.5000) -- (198.1000,85.5000) -- (198.2000,85.5000) -- (198.3000,85.5000) -- (198.4000,85.5000) -- (198.4000,85.5000) -- (198.5000,85.5000) -- (198.6000,85.4000) -- (198.7000,85.4000) -- (198.8000,85.3000) -- (198.9000,85.3000) -- (199.0000,85.2000) -- (199.0000,85.2000) -- (199.1000,85.2000) -- (199.2000,85.1000) -- (199.3000,85.1000) -- (199.4000,85.1000) -- (199.5000,85.1000) -- (199.6000,85.1000) -- (199.7000,85.1000) -- (199.7000,85.1000) -- (199.8000,85.2000) -- (199.9000,85.2000) -- (200.0000,85.2000) -- (200.1000,85.3000) -- (200.2000,85.4000) -- (200.3000,85.5000) -- (200.3000,85.5000) -- (200.4000,85.6000) -- (200.5000,85.7000) -- (200.6000,85.8000) -- (200.7000,85.9000) -- (200.8000,86.0000) -- (200.9000,86.1000) -- (201.0000,86.2000) -- (201.0000,86.3000) -- (201.1000,86.4000) -- (201.2000,86.5000) -- (201.3000,86.5000) -- (201.4000,86.6000) -- (201.5000,86.6000) -- (201.6000,86.6000) -- (201.7000,86.6000) -- (201.7000,86.5000) -- (201.8000,86.4000) -- (201.9000,86.3000) -- (202.0000,86.2000) -- (202.1000,86.0000) -- (202.2000,85.8000) -- (202.3000,85.6000) -- (202.3000,85.3000) -- (202.4000,85.0000) -- (202.5000,84.6000) -- (202.6000,84.2000) -- (202.7000,83.8000) -- (202.8000,83.3000) -- (202.9000,82.8000) -- (203.0000,82.2000) -- (203.0000,81.6000) -- (203.1000,80.9000) -- (203.2000,80.2000) -- (203.3000,79.5000) -- (203.4000,78.7000) -- (203.5000,77.9000) -- (203.6000,77.0000) -- (203.6000,76.2000) -- (203.7000,75.2000) -- (203.8000,74.3000) -- (203.9000,73.3000) -- (204.0000,72.3000) -- (204.1000,71.3000) -- (204.2000,70.3000) -- (204.3000,69.2000) -- (204.3000,68.1000) -- (204.4000,67.1000) -- (204.5000,66.0000) -- (204.6000,64.9000) -- (204.7000,63.8000) -- (204.8000,62.7000) -- (204.9000,61.6000) -- (204.9000,60.6000) -- (205.0000,59.5000) -- (205.1000,58.5000) -- (205.2000,57.5000) -- (205.3000,56.5000) -- (205.4000,55.5000) -- (205.5000,54.6000) -- (205.6000,53.7000) -- (205.6000,52.8000) -- (205.7000,51.9000) -- (205.8000,51.1000) -- (205.9000,50.4000) -- (206.0000,49.7000) -- (206.1000,49.0000) -- (206.2000,48.4000) -- (206.3000,47.8000) -- (206.3000,47.3000) -- (206.4000,46.8000) -- (206.5000,46.4000) -- (206.6000,46.0000) -- (206.7000,45.7000) -- (206.8000,45.4000) -- (206.9000,45.1000) -- (206.9000,45.0000) -- (207.0000,44.8000) -- (207.1000,44.7000) -- (207.2000,44.6000) -- (207.3000,44.6000) -- (207.4000,44.6000) -- (207.5000,44.7000) -- (207.6000,44.8000) -- (207.6000,44.9000) -- (207.7000,45.0000) -- (207.8000,45.2000) -- (207.9000,45.4000) -- (208.0000,45.6000) -- (208.1000,45.8000) -- (208.2000,46.0000) -- (208.2000,46.3000) -- (208.3000,46.5000) -- (208.4000,46.8000) -- (208.5000,47.1000) -- (208.6000,47.3000) -- (208.7000,47.5000) -- (208.8000,47.8000) -- (208.9000,48.0000) -- (208.9000,48.2000) -- (209.0000,48.4000) -- (209.1000,48.6000) -- (209.2000,48.8000) -- (209.3000,48.9000) -- (209.4000,49.0000) -- (209.5000,49.1000) -- (209.5000,49.1000) -- (209.6000,49.2000) -- (209.7000,49.2000) -- (209.8000,49.1000) -- (209.9000,49.1000) -- (210.0000,49.0000) -- (210.1000,48.9000) -- (210.2000,48.7000) -- (210.2000,48.5000) -- (210.3000,48.3000) -- (210.4000,48.1000) -- (210.5000,47.9000) -- (210.6000,47.6000) -- (210.7000,47.3000) -- (210.8000,47.0000) -- (210.9000,46.6000) -- (210.9000,46.3000) -- (211.0000,45.9000) -- (211.1000,45.6000) -- (211.2000,45.2000) -- (211.3000,44.8000) -- (211.4000,44.5000) -- (211.5000,44.1000) -- (211.5000,43.8000) -- (211.6000,43.4000) -- (211.7000,43.1000) -- (211.8000,42.8000) -- (211.9000,42.5000) -- (212.0000,42.2000) -- (212.1000,42.0000) -- (212.2000,41.8000) -- (212.2000,41.6000) -- (212.3000,41.4000) -- (212.4000,41.3000) -- (212.5000,41.3000) -- (212.6000,41.3000) -- (212.7000,41.3000) -- (212.8000,41.4000) -- (212.8000,41.5000) -- (212.9000,41.6000) -- (213.0000,41.9000) -- (213.1000,42.1000) -- (213.2000,42.5000) -- (213.3000,42.8000) -- (213.4000,43.3000) -- (213.5000,43.7000) -- (213.5000,44.3000) -- (213.6000,44.9000) -- (213.7000,45.5000) -- (213.8000,46.2000) -- (213.9000,46.9000) -- (214.0000,47.7000) -- (214.1000,48.5000) -- (214.1000,49.4000) -- (214.2000,50.3000) -- (214.3000,51.2000) -- (214.4000,52.2000) -- (214.5000,53.2000) -- (214.6000,54.2000) -- (214.7000,55.3000) -- (214.8000,56.4000) -- (214.8000,57.5000) -- (214.9000,58.6000) -- (215.0000,59.7000) -- (215.1000,60.9000) -- (215.2000,62.0000) -- (215.3000,63.1000) -- (215.4000,64.3000) -- (215.5000,65.4000) -- (215.5000,66.5000) -- (215.6000,67.6000) -- (215.7000,68.7000) -- (215.8000,69.8000) -- (215.9000,70.8000) -- (216.0000,71.9000) -- (216.1000,72.9000) -- (216.1000,73.8000) -- (216.2000,74.8000) -- (216.3000,75.7000) -- (216.4000,76.5000) -- (216.5000,77.4000) -- (216.6000,78.1000) -- (216.7000,78.9000) -- (216.8000,79.6000) -- (216.8000,80.2000) -- (216.9000,80.9000) -- (217.0000,81.4000) -- (217.1000,82.0000) -- (217.2000,82.5000) -- (217.3000,82.9000) -- (217.4000,83.3000) -- (217.4000,83.7000) -- (217.5000,84.0000) -- (217.6000,84.3000) -- (217.7000,84.6000) -- (217.8000,84.8000) -- (217.9000,85.0000) -- (218.0000,85.1000) -- (218.1000,85.3000) -- (218.1000,85.4000) -- (218.2000,85.4000) -- (218.3000,85.5000) -- (218.4000,85.5000) -- (218.5000,85.5000) -- (218.6000,85.5000) -- (218.7000,85.5000) -- (218.7000,85.5000) -- (218.8000,85.5000) -- (218.9000,85.4000) -- (219.0000,85.4000) -- (219.1000,85.3000) -- (219.2000,85.3000) -- (219.3000,85.2000) -- (219.4000,85.2000) -- (219.4000,85.2000) -- (219.5000,85.1000) -- (219.6000,85.1000) -- (219.7000,85.1000) -- (219.8000,85.1000) -- (219.9000,85.1000) -- (220.0000,85.1000) -- (220.1000,85.1000) -- (220.1000,85.2000) -- (220.2000,85.2000) -- (220.3000,85.3000) -- (220.4000,85.3000) -- (220.5000,85.4000) -- (220.6000,85.5000) -- (220.7000,85.6000) -- (220.7000,85.7000) -- (220.8000,85.8000) -- (220.9000,85.9000) -- (221.0000,86.0000) -- (221.1000,86.1000) -- (221.2000,86.2000) -- (221.3000,86.3000) -- (221.4000,86.4000) -- (221.4000,86.4000) -- (221.5000,86.5000) -- (221.6000,86.5000) -- (221.7000,86.6000) -- (221.8000,86.6000) -- (221.9000,86.6000) -- (222.0000,86.6000) -- (222.0000,86.5000) -- (222.1000,86.4000) -- (222.2000,86.3000) -- (222.3000,86.1000) -- (222.4000,86.0000) -- (222.5000,85.7000) -- (222.6000,85.5000) -- (222.7000,85.2000) -- (222.7000,84.9000) -- (222.8000,84.5000) -- (222.9000,84.1000) -- (223.0000,83.6000) -- (223.1000,83.1000) -- (223.2000,82.6000) -- (223.3000,82.0000) -- (223.3000,81.3000) -- (223.4000,80.7000) -- (223.5000,80.0000) -- (223.6000,79.2000) -- (223.7000,78.4000) -- (223.8000,77.6000) -- (223.9000,76.7000) -- (224.0000,75.8000) -- (224.0000,74.9000) -- (224.1000,73.9000) -- (224.2000,73.0000) -- (224.3000,71.9000) -- (224.4000,70.9000) -- (224.5000,69.9000) -- (224.6000,68.8000) -- (224.7000,67.7000) -- (224.7000,66.6000) -- (224.8000,65.6000) -- (224.9000,64.5000) -- (225.0000,63.4000) -- (225.1000,62.3000) -- (225.2000,61.2000) -- (225.3000,60.1000) -- (225.3000,59.1000) -- (225.4000,58.1000) -- (225.5000,57.1000) -- (225.6000,56.1000) -- (225.7000,55.1000) -- (225.8000,54.2000) -- (225.9000,53.3000) -- (226.0000,52.4000) -- (226.0000,51.6000) -- (226.1000,50.8000) -- (226.2000,50.1000) -- (226.3000,49.4000) -- (226.4000,48.7000) -- (226.5000,48.1000) -- (226.6000,47.6000) -- (226.6000,47.1000) -- (226.7000,46.6000) -- (226.8000,46.2000) -- (226.9000,45.8000) -- (227.0000,45.5000) -- (227.1000,45.3000) -- (227.2000,45.0000) -- (227.3000,44.9000) -- (227.3000,44.7000) -- (227.4000,44.6000) -- (227.5000,44.6000) -- (227.6000,44.6000) -- (227.7000,44.6000) -- (227.8000,44.7000) -- (227.9000,44.8000) -- (227.9000,44.9000) -- (228.0000,45.1000) -- (228.1000,45.2000) -- (228.2000,45.4000) -- (228.3000,45.7000) -- (228.4000,45.9000) -- (228.5000,46.1000) -- (228.6000,46.4000) -- (228.6000,46.6000) -- (228.7000,46.9000) -- (228.8000,47.2000) -- (228.9000,47.4000) -- (229.0000,47.6000) -- (229.1000,47.9000) -- (229.2000,48.1000) -- (229.2000,48.3000) -- (229.3000,48.5000) -- (229.4000,48.7000) -- (229.5000,48.8000) -- (229.6000,48.9000) -- (229.7000,49.0000) -- (229.8000,49.1000) -- (229.9000,49.2000) -- (229.9000,49.2000) -- (230.0000,49.2000) -- (230.1000,49.1000) -- (230.2000,49.1000) -- (230.3000,49.0000) -- (230.4000,48.8000) -- (230.5000,48.7000) -- (230.6000,48.5000) -- (230.6000,48.3000) -- (230.7000,48.0000) -- (230.8000,47.8000) -- (230.9000,47.5000) -- (231.0000,47.2000) -- (231.1000,46.9000) -- (231.2000,46.5000) -- (231.2000,46.2000) -- (231.3000,45.8000) -- (231.4000,45.4000) -- (231.5000,45.1000) -- (231.6000,44.7000) -- (231.7000,44.3000) -- (231.8000,44.0000) -- (231.9000,43.6000) -- (231.9000,43.3000) -- (232.0000,43.0000) -- (232.1000,42.6000) -- (232.2000,42.4000) -- (232.3000,42.1000) -- (232.4000,41.9000) -- (232.5000,41.7000) -- (232.5000,41.5000) -- (232.6000,41.4000) -- (232.7000,41.3000) -- (232.8000,41.2000) -- (232.9000,41.2000) -- (233.0000,41.3000) -- (233.1000,41.4000) -- (233.2000,41.5000) -- (233.2000,41.7000) -- (233.3000,41.9000) -- (233.4000,42.2000) -- (233.5000,42.6000) -- (233.6000,43.0000) -- (233.7000,43.4000) -- (233.8000,43.9000) -- (233.8000,44.5000) -- (233.9000,45.1000) -- (234.0000,45.7000) -- (234.1000,46.4000) -- (234.2000,47.2000) -- (234.3000,48.0000) -- (234.4000,48.8000) -- (234.5000,49.7000) -- (234.5000,50.6000) -- (234.6000,51.6000) -- (234.7000,52.6000) -- (234.8000,53.6000) -- (234.9000,54.6000) -- (235.0000,55.7000) -- (235.1000,56.8000) -- (235.1000,57.9000) -- (235.2000,59.0000) -- (235.3000,60.2000) -- (235.4000,61.3000) -- (235.5000,62.4000) -- (235.6000,63.6000) -- (235.7000,64.7000) -- (235.8000,65.8000) -- (235.8000,67.0000) -- (235.9000,68.1000) -- (236.0000,69.2000) -- (236.1000,70.2000) -- (236.2000,71.3000) -- (236.3000,72.3000) -- (236.4000,73.3000) -- (236.5000,74.2000) -- (236.5000,75.1000) -- (236.6000,76.0000) -- (236.7000,76.9000) -- (236.8000,77.7000) -- (236.9000,78.5000) -- (237.0000,79.2000) -- (237.1000,79.9000) -- (237.1000,80.5000) -- (237.2000,81.1000) -- (237.3000,81.7000) -- (237.4000,82.2000) -- (237.5000,82.7000) -- (237.6000,83.1000) -- (237.7000,83.5000) -- (237.8000,83.9000) -- (237.8000,84.2000) -- (237.9000,84.4000) -- (238.0000,84.7000) -- (238.1000,84.9000) -- (238.2000,85.1000) -- (238.3000,85.2000) -- (238.4000,85.3000) -- (238.4000,85.4000) -- (238.5000,85.5000) -- (238.6000,85.5000) -- (238.7000,85.5000) -- (238.8000,85.5000) -- (238.9000,85.5000) -- (239.0000,85.5000) -- (239.1000,85.5000) -- (239.1000,85.4000) -- (239.2000,85.4000) -- (239.3000,85.4000) -- (239.4000,85.3000) -- (239.5000,85.3000) -- (239.6000,85.2000) -- (239.7000,85.2000) -- (239.7000,85.1000) -- (239.8000,85.1000) -- (239.9000,85.1000) -- (240.0000,85.1000) -- (240.1000,85.1000) -- (240.2000,85.1000) -- (240.3000,85.1000) -- (240.4000,85.1000) -- (240.4000,85.2000) -- (240.5000,85.2000) -- (240.6000,85.3000) -- (240.7000,85.4000) -- (240.8000,85.4000) -- (240.9000,85.5000) -- (241.0000,85.6000) -- (241.0000,85.7000) -- (241.1000,85.8000) -- (241.2000,85.9000) -- (241.3000,86.0000) -- (241.4000,86.1000) -- (241.5000,86.2000) -- (241.6000,86.3000) -- (241.7000,86.4000) -- (241.7000,86.5000) -- (241.8000,86.5000) -- (241.9000,86.6000) -- (242.0000,86.6000) -- (242.1000,86.6000) -- (242.2000,86.6000) -- (242.3000,86.5000) -- (242.4000,86.5000) -- (242.4000,86.4000) -- (242.5000,86.3000) -- (242.6000,86.1000) -- (242.7000,85.9000) -- (242.8000,85.7000) -- (242.9000,85.4000) -- (243.0000,85.1000) -- (243.0000,84.7000) -- (243.1000,84.3000) -- (243.2000,83.9000) -- (243.3000,83.4000) -- (243.4000,82.9000) -- (243.5000,82.4000) -- (243.6000,81.8000) -- (243.7000,81.1000) -- (243.7000,80.4000) -- (243.8000,79.7000) -- (243.9000,78.9000) -- (244.0000,78.1000) -- (244.1000,77.3000) -- (244.2000,76.4000) -- (244.3000,75.5000) -- (244.3000,74.5000) -- (244.4000,73.6000) -- (244.5000,72.6000) -- (244.6000,71.6000) -- (244.7000,70.5000) -- (244.8000,69.5000) -- (244.9000,68.4000) -- (245.0000,67.3000) -- (245.0000,66.2000) -- (245.1000,65.1000) -- (245.2000,64.0000) -- (245.3000,62.9000) -- (245.4000,61.9000) -- (245.5000,60.8000) -- (245.6000,59.7000) -- (245.7000,58.7000) -- (245.7000,57.7000) -- (245.8000,56.7000) -- (245.9000,55.7000) -- (246.0000,54.7000) -- (246.1000,53.8000) -- (246.2000,52.9000) -- (246.3000,52.1000) -- (246.3000,51.3000) -- (246.4000,50.5000) -- (246.5000,49.8000) -- (246.6000,49.1000) -- (246.7000,48.5000) -- (246.8000,47.9000) -- (246.9000,47.3000) -- (247.0000,46.9000) -- (247.0000,46.4000) -- (247.1000,46.0000) -- (247.2000,45.7000) -- (247.3000,45.4000) -- (247.4000,45.1000) -- (247.5000,44.9000) -- (247.6000,44.8000) -- (247.6000,44.7000) -- (247.7000,44.6000) -- (247.8000,44.6000) -- (247.9000,44.6000) -- (248.0000,44.6000) -- (248.1000,44.7000) -- (248.2000,44.8000) -- (248.3000,45.0000) -- (248.3000,45.1000) -- (248.4000,45.3000) -- (248.5000,45.5000) -- (248.6000,45.7000) -- (248.7000,46.0000) -- (248.8000,46.2000) -- (248.9000,46.5000) -- (248.9000,46.7000) -- (249.0000,47.0000) -- (249.1000,47.3000) -- (249.2000,47.5000) -- (249.3000,47.7000) -- (249.4000,48.0000) -- (249.5000,48.2000) -- (249.6000,48.4000) -- (249.6000,48.6000) -- (249.7000,48.8000) -- (249.8000,48.9000) -- (249.9000,49.0000) -- (250.0000,49.1000) -- (250.1000,49.2000) -- (250.2000,49.2000) -- (250.3000,49.2000) -- (250.3000,49.2000) -- (250.4000,49.1000) -- (250.5000,49.0000) -- (250.6000,48.9000) -- (250.7000,48.8000) -- (250.8000,48.6000) -- (250.9000,48.4000) -- (250.9000,48.2000) -- (251.0000,47.9000) -- (251.1000,47.7000) -- (251.2000,47.4000) -- (251.3000,47.1000) -- (251.4000,46.7000) -- (251.5000,46.4000) -- (251.6000,46.0000) -- (251.6000,45.7000) -- (251.7000,45.3000) -- (251.8000,44.9000) -- (251.9000,44.5000);



  \end{scope}
  \begin{scope}[scale=1.006,draw=black,line join=bevel,line cap=rect,line width=0.800pt]
  \end{scope}
  \begin{scope}[scale=1.006,draw=black,line join=bevel,line cap=rect,line width=0.800pt]
  \end{scope}
  \begin{scope}[cm={{1.00588,0.0,0.0,1.00588,(-444.19503,-9.0)}},draw=black,line join=round,line cap=round,line width=0.480pt]
    \path[draw] (165.5000,13.5000) -- (165.5000,95.5000) -- (251.5000,95.5000) -- (251.5000,13.5000) -- (165.5000,13.5000);



  \end{scope}
  \begin{scope}[cm={{1.00588,0.0,0.0,1.00588,(-334.50007,118.4584)}},draw=ca0a0a4,dash pattern=on 0.40pt off 0.80pt,line join=round,line cap=round,line width=0.400pt]
    \path[draw] (56.5000,194.5000) -- (251.5000,194.5000);



  \end{scope}
  \begin{scope}[cm={{1.00588,0.0,0.0,1.00588,(-334.50007,118.4584)}},draw=black,line join=round,line cap=round,line width=0.480pt]
    \path[draw] (56.5000,194.5000) -- (59.5000,194.5000);



    \path[draw] (251.5000,194.5000) -- (248.5000,194.5000);



  \end{scope}
  \begin{scope}[scale=1.006,draw=black,line join=bevel,line cap=rect,line width=0.800pt]
  \end{scope}
  \begin{scope}[cm={{1.00588,0.0,0.0,1.00588,(39.2294,199.165)}},draw=black,line join=bevel,line cap=rect,line width=0.800pt]
  \end{scope}
  \begin{scope}[cm={{1.00588,0.0,0.0,1.00588,(39.2294,199.165)}},draw=black,line join=bevel,line cap=rect,line width=0.800pt]
  \end{scope}
  \begin{scope}[cm={{1.00588,0.0,0.0,1.00588,(39.2294,199.165)}},draw=black,line join=bevel,line cap=rect,line width=0.800pt]
  \end{scope}
  \begin{scope}[cm={{1.00588,0.0,0.0,1.00588,(39.2294,199.165)}},draw=black,line join=bevel,line cap=rect,line width=0.800pt]
  \end{scope}
  \begin{scope}[cm={{1.00588,0.0,0.0,1.00588,(39.2294,199.165)}},draw=black,line join=bevel,line cap=rect,line width=0.800pt]
  \end{scope}
  \begin{scope}[cm={{1.00588,0.0,0.0,1.00588,(-293.82403,314.665)}},draw=black,line join=bevel,line cap=rect,line width=0.800pt]
    \path[fill=black] (0.0000,0.0000) node[above right] (text658) {27};



  \end{scope}
  \begin{scope}[cm={{1.00588,0.0,0.0,1.00588,(39.2294,199.165)}},draw=black,line join=bevel,line cap=rect,line width=0.800pt]
  \end{scope}
  \begin{scope}[scale=1.006,draw=black,line join=bevel,line cap=rect,line width=0.800pt]
  \end{scope}
  \begin{scope}[cm={{1.00588,0.0,0.0,1.00588,(-334.50007,118.4584)}},draw=ca0a0a4,dash pattern=on 0.40pt off 0.80pt,line join=round,line cap=round,line width=0.400pt]
    \path[draw] (56.5000,171.5000) -- (251.5000,171.5000);



  \end{scope}
  \begin{scope}[cm={{1.00588,0.0,0.0,1.00588,(-334.50007,118.4584)}},draw=black,line join=round,line cap=round,line width=0.480pt]
    \path[draw] (56.5000,171.5000) -- (59.5000,171.5000);



    \path[draw] (251.5000,171.5000) -- (248.5000,171.5000);



  \end{scope}
  \begin{scope}[scale=1.006,draw=black,line join=bevel,line cap=rect,line width=0.800pt]
  \end{scope}
  \begin{scope}[cm={{1.00588,0.0,0.0,1.00588,(40.2353,177.035)}},draw=black,line join=bevel,line cap=rect,line width=0.800pt]
  \end{scope}
  \begin{scope}[cm={{1.00588,0.0,0.0,1.00588,(40.2353,177.035)}},draw=black,line join=bevel,line cap=rect,line width=0.800pt]
  \end{scope}
  \begin{scope}[cm={{1.00588,0.0,0.0,1.00588,(40.2353,177.035)}},draw=black,line join=bevel,line cap=rect,line width=0.800pt]
  \end{scope}
  \begin{scope}[cm={{1.00588,0.0,0.0,1.00588,(40.2353,177.035)}},draw=black,line join=bevel,line cap=rect,line width=0.800pt]
  \end{scope}
  \begin{scope}[cm={{1.00588,0.0,0.0,1.00588,(40.2353,177.035)}},draw=black,line join=bevel,line cap=rect,line width=0.800pt]
  \end{scope}
  \begin{scope}[cm={{1.00588,0.0,0.0,1.00588,(-293.9107,292.535)}},draw=black,line join=bevel,line cap=rect,line width=0.800pt]
    \path[fill=black] (0.5262,0.0000) node[above right] (text688) {31};



  \end{scope}
  \begin{scope}[cm={{1.00588,0.0,0.0,1.00588,(40.2353,177.035)}},draw=black,line join=bevel,line cap=rect,line width=0.800pt]
  \end{scope}
  \begin{scope}[scale=1.006,draw=black,line join=bevel,line cap=rect,line width=0.800pt]
  \end{scope}
  \begin{scope}[cm={{1.00588,0.0,0.0,1.00588,(-334.50007,118.4584)}},draw=ca0a0a4,dash pattern=on 0.40pt off 0.80pt,line join=round,line cap=round,line width=0.400pt]
    \path[draw] (56.5000,149.5000) -- (251.5000,149.5000);



  \end{scope}
  \begin{scope}[cm={{1.00588,0.0,0.0,1.00588,(-334.50007,118.4584)}},draw=black,line join=round,line cap=round,line width=0.480pt]
    \path[draw] (56.5000,149.5000) -- (59.5000,149.5000);



    \path[draw] (251.5000,149.5000) -- (248.5000,149.5000);



  \end{scope}
  \begin{scope}[scale=1.006,draw=black,line join=bevel,line cap=rect,line width=0.800pt]
  \end{scope}
  \begin{scope}[cm={{1.00588,0.0,0.0,1.00588,(40.2353,153.9)}},draw=black,line join=bevel,line cap=rect,line width=0.800pt]
  \end{scope}
  \begin{scope}[cm={{1.00588,0.0,0.0,1.00588,(40.2353,153.9)}},draw=black,line join=bevel,line cap=rect,line width=0.800pt]
  \end{scope}
  \begin{scope}[cm={{1.00588,0.0,0.0,1.00588,(40.2353,153.9)}},draw=black,line join=bevel,line cap=rect,line width=0.800pt]
  \end{scope}
  \begin{scope}[cm={{1.00588,0.0,0.0,1.00588,(40.2353,153.9)}},draw=black,line join=bevel,line cap=rect,line width=0.800pt]
  \end{scope}
  \begin{scope}[cm={{1.00588,0.0,0.0,1.00588,(40.2353,153.9)}},draw=black,line join=bevel,line cap=rect,line width=0.800pt]
  \end{scope}
  \begin{scope}[cm={{1.00588,0.0,0.0,1.00588,(-293.73551,270.9)}},draw=black,line join=bevel,line cap=rect,line width=0.800pt]
    \path[fill=black] (0.0000,0.0000) node[above right] (text718) {35};



  \end{scope}
  \begin{scope}[cm={{1.00588,0.0,0.0,1.00588,(40.2353,153.9)}},draw=black,line join=bevel,line cap=rect,line width=0.800pt]
  \end{scope}
  \begin{scope}[scale=1.006,draw=black,line join=bevel,line cap=rect,line width=0.800pt]
  \end{scope}
  \begin{scope}[cm={{1.00588,0.0,0.0,1.00588,(-334.50007,-2.87674)}},draw=ca0a0a4,dash pattern=on 0.40pt off 0.80pt,line join=round,line cap=round,line width=0.400pt]
    \path[draw] (56.5000,126.5000) -- (195.5000,126.5000);



    \path[draw] (246.5000,126.5000) -- (251.5000,126.5000);



  \end{scope}
  \begin{scope}[cm={{1.00588,0.0,0.0,1.00588,(-334.50007,118.4584)}},draw=black,line join=round,line cap=round,line width=0.480pt]
    \path[draw] (56.5000,126.5000) -- (59.5000,126.5000);



    \path[draw] (251.5000,126.5000) -- (248.5000,126.5000);



  \end{scope}
  \begin{scope}[scale=1.006,draw=black,line join=bevel,line cap=rect,line width=0.800pt]
  \end{scope}
  \begin{scope}[cm={{1.00588,0.0,0.0,1.00588,(39.2294,131.771)}},draw=black,line join=bevel,line cap=rect,line width=0.800pt]
  \end{scope}
  \begin{scope}[cm={{1.00588,0.0,0.0,1.00588,(39.2294,131.771)}},draw=black,line join=bevel,line cap=rect,line width=0.800pt]
  \end{scope}
  \begin{scope}[cm={{1.00588,0.0,0.0,1.00588,(39.2294,131.771)}},draw=black,line join=bevel,line cap=rect,line width=0.800pt]
  \end{scope}
  \begin{scope}[cm={{1.00588,0.0,0.0,1.00588,(39.2294,131.771)}},draw=black,line join=bevel,line cap=rect,line width=0.800pt]
  \end{scope}
  \begin{scope}[cm={{1.00588,0.0,0.0,1.00588,(39.2294,131.771)}},draw=black,line join=bevel,line cap=rect,line width=0.800pt]
  \end{scope}
  \begin{scope}[cm={{1.00588,0.0,0.0,1.00588,(-293.9045,245.771)}},draw=black,line join=bevel,line cap=rect,line width=0.800pt]
    \path[fill=black] (0.0000,0.0000) node[above right] (text750) {39};



  \end{scope}
  \begin{scope}[cm={{1.00588,0.0,0.0,1.00588,(39.2294,131.771)}},draw=black,line join=bevel,line cap=rect,line width=0.800pt]
  \end{scope}
  \begin{scope}[scale=1.006,draw=black,line join=bevel,line cap=rect,line width=0.800pt]
  \end{scope}
  \begin{scope}[cm={{1.00588,0.0,0.0,1.00588,(-334.50007,118.4584)}},draw=ca0a0a4,dash pattern=on 0.40pt off 0.80pt,line join=round,line cap=round,line width=0.400pt]
    \path[draw] (56.5000,200.5000) -- (56.5000,115.5000);



  \end{scope}
  \begin{scope}[cm={{1.00588,0.0,0.0,1.00588,(-334.50007,118.4584)}},draw=black,line join=round,line cap=round,line width=0.480pt]
    \path[draw] (56.5000,200.5000) -- (56.5000,198.5000);



    \path[draw] (56.5000,115.5000) -- (56.5000,118.5000);



  \end{scope}
  \begin{scope}[scale=1.006,draw=black,line join=bevel,line cap=rect,line width=0.800pt]
  \end{scope}
  \begin{scope}[cm={{1.00588,0.0,0.0,1.00588,(53.3118,217.271)}},draw=black,line join=bevel,line cap=rect,line width=0.800pt]
  \end{scope}
  \begin{scope}[cm={{1.00588,0.0,0.0,1.00588,(53.3118,217.271)}},draw=black,line join=bevel,line cap=rect,line width=0.800pt]
  \end{scope}
  \begin{scope}[cm={{1.00588,0.0,0.0,1.00588,(53.3118,217.271)}},draw=black,line join=bevel,line cap=rect,line width=0.800pt]
  \end{scope}
  \begin{scope}[cm={{1.00588,0.0,0.0,1.00588,(53.3118,217.271)}},draw=black,line join=bevel,line cap=rect,line width=0.800pt]
  \end{scope}
  \begin{scope}[cm={{1.00588,0.0,0.0,1.00588,(53.3118,217.271)}},draw=black,line join=bevel,line cap=rect,line width=0.800pt]
  \end{scope}
  \begin{scope}[cm={{1.00588,0.0,0.0,1.00588,(-279.68826,331.271)}},draw=black,line join=bevel,line cap=rect,line width=0.800pt]
    \path[fill=black] (0.0000,0.0000) node[above right] (text780) {0};



  \end{scope}
  \begin{scope}[cm={{1.00588,0.0,0.0,1.00588,(53.3118,217.271)}},draw=black,line join=bevel,line cap=rect,line width=0.800pt]
  \end{scope}
  \begin{scope}[scale=1.006,draw=black,line join=bevel,line cap=rect,line width=0.800pt]
  \end{scope}
  \begin{scope}[cm={{1.00588,0.0,0.0,1.00588,(-334.50007,118.4584)}},draw=ca0a0a4,dash pattern=on 0.40pt off 0.80pt,line join=round,line cap=round,line width=0.400pt]
    \path[draw] (85.5000,200.5000) -- (85.5000,115.5000);



  \end{scope}
  \begin{scope}[cm={{1.00588,0.0,0.0,1.00588,(-334.50007,118.4584)}},draw=black,line join=round,line cap=round,line width=0.480pt]
    \path[draw] (85.5000,200.5000) -- (85.5000,198.5000);



    \path[draw] (85.5000,115.5000) -- (85.5000,118.5000);



  \end{scope}
  \begin{scope}[scale=1.006,draw=black,line join=bevel,line cap=rect,line width=0.800pt]
  \end{scope}
  \begin{scope}[cm={{1.00588,0.0,0.0,1.00588,(82.4824,217.271)}},draw=black,line join=bevel,line cap=rect,line width=0.800pt]
  \end{scope}
  \begin{scope}[cm={{1.00588,0.0,0.0,1.00588,(82.4824,217.271)}},draw=black,line join=bevel,line cap=rect,line width=0.800pt]
  \end{scope}
  \begin{scope}[cm={{1.00588,0.0,0.0,1.00588,(82.4824,217.271)}},draw=black,line join=bevel,line cap=rect,line width=0.800pt]
  \end{scope}
  \begin{scope}[cm={{1.00588,0.0,0.0,1.00588,(82.4824,217.271)}},draw=black,line join=bevel,line cap=rect,line width=0.800pt]
  \end{scope}
  \begin{scope}[cm={{1.00588,0.0,0.0,1.00588,(82.4824,217.271)}},draw=black,line join=bevel,line cap=rect,line width=0.800pt]
  \end{scope}
  \begin{scope}[cm={{1.00588,0.0,0.0,1.00588,(-250.51767,331.38366)}},draw=black,line join=bevel,line cap=rect,line width=0.800pt]
    \path[fill=black] (0.0000,0.0000) node[above right] (text810) {1};



  \end{scope}
  \begin{scope}[cm={{1.00588,0.0,0.0,1.00588,(82.4824,217.271)}},draw=black,line join=bevel,line cap=rect,line width=0.800pt]
  \end{scope}
  \begin{scope}[scale=1.006,draw=black,line join=bevel,line cap=rect,line width=0.800pt]
  \end{scope}
  \begin{scope}[cm={{1.00588,0.0,0.0,1.00588,(-334.50007,118.4584)}},draw=ca0a0a4,dash pattern=on 0.40pt off 0.80pt,line join=round,line cap=round,line width=0.400pt]
    \path[draw] (114.5000,200.5000) -- (114.5000,115.5000);



  \end{scope}
  \begin{scope}[cm={{1.00588,0.0,0.0,1.00588,(-334.50007,118.4584)}},draw=black,line join=round,line cap=round,line width=0.480pt]
    \path[draw] (114.5000,200.5000) -- (114.5000,198.5000);



    \path[draw] (114.5000,115.5000) -- (114.5000,118.5000);



  \end{scope}
  \begin{scope}[scale=1.006,draw=black,line join=bevel,line cap=rect,line width=0.800pt]
  \end{scope}
  \begin{scope}[cm={{1.00588,0.0,0.0,1.00588,(111.653,217.271)}},draw=black,line join=bevel,line cap=rect,line width=0.800pt]
  \end{scope}
  \begin{scope}[cm={{1.00588,0.0,0.0,1.00588,(111.653,217.271)}},draw=black,line join=bevel,line cap=rect,line width=0.800pt]
  \end{scope}
  \begin{scope}[cm={{1.00588,0.0,0.0,1.00588,(111.653,217.271)}},draw=black,line join=bevel,line cap=rect,line width=0.800pt]
  \end{scope}
  \begin{scope}[cm={{1.00588,0.0,0.0,1.00588,(111.653,217.271)}},draw=black,line join=bevel,line cap=rect,line width=0.800pt]
  \end{scope}
  \begin{scope}[cm={{1.00588,0.0,0.0,1.00588,(111.653,217.271)}},draw=black,line join=bevel,line cap=rect,line width=0.800pt]
  \end{scope}
  \begin{scope}[cm={{1.00588,0.0,0.0,1.00588,(-221.34706,331.38366)}},draw=black,line join=bevel,line cap=rect,line width=0.800pt]
    \path[fill=black] (0.0000,0.0000) node[above right] (text840) {2};



  \end{scope}
  \begin{scope}[cm={{1.00588,0.0,0.0,1.00588,(111.653,217.271)}},draw=black,line join=bevel,line cap=rect,line width=0.800pt]
  \end{scope}
  \begin{scope}[scale=1.006,draw=black,line join=bevel,line cap=rect,line width=0.800pt]
  \end{scope}
  \begin{scope}[cm={{1.00588,0.0,0.0,1.00588,(-334.50007,118.4584)}},draw=ca0a0a4,dash pattern=on 0.40pt off 0.80pt,line join=round,line cap=round,line width=0.400pt]
    \path[draw] (143.5000,200.5000) -- (143.5000,115.5000);



  \end{scope}
  \begin{scope}[cm={{1.00588,0.0,0.0,1.00588,(-334.50007,118.4584)}},draw=black,line join=round,line cap=round,line width=0.480pt]
    \path[draw] (143.5000,200.5000) -- (143.5000,198.5000);



    \path[draw] (143.5000,115.5000) -- (143.5000,118.5000);



  \end{scope}
  \begin{scope}[scale=1.006,draw=black,line join=bevel,line cap=rect,line width=0.800pt]
  \end{scope}
  \begin{scope}[cm={{1.00588,0.0,0.0,1.00588,(142.332,217.271)}},draw=black,line join=bevel,line cap=rect,line width=0.800pt]
  \end{scope}
  \begin{scope}[cm={{1.00588,0.0,0.0,1.00588,(142.332,217.271)}},draw=black,line join=bevel,line cap=rect,line width=0.800pt]
  \end{scope}
  \begin{scope}[cm={{1.00588,0.0,0.0,1.00588,(142.332,217.271)}},draw=black,line join=bevel,line cap=rect,line width=0.800pt]
  \end{scope}
  \begin{scope}[cm={{1.00588,0.0,0.0,1.00588,(142.332,217.271)}},draw=black,line join=bevel,line cap=rect,line width=0.800pt]
  \end{scope}
  \begin{scope}[cm={{1.00588,0.0,0.0,1.00588,(142.332,217.271)}},draw=black,line join=bevel,line cap=rect,line width=0.800pt]
  \end{scope}
  \begin{scope}[cm={{1.00588,0.0,0.0,1.00588,(-190.66806,331.271)}},draw=black,line join=bevel,line cap=rect,line width=0.800pt]
    \path[fill=black] (0.0000,0.0000) node[above right] (text870) {3};



  \end{scope}
  \begin{scope}[cm={{1.00588,0.0,0.0,1.00588,(142.332,217.271)}},draw=black,line join=bevel,line cap=rect,line width=0.800pt]
  \end{scope}
  \begin{scope}[scale=1.006,draw=black,line join=bevel,line cap=rect,line width=0.800pt]
  \end{scope}
  \begin{scope}[cm={{1.00588,0.0,0.0,1.00588,(-334.50007,118.4584)}},draw=ca0a0a4,dash pattern=on 0.40pt off 0.80pt,line join=round,line cap=round,line width=0.400pt]
    \path[draw] (172.5000,200.5000) -- (172.5000,115.5000);



  \end{scope}
  \begin{scope}[cm={{1.00588,0.0,0.0,1.00588,(-334.50007,118.4584)}},draw=black,line join=round,line cap=round,line width=0.480pt]
    \path[draw] (172.5000,200.5000) -- (172.5000,198.5000);



    \path[draw] (172.5000,115.5000) -- (172.5000,118.5000);



  \end{scope}
  \begin{scope}[scale=1.006,draw=black,line join=bevel,line cap=rect,line width=0.800pt]
  \end{scope}
  \begin{scope}[cm={{1.00588,0.0,0.0,1.00588,(171.503,217.271)}},draw=black,line join=bevel,line cap=rect,line width=0.800pt]
  \end{scope}
  \begin{scope}[cm={{1.00588,0.0,0.0,1.00588,(171.503,217.271)}},draw=black,line join=bevel,line cap=rect,line width=0.800pt]
  \end{scope}
  \begin{scope}[cm={{1.00588,0.0,0.0,1.00588,(171.503,217.271)}},draw=black,line join=bevel,line cap=rect,line width=0.800pt]
  \end{scope}
  \begin{scope}[cm={{1.00588,0.0,0.0,1.00588,(171.503,217.271)}},draw=black,line join=bevel,line cap=rect,line width=0.800pt]
  \end{scope}
  \begin{scope}[cm={{1.00588,0.0,0.0,1.00588,(171.503,217.271)}},draw=black,line join=bevel,line cap=rect,line width=0.800pt]
  \end{scope}
  \begin{scope}[cm={{1.00588,0.0,0.0,1.00588,(-161.49705,331.38366)}},draw=black,line join=bevel,line cap=rect,line width=0.800pt]
    \path[fill=black] (0.0000,0.0000) node[above right] (text900) {4};



  \end{scope}
  \begin{scope}[cm={{1.00588,0.0,0.0,1.00588,(171.503,217.271)}},draw=black,line join=bevel,line cap=rect,line width=0.800pt]
  \end{scope}
  \begin{scope}[scale=1.006,draw=black,line join=bevel,line cap=rect,line width=0.800pt]
  \end{scope}
  \begin{scope}[cm={{1.00588,0.0,0.0,1.00588,(-337.51771,-2.87674)}},draw=ca0a0a4,dash pattern=on 0.40pt off 0.80pt,line join=round,line cap=round,line width=0.400pt]
    \path[draw] (201.5000,200.5000) -- (201.5000,129.5000);



    \path[draw] (201.5000,121.5000) -- (201.5000,115.5000);



  \end{scope}
  \begin{scope}[cm={{1.00588,0.0,0.0,1.00588,(-331.48243,11.83512)}},draw=ca0a0a4,dash pattern=on 0.40pt off 0.80pt,line join=round,line cap=round,line width=0.400pt]
    \path[draw] (198.5000,306.5000) -- (198.5000,221.5000);



  \end{scope}
  \begin{scope}[cm={{1.00588,0.0,0.0,1.00588,(-334.50007,118.4584)}},draw=black,line join=round,line cap=round,line width=0.480pt]
    \path[draw] (201.5000,200.5000) -- (201.5000,198.5000);



    \path[draw] (201.5000,115.5000) -- (201.5000,118.5000);



  \end{scope}
  \begin{scope}[scale=1.006,draw=black,line join=bevel,line cap=rect,line width=0.800pt]
  \end{scope}
  \begin{scope}[cm={{1.00588,0.0,0.0,1.00588,(200.674,217.271)}},draw=black,line join=bevel,line cap=rect,line width=0.800pt]
  \end{scope}
  \begin{scope}[cm={{1.00588,0.0,0.0,1.00588,(200.674,217.271)}},draw=black,line join=bevel,line cap=rect,line width=0.800pt]
  \end{scope}
  \begin{scope}[cm={{1.00588,0.0,0.0,1.00588,(200.674,217.271)}},draw=black,line join=bevel,line cap=rect,line width=0.800pt]
  \end{scope}
  \begin{scope}[cm={{1.00588,0.0,0.0,1.00588,(200.674,217.271)}},draw=black,line join=bevel,line cap=rect,line width=0.800pt]
  \end{scope}
  \begin{scope}[cm={{1.00588,0.0,0.0,1.00588,(200.674,217.271)}},draw=black,line join=bevel,line cap=rect,line width=0.800pt]
  \end{scope}
  \begin{scope}[cm={{1.00588,0.0,0.0,1.00588,(-133.82605,331.271)}},draw=black,line join=bevel,line cap=rect,line width=0.800pt]
    \path[fill=black] (1.4912,0.0000) node[above right] (text932) {5};



  \end{scope}
  \begin{scope}[cm={{1.00588,0.0,0.0,1.00588,(200.674,217.271)}},draw=black,line join=bevel,line cap=rect,line width=0.800pt]
  \end{scope}
  \begin{scope}[scale=1.006,draw=black,line join=bevel,line cap=rect,line width=0.800pt]
  \end{scope}
  \begin{scope}[cm={{1.00588,0.0,0.0,1.00588,(-332.48831,-2.87674)}},draw=ca0a0a4,dash pattern=on 0.40pt off 0.80pt,line join=round,line cap=round,line width=0.400pt]
    \path[draw] (231.5000,200.5000) -- (231.5000,129.5000);



    \path[draw] (231.5000,121.5000) -- (231.5000,115.5000);



  \end{scope}
  \begin{scope}[cm={{1.00588,0.0,0.0,1.00588,(-336.51183,11.83512)}},draw=ca0a0a4,dash pattern=on 0.40pt off 0.80pt,line join=round,line cap=round,line width=0.400pt]
    \path[draw] (233.5000,306.5000) -- (233.5000,221.5000);



  \end{scope}
  \begin{scope}[cm={{1.00588,0.0,0.0,1.00588,(-334.50007,118.4584)}},draw=black,line join=round,line cap=round,line width=0.480pt]
    \path[draw] (231.5000,200.5000) -- (231.5000,198.5000);



    \path[draw] (231.5000,115.5000) -- (231.5000,118.5000);



  \end{scope}
  \begin{scope}[scale=1.006,draw=black,line join=bevel,line cap=rect,line width=0.800pt]
  \end{scope}
  \begin{scope}[cm={{1.00588,0.0,0.0,1.00588,(229.341,217.271)}},draw=black,line join=bevel,line cap=rect,line width=0.800pt]
  \end{scope}
  \begin{scope}[cm={{1.00588,0.0,0.0,1.00588,(229.341,217.271)}},draw=black,line join=bevel,line cap=rect,line width=0.800pt]
  \end{scope}
  \begin{scope}[cm={{1.00588,0.0,0.0,1.00588,(229.341,217.271)}},draw=black,line join=bevel,line cap=rect,line width=0.800pt]
  \end{scope}
  \begin{scope}[cm={{1.00588,0.0,0.0,1.00588,(229.341,217.271)}},draw=black,line join=bevel,line cap=rect,line width=0.800pt]
  \end{scope}
  \begin{scope}[cm={{1.00588,0.0,0.0,1.00588,(229.341,217.271)}},draw=black,line join=bevel,line cap=rect,line width=0.800pt]
  \end{scope}
  \begin{scope}[cm={{1.00588,0.0,0.0,1.00588,(-103.65906,331.271)}},draw=black,line join=bevel,line cap=rect,line width=0.800pt]
    \path[fill=black] (0.0000,0.0000) node[above right] (text964) {6};



  \end{scope}
  \begin{scope}[cm={{1.00588,0.0,0.0,1.00588,(229.341,217.271)}},draw=black,line join=bevel,line cap=rect,line width=0.800pt]
  \end{scope}
  \begin{scope}[scale=1.006,draw=black,line join=bevel,line cap=rect,line width=0.800pt]
  \end{scope}
  \begin{scope}[cm={{1.00588,0.0,0.0,1.00588,(-334.50007,118.4584)}},draw=black,line join=round,line cap=round,line width=0.480pt]
    \path[draw] (56.5000,115.5000) -- (56.5000,200.5000) -- (251.5000,200.5000) -- (251.5000,115.5000) -- (56.5000,115.5000);



  \end{scope}
  \begin{scope}[scale=1.006,draw=black,line join=bevel,line cap=rect,line width=0.800pt]
  \end{scope}
  \begin{scope}[scale=1.006,draw=black,line join=bevel,line cap=rect,line width=0.800pt]
  \end{scope}
  \begin{scope}[cm={{1.00588,0.0,0.0,1.00588,(-340.50007,118.4584)}},draw=c00ff00,line join=round,line cap=round,line width=0.480pt]
    \path[draw,even odd rule] (215.5000,6.2015) -- (241.5000,6.2015);



  \end{scope}
  \begin{scope}[cm={{1.00588,0.0,0.0,1.00588,(-331.50007,118.4584)}},fill=cffffff]
    \path[fill,rounded corners=0.0000cm] (231.0000,120.0000) rectangle (242.0000,136.0000);



  \end{scope}
  \begin{scope}[scale=1.006,draw=black,line join=bevel,line cap=rect,line width=0.800pt]
  \end{scope}
  \begin{scope}[scale=1.006,draw=black,line join=bevel,line cap=rect,line width=0.800pt]
  \end{scope}
  \begin{scope}[cm={{1.00588,0.0,0.0,1.00588,(-331.50007,118.4584)}},draw=black,line join=round,line cap=round,line width=0.800pt]
    \path[draw] (231.5000,135.5000) -- (231.5000,119.5000) -- (242.5000,119.5000) -- (242.5000,135.5000) -- (231.5000,135.5000);



  \end{scope}
  \begin{scope}[scale=1.006,draw=black,line join=bevel,line cap=rect,line width=0.800pt]
  \end{scope}
  \begin{scope}[cm={{1.00588,0.0,0.0,1.00588,(236.382,132.776)}},draw=black,line join=bevel,line cap=rect,line width=0.800pt]
  \end{scope}
  \begin{scope}[cm={{1.00588,0.0,0.0,1.00588,(236.382,132.776)}},draw=black,line join=bevel,line cap=rect,line width=0.800pt]
  \end{scope}
  \begin{scope}[cm={{1.00588,0.0,0.0,1.00588,(236.382,132.776)}},draw=black,line join=bevel,line cap=rect,line width=0.800pt]
  \end{scope}
  \begin{scope}[cm={{1.00588,0.0,0.0,1.00588,(236.382,132.776)}},draw=black,line join=bevel,line cap=rect,line width=0.800pt]
  \end{scope}
  \begin{scope}[cm={{1.00588,0.0,0.0,1.00588,(236.382,132.776)}},draw=black,line join=bevel,line cap=rect,line width=0.800pt]
  \end{scope}
  \begin{scope}[cm={{1.00588,0.0,0.0,1.00588,(-94.70451,249.99373)}},draw=black,line join=bevel,line cap=rect,line width=0.800pt]
    \path[fill=black] (0.5272,0.0000) node[above right] (text1004) {\label{fig:ener-dyn-i}i};



  \end{scope}
  \begin{scope}[cm={{1.00588,0.0,0.0,1.00588,(236.382,132.776)}},draw=black,line join=bevel,line cap=rect,line width=0.800pt]
  \end{scope}
  \begin{scope}[scale=1.006,draw=black,line join=bevel,line cap=rect,line width=0.800pt]
  \end{scope}
  \begin{scope}[scale=1.006,draw=black,line join=bevel,line cap=rect,line width=0.800pt]
  \end{scope}
  \begin{scope}[scale=1.006,draw=black,line join=bevel,line cap=rect,line width=0.800pt]
  \end{scope}
  \begin{scope}[cm={{1.00588,0.0,0.0,1.00588,(-334.50007,118.4584)}},draw=black,line join=round,line cap=round,line width=0.480pt]
    \path[draw] (56.1000,127.5000) -- (56.1000,127.5000) -- (56.3000,142.6000) -- (56.5000,149.0000) -- (56.7000,148.4000) -- (56.9000,141.3000) -- (57.1000,134.7000) -- (57.3000,131.3000) -- (57.5000,130.4000) -- (57.7000,131.2000) -- (57.9000,132.8000) -- (58.1000,135.0000) -- (58.3000,137.3000) -- (58.4000,139.1000) -- (58.6000,140.1000) -- (58.8000,140.4000) -- (59.0000,140.4000) -- (59.2000,140.1000) -- (59.4000,139.9000) -- (59.6000,139.7000) -- (59.8000,139.6000) -- (60.0000,139.5000) -- (60.2000,139.5000) -- (60.4000,139.5000) -- (60.6000,139.5000) -- (60.8000,139.5000) -- (61.0000,139.5000) -- (61.2000,139.5000) -- (61.4000,139.5000) -- (61.6000,139.5000) -- (61.8000,139.5000) -- (62.0000,139.5000) -- (62.2000,139.5000) -- (62.4000,139.5000) -- (62.6000,139.5000) -- (62.8000,139.6000) -- (63.0000,140.3000) -- (63.1000,141.9000) -- (63.3000,143.7000) -- (63.5000,145.6000) -- (63.7000,147.5000) -- (63.9000,149.3000) -- (64.1000,151.1000) -- (64.3000,152.8000) -- (64.5000,154.5000) -- (64.7000,156.1000) -- (64.9000,157.6000) -- (65.1000,159.1000) -- (65.3000,160.5000) -- (65.5000,161.7000) -- (65.7000,162.7000) -- (65.9000,163.6000) -- (66.1000,164.2000) -- (66.3000,164.6000) -- (66.5000,164.7000) -- (66.7000,164.5000) -- (66.9000,164.0000) -- (67.1000,163.2000) -- (67.3000,162.2000) -- (67.5000,161.0000) -- (67.7000,159.7000) -- (67.9000,158.2000) -- (68.0000,156.6000) -- (68.2000,154.9000) -- (68.4000,153.2000) -- (68.6000,151.5000) -- (68.8000,149.7000) -- (69.0000,147.3000) -- (69.2000,144.6000) -- (69.4000,142.0000) -- (69.6000,140.1000) -- (69.8000,139.0000) -- (70.0000,138.5000) -- (70.2000,138.5000) -- (70.4000,138.7000) -- (70.6000,139.0000) -- (70.8000,139.2000) -- (71.0000,139.4000) -- (71.2000,139.5000) -- (71.4000,139.5000) -- (71.6000,139.6000) -- (71.8000,139.5000) -- (72.0000,139.5000) -- (72.2000,139.5000) -- (72.4000,139.5000) -- (72.6000,139.5000) -- (72.7000,139.5000) -- (72.9000,139.5000) -- (73.1000,139.5000) -- (73.3000,139.5000) -- (73.5000,139.4000) -- (73.7000,140.8000) -- (73.9000,142.7000) -- (74.1000,142.4000) -- (74.3000,141.0000) -- (74.5000,139.2000) -- (74.7000,137.3000) -- (74.9000,135.6000) -- (75.1000,134.1000) -- (75.3000,132.8000) -- (75.5000,131.6000) -- (75.7000,130.7000) -- (75.9000,129.8000) -- (76.1000,129.1000) -- (76.3000,128.4000) -- (76.5000,127.9000) -- (76.7000,127.4000) -- (76.9000,127.0000) -- (77.1000,126.7000) -- (77.3000,126.4000) -- (77.5000,126.3000) -- (77.6000,126.2000) -- (77.8000,126.2000) -- (78.0000,126.3000) -- (78.2000,126.5000) -- (78.4000,126.7000) -- (78.6000,127.1000) -- (78.8000,127.5000) -- (79.0000,128.0000) -- (79.2000,128.6000) -- (79.4000,129.3000) -- (79.6000,130.0000) -- (79.8000,130.9000) -- (80.0000,131.9000) -- (80.2000,132.9000) -- (80.4000,134.1000) -- (80.6000,135.3000) -- (80.8000,137.0000) -- (81.0000,138.6000) -- (81.2000,139.6000) -- (81.4000,140.1000) -- (81.6000,140.2000) -- (81.8000,140.1000) -- (82.0000,140.0000) -- (82.2000,139.8000) -- (82.4000,139.6000) -- (82.5000,139.6000) -- (82.7000,139.5000) -- (82.9000,139.5000) -- (83.1000,139.5000) -- (83.3000,139.5000) -- (83.5000,139.5000) -- (83.7000,139.5000) -- (83.9000,139.5000) -- (84.1000,139.5000) -- (84.3000,139.5000) -- (84.5000,139.5000) -- (84.7000,139.5000) -- (84.9000,139.5000) -- (85.1000,139.5000) -- (85.3000,139.5000) -- (85.5000,139.7000) -- (85.7000,140.9000) -- (85.9000,142.7000) -- (86.1000,144.6000) -- (86.3000,146.6000) -- (86.5000,148.5000) -- (86.7000,150.3000) -- (86.9000,152.1000) -- (87.1000,153.8000) -- (87.2000,155.5000) -- (87.4000,157.0000) -- (87.6000,158.6000) -- (87.8000,160.0000) -- (88.0000,161.3000) -- (88.2000,162.4000) -- (88.4000,163.3000) -- (88.6000,164.1000) -- (88.8000,164.5000) -- (89.0000,164.7000) -- (89.2000,164.6000) -- (89.4000,164.1000) -- (89.6000,163.4000) -- (89.8000,162.4000) -- (90.0000,161.2000) -- (90.2000,159.8000) -- (90.4000,158.3000) -- (90.6000,156.6000) -- (90.8000,154.9000) -- (91.0000,153.2000) -- (91.2000,151.5000) -- (91.4000,149.6000) -- (91.6000,147.3000) -- (91.8000,144.8000) -- (92.0000,142.3000) -- (92.1000,140.4000) -- (92.3000,139.2000) -- (92.5000,138.7000) -- (92.7000,138.6000) -- (92.9000,138.7000) -- (93.1000,139.0000) -- (93.3000,139.2000) -- (93.5000,139.4000) -- (93.7000,139.5000) -- (93.9000,139.5000) -- (94.1000,139.5000) -- (94.3000,139.5000) -- (94.5000,139.5000) -- (94.7000,139.5000) -- (94.9000,139.5000) -- (95.1000,139.5000) -- (95.3000,139.5000) -- (95.5000,139.5000) -- (95.7000,139.5000) -- (95.9000,139.5000) -- (96.1000,139.5000) -- (96.3000,141.0000) -- (96.5000,142.6000) -- (96.7000,142.3000) -- (96.8000,140.8000) -- (97.0000,139.0000) -- (97.2000,137.1000) -- (97.4000,135.4000) -- (97.6000,133.9000) -- (97.8000,132.6000) -- (98.0000,131.4000) -- (98.2000,130.5000) -- (98.4000,129.6000) -- (98.6000,128.9000) -- (98.8000,128.3000) -- (99.0000,127.7000) -- (99.2000,127.3000) -- (99.4000,126.9000) -- (99.6000,126.6000) -- (99.8000,126.4000) -- (100.0000,126.2000) -- (100.2000,126.2000) -- (100.4000,126.2000) -- (100.6000,126.3000) -- (100.8000,126.6000) -- (101.0000,126.8000) -- (101.2000,127.3000) -- (101.4000,127.9000) -- (101.6000,128.6000) -- (101.7000,129.3000) -- (101.9000,130.1000) -- (102.1000,130.9000) -- (102.3000,131.9000) -- (102.5000,132.9000) -- (102.7000,134.0000) -- (102.9000,135.3000) -- (103.1000,137.1000) -- (103.3000,138.7000) -- (103.5000,139.8000) -- (103.7000,140.2000) -- (103.9000,140.3000) -- (104.1000,140.2000) -- (104.3000,140.0000) -- (104.5000,139.8000) -- (104.7000,139.6000) -- (104.9000,139.5000) -- (105.1000,139.5000) -- (105.3000,139.5000) -- (105.5000,139.5000) -- (105.7000,139.5000) -- (105.9000,139.5000) -- (106.1000,139.5000) -- (106.3000,139.5000) -- (106.4000,139.5000) -- (106.6000,139.5000) -- (106.8000,139.5000) -- (107.0000,139.5000) -- (107.2000,139.5000) -- (107.4000,139.5000) -- (107.6000,139.5000) -- (107.8000,139.7000) -- (108.0000,140.8000) -- (108.2000,142.5000) -- (108.4000,144.4000) -- (108.6000,146.3000) -- (108.8000,148.1000) -- (109.0000,149.8000) -- (109.2000,151.5000) -- (109.4000,153.2000) -- (109.6000,154.8000) -- (109.8000,156.4000) -- (110.0000,157.9000) -- (110.2000,159.3000) -- (110.4000,160.6000) -- (110.6000,161.7000) -- (110.8000,162.7000) -- (111.0000,163.5000) -- (111.1000,164.1000) -- (111.3000,164.5000) -- (111.5000,164.6000) -- (111.7000,164.5000) -- (111.9000,164.1000) -- (112.1000,163.5000) -- (112.3000,162.7000) -- (112.5000,161.7000) -- (112.7000,160.6000) -- (112.9000,159.3000) -- (113.1000,157.9000) -- (113.3000,156.4000) -- (113.5000,154.9000) -- (113.7000,153.3000) -- (113.9000,151.7000) -- (114.1000,149.9000) -- (114.3000,146.7000) -- (114.5000,143.0000) -- (114.7000,140.3000) -- (114.9000,144.4000) -- (115.1000,145.2000) -- (115.3000,145.1000) -- (115.5000,145.4000) -- (115.7000,145.8000) -- (115.9000,146.2000) -- (116.0000,146.4000) -- (116.2000,146.5000) -- (116.4000,146.6000) -- (116.6000,146.6000) -- (116.8000,146.6000) -- (117.0000,146.6000) -- (117.2000,146.6000) -- (117.4000,146.6000) -- (117.6000,146.5000) -- (117.8000,146.5000) -- (118.0000,146.5000) -- (118.2000,146.5000) -- (118.4000,146.5000) -- (118.6000,146.5000) -- (118.8000,146.5000) -- (119.0000,146.7000) -- (119.2000,147.4000) -- (119.4000,147.9000) -- (119.6000,147.5000) -- (119.8000,146.2000) -- (120.0000,144.3000) -- (120.2000,146.9000) -- (120.4000,148.4000) -- (120.6000,147.9000) -- (120.8000,152.5000) -- (120.9000,151.7000) -- (121.1000,150.8000) -- (121.3000,150.0000) -- (121.5000,149.4000) -- (121.7000,148.8000) -- (121.9000,148.1000) -- (122.1000,152.5000) -- (122.3000,154.9000) -- (122.5000,154.5000) -- (122.7000,154.3000) -- (122.9000,154.3000) -- (123.1000,154.3000) -- (123.3000,154.5000) -- (123.5000,154.7000) -- (123.7000,155.0000) -- (123.9000,155.4000) -- (124.1000,155.8000) -- (124.3000,156.6000) -- (124.5000,155.2000) -- (124.7000,150.6000) -- (124.9000,151.6000) -- (125.1000,152.5000) -- (125.3000,153.6000) -- (125.5000,154.7000) -- (125.7000,156.0000) -- (125.8000,157.7000) -- (126.0000,153.9000) -- (126.2000,153.5000) -- (126.4000,154.2000) -- (126.6000,154.4000) -- (126.8000,154.4000) -- (127.0000,159.8000) -- (127.2000,161.1000) -- (127.4000,160.7000) -- (127.6000,160.6000) -- (127.8000,160.6000) -- (128.0000,160.6000) -- (128.2000,160.6000) -- (128.4000,160.6000) -- (128.6000,160.6000) -- (128.8000,160.6000) -- (129.0000,160.6000) -- (129.2000,160.6000) -- (129.4000,160.6000) -- (129.6000,160.6000) -- (129.8000,160.6000) -- (130.0000,160.6000) -- (130.2000,160.6000) -- (130.4000,160.6000) -- (130.5000,160.7000) -- (130.7000,161.7000) -- (130.9000,163.4000) -- (131.1000,165.5000) -- (131.3000,163.4000) -- (131.5000,161.8000) -- (131.7000,163.0000) -- (131.9000,158.9000) -- (132.1000,160.1000) -- (132.3000,161.9000) -- (132.5000,163.6000) -- (132.7000,164.1000) -- (132.9000,159.7000) -- (133.1000,160.7000) -- (133.3000,161.9000) -- (133.5000,162.9000) -- (133.7000,163.7000) -- (133.9000,164.3000) -- (134.1000,164.6000) -- (134.3000,164.6000) -- (134.5000,164.2000) -- (134.7000,165.2000) -- (134.9000,170.0000) -- (135.1000,169.0000) -- (135.2000,167.8000) -- (135.4000,166.2000) -- (135.6000,166.4000) -- (135.8000,170.3000) -- (136.0000,168.4000) -- (136.2000,171.9000) -- (136.4000,172.4000) -- (136.6000,170.3000) -- (136.8000,167.8000) -- (137.0000,165.0000) -- (137.2000,167.9000) -- (137.4000,168.1000) -- (137.6000,166.9000) -- (137.8000,166.6000) -- (138.0000,166.7000) -- (138.2000,166.9000) -- (138.4000,167.2000) -- (138.6000,167.4000) -- (138.8000,167.5000) -- (139.0000,167.6000) -- (139.2000,167.7000) -- (139.4000,167.7000) -- (139.6000,167.7000) -- (139.8000,167.7000) -- (140.0000,167.6000) -- (140.1000,167.6000) -- (140.3000,167.6000) -- (140.5000,167.6000) -- (140.7000,167.6000) -- (140.9000,167.6000) -- (141.1000,167.6000) -- (141.3000,167.7000) -- (141.5000,169.6000) -- (141.7000,171.2000) -- (141.9000,170.6000) -- (142.1000,169.0000) -- (142.3000,167.1000) -- (142.5000,165.1000) -- (142.7000,163.3000) -- (142.9000,161.8000) -- (143.1000,160.5000) -- (143.3000,159.3000) -- (143.5000,158.3000) -- (143.7000,157.5000) -- (143.9000,156.8000) -- (144.1000,156.1000) -- (144.3000,155.6000) -- (144.5000,155.2000) -- (144.7000,154.8000) -- (144.9000,154.6000) -- (145.0000,154.4000) -- (145.2000,154.3000) -- (145.4000,154.3000) -- (145.6000,154.4000) -- (145.8000,154.6000) -- (146.0000,154.8000) -- (146.2000,155.2000) -- (146.4000,155.7000) -- (146.6000,156.2000) -- (146.8000,156.9000) -- (147.0000,157.6000) -- (147.2000,158.4000) -- (147.4000,159.4000) -- (147.6000,160.4000) -- (147.8000,161.6000) -- (148.0000,162.8000) -- (148.2000,164.5000) -- (148.4000,166.2000) -- (148.6000,167.5000) -- (148.8000,168.2000) -- (149.0000,168.4000) -- (149.2000,168.4000) -- (149.4000,168.2000) -- (149.6000,168.0000) -- (149.8000,167.8000) -- (149.9000,167.7000) -- (150.1000,167.6000) -- (150.3000,167.6000) -- (150.5000,167.6000) -- (150.7000,167.6000) -- (150.9000,167.6000) -- (151.1000,167.6000) -- (151.3000,167.6000) -- (151.5000,167.6000) -- (151.7000,167.6000) -- (151.9000,167.6000) -- (152.1000,167.6000) -- (152.3000,167.6000) -- (152.5000,167.6000) -- (152.7000,167.6000) -- (152.9000,167.7000) -- (153.1000,168.6000) -- (153.3000,170.3000) -- (153.5000,172.2000) -- (153.7000,174.1000) -- (153.9000,176.0000) -- (154.1000,177.9000) -- (154.3000,179.6000) -- (154.5000,181.4000) -- (154.6000,183.0000) -- (154.8000,184.6000) -- (155.0000,186.2000) -- (155.2000,187.6000) -- (155.4000,189.0000) -- (155.6000,190.2000) -- (155.8000,191.2000) -- (156.0000,192.0000) -- (156.2000,192.5000) -- (156.4000,192.8000) -- (156.6000,192.8000) -- (156.8000,192.4000) -- (157.0000,191.8000) -- (157.2000,190.9000) -- (157.4000,189.8000) -- (157.6000,188.5000) -- (157.8000,187.0000) -- (158.0000,185.5000) -- (158.2000,183.8000) -- (158.4000,182.1000) -- (158.6000,180.3000) -- (158.8000,178.6000) -- (159.0000,176.5000) -- (159.2000,174.0000) -- (159.3000,171.4000) -- (159.5000,169.2000) -- (159.7000,167.7000) -- (159.9000,166.9000) -- (160.1000,166.7000) -- (160.3000,166.8000) -- (160.5000,167.0000) -- (160.7000,167.2000) -- (160.9000,167.4000) -- (161.1000,167.6000) -- (161.3000,167.6000) -- (161.5000,167.7000) -- (161.7000,167.7000) -- (161.9000,167.7000) -- (162.1000,167.6000) -- (162.3000,167.6000) -- (162.5000,167.6000) -- (162.7000,167.6000) -- (162.9000,167.6000) -- (163.1000,167.6000) -- (163.3000,167.6000) -- (163.5000,167.6000) -- (163.7000,168.2000) -- (163.9000,170.6000) -- (164.1000,171.2000) -- (164.2000,170.0000) -- (164.4000,168.2000) -- (164.6000,166.2000) -- (164.8000,164.3000) -- (165.0000,162.6000) -- (165.2000,161.2000) -- (165.4000,159.9000) -- (165.6000,158.8000) -- (165.8000,157.9000) -- (166.0000,157.1000) -- (166.2000,156.5000) -- (166.4000,155.9000) -- (166.6000,155.4000) -- (166.8000,155.0000) -- (167.0000,154.7000) -- (167.2000,154.5000) -- (167.4000,154.3000) -- (167.6000,154.3000) -- (167.8000,154.3000) -- (168.0000,154.5000) -- (168.2000,154.7000) -- (168.4000,155.0000) -- (168.6000,155.5000) -- (168.8000,156.0000) -- (169.0000,156.6000) -- (169.1000,157.3000) -- (169.3000,158.1000) -- (169.5000,159.0000) -- (169.7000,160.0000) -- (169.9000,161.2000) -- (170.1000,162.4000) -- (170.3000,163.9000) -- (170.5000,165.7000) -- (170.7000,167.1000) -- (170.9000,168.0000) -- (171.1000,168.4000) -- (171.3000,168.4000) -- (171.5000,168.2000) -- (171.7000,168.0000) -- (171.9000,167.8000) -- (172.1000,167.7000) -- (172.3000,167.7000) -- (172.5000,167.6000) -- (172.7000,167.6000) -- (172.9000,167.6000) -- (173.1000,167.6000) -- (173.3000,167.6000) -- (173.5000,167.6000) -- (173.7000,167.6000) -- (173.9000,167.6000) -- (174.0000,167.6000) -- (174.2000,167.6000) -- (174.4000,167.6000) -- (174.6000,167.6000) -- (174.8000,167.6000) -- (175.0000,167.6000) -- (175.2000,168.2000) -- (175.4000,169.6000) -- (175.6000,171.5000) -- (175.8000,173.5000) -- (176.0000,175.5000) -- (176.2000,177.4000) -- (176.4000,179.2000) -- (176.6000,180.9000) -- (176.8000,182.6000) -- (177.0000,184.2000) -- (177.2000,185.8000) -- (177.4000,187.2000) -- (177.6000,188.6000) -- (177.8000,189.8000) -- (178.0000,190.9000) -- (178.2000,191.8000) -- (178.4000,192.4000) -- (178.6000,192.7000) -- (178.7000,192.8000) -- (178.9000,192.5000) -- (179.1000,192.0000) -- (179.3000,191.2000) -- (179.5000,190.1000) -- (179.7000,188.8000) -- (179.9000,187.3000) -- (180.1000,185.7000) -- (180.3000,184.1000) -- (180.5000,182.4000) -- (180.7000,180.6000) -- (180.9000,178.9000) -- (181.1000,176.9000) -- (181.3000,174.5000) -- (181.5000,171.9000) -- (181.7000,169.6000) -- (181.9000,168.0000) -- (182.1000,167.0000) -- (182.3000,166.7000) -- (182.5000,166.8000) -- (182.7000,167.0000) -- (182.9000,167.2000) -- (183.1000,167.4000) -- (183.3000,167.5000) -- (183.4000,167.6000) -- (183.6000,167.7000) -- (183.8000,167.7000) -- (184.0000,167.7000) -- (184.2000,167.7000) -- (184.4000,167.6000) -- (184.6000,167.6000) -- (184.8000,167.6000) -- (185.0000,167.6000) -- (185.2000,167.6000) -- (185.4000,167.6000) -- (185.6000,167.6000) -- (185.8000,168.0000) -- (186.0000,170.4000) -- (186.2000,171.2000) -- (186.4000,170.3000) -- (186.6000,168.5000) -- (186.8000,166.5000) -- (187.0000,164.6000) -- (187.2000,162.8000) -- (187.4000,161.2000) -- (187.6000,159.9000) -- (187.8000,158.7000) -- (188.0000,157.7000) -- (188.2000,156.9000) -- (188.3000,156.2000) -- (188.5000,155.6000) -- (188.7000,155.1000) -- (188.9000,154.7000) -- (189.1000,154.5000) -- (189.3000,154.3000) -- (189.5000,154.2000) -- (189.7000,154.3000) -- (189.9000,154.4000) -- (190.1000,154.7000) -- (190.3000,155.0000) -- (190.5000,155.5000) -- (190.7000,156.0000) -- (190.9000,156.7000) -- (191.1000,157.5000) -- (191.3000,158.4000) -- (191.5000,159.4000) -- (191.7000,160.6000) -- (191.9000,161.8000) -- (192.1000,163.3000) -- (192.3000,165.2000) -- (192.5000,166.8000) -- (192.7000,167.8000) -- (192.9000,168.3000) -- (193.1000,168.4000) -- (193.2000,168.3000) -- (193.4000,168.1000) -- (193.6000,167.9000) -- (193.8000,167.8000) -- (194.0000,167.7000) -- (194.2000,167.6000) -- (194.4000,167.6000) -- (194.6000,167.6000) -- (194.8000,167.6000) -- (195.0000,167.6000) -- (195.2000,167.6000) -- (195.4000,167.6000) -- (195.6000,167.6000) -- (195.8000,167.6000) -- (196.0000,167.6000) -- (196.2000,167.6000) -- (196.4000,167.6000) -- (196.6000,167.6000) -- (196.8000,167.6000) -- (197.0000,168.0000) -- (197.2000,169.3000) -- (197.4000,171.2000) -- (197.6000,173.2000) -- (197.8000,175.2000) -- (197.9000,177.1000) -- (198.1000,178.9000) -- (198.3000,180.7000) -- (198.5000,182.3000) -- (198.7000,184.0000) -- (198.9000,185.5000) -- (199.1000,187.0000) -- (199.3000,188.4000) -- (199.5000,189.7000) -- (199.7000,190.8000) -- (199.9000,191.7000) -- (200.1000,192.3000) -- (200.3000,192.7000) -- (200.5000,192.8000) -- (200.7000,192.6000) -- (200.9000,192.1000) -- (201.1000,191.3000) -- (201.3000,190.2000) -- (201.5000,188.9000) -- (201.7000,187.5000) -- (201.9000,185.9000) -- (202.1000,184.3000) -- (202.3000,182.5000) -- (202.5000,180.8000) -- (202.7000,179.0000) -- (202.8000,177.1000) -- (203.0000,174.7000) -- (203.2000,172.1000) -- (203.4000,169.8000) -- (203.6000,168.1000) -- (203.8000,167.1000) -- (204.0000,166.7000) -- (204.2000,166.7000) -- (204.4000,166.9000) -- (204.6000,167.2000) -- (204.8000,167.4000) -- (205.0000,167.5000) -- (205.2000,167.6000) -- (205.4000,167.7000) -- (205.6000,167.7000) -- (205.8000,167.7000) -- (206.0000,167.7000) -- (206.2000,167.6000) -- (206.4000,167.6000) -- (206.6000,167.6000) -- (206.8000,167.6000) -- (207.0000,167.6000) -- (207.2000,167.6000) -- (207.4000,167.6000) -- (207.5000,167.8000) -- (207.7000,170.3000) -- (207.9000,171.6000) -- (208.1000,170.8000) -- (208.3000,169.0000) -- (208.5000,166.9000) -- (208.7000,164.8000) -- (208.9000,162.9000) -- (209.1000,161.3000) -- (209.3000,160.0000) -- (209.5000,158.8000) -- (209.7000,157.8000) -- (209.9000,157.0000) -- (210.1000,156.3000) -- (210.3000,155.7000) -- (210.5000,155.2000) -- (210.7000,154.8000) -- (210.9000,154.5000) -- (211.1000,154.3000) -- (211.3000,154.3000) -- (211.5000,154.3000) -- (211.7000,154.4000) -- (211.9000,154.6000) -- (212.1000,154.9000) -- (212.3000,155.3000) -- (212.4000,155.9000) -- (212.6000,156.5000) -- (212.8000,157.2000) -- (213.0000,158.1000) -- (213.2000,159.0000) -- (213.4000,160.1000) -- (213.6000,161.3000) -- (213.8000,162.7000) -- (214.0000,164.4000) -- (214.2000,166.3000) -- (214.4000,167.6000) -- (214.6000,168.3000) -- (214.8000,168.5000) -- (215.0000,168.4000) -- (215.2000,168.2000) -- (215.4000,168.0000) -- (215.6000,167.8000) -- (215.8000,167.7000) -- (216.0000,167.6000) -- (216.2000,167.6000) -- (216.4000,167.6000) -- (216.6000,167.6000) -- (216.8000,167.6000) -- (217.0000,167.6000) -- (217.1000,167.6000) -- (217.3000,167.6000) -- (217.5000,167.6000) -- (217.7000,167.6000) -- (217.9000,167.6000) -- (218.1000,167.6000) -- (218.3000,167.6000) -- (218.5000,167.6000) -- (218.7000,167.7000) -- (218.9000,168.7000) -- (219.1000,170.4000) -- (219.3000,172.4000) -- (219.5000,174.3000) -- (219.7000,176.2000) -- (219.9000,178.1000) -- (220.1000,179.8000) -- (220.3000,181.6000) -- (220.5000,183.2000) -- (220.7000,184.8000) -- (220.9000,186.4000) -- (221.1000,187.8000) -- (221.3000,189.1000) -- (221.5000,190.3000) -- (221.7000,191.3000) -- (221.9000,192.0000) -- (222.0000,192.5000) -- (222.2000,192.8000) -- (222.4000,192.7000) -- (222.6000,192.4000) -- (222.8000,191.7000) -- (223.0000,190.8000) -- (223.2000,189.7000) -- (223.4000,188.3000) -- (223.6000,186.8000) -- (223.8000,185.2000) -- (224.0000,183.6000) -- (224.2000,181.8000) -- (224.4000,180.1000) -- (224.6000,178.3000) -- (224.8000,176.2000) -- (225.0000,173.7000) -- (225.2000,171.1000) -- (225.4000,169.0000) -- (225.6000,167.6000) -- (225.8000,166.9000) -- (226.0000,166.7000) -- (226.2000,166.8000) -- (226.4000,167.0000) -- (226.6000,167.3000) -- (226.8000,167.4000) -- (226.9000,167.6000) -- (227.1000,167.6000) -- (227.3000,167.7000) -- (227.5000,167.7000) -- (227.7000,167.7000) -- (227.9000,167.6000) -- (228.1000,167.6000) -- (228.3000,167.6000) -- (228.5000,167.6000) -- (228.7000,167.6000) -- (228.9000,167.6000) -- (229.1000,167.6000) -- (229.3000,167.5000) -- (229.5000,168.5000) -- (229.7000,171.2000) -- (229.9000,171.5000) -- (230.1000,170.2000) -- (230.3000,168.2000) -- (230.5000,166.1000) -- (230.7000,164.1000) -- (230.9000,162.3000) -- (231.1000,160.8000) -- (231.3000,159.5000) -- (231.4000,158.4000) -- (231.6000,157.5000) -- (231.8000,156.7000) -- (232.0000,156.1000) -- (232.2000,155.5000) -- (232.4000,155.1000) -- (232.6000,154.7000) -- (232.8000,154.5000) -- (233.0000,154.3000) -- (233.2000,154.2000) -- (233.4000,154.3000) -- (233.6000,154.4000) -- (233.8000,154.7000) -- (234.0000,155.0000) -- (234.2000,155.5000) -- (234.4000,156.1000) -- (234.6000,156.7000) -- (234.8000,157.5000) -- (235.0000,158.4000) -- (235.2000,159.4000) -- (235.4000,160.5000) -- (235.6000,161.7000) -- (235.8000,163.2000) -- (236.0000,165.1000) -- (236.2000,166.8000) -- (236.3000,167.9000) -- (236.5000,168.4000) -- (236.7000,168.5000) -- (236.9000,168.3000) -- (237.1000,168.1000) -- (237.3000,167.9000) -- (237.5000,167.8000) -- (237.7000,167.7000) -- (237.9000,167.6000) -- (238.1000,167.6000) -- (238.3000,167.6000) -- (238.5000,167.6000) -- (238.7000,167.6000) -- (238.9000,167.6000) -- (239.1000,167.6000) -- (239.3000,167.6000) -- (239.5000,167.6000) -- (239.7000,167.6000) -- (239.9000,167.6000) -- (240.1000,167.6000) -- (240.3000,167.6000) -- (240.5000,167.6000) -- (240.7000,168.0000) -- (240.9000,169.3000) -- (241.1000,171.1000) -- (241.2000,173.0000) -- (241.4000,174.9000) -- (241.6000,176.8000) -- (241.8000,178.6000) -- (242.0000,180.4000) -- (242.2000,182.1000) -- (242.4000,183.7000) -- (242.6000,185.3000) -- (242.8000,186.8000) -- (243.0000,188.2000) -- (243.2000,189.5000) -- (243.4000,190.6000) -- (243.6000,191.5000) -- (243.8000,192.2000) -- (244.0000,192.7000) -- (244.2000,192.8000) -- (244.4000,192.6000) -- (244.6000,192.2000) -- (244.8000,191.5000) -- (245.0000,190.5000) -- (245.2000,189.3000) -- (245.4000,187.9000) -- (245.6000,186.4000) -- (245.8000,184.7000) -- (246.0000,183.0000) -- (246.1000,181.3000) -- (246.3000,179.6000) -- (246.5000,177.8000) -- (246.7000,175.5000) -- (246.9000,172.9000) -- (247.1000,170.4000) -- (247.3000,168.5000) -- (247.5000,167.3000) -- (247.7000,166.8000) -- (247.9000,166.7000) -- (248.1000,166.9000) -- (248.3000,167.1000) -- (248.5000,167.3000) -- (248.7000,167.5000) -- (248.9000,167.6000) -- (249.1000,167.6000) -- (249.3000,167.7000) -- (249.5000,167.7000) -- (249.7000,167.7000) -- (249.9000,167.6000) -- (250.1000,167.6000) -- (250.3000,167.6000) -- (250.5000,167.6000) -- (250.7000,167.6000) -- (250.9000,167.6000) -- (251.0000,167.6000) -- (251.2000,167.6000) -- (251.4000,169.4000) -- (251.6000,171.5000) -- (251.8000,171.2000);



  \end{scope}
  \begin{scope}[scale=1.006,draw=black,line join=bevel,line cap=rect,line width=0.800pt]
  \end{scope}
  \begin{scope}[cm={{1.00588,0.0,0.0,1.00588,(197.153,129.759)}},draw=black,line join=bevel,line cap=rect,line width=0.800pt]
  \end{scope}
  \begin{scope}[cm={{1.00588,0.0,0.0,1.00588,(197.153,129.759)}},draw=black,line join=bevel,line cap=rect,line width=0.800pt]
  \end{scope}
  \begin{scope}[cm={{1.00588,0.0,0.0,1.00588,(197.153,129.759)}},draw=black,line join=bevel,line cap=rect,line width=0.800pt]
  \end{scope}
  \begin{scope}[cm={{1.00588,0.0,0.0,1.00588,(197.153,129.759)}},draw=black,line join=bevel,line cap=rect,line width=0.800pt]
  \end{scope}
  \begin{scope}[cm={{1.00588,0.0,0.0,1.00588,(197.153,129.759)}},draw=black,line join=bevel,line cap=rect,line width=0.800pt]
  \end{scope}
  \begin{scope}[cm={{1.00588,0.0,0.0,1.00588,(-144.59319,129.10182)}},draw=black,line join=bevel,line cap=rect,line width=0.800pt]
    \path[fill=black] (0.0000,0.0000) node[above right] (text1032) {\scriptsize $b_0(i)$};



  \end{scope}
  \begin{scope}[cm={{1.00588,0.0,0.0,1.00588,(197.153,129.759)}},draw=black,line join=bevel,line cap=rect,line width=0.800pt]
  \end{scope}
  \begin{scope}[scale=1.006,draw=black,line join=bevel,line cap=rect,line width=0.800pt]
  \end{scope}
  \begin{scope}[scale=1.006,draw=black,line join=bevel,line cap=rect,line width=0.800pt]
  \end{scope}
  \begin{scope}[scale=1.006,draw=black,line join=bevel,line cap=rect,line width=0.800pt]
  \end{scope}
  \begin{scope}[scale=1.006,draw=black,line join=bevel,line cap=rect,line width=0.800pt]
  \end{scope}
  \begin{scope}[scale=1.006,draw=black,line join=bevel,line cap=rect,line width=0.800pt]
  \end{scope}
  \begin{scope}[cm={{1.00588,0.0,0.0,1.00588,(-334.50007,118.4584)}},draw=c00ff00,line join=round,line cap=round,line width=0.480pt]
    \path[draw] (101.0000,115.7000) -- (101.2000,122.5000) -- (101.7000,123.1000) -- (102.2000,123.7000) -- (102.7000,124.3000) -- (103.2000,124.9000) -- (103.6000,125.5000) -- (104.1000,126.1000) -- (104.6000,126.7000) -- (105.1000,127.3000) -- (105.6000,127.9000) -- (106.1000,128.4000) -- (106.6000,129.0000) -- (107.0000,129.6000) -- (107.5000,130.2000) -- (108.0000,130.8000) -- (108.5000,131.4000) -- (109.0000,132.0000) -- (109.5000,132.6000) -- (110.0000,133.2000) -- (110.4000,133.8000) -- (110.9000,134.3000) -- (111.4000,134.9000) -- (111.9000,135.5000) -- (112.4000,136.1000) -- (112.9000,136.7000) -- (113.4000,137.3000) -- (113.9000,137.9000) -- (114.3000,138.5000) -- (114.8000,139.1000) -- (115.3000,139.7000) -- (115.8000,140.2000) -- (116.3000,140.8000) -- (116.8000,141.4000) -- (117.3000,142.0000) -- (117.7000,142.6000) -- (118.2000,143.2000) -- (118.7000,143.8000) -- (119.2000,144.4000) -- (119.7000,145.0000) -- (120.2000,145.6000) -- (120.7000,146.1000) -- (121.1000,146.7000) -- (121.6000,147.3000) -- (122.1000,147.9000) -- (122.6000,148.5000) -- (123.1000,149.1000) -- (123.6000,149.7000) -- (124.1000,150.3000) -- (124.5000,150.9000) -- (125.0000,151.5000) -- (125.5000,152.0000) -- (126.0000,152.6000) -- (126.5000,153.2000) -- (127.0000,153.8000) -- (127.5000,154.4000) -- (128.0000,155.0000) -- (128.4000,155.6000) -- (128.9000,156.2000) -- (129.4000,156.8000) -- (129.9000,157.4000) -- (130.4000,157.9000) -- (130.9000,158.5000) -- (131.4000,159.1000) -- (131.8000,159.7000) -- (132.3000,160.3000) -- (132.8000,160.9000) -- (133.3000,161.5000) -- (133.8000,162.1000) -- (134.3000,162.7000) -- (134.8000,163.3000) -- (135.2000,163.9000) -- (135.7000,164.4000) -- (136.2000,165.0000) -- (136.7000,165.6000) -- (137.2000,166.2000) -- (137.7000,166.8000) -- (138.2000,167.4000) -- (138.6000,168.0000) -- (139.1000,168.6000) -- (139.6000,169.2000) -- (140.1000,169.7000) -- (140.6000,170.3000) -- (141.1000,170.9000) -- (141.6000,171.5000) -- (142.1000,172.1000) -- (142.5000,172.7000) -- (143.0000,173.3000) -- (143.5000,173.9000) -- (144.0000,174.5000) -- (144.5000,175.1000) -- (145.0000,175.7000) -- (145.5000,176.2000) -- (145.9000,176.8000) -- (146.4000,177.4000) -- (146.9000,178.0000) -- (147.4000,178.6000) -- (147.9000,179.2000) -- (148.4000,179.8000) -- (148.9000,180.4000) -- (149.3000,181.0000) -- (149.8000,181.6000) -- (150.3000,182.1000) -- (150.8000,182.7000) -- (151.3000,183.3000) -- (151.8000,183.9000) -- (152.3000,184.5000) -- (152.7000,185.1000) -- (153.2000,185.7000) -- (153.7000,186.3000) -- (154.2000,186.9000) -- (154.7000,187.5000) -- (155.2000,188.0000) -- (155.7000,188.6000) -- (156.2000,189.2000) -- (156.6000,189.8000) -- (157.1000,190.4000) -- (157.6000,191.0000) -- (158.1000,191.6000) -- (158.6000,192.2000) -- (159.1000,192.8000) -- (159.6000,193.4000) -- (160.0000,193.9000) -- (160.5000,194.5000) -- (161.0000,195.1000) -- (161.5000,195.7000) -- (162.0000,196.3000) -- (162.5000,196.9000) -- (163.0000,197.5000) -- (163.4000,198.1000) -- (163.9000,198.7000) -- (164.4000,199.3000) -- (164.9000,199.8000) -- (165.4000,200.4000) -- (165.5000,200.6000);



  \end{scope}
  \begin{scope}[scale=1.006,draw=black,line join=bevel,line cap=rect,line width=0.800pt]
  \end{scope}
  \begin{scope}[scale=1.006,draw=black,line join=bevel,line cap=rect,line width=0.800pt]
  \end{scope}
  \begin{scope}[cm={{1.00588,0.0,0.0,1.00588,(-334.50007,117.00002)}},draw=black,line join=round,line cap=round,line width=0.480pt]
    \path[shift={(0,1.44988)},draw] (56.5000,115.5000) -- (56.5000,200.5000) -- (251.5000,200.5000) -- (251.5000,115.5000) -- (56.5000,115.5000);



  \end{scope}
  \begin{scope}[cm={{1.00588,0.0,0.0,1.00588,(-334.50007,-109.50002)}},draw=ca0a0a4,dash pattern=on 0.40pt off 0.80pt,line join=round,line cap=round,line width=0.400pt]
    \path[draw] (56.5000,299.5000) -- (251.5000,299.5000);



  \end{scope}
  \begin{scope}[cm={{1.00588,0.0,0.0,1.00588,(-334.50007,-109.50002)}},draw=black,line join=round,line cap=round,line width=0.480pt]
    \path[draw] (56.5000,299.5000) -- (59.5000,299.5000);



    \path[draw] (251.5000,299.5000) -- (248.5000,299.5000);



  \end{scope}
  \begin{scope}[scale=1.006,draw=black,line join=bevel,line cap=rect,line width=0.800pt]
  \end{scope}
  \begin{scope}[cm={{1.00588,0.0,0.0,1.00588,(39.2294,305.788)}},draw=black,line join=bevel,line cap=rect,line width=0.800pt]
  \end{scope}
  \begin{scope}[cm={{1.00588,0.0,0.0,1.00588,(39.2294,305.788)}},draw=black,line join=bevel,line cap=rect,line width=0.800pt]
  \end{scope}
  \begin{scope}[cm={{1.00588,0.0,0.0,1.00588,(39.2294,305.788)}},draw=black,line join=bevel,line cap=rect,line width=0.800pt]
  \end{scope}
  \begin{scope}[cm={{1.00588,0.0,0.0,1.00588,(39.2294,305.788)}},draw=black,line join=bevel,line cap=rect,line width=0.800pt]
  \end{scope}
  \begin{scope}[cm={{1.00588,0.0,0.0,1.00588,(39.2294,305.788)}},draw=black,line join=bevel,line cap=rect,line width=0.800pt]
  \end{scope}
  \begin{scope}[cm={{1.00588,0.0,0.0,1.00588,(-293.82403,194.788)}},draw=black,line join=bevel,line cap=rect,line width=0.800pt]
    \path[fill=black] (0.0000,0.0000) node[above right] (text1086) {27};



  \end{scope}
  \begin{scope}[cm={{1.00588,0.0,0.0,1.00588,(39.2294,305.788)}},draw=black,line join=bevel,line cap=rect,line width=0.800pt]
  \end{scope}
  \begin{scope}[scale=1.006,draw=black,line join=bevel,line cap=rect,line width=0.800pt]
  \end{scope}
  \begin{scope}[cm={{1.00588,0.0,0.0,1.00588,(-334.50007,-109.50002)}},draw=ca0a0a4,dash pattern=on 0.40pt off 0.80pt,line join=round,line cap=round,line width=0.400pt]
    \path[draw] (56.5000,277.5000) -- (251.5000,277.5000);



  \end{scope}
  \begin{scope}[cm={{1.00588,0.0,0.0,1.00588,(-334.50007,-109.50002)}},draw=black,line join=round,line cap=round,line width=0.480pt]
    \path[draw] (56.5000,277.5000) -- (59.5000,277.5000);



    \path[draw] (251.5000,277.5000) -- (248.5000,277.5000);



  \end{scope}
  \begin{scope}[scale=1.006,draw=black,line join=bevel,line cap=rect,line width=0.800pt]
  \end{scope}
  \begin{scope}[cm={{1.00588,0.0,0.0,1.00588,(40.2353,282.653)}},draw=black,line join=bevel,line cap=rect,line width=0.800pt]
  \end{scope}
  \begin{scope}[cm={{1.00588,0.0,0.0,1.00588,(40.2353,282.653)}},draw=black,line join=bevel,line cap=rect,line width=0.800pt]
  \end{scope}
  \begin{scope}[cm={{1.00588,0.0,0.0,1.00588,(40.2353,282.653)}},draw=black,line join=bevel,line cap=rect,line width=0.800pt]
  \end{scope}
  \begin{scope}[cm={{1.00588,0.0,0.0,1.00588,(40.2353,282.653)}},draw=black,line join=bevel,line cap=rect,line width=0.800pt]
  \end{scope}
  \begin{scope}[cm={{1.00588,0.0,0.0,1.00588,(40.2353,282.653)}},draw=black,line join=bevel,line cap=rect,line width=0.800pt]
  \end{scope}
  \begin{scope}[cm={{1.00588,0.0,0.0,1.00588,(-293.73551,171.653)}},draw=black,line join=bevel,line cap=rect,line width=0.800pt]
    \path[fill=black] (0.3520,0.0000) node[above right] (text1116) {31};



  \end{scope}
  \begin{scope}[cm={{1.00588,0.0,0.0,1.00588,(40.2353,282.653)}},draw=black,line join=bevel,line cap=rect,line width=0.800pt]
  \end{scope}
  \begin{scope}[scale=1.006,draw=black,line join=bevel,line cap=rect,line width=0.800pt]
  \end{scope}
  \begin{scope}[cm={{1.00588,0.0,0.0,1.00588,(-334.50007,-109.50002)}},draw=ca0a0a4,dash pattern=on 0.40pt off 0.80pt,line join=round,line cap=round,line width=0.400pt]
    \path[draw] (56.5000,254.5000) -- (251.5000,254.5000);



  \end{scope}
  \begin{scope}[cm={{1.00588,0.0,0.0,1.00588,(-334.50007,-109.50002)}},draw=black,line join=round,line cap=round,line width=0.480pt]
    \path[draw] (56.5000,254.5000) -- (59.5000,254.5000);



    \path[draw] (251.5000,254.5000) -- (248.5000,254.5000);



  \end{scope}
  \begin{scope}[scale=1.006,draw=black,line join=bevel,line cap=rect,line width=0.800pt]
  \end{scope}
  \begin{scope}[cm={{1.00588,0.0,0.0,1.00588,(40.2353,260.524)}},draw=black,line join=bevel,line cap=rect,line width=0.800pt]
  \end{scope}
  \begin{scope}[cm={{1.00588,0.0,0.0,1.00588,(40.2353,260.524)}},draw=black,line join=bevel,line cap=rect,line width=0.800pt]
  \end{scope}
  \begin{scope}[cm={{1.00588,0.0,0.0,1.00588,(40.2353,260.524)}},draw=black,line join=bevel,line cap=rect,line width=0.800pt]
  \end{scope}
  \begin{scope}[cm={{1.00588,0.0,0.0,1.00588,(40.2353,260.524)}},draw=black,line join=bevel,line cap=rect,line width=0.800pt]
  \end{scope}
  \begin{scope}[cm={{1.00588,0.0,0.0,1.00588,(40.2353,260.524)}},draw=black,line join=bevel,line cap=rect,line width=0.800pt]
  \end{scope}
  \begin{scope}[cm={{1.00588,0.0,0.0,1.00588,(-293.73551,149.524)}},draw=black,line join=bevel,line cap=rect,line width=0.800pt]
    \path[fill=black] (0.0000,0.0000) node[above right] (text1146) {35};



  \end{scope}
  \begin{scope}[cm={{1.00588,0.0,0.0,1.00588,(40.2353,260.524)}},draw=black,line join=bevel,line cap=rect,line width=0.800pt]
  \end{scope}
  \begin{scope}[scale=1.006,draw=black,line join=bevel,line cap=rect,line width=0.800pt]
  \end{scope}
  \begin{scope}[cm={{1.00588,0.0,0.0,1.00588,(-334.50007,-109.50002)}},draw=black,line join=round,line cap=round,line width=0.480pt]
    \path[draw] (56.5000,232.5000) -- (59.5000,232.5000);



    \path[draw] (251.5000,232.5000) -- (248.5000,232.5000);



  \end{scope}
  \begin{scope}[scale=1.006,draw=black,line join=bevel,line cap=rect,line width=0.800pt]
  \end{scope}
  \begin{scope}[cm={{1.00588,0.0,0.0,1.00588,(39.2294,237.388)}},draw=black,line join=bevel,line cap=rect,line width=0.800pt]
  \end{scope}
  \begin{scope}[cm={{1.00588,0.0,0.0,1.00588,(39.2294,237.388)}},draw=black,line join=bevel,line cap=rect,line width=0.800pt]
  \end{scope}
  \begin{scope}[cm={{1.00588,0.0,0.0,1.00588,(39.2294,237.388)}},draw=black,line join=bevel,line cap=rect,line width=0.800pt]
  \end{scope}
  \begin{scope}[cm={{1.00588,0.0,0.0,1.00588,(39.2294,237.388)}},draw=black,line join=bevel,line cap=rect,line width=0.800pt]
  \end{scope}
  \begin{scope}[cm={{1.00588,0.0,0.0,1.00588,(39.2294,237.388)}},draw=black,line join=bevel,line cap=rect,line width=0.800pt]
  \end{scope}
  \begin{scope}[cm={{1.00588,0.0,0.0,1.00588,(-293.9045,126.388)}},draw=black,line join=bevel,line cap=rect,line width=0.800pt]
    \path[fill=black] (0.0000,0.0000) node[above right] (text1176) {39};



  \end{scope}
  \begin{scope}[cm={{1.00588,0.0,0.0,1.00588,(39.2294,237.388)}},draw=black,line join=bevel,line cap=rect,line width=0.800pt]
  \end{scope}
  \begin{scope}[scale=1.006,draw=black,line join=bevel,line cap=rect,line width=0.800pt]
  \end{scope}
  \begin{scope}[cm={{1.00588,0.0,0.0,1.00588,(-334.50007,-109.50002)}},draw=ca0a0a4,dash pattern=on 0.40pt off 0.80pt,line join=round,line cap=round,line width=0.400pt]
    \path[draw] (56.5000,306.5000) -- (56.5000,221.5000);



  \end{scope}
  \begin{scope}[cm={{1.00588,0.0,0.0,1.00588,(-334.50007,-109.50002)}},draw=black,line join=round,line cap=round,line width=0.480pt]
    \path[draw] (56.5000,306.5000) -- (56.5000,303.5000);



    \path[draw] (56.5000,221.5000) -- (56.5000,223.5000);



  \end{scope}
  \begin{scope}[scale=1.006,draw=black,line join=bevel,line cap=rect,line width=0.800pt]
  \end{scope}
  \begin{scope}[cm={{1.00588,0.0,0.0,1.00588,(53.3118,322.888)}},draw=black,line join=bevel,line cap=rect,line width=0.800pt]
  \end{scope}
  \begin{scope}[cm={{1.00588,0.0,0.0,1.00588,(53.3118,322.888)}},draw=black,line join=bevel,line cap=rect,line width=0.800pt]
  \end{scope}
  \begin{scope}[cm={{1.00588,0.0,0.0,1.00588,(53.3118,322.888)}},draw=black,line join=bevel,line cap=rect,line width=0.800pt]
  \end{scope}
  \begin{scope}[cm={{1.00588,0.0,0.0,1.00588,(53.3118,322.888)}},draw=black,line join=bevel,line cap=rect,line width=0.800pt]
  \end{scope}
  \begin{scope}[cm={{1.00588,0.0,0.0,1.00588,(53.3118,322.888)}},draw=black,line join=bevel,line cap=rect,line width=0.800pt]
  \end{scope}
  \begin{scope}[cm={{1.00588,0.0,0.0,1.00588,(-281.18826,210.388)}},draw=black,line join=bevel,line cap=rect,line width=0.800pt]
    \path[fill=black] (0.0000,0.0000) node[above right] (text1206) {0};



  \end{scope}
  \begin{scope}[cm={{1.00588,0.0,0.0,1.00588,(53.3118,322.888)}},draw=black,line join=bevel,line cap=rect,line width=0.800pt]
  \end{scope}
  \begin{scope}[scale=1.006,draw=black,line join=bevel,line cap=rect,line width=0.800pt]
  \end{scope}
  \begin{scope}[cm={{1.00588,0.0,0.0,1.00588,(-334.50007,-109.50002)}},draw=ca0a0a4,dash pattern=on 0.40pt off 0.80pt,line join=round,line cap=round,line width=0.400pt]
    \path[draw] (91.5000,306.5000) -- (91.5000,221.5000);



  \end{scope}
  \begin{scope}[cm={{1.00588,0.0,0.0,1.00588,(-334.50007,-109.50002)}},draw=black,line join=round,line cap=round,line width=0.480pt]
    \path[draw] (91.5000,306.5000) -- (91.5000,303.5000);



    \path[draw] (91.5000,221.5000) -- (91.5000,223.5000);



  \end{scope}
  \begin{scope}[scale=1.006,draw=black,line join=bevel,line cap=rect,line width=0.800pt]
  \end{scope}
  \begin{scope}[cm={{1.00588,0.0,0.0,1.00588,(89.5235,322.888)}},draw=black,line join=bevel,line cap=rect,line width=0.800pt]
  \end{scope}
  \begin{scope}[cm={{1.00588,0.0,0.0,1.00588,(89.5235,322.888)}},draw=black,line join=bevel,line cap=rect,line width=0.800pt]
  \end{scope}
  \begin{scope}[cm={{1.00588,0.0,0.0,1.00588,(89.5235,322.888)}},draw=black,line join=bevel,line cap=rect,line width=0.800pt]
  \end{scope}
  \begin{scope}[cm={{1.00588,0.0,0.0,1.00588,(89.5235,322.888)}},draw=black,line join=bevel,line cap=rect,line width=0.800pt]
  \end{scope}
  \begin{scope}[cm={{1.00588,0.0,0.0,1.00588,(89.5235,322.888)}},draw=black,line join=bevel,line cap=rect,line width=0.800pt]
  \end{scope}
  \begin{scope}[cm={{1.00588,0.0,0.0,1.00588,(-244.97657,210.388)}},draw=black,line join=bevel,line cap=rect,line width=0.800pt]
    \path[fill=black] (0.0000,0.0000) node[above right] (text1236) {2};



  \end{scope}
  \begin{scope}[cm={{1.00588,0.0,0.0,1.00588,(89.5235,322.888)}},draw=black,line join=bevel,line cap=rect,line width=0.800pt]
  \end{scope}
  \begin{scope}[scale=1.006,draw=black,line join=bevel,line cap=rect,line width=0.800pt]
  \end{scope}
  \begin{scope}[cm={{1.00588,0.0,0.0,1.00588,(-334.50007,-109.50002)}},draw=ca0a0a4,dash pattern=on 0.40pt off 0.80pt,line join=round,line cap=round,line width=0.400pt]
    \path[draw] (127.5000,306.5000) -- (127.5000,221.5000);



  \end{scope}
  \begin{scope}[cm={{1.00588,0.0,0.0,1.00588,(-334.50007,-109.50002)}},draw=black,line join=round,line cap=round,line width=0.480pt]
    \path[draw] (127.5000,306.5000) -- (127.5000,303.5000);



    \path[draw] (127.5000,221.5000) -- (127.5000,223.5000);



  \end{scope}
  \begin{scope}[scale=1.006,draw=black,line join=bevel,line cap=rect,line width=0.800pt]
  \end{scope}
  \begin{scope}[cm={{1.00588,0.0,0.0,1.00588,(125.232,322.888)}},draw=black,line join=bevel,line cap=rect,line width=0.800pt]
  \end{scope}
  \begin{scope}[cm={{1.00588,0.0,0.0,1.00588,(125.232,322.888)}},draw=black,line join=bevel,line cap=rect,line width=0.800pt]
  \end{scope}
  \begin{scope}[cm={{1.00588,0.0,0.0,1.00588,(125.232,322.888)}},draw=black,line join=bevel,line cap=rect,line width=0.800pt]
  \end{scope}
  \begin{scope}[cm={{1.00588,0.0,0.0,1.00588,(125.232,322.888)}},draw=black,line join=bevel,line cap=rect,line width=0.800pt]
  \end{scope}
  \begin{scope}[cm={{1.00588,0.0,0.0,1.00588,(125.232,322.888)}},draw=black,line join=bevel,line cap=rect,line width=0.800pt]
  \end{scope}
  \begin{scope}[cm={{1.00588,0.0,0.0,1.00588,(-209.26806,210.388)}},draw=black,line join=bevel,line cap=rect,line width=0.800pt]
    \path[fill=black] (0.0000,0.0000) node[above right] (text1266) {4};



  \end{scope}
  \begin{scope}[cm={{1.00588,0.0,0.0,1.00588,(125.232,322.888)}},draw=black,line join=bevel,line cap=rect,line width=0.800pt]
  \end{scope}
  \begin{scope}[scale=1.006,draw=black,line join=bevel,line cap=rect,line width=0.800pt]
  \end{scope}
  \begin{scope}[cm={{1.00588,0.0,0.0,1.00588,(-334.50007,-109.50002)}},draw=ca0a0a4,dash pattern=on 0.40pt off 0.80pt,line join=round,line cap=round,line width=0.400pt]
    \path[draw] (162.5000,306.5000) -- (162.5000,221.5000);



  \end{scope}
  \begin{scope}[cm={{1.00588,0.0,0.0,1.00588,(-334.50007,-109.50002)}},draw=black,line join=round,line cap=round,line width=0.480pt]
    \path[draw] (162.5000,306.5000) -- (162.5000,303.5000);



    \path[draw] (162.5000,221.5000) -- (162.5000,223.5000);



  \end{scope}
  \begin{scope}[scale=1.006,draw=black,line join=bevel,line cap=rect,line width=0.800pt]
  \end{scope}
  \begin{scope}[cm={{1.00588,0.0,0.0,1.00588,(160.941,322.888)}},draw=black,line join=bevel,line cap=rect,line width=0.800pt]
  \end{scope}
  \begin{scope}[cm={{1.00588,0.0,0.0,1.00588,(160.941,322.888)}},draw=black,line join=bevel,line cap=rect,line width=0.800pt]
  \end{scope}
  \begin{scope}[cm={{1.00588,0.0,0.0,1.00588,(160.941,322.888)}},draw=black,line join=bevel,line cap=rect,line width=0.800pt]
  \end{scope}
  \begin{scope}[cm={{1.00588,0.0,0.0,1.00588,(160.941,322.888)}},draw=black,line join=bevel,line cap=rect,line width=0.800pt]
  \end{scope}
  \begin{scope}[cm={{1.00588,0.0,0.0,1.00588,(160.941,322.888)}},draw=black,line join=bevel,line cap=rect,line width=0.800pt]
  \end{scope}
  \begin{scope}[cm={{1.00588,0.0,0.0,1.00588,(-173.55905,210.388)}},draw=black,line join=bevel,line cap=rect,line width=0.800pt]
    \path[fill=black] (0.0000,0.0000) node[above right] (text1296) {6};



  \end{scope}
  \begin{scope}[cm={{1.00588,0.0,0.0,1.00588,(160.941,322.888)}},draw=black,line join=bevel,line cap=rect,line width=0.800pt]
  \end{scope}
  \begin{scope}[scale=1.006,draw=black,line join=bevel,line cap=rect,line width=0.800pt]
  \end{scope}
  \begin{scope}[cm={{1.00588,0.0,0.0,1.00588,(-334.50007,-109.50002)}},draw=black,line join=round,line cap=round,line width=0.480pt]
    \path[draw] (198.5000,306.5000) -- (198.5000,303.5000);



    \path[draw] (198.5000,221.5000) -- (198.5000,223.5000);



  \end{scope}
  \begin{scope}[scale=1.006,draw=black,line join=bevel,line cap=rect,line width=0.800pt]
  \end{scope}
  \begin{scope}[cm={{1.00588,0.0,0.0,1.00588,(196.147,322.888)}},draw=black,line join=bevel,line cap=rect,line width=0.800pt]
  \end{scope}
  \begin{scope}[cm={{1.00588,0.0,0.0,1.00588,(196.147,322.888)}},draw=black,line join=bevel,line cap=rect,line width=0.800pt]
  \end{scope}
  \begin{scope}[cm={{1.00588,0.0,0.0,1.00588,(196.147,322.888)}},draw=black,line join=bevel,line cap=rect,line width=0.800pt]
  \end{scope}
  \begin{scope}[cm={{1.00588,0.0,0.0,1.00588,(196.147,322.888)}},draw=black,line join=bevel,line cap=rect,line width=0.800pt]
  \end{scope}
  \begin{scope}[cm={{1.00588,0.0,0.0,1.00588,(196.147,322.888)}},draw=black,line join=bevel,line cap=rect,line width=0.800pt]
  \end{scope}
  \begin{scope}[cm={{1.00588,0.0,0.0,1.00588,(-138.35305,210.388)}},draw=black,line join=bevel,line cap=rect,line width=0.800pt]
    \path[fill=black] (0.0000,0.0000) node[above right] (text1326) {8};



  \end{scope}
  \begin{scope}[cm={{1.00588,0.0,0.0,1.00588,(196.147,322.888)}},draw=black,line join=bevel,line cap=rect,line width=0.800pt]
  \end{scope}
  \begin{scope}[scale=1.006,draw=black,line join=bevel,line cap=rect,line width=0.800pt]
  \end{scope}
  \begin{scope}[cm={{1.00588,0.0,0.0,1.00588,(-334.50007,-109.50002)}},draw=black,line join=round,line cap=round,line width=0.480pt]
    \path[draw] (233.5000,306.5000) -- (233.5000,303.5000);



    \path[draw] (233.5000,221.5000) -- (233.5000,223.5000);



  \end{scope}
  \begin{scope}[scale=1.006,draw=black,line join=bevel,line cap=rect,line width=0.800pt]
  \end{scope}
  \begin{scope}[cm={{1.00588,0.0,0.0,1.00588,(228.838,322.888)}},draw=black,line join=bevel,line cap=rect,line width=0.800pt]
  \end{scope}
  \begin{scope}[cm={{1.00588,0.0,0.0,1.00588,(228.838,322.888)}},draw=black,line join=bevel,line cap=rect,line width=0.800pt]
  \end{scope}
  \begin{scope}[cm={{1.00588,0.0,0.0,1.00588,(228.838,322.888)}},draw=black,line join=bevel,line cap=rect,line width=0.800pt]
  \end{scope}
  \begin{scope}[cm={{1.00588,0.0,0.0,1.00588,(228.838,322.888)}},draw=black,line join=bevel,line cap=rect,line width=0.800pt]
  \end{scope}
  \begin{scope}[cm={{1.00588,0.0,0.0,1.00588,(228.838,322.888)}},draw=black,line join=bevel,line cap=rect,line width=0.800pt]
  \end{scope}
  \begin{scope}[cm={{1.00588,0.0,0.0,1.00588,(-105.66205,210.388)}},draw=black,line join=bevel,line cap=rect,line width=0.800pt]
    \path[fill=black] (0.0000,0.0000) node[above right] (text1356) {10};



  \end{scope}
  \begin{scope}[cm={{1.00588,0.0,0.0,1.00588,(228.838,322.888)}},draw=black,line join=bevel,line cap=rect,line width=0.800pt]
  \end{scope}
  \begin{scope}[scale=1.006,draw=black,line join=bevel,line cap=rect,line width=0.800pt]
  \end{scope}
  \begin{scope}[cm={{1.00588,0.0,0.0,1.00588,(-334.50007,-109.50002)}},draw=black,line join=round,line cap=round,line width=0.480pt]
    \path[draw] (56.5000,221.5000) -- (56.5000,306.5000) -- (251.5000,306.5000) -- (251.5000,221.5000) -- (56.5000,221.5000);



  \end{scope}
  \begin{scope}[scale=1.006,draw=black,line join=bevel,line cap=rect,line width=0.800pt]
  \end{scope}
  \begin{scope}[scale=1.006,draw=black,line join=bevel,line cap=rect,line width=0.800pt]
  \end{scope}
  \begin{scope}[cm={{1.00588,0.0,0.0,1.00588,(-331.50007,-109.50002)}},fill=cebebeb]
    \path[fill=cebebeb,rounded corners=0.0000cm] (226.0000,225.0000) rectangle (242.0000,241.0000);



  \end{scope}
  \begin{scope}[scale=1.006,draw=black,line join=bevel,line cap=rect,line width=0.800pt]
  \end{scope}
  \begin{scope}[scale=1.006,draw=black,line join=bevel,line cap=rect,line width=0.800pt]
  \end{scope}
  \begin{scope}[cm={{1.00588,0.0,0.0,1.00588,(-331.50007,-109.50002)}},draw=black,line join=round,line cap=round,line width=0.800pt]
    \path[draw] (225.5000,241.5000) -- (225.5000,225.5000) -- (241.5000,225.5000) -- (241.5000,241.5000) -- (225.5000,241.5000);



  \end{scope}
  \begin{scope}[scale=1.006,draw=black,line join=bevel,line cap=rect,line width=0.800pt]
  \end{scope}
  \begin{scope}[cm={{1.00588,0.0,0.0,1.00588,(232.359,238.394)}},draw=black,line join=bevel,line cap=rect,line width=0.800pt]
  \end{scope}
  \begin{scope}[cm={{1.00588,0.0,0.0,1.00588,(232.359,238.394)}},draw=black,line join=bevel,line cap=rect,line width=0.800pt]
  \end{scope}
  \begin{scope}[cm={{1.00588,0.0,0.0,1.00588,(232.359,238.394)}},draw=black,line join=bevel,line cap=rect,line width=0.800pt]
  \end{scope}
  \begin{scope}[cm={{1.00588,0.0,0.0,1.00588,(232.359,238.394)}},draw=black,line join=bevel,line cap=rect,line width=0.800pt]
  \end{scope}
  \begin{scope}[cm={{1.00588,0.0,0.0,1.00588,(232.359,238.394)}},draw=black,line join=bevel,line cap=rect,line width=0.800pt]
  \end{scope}
  \begin{scope}[cm={{1.00588,0.0,0.0,1.00588,(-100.19038,128.3426)}},draw=black,line join=bevel,line cap=rect,line width=0.800pt]
    \path[fill=black] (0.7004,0.0000) node[above right] (text1396) {\label{fig:ener-dyn-ii}ii};



  \end{scope}
  \begin{scope}[cm={{1.00588,0.0,0.0,1.00588,(232.359,238.394)}},draw=black,line join=bevel,line cap=rect,line width=0.800pt]
  \end{scope}
  \begin{scope}[cm={{1.00588,0.0,0.0,1.00588,(130.262,337.976)}},draw=black,line join=bevel,line cap=rect,line width=0.800pt]
  \end{scope}
  \begin{scope}[cm={{1.00588,0.0,0.0,1.00588,(130.262,337.976)}},draw=black,line join=bevel,line cap=rect,line width=0.800pt]
  \end{scope}
  \begin{scope}[cm={{1.00588,0.0,0.0,1.00588,(130.262,337.976)}},draw=black,line join=bevel,line cap=rect,line width=0.800pt]
  \end{scope}
  \begin{scope}[cm={{1.00588,0.0,0.0,1.00588,(130.262,337.976)}},draw=black,line join=bevel,line cap=rect,line width=0.800pt]
  \end{scope}
  \begin{scope}[cm={{1.00588,0.0,0.0,1.00588,(130.262,337.976)}},draw=black,line join=bevel,line cap=rect,line width=0.800pt]
  \end{scope}
  \begin{scope}[cm={{1.00588,0.0,0.0,1.00588,(-149.23807,343.976)}},draw=black,line join=bevel,line cap=rect,line width=0.800pt]
    \path[fill=black] (0.0000,0.0000) node[above right] (text1412) {Time (min)};



  \end{scope}
  \begin{scope}[cm={{1.00588,0.0,0.0,1.00588,(130.262,337.976)}},draw=black,line join=bevel,line cap=rect,line width=0.800pt]
  \end{scope}
  \begin{scope}[scale=1.006,draw=black,line join=bevel,line cap=rect,line width=0.800pt]
  \end{scope}
  \begin{scope}[scale=1.006,draw=black,line join=bevel,line cap=rect,line width=0.800pt]
  \end{scope}
  \begin{scope}[scale=1.006,draw=black,line join=bevel,line cap=rect,line width=0.800pt]
  \end{scope}
  \begin{scope}[cm={{1.00588,0.0,0.0,1.00588,(-334.50007,-109.50002)}},draw=black,line join=round,line cap=round,line width=0.480pt]
    \path[draw] (56.1000,255.4000) -- (56.1000,255.4000) -- (56.3000,261.0000) -- (56.5000,262.0000) -- (56.7000,262.2000) -- (56.9000,262.4000) -- (57.1000,261.4000) -- (57.3000,260.3000) -- (57.5000,259.7000) -- (57.7000,259.7000) -- (57.9000,259.8000) -- (58.1000,259.9000) -- (58.3000,260.0000) -- (58.4000,260.0000) -- (58.6000,260.0000) -- (58.8000,260.0000) -- (59.0000,260.0000) -- (59.2000,260.0000) -- (59.4000,260.0000) -- (59.6000,260.0000) -- (59.8000,260.0000) -- (60.0000,260.0000) -- (60.2000,260.0000) -- (60.4000,260.0000) -- (60.6000,260.2000) -- (60.8000,260.7000) -- (61.0000,261.3000) -- (61.2000,262.2000) -- (61.4000,263.3000) -- (61.6000,264.6000) -- (61.8000,266.1000) -- (62.0000,267.8000) -- (62.2000,269.7000) -- (62.4000,271.8000) -- (62.6000,274.1000) -- (62.8000,276.6000) -- (63.0000,279.2000) -- (63.1000,281.9000) -- (63.3000,284.7000) -- (63.5000,287.5000) -- (63.7000,290.1000) -- (63.9000,292.8000) -- (64.1000,295.5000) -- (64.3000,297.5000) -- (64.5000,298.2000) -- (64.7000,298.2000) -- (64.9000,297.9000) -- (65.1000,297.6000) -- (65.3000,297.5000) -- (65.5000,297.4000) -- (65.7000,297.4000) -- (65.9000,297.4000) -- (66.1000,297.4000) -- (66.3000,297.5000) -- (66.5000,297.5000) -- (66.7000,297.5000) -- (66.9000,297.5000) -- (67.1000,297.4000) -- (67.3000,297.1000) -- (67.5000,297.2000) -- (67.7000,297.2000) -- (67.9000,296.1000) -- (68.0000,294.1000) -- (68.2000,291.2000) -- (68.4000,288.0000) -- (68.6000,284.5000) -- (68.8000,281.1000) -- (69.0000,277.8000) -- (69.2000,274.8000) -- (69.4000,272.0000) -- (69.6000,269.4000) -- (69.8000,267.2000) -- (70.0000,265.3000) -- (70.2000,263.7000) -- (70.4000,262.4000) -- (70.6000,261.3000) -- (70.8000,260.5000) -- (71.0000,260.0000) -- (71.2000,259.6000) -- (71.4000,259.7000) -- (71.6000,259.8000) -- (71.8000,259.9000) -- (72.0000,260.0000) -- (72.2000,260.0000) -- (72.4000,260.0000) -- (72.6000,260.0000) -- (72.7000,260.0000) -- (72.9000,260.0000) -- (73.1000,260.0000) -- (73.3000,260.0000) -- (73.5000,260.0000) -- (73.7000,260.0000) -- (73.9000,260.0000) -- (74.1000,260.0000) -- (74.3000,260.1000) -- (74.5000,260.5000) -- (74.7000,261.1000) -- (74.9000,261.9000) -- (75.1000,262.9000) -- (75.3000,264.1000) -- (75.5000,265.4000) -- (75.7000,267.0000) -- (75.9000,268.9000) -- (76.1000,270.9000) -- (76.3000,273.1000) -- (76.5000,275.5000) -- (76.7000,278.0000) -- (76.9000,280.7000) -- (77.1000,283.4000) -- (77.3000,286.2000) -- (77.4000,288.9000) -- (77.6000,291.5000) -- (77.8000,294.2000) -- (78.0000,296.7000) -- (78.2000,298.1000) -- (78.4000,298.3000) -- (78.6000,298.0000) -- (78.8000,297.7000) -- (79.0000,297.5000) -- (79.2000,297.4000) -- (79.4000,297.4000) -- (79.6000,297.4000) -- (79.8000,297.4000) -- (80.0000,297.4000) -- (80.2000,297.5000) -- (80.4000,297.5000) -- (80.6000,297.5000) -- (80.8000,297.5000) -- (81.0000,298.3000) -- (81.2000,280.4000) -- (81.4000,268.1000) -- (81.6000,268.7000) -- (81.8000,267.3000) -- (82.0000,265.2000) -- (82.2000,262.6000) -- (82.3000,259.7000) -- (82.5000,256.7000) -- (82.7000,253.7000) -- (82.9000,250.7000) -- (83.1000,247.9000) -- (83.3000,245.3000) -- (83.5000,242.9000) -- (83.7000,240.7000) -- (83.9000,238.8000) -- (84.1000,237.2000) -- (84.3000,235.7000) -- (84.5000,234.5000) -- (84.7000,233.5000) -- (84.9000,232.7000) -- (85.1000,232.2000) -- (85.3000,231.7000) -- (85.5000,231.6000) -- (85.7000,231.6000) -- (85.9000,231.7000) -- (86.1000,231.8000) -- (86.3000,231.9000) -- (86.5000,231.9000) -- (86.7000,231.9000) -- (86.9000,231.9000) -- (87.0000,231.9000) -- (87.2000,231.9000) -- (87.4000,231.9000) -- (87.6000,231.9000) -- (87.8000,231.9000) -- (88.0000,231.9000) -- (88.2000,231.9000) -- (88.4000,231.9000) -- (88.6000,232.1000) -- (88.8000,232.5000) -- (89.0000,233.1000) -- (89.2000,234.0000) -- (89.4000,235.0000) -- (89.6000,236.2000) -- (89.8000,237.7000) -- (90.0000,239.3000) -- (90.2000,241.2000) -- (90.4000,243.2000) -- (90.6000,245.4000) -- (90.8000,247.9000) -- (91.0000,250.4000) -- (91.2000,253.1000) -- (91.4000,255.8000) -- (91.6000,258.6000) -- (91.8000,261.3000) -- (91.9000,263.9000) -- (92.1000,266.6000) -- (92.3000,269.0000) -- (92.5000,270.1000) -- (92.7000,270.2000) -- (92.9000,269.8000) -- (93.1000,269.5000) -- (93.3000,269.4000) -- (93.5000,269.3000) -- (93.7000,269.3000) -- (93.9000,269.3000) -- (94.1000,269.3000) -- (94.3000,269.3000) -- (94.5000,269.3000) -- (94.7000,269.3000) -- (94.9000,269.3000) -- (95.1000,269.3000) -- (95.3000,269.1000) -- (95.5000,269.1000) -- (95.7000,269.2000) -- (95.9000,268.5000) -- (96.1000,266.9000) -- (96.3000,264.6000) -- (96.5000,261.9000) -- (96.6000,259.0000) -- (96.8000,255.9000) -- (97.0000,252.9000) -- (97.2000,250.0000) -- (97.4000,247.3000) -- (97.6000,244.7000) -- (97.8000,242.4000) -- (98.0000,240.3000) -- (98.2000,238.4000) -- (98.4000,236.8000) -- (98.6000,235.4000) -- (98.8000,234.2000) -- (99.0000,233.3000) -- (99.2000,232.6000) -- (99.4000,232.0000) -- (99.6000,231.7000) -- (99.8000,231.6000) -- (100.0000,231.7000) -- (100.2000,231.8000) -- (100.4000,231.8000) -- (100.6000,231.9000) -- (100.8000,231.9000) -- (101.0000,231.9000) -- (101.2000,231.9000) -- (101.3000,231.9000) -- (101.5000,231.9000) -- (101.7000,231.9000) -- (101.9000,231.9000) -- (102.1000,231.9000) -- (102.3000,231.9000) -- (102.5000,231.9000) -- (102.7000,231.9000) -- (102.9000,232.2000) -- (103.1000,232.7000) -- (103.3000,233.4000) -- (103.5000,234.3000) -- (103.7000,235.4000) -- (103.9000,236.7000) -- (104.1000,238.2000) -- (104.3000,239.9000) -- (104.5000,241.8000) -- (104.7000,244.0000) -- (104.9000,246.3000) -- (105.1000,248.7000) -- (105.3000,251.4000) -- (105.5000,254.1000) -- (105.7000,256.8000) -- (105.9000,259.6000) -- (106.0000,262.2000) -- (106.2000,264.9000) -- (106.4000,267.6000) -- (106.6000,269.5000) -- (106.8000,270.2000) -- (107.0000,270.0000) -- (107.2000,269.7000) -- (107.4000,269.5000) -- (107.6000,269.3000) -- (107.8000,269.3000) -- (108.0000,269.3000) -- (108.2000,269.3000) -- (108.4000,269.3000) -- (108.6000,269.3000) -- (108.8000,269.3000) -- (109.0000,269.3000) -- (109.2000,269.3000) -- (109.4000,269.2000) -- (109.6000,269.0000) -- (109.8000,269.2000) -- (110.0000,269.0000) -- (110.2000,268.0000) -- (110.4000,266.1000) -- (110.6000,263.6000) -- (110.8000,260.8000) -- (110.9000,257.8000) -- (111.1000,254.8000) -- (111.3000,251.8000) -- (111.5000,248.9000) -- (111.7000,246.2000) -- (111.9000,243.8000) -- (112.1000,241.5000) -- (112.3000,239.5000) -- (112.5000,237.7000) -- (112.7000,236.2000) -- (112.9000,234.9000) -- (113.1000,233.8000) -- (113.3000,233.0000) -- (113.5000,232.3000) -- (113.7000,231.9000) -- (113.9000,231.6000) -- (114.1000,231.6000) -- (114.3000,231.7000) -- (114.5000,231.8000) -- (114.7000,231.9000) -- (114.9000,231.9000) -- (115.1000,231.9000) -- (115.3000,231.9000) -- (115.5000,231.9000) -- (115.6000,231.9000) -- (115.8000,231.9000) -- (116.0000,231.9000) -- (116.2000,231.9000) -- (116.4000,231.9000) -- (116.6000,231.9000) -- (116.8000,231.9000) -- (117.0000,232.0000) -- (117.2000,232.3000) -- (117.4000,232.9000) -- (117.6000,233.7000) -- (117.8000,234.7000) -- (118.0000,235.9000) -- (118.2000,237.2000) -- (118.4000,238.8000) -- (118.6000,240.6000) -- (118.8000,242.6000) -- (119.0000,244.8000) -- (119.2000,247.2000) -- (119.4000,249.8000) -- (119.6000,252.4000) -- (119.8000,255.2000) -- (120.0000,257.9000) -- (120.2000,260.6000) -- (120.4000,263.2000) -- (120.5000,266.0000) -- (120.7000,268.5000) -- (120.9000,269.9000) -- (121.1000,270.2000) -- (121.3000,269.9000) -- (121.5000,269.6000) -- (121.7000,269.4000) -- (121.9000,269.3000) -- (122.1000,269.3000) -- (122.3000,269.3000) -- (122.5000,269.3000) -- (122.7000,269.3000) -- (122.9000,269.3000) -- (123.1000,269.3000) -- (123.3000,269.3000) -- (123.5000,269.3000) -- (123.7000,269.1000) -- (123.9000,269.1000) -- (124.1000,269.2000) -- (124.3000,268.7000) -- (124.5000,267.3000) -- (124.7000,265.2000) -- (124.9000,262.5000) -- (125.1000,259.6000) -- (125.2000,256.6000) -- (125.4000,253.6000) -- (125.6000,250.6000) -- (125.8000,247.8000) -- (126.0000,245.2000) -- (126.2000,242.8000) -- (126.4000,240.7000) -- (126.6000,238.8000) -- (126.8000,237.1000) -- (127.0000,235.7000) -- (127.2000,234.5000) -- (127.4000,233.5000) -- (127.6000,232.7000) -- (127.8000,232.1000) -- (128.0000,231.7000) -- (128.2000,231.6000) -- (128.4000,231.6000) -- (128.6000,231.8000) -- (128.8000,231.8000) -- (129.0000,231.9000) -- (129.2000,231.9000) -- (129.4000,231.9000) -- (129.6000,231.9000) -- (129.8000,231.9000) -- (129.9000,231.9000) -- (130.1000,231.9000) -- (130.3000,231.9000) -- (130.5000,231.9000) -- (130.7000,231.9000) -- (130.9000,231.9000) -- (131.1000,231.9000) -- (131.3000,232.1000) -- (131.5000,232.5000) -- (131.7000,233.2000) -- (131.9000,234.1000) -- (132.1000,235.1000) -- (132.3000,236.4000) -- (132.5000,237.9000) -- (132.7000,239.6000) -- (132.9000,241.4000) -- (133.1000,243.5000) -- (133.3000,245.8000) -- (133.5000,248.2000) -- (133.7000,250.8000) -- (133.9000,253.5000) -- (134.1000,256.3000) -- (134.3000,259.1000) -- (134.5000,261.7000) -- (134.7000,264.3000) -- (134.8000,267.1000) -- (135.0000,269.2000) -- (135.2000,270.1000) -- (135.4000,270.1000) -- (135.6000,269.8000) -- (135.8000,269.5000) -- (136.0000,269.3000) -- (136.2000,269.3000) -- (136.4000,269.3000) -- (136.6000,269.3000) -- (136.8000,269.3000) -- (137.0000,269.3000) -- (137.2000,269.3000) -- (137.4000,269.3000) -- (137.6000,269.3000) -- (137.8000,269.3000) -- (138.0000,269.0000) -- (138.2000,269.2000) -- (138.4000,269.1000) -- (138.6000,268.2000) -- (138.8000,266.5000) -- (139.0000,264.1000) -- (139.2000,261.3000) -- (139.4000,258.4000) -- (139.5000,255.3000) -- (139.7000,252.3000) -- (139.9000,249.4000) -- (140.1000,246.7000) -- (140.3000,244.2000) -- (140.5000,241.9000) -- (140.7000,239.9000) -- (140.9000,238.1000) -- (141.1000,236.5000) -- (141.3000,235.1000) -- (141.5000,234.0000) -- (141.7000,233.1000) -- (141.9000,232.4000) -- (142.1000,231.9000) -- (142.3000,231.6000) -- (142.5000,231.6000) -- (142.7000,231.7000) -- (142.9000,231.8000) -- (143.1000,231.8000) -- (143.3000,231.9000) -- (143.5000,231.9000) -- (143.7000,231.9000) -- (143.9000,231.9000) -- (144.1000,231.9000) -- (144.3000,231.9000) -- (144.4000,231.9000) -- (144.6000,231.9000) -- (144.8000,231.9000) -- (145.0000,231.9000) -- (145.2000,231.9000) -- (145.4000,232.0000) -- (145.6000,232.3000) -- (145.8000,232.8000) -- (146.0000,233.5000) -- (146.2000,234.5000) -- (146.4000,235.6000) -- (146.6000,237.0000) -- (146.8000,238.5000) -- (147.0000,240.3000) -- (147.2000,242.3000) -- (147.4000,244.4000) -- (147.6000,246.8000) -- (147.8000,249.3000) -- (148.0000,252.0000) -- (148.2000,254.7000) -- (148.4000,257.5000) -- (148.6000,260.2000) -- (148.8000,262.8000) -- (149.0000,265.5000) -- (149.1000,268.1000) -- (149.3000,269.8000) -- (149.5000,270.2000) -- (149.7000,270.0000) -- (149.9000,269.6000) -- (150.1000,269.4000) -- (150.3000,269.3000) -- (150.5000,269.3000) -- (150.7000,269.3000) -- (150.9000,269.3000) -- (151.1000,269.3000) -- (151.3000,269.3000) -- (151.5000,269.3000) -- (151.7000,269.3000) -- (151.9000,269.3000) -- (152.1000,269.2000) -- (152.3000,269.0000) -- (152.5000,269.2000) -- (152.7000,268.9000) -- (152.9000,267.6000) -- (153.1000,265.6000) -- (153.3000,263.0000) -- (153.5000,260.1000) -- (153.7000,257.1000) -- (153.8000,254.1000) -- (154.0000,251.1000) -- (154.2000,248.3000) -- (154.4000,245.7000) -- (154.6000,243.2000) -- (154.8000,241.0000) -- (155.0000,239.1000) -- (155.2000,237.1000) -- (155.4000,238.9000) -- (155.6000,242.0000) -- (155.8000,240.7000) -- (156.0000,239.8000) -- (156.2000,239.2000) -- (156.4000,238.8000) -- (156.6000,238.6000) -- (156.8000,238.6000) -- (157.0000,238.8000) -- (157.2000,238.9000) -- (157.4000,238.9000) -- (157.6000,238.9000) -- (157.8000,238.9000) -- (158.0000,238.9000) -- (158.2000,238.9000) -- (158.4000,238.9000) -- (158.6000,238.9000) -- (158.7000,238.9000) -- (158.9000,238.9000) -- (159.1000,238.9000) -- (159.3000,238.9000) -- (159.5000,238.9000) -- (159.7000,239.1000) -- (159.9000,239.5000) -- (160.1000,240.1000) -- (160.3000,240.9000) -- (160.5000,242.2000) -- (160.7000,241.5000) -- (160.9000,237.5000) -- (161.1000,239.2000) -- (161.3000,241.1000) -- (161.5000,243.1000) -- (161.7000,245.4000) -- (161.9000,247.8000) -- (162.1000,250.4000) -- (162.3000,253.0000) -- (162.5000,255.8000) -- (162.7000,258.6000) -- (162.9000,261.3000) -- (163.1000,263.8000) -- (163.3000,266.6000) -- (163.4000,268.9000) -- (163.6000,270.1000) -- (163.8000,270.1000) -- (164.0000,269.8000) -- (164.2000,269.5000) -- (164.4000,269.4000) -- (164.6000,269.3000) -- (164.8000,269.3000) -- (165.0000,269.3000) -- (165.2000,269.3000) -- (165.4000,269.3000) -- (165.6000,269.3000) -- (165.8000,269.3000) -- (166.0000,269.3000) -- (166.2000,269.3000) -- (166.4000,269.0000) -- (166.6000,269.1000) -- (166.8000,269.2000) -- (167.0000,268.5000) -- (167.2000,266.9000) -- (167.4000,264.6000) -- (167.6000,261.9000) -- (167.8000,258.9000) -- (168.0000,255.9000) -- (168.2000,252.9000) -- (168.3000,250.0000) -- (168.5000,246.9000) -- (168.7000,247.8000) -- (168.9000,249.7000) -- (169.1000,247.3000) -- (169.3000,245.3000) -- (169.5000,244.1000) -- (169.7000,248.5000) -- (169.9000,248.4000) -- (170.1000,247.3000) -- (170.3000,246.6000) -- (170.5000,246.1000) -- (170.7000,245.7000) -- (170.9000,245.6000) -- (171.1000,245.7000) -- (171.3000,245.8000) -- (171.5000,245.9000) -- (171.7000,245.9000) -- (171.9000,245.9000) -- (172.1000,245.9000) -- (172.3000,245.9000) -- (172.5000,245.9000) -- (172.7000,245.9000) -- (172.9000,245.9000) -- (173.0000,245.9000) -- (173.2000,245.9000) -- (173.4000,245.9000) -- (173.6000,245.9000) -- (173.8000,246.0000) -- (174.0000,246.2000) -- (174.2000,246.7000) -- (174.4000,247.4000) -- (174.6000,248.4000) -- (174.8000,249.6000) -- (175.0000,245.8000) -- (175.2000,245.0000) -- (175.4000,247.0000) -- (175.6000,249.1000) -- (175.8000,249.5000) -- (176.0000,246.3000) -- (176.2000,248.8000) -- (176.4000,251.4000) -- (176.6000,254.1000) -- (176.8000,256.9000) -- (177.0000,259.6000) -- (177.2000,262.3000) -- (177.4000,264.9000) -- (177.6000,267.7000) -- (177.7000,269.5000) -- (177.9000,270.2000) -- (178.1000,270.0000) -- (178.3000,269.7000) -- (178.5000,269.5000) -- (178.7000,269.3000) -- (178.9000,269.3000) -- (179.1000,269.3000) -- (179.3000,269.3000) -- (179.5000,269.3000) -- (179.7000,269.3000) -- (179.9000,269.3000) -- (180.1000,269.3000) -- (180.3000,269.3000) -- (180.5000,269.2000) -- (180.7000,269.0000) -- (180.9000,269.2000) -- (181.1000,269.0000) -- (181.3000,267.9000) -- (181.5000,266.1000) -- (181.7000,263.6000) -- (181.9000,260.7000) -- (182.1000,257.6000) -- (182.3000,255.1000) -- (182.4000,258.0000) -- (182.6000,255.9000) -- (182.8000,253.5000) -- (183.0000,256.9000) -- (183.2000,255.7000) -- (183.4000,253.5000) -- (183.6000,251.5000) -- (183.8000,253.1000) -- (184.0000,256.2000) -- (184.2000,254.9000) -- (184.4000,254.0000) -- (184.6000,253.4000) -- (184.8000,252.9000) -- (185.0000,252.7000) -- (185.2000,252.7000) -- (185.4000,252.8000) -- (185.6000,252.9000) -- (185.8000,252.9000) -- (186.0000,253.0000) -- (186.2000,252.9000) -- (186.4000,252.9000) -- (186.6000,252.9000) -- (186.8000,252.9000) -- (187.0000,252.9000) -- (187.2000,252.9000) -- (187.3000,252.9000) -- (187.5000,252.9000) -- (187.7000,252.9000) -- (187.9000,253.0000) -- (188.1000,253.1000) -- (188.3000,253.4000) -- (188.5000,254.0000) -- (188.7000,254.8000) -- (188.9000,256.0000) -- (189.1000,255.4000) -- (189.3000,251.3000) -- (189.5000,252.9000) -- (189.7000,254.8000) -- (189.9000,257.0000) -- (190.1000,254.2000) -- (190.3000,254.1000) -- (190.5000,257.1000) -- (190.7000,254.9000) -- (190.9000,255.0000) -- (191.1000,258.0000) -- (191.3000,260.7000) -- (191.5000,263.3000) -- (191.7000,266.0000) -- (191.9000,268.6000) -- (192.0000,269.9000) -- (192.2000,270.2000) -- (192.4000,269.9000) -- (192.6000,269.6000) -- (192.8000,269.4000) -- (193.0000,269.3000) -- (193.2000,269.3000) -- (193.4000,269.3000) -- (193.6000,269.3000) -- (193.8000,269.3000) -- (194.0000,269.3000) -- (194.2000,269.3000) -- (194.4000,269.3000) -- (194.6000,269.3000) -- (194.8000,269.1000) -- (195.0000,269.1000) -- (195.2000,269.2000) -- (195.4000,268.7000) -- (195.6000,267.3000) -- (195.8000,265.1000) -- (196.0000,262.4000) -- (196.2000,259.9000) -- (196.4000,262.5000) -- (196.6000,263.8000) -- (196.8000,264.9000) -- (196.9000,261.5000) -- (197.1000,262.2000) -- (197.3000,264.2000) -- (197.5000,261.7000) -- (197.7000,259.8000) -- (197.9000,258.4000) -- (198.1000,262.7000) -- (198.3000,262.7000) -- (198.5000,261.6000) -- (198.7000,260.8000) -- (198.9000,260.2000) -- (199.1000,259.8000) -- (199.3000,259.7000) -- (199.5000,259.8000) -- (199.7000,259.9000) -- (199.9000,260.0000) -- (200.1000,260.0000) -- (200.3000,260.0000) -- (200.5000,260.0000) -- (200.7000,260.0000) -- (200.9000,260.0000) -- (201.1000,260.0000) -- (201.3000,260.0000) -- (201.5000,260.0000) -- (201.6000,260.0000) -- (201.8000,260.0000) -- (202.0000,260.0000) -- (202.2000,260.0000) -- (202.4000,260.2000) -- (202.6000,260.7000) -- (202.8000,261.3000) -- (203.0000,262.2000) -- (203.2000,263.5000) -- (203.4000,259.7000) -- (203.6000,258.7000) -- (203.8000,260.7000) -- (204.0000,262.7000) -- (204.2000,263.3000) -- (204.4000,259.9000) -- (204.6000,262.5000) -- (204.8000,263.7000) -- (205.0000,260.7000) -- (205.2000,263.5000) -- (205.4000,265.0000) -- (205.6000,261.9000) -- (205.8000,264.3000) -- (206.0000,267.1000) -- (206.2000,269.3000) -- (206.3000,270.1000) -- (206.5000,270.1000) -- (206.7000,269.8000) -- (206.9000,269.5000) -- (207.1000,269.3000) -- (207.3000,269.3000) -- (207.5000,269.3000) -- (207.7000,269.3000) -- (207.9000,269.3000) -- (208.1000,269.3000) -- (208.3000,269.3000) -- (208.5000,269.3000) -- (208.7000,269.3000) -- (208.9000,269.3000) -- (209.1000,269.0000) -- (209.3000,269.2000) -- (209.5000,269.1000) -- (209.7000,268.2000) -- (209.9000,266.4000) -- (210.1000,264.5000) -- (210.3000,266.7000) -- (210.5000,274.9000) -- (210.7000,284.3000) -- (210.9000,280.4000) -- (211.1000,277.6000) -- (211.2000,274.8000) -- (211.4000,272.3000) -- (211.6000,270.0000) -- (211.8000,268.0000) -- (212.0000,266.2000) -- (212.2000,264.6000) -- (212.4000,263.3000) -- (212.6000,262.1000) -- (212.8000,261.2000) -- (213.0000,260.6000) -- (213.2000,260.1000) -- (213.4000,259.7000) -- (213.6000,259.7000) -- (213.8000,259.8000) -- (214.0000,259.9000) -- (214.2000,260.0000) -- (214.4000,260.0000) -- (214.6000,260.0000) -- (214.8000,260.0000) -- (215.0000,260.0000) -- (215.2000,260.0000) -- (215.4000,260.0000) -- (215.6000,260.0000) -- (215.8000,260.0000) -- (215.9000,260.0000) -- (216.1000,260.0000) -- (216.3000,260.0000) -- (216.5000,260.1000) -- (216.7000,260.4000) -- (216.9000,260.9000) -- (217.1000,261.7000) -- (217.3000,262.6000) -- (217.5000,263.8000) -- (217.7000,265.1000) -- (217.9000,266.7000) -- (218.1000,268.4000) -- (218.3000,270.6000) -- (218.5000,268.0000) -- (218.7000,267.6000) -- (218.9000,270.6000) -- (219.1000,268.6000) -- (219.3000,268.5000) -- (219.5000,271.8000) -- (219.7000,269.9000) -- (219.9000,269.5000) -- (220.1000,272.8000) -- (220.3000,271.0000) -- (220.5000,269.5000) -- (220.7000,270.2000) -- (220.8000,270.0000) -- (221.0000,269.6000) -- (221.2000,269.4000) -- (221.4000,269.3000) -- (221.6000,269.3000) -- (221.8000,269.3000) -- (222.0000,269.3000) -- (222.2000,269.3000) -- (222.4000,269.3000) -- (222.6000,269.2000) -- (222.8000,269.9000) -- (223.0000,275.8000) -- (223.2000,276.3000) -- (223.4000,276.1000) -- (223.6000,276.3000) -- (223.8000,275.9000) -- (224.0000,274.7000) -- (224.2000,271.8000) -- (224.4000,279.4000) -- (224.6000,289.1000) -- (224.8000,285.2000) -- (225.0000,282.2000) -- (225.2000,279.3000) -- (225.4000,276.4000) -- (225.5000,273.8000) -- (225.7000,271.4000) -- (225.9000,269.2000) -- (226.1000,267.2000) -- (226.3000,265.5000) -- (226.5000,264.0000) -- (226.7000,262.8000) -- (226.9000,261.8000) -- (227.1000,260.9000) -- (227.3000,260.3000) -- (227.5000,259.9000) -- (227.7000,259.7000) -- (227.9000,259.7000) -- (228.1000,259.9000) -- (228.3000,259.9000) -- (228.5000,260.0000) -- (228.7000,260.0000) -- (228.9000,260.0000) -- (229.1000,260.0000) -- (229.3000,260.0000) -- (229.5000,260.0000) -- (229.7000,260.0000) -- (229.9000,260.0000) -- (230.1000,260.0000) -- (230.3000,260.0000) -- (230.4000,260.0000) -- (230.6000,260.0000) -- (230.8000,260.2000) -- (231.0000,260.6000) -- (231.2000,261.2000) -- (231.4000,262.0000) -- (231.6000,263.0000) -- (231.8000,264.3000) -- (232.0000,265.7000) -- (232.2000,267.3000) -- (232.4000,269.2000) -- (232.6000,271.2000) -- (232.8000,273.5000) -- (233.0000,276.0000) -- (233.2000,277.3000) -- (233.4000,274.3000) -- (233.6000,277.0000) -- (233.8000,278.6000) -- (234.0000,275.5000) -- (234.2000,277.9000) -- (234.4000,279.7000) -- (234.6000,276.2000) -- (234.8000,277.0000) -- (234.9000,277.2000) -- (235.1000,276.9000) -- (235.3000,276.6000) -- (235.5000,276.4000) -- (235.7000,276.3000) -- (235.9000,276.3000) -- (236.1000,276.3000) -- (236.3000,276.4000) -- (236.5000,276.3000) -- (236.7000,276.9000) -- (236.9000,282.8000) -- (237.1000,283.5000) -- (237.3000,283.4000) -- (237.5000,283.1000) -- (237.7000,283.0000) -- (237.9000,284.0000) -- (238.1000,295.1000) -- (238.3000,295.2000) -- (238.5000,292.8000) -- (238.7000,290.0000) -- (238.9000,287.1000) -- (239.1000,284.0000) -- (239.3000,281.0000) -- (239.5000,278.1000) -- (239.7000,275.4000) -- (239.8000,272.8000) -- (240.0000,270.5000) -- (240.2000,268.4000) -- (240.4000,266.5000) -- (240.6000,264.9000) -- (240.8000,263.5000) -- (241.0000,262.3000) -- (241.2000,261.4000) -- (241.4000,260.7000) -- (241.6000,260.1000) -- (241.8000,259.8000) -- (242.0000,259.7000) -- (242.2000,259.8000) -- (242.4000,259.9000) -- (242.6000,260.0000) -- (242.8000,260.0000) -- (243.0000,260.0000) -- (243.2000,260.0000) -- (243.4000,260.0000) -- (243.6000,260.0000) -- (243.8000,260.0000) -- (244.0000,260.0000) -- (244.2000,260.0000) -- (244.4000,260.0000) -- (244.5000,260.0000) -- (244.7000,260.0000) -- (244.9000,260.0000) -- (245.1000,260.3000) -- (245.3000,260.8000) -- (245.5000,261.5000) -- (245.7000,262.4000) -- (245.9000,263.5000) -- (246.1000,264.8000) -- (246.3000,266.3000) -- (246.5000,268.0000) -- (246.7000,270.0000) -- (246.9000,272.1000) -- (247.1000,274.4000) -- (247.3000,276.9000) -- (247.5000,279.5000) -- (247.7000,282.2000) -- (247.9000,285.3000) -- (248.1000,283.6000) -- (248.3000,283.1000) -- (248.5000,286.3000) -- (248.7000,284.7000) -- (248.9000,283.3000) -- (249.1000,284.2000) -- (249.3000,284.1000) -- (249.4000,283.8000) -- (249.6000,283.5000) -- (249.8000,283.4000) -- (250.0000,283.4000) -- (250.2000,283.4000) -- (250.4000,283.4000) -- (250.6000,283.4000) -- (250.8000,283.1000) -- (251.0000,286.8000) -- (251.2000,290.7000) -- (251.4000,290.1000) -- (251.6000,293.6000) -- (251.8000,297.4000);



  \end{scope}
  \begin{scope}[scale=1.006,draw=black,line join=bevel,line cap=rect,line width=0.800pt]
  \end{scope}
  \begin{scope}[scale=1.006,draw=black,line join=bevel,line cap=rect,line width=0.800pt]
  \end{scope}
  \begin{scope}[scale=1.006,draw=black,line join=bevel,line cap=rect,line width=0.800pt]
  \end{scope}
  \begin{scope}[scale=1.006,draw=black,line join=bevel,line cap=rect,line width=0.800pt]
  \end{scope}
  \begin{scope}[cm={{1.00588,0.0,0.0,1.00588,(-334.50007,-109.50002)}},draw=c00ff00,dash pattern=on 0.48pt off 0.48pt,line join=round,line cap=round,miter limit=4.00,line width=0.480pt]
    \path[draw,dash pattern=on 0.48pt off 0.48pt,miter limit=4.00,line width=0.480pt] (127.3000,221.1000) -- (127.6000,221.2000) -- (127.9000,221.4000) -- (128.2000,221.5000) -- (128.5000,221.7000) -- (128.8000,221.8000) -- (129.1000,222.0000) -- (129.4000,222.1000) -- (129.7000,222.3000) -- (130.0000,222.4000) -- (130.3000,222.6000) -- (130.6000,222.7000) -- (130.9000,222.9000) -- (131.2000,223.0000) -- (131.5000,223.2000) -- (131.8000,223.3000) -- (132.1000,223.5000) -- (132.3000,223.6000) -- (132.6000,223.8000) -- (132.9000,223.9000) -- (133.2000,224.1000) -- (133.5000,224.2000) -- (133.8000,224.4000) -- (134.1000,224.5000) -- (134.4000,224.7000) -- (134.7000,224.8000) -- (135.0000,225.0000) -- (135.3000,225.1000) -- (135.6000,225.2000) -- (135.9000,225.4000) -- (136.2000,225.5000) -- (136.5000,225.7000) -- (136.8000,225.8000) -- (137.1000,226.0000) -- (137.4000,226.1000) -- (137.7000,226.3000) -- (138.0000,226.4000) -- (138.3000,226.6000) -- (138.6000,226.7000) -- (138.9000,226.9000) -- (139.2000,227.0000) -- (139.4000,227.2000) -- (139.7000,227.3000) -- (140.0000,227.5000) -- (140.3000,227.6000) -- (140.6000,227.8000) -- (140.9000,227.9000) -- (141.2000,228.1000) -- (141.5000,228.2000) -- (141.8000,228.4000) -- (142.1000,228.5000) -- (142.4000,228.7000) -- (142.7000,228.8000) -- (143.0000,229.0000) -- (143.3000,229.1000) -- (143.6000,229.3000) -- (143.9000,229.4000) -- (144.2000,229.6000) -- (144.5000,229.7000) -- (144.8000,229.8000) -- (145.1000,230.0000) -- (145.4000,230.1000) -- (145.7000,230.3000) -- (146.0000,230.4000) -- (146.3000,230.6000) -- (146.6000,230.7000) -- (146.8000,230.9000) -- (147.1000,231.0000) -- (147.4000,231.2000) -- (147.7000,231.3000) -- (148.0000,231.5000) -- (148.3000,231.6000) -- (148.6000,231.8000) -- (148.9000,231.9000) -- (149.2000,232.1000) -- (149.5000,232.2000) -- (149.8000,232.4000) -- (150.1000,232.5000) -- (150.4000,232.7000) -- (150.7000,232.8000) -- (151.0000,233.0000) -- (151.3000,233.1000) -- (151.6000,233.3000) -- (151.9000,233.4000) -- (152.2000,233.6000) -- (152.5000,233.7000) -- (152.8000,233.9000) -- (153.1000,234.0000) -- (153.4000,234.1000) -- (153.7000,234.3000) -- (153.9000,234.4000) -- (154.2000,234.6000) -- (154.5000,234.7000) -- (154.8000,234.9000) -- (155.1000,235.0000) -- (155.4000,235.2000) -- (155.7000,235.3000) -- (156.0000,235.5000) -- (156.3000,235.6000) -- (156.6000,235.8000) -- (156.9000,235.9000) -- (157.2000,236.1000) -- (157.5000,236.2000) -- (157.8000,236.4000) -- (158.1000,236.5000) -- (158.4000,236.7000) -- (158.7000,236.8000) -- (159.0000,237.0000) -- (159.3000,237.1000) -- (159.6000,237.3000) -- (159.9000,237.4000) -- (160.2000,237.6000) -- (160.5000,237.7000) -- (160.8000,237.9000) -- (161.0000,238.0000) -- (161.3000,238.2000) -- (161.6000,238.3000) -- (161.9000,238.5000) -- (162.2000,238.6000) -- (162.5000,238.6000) -- (162.8000,238.7000) -- (163.1000,238.9000) -- (163.4000,239.0000) -- (163.7000,239.2000) -- (164.0000,239.3000) -- (164.3000,239.5000) -- (164.6000,239.6000) -- (164.9000,239.8000) -- (165.2000,239.9000) -- (165.5000,240.1000) -- (165.8000,240.2000) -- (166.1000,240.4000) -- (166.4000,240.5000) -- (166.7000,240.7000) -- (167.0000,240.8000) -- (167.3000,241.0000) -- (167.6000,241.1000) -- (167.9000,241.3000) -- (168.2000,241.4000) -- (168.4000,241.6000) -- (168.7000,241.7000) -- (169.0000,241.9000) -- (169.3000,242.0000) -- (169.6000,242.2000) -- (169.9000,242.3000) -- (170.2000,242.5000) -- (170.5000,242.6000) -- (170.8000,242.8000) -- (171.1000,242.9000) -- (171.4000,243.1000) -- (171.7000,243.2000) -- (172.0000,243.3000) -- (172.3000,243.5000) -- (172.6000,243.6000) -- (172.9000,243.8000) -- (173.2000,243.9000) -- (173.5000,244.1000) -- (173.8000,244.2000) -- (174.1000,244.4000) -- (174.4000,244.5000) -- (174.7000,244.7000) -- (175.0000,244.8000) -- (175.3000,245.0000) -- (175.5000,245.1000) -- (175.8000,245.3000) -- (176.1000,245.4000) -- (176.4000,245.6000) -- (176.7000,245.7000) -- (177.0000,245.9000) -- (177.3000,246.0000) -- (177.6000,246.2000) -- (177.9000,246.3000) -- (178.2000,246.5000) -- (178.5000,246.6000) -- (178.8000,246.8000) -- (179.1000,246.9000) -- (179.4000,247.1000) -- (179.7000,247.2000) -- (180.0000,247.4000) -- (180.3000,247.5000) -- (180.6000,247.6000) -- (180.9000,247.8000) -- (181.2000,247.9000) -- (181.5000,248.1000) -- (181.8000,248.2000) -- (182.1000,248.4000) -- (182.4000,248.5000) -- (182.7000,248.7000) -- (182.9000,248.8000) -- (183.2000,249.0000) -- (183.5000,249.1000) -- (183.8000,249.3000) -- (184.1000,249.4000) -- (184.4000,249.6000) -- (184.7000,249.7000) -- (185.0000,249.9000) -- (185.3000,250.0000) -- (185.6000,250.2000) -- (185.9000,250.3000) -- (186.2000,250.5000) -- (186.5000,250.6000) -- (186.8000,250.8000) -- (187.1000,250.9000) -- (187.4000,251.1000) -- (187.7000,251.2000) -- (188.0000,251.4000) -- (188.3000,251.5000) -- (188.6000,251.7000) -- (188.9000,251.8000) -- (189.2000,252.0000) -- (189.5000,252.1000) -- (189.8000,252.2000) -- (190.0000,252.4000) -- (190.3000,252.5000) -- (190.6000,252.7000) -- (190.9000,252.8000) -- (191.2000,253.0000) -- (191.5000,253.1000) -- (191.8000,253.3000) -- (192.1000,253.4000) -- (192.4000,253.6000) -- (192.7000,253.7000) -- (193.0000,253.9000) -- (193.3000,254.0000) -- (193.6000,254.2000) -- (193.9000,254.3000) -- (194.2000,254.5000) -- (194.5000,254.6000) -- (194.8000,254.8000) -- (195.1000,254.9000) -- (195.4000,255.1000) -- (195.7000,255.2000) -- (196.0000,255.4000) -- (196.3000,255.5000) -- (196.6000,255.7000) -- (196.9000,255.8000) -- (197.2000,256.0000) -- (197.4000,256.1000) -- (197.7000,256.3000) -- (198.0000,256.4000) -- (198.3000,256.5000) -- (198.6000,256.7000) -- (198.9000,256.8000) -- (199.2000,257.0000) -- (199.5000,257.1000) -- (199.8000,257.3000) -- (200.1000,257.4000) -- (200.4000,257.6000) -- (200.7000,257.7000) -- (201.0000,257.9000) -- (201.3000,258.0000) -- (201.6000,258.2000) -- (201.9000,258.3000) -- (202.2000,258.5000) -- (202.5000,258.6000) -- (202.8000,258.8000) -- (203.1000,258.9000) -- (203.4000,259.1000) -- (203.7000,259.2000) -- (204.0000,259.4000) -- (204.3000,259.5000) -- (204.5000,259.7000) -- (204.8000,259.8000) -- (205.1000,260.0000) -- (205.4000,260.1000) -- (205.7000,260.3000) -- (206.0000,260.4000) -- (206.3000,260.6000) -- (206.6000,260.7000) -- (206.9000,260.9000) -- (207.2000,261.0000) -- (207.5000,261.1000) -- (207.8000,261.3000) -- (208.1000,261.4000) -- (208.4000,261.6000) -- (208.7000,261.7000) -- (209.0000,261.9000) -- (209.3000,262.0000) -- (209.6000,262.2000) -- (209.9000,262.3000) -- (210.2000,262.5000) -- (210.5000,262.6000) -- (210.8000,262.8000) -- (211.1000,262.9000) -- (211.4000,263.1000) -- (211.7000,263.2000) -- (211.9000,263.4000) -- (212.2000,263.5000) -- (212.5000,263.7000) -- (212.8000,263.8000) -- (213.1000,264.0000) -- (213.4000,264.1000) -- (213.7000,264.3000) -- (214.0000,264.4000) -- (214.3000,264.6000) -- (214.6000,264.7000) -- (214.9000,264.9000) -- (215.2000,265.0000) -- (215.5000,265.2000) -- (215.8000,265.3000) -- (216.1000,265.5000) -- (216.4000,265.6000) -- (216.7000,265.7000) -- (217.0000,265.9000) -- (217.3000,266.0000) -- (217.6000,266.2000) -- (217.9000,266.3000) -- (218.2000,266.5000) -- (218.5000,266.6000) -- (218.8000,266.8000) -- (219.0000,266.9000) -- (219.3000,267.1000) -- (219.6000,267.2000) -- (219.9000,267.4000) -- (220.2000,267.5000) -- (220.5000,267.7000) -- (220.8000,267.8000) -- (221.1000,268.0000) -- (221.4000,268.1000) -- (221.7000,268.3000) -- (222.0000,268.4000) -- (222.3000,268.6000) -- (222.6000,268.7000) -- (222.9000,268.9000) -- (223.2000,269.0000) -- (223.5000,269.2000) -- (223.8000,269.3000) -- (224.1000,269.5000) -- (224.4000,269.6000) -- (224.7000,269.8000) -- (225.0000,269.9000) -- (225.3000,270.0000) -- (225.6000,270.2000) -- (225.9000,270.3000) -- (226.1000,270.5000) -- (226.4000,270.6000) -- (226.7000,270.8000) -- (227.0000,270.9000) -- (227.3000,271.1000) -- (227.6000,271.2000) -- (227.9000,271.4000) -- (228.2000,271.5000) -- (228.5000,271.7000) -- (228.8000,271.8000) -- (229.1000,272.0000) -- (229.4000,272.1000) -- (229.7000,272.3000) -- (230.0000,272.4000) -- (230.3000,272.6000) -- (230.6000,272.7000) -- (230.9000,272.9000) -- (231.2000,273.0000) -- (231.5000,273.2000) -- (231.8000,273.3000) -- (232.1000,273.5000) -- (232.4000,273.6000) -- (232.7000,273.8000) -- (233.0000,273.9000) -- (233.3000,274.1000) -- (233.5000,274.2000) -- (233.8000,274.4000) -- (234.1000,274.5000) -- (234.4000,274.6000) -- (234.7000,274.8000) -- (235.0000,274.9000) -- (235.3000,275.1000) -- (235.6000,275.2000) -- (235.9000,275.4000) -- (236.2000,275.5000) -- (236.5000,275.7000) -- (236.8000,275.8000) -- (237.1000,276.0000) -- (237.4000,276.1000) -- (237.7000,276.3000) -- (238.0000,276.4000) -- (238.3000,276.6000) -- (238.6000,276.7000) -- (238.9000,276.9000) -- (239.2000,277.0000) -- (239.5000,277.2000) -- (239.8000,277.3000) -- (240.1000,277.5000) -- (240.4000,277.6000) -- (240.6000,277.8000) -- (240.9000,277.9000) -- (241.2000,278.1000) -- (241.5000,278.2000) -- (241.8000,278.4000) -- (242.1000,278.5000) -- (242.4000,278.7000) -- (242.7000,278.8000) -- (243.0000,279.0000) -- (243.3000,279.1000) -- (243.6000,279.2000) -- (243.9000,279.4000) -- (244.2000,279.5000) -- (244.5000,279.7000) -- (244.8000,279.8000) -- (245.1000,280.0000) -- (245.4000,280.1000) -- (245.7000,280.3000) -- (246.0000,280.4000) -- (246.3000,280.6000) -- (246.6000,280.7000) -- (246.9000,280.9000) -- (247.2000,281.0000) -- (247.5000,281.2000) -- (247.8000,281.3000) -- (248.0000,281.5000) -- (248.3000,281.6000) -- (248.6000,281.8000) -- (248.9000,281.9000) -- (249.2000,282.1000) -- (249.5000,282.2000) -- (249.8000,282.4000) -- (250.1000,282.5000) -- (250.4000,282.7000) -- (250.7000,282.8000) -- (251.0000,283.0000) -- (251.3000,283.1000) -- (251.6000,283.3000) -- (251.9000,283.4000);



  \end{scope}
  \begin{scope}[scale=1.006,draw=black,line join=bevel,line cap=rect,line width=0.800pt]
  \end{scope}
  \begin{scope}[draw=black,line join=bevel,line cap=rect,line width=0.800pt]
  \end{scope}
  \begin{scope}[cm={{1.00588,0.0,0.0,1.00588,(-114.00002,89.46968)}},draw=ca0a0a4,dash pattern=on 0.40pt off 0.80pt,line join=round,line cap=round,line width=0.400pt]
    \path[draw] (56.5000,88.5000) -- (142.5000,88.5000);



  \end{scope}
  \begin{scope}[cm={{1.00588,0.0,0.0,1.00588,(-114.00002,89.46968)}},draw=black,line join=round,line cap=round,line width=0.480pt]
    \path[draw] (56.5000,88.5000) -- (59.5000,88.5000);



    \path[draw] (142.5000,88.5000) -- (139.5000,88.5000);



  \end{scope}
  \begin{scope}[cm={{1.00588,0.0,0.0,1.00588,(-114.00002,89.46968)}},draw=ca0a0a4,dash pattern=on 0.40pt off 0.80pt,line join=round,line cap=round,line width=0.400pt]
    \path[draw] (56.5000,63.5000) -- (142.5000,63.5000);



  \end{scope}
  \begin{scope}[cm={{1.00588,0.0,0.0,1.00588,(-114.00002,89.46968)}},draw=black,line join=round,line cap=round,line width=0.480pt]
    \path[draw] (56.5000,63.5000) -- (59.5000,63.5000);



    \path[draw] (142.5000,63.5000) -- (139.5000,63.5000);



  \end{scope}
  \begin{scope}[cm={{1.00588,0.0,0.0,1.00588,(-114.00002,89.46968)}},draw=ca0a0a4,dash pattern=on 0.40pt off 0.80pt,line join=round,line cap=round,line width=0.400pt]
    \path[draw] (56.5000,38.5000) -- (142.5000,38.5000);



  \end{scope}
  \begin{scope}[cm={{1.00588,0.0,0.0,1.00588,(-114.00002,89.46968)}},draw=black,line join=round,line cap=round,line width=0.480pt]
    \path[draw] (56.5000,38.5000) -- (59.5000,38.5000);



    \path[draw] (142.5000,38.5000) -- (139.5000,38.5000);



  \end{scope}
  \begin{scope}[cm={{1.00588,0.0,0.0,1.00588,(-114.00002,89.46968)}},draw=ca0a0a4,dash pattern=on 0.40pt off 0.80pt,line join=round,line cap=round,line width=0.400pt]
    \path[draw] (56.5000,95.5000) -- (56.5000,13.5000);



  \end{scope}
  \begin{scope}[cm={{1.00588,0.0,0.0,1.00588,(-114.00002,89.46968)}},draw=black,line join=round,line cap=round,line width=0.480pt]
    \path[draw] (56.5000,95.5000) -- (56.5000,92.5000);



    \path[draw] (56.5000,13.5000) -- (56.5000,16.5000);



  \end{scope}
  \begin{scope}[cm={{1.00588,0.0,0.0,1.00588,(-114.00002,89.46968)}},draw=black,line join=round,line cap=round,line width=0.480pt]
    \path[draw] (82.5000,95.5000) -- (82.5000,92.5000);



    \path[draw] (82.5000,13.5000) -- (82.5000,16.5000);



  \end{scope}
  \begin{scope}[cm={{1.00588,0.0,0.0,1.00588,(-114.00002,89.46968)}},draw=black,line join=round,line cap=round,line width=0.480pt]
    \path[draw] (108.5000,95.5000) -- (108.5000,92.5000);



    \path[draw] (108.5000,13.5000) -- (108.5000,16.5000);



  \end{scope}
  \begin{scope}[cm={{1.00588,0.0,0.0,1.00588,(-114.00002,89.46968)}},draw=ca0a0a4,dash pattern=on 0.40pt off 0.80pt,line join=round,line cap=round,line width=0.400pt]
    \path[draw] (134.5000,95.5000) -- (134.5000,13.5000);



  \end{scope}
  \begin{scope}[cm={{1.00588,0.0,0.0,1.00588,(-114.00002,89.46968)}},draw=black,line join=round,line cap=round,line width=0.480pt]
    \path[draw] (134.5000,95.5000) -- (134.5000,92.5000);



    \path[draw] (134.5000,13.5000) -- (134.5000,16.5000);



  \end{scope}
  \begin{scope}[cm={{1.00588,0.0,0.0,1.00588,(-114.00002,89.46968)}},fill=cffffff]
    \path[fill,rounded corners=0.0000cm] (124.0000,17.0000) rectangle (135.0000,33.0000);



  \end{scope}
  \begin{scope}[cm={{1.00588,0.0,0.0,1.00588,(-114.00002,89.46968)}},draw=black,line join=round,line cap=round,line width=0.800pt]
    \path[draw] (123.5000,33.5000) -- (123.5000,17.5000) -- (134.5000,17.5000) -- (134.5000,33.5000) -- (123.5000,33.5000);



  \end{scope}
  \begin{scope}[cm={{1.00588,0.0,0.0,1.00588,(14.61513,118.22675)}},draw=black,line join=bevel,line cap=rect,line width=0.800pt]
    \path[fill=black] (0.0000,0.0000) node[above right] (text258) {\label{fig:ener:static-I}I};



  \end{scope}
  \begin{scope}[cm={{1.00588,0.0,0.0,1.00588,(-272.27196,19.63437)}},draw=black,line join=bevel,line cap=rect,line width=0.800pt]
    \path[fill=black] (0.0000,-1.4912) node[above right] (text290) {\scriptsize $\Upsilon(i)$};



  \end{scope}
  \begin{scope}[cm={{1.00588,0.0,0.0,1.00588,(-339.00002,-9.53034)}},draw=black,line join=round,line cap=round,line width=0.480pt]
    \path[draw,even odd rule] (86.5000,24.5000) -- (112.5000,24.5000);



  \end{scope}
  \begin{scope}[cm={{1.00588,0.0,0.0,1.00588,(-114.00002,89.46968)}},draw=black,line join=round,line cap=round,line width=0.480pt]
    \path[draw] (56.1000,52.3000) -- (56.1000,52.3000) -- (56.5000,69.8000) -- (56.8000,61.5000) -- (57.1000,50.5000) -- (57.4000,49.2000) -- (57.7000,52.5000) -- (58.0000,57.0000) -- (58.4000,59.6000) -- (58.7000,60.0000) -- (59.0000,59.6000) -- (59.3000,59.2000) -- (59.6000,59.1000) -- (59.9000,59.0000) -- (60.2000,59.0000) -- (60.6000,59.1000) -- (60.9000,59.1000) -- (61.2000,59.1000) -- (61.5000,59.1000) -- (61.8000,59.0000) -- (62.1000,59.5000) -- (62.4000,62.4000) -- (62.8000,66.1000) -- (63.1000,69.9000) -- (63.4000,73.4000) -- (63.7000,76.7000) -- (64.0000,79.8000) -- (64.3000,82.6000) -- (64.6000,84.9000) -- (65.0000,86.4000) -- (65.3000,87.1000) -- (65.6000,86.7000) -- (65.9000,85.4000) -- (66.2000,83.2000) -- (66.5000,80.3000) -- (66.8000,77.1000) -- (67.2000,73.7000) -- (67.5000,70.1000) -- (67.8000,65.0000) -- (68.1000,60.3000) -- (68.4000,58.1000) -- (68.7000,58.0000) -- (69.0000,58.5000) -- (69.4000,58.9000) -- (69.7000,59.1000) -- (70.0000,59.1000) -- (70.3000,59.1000) -- (70.6000,59.1000) -- (70.9000,59.1000) -- (71.2000,59.1000) -- (71.5000,59.0000) -- (71.9000,61.2000) -- (72.2000,62.2000) -- (72.5000,59.0000) -- (72.8000,55.3000) -- (73.1000,52.3000) -- (73.4000,50.0000) -- (73.7000,48.2000) -- (74.1000,46.8000) -- (74.4000,45.7000) -- (74.7000,44.9000) -- (75.0000,44.4000) -- (75.3000,44.2000) -- (75.6000,44.3000) -- (75.9000,44.7000) -- (76.3000,45.4000) -- (76.6000,46.4000) -- (76.9000,47.7000) -- (77.2000,49.3000) -- (77.5000,51.3000) -- (77.8000,53.6000) -- (78.1000,56.7000) -- (78.5000,59.2000) -- (78.8000,59.9000) -- (79.1000,59.6000) -- (79.4000,59.3000) -- (79.7000,59.1000) -- (80.0000,59.0000) -- (80.3000,59.0000) -- (80.7000,59.1000) -- (81.0000,59.1000) -- (81.3000,59.1000) -- (81.6000,59.1000) -- (81.9000,59.0000) -- (82.2000,59.1000) -- (82.5000,61.3000) -- (82.9000,65.2000) -- (83.2000,69.1000) -- (83.5000,72.7000) -- (83.8000,76.1000) -- (84.1000,79.3000) -- (84.4000,82.1000) -- (84.7000,84.6000) -- (85.1000,86.3000) -- (85.4000,87.1000) -- (85.7000,86.8000) -- (86.0000,85.5000) -- (86.3000,83.3000) -- (86.6000,80.4000) -- (86.9000,77.1000) -- (87.3000,73.6000) -- (87.6000,69.9000) -- (87.9000,65.0000) -- (88.2000,60.5000) -- (88.5000,58.3000) -- (88.8000,58.1000) -- (89.1000,58.5000) -- (89.5000,58.9000) -- (89.8000,59.1000) -- (90.1000,59.1000) -- (90.4000,59.1000) -- (90.7000,59.1000) -- (91.0000,59.1000) -- (91.3000,59.1000) -- (91.7000,59.0000) -- (92.0000,61.4000) -- (92.3000,62.0000) -- (92.6000,58.6000) -- (92.9000,55.0000) -- (93.2000,52.0000) -- (93.5000,49.7000) -- (93.9000,47.9000) -- (94.2000,46.6000) -- (94.5000,45.5000) -- (94.8000,44.8000) -- (95.1000,44.3000) -- (95.4000,44.2000) -- (95.7000,44.4000) -- (96.1000,44.8000) -- (96.4000,45.6000) -- (96.7000,46.7000) -- (97.0000,48.2000) -- (97.3000,50.0000) -- (97.6000,52.1000) -- (97.9000,54.7000) -- (98.3000,57.7000) -- (98.6000,59.4000) -- (98.9000,59.7000) -- (99.2000,59.5000) -- (99.5000,59.2000) -- (99.8000,59.1000) -- (100.1000,59.0000) -- (100.5000,59.1000) -- (100.8000,59.1000) -- (101.1000,59.1000) -- (101.4000,59.1000) -- (101.7000,59.1000) -- (102.0000,59.0000) -- (102.3000,59.5000) -- (102.7000,62.6000) -- (103.0000,66.8000) -- (103.3000,70.7000) -- (103.6000,74.2000) -- (103.9000,77.5000) -- (104.2000,80.5000) -- (104.5000,83.2000) -- (104.9000,85.3000) -- (105.2000,86.7000) -- (105.5000,87.1000) -- (105.8000,86.3000) -- (106.1000,84.5000) -- (106.4000,81.8000) -- (106.7000,78.6000) -- (107.0000,75.1000) -- (107.4000,71.6000) -- (107.7000,67.2000) -- (108.0000,62.3000) -- (108.3000,59.0000) -- (108.6000,58.0000) -- (108.9000,58.3000) -- (109.3000,58.7000) -- (109.6000,59.0000) -- (109.9000,59.1000) -- (110.2000,59.1000) -- (110.5000,59.1000) -- (110.8000,59.1000) -- (111.1000,59.1000) -- (111.4000,59.0000) -- (111.8000,60.0000) -- (112.1000,62.4000) -- (112.4000,60.2000) -- (112.7000,56.4000) -- (113.0000,53.1000) -- (113.3000,50.5000) -- (113.6000,48.6000) -- (114.0000,47.1000) -- (114.3000,45.9000) -- (114.6000,45.0000) -- (114.9000,44.5000) -- (115.2000,44.2000) -- (115.5000,44.3000) -- (115.8000,44.6000) -- (116.2000,45.3000) -- (116.5000,46.3000) -- (116.8000,47.6000) -- (117.1000,49.4000) -- (117.4000,51.4000) -- (117.7000,53.9000) -- (118.0000,56.8000) -- (118.4000,59.0000) -- (118.7000,59.7000) -- (119.0000,59.6000) -- (119.3000,59.3000) -- (119.6000,59.1000) -- (119.9000,59.1000) -- (120.2000,59.1000) -- (120.6000,59.1000) -- (120.9000,59.1000) -- (121.2000,59.1000) -- (121.5000,59.1000) -- (121.8000,59.0000) -- (122.1000,59.2000) -- (122.4000,61.3000) -- (122.8000,65.4000) -- (123.1000,69.6000) -- (123.4000,73.3000) -- (123.7000,76.6000) -- (124.0000,79.6000) -- (124.3000,82.4000) -- (124.6000,84.7000) -- (125.0000,86.4000) -- (125.3000,87.1000) -- (125.6000,86.7000) -- (125.9000,85.1000) -- (126.2000,82.5000) -- (126.5000,79.4000) -- (126.8000,76.0000) -- (127.2000,72.5000) -- (127.5000,68.4000) -- (127.8000,63.4000) -- (128.1000,59.6000) -- (128.4000,58.1000) -- (128.7000,58.2000) -- (129.0000,58.6000) -- (129.4000,59.0000) -- (129.7000,59.1000) -- (130.0000,59.1000) -- (130.3000,59.1000) -- (130.6000,59.1000) -- (130.9000,59.1000) -- (131.2000,59.0000) -- (131.6000,59.4000) -- (131.9000,62.2000) -- (132.2000,61.0000) -- (132.5000,57.3000) -- (132.8000,53.8000) -- (133.1000,51.1000) -- (133.4000,49.0000) -- (133.8000,47.4000) -- (134.1000,46.1000) -- (134.4000,45.2000) -- (134.7000,44.6000) -- (135.0000,44.2000) -- (135.3000,44.2000) -- (135.6000,44.5000) -- (136.0000,45.1000) -- (136.3000,46.0000) -- (136.6000,47.3000) -- (136.9000,49.0000) -- (137.2000,51.0000) -- (137.5000,53.3000) -- (137.8000,56.2000) -- (138.2000,58.6000) -- (138.5000,59.6000) -- (138.8000,59.6000) -- (139.1000,59.3000) -- (139.4000,59.1000) -- (139.7000,59.1000) -- (140.0000,59.1000) -- (140.4000,59.1000) -- (140.7000,59.1000) -- (141.0000,59.1000) -- (141.3000,59.1000) -- (141.6000,59.1000) -- (141.9000,59.0000) -- (142.2000,60.6000) -- (142.6000,64.5000) -- (142.7000,66.5000);



  \end{scope}
  \begin{scope}[cm={{1.00588,0.0,0.0,1.00588,(-270.14713,27.68147)}},draw=black,line join=bevel,line cap=rect,line width=0.800pt]
    \path[fill=black] (0.0000,0.0000) node[above right] (text326) {\scriptsize $y(i)$};



  \end{scope}
  \begin{scope}[cm={{1.00588,0.0,0.0,1.00588,(-114.00002,89.46968)}},draw=cff0000,line join=round,line cap=round,line width=0.480pt]
    \path[draw,even odd rule] (-137.1848,-65.9213) -- (-111.1848,-65.9213);



  \end{scope}
  \begin{scope}[cm={{1.00588,0.0,0.0,1.00588,(-114.00002,89.46968)}},draw=cff0000,line join=round,line cap=round,line width=0.480pt]
    \path[draw] (56.0000,44.3000) -- (56.0000,44.3000) -- (56.1000,44.4000) -- (56.2000,44.6000) -- (56.3000,44.8000) -- (56.3000,45.1000) -- (56.4000,45.4000) -- (56.5000,45.7000) -- (56.6000,46.1000) -- (56.7000,46.5000) -- (56.8000,47.0000) -- (56.9000,47.4000) -- (57.0000,47.9000) -- (57.0000,48.4000) -- (57.1000,48.9000) -- (57.2000,49.5000) -- (57.3000,50.0000) -- (57.4000,50.6000) -- (57.5000,51.2000) -- (57.6000,51.8000) -- (57.6000,52.3000) -- (57.7000,52.9000) -- (57.8000,53.5000) -- (57.9000,54.0000) -- (58.0000,54.6000) -- (58.1000,55.1000) -- (58.2000,55.6000) -- (58.3000,56.1000) -- (58.3000,56.6000) -- (58.4000,57.1000) -- (58.5000,57.5000) -- (58.6000,57.9000) -- (58.7000,58.2000) -- (58.8000,58.6000) -- (58.9000,58.9000) -- (59.0000,59.1000) -- (59.0000,59.4000) -- (59.1000,59.6000) -- (59.2000,59.8000) -- (59.3000,59.9000) -- (59.4000,60.0000) -- (59.5000,60.1000) -- (59.6000,60.1000) -- (59.6000,60.1000) -- (59.7000,60.1000) -- (59.8000,60.1000) -- (59.9000,60.0000) -- (60.0000,60.0000) -- (60.1000,59.9000) -- (60.2000,59.7000) -- (60.3000,59.6000) -- (60.3000,59.5000) -- (60.4000,59.3000) -- (60.5000,59.2000) -- (60.6000,59.0000) -- (60.7000,58.9000) -- (60.8000,58.8000) -- (60.9000,58.6000) -- (60.9000,58.5000) -- (61.0000,58.4000) -- (61.1000,58.3000) -- (61.2000,58.2000) -- (61.3000,58.2000) -- (61.4000,58.2000) -- (61.5000,58.2000) -- (61.6000,58.3000) -- (61.6000,58.3000) -- (61.7000,58.5000) -- (61.8000,58.6000) -- (61.9000,58.8000) -- (62.0000,59.1000) -- (62.1000,59.3000) -- (62.2000,59.7000) -- (62.2000,60.0000) -- (62.3000,60.4000) -- (62.4000,60.9000) -- (62.5000,61.4000) -- (62.6000,61.9000) -- (62.7000,62.5000) -- (62.8000,63.1000) -- (62.9000,63.8000) -- (62.9000,64.4000) -- (63.0000,65.2000) -- (63.1000,65.9000) -- (63.2000,66.7000) -- (63.3000,67.5000) -- (63.4000,68.3000) -- (63.5000,69.2000) -- (63.6000,70.0000) -- (63.6000,70.9000) -- (63.7000,71.8000) -- (63.8000,72.7000) -- (63.9000,73.6000) -- (64.0000,74.4000) -- (64.1000,75.3000) -- (64.2000,76.2000) -- (64.2000,77.0000) -- (64.3000,77.8000) -- (64.4000,78.6000) -- (64.5000,79.4000) -- (64.6000,80.1000) -- (64.7000,80.8000) -- (64.8000,81.5000) -- (64.9000,82.1000) -- (64.9000,82.7000) -- (65.0000,83.2000) -- (65.1000,83.7000) -- (65.2000,84.1000) -- (65.3000,84.4000) -- (65.4000,84.7000) -- (65.5000,85.0000) -- (65.5000,85.2000) -- (65.6000,85.3000) -- (65.7000,85.3000) -- (65.8000,85.3000) -- (65.9000,85.3000) -- (66.0000,85.1000) -- (66.1000,84.9000) -- (66.2000,84.7000) -- (66.2000,84.3000) -- (66.3000,84.0000) -- (66.4000,83.5000) -- (66.5000,83.1000) -- (66.6000,82.5000) -- (66.7000,81.9000) -- (66.8000,81.3000) -- (66.8000,80.6000) -- (66.9000,79.9000) -- (67.0000,79.2000) -- (67.1000,78.4000) -- (67.2000,77.6000) -- (67.3000,76.7000) -- (67.4000,75.9000) -- (67.5000,75.0000) -- (67.5000,74.1000) -- (67.6000,73.2000) -- (67.7000,72.3000) -- (67.8000,71.4000) -- (67.9000,70.5000) -- (68.0000,69.6000) -- (68.1000,68.7000) -- (68.2000,67.9000) -- (68.2000,67.0000) -- (68.3000,66.2000) -- (68.4000,65.4000) -- (68.5000,64.7000) -- (68.6000,63.9000) -- (68.7000,63.2000) -- (68.8000,62.5000) -- (68.8000,61.9000) -- (68.9000,61.3000) -- (69.0000,60.8000) -- (69.1000,60.2000) -- (69.2000,59.8000) -- (69.3000,59.4000) -- (69.4000,59.0000) -- (69.5000,58.6000) -- (69.5000,58.3000) -- (69.6000,58.1000) -- (69.7000,57.8000) -- (69.8000,57.7000) -- (69.9000,57.5000) -- (70.0000,57.4000) -- (70.1000,57.3000) -- (70.1000,57.3000) -- (70.2000,57.3000) -- (70.3000,57.3000) -- (70.4000,57.3000) -- (70.5000,57.4000) -- (70.6000,57.5000) -- (70.7000,57.6000) -- (70.8000,57.7000) -- (70.8000,57.8000) -- (70.9000,57.9000) -- (71.0000,58.0000) -- (71.1000,58.2000) -- (71.2000,58.3000) -- (71.3000,58.4000) -- (71.4000,58.5000) -- (71.4000,58.6000) -- (71.5000,58.7000) -- (71.6000,58.7000) -- (71.7000,58.8000) -- (71.8000,58.8000) -- (71.9000,58.8000) -- (72.0000,58.8000) -- (72.1000,58.7000) -- (72.1000,58.6000) -- (72.2000,58.5000) -- (72.3000,58.3000) -- (72.4000,58.2000) -- (72.5000,57.9000) -- (72.6000,57.7000) -- (72.7000,57.4000) -- (72.8000,57.1000) -- (72.8000,56.8000) -- (72.9000,56.4000) -- (73.0000,56.0000) -- (73.1000,55.6000) -- (73.2000,55.2000) -- (73.3000,54.7000) -- (73.4000,54.3000) -- (73.4000,53.8000) -- (73.5000,53.3000) -- (73.6000,52.7000) -- (73.7000,52.2000) -- (73.8000,51.7000) -- (73.9000,51.1000) -- (74.0000,50.6000) -- (74.1000,50.0000) -- (74.1000,49.5000) -- (74.2000,49.0000) -- (74.3000,48.5000) -- (74.4000,48.0000) -- (74.5000,47.5000) -- (74.6000,47.1000) -- (74.7000,46.6000) -- (74.7000,46.2000) -- (74.8000,45.8000) -- (74.9000,45.5000) -- (75.0000,45.2000) -- (75.1000,44.9000) -- (75.2000,44.7000) -- (75.3000,44.5000) -- (75.4000,44.3000) -- (75.4000,44.2000) -- (75.5000,44.1000) -- (75.6000,44.1000) -- (75.7000,44.1000) -- (75.8000,44.2000) -- (75.9000,44.3000) -- (76.0000,44.4000) -- (76.0000,44.6000) -- (76.1000,44.8000) -- (76.2000,45.1000) -- (76.3000,45.4000) -- (76.4000,45.8000) -- (76.5000,46.1000) -- (76.6000,46.5000) -- (76.7000,47.0000) -- (76.7000,47.4000) -- (76.8000,47.9000) -- (76.9000,48.5000) -- (77.0000,49.0000) -- (77.1000,49.5000) -- (77.2000,50.1000) -- (77.3000,50.7000) -- (77.3000,51.2000) -- (77.4000,51.8000) -- (77.5000,52.4000) -- (77.6000,53.0000) -- (77.7000,53.5000) -- (77.8000,54.1000) -- (77.9000,54.6000) -- (78.0000,55.2000) -- (78.0000,55.7000) -- (78.1000,56.2000) -- (78.2000,56.7000) -- (78.3000,57.1000) -- (78.4000,57.5000) -- (78.5000,57.9000) -- (78.6000,58.3000) -- (78.7000,58.6000) -- (78.7000,58.9000) -- (78.8000,59.2000) -- (78.9000,59.4000) -- (79.0000,59.6000) -- (79.1000,59.8000) -- (79.2000,59.9000) -- (79.3000,60.0000) -- (79.3000,60.1000) -- (79.4000,60.1000) -- (79.5000,60.2000) -- (79.6000,60.1000) -- (79.7000,60.1000) -- (79.8000,60.0000) -- (79.9000,60.0000) -- (80.0000,59.9000) -- (80.0000,59.7000) -- (80.1000,59.6000) -- (80.2000,59.5000) -- (80.3000,59.3000) -- (80.4000,59.2000) -- (80.5000,59.0000) -- (80.6000,58.9000) -- (80.6000,58.7000) -- (80.7000,58.6000) -- (80.8000,58.5000) -- (80.9000,58.4000) -- (81.0000,58.3000) -- (81.1000,58.2000) -- (81.2000,58.2000) -- (81.3000,58.2000) -- (81.3000,58.2000) -- (81.4000,58.2000) -- (81.5000,58.3000) -- (81.6000,58.5000) -- (81.7000,58.6000) -- (81.8000,58.8000) -- (81.9000,59.1000) -- (81.9000,59.3000) -- (82.0000,59.7000) -- (82.1000,60.0000) -- (82.2000,60.4000) -- (82.3000,60.9000) -- (82.4000,61.4000) -- (82.5000,61.9000) -- (82.6000,62.5000) -- (82.6000,63.1000) -- (82.7000,63.8000) -- (82.8000,64.5000) -- (82.9000,65.2000) -- (83.0000,66.0000) -- (83.1000,66.8000) -- (83.2000,67.6000) -- (83.3000,68.4000) -- (83.3000,69.3000) -- (83.4000,70.1000) -- (83.5000,71.0000) -- (83.6000,71.9000) -- (83.7000,72.8000) -- (83.8000,73.7000) -- (83.9000,74.5000) -- (83.9000,75.4000) -- (84.0000,76.3000) -- (84.1000,77.1000) -- (84.2000,77.9000) -- (84.3000,78.7000) -- (84.4000,79.5000) -- (84.5000,80.2000) -- (84.6000,80.9000) -- (84.6000,81.6000) -- (84.7000,82.2000) -- (84.8000,82.8000) -- (84.9000,83.3000) -- (85.0000,83.7000) -- (85.1000,84.2000) -- (85.2000,84.5000) -- (85.2000,84.8000) -- (85.3000,85.0000) -- (85.4000,85.2000) -- (85.5000,85.3000) -- (85.6000,85.4000) -- (85.7000,85.4000) -- (85.8000,85.3000) -- (85.9000,85.1000) -- (85.9000,84.9000) -- (86.0000,84.7000) -- (86.1000,84.4000) -- (86.2000,84.0000) -- (86.3000,83.5000) -- (86.4000,83.0000) -- (86.5000,82.5000) -- (86.5000,81.9000) -- (86.6000,81.3000) -- (86.7000,80.6000) -- (86.8000,79.9000) -- (86.9000,79.1000) -- (87.0000,78.3000) -- (87.1000,77.5000) -- (87.2000,76.7000) -- (87.2000,75.8000) -- (87.3000,74.9000) -- (87.4000,74.0000) -- (87.5000,73.1000) -- (87.6000,72.2000) -- (87.7000,71.3000) -- (87.8000,70.4000) -- (87.9000,69.5000) -- (87.9000,68.7000) -- (88.0000,67.8000) -- (88.1000,67.0000) -- (88.2000,66.1000) -- (88.3000,65.3000) -- (88.4000,64.6000) -- (88.5000,63.8000) -- (88.5000,63.1000) -- (88.6000,62.5000) -- (88.7000,61.8000) -- (88.8000,61.2000) -- (88.9000,60.7000) -- (89.0000,60.2000) -- (89.1000,59.7000) -- (89.2000,59.3000) -- (89.2000,58.9000) -- (89.3000,58.6000) -- (89.4000,58.3000) -- (89.5000,58.0000) -- (89.6000,57.8000) -- (89.7000,57.6000) -- (89.8000,57.5000) -- (89.8000,57.4000) -- (89.9000,57.3000) -- (90.0000,57.3000) -- (90.1000,57.3000) -- (90.2000,57.3000) -- (90.3000,57.3000) -- (90.4000,57.4000) -- (90.5000,57.5000) -- (90.5000,57.6000) -- (90.6000,57.7000) -- (90.7000,57.8000) -- (90.8000,57.9000) -- (90.9000,58.1000) -- (91.0000,58.2000) -- (91.1000,58.3000) -- (91.1000,58.4000) -- (91.2000,58.5000) -- (91.3000,58.6000) -- (91.4000,58.7000) -- (91.5000,58.8000) -- (91.6000,58.8000) -- (91.7000,58.8000) -- (91.8000,58.8000) -- (91.8000,58.8000) -- (91.9000,58.7000) -- (92.0000,58.6000) -- (92.1000,58.5000) -- (92.2000,58.3000) -- (92.3000,58.2000) -- (92.4000,57.9000) -- (92.5000,57.7000) -- (92.5000,57.4000) -- (92.6000,57.1000) -- (92.7000,56.8000) -- (92.8000,56.4000) -- (92.9000,56.0000) -- (93.0000,55.6000) -- (93.1000,55.2000) -- (93.1000,54.7000) -- (93.2000,54.2000) -- (93.3000,53.7000) -- (93.4000,53.2000) -- (93.5000,52.7000) -- (93.6000,52.2000) -- (93.7000,51.6000) -- (93.8000,51.1000) -- (93.8000,50.5000) -- (93.9000,50.0000) -- (94.0000,49.5000) -- (94.1000,48.9000) -- (94.2000,48.4000) -- (94.3000,47.9000) -- (94.4000,47.5000) -- (94.4000,47.0000) -- (94.5000,46.6000) -- (94.6000,46.2000) -- (94.7000,45.8000) -- (94.8000,45.4000) -- (94.9000,45.1000) -- (95.0000,44.9000) -- (95.1000,44.6000) -- (95.1000,44.4000) -- (95.2000,44.3000) -- (95.3000,44.2000) -- (95.4000,44.1000) -- (95.5000,44.1000) -- (95.6000,44.1000) -- (95.7000,44.2000) -- (95.7000,44.3000) -- (95.8000,44.4000) -- (95.9000,44.6000) -- (96.0000,44.8000) -- (96.1000,45.1000) -- (96.2000,45.4000) -- (96.3000,45.8000) -- (96.4000,46.2000) -- (96.4000,46.6000) -- (96.5000,47.0000) -- (96.6000,47.5000) -- (96.7000,48.0000) -- (96.8000,48.5000) -- (96.9000,49.0000) -- (97.0000,49.6000) -- (97.1000,50.1000) -- (97.1000,50.7000) -- (97.2000,51.3000) -- (97.3000,51.9000) -- (97.4000,52.4000) -- (97.5000,53.0000) -- (97.6000,53.6000) -- (97.7000,54.2000) -- (97.7000,54.7000) -- (97.8000,55.2000) -- (97.9000,55.8000) -- (98.0000,56.2000) -- (98.1000,56.7000) -- (98.2000,57.2000) -- (98.3000,57.6000) -- (98.4000,58.0000) -- (98.4000,58.3000) -- (98.5000,58.7000) -- (98.6000,59.0000) -- (98.7000,59.2000) -- (98.8000,59.5000) -- (98.9000,59.7000) -- (99.0000,59.8000) -- (99.0000,60.0000) -- (99.1000,60.1000) -- (99.2000,60.1000) -- (99.3000,60.2000) -- (99.4000,60.2000) -- (99.5000,60.2000) -- (99.6000,60.1000) -- (99.7000,60.0000) -- (99.7000,60.0000) -- (99.8000,59.9000) -- (99.9000,59.7000) -- (100.0000,59.6000) -- (100.1000,59.5000) -- (100.2000,59.3000) -- (100.3000,59.2000) -- (100.3000,59.0000) -- (100.4000,58.9000) -- (100.5000,58.7000) -- (100.6000,58.6000) -- (100.7000,58.5000) -- (100.8000,58.4000) -- (100.9000,58.3000) -- (101.0000,58.2000) -- (101.0000,58.2000) -- (101.1000,58.2000) -- (101.2000,58.2000) -- (101.3000,58.2000) -- (101.4000,58.3000) -- (101.5000,58.4000) -- (101.6000,58.6000) -- (101.7000,58.8000) -- (101.7000,59.1000) -- (101.8000,59.3000) -- (101.9000,59.7000) -- (102.0000,60.0000) -- (102.1000,60.5000) -- (102.2000,60.9000) -- (102.3000,61.4000) -- (102.3000,62.0000) -- (102.4000,62.6000) -- (102.5000,63.2000) -- (102.6000,63.8000) -- (102.7000,64.5000) -- (102.8000,65.3000) -- (102.9000,66.0000) -- (103.0000,66.8000) -- (103.0000,67.6000) -- (103.1000,68.5000) -- (103.2000,69.3000) -- (103.3000,70.2000) -- (103.4000,71.1000) -- (103.5000,72.0000) -- (103.6000,72.8000) -- (103.6000,73.7000) -- (103.7000,74.6000) -- (103.8000,75.5000) -- (103.9000,76.4000) -- (104.0000,77.2000) -- (104.1000,78.0000) -- (104.2000,78.8000) -- (104.3000,79.6000) -- (104.3000,80.3000) -- (104.4000,81.0000) -- (104.5000,81.7000) -- (104.6000,82.3000) -- (104.7000,82.8000) -- (104.8000,83.4000) -- (104.9000,83.8000) -- (104.9000,84.2000) -- (105.0000,84.6000) -- (105.1000,84.9000) -- (105.2000,85.1000) -- (105.3000,85.3000) -- (105.4000,85.4000) -- (105.5000,85.4000) -- (105.6000,85.4000) -- (105.6000,85.3000) -- (105.7000,85.2000) -- (105.8000,85.0000) -- (105.9000,84.7000) -- (106.0000,84.4000) -- (106.1000,84.0000) -- (106.2000,83.5000) -- (106.3000,83.0000) -- (106.3000,82.5000) -- (106.4000,81.9000) -- (106.5000,81.2000) -- (106.6000,80.5000) -- (106.7000,79.8000) -- (106.8000,79.1000) -- (106.9000,78.3000) -- (106.9000,77.4000) -- (107.0000,76.6000) -- (107.1000,75.7000) -- (107.2000,74.8000) -- (107.3000,73.9000) -- (107.4000,73.0000) -- (107.5000,72.1000) -- (107.6000,71.2000) -- (107.6000,70.3000) -- (107.7000,69.4000) -- (107.8000,68.6000) -- (107.9000,67.7000) -- (108.0000,66.9000) -- (108.1000,66.0000) -- (108.2000,65.2000) -- (108.2000,64.5000) -- (108.3000,63.7000) -- (108.4000,63.0000) -- (108.5000,62.4000) -- (108.6000,61.8000) -- (108.7000,61.2000) -- (108.8000,60.6000) -- (108.9000,60.1000) -- (108.9000,59.6000) -- (109.0000,59.2000) -- (109.1000,58.9000) -- (109.2000,58.5000) -- (109.3000,58.2000) -- (109.4000,58.0000) -- (109.5000,57.8000) -- (109.5000,57.6000) -- (109.6000,57.5000) -- (109.7000,57.4000) -- (109.8000,57.3000) -- (109.9000,57.3000) -- (110.0000,57.3000) -- (110.1000,57.3000) -- (110.2000,57.3000) -- (110.2000,57.4000) -- (110.3000,57.5000) -- (110.4000,57.6000) -- (110.5000,57.7000) -- (110.6000,57.8000) -- (110.7000,57.9000) -- (110.8000,58.1000) -- (110.9000,58.2000) -- (110.9000,58.3000) -- (111.0000,58.4000) -- (111.1000,58.5000) -- (111.2000,58.6000) -- (111.3000,58.7000) -- (111.4000,58.8000) -- (111.5000,58.8000) -- (111.5000,58.8000) -- (111.6000,58.8000) -- (111.7000,58.8000) -- (111.8000,58.7000) -- (111.9000,58.6000) -- (112.0000,58.5000) -- (112.1000,58.3000) -- (112.2000,58.2000) -- (112.2000,57.9000) -- (112.3000,57.7000) -- (112.4000,57.4000) -- (112.5000,57.1000) -- (112.6000,56.8000) -- (112.7000,56.4000) -- (112.8000,56.0000) -- (112.8000,55.6000) -- (112.9000,55.1000) -- (113.0000,54.7000) -- (113.1000,54.2000) -- (113.2000,53.7000) -- (113.3000,53.2000) -- (113.4000,52.7000) -- (113.5000,52.1000) -- (113.5000,51.6000) -- (113.6000,51.0000) -- (113.7000,50.5000) -- (113.8000,49.9000) -- (113.9000,49.4000) -- (114.0000,48.9000) -- (114.1000,48.4000) -- (114.1000,47.9000) -- (114.2000,47.4000) -- (114.3000,46.9000) -- (114.4000,46.5000) -- (114.5000,46.1000) -- (114.6000,45.7000) -- (114.7000,45.4000) -- (114.8000,45.1000) -- (114.8000,44.8000) -- (114.9000,44.6000) -- (115.0000,44.4000) -- (115.1000,44.3000) -- (115.2000,44.1000) -- (115.3000,44.1000) -- (115.4000,44.1000) -- (115.5000,44.1000) -- (115.5000,44.1000) -- (115.6000,44.3000) -- (115.7000,44.4000) -- (115.8000,44.6000) -- (115.9000,44.8000) -- (116.0000,45.1000) -- (116.1000,45.4000) -- (116.1000,45.8000) -- (116.2000,46.2000) -- (116.3000,46.6000) -- (116.4000,47.0000) -- (116.5000,47.5000) -- (116.6000,48.0000) -- (116.7000,48.5000) -- (116.8000,49.1000) -- (116.8000,49.6000) -- (116.9000,50.2000) -- (117.0000,50.8000) -- (117.1000,51.3000) -- (117.2000,51.9000) -- (117.3000,52.5000) -- (117.4000,53.1000) -- (117.4000,53.6000) -- (117.5000,54.2000) -- (117.6000,54.8000) -- (117.7000,55.3000) -- (117.8000,55.8000) -- (117.9000,56.3000) -- (118.0000,56.8000) -- (118.1000,57.2000) -- (118.1000,57.6000) -- (118.2000,58.0000) -- (118.3000,58.4000) -- (118.4000,58.7000) -- (118.5000,59.0000) -- (118.6000,59.3000) -- (118.7000,59.5000) -- (118.7000,59.7000) -- (118.8000,59.9000) -- (118.9000,60.0000) -- (119.0000,60.1000) -- (119.1000,60.1000) -- (119.2000,60.2000) -- (119.3000,60.2000) -- (119.4000,60.2000) -- (119.4000,60.1000) -- (119.5000,60.1000) -- (119.6000,60.0000) -- (119.7000,59.9000) -- (119.8000,59.7000) -- (119.9000,59.6000) -- (120.0000,59.5000) -- (120.0000,59.3000) -- (120.1000,59.1000) -- (120.2000,59.0000) -- (120.3000,58.8000) -- (120.4000,58.7000) -- (120.5000,58.6000) -- (120.6000,58.4000) -- (120.7000,58.3000) -- (120.7000,58.2000) -- (120.8000,58.2000) -- (120.9000,58.2000) -- (121.0000,58.1000) -- (121.1000,58.2000) -- (121.2000,58.2000) -- (121.3000,58.3000) -- (121.4000,58.4000) -- (121.4000,58.6000) -- (121.5000,58.8000) -- (121.6000,59.1000) -- (121.7000,59.3000) -- (121.8000,59.7000) -- (121.9000,60.1000) -- (122.0000,60.5000) -- (122.0000,60.9000) -- (122.1000,61.5000) -- (122.2000,62.0000) -- (122.3000,62.6000) -- (122.4000,63.2000) -- (122.5000,63.9000) -- (122.6000,64.6000) -- (122.7000,65.3000) -- (122.7000,66.1000) -- (122.8000,66.9000) -- (122.9000,67.7000) -- (123.0000,68.5000) -- (123.1000,69.4000) -- (123.2000,70.3000) -- (123.3000,71.1000) -- (123.3000,72.0000) -- (123.4000,72.9000) -- (123.5000,73.8000) -- (123.6000,74.7000) -- (123.7000,75.6000) -- (123.8000,76.4000) -- (123.9000,77.3000) -- (124.0000,78.1000) -- (124.0000,78.9000) -- (124.1000,79.7000) -- (124.2000,80.4000) -- (124.3000,81.1000) -- (124.4000,81.7000) -- (124.5000,82.4000) -- (124.6000,82.9000) -- (124.6000,83.4000) -- (124.7000,83.9000) -- (124.8000,84.3000) -- (124.9000,84.6000) -- (125.0000,84.9000) -- (125.1000,85.2000) -- (125.2000,85.3000) -- (125.3000,85.4000) -- (125.3000,85.5000) -- (125.4000,85.4000) -- (125.5000,85.3000) -- (125.6000,85.2000) -- (125.7000,85.0000) -- (125.8000,84.7000) -- (125.9000,84.4000) -- (125.9000,84.0000) -- (126.0000,83.5000) -- (126.1000,83.0000) -- (126.2000,82.4000) -- (126.3000,81.8000) -- (126.4000,81.2000) -- (126.5000,80.5000) -- (126.6000,79.8000) -- (126.6000,79.0000) -- (126.7000,78.2000) -- (126.8000,77.4000) -- (126.9000,76.5000) -- (127.0000,75.7000) -- (127.1000,74.8000) -- (127.2000,73.9000) -- (127.3000,73.0000) -- (127.3000,72.1000) -- (127.4000,71.2000) -- (127.5000,70.3000) -- (127.6000,69.4000) -- (127.7000,68.5000) -- (127.8000,67.6000) -- (127.9000,66.8000) -- (127.9000,66.0000) -- (128.0000,65.2000) -- (128.1000,64.4000) -- (128.2000,63.7000) -- (128.3000,63.0000) -- (128.4000,62.3000) -- (128.5000,61.7000) -- (128.6000,61.1000) -- (128.6000,60.5000) -- (128.7000,60.0000) -- (128.8000,59.6000) -- (128.9000,59.2000) -- (129.0000,58.8000) -- (129.1000,58.5000) -- (129.2000,58.2000) -- (129.2000,57.9000) -- (129.3000,57.7000) -- (129.4000,57.6000) -- (129.5000,57.4000) -- (129.6000,57.3000) -- (129.7000,57.3000) -- (129.8000,57.2000) -- (129.9000,57.3000) -- (129.9000,57.3000) -- (130.0000,57.3000) -- (130.1000,57.4000) -- (130.2000,57.5000) -- (130.3000,57.6000) -- (130.4000,57.7000) -- (130.5000,57.8000) -- (130.5000,58.0000) -- (130.6000,58.1000) -- (130.7000,58.2000) -- (130.8000,58.3000) -- (130.9000,58.5000) -- (131.0000,58.6000) -- (131.1000,58.7000) -- (131.2000,58.7000) -- (131.2000,58.8000) -- (131.3000,58.8000) -- (131.4000,58.8000) -- (131.5000,58.8000) -- (131.6000,58.8000) -- (131.7000,58.7000) -- (131.8000,58.6000) -- (131.8000,58.5000) -- (131.9000,58.3000) -- (132.0000,58.2000) -- (132.1000,57.9000) -- (132.2000,57.7000) -- (132.3000,57.4000) -- (132.4000,57.1000) -- (132.5000,56.8000) -- (132.5000,56.4000) -- (132.6000,56.0000) -- (132.7000,55.6000) -- (132.8000,55.1000) -- (132.9000,54.6000) -- (133.0000,54.2000) -- (133.1000,53.7000) -- (133.2000,53.1000) -- (133.2000,52.6000) -- (133.3000,52.1000) -- (133.4000,51.5000) -- (133.5000,51.0000) -- (133.6000,50.4000) -- (133.7000,49.9000) -- (133.8000,49.4000) -- (133.8000,48.8000) -- (133.9000,48.3000) -- (134.0000,47.8000) -- (134.1000,47.3000) -- (134.2000,46.9000) -- (134.3000,46.5000) -- (134.4000,46.1000) -- (134.5000,45.7000) -- (134.5000,45.3000) -- (134.6000,45.0000) -- (134.7000,44.8000) -- (134.8000,44.6000) -- (134.9000,44.4000) -- (135.0000,44.2000) -- (135.1000,44.1000) -- (135.1000,44.1000) -- (135.2000,44.0000) -- (135.3000,44.1000) -- (135.4000,44.1000) -- (135.5000,44.2000) -- (135.6000,44.4000) -- (135.7000,44.6000) -- (135.8000,44.8000) -- (135.8000,45.1000) -- (135.9000,45.4000) -- (136.0000,45.8000) -- (136.1000,46.2000) -- (136.2000,46.6000) -- (136.3000,47.1000) -- (136.4000,47.5000) -- (136.5000,48.0000) -- (136.5000,48.6000) -- (136.6000,49.1000) -- (136.7000,49.7000) -- (136.8000,50.2000) -- (136.9000,50.8000) -- (137.0000,51.4000) -- (137.1000,52.0000) -- (137.1000,52.6000) -- (137.2000,53.1000) -- (137.3000,53.7000) -- (137.4000,54.3000) -- (137.5000,54.8000) -- (137.6000,55.4000) -- (137.7000,55.9000) -- (137.8000,56.4000) -- (137.8000,56.8000) -- (137.9000,57.3000) -- (138.0000,57.7000) -- (138.1000,58.1000) -- (138.2000,58.4000) -- (138.3000,58.8000) -- (138.4000,59.1000) -- (138.4000,59.3000) -- (138.5000,59.5000) -- (138.6000,59.7000) -- (138.7000,59.9000) -- (138.8000,60.0000) -- (138.9000,60.1000) -- (139.0000,60.2000) -- (139.1000,60.2000) -- (139.1000,60.2000) -- (139.2000,60.2000) -- (139.3000,60.1000) -- (139.4000,60.1000) -- (139.5000,60.0000) -- (139.6000,59.9000) -- (139.7000,59.7000) -- (139.7000,59.6000) -- (139.8000,59.4000) -- (139.9000,59.3000) -- (140.0000,59.1000) -- (140.1000,59.0000) -- (140.2000,58.8000) -- (140.3000,58.7000) -- (140.4000,58.5000) -- (140.4000,58.4000) -- (140.5000,58.3000) -- (140.6000,58.2000) -- (140.7000,58.2000) -- (140.8000,58.1000) -- (140.9000,58.1000) -- (141.0000,58.1000) -- (141.1000,58.2000) -- (141.1000,58.3000) -- (141.2000,58.4000) -- (141.3000,58.6000) -- (141.4000,58.8000) -- (141.5000,59.1000) -- (141.6000,59.4000) -- (141.7000,59.7000) -- (141.7000,60.1000) -- (141.8000,60.5000) -- (141.9000,61.0000) -- (142.0000,61.5000) -- (142.1000,62.0000) -- (142.2000,62.6000) -- (142.3000,63.3000) -- (142.4000,63.9000) -- (142.4000,64.6000) -- (142.5000,65.4000) -- (142.6000,66.1000) -- (142.7000,67.0000);



  \end{scope}
  \begin{scope}[cm={{1.00588,0.0,0.0,1.00588,(-114.00002,89.46968)}},draw=black,line join=round,line cap=round,line width=0.480pt]
    \path[draw] (56.5000,13.5000) -- (56.5000,95.5000) -- (142.5000,95.5000) -- (142.5000,13.5000) -- (56.5000,13.5000);



  \end{scope}
\end{scope}

\end{tikzpicture}


%    \caption{Energies, model states ($\alpha_0,\dots\in\mathbf{q}$), and parameters ($T,b_0$) evolutions}
%    \label{fig:ener}
%  \end{subfigure}
%  \footnotesize
%  \caption{CPP with novel Zamboni-like motion (\hyperref[fig:trajs-I-static]{I},\hyperref[fig:trajs-II-static]{II}) and Planning-scheduling of CPP and ground patterns detections with PedNet CNN (\hyperref[fig:trajs-dyn-i]{i},\hyperref[fig:trajs-dyn-ii]{ii}) in terms of the path, energies, and plans-schedules under different conditions (\hyperref[fig:trajs-I-static]{I}--\hyperref[fig:trajs-dyn-i]{i},\hyperref[fig:trajs-II-static]{II}--\hyperref[fig:trajs-dyn-ii]{ii}): wind speed and direction, battery behavior, and parameters initial values.}
%  \label{fig:res}
%  \vspace*{-3ex}
%\end{figure*}
\begin{figure*}
  \centering
  \footnotesize
  \begin{minipage}[t]{0.63\columnwidth}
    
% ! replace blue with black
\definecolor{cd9d9d9}{RGB}{235,235,235}
\definecolor{cffffff}{RGB}{255,255,255}
\definecolor{ca0a0a4}{RGB}{160,160,164}
\definecolor{ce10000}{RGB}{225,0,0}
\definecolor{cff0000}{RGB}{255,0,0}


\def \globalscale {0.940000}
\begin{tikzpicture}[y=0.80pt, x=0.80pt, yscale=-1*\globalscale, xscale=1*\globalscale, inner sep=0pt, outer sep=0pt]
\color{blue}
\begin{scope}[shift={(598.26327,165.72719)},draw=blue,even odd rule,line cap=rect,line join=bevel,line width=0.800pt]
  \path[draw=cd9d9d9,line cap=butt,line join=miter,line width=1.440pt,miter limit=4.00] (-302.5903,-136.2302) -- (-281.2030,-152.5408);



  \path[draw=cd9d9d9,line cap=butt,line join=miter,line width=1.440pt,miter limit=4.00] (-303.0158,-64.4976) -- (-281.5953,-47.1121);



  \path[fill=cd9d9d9,dash pattern=on 0.99pt off 0.99pt,even odd rule,line cap=round,line width=0.248pt,miter limit=4.00,rounded corners=0.0000cm] (-617.3423,-41.6805) rectangle (-187.4691,79.0929);



  \path[draw=cffffff,line cap=butt,line join=miter,line width=1.440pt,miter limit=4.00] (-302.9246,46.2884) -- (-281.1456,66.6309);



  \begin{scope}[draw=blue,line cap=rect,line join=bevel,line width=0.800pt]
  \end{scope}
  \begin{scope}[scale=1.006,draw=blue,line cap=rect,line join=bevel,line width=0.800pt]
  \end{scope}
  \begin{scope}[scale=1.006,draw=blue,line cap=rect,line join=bevel,line width=0.800pt]
  \end{scope}
  \begin{scope}[cm={{1.00588,0.0,0.0,1.00588,(39.2294,93.5471)}},draw=blue,line cap=rect,line join=bevel,line width=0.800pt]
  \end{scope}
  \begin{scope}[cm={{1.00588,0.0,0.0,1.00588,(39.2294,93.5471)}},draw=blue,line cap=rect,line join=bevel,line width=0.800pt]
  \end{scope}
  \begin{scope}[cm={{1.00588,0.0,0.0,1.00588,(39.2294,93.5471)}},draw=blue,line cap=rect,line join=bevel,line width=0.800pt]
  \end{scope}
  \begin{scope}[cm={{1.00588,0.0,0.0,1.00588,(39.2294,93.5471)}},draw=blue,line cap=rect,line join=bevel,line width=0.800pt]
  \end{scope}
  \begin{scope}[cm={{1.00588,0.0,0.0,1.00588,(39.2294,93.5471)}},draw=blue,line cap=rect,line join=bevel,line width=0.800pt]
  \end{scope}
  \begin{scope}[cm={{1.00588,0.0,0.0,1.00588,(-426.03325,-54.43023)}},draw=blue,line cap=rect,line join=bevel,line width=0.800pt]
    \path[fill=blue] (0.0000,0.0000) node[above right] (text34) {32};



  \end{scope}
  \begin{scope}[cm={{1.00588,0.0,0.0,1.00588,(39.2294,93.5471)}},draw=blue,line cap=rect,line join=bevel,line width=0.800pt]
  \end{scope}
  \begin{scope}[scale=1.006,draw=blue,line cap=rect,line join=bevel,line width=0.800pt]
  \end{scope}
  \begin{scope}[scale=1.006,draw=blue,line cap=rect,line join=bevel,line width=0.800pt]
  \end{scope}
  \begin{scope}[cm={{1.00588,0.0,0.0,1.00588,(39.2294,68.4)}},draw=blue,line cap=rect,line join=bevel,line width=0.800pt]
  \end{scope}
  \begin{scope}[cm={{1.00588,0.0,0.0,1.00588,(39.2294,68.4)}},draw=blue,line cap=rect,line join=bevel,line width=0.800pt]
  \end{scope}
  \begin{scope}[cm={{1.00588,0.0,0.0,1.00588,(39.2294,68.4)}},draw=blue,line cap=rect,line join=bevel,line width=0.800pt]
  \end{scope}
  \begin{scope}[cm={{1.00588,0.0,0.0,1.00588,(39.2294,68.4)}},draw=blue,line cap=rect,line join=bevel,line width=0.800pt]
  \end{scope}
  \begin{scope}[cm={{1.00588,0.0,0.0,1.00588,(39.2294,68.4)}},draw=blue,line cap=rect,line join=bevel,line width=0.800pt]
  \end{scope}
  \begin{scope}[cm={{1.00588,0.0,0.0,1.00588,(-425.97692,-85.57732)}},draw=blue,line cap=rect,line join=bevel,line width=0.800pt]
    \path[fill=blue] (0.0000,0.0000) node[above right] (text64) {36};



  \end{scope}
  \begin{scope}[cm={{1.00588,0.0,0.0,1.00588,(39.2294,68.4)}},draw=blue,line cap=rect,line join=bevel,line width=0.800pt]
  \end{scope}
  \begin{scope}[scale=1.006,draw=blue,line cap=rect,line join=bevel,line width=0.800pt]
  \end{scope}
  \begin{scope}[scale=1.006,draw=blue,line cap=rect,line join=bevel,line width=0.800pt]
  \end{scope}
  \begin{scope}[cm={{1.00588,0.0,0.0,1.00588,(39.2294,43.2529)}},draw=blue,line cap=rect,line join=bevel,line width=0.800pt]
  \end{scope}
  \begin{scope}[cm={{1.00588,0.0,0.0,1.00588,(39.2294,43.2529)}},draw=blue,line cap=rect,line join=bevel,line width=0.800pt]
  \end{scope}
  \begin{scope}[cm={{1.00588,0.0,0.0,1.00588,(39.2294,43.2529)}},draw=blue,line cap=rect,line join=bevel,line width=0.800pt]
  \end{scope}
  \begin{scope}[cm={{1.00588,0.0,0.0,1.00588,(39.2294,43.2529)}},draw=blue,line cap=rect,line join=bevel,line width=0.800pt]
  \end{scope}
  \begin{scope}[cm={{1.00588,0.0,0.0,1.00588,(39.2294,43.2529)}},draw=blue,line cap=rect,line join=bevel,line width=0.800pt]
  \end{scope}
  \begin{scope}[cm={{1.00588,0.0,0.0,1.00588,(-426.0413,-118.22444)}},draw=blue,line cap=rect,line join=bevel,line width=0.800pt]
    \path[fill=blue] (0.0000,0.0000) node[above right] (text94) {40};



  \end{scope}
  \begin{scope}[cm={{1.00588,0.0,0.0,1.00588,(39.2294,43.2529)}},draw=blue,line cap=rect,line join=bevel,line width=0.800pt]
  \end{scope}
  \begin{scope}[scale=1.006,draw=blue,line cap=rect,line join=bevel,line width=0.800pt]
  \end{scope}
  \begin{scope}[scale=1.006,draw=blue,line cap=rect,line join=bevel,line width=0.800pt]
  \end{scope}
  \begin{scope}[cm={{1.00588,0.0,0.0,1.00588,(53.3118,110.647)}},draw=blue,line cap=rect,line join=bevel,line width=0.800pt]
  \end{scope}
  \begin{scope}[cm={{1.00588,0.0,0.0,1.00588,(53.3118,110.647)}},draw=blue,line cap=rect,line join=bevel,line width=0.800pt]
  \end{scope}
  \begin{scope}[cm={{1.00588,0.0,0.0,1.00588,(53.3118,110.647)}},draw=blue,line cap=rect,line join=bevel,line width=0.800pt]
  \end{scope}
  \begin{scope}[cm={{1.00588,0.0,0.0,1.00588,(53.3118,110.647)}},draw=blue,line cap=rect,line join=bevel,line width=0.800pt]
  \end{scope}
  \begin{scope}[cm={{1.00588,0.0,0.0,1.00588,(53.3118,110.647)}},draw=blue,line cap=rect,line join=bevel,line width=0.800pt]
  \end{scope}
  \begin{scope}[cm={{1.00588,0.0,0.0,1.00588,(-170.37572,77.03586)}},draw=blue,line cap=rect,line join=bevel,line width=0.800pt]
    \path[fill=blue] (0.0000,0.0000) node[above right] (text124) {\scriptsize 0};



  \end{scope}
  \begin{scope}[cm={{1.00588,0.0,0.0,1.00588,(53.3118,110.647)}},draw=blue,line cap=rect,line join=bevel,line width=0.800pt]
  \end{scope}
  \begin{scope}[scale=1.006,draw=blue,line cap=rect,line join=bevel,line width=0.800pt]
  \end{scope}
  \begin{scope}[scale=1.006,draw=blue,line cap=rect,line join=bevel,line width=0.800pt]
  \end{scope}
  \begin{scope}[cm={{1.00588,0.0,0.0,1.00588,(79.4647,110.647)}},draw=blue,line cap=rect,line join=bevel,line width=0.800pt]
  \end{scope}
  \begin{scope}[cm={{1.00588,0.0,0.0,1.00588,(79.4647,110.647)}},draw=blue,line cap=rect,line join=bevel,line width=0.800pt]
  \end{scope}
  \begin{scope}[cm={{1.00588,0.0,0.0,1.00588,(79.4647,110.647)}},draw=blue,line cap=rect,line join=bevel,line width=0.800pt]
  \end{scope}
  \begin{scope}[cm={{1.00588,0.0,0.0,1.00588,(79.4647,110.647)}},draw=blue,line cap=rect,line join=bevel,line width=0.800pt]
  \end{scope}
  \begin{scope}[cm={{1.00588,0.0,0.0,1.00588,(79.4647,110.647)}},draw=blue,line cap=rect,line join=bevel,line width=0.800pt]
  \end{scope}
  \begin{scope}[cm={{1.00588,0.0,0.0,1.00588,(-144.22284,77.03586)}},draw=blue,line cap=rect,line join=bevel,line width=0.800pt]
    \path[fill=blue] (0.0000,0.0000) node[above right] (text156) {\scriptsize 1};



  \end{scope}
  \begin{scope}[cm={{1.00588,0.0,0.0,1.00588,(79.4647,110.647)}},draw=blue,line cap=rect,line join=bevel,line width=0.800pt]
  \end{scope}
  \begin{scope}[scale=1.006,draw=blue,line cap=rect,line join=bevel,line width=0.800pt]
  \end{scope}
  \begin{scope}[scale=1.006,draw=blue,line cap=rect,line join=bevel,line width=0.800pt]
  \end{scope}
  \begin{scope}[cm={{1.00588,0.0,0.0,1.00588,(105.618,110.647)}},draw=blue,line cap=rect,line join=bevel,line width=0.800pt]
  \end{scope}
  \begin{scope}[cm={{1.00588,0.0,0.0,1.00588,(105.618,110.647)}},draw=blue,line cap=rect,line join=bevel,line width=0.800pt]
  \end{scope}
  \begin{scope}[cm={{1.00588,0.0,0.0,1.00588,(105.618,110.647)}},draw=blue,line cap=rect,line join=bevel,line width=0.800pt]
  \end{scope}
  \begin{scope}[cm={{1.00588,0.0,0.0,1.00588,(105.618,110.647)}},draw=blue,line cap=rect,line join=bevel,line width=0.800pt]
  \end{scope}
  \begin{scope}[cm={{1.00588,0.0,0.0,1.00588,(105.618,110.647)}},draw=blue,line cap=rect,line join=bevel,line width=0.800pt]
  \end{scope}
  \begin{scope}[cm={{1.00588,0.0,0.0,1.00588,(-118.06953,77.03586)}},draw=blue,line cap=rect,line join=bevel,line width=0.800pt]
    \path[fill=blue] (0.0000,0.0000) node[above right] (text188) {\scriptsize 2};



  \end{scope}
  \begin{scope}[cm={{1.00588,0.0,0.0,1.00588,(105.618,110.647)}},draw=blue,line cap=rect,line join=bevel,line width=0.800pt]
  \end{scope}
  \begin{scope}[scale=1.006,draw=blue,line cap=rect,line join=bevel,line width=0.800pt]
  \end{scope}
  \begin{scope}[scale=1.006,draw=blue,line cap=rect,line join=bevel,line width=0.800pt]
  \end{scope}
  \begin{scope}[cm={{1.00588,0.0,0.0,1.00588,(132.274,110.647)}},draw=blue,line cap=rect,line join=bevel,line width=0.800pt]
  \end{scope}
  \begin{scope}[cm={{1.00588,0.0,0.0,1.00588,(132.274,110.647)}},draw=blue,line cap=rect,line join=bevel,line width=0.800pt]
  \end{scope}
  \begin{scope}[cm={{1.00588,0.0,0.0,1.00588,(132.274,110.647)}},draw=blue,line cap=rect,line join=bevel,line width=0.800pt]
  \end{scope}
  \begin{scope}[cm={{1.00588,0.0,0.0,1.00588,(132.274,110.647)}},draw=blue,line cap=rect,line join=bevel,line width=0.800pt]
  \end{scope}
  \begin{scope}[cm={{1.00588,0.0,0.0,1.00588,(132.274,110.647)}},draw=blue,line cap=rect,line join=bevel,line width=0.800pt]
  \end{scope}
  \begin{scope}[cm={{1.00588,0.0,0.0,1.00588,(-91.41353,-12.96416)}},draw=blue,line cap=rect,line join=bevel,line width=0.800pt]
    \path[fill=blue] (0.0000,89.4739) node[above right] (text218) {\scriptsize 3};



  \end{scope}
  \begin{scope}[cm={{1.00588,0.0,0.0,1.00588,(132.274,110.647)}},draw=blue,line cap=rect,line join=bevel,line width=0.800pt]
  \end{scope}
  \begin{scope}[scale=1.006,draw=blue,line cap=rect,line join=bevel,line width=0.800pt]
  \end{scope}
  \begin{scope}[scale=1.006,draw=blue,line cap=rect,line join=bevel,line width=0.800pt]
  \end{scope}
  \begin{scope}[scale=1.006,draw=blue,line cap=rect,line join=bevel,line width=0.800pt]
  \end{scope}
  \begin{scope}[scale=1.006,draw=blue,line cap=rect,line join=bevel,line width=0.800pt]
  \end{scope}
  \begin{scope}[scale=1.006,draw=blue,line cap=rect,line join=bevel,line width=0.800pt]
  \end{scope}
  \begin{scope}[scale=1.006,draw=blue,line cap=rect,line join=bevel,line width=0.800pt]
  \end{scope}
  \begin{scope}[cm={{1.00588,0.0,0.0,1.00588,(128.753,29.1706)}},draw=blue,line cap=rect,line join=bevel,line width=0.800pt]
  \end{scope}
  \begin{scope}[cm={{1.00588,0.0,0.0,1.00588,(128.753,29.1706)}},draw=blue,line cap=rect,line join=bevel,line width=0.800pt]
  \end{scope}
  \begin{scope}[cm={{1.00588,0.0,0.0,1.00588,(128.753,29.1706)}},draw=blue,line cap=rect,line join=bevel,line width=0.800pt]
  \end{scope}
  \begin{scope}[cm={{1.00588,0.0,0.0,1.00588,(128.753,29.1706)}},draw=blue,line cap=rect,line join=bevel,line width=0.800pt]
  \end{scope}
  \begin{scope}[cm={{1.00588,0.0,0.0,1.00588,(128.753,29.1706)}},draw=blue,line cap=rect,line join=bevel,line width=0.800pt]
  \end{scope}
  \begin{scope}[cm={{1.00588,0.0,0.0,1.00588,(128.753,29.1706)}},draw=blue,line cap=rect,line join=bevel,line width=0.800pt]
  \end{scope}
  \begin{scope}[cm={{0.0,-1.00588,1.00588,0.0,(29.1706,189.106)}},draw=blue,line cap=rect,line join=bevel,line width=0.800pt]
  \end{scope}
  \begin{scope}[cm={{0.0,-1.00588,1.00588,0.0,(29.1706,189.106)}},draw=blue,line cap=rect,line join=bevel,line width=0.800pt]
  \end{scope}
  \begin{scope}[cm={{0.0,-1.00588,1.00588,0.0,(29.1706,189.106)}},draw=blue,line cap=rect,line join=bevel,line width=0.800pt]
  \end{scope}
  \begin{scope}[cm={{0.0,-1.00588,1.00588,0.0,(29.1706,189.106)}},draw=blue,line cap=rect,line join=bevel,line width=0.800pt]
  \end{scope}
  \begin{scope}[cm={{0.0,-1.00588,1.00588,0.0,(29.1706,189.106)}},draw=blue,line cap=rect,line join=bevel,line width=0.800pt]
  \end{scope}
  \begin{scope}[cm={{0.0,-1.00588,1.00588,0.0,(29.1706,189.106)}},draw=blue,line cap=rect,line join=bevel,line width=0.800pt]
  \end{scope}
  \begin{scope}[cm={{1.00588,0.0,0.0,1.00588,(62.3647,28.1647)}},draw=blue,line cap=rect,line join=bevel,line width=0.800pt]
  \end{scope}
  \begin{scope}[cm={{1.00588,0.0,0.0,1.00588,(62.3647,28.1647)}},draw=blue,line cap=rect,line join=bevel,line width=0.800pt]
  \end{scope}
  \begin{scope}[cm={{1.00588,0.0,0.0,1.00588,(62.3647,28.1647)}},draw=blue,line cap=rect,line join=bevel,line width=0.800pt]
  \end{scope}
  \begin{scope}[cm={{1.00588,0.0,0.0,1.00588,(62.3647,28.1647)}},draw=blue,line cap=rect,line join=bevel,line width=0.800pt]
  \end{scope}
  \begin{scope}[cm={{1.00588,0.0,0.0,1.00588,(62.3647,28.1647)}},draw=blue,line cap=rect,line join=bevel,line width=0.800pt]
  \end{scope}
  \begin{scope}[cm={{1.00588,0.0,0.0,1.00588,(62.3647,28.1647)}},draw=blue,line cap=rect,line join=bevel,line width=0.800pt]
  \end{scope}
  \begin{scope}[scale=1.006,draw=blue,line cap=rect,line join=bevel,line width=0.800pt]
  \end{scope}
  \begin{scope}[scale=1.006,draw=blue,line cap=rect,line join=bevel,line width=0.800pt]
  \end{scope}
  \begin{scope}[scale=1.006,draw=blue,line cap=rect,line join=bevel,line width=0.800pt]
  \end{scope}
  \begin{scope}[scale=1.006,draw=blue,line cap=rect,line join=bevel,line width=0.800pt]
  \end{scope}
  \begin{scope}[scale=1.006,draw=blue,line cap=rect,line join=bevel,line width=0.800pt]
  \end{scope}
  \begin{scope}[scale=1.006,draw=blue,line cap=rect,line join=bevel,line width=0.800pt]
  \end{scope}
  \begin{scope}[cm={{1.00588,0.0,0.0,1.00588,(60.3529,36.2118)}},draw=blue,line cap=rect,line join=bevel,line width=0.800pt]
  \end{scope}
  \begin{scope}[cm={{1.00588,0.0,0.0,1.00588,(60.3529,36.2118)}},draw=blue,line cap=rect,line join=bevel,line width=0.800pt]
  \end{scope}
  \begin{scope}[cm={{1.00588,0.0,0.0,1.00588,(60.3529,36.2118)}},draw=blue,line cap=rect,line join=bevel,line width=0.800pt]
  \end{scope}
  \begin{scope}[cm={{1.00588,0.0,0.0,1.00588,(60.3529,36.2118)}},draw=blue,line cap=rect,line join=bevel,line width=0.800pt]
  \end{scope}
  \begin{scope}[cm={{1.00588,0.0,0.0,1.00588,(60.3529,36.2118)}},draw=blue,line cap=rect,line join=bevel,line width=0.800pt]
  \end{scope}
  \begin{scope}[cm={{1.00588,0.0,0.0,1.00588,(60.3529,36.2118)}},draw=blue,line cap=rect,line join=bevel,line width=0.800pt]
  \end{scope}
  \begin{scope}[scale=1.006,draw=blue,line cap=rect,line join=bevel,line width=0.800pt]
  \end{scope}
  \begin{scope}[scale=1.006,draw=blue,line cap=rect,line join=bevel,line width=0.800pt]
  \end{scope}
  \begin{scope}[scale=1.006,draw=blue,line cap=rect,line join=bevel,line width=0.800pt]
  \end{scope}
  \begin{scope}[scale=1.006,draw=blue,line cap=rect,line join=bevel,line width=0.800pt]
  \end{scope}
  \begin{scope}[scale=1.006,draw=blue,line cap=rect,line join=bevel,line width=0.800pt]
  \end{scope}
  \begin{scope}[scale=1.006,draw=blue,line cap=rect,line join=bevel,line width=0.800pt]
  \end{scope}
  \begin{scope}[scale=1.006,draw=blue,line cap=rect,line join=bevel,line width=0.800pt]
  \end{scope}
  \begin{scope}[scale=1.006,draw=blue,line cap=round,line join=round,line width=0.480pt]
    \path[cm={{1.25275,0.0,0.0,1.25275,(-479.34266,-167.80381)}},draw] (56.5000,13.5000) -- (56.5000,95.5000) -- (142.5000,95.5000) -- (142.5000,13.5000) -- (56.5000,13.5000);



    \begin{scope}[cm={{0.99623,0.0,0.0,1.3704,(-430.13329,-152.91751)}},draw=ca0a0a4,dash pattern=on 1.02pt off 1.02pt,line cap=round,line join=round,line width=0.255pt,miter limit=4.00]
      \path[shift={(110.39113,-5.49717)},draw,dash pattern=on 1.02pt off 1.02pt,line width=0.255pt,miter limit=4.00] (70.5000,164.5000) -- (70.5000,88.5000);



    \end{scope}
    \begin{scope}[cm={{0.99623,0.0,0.0,1.3704,(-320.15845,-159.02501)}},draw=ca0a0a4,dash pattern=on 1.02pt off 1.02pt,line cap=round,line join=round,line width=0.255pt,miter limit=4.00]
      \path[draw,dash pattern=on 1.02pt off 1.02pt,line width=0.255pt,miter limit=4.00] (98.5000,164.5000) -- (98.5000,88.5000);



    \end{scope}
    \begin{scope}[cm={{0.74279,0.0,0.0,1.28515,(-186.22138,-161.30028)}},draw=ca0a0a4,dash pattern=on 1.22pt off 1.22pt,line cap=round,line join=round,line width=0.305pt,miter limit=4.00]
      \path[draw,dash pattern=on 1.22pt off 1.22pt,line width=0.305pt,miter limit=4.00] (25.5000,26.5000) -- (108.5000,26.5000);



      \path[draw,dash pattern=on 1.22pt off 1.22pt,line width=0.305pt,miter limit=4.00] (137.5000,26.5000) -- (142.5000,26.5000);



    \end{scope}
    \begin{scope}[cm={{0.74279,0.0,0.0,1.28515,(-186.22138,-161.30028)}},draw=blue,line cap=round,line join=round,line width=0.480pt]
      \path[cm={{1.54975,0.0,0.0,1.0,(-13.85377,0.0)}},draw] (25.5000,26.5000) -- (28.5000,26.5000);



      \path[cm={{1.54975,0.0,0.0,1.0,(-78.53317,0.0)}},draw] (142.5000,26.5000) -- (139.5000,26.5000);



    \end{scope}
    \begin{scope}[cm={{0.95389,0.0,0.0,0.95389,(-180.91222,-124.57711)}},draw=blue,fill=ce10000,line cap=rect,line join=bevel,line width=0.800pt]
      \path[fill=ce10000] (0.0000,0.0000) node[above right] (text36) {\scriptsize 30};



    \end{scope}
    \begin{scope}[cm={{0.74279,0.0,0.0,1.28515,(-186.22138,-161.30028)}},draw=ca0a0a4,dash pattern=on 1.22pt off 1.22pt,line cap=round,line join=round,line width=0.305pt,miter limit=4.00]
      \path[draw,dash pattern=on 1.22pt off 1.22pt,line width=0.305pt,miter limit=4.00] (25.5000,11.5000) -- (142.5000,11.5000);



    \end{scope}
    \begin{scope}[cm={{0.74279,0.0,0.0,1.28515,(-186.22138,-161.30028)}},draw=blue,line cap=round,line join=round,line width=0.480pt]
      \path[cm={{1.54975,0.0,0.0,1.0,(-13.85377,0.0)}},draw] (25.5000,11.5000) -- (28.5000,11.5000);



      \path[cm={{1.54975,0.0,0.0,1.0,(-78.53317,0.0)}},draw] (142.5000,11.5000) -- (139.5000,11.5000);



    \end{scope}
    \begin{scope}[cm={{0.95389,0.0,0.0,0.95389,(-180.91222,-144.27915)}},draw=blue,fill=ce10000,line cap=rect,line join=bevel,line width=0.800pt]
      \path[fill=ce10000] (0.0000,0.0000) node[above right] (text66) {\scriptsize 40};



    \end{scope}
    \begin{scope}[cm={{0.74279,0.0,0.0,1.28515,(-186.22138,-161.30028)}},draw=ca0a0a4,dash pattern=on 0.40pt off 0.80pt,line cap=round,line join=round,line width=0.400pt]
      \path[draw] (25.5000,32.5000) -- (25.5000,8.5000);



    \end{scope}
    \begin{scope}[cm={{0.74279,0.0,0.0,1.28515,(-186.22138,-161.30028)}},draw=blue,line cap=round,line join=round,line width=0.480pt]
      \path[draw] (25.5000,32.5000) -- (25.5000,28.5000);



      \path[draw] (25.5000,8.5000) -- (25.5000,11.5000);



    \end{scope}
    \begin{scope}[cm={{0.74279,0.0,0.0,1.28515,(-186.22138,-161.30028)}},draw=ca0a0a4,dash pattern=on 1.22pt off 1.22pt,line cap=round,line join=round,line width=0.305pt,miter limit=4.00]
      \path[draw,dash pattern=on 1.22pt off 1.22pt,line width=0.305pt,miter limit=4.00] (60.5000,32.5000) -- (60.5000,8.5000);



    \end{scope}
    \begin{scope}[cm={{0.74279,0.0,0.0,1.28515,(-186.22138,-161.30028)}},draw=blue,line cap=round,line join=round,line width=0.480pt]
      \path[cm={{1.0,0.0,0.0,0.70101,(0.0,9.59194)}},draw] (60.5000,32.5000) -- (60.5000,28.5000);



      \path[cm={{1.0,0.0,0.0,0.89573,(0.0,0.855)}},draw] (60.5000,8.5000) -- (60.5000,11.5000);



    \end{scope}
    \begin{scope}[cm={{0.74279,0.0,0.0,1.28515,(-186.22138,-161.30028)}},draw=ca0a0a4,dash pattern=on 1.22pt off 1.22pt,line cap=round,line join=round,line width=0.305pt,miter limit=4.00]
      \path[draw,dash pattern=on 1.22pt off 1.22pt,line width=0.305pt,miter limit=4.00] (95.5000,32.5000) -- (95.5000,8.5000);



    \end{scope}
    \begin{scope}[cm={{0.74279,0.0,0.0,1.28515,(-186.22138,-161.30028)}},draw=blue,line cap=round,line join=round,line width=0.480pt]
      \path[cm={{1.0,0.0,0.0,0.70101,(0.0,9.59194)}},draw] (95.5000,32.5000) -- (95.5000,28.5000);



      \path[cm={{1.0,0.0,0.0,0.89573,(0.0,0.855)}},draw] (95.5000,8.5000) -- (95.5000,11.5000);



    \end{scope}
    \begin{scope}[cm={{0.74279,0.0,0.0,1.28515,(-186.22138,-161.30028)}},draw=ca0a0a4,dash pattern=on 1.22pt off 1.22pt,line cap=round,line join=round,line width=0.305pt,miter limit=4.00]
      \path[draw,dash pattern=on 1.22pt off 1.22pt,line width=0.305pt,miter limit=4.00] (130.5000,20.5000) -- (130.5000,8.5000);



    \end{scope}
    \begin{scope}[cm={{0.74279,0.0,0.0,1.28515,(-186.22138,-161.30028)}},draw=blue,line cap=round,line join=round,line width=0.480pt]
      \path[cm={{1.0,0.0,0.0,0.70101,(0.0,9.59194)}},draw] (130.5000,32.5000) -- (130.5000,28.5000);



      \path[cm={{1.0,0.0,0.0,0.89573,(0.0,0.855)}},draw] (130.5000,8.5000) -- (130.5000,11.5000);



    \end{scope}
    \begin{scope}[cm={{0.74279,0.0,0.0,1.28515,(-186.22138,-161.30028)}},draw=blue,line cap=round,line join=round,line width=0.480pt]
      \path[draw] (25.5000,8.5000) -- (25.5000,32.5000) -- (142.5000,32.5000) -- (142.5000,8.5000) -- (25.5000,8.5000);



    \end{scope}
    \begin{scope}[cm={{0.95389,0.0,0.0,0.95389,(-104.99275,-127.89091)}},draw=blue,line cap=rect,line join=bevel,line width=0.800pt]
      \path[fill=blue] (0.0000,0.0000) node[above right] (text194) {\scriptsize $\alpha_0$};



    \end{scope}
    \begin{scope}[cm={{0.74279,0.0,0.0,1.28515,(-184.07401,-166.61499)}},draw=blue,line cap=round,line join=round,line width=0.480pt]
      \path[draw,even odd rule] (123.5000,28.5000) -- (132.5000,28.5000);



    \end{scope}
    \begin{scope}[cm={{0.74279,0.0,0.0,1.28515,(-186.22138,-161.30028)}},draw=blue,line cap=round,line join=round,line width=0.480pt]
      \path[draw] (25.8000,32.0000) -- (26.2000,14.3000) -- (26.7000,17.2000) -- (27.1000,18.1000) -- (27.5000,14.6000) -- (27.9000,12.6000) -- (28.3000,14.9000) -- (28.8000,17.3000) -- (29.2000,10.3000) -- (29.6000,14.5000) -- (30.0000,19.1000) -- (30.5000,15.7000) -- (30.9000,12.7000) -- (31.3000,15.0000) -- (31.7000,18.7000) -- (32.2000,19.3000) -- (32.6000,16.6000) -- (33.0000,13.7000) -- (33.4000,12.8000) -- (33.8000,14.3000) -- (34.3000,16.9000) -- (34.7000,18.9000) -- (35.1000,19.4000) -- (35.5000,18.6000) -- (36.0000,17.2000) -- (36.4000,15.9000) -- (36.8000,15.5000) -- (37.2000,15.9000) -- (37.6000,17.1000) -- (38.1000,18.6000) -- (38.5000,20.0000) -- (38.9000,21.0000) -- (39.3000,21.3000) -- (39.8000,21.0000) -- (40.2000,20.3000) -- (40.6000,19.3000) -- (41.0000,18.3000) -- (41.5000,17.4000) -- (41.9000,16.9000) -- (42.3000,16.8000) -- (42.7000,17.1000) -- (43.1000,17.6000) -- (43.6000,18.3000) -- (44.0000,19.0000) -- (44.4000,19.7000) -- (44.8000,20.1000) -- (45.3000,20.4000) -- (45.7000,20.5000) -- (46.1000,20.4000) -- (46.5000,20.1000) -- (47.0000,19.8000) -- (47.4000,19.3000) -- (47.8000,18.5000) -- (48.2000,17.8000) -- (48.6000,17.2000) -- (49.1000,16.7000) -- (49.5000,16.5000) -- (49.9000,16.4000) -- (50.3000,16.4000) -- (50.8000,16.5000) -- (51.2000,16.7000) -- (51.6000,17.0000) -- (52.0000,17.2000) -- (52.4000,17.4000) -- (52.9000,17.5000) -- (53.3000,17.5000) -- (53.7000,17.3000) -- (54.1000,17.0000) -- (54.6000,16.6000) -- (55.0000,16.1000) -- (55.4000,15.6000) -- (55.8000,15.0000) -- (56.3000,14.4000) -- (56.7000,13.9000) -- (57.1000,13.5000) -- (57.5000,13.2000) -- (57.9000,13.2000) -- (58.4000,13.3000) -- (58.8000,13.5000) -- (59.2000,13.9000) -- (59.6000,14.3000) -- (60.1000,14.7000) -- (60.5000,15.0000) -- (60.9000,15.4000) -- (61.3000,15.7000) -- (61.8000,16.0000) -- (62.2000,16.2000) -- (62.6000,16.3000) -- (63.0000,16.2000) -- (63.4000,16.2000) -- (63.9000,16.1000) -- (64.3000,16.0000) -- (64.7000,15.9000) -- (65.1000,15.9000) -- (65.6000,15.9000) -- (66.0000,16.0000) -- (66.4000,16.2000) -- (66.8000,16.4000) -- (67.3000,16.7000) -- (67.7000,16.9000) -- (68.1000,17.2000) -- (68.5000,17.3000) -- (68.9000,17.4000) -- (69.4000,17.5000) -- (69.8000,17.7000) -- (70.2000,17.7000) -- (70.6000,17.7000) -- (71.1000,17.6000) -- (71.5000,17.5000) -- (71.9000,17.3000) -- (72.3000,17.1000) -- (72.7000,17.0000) -- (73.2000,16.9000) -- (73.6000,16.8000) -- (74.0000,16.9000) -- (74.4000,17.0000) -- (74.9000,16.9000) -- (75.3000,16.8000) -- (75.7000,16.8000) -- (76.1000,16.8000) -- (76.6000,16.9000) -- (77.0000,17.1000) -- (77.4000,17.2000) -- (77.8000,17.3000) -- (78.2000,17.4000) -- (78.7000,17.4000) -- (79.1000,17.4000) -- (79.5000,17.4000) -- (79.9000,17.4000) -- (80.4000,17.3000) -- (80.8000,17.2000) -- (81.2000,17.1000) -- (81.6000,17.1000) -- (82.1000,17.0000) -- (82.5000,17.0000) -- (82.9000,16.9000) -- (83.3000,16.7000) -- (83.7000,16.6000) -- (84.2000,16.4000) -- (84.6000,16.4000) -- (85.0000,16.3000) -- (85.4000,16.3000) -- (85.9000,16.3000) -- (86.3000,16.3000) -- (86.7000,16.2000) -- (87.1000,16.2000) -- (87.6000,16.1000) -- (88.0000,16.1000) -- (88.4000,16.2000) -- (88.8000,16.3000) -- (89.2000,16.3000) -- (89.7000,16.3000) -- (90.1000,16.3000) -- (90.5000,16.3000) -- (90.9000,16.3000) -- (91.4000,16.3000) -- (91.8000,16.3000) -- (92.2000,16.3000) -- (92.6000,16.3000) -- (93.0000,16.2000) -- (93.5000,16.2000) -- (93.9000,16.2000) -- (94.3000,16.2000) -- (94.7000,16.2000) -- (95.2000,16.1000) -- (95.6000,16.1000) -- (96.0000,16.1000) -- (96.4000,16.2000) -- (96.9000,16.4000) -- (97.3000,16.5000) -- (97.7000,16.5000) -- (98.1000,16.6000) -- (98.5000,16.6000) -- (99.0000,16.6000) -- (99.4000,16.6000) -- (99.8000,16.5000) -- (100.2000,16.5000) -- (100.7000,16.6000) -- (101.1000,16.7000) -- (101.5000,16.6000) -- (101.9000,16.5000) -- (102.3000,16.3000) -- (102.8000,16.3000) -- (103.2000,16.3000) -- (103.6000,16.3000) -- (104.0000,16.3000) -- (104.5000,16.3000) -- (104.9000,16.3000) -- (105.3000,16.3000) -- (105.7000,16.2000) -- (106.2000,16.2000) -- (106.6000,16.2000) -- (107.0000,16.2000) -- (107.4000,16.2000) -- (107.8000,16.2000) -- (108.3000,16.3000) -- (108.7000,16.3000) -- (109.1000,16.4000) -- (109.5000,16.5000) -- (110.0000,16.5000) -- (110.4000,16.5000) -- (110.8000,16.5000) -- (111.2000,16.5000) -- (111.7000,16.5000) -- (112.1000,16.6000) -- (112.5000,16.6000) -- (112.9000,16.6000) -- (113.3000,16.5000) -- (113.8000,16.5000) -- (114.2000,16.4000) -- (114.6000,16.3000) -- (115.0000,16.3000) -- (115.5000,16.4000) -- (115.9000,16.5000) -- (116.3000,16.5000) -- (116.7000,16.5000) -- (117.2000,16.5000) -- (117.6000,16.5000) -- (118.0000,16.5000) -- (118.4000,16.5000) -- (118.8000,16.6000) -- (119.3000,16.6000) -- (119.7000,16.6000) -- (120.1000,16.5000) -- (120.5000,16.5000) -- (121.0000,16.5000) -- (121.4000,16.5000) -- (121.8000,16.5000) -- (122.2000,16.3000) -- (122.6000,16.3000) -- (123.1000,16.3000) -- (123.5000,16.4000) -- (123.9000,16.4000) -- (124.3000,16.4000) -- (124.8000,16.4000) -- (125.2000,16.4000) -- (125.6000,16.3000) -- (126.0000,16.3000) -- (126.5000,16.2000) -- (126.9000,16.2000) -- (127.3000,16.3000) -- (127.7000,16.5000) -- (128.1000,16.5000) -- (128.6000,16.4000) -- (129.0000,16.3000) -- (129.4000,16.3000) -- (129.8000,16.4000) -- (130.3000,16.5000) -- (130.7000,16.5000) -- (131.1000,16.6000) -- (131.5000,16.6000) -- (131.9000,16.6000) -- (132.4000,16.6000) -- (132.8000,16.5000) -- (133.2000,16.5000) -- (133.6000,16.4000) -- (134.1000,16.4000) -- (134.5000,16.3000) -- (134.9000,16.3000) -- (135.3000,16.3000) -- (135.8000,16.4000) -- (136.2000,16.4000) -- (136.6000,16.4000) -- (137.0000,16.4000) -- (137.4000,16.4000) -- (137.9000,16.4000) -- (138.3000,16.4000) -- (138.7000,16.5000) -- (139.1000,16.5000) -- (139.6000,16.5000) -- (140.0000,16.5000) -- (140.4000,16.5000) -- (140.8000,16.4000) -- (141.3000,16.4000) -- (141.7000,16.3000) -- (142.1000,16.4000) -- (142.3000,16.4000);



    \end{scope}
    \begin{scope}[cm={{0.74279,0.0,0.0,1.28515,(-186.22138,-161.30028)}},draw=blue,line cap=round,line join=round,line width=0.480pt]
      \path[draw] (25.5000,8.5000) -- (25.5000,32.5000) -- (142.5000,32.5000) -- (142.5000,8.5000) -- (25.5000,8.5000);



    \end{scope}
    \begin{scope}[cm={{0.74279,0.0,0.0,1.28515,(-186.22138,-161.30028)}},draw=ca0a0a4,dash pattern=on 1.22pt off 1.22pt,line cap=round,line join=round,line width=0.305pt,miter limit=4.00]
      \path[draw,dash pattern=on 1.22pt off 1.22pt,line width=0.305pt,miter limit=4.00] (25.5000,46.5000) -- (108.5000,46.5000);



      \path[draw,dash pattern=on 1.22pt off 1.22pt,line width=0.305pt,miter limit=4.00] (137.5000,46.5000) -- (142.5000,46.5000);



    \end{scope}
    \begin{scope}[cm={{0.74279,0.0,0.0,1.28515,(-186.22138,-161.30028)}},draw=blue,line cap=round,line join=round,line width=0.480pt]
      \path[cm={{1.54975,0.0,0.0,1.0,(-13.85377,0.0)}},draw] (25.5000,46.5000) -- (28.5000,46.5000);



      \path[cm={{1.54975,0.0,0.0,1.0,(-78.53317,0.0)}},draw] (142.5000,46.5000) -- (139.5000,46.5000);



    \end{scope}
    \begin{scope}[cm={{0.95389,0.0,0.0,0.95389,(-183.45337,-99.34397)}},draw=blue,fill=ce10000,line cap=rect,line join=bevel,line width=0.800pt]
      \path[fill=ce10000] (0.0000,0.0000) node[above right] (text250) {\scriptsize -10};



    \end{scope}
    \begin{scope}[cm={{0.74279,0.0,0.0,1.28515,(-186.22138,-161.30028)}},draw=ca0a0a4,dash pattern=on 1.22pt off 1.22pt,line cap=round,line join=round,line width=0.305pt,miter limit=4.00]
      \path[draw,dash pattern=on 1.22pt off 1.22pt,line width=0.305pt,miter limit=4.00] (25.5000,34.5000) -- (142.5000,34.5000);



    \end{scope}
    \begin{scope}[cm={{0.74279,0.0,0.0,1.28515,(-186.22138,-161.30028)}},draw=blue,line cap=round,line join=round,line width=0.480pt]
      \path[cm={{1.54975,0.0,0.0,1.0,(-13.85377,0.0)}},draw] (25.5000,34.5000) -- (28.5000,34.5000);



      \path[cm={{1.54975,0.0,0.0,1.0,(-78.53317,0.0)}},draw] (142.5000,34.5000) -- (139.5000,34.5000);



    \end{scope}
    \begin{scope}[cm={{0.95389,0.0,0.0,0.95389,(-180.91222,-114.12005)}},draw=blue,fill=ce10000,line cap=rect,line join=bevel,line width=0.800pt]
      \path[fill=ce10000] (0.0000,0.0000) node[above right] (text280) {\scriptsize 10};



    \end{scope}
    \begin{scope}[cm={{0.74279,0.0,0.0,1.28515,(-186.22138,-161.30028)}},draw=ca0a0a4,dash pattern=on 0.40pt off 0.80pt,line cap=round,line join=round,line width=0.400pt]
      \path[draw] (25.5000,56.5000) -- (25.5000,32.5000);



    \end{scope}
    \begin{scope}[cm={{0.74279,0.0,0.0,1.28515,(-186.22138,-161.30028)}},draw=blue,line cap=round,line join=round,line width=0.480pt]
      \path[draw] (25.5000,56.5000) -- (25.5000,51.5000);



      \path[draw] (25.5000,32.5000) -- (25.5000,36.5000);



    \end{scope}
    \begin{scope}[cm={{0.74279,0.0,0.0,1.28515,(-186.22138,-161.30028)}},draw=ca0a0a4,dash pattern=on 1.22pt off 1.22pt,line cap=round,line join=round,line width=0.305pt,miter limit=4.00]
      \path[draw,dash pattern=on 1.22pt off 1.22pt,line width=0.305pt,miter limit=4.00] (60.5000,56.5000) -- (60.5000,32.5000);



    \end{scope}
    \begin{scope}[cm={{0.74279,0.0,0.0,1.28515,(-186.22138,-161.30028)}},draw=blue,line cap=round,line join=round,line width=0.480pt]
      \path[cm={{1.0,0.0,0.0,0.57583,(0.0,24.03834)}},draw] (60.5000,56.5000) -- (60.5000,51.5000);



      \path[cm={{1.0,0.0,0.0,0.70101,(0.0,9.62756)}},draw] (60.5000,32.5000) -- (60.5000,36.5000);



    \end{scope}
    \begin{scope}[cm={{0.74279,0.0,0.0,1.28515,(-186.22138,-161.30028)}},draw=ca0a0a4,dash pattern=on 1.22pt off 1.22pt,line cap=round,line join=round,line width=0.305pt,miter limit=4.00]
      \path[draw,dash pattern=on 1.22pt off 1.22pt,line width=0.305pt,miter limit=4.00] (95.5000,56.5000) -- (95.5000,32.5000);



    \end{scope}
    \begin{scope}[cm={{0.74279,0.0,0.0,1.28515,(-186.22138,-161.30028)}},draw=blue,line cap=round,line join=round,line width=0.480pt]
      \path[cm={{1.0,0.0,0.0,0.57583,(0.0,24.03834)}},draw] (95.5000,56.5000) -- (95.5000,51.5000);



      \path[cm={{1.0,0.0,0.0,0.70101,(0.0,9.62756)}},draw] (95.5000,32.5000) -- (95.5000,36.5000);



    \end{scope}
    \begin{scope}[cm={{0.74279,0.0,0.0,1.28515,(-186.22138,-161.30028)}},draw=ca0a0a4,dash pattern=on 1.22pt off 1.22pt,line cap=round,line join=round,line width=0.305pt,miter limit=4.00]
      \path[draw,dash pattern=on 1.22pt off 1.22pt,line width=0.305pt,miter limit=4.00] (130.5000,44.5000) -- (130.5000,32.5000);



    \end{scope}
    \begin{scope}[cm={{0.74279,0.0,0.0,1.28515,(-186.22138,-161.30028)}},draw=blue,line cap=round,line join=round,line width=0.480pt]
      \path[cm={{1.0,0.0,0.0,0.57583,(0.0,24.03834)}},draw] (130.5000,56.5000) -- (130.5000,51.5000);



      \path[cm={{1.0,0.0,0.0,0.70101,(0.0,9.62756)}},draw] (130.5000,32.5000) -- (130.5000,36.5000);



    \end{scope}
    \begin{scope}[cm={{0.74279,0.0,0.0,1.28515,(-186.22138,-161.30028)}},draw=blue,line cap=round,line join=round,line width=0.480pt]
      \path[draw] (25.5000,32.5000) -- (25.5000,56.5000) -- (142.5000,56.5000) -- (142.5000,32.5000) -- (25.5000,32.5000);



    \end{scope}
    \begin{scope}[cm={{0.95389,0.0,0.0,0.95389,(-104.99275,-97.16859)}},draw=blue,line cap=rect,line join=bevel,line width=0.800pt]
      \path[fill=blue] (0.0000,0.0000) node[above right] (text408) {\scriptsize $\alpha_1$};



    \end{scope}
    \begin{scope}[cm={{0.74279,0.0,0.0,1.28515,(-184.07401,-166.61499)}},draw=blue,line cap=round,line join=round,line width=0.480pt]
      \path[draw,even odd rule] (123.5000,52.5000) -- (132.5000,52.5000);



    \end{scope}
    \begin{scope}[cm={{0.74279,0.0,0.0,1.28515,(-186.22138,-161.30028)}},draw=blue,line cap=round,line join=round,line width=0.480pt]
      \path[draw] (25.8000,36.4000) -- (25.8000,36.4000) -- (26.2000,41.9000) -- (26.7000,39.4000) -- (27.1000,40.5000) -- (27.5000,40.1000) -- (27.9000,39.9000) -- (28.3000,40.6000) -- (28.8000,43.1000) -- (29.2000,39.4000) -- (29.6000,39.6000) -- (30.0000,41.8000) -- (30.5000,40.8000) -- (30.9000,39.0000) -- (31.3000,39.5000) -- (31.7000,41.2000) -- (32.2000,42.0000) -- (32.6000,41.2000) -- (33.0000,39.7000) -- (33.4000,38.9000) -- (33.8000,39.1000) -- (34.3000,40.2000) -- (34.7000,41.3000) -- (35.1000,41.9000) -- (35.5000,41.9000) -- (36.0000,41.5000) -- (36.4000,40.9000) -- (36.8000,40.4000) -- (37.2000,40.4000) -- (37.6000,40.7000) -- (38.1000,41.2000) -- (38.5000,41.8000) -- (38.9000,42.2000) -- (39.3000,42.3000) -- (39.8000,42.2000) -- (40.2000,41.8000) -- (40.6000,41.3000) -- (41.0000,40.6000) -- (41.5000,39.9000) -- (41.9000,39.3000) -- (42.3000,38.9000) -- (42.7000,38.7000) -- (43.1000,38.7000) -- (43.6000,38.8000) -- (44.0000,39.0000) -- (44.4000,39.2000) -- (44.8000,39.4000) -- (45.3000,39.6000) -- (45.7000,39.8000) -- (46.1000,39.8000) -- (46.5000,39.8000) -- (47.0000,39.8000) -- (47.4000,39.7000) -- (47.8000,39.5000) -- (48.2000,39.3000) -- (48.6000,39.1000) -- (49.1000,39.0000) -- (49.5000,38.9000) -- (49.9000,38.9000) -- (50.3000,39.0000) -- (50.8000,39.2000) -- (51.2000,39.4000) -- (51.6000,39.7000) -- (52.0000,39.9000) -- (52.4000,40.2000) -- (52.9000,40.5000) -- (53.3000,40.7000) -- (53.7000,40.8000) -- (54.1000,40.9000) -- (54.6000,40.9000) -- (55.0000,40.9000) -- (55.4000,40.8000) -- (55.8000,40.6000) -- (56.3000,40.5000) -- (56.7000,40.3000) -- (57.1000,40.1000) -- (57.5000,40.0000) -- (57.9000,40.0000) -- (58.4000,40.0000) -- (58.8000,40.1000) -- (59.2000,40.3000) -- (59.6000,40.5000) -- (60.1000,40.7000) -- (60.5000,40.9000) -- (60.9000,41.1000) -- (61.3000,41.4000) -- (61.8000,41.6000) -- (62.2000,41.7000) -- (62.6000,41.8000) -- (63.0000,41.9000) -- (63.4000,41.9000) -- (63.9000,41.8000) -- (64.3000,41.8000) -- (64.7000,41.7000) -- (65.1000,41.6000) -- (65.6000,41.6000) -- (66.0000,41.5000) -- (66.4000,41.4000) -- (66.8000,41.4000) -- (67.3000,41.4000) -- (67.7000,41.3000) -- (68.1000,41.3000) -- (68.5000,41.2000) -- (68.9000,41.1000) -- (69.4000,41.0000) -- (69.8000,40.9000) -- (70.2000,40.8000) -- (70.6000,40.7000) -- (71.1000,40.5000) -- (71.5000,40.3000) -- (71.9000,40.1000) -- (72.3000,39.9000) -- (72.7000,39.7000) -- (73.2000,39.5000) -- (73.6000,39.4000) -- (74.0000,39.3000) -- (74.4000,39.2000) -- (74.9000,39.0000) -- (75.3000,38.9000) -- (75.7000,38.9000) -- (76.1000,38.8000) -- (76.6000,38.8000) -- (77.0000,38.9000) -- (77.4000,38.9000) -- (77.8000,39.0000) -- (78.2000,39.0000) -- (78.7000,39.1000) -- (79.1000,39.2000) -- (79.5000,39.3000) -- (79.9000,39.3000) -- (80.4000,39.4000) -- (80.8000,39.5000) -- (81.2000,39.6000) -- (81.6000,39.7000) -- (82.1000,39.8000) -- (82.5000,39.9000) -- (82.9000,40.0000) -- (83.3000,40.1000) -- (83.7000,40.1000) -- (84.2000,40.2000) -- (84.6000,40.3000) -- (85.0000,40.4000) -- (85.4000,40.5000) -- (85.9000,40.7000) -- (86.3000,40.8000) -- (86.7000,40.9000) -- (87.1000,41.0000) -- (87.6000,41.1000) -- (88.0000,41.1000) -- (88.4000,41.3000) -- (88.8000,41.4000) -- (89.2000,41.5000) -- (89.7000,41.6000) -- (90.1000,41.6000) -- (90.5000,41.7000) -- (90.9000,41.7000) -- (91.4000,41.7000) -- (91.8000,41.7000) -- (92.2000,41.7000) -- (92.6000,41.7000) -- (93.0000,41.6000) -- (93.5000,41.6000) -- (93.9000,41.5000) -- (94.3000,41.5000) -- (94.7000,41.4000) -- (95.2000,41.3000) -- (95.6000,41.2000) -- (96.0000,41.1000) -- (96.4000,41.0000) -- (96.9000,41.0000) -- (97.3000,40.9000) -- (97.7000,40.8000) -- (98.1000,40.7000) -- (98.5000,40.6000) -- (99.0000,40.5000) -- (99.4000,40.4000) -- (99.8000,40.3000) -- (100.2000,40.1000) -- (100.7000,40.1000) -- (101.1000,40.0000) -- (101.5000,39.9000) -- (101.9000,39.7000) -- (102.3000,39.6000) -- (102.8000,39.5000) -- (103.2000,39.4000) -- (103.6000,39.3000) -- (104.0000,39.2000) -- (104.5000,39.2000) -- (104.9000,39.1000) -- (105.3000,39.1000) -- (105.7000,39.1000) -- (106.2000,39.1000) -- (106.6000,39.1000) -- (107.0000,39.1000) -- (107.4000,39.2000) -- (107.8000,39.2000) -- (108.3000,39.3000) -- (108.7000,39.4000) -- (109.1000,39.5000) -- (109.5000,39.7000) -- (110.0000,39.8000) -- (110.4000,39.9000) -- (110.8000,40.0000) -- (111.2000,40.1000) -- (111.7000,40.3000) -- (112.1000,40.4000) -- (112.5000,40.6000) -- (112.9000,40.7000) -- (113.3000,40.8000) -- (113.8000,40.9000) -- (114.2000,41.0000) -- (114.6000,41.1000) -- (115.0000,41.2000) -- (115.5000,41.3000) -- (115.9000,41.4000) -- (116.3000,41.5000) -- (116.7000,41.6000) -- (117.2000,41.6000) -- (117.6000,41.7000) -- (118.0000,41.7000) -- (118.4000,41.8000) -- (118.8000,41.8000) -- (119.3000,41.8000) -- (119.7000,41.8000) -- (120.1000,41.7000) -- (120.5000,41.7000) -- (121.0000,41.6000) -- (121.4000,41.6000) -- (121.8000,41.5000) -- (122.2000,41.3000) -- (122.6000,41.2000) -- (123.1000,41.1000) -- (123.5000,41.0000) -- (123.9000,40.9000) -- (124.3000,40.8000) -- (124.8000,40.7000) -- (125.2000,40.5000) -- (125.6000,40.4000) -- (126.0000,40.2000) -- (126.5000,40.1000) -- (126.9000,39.9000) -- (127.3000,39.8000) -- (127.7000,39.8000) -- (128.1000,39.7000) -- (128.6000,39.5000) -- (129.0000,39.4000) -- (129.4000,39.4000) -- (129.8000,39.3000) -- (130.3000,39.3000) -- (130.7000,39.2000) -- (131.1000,39.2000) -- (131.5000,39.2000) -- (131.9000,39.2000) -- (132.4000,39.2000) -- (132.8000,39.2000) -- (133.2000,39.2000) -- (133.6000,39.2000) -- (134.1000,39.3000) -- (134.5000,39.3000) -- (134.9000,39.4000) -- (135.3000,39.5000) -- (135.8000,39.6000) -- (136.2000,39.7000) -- (136.6000,39.8000) -- (137.0000,39.9000) -- (137.4000,40.0000) -- (137.9000,40.1000) -- (138.3000,40.3000) -- (138.7000,40.4000) -- (139.1000,40.6000) -- (139.6000,40.7000) -- (140.0000,40.8000) -- (140.4000,40.9000) -- (140.8000,41.0000) -- (141.3000,41.1000) -- (141.7000,41.2000) -- (142.1000,41.3000) -- (142.3000,41.4000);



    \end{scope}
    \begin{scope}[cm={{0.74279,0.0,0.0,1.28515,(-186.22138,-161.30028)}},draw=blue,line cap=round,line join=round,line width=0.480pt]
      \path[draw] (25.5000,32.5000) -- (25.5000,56.5000) -- (142.5000,56.5000) -- (142.5000,32.5000) -- (25.5000,32.5000);



    \end{scope}
    \begin{scope}[cm={{0.74279,0.0,0.0,1.28515,(-186.22138,-161.30028)}},draw=ca0a0a4,dash pattern=on 1.22pt off 1.22pt,line cap=round,line join=round,line width=0.305pt,miter limit=4.00]
      \path[draw,dash pattern=on 1.22pt off 1.22pt,line width=0.305pt,miter limit=4.00] (25.5000,70.5000) -- (108.5000,70.5000);



      \path[draw,dash pattern=on 1.22pt off 1.22pt,line width=0.305pt,miter limit=4.00] (137.5000,70.5000) -- (142.5000,70.5000);



    \end{scope}
    \begin{scope}[cm={{0.74279,0.0,0.0,1.28515,(-186.22138,-161.30028)}},draw=blue,line cap=round,line join=round,line width=0.480pt]
      \path[cm={{1.54975,0.0,0.0,1.0,(-13.85377,0.0)}},draw] (25.5000,70.5000) -- (28.5000,70.5000);



      \path[cm={{1.54975,0.0,0.0,1.0,(-78.53317,0.15196)}},draw] (142.5000,70.5000) -- (139.5000,70.5000);



    \end{scope}
    \begin{scope}[cm={{0.95389,0.0,0.0,0.95389,(-183.45337,-67.69359)}},draw=blue,fill=ce10000,line cap=rect,line join=bevel,line width=0.800pt]
      \path[fill=ce10000] (0.0000,0.0000) node[above right] (text464) {\scriptsize -10};



    \end{scope}
    \begin{scope}[cm={{0.74279,0.0,0.0,1.28515,(-186.22138,-161.30028)}},draw=ca0a0a4,dash pattern=on 1.22pt off 1.22pt,line cap=round,line join=round,line width=0.305pt,miter limit=4.00]
      \path[draw,dash pattern=on 1.22pt off 1.22pt,line width=0.305pt,miter limit=4.00] (25.5000,58.5000) -- (142.5000,58.5000);



    \end{scope}
    \begin{scope}[cm={{0.74279,0.0,0.0,1.28515,(-186.22138,-161.30028)}},draw=blue,line cap=round,line join=round,line width=0.480pt]
      \path[cm={{1.54975,0.0,0.0,1.0,(-13.85377,0.0)}},draw] (25.5000,58.5000) -- (28.5000,58.5000);



      \path[cm={{1.54975,0.0,0.0,1.0,(-78.53317,0.0)}},draw] (142.5000,58.5000) -- (139.5000,58.5000);



    \end{scope}
    \begin{scope}[cm={{0.95389,0.0,0.0,0.95389,(-180.91222,-84.15133)}},draw=blue,fill=ce10000,line cap=rect,line join=bevel,line width=0.800pt]
      \path[fill=ce10000] (0.0000,0.0000) node[above right] (text494) {\scriptsize 10};



    \end{scope}
    \begin{scope}[cm={{0.74279,0.0,0.0,1.28515,(-186.22138,-161.30028)}},draw=ca0a0a4,dash pattern=on 0.40pt off 0.80pt,line cap=round,line join=round,line width=0.400pt]
      \path[draw] (25.5000,80.5000) -- (25.5000,56.5000);



    \end{scope}
    \begin{scope}[cm={{0.74279,0.0,0.0,1.28515,(-186.22138,-161.30028)}},draw=blue,line cap=round,line join=round,line width=0.480pt]
      \path[draw] (25.5000,80.5000) -- (25.5000,75.5000);



      \path[draw] (25.5000,56.5000) -- (25.5000,60.5000);



    \end{scope}
    \begin{scope}[cm={{0.74279,0.0,0.0,1.28515,(-186.22138,-161.30028)}},draw=ca0a0a4,dash pattern=on 1.22pt off 1.22pt,line cap=round,line join=round,line width=0.305pt,miter limit=4.00]
      \path[draw,dash pattern=on 1.22pt off 1.22pt,line width=0.305pt,miter limit=4.00] (60.5000,80.5000) -- (60.5000,56.5000);



    \end{scope}
    \begin{scope}[cm={{0.74279,0.0,0.0,1.28515,(-186.22138,-161.30028)}},draw=blue,line cap=round,line join=round,line width=0.480pt]
      \path[cm={{1.0,0.0,0.0,0.57583,(0.0,34.21848)}},draw] (60.5000,80.5000) -- (60.5000,75.5000);



      \path[cm={{1.0,0.0,0.0,0.70101,(0.0,16.80337)}},draw] (60.5000,56.5000) -- (60.5000,60.5000);



    \end{scope}
    \begin{scope}[cm={{0.74279,0.0,0.0,1.28515,(-186.22138,-161.30028)}},draw=ca0a0a4,dash pattern=on 1.22pt off 1.22pt,line cap=round,line join=round,line width=0.305pt,miter limit=4.00]
      \path[draw,dash pattern=on 1.22pt off 1.22pt,line width=0.305pt,miter limit=4.00] (95.5000,80.5000) -- (95.5000,56.5000);



    \end{scope}
    \begin{scope}[cm={{0.74279,0.0,0.0,1.28515,(-186.22138,-161.30028)}},draw=blue,line cap=round,line join=round,line width=0.480pt]
      \path[cm={{1.0,0.0,0.0,0.57583,(0.0,34.21848)}},draw] (95.5000,80.5000) -- (95.5000,75.5000);



      \path[cm={{1.0,0.0,0.0,0.70101,(0.0,16.80337)}},draw] (95.5000,56.5000) -- (95.5000,60.5000);



    \end{scope}
    \begin{scope}[cm={{0.74279,0.0,0.0,1.28515,(-186.22138,-161.30028)}},draw=ca0a0a4,dash pattern=on 1.22pt off 1.22pt,line cap=round,line join=round,line width=0.305pt,miter limit=4.00]
      \path[draw,dash pattern=on 1.22pt off 1.22pt,line width=0.305pt,miter limit=4.00] (130.5000,68.5000) -- (130.5000,56.5000);



    \end{scope}
    \begin{scope}[cm={{0.74279,0.0,0.0,1.28515,(-186.22138,-161.30028)}},draw=blue,line cap=round,line join=round,line width=0.480pt]
      \path[cm={{1.0,0.0,0.0,0.57583,(0.0,34.21848)}},draw] (130.5000,80.5000) -- (130.5000,75.5000);



      \path[cm={{1.0,0.0,0.0,0.70101,(0.0,16.80337)}},draw] (130.5000,56.5000) -- (130.5000,60.5000);



    \end{scope}
    \begin{scope}[cm={{0.74279,0.0,0.0,1.28515,(-186.22138,-161.30028)}},draw=blue,line cap=round,line join=round,line width=0.480pt]
      \path[draw] (25.5000,56.5000) -- (25.5000,80.5000) -- (142.5000,80.5000) -- (142.5000,56.5000) -- (25.5000,56.5000);



    \end{scope}
    \begin{scope}[cm={{0.95389,0.0,0.0,0.95389,(-104.73329,-65.32779)}},draw=blue,line cap=rect,line join=bevel,line width=0.800pt]
      \path[fill=blue] (0.0000,0.0000) node[above right] (text622) {\scriptsize $\beta_1$};



    \end{scope}
    \begin{scope}[cm={{0.74279,0.0,0.0,1.28515,(-184.07401,-166.61499)}},draw=blue,line cap=round,line join=round,line width=0.480pt]
      \path[draw,even odd rule] (123.5000,76.5000) -- (132.5000,76.5000);



    \end{scope}
    \begin{scope}[cm={{0.74279,0.0,0.0,1.28515,(-186.22138,-161.30028)}},draw=blue,line cap=round,line join=round,line width=0.480pt]
      \path[draw] (25.8000,67.9000) -- (25.8000,67.9000) -- (26.2000,64.4000) -- (26.7000,63.4000) -- (27.1000,65.9000) -- (27.5000,64.8000) -- (27.9000,64.0000) -- (28.3000,65.9000) -- (28.8000,65.0000) -- (29.2000,61.7000) -- (29.6000,64.2000) -- (30.0000,65.6000) -- (30.5000,63.7000) -- (30.9000,62.9000) -- (31.3000,64.5000) -- (31.7000,65.9000) -- (32.2000,65.6000) -- (32.6000,64.1000) -- (33.0000,62.9000) -- (33.4000,63.0000) -- (33.8000,64.1000) -- (34.3000,65.3000) -- (34.7000,66.0000) -- (35.1000,65.8000) -- (35.5000,65.1000) -- (36.0000,64.3000) -- (36.4000,63.6000) -- (36.8000,63.4000) -- (37.2000,63.6000) -- (37.6000,64.0000) -- (38.1000,64.5000) -- (38.5000,64.8000) -- (38.9000,64.8000) -- (39.3000,64.6000) -- (39.8000,64.1000) -- (40.2000,63.5000) -- (40.6000,63.0000) -- (41.0000,62.5000) -- (41.5000,62.2000) -- (41.9000,62.1000) -- (42.3000,62.3000) -- (42.7000,62.7000) -- (43.1000,63.2000) -- (43.6000,63.7000) -- (44.0000,64.1000) -- (44.4000,64.6000) -- (44.8000,64.9000) -- (45.3000,65.1000) -- (45.7000,65.3000) -- (46.1000,65.3000) -- (46.5000,65.3000) -- (47.0000,65.2000) -- (47.4000,65.1000) -- (47.8000,64.9000) -- (48.2000,64.8000) -- (48.6000,64.7000) -- (49.1000,64.8000) -- (49.5000,64.8000) -- (49.9000,65.0000) -- (50.3000,65.2000) -- (50.8000,65.4000) -- (51.2000,65.6000) -- (51.6000,65.8000) -- (52.0000,66.0000) -- (52.4000,66.1000) -- (52.9000,66.2000) -- (53.3000,66.1000) -- (53.7000,66.0000) -- (54.1000,65.9000) -- (54.6000,65.7000) -- (55.0000,65.4000) -- (55.4000,65.2000) -- (55.8000,64.9000) -- (56.3000,64.7000) -- (56.7000,64.5000) -- (57.1000,64.4000) -- (57.5000,64.3000) -- (57.9000,64.4000) -- (58.4000,64.5000) -- (58.8000,64.6000) -- (59.2000,64.8000) -- (59.6000,64.9000) -- (60.1000,65.0000) -- (60.5000,65.1000) -- (60.9000,65.2000) -- (61.3000,65.2000) -- (61.8000,65.2000) -- (62.2000,65.1000) -- (62.6000,65.0000) -- (63.0000,64.8000) -- (63.4000,64.6000) -- (63.9000,64.3000) -- (64.3000,64.1000) -- (64.7000,63.9000) -- (65.1000,63.8000) -- (65.6000,63.7000) -- (66.0000,63.6000) -- (66.4000,63.5000) -- (66.8000,63.4000) -- (67.3000,63.4000) -- (67.7000,63.4000) -- (68.1000,63.4000) -- (68.5000,63.3000) -- (68.9000,63.3000) -- (69.4000,63.3000) -- (69.8000,63.3000) -- (70.2000,63.2000) -- (70.6000,63.2000) -- (71.1000,63.1000) -- (71.5000,63.1000) -- (71.9000,63.0000) -- (72.3000,63.0000) -- (72.7000,63.0000) -- (73.2000,63.1000) -- (73.6000,63.2000) -- (74.0000,63.3000) -- (74.4000,63.4000) -- (74.9000,63.5000) -- (75.3000,63.6000) -- (75.7000,63.8000) -- (76.1000,64.0000) -- (76.6000,64.2000) -- (77.0000,64.4000) -- (77.4000,64.6000) -- (77.8000,64.8000) -- (78.2000,64.9000) -- (78.7000,65.1000) -- (79.1000,65.2000) -- (79.5000,65.3000) -- (79.9000,65.4000) -- (80.4000,65.5000) -- (80.8000,65.6000) -- (81.2000,65.6000) -- (81.6000,65.7000) -- (82.1000,65.7000) -- (82.5000,65.8000) -- (82.9000,65.8000) -- (83.3000,65.8000) -- (83.7000,65.8000) -- (84.2000,65.7000) -- (84.6000,65.7000) -- (85.0000,65.7000) -- (85.4000,65.7000) -- (85.9000,65.7000) -- (86.3000,65.6000) -- (86.7000,65.6000) -- (87.1000,65.5000) -- (87.6000,65.5000) -- (88.0000,65.4000) -- (88.4000,65.3000) -- (88.8000,65.3000) -- (89.2000,65.2000) -- (89.7000,65.1000) -- (90.1000,65.0000) -- (90.5000,64.9000) -- (90.9000,64.7000) -- (91.4000,64.6000) -- (91.8000,64.5000) -- (92.2000,64.4000) -- (92.6000,64.2000) -- (93.0000,64.1000) -- (93.5000,64.0000) -- (93.9000,63.8000) -- (94.3000,63.7000) -- (94.7000,63.6000) -- (95.2000,63.5000) -- (95.6000,63.4000) -- (96.0000,63.4000) -- (96.4000,63.4000) -- (96.9000,63.4000) -- (97.3000,63.4000) -- (97.7000,63.3000) -- (98.1000,63.3000) -- (98.5000,63.3000) -- (99.0000,63.3000) -- (99.4000,63.3000) -- (99.8000,63.3000) -- (100.2000,63.3000) -- (100.7000,63.3000) -- (101.1000,63.4000) -- (101.5000,63.4000) -- (101.9000,63.4000) -- (102.3000,63.5000) -- (102.8000,63.6000) -- (103.2000,63.7000) -- (103.6000,63.8000) -- (104.0000,63.9000) -- (104.5000,64.0000) -- (104.9000,64.1000) -- (105.3000,64.3000) -- (105.7000,64.4000) -- (106.2000,64.5000) -- (106.6000,64.6000) -- (107.0000,64.8000) -- (107.4000,64.9000) -- (107.8000,65.0000) -- (108.3000,65.2000) -- (108.7000,65.3000) -- (109.1000,65.4000) -- (109.5000,65.5000) -- (110.0000,65.6000) -- (110.4000,65.7000) -- (110.8000,65.7000) -- (111.2000,65.7000) -- (111.7000,65.8000) -- (112.1000,65.8000) -- (112.5000,65.8000) -- (112.9000,65.8000) -- (113.3000,65.7000) -- (113.8000,65.7000) -- (114.2000,65.6000) -- (114.6000,65.5000) -- (115.0000,65.4000) -- (115.5000,65.4000) -- (115.9000,65.3000) -- (116.3000,65.2000) -- (116.7000,65.1000) -- (117.2000,65.0000) -- (117.6000,64.9000) -- (118.0000,64.7000) -- (118.4000,64.6000) -- (118.8000,64.5000) -- (119.3000,64.4000) -- (119.7000,64.2000) -- (120.1000,64.1000) -- (120.5000,64.0000) -- (121.0000,63.8000) -- (121.4000,63.7000) -- (121.8000,63.6000) -- (122.2000,63.4000) -- (122.6000,63.3000) -- (123.1000,63.3000) -- (123.5000,63.2000) -- (123.9000,63.2000) -- (124.3000,63.2000) -- (124.8000,63.1000) -- (125.2000,63.1000) -- (125.6000,63.1000) -- (126.0000,63.1000) -- (126.5000,63.1000) -- (126.9000,63.1000) -- (127.3000,63.2000) -- (127.7000,63.4000) -- (128.1000,63.4000) -- (128.6000,63.5000) -- (129.0000,63.5000) -- (129.4000,63.6000) -- (129.8000,63.8000) -- (130.3000,63.9000) -- (130.7000,64.1000) -- (131.1000,64.2000) -- (131.5000,64.3000) -- (131.9000,64.4000) -- (132.4000,64.6000) -- (132.8000,64.7000) -- (133.2000,64.8000) -- (133.6000,64.9000) -- (134.1000,65.0000) -- (134.5000,65.1000) -- (134.9000,65.2000) -- (135.3000,65.3000) -- (135.8000,65.4000) -- (136.2000,65.5000) -- (136.6000,65.6000) -- (137.0000,65.6000) -- (137.4000,65.6000) -- (137.9000,65.7000) -- (138.3000,65.7000) -- (138.7000,65.8000) -- (139.1000,65.8000) -- (139.6000,65.8000) -- (140.0000,65.7000) -- (140.4000,65.7000) -- (140.8000,65.6000) -- (141.3000,65.5000) -- (141.7000,65.4000) -- (142.1000,65.4000) -- (142.3000,65.3000);



    \end{scope}
    \begin{scope}[cm={{0.74279,0.0,0.0,1.28515,(-186.22138,-161.30028)}},draw=blue,line cap=round,line join=round,line width=0.480pt]
      \path[draw] (25.5000,56.5000) -- (25.5000,80.5000) -- (142.5000,80.5000) -- (142.5000,56.5000) -- (25.5000,56.5000);



    \end{scope}
    \begin{scope}[cm={{0.74279,0.0,0.0,1.28515,(-186.22138,-161.30028)}},draw=ca0a0a4,dash pattern=on 1.22pt off 1.22pt,line cap=round,line join=round,line width=0.305pt,miter limit=4.00]
      \path[draw,dash pattern=on 1.22pt off 1.22pt,line width=0.305pt,miter limit=4.00] (25.5000,94.5000) -- (108.5000,94.5000);



      \path[draw,dash pattern=on 1.22pt off 1.22pt,line width=0.305pt,miter limit=4.00] (137.5000,94.5000) -- (142.5000,94.5000);



    \end{scope}
    \begin{scope}[cm={{0.74279,0.0,0.0,1.28515,(-186.22138,-161.30028)}},draw=blue,line cap=round,line join=round,line width=0.480pt]
      \path[cm={{1.54975,0.0,0.0,1.0,(-13.85377,0.0)}},draw] (25.5000,94.5000) -- (28.5000,94.5000);



      \path[cm={{1.54975,0.0,0.0,1.0,(-78.53317,0.0)}},draw] (142.5000,94.5000) -- (139.5000,94.5000);



    \end{scope}
    \begin{scope}[cm={{0.95389,0.0,0.0,0.95389,(-183.45337,-37.34406)}},draw=blue,fill=ce10000,line cap=rect,line join=bevel,line width=0.800pt]
      \path[fill=ce10000] (0.0000,0.0000) node[above right] (text678) {\scriptsize -10};



    \end{scope}
    \begin{scope}[cm={{0.74279,0.0,0.0,1.28515,(-186.22138,-161.30028)}},draw=ca0a0a4,dash pattern=on 1.22pt off 1.22pt,line cap=round,line join=round,line width=0.305pt,miter limit=4.00]
      \path[draw,dash pattern=on 1.22pt off 1.22pt,line width=0.305pt,miter limit=4.00] (25.5000,82.5000) -- (142.5000,82.5000);



    \end{scope}
    \begin{scope}[cm={{0.74279,0.0,0.0,1.28515,(-186.22138,-161.30028)}},draw=blue,line cap=round,line join=round,line width=0.480pt]
      \path[cm={{1.54975,0.0,0.0,1.0,(-13.85377,0.0)}},draw] (25.5000,82.5000) -- (28.5000,82.5000);



      \path[cm={{1.54975,0.0,0.0,1.0,(-78.53317,0.0)}},draw] (142.5000,82.5000) -- (139.5000,82.5000);



    \end{scope}
    \begin{scope}[cm={{0.95389,0.0,0.0,0.95389,(-180.91222,-52.31054)}},draw=blue,fill=ce10000,line cap=rect,line join=bevel,line width=0.800pt]
      \path[fill=ce10000] (0.0000,0.0000) node[above right] (text708) {\scriptsize 10};



    \end{scope}
    \begin{scope}[cm={{0.74279,0.0,0.0,1.28515,(-186.22138,-161.30028)}},draw=ca0a0a4,dash pattern=on 0.40pt off 0.80pt,line cap=round,line join=round,line width=0.400pt]
      \path[draw] (25.5000,104.5000) -- (25.5000,80.5000);



    \end{scope}
    \begin{scope}[cm={{0.74279,0.0,0.0,1.28515,(-186.22138,-161.30028)}},draw=blue,line cap=round,line join=round,line width=0.480pt]
      \path[draw] (25.5000,104.5000) -- (25.5000,99.5000);



      \path[draw] (25.5000,80.5000) -- (25.5000,84.5000);



    \end{scope}
    \begin{scope}[cm={{0.74279,0.0,0.0,1.28515,(-186.22138,-161.30028)}},draw=ca0a0a4,dash pattern=on 1.22pt off 1.22pt,line cap=round,line join=round,line width=0.305pt,miter limit=4.00]
      \path[draw,dash pattern=on 1.22pt off 1.22pt,line width=0.305pt,miter limit=4.00] (60.5000,104.5000) -- (60.5000,80.5000);



    \end{scope}
    \begin{scope}[cm={{0.74279,0.0,0.0,1.28515,(-186.22138,-161.30028)}},draw=blue,line cap=round,line join=round,line width=0.480pt]
      \path[cm={{1.0,0.0,0.0,0.57583,(0.0,44.39861)}},draw] (60.5000,104.5000) -- (60.5000,99.5000);



      \path[cm={{1.0,0.0,0.0,0.70101,(0.0,23.97919)}},draw] (60.5000,80.5000) -- (60.5000,84.5000);



    \end{scope}
    \begin{scope}[cm={{0.74279,0.0,0.0,1.28515,(-186.22138,-161.30028)}},draw=ca0a0a4,dash pattern=on 1.22pt off 1.22pt,line cap=round,line join=round,line width=0.305pt,miter limit=4.00]
      \path[draw,dash pattern=on 1.22pt off 1.22pt,line width=0.305pt,miter limit=4.00] (95.5000,104.5000) -- (95.5000,80.5000);



    \end{scope}
    \begin{scope}[cm={{0.74279,0.0,0.0,1.28515,(-186.22138,-161.30028)}},draw=blue,line cap=round,line join=round,line width=0.480pt]
      \path[cm={{1.0,0.0,0.0,0.57583,(0.0,44.39861)}},draw] (95.5000,104.5000) -- (95.5000,99.5000);



      \path[cm={{1.0,0.0,0.0,0.70101,(0.0,23.97919)}},draw] (95.5000,80.5000) -- (95.5000,84.5000);



    \end{scope}
    \begin{scope}[cm={{0.74279,0.0,0.0,1.28515,(-186.22138,-161.30028)}},draw=ca0a0a4,dash pattern=on 1.22pt off 1.22pt,line cap=round,line join=round,line width=0.305pt,miter limit=4.00]
      \path[draw,dash pattern=on 1.22pt off 1.22pt,line width=0.305pt,miter limit=4.00] (130.5000,92.5000) -- (130.5000,80.5000);



    \end{scope}
    \begin{scope}[cm={{0.74279,0.0,0.0,1.28515,(-186.22138,-161.30028)}},draw=blue,line cap=round,line join=round,line width=0.480pt]
      \path[cm={{1.0,0.0,0.0,0.57583,(0.0,44.39861)}},draw] (130.5000,104.5000) -- (130.5000,99.5000);



      \path[cm={{1.0,0.0,0.0,0.70101,(0.0,23.97919)}},draw] (130.5000,80.5000) -- (130.5000,84.5000);



    \end{scope}
    \begin{scope}[cm={{0.74279,0.0,0.0,1.28515,(-186.22138,-161.30028)}},draw=blue,line cap=round,line join=round,line width=0.480pt]
      \path[draw] (25.5000,80.5000) -- (25.5000,104.5000) -- (142.5000,104.5000) -- (142.5000,80.5000) -- (25.5000,80.5000);



    \end{scope}
    \begin{scope}[cm={{0.95389,0.0,0.0,0.95389,(-104.99275,-34.97829)}},draw=blue,line cap=rect,line join=bevel,line width=0.800pt]
      \path[fill=blue] (0.0000,0.0000) node[above right] (text836) {\scriptsize $\alpha_2$};



    \end{scope}
    \begin{scope}[cm={{0.74279,0.0,0.0,1.28515,(-184.07401,-166.61499)}},draw=blue,line cap=round,line join=round,line width=0.480pt]
      \path[draw,even odd rule] (123.5000,100.5000) -- (132.5000,100.5000);



    \end{scope}
    \begin{scope}[cm={{0.74279,0.0,0.0,1.28515,(-186.22138,-161.30028)}},draw=blue,line cap=round,line join=round,line width=0.480pt]
      \path[draw] (25.8000,86.8000) -- (25.8000,86.8000) -- (26.2000,88.5000) -- (26.7000,88.7000) -- (27.1000,88.0000) -- (27.5000,88.0000) -- (27.9000,90.0000) -- (28.3000,85.4000) -- (28.8000,92.2000) -- (29.2000,86.5000) -- (29.6000,86.2000) -- (30.0000,92.2000) -- (30.5000,91.0000) -- (30.9000,85.7000) -- (31.3000,84.6000) -- (31.7000,88.3000) -- (32.2000,91.6000) -- (32.6000,91.4000) -- (33.0000,88.6000) -- (33.4000,86.0000) -- (33.8000,85.5000) -- (34.3000,87.1000) -- (34.7000,89.4000) -- (35.1000,91.2000) -- (35.5000,91.5000) -- (36.0000,90.5000) -- (36.4000,88.8000) -- (36.8000,87.2000) -- (37.2000,86.2000) -- (37.6000,86.2000) -- (38.1000,86.9000) -- (38.5000,88.0000) -- (38.9000,89.2000) -- (39.3000,90.2000) -- (39.8000,90.6000) -- (40.2000,90.5000) -- (40.6000,89.9000) -- (41.0000,89.0000) -- (41.5000,88.0000) -- (41.9000,87.1000) -- (42.3000,86.4000) -- (42.7000,86.1000) -- (43.1000,86.2000) -- (43.6000,86.6000) -- (44.0000,87.3000) -- (44.4000,88.1000) -- (44.8000,88.9000) -- (45.3000,89.6000) -- (45.7000,90.1000) -- (46.1000,90.5000) -- (46.5000,90.6000) -- (47.0000,90.6000) -- (47.4000,90.3000) -- (47.8000,89.7000) -- (48.2000,89.0000) -- (48.6000,88.3000) -- (49.1000,87.6000) -- (49.5000,86.9000) -- (49.9000,86.4000) -- (50.3000,86.0000) -- (50.8000,85.9000) -- (51.2000,86.0000) -- (51.6000,86.2000) -- (52.0000,86.6000) -- (52.4000,87.2000) -- (52.9000,87.8000) -- (53.3000,88.3000) -- (53.7000,88.9000) -- (54.1000,89.3000) -- (54.6000,89.6000) -- (55.0000,89.8000) -- (55.4000,89.8000) -- (55.8000,89.6000) -- (56.3000,89.4000) -- (56.7000,89.0000) -- (57.1000,88.6000) -- (57.5000,88.2000) -- (57.9000,87.9000) -- (58.4000,87.7000) -- (58.8000,87.6000) -- (59.2000,87.6000) -- (59.6000,87.7000) -- (60.1000,87.9000) -- (60.5000,88.2000) -- (60.9000,88.5000) -- (61.3000,88.8000) -- (61.8000,89.1000) -- (62.2000,89.4000) -- (62.6000,89.5000) -- (63.0000,89.6000) -- (63.4000,89.6000) -- (63.9000,89.4000) -- (64.3000,89.2000) -- (64.7000,89.0000) -- (65.1000,88.7000) -- (65.6000,88.5000) -- (66.0000,88.2000) -- (66.4000,88.1000) -- (66.8000,87.9000) -- (67.3000,87.9000) -- (67.7000,87.9000) -- (68.1000,87.9000) -- (68.5000,88.0000) -- (68.9000,88.1000) -- (69.4000,88.2000) -- (69.8000,88.4000) -- (70.2000,88.5000) -- (70.6000,88.6000) -- (71.1000,88.7000) -- (71.5000,88.7000) -- (71.9000,88.7000) -- (72.3000,88.6000) -- (72.7000,88.6000) -- (73.2000,88.5000) -- (73.6000,88.5000) -- (74.0000,88.5000) -- (74.4000,88.5000) -- (74.9000,88.4000) -- (75.3000,88.4000) -- (75.7000,88.3000) -- (76.1000,88.3000) -- (76.6000,88.4000) -- (77.0000,88.4000) -- (77.4000,88.4000) -- (77.8000,88.5000) -- (78.2000,88.5000) -- (78.7000,88.6000) -- (79.1000,88.6000) -- (79.5000,88.6000) -- (79.9000,88.6000) -- (80.4000,88.6000) -- (80.8000,88.5000) -- (81.2000,88.5000) -- (81.6000,88.4000) -- (82.1000,88.3000) -- (82.5000,88.3000) -- (82.9000,88.2000) -- (83.3000,88.1000) -- (83.7000,88.0000) -- (84.2000,88.0000) -- (84.6000,87.9000) -- (85.0000,87.9000) -- (85.4000,87.9000) -- (85.9000,87.9000) -- (86.3000,88.0000) -- (86.7000,88.0000) -- (87.1000,88.1000) -- (87.6000,88.2000) -- (88.0000,88.3000) -- (88.4000,88.4000) -- (88.8000,88.5000) -- (89.2000,88.7000) -- (89.7000,88.8000) -- (90.1000,88.9000) -- (90.5000,89.0000) -- (90.9000,89.0000) -- (91.4000,89.1000) -- (91.8000,89.1000) -- (92.2000,89.0000) -- (92.6000,89.0000) -- (93.0000,88.9000) -- (93.5000,88.8000) -- (93.9000,88.7000) -- (94.3000,88.6000) -- (94.7000,88.5000) -- (95.2000,88.3000) -- (95.6000,88.2000) -- (96.0000,88.1000) -- (96.4000,88.0000) -- (96.9000,88.0000) -- (97.3000,88.0000) -- (97.7000,88.0000) -- (98.1000,88.0000) -- (98.5000,88.0000) -- (99.0000,88.1000) -- (99.4000,88.1000) -- (99.8000,88.2000) -- (100.2000,88.2000) -- (100.7000,88.3000) -- (101.1000,88.5000) -- (101.5000,88.5000) -- (101.9000,88.5000) -- (102.3000,88.5000) -- (102.8000,88.6000) -- (103.2000,88.6000) -- (103.6000,88.6000) -- (104.0000,88.7000) -- (104.5000,88.7000) -- (104.9000,88.7000) -- (105.3000,88.7000) -- (105.7000,88.7000) -- (106.2000,88.6000) -- (106.6000,88.6000) -- (107.0000,88.6000) -- (107.4000,88.5000) -- (107.8000,88.5000) -- (108.3000,88.4000) -- (108.7000,88.4000) -- (109.1000,88.4000) -- (109.5000,88.4000) -- (110.0000,88.4000) -- (110.4000,88.4000) -- (110.8000,88.3000) -- (111.2000,88.3000) -- (111.7000,88.3000) -- (112.1000,88.3000) -- (112.5000,88.3000) -- (112.9000,88.3000) -- (113.3000,88.3000) -- (113.8000,88.3000) -- (114.2000,88.3000) -- (114.6000,88.3000) -- (115.0000,88.3000) -- (115.5000,88.3000) -- (115.9000,88.4000) -- (116.3000,88.4000) -- (116.7000,88.4000) -- (117.2000,88.5000) -- (117.6000,88.5000) -- (118.0000,88.6000) -- (118.4000,88.6000) -- (118.8000,88.6000) -- (119.3000,88.7000) -- (119.7000,88.7000) -- (120.1000,88.7000) -- (120.5000,88.6000) -- (121.0000,88.6000) -- (121.4000,88.6000) -- (121.8000,88.5000) -- (122.2000,88.4000) -- (122.6000,88.4000) -- (123.1000,88.3000) -- (123.5000,88.3000) -- (123.9000,88.3000) -- (124.3000,88.3000) -- (124.8000,88.2000) -- (125.2000,88.2000) -- (125.6000,88.2000) -- (126.0000,88.1000) -- (126.5000,88.1000) -- (126.9000,88.2000) -- (127.3000,88.2000) -- (127.7000,88.3000) -- (128.1000,88.4000) -- (128.6000,88.4000) -- (129.0000,88.4000) -- (129.4000,88.5000) -- (129.8000,88.6000) -- (130.3000,88.6000) -- (130.7000,88.7000) -- (131.1000,88.8000) -- (131.5000,88.8000) -- (131.9000,88.8000) -- (132.4000,88.8000) -- (132.8000,88.8000) -- (133.2000,88.7000) -- (133.6000,88.7000) -- (134.1000,88.6000) -- (134.5000,88.5000) -- (134.9000,88.4000) -- (135.3000,88.4000) -- (135.8000,88.3000) -- (136.2000,88.3000) -- (136.6000,88.2000) -- (137.0000,88.2000) -- (137.4000,88.1000) -- (137.9000,88.1000) -- (138.3000,88.1000) -- (138.7000,88.1000) -- (139.1000,88.2000) -- (139.6000,88.2000) -- (140.0000,88.3000) -- (140.4000,88.3000) -- (140.8000,88.3000) -- (141.3000,88.4000) -- (141.7000,88.4000) -- (142.1000,88.5000) -- (142.3000,88.5000);



    \end{scope}
    \begin{scope}[cm={{0.74279,0.0,0.0,1.28515,(-186.22138,-161.30028)}},draw=blue,line cap=round,line join=round,line width=0.480pt]
      \path[draw] (25.5000,80.5000) -- (25.5000,104.5000) -- (142.5000,104.5000) -- (142.5000,80.5000) -- (25.5000,80.5000);



    \end{scope}
    \begin{scope}[cm={{0.74279,0.0,0.0,1.28515,(-186.22138,-161.30028)}},draw=ca0a0a4,dash pattern=on 1.22pt off 1.22pt,line cap=round,line join=round,line width=0.305pt,miter limit=4.00]
      \path[draw,dash pattern=on 1.22pt off 1.22pt,line width=0.305pt,miter limit=4.00] (25.5000,118.5000) -- (108.5000,118.5000);



      \path[draw,dash pattern=on 1.22pt off 1.22pt,line width=0.305pt,miter limit=4.00] (137.5000,118.5000) -- (142.5000,118.5000);



    \end{scope}
    \begin{scope}[cm={{0.74279,0.0,0.0,1.28515,(-186.22138,-161.30028)}},draw=blue,line cap=round,line join=round,line width=0.480pt]
      \path[cm={{1.54975,0.0,0.0,1.0,(-13.85377,0.0)}},draw] (25.5000,118.5000) -- (28.5000,118.5000);



      \path[cm={{1.54975,0.0,0.0,1.0,(-78.53317,0.0)}},draw] (142.5000,118.5000) -- (139.5000,118.5000);



    \end{scope}
    \begin{scope}[cm={{0.95389,0.0,0.0,0.95389,(-183.45337,-6.99453)}},draw=blue,fill=ce10000,line cap=rect,line join=bevel,line width=0.800pt]
      \path[fill=ce10000] (0.0000,0.0000) node[above right] (text892) {\scriptsize -10};



    \end{scope}
    \begin{scope}[cm={{0.74279,0.0,0.0,1.28515,(-186.22138,-161.30028)}},draw=ca0a0a4,dash pattern=on 1.22pt off 1.22pt,line cap=round,line join=round,line width=0.305pt,miter limit=4.00]
      \path[draw,dash pattern=on 1.22pt off 1.22pt,line width=0.305pt,miter limit=4.00] (25.5000,106.5000) -- (142.5000,106.5000);



    \end{scope}
    \begin{scope}[cm={{0.74279,0.0,0.0,1.28515,(-186.22138,-161.30028)}},draw=blue,line cap=round,line join=round,line width=0.480pt]
      \path[cm={{1.54975,0.0,0.0,1.0,(-13.85377,0.0)}},draw] (25.5000,106.5000) -- (28.5000,106.5000);



      \path[cm={{1.54975,0.0,0.0,1.0,(-78.53317,0.0)}},draw] (142.5000,106.5000) -- (139.5000,106.5000);



    \end{scope}
    \begin{scope}[cm={{0.95389,0.0,0.0,0.95389,(-180.91222,-21.96101)}},draw=blue,fill=ce10000,line cap=rect,line join=bevel,line width=0.800pt]
      \path[fill=ce10000] (0.0000,0.0000) node[above right] (text922) {\scriptsize 10};



    \end{scope}
    \begin{scope}[cm={{0.74279,0.0,0.0,1.28515,(-186.22138,-161.30028)}},draw=ca0a0a4,dash pattern=on 0.40pt off 0.80pt,line cap=round,line join=round,line width=0.400pt]
      \path[draw] (25.5000,128.5000) -- (25.5000,104.5000);



    \end{scope}
    \begin{scope}[cm={{0.74279,0.0,0.0,1.28515,(-186.22138,-161.30028)}},draw=blue,line cap=round,line join=round,line width=0.480pt]
      \path[draw] (25.5000,128.5000) -- (25.5000,123.5000);



      \path[draw] (25.5000,104.5000) -- (25.5000,108.5000);



    \end{scope}
    \begin{scope}[cm={{0.74279,0.0,0.0,1.28515,(-186.22138,-161.30028)}},draw=ca0a0a4,dash pattern=on 1.22pt off 1.22pt,line cap=round,line join=round,line width=0.305pt,miter limit=4.00]
      \path[draw,dash pattern=on 1.22pt off 1.22pt,line width=0.305pt,miter limit=4.00] (60.5000,128.5000) -- (60.5000,104.5000);



    \end{scope}
    \begin{scope}[cm={{0.74279,0.0,0.0,1.28515,(-186.22138,-161.30028)}},draw=blue,line cap=round,line join=round,line width=0.480pt]
      \path[cm={{1.0,0.0,0.0,0.57583,(0.0,54.57875)}},draw] (60.5000,128.5000) -- (60.5000,123.5000);



      \path[cm={{1.0,0.0,0.0,0.70101,(0.0,31.15501)}},draw] (60.5000,104.5000) -- (60.5000,108.5000);



    \end{scope}
    \begin{scope}[cm={{0.74279,0.0,0.0,1.28515,(-186.22138,-161.30028)}},draw=ca0a0a4,dash pattern=on 1.22pt off 1.22pt,line cap=round,line join=round,line width=0.305pt,miter limit=4.00]
      \path[draw,dash pattern=on 1.22pt off 1.22pt,line width=0.305pt,miter limit=4.00] (95.5000,128.5000) -- (95.5000,104.5000);



    \end{scope}
    \begin{scope}[cm={{0.74279,0.0,0.0,1.28515,(-147.4494,-161.30028)}},draw=blue,line cap=round,line join=round,line width=0.480pt]
      \path[cm={{1.0,0.0,0.0,0.57583,(-52.1975,54.57875)}},draw] (95.5000,128.5000) -- (95.5000,123.5000);



      \path[cm={{1.0,0.0,0.0,0.70101,(-52.1975,31.15501)}},draw] (95.5000,104.5000) -- (95.5000,108.5000);



    \end{scope}
    \begin{scope}[cm={{0.74279,0.0,0.0,1.28515,(-186.22138,-161.30028)}},draw=ca0a0a4,dash pattern=on 1.22pt off 1.22pt,line cap=round,line join=round,line width=0.305pt,miter limit=4.00]
      \path[draw,dash pattern=on 1.22pt off 1.22pt,line width=0.305pt,miter limit=4.00] (130.5000,116.5000) -- (130.5000,104.5000);



    \end{scope}
    \begin{scope}[cm={{0.74279,0.0,0.0,1.28515,(-186.22138,-161.30028)}},draw=blue,line cap=round,line join=round,line width=0.480pt]
      \path[cm={{1.0,0.0,0.0,0.57583,(0.0,54.57875)}},draw] (130.5000,128.5000) -- (130.5000,123.5000);



      \path[cm={{1.0,0.0,0.0,0.70101,(0.0,31.15501)}},draw] (130.5000,104.5000) -- (130.5000,108.5000);



    \end{scope}
    \begin{scope}[cm={{0.74279,0.0,0.0,1.28515,(-186.22138,-161.30028)}},draw=blue,line cap=round,line join=round,line width=0.480pt]
      \path[draw] (25.5000,104.5000) -- (25.5000,128.5000) -- (142.5000,128.5000) -- (142.5000,104.5000) -- (25.5000,104.5000);



    \end{scope}
    \begin{scope}[cm={{0.95389,0.0,0.0,0.95389,(-104.73329,-4.06956)}},draw=blue,line cap=rect,line join=bevel,line width=0.800pt]
      \path[fill=blue] (0.0000,0.0000) node[above right] (text1050) {\scriptsize $\beta_2$};



    \end{scope}
    \begin{scope}[cm={{0.74279,0.0,0.0,1.28515,(-184.07401,-166.61499)}},draw=blue,line cap=round,line join=round,line width=0.480pt]
      \path[draw,even odd rule] (123.5000,124.5000) -- (132.5000,124.5000);



    \end{scope}
    \begin{scope}[cm={{0.74279,0.0,0.0,1.28515,(-186.22138,-161.30028)}},draw=blue,line cap=round,line join=round,line width=0.480pt]
      \path[draw] (25.8000,114.2000) -- (25.8000,114.2000) -- (26.2000,112.5000) -- (26.7000,112.2000) -- (27.1000,112.0000) -- (27.5000,115.1000) -- (27.9000,109.2000) -- (28.3000,114.1000) -- (28.8000,113.9000) -- (29.2000,108.7000) -- (29.6000,114.6000) -- (30.0000,115.3000) -- (30.5000,109.5000) -- (30.9000,108.5000) -- (31.3000,113.0000) -- (31.7000,116.4000) -- (32.2000,115.2000) -- (32.6000,111.5000) -- (33.0000,109.2000) -- (33.4000,109.7000) -- (33.8000,112.1000) -- (34.3000,114.6000) -- (34.7000,115.4000) -- (35.1000,114.5000) -- (35.5000,112.6000) -- (36.0000,110.6000) -- (36.4000,109.5000) -- (36.8000,109.5000) -- (37.2000,110.5000) -- (37.6000,112.0000) -- (38.1000,113.4000) -- (38.5000,114.4000) -- (38.9000,114.7000) -- (39.3000,114.3000) -- (39.8000,113.4000) -- (40.2000,112.3000) -- (40.6000,111.3000) -- (41.0000,110.6000) -- (41.5000,110.2000) -- (41.9000,110.3000) -- (42.3000,110.8000) -- (42.7000,111.6000) -- (43.1000,112.5000) -- (43.6000,113.4000) -- (44.0000,114.1000) -- (44.4000,114.5000) -- (44.8000,114.7000) -- (45.3000,114.6000) -- (45.7000,114.3000) -- (46.1000,113.7000) -- (46.5000,113.0000) -- (47.0000,112.4000) -- (47.4000,111.7000) -- (47.8000,111.0000) -- (48.2000,110.5000) -- (48.6000,110.2000) -- (49.1000,110.2000) -- (49.5000,110.4000) -- (49.9000,110.8000) -- (50.3000,111.4000) -- (50.8000,112.0000) -- (51.2000,112.7000) -- (51.6000,113.4000) -- (52.0000,114.0000) -- (52.4000,114.5000) -- (52.9000,114.8000) -- (53.3000,114.8000) -- (53.7000,114.7000) -- (54.1000,114.5000) -- (54.6000,114.0000) -- (55.0000,113.5000) -- (55.4000,113.0000) -- (55.8000,112.5000) -- (56.3000,112.0000) -- (56.7000,111.6000) -- (57.1000,111.4000) -- (57.5000,111.3000) -- (57.9000,111.3000) -- (58.4000,111.5000) -- (58.8000,111.8000) -- (59.2000,112.2000) -- (59.6000,112.5000) -- (60.1000,112.8000) -- (60.5000,113.0000) -- (60.9000,113.2000) -- (61.3000,113.2000) -- (61.8000,113.2000) -- (62.2000,113.1000) -- (62.6000,112.8000) -- (63.0000,112.6000) -- (63.4000,112.2000) -- (63.9000,111.9000) -- (64.3000,111.7000) -- (64.7000,111.5000) -- (65.1000,111.4000) -- (65.6000,111.3000) -- (66.0000,111.4000) -- (66.4000,111.5000) -- (66.8000,111.7000) -- (67.3000,112.0000) -- (67.7000,112.2000) -- (68.1000,112.4000) -- (68.5000,112.6000) -- (68.9000,112.7000) -- (69.4000,112.8000) -- (69.8000,112.9000) -- (70.2000,112.9000) -- (70.6000,112.9000) -- (71.1000,112.8000) -- (71.5000,112.7000) -- (71.9000,112.6000) -- (72.3000,112.5000) -- (72.7000,112.4000) -- (73.2000,112.3000) -- (73.6000,112.3000) -- (74.0000,112.3000) -- (74.4000,112.3000) -- (74.9000,112.3000) -- (75.3000,112.3000) -- (75.7000,112.3000) -- (76.1000,112.3000) -- (76.6000,112.4000) -- (77.0000,112.5000) -- (77.4000,112.5000) -- (77.8000,112.5000) -- (78.2000,112.5000) -- (78.7000,112.5000) -- (79.1000,112.5000) -- (79.5000,112.5000) -- (79.9000,112.4000) -- (80.4000,112.4000) -- (80.8000,112.3000) -- (81.2000,112.3000) -- (81.6000,112.3000) -- (82.1000,112.2000) -- (82.5000,112.3000) -- (82.9000,112.3000) -- (83.3000,112.3000) -- (83.7000,112.3000) -- (84.2000,112.3000) -- (84.6000,112.4000) -- (85.0000,112.5000) -- (85.4000,112.6000) -- (85.9000,112.7000) -- (86.3000,112.8000) -- (86.7000,112.8000) -- (87.1000,112.9000) -- (87.6000,112.9000) -- (88.0000,113.0000) -- (88.4000,113.0000) -- (88.8000,113.0000) -- (89.2000,113.0000) -- (89.7000,113.0000) -- (90.1000,112.9000) -- (90.5000,112.8000) -- (90.9000,112.7000) -- (91.4000,112.6000) -- (91.8000,112.4000) -- (92.2000,112.3000) -- (92.6000,112.2000) -- (93.0000,112.0000) -- (93.5000,111.9000) -- (93.9000,111.9000) -- (94.3000,111.8000) -- (94.7000,111.8000) -- (95.2000,111.8000) -- (95.6000,111.8000) -- (96.0000,111.9000) -- (96.4000,112.0000) -- (96.9000,112.2000) -- (97.3000,112.3000) -- (97.7000,112.4000) -- (98.1000,112.5000) -- (98.5000,112.6000) -- (99.0000,112.6000) -- (99.4000,112.7000) -- (99.8000,112.7000) -- (100.2000,112.8000) -- (100.7000,112.8000) -- (101.1000,112.9000) -- (101.5000,112.9000) -- (101.9000,112.8000) -- (102.3000,112.7000) -- (102.8000,112.7000) -- (103.2000,112.6000) -- (103.6000,112.6000) -- (104.0000,112.6000) -- (104.5000,112.5000) -- (104.9000,112.5000) -- (105.3000,112.4000) -- (105.7000,112.3000) -- (106.2000,112.3000) -- (106.6000,112.3000) -- (107.0000,112.2000) -- (107.4000,112.2000) -- (107.8000,112.2000) -- (108.3000,112.2000) -- (108.7000,112.2000) -- (109.1000,112.3000) -- (109.5000,112.3000) -- (110.0000,112.3000) -- (110.4000,112.3000) -- (110.8000,112.3000) -- (111.2000,112.3000) -- (111.7000,112.4000) -- (112.1000,112.4000) -- (112.5000,112.4000) -- (112.9000,112.5000) -- (113.3000,112.5000) -- (113.8000,112.5000) -- (114.2000,112.5000) -- (114.6000,112.5000) -- (115.0000,112.5000) -- (115.5000,112.6000) -- (115.9000,112.6000) -- (116.3000,112.6000) -- (116.7000,112.6000) -- (117.2000,112.6000) -- (117.6000,112.6000) -- (118.0000,112.6000) -- (118.4000,112.6000) -- (118.8000,112.6000) -- (119.3000,112.5000) -- (119.7000,112.5000) -- (120.1000,112.4000) -- (120.5000,112.4000) -- (121.0000,112.3000) -- (121.4000,112.3000) -- (121.8000,112.2000) -- (122.2000,112.2000) -- (122.6000,112.2000) -- (123.1000,112.2000) -- (123.5000,112.2000) -- (123.9000,112.3000) -- (124.3000,112.3000) -- (124.8000,112.3000) -- (125.2000,112.4000) -- (125.6000,112.4000) -- (126.0000,112.4000) -- (126.5000,112.5000) -- (126.9000,112.5000) -- (127.3000,112.6000) -- (127.7000,112.7000) -- (128.1000,112.7000) -- (128.6000,112.7000) -- (129.0000,112.7000) -- (129.4000,112.7000) -- (129.8000,112.7000) -- (130.3000,112.7000) -- (130.7000,112.7000) -- (131.1000,112.6000) -- (131.5000,112.6000) -- (131.9000,112.5000) -- (132.4000,112.4000) -- (132.8000,112.3000) -- (133.2000,112.2000) -- (133.6000,112.2000) -- (134.1000,112.1000) -- (134.5000,112.1000) -- (134.9000,112.1000) -- (135.3000,112.1000) -- (135.8000,112.1000) -- (136.2000,112.2000) -- (136.6000,112.2000) -- (137.0000,112.2000) -- (137.4000,112.3000) -- (137.9000,112.4000) -- (138.3000,112.4000) -- (138.7000,112.5000) -- (139.1000,112.6000) -- (139.6000,112.7000) -- (140.0000,112.7000) -- (140.4000,112.7000) -- (140.8000,112.7000) -- (141.3000,112.7000) -- (141.7000,112.7000) -- (142.1000,112.7000) -- (142.3000,112.7000);



    \end{scope}
    \begin{scope}[cm={{0.74279,0.0,0.0,1.28515,(-186.22138,-161.30028)}},draw=blue,line cap=round,line join=round,line width=0.480pt]
      \path[draw] (25.5000,104.5000) -- (25.5000,128.5000) -- (142.5000,128.5000) -- (142.5000,104.5000) -- (25.5000,104.5000);



    \end{scope}
    \begin{scope}[cm={{0.74279,0.0,0.0,1.28515,(-186.22138,-161.30028)}},draw=ca0a0a4,dash pattern=on 1.22pt off 1.22pt,line cap=round,line join=round,line width=0.305pt,miter limit=4.00]
      \path[draw,dash pattern=on 1.22pt off 1.22pt,line width=0.305pt,miter limit=4.00] (25.5000,144.5000) -- (108.5000,144.5000);



      \path[draw,dash pattern=on 1.22pt off 1.22pt,line width=0.305pt,miter limit=4.00] (137.5000,144.5000) -- (142.5000,144.5000);



    \end{scope}
    \begin{scope}[cm={{0.74279,0.0,0.0,1.28515,(-186.22138,-161.30028)}},draw=blue,line cap=round,line join=round,line width=0.480pt]
      \path[cm={{1.54975,0.0,0.0,1.0,(-13.85377,0.0)}},draw] (25.5000,144.5000) -- (28.5000,144.5000);



      \path[cm={{1.54975,0.0,0.0,1.0,(-78.53317,0.0)}},draw] (142.5000,144.5000) -- (139.5000,144.5000);



    \end{scope}
    \begin{scope}[cm={{0.95389,0.0,0.0,0.95389,(-183.45337,27.70791)}},draw=blue,fill=ce10000,line cap=rect,line join=bevel,line width=0.800pt]
      \path[fill=ce10000] (0.0000,0.0000) node[above right] (text1106) {\scriptsize -20};



    \end{scope}
    \begin{scope}[cm={{0.74279,0.0,0.0,1.28515,(-186.22138,-161.30028)}},draw=ca0a0a4,dash pattern=on 1.22pt off 1.22pt,line cap=round,line join=round,line width=0.305pt,miter limit=4.00]
      \path[draw,dash pattern=on 1.22pt off 1.22pt,line width=0.305pt,miter limit=4.00] (25.5000,129.5000) -- (142.5000,129.5000);



    \end{scope}
    \begin{scope}[cm={{0.74279,0.0,0.0,1.28515,(-186.22138,-161.30028)}},draw=blue,line cap=round,line join=round,line width=0.480pt]
      \path[cm={{1.54975,0.0,0.0,1.0,(-13.85377,0.0)}},draw] (25.5000,129.5000) -- (28.5000,129.5000);



      \path[cm={{1.54975,0.0,0.0,1.0,(-78.53317,0.0)}},draw] (142.5000,129.5000) -- (139.5000,129.5000);



    \end{scope}
    \begin{scope}[cm={{0.95389,0.0,0.0,0.95389,(-180.91222,7.43463)}},draw=blue,fill=ce10000,line cap=rect,line join=bevel,line width=0.800pt]
      \path[fill=ce10000] (0.0000,0.0000) node[above right] (text1136) {\scriptsize 20};



    \end{scope}
    \begin{scope}[cm={{0.74279,0.0,0.0,1.28515,(-186.22138,-161.30028)}},draw=ca0a0a4,dash pattern=on 0.40pt off 0.80pt,line cap=round,line join=round,line width=0.400pt]
      \path[draw] (25.5000,152.5000) -- (25.5000,128.5000);



    \end{scope}
    \begin{scope}[cm={{0.74279,0.0,0.0,1.28515,(-186.22138,-161.30028)}},draw=blue,line cap=round,line join=round,line width=0.480pt]
      \path[draw] (25.5000,152.5000) -- (25.5000,147.5000);



      \path[draw] (25.5000,128.5000) -- (25.5000,132.5000);



    \end{scope}
    \begin{scope}[cm={{0.74279,0.0,0.0,1.28515,(-186.22138,-161.30028)}},draw=ca0a0a4,dash pattern=on 1.22pt off 1.22pt,line cap=round,line join=round,line width=0.305pt,miter limit=4.00]
      \path[draw,dash pattern=on 1.22pt off 1.22pt,line width=0.305pt,miter limit=4.00] (60.5000,152.5000) -- (60.5000,128.5000);



    \end{scope}
    \begin{scope}[cm={{0.74279,0.0,0.0,1.28515,(-186.22138,-161.30028)}},draw=blue,line cap=round,line join=round,line width=0.480pt]
      \path[cm={{1.0,0.0,0.0,0.57583,(0.0,64.75889)}},draw] (60.5000,152.5000) -- (60.5000,147.5000);



      \path[cm={{1.0,0.0,0.0,0.70101,(0.0,38.33083)}},draw] (60.5000,128.5000) -- (60.5000,132.5000);



    \end{scope}
    \begin{scope}[cm={{0.74279,0.0,0.0,1.28515,(-186.22138,-161.30028)}},draw=ca0a0a4,dash pattern=on 1.22pt off 1.22pt,line cap=round,line join=round,line width=0.305pt,miter limit=4.00]
      \path[draw,dash pattern=on 1.22pt off 1.22pt,line width=0.305pt,miter limit=4.00] (95.5000,152.5000) -- (95.5000,128.5000);



    \end{scope}
    \begin{scope}[cm={{0.74279,0.0,0.0,1.28515,(-186.22138,-161.30028)}},draw=blue,line cap=round,line join=round,line width=0.480pt]
      \path[cm={{1.0,0.0,0.0,0.57583,(0.0,64.75889)}},draw] (95.5000,152.5000) -- (95.5000,147.5000);



      \path[cm={{1.0,0.0,0.0,0.70101,(0.0,38.33083)}},draw] (95.5000,128.5000) -- (95.5000,132.5000);



    \end{scope}
    \begin{scope}[cm={{0.74279,0.0,0.0,1.28515,(-186.22138,-161.30028)}},draw=ca0a0a4,dash pattern=on 1.22pt off 1.22pt,line cap=round,line join=round,line width=0.305pt,miter limit=4.00]
      \path[draw,dash pattern=on 1.22pt off 1.22pt,line width=0.305pt,miter limit=4.00] (130.5000,140.5000) -- (130.5000,128.5000);



    \end{scope}
    \begin{scope}[cm={{0.99623,0.0,0.0,1.3704,(-320.15815,-158.86873)}},draw=ca0a0a4,dash pattern=on 1.02pt off 1.02pt,even odd rule,line cap=round,line join=round,line width=0.255pt,miter limit=4.00]
      \path[draw,dash pattern=on 1.02pt off 1.02pt,even odd rule,line cap=round,line width=0.255pt,miter limit=4.00] (41.5000,101.5000) -- (127.5000,101.5000);



    \end{scope}
    \begin{scope}[cm={{0.95389,0.0,0.0,0.95389,(-228.06584,308.36388)}},draw=blue,line cap=rect,line join=bevel,line width=0.800pt]
      \begin{scope}[cm={{1.04485,0.0,0.0,1.39271,(-96.58638,-490.878)}},draw=ca0a0a4,dash pattern=on 1.56pt off 1.56pt,line cap=round,line join=round,line width=0.259pt,miter limit=4.00]
        \path[draw,dash pattern=on 1.56pt off 1.56pt,line width=0.259pt,miter limit=4.00] (70.5000,84.5000) -- (70.5000,30.5000);



        \path[draw,dash pattern=on 1.56pt off 1.56pt,line width=0.259pt,miter limit=4.00] (70.5000,14.5000) -- (70.5000,8.5000);



      \end{scope}
      \begin{scope}[cm={{1.04485,0.0,0.0,1.39271,(-96.58638,-490.7838)}},draw=ca0a0a4,dash pattern=on 1.56pt off 1.56pt,line cap=round,line join=round,line width=0.259pt,miter limit=4.00]
        \path[draw,dash pattern=on 1.56pt off 1.56pt,line width=0.259pt,miter limit=4.00] (98.5000,84.5000) -- (98.5000,30.5000);



        \path[draw,dash pattern=on 1.56pt off 1.56pt,line width=0.259pt,miter limit=4.00] (98.5000,14.5000) -- (98.5000,8.5000);



      \end{scope}
      \path[fill=ce10000] (46.7690,-262.2919) node[above right] (text1320) {\scriptsize -20};



      \begin{scope}[cm={{1.04439,0.0,0.0,1.41697,(-96.54445,-493.13085)}},draw=blue,line cap=round,line join=round,line width=0.480pt]
        \path[draw=cd9d9d9,line cap=rect,line join=miter,line width=2.127pt,miter limit=4.00] (41.5000,8.5000) -- (41.5000,84.5000) -- (127.5000,84.5000) -- (127.5000,8.5000) -- (41.5000,8.5000);



      \end{scope}
      \begin{scope}[cm={{1.04439,0.0,0.0,1.41697,(-96.54445,-493.13085)}},draw=ca0a0a4,dash pattern=on 1.03pt off 1.03pt,line cap=round,line join=round,line width=0.257pt,miter limit=4.00]
        \path[draw,dash pattern=on 1.03pt off 1.03pt,line width=0.257pt,miter limit=4.00] (41.5000,74.5000) -- (127.5000,74.5000);



      \end{scope}
      \begin{scope}[cm={{1.04439,0.0,0.0,1.41697,(-96.54445,-493.13085)}},draw=blue,line cap=round,line join=round,line width=0.480pt]
        \path[cm={{1.15551,0.0,0.0,1.0,(-6.40682,0.0)}},draw] (41.5000,74.5000) -- (44.5000,74.5000);



        \path[cm={{1.15551,0.0,0.0,1.0,(-19.98953,0.0)}},draw] (127.5000,74.5000) -- (124.5000,74.5000);



      \end{scope}
      \begin{scope}[cm={{1.05023,0.0,0.0,1.05023,(-193.26478,-275.39592)}},draw=blue,line cap=rect,line join=bevel,line width=0.800pt]
        \path[fill=blue] (119.0842,-104.1987) node[above right] (text34-5) {\scriptsize 32};



      \end{scope}
      \begin{scope}[cm={{1.04439,0.0,0.0,1.41697,(-96.54445,-493.13085)}},draw=ca0a0a4,dash pattern=on 1.03pt off 1.03pt,line cap=round,line join=round,line width=0.257pt,miter limit=4.00]
        \path[draw,dash pattern=on 1.03pt off 1.03pt,line width=0.257pt,miter limit=4.00] (41.5000,48.5000) -- (127.5000,48.5000);



      \end{scope}
      \begin{scope}[cm={{1.04439,0.0,0.0,1.41697,(-96.54445,-493.13085)}},draw=blue,line cap=round,line join=round,line width=0.480pt]
        \path[cm={{1.15551,0.0,0.0,1.0,(-6.40682,0.0)}},draw] (41.5000,48.5000) -- (44.5000,48.5000);



        \path[cm={{1.15551,0.0,0.0,1.0,(-19.98953,0.0)}},draw] (127.5000,48.5000) -- (124.5000,48.5000);



      \end{scope}
      \begin{scope}[cm={{1.05023,0.0,0.0,1.05023,(-68.14069,-421.00079)}},draw=blue,line cap=rect,line join=bevel,line width=0.800pt]
        \path[fill=blue] (0.0000,0.0000) node[above right] (text64-7) {\scriptsize 36};



      \end{scope}
      \begin{scope}[cm={{1.04485,0.0,0.0,1.39271,(-88.8098,-492.60283)}},draw=ca0a0a4,dash pattern=on 1.56pt off 1.56pt,line cap=round,line join=round,line width=0.259pt,miter limit=4.00]
        \path[shift={(-7.44276,0)},draw,dash pattern=on 1.56pt off 1.56pt,line width=0.259pt,miter limit=4.00] (41.5000,22.5000) -- (46.5000,22.5000);



        \path[shift={(-7.44276,0)},draw,dash pattern=on 1.56pt off 1.56pt,line width=0.259pt,miter limit=4.00] (103.5000,22.5000) -- (127.5000,22.5000);



      \end{scope}
      \begin{scope}[cm={{1.04439,0.0,0.0,1.41697,(-96.54445,-493.13085)}},draw=blue,line cap=round,line join=round,line width=0.480pt]
        \path[cm={{1.15551,0.0,0.0,1.0,(-6.40682,0.0)}},draw] (41.5000,22.5000) -- (44.5000,22.5000);



        \path[cm={{1.15551,0.0,0.0,1.0,(-19.98953,0.0)}},draw] (127.5000,22.5000) -- (124.5000,22.5000);



      \end{scope}
      \begin{scope}[cm={{1.05023,0.0,0.0,1.05023,(-68.2079,-459.24995)}},draw=blue,line cap=rect,line join=bevel,line width=0.800pt]
        \path[fill=blue] (0.0000,0.0000) node[above right] (text96) {\scriptsize 40};



      \end{scope}
      \begin{scope}[cm={{1.04439,0.0,0.0,1.41697,(-96.54445,-493.13085)}},draw=ca0a0a4,dash pattern=on 0.40pt off 0.80pt,line cap=round,line join=round,line width=0.400pt]
        \path[draw] (41.5000,84.5000) -- (41.5000,8.5000);



      \end{scope}
      \begin{scope}[cm={{1.04439,0.0,0.0,1.41697,(-96.54445,-493.13085)}},draw=blue,line cap=round,line join=round,line width=0.480pt]
        \path[draw] (41.5000,84.5000) -- (41.5000,80.5000);



        \path[draw] (41.5000,8.5000) -- (41.5000,11.5000);



      \end{scope}
      \begin{scope}[cm={{1.04439,0.0,0.0,1.41697,(-96.54445,-493.13085)}},draw=blue,line cap=round,line join=round,line width=0.480pt]
        \path[cm={{1.0,0.0,0.0,0.66652,(0.0,27.84811)}},draw] (70.5000,84.5000) -- (70.5000,80.5000);



        \path[cm={{1.0,0.0,0.0,0.85167,(0.00012,1.21632)}},draw] (70.5000,8.5000) -- (70.5000,11.5000);



      \end{scope}
      \begin{scope}[cm={{1.04439,0.0,0.0,1.41697,(-96.54445,-493.13085)}},draw=blue,line cap=round,line join=round,line width=0.480pt]
        \path[cm={{1.0,0.0,0.0,0.66652,(0.0,27.84811)}},draw] (98.5000,84.5000) -- (98.5000,80.5000);



        \path[cm={{1.0,0.0,0.0,0.85167,(0.14844,1.21632)}},draw] (98.5000,8.5000) -- (98.5000,11.5000);



      \end{scope}
      \begin{scope}[cm={{1.04439,0.0,0.0,1.41697,(-96.54445,-493.13085)}},draw=ca0a0a4,dash pattern=on 0.40pt off 0.80pt,line cap=round,line join=round,line width=0.400pt]
        \path[draw] (127.5000,84.5000) -- (127.5000,8.5000);



      \end{scope}
      \begin{scope}[cm={{1.04439,0.0,0.0,1.41697,(-96.54445,-493.13085)}},draw=blue,line cap=round,line join=round,line width=0.480pt]
        \path[draw] (127.5000,84.5000) -- (127.5000,80.5000);



        \path[draw] (127.5000,8.5000) -- (127.5000,11.5000);



      \end{scope}
      \begin{scope}[cm={{1.05016,0.0,0.0,1.02141,(-35.68793,-461.26773)}},draw=blue,line cap=rect,line join=bevel,line width=0.800pt]
        \path[fill=blue] (0.0000,0.0000) node[above right] (text276) {\scriptsize $\Upsilon(t)$};



      \end{scope}
      \begin{scope}[cm={{0.76164,0.0,0.0,1.39271,(-67.04184,-490.7838)}},draw=blue,line cap=round,line join=round,line width=0.480pt]
        \path[draw,even odd rule] (71.5000,18.5000) -- (98.5000,18.5000);



      \end{scope}
      \begin{scope}[cm={{1.04439,0.0,0.0,1.41697,(-96.54445,-493.13085)}},draw=blue,line cap=round,line join=round,line width=0.480pt]
        \path[draw] (41.6000,17.0000) -- (41.6000,17.0000) -- (41.7000,17.8000) -- (41.9000,18.6000) -- (42.0000,19.4000) -- (42.2000,20.2000) -- (42.3000,20.9000) -- (42.5000,21.6000) -- (42.6000,22.4000) -- (42.7000,23.1000) -- (42.9000,23.8000) -- (43.0000,24.5000) -- (43.2000,25.2000) -- (43.3000,25.8000) -- (43.5000,26.5000) -- (43.6000,27.1000) -- (43.7000,27.7000) -- (43.9000,28.4000) -- (44.0000,29.0000) -- (44.2000,29.6000) -- (44.3000,30.1000) -- (44.5000,30.7000) -- (44.6000,31.3000) -- (44.7000,31.8000) -- (44.9000,32.4000) -- (45.0000,32.9000) -- (45.2000,33.4000) -- (45.3000,34.0000) -- (45.5000,34.5000) -- (45.6000,35.0000) -- (45.7000,35.5000) -- (45.9000,35.9000) -- (46.0000,36.4000) -- (46.2000,36.9000) -- (46.3000,37.3000) -- (46.5000,37.8000) -- (46.6000,38.2000) -- (46.7000,38.7000) -- (46.9000,39.1000) -- (47.0000,39.5000) -- (47.2000,39.9000) -- (47.3000,40.3000) -- (47.5000,40.7000) -- (47.6000,41.1000) -- (47.7000,41.5000) -- (47.9000,41.9000) -- (48.0000,42.2000) -- (48.2000,42.6000) -- (48.3000,43.0000) -- (48.5000,43.3000) -- (48.6000,43.6000) -- (48.7000,44.0000) -- (48.9000,44.3000) -- (49.0000,44.6000) -- (49.2000,45.0000) -- (49.3000,45.3000) -- (49.5000,45.6000) -- (49.6000,45.9000) -- (49.7000,46.2000) -- (49.9000,46.5000) -- (50.0000,46.8000) -- (50.2000,47.0000) -- (50.3000,47.3000) -- (50.5000,47.6000) -- (50.6000,47.8000) -- (50.7000,48.1000) -- (50.9000,48.4000) -- (51.0000,48.6000) -- (51.2000,48.8000) -- (51.3000,49.1000) -- (51.5000,49.3000) -- (51.6000,49.5000) -- (51.7000,49.8000) -- (51.9000,50.0000) -- (52.0000,50.2000) -- (52.2000,50.4000) -- (52.3000,50.6000) -- (52.5000,50.8000) -- (52.6000,51.0000) -- (52.7000,51.2000) -- (52.9000,51.4000) -- (53.0000,51.6000) -- (53.2000,51.7000) -- (53.3000,51.9000) -- (53.5000,52.1000) -- (53.6000,52.3000) -- (53.7000,52.4000) -- (53.9000,52.6000) -- (54.0000,52.7000) -- (54.2000,52.9000) -- (54.3000,53.0000) -- (54.5000,53.1000) -- (54.6000,53.3000) -- (54.7000,53.4000) -- (54.9000,53.5000) -- (55.0000,53.7000) -- (55.2000,53.8000) -- (55.3000,53.9000) -- (55.5000,54.0000) -- (55.6000,54.1000) -- (55.7000,54.2000) -- (55.9000,54.3000) -- (56.0000,54.4000) -- (56.2000,54.5000) -- (56.3000,54.6000) -- (56.5000,54.6000) -- (56.6000,54.7000) -- (56.7000,54.8000) -- (56.9000,54.9000) -- (57.0000,54.9000) -- (57.2000,55.0000) -- (57.3000,55.0000) -- (57.5000,55.1000) -- (57.6000,55.1000) -- (57.7000,55.2000) -- (57.9000,55.2000) -- (58.0000,55.2000) -- (58.2000,55.3000) -- (58.3000,55.3000) -- (58.5000,55.3000) -- (58.6000,55.3000) -- (58.7000,55.3000) -- (58.9000,55.3000) -- (59.0000,55.3000) -- (59.2000,55.3000) -- (59.3000,55.3000) -- (59.5000,55.3000) -- (59.6000,55.2000) -- (59.7000,55.2000) -- (59.9000,55.2000) -- (60.0000,55.1000) -- (60.2000,55.1000) -- (60.3000,55.0000) -- (60.5000,54.9000) -- (60.6000,54.9000) -- (60.7000,54.8000) -- (60.9000,54.7000) -- (61.0000,54.6000) -- (61.2000,54.5000) -- (61.3000,54.4000) -- (61.5000,54.2000) -- (61.6000,54.1000) -- (61.7000,54.0000) -- (61.9000,53.8000) -- (62.0000,53.7000) -- (62.2000,53.5000) -- (62.3000,53.3000) -- (62.5000,53.1000) -- (62.6000,53.0000) -- (62.7000,52.8000) -- (62.9000,52.6000) -- (63.0000,52.4000) -- (63.2000,52.2000) -- (63.3000,52.0000) -- (63.5000,51.7000) -- (63.6000,51.5000) -- (63.7000,51.3000) -- (63.9000,51.1000) -- (64.0000,50.8000) -- (64.2000,50.6000) -- (64.3000,50.4000) -- (64.5000,50.1000) -- (64.6000,49.9000) -- (64.7000,49.7000) -- (64.9000,49.4000) -- (65.0000,49.2000) -- (65.2000,49.0000) -- (65.3000,48.7000) -- (65.5000,48.5000) -- (65.6000,48.2000) -- (65.7000,48.0000) -- (65.9000,47.8000) -- (66.0000,47.5000) -- (66.2000,47.3000) -- (66.3000,47.1000) -- (66.5000,46.8000) -- (66.6000,46.6000) -- (66.7000,46.4000) -- (66.9000,46.1000) -- (67.0000,45.9000) -- (67.2000,45.7000) -- (67.3000,45.4000) -- (67.5000,45.2000) -- (67.6000,45.0000) -- (67.7000,44.8000) -- (67.9000,44.5000) -- (68.0000,44.3000) -- (68.2000,44.1000) -- (68.3000,43.9000) -- (68.5000,43.7000) -- (68.6000,43.5000) -- (68.7000,43.2000) -- (68.9000,43.0000) -- (69.0000,42.8000) -- (69.2000,42.6000) -- (69.3000,42.4000) -- (69.5000,42.2000) -- (69.6000,42.0000) -- (69.7000,41.8000) -- (69.9000,41.6000) -- (70.0000,41.4000) -- (70.2000,41.2000) -- (70.3000,41.0000) -- (70.5000,40.9000) -- (70.6000,40.7000) -- (70.7000,40.5000) -- (70.9000,40.3000) -- (71.0000,40.1000) -- (71.2000,40.0000) -- (71.3000,39.8000) -- (71.5000,39.6000) -- (71.6000,39.4000) -- (71.7000,39.3000) -- (71.9000,39.1000) -- (72.0000,38.9000) -- (72.2000,38.8000) -- (72.3000,38.6000) -- (72.5000,38.5000) -- (72.6000,38.3000) -- (72.7000,38.1000) -- (72.9000,38.0000) -- (73.0000,37.9000) -- (73.2000,37.7000) -- (73.3000,37.6000) -- (73.5000,37.4000) -- (73.6000,37.3000) -- (73.7000,37.1000) -- (73.9000,37.0000) -- (74.0000,36.9000) -- (74.2000,36.8000) -- (74.3000,36.6000) -- (74.5000,36.5000) -- (74.6000,36.4000) -- (74.7000,36.2000) -- (74.9000,36.1000) -- (75.0000,36.0000) -- (75.2000,35.9000) -- (75.3000,35.8000) -- (75.5000,35.7000) -- (75.6000,35.6000) -- (75.7000,35.5000) -- (75.9000,35.4000) -- (76.0000,35.3000) -- (76.2000,35.2000) -- (76.3000,35.1000) -- (76.5000,35.0000) -- (76.6000,34.9000) -- (76.7000,34.8000) -- (76.9000,34.7000) -- (77.0000,34.6000) -- (77.2000,34.5000) -- (77.3000,34.4000) -- (77.5000,34.4000) -- (77.6000,34.3000) -- (77.7000,34.2000) -- (77.9000,34.1000) -- (78.0000,34.1000) -- (78.2000,34.0000) -- (78.3000,33.9000) -- (78.5000,33.9000) -- (78.6000,33.8000) -- (78.7000,33.7000) -- (78.9000,33.7000) -- (79.0000,33.6000) -- (79.2000,33.6000) -- (79.3000,33.5000) -- (79.5000,33.5000) -- (79.6000,33.4000) -- (79.7000,33.4000) -- (79.9000,33.3000) -- (80.0000,33.3000) -- (80.2000,33.2000) -- (80.3000,33.2000) -- (80.5000,33.1000) -- (80.6000,33.1000) -- (80.7000,33.1000) -- (80.9000,33.0000) -- (81.0000,33.0000) -- (81.2000,33.0000) -- (81.3000,32.9000) -- (81.5000,32.9000) -- (81.6000,32.9000) -- (81.7000,32.9000) -- (81.9000,32.8000) -- (82.0000,32.8000) -- (82.2000,32.8000) -- (82.3000,32.8000) -- (82.5000,32.8000) -- (82.6000,32.7000) -- (82.7000,32.7000) -- (82.9000,32.7000) -- (83.0000,32.7000) -- (83.2000,32.7000) -- (83.3000,32.7000) -- (83.5000,32.7000) -- (83.6000,32.7000) -- (83.7000,32.7000) -- (83.9000,32.6000) -- (84.0000,32.6000) -- (84.2000,32.6000) -- (84.3000,32.6000) -- (84.5000,32.6000) -- (84.6000,32.6000) -- (84.7000,32.6000) -- (84.9000,32.7000) -- (85.0000,32.7000) -- (85.2000,32.7000) -- (85.3000,32.7000) -- (85.4000,32.7000) -- (85.6000,32.7000) -- (85.7000,32.7000) -- (85.9000,32.7000) -- (86.0000,32.7000) -- (86.2000,32.7000) -- (86.3000,32.8000) -- (86.4000,32.8000) -- (86.6000,32.8000) -- (86.7000,32.8000) -- (86.9000,32.8000) -- (87.0000,32.8000) -- (87.2000,32.9000) -- (87.3000,32.9000) -- (87.4000,32.9000) -- (87.6000,32.9000) -- (87.7000,33.0000) -- (87.9000,33.0000) -- (88.0000,33.0000) -- (88.2000,33.0000) -- (88.3000,33.1000) -- (88.4000,33.1000) -- (88.6000,33.1000) -- (88.7000,33.1000) -- (88.9000,33.2000) -- (89.0000,33.2000) -- (89.2000,33.2000) -- (89.3000,33.3000) -- (89.4000,33.3000) -- (89.6000,33.3000) -- (89.7000,33.4000) -- (89.9000,33.4000) -- (90.0000,33.4000) -- (90.2000,33.5000) -- (90.3000,33.5000) -- (90.4000,33.6000) -- (90.6000,33.6000) -- (90.7000,33.6000) -- (90.9000,33.7000) -- (91.0000,33.7000) -- (91.2000,33.8000) -- (91.3000,33.8000) -- (91.4000,33.8000) -- (91.6000,33.9000) -- (91.7000,33.9000) -- (91.9000,34.0000) -- (92.0000,34.0000) -- (92.2000,34.1000) -- (92.3000,34.1000) -- (92.4000,34.1000) -- (92.6000,34.2000) -- (92.7000,34.2000) -- (92.9000,34.3000) -- (93.0000,34.3000) -- (93.2000,34.4000) -- (93.3000,34.4000) -- (93.4000,34.5000) -- (93.6000,34.5000) -- (93.7000,34.6000) -- (93.9000,34.6000) -- (94.0000,34.7000) -- (94.2000,34.7000) -- (94.3000,34.8000) -- (94.4000,34.8000) -- (94.6000,34.9000) -- (94.7000,34.9000) -- (94.9000,35.0000) -- (95.0000,35.0000) -- (95.2000,35.1000) -- (95.3000,35.1000) -- (95.4000,35.2000) -- (95.6000,35.2000) -- (95.7000,35.3000) -- (95.9000,35.3000) -- (96.0000,35.4000) -- (96.2000,35.4000) -- (96.3000,35.5000) -- (96.4000,35.5000) -- (96.6000,35.6000) -- (96.7000,35.6000) -- (96.9000,35.7000) -- (97.0000,35.7000) -- (97.2000,35.8000) -- (97.3000,35.9000) -- (97.4000,35.9000) -- (97.6000,36.0000) -- (97.7000,36.0000) -- (97.9000,36.1000) -- (98.0000,36.1000) -- (98.2000,36.2000) -- (98.3000,36.2000) -- (98.4000,36.3000) -- (98.6000,36.4000) -- (98.7000,36.4000) -- (98.9000,36.5000) -- (99.0000,36.6000) -- (99.2000,36.6000) -- (99.3000,36.7000) -- (99.4000,36.8000) -- (99.6000,36.8000) -- (99.7000,36.9000) -- (99.9000,37.0000) -- (100.0000,37.0000) -- (100.2000,37.1000) -- (100.3000,37.2000) -- (100.4000,37.2000) -- (100.6000,37.3000) -- (100.7000,37.4000) -- (100.9000,37.4000) -- (101.0000,37.5000) -- (101.2000,37.6000) -- (101.3000,37.6000) -- (101.4000,37.7000) -- (101.6000,37.8000) -- (101.7000,37.9000) -- (101.9000,37.9000) -- (102.0000,38.0000) -- (102.2000,38.1000) -- (102.3000,38.2000) -- (102.4000,38.2000) -- (102.6000,38.3000) -- (102.7000,38.4000) -- (102.9000,38.4000) -- (103.0000,38.5000) -- (103.2000,38.6000) -- (103.3000,38.7000) -- (103.4000,38.7000) -- (103.6000,38.8000) -- (103.7000,38.9000) -- (103.9000,38.9000) -- (104.0000,39.0000) -- (104.2000,39.1000) -- (104.3000,39.2000) -- (104.4000,39.2000) -- (104.6000,39.3000) -- (104.7000,39.4000) -- (104.9000,39.4000) -- (105.0000,39.5000) -- (105.2000,39.6000) -- (105.3000,39.7000) -- (105.4000,39.7000) -- (105.6000,39.8000) -- (105.7000,39.9000) -- (105.9000,39.9000) -- (106.0000,40.0000) -- (106.2000,40.1000) -- (106.3000,40.1000) -- (106.4000,40.2000) -- (106.6000,40.3000) -- (106.7000,40.3000) -- (106.9000,40.4000) -- (107.0000,40.5000) -- (107.2000,40.5000) -- (107.3000,40.6000) -- (107.4000,40.6000) -- (107.6000,40.7000) -- (107.7000,40.8000) -- (107.9000,40.8000) -- (108.0000,40.9000) -- (108.2000,41.0000) -- (108.3000,41.0000) -- (108.4000,41.1000) -- (108.6000,41.1000) -- (108.7000,41.2000) -- (108.9000,41.3000) -- (109.0000,41.3000) -- (109.2000,41.4000) -- (109.3000,41.4000) -- (109.4000,41.5000) -- (109.6000,41.5000) -- (109.7000,41.6000) -- (109.9000,41.6000) -- (110.0000,41.7000) -- (110.2000,41.7000) -- (110.3000,41.8000) -- (110.4000,41.9000) -- (110.6000,41.9000) -- (110.7000,42.0000) -- (110.9000,42.0000) -- (111.0000,42.1000) -- (111.2000,42.1000) -- (111.3000,42.1000) -- (111.4000,42.2000) -- (111.6000,42.2000) -- (111.7000,42.3000) -- (111.9000,42.3000) -- (112.0000,42.4000) -- (112.2000,42.4000) -- (112.3000,42.5000) -- (112.4000,42.5000) -- (112.6000,42.6000) -- (112.7000,42.6000) -- (112.9000,42.6000) -- (113.0000,42.7000) -- (113.2000,42.7000) -- (113.3000,42.8000) -- (113.4000,42.8000) -- (113.6000,42.8000) -- (113.7000,42.9000) -- (113.9000,42.9000) -- (114.0000,43.0000) -- (114.2000,43.0000) -- (114.3000,43.0000) -- (114.4000,43.1000) -- (114.6000,43.1000) -- (114.7000,43.1000) -- (114.9000,43.2000) -- (115.0000,43.2000) -- (115.2000,43.2000) -- (115.3000,43.3000) -- (115.4000,43.3000) -- (115.6000,43.3000) -- (115.7000,43.4000) -- (115.9000,43.4000) -- (116.0000,43.4000) -- (116.2000,43.4000) -- (116.3000,43.5000) -- (116.4000,43.5000) -- (116.6000,43.5000) -- (116.7000,43.6000) -- (116.9000,43.6000) -- (117.0000,43.6000) -- (117.2000,43.6000) -- (117.3000,43.7000) -- (117.4000,43.7000) -- (117.6000,43.7000) -- (117.7000,43.7000) -- (117.9000,43.7000) -- (118.0000,43.8000) -- (118.2000,43.8000) -- (118.3000,43.8000) -- (118.4000,43.8000) -- (118.6000,43.9000) -- (118.7000,43.9000) -- (118.9000,43.9000) -- (119.0000,43.9000) -- (119.2000,43.9000) -- (119.3000,44.0000) -- (119.4000,44.0000) -- (119.6000,44.0000) -- (119.7000,44.0000) -- (119.9000,44.0000) -- (120.0000,44.0000) -- (120.2000,44.0000) -- (120.3000,44.1000) -- (120.4000,44.1000) -- (120.6000,44.1000) -- (120.7000,44.1000) -- (120.9000,44.1000) -- (121.0000,44.1000) -- (121.2000,44.1000) -- (121.3000,44.2000) -- (121.4000,44.2000) -- (121.6000,44.2000) -- (121.7000,44.2000) -- (121.9000,44.2000) -- (122.0000,44.2000) -- (122.2000,44.2000) -- (122.3000,44.2000) -- (122.4000,44.2000) -- (122.6000,44.2000) -- (122.7000,44.3000) -- (122.9000,44.3000) -- (123.0000,44.3000) -- (123.2000,44.3000) -- (123.3000,44.3000) -- (123.4000,44.3000) -- (123.6000,44.3000) -- (123.7000,44.3000) -- (123.9000,44.3000) -- (124.0000,44.3000) -- (124.2000,44.3000) -- (124.3000,44.3000) -- (124.4000,44.3000) -- (124.6000,44.3000) -- (124.7000,44.3000) -- (124.9000,44.3000) -- (125.0000,44.3000) -- (125.2000,44.3000) -- (125.3000,44.3000) -- (125.4000,44.4000) -- (125.6000,44.4000) -- (125.7000,44.4000) -- (125.9000,44.4000) -- (126.0000,44.4000) -- (126.2000,44.4000) -- (126.3000,44.4000) -- (126.4000,44.4000) -- (126.6000,44.4000) -- (126.7000,44.4000) -- (126.9000,44.4000) -- (127.0000,44.4000) -- (127.2000,44.4000) -- (127.3000,44.4000);



      \end{scope}
      \begin{scope}[cm={{1.05016,0.0,0.0,1.02141,(-33.29971,-448.94892)}},draw=blue,line cap=rect,line join=bevel,line width=0.800pt]
        \path[fill=blue] (0.0000,0.0000) node[above right] (text312) {\scriptsize $\hat{y}(t)$};



      \end{scope}
      \begin{scope}[cm={{1.04485,0.0,0.0,1.39271,(-87.20649,-490.7838)}},draw=cff0000,line cap=round,line join=round,line width=0.480pt]
        \path[draw,even odd rule,line width=0.410pt] (71.4187,26.5000) -- (91.1002,26.5000);



      \end{scope}
      \begin{scope}[cm={{1.04439,0.0,0.0,1.41697,(-96.54445,-493.13085)}},draw=cff0000,line cap=round,line join=round,line width=0.480pt]
        \path[draw] (41.6000,21.8000) -- (41.6000,21.8000) -- (41.7000,17.5000) -- (41.9000,18.2000) -- (42.0000,19.1000) -- (42.2000,20.0000) -- (42.3000,20.9000) -- (42.5000,21.7000) -- (42.6000,22.5000) -- (42.7000,23.3000) -- (42.9000,24.1000) -- (43.0000,24.8000) -- (43.2000,25.5000) -- (43.3000,26.2000) -- (43.5000,26.9000) -- (43.6000,27.6000) -- (43.7000,28.2000) -- (43.9000,28.8000) -- (44.0000,29.4000) -- (44.2000,30.0000) -- (44.3000,30.5000) -- (44.5000,31.1000) -- (44.6000,31.6000) -- (44.7000,32.1000) -- (44.9000,32.7000) -- (45.0000,33.2000) -- (45.2000,33.7000) -- (45.3000,34.2000) -- (45.5000,34.6000) -- (45.6000,35.1000) -- (45.7000,35.6000) -- (45.9000,36.1000) -- (46.0000,36.5000) -- (46.2000,37.0000) -- (46.3000,37.5000) -- (46.5000,37.9000) -- (46.6000,38.4000) -- (46.7000,38.8000) -- (46.9000,39.3000) -- (47.0000,39.7000) -- (47.2000,40.1000) -- (47.3000,40.6000) -- (47.5000,41.0000) -- (47.6000,41.4000) -- (47.7000,41.8000) -- (47.9000,42.2000) -- (48.0000,42.5000) -- (48.2000,42.9000) -- (48.3000,43.3000) -- (48.5000,43.6000) -- (48.6000,43.9000) -- (48.7000,44.3000) -- (48.9000,44.6000) -- (49.0000,44.9000) -- (49.2000,45.2000) -- (49.3000,45.5000) -- (49.5000,45.8000) -- (49.6000,46.0000) -- (49.7000,46.3000) -- (49.9000,46.6000) -- (50.0000,46.8000) -- (50.2000,47.1000) -- (50.3000,47.4000) -- (50.5000,47.6000) -- (50.6000,47.9000) -- (50.7000,48.1000) -- (50.9000,48.4000) -- (51.0000,48.7000) -- (51.2000,48.9000) -- (51.3000,49.2000) -- (51.5000,49.4000) -- (51.6000,49.7000) -- (51.7000,50.0000) -- (51.9000,50.2000) -- (52.0000,50.5000) -- (52.2000,50.8000) -- (52.3000,51.0000) -- (52.5000,51.3000) -- (52.6000,51.5000) -- (52.7000,51.7000) -- (52.9000,51.9000) -- (53.0000,52.1000) -- (53.2000,52.3000) -- (53.3000,52.5000) -- (53.5000,52.7000) -- (53.6000,52.8000) -- (53.7000,52.9000) -- (53.9000,53.1000) -- (54.0000,53.2000) -- (54.2000,53.3000) -- (54.3000,53.3000) -- (54.5000,53.4000) -- (54.6000,53.5000) -- (54.7000,53.5000) -- (54.9000,53.6000) -- (55.0000,53.6000) -- (55.2000,53.6000) -- (55.3000,53.7000) -- (55.5000,53.7000) -- (55.6000,53.7000) -- (55.7000,53.8000) -- (55.9000,53.8000) -- (56.0000,53.8000) -- (56.2000,53.9000) -- (56.3000,54.0000) -- (56.5000,54.0000) -- (56.6000,54.1000) -- (56.7000,54.2000) -- (56.9000,54.3000) -- (57.0000,54.4000) -- (57.2000,54.5000) -- (57.3000,54.6000) -- (57.5000,54.7000) -- (57.6000,54.8000) -- (57.7000,54.9000) -- (57.9000,55.0000) -- (58.0000,55.1000) -- (58.2000,55.2000) -- (58.3000,55.3000) -- (58.5000,55.3000) -- (58.6000,55.4000) -- (58.7000,55.4000) -- (58.9000,55.5000) -- (59.0000,55.5000) -- (59.2000,55.5000) -- (59.3000,55.5000) -- (59.5000,55.5000) -- (59.6000,55.5000) -- (59.7000,55.5000) -- (59.9000,55.4000) -- (60.0000,55.3000) -- (60.2000,55.3000) -- (60.3000,55.2000) -- (60.5000,55.1000) -- (60.6000,55.0000) -- (60.7000,54.9000) -- (60.9000,54.8000) -- (61.0000,54.6000) -- (61.2000,54.5000) -- (61.3000,54.4000) -- (61.5000,54.2000) -- (61.6000,54.0000) -- (61.7000,53.9000) -- (61.9000,53.7000) -- (62.0000,53.5000) -- (62.2000,53.3000) -- (62.3000,53.1000) -- (62.5000,52.9000) -- (62.6000,52.7000) -- (62.7000,52.5000) -- (62.9000,52.3000) -- (63.0000,52.1000) -- (63.2000,51.9000) -- (63.3000,51.7000) -- (63.5000,51.5000) -- (63.6000,51.3000) -- (63.7000,51.1000) -- (63.9000,50.9000) -- (64.0000,50.7000) -- (64.2000,50.5000) -- (64.3000,50.3000) -- (64.5000,50.0000) -- (64.6000,49.8000) -- (64.7000,49.6000) -- (64.9000,49.4000) -- (65.0000,49.2000) -- (65.2000,49.0000) -- (65.3000,48.8000) -- (65.5000,48.6000) -- (65.6000,48.3000) -- (65.7000,48.1000) -- (65.9000,47.9000) -- (66.0000,47.7000) -- (66.2000,47.5000) -- (66.3000,47.3000) -- (66.5000,47.0000) -- (66.6000,46.8000) -- (66.7000,46.6000) -- (66.9000,46.4000) -- (67.0000,46.2000) -- (67.2000,45.9000) -- (67.3000,45.7000) -- (67.5000,45.5000) -- (67.6000,45.3000) -- (67.7000,45.0000) -- (67.9000,44.8000) -- (68.0000,44.6000) -- (68.2000,44.3000) -- (68.3000,44.1000) -- (68.5000,43.9000) -- (68.6000,43.6000) -- (68.7000,43.4000) -- (68.9000,43.2000) -- (69.0000,43.0000) -- (69.2000,42.7000) -- (69.3000,42.5000) -- (69.5000,42.3000) -- (69.6000,42.1000) -- (69.7000,41.8000) -- (69.9000,41.6000) -- (70.0000,41.4000) -- (70.2000,41.2000) -- (70.3000,41.0000) -- (70.5000,40.8000) -- (70.6000,40.6000) -- (70.7000,40.4000) -- (70.9000,40.2000) -- (71.0000,40.0000) -- (71.2000,39.8000) -- (71.3000,39.6000) -- (71.5000,39.4000) -- (71.6000,39.2000) -- (71.7000,39.0000) -- (71.9000,38.9000) -- (72.0000,38.7000) -- (72.2000,38.5000) -- (72.3000,38.3000) -- (72.5000,38.2000) -- (72.6000,38.0000) -- (72.7000,37.9000) -- (72.9000,37.7000) -- (73.0000,37.6000) -- (73.2000,37.4000) -- (73.3000,37.3000) -- (73.5000,37.2000) -- (73.6000,37.0000) -- (73.7000,36.9000) -- (73.9000,36.8000) -- (74.0000,36.7000) -- (74.2000,36.5000) -- (74.3000,36.4000) -- (74.5000,36.3000) -- (74.6000,36.2000) -- (74.7000,36.1000) -- (74.9000,36.0000) -- (75.0000,35.9000) -- (75.2000,35.8000) -- (75.3000,35.7000) -- (75.5000,35.6000) -- (75.6000,35.5000) -- (75.7000,35.4000) -- (75.9000,35.3000) -- (76.0000,35.3000) -- (76.2000,35.2000) -- (76.3000,35.1000) -- (76.5000,35.0000) -- (76.6000,34.9000) -- (76.7000,34.9000) -- (76.9000,34.8000) -- (77.0000,34.7000) -- (77.2000,34.6000) -- (77.3000,34.6000) -- (77.5000,34.5000) -- (77.6000,34.4000) -- (77.7000,34.4000) -- (77.9000,34.3000) -- (78.0000,34.3000) -- (78.2000,34.2000) -- (78.3000,34.1000) -- (78.5000,34.1000) -- (78.6000,34.0000) -- (78.7000,34.0000) -- (78.9000,33.9000) -- (79.0000,33.9000) -- (79.2000,33.8000) -- (79.3000,33.7000) -- (79.5000,33.7000) -- (79.6000,33.6000) -- (79.7000,33.6000) -- (79.9000,33.5000) -- (80.0000,33.5000) -- (80.2000,33.5000) -- (80.3000,33.4000) -- (80.5000,33.4000) -- (80.6000,33.3000) -- (80.7000,33.3000) -- (80.9000,33.2000) -- (81.0000,33.2000) -- (81.2000,33.2000) -- (81.3000,33.1000) -- (81.5000,33.1000) -- (81.6000,33.0000) -- (81.7000,33.0000) -- (81.9000,33.0000) -- (82.0000,32.9000) -- (82.2000,32.9000) -- (82.3000,32.9000) -- (82.5000,32.9000) -- (82.6000,32.8000) -- (82.7000,32.8000) -- (82.9000,32.8000) -- (83.0000,32.8000) -- (83.2000,32.7000) -- (83.3000,32.7000) -- (83.5000,32.7000) -- (83.6000,32.7000) -- (83.7000,32.7000) -- (83.9000,32.7000) -- (84.0000,32.6000) -- (84.2000,32.6000) -- (84.3000,32.6000) -- (84.5000,32.6000) -- (84.6000,32.6000) -- (84.7000,32.6000) -- (84.9000,32.6000) -- (85.0000,32.6000) -- (85.2000,32.6000) -- (85.3000,32.6000) -- (85.4000,32.6000) -- (85.6000,32.6000) -- (85.7000,32.6000) -- (85.9000,32.6000) -- (86.0000,32.6000) -- (86.2000,32.6000) -- (86.3000,32.6000) -- (86.4000,32.7000) -- (86.6000,32.7000) -- (86.7000,32.7000) -- (86.9000,32.7000) -- (87.0000,32.7000) -- (87.2000,32.7000) -- (87.3000,32.8000) -- (87.4000,32.8000) -- (87.6000,32.8000) -- (87.7000,32.8000) -- (87.9000,32.8000) -- (88.0000,32.9000) -- (88.2000,32.9000) -- (88.3000,32.9000) -- (88.4000,33.0000) -- (88.6000,33.0000) -- (88.7000,33.0000) -- (88.9000,33.1000) -- (89.0000,33.1000) -- (89.2000,33.1000) -- (89.3000,33.2000) -- (89.4000,33.2000) -- (89.6000,33.2000) -- (89.7000,33.3000) -- (89.9000,33.3000) -- (90.0000,33.4000) -- (90.2000,33.4000) -- (90.3000,33.5000) -- (90.4000,33.5000) -- (90.6000,33.5000) -- (90.7000,33.6000) -- (90.9000,33.6000) -- (91.0000,33.7000) -- (91.2000,33.7000) -- (91.3000,33.8000) -- (91.4000,33.8000) -- (91.6000,33.9000) -- (91.7000,33.9000) -- (91.9000,34.0000) -- (92.0000,34.0000) -- (92.2000,34.1000) -- (92.3000,34.1000) -- (92.4000,34.2000) -- (92.6000,34.2000) -- (92.7000,34.3000) -- (92.9000,34.3000) -- (93.0000,34.4000) -- (93.2000,34.4000) -- (93.3000,34.5000) -- (93.4000,34.5000) -- (93.6000,34.6000) -- (93.7000,34.6000) -- (93.9000,34.7000) -- (94.0000,34.7000) -- (94.2000,34.8000) -- (94.3000,34.8000) -- (94.4000,34.9000) -- (94.6000,34.9000) -- (94.7000,35.0000) -- (94.9000,35.1000) -- (95.0000,35.1000) -- (95.2000,35.2000) -- (95.3000,35.2000) -- (95.4000,35.3000) -- (95.6000,35.3000) -- (95.7000,35.4000) -- (95.9000,35.4000) -- (96.0000,35.5000) -- (96.2000,35.5000) -- (96.3000,35.6000) -- (96.4000,35.6000) -- (96.6000,35.7000) -- (96.7000,35.7000) -- (96.9000,35.8000) -- (97.0000,35.9000) -- (97.2000,35.9000) -- (97.3000,36.0000) -- (97.4000,36.0000) -- (97.6000,36.1000) -- (97.7000,36.1000) -- (97.9000,36.2000) -- (98.0000,36.2000) -- (98.2000,36.3000) -- (98.3000,36.3000) -- (98.4000,36.4000) -- (98.6000,36.5000) -- (98.7000,36.5000) -- (98.9000,36.6000) -- (99.0000,36.6000) -- (99.2000,36.7000) -- (99.3000,36.8000) -- (99.4000,36.8000) -- (99.6000,36.9000) -- (99.7000,37.0000) -- (99.9000,37.0000) -- (100.0000,37.1000) -- (100.2000,37.1000) -- (100.3000,37.2000) -- (100.4000,37.3000) -- (100.6000,37.3000) -- (100.7000,37.4000) -- (100.9000,37.5000) -- (101.0000,37.5000) -- (101.2000,37.6000) -- (101.3000,37.7000) -- (101.4000,37.7000) -- (101.6000,37.8000) -- (101.7000,37.9000) -- (101.9000,37.9000) -- (102.0000,38.0000) -- (102.2000,38.1000) -- (102.3000,38.1000) -- (102.4000,38.2000) -- (102.6000,38.3000) -- (102.7000,38.4000) -- (102.9000,38.4000) -- (103.0000,38.5000) -- (103.2000,38.6000) -- (103.3000,38.6000) -- (103.4000,38.7000) -- (103.6000,38.8000) -- (103.7000,38.8000) -- (103.9000,38.9000) -- (104.0000,39.0000) -- (104.2000,39.0000) -- (104.3000,39.1000) -- (104.4000,39.2000) -- (104.6000,39.2000) -- (104.7000,39.3000) -- (104.9000,39.4000) -- (105.0000,39.4000) -- (105.2000,39.5000) -- (105.3000,39.6000) -- (105.4000,39.6000) -- (105.6000,39.7000) -- (105.7000,39.8000) -- (105.9000,39.8000) -- (106.0000,39.9000) -- (106.2000,40.0000) -- (106.3000,40.0000) -- (106.4000,40.1000) -- (106.6000,40.2000) -- (106.7000,40.2000) -- (106.9000,40.3000) -- (107.0000,40.4000) -- (107.2000,40.4000) -- (107.3000,40.5000) -- (107.4000,40.5000) -- (107.6000,40.6000) -- (107.7000,40.7000) -- (107.9000,40.7000) -- (108.0000,40.8000) -- (108.2000,40.9000) -- (108.3000,40.9000) -- (108.4000,41.0000) -- (108.6000,41.0000) -- (108.7000,41.1000) -- (108.9000,41.2000) -- (109.0000,41.2000) -- (109.2000,41.3000) -- (109.3000,41.3000) -- (109.4000,41.4000) -- (109.6000,41.4000) -- (109.7000,41.5000) -- (109.9000,41.5000) -- (110.0000,41.6000) -- (110.2000,41.7000) -- (110.3000,41.7000) -- (110.4000,41.8000) -- (110.6000,41.8000) -- (110.7000,41.9000) -- (110.9000,41.9000) -- (111.0000,42.0000) -- (111.2000,42.0000) -- (111.3000,42.1000) -- (111.4000,42.1000) -- (111.6000,42.2000) -- (111.7000,42.2000) -- (111.9000,42.3000) -- (112.0000,42.3000) -- (112.2000,42.4000) -- (112.3000,42.4000) -- (112.4000,42.5000) -- (112.6000,42.5000) -- (112.7000,42.6000) -- (112.9000,42.6000) -- (113.0000,42.6000) -- (113.2000,42.7000) -- (113.3000,42.7000) -- (113.4000,42.8000) -- (113.6000,42.8000) -- (113.7000,42.9000) -- (113.9000,42.9000) -- (114.0000,42.9000) -- (114.2000,43.0000) -- (114.3000,43.0000) -- (114.4000,43.1000) -- (114.6000,43.1000) -- (114.7000,43.1000) -- (114.9000,43.2000) -- (115.0000,43.2000) -- (115.2000,43.2000) -- (115.3000,43.3000) -- (115.4000,43.3000) -- (115.6000,43.3000) -- (115.7000,43.4000) -- (115.9000,43.4000) -- (116.0000,43.4000) -- (116.2000,43.5000) -- (116.3000,43.5000) -- (116.4000,43.5000) -- (116.6000,43.6000) -- (116.7000,43.6000) -- (116.9000,43.6000) -- (117.0000,43.6000) -- (117.2000,43.7000) -- (117.3000,43.7000) -- (117.4000,43.7000) -- (117.6000,43.7000) -- (117.7000,43.8000) -- (117.9000,43.8000) -- (118.0000,43.8000) -- (118.2000,43.8000) -- (118.3000,43.9000) -- (118.4000,43.9000) -- (118.6000,43.9000) -- (118.7000,43.9000) -- (118.9000,44.0000) -- (119.0000,44.0000) -- (119.2000,44.0000) -- (119.3000,44.0000) -- (119.4000,44.0000) -- (119.6000,44.1000) -- (119.7000,44.1000) -- (119.9000,44.1000) -- (120.0000,44.1000) -- (120.2000,44.1000) -- (120.3000,44.1000) -- (120.4000,44.1000) -- (120.6000,44.2000) -- (120.7000,44.2000) -- (120.9000,44.2000) -- (121.0000,44.2000) -- (121.2000,44.2000) -- (121.3000,44.2000) -- (121.4000,44.2000) -- (121.6000,44.3000) -- (121.7000,44.3000) -- (121.9000,44.3000) -- (122.0000,44.3000) -- (122.2000,44.3000) -- (122.3000,44.3000) -- (122.4000,44.3000) -- (122.6000,44.3000) -- (122.7000,44.3000) -- (122.9000,44.3000) -- (123.0000,44.3000) -- (123.2000,44.4000) -- (123.3000,44.4000) -- (123.4000,44.4000) -- (123.6000,44.4000) -- (123.7000,44.4000) -- (123.9000,44.4000) -- (124.0000,44.4000) -- (124.2000,44.4000) -- (124.3000,44.4000) -- (124.4000,44.4000) -- (124.6000,44.4000) -- (124.7000,44.4000) -- (124.9000,44.4000) -- (125.0000,44.4000) -- (125.2000,44.4000) -- (125.3000,44.4000) -- (125.4000,44.4000) -- (125.6000,44.4000) -- (125.7000,44.4000) -- (125.9000,44.4000) -- (126.0000,44.4000) -- (126.2000,44.4000) -- (126.3000,44.4000) -- (126.4000,44.4000) -- (126.6000,44.4000) -- (126.7000,44.4000) -- (126.9000,44.4000) -- (127.0000,44.4000) -- (127.2000,44.4000) -- (127.3000,44.4000);



      \end{scope}
      \begin{scope}[cm={{1.04439,0.0,0.0,1.41697,(-96.54445,-495.32852)}},draw=blue,line cap=round,line join=round,line width=0.480pt]
        \path[draw=blue] (41.5000,10.0509) -- (41.5000,86.0509) -- (127.5000,86.0509) -- (127.5000,10.0509) -- (41.5000,10.0509);



      \end{scope}
      \begin{scope}[cm={{1.27491,0.0,0.0,1.41542,(-124.58597,-493.0531)}},fill=cffffff]
        \path[fill,rounded corners=0.0000cm] (81.0000,50.0000) rectangle (121.0000,78.0000);



      \end{scope}
      \begin{scope}[cm={{1.27491,0.0,0.0,1.41542,(-124.58597,-493.0531)}},draw=ca0a0a4,dash pattern=on 0.93pt off 0.93pt,line cap=round,line join=round,line width=0.233pt,miter limit=4.00]
        \path[draw,dash pattern=on 0.93pt off 0.93pt,line width=0.233pt,miter limit=4.00] (81.5000,55.5000) -- (121.5000,55.5000);



      \end{scope}
      \begin{scope}[cm={{1.27491,0.0,0.0,1.41542,(-124.58597,-493.0531)}},draw=blue,line cap=round,line join=round,line width=0.480pt]
        \path[draw] (81.5000,55.5000) -- (82.6777,55.5000);



        \path[draw] (121.5000,55.5000) -- (120.1220,55.5000);



      \end{scope}
      \begin{scope}[cm={{1.27491,0.0,0.0,1.41542,(-124.58597,-493.0531)}},draw=ca0a0a4,dash pattern=on 0.93pt off 0.93pt,line cap=round,line join=round,line width=0.233pt,miter limit=4.00]
        \path[draw,dash pattern=on 0.93pt off 0.93pt,line width=0.233pt,miter limit=4.00] (97.5000,78.5000) -- (97.5000,50.5000);



      \end{scope}
      \begin{scope}[cm={{1.27491,0.0,0.0,1.41542,(-124.58597,-493.0531)}},draw=blue,line cap=round,line join=round,line width=0.480pt]
        \path[draw] (97.5000,50.5000) -- (97.5000,50.5000) -- (97.5000,51.6564);



      \end{scope}
      \begin{scope}[cm={{1.27491,0.0,0.0,1.41542,(-124.58597,-493.0531)}},draw=blue,line cap=round,line join=round,line width=0.480pt]
        \path[draw] (81.5000,50.5000) -- (81.5000,78.5000) -- (121.5000,78.5000) -- (121.5000,50.5000) -- (81.5000,50.5000);



      \end{scope}
      \begin{scope}[cm={{1.27491,0.0,0.0,1.41542,(-124.58597,-493.0531)}},draw=blue,line cap=round,line join=round,line width=0.480pt]
        \path[draw] (81.2000,77.5000) -- (81.2000,77.5000) -- (81.2000,77.5000) -- (81.2000,77.5000) -- (81.2000,77.5000) -- (81.2000,77.5000) -- (81.2000,77.5000) -- (81.2000,77.5000) -- (81.2000,77.5000) -- (81.2000,77.5000) -- (81.2000,77.5000) -- (81.2000,77.5000) -- (81.2000,77.5000) -- (81.2000,77.5000) -- (81.2000,77.5000) -- (81.2000,77.5000) -- (81.2000,77.4000) -- (81.2000,77.4000) -- (81.2000,77.4000) -- (81.2000,77.4000) -- (81.2000,77.4000) -- (81.2000,77.4000) -- (81.2000,77.4000) -- (81.2000,77.4000) -- (81.2000,77.4000) -- (81.2000,77.4000) -- (81.3000,77.4000) -- (81.3000,77.4000) -- (81.3000,77.4000) -- (81.3000,77.4000) -- (81.3000,77.4000) -- (81.3000,77.4000) -- (81.3000,77.4000) -- (81.3000,77.4000) -- (81.3000,77.4000) -- (81.3000,77.4000) -- (81.3000,77.4000) -- (81.3000,77.3000) -- (81.3000,77.3000) -- (81.3000,77.3000) -- (81.3000,77.3000) -- (81.3000,77.3000) -- (81.3000,77.3000) -- (81.3000,77.3000) -- (81.3000,77.3000) -- (81.3000,77.3000) -- (81.3000,77.3000) -- (81.3000,77.3000) -- (81.3000,77.3000) -- (81.3000,77.3000) -- (81.3000,77.3000) -- (81.3000,77.3000) -- (81.3000,77.3000) -- (81.3000,77.3000) -- (81.3000,77.3000) -- (81.3000,77.3000) -- (81.3000,77.3000) -- (81.3000,77.2000) -- (81.3000,77.2000) -- (81.3000,77.2000) -- (81.3000,77.2000) -- (81.3000,77.2000) -- (81.3000,77.2000) -- (81.3000,77.2000) -- (81.3000,77.2000) -- (81.3000,77.2000) -- (81.3000,77.2000) -- (81.3000,77.2000) -- (81.3000,77.2000) -- (81.3000,77.2000) -- (81.3000,77.2000) -- (81.3000,77.2000) -- (81.3000,77.2000) -- (81.3000,77.2000) -- (81.3000,77.2000) -- (81.3000,77.2000) -- (81.4000,77.2000) -- (81.4000,77.2000) -- (81.4000,77.1000) -- (81.4000,77.1000) -- (81.4000,77.1000) -- (81.4000,77.1000) -- (81.4000,77.1000) -- (81.4000,77.1000) -- (81.4000,77.1000) -- (81.4000,77.1000) -- (81.4000,77.1000) -- (81.4000,77.1000) -- (81.4000,77.1000) -- (81.4000,77.1000) -- (81.4000,77.1000) -- (81.4000,77.1000) -- (81.4000,77.1000) -- (81.4000,77.1000) -- (81.4000,77.1000) -- (81.4000,77.1000) -- (81.4000,77.1000) -- (81.4000,77.1000) -- (81.4000,77.1000) -- (81.4000,77.0000) -- (81.4000,77.0000) -- (81.4000,77.0000) -- (81.4000,77.0000) -- (81.4000,77.0000) -- (81.4000,77.0000) -- (81.4000,77.0000) -- (81.4000,77.0000) -- (81.4000,77.0000) -- (81.4000,77.0000) -- (81.4000,77.0000) -- (81.4000,77.0000) -- (81.4000,77.0000) -- (81.4000,77.0000) -- (81.4000,77.0000) -- (81.4000,77.0000) -- (81.4000,77.0000) -- (81.4000,77.0000) -- (81.4000,77.0000) -- (81.4000,77.0000) -- (81.4000,77.0000) -- (81.4000,76.9000) -- (81.4000,76.9000) -- (81.4000,76.9000) -- (81.4000,76.9000) -- (81.4000,76.9000) -- (81.5000,76.9000) -- (81.5000,76.9000) -- (81.5000,76.9000) -- (81.5000,76.9000) -- (81.5000,76.9000) -- (81.5000,76.9000) -- (81.5000,76.9000) -- (81.5000,76.9000) -- (81.5000,76.9000) -- (81.5000,76.9000) -- (81.5000,76.9000) -- (81.5000,76.9000) -- (81.5000,76.9000) -- (81.5000,76.9000) -- (81.5000,76.9000) -- (81.5000,76.9000) -- (81.5000,76.8000) -- (81.5000,76.8000) -- (81.5000,76.8000) -- (81.5000,76.8000) -- (81.5000,76.8000) -- (81.5000,76.8000) -- (81.5000,76.8000) -- (81.5000,76.8000) -- (81.5000,76.8000) -- (81.5000,76.8000) -- (81.5000,76.8000) -- (81.5000,76.8000) -- (81.5000,76.8000) -- (81.5000,76.8000) -- (81.5000,76.8000) -- (81.5000,76.8000) -- (81.5000,76.8000) -- (81.5000,76.8000) -- (81.5000,76.8000) -- (81.5000,76.8000) -- (81.5000,76.7000) -- (81.5000,76.7000) -- (81.5000,76.7000) -- (81.5000,76.7000) -- (81.5000,76.7000) -- (81.5000,76.7000) -- (81.5000,76.7000) -- (81.5000,76.7000) -- (81.5000,76.7000) -- (81.5000,76.7000) -- (81.5000,76.7000) -- (81.5000,76.7000) -- (81.5000,76.7000) -- (81.5000,76.7000) -- (81.6000,76.7000) -- (81.6000,76.7000) -- (81.6000,76.7000) -- (81.6000,76.7000) -- (81.6000,76.7000) -- (81.6000,76.7000) -- (81.6000,76.7000) -- (81.6000,76.6000) -- (81.6000,76.6000) -- (81.6000,76.6000) -- (81.6000,76.6000) -- (81.6000,76.6000) -- (81.6000,76.6000) -- (81.6000,76.6000) -- (81.6000,76.6000) -- (81.6000,76.6000) -- (81.6000,76.6000) -- (81.6000,76.6000) -- (81.6000,76.6000) -- (81.6000,76.6000) -- (81.6000,76.6000) -- (81.6000,76.6000) -- (81.6000,76.6000) -- (81.6000,76.6000) -- (81.6000,76.6000) -- (81.6000,76.6000) -- (81.6000,76.6000) -- (81.6000,76.6000) -- (81.6000,76.5000) -- (81.6000,76.5000) -- (81.6000,76.5000) -- (81.6000,76.5000) -- (81.6000,76.5000) -- (81.6000,76.5000) -- (81.6000,76.5000) -- (81.6000,76.5000) -- (81.6000,76.5000) -- (81.6000,76.5000) -- (81.6000,76.5000) -- (81.6000,76.5000) -- (81.6000,76.5000) -- (81.6000,76.5000) -- (81.6000,76.5000) -- (81.6000,76.5000) -- (81.6000,76.5000) -- (81.6000,76.5000) -- (81.6000,76.5000) -- (81.6000,76.5000) -- (81.6000,76.5000) -- (81.7000,76.4000) -- (81.7000,76.4000) -- (81.7000,76.4000) -- (81.7000,76.4000) -- (81.7000,76.4000) -- (81.7000,76.4000) -- (81.7000,76.4000) -- (81.7000,76.4000) -- (81.7000,76.4000) -- (81.7000,76.4000) -- (81.7000,76.4000) -- (81.7000,76.4000) -- (81.7000,76.4000) -- (81.7000,76.4000) -- (81.7000,76.4000) -- (81.7000,76.4000) -- (81.7000,76.4000) -- (81.7000,76.4000) -- (81.7000,76.4000) -- (81.7000,76.4000) -- (81.7000,76.4000) -- (81.7000,76.3000) -- (81.7000,76.3000) -- (81.7000,76.3000) -- (81.7000,76.3000) -- (81.7000,76.3000) -- (81.7000,76.3000) -- (81.7000,76.3000) -- (81.7000,76.3000) -- (81.7000,76.3000) -- (81.7000,76.3000) -- (81.7000,76.3000) -- (81.7000,76.3000) -- (81.7000,76.3000) -- (81.7000,76.3000) -- (81.7000,76.3000) -- (81.7000,76.3000) -- (81.7000,76.3000) -- (81.7000,76.3000) -- (81.7000,76.3000) -- (81.7000,76.3000) -- (81.7000,76.2000) -- (81.7000,76.2000) -- (81.7000,76.2000) -- (81.7000,76.2000) -- (81.7000,76.2000) -- (81.7000,76.2000) -- (81.7000,76.2000) -- (81.7000,76.2000) -- (81.7000,76.2000) -- (81.8000,76.2000) -- (81.8000,76.2000) -- (81.8000,76.2000) -- (81.8000,76.2000) -- (81.8000,76.2000) -- (81.8000,76.2000) -- (81.8000,76.2000) -- (81.8000,76.2000) -- (81.8000,76.2000) -- (81.8000,76.2000) -- (81.8000,76.2000) -- (81.8000,76.2000) -- (81.8000,76.1000) -- (81.8000,76.1000) -- (81.8000,76.1000) -- (81.8000,76.1000) -- (81.8000,76.1000) -- (81.8000,76.1000) -- (81.8000,76.1000) -- (81.8000,76.1000) -- (81.8000,76.1000) -- (81.8000,76.1000) -- (81.8000,76.1000) -- (81.8000,76.1000) -- (81.8000,76.1000) -- (81.8000,76.1000) -- (81.8000,76.1000) -- (81.8000,76.1000) -- (81.8000,76.1000) -- (81.8000,76.1000) -- (81.8000,76.1000) -- (81.8000,76.1000) -- (81.8000,76.1000) -- (81.8000,76.0000) -- (81.8000,76.0000) -- (81.8000,76.0000) -- (81.8000,76.0000) -- (81.8000,76.0000) -- (81.8000,76.0000) -- (81.8000,76.0000) -- (81.8000,76.0000) -- (81.8000,76.0000) -- (81.8000,76.0000) -- (81.8000,76.0000) -- (81.8000,76.0000) -- (81.8000,76.0000) -- (81.8000,76.0000) -- (81.8000,76.0000) -- (81.8000,76.0000) -- (81.9000,76.0000) -- (81.9000,76.0000) -- (81.9000,76.0000) -- (81.9000,76.0000) -- (81.9000,76.0000) -- (81.9000,75.9000) -- (81.9000,75.9000) -- (81.9000,75.9000) -- (81.9000,75.9000) -- (81.9000,75.9000) -- (81.9000,75.9000) -- (81.9000,75.9000) -- (81.9000,75.9000) -- (81.9000,75.9000) -- (81.9000,75.9000) -- (81.9000,75.9000) -- (81.9000,75.9000) -- (81.9000,75.9000) -- (81.9000,75.9000) -- (81.9000,75.9000) -- (81.9000,75.9000) -- (81.9000,75.9000) -- (81.9000,75.9000) -- (81.9000,75.9000) -- (81.9000,75.9000) -- (81.9000,75.8000) -- (81.9000,75.8000) -- (81.9000,75.8000) -- (81.9000,75.8000) -- (81.9000,75.8000) -- (81.9000,75.8000) -- (81.9000,75.8000) -- (81.9000,75.8000) -- (81.9000,75.8000) -- (81.9000,75.8000) -- (81.9000,75.8000) -- (81.9000,75.8000) -- (81.9000,75.8000) -- (81.9000,75.8000) -- (81.9000,75.8000) -- (81.9000,75.8000) -- (81.9000,75.8000) -- (81.9000,75.8000) -- (81.9000,75.8000) -- (81.9000,75.8000) -- (81.9000,75.8000) -- (81.9000,75.7000) -- (81.9000,75.7000) -- (81.9000,75.7000) -- (81.9000,75.7000) -- (82.0000,75.7000) -- (82.0000,75.7000) -- (82.0000,75.7000) -- (82.0000,75.7000) -- (82.0000,75.7000) -- (82.0000,75.7000) -- (82.0000,75.7000) -- (82.0000,75.7000) -- (82.0000,75.7000) -- (82.0000,75.7000) -- (82.0000,75.7000) -- (82.0000,75.7000) -- (82.0000,75.7000) -- (82.0000,75.7000) -- (82.0000,75.7000) -- (82.0000,75.7000) -- (82.0000,75.7000) -- (82.0000,75.6000) -- (82.0000,75.6000) -- (82.0000,75.6000) -- (82.0000,75.6000) -- (82.0000,75.6000) -- (82.0000,75.6000) -- (82.0000,75.6000) -- (82.0000,75.6000) -- (82.0000,75.6000) -- (82.0000,75.6000) -- (82.0000,75.6000) -- (82.0000,75.6000) -- (82.0000,75.6000) -- (82.0000,75.6000) -- (82.0000,75.6000) -- (82.0000,75.6000) -- (82.0000,75.6000) -- (82.0000,75.6000) -- (82.0000,75.6000) -- (82.0000,75.6000) -- (82.0000,75.6000) -- (82.0000,75.5000) -- (82.0000,75.5000) -- (82.0000,75.5000) -- (82.0000,75.5000) -- (82.0000,75.5000) -- (82.0000,75.5000) -- (82.0000,75.5000) -- (82.0000,75.5000) -- (82.0000,75.5000) -- (82.0000,75.5000) -- (82.0000,75.5000) -- (82.1000,75.5000) -- (82.1000,75.5000) -- (82.1000,75.5000) -- (82.1000,75.5000) -- (82.1000,75.5000) -- (82.1000,75.5000) -- (82.1000,75.5000) -- (82.1000,75.5000) -- (82.1000,75.5000) -- (82.1000,75.5000) -- (82.1000,75.4000) -- (82.1000,75.4000) -- (82.1000,75.4000) -- (82.1000,75.4000) -- (82.1000,75.4000) -- (82.1000,75.4000) -- (82.1000,75.4000) -- (82.1000,75.4000) -- (82.1000,75.4000) -- (82.1000,75.4000) -- (82.1000,75.4000) -- (82.1000,75.4000) -- (82.1000,75.4000) -- (82.1000,75.4000) -- (82.1000,75.4000) -- (82.1000,75.4000) -- (82.1000,75.4000) -- (82.1000,75.4000) -- (82.1000,75.4000) -- (82.1000,75.4000) -- (82.1000,75.3000) -- (82.1000,75.3000) -- (82.1000,75.3000) -- (82.1000,75.3000) -- (82.1000,75.3000) -- (82.1000,75.3000) -- (82.1000,75.3000) -- (82.1000,75.3000) -- (82.1000,75.3000) -- (82.1000,75.3000) -- (82.1000,75.3000) -- (82.1000,75.3000) -- (82.1000,75.3000) -- (82.1000,75.3000) -- (82.1000,75.3000) -- (82.1000,75.3000) -- (82.1000,75.3000) -- (82.1000,75.3000) -- (82.1000,75.3000) -- (82.1000,75.3000) -- (82.2000,75.3000) -- (82.2000,75.2000) -- (82.2000,75.2000) -- (82.2000,75.2000) -- (82.2000,75.2000) -- (82.2000,75.2000) -- (82.2000,75.2000) -- (82.2000,75.2000) -- (82.2000,75.2000) -- (82.2000,75.2000) -- (82.2000,75.2000) -- (82.2000,75.2000) -- (82.2000,75.2000) -- (82.2000,75.2000) -- (82.2000,75.2000) -- (82.2000,75.2000) -- (82.2000,75.2000) -- (82.2000,75.2000) -- (82.2000,75.2000) -- (82.2000,75.2000) -- (82.2000,75.2000) -- (82.2000,75.2000) -- (82.2000,75.1000) -- (82.2000,75.1000) -- (82.2000,75.1000) -- (82.2000,75.1000) -- (82.2000,75.1000) -- (82.2000,75.1000) -- (82.2000,75.1000) -- (82.2000,75.1000) -- (82.2000,75.1000) -- (82.2000,75.1000) -- (82.2000,75.1000) -- (82.2000,75.1000) -- (82.2000,75.1000) -- (82.2000,75.1000) -- (82.2000,75.1000) -- (82.2000,75.1000) -- (82.2000,75.1000) -- (82.2000,75.1000) -- (82.2000,75.1000) -- (82.2000,75.1000) -- (82.2000,75.1000) -- (82.2000,75.0000) -- (82.2000,75.0000) -- (82.2000,75.0000) -- (82.2000,75.0000) -- (82.2000,75.0000) -- (82.2000,75.0000) -- (82.3000,75.0000) -- (82.3000,75.0000) -- (82.3000,75.0000) -- (82.3000,75.0000) -- (82.3000,75.0000) -- (82.3000,75.0000) -- (82.3000,75.0000) -- (82.3000,75.0000) -- (82.3000,75.0000) -- (82.3000,75.0000) -- (82.3000,75.0000) -- (82.3000,75.0000) -- (82.3000,75.0000) -- (82.3000,75.0000) -- (82.3000,75.0000) -- (82.3000,74.9000) -- (82.3000,74.9000) -- (82.3000,74.9000) -- (82.3000,74.9000) -- (82.3000,74.9000) -- (82.3000,74.9000) -- (82.3000,74.9000) -- (82.3000,74.9000) -- (82.3000,74.9000) -- (82.3000,74.9000) -- (82.3000,74.9000) -- (82.3000,74.9000) -- (82.3000,74.9000) -- (82.3000,74.9000) -- (82.3000,74.9000) -- (82.3000,74.9000) -- (82.3000,74.9000) -- (82.3000,74.9000) -- (82.3000,74.9000) -- (82.3000,74.9000) -- (82.3000,74.8000) -- (82.3000,74.8000) -- (82.3000,74.8000) -- (82.3000,74.8000) -- (82.3000,74.8000) -- (82.3000,74.8000) -- (82.3000,74.8000) -- (82.3000,74.8000) -- (82.3000,74.8000) -- (82.3000,74.8000) -- (82.3000,74.8000) -- (82.3000,74.8000) -- (82.3000,74.8000) -- (82.3000,74.8000) -- (82.3000,74.8000) -- (82.4000,74.8000) -- (82.4000,74.8000) -- (82.4000,74.8000) -- (82.4000,74.8000) -- (82.4000,74.8000) -- (82.4000,74.8000) -- (82.4000,74.7000) -- (82.4000,74.7000) -- (82.4000,74.7000) -- (82.4000,74.7000) -- (82.4000,74.7000) -- (82.4000,74.7000) -- (82.4000,74.7000) -- (82.4000,74.7000) -- (82.4000,74.7000) -- (82.4000,74.7000) -- (82.4000,74.7000) -- (82.4000,74.7000) -- (82.4000,74.7000) -- (82.4000,74.7000) -- (82.4000,74.7000) -- (82.4000,74.7000) -- (82.4000,74.7000) -- (82.4000,74.7000) -- (82.4000,74.7000) -- (82.4000,74.7000) -- (82.4000,74.7000) -- (82.4000,74.6000) -- (82.4000,74.6000) -- (82.4000,74.6000) -- (82.4000,74.6000) -- (82.4000,74.6000) -- (82.4000,74.6000) -- (82.4000,74.6000) -- (82.4000,74.6000) -- (82.4000,74.6000) -- (82.4000,74.6000) -- (82.4000,74.6000) -- (82.4000,74.6000) -- (82.4000,74.6000) -- (82.4000,74.6000) -- (82.4000,74.6000) -- (82.4000,74.6000) -- (82.4000,74.6000) -- (82.4000,74.6000) -- (82.4000,74.6000) -- (82.4000,74.6000) -- (82.4000,74.6000) -- (82.4000,74.5000) -- (82.5000,74.5000) -- (82.5000,74.5000) -- (82.5000,74.5000) -- (82.5000,74.5000) -- (82.5000,74.5000) -- (82.5000,74.5000) -- (82.5000,74.5000) -- (82.5000,74.5000) -- (82.5000,74.5000) -- (82.5000,74.5000) -- (82.5000,74.5000) -- (82.5000,74.5000) -- (82.5000,74.5000) -- (82.5000,74.5000) -- (82.5000,74.5000) -- (82.5000,74.5000) -- (82.5000,74.5000) -- (82.5000,74.5000) -- (82.5000,74.5000) -- (82.5000,74.4000) -- (82.5000,74.4000) -- (82.5000,74.4000) -- (82.5000,74.4000) -- (82.5000,74.4000) -- (82.5000,74.4000) -- (82.5000,74.4000) -- (82.5000,74.4000) -- (82.5000,74.4000) -- (82.5000,74.4000) -- (82.5000,74.4000) -- (82.5000,74.4000) -- (82.5000,74.4000) -- (82.5000,74.4000) -- (82.5000,74.4000) -- (82.5000,74.4000) -- (82.5000,74.4000) -- (82.5000,74.4000) -- (82.5000,74.4000) -- (82.5000,74.4000) -- (82.5000,74.4000) -- (82.5000,74.3000) -- (82.5000,74.3000) -- (82.5000,74.3000) -- (82.5000,74.3000) -- (82.5000,74.3000) -- (82.5000,74.3000) -- (82.5000,74.3000) -- (82.5000,74.3000) -- (82.5000,74.3000) -- (82.5000,74.3000) -- (82.6000,74.3000) -- (82.6000,74.3000) -- (82.6000,74.3000) -- (82.6000,74.3000) -- (82.6000,74.3000) -- (82.6000,74.3000) -- (82.6000,74.3000) -- (82.6000,74.3000) -- (82.6000,74.3000) -- (82.6000,74.3000) -- (82.6000,74.3000) -- (82.6000,74.2000) -- (82.6000,74.2000) -- (82.6000,74.2000) -- (82.6000,74.2000) -- (82.6000,74.2000) -- (82.6000,74.2000) -- (82.6000,74.2000) -- (82.6000,74.2000) -- (82.6000,74.2000) -- (82.6000,74.2000) -- (82.6000,74.2000) -- (82.6000,74.2000) -- (82.6000,74.2000) -- (82.6000,74.2000) -- (82.6000,74.2000) -- (82.6000,74.2000) -- (82.6000,74.2000) -- (82.6000,74.2000) -- (82.6000,74.2000) -- (82.6000,74.2000) -- (82.6000,74.2000) -- (82.6000,74.1000) -- (82.6000,74.1000) -- (82.6000,74.1000) -- (82.6000,74.1000) -- (82.6000,74.1000) -- (82.6000,74.1000) -- (82.6000,74.1000) -- (82.6000,74.1000) -- (82.6000,74.1000) -- (82.6000,74.1000) -- (82.6000,74.1000) -- (82.6000,74.1000) -- (82.6000,74.1000) -- (82.6000,74.1000) -- (82.6000,74.1000) -- (82.6000,74.1000) -- (82.6000,74.1000) -- (82.7000,74.1000) -- (82.7000,74.1000) -- (82.7000,74.1000) -- (82.7000,74.1000) -- (82.7000,74.0000) -- (82.7000,74.0000) -- (82.7000,74.0000) -- (82.7000,74.0000) -- (82.7000,74.0000) -- (82.7000,74.0000) -- (82.7000,74.0000) -- (82.7000,74.0000) -- (82.7000,74.0000) -- (82.7000,74.0000) -- (82.7000,74.0000) -- (82.7000,74.0000) -- (82.7000,74.0000) -- (82.7000,74.0000) -- (82.7000,74.0000) -- (82.7000,74.0000) -- (82.7000,74.0000) -- (82.7000,74.0000) -- (82.7000,74.0000) -- (82.7000,74.0000) -- (82.7000,73.9000) -- (82.7000,73.9000) -- (82.7000,73.9000) -- (82.7000,73.9000) -- (82.7000,73.9000) -- (82.7000,73.9000) -- (82.7000,73.9000) -- (82.7000,73.9000) -- (82.7000,73.9000) -- (82.7000,73.9000) -- (82.7000,73.9000) -- (82.7000,73.9000) -- (82.7000,73.9000) -- (82.7000,73.9000) -- (82.7000,73.9000) -- (82.7000,73.9000) -- (82.7000,73.9000) -- (82.7000,73.9000) -- (82.7000,73.9000) -- (82.7000,73.9000) -- (82.7000,73.9000) -- (82.7000,73.8000) -- (82.7000,73.8000) -- (82.7000,73.8000) -- (82.7000,73.8000) -- (82.7000,73.8000) -- (82.8000,73.8000) -- (82.8000,73.8000) -- (82.8000,73.8000) -- (82.8000,73.8000) -- (82.8000,73.8000) -- (82.8000,73.8000) -- (82.8000,73.8000) -- (82.8000,73.8000) -- (82.8000,73.8000) -- (82.8000,73.8000) -- (82.8000,73.8000) -- (82.8000,73.8000) -- (82.8000,73.8000) -- (82.8000,73.8000) -- (82.8000,73.8000) -- (82.8000,73.8000) -- (82.8000,73.7000) -- (82.8000,73.7000) -- (82.8000,73.7000) -- (82.8000,73.7000) -- (82.8000,73.7000) -- (82.8000,73.7000) -- (82.8000,73.7000) -- (82.8000,73.7000) -- (82.8000,73.7000) -- (82.8000,73.7000) -- (82.8000,73.7000) -- (82.8000,73.7000) -- (82.8000,73.7000) -- (82.8000,73.7000) -- (82.8000,73.7000) -- (82.8000,73.7000) -- (82.8000,73.7000) -- (82.8000,73.7000) -- (82.8000,73.7000) -- (82.8000,73.7000) -- (82.8000,73.7000) -- (82.8000,73.6000) -- (82.8000,73.6000) -- (82.8000,73.6000) -- (82.8000,73.6000) -- (82.8000,73.6000) -- (82.8000,73.6000) -- (82.8000,73.6000) -- (82.8000,73.6000) -- (82.8000,73.6000) -- (82.8000,73.6000) -- (82.8000,73.6000) -- (82.8000,73.6000) -- (82.9000,73.6000) -- (82.9000,73.6000) -- (82.9000,73.6000) -- (82.9000,73.6000) -- (82.9000,73.6000) -- (82.9000,73.6000) -- (82.9000,73.6000) -- (82.9000,73.6000) -- (82.9000,73.6000) -- (82.9000,73.5000) -- (82.9000,73.5000) -- (82.9000,73.5000) -- (82.9000,73.5000) -- (82.9000,73.5000) -- (82.9000,73.5000) -- (82.9000,73.5000) -- (82.9000,73.5000) -- (82.9000,73.5000) -- (82.9000,73.5000) -- (82.9000,73.5000) -- (82.9000,73.5000) -- (82.9000,73.5000) -- (82.9000,73.5000) -- (82.9000,73.5000) -- (82.9000,73.5000) -- (82.9000,73.5000) -- (82.9000,73.5000) -- (82.9000,73.5000) -- (82.9000,73.5000) -- (82.9000,73.4000) -- (82.9000,73.4000) -- (82.9000,73.4000) -- (82.9000,73.4000) -- (82.9000,73.4000) -- (82.9000,73.4000) -- (82.9000,73.4000) -- (82.9000,73.4000) -- (82.9000,73.4000) -- (82.9000,73.4000) -- (82.9000,73.4000) -- (82.9000,73.4000) -- (82.9000,73.4000) -- (82.9000,73.4000) -- (82.9000,73.4000) -- (82.9000,73.4000) -- (82.9000,73.4000) -- (82.9000,73.4000) -- (82.9000,73.4000) -- (82.9000,73.4000) -- (82.9000,73.4000) -- (83.0000,73.3000) -- (83.0000,73.3000) -- (83.0000,73.3000) -- (83.0000,73.3000) -- (83.0000,73.3000) -- (83.0000,73.3000) -- (83.0000,73.3000) -- (83.0000,73.3000) -- (83.0000,73.3000) -- (83.0000,73.3000) -- (83.0000,73.3000) -- (83.0000,73.3000) -- (83.0000,73.3000) -- (83.0000,73.3000) -- (83.0000,73.3000) -- (83.0000,73.3000) -- (83.0000,73.3000) -- (83.0000,73.3000) -- (83.0000,73.3000) -- (83.0000,73.3000) -- (83.0000,73.3000) -- (83.0000,73.2000) -- (83.0000,73.2000) -- (83.0000,73.2000) -- (83.0000,73.2000) -- (83.0000,73.2000) -- (83.0000,73.2000) -- (83.0000,73.2000) -- (83.0000,73.2000) -- (83.0000,73.2000) -- (83.0000,73.2000) -- (83.0000,73.2000) -- (83.0000,73.2000) -- (83.0000,73.2000) -- (83.0000,73.2000) -- (83.0000,73.2000) -- (83.0000,73.2000) -- (83.0000,73.2000) -- (83.0000,73.2000) -- (83.0000,73.2000) -- (83.0000,73.2000) -- (83.0000,73.2000) -- (83.0000,73.1000) -- (83.0000,73.1000) -- (83.0000,73.1000) -- (83.0000,73.1000) -- (83.0000,73.1000) -- (83.0000,73.1000) -- (83.0000,73.1000) -- (83.1000,73.1000) -- (83.1000,73.1000) -- (83.1000,73.1000) -- (83.1000,73.1000) -- (83.1000,73.1000) -- (83.1000,73.1000) -- (83.1000,73.1000) -- (83.1000,73.1000) -- (83.1000,73.1000) -- (83.1000,73.1000) -- (83.1000,73.1000) -- (83.1000,73.1000) -- (83.1000,73.1000) -- (83.1000,73.1000) -- (83.1000,73.0000) -- (83.1000,73.0000) -- (83.1000,73.0000) -- (83.1000,73.0000) -- (83.1000,73.0000) -- (83.1000,73.0000) -- (83.1000,73.0000) -- (83.1000,73.0000) -- (83.1000,73.0000) -- (83.1000,73.0000) -- (83.1000,73.0000) -- (83.1000,73.0000) -- (83.1000,73.0000) -- (83.1000,73.0000) -- (83.1000,73.0000) -- (83.1000,73.0000) -- (83.1000,73.0000) -- (83.1000,73.0000) -- (83.1000,73.0000) -- (83.1000,73.0000) -- (83.1000,72.9000) -- (83.1000,72.9000) -- (83.1000,72.9000) -- (83.1000,72.9000) -- (83.1000,72.9000) -- (83.1000,72.9000) -- (83.1000,72.9000) -- (83.1000,72.9000) -- (83.1000,72.9000) -- (83.1000,72.9000) -- (83.1000,72.9000) -- (83.1000,72.9000) -- (83.1000,72.9000) -- (83.1000,72.9000) -- (83.1000,72.9000) -- (83.1000,72.9000) -- (83.2000,72.9000) -- (83.2000,72.9000) -- (83.2000,72.9000) -- (83.2000,72.9000) -- (83.2000,72.9000) -- (83.2000,72.8000) -- (83.2000,72.8000) -- (83.2000,72.8000) -- (83.2000,72.8000) -- (83.2000,72.8000) -- (83.2000,72.8000) -- (83.2000,72.8000) -- (83.2000,72.8000) -- (83.2000,72.8000) -- (83.2000,72.8000) -- (83.2000,72.8000) -- (83.2000,72.8000) -- (83.2000,72.8000) -- (83.2000,72.8000) -- (83.2000,72.8000) -- (83.2000,72.8000) -- (83.2000,72.8000) -- (83.2000,72.8000) -- (83.2000,72.8000) -- (83.2000,72.8000) -- (83.2000,72.8000) -- (83.2000,72.7000) -- (83.2000,72.7000) -- (83.2000,72.7000) -- (83.2000,72.7000) -- (83.2000,72.7000) -- (83.2000,72.7000) -- (83.2000,72.7000) -- (83.2000,72.7000) -- (83.2000,72.7000) -- (83.2000,72.7000) -- (83.2000,72.7000) -- (83.2000,72.7000) -- (83.2000,72.7000) -- (83.2000,72.7000) -- (83.2000,72.7000) -- (83.2000,72.7000) -- (83.2000,72.7000) -- (83.2000,72.7000) -- (83.2000,72.7000) -- (83.2000,72.7000) -- (83.2000,72.7000) -- (83.2000,72.6000) -- (83.2000,72.6000) -- (83.3000,72.6000) -- (83.3000,72.6000) -- (83.3000,72.6000) -- (83.3000,72.6000) -- (83.3000,72.6000) -- (83.3000,72.6000) -- (83.3000,72.6000) -- (83.3000,72.6000) -- (83.3000,72.6000) -- (83.3000,72.6000) -- (83.3000,72.6000) -- (83.3000,72.6000) -- (83.3000,72.6000) -- (83.3000,72.6000) -- (83.3000,72.6000) -- (83.3000,72.6000) -- (83.3000,72.6000) -- (83.3000,72.6000) -- (83.3000,72.5000) -- (83.3000,72.5000) -- (83.3000,72.5000) -- (83.3000,72.5000) -- (83.3000,72.5000) -- (83.3000,72.5000) -- (83.3000,72.5000) -- (83.3000,72.5000) -- (83.3000,72.5000) -- (83.3000,72.5000) -- (83.3000,72.5000) -- (83.3000,72.5000) -- (83.3000,72.5000) -- (83.3000,72.5000) -- (83.3000,72.5000) -- (83.3000,72.5000) -- (83.3000,72.5000) -- (83.3000,72.5000) -- (83.3000,72.5000) -- (83.3000,72.5000) -- (83.3000,72.5000) -- (83.3000,72.4000) -- (83.3000,72.4000) -- (83.3000,72.4000) -- (83.3000,72.4000) -- (83.3000,72.4000) -- (83.3000,72.4000) -- (83.3000,72.4000) -- (83.3000,72.4000) -- (83.3000,72.4000) -- (83.3000,72.4000) -- (83.3000,72.4000) -- (83.4000,72.4000) -- (83.4000,72.4000) -- (83.4000,72.4000) -- (83.4000,72.4000) -- (83.4000,72.4000) -- (83.4000,72.4000) -- (83.4000,72.4000) -- (83.4000,72.4000) -- (83.4000,72.4000) -- (83.4000,72.4000) -- (83.4000,72.3000) -- (83.4000,72.3000) -- (83.4000,72.3000) -- (83.4000,72.3000) -- (83.4000,72.3000) -- (83.4000,72.3000) -- (83.4000,72.3000) -- (83.4000,72.3000) -- (83.4000,72.3000) -- (83.4000,72.3000) -- (83.4000,72.3000) -- (83.4000,72.3000) -- (83.4000,72.3000) -- (83.4000,72.3000) -- (83.4000,72.3000) -- (83.4000,72.3000) -- (83.4000,72.3000) -- (83.4000,72.3000) -- (83.4000,72.3000) -- (83.4000,72.3000) -- (83.4000,72.3000) -- (83.4000,72.2000) -- (83.4000,72.2000) -- (83.4000,72.2000) -- (83.4000,72.2000) -- (83.4000,72.2000) -- (83.4000,72.2000) -- (83.4000,72.2000) -- (83.4000,72.2000) -- (83.4000,72.2000) -- (83.4000,72.2000) -- (83.4000,72.2000) -- (83.4000,72.2000) -- (83.4000,72.2000) -- (83.4000,72.2000) -- (83.4000,72.2000) -- (83.4000,72.2000) -- (83.4000,72.2000) -- (83.4000,72.2000) -- (83.5000,72.2000) -- (83.5000,72.2000) -- (83.5000,72.2000) -- (83.5000,72.1000) -- (83.5000,72.1000) -- (83.5000,72.1000) -- (83.5000,72.1000) -- (83.5000,72.1000) -- (83.5000,72.1000) -- (83.5000,72.1000) -- (83.5000,72.1000) -- (83.5000,72.1000) -- (83.5000,72.1000) -- (83.5000,72.1000) -- (83.5000,72.1000) -- (83.5000,72.1000) -- (83.5000,72.1000) -- (83.5000,72.1000) -- (83.5000,72.1000) -- (83.5000,72.1000) -- (83.5000,72.1000) -- (83.5000,72.1000) -- (83.5000,72.1000) -- (83.5000,72.0000) -- (83.5000,72.0000) -- (83.5000,72.0000) -- (83.5000,72.0000) -- (83.5000,72.0000) -- (83.5000,72.0000) -- (83.5000,72.0000) -- (83.5000,72.0000) -- (83.5000,72.0000) -- (83.5000,72.0000) -- (83.5000,72.0000) -- (83.5000,72.0000) -- (83.5000,72.0000) -- (83.5000,72.0000) -- (83.5000,72.0000) -- (83.5000,72.0000) -- (83.5000,72.0000) -- (83.5000,72.0000) -- (83.5000,72.0000) -- (83.5000,72.0000) -- (83.5000,72.0000) -- (83.5000,71.9000) -- (83.5000,71.9000) -- (83.5000,71.9000) -- (83.5000,71.9000) -- (83.5000,71.9000) -- (83.5000,71.9000) -- (83.6000,71.9000) -- (83.6000,71.9000) -- (83.6000,71.9000) -- (83.6000,71.9000) -- (83.6000,71.9000) -- (83.6000,71.9000) -- (83.6000,71.9000) -- (83.6000,71.9000) -- (83.6000,71.9000) -- (83.6000,71.9000) -- (83.6000,71.9000) -- (83.6000,71.9000) -- (83.6000,71.9000) -- (83.6000,71.9000) -- (83.6000,71.9000) -- (83.6000,71.8000) -- (83.6000,71.8000) -- (83.6000,71.8000) -- (83.6000,71.8000) -- (83.6000,71.8000) -- (83.6000,71.8000) -- (83.6000,71.8000) -- (83.6000,71.8000) -- (83.6000,71.8000) -- (83.6000,71.8000) -- (83.6000,71.8000) -- (83.6000,71.8000) -- (83.6000,71.8000) -- (83.6000,71.8000) -- (83.6000,71.8000) -- (83.6000,71.8000) -- (83.6000,71.8000) -- (83.6000,71.8000) -- (83.6000,71.8000) -- (83.6000,71.8000) -- (83.6000,71.8000) -- (83.6000,71.7000) -- (83.6000,71.7000) -- (83.6000,71.7000) -- (83.6000,71.7000) -- (83.6000,71.7000) -- (83.6000,71.7000) -- (83.6000,71.7000) -- (83.6000,71.7000) -- (83.6000,71.7000) -- (83.6000,71.7000) -- (83.6000,71.7000) -- (83.6000,71.7000) -- (83.6000,71.7000) -- (83.7000,71.7000) -- (83.7000,71.7000) -- (83.7000,71.7000) -- (83.7000,71.7000) -- (83.7000,71.7000) -- (83.7000,71.7000) -- (83.7000,71.7000) -- (83.7000,71.7000) -- (83.7000,71.6000) -- (83.7000,71.6000) -- (83.7000,71.6000) -- (83.7000,71.6000) -- (83.7000,71.6000) -- (83.7000,71.6000) -- (83.7000,71.6000) -- (83.7000,71.6000) -- (83.7000,71.6000) -- (83.7000,71.6000) -- (83.7000,71.6000) -- (83.7000,71.6000) -- (83.7000,71.6000) -- (83.7000,71.6000) -- (83.7000,71.6000) -- (83.7000,71.6000) -- (83.7000,71.6000) -- (83.7000,71.6000) -- (83.7000,71.6000) -- (83.7000,71.6000) -- (83.7000,71.5000) -- (83.7000,71.5000) -- (83.7000,71.5000) -- (83.7000,71.5000) -- (83.7000,71.5000) -- (83.7000,71.5000) -- (83.7000,71.5000) -- (83.7000,71.5000) -- (83.7000,71.5000) -- (83.7000,71.5000) -- (83.7000,71.5000) -- (83.7000,71.5000) -- (83.7000,71.5000) -- (83.7000,71.5000) -- (83.7000,71.5000) -- (83.7000,71.5000) -- (83.7000,71.5000) -- (83.7000,71.5000) -- (83.7000,71.5000) -- (83.7000,71.5000) -- (83.7000,71.5000) -- (83.7000,71.4000) -- (83.8000,71.4000) -- (83.8000,71.4000) -- (83.8000,71.4000) -- (83.8000,71.4000) -- (83.8000,71.4000) -- (83.8000,71.4000) -- (83.8000,71.4000) -- (83.8000,71.4000) -- (83.8000,71.4000) -- (83.8000,71.4000) -- (83.8000,71.4000) -- (83.8000,71.4000) -- (83.8000,71.4000) -- (83.8000,71.4000) -- (83.8000,71.4000) -- (83.8000,71.4000) -- (83.8000,71.4000) -- (83.8000,71.4000) -- (83.8000,71.4000) -- (83.8000,71.4000) -- (83.8000,71.3000) -- (83.8000,71.3000) -- (83.8000,71.3000) -- (83.8000,71.3000) -- (83.8000,71.3000) -- (83.8000,71.3000) -- (83.8000,71.3000) -- (83.8000,71.3000) -- (83.8000,71.3000) -- (83.8000,71.3000) -- (83.8000,71.3000) -- (83.8000,71.3000) -- (83.8000,71.3000) -- (83.8000,71.3000) -- (83.8000,71.3000) -- (83.8000,71.3000) -- (83.8000,71.3000) -- (83.8000,71.3000) -- (83.8000,71.3000) -- (83.8000,71.3000) -- (83.8000,71.3000) -- (83.8000,71.2000) -- (83.8000,71.2000) -- (83.8000,71.2000) -- (83.8000,71.2000) -- (83.8000,71.2000) -- (83.8000,71.2000) -- (83.8000,71.2000) -- (83.8000,71.2000) -- (83.9000,71.2000) -- (83.9000,71.2000) -- (83.9000,71.2000) -- (83.9000,71.2000) -- (83.9000,71.2000) -- (83.9000,71.2000) -- (83.9000,71.2000) -- (83.9000,71.2000) -- (83.9000,71.2000) -- (83.9000,71.2000) -- (83.9000,71.2000) -- (83.9000,71.2000) -- (83.9000,71.2000) -- (83.9000,71.1000) -- (83.9000,71.1000) -- (83.9000,71.1000) -- (83.9000,71.1000) -- (83.9000,71.1000) -- (83.9000,71.1000) -- (83.9000,71.1000) -- (83.9000,71.1000) -- (83.9000,71.1000) -- (83.9000,71.1000) -- (83.9000,71.1000) -- (83.9000,71.1000) -- (83.9000,71.1000) -- (83.9000,71.1000) -- (83.9000,71.1000) -- (83.9000,71.1000) -- (83.9000,71.1000) -- (83.9000,71.1000) -- (83.9000,71.1000) -- (83.9000,71.1000) -- (83.9000,71.0000) -- (83.9000,71.0000) -- (83.9000,71.0000) -- (83.9000,71.0000) -- (83.9000,71.0000) -- (83.9000,71.0000) -- (83.9000,71.0000) -- (83.9000,71.0000) -- (83.9000,71.0000) -- (83.9000,71.0000) -- (83.9000,71.0000) -- (83.9000,71.0000) -- (83.9000,71.0000) -- (83.9000,71.0000) -- (83.9000,71.0000) -- (83.9000,71.0000) -- (83.9000,71.0000) -- (84.0000,71.0000) -- (84.0000,71.0000) -- (84.0000,71.0000) -- (84.0000,71.0000) -- (84.0000,70.9000) -- (84.0000,70.9000) -- (84.0000,70.9000) -- (84.0000,70.9000) -- (84.0000,70.9000) -- (84.0000,70.9000) -- (84.0000,70.9000) -- (84.0000,70.9000) -- (84.0000,70.9000) -- (84.0000,70.9000) -- (84.0000,70.9000) -- (84.0000,70.9000) -- (84.0000,70.9000) -- (84.0000,70.9000) -- (84.0000,70.9000) -- (84.0000,70.9000) -- (84.0000,70.9000) -- (84.0000,70.9000) -- (84.0000,70.9000) -- (84.0000,70.9000) -- (84.0000,70.9000) -- (84.0000,70.8000) -- (84.0000,70.8000) -- (84.0000,70.8000) -- (84.0000,70.8000) -- (84.0000,70.8000) -- (84.0000,70.8000) -- (84.0000,70.8000) -- (84.0000,70.8000) -- (84.0000,70.8000) -- (84.0000,70.8000) -- (84.0000,70.8000) -- (84.0000,70.8000) -- (84.0000,70.8000) -- (84.0000,70.8000) -- (84.0000,70.8000) -- (84.0000,70.8000) -- (84.0000,70.8000) -- (84.0000,70.8000) -- (84.0000,70.8000) -- (84.0000,70.8000) -- (84.0000,70.8000) -- (84.0000,70.7000) -- (84.0000,70.7000) -- (84.0000,70.7000) -- (84.1000,70.7000) -- (84.1000,70.7000) -- (84.1000,70.7000) -- (84.1000,70.7000) -- (84.1000,70.7000) -- (84.1000,70.7000) -- (84.1000,70.7000) -- (84.1000,70.7000) -- (84.1000,70.7000) -- (84.1000,70.7000) -- (84.1000,70.7000) -- (84.1000,70.7000) -- (84.1000,70.7000) -- (84.1000,70.7000) -- (84.1000,70.7000) -- (84.1000,70.7000) -- (84.1000,70.7000) -- (84.1000,70.6000) -- (84.1000,70.6000) -- (84.1000,70.6000) -- (84.1000,70.6000) -- (84.1000,70.6000) -- (84.1000,70.6000) -- (84.1000,70.6000) -- (84.1000,70.6000) -- (84.1000,70.6000) -- (84.1000,70.6000) -- (84.1000,70.6000) -- (84.1000,70.6000) -- (84.1000,70.6000) -- (84.1000,70.6000) -- (84.1000,70.6000) -- (84.1000,70.6000) -- (84.1000,70.6000) -- (84.1000,70.6000) -- (84.1000,70.6000) -- (84.1000,70.6000) -- (84.1000,70.6000) -- (84.1000,70.5000) -- (84.1000,70.5000) -- (84.1000,70.5000) -- (84.1000,70.5000) -- (84.1000,70.5000) -- (84.1000,70.5000) -- (84.1000,70.5000) -- (84.1000,70.5000) -- (84.1000,70.5000) -- (84.1000,70.5000) -- (84.1000,70.5000) -- (84.1000,70.5000) -- (84.2000,70.5000) -- (84.2000,70.5000) -- (84.2000,70.5000) -- (84.2000,70.5000) -- (84.2000,70.5000) -- (84.2000,70.5000) -- (84.2000,70.5000) -- (84.2000,70.5000) -- (84.2000,70.5000) -- (84.2000,70.4000) -- (84.2000,70.4000) -- (84.2000,70.4000) -- (84.2000,70.4000) -- (84.2000,70.4000) -- (84.2000,70.4000) -- (84.2000,70.4000) -- (84.2000,70.4000) -- (84.2000,70.4000) -- (84.2000,70.4000) -- (84.2000,70.4000) -- (84.2000,70.4000) -- (84.2000,70.4000) -- (84.2000,70.4000) -- (84.2000,70.4000) -- (84.2000,70.4000) -- (84.2000,70.4000) -- (84.2000,70.4000) -- (84.2000,70.4000) -- (84.2000,70.4000) -- (84.2000,70.4000) -- (84.2000,70.3000) -- (84.2000,70.3000) -- (84.2000,70.3000) -- (84.2000,70.3000) -- (84.2000,70.3000) -- (84.2000,70.3000) -- (84.2000,70.3000) -- (84.2000,70.3000) -- (84.2000,70.3000) -- (84.2000,70.3000) -- (84.2000,70.3000) -- (84.2000,70.3000) -- (84.2000,70.3000) -- (84.2000,70.3000) -- (84.2000,70.3000) -- (84.2000,70.3000) -- (84.2000,70.3000) -- (84.2000,70.3000) -- (84.2000,70.3000) -- (84.3000,70.3000) -- (84.3000,70.3000) -- (84.3000,70.2000) -- (84.3000,70.2000) -- (84.3000,70.2000) -- (84.3000,70.2000) -- (84.3000,70.2000) -- (84.3000,70.2000) -- (84.3000,70.2000) -- (84.3000,70.2000) -- (84.3000,70.2000) -- (84.3000,70.2000) -- (84.3000,70.2000) -- (84.3000,70.2000) -- (84.3000,70.2000) -- (84.3000,70.2000) -- (84.3000,70.2000) -- (84.3000,70.2000) -- (84.3000,70.2000) -- (84.3000,70.2000) -- (84.3000,70.2000) -- (84.3000,70.2000) -- (84.3000,70.1000) -- (84.3000,70.1000) -- (84.3000,70.1000) -- (84.3000,70.1000) -- (84.3000,70.1000) -- (84.3000,70.1000) -- (84.3000,70.1000) -- (84.3000,70.1000) -- (84.3000,70.1000) -- (84.3000,70.1000) -- (84.3000,70.1000) -- (84.3000,70.1000) -- (84.3000,70.1000) -- (84.3000,70.1000) -- (84.3000,70.1000) -- (84.3000,70.1000) -- (84.3000,70.1000) -- (84.3000,70.1000) -- (84.3000,70.1000) -- (84.3000,70.1000) -- (84.3000,70.1000) -- (84.3000,70.0000) -- (84.3000,70.0000) -- (84.3000,70.0000) -- (84.3000,70.0000) -- (84.3000,70.0000) -- (84.3000,70.0000) -- (84.3000,70.0000) -- (84.4000,70.0000) -- (84.4000,70.0000) -- (84.4000,70.0000) -- (84.4000,70.0000) -- (84.4000,70.0000) -- (84.4000,70.0000) -- (84.4000,70.0000) -- (84.4000,70.0000) -- (84.4000,70.0000) -- (84.4000,70.0000) -- (84.4000,70.0000) -- (84.4000,70.0000) -- (84.4000,70.0000) -- (84.4000,70.0000) -- (84.4000,69.9000) -- (84.4000,69.9000) -- (84.4000,69.9000) -- (84.4000,69.9000) -- (84.4000,69.9000) -- (84.4000,69.9000) -- (84.4000,69.9000) -- (84.4000,69.9000) -- (84.4000,69.9000) -- (84.4000,69.9000) -- (84.4000,69.9000) -- (84.4000,69.9000) -- (84.4000,69.9000) -- (84.4000,69.9000) -- (84.4000,69.9000) -- (84.4000,69.9000) -- (84.4000,69.9000) -- (84.4000,69.9000) -- (84.4000,69.9000) -- (84.4000,69.9000) -- (84.4000,69.9000) -- (84.4000,69.8000) -- (84.4000,69.8000) -- (84.4000,69.8000) -- (84.4000,69.8000) -- (84.4000,69.8000) -- (84.4000,69.8000) -- (84.4000,69.8000) -- (84.4000,69.8000) -- (84.4000,69.8000) -- (84.4000,69.8000) -- (84.4000,69.8000) -- (84.4000,69.8000) -- (84.4000,69.8000) -- (84.4000,69.8000) -- (84.5000,69.8000) -- (84.5000,69.8000) -- (84.5000,69.8000) -- (84.5000,69.8000) -- (84.5000,69.8000) -- (84.5000,69.8000) -- (84.5000,69.8000) -- (84.5000,69.7000) -- (84.5000,69.7000) -- (84.5000,69.7000) -- (84.5000,69.7000) -- (84.5000,69.7000) -- (84.5000,69.7000) -- (84.5000,69.7000) -- (84.5000,69.7000) -- (84.5000,69.7000) -- (84.5000,69.7000) -- (84.5000,69.7000) -- (84.5000,69.7000) -- (84.5000,69.7000) -- (84.5000,69.7000) -- (84.5000,69.7000) -- (84.5000,69.7000) -- (84.5000,69.7000) -- (84.5000,69.7000) -- (84.5000,69.7000) -- (84.5000,69.7000) -- (84.5000,69.6000) -- (84.5000,69.6000) -- (84.5000,69.6000) -- (84.5000,69.6000) -- (84.5000,69.6000) -- (84.5000,69.6000) -- (84.5000,69.6000) -- (84.5000,69.6000) -- (84.5000,69.6000) -- (84.5000,69.6000) -- (84.5000,69.6000) -- (84.5000,69.6000) -- (84.5000,69.6000) -- (84.5000,69.6000) -- (84.5000,69.6000) -- (84.5000,69.6000) -- (84.5000,69.6000) -- (84.5000,69.6000) -- (84.5000,69.6000) -- (84.5000,69.6000) -- (84.5000,69.6000) -- (84.5000,69.5000) -- (84.5000,69.5000) -- (84.6000,69.5000) -- (84.6000,69.5000) -- (84.6000,69.5000) -- (84.6000,69.5000) -- (84.6000,69.5000) -- (84.6000,69.5000) -- (84.6000,69.5000) -- (84.6000,69.5000) -- (84.6000,69.5000) -- (84.6000,69.5000) -- (84.6000,69.5000) -- (84.6000,69.5000) -- (84.6000,69.5000) -- (84.6000,69.5000) -- (84.6000,69.5000) -- (84.6000,69.5000) -- (84.6000,69.5000) -- (84.6000,69.5000) -- (84.6000,69.5000) -- (84.6000,69.4000) -- (84.6000,69.4000) -- (84.6000,69.4000) -- (84.6000,69.4000) -- (84.6000,69.4000) -- (84.6000,69.4000) -- (84.6000,69.4000) -- (84.6000,69.4000) -- (84.6000,69.4000) -- (84.6000,69.4000) -- (84.6000,69.4000) -- (84.6000,69.4000) -- (84.6000,69.4000) -- (84.6000,69.4000) -- (84.6000,69.4000) -- (84.6000,69.4000) -- (84.6000,69.4000) -- (84.6000,69.4000) -- (84.6000,69.4000) -- (84.6000,69.4000) -- (84.6000,69.4000) -- (84.6000,69.3000) -- (84.6000,69.3000) -- (84.6000,69.3000) -- (84.6000,69.3000) -- (84.6000,69.3000) -- (84.6000,69.3000) -- (84.6000,69.3000) -- (84.6000,69.3000) -- (84.6000,69.3000) -- (84.7000,69.3000) -- (84.7000,69.3000) -- (84.7000,69.3000) -- (84.7000,69.3000) -- (84.7000,69.3000) -- (84.7000,69.3000) -- (84.7000,69.3000) -- (84.7000,69.3000) -- (84.7000,69.3000) -- (84.7000,69.3000) -- (84.7000,69.3000) -- (84.7000,69.2000) -- (84.7000,69.2000) -- (84.7000,69.2000) -- (84.7000,69.2000) -- (84.7000,69.2000) -- (84.7000,69.2000) -- (84.7000,69.2000) -- (84.7000,69.2000) -- (84.7000,69.2000) -- (84.7000,69.2000) -- (84.7000,69.2000) -- (84.7000,69.2000) -- (84.7000,69.2000) -- (84.7000,69.2000) -- (84.7000,69.2000) -- (84.7000,69.2000) -- (84.7000,69.2000) -- (84.7000,69.2000) -- (84.7000,69.2000) -- (84.7000,69.2000) -- (84.7000,69.2000) -- (84.7000,69.1000) -- (84.7000,69.1000) -- (84.7000,69.1000) -- (84.7000,69.1000) -- (84.7000,69.1000) -- (84.7000,69.1000) -- (84.7000,69.1000) -- (84.7000,69.1000) -- (84.7000,69.1000) -- (84.7000,69.1000) -- (84.7000,69.1000) -- (84.7000,69.1000) -- (84.7000,69.1000) -- (84.7000,69.1000) -- (84.7000,69.1000) -- (84.7000,69.1000) -- (84.7000,69.1000) -- (84.7000,69.1000) -- (84.8000,69.1000) -- (84.8000,69.1000) -- (84.8000,69.1000) -- (84.8000,69.0000) -- (84.8000,69.0000) -- (84.8000,69.0000) -- (84.8000,69.0000) -- (84.8000,69.0000) -- (84.8000,69.0000) -- (84.8000,69.0000) -- (84.8000,69.0000) -- (84.8000,69.0000) -- (84.8000,69.0000) -- (84.8000,69.0000) -- (84.8000,69.0000) -- (84.8000,69.0000) -- (84.8000,69.0000) -- (84.8000,69.0000) -- (84.8000,69.0000) -- (84.8000,69.0000) -- (84.8000,69.0000) -- (84.8000,69.0000) -- (84.8000,69.0000) -- (84.8000,69.0000) -- (84.8000,68.9000) -- (84.8000,68.9000) -- (84.8000,68.9000) -- (84.8000,68.9000) -- (84.8000,68.9000) -- (84.8000,68.9000) -- (84.8000,68.9000) -- (84.8000,68.9000) -- (84.8000,68.9000) -- (84.8000,68.9000) -- (84.8000,68.9000) -- (84.8000,68.9000) -- (84.8000,68.9000) -- (84.8000,68.9000) -- (84.8000,68.9000) -- (84.8000,68.9000) -- (84.8000,68.9000) -- (84.8000,68.9000) -- (84.8000,68.9000) -- (84.8000,68.9000) -- (84.8000,68.9000) -- (84.8000,68.8000) -- (84.8000,68.8000) -- (84.8000,68.8000) -- (84.8000,68.8000) -- (84.9000,68.8000) -- (84.9000,68.8000) -- (84.9000,68.8000) -- (84.9000,68.8000) -- (84.9000,68.8000) -- (84.9000,68.8000) -- (84.9000,68.8000) -- (84.9000,68.8000) -- (84.9000,68.8000) -- (84.9000,68.8000) -- (84.9000,68.8000) -- (84.9000,68.8000) -- (84.9000,68.8000) -- (84.9000,68.8000) -- (84.9000,68.8000) -- (84.9000,68.8000) -- (84.9000,68.7000) -- (84.9000,68.7000) -- (84.9000,68.7000) -- (84.9000,68.7000) -- (84.9000,68.7000) -- (84.9000,68.7000) -- (84.9000,68.7000) -- (84.9000,68.7000) -- (84.9000,68.7000) -- (84.9000,68.7000) -- (84.9000,68.7000) -- (84.9000,68.7000) -- (84.9000,68.7000) -- (84.9000,68.7000) -- (84.9000,68.7000) -- (84.9000,68.7000) -- (84.9000,68.7000) -- (84.9000,68.7000) -- (84.9000,68.7000) -- (84.9000,68.7000) -- (84.9000,68.7000) -- (84.9000,68.6000) -- (84.9000,68.6000) -- (84.9000,68.6000) -- (84.9000,68.6000) -- (84.9000,68.6000) -- (84.9000,68.6000) -- (84.9000,68.6000) -- (84.9000,68.6000) -- (84.9000,68.6000) -- (84.9000,68.6000) -- (84.9000,68.6000) -- (84.9000,68.6000) -- (84.9000,68.6000) -- (85.0000,68.6000) -- (85.0000,68.6000) -- (85.0000,68.6000) -- (85.0000,68.6000) -- (85.0000,68.6000) -- (85.0000,68.6000) -- (85.0000,68.6000) -- (85.0000,68.6000) -- (85.0000,68.5000) -- (85.0000,68.5000) -- (85.0000,68.5000) -- (85.0000,68.5000) -- (85.0000,68.5000) -- (85.0000,68.5000) -- (85.0000,68.5000) -- (85.0000,68.5000) -- (85.0000,68.5000) -- (85.0000,68.5000) -- (85.0000,68.5000) -- (85.0000,68.5000) -- (85.0000,68.5000) -- (85.0000,68.5000) -- (85.0000,68.5000) -- (85.0000,68.5000) -- (85.0000,68.5000) -- (85.0000,68.5000) -- (85.0000,68.5000) -- (85.0000,68.5000) -- (85.0000,68.5000) -- (85.0000,68.4000) -- (85.0000,68.4000) -- (85.0000,68.4000) -- (85.0000,68.4000) -- (85.0000,68.4000) -- (85.0000,68.4000) -- (85.0000,68.4000) -- (85.0000,68.4000) -- (85.0000,68.4000) -- (85.0000,68.4000) -- (85.0000,68.4000) -- (85.0000,68.4000) -- (85.0000,68.4000) -- (85.0000,68.4000) -- (85.0000,68.4000) -- (85.0000,68.4000) -- (85.0000,68.4000) -- (85.0000,68.4000) -- (85.0000,68.4000) -- (85.0000,68.4000) -- (85.1000,68.4000) -- (85.1000,68.3000) -- (85.1000,68.3000) -- (85.1000,68.3000) -- (85.1000,68.3000) -- (85.1000,68.3000) -- (85.1000,68.3000) -- (85.1000,68.3000) -- (85.1000,68.3000) -- (85.1000,68.3000) -- (85.1000,68.3000) -- (85.1000,68.3000) -- (85.1000,68.3000) -- (85.1000,68.3000) -- (85.1000,68.3000) -- (85.1000,68.3000) -- (85.1000,68.3000) -- (85.1000,68.3000) -- (85.1000,68.3000) -- (85.1000,68.3000) -- (85.1000,68.3000) -- (85.1000,68.2000) -- (85.1000,68.2000) -- (85.1000,68.2000) -- (85.1000,68.2000) -- (85.1000,68.2000) -- (85.1000,68.2000) -- (85.1000,68.2000) -- (85.1000,68.2000) -- (85.1000,68.2000) -- (85.1000,68.2000) -- (85.1000,68.2000) -- (85.1000,68.2000) -- (85.1000,68.2000) -- (85.1000,68.2000) -- (85.1000,68.2000) -- (85.1000,68.2000) -- (85.1000,68.2000) -- (85.1000,68.2000) -- (85.1000,68.2000) -- (85.1000,68.2000) -- (85.1000,68.2000) -- (85.1000,68.1000) -- (85.1000,68.1000) -- (85.1000,68.1000) -- (85.1000,68.1000) -- (85.1000,68.1000) -- (85.1000,68.1000) -- (85.1000,68.1000) -- (85.1000,68.1000) -- (85.2000,68.1000) -- (85.2000,68.1000) -- (85.2000,68.1000) -- (85.2000,68.1000) -- (85.2000,68.1000) -- (85.2000,68.1000) -- (85.2000,68.1000) -- (85.2000,68.1000) -- (85.2000,68.1000) -- (85.2000,68.1000) -- (85.2000,68.1000) -- (85.2000,68.1000) -- (85.2000,68.1000) -- (85.2000,68.0000) -- (85.2000,68.0000) -- (85.2000,68.0000) -- (85.2000,68.0000) -- (85.2000,68.0000) -- (85.2000,68.0000) -- (85.2000,68.0000) -- (85.2000,68.0000) -- (85.2000,68.0000) -- (85.2000,68.0000) -- (85.2000,68.0000) -- (85.2000,68.0000) -- (85.2000,68.0000) -- (85.2000,68.0000) -- (85.2000,68.0000) -- (85.2000,68.0000) -- (85.2000,68.0000) -- (85.2000,68.0000) -- (85.2000,68.0000) -- (85.2000,68.0000) -- (85.2000,68.0000) -- (85.2000,67.9000) -- (85.2000,67.9000) -- (85.2000,67.9000) -- (85.2000,67.9000) -- (85.2000,67.9000) -- (85.2000,67.9000) -- (85.2000,67.9000) -- (85.2000,67.9000) -- (85.2000,67.9000) -- (85.2000,67.9000) -- (85.2000,67.9000) -- (85.2000,67.9000) -- (85.2000,67.9000) -- (85.2000,67.9000) -- (85.2000,67.9000) -- (85.3000,67.9000) -- (85.3000,67.9000) -- (85.3000,67.9000) -- (85.3000,67.9000) -- (85.3000,67.9000) -- (85.3000,67.9000) -- (85.3000,67.8000) -- (85.3000,67.8000) -- (85.3000,67.8000) -- (85.3000,67.8000) -- (85.3000,67.8000) -- (85.3000,67.8000) -- (85.3000,67.8000) -- (85.3000,67.8000) -- (85.3000,67.8000) -- (85.3000,67.8000) -- (85.3000,67.8000) -- (85.3000,67.8000) -- (85.3000,67.8000) -- (85.3000,67.8000) -- (85.3000,67.8000) -- (85.3000,67.8000) -- (85.3000,67.8000) -- (85.3000,67.8000) -- (85.3000,67.8000) -- (85.3000,67.8000) -- (85.3000,67.7000) -- (85.3000,67.7000) -- (85.3000,67.7000) -- (85.3000,67.7000) -- (85.3000,67.7000) -- (85.3000,67.7000) -- (85.3000,67.7000) -- (85.3000,67.7000) -- (85.3000,67.7000) -- (85.3000,67.7000) -- (85.3000,67.7000) -- (85.3000,67.7000) -- (85.3000,67.7000) -- (85.3000,67.7000) -- (85.3000,67.7000) -- (85.3000,67.7000) -- (85.3000,67.7000) -- (85.3000,67.7000) -- (85.3000,67.7000) -- (85.3000,67.7000) -- (85.3000,67.7000) -- (85.3000,67.6000) -- (85.3000,67.6000) -- (85.3000,67.6000) -- (85.4000,67.6000) -- (85.4000,67.6000) -- (85.4000,67.6000) -- (85.4000,67.6000) -- (85.4000,67.6000) -- (85.4000,67.6000) -- (85.4000,67.6000) -- (85.4000,67.6000) -- (85.4000,67.6000) -- (85.4000,67.6000) -- (85.4000,67.6000) -- (85.4000,67.6000) -- (85.4000,67.6000) -- (85.4000,67.6000) -- (85.4000,67.6000) -- (85.4000,67.6000) -- (85.4000,67.6000) -- (85.4000,67.6000) -- (85.4000,67.5000) -- (85.4000,67.5000) -- (85.4000,67.5000) -- (85.4000,67.5000) -- (85.4000,67.5000) -- (85.4000,67.5000) -- (85.4000,67.5000) -- (85.4000,67.5000) -- (85.4000,67.5000) -- (85.4000,67.5000) -- (85.4000,67.5000) -- (85.4000,67.5000) -- (85.4000,67.5000) -- (85.4000,67.5000) -- (85.4000,67.5000) -- (85.4000,67.5000) -- (85.4000,67.5000) -- (85.4000,67.5000) -- (85.4000,67.5000) -- (85.4000,67.5000) -- (85.4000,67.5000) -- (85.4000,67.4000) -- (85.4000,67.4000) -- (85.4000,67.4000) -- (85.4000,67.4000) -- (85.4000,67.4000) -- (85.4000,67.4000) -- (85.4000,67.4000) -- (85.4000,67.4000) -- (85.4000,67.4000) -- (85.4000,67.4000) -- (85.5000,67.4000) -- (85.5000,67.4000) -- (85.5000,67.4000) -- (85.5000,67.4000) -- (85.5000,67.4000) -- (85.5000,67.4000) -- (85.5000,67.4000) -- (85.5000,67.4000) -- (85.5000,67.4000) -- (85.5000,67.4000) -- (85.5000,67.3000) -- (85.5000,67.3000) -- (85.5000,67.3000) -- (85.5000,67.3000) -- (85.5000,67.3000) -- (85.5000,67.3000) -- (85.5000,67.3000) -- (85.5000,67.3000) -- (85.5000,67.3000) -- (85.5000,67.3000) -- (85.5000,67.3000) -- (85.5000,67.3000) -- (85.5000,67.3000) -- (85.5000,67.3000) -- (85.5000,67.3000) -- (85.5000,67.3000) -- (85.5000,67.3000) -- (85.5000,67.3000) -- (85.5000,67.3000) -- (85.5000,67.3000) -- (85.5000,67.3000) -- (85.5000,67.2000) -- (85.5000,67.2000) -- (85.5000,67.2000) -- (85.5000,67.2000) -- (85.5000,67.2000) -- (85.5000,67.2000) -- (85.5000,67.2000) -- (85.5000,67.2000) -- (85.5000,67.2000) -- (85.5000,67.2000) -- (85.5000,67.2000) -- (85.5000,67.2000) -- (85.5000,67.2000) -- (85.5000,67.2000) -- (85.5000,67.2000) -- (85.5000,67.2000) -- (85.5000,67.2000) -- (85.5000,67.2000) -- (85.5000,67.2000) -- (85.6000,67.2000) -- (85.6000,67.2000) -- (85.6000,67.1000) -- (85.6000,67.1000) -- (85.6000,67.1000) -- (85.6000,67.1000) -- (85.6000,67.1000) -- (85.6000,67.1000) -- (85.6000,67.1000) -- (85.6000,67.1000) -- (85.6000,67.1000) -- (85.6000,67.1000) -- (85.6000,67.1000) -- (85.6000,67.1000) -- (85.6000,67.1000) -- (85.6000,67.1000) -- (85.6000,67.1000) -- (85.6000,67.1000) -- (85.6000,67.1000) -- (85.6000,67.1000) -- (85.6000,67.1000) -- (85.6000,67.1000) -- (85.6000,67.1000) -- (85.6000,67.0000) -- (85.6000,67.0000) -- (85.6000,67.0000) -- (85.6000,67.0000) -- (85.6000,67.0000) -- (85.6000,67.0000) -- (85.6000,67.0000) -- (85.6000,67.0000) -- (85.6000,67.0000) -- (85.6000,67.0000) -- (85.6000,67.0000) -- (85.6000,67.0000) -- (85.6000,67.0000) -- (85.6000,67.0000) -- (85.6000,67.0000) -- (85.6000,67.0000) -- (85.6000,67.0000) -- (85.6000,67.0000) -- (85.6000,67.0000) -- (85.6000,67.0000) -- (85.6000,67.0000) -- (85.6000,66.9000) -- (85.6000,66.9000) -- (85.6000,66.9000) -- (85.6000,66.9000) -- (85.6000,66.9000) -- (85.7000,66.9000) -- (85.7000,66.9000) -- (85.7000,66.9000) -- (85.7000,66.9000) -- (85.7000,66.9000) -- (85.7000,66.9000) -- (85.7000,66.9000) -- (85.7000,66.9000) -- (85.7000,66.9000) -- (85.7000,66.9000) -- (85.7000,66.9000) -- (85.7000,66.9000) -- (85.7000,66.9000) -- (85.7000,66.9000) -- (85.7000,66.9000) -- (85.7000,66.8000) -- (85.7000,66.8000) -- (85.7000,66.8000) -- (85.7000,66.8000) -- (85.7000,66.8000) -- (85.7000,66.8000) -- (85.7000,66.8000) -- (85.7000,66.8000) -- (85.7000,66.8000) -- (85.7000,66.8000) -- (85.7000,66.8000) -- (85.7000,66.8000) -- (85.7000,66.8000) -- (85.7000,66.8000) -- (85.7000,66.8000) -- (85.7000,66.8000) -- (85.7000,66.8000) -- (85.7000,66.8000) -- (85.7000,66.8000) -- (85.7000,66.8000) -- (85.7000,66.8000) -- (85.7000,66.7000) -- (85.7000,66.7000) -- (85.7000,66.7000) -- (85.7000,66.7000) -- (85.7000,66.7000) -- (85.7000,66.7000) -- (85.7000,66.7000) -- (85.7000,66.7000) -- (85.7000,66.7000) -- (85.7000,66.7000) -- (85.7000,66.7000) -- (85.7000,66.7000) -- (85.7000,66.7000) -- (85.7000,66.7000) -- (85.8000,66.7000) -- (85.8000,66.7000) -- (85.8000,66.7000) -- (85.8000,66.7000) -- (85.8000,66.7000) -- (85.8000,66.7000) -- (85.8000,66.7000) -- (85.8000,66.6000) -- (85.8000,66.6000) -- (85.8000,66.6000) -- (85.8000,66.6000) -- (85.8000,66.6000) -- (85.8000,66.6000) -- (85.8000,66.6000) -- (85.8000,66.6000) -- (85.8000,66.6000) -- (85.8000,66.6000) -- (85.8000,66.6000) -- (85.8000,66.6000) -- (85.8000,66.6000) -- (85.8000,66.6000) -- (85.8000,66.6000) -- (85.8000,66.6000) -- (85.8000,66.6000) -- (85.8000,66.6000) -- (85.8000,66.6000) -- (85.8000,66.6000) -- (85.8000,66.6000) -- (85.8000,66.5000) -- (85.8000,66.5000) -- (85.8000,66.5000) -- (85.8000,66.5000) -- (85.8000,66.5000) -- (85.8000,66.5000) -- (85.8000,66.5000) -- (85.8000,66.5000) -- (85.8000,66.5000) -- (85.8000,66.5000) -- (85.8000,66.5000) -- (85.8000,66.5000) -- (85.8000,66.5000) -- (85.8000,66.5000) -- (85.8000,66.5000) -- (85.8000,66.5000) -- (85.8000,66.5000) -- (85.8000,66.5000) -- (85.8000,66.5000) -- (85.8000,66.5000) -- (85.8000,66.5000) -- (85.9000,66.4000) -- (85.9000,66.4000) -- (85.9000,66.4000) -- (85.9000,66.4000) -- (85.9000,66.4000) -- (85.9000,66.4000) -- (85.9000,66.4000) -- (85.9000,66.4000) -- (85.9000,66.4000) -- (85.9000,66.4000) -- (85.9000,66.4000) -- (85.9000,66.4000) -- (85.9000,66.4000) -- (85.9000,66.4000) -- (85.9000,66.4000) -- (85.9000,66.4000) -- (85.9000,66.4000) -- (85.9000,66.4000) -- (85.9000,66.4000) -- (85.9000,66.4000) -- (85.9000,66.3000) -- (85.9000,66.3000) -- (85.9000,66.3000) -- (85.9000,66.3000) -- (85.9000,66.3000) -- (85.9000,66.3000) -- (85.9000,66.3000) -- (85.9000,66.3000) -- (85.9000,66.3000) -- (85.9000,66.3000) -- (85.9000,66.3000) -- (85.9000,66.3000) -- (85.9000,66.3000) -- (85.9000,66.3000) -- (85.9000,66.3000) -- (85.9000,66.3000) -- (85.9000,66.3000) -- (85.9000,66.3000) -- (85.9000,66.3000) -- (85.9000,66.3000) -- (85.9000,66.3000) -- (85.9000,66.2000) -- (85.9000,66.2000) -- (85.9000,66.2000) -- (85.9000,66.2000) -- (85.9000,66.2000) -- (85.9000,66.2000) -- (85.9000,66.2000) -- (85.9000,66.2000) -- (85.9000,66.2000) -- (86.0000,66.2000) -- (86.0000,66.2000) -- (86.0000,66.2000) -- (86.0000,66.2000) -- (86.0000,66.2000) -- (86.0000,66.2000) -- (86.0000,66.2000) -- (86.0000,66.2000) -- (86.0000,66.2000) -- (86.0000,66.2000) -- (86.0000,66.2000) -- (86.0000,66.2000) -- (86.0000,66.1000) -- (86.0000,66.1000) -- (86.0000,66.1000) -- (86.0000,66.1000) -- (86.0000,66.1000) -- (86.0000,66.1000) -- (86.0000,66.1000) -- (86.0000,66.1000) -- (86.0000,66.1000) -- (86.0000,66.1000) -- (86.0000,66.1000) -- (86.0000,66.1000) -- (86.0000,66.1000) -- (86.0000,66.1000) -- (86.0000,66.1000) -- (86.0000,66.1000) -- (86.0000,66.1000) -- (86.0000,66.1000) -- (86.0000,66.1000) -- (86.0000,66.1000) -- (86.0000,66.1000) -- (86.0000,66.0000) -- (86.0000,66.0000) -- (86.0000,66.0000) -- (86.0000,66.0000) -- (86.0000,66.0000) -- (86.0000,66.0000) -- (86.0000,66.0000) -- (86.0000,66.0000) -- (86.0000,66.0000) -- (86.0000,66.0000) -- (86.0000,66.0000) -- (86.0000,66.0000) -- (86.0000,66.0000) -- (86.0000,66.0000) -- (86.0000,66.0000) -- (86.0000,66.0000) -- (86.1000,66.0000) -- (86.1000,66.0000) -- (86.1000,66.0000) -- (86.1000,66.0000) -- (86.1000,66.0000) -- (86.1000,65.9000) -- (86.1000,65.9000) -- (86.1000,65.9000) -- (86.1000,65.9000) -- (86.1000,65.9000) -- (86.1000,65.9000) -- (86.1000,65.9000) -- (86.1000,65.9000) -- (86.1000,65.9000) -- (86.1000,65.9000) -- (86.1000,65.9000) -- (86.1000,65.9000) -- (86.1000,65.9000) -- (86.1000,65.9000) -- (86.1000,65.9000) -- (86.1000,65.9000) -- (86.1000,65.9000) -- (86.1000,65.9000) -- (86.1000,65.9000) -- (86.1000,65.9000) -- (86.1000,65.8000) -- (86.1000,65.8000) -- (86.1000,65.8000) -- (86.1000,65.8000) -- (86.1000,65.8000) -- (86.1000,65.8000) -- (86.1000,65.8000) -- (86.1000,65.8000) -- (86.1000,65.8000) -- (86.1000,65.8000) -- (86.1000,65.8000) -- (86.1000,65.8000) -- (86.1000,65.8000) -- (86.1000,65.8000) -- (86.1000,65.8000) -- (86.1000,65.8000) -- (86.1000,65.8000) -- (86.1000,65.8000) -- (86.1000,65.8000) -- (86.1000,65.8000) -- (86.1000,65.8000) -- (86.1000,65.7000) -- (86.1000,65.7000) -- (86.1000,65.7000) -- (86.1000,65.7000) -- (86.2000,65.7000) -- (86.2000,65.7000) -- (86.2000,65.7000) -- (86.2000,65.7000) -- (86.2000,65.7000) -- (86.2000,65.7000) -- (86.2000,65.7000) -- (86.2000,65.7000) -- (86.2000,65.7000) -- (86.2000,65.7000) -- (86.2000,65.7000) -- (86.2000,65.7000) -- (86.2000,65.7000) -- (86.2000,65.7000) -- (86.2000,65.7000) -- (86.2000,65.7000) -- (86.2000,65.7000) -- (86.2000,65.6000) -- (86.2000,65.6000) -- (86.2000,65.6000) -- (86.2000,65.6000) -- (86.2000,65.6000) -- (86.2000,65.6000) -- (86.2000,65.6000) -- (86.2000,65.6000) -- (86.2000,65.6000) -- (86.2000,65.6000) -- (86.2000,65.6000) -- (86.2000,65.6000) -- (86.2000,65.6000) -- (86.2000,65.6000) -- (86.2000,65.6000) -- (86.2000,65.6000) -- (86.2000,65.6000) -- (86.2000,65.6000) -- (86.2000,65.6000) -- (86.2000,65.6000) -- (86.2000,65.6000) -- (86.2000,65.5000) -- (86.2000,65.5000) -- (86.2000,65.5000) -- (86.2000,65.5000) -- (86.2000,65.5000) -- (86.2000,65.5000) -- (86.2000,65.5000) -- (86.2000,65.5000) -- (86.2000,65.5000) -- (86.2000,65.5000) -- (86.2000,65.5000) -- (86.2000,65.5000) -- (86.3000,65.5000) -- (86.3000,65.5000) -- (86.3000,65.5000) -- (86.3000,65.5000) -- (86.3000,65.5000) -- (86.3000,65.5000) -- (86.3000,65.5000) -- (86.3000,65.5000) -- (86.3000,65.4000) -- (86.3000,65.4000) -- (86.3000,65.4000) -- (86.3000,65.4000) -- (86.3000,65.4000) -- (86.3000,65.4000) -- (86.3000,65.4000) -- (86.3000,65.4000) -- (86.3000,65.4000) -- (86.3000,65.4000) -- (86.3000,65.4000) -- (86.3000,65.4000) -- (86.3000,65.4000) -- (86.3000,65.4000) -- (86.3000,65.4000) -- (86.3000,65.4000) -- (86.3000,65.4000) -- (86.3000,65.4000) -- (86.3000,65.4000) -- (86.3000,65.4000) -- (86.3000,65.4000) -- (86.3000,65.3000) -- (86.3000,65.3000) -- (86.3000,65.3000) -- (86.3000,65.3000) -- (86.3000,65.3000) -- (86.3000,65.3000) -- (86.3000,65.3000) -- (86.3000,65.3000) -- (86.3000,65.3000) -- (86.3000,65.3000) -- (86.3000,65.3000) -- (86.3000,65.3000) -- (86.3000,65.3000) -- (86.3000,65.3000) -- (86.3000,65.3000) -- (86.3000,65.3000) -- (86.3000,65.3000) -- (86.3000,65.3000) -- (86.3000,65.3000) -- (86.3000,65.3000) -- (86.4000,65.3000) -- (86.4000,65.2000) -- (86.4000,65.2000) -- (86.4000,65.2000) -- (86.4000,65.2000) -- (86.4000,65.2000) -- (86.4000,65.2000) -- (86.4000,65.2000) -- (86.4000,65.2000) -- (86.4000,65.2000) -- (86.4000,65.2000) -- (86.4000,65.2000) -- (86.4000,65.2000) -- (86.4000,65.2000) -- (86.4000,65.2000) -- (86.4000,65.2000) -- (86.4000,65.2000) -- (86.4000,65.2000) -- (86.4000,65.2000) -- (86.4000,65.2000) -- (86.4000,65.2000) -- (86.4000,65.2000) -- (86.4000,65.1000) -- (86.4000,65.1000) -- (86.4000,65.1000) -- (86.4000,65.1000) -- (86.4000,65.1000) -- (86.4000,65.1000) -- (86.4000,65.1000) -- (86.4000,65.1000) -- (86.4000,65.1000) -- (86.4000,65.1000) -- (86.4000,65.1000) -- (86.4000,65.1000) -- (86.4000,65.1000) -- (86.4000,65.1000) -- (86.4000,65.1000) -- (86.4000,65.1000) -- (86.4000,65.1000) -- (86.4000,65.1000) -- (86.4000,65.1000) -- (86.4000,65.1000) -- (86.4000,65.1000) -- (86.4000,65.0000) -- (86.4000,65.0000) -- (86.4000,65.0000) -- (86.4000,65.0000) -- (86.4000,65.0000) -- (86.4000,65.0000) -- (86.4000,65.0000) -- (86.5000,65.0000) -- (86.5000,65.0000) -- (86.5000,65.0000) -- (86.5000,65.0000) -- (86.5000,65.0000) -- (86.5000,65.0000) -- (86.5000,65.0000) -- (86.5000,65.0000) -- (86.5000,65.0000) -- (86.5000,65.0000) -- (86.5000,65.0000) -- (86.5000,65.0000) -- (86.5000,65.0000) -- (86.5000,64.9000) -- (86.5000,64.9000) -- (86.5000,64.9000) -- (86.5000,64.9000) -- (86.5000,64.9000) -- (86.5000,64.9000) -- (86.5000,64.9000) -- (86.5000,64.9000) -- (86.5000,64.9000) -- (86.5000,64.9000) -- (86.5000,64.9000) -- (86.5000,64.9000) -- (86.5000,64.9000) -- (86.5000,64.9000) -- (86.5000,64.9000) -- (86.5000,64.9000) -- (86.5000,64.9000) -- (86.5000,64.9000) -- (86.5000,64.9000) -- (86.5000,64.9000) -- (86.5000,64.9000) -- (86.5000,64.8000) -- (86.5000,64.8000) -- (86.5000,64.8000) -- (86.5000,64.8000) -- (86.5000,64.8000) -- (86.5000,64.8000) -- (86.5000,64.8000) -- (86.5000,64.8000) -- (86.5000,64.8000) -- (86.5000,64.8000) -- (86.5000,64.8000) -- (86.5000,64.8000) -- (86.5000,64.8000) -- (86.5000,64.8000) -- (86.5000,64.8000) -- (86.6000,64.8000) -- (86.6000,64.8000) -- (86.6000,64.8000) -- (86.6000,64.8000) -- (86.6000,64.8000) -- (86.6000,64.8000) -- (86.6000,64.7000) -- (86.6000,64.7000) -- (86.6000,64.7000) -- (86.6000,64.7000) -- (86.6000,64.7000) -- (86.6000,64.7000) -- (86.6000,64.7000) -- (86.6000,64.7000) -- (86.6000,64.7000) -- (86.6000,64.7000) -- (86.6000,64.7000) -- (86.6000,64.7000) -- (86.6000,64.7000) -- (86.6000,64.7000) -- (86.6000,64.7000) -- (86.6000,64.7000) -- (86.6000,64.7000) -- (86.6000,64.7000) -- (86.6000,64.7000) -- (86.6000,64.7000) -- (86.6000,64.7000) -- (86.6000,64.6000) -- (86.6000,64.6000) -- (86.6000,64.6000) -- (86.6000,64.6000) -- (86.6000,64.6000) -- (86.6000,64.6000) -- (86.6000,64.6000) -- (86.6000,64.6000) -- (86.6000,64.6000) -- (86.6000,64.6000) -- (86.6000,64.6000) -- (86.6000,64.6000) -- (86.6000,64.6000) -- (86.6000,64.6000) -- (86.6000,64.6000) -- (86.6000,64.6000) -- (86.6000,64.6000) -- (86.6000,64.6000) -- (86.6000,64.6000) -- (86.6000,64.6000) -- (86.6000,64.6000) -- (86.6000,64.5000) -- (86.6000,64.5000) -- (86.7000,64.5000) -- (86.7000,64.5000) -- (86.7000,64.5000) -- (86.7000,64.5000) -- (86.7000,64.5000) -- (86.7000,64.5000) -- (86.7000,64.5000) -- (86.7000,64.5000) -- (86.7000,64.5000) -- (86.7000,64.5000) -- (86.7000,64.5000) -- (86.7000,64.5000) -- (86.7000,64.5000) -- (86.7000,64.5000) -- (86.7000,64.5000) -- (86.7000,64.5000) -- (86.7000,64.5000) -- (86.7000,64.5000) -- (86.7000,64.4000) -- (86.7000,64.4000) -- (86.7000,64.4000) -- (86.7000,64.4000) -- (86.7000,64.4000) -- (86.7000,64.4000) -- (86.7000,64.4000) -- (86.7000,64.4000) -- (86.7000,64.4000) -- (86.7000,64.4000) -- (86.7000,64.4000) -- (86.7000,64.4000) -- (86.7000,64.4000) -- (86.7000,64.4000) -- (86.7000,64.4000) -- (86.7000,64.4000) -- (86.7000,64.4000) -- (86.7000,64.4000) -- (86.7000,64.4000) -- (86.7000,64.4000) -- (86.7000,64.4000) -- (86.7000,64.3000) -- (86.7000,64.3000) -- (86.7000,64.3000) -- (86.7000,64.3000) -- (86.7000,64.3000) -- (86.7000,64.3000) -- (86.7000,64.3000) -- (86.7000,64.3000) -- (86.7000,64.3000) -- (86.7000,64.3000) -- (86.8000,64.3000) -- (86.8000,64.3000) -- (86.8000,64.3000) -- (86.8000,64.3000) -- (86.8000,64.3000) -- (86.8000,64.3000) -- (86.8000,64.3000) -- (86.8000,64.3000) -- (86.8000,64.3000) -- (86.8000,64.3000) -- (86.8000,64.3000) -- (86.8000,64.2000) -- (86.8000,64.2000) -- (86.8000,64.2000) -- (86.8000,64.2000) -- (86.8000,64.2000) -- (86.8000,64.2000) -- (86.8000,64.2000) -- (86.8000,64.2000) -- (86.8000,64.2000) -- (86.8000,64.2000) -- (86.8000,64.2000) -- (86.8000,64.2000) -- (86.8000,64.2000) -- (86.8000,64.2000) -- (86.8000,64.2000) -- (86.8000,64.2000) -- (86.8000,64.2000) -- (86.8000,64.2000) -- (86.8000,64.2000) -- (86.8000,64.2000) -- (86.8000,64.2000) -- (86.8000,64.1000) -- (86.8000,64.1000) -- (86.8000,64.1000) -- (86.8000,64.1000) -- (86.8000,64.1000) -- (86.8000,64.1000) -- (86.8000,64.1000) -- (86.8000,64.1000) -- (86.8000,64.1000) -- (86.8000,64.1000) -- (86.8000,64.1000) -- (86.8000,64.1000) -- (86.8000,64.1000) -- (86.8000,64.1000) -- (86.8000,64.1000) -- (86.8000,64.1000) -- (86.8000,64.1000) -- (86.8000,64.1000) -- (86.9000,64.1000) -- (86.9000,64.1000) -- (86.9000,64.0000) -- (86.9000,64.0000) -- (86.9000,64.0000) -- (86.9000,64.0000) -- (86.9000,64.0000) -- (86.9000,64.0000) -- (86.9000,64.0000) -- (86.9000,64.0000) -- (86.9000,64.0000) -- (86.9000,64.0000) -- (86.9000,64.0000) -- (86.9000,64.0000) -- (86.9000,64.0000) -- (86.9000,64.0000) -- (86.9000,64.0000) -- (86.9000,64.0000) -- (86.9000,64.0000) -- (86.9000,64.0000) -- (86.9000,64.0000) -- (86.9000,64.0000) -- (86.9000,64.0000) -- (86.9000,63.9000) -- (86.9000,63.9000) -- (86.9000,63.9000) -- (86.9000,63.9000) -- (86.9000,63.9000) -- (86.9000,63.9000) -- (86.9000,63.9000) -- (86.9000,63.9000) -- (86.9000,63.9000) -- (86.9000,63.9000) -- (86.9000,63.9000) -- (86.9000,63.9000) -- (86.9000,63.9000) -- (86.9000,63.9000) -- (86.9000,63.9000) -- (86.9000,63.9000) -- (86.9000,63.9000) -- (86.9000,63.9000) -- (86.9000,63.9000) -- (86.9000,63.9000) -- (86.9000,63.9000) -- (86.9000,63.8000) -- (86.9000,63.8000) -- (86.9000,63.8000) -- (86.9000,63.8000) -- (86.9000,63.8000) -- (87.0000,63.8000) -- (87.0000,63.8000) -- (87.0000,63.8000) -- (87.0000,63.8000) -- (87.0000,63.8000) -- (87.0000,63.8000) -- (87.0000,63.8000) -- (87.0000,63.8000) -- (87.0000,63.8000) -- (87.0000,63.8000) -- (87.0000,63.8000) -- (87.0000,63.8000) -- (87.0000,63.8000) -- (87.0000,63.8000) -- (87.0000,63.8000) -- (87.0000,63.8000) -- (87.0000,63.7000) -- (87.0000,63.7000) -- (87.0000,63.7000) -- (87.0000,63.7000) -- (87.0000,63.7000) -- (87.0000,63.7000) -- (87.0000,63.7000) -- (87.0000,63.7000) -- (87.0000,63.7000) -- (87.0000,63.7000) -- (87.0000,63.7000) -- (87.0000,63.7000) -- (87.0000,63.7000) -- (87.0000,63.7000) -- (87.0000,63.7000) -- (87.0000,63.7000) -- (87.0000,63.7000) -- (87.0000,63.7000) -- (87.0000,63.7000) -- (87.0000,63.7000) -- (87.0000,63.7000) -- (87.0000,63.6000) -- (87.0000,63.6000) -- (87.0000,63.6000) -- (87.0000,63.6000) -- (87.0000,63.6000) -- (87.0000,63.6000) -- (87.0000,63.6000) -- (87.0000,63.6000) -- (87.0000,63.6000) -- (87.0000,63.6000) -- (87.0000,63.6000) -- (87.0000,63.6000) -- (87.0000,63.6000) -- (87.1000,63.6000) -- (87.1000,63.6000) -- (87.1000,63.6000) -- (87.1000,63.6000) -- (87.1000,63.6000) -- (87.1000,63.6000) -- (87.1000,63.6000) -- (87.1000,63.5000) -- (87.1000,63.5000) -- (87.1000,63.5000) -- (87.1000,63.5000) -- (87.1000,63.5000) -- (87.1000,63.5000) -- (87.1000,63.5000) -- (87.1000,63.5000) -- (87.1000,63.5000) -- (87.1000,63.5000) -- (87.1000,63.5000) -- (87.1000,63.5000) -- (87.1000,63.5000) -- (87.1000,63.5000) -- (87.1000,63.5000) -- (87.1000,63.5000) -- (87.1000,63.5000) -- (87.1000,63.5000) -- (87.1000,63.5000) -- (87.1000,63.5000) -- (87.1000,63.5000) -- (87.1000,63.4000) -- (87.1000,63.4000) -- (87.1000,63.4000) -- (87.1000,63.4000) -- (87.1000,63.4000) -- (87.1000,63.4000) -- (87.1000,63.4000) -- (87.1000,63.4000) -- (87.1000,63.4000) -- (87.1000,63.4000) -- (87.1000,63.4000) -- (87.1000,63.4000) -- (87.1000,63.4000) -- (87.1000,63.4000) -- (87.1000,63.4000) -- (87.1000,63.4000) -- (87.1000,63.4000) -- (87.1000,63.4000) -- (87.1000,63.4000) -- (87.1000,63.4000) -- (87.1000,63.4000) -- (87.2000,63.3000) -- (87.2000,63.3000) -- (87.2000,63.3000) -- (87.2000,63.3000) -- (87.2000,63.3000) -- (87.2000,63.3000) -- (87.2000,63.3000) -- (87.2000,63.3000) -- (87.2000,63.3000) -- (87.2000,63.3000) -- (87.2000,63.3000) -- (87.2000,63.3000) -- (87.2000,63.3000) -- (87.2000,63.3000) -- (87.2000,63.3000) -- (87.2000,63.3000) -- (87.2000,63.3000) -- (87.2000,63.3000) -- (87.2000,63.3000) -- (87.2000,63.3000) -- (87.2000,63.3000) -- (87.2000,63.2000) -- (87.2000,63.2000) -- (87.2000,63.2000) -- (87.2000,63.2000) -- (87.2000,63.2000) -- (87.2000,63.2000) -- (87.2000,63.2000) -- (87.2000,63.2000) -- (87.2000,63.2000) -- (87.2000,63.2000) -- (87.2000,63.2000) -- (87.2000,63.2000) -- (87.2000,63.2000) -- (87.2000,63.2000) -- (87.2000,63.2000) -- (87.2000,63.2000) -- (87.2000,63.2000) -- (87.2000,63.2000) -- (87.2000,63.2000) -- (87.2000,63.2000) -- (87.2000,63.2000) -- (87.2000,63.1000) -- (87.2000,63.1000) -- (87.2000,63.1000) -- (87.2000,63.1000) -- (87.2000,63.1000) -- (87.2000,63.1000) -- (87.2000,63.1000) -- (87.2000,63.1000) -- (87.3000,63.1000) -- (87.3000,63.1000) -- (87.3000,63.1000) -- (87.3000,63.1000) -- (87.3000,63.1000) -- (87.3000,63.1000) -- (87.3000,63.1000) -- (87.3000,63.1000) -- (87.3000,63.1000) -- (87.3000,63.1000) -- (87.3000,63.1000) -- (87.3000,63.1000) -- (87.3000,63.0000) -- (87.3000,63.0000) -- (87.3000,63.0000) -- (87.3000,63.0000) -- (87.3000,63.0000) -- (87.3000,63.0000) -- (87.3000,63.0000) -- (87.3000,63.0000) -- (87.3000,63.0000) -- (87.3000,63.0000) -- (87.3000,63.0000) -- (87.3000,63.0000) -- (87.3000,63.0000) -- (87.3000,63.0000) -- (87.3000,63.0000) -- (87.3000,63.0000) -- (87.3000,63.0000) -- (87.3000,63.0000) -- (87.3000,63.0000) -- (87.3000,63.0000) -- (87.3000,63.0000) -- (87.3000,62.9000) -- (87.3000,62.9000) -- (87.3000,62.9000) -- (87.3000,62.9000) -- (87.3000,62.9000) -- (87.3000,62.9000) -- (87.3000,62.9000) -- (87.3000,62.9000) -- (87.3000,62.9000) -- (87.3000,62.9000) -- (87.3000,62.9000) -- (87.3000,62.9000) -- (87.3000,62.9000) -- (87.3000,62.9000) -- (87.3000,62.9000) -- (87.3000,62.9000) -- (87.4000,62.9000) -- (87.4000,62.9000) -- (87.4000,62.9000) -- (87.4000,62.9000) -- (87.4000,62.9000) -- (87.4000,62.8000) -- (87.4000,62.8000) -- (87.4000,62.8000) -- (87.4000,62.8000) -- (87.4000,62.8000) -- (87.4000,62.8000) -- (87.4000,62.8000) -- (87.4000,62.8000) -- (87.4000,62.8000) -- (87.4000,62.8000) -- (87.4000,62.8000) -- (87.4000,62.8000) -- (87.4000,62.8000) -- (87.4000,62.8000) -- (87.4000,62.8000) -- (87.4000,62.8000) -- (87.4000,62.8000) -- (87.4000,62.8000) -- (87.4000,62.8000) -- (87.4000,62.8000) -- (87.4000,62.8000) -- (87.4000,62.7000) -- (87.4000,62.7000) -- (87.4000,62.7000) -- (87.4000,62.7000) -- (87.4000,62.7000) -- (87.4000,62.7000) -- (87.4000,62.7000) -- (87.4000,62.7000) -- (87.4000,62.7000) -- (87.4000,62.7000) -- (87.4000,62.7000) -- (87.4000,62.7000) -- (87.4000,62.7000) -- (87.4000,62.7000) -- (87.4000,62.7000) -- (87.4000,62.7000) -- (87.4000,62.7000) -- (87.4000,62.7000) -- (87.4000,62.7000) -- (87.4000,62.7000) -- (87.4000,62.7000) -- (87.4000,62.6000) -- (87.4000,62.6000) -- (87.4000,62.6000) -- (87.5000,62.6000) -- (87.5000,62.6000) -- (87.5000,62.6000) -- (87.5000,62.6000) -- (87.5000,62.6000) -- (87.5000,62.6000) -- (87.5000,62.6000) -- (87.5000,62.6000) -- (87.5000,62.6000) -- (87.5000,62.6000) -- (87.5000,62.6000) -- (87.5000,62.6000) -- (87.5000,62.6000) -- (87.5000,62.6000) -- (87.5000,62.6000) -- (87.5000,62.6000) -- (87.5000,62.6000) -- (87.5000,62.5000) -- (87.5000,62.5000) -- (87.5000,62.5000) -- (87.5000,62.5000) -- (87.5000,62.5000) -- (87.5000,62.5000) -- (87.5000,62.5000) -- (87.5000,62.5000) -- (87.5000,62.5000) -- (87.5000,62.5000) -- (87.5000,62.5000) -- (87.5000,62.5000) -- (87.5000,62.5000) -- (87.5000,62.5000) -- (87.5000,62.5000) -- (87.5000,62.5000) -- (87.5000,62.5000) -- (87.5000,62.5000) -- (87.5000,62.5000) -- (87.5000,62.5000) -- (87.5000,62.5000) -- (87.5000,62.4000) -- (87.5000,62.4000) -- (87.5000,62.4000) -- (87.5000,62.4000) -- (87.5000,62.4000) -- (87.5000,62.4000) -- (87.5000,62.4000) -- (87.5000,62.4000) -- (87.5000,62.4000) -- (87.5000,62.4000) -- (87.5000,62.4000) -- (87.6000,62.4000) -- (87.6000,62.4000) -- (87.6000,62.4000) -- (87.6000,62.4000) -- (87.6000,62.4000) -- (87.6000,62.4000) -- (87.6000,62.4000) -- (87.6000,62.4000) -- (87.6000,62.4000) -- (87.6000,62.4000) -- (87.6000,62.3000) -- (87.6000,62.3000) -- (87.6000,62.3000) -- (87.6000,62.3000) -- (87.6000,62.3000) -- (87.6000,62.3000) -- (87.6000,62.3000) -- (87.6000,62.3000) -- (87.6000,62.3000) -- (87.6000,62.3000) -- (87.6000,62.3000) -- (87.6000,62.3000) -- (87.6000,62.3000) -- (87.6000,62.3000) -- (87.6000,62.3000) -- (87.6000,62.3000) -- (87.6000,62.3000) -- (87.6000,62.3000) -- (87.6000,62.3000) -- (87.6000,62.3000) -- (87.6000,62.3000) -- (87.6000,62.2000) -- (87.6000,62.2000) -- (87.6000,62.2000) -- (87.6000,62.2000) -- (87.6000,62.2000) -- (87.6000,62.2000) -- (87.6000,62.2000) -- (87.6000,62.2000) -- (87.6000,62.2000) -- (87.6000,62.2000) -- (87.6000,62.2000) -- (87.6000,62.2000) -- (87.6000,62.2000) -- (87.6000,62.2000) -- (87.6000,62.2000) -- (87.6000,62.2000) -- (87.6000,62.2000) -- (87.6000,62.2000) -- (87.6000,62.2000) -- (87.7000,62.2000) -- (87.7000,62.1000) -- (87.7000,62.1000) -- (87.7000,62.1000) -- (87.7000,62.1000) -- (87.7000,62.1000) -- (87.7000,62.1000) -- (87.7000,62.1000) -- (87.7000,62.1000) -- (87.7000,62.1000) -- (87.7000,62.1000) -- (87.7000,62.1000) -- (87.7000,62.1000) -- (87.7000,62.1000) -- (87.7000,62.1000) -- (87.7000,62.1000) -- (87.7000,62.1000) -- (87.7000,62.1000) -- (87.7000,62.1000) -- (87.7000,62.1000) -- (87.7000,62.1000) -- (87.7000,62.1000) -- (87.7000,62.0000) -- (87.7000,62.0000) -- (87.7000,62.0000) -- (87.7000,62.0000) -- (87.7000,62.0000) -- (87.7000,62.0000) -- (87.7000,62.0000) -- (87.7000,62.0000) -- (87.7000,62.0000) -- (87.7000,62.0000) -- (87.7000,62.0000) -- (87.7000,62.0000) -- (87.7000,62.0000) -- (87.7000,62.0000) -- (87.7000,62.0000) -- (87.7000,62.0000) -- (87.7000,62.0000) -- (87.7000,62.0000) -- (87.7000,62.0000) -- (87.7000,62.0000) -- (87.7000,62.0000) -- (87.7000,61.9000) -- (87.7000,61.9000) -- (87.7000,61.9000) -- (87.7000,61.9000) -- (87.7000,61.9000) -- (87.7000,61.9000) -- (87.8000,61.9000) -- (87.8000,61.9000) -- (87.8000,61.9000) -- (87.8000,61.9000) -- (87.8000,61.9000) -- (87.8000,61.9000) -- (87.8000,61.9000) -- (87.8000,61.9000) -- (87.8000,61.9000) -- (87.8000,61.9000) -- (87.8000,61.9000) -- (87.8000,61.9000) -- (87.8000,61.9000) -- (87.8000,61.9000) -- (87.8000,61.9000) -- (87.8000,61.8000) -- (87.8000,61.8000) -- (87.8000,61.8000) -- (87.8000,61.8000) -- (87.8000,61.8000) -- (87.8000,61.8000) -- (87.8000,61.8000) -- (87.8000,61.8000) -- (87.8000,61.8000) -- (87.8000,61.8000) -- (87.8000,61.8000) -- (87.8000,61.8000) -- (87.8000,61.8000) -- (87.8000,61.8000) -- (87.8000,61.8000) -- (87.8000,61.8000) -- (87.8000,61.8000) -- (87.8000,61.8000) -- (87.8000,61.8000) -- (87.8000,61.8000) -- (87.8000,61.8000) -- (87.8000,61.7000) -- (87.8000,61.7000) -- (87.8000,61.7000) -- (87.8000,61.7000) -- (87.8000,61.7000) -- (87.8000,61.7000) -- (87.8000,61.7000) -- (87.8000,61.7000) -- (87.8000,61.7000) -- (87.8000,61.7000) -- (87.8000,61.7000) -- (87.8000,61.7000) -- (87.8000,61.7000) -- (87.8000,61.7000) -- (87.9000,61.7000) -- (87.9000,61.7000) -- (87.9000,61.7000) -- (87.9000,61.7000) -- (87.9000,61.7000) -- (87.9000,61.7000) -- (87.9000,61.6000) -- (87.9000,61.6000) -- (87.9000,61.6000) -- (87.9000,61.6000) -- (87.9000,61.6000) -- (87.9000,61.6000) -- (87.9000,61.6000) -- (87.9000,61.6000) -- (87.9000,61.6000) -- (87.9000,61.6000) -- (87.9000,61.6000) -- (87.9000,61.6000) -- (87.9000,61.6000) -- (87.9000,61.6000) -- (87.9000,61.6000) -- (87.9000,61.6000) -- (87.9000,61.6000) -- (87.9000,61.6000) -- (87.9000,61.6000) -- (87.9000,61.6000) -- (87.9000,61.6000) -- (87.9000,61.5000) -- (87.9000,61.5000) -- (87.9000,61.5000) -- (87.9000,61.5000) -- (87.9000,61.5000) -- (87.9000,61.5000) -- (87.9000,61.5000) -- (87.9000,61.5000) -- (87.9000,61.5000) -- (87.9000,61.5000) -- (87.9000,61.5000) -- (87.9000,61.5000) -- (87.9000,61.5000) -- (87.9000,61.5000) -- (87.9000,61.5000) -- (87.9000,61.5000) -- (87.9000,61.5000) -- (87.9000,61.5000) -- (87.9000,61.5000) -- (87.9000,61.5000) -- (87.9000,61.5000) -- (87.9000,61.4000) -- (88.0000,61.4000) -- (88.0000,61.4000) -- (88.0000,61.4000) -- (88.0000,61.4000) -- (88.0000,61.4000) -- (88.0000,61.4000) -- (88.0000,61.4000) -- (88.0000,61.4000) -- (88.0000,61.4000) -- (88.0000,61.4000) -- (88.0000,61.4000) -- (88.0000,61.4000) -- (88.0000,61.4000) -- (88.0000,61.4000) -- (88.0000,61.4000) -- (88.0000,61.4000) -- (88.0000,61.4000) -- (88.0000,61.4000) -- (88.0000,61.4000) -- (88.0000,61.4000) -- (88.0000,61.3000) -- (88.0000,61.3000) -- (88.0000,61.3000) -- (88.0000,61.3000) -- (88.0000,61.3000) -- (88.0000,61.3000) -- (88.0000,61.3000) -- (88.0000,61.3000) -- (88.0000,61.3000) -- (88.0000,61.3000) -- (88.0000,61.3000) -- (88.0000,61.3000) -- (88.0000,61.3000) -- (88.0000,61.3000) -- (88.0000,61.3000) -- (88.0000,61.3000) -- (88.0000,61.3000) -- (88.0000,61.3000) -- (88.0000,61.3000) -- (88.0000,61.3000) -- (88.0000,61.3000) -- (88.0000,61.2000) -- (88.0000,61.2000) -- (88.0000,61.2000) -- (88.0000,61.2000) -- (88.0000,61.2000) -- (88.0000,61.2000) -- (88.0000,61.2000) -- (88.0000,61.2000) -- (88.0000,61.2000) -- (88.1000,61.2000) -- (88.1000,61.2000) -- (88.1000,61.2000) -- (88.1000,61.2000) -- (88.1000,61.2000) -- (88.1000,61.2000) -- (88.1000,61.2000) -- (88.1000,61.2000) -- (88.1000,61.2000) -- (88.1000,61.2000) -- (88.1000,61.2000) -- (88.1000,61.1000) -- (88.1000,61.1000) -- (88.1000,61.1000) -- (88.1000,61.1000) -- (88.1000,61.1000) -- (88.1000,61.1000) -- (88.1000,61.1000) -- (88.1000,61.1000) -- (88.1000,61.1000) -- (88.1000,61.1000) -- (88.1000,61.1000) -- (88.1000,61.1000) -- (88.1000,61.1000) -- (88.1000,61.1000) -- (88.1000,61.1000) -- (88.1000,61.1000) -- (88.1000,61.1000) -- (88.1000,61.1000) -- (88.1000,61.1000) -- (88.1000,61.1000) -- (88.1000,61.1000) -- (88.1000,61.0000) -- (88.1000,61.0000) -- (88.1000,61.0000) -- (88.1000,61.0000) -- (88.1000,61.0000) -- (88.1000,61.0000) -- (88.1000,61.0000) -- (88.1000,61.0000) -- (88.1000,61.0000) -- (88.1000,61.0000) -- (88.1000,61.0000) -- (88.1000,61.0000) -- (88.1000,61.0000) -- (88.1000,61.0000) -- (88.1000,61.0000) -- (88.1000,61.0000) -- (88.1000,61.0000) -- (88.2000,61.0000) -- (88.2000,61.0000) -- (88.2000,61.0000) -- (88.2000,61.0000) -- (88.2000,60.9000) -- (88.2000,60.9000) -- (88.2000,60.9000) -- (88.2000,60.9000) -- (88.2000,60.9000) -- (88.2000,60.9000) -- (88.2000,60.9000) -- (88.2000,60.9000) -- (88.2000,60.9000) -- (88.2000,60.9000) -- (88.2000,60.9000) -- (88.2000,60.9000) -- (88.2000,60.9000) -- (88.2000,60.9000) -- (88.2000,60.9000) -- (88.2000,60.9000) -- (88.2000,60.9000) -- (88.2000,60.9000) -- (88.2000,60.9000) -- (88.2000,60.9000) -- (88.2000,60.9000) -- (88.2000,60.8000) -- (88.2000,60.8000) -- (88.2000,60.8000) -- (88.2000,60.8000) -- (88.2000,60.8000) -- (88.2000,60.8000) -- (88.2000,60.8000) -- (88.2000,60.8000) -- (88.2000,60.8000) -- (88.2000,60.8000) -- (88.2000,60.8000) -- (88.2000,60.8000) -- (88.2000,60.8000) -- (88.2000,60.8000) -- (88.2000,60.8000) -- (88.2000,60.8000) -- (88.2000,60.8000) -- (88.2000,60.8000) -- (88.2000,60.8000) -- (88.2000,60.8000) -- (88.2000,60.8000) -- (88.2000,60.7000) -- (88.2000,60.7000) -- (88.2000,60.7000) -- (88.2000,60.7000) -- (88.3000,60.7000) -- (88.3000,60.7000) -- (88.3000,60.7000) -- (88.3000,60.7000) -- (88.3000,60.7000) -- (88.3000,60.7000) -- (88.3000,60.7000) -- (88.3000,60.7000) -- (88.3000,60.7000) -- (88.3000,60.7000) -- (88.3000,60.7000) -- (88.3000,60.7000) -- (88.3000,60.7000) -- (88.3000,60.7000) -- (88.3000,60.7000) -- (88.3000,60.7000) -- (88.3000,60.6000) -- (88.3000,60.6000) -- (88.3000,60.6000) -- (88.3000,60.6000) -- (88.3000,60.6000) -- (88.3000,60.6000) -- (88.3000,60.6000) -- (88.3000,60.6000) -- (88.3000,60.6000) -- (88.3000,60.6000) -- (88.3000,60.6000) -- (88.3000,60.6000) -- (88.3000,60.6000) -- (88.3000,60.6000) -- (88.3000,60.6000) -- (88.3000,60.6000) -- (88.3000,60.6000) -- (88.3000,60.6000) -- (88.3000,60.6000) -- (88.3000,60.6000) -- (88.3000,60.6000) -- (95.7000,60.5000) -- (95.7000,60.5000) -- (95.7000,60.5000) -- (95.7000,60.5000) -- (95.7000,60.5000) -- (95.7000,60.5000) -- (95.7000,60.5000) -- (95.7000,60.5000) -- (95.7000,60.5000) -- (95.7000,60.5000) -- (95.7000,60.5000) -- (95.7000,60.5000) -- (95.7000,60.5000) -- (95.7000,60.5000) -- (95.7000,60.5000) -- (95.7000,60.5000) -- (95.7000,60.5000) -- (95.7000,60.5000) -- (95.7000,60.5000) -- (95.7000,60.5000) -- (95.7000,60.5000) -- (95.7000,60.4000) -- (95.7000,60.4000) -- (95.7000,60.4000) -- (95.7000,60.4000) -- (95.7000,60.4000) -- (95.7000,60.4000) -- (95.7000,60.4000) -- (95.7000,60.4000) -- (95.7000,60.4000) -- (95.7000,60.4000) -- (95.7000,60.4000) -- (95.7000,60.4000) -- (95.7000,60.4000) -- (95.7000,60.4000) -- (95.7000,60.4000) -- (95.7000,60.4000) -- (95.7000,60.4000) -- (95.7000,60.4000) -- (95.7000,60.4000) -- (95.7000,60.4000) -- (95.7000,60.4000) -- (95.7000,60.3000) -- (95.7000,60.3000) -- (95.7000,60.3000) -- (95.7000,60.3000) -- (95.7000,60.3000) -- (95.7000,60.3000) -- (95.8000,60.3000) -- (95.8000,60.3000) -- (95.8000,60.3000) -- (95.8000,60.3000) -- (95.8000,60.3000) -- (95.8000,60.3000) -- (95.8000,60.3000) -- (95.8000,60.3000) -- (95.8000,60.3000) -- (95.8000,60.3000) -- (95.8000,60.3000) -- (95.8000,60.3000) -- (95.8000,60.3000) -- (95.8000,60.3000) -- (95.8000,60.2000) -- (95.8000,60.2000) -- (95.8000,60.2000) -- (95.8000,60.2000) -- (95.8000,60.2000) -- (95.8000,60.2000) -- (95.8000,60.2000) -- (95.8000,60.2000) -- (95.8000,60.2000) -- (95.8000,60.2000) -- (95.8000,60.2000) -- (95.8000,60.2000) -- (95.8000,60.2000) -- (95.8000,60.2000) -- (95.8000,60.2000) -- (95.8000,60.2000) -- (95.8000,60.2000) -- (95.8000,60.2000) -- (95.8000,60.2000) -- (95.8000,60.2000) -- (95.8000,60.2000) -- (95.8000,60.1000) -- (95.8000,60.1000) -- (95.8000,60.1000) -- (95.8000,60.1000) -- (95.8000,60.1000) -- (95.8000,60.1000) -- (95.8000,60.1000) -- (95.8000,60.1000) -- (95.8000,60.1000) -- (95.8000,60.1000) -- (95.8000,60.1000) -- (95.8000,60.1000) -- (95.8000,60.1000) -- (95.8000,60.1000) -- (95.8000,60.1000) -- (95.9000,60.1000) -- (95.9000,60.1000) -- (95.9000,60.1000) -- (95.9000,60.1000) -- (95.9000,60.1000) -- (95.9000,60.1000) -- (95.9000,60.0000) -- (95.9000,60.0000) -- (95.9000,60.0000) -- (95.9000,60.0000) -- (95.9000,60.0000) -- (95.9000,60.0000) -- (95.9000,60.0000) -- (95.9000,60.0000) -- (95.9000,60.0000) -- (95.9000,60.0000) -- (95.9000,60.0000) -- (95.9000,60.0000) -- (95.9000,60.0000) -- (95.9000,60.0000) -- (95.9000,60.0000) -- (95.9000,60.0000) -- (95.9000,60.0000) -- (95.9000,60.0000) -- (95.9000,60.0000) -- (95.9000,60.0000) -- (95.9000,60.0000) -- (95.9000,59.9000) -- (95.9000,59.9000) -- (95.9000,59.9000) -- (95.9000,59.9000) -- (95.9000,59.9000) -- (95.9000,59.9000) -- (95.9000,59.9000) -- (95.9000,59.9000) -- (95.9000,59.9000) -- (95.9000,59.9000) -- (95.9000,59.9000) -- (95.9000,59.9000) -- (95.9000,59.9000) -- (95.9000,59.9000) -- (95.9000,59.9000) -- (95.9000,59.9000) -- (95.9000,59.9000) -- (95.9000,59.9000) -- (95.9000,59.9000) -- (95.9000,59.9000) -- (95.9000,59.9000) -- (95.9000,59.8000) -- (96.0000,59.8000) -- (96.0000,59.8000) -- (96.0000,59.8000) -- (96.0000,59.8000) -- (96.0000,59.8000) -- (96.0000,59.8000) -- (96.0000,59.8000) -- (96.0000,59.8000) -- (96.0000,59.8000) -- (96.0000,59.8000) -- (96.0000,59.8000) -- (96.0000,59.8000) -- (96.0000,59.8000) -- (96.0000,59.8000) -- (96.0000,59.8000) -- (96.0000,59.8000) -- (96.0000,59.8000) -- (96.0000,59.8000) -- (96.0000,59.8000) -- (96.0000,59.7000) -- (96.0000,59.7000) -- (96.0000,59.7000) -- (96.0000,59.7000) -- (96.0000,59.7000) -- (96.0000,59.7000) -- (96.0000,59.7000) -- (96.0000,59.7000) -- (96.0000,59.7000) -- (96.0000,59.7000) -- (96.0000,59.7000) -- (96.0000,59.7000) -- (96.0000,59.7000) -- (96.0000,59.7000) -- (96.0000,59.7000) -- (96.0000,59.7000) -- (96.0000,59.7000) -- (96.0000,59.7000) -- (96.0000,59.7000) -- (96.0000,59.7000) -- (96.0000,59.7000) -- (96.0000,59.6000) -- (96.0000,59.6000) -- (96.0000,59.6000) -- (96.0000,59.6000) -- (96.0000,59.6000) -- (96.0000,59.6000) -- (96.0000,59.6000) -- (96.0000,59.6000) -- (96.0000,59.6000) -- (96.0000,59.6000) -- (96.1000,59.6000) -- (96.1000,59.6000) -- (96.1000,59.6000) -- (96.1000,59.6000) -- (96.1000,59.6000) -- (96.1000,59.6000) -- (96.1000,59.6000) -- (96.1000,59.6000) -- (96.1000,59.6000) -- (96.1000,59.6000) -- (96.1000,59.6000) -- (96.1000,59.5000) -- (96.1000,59.5000) -- (96.1000,59.5000) -- (96.1000,59.5000) -- (96.1000,59.5000) -- (96.1000,59.5000) -- (96.1000,59.5000) -- (96.1000,59.5000) -- (96.1000,59.5000) -- (96.1000,59.5000) -- (96.1000,59.5000) -- (96.1000,59.5000) -- (96.1000,59.5000) -- (96.1000,59.5000) -- (96.1000,59.5000) -- (96.1000,59.5000) -- (96.1000,59.5000) -- (96.1000,59.5000) -- (96.1000,59.5000) -- (96.1000,59.5000) -- (96.1000,59.5000) -- (96.1000,59.4000) -- (96.1000,59.4000) -- (96.1000,59.4000) -- (96.1000,59.4000) -- (96.1000,59.4000) -- (96.1000,59.4000) -- (96.1000,59.4000) -- (96.1000,59.4000) -- (96.1000,59.4000) -- (96.1000,59.4000) -- (96.1000,59.4000) -- (96.1000,59.4000) -- (96.1000,59.4000) -- (96.1000,59.4000) -- (96.1000,59.4000) -- (96.1000,59.4000) -- (96.1000,59.4000) -- (96.2000,59.4000) -- (96.2000,59.4000) -- (96.2000,59.4000) -- (96.2000,59.4000) -- (96.2000,59.3000) -- (96.2000,59.3000) -- (96.2000,59.3000) -- (96.2000,59.3000) -- (96.2000,59.3000) -- (96.2000,59.3000) -- (96.2000,59.3000) -- (96.2000,59.3000) -- (96.2000,59.3000) -- (96.2000,59.3000) -- (96.2000,59.3000) -- (96.2000,59.3000) -- (96.2000,59.3000) -- (96.2000,59.3000) -- (96.2000,59.3000) -- (96.2000,59.3000) -- (96.2000,59.3000) -- (96.2000,59.3000) -- (96.2000,59.3000) -- (96.2000,59.3000) -- (96.2000,59.2000) -- (96.2000,59.2000) -- (96.2000,59.2000) -- (96.2000,59.2000) -- (96.2000,59.2000) -- (96.2000,59.2000) -- (96.2000,59.2000) -- (96.2000,59.2000) -- (96.2000,59.2000) -- (96.2000,59.2000) -- (96.2000,59.2000) -- (96.2000,59.2000) -- (96.2000,59.2000) -- (96.2000,59.2000) -- (96.2000,59.2000) -- (96.2000,59.2000) -- (96.2000,59.2000) -- (96.2000,59.2000) -- (96.2000,59.2000) -- (96.2000,59.2000) -- (96.2000,59.2000) -- (96.2000,59.1000) -- (96.2000,59.1000) -- (96.2000,59.1000) -- (96.2000,59.1000) -- (96.2000,59.1000) -- (96.3000,59.1000) -- (96.3000,59.1000) -- (96.3000,59.1000) -- (96.3000,59.1000) -- (96.3000,59.1000) -- (96.3000,59.1000) -- (96.3000,59.1000) -- (96.3000,59.1000) -- (96.3000,59.1000) -- (96.3000,59.1000) -- (96.3000,59.1000) -- (96.3000,59.1000) -- (96.3000,59.1000) -- (96.3000,59.1000) -- (96.3000,59.1000) -- (96.3000,59.1000) -- (96.3000,59.0000) -- (96.3000,59.0000) -- (96.3000,59.0000) -- (96.3000,59.0000) -- (96.3000,59.0000) -- (96.3000,59.0000) -- (96.3000,59.0000) -- (96.3000,59.0000) -- (96.3000,59.0000) -- (96.3000,59.0000) -- (96.3000,59.0000) -- (96.3000,59.0000) -- (96.3000,59.0000) -- (96.3000,59.0000) -- (96.3000,59.0000) -- (96.3000,59.0000) -- (96.3000,59.0000) -- (96.3000,59.0000) -- (96.3000,59.0000) -- (96.3000,59.0000) -- (96.3000,59.0000) -- (96.3000,58.9000) -- (96.3000,58.9000) -- (96.3000,58.9000) -- (96.3000,58.9000) -- (96.3000,58.9000) -- (96.3000,58.9000) -- (96.3000,58.9000) -- (96.3000,58.9000) -- (96.3000,58.9000) -- (96.3000,58.9000) -- (96.3000,58.9000) -- (96.3000,58.9000) -- (96.3000,58.9000) -- (96.4000,58.9000) -- (96.4000,58.9000) -- (96.4000,58.9000) -- (96.4000,58.9000) -- (96.4000,58.9000) -- (96.4000,58.9000) -- (96.4000,58.9000) -- (96.4000,58.9000) -- (96.4000,58.8000) -- (96.4000,58.8000) -- (96.4000,58.8000) -- (96.4000,58.8000) -- (96.4000,58.8000) -- (96.4000,58.8000) -- (96.4000,58.8000) -- (96.4000,58.8000) -- (96.4000,58.8000) -- (96.4000,58.8000) -- (96.4000,58.8000) -- (96.4000,58.8000) -- (96.4000,58.8000) -- (96.4000,58.8000) -- (96.4000,58.8000) -- (96.4000,58.8000) -- (96.4000,58.8000) -- (96.4000,58.8000) -- (96.4000,58.8000) -- (96.4000,58.8000) -- (96.4000,58.7000) -- (96.4000,58.7000) -- (96.4000,58.7000) -- (96.4000,58.7000) -- (96.4000,58.7000) -- (96.4000,58.7000) -- (96.4000,58.7000) -- (96.4000,58.7000) -- (96.4000,58.7000) -- (96.4000,58.7000) -- (96.4000,58.7000) -- (96.4000,58.7000) -- (96.4000,58.7000) -- (96.4000,58.7000) -- (96.4000,58.7000) -- (96.4000,58.7000) -- (96.4000,58.7000) -- (96.4000,58.7000) -- (96.4000,58.7000) -- (96.4000,58.7000) -- (96.4000,58.7000) -- (96.5000,58.6000) -- (96.5000,58.6000) -- (96.5000,58.6000) -- (96.5000,58.6000) -- (96.5000,58.6000) -- (96.5000,58.6000) -- (96.5000,58.6000) -- (96.5000,58.6000) -- (96.5000,58.6000) -- (96.5000,58.6000) -- (96.5000,58.6000) -- (96.5000,58.6000) -- (96.5000,58.6000) -- (96.5000,58.6000) -- (96.5000,58.6000) -- (96.5000,58.6000) -- (96.5000,58.6000) -- (96.5000,58.6000) -- (96.5000,58.6000) -- (96.5000,58.6000) -- (96.5000,58.6000) -- (96.5000,58.5000) -- (96.5000,58.5000) -- (96.5000,58.5000) -- (96.5000,58.5000) -- (96.5000,58.5000) -- (96.5000,58.5000) -- (96.5000,58.5000) -- (96.5000,58.5000) -- (96.5000,58.5000) -- (96.5000,58.5000) -- (96.5000,58.5000) -- (96.5000,58.5000) -- (96.5000,58.5000) -- (96.5000,58.5000) -- (96.5000,58.5000) -- (96.5000,58.5000) -- (96.5000,58.5000) -- (96.5000,58.5000) -- (96.5000,58.5000) -- (96.5000,58.5000) -- (96.5000,58.5000) -- (96.5000,58.4000) -- (96.5000,58.4000) -- (96.5000,58.4000) -- (96.5000,58.4000) -- (96.5000,58.4000) -- (96.5000,58.4000) -- (96.5000,58.4000) -- (96.5000,58.4000) -- (96.6000,58.4000) -- (96.6000,58.4000) -- (96.6000,58.4000) -- (96.6000,58.4000) -- (96.6000,58.4000) -- (96.6000,58.4000) -- (96.6000,58.4000) -- (96.6000,58.4000) -- (96.6000,58.4000) -- (96.6000,58.4000) -- (96.6000,58.4000) -- (96.6000,58.4000) -- (96.6000,58.3000) -- (96.6000,58.3000) -- (96.6000,58.3000) -- (96.6000,58.3000) -- (96.6000,58.3000) -- (96.6000,58.3000) -- (96.6000,58.3000) -- (96.6000,58.3000) -- (96.6000,58.3000) -- (96.6000,58.3000) -- (96.6000,58.3000) -- (96.6000,58.3000) -- (96.6000,58.3000) -- (96.6000,58.3000) -- (96.6000,58.3000) -- (96.6000,58.3000) -- (96.6000,58.3000) -- (96.6000,58.3000) -- (96.6000,58.3000) -- (96.6000,58.3000) -- (96.6000,58.3000) -- (96.6000,58.2000) -- (96.6000,58.2000) -- (96.6000,58.2000) -- (96.6000,58.2000) -- (96.6000,58.2000) -- (96.6000,58.2000) -- (96.6000,58.2000) -- (96.6000,58.2000) -- (96.6000,58.2000) -- (96.6000,58.2000) -- (96.6000,58.2000) -- (96.6000,58.2000) -- (96.6000,58.2000) -- (96.6000,58.2000) -- (96.6000,58.2000) -- (96.6000,58.2000) -- (96.7000,58.2000) -- (96.7000,58.2000) -- (96.7000,58.2000) -- (96.7000,58.2000) -- (96.7000,58.2000) -- (96.7000,58.1000) -- (96.7000,58.1000) -- (96.7000,58.1000) -- (96.7000,58.1000) -- (96.7000,58.1000) -- (96.7000,58.1000) -- (96.7000,58.1000) -- (96.7000,58.1000) -- (96.7000,58.1000) -- (96.7000,58.1000) -- (96.7000,58.1000) -- (96.7000,58.1000) -- (96.7000,58.1000) -- (96.7000,58.1000) -- (96.7000,58.1000) -- (96.7000,58.1000) -- (96.7000,58.1000) -- (96.7000,58.1000) -- (96.7000,58.1000) -- (96.7000,58.1000) -- (96.7000,58.1000) -- (96.7000,58.0000) -- (96.7000,58.0000) -- (96.7000,58.0000) -- (96.7000,58.0000) -- (96.7000,58.0000) -- (96.7000,58.0000) -- (96.7000,58.0000) -- (96.7000,58.0000) -- (96.7000,58.0000) -- (96.7000,58.0000) -- (96.7000,58.0000) -- (96.7000,58.0000) -- (96.7000,58.0000) -- (96.7000,58.0000) -- (96.7000,58.0000) -- (96.7000,58.0000) -- (96.7000,58.0000) -- (96.7000,58.0000) -- (96.7000,58.0000) -- (96.7000,58.0000) -- (96.7000,58.0000) -- (96.7000,57.9000) -- (96.7000,57.9000) -- (96.7000,57.9000) -- (96.8000,57.9000) -- (96.8000,57.9000) -- (96.8000,57.9000) -- (96.8000,57.9000) -- (96.8000,57.9000) -- (96.8000,57.9000) -- (96.8000,57.9000) -- (96.8000,57.9000) -- (96.8000,57.9000) -- (96.8000,57.9000) -- (96.8000,57.9000) -- (96.8000,57.9000) -- (96.8000,57.9000) -- (96.8000,57.9000) -- (96.8000,57.9000) -- (96.8000,57.9000) -- (96.8000,57.9000) -- (96.8000,57.8000) -- (96.8000,57.8000) -- (96.8000,57.8000) -- (96.8000,57.8000) -- (96.8000,57.8000) -- (96.8000,57.8000) -- (96.8000,57.8000) -- (96.8000,57.8000) -- (96.8000,57.8000) -- (96.8000,57.8000) -- (96.8000,57.8000) -- (96.8000,57.8000) -- (96.8000,57.8000) -- (96.8000,57.8000) -- (96.8000,57.8000) -- (96.8000,57.8000) -- (96.8000,57.8000) -- (96.8000,57.8000) -- (96.8000,57.8000) -- (96.8000,57.8000) -- (96.8000,57.8000) -- (96.8000,57.7000) -- (96.8000,57.7000) -- (96.8000,57.7000) -- (96.8000,57.7000) -- (96.8000,57.7000) -- (96.8000,57.7000) -- (96.8000,57.7000) -- (96.8000,57.7000) -- (96.8000,57.7000) -- (96.8000,57.7000) -- (96.8000,57.7000) -- (96.9000,57.7000) -- (96.9000,57.7000) -- (96.9000,57.7000) -- (96.9000,57.7000) -- (96.9000,57.7000) -- (96.9000,57.7000) -- (96.9000,57.7000) -- (96.9000,57.7000) -- (96.9000,57.7000) -- (96.9000,57.7000) -- (96.9000,57.6000) -- (96.9000,57.6000) -- (96.9000,57.6000) -- (96.9000,57.6000) -- (96.9000,57.6000) -- (96.9000,57.6000) -- (96.9000,57.6000) -- (96.9000,57.6000) -- (96.9000,57.6000) -- (96.9000,57.6000) -- (96.9000,57.6000) -- (96.9000,57.6000) -- (96.9000,57.6000) -- (96.9000,57.6000) -- (96.9000,57.6000) -- (96.9000,57.6000) -- (96.9000,57.6000) -- (96.9000,57.6000) -- (96.9000,57.6000) -- (96.9000,57.6000) -- (96.9000,57.6000) -- (96.9000,57.5000) -- (96.9000,57.5000) -- (96.9000,57.5000) -- (96.9000,57.5000) -- (96.9000,57.5000) -- (96.9000,57.5000) -- (96.9000,57.5000) -- (96.9000,57.5000) -- (96.9000,57.5000) -- (96.9000,57.5000) -- (96.9000,57.5000) -- (96.9000,57.5000) -- (96.9000,57.5000) -- (96.9000,57.5000) -- (96.9000,57.5000) -- (96.9000,57.5000) -- (96.9000,57.5000) -- (96.9000,57.5000) -- (96.9000,57.5000) -- (97.0000,57.5000) -- (97.0000,57.5000) -- (97.0000,57.4000) -- (97.0000,57.4000) -- (97.0000,57.4000) -- (97.0000,57.4000) -- (97.0000,57.4000) -- (97.0000,57.4000) -- (97.0000,57.4000) -- (97.0000,57.4000) -- (97.0000,57.4000) -- (97.0000,57.4000) -- (97.0000,57.4000) -- (97.0000,57.4000) -- (97.0000,57.4000) -- (97.0000,57.4000) -- (97.0000,57.4000) -- (97.0000,57.4000) -- (97.0000,57.4000) -- (97.0000,57.4000) -- (97.0000,57.4000) -- (97.0000,57.4000) -- (97.0000,57.3000) -- (97.0000,57.3000) -- (97.0000,57.3000) -- (97.0000,57.3000) -- (97.0000,57.3000) -- (97.0000,57.3000) -- (97.0000,57.3000) -- (97.0000,57.3000) -- (97.0000,57.3000) -- (97.0000,57.3000) -- (97.0000,57.3000) -- (97.0000,57.3000) -- (97.0000,57.3000) -- (97.0000,57.3000) -- (97.0000,57.3000) -- (97.0000,57.3000) -- (97.0000,57.3000) -- (97.0000,57.3000) -- (97.0000,57.3000) -- (97.0000,57.3000) -- (97.0000,57.3000) -- (97.0000,57.2000) -- (97.0000,57.2000) -- (97.0000,57.2000) -- (97.0000,57.2000) -- (97.0000,57.2000) -- (97.0000,57.2000) -- (97.1000,57.2000) -- (97.1000,57.2000) -- (97.1000,57.2000) -- (97.1000,57.2000) -- (97.1000,57.2000) -- (97.1000,57.2000) -- (97.1000,57.2000) -- (97.1000,57.2000) -- (97.1000,57.2000) -- (97.1000,57.2000) -- (97.1000,57.2000) -- (97.1000,57.2000) -- (97.1000,57.2000) -- (97.1000,57.2000) -- (97.1000,57.2000) -- (97.1000,57.1000) -- (97.1000,57.1000) -- (97.1000,57.1000) -- (97.1000,57.1000) -- (97.1000,57.1000) -- (97.1000,57.1000) -- (97.1000,57.1000) -- (97.1000,57.1000) -- (97.1000,57.1000) -- (97.1000,57.1000) -- (97.1000,57.1000) -- (97.1000,57.1000) -- (97.1000,57.1000) -- (97.1000,57.1000) -- (97.1000,57.1000) -- (97.1000,57.1000) -- (97.1000,57.1000) -- (97.1000,57.1000) -- (97.1000,57.1000) -- (97.1000,57.1000) -- (97.1000,57.1000) -- (97.1000,57.0000) -- (97.1000,57.0000) -- (97.1000,57.0000) -- (97.1000,57.0000) -- (97.1000,57.0000) -- (97.1000,57.0000) -- (97.1000,57.0000) -- (97.1000,57.0000) -- (97.1000,57.0000) -- (97.1000,57.0000) -- (97.1000,57.0000) -- (97.1000,57.0000) -- (97.1000,57.0000) -- (97.1000,57.0000) -- (97.2000,57.0000) -- (97.2000,57.0000) -- (97.2000,57.0000) -- (97.2000,57.0000) -- (97.2000,57.0000) -- (97.2000,57.0000) -- (97.2000,56.9000) -- (97.2000,56.9000) -- (97.2000,56.9000) -- (97.2000,56.9000) -- (97.2000,56.9000) -- (97.2000,56.9000) -- (97.2000,56.9000) -- (97.2000,56.9000) -- (97.2000,56.9000) -- (97.2000,56.9000) -- (97.2000,56.9000) -- (97.2000,56.9000) -- (97.2000,56.9000) -- (97.2000,56.9000) -- (97.2000,56.9000) -- (97.2000,56.9000) -- (97.2000,56.9000) -- (97.2000,56.9000) -- (97.2000,56.9000) -- (97.2000,56.9000) -- (97.2000,56.9000) -- (97.2000,56.8000) -- (97.2000,56.8000) -- (97.2000,56.8000) -- (97.2000,56.8000) -- (97.2000,56.8000) -- (97.2000,56.8000) -- (97.2000,56.8000) -- (97.2000,56.8000) -- (97.2000,56.8000) -- (97.2000,56.8000) -- (97.2000,56.8000) -- (97.2000,56.8000) -- (97.2000,56.8000) -- (97.2000,56.8000) -- (97.2000,56.8000) -- (97.2000,56.8000) -- (97.2000,56.8000) -- (97.2000,56.8000) -- (97.2000,56.8000) -- (97.2000,56.8000) -- (97.2000,56.8000) -- (97.2000,56.7000) -- (97.3000,56.7000) -- (97.3000,56.7000) -- (97.3000,56.7000) -- (97.3000,56.7000) -- (97.3000,56.7000) -- (97.3000,56.7000) -- (97.3000,56.7000) -- (97.3000,56.7000) -- (97.3000,56.7000) -- (97.3000,56.7000) -- (97.3000,56.7000) -- (97.3000,56.7000) -- (97.3000,56.7000) -- (97.3000,56.7000) -- (97.3000,56.7000) -- (97.3000,56.7000) -- (97.3000,56.7000) -- (97.3000,56.7000) -- (97.3000,56.7000) -- (97.3000,56.7000) -- (97.3000,56.6000) -- (97.3000,56.6000) -- (97.3000,56.6000) -- (97.3000,56.6000) -- (97.3000,56.6000) -- (97.3000,56.6000) -- (97.3000,56.6000) -- (97.3000,56.6000) -- (97.3000,56.6000) -- (97.3000,56.6000) -- (97.3000,56.6000) -- (97.3000,56.6000) -- (97.3000,56.6000) -- (97.3000,56.6000) -- (97.3000,56.6000) -- (97.3000,56.6000) -- (97.3000,56.6000) -- (97.3000,56.6000) -- (97.3000,56.6000) -- (97.3000,56.6000) -- (97.3000,56.6000) -- (97.3000,56.5000) -- (97.3000,56.5000) -- (97.3000,56.5000) -- (97.3000,56.5000) -- (97.3000,56.5000) -- (97.3000,56.5000) -- (97.3000,56.5000) -- (97.3000,56.5000) -- (97.3000,56.5000) -- (97.4000,56.5000) -- (97.4000,56.5000) -- (97.4000,56.5000) -- (97.4000,56.5000) -- (97.4000,56.5000) -- (97.4000,56.5000) -- (97.4000,56.5000) -- (97.4000,56.5000) -- (97.4000,56.5000) -- (97.4000,56.5000) -- (97.4000,56.5000) -- (97.4000,56.4000) -- (97.4000,56.4000) -- (97.4000,56.4000) -- (97.4000,56.4000) -- (97.4000,56.4000) -- (97.4000,56.4000) -- (97.4000,56.4000) -- (97.4000,56.4000) -- (97.4000,56.4000) -- (97.4000,56.4000) -- (97.4000,56.4000) -- (97.4000,56.4000) -- (97.4000,56.4000) -- (97.4000,56.4000) -- (97.4000,56.4000) -- (97.4000,56.4000) -- (97.4000,56.4000) -- (97.4000,56.4000) -- (97.4000,56.4000) -- (97.4000,56.4000) -- (97.4000,56.4000) -- (97.4000,56.3000) -- (97.4000,56.3000) -- (97.4000,56.3000) -- (97.4000,56.3000) -- (97.4000,56.3000) -- (97.4000,56.3000) -- (97.4000,56.3000) -- (97.4000,56.3000) -- (97.4000,56.3000) -- (97.4000,56.3000) -- (97.4000,56.3000) -- (97.4000,56.3000) -- (97.4000,56.3000) -- (97.4000,56.3000) -- (97.4000,56.3000) -- (97.4000,56.3000) -- (97.4000,56.3000) -- (97.5000,56.3000) -- (97.5000,56.3000) -- (97.5000,56.3000) -- (97.5000,56.3000) -- (97.5000,56.2000) -- (97.5000,56.2000) -- (97.5000,56.2000) -- (97.5000,56.2000) -- (97.5000,56.2000) -- (97.5000,56.2000) -- (97.5000,56.2000) -- (97.5000,56.2000) -- (97.5000,56.2000) -- (97.5000,56.2000) -- (97.5000,56.2000) -- (97.5000,56.2000) -- (97.5000,56.2000) -- (97.5000,56.2000) -- (97.5000,56.2000) -- (97.5000,56.2000) -- (97.5000,56.2000) -- (97.5000,56.2000) -- (97.5000,56.2000) -- (97.5000,56.2000) -- (97.5000,56.2000) -- (97.5000,56.1000) -- (97.5000,56.1000) -- (97.5000,56.1000) -- (97.5000,56.1000) -- (97.5000,56.1000) -- (97.5000,56.1000) -- (97.5000,56.1000) -- (97.5000,56.1000) -- (97.5000,56.1000) -- (97.5000,56.1000) -- (97.5000,56.1000) -- (97.5000,56.1000) -- (97.5000,56.1000) -- (97.5000,56.1000) -- (97.5000,56.1000) -- (97.5000,56.1000) -- (97.5000,56.1000) -- (97.5000,56.1000) -- (97.5000,56.1000) -- (97.5000,56.1000) -- (97.5000,56.1000) -- (97.5000,56.0000) -- (97.5000,56.0000) -- (97.5000,56.0000) -- (97.5000,56.0000) -- (97.6000,56.0000) -- (97.6000,56.0000) -- (97.6000,56.0000) -- (97.6000,56.0000) -- (97.6000,56.0000) -- (97.6000,56.0000) -- (97.6000,56.0000) -- (97.6000,56.0000) -- (97.6000,56.0000) -- (97.6000,56.0000) -- (97.6000,56.0000) -- (97.6000,56.0000) -- (97.6000,56.0000) -- (97.6000,56.0000) -- (97.6000,56.0000) -- (97.6000,56.0000) -- (97.6000,55.9000) -- (97.6000,55.9000) -- (97.6000,55.9000) -- (97.6000,55.9000) -- (97.6000,55.9000) -- (97.6000,55.9000) -- (97.6000,55.9000) -- (97.6000,55.9000) -- (97.6000,55.9000) -- (97.6000,55.9000) -- (97.6000,55.9000) -- (97.6000,55.9000) -- (97.6000,55.9000) -- (97.6000,55.9000) -- (97.6000,55.9000) -- (97.6000,55.9000) -- (97.6000,55.9000) -- (97.6000,55.9000) -- (97.6000,55.9000) -- (97.6000,55.9000) -- (97.6000,55.9000) -- (97.6000,55.8000) -- (97.6000,55.8000) -- (97.6000,55.8000) -- (97.6000,55.8000) -- (97.6000,55.8000) -- (97.6000,55.8000) -- (97.6000,55.8000) -- (97.6000,55.8000) -- (97.6000,55.8000) -- (97.6000,55.8000) -- (97.6000,55.8000) -- (97.6000,55.8000) -- (97.7000,55.8000) -- (97.7000,55.8000) -- (97.7000,55.8000) -- (97.7000,55.8000) -- (97.7000,55.8000) -- (97.7000,55.8000) -- (97.7000,55.8000) -- (97.7000,55.8000) -- (97.7000,55.8000) -- (97.7000,55.7000) -- (97.7000,55.7000) -- (97.7000,55.7000) -- (97.7000,55.7000) -- (97.7000,55.7000) -- (97.7000,55.7000) -- (97.7000,55.7000) -- (121.4000,55.9000);



      \end{scope}
      \begin{scope}[cm={{1.27491,0.0,0.0,1.41542,(-124.58597,-493.0531)}},draw=blue,line cap=round,line join=round,line width=0.480pt]
        \path[draw] (81.5000,50.5000) -- (81.5000,78.5000) -- (121.5000,78.5000) -- (121.5000,50.5000) -- (81.5000,50.5000);



      \end{scope}
      \begin{scope}[cm={{1.05023,0.0,0.0,1.05023,(-51.68473,-410.86372)}},draw=blue,line cap=rect,line join=bevel,line width=0.800pt]
        \path[fill=blue] (0.0000,0.0000) node[above right] (text64-7-2) {\scriptsize $T$\hspace{.5ex}=\hspace{.5ex}46};



      \end{scope}
      \begin{scope}[cm={{1.05023,0.0,0.0,1.05023,(-3.79554,-425.35157)}},draw=blue,line cap=rect,line join=bevel,line width=0.800pt]
        \path[fill=blue] (0.0000,0.0000) node[above right] (text64-7-8) {\scriptsize 83};



      \end{scope}
    \end{scope}
    \begin{scope}[cm={{0.74279,0.0,0.0,1.28515,(-186.22138,-161.30028)}},draw=blue,line cap=round,line join=round,line width=0.480pt]
      \path[cm={{1.0,0.0,0.0,0.57583,(0.0,64.75889)}},draw] (130.5000,152.5000) -- (130.5000,147.5000);



      \path[cm={{1.0,0.0,0.0,0.70101,(-0.0,38.33083)}},draw] (130.5000,128.5000) -- (130.5000,132.5000);



    \end{scope}
    \begin{scope}[cm={{0.74279,0.0,0.0,1.28515,(-186.22138,-161.30028)}},draw=blue,line cap=round,line join=round,line width=0.480pt]
      \path[draw] (25.5000,128.5000) -- (25.5000,152.5000) -- (142.5000,152.5000) -- (142.5000,128.5000) -- (25.5000,128.5000);



    \end{scope}
    \begin{scope}[cm={{0.95389,0.0,0.0,0.95389,(-108.68182,28.37633)}},draw=blue,line cap=rect,line join=bevel,line width=0.800pt]
      \path[fill=blue] (3.8674,-1.6114) node[above right] (text1264) {\scriptsize $\alpha_3$};



    \end{scope}
    \begin{scope}[cm={{0.74279,0.0,0.0,1.28515,(-184.07401,-166.61499)}},draw=blue,line cap=round,line join=round,line width=0.480pt]
      \path[draw,even odd rule] (123.5000,148.5000) -- (132.5000,148.5000);



    \end{scope}
    \begin{scope}[cm={{0.74279,0.0,0.0,1.28515,(-186.22138,-161.30028)}},draw=blue,line cap=round,line join=round,line width=0.480pt]
      \path[draw] (25.8000,136.6000) -- (25.8000,136.6000) -- (26.2000,137.3000) -- (26.7000,137.7000) -- (27.1000,136.6000) -- (27.5000,137.5000) -- (27.9000,137.0000) -- (28.3000,139.3000) -- (28.8000,132.8000) -- (29.2000,140.7000) -- (29.6000,139.7000) -- (30.0000,133.1000) -- (30.5000,135.4000) -- (30.9000,140.8000) -- (31.3000,140.7000) -- (31.7000,136.2000) -- (32.2000,133.3000) -- (32.6000,134.6000) -- (33.0000,138.2000) -- (33.4000,140.7000) -- (33.8000,140.6000) -- (34.3000,138.4000) -- (34.7000,135.8000) -- (35.1000,134.3000) -- (35.5000,134.4000) -- (36.0000,136.0000) -- (36.4000,138.0000) -- (36.8000,139.6000) -- (37.2000,140.3000) -- (37.6000,140.0000) -- (38.1000,138.8000) -- (38.5000,137.3000) -- (38.9000,135.9000) -- (39.3000,135.0000) -- (39.8000,134.7000) -- (40.2000,135.0000) -- (40.6000,135.8000) -- (41.0000,136.8000) -- (41.5000,137.9000) -- (41.9000,138.8000) -- (42.3000,139.4000) -- (42.7000,139.6000) -- (43.1000,139.4000) -- (43.6000,139.0000) -- (44.0000,138.3000) -- (44.4000,137.4000) -- (44.8000,136.6000) -- (45.3000,136.0000) -- (45.7000,135.5000) -- (46.1000,135.2000) -- (46.5000,135.2000) -- (47.0000,135.5000) -- (47.4000,135.9000) -- (47.8000,136.4000) -- (48.2000,137.0000) -- (48.6000,137.6000) -- (49.1000,138.1000) -- (49.5000,138.5000) -- (49.9000,138.8000) -- (50.3000,138.9000) -- (50.8000,138.8000) -- (51.2000,138.5000) -- (51.6000,138.1000) -- (52.0000,137.6000) -- (52.4000,137.1000) -- (52.9000,136.5000) -- (53.3000,136.1000) -- (53.7000,135.7000) -- (54.1000,135.6000) -- (54.6000,135.6000) -- (55.0000,135.8000) -- (55.4000,136.1000) -- (55.8000,136.6000) -- (56.3000,137.1000) -- (56.7000,137.6000) -- (57.1000,138.0000) -- (57.5000,138.3000) -- (57.9000,138.6000) -- (58.4000,138.7000) -- (58.8000,138.6000) -- (59.2000,138.4000) -- (59.6000,138.1000) -- (60.1000,137.8000) -- (60.5000,137.3000) -- (60.9000,136.9000) -- (61.3000,136.6000) -- (61.8000,136.4000) -- (62.2000,136.3000) -- (62.6000,136.4000) -- (63.0000,136.5000) -- (63.4000,136.7000) -- (63.9000,137.0000) -- (64.3000,137.4000) -- (64.7000,137.7000) -- (65.1000,138.0000) -- (65.6000,138.2000) -- (66.0000,138.3000) -- (66.4000,138.2000) -- (66.8000,138.1000) -- (67.3000,137.9000) -- (67.7000,137.7000) -- (68.1000,137.4000) -- (68.5000,137.1000) -- (68.9000,136.8000) -- (69.4000,136.6000) -- (69.8000,136.5000) -- (70.2000,136.5000) -- (70.6000,136.6000) -- (71.1000,136.7000) -- (71.5000,136.9000) -- (71.9000,137.0000) -- (72.3000,137.2000) -- (72.7000,137.4000) -- (73.2000,137.6000) -- (73.6000,137.8000) -- (74.0000,137.9000) -- (74.4000,138.0000) -- (74.9000,138.0000) -- (75.3000,137.9000) -- (75.7000,137.8000) -- (76.1000,137.7000) -- (76.6000,137.5000) -- (77.0000,137.3000) -- (77.4000,137.2000) -- (77.8000,137.1000) -- (78.2000,136.9000) -- (78.7000,136.8000) -- (79.1000,136.8000) -- (79.5000,136.8000) -- (79.9000,136.8000) -- (80.4000,136.8000) -- (80.8000,136.9000) -- (81.2000,137.0000) -- (81.6000,137.1000) -- (82.1000,137.3000) -- (82.5000,137.4000) -- (82.9000,137.5000) -- (83.3000,137.6000) -- (83.7000,137.6000) -- (84.2000,137.6000) -- (84.6000,137.6000) -- (85.0000,137.6000) -- (85.4000,137.5000) -- (85.9000,137.4000) -- (86.3000,137.3000) -- (86.7000,137.2000) -- (87.1000,137.1000) -- (87.6000,137.0000) -- (88.0000,136.9000) -- (88.4000,136.9000) -- (88.8000,137.0000) -- (89.2000,137.0000) -- (89.7000,137.1000) -- (90.1000,137.2000) -- (90.5000,137.3000) -- (90.9000,137.4000) -- (91.4000,137.5000) -- (91.8000,137.5000) -- (92.2000,137.6000) -- (92.6000,137.6000) -- (93.0000,137.6000) -- (93.5000,137.5000) -- (93.9000,137.5000) -- (94.3000,137.4000) -- (94.7000,137.3000) -- (95.2000,137.2000) -- (95.6000,137.1000) -- (96.0000,137.1000) -- (96.4000,137.0000) -- (96.9000,137.1000) -- (97.3000,137.1000) -- (97.7000,137.1000) -- (98.1000,137.2000) -- (98.5000,137.3000) -- (99.0000,137.3000) -- (99.4000,137.4000) -- (99.8000,137.4000) -- (100.2000,137.5000) -- (100.7000,137.5000) -- (101.1000,137.6000) -- (101.5000,137.5000) -- (101.9000,137.4000) -- (102.3000,137.3000) -- (102.8000,137.3000) -- (103.2000,137.2000) -- (103.6000,137.1000) -- (104.0000,137.1000) -- (104.5000,137.0000) -- (104.9000,137.0000) -- (105.3000,137.0000) -- (105.7000,137.1000) -- (106.2000,137.1000) -- (106.6000,137.2000) -- (107.0000,137.3000) -- (107.4000,137.3000) -- (107.8000,137.4000) -- (108.3000,137.5000) -- (108.7000,137.6000) -- (109.1000,137.6000) -- (109.5000,137.7000) -- (110.0000,137.6000) -- (110.4000,137.6000) -- (110.8000,137.5000) -- (111.2000,137.4000) -- (111.7000,137.3000) -- (112.1000,137.2000) -- (112.5000,137.1000) -- (112.9000,137.0000) -- (113.3000,137.0000) -- (113.8000,136.9000) -- (114.2000,136.9000) -- (114.6000,136.9000) -- (115.0000,136.9000) -- (115.5000,137.0000) -- (115.9000,137.2000) -- (116.3000,137.3000) -- (116.7000,137.4000) -- (117.2000,137.5000) -- (117.6000,137.6000) -- (118.0000,137.7000) -- (118.4000,137.7000) -- (118.8000,137.7000) -- (119.3000,137.7000) -- (119.7000,137.6000) -- (120.1000,137.5000) -- (120.5000,137.4000) -- (121.0000,137.2000) -- (121.4000,137.1000) -- (121.8000,137.0000) -- (122.2000,136.8000) -- (122.6000,136.8000) -- (123.1000,136.8000) -- (123.5000,136.8000) -- (123.9000,136.9000) -- (124.3000,137.0000) -- (124.8000,137.2000) -- (125.2000,137.3000) -- (125.6000,137.4000) -- (126.0000,137.6000) -- (126.5000,137.7000) -- (126.9000,137.7000) -- (127.3000,137.8000) -- (127.7000,137.8000) -- (128.1000,137.8000) -- (128.6000,137.7000) -- (129.0000,137.5000) -- (129.4000,137.4000) -- (129.8000,137.3000) -- (130.3000,137.1000) -- (130.7000,137.0000) -- (131.1000,136.9000) -- (131.5000,136.9000) -- (131.9000,136.8000) -- (132.4000,136.8000) -- (132.8000,136.9000) -- (133.2000,137.0000) -- (133.6000,137.1000) -- (134.1000,137.2000) -- (134.5000,137.3000) -- (134.9000,137.4000) -- (135.3000,137.6000) -- (135.8000,137.7000) -- (136.2000,137.8000) -- (136.6000,137.8000) -- (137.0000,137.8000) -- (137.4000,137.7000) -- (137.9000,137.6000) -- (138.3000,137.5000) -- (138.7000,137.4000) -- (139.1000,137.3000) -- (139.6000,137.1000) -- (140.0000,137.0000) -- (140.4000,136.9000) -- (140.8000,136.8000) -- (141.3000,136.8000) -- (141.7000,136.8000) -- (142.1000,136.9000) -- (142.3000,136.9000);



    \end{scope}
    \begin{scope}[cm={{0.7438,0.0,0.0,0.77563,(-64.025,-157.04212)}},draw=blue,line cap=round,line join=round,line width=0.480pt]
    \end{scope}
    \begin{scope}[cm={{0.74279,0.0,0.0,1.28515,(-186.22138,-161.30028)}},draw=ca0a0a4,dash pattern=on 1.22pt off 1.22pt,line cap=round,line join=round,line width=0.305pt,miter limit=4.00]
      \path[draw,dash pattern=on 1.22pt off 1.22pt,line width=0.305pt,miter limit=4.00] (25.5000,168.5000) -- (108.5000,168.5000);



      \path[draw,dash pattern=on 1.22pt off 1.22pt,line width=0.305pt,miter limit=4.00] (137.5000,168.5000) -- (142.5000,168.5000);



    \end{scope}
    \begin{scope}[cm={{0.74279,0.0,0.0,1.28515,(-186.22138,-161.30028)}},draw=blue,line cap=round,line join=round,line width=0.480pt]
      \path[cm={{1.54975,0.0,0.0,1.0,(-13.85377,0.0)}},draw] (25.5000,168.5000) -- (28.5000,168.5000);



      \path[cm={{1.54975,0.0,0.0,1.0,(-78.53317,0.0)}},draw] (142.5000,168.5000) -- (139.5000,168.5000);



    \end{scope}
    \begin{scope}[cm={{0.74279,0.0,0.0,1.28515,(-186.22138,-161.30028)}},draw=ca0a0a4,dash pattern=on 1.22pt off 1.22pt,line cap=round,line join=round,line width=0.305pt,miter limit=4.00]
      \path[draw,dash pattern=on 1.22pt off 1.22pt,line width=0.305pt,miter limit=4.00] (25.5000,153.5000) -- (142.5000,153.5000);



    \end{scope}
    \begin{scope}[cm={{0.74279,0.0,0.0,1.28515,(-186.22138,-161.30028)}},draw=blue,line cap=round,line join=round,line width=0.480pt]
      \path[cm={{1.54975,0.0,0.0,1.0,(-13.85377,0.0)}},draw] (25.5000,153.5000) -- (28.5000,153.5000);



      \path[cm={{1.54975,0.0,0.0,1.0,(-78.53317,0.0)}},draw] (142.5000,153.5000) -- (139.5000,153.5000);



    \end{scope}
    \begin{scope}[cm={{0.95389,0.0,0.0,0.95389,(-180.91222,39.27539)}},draw=blue,fill=ce10000,line cap=rect,line join=bevel,line width=0.800pt]
      \path[fill=ce10000] (0.0000,0.0000) node[above right] (text1350) {\scriptsize 20};



    \end{scope}
    \begin{scope}[cm={{0.74279,0.0,0.0,1.28515,(-186.22138,-161.30028)}},draw=ca0a0a4,dash pattern=on 0.40pt off 0.80pt,line cap=round,line join=round,line width=0.400pt]
      \path[draw] (25.5000,176.5000) -- (25.5000,152.5000);



    \end{scope}
    \begin{scope}[cm={{0.74279,0.0,0.0,1.28515,(-186.22138,-161.30028)}},draw=blue,line cap=round,line join=round,line width=0.480pt]
      \path[draw] (25.5000,176.5000) -- (25.5000,171.5000);



      \path[draw] (25.5000,152.5000) -- (25.5000,156.5000);



    \end{scope}
    \begin{scope}[cm={{0.74279,0.0,0.0,1.28515,(-186.22138,-161.30028)}},draw=ca0a0a4,dash pattern=on 1.22pt off 1.22pt,line cap=round,line join=round,line width=0.305pt,miter limit=4.00]
      \path[draw,dash pattern=on 1.22pt off 1.22pt,line width=0.305pt,miter limit=4.00] (60.5000,176.5000) -- (60.5000,152.5000);



    \end{scope}
    \begin{scope}[cm={{0.74279,0.0,0.0,1.28515,(-186.22138,-161.30028)}},draw=blue,line cap=round,line join=round,line width=0.480pt]
      \path[cm={{1.0,0.0,0.0,0.57583,(0.0,74.93902)}},draw] (60.5000,176.5000) -- (60.5000,171.5000);



      \path[cm={{1.0,0.0,0.0,0.70101,(0.0,45.50665)}},draw] (60.5000,152.5000) -- (60.5000,156.5000);



    \end{scope}
    \begin{scope}[cm={{0.74279,0.0,0.0,1.28515,(-186.22138,-161.30028)}},draw=ca0a0a4,dash pattern=on 1.22pt off 1.22pt,line cap=round,line join=round,line width=0.305pt,miter limit=4.00]
      \path[draw,dash pattern=on 1.22pt off 1.22pt,line width=0.305pt,miter limit=4.00] (95.5000,176.5000) -- (95.5000,152.5000);



    \end{scope}
    \begin{scope}[cm={{0.74279,0.0,0.0,1.28515,(-186.22138,-161.30028)}},draw=blue,line cap=round,line join=round,line width=0.480pt]
      \path[cm={{1.0,0.0,0.0,0.57583,(0.0,74.93902)}},draw] (95.5000,176.5000) -- (95.5000,171.5000);



      \path[cm={{1.0,0.0,0.0,0.70101,(0.0,45.50665)}},draw] (95.5000,152.5000) -- (95.5000,156.5000);



    \end{scope}
    \begin{scope}[cm={{0.74279,0.0,0.0,1.28515,(-186.22138,-161.30028)}},draw=ca0a0a4,dash pattern=on 1.22pt off 1.22pt,line cap=round,line join=round,line width=0.305pt,miter limit=4.00]
      \path[draw,dash pattern=on 1.22pt off 1.22pt,line width=0.305pt,miter limit=4.00] (130.5000,164.5000) -- (130.5000,152.5000);



    \end{scope}
    \begin{scope}[cm={{0.74279,0.0,0.0,1.28515,(-186.22138,-161.30028)}},draw=blue,line cap=round,line join=round,line width=0.480pt]
      \path[cm={{1.0,0.0,0.0,0.57583,(0.0,74.93902)}},draw] (130.5000,176.5000) -- (130.5000,171.5000);



      \path[cm={{1.0,0.0,0.0,0.70101,(0.0,45.50665)}},draw] (130.5000,152.5000) -- (130.5000,156.5000);



    \end{scope}
    \begin{scope}[cm={{0.74279,0.0,0.0,1.28515,(-186.22138,-161.30028)}},draw=blue,line cap=round,line join=round,line width=0.480pt]
      \path[draw] (25.5000,152.5000) -- (25.5000,176.5000) -- (142.5000,176.5000) -- (142.5000,152.5000) -- (25.5000,152.5000);



    \end{scope}
    \begin{scope}[cm={{0.95389,0.0,0.0,0.95389,(-104.73329,57.56154)}},draw=blue,line cap=rect,line join=bevel,line width=0.800pt]
      \path[fill=blue] (0.0000,0.0000) node[above right] (text1494) {\scriptsize $\beta_3$};



    \end{scope}
    \begin{scope}[cm={{0.74279,0.0,0.0,1.28515,(-184.07401,-166.61499)}},draw=blue,line cap=round,line join=round,line width=0.480pt]
      \path[draw,even odd rule] (123.5000,172.5000) -- (132.5000,172.5000);



    \end{scope}
    \begin{scope}[cm={{0.74279,0.0,0.0,1.28515,(-186.22138,-161.30028)}},draw=blue,line cap=round,line join=round,line width=0.480pt]
      \path[draw] (25.8000,162.1000) -- (25.8000,162.1000) -- (26.2000,161.2000) -- (26.7000,161.1000) -- (27.1000,162.2000) -- (27.5000,159.4000) -- (27.9000,164.7000) -- (28.3000,157.2000) -- (28.8000,163.3000) -- (29.2000,164.4000) -- (29.6000,156.6000) -- (30.0000,159.9000) -- (30.5000,165.6000) -- (30.9000,163.5000) -- (31.3000,158.2000) -- (31.7000,157.3000) -- (32.2000,161.0000) -- (32.6000,164.7000) -- (33.0000,165.0000) -- (33.4000,162.4000) -- (33.8000,159.2000) -- (34.3000,157.7000) -- (34.7000,158.4000) -- (35.1000,160.6000) -- (35.5000,162.9000) -- (36.0000,164.3000) -- (36.4000,164.2000) -- (36.8000,163.0000) -- (37.2000,161.2000) -- (37.6000,159.5000) -- (38.1000,158.5000) -- (38.5000,158.3000) -- (38.9000,159.0000) -- (39.3000,160.2000) -- (39.8000,161.6000) -- (40.2000,162.8000) -- (40.6000,163.6000) -- (41.0000,163.9000) -- (41.5000,163.6000) -- (41.9000,163.0000) -- (42.3000,162.1000) -- (42.7000,161.1000) -- (43.1000,160.2000) -- (43.6000,159.5000) -- (44.0000,159.1000) -- (44.4000,159.0000) -- (44.8000,159.2000) -- (45.3000,159.7000) -- (45.7000,160.4000) -- (46.1000,161.1000) -- (46.5000,161.8000) -- (47.0000,162.5000) -- (47.4000,163.0000) -- (47.8000,163.2000) -- (48.2000,163.3000) -- (48.6000,163.1000) -- (49.1000,162.8000) -- (49.5000,162.4000) -- (49.9000,161.8000) -- (50.3000,161.2000) -- (50.8000,160.7000) -- (51.2000,160.2000) -- (51.6000,159.9000) -- (52.0000,159.7000) -- (52.4000,159.7000) -- (52.9000,159.9000) -- (53.3000,160.3000) -- (53.7000,160.8000) -- (54.1000,161.3000) -- (54.6000,161.9000) -- (55.0000,162.4000) -- (55.4000,162.7000) -- (55.8000,163.0000) -- (56.3000,163.0000) -- (56.7000,162.8000) -- (57.1000,162.6000) -- (57.5000,162.2000) -- (57.9000,161.7000) -- (58.4000,161.2000) -- (58.8000,160.8000) -- (59.2000,160.4000) -- (59.6000,160.1000) -- (60.1000,160.0000) -- (60.5000,160.0000) -- (60.9000,160.1000) -- (61.3000,160.4000) -- (61.8000,160.7000) -- (62.2000,161.1000) -- (62.6000,161.5000) -- (63.0000,161.8000) -- (63.4000,162.1000) -- (63.9000,162.2000) -- (64.3000,162.2000) -- (64.7000,162.1000) -- (65.1000,161.9000) -- (65.6000,161.6000) -- (66.0000,161.3000) -- (66.4000,161.0000) -- (66.8000,160.7000) -- (67.3000,160.5000) -- (67.7000,160.4000) -- (68.1000,160.3000) -- (68.5000,160.4000) -- (68.9000,160.6000) -- (69.4000,160.8000) -- (69.8000,161.1000) -- (70.2000,161.4000) -- (70.6000,161.6000) -- (71.1000,161.8000) -- (71.5000,162.0000) -- (71.9000,162.0000) -- (72.3000,162.0000) -- (72.7000,162.0000) -- (73.2000,161.9000) -- (73.6000,161.8000) -- (74.0000,161.6000) -- (74.4000,161.5000) -- (74.9000,161.2000) -- (75.3000,161.0000) -- (75.7000,160.8000) -- (76.1000,160.7000) -- (76.6000,160.7000) -- (77.0000,160.7000) -- (77.4000,160.7000) -- (77.8000,160.8000) -- (78.2000,160.9000) -- (78.7000,161.0000) -- (79.1000,161.2000) -- (79.5000,161.3000) -- (79.9000,161.5000) -- (80.4000,161.6000) -- (80.8000,161.7000) -- (81.2000,161.8000) -- (81.6000,161.8000) -- (82.1000,161.8000) -- (82.5000,161.8000) -- (82.9000,161.7000) -- (83.3000,161.6000) -- (83.7000,161.5000) -- (84.2000,161.3000) -- (84.6000,161.2000) -- (85.0000,161.1000) -- (85.4000,161.0000) -- (85.9000,161.0000) -- (86.3000,161.0000) -- (86.7000,161.0000) -- (87.1000,161.0000) -- (87.6000,161.1000) -- (88.0000,161.1000) -- (88.4000,161.3000) -- (88.8000,161.4000) -- (89.2000,161.5000) -- (89.7000,161.6000) -- (90.1000,161.6000) -- (90.5000,161.6000) -- (90.9000,161.6000) -- (91.4000,161.6000) -- (91.8000,161.5000) -- (92.2000,161.4000) -- (92.6000,161.3000) -- (93.0000,161.2000) -- (93.5000,161.1000) -- (93.9000,161.1000) -- (94.3000,161.0000) -- (94.7000,161.0000) -- (95.2000,161.0000) -- (95.6000,161.0000) -- (96.0000,161.1000) -- (96.4000,161.2000) -- (96.9000,161.3000) -- (97.3000,161.4000) -- (97.7000,161.5000) -- (98.1000,161.5000) -- (98.5000,161.5000) -- (99.0000,161.5000) -- (99.4000,161.5000) -- (99.8000,161.4000) -- (100.2000,161.4000) -- (100.7000,161.3000) -- (101.1000,161.3000) -- (101.5000,161.2000) -- (101.9000,161.1000) -- (102.3000,161.1000) -- (102.8000,161.0000) -- (103.2000,161.1000) -- (103.6000,161.1000) -- (104.0000,161.2000) -- (104.5000,161.2000) -- (104.9000,161.3000) -- (105.3000,161.4000) -- (105.7000,161.4000) -- (106.2000,161.5000) -- (106.6000,161.5000) -- (107.0000,161.6000) -- (107.4000,161.6000) -- (107.8000,161.5000) -- (108.3000,161.5000) -- (108.7000,161.4000) -- (109.1000,161.4000) -- (109.5000,161.3000) -- (110.0000,161.2000) -- (110.4000,161.1000) -- (110.8000,161.0000) -- (111.2000,161.0000) -- (111.7000,160.9000) -- (112.1000,160.9000) -- (112.5000,161.0000) -- (112.9000,161.0000) -- (113.3000,161.1000) -- (113.8000,161.2000) -- (114.2000,161.3000) -- (114.6000,161.4000) -- (115.0000,161.5000) -- (115.5000,161.6000) -- (115.9000,161.7000) -- (116.3000,161.7000) -- (116.7000,161.7000) -- (117.2000,161.6000) -- (117.6000,161.6000) -- (118.0000,161.5000) -- (118.4000,161.3000) -- (118.8000,161.2000) -- (119.3000,161.1000) -- (119.7000,161.0000) -- (120.1000,160.9000) -- (120.5000,160.8000) -- (121.0000,160.8000) -- (121.4000,160.9000) -- (121.8000,160.9000) -- (122.2000,161.0000) -- (122.6000,161.2000) -- (123.1000,161.3000) -- (123.5000,161.5000) -- (123.9000,161.6000) -- (124.3000,161.7000) -- (124.8000,161.8000) -- (125.2000,161.8000) -- (125.6000,161.8000) -- (126.0000,161.7000) -- (126.5000,161.6000) -- (126.9000,161.5000) -- (127.3000,161.4000) -- (127.7000,161.3000) -- (128.1000,161.1000) -- (128.6000,160.9000) -- (129.0000,160.8000) -- (129.4000,160.8000) -- (129.8000,160.8000) -- (130.3000,160.8000) -- (130.7000,160.9000) -- (131.1000,161.0000) -- (131.5000,161.1000) -- (131.9000,161.3000) -- (132.4000,161.4000) -- (132.8000,161.5000) -- (133.2000,161.6000) -- (133.6000,161.7000) -- (134.1000,161.7000) -- (134.5000,161.8000) -- (134.9000,161.7000) -- (135.3000,161.7000) -- (135.8000,161.6000) -- (136.2000,161.5000) -- (136.6000,161.3000) -- (137.0000,161.2000) -- (137.4000,161.0000) -- (137.9000,160.9000) -- (138.3000,160.9000) -- (138.7000,160.8000) -- (139.1000,160.8000) -- (139.6000,160.9000) -- (140.0000,160.9000) -- (140.4000,161.0000) -- (140.8000,161.1000) -- (141.3000,161.3000) -- (141.7000,161.4000) -- (142.1000,161.6000) -- (142.3000,161.6000);



    \end{scope}
    \path[draw=blue,line cap=butt,line join=miter,line width=0.747pt] (-429.4395,-113.5288) -- cycle;



    \begin{scope}[cm={{0.99667,0.0,0.0,1.34693,(-320.19845,-156.47287)}},draw=ca0a0a4,dash pattern=on 1.54pt off 1.54pt,line cap=round,line join=round,line width=0.257pt,miter limit=4.00]
      \path[draw,dash pattern=on 1.54pt off 1.54pt,line width=0.257pt,miter limit=4.00] (41.5000,153.5000) -- (127.5000,153.5000);



    \end{scope}
    \begin{scope}[cm={{0.99667,0.0,0.0,1.34693,(-320.19845,-156.47287)}},draw=cd9d9d9,line cap=rect,line join=miter,line width=2.113pt,miter limit=10.00]
      \path[draw=cffffff,line cap=rect,line join=miter,line width=2.113pt,miter limit=10.00] (41.5000,88.5000) -- (41.5000,164.5000) -- (127.5000,164.5000) -- (127.5000,88.5000) -- (41.5000,88.5000);



    \end{scope}
    \begin{scope}[cm={{0.99667,0.0,0.0,1.34693,(-320.19845,-156.47287)}},draw=blue,line cap=round,line join=round,line width=0.480pt]
      \path[cm={{1.155,0.0,0.0,1.0,(-6.38582,0.0)}},draw] (41.5000,153.5000) -- (44.5000,153.5000);



      \path[cm={{1.155,0.0,0.0,1.0,(-20.06101,0.0)}},draw] (127.5000,153.5000) -- (124.5000,153.5000);



    \end{scope}
    \begin{scope}[cm={{1.0018,0.0,0.0,1.0018,(-293.12869,52.89918)}},draw=blue,line cap=rect,line join=bevel,line width=0.800pt]
      \path[fill=blue] (0.0000,0.0000) node[above right] (text366) {\scriptsize 30};



    \end{scope}
    \begin{scope}[cm={{0.99667,0.0,0.0,1.34693,(-320.19845,-156.47287)}},draw=ca0a0a4,dash pattern=on 1.54pt off 1.54pt,line cap=round,line join=round,line width=0.257pt,miter limit=4.00]
      \path[draw,dash pattern=on 1.54pt off 1.54pt,line width=0.257pt,miter limit=4.00] (41.5000,127.5000) -- (127.5000,127.5000);



    \end{scope}
    \begin{scope}[cm={{0.99667,0.0,0.0,1.34693,(-320.19845,-156.47287)}},draw=blue,line cap=round,line join=round,line width=0.480pt]
      \path[cm={{1.155,0.0,0.0,1.0,(-6.38582,0.0)}},draw] (41.5000,127.5000) -- (44.5000,127.5000);



      \path[cm={{1.155,0.0,0.0,1.0,(-20.06101,0.0)}},draw] (127.5000,127.5000) -- (124.5000,127.5000);



    \end{scope}
    \begin{scope}[cm={{1.0018,0.0,0.0,1.0018,(-292.76805,19.17029)}},draw=blue,line cap=rect,line join=bevel,line width=0.800pt]
      \path[fill=blue] (0.0000,0.0000) node[above right] (text396) {\scriptsize 33};



    \end{scope}
    \begin{scope}[cm={{0.99667,0.0,0.0,1.34693,(-320.19845,-156.47287)}},draw=blue,line cap=round,line join=round,line width=0.480pt]
      \path[cm={{1.155,0.0,0.0,1.0,(-6.38582,0.0)}},draw] (41.5000,101.5000) -- (44.5000,101.5000);



      \path[cm={{1.155,0.0,0.0,1.0,(-20.06101,0.0)}},draw] (127.5000,101.5000) -- (124.5000,101.5000);



    \end{scope}
    \begin{scope}[cm={{1.0018,0.0,0.0,1.0018,(-293.06458,-16.50733)}},draw=blue,line cap=rect,line join=bevel,line width=0.800pt]
      \path[fill=blue] (0.0000,0.0000) node[above right] (text426) {\scriptsize 36};



    \end{scope}
    \begin{scope}[cm={{0.99667,0.0,0.0,1.34693,(-320.19845,-156.47287)}},draw=ca0a0a4,dash pattern=on 0.40pt off 0.80pt,line cap=round,line join=round,line width=0.400pt]
      \path[draw] (41.5000,164.5000) -- (41.5000,88.5000);



    \end{scope}
    \begin{scope}[cm={{0.99667,0.0,0.0,1.34693,(-320.19845,-156.47287)}},draw=blue,line cap=round,line join=round,line width=0.480pt]
      \path[draw] (41.5000,164.5000) -- (41.5000,159.5000);



      \path[draw] (41.5000,88.5000) -- (41.5000,92.5000);



    \end{scope}
    \begin{scope}[cm={{1.00174,0.0,0.0,1.01515,(-281.55885,76.66745)}},draw=blue,line cap=rect,line join=bevel,line width=0.800pt]
      \path[fill=blue] (0.0000,0.0000) node[above right] (text456) {\scriptsize 0};



    \end{scope}
    \begin{scope}[cm={{0.99667,0.0,0.0,1.34693,(-320.19845,-156.47287)}},draw=blue,line cap=round,line join=round,line width=0.480pt]
      \path[cm={{1.0,0.0,0.0,0.54942,(0.0,73.94736)}},draw] (70.5000,164.5000) -- (70.5000,159.5000);



      \path[cm={{1.0,0.0,0.0,0.66885,(0.0,29.20714)}},draw] (70.5000,88.5000) -- (70.5000,92.5000);



    \end{scope}
    \begin{scope}[cm={{1.00174,0.0,0.0,1.01515,(-253.51015,76.66745)}},draw=blue,line cap=rect,line join=bevel,line width=0.800pt]
      \path[fill=blue] (0.0000,0.0000) node[above right] (text486) {\scriptsize 2};



    \end{scope}
    \begin{scope}[cm={{0.99667,0.0,0.0,1.34693,(-320.19845,-156.47287)}},draw=blue,line cap=round,line join=round,line width=0.480pt]
      \path[cm={{1.0,0.0,0.0,0.54942,(0.0,73.94736)}},draw] (98.5000,164.5000) -- (98.5000,159.5000);



      \path[cm={{1.0,0.0,0.0,0.66885,(0.0,29.20714)}},draw] (98.5000,88.5000) -- (98.5000,92.5000);



    \end{scope}
    \begin{scope}[cm={{1.00174,0.0,0.0,1.01515,(-223.95891,76.66745)}},draw=blue,line cap=rect,line join=bevel,line width=0.800pt]
      \path[fill=blue] (0.0000,0.0000) node[above right] (text516) {\scriptsize 4};



    \end{scope}
    \begin{scope}[cm={{0.99667,0.0,0.0,1.34693,(-320.19845,-156.47287)}},draw=ca0a0a4,dash pattern=on 0.40pt off 0.80pt,line cap=round,line join=round,line width=0.400pt]
      \path[draw] (127.5000,164.5000) -- (127.5000,88.5000);



    \end{scope}
    \begin{scope}[cm={{0.99667,0.0,0.0,1.34693,(-320.19845,-156.47287)}},draw=blue,line cap=round,line join=round,line width=0.480pt]
      \path[draw] (127.5000,164.5000) -- (127.5000,159.5000);



      \path[draw] (127.5000,88.5000) -- (127.5000,92.5000);



    \end{scope}
    \begin{scope}[cm={{1.00174,0.0,0.0,1.01515,(-196.411,76.73242)}},draw=blue,line cap=rect,line join=bevel,line width=0.800pt]
      \path[fill=blue] (0.0000,0.0000) node[above right] (text546) {\scriptsize 6};



    \end{scope}
    \begin{scope}[cm={{0.99667,0.0,0.0,1.34693,(-320.19845,-156.47287)}},draw=blue,line cap=round,line join=round,line width=0.480pt]
      \path[draw] (41.6000,103.3000) -- (41.6000,103.3000) -- (41.7000,103.6000) -- (41.9000,104.0000) -- (42.0000,104.3000) -- (42.2000,104.6000) -- (42.3000,104.9000) -- (42.5000,105.1000) -- (42.6000,105.4000) -- (42.7000,105.7000) -- (42.9000,106.0000) -- (43.0000,106.3000) -- (43.2000,106.5000) -- (43.3000,106.8000) -- (43.5000,107.1000) -- (43.6000,107.3000) -- (43.7000,107.6000) -- (43.9000,107.9000) -- (44.0000,108.1000) -- (44.2000,108.4000) -- (44.3000,108.6000) -- (44.5000,108.8000) -- (44.6000,109.1000) -- (44.7000,109.3000) -- (44.9000,109.5000) -- (45.0000,109.8000) -- (45.2000,110.0000) -- (45.3000,110.2000) -- (45.5000,110.4000) -- (45.6000,110.7000) -- (45.7000,110.9000) -- (45.9000,111.1000) -- (46.0000,111.3000) -- (46.2000,111.5000) -- (46.3000,111.7000) -- (46.5000,111.9000) -- (46.6000,112.1000) -- (46.7000,112.3000) -- (46.9000,112.4000) -- (47.0000,112.6000) -- (47.2000,112.8000) -- (47.3000,113.0000) -- (47.5000,113.2000) -- (47.6000,113.3000) -- (47.7000,113.5000) -- (47.9000,113.7000) -- (48.0000,113.9000) -- (48.2000,114.0000) -- (48.3000,114.2000) -- (48.5000,114.3000) -- (48.6000,114.5000) -- (48.7000,114.7000) -- (48.9000,114.8000) -- (49.0000,115.0000) -- (49.2000,115.1000) -- (49.3000,115.2000) -- (49.5000,115.4000) -- (49.6000,115.5000) -- (49.7000,115.7000) -- (49.9000,115.8000) -- (50.0000,115.9000) -- (50.2000,116.1000) -- (50.3000,116.2000) -- (50.5000,116.3000) -- (50.6000,116.4000) -- (50.7000,116.6000) -- (50.9000,116.7000) -- (51.0000,116.8000) -- (51.2000,116.9000) -- (51.3000,117.0000) -- (51.5000,117.1000) -- (51.6000,117.2000) -- (51.7000,117.4000) -- (51.9000,117.5000) -- (52.0000,117.6000) -- (52.2000,117.7000) -- (52.3000,117.8000) -- (52.5000,117.9000) -- (52.6000,118.0000) -- (52.7000,118.0000) -- (52.9000,118.1000) -- (53.0000,118.2000) -- (53.2000,118.3000) -- (53.3000,118.4000) -- (53.5000,118.5000) -- (53.6000,118.6000) -- (53.7000,118.7000) -- (53.9000,118.7000) -- (54.0000,118.8000) -- (54.2000,118.9000) -- (54.3000,119.0000) -- (54.5000,119.0000) -- (54.6000,119.1000) -- (54.7000,119.2000) -- (54.9000,119.2000) -- (55.0000,119.3000) -- (55.2000,119.4000) -- (55.3000,119.4000) -- (55.5000,119.5000) -- (55.6000,119.6000) -- (55.7000,119.6000) -- (55.9000,119.7000) -- (56.0000,119.7000) -- (56.2000,119.8000) -- (56.3000,119.8000) -- (56.5000,119.9000) -- (56.6000,119.9000) -- (56.7000,120.0000) -- (56.9000,120.0000) -- (57.0000,120.1000) -- (57.2000,120.1000) -- (57.3000,120.2000) -- (57.5000,120.2000) -- (57.6000,120.2000) -- (57.7000,120.3000) -- (57.9000,120.3000) -- (58.0000,120.4000) -- (58.2000,120.4000) -- (58.3000,120.4000) -- (58.5000,120.5000) -- (58.6000,120.5000) -- (58.7000,120.5000) -- (58.9000,120.6000) -- (59.0000,120.6000) -- (59.2000,120.6000) -- (59.3000,120.6000) -- (59.5000,120.7000) -- (59.6000,120.7000) -- (59.7000,120.7000) -- (59.9000,120.7000) -- (60.0000,120.8000) -- (60.2000,120.8000) -- (60.3000,120.8000) -- (60.5000,120.8000) -- (60.6000,120.8000) -- (60.7000,120.9000) -- (60.9000,120.9000) -- (61.0000,120.9000) -- (61.2000,120.9000) -- (61.3000,120.9000) -- (61.5000,120.9000) -- (61.6000,120.9000) -- (61.7000,121.0000) -- (61.9000,121.0000) -- (62.0000,121.0000) -- (62.2000,121.0000) -- (62.3000,121.0000) -- (62.5000,121.0000) -- (62.6000,121.0000) -- (62.7000,121.0000) -- (62.9000,121.0000) -- (63.0000,121.0000) -- (63.2000,121.0000) -- (63.3000,121.0000) -- (63.5000,121.0000) -- (63.6000,121.0000) -- (63.7000,121.0000) -- (63.9000,121.0000) -- (64.0000,121.0000) -- (64.2000,121.0000) -- (64.3000,121.0000) -- (64.5000,121.0000) -- (64.6000,121.0000) -- (64.7000,121.0000) -- (64.9000,121.0000) -- (65.0000,121.0000) -- (65.2000,121.0000) -- (65.3000,121.0000) -- (65.5000,121.0000) -- (65.6000,121.0000) -- (65.7000,121.0000) -- (65.9000,121.0000) -- (66.0000,121.0000) -- (66.2000,121.0000) -- (66.3000,121.0000) -- (66.5000,120.9000) -- (66.6000,120.9000) -- (66.7000,120.9000) -- (66.9000,120.9000) -- (67.0000,120.9000) -- (67.2000,120.9000) -- (67.3000,120.9000) -- (67.5000,120.9000) -- (67.6000,120.9000) -- (67.7000,120.9000) -- (67.9000,120.9000) -- (68.0000,120.9000) -- (68.2000,120.8000) -- (68.3000,120.8000) -- (68.5000,120.8000) -- (68.6000,120.8000) -- (68.7000,120.8000) -- (68.9000,120.8000) -- (69.0000,120.8000) -- (69.2000,120.8000) -- (69.3000,120.8000) -- (69.5000,120.8000) -- (69.6000,120.8000) -- (69.7000,120.8000) -- (69.9000,120.8000) -- (70.0000,120.8000) -- (70.2000,120.8000) -- (70.3000,120.8000) -- (70.5000,120.8000) -- (70.6000,120.8000) -- (70.7000,120.8000) -- (70.9000,120.8000) -- (71.0000,120.8000) -- (71.2000,120.8000) -- (71.3000,120.8000) -- (71.5000,120.9000) -- (71.6000,120.9000) -- (71.7000,120.9000) -- (71.9000,120.9000) -- (72.0000,120.9000) -- (72.2000,120.9000) -- (72.3000,121.0000) -- (72.5000,121.0000) -- (72.6000,121.0000) -- (72.7000,121.0000) -- (72.9000,121.1000) -- (73.0000,121.1000) -- (73.2000,121.1000) -- (73.3000,121.2000) -- (73.5000,121.2000) -- (73.6000,121.2000) -- (73.7000,121.2000) -- (73.9000,121.3000) -- (74.0000,121.3000) -- (74.2000,121.3000) -- (74.3000,121.3000) -- (74.5000,121.4000) -- (74.6000,121.4000) -- (74.7000,121.4000) -- (74.9000,121.5000) -- (75.0000,121.5000) -- (75.2000,121.5000) -- (75.3000,121.5000) -- (75.5000,121.6000) -- (75.6000,121.6000) -- (75.7000,121.6000) -- (75.9000,121.6000) -- (76.0000,121.7000) -- (76.2000,121.7000) -- (76.3000,121.7000) -- (76.5000,121.7000) -- (76.6000,121.7000) -- (76.7000,121.7000) -- (76.9000,121.8000) -- (77.0000,121.8000) -- (77.2000,121.8000) -- (77.3000,121.8000) -- (77.5000,121.8000) -- (77.6000,121.8000) -- (77.7000,121.9000) -- (77.9000,121.9000) -- (78.0000,121.9000) -- (78.2000,121.9000) -- (78.3000,121.9000) -- (78.5000,121.9000) -- (78.6000,121.9000) -- (78.7000,121.9000) -- (78.9000,121.9000) -- (79.0000,121.9000) -- (79.2000,121.9000) -- (79.3000,121.9000) -- (79.5000,121.9000) -- (79.6000,121.9000) -- (79.7000,122.0000) -- (79.9000,122.0000) -- (80.0000,122.0000) -- (80.2000,122.0000) -- (80.3000,122.0000) -- (80.5000,121.9000) -- (80.6000,121.9000) -- (80.7000,121.9000) -- (80.9000,121.9000) -- (81.0000,121.9000) -- (81.2000,121.9000) -- (81.3000,121.9000) -- (81.5000,121.9000) -- (81.6000,121.9000) -- (81.7000,121.9000) -- (81.9000,121.9000) -- (82.0000,121.9000) -- (82.2000,121.9000) -- (82.3000,121.9000) -- (82.5000,121.9000) -- (82.6000,121.9000) -- (82.7000,121.8000) -- (82.9000,121.8000) -- (83.0000,121.8000) -- (83.2000,121.8000) -- (83.3000,121.8000) -- (83.5000,121.8000) -- (83.6000,121.8000) -- (83.7000,121.8000) -- (83.9000,121.7000) -- (84.0000,121.7000) -- (84.2000,121.7000) -- (84.3000,121.7000) -- (84.5000,121.7000) -- (84.6000,121.7000) -- (84.7000,121.6000) -- (84.9000,121.6000) -- (85.0000,121.6000) -- (85.2000,121.6000) -- (85.3000,121.6000) -- (85.5000,121.5000) -- (85.6000,121.5000) -- (85.7000,121.5000) -- (85.9000,121.5000) -- (86.0000,121.5000) -- (86.2000,121.5000) -- (86.3000,121.4000) -- (86.5000,121.4000) -- (86.6000,121.4000) -- (86.7000,121.4000) -- (86.9000,121.3000) -- (87.0000,121.3000) -- (87.2000,121.3000) -- (87.3000,121.3000) -- (87.5000,121.3000) -- (87.6000,121.2000) -- (87.7000,121.2000) -- (87.9000,121.2000) -- (88.0000,121.2000) -- (88.2000,121.1000) -- (88.3000,121.1000) -- (88.5000,121.1000) -- (88.6000,121.1000) -- (88.7000,121.0000) -- (88.9000,121.0000) -- (89.0000,121.0000) -- (89.2000,121.0000) -- (89.3000,120.9000) -- (89.5000,120.9000) -- (89.6000,120.9000) -- (89.7000,120.9000) -- (89.9000,120.8000) -- (90.0000,120.8000) -- (90.2000,120.8000) -- (90.3000,120.8000) -- (90.5000,120.7000) -- (90.6000,120.7000) -- (90.7000,120.7000) -- (90.9000,120.6000) -- (91.0000,120.6000) -- (91.2000,120.6000) -- (91.3000,120.6000) -- (91.5000,120.5000) -- (91.6000,120.5000) -- (91.7000,120.5000) -- (91.9000,120.5000) -- (92.0000,120.4000) -- (92.2000,120.4000) -- (92.3000,120.4000) -- (92.5000,120.4000) -- (92.6000,120.3000) -- (92.7000,120.3000) -- (92.9000,120.3000) -- (93.0000,120.2000) -- (93.2000,120.2000) -- (93.3000,120.2000) -- (93.4000,120.2000) -- (93.6000,120.1000) -- (93.7000,120.1000) -- (93.9000,120.1000) -- (94.0000,120.1000) -- (94.2000,120.0000) -- (94.3000,120.0000) -- (94.4000,120.0000) -- (94.6000,120.0000) -- (94.7000,119.9000) -- (94.9000,119.9000) -- (95.0000,119.9000) -- (95.2000,119.8000) -- (95.3000,119.8000) -- (95.4000,119.8000) -- (95.6000,119.8000) -- (95.7000,119.7000) -- (95.9000,119.7000) -- (96.0000,119.7000) -- (96.2000,119.7000) -- (96.3000,119.6000) -- (96.4000,119.6000) -- (96.6000,119.6000) -- (96.7000,119.6000) -- (96.9000,119.5000) -- (97.0000,119.5000) -- (97.2000,119.5000) -- (97.3000,119.5000) -- (97.4000,119.4000) -- (97.6000,119.4000) -- (97.7000,119.4000) -- (97.9000,119.3000) -- (98.0000,119.3000) -- (98.2000,119.3000) -- (98.3000,119.3000) -- (98.4000,119.2000) -- (98.6000,119.2000) -- (98.7000,119.2000) -- (98.9000,119.1000) -- (99.0000,119.1000) -- (99.2000,119.1000) -- (99.3000,119.1000) -- (99.4000,119.0000) -- (99.6000,119.0000) -- (99.7000,119.0000) -- (99.9000,119.0000) -- (100.0000,118.9000) -- (100.2000,118.9000) -- (100.3000,118.9000) -- (100.4000,118.9000) -- (100.6000,118.8000) -- (100.7000,118.8000) -- (100.9000,118.8000) -- (101.0000,118.8000) -- (101.2000,118.7000) -- (101.3000,118.7000) -- (101.4000,118.7000) -- (101.6000,118.7000) -- (101.7000,118.6000) -- (101.9000,118.6000) -- (102.0000,118.6000) -- (102.2000,118.6000) -- (102.3000,118.5000) -- (102.4000,118.5000) -- (102.6000,118.5000) -- (102.7000,118.5000) -- (102.9000,118.4000) -- (103.0000,118.4000) -- (103.2000,118.4000) -- (103.3000,118.4000) -- (103.4000,118.4000) -- (103.6000,118.3000) -- (103.7000,118.3000) -- (103.9000,118.3000) -- (104.0000,118.3000) -- (104.2000,118.3000) -- (104.3000,118.2000) -- (104.4000,118.2000) -- (104.6000,118.2000) -- (104.7000,118.2000) -- (104.9000,118.2000) -- (105.0000,118.1000) -- (105.2000,118.1000) -- (105.3000,118.1000) -- (105.4000,118.1000) -- (105.6000,118.1000) -- (105.7000,118.0000) -- (105.9000,118.0000) -- (106.0000,118.0000) -- (106.2000,118.0000) -- (106.3000,118.0000) -- (106.4000,118.0000) -- (106.6000,117.9000) -- (106.7000,117.9000) -- (106.9000,117.9000) -- (107.0000,117.9000) -- (107.2000,117.9000) -- (107.3000,117.9000) -- (107.4000,117.9000) -- (107.6000,117.8000) -- (107.7000,117.8000) -- (107.9000,117.8000) -- (108.0000,117.8000) -- (108.2000,117.8000) -- (108.3000,117.8000) -- (108.4000,117.8000) -- (108.6000,117.8000) -- (108.7000,117.7000) -- (108.9000,117.7000) -- (109.0000,117.7000) -- (109.2000,117.7000) -- (109.3000,117.7000) -- (109.4000,117.7000) -- (109.6000,117.7000) -- (109.7000,117.7000) -- (109.9000,117.7000) -- (110.0000,117.6000) -- (110.2000,117.6000) -- (110.3000,117.6000) -- (110.4000,117.6000) -- (110.6000,117.6000) -- (110.7000,117.6000) -- (110.9000,117.6000) -- (111.0000,117.6000) -- (111.2000,117.6000) -- (111.3000,117.6000) -- (111.4000,117.6000) -- (111.6000,117.6000) -- (111.7000,117.5000) -- (111.9000,117.5000) -- (112.0000,117.5000) -- (112.2000,117.5000) -- (112.3000,117.5000) -- (112.4000,117.5000) -- (112.6000,117.5000) -- (112.7000,117.5000) -- (112.9000,117.5000) -- (113.0000,117.5000) -- (113.2000,117.5000) -- (113.3000,117.5000) -- (113.4000,117.5000) -- (113.6000,117.5000) -- (113.7000,117.5000) -- (113.9000,117.5000) -- (114.0000,117.5000) -- (114.2000,117.5000) -- (114.3000,117.4000) -- (114.4000,117.4000) -- (114.6000,117.4000) -- (114.7000,117.4000) -- (114.9000,117.4000) -- (115.0000,117.4000) -- (115.2000,117.4000) -- (115.3000,117.4000) -- (115.4000,117.4000) -- (115.6000,117.4000) -- (115.7000,117.4000) -- (115.9000,117.4000) -- (116.0000,117.4000) -- (116.2000,117.4000) -- (116.3000,117.4000) -- (116.4000,117.4000) -- (116.6000,117.4000) -- (116.7000,117.4000) -- (116.9000,117.4000) -- (117.0000,117.4000) -- (117.2000,117.4000) -- (117.3000,117.4000) -- (117.4000,117.4000) -- (117.6000,117.4000) -- (117.7000,117.4000) -- (117.9000,117.4000) -- (118.0000,117.4000) -- (118.2000,117.4000) -- (118.3000,117.4000) -- (118.4000,117.4000) -- (118.6000,117.4000) -- (118.7000,117.4000) -- (118.9000,117.4000) -- (119.0000,117.4000) -- (119.2000,117.4000) -- (119.3000,117.4000) -- (119.4000,117.4000) -- (119.6000,117.4000) -- (119.7000,117.4000) -- (119.9000,117.4000) -- (120.0000,117.4000) -- (120.2000,117.4000) -- (120.3000,117.4000) -- (120.4000,117.4000) -- (120.6000,117.4000) -- (120.7000,117.4000) -- (120.9000,117.4000) -- (121.0000,117.4000) -- (121.2000,117.4000) -- (121.3000,117.4000) -- (121.4000,117.4000) -- (121.6000,117.4000) -- (121.7000,117.4000) -- (121.9000,117.4000) -- (122.0000,117.4000) -- (122.2000,117.4000) -- (122.3000,117.4000) -- (122.4000,117.4000) -- (122.6000,117.4000) -- (122.7000,117.4000) -- (122.9000,117.4000) -- (123.0000,117.4000) -- (123.2000,117.4000) -- (123.3000,117.4000) -- (123.4000,117.4000) -- (123.6000,117.4000) -- (123.7000,117.4000) -- (123.9000,117.4000) -- (124.0000,117.4000) -- (124.2000,117.4000) -- (124.3000,117.4000) -- (124.4000,117.5000) -- (124.6000,117.5000) -- (124.7000,117.5000) -- (124.9000,117.5000) -- (125.0000,117.5000) -- (125.2000,117.5000) -- (125.3000,117.5000) -- (125.4000,117.5000) -- (125.6000,117.5000) -- (125.7000,117.5000) -- (125.9000,117.5000) -- (126.0000,117.5000) -- (126.2000,117.5000) -- (126.3000,117.5000) -- (126.4000,117.5000) -- (126.6000,117.5000) -- (126.7000,117.5000) -- (126.9000,117.5000) -- (127.0000,117.5000) -- (127.2000,117.5000) -- (127.3000,117.5000);



    \end{scope}
    \begin{scope}[cm={{0.99667,0.0,0.0,1.34693,(-320.19845,-156.47287)}},draw=cff0000,line cap=round,line join=bevel,line width=0.480pt,miter limit=4.00]
      \path[draw,line cap=round,line join=round,line width=0.480pt,miter limit=4.00] (41.6000,109.8000) -- (41.6000,109.8000) -- (41.7000,103.2000) -- (41.9000,103.5000) -- (42.0000,103.9000) -- (42.2000,104.4000) -- (42.3000,104.8000) -- (42.5000,105.2000) -- (42.6000,105.6000) -- (42.7000,106.0000) -- (42.9000,106.4000) -- (43.0000,106.7000) -- (43.2000,107.1000) -- (43.3000,107.4000) -- (43.5000,107.7000) -- (43.6000,107.9000) -- (43.7000,108.2000) -- (43.9000,108.5000) -- (44.0000,108.7000) -- (44.2000,108.9000) -- (44.3000,109.1000) -- (44.5000,109.4000) -- (44.6000,109.6000) -- (44.7000,109.7000) -- (44.9000,109.9000) -- (45.0000,110.1000) -- (45.2000,110.3000) -- (45.3000,110.5000) -- (45.5000,110.7000) -- (45.6000,110.9000) -- (45.7000,111.1000) -- (45.9000,111.3000) -- (46.0000,111.5000) -- (46.2000,111.7000) -- (46.3000,111.9000) -- (46.5000,112.1000) -- (46.6000,112.3000) -- (46.7000,112.5000) -- (46.9000,112.7000) -- (47.0000,112.9000) -- (47.2000,113.1000) -- (47.3000,113.3000) -- (47.5000,113.5000) -- (47.6000,113.7000) -- (47.7000,113.9000) -- (47.9000,114.1000) -- (48.0000,114.3000) -- (48.2000,114.5000) -- (48.3000,114.6000) -- (48.5000,114.8000) -- (48.6000,114.9000) -- (48.7000,115.1000) -- (48.9000,115.2000) -- (49.0000,115.3000) -- (49.2000,115.4000) -- (49.3000,115.6000) -- (49.5000,115.7000) -- (49.6000,115.8000) -- (49.7000,115.9000) -- (49.9000,116.0000) -- (50.0000,116.1000) -- (50.2000,116.2000) -- (50.3000,116.3000) -- (50.5000,116.4000) -- (50.6000,116.5000) -- (50.7000,116.6000) -- (50.9000,116.8000) -- (51.0000,116.9000) -- (51.2000,117.0000) -- (51.3000,117.2000) -- (51.5000,117.3000) -- (51.6000,117.5000) -- (51.7000,117.7000) -- (51.9000,117.8000) -- (52.0000,118.0000) -- (52.2000,118.2000) -- (52.3000,118.3000) -- (52.5000,118.5000) -- (52.6000,118.6000) -- (52.7000,118.8000) -- (52.9000,118.9000) -- (53.0000,119.0000) -- (53.2000,119.1000) -- (53.3000,119.2000) -- (53.5000,119.3000) -- (53.6000,119.4000) -- (53.7000,119.4000) -- (53.9000,119.4000) -- (54.0000,119.5000) -- (54.2000,119.5000) -- (54.3000,119.5000) -- (54.5000,119.4000) -- (54.6000,119.4000) -- (54.7000,119.4000) -- (54.9000,119.3000) -- (55.0000,119.3000) -- (55.2000,119.2000) -- (55.3000,119.2000) -- (55.5000,119.1000) -- (55.6000,119.1000) -- (55.7000,119.1000) -- (55.9000,119.0000) -- (56.0000,119.0000) -- (56.2000,119.0000) -- (56.3000,119.1000) -- (56.5000,119.1000) -- (56.6000,119.2000) -- (56.7000,119.2000) -- (56.9000,119.3000) -- (57.0000,119.4000) -- (57.2000,119.5000) -- (57.3000,119.6000) -- (57.5000,119.7000) -- (57.6000,119.8000) -- (57.7000,119.9000) -- (57.9000,120.1000) -- (58.0000,120.2000) -- (58.2000,120.3000) -- (58.3000,120.4000) -- (58.5000,120.5000) -- (58.6000,120.6000) -- (58.7000,120.7000) -- (58.9000,120.8000) -- (59.0000,120.8000) -- (59.2000,120.9000) -- (59.3000,121.0000) -- (59.5000,121.0000) -- (59.6000,121.0000) -- (59.7000,121.1000) -- (59.9000,121.1000) -- (60.0000,121.1000) -- (60.2000,121.1000) -- (60.3000,121.1000) -- (60.5000,121.1000) -- (60.6000,121.1000) -- (60.7000,121.0000) -- (60.9000,121.0000) -- (61.0000,121.0000) -- (61.2000,121.0000) -- (61.3000,120.9000) -- (61.5000,120.9000) -- (61.6000,120.9000) -- (61.7000,120.8000) -- (61.9000,120.8000) -- (62.0000,120.8000) -- (62.2000,120.8000) -- (62.3000,120.7000) -- (62.5000,120.7000) -- (62.6000,120.7000) -- (62.7000,120.7000) -- (62.9000,120.7000) -- (63.0000,120.7000) -- (63.2000,120.7000) -- (63.3000,120.7000) -- (63.5000,120.7000) -- (63.6000,120.8000) -- (63.7000,120.8000) -- (63.9000,120.8000) -- (64.0000,120.8000) -- (64.2000,120.8000) -- (64.3000,120.9000) -- (64.5000,120.9000) -- (64.6000,120.9000) -- (64.7000,121.0000) -- (64.9000,121.0000) -- (65.0000,121.0000) -- (65.2000,121.0000) -- (65.3000,121.1000) -- (65.5000,121.1000) -- (65.6000,121.1000) -- (65.7000,121.1000) -- (65.9000,121.2000) -- (66.0000,121.2000) -- (66.2000,121.2000) -- (66.3000,121.2000) -- (66.5000,121.2000) -- (66.6000,121.2000) -- (66.7000,121.2000) -- (66.9000,121.2000) -- (67.0000,121.2000) -- (67.2000,121.2000) -- (67.3000,121.2000) -- (67.5000,121.2000) -- (67.6000,121.2000) -- (67.7000,121.2000) -- (67.9000,121.2000) -- (68.0000,121.2000) -- (68.2000,121.1000) -- (68.3000,121.1000) -- (68.5000,121.1000) -- (68.6000,121.1000) -- (68.7000,121.0000) -- (68.9000,121.0000) -- (69.0000,121.0000) -- (69.2000,120.9000) -- (69.3000,120.9000) -- (69.5000,120.9000) -- (69.6000,120.8000) -- (69.7000,120.8000) -- (69.9000,120.8000) -- (70.0000,120.7000) -- (70.2000,120.7000) -- (70.3000,120.7000) -- (70.5000,120.7000) -- (70.6000,120.6000) -- (70.7000,120.6000) -- (70.9000,120.6000) -- (71.0000,120.6000) -- (71.2000,120.6000) -- (71.3000,120.6000) -- (71.5000,120.6000) -- (71.6000,120.6000) -- (71.7000,120.6000) -- (71.9000,120.6000) -- (72.0000,120.6000) -- (72.2000,120.6000) -- (72.3000,120.6000) -- (72.5000,120.6000) -- (72.6000,120.6000) -- (72.7000,120.7000) -- (72.9000,120.7000) -- (73.0000,120.7000) -- (73.2000,120.8000) -- (73.3000,120.8000) -- (73.5000,120.8000) -- (73.6000,120.9000) -- (73.7000,120.9000) -- (73.9000,120.9000) -- (74.0000,121.0000) -- (74.2000,121.0000) -- (74.3000,121.1000) -- (74.5000,121.1000) -- (74.6000,121.2000) -- (74.7000,121.2000) -- (74.9000,121.3000) -- (75.0000,121.3000) -- (75.2000,121.3000) -- (75.3000,121.4000) -- (75.5000,121.4000) -- (75.6000,121.5000) -- (75.7000,121.5000) -- (75.9000,121.6000) -- (76.0000,121.6000) -- (76.2000,121.7000) -- (76.3000,121.7000) -- (76.5000,121.7000) -- (76.6000,121.8000) -- (76.7000,121.8000) -- (76.9000,121.9000) -- (77.0000,121.9000) -- (77.2000,121.9000) -- (77.3000,122.0000) -- (77.5000,122.0000) -- (77.6000,122.0000) -- (77.7000,122.1000) -- (77.9000,122.1000) -- (78.0000,122.1000) -- (78.2000,122.1000) -- (78.3000,122.1000) -- (78.5000,122.2000) -- (78.6000,122.2000) -- (78.7000,122.2000) -- (78.9000,122.2000) -- (79.0000,122.2000) -- (79.2000,122.2000) -- (79.3000,122.2000) -- (79.5000,122.2000) -- (79.6000,122.2000) -- (79.7000,122.2000) -- (79.9000,122.2000) -- (80.0000,122.2000) -- (80.2000,122.2000) -- (80.3000,122.2000) -- (80.5000,122.2000) -- (80.6000,122.2000) -- (80.7000,122.2000) -- (80.9000,122.2000) -- (81.0000,122.2000) -- (81.2000,122.2000) -- (81.3000,122.2000) -- (81.5000,122.1000) -- (81.6000,122.1000) -- (81.7000,122.1000) -- (81.9000,122.1000) -- (82.0000,122.1000) -- (82.2000,122.0000) -- (82.3000,122.0000) -- (82.5000,122.0000) -- (82.6000,122.0000) -- (82.7000,121.9000) -- (82.9000,121.9000) -- (83.0000,121.9000) -- (83.2000,121.9000) -- (83.3000,121.8000) -- (83.5000,121.8000) -- (83.6000,121.8000) -- (83.7000,121.8000) -- (83.9000,121.7000) -- (84.0000,121.7000) -- (84.2000,121.7000) -- (84.3000,121.7000) -- (84.5000,121.6000) -- (84.6000,121.6000) -- (84.7000,121.6000) -- (84.9000,121.5000) -- (85.0000,121.5000) -- (85.2000,121.5000) -- (85.3000,121.5000) -- (85.5000,121.4000) -- (85.6000,121.4000) -- (85.7000,121.4000) -- (85.9000,121.3000) -- (86.0000,121.3000) -- (86.2000,121.3000) -- (86.3000,121.3000) -- (86.5000,121.2000) -- (86.6000,121.2000) -- (86.7000,121.2000) -- (86.9000,121.2000) -- (87.0000,121.1000) -- (87.2000,121.1000) -- (87.3000,121.1000) -- (87.5000,121.1000) -- (87.6000,121.1000) -- (87.7000,121.0000) -- (87.9000,121.0000) -- (88.0000,121.0000) -- (88.2000,121.0000) -- (88.3000,120.9000) -- (88.5000,120.9000) -- (88.6000,120.9000) -- (88.7000,120.9000) -- (88.9000,120.9000) -- (89.0000,120.8000) -- (89.2000,120.8000) -- (89.3000,120.8000) -- (89.5000,120.8000) -- (89.6000,120.8000) -- (89.7000,120.7000) -- (89.9000,120.7000) -- (90.0000,120.7000) -- (90.2000,120.7000) -- (90.3000,120.7000) -- (90.5000,120.6000) -- (90.6000,120.6000) -- (90.7000,120.6000) -- (90.9000,120.6000) -- (91.0000,120.6000) -- (91.2000,120.5000) -- (91.3000,120.5000) -- (91.5000,120.5000) -- (91.6000,120.5000) -- (91.7000,120.5000) -- (91.9000,120.4000) -- (92.0000,120.4000) -- (92.2000,120.4000) -- (92.3000,120.4000) -- (92.5000,120.4000) -- (92.6000,120.4000) -- (92.7000,120.3000) -- (92.9000,120.3000) -- (93.0000,120.3000) -- (93.2000,120.3000) -- (93.3000,120.3000) -- (93.4000,120.2000) -- (93.6000,120.2000) -- (93.7000,120.2000) -- (93.9000,120.2000) -- (94.0000,120.2000) -- (94.2000,120.1000) -- (94.3000,120.1000) -- (94.4000,120.1000) -- (94.6000,120.1000) -- (94.7000,120.0000) -- (94.9000,120.0000) -- (95.0000,120.0000) -- (95.2000,120.0000) -- (95.3000,120.0000) -- (95.4000,119.9000) -- (95.6000,119.9000) -- (95.7000,119.9000) -- (95.9000,119.9000) -- (96.0000,119.8000) -- (96.2000,119.8000) -- (96.3000,119.8000) -- (96.4000,119.8000) -- (96.6000,119.7000) -- (96.7000,119.7000) -- (96.9000,119.7000) -- (97.0000,119.7000) -- (97.2000,119.6000) -- (97.3000,119.6000) -- (97.4000,119.6000) -- (97.6000,119.5000) -- (97.7000,119.5000) -- (97.9000,119.5000) -- (98.0000,119.5000) -- (98.2000,119.4000) -- (98.3000,119.4000) -- (98.4000,119.4000) -- (98.6000,119.3000) -- (98.7000,119.3000) -- (98.9000,119.3000) -- (99.0000,119.2000) -- (99.2000,119.2000) -- (99.3000,119.2000) -- (99.4000,119.1000) -- (99.6000,119.1000) -- (99.7000,119.1000) -- (99.9000,119.1000) -- (100.0000,119.0000) -- (100.2000,119.0000) -- (100.3000,119.0000) -- (100.4000,118.9000) -- (100.6000,118.9000) -- (100.7000,118.9000) -- (100.9000,118.8000) -- (101.0000,118.8000) -- (101.2000,118.8000) -- (101.3000,118.7000) -- (101.4000,118.7000) -- (101.6000,118.7000) -- (101.7000,118.6000) -- (101.9000,118.6000) -- (102.0000,118.6000) -- (102.2000,118.6000) -- (102.3000,118.5000) -- (102.4000,118.5000) -- (102.6000,118.5000) -- (102.7000,118.4000) -- (102.9000,118.4000) -- (103.0000,118.4000) -- (103.2000,118.4000) -- (103.3000,118.3000) -- (103.4000,118.3000) -- (103.6000,118.3000) -- (103.7000,118.3000) -- (103.9000,118.2000) -- (104.0000,118.2000) -- (104.2000,118.2000) -- (104.3000,118.2000) -- (104.4000,118.1000) -- (104.6000,118.1000) -- (104.7000,118.1000) -- (104.9000,118.1000) -- (105.0000,118.0000) -- (105.2000,118.0000) -- (105.3000,118.0000) -- (105.4000,118.0000) -- (105.6000,118.0000) -- (105.7000,117.9000) -- (105.9000,117.9000) -- (106.0000,117.9000) -- (106.2000,117.9000) -- (106.3000,117.9000) -- (106.4000,117.8000) -- (106.6000,117.8000) -- (106.7000,117.8000) -- (106.9000,117.8000) -- (107.0000,117.8000) -- (107.2000,117.8000) -- (107.3000,117.7000) -- (107.4000,117.7000) -- (107.6000,117.7000) -- (107.7000,117.7000) -- (107.9000,117.7000) -- (108.0000,117.7000) -- (108.2000,117.7000) -- (108.3000,117.6000) -- (108.4000,117.6000) -- (108.6000,117.6000) -- (108.7000,117.6000) -- (108.9000,117.6000) -- (109.0000,117.6000) -- (109.2000,117.6000) -- (109.3000,117.6000) -- (109.4000,117.6000) -- (109.6000,117.6000) -- (109.7000,117.5000) -- (109.9000,117.5000) -- (110.0000,117.5000) -- (110.2000,117.5000) -- (110.3000,117.5000) -- (110.4000,117.5000) -- (110.6000,117.5000) -- (110.7000,117.5000) -- (110.9000,117.5000) -- (111.0000,117.5000) -- (111.2000,117.5000) -- (111.3000,117.5000) -- (111.4000,117.5000) -- (111.6000,117.5000) -- (111.7000,117.5000) -- (111.9000,117.5000) -- (112.0000,117.5000) -- (112.2000,117.5000) -- (112.3000,117.4000) -- (112.4000,117.4000) -- (112.6000,117.4000) -- (112.7000,117.4000) -- (112.9000,117.4000) -- (113.0000,117.4000) -- (113.2000,117.4000) -- (113.3000,117.4000) -- (113.4000,117.4000) -- (113.6000,117.4000) -- (113.7000,117.4000) -- (113.9000,117.4000) -- (114.0000,117.4000) -- (114.2000,117.4000) -- (114.3000,117.4000) -- (114.4000,117.4000) -- (114.6000,117.4000) -- (114.7000,117.4000) -- (114.9000,117.4000) -- (115.0000,117.4000) -- (115.2000,117.4000) -- (115.3000,117.4000) -- (115.4000,117.4000) -- (115.6000,117.4000) -- (115.7000,117.4000) -- (115.9000,117.4000) -- (116.0000,117.4000) -- (116.2000,117.4000) -- (116.3000,117.4000) -- (116.4000,117.4000) -- (116.6000,117.4000) -- (116.7000,117.4000) -- (116.9000,117.5000) -- (117.0000,117.5000) -- (117.2000,117.5000) -- (117.3000,117.5000) -- (117.4000,117.5000) -- (117.6000,117.5000) -- (117.7000,117.5000) -- (117.9000,117.5000) -- (118.0000,117.5000) -- (118.2000,117.5000) -- (118.3000,117.5000) -- (118.4000,117.5000) -- (118.6000,117.5000) -- (118.7000,117.5000) -- (118.9000,117.5000) -- (119.0000,117.5000) -- (119.2000,117.5000) -- (119.3000,117.5000) -- (119.4000,117.5000) -- (119.6000,117.5000) -- (119.7000,117.5000) -- (119.9000,117.5000) -- (120.0000,117.5000) -- (120.2000,117.5000) -- (120.3000,117.5000) -- (120.4000,117.5000) -- (120.6000,117.5000) -- (120.7000,117.5000) -- (120.9000,117.5000) -- (121.0000,117.5000) -- (121.2000,117.5000) -- (121.3000,117.5000) -- (121.4000,117.5000) -- (121.6000,117.5000) -- (121.7000,117.5000) -- (121.9000,117.5000) -- (122.0000,117.5000) -- (122.2000,117.5000) -- (122.3000,117.5000) -- (122.4000,117.5000) -- (122.6000,117.5000) -- (122.7000,117.5000) -- (122.9000,117.5000) -- (123.0000,117.5000) -- (123.2000,117.5000) -- (123.3000,117.5000) -- (123.4000,117.5000) -- (123.6000,117.5000) -- (123.7000,117.5000) -- (123.9000,117.6000) -- (124.0000,117.6000) -- (124.2000,117.6000) -- (124.3000,117.6000) -- (124.4000,117.6000) -- (124.6000,117.6000) -- (124.7000,117.6000) -- (124.9000,117.6000) -- (125.0000,117.6000) -- (125.2000,117.6000) -- (125.3000,117.6000) -- (125.4000,117.6000) -- (125.6000,117.6000) -- (125.7000,117.6000) -- (125.9000,117.6000) -- (126.0000,117.6000) -- (126.2000,117.6000) -- (126.3000,117.6000) -- (126.4000,117.6000) -- (126.6000,117.6000) -- (126.7000,117.6000) -- (126.9000,117.6000) -- (127.0000,117.6000) -- (127.2000,117.6000) -- (127.3000,117.6000);



    \end{scope}
    \begin{scope}[cm={{1.21653,0.0,0.0,1.34548,(-346.94368,-156.28627)}},fill=cffffff]
      \path[fill=cd9d9d9,rounded corners=0.0000cm] (81.0000,129.0000) rectangle (121.0000,157.0000);



    \end{scope}
    \begin{scope}[cm={{1.21653,0.0,0.0,1.34548,(-346.94368,-156.28627)}},draw=ca0a0a4,dash pattern=on 1.40pt off 1.40pt,line cap=round,line join=round,line width=0.233pt,miter limit=4.00]
      \path[draw,dash pattern=on 1.40pt off 1.40pt,line width=0.233pt,miter limit=4.00] (81.5000,135.5000) -- (121.5000,135.5000);



    \end{scope}
    \begin{scope}[cm={{1.21653,0.0,0.0,1.34548,(-346.94368,-156.28627)}},draw=blue,line cap=round,line join=round,line width=0.480pt]
      \path[draw] (81.5000,135.5000) -- (82.6460,135.5000);



      \path[draw] (121.5000,135.5000) -- (120.1220,135.5000);



    \end{scope}
    \begin{scope}[cm={{1.00174,0.0,0.0,1.01515,(-276.4622,28.89387)}},draw=blue,line cap=rect,line join=bevel,line width=0.800pt]
      \path[fill=blue] (0.0000,0.0000) node[above right] (text766) {\scriptsize $T$\hspace{.5ex}=\hspace{.5ex}47};



    \end{scope}
    \begin{scope}[cm={{1.21653,0.0,0.0,1.34548,(-346.94368,-156.28627)}},draw=ca0a0a4,dash pattern=on 1.40pt off 1.40pt,line cap=round,line join=round,line width=0.233pt,miter limit=4.00]
      \path[draw,dash pattern=on 1.40pt off 1.40pt,line width=0.233pt,miter limit=4.00] (97.5000,157.5000) -- (97.5000,129.5000);



    \end{scope}
    \begin{scope}[cm={{1.21653,0.0,0.0,1.34548,(-346.94368,-156.28627)}},draw=blue,line cap=round,line join=round,line width=0.480pt]
      \path[draw] (97.5000,129.5000) -- (97.5000,129.5000) -- (97.5000,130.6564);



    \end{scope}
    \begin{scope}[cm={{1.00174,0.0,0.0,1.01515,(-232.9405,15.55865)}},draw=blue,line cap=rect,line join=bevel,line width=0.800pt]
      \path[fill=blue] (0.0000,0.0000) node[above right] (text794) {\scriptsize 84};



    \end{scope}
    \begin{scope}[cm={{1.21653,0.0,0.0,1.34548,(-346.94368,-156.28627)}},draw=blue,line cap=round,line join=round,line width=0.480pt]
      \path[draw] (81.5000,129.5000) -- (81.5000,157.5000) -- (121.5000,157.5000) -- (121.5000,129.5000) -- (81.5000,129.5000);



    \end{scope}
    \begin{scope}[cm={{1.21653,0.0,0.0,1.34548,(-346.94368,-156.28627)}},draw=blue,line cap=round,line join=round,line width=0.480pt]
      \path[draw] (81.2000,156.5000) -- (81.2000,156.5000) -- (81.2000,156.5000) -- (81.2000,156.5000) -- (81.2000,156.5000) -- (81.2000,156.5000) -- (81.2000,156.5000) -- (81.2000,156.5000) -- (81.2000,156.5000) -- (81.2000,156.5000) -- (81.2000,156.5000) -- (81.2000,156.5000) -- (81.2000,156.5000) -- (81.2000,156.5000) -- (81.2000,156.5000) -- (81.2000,156.5000) -- (81.2000,156.5000) -- (81.2000,156.5000) -- (81.2000,156.5000) -- (81.2000,156.5000) -- (81.2000,156.4000) -- (81.2000,156.4000) -- (81.2000,156.4000) -- (81.2000,156.4000) -- (81.2000,156.4000) -- (81.2000,156.4000) -- (81.3000,156.4000) -- (81.3000,156.4000) -- (81.3000,156.4000) -- (81.3000,156.4000) -- (81.3000,156.4000) -- (81.3000,156.4000) -- (81.3000,156.4000) -- (81.3000,156.4000) -- (81.3000,156.4000) -- (81.3000,156.4000) -- (81.3000,156.4000) -- (81.3000,156.4000) -- (81.3000,156.4000) -- (81.3000,156.4000) -- (81.3000,156.4000) -- (81.3000,156.3000) -- (81.3000,156.3000) -- (81.3000,156.3000) -- (81.3000,156.3000) -- (81.3000,156.3000) -- (81.3000,156.3000) -- (81.3000,156.3000) -- (81.3000,156.3000) -- (81.3000,156.3000) -- (81.3000,156.3000) -- (81.3000,156.3000) -- (81.3000,156.3000) -- (81.3000,156.3000) -- (81.3000,156.3000) -- (81.3000,156.3000) -- (81.3000,156.3000) -- (81.3000,156.3000) -- (81.3000,156.3000) -- (81.3000,156.3000) -- (81.3000,156.3000) -- (81.3000,156.3000) -- (81.3000,156.3000) -- (81.3000,156.2000) -- (81.3000,156.2000) -- (81.3000,156.2000) -- (81.3000,156.2000) -- (81.3000,156.2000) -- (81.3000,156.2000) -- (81.3000,156.2000) -- (81.3000,156.2000) -- (81.3000,156.2000) -- (81.3000,156.2000) -- (81.3000,156.2000) -- (81.3000,156.2000) -- (81.3000,156.2000) -- (81.4000,156.2000) -- (81.4000,156.2000) -- (81.4000,156.2000) -- (81.4000,156.2000) -- (81.4000,156.2000) -- (81.4000,156.2000) -- (81.4000,156.2000) -- (81.4000,156.2000) -- (81.4000,156.1000) -- (81.4000,156.1000) -- (81.4000,156.1000) -- (81.4000,156.1000) -- (81.4000,156.1000) -- (81.4000,156.1000) -- (81.4000,156.1000) -- (81.4000,156.1000) -- (81.4000,156.1000) -- (81.4000,156.1000) -- (81.4000,156.1000) -- (81.4000,156.1000) -- (81.4000,156.1000) -- (81.4000,156.1000) -- (81.4000,156.1000) -- (81.4000,156.1000) -- (81.4000,156.1000) -- (81.4000,156.1000) -- (81.4000,156.1000) -- (81.4000,156.1000) -- (81.4000,156.1000) -- (81.4000,156.1000) -- (81.4000,156.0000) -- (81.4000,156.0000) -- (81.4000,156.0000) -- (81.4000,156.0000) -- (81.4000,156.0000) -- (81.4000,156.0000) -- (81.4000,156.0000) -- (81.4000,156.0000) -- (81.4000,156.0000) -- (81.4000,156.0000) -- (81.4000,156.0000) -- (81.4000,156.0000) -- (81.4000,156.0000) -- (81.4000,156.0000) -- (81.4000,156.0000) -- (81.4000,156.0000) -- (81.4000,156.0000) -- (81.4000,156.0000) -- (81.4000,156.0000) -- (81.5000,156.0000) -- (81.5000,156.0000) -- (81.5000,155.9000) -- (81.5000,155.9000) -- (81.5000,155.9000) -- (81.5000,155.9000) -- (81.5000,155.9000) -- (81.5000,155.9000) -- (81.5000,155.9000) -- (81.5000,155.9000) -- (81.5000,155.9000) -- (81.5000,155.9000) -- (81.5000,155.9000) -- (81.5000,155.9000) -- (81.5000,155.9000) -- (81.5000,155.9000) -- (81.5000,155.9000) -- (81.5000,155.9000) -- (81.5000,155.9000) -- (81.5000,155.9000) -- (81.5000,155.9000) -- (81.5000,155.9000) -- (81.5000,155.9000) -- (81.5000,155.9000) -- (81.5000,155.8000) -- (81.5000,155.8000) -- (81.5000,155.8000) -- (81.5000,155.8000) -- (81.5000,155.8000) -- (81.5000,155.8000) -- (81.5000,155.8000) -- (81.5000,155.8000) -- (81.5000,155.8000) -- (81.5000,155.8000) -- (81.5000,155.8000) -- (81.5000,155.8000) -- (81.5000,155.8000) -- (81.5000,155.8000) -- (81.5000,155.8000) -- (81.5000,155.8000) -- (81.5000,155.8000) -- (81.5000,155.8000) -- (81.5000,155.8000) -- (81.5000,155.8000) -- (81.5000,155.8000) -- (81.5000,155.7000) -- (81.5000,155.7000) -- (81.5000,155.7000) -- (81.5000,155.7000) -- (81.5000,155.7000) -- (81.6000,155.7000) -- (81.6000,155.7000) -- (81.6000,155.7000) -- (81.6000,155.7000) -- (81.6000,155.7000) -- (81.6000,155.7000) -- (81.6000,155.7000) -- (81.6000,155.7000) -- (81.6000,155.7000) -- (81.6000,155.7000) -- (81.6000,155.7000) -- (81.6000,155.7000) -- (81.6000,155.7000) -- (81.6000,155.7000) -- (81.6000,155.7000) -- (81.6000,155.7000) -- (81.6000,155.7000) -- (81.6000,155.6000) -- (81.6000,155.6000) -- (81.6000,155.6000) -- (81.6000,155.6000) -- (81.6000,155.6000) -- (81.6000,155.6000) -- (81.6000,155.6000) -- (81.6000,155.6000) -- (81.6000,155.6000) -- (81.6000,155.6000) -- (81.6000,155.6000) -- (81.6000,155.6000) -- (81.6000,155.6000) -- (81.6000,155.6000) -- (81.6000,155.6000) -- (81.6000,155.6000) -- (81.6000,155.6000) -- (81.6000,155.6000) -- (81.6000,155.6000) -- (81.6000,155.6000) -- (81.6000,155.6000) -- (81.6000,155.5000) -- (81.6000,155.5000) -- (81.6000,155.5000) -- (81.6000,155.5000) -- (81.6000,155.5000) -- (81.6000,155.5000) -- (81.6000,155.5000) -- (81.6000,155.5000) -- (81.6000,155.5000) -- (81.6000,155.5000) -- (81.6000,155.5000) -- (81.7000,155.5000) -- (81.7000,155.5000) -- (81.7000,155.5000) -- (81.7000,155.5000) -- (81.7000,155.5000) -- (81.7000,155.5000) -- (81.7000,155.5000) -- (81.7000,155.5000) -- (81.7000,155.5000) -- (81.7000,155.5000) -- (81.7000,155.5000) -- (81.7000,155.4000) -- (81.7000,155.4000) -- (81.7000,155.4000) -- (81.7000,155.4000) -- (81.7000,155.4000) -- (81.7000,155.4000) -- (81.7000,155.4000) -- (81.7000,155.4000) -- (81.7000,155.4000) -- (81.7000,155.4000) -- (81.7000,155.4000) -- (81.7000,155.4000) -- (81.7000,155.4000) -- (81.7000,155.4000) -- (81.7000,155.4000) -- (81.7000,155.4000) -- (81.7000,155.4000) -- (81.7000,155.4000) -- (81.7000,155.4000) -- (81.7000,155.4000) -- (81.7000,155.4000) -- (81.7000,155.3000) -- (81.7000,155.3000) -- (81.7000,155.3000) -- (81.7000,155.3000) -- (81.7000,155.3000) -- (81.7000,155.3000) -- (81.7000,155.3000) -- (81.7000,155.3000) -- (81.7000,155.3000) -- (81.7000,155.3000) -- (81.7000,155.3000) -- (81.7000,155.3000) -- (81.7000,155.3000) -- (81.7000,155.3000) -- (81.7000,155.3000) -- (81.7000,155.3000) -- (81.7000,155.3000) -- (81.7000,155.3000) -- (81.8000,155.3000) -- (81.8000,155.3000) -- (81.8000,155.3000) -- (81.8000,155.3000) -- (81.8000,155.2000) -- (81.8000,155.2000) -- (81.8000,155.2000) -- (81.8000,155.2000) -- (81.8000,155.2000) -- (81.8000,155.2000) -- (81.8000,155.2000) -- (81.8000,155.2000) -- (81.8000,155.2000) -- (81.8000,155.2000) -- (81.8000,155.2000) -- (81.8000,155.2000) -- (81.8000,155.2000) -- (81.8000,155.2000) -- (81.8000,155.2000) -- (81.8000,155.2000) -- (81.8000,155.2000) -- (81.8000,155.2000) -- (81.8000,155.2000) -- (81.8000,155.2000) -- (81.8000,155.2000) -- (81.8000,155.1000) -- (81.8000,155.1000) -- (81.8000,155.1000) -- (81.8000,155.1000) -- (81.8000,155.1000) -- (81.8000,155.1000) -- (81.8000,155.1000) -- (81.8000,155.1000) -- (81.8000,155.1000) -- (81.8000,155.1000) -- (81.8000,155.1000) -- (81.8000,155.1000) -- (81.8000,155.1000) -- (81.8000,155.1000) -- (81.8000,155.1000) -- (81.8000,155.1000) -- (81.8000,155.1000) -- (81.8000,155.1000) -- (81.8000,155.1000) -- (81.8000,155.1000) -- (81.8000,155.1000) -- (81.8000,155.1000) -- (81.8000,155.0000) -- (81.8000,155.0000) -- (81.9000,155.0000) -- (81.9000,155.0000) -- (81.9000,155.0000) -- (81.9000,155.0000) -- (81.9000,155.0000) -- (81.9000,155.0000) -- (81.9000,155.0000) -- (81.9000,155.0000) -- (81.9000,155.0000) -- (81.9000,155.0000) -- (81.9000,155.0000) -- (81.9000,155.0000) -- (81.9000,155.0000) -- (81.9000,155.0000) -- (81.9000,155.0000) -- (81.9000,155.0000) -- (81.9000,155.0000) -- (81.9000,155.0000) -- (81.9000,155.0000) -- (81.9000,154.9000) -- (81.9000,154.9000) -- (81.9000,154.9000) -- (81.9000,154.9000) -- (81.9000,154.9000) -- (81.9000,154.9000) -- (81.9000,154.9000) -- (81.9000,154.9000) -- (81.9000,154.9000) -- (81.9000,154.9000) -- (81.9000,154.9000) -- (81.9000,154.9000) -- (81.9000,154.9000) -- (81.9000,154.9000) -- (81.9000,154.9000) -- (81.9000,154.9000) -- (81.9000,154.9000) -- (81.9000,154.9000) -- (81.9000,154.9000) -- (81.9000,154.9000) -- (81.9000,154.9000) -- (81.9000,154.9000) -- (81.9000,154.8000) -- (81.9000,154.8000) -- (81.9000,154.8000) -- (81.9000,154.8000) -- (81.9000,154.8000) -- (81.9000,154.8000) -- (81.9000,154.8000) -- (81.9000,154.8000) -- (81.9000,154.8000) -- (82.0000,154.8000) -- (82.0000,154.8000) -- (82.0000,154.8000) -- (82.0000,154.8000) -- (82.0000,154.8000) -- (82.0000,154.8000) -- (82.0000,154.8000) -- (82.0000,154.8000) -- (82.0000,154.8000) -- (82.0000,154.8000) -- (82.0000,154.8000) -- (82.0000,154.8000) -- (82.0000,154.7000) -- (82.0000,154.7000) -- (82.0000,154.7000) -- (82.0000,154.7000) -- (82.0000,154.7000) -- (82.0000,154.7000) -- (82.0000,154.7000) -- (82.0000,154.7000) -- (82.0000,154.7000) -- (82.0000,154.7000) -- (82.0000,154.7000) -- (82.0000,154.7000) -- (82.0000,154.7000) -- (82.0000,154.7000) -- (82.0000,154.7000) -- (82.0000,154.7000) -- (82.0000,154.7000) -- (82.0000,154.7000) -- (82.0000,154.7000) -- (82.0000,154.7000) -- (82.0000,154.7000) -- (82.0000,154.7000) -- (82.0000,154.6000) -- (82.0000,154.6000) -- (82.0000,154.6000) -- (82.0000,154.6000) -- (82.0000,154.6000) -- (82.0000,154.6000) -- (82.0000,154.6000) -- (82.0000,154.6000) -- (82.0000,154.6000) -- (82.0000,154.6000) -- (82.0000,154.6000) -- (82.0000,154.6000) -- (82.0000,154.6000) -- (82.0000,154.6000) -- (82.0000,154.6000) -- (82.1000,154.6000) -- (82.1000,154.6000) -- (82.1000,154.6000) -- (82.1000,154.6000) -- (82.1000,154.6000) -- (82.1000,154.6000) -- (82.1000,154.5000) -- (82.1000,154.5000) -- (82.1000,154.5000) -- (82.1000,154.5000) -- (82.1000,154.5000) -- (82.1000,154.5000) -- (82.1000,154.5000) -- (82.1000,154.5000) -- (82.1000,154.5000) -- (82.1000,154.5000) -- (82.1000,154.5000) -- (82.1000,154.5000) -- (82.1000,154.5000) -- (82.1000,154.5000) -- (82.1000,154.5000) -- (82.1000,154.5000) -- (82.1000,154.5000) -- (82.1000,154.5000) -- (82.1000,154.5000) -- (82.1000,154.5000) -- (82.1000,154.5000) -- (82.1000,154.5000) -- (82.1000,154.4000) -- (82.1000,154.4000) -- (82.1000,154.4000) -- (82.1000,154.4000) -- (82.1000,154.4000) -- (82.1000,154.4000) -- (82.1000,154.4000) -- (82.1000,154.4000) -- (82.1000,154.4000) -- (82.1000,154.4000) -- (82.1000,154.4000) -- (82.1000,154.4000) -- (82.1000,154.4000) -- (82.1000,154.4000) -- (82.1000,154.4000) -- (82.1000,154.4000) -- (82.1000,154.4000) -- (82.1000,154.4000) -- (82.1000,154.4000) -- (82.1000,154.4000) -- (82.1000,154.4000) -- (82.1000,154.3000) -- (82.2000,154.3000) -- (82.2000,154.3000) -- (82.2000,154.3000) -- (82.2000,154.3000) -- (82.2000,154.3000) -- (82.2000,154.3000) -- (82.2000,154.3000) -- (82.2000,154.3000) -- (82.2000,154.3000) -- (82.2000,154.3000) -- (82.2000,154.3000) -- (82.2000,154.3000) -- (82.2000,154.3000) -- (82.2000,154.3000) -- (82.2000,154.3000) -- (82.2000,154.3000) -- (82.2000,154.3000) -- (82.2000,154.3000) -- (82.2000,154.3000) -- (82.2000,154.3000) -- (82.2000,154.3000) -- (82.2000,154.2000) -- (82.2000,154.2000) -- (82.2000,154.2000) -- (82.2000,154.2000) -- (82.2000,154.2000) -- (82.2000,154.2000) -- (82.2000,154.2000) -- (82.2000,154.2000) -- (82.2000,154.2000) -- (82.2000,154.2000) -- (82.2000,154.2000) -- (82.2000,154.2000) -- (82.2000,154.2000) -- (82.2000,154.2000) -- (82.2000,154.2000) -- (82.2000,154.2000) -- (82.2000,154.2000) -- (82.2000,154.2000) -- (82.2000,154.2000) -- (82.2000,154.2000) -- (82.2000,154.2000) -- (82.2000,154.1000) -- (82.2000,154.1000) -- (82.2000,154.1000) -- (82.2000,154.1000) -- (82.2000,154.1000) -- (82.2000,154.1000) -- (82.2000,154.1000) -- (82.3000,154.1000) -- (82.3000,154.1000) -- (82.3000,154.1000) -- (82.3000,154.1000) -- (82.3000,154.1000) -- (82.3000,154.1000) -- (82.3000,154.1000) -- (82.3000,154.1000) -- (82.3000,154.1000) -- (82.3000,154.1000) -- (82.3000,154.1000) -- (82.3000,154.1000) -- (82.3000,154.1000) -- (82.3000,154.1000) -- (82.3000,154.1000) -- (82.3000,154.0000) -- (82.3000,154.0000) -- (82.3000,154.0000) -- (82.3000,154.0000) -- (82.3000,154.0000) -- (82.3000,154.0000) -- (82.3000,154.0000) -- (82.3000,154.0000) -- (82.3000,154.0000) -- (82.3000,154.0000) -- (82.3000,154.0000) -- (82.3000,154.0000) -- (82.3000,154.0000) -- (82.3000,154.0000) -- (82.3000,154.0000) -- (82.3000,154.0000) -- (82.3000,154.0000) -- (82.3000,154.0000) -- (82.3000,154.0000) -- (82.3000,154.0000) -- (82.3000,154.0000) -- (82.3000,153.9000) -- (82.3000,153.9000) -- (82.3000,153.9000) -- (82.3000,153.9000) -- (82.3000,153.9000) -- (82.3000,153.9000) -- (82.3000,153.9000) -- (82.3000,153.9000) -- (82.3000,153.9000) -- (82.3000,153.9000) -- (82.3000,153.9000) -- (82.3000,153.9000) -- (82.3000,153.9000) -- (82.3000,153.9000) -- (82.4000,153.9000) -- (82.4000,153.9000) -- (82.4000,153.9000) -- (82.4000,153.9000) -- (82.4000,153.9000) -- (82.4000,153.9000) -- (82.4000,153.9000) -- (82.4000,153.9000) -- (82.4000,153.8000) -- (82.4000,153.8000) -- (82.4000,153.8000) -- (82.4000,153.8000) -- (82.4000,153.8000) -- (82.4000,153.8000) -- (82.4000,153.8000) -- (82.4000,153.8000) -- (82.4000,153.8000) -- (82.4000,153.8000) -- (82.4000,153.8000) -- (82.4000,153.8000) -- (82.4000,153.8000) -- (82.4000,153.8000) -- (82.4000,153.8000) -- (82.4000,153.8000) -- (82.4000,153.8000) -- (82.4000,153.8000) -- (82.4000,153.8000) -- (82.4000,153.8000) -- (82.4000,153.8000) -- (82.4000,153.7000) -- (82.4000,153.7000) -- (82.4000,153.7000) -- (82.4000,153.7000) -- (82.4000,153.7000) -- (82.4000,153.7000) -- (82.4000,153.7000) -- (82.4000,153.7000) -- (82.4000,153.7000) -- (82.4000,153.7000) -- (82.4000,153.7000) -- (82.4000,153.7000) -- (82.4000,153.7000) -- (82.4000,153.7000) -- (82.4000,153.7000) -- (82.4000,153.7000) -- (82.4000,153.7000) -- (82.4000,153.7000) -- (82.4000,153.7000) -- (82.4000,153.7000) -- (82.5000,153.7000) -- (82.5000,153.7000) -- (82.5000,153.6000) -- (82.5000,153.6000) -- (82.5000,153.6000) -- (82.5000,153.6000) -- (82.5000,153.6000) -- (82.5000,153.6000) -- (82.5000,153.6000) -- (82.5000,153.6000) -- (82.5000,153.6000) -- (82.5000,153.6000) -- (82.5000,153.6000) -- (82.5000,153.6000) -- (82.5000,153.6000) -- (82.5000,153.6000) -- (82.5000,153.6000) -- (82.5000,153.6000) -- (82.5000,153.6000) -- (82.5000,153.6000) -- (82.5000,153.6000) -- (82.5000,153.6000) -- (82.5000,153.6000) -- (82.5000,153.5000) -- (82.5000,153.5000) -- (82.5000,153.5000) -- (82.5000,153.5000) -- (82.5000,153.5000) -- (82.5000,153.5000) -- (82.5000,153.5000) -- (82.5000,153.5000) -- (82.5000,153.5000) -- (82.5000,153.5000) -- (82.5000,153.5000) -- (82.5000,153.5000) -- (82.5000,153.5000) -- (82.5000,153.5000) -- (82.5000,153.5000) -- (82.5000,153.5000) -- (82.5000,153.5000) -- (82.5000,153.5000) -- (82.5000,153.5000) -- (82.5000,153.5000) -- (82.5000,153.5000) -- (82.5000,153.5000) -- (82.5000,153.4000) -- (82.5000,153.4000) -- (82.5000,153.4000) -- (82.5000,153.4000) -- (82.5000,153.4000) -- (82.6000,153.4000) -- (82.6000,153.4000) -- (82.6000,153.4000) -- (82.6000,153.4000) -- (82.6000,153.4000) -- (82.6000,153.4000) -- (82.6000,153.4000) -- (82.6000,153.4000) -- (82.6000,153.4000) -- (82.6000,153.4000) -- (82.6000,153.4000) -- (82.6000,153.4000) -- (82.6000,153.4000) -- (82.6000,153.4000) -- (82.6000,153.4000) -- (82.6000,153.4000) -- (82.6000,153.3000) -- (82.6000,153.3000) -- (82.6000,153.3000) -- (82.6000,153.3000) -- (82.6000,153.3000) -- (82.6000,153.3000) -- (82.6000,153.3000) -- (82.6000,153.3000) -- (82.6000,153.3000) -- (82.6000,153.3000) -- (82.6000,153.3000) -- (82.6000,153.3000) -- (82.6000,153.3000) -- (82.6000,153.3000) -- (82.6000,153.3000) -- (82.6000,153.3000) -- (82.6000,153.3000) -- (82.6000,153.3000) -- (82.6000,153.3000) -- (82.6000,153.3000) -- (82.6000,153.3000) -- (82.6000,153.3000) -- (82.6000,153.2000) -- (82.6000,153.2000) -- (82.6000,153.2000) -- (82.6000,153.2000) -- (82.6000,153.2000) -- (82.6000,153.2000) -- (82.6000,153.2000) -- (82.6000,153.2000) -- (82.6000,153.2000) -- (82.6000,153.2000) -- (82.6000,153.2000) -- (82.7000,153.2000) -- (82.7000,153.2000) -- (82.7000,153.2000) -- (82.7000,153.2000) -- (82.7000,153.2000) -- (82.7000,153.2000) -- (82.7000,153.2000) -- (82.7000,153.2000) -- (82.7000,153.2000) -- (82.7000,153.2000) -- (82.7000,153.1000) -- (82.7000,153.1000) -- (82.7000,153.1000) -- (82.7000,153.1000) -- (82.7000,153.1000) -- (82.7000,153.1000) -- (82.7000,153.1000) -- (82.7000,153.1000) -- (82.7000,153.1000) -- (82.7000,153.1000) -- (82.7000,153.1000) -- (82.7000,153.1000) -- (82.7000,153.1000) -- (82.7000,153.1000) -- (82.7000,153.1000) -- (82.7000,153.1000) -- (82.7000,153.1000) -- (82.7000,153.1000) -- (82.7000,153.1000) -- (82.7000,153.1000) -- (82.7000,153.1000) -- (82.7000,153.1000) -- (82.7000,153.0000) -- (82.7000,153.0000) -- (82.7000,153.0000) -- (82.7000,153.0000) -- (82.7000,153.0000) -- (82.7000,153.0000) -- (82.7000,153.0000) -- (82.7000,153.0000) -- (82.7000,153.0000) -- (82.7000,153.0000) -- (82.7000,153.0000) -- (82.7000,153.0000) -- (82.7000,153.0000) -- (82.7000,153.0000) -- (82.7000,153.0000) -- (82.7000,153.0000) -- (82.7000,153.0000) -- (82.7000,153.0000) -- (82.8000,153.0000) -- (82.8000,153.0000) -- (82.8000,153.0000) -- (82.8000,152.9000) -- (82.8000,152.9000) -- (82.8000,152.9000) -- (82.8000,152.9000) -- (82.8000,152.9000) -- (82.8000,152.9000) -- (82.8000,152.9000) -- (82.8000,152.9000) -- (82.8000,152.9000) -- (82.8000,152.9000) -- (82.8000,152.9000) -- (82.8000,152.9000) -- (82.8000,152.9000) -- (82.8000,152.9000) -- (82.8000,152.9000) -- (82.8000,152.9000) -- (82.8000,152.9000) -- (82.8000,152.9000) -- (82.8000,152.9000) -- (82.8000,152.9000) -- (82.8000,152.9000) -- (82.8000,152.9000) -- (82.8000,152.8000) -- (82.8000,152.8000) -- (82.8000,152.8000) -- (82.8000,152.8000) -- (82.8000,152.8000) -- (82.8000,152.8000) -- (82.8000,152.8000) -- (82.8000,152.8000) -- (82.8000,152.8000) -- (82.8000,152.8000) -- (82.8000,152.8000) -- (82.8000,152.8000) -- (82.8000,152.8000) -- (82.8000,152.8000) -- (82.8000,152.8000) -- (82.8000,152.8000) -- (82.8000,152.8000) -- (82.8000,152.8000) -- (82.8000,152.8000) -- (82.8000,152.8000) -- (82.8000,152.8000) -- (82.8000,152.7000) -- (82.8000,152.7000) -- (82.8000,152.7000) -- (82.9000,152.7000) -- (82.9000,152.7000) -- (82.9000,152.7000) -- (82.9000,152.7000) -- (82.9000,152.7000) -- (82.9000,152.7000) -- (82.9000,152.7000) -- (82.9000,152.7000) -- (82.9000,152.7000) -- (82.9000,152.7000) -- (82.9000,152.7000) -- (82.9000,152.7000) -- (82.9000,152.7000) -- (82.9000,152.7000) -- (82.9000,152.7000) -- (82.9000,152.7000) -- (82.9000,152.7000) -- (82.9000,152.7000) -- (82.9000,152.7000) -- (82.9000,152.6000) -- (82.9000,152.6000) -- (82.9000,152.6000) -- (82.9000,152.6000) -- (82.9000,152.6000) -- (82.9000,152.6000) -- (82.9000,152.6000) -- (82.9000,152.6000) -- (82.9000,152.6000) -- (82.9000,152.6000) -- (82.9000,152.6000) -- (82.9000,152.6000) -- (82.9000,152.6000) -- (82.9000,152.6000) -- (82.9000,152.6000) -- (82.9000,152.6000) -- (82.9000,152.6000) -- (82.9000,152.6000) -- (82.9000,152.6000) -- (82.9000,152.6000) -- (82.9000,152.6000) -- (82.9000,152.5000) -- (82.9000,152.5000) -- (82.9000,152.5000) -- (82.9000,152.5000) -- (82.9000,152.5000) -- (82.9000,152.5000) -- (82.9000,152.5000) -- (82.9000,152.5000) -- (82.9000,152.5000) -- (82.9000,152.5000) -- (83.0000,152.5000) -- (83.0000,152.5000) -- (83.0000,152.5000) -- (83.0000,152.5000) -- (83.0000,152.5000) -- (83.0000,152.5000) -- (83.0000,152.5000) -- (83.0000,152.5000) -- (83.0000,152.5000) -- (83.0000,152.5000) -- (83.0000,152.5000) -- (83.0000,152.5000) -- (83.0000,152.4000) -- (83.0000,152.4000) -- (83.0000,152.4000) -- (83.0000,152.4000) -- (83.0000,152.4000) -- (83.0000,152.4000) -- (83.0000,152.4000) -- (83.0000,152.4000) -- (83.0000,152.4000) -- (83.0000,152.4000) -- (83.0000,152.4000) -- (83.0000,152.4000) -- (83.0000,152.4000) -- (83.0000,152.4000) -- (83.0000,152.4000) -- (83.0000,152.4000) -- (83.0000,152.4000) -- (83.0000,152.4000) -- (83.0000,152.4000) -- (83.0000,152.4000) -- (83.0000,152.4000) -- (83.0000,152.3000) -- (83.0000,152.3000) -- (83.0000,152.3000) -- (83.0000,152.3000) -- (83.0000,152.3000) -- (83.0000,152.3000) -- (83.0000,152.3000) -- (83.0000,152.3000) -- (83.0000,152.3000) -- (83.0000,152.3000) -- (83.0000,152.3000) -- (83.0000,152.3000) -- (83.0000,152.3000) -- (83.0000,152.3000) -- (83.0000,152.3000) -- (83.0000,152.3000) -- (83.1000,152.3000) -- (83.1000,152.3000) -- (83.1000,152.3000) -- (83.1000,152.3000) -- (83.1000,152.3000) -- (83.1000,152.3000) -- (83.1000,152.2000) -- (83.1000,152.2000) -- (83.1000,152.2000) -- (83.1000,152.2000) -- (83.1000,152.2000) -- (83.1000,152.2000) -- (83.1000,152.2000) -- (83.1000,152.2000) -- (83.1000,152.2000) -- (83.1000,152.2000) -- (83.1000,152.2000) -- (83.1000,152.2000) -- (83.1000,152.2000) -- (83.1000,152.2000) -- (83.1000,152.2000) -- (83.1000,152.2000) -- (83.1000,152.2000) -- (83.1000,152.2000) -- (83.1000,152.2000) -- (83.1000,152.2000) -- (83.1000,152.2000) -- (83.1000,152.2000) -- (83.1000,152.1000) -- (83.1000,152.1000) -- (83.1000,152.1000) -- (83.1000,152.1000) -- (83.1000,152.1000) -- (83.1000,152.1000) -- (83.1000,152.1000) -- (83.1000,152.1000) -- (83.1000,152.1000) -- (83.1000,152.1000) -- (83.1000,152.1000) -- (83.1000,152.1000) -- (83.1000,152.1000) -- (83.1000,152.1000) -- (83.1000,152.1000) -- (83.1000,152.1000) -- (83.1000,152.1000) -- (83.1000,152.1000) -- (83.1000,152.1000) -- (83.1000,152.1000) -- (83.1000,152.1000) -- (83.1000,152.0000) -- (83.2000,152.0000) -- (83.2000,152.0000) -- (83.2000,152.0000) -- (83.2000,152.0000) -- (83.2000,152.0000) -- (83.2000,152.0000) -- (83.2000,152.0000) -- (83.2000,152.0000) -- (83.2000,152.0000) -- (83.2000,152.0000) -- (83.2000,152.0000) -- (83.2000,152.0000) -- (83.2000,152.0000) -- (83.2000,152.0000) -- (83.2000,152.0000) -- (83.2000,152.0000) -- (83.2000,152.0000) -- (83.2000,152.0000) -- (83.2000,152.0000) -- (83.2000,152.0000) -- (83.2000,152.0000) -- (83.2000,151.9000) -- (83.2000,151.9000) -- (83.2000,151.9000) -- (83.2000,151.9000) -- (83.2000,151.9000) -- (83.2000,151.9000) -- (83.2000,151.9000) -- (83.2000,151.9000) -- (83.2000,151.9000) -- (83.2000,151.9000) -- (83.2000,151.9000) -- (83.2000,151.9000) -- (83.2000,151.9000) -- (83.2000,151.9000) -- (83.2000,151.9000) -- (83.2000,151.9000) -- (83.2000,151.9000) -- (83.2000,151.9000) -- (83.2000,151.9000) -- (83.2000,151.9000) -- (83.2000,151.9000) -- (83.2000,151.8000) -- (83.2000,151.8000) -- (83.2000,151.8000) -- (83.2000,151.8000) -- (83.2000,151.8000) -- (83.2000,151.8000) -- (83.2000,151.8000) -- (83.3000,151.8000) -- (83.3000,151.8000) -- (83.3000,151.8000) -- (83.3000,151.8000) -- (83.3000,151.8000) -- (83.3000,151.8000) -- (83.3000,151.8000) -- (83.3000,151.8000) -- (83.3000,151.8000) -- (83.3000,151.8000) -- (83.3000,151.8000) -- (83.3000,151.8000) -- (83.3000,151.8000) -- (83.3000,151.8000) -- (83.3000,151.8000) -- (83.3000,151.7000) -- (83.3000,151.7000) -- (83.3000,151.7000) -- (83.3000,151.7000) -- (83.3000,151.7000) -- (83.3000,151.7000) -- (83.3000,151.7000) -- (83.3000,151.7000) -- (83.3000,151.7000) -- (83.3000,151.7000) -- (83.3000,151.7000) -- (83.3000,151.7000) -- (83.3000,151.7000) -- (83.3000,151.7000) -- (83.3000,151.7000) -- (83.3000,151.7000) -- (83.3000,151.7000) -- (83.3000,151.7000) -- (83.3000,151.7000) -- (83.3000,151.7000) -- (83.3000,151.7000) -- (83.3000,151.6000) -- (83.3000,151.6000) -- (83.3000,151.6000) -- (83.3000,151.6000) -- (83.3000,151.6000) -- (83.3000,151.6000) -- (83.3000,151.6000) -- (83.3000,151.6000) -- (83.3000,151.6000) -- (83.3000,151.6000) -- (83.3000,151.6000) -- (83.3000,151.6000) -- (83.3000,151.6000) -- (83.3000,151.6000) -- (83.4000,151.6000) -- (83.4000,151.6000) -- (83.4000,151.6000) -- (83.4000,151.6000) -- (83.4000,151.6000) -- (83.4000,151.6000) -- (83.4000,151.6000) -- (83.4000,151.6000) -- (83.4000,151.5000) -- (83.4000,151.5000) -- (83.4000,151.5000) -- (83.4000,151.5000) -- (83.4000,151.5000) -- (83.4000,151.5000) -- (83.4000,151.5000) -- (83.4000,151.5000) -- (83.4000,151.5000) -- (83.4000,151.5000) -- (83.4000,151.5000) -- (83.4000,151.5000) -- (83.4000,151.5000) -- (83.4000,151.5000) -- (83.4000,151.5000) -- (83.4000,151.5000) -- (83.4000,151.5000) -- (83.4000,151.5000) -- (83.4000,151.5000) -- (83.4000,151.5000) -- (83.4000,151.5000) -- (83.4000,151.4000) -- (83.4000,151.4000) -- (83.4000,151.4000) -- (83.4000,151.4000) -- (83.4000,151.4000) -- (83.4000,151.4000) -- (83.4000,151.4000) -- (83.4000,151.4000) -- (83.4000,151.4000) -- (83.4000,151.4000) -- (83.4000,151.4000) -- (83.4000,151.4000) -- (83.4000,151.4000) -- (83.4000,151.4000) -- (83.4000,151.4000) -- (83.4000,151.4000) -- (83.4000,151.4000) -- (83.4000,151.4000) -- (83.4000,151.4000) -- (83.4000,151.4000) -- (83.5000,151.4000) -- (83.5000,151.4000) -- (83.5000,151.3000) -- (83.5000,151.3000) -- (83.5000,151.3000) -- (83.5000,151.3000) -- (83.5000,151.3000) -- (83.5000,151.3000) -- (83.5000,151.3000) -- (83.5000,151.3000) -- (83.5000,151.3000) -- (83.5000,151.3000) -- (83.5000,151.3000) -- (83.5000,151.3000) -- (83.5000,151.3000) -- (83.5000,151.3000) -- (83.5000,151.3000) -- (83.5000,151.3000) -- (83.5000,151.3000) -- (83.5000,151.3000) -- (83.5000,151.3000) -- (83.5000,151.3000) -- (83.5000,151.3000) -- (83.5000,151.2000) -- (83.5000,151.2000) -- (83.5000,151.2000) -- (83.5000,151.2000) -- (83.5000,151.2000) -- (83.5000,151.2000) -- (83.5000,151.2000) -- (83.5000,151.2000) -- (83.5000,151.2000) -- (83.5000,151.2000) -- (83.5000,151.2000) -- (83.5000,151.2000) -- (83.5000,151.2000) -- (83.5000,151.2000) -- (83.5000,151.2000) -- (83.5000,151.2000) -- (83.5000,151.2000) -- (83.5000,151.2000) -- (83.5000,151.2000) -- (83.5000,151.2000) -- (83.5000,151.2000) -- (83.5000,151.2000) -- (83.5000,151.1000) -- (83.5000,151.1000) -- (83.5000,151.1000) -- (83.5000,151.1000) -- (83.5000,151.1000) -- (83.6000,151.1000) -- (83.6000,151.1000) -- (83.6000,151.1000) -- (83.6000,151.1000) -- (83.6000,151.1000) -- (83.6000,151.1000) -- (83.6000,151.1000) -- (83.6000,151.1000) -- (83.6000,151.1000) -- (83.6000,151.1000) -- (83.6000,151.1000) -- (83.6000,151.1000) -- (83.6000,151.1000) -- (83.6000,151.1000) -- (83.6000,151.1000) -- (83.6000,151.1000) -- (83.6000,151.0000) -- (83.6000,151.0000) -- (83.6000,151.0000) -- (83.6000,151.0000) -- (83.6000,151.0000) -- (83.6000,151.0000) -- (83.6000,151.0000) -- (83.6000,151.0000) -- (83.6000,151.0000) -- (83.6000,151.0000) -- (83.6000,151.0000) -- (83.6000,151.0000) -- (83.6000,151.0000) -- (83.6000,151.0000) -- (83.6000,151.0000) -- (83.6000,151.0000) -- (83.6000,151.0000) -- (83.6000,151.0000) -- (83.6000,151.0000) -- (83.6000,151.0000) -- (83.6000,151.0000) -- (83.6000,151.0000) -- (83.6000,150.9000) -- (83.6000,150.9000) -- (83.6000,150.9000) -- (83.6000,150.9000) -- (83.6000,150.9000) -- (83.6000,150.9000) -- (83.6000,150.9000) -- (83.6000,150.9000) -- (83.6000,150.9000) -- (83.6000,150.9000) -- (83.6000,150.9000) -- (83.7000,150.9000) -- (83.7000,150.9000) -- (83.7000,150.9000) -- (83.7000,150.9000) -- (83.7000,150.9000) -- (83.7000,150.9000) -- (83.7000,150.9000) -- (83.7000,150.9000) -- (83.7000,150.9000) -- (83.7000,150.9000) -- (83.7000,150.8000) -- (83.7000,150.8000) -- (83.7000,150.8000) -- (83.7000,150.8000) -- (83.7000,150.8000) -- (83.7000,150.8000) -- (83.7000,150.8000) -- (83.7000,150.8000) -- (83.7000,150.8000) -- (83.7000,150.8000) -- (83.7000,150.8000) -- (83.7000,150.8000) -- (83.7000,150.8000) -- (83.7000,150.8000) -- (83.7000,150.8000) -- (83.7000,150.8000) -- (83.7000,150.8000) -- (83.7000,150.8000) -- (83.7000,150.8000) -- (83.7000,150.8000) -- (83.7000,150.8000) -- (83.7000,150.8000) -- (83.7000,150.7000) -- (83.7000,150.7000) -- (83.7000,150.7000) -- (83.7000,150.7000) -- (83.7000,150.7000) -- (83.7000,150.7000) -- (83.7000,150.7000) -- (83.7000,150.7000) -- (83.7000,150.7000) -- (83.7000,150.7000) -- (83.7000,150.7000) -- (83.7000,150.7000) -- (83.7000,150.7000) -- (83.7000,150.7000) -- (83.7000,150.7000) -- (83.7000,150.7000) -- (83.7000,150.7000) -- (83.7000,150.7000) -- (83.8000,150.7000) -- (83.8000,150.7000) -- (83.8000,150.7000) -- (83.8000,150.6000) -- (83.8000,150.6000) -- (83.8000,150.6000) -- (83.8000,150.6000) -- (83.8000,150.6000) -- (83.8000,150.6000) -- (83.8000,150.6000) -- (83.8000,150.6000) -- (83.8000,150.6000) -- (83.8000,150.6000) -- (83.8000,150.6000) -- (83.8000,150.6000) -- (83.8000,150.6000) -- (83.8000,150.6000) -- (83.8000,150.6000) -- (83.8000,150.6000) -- (83.8000,150.6000) -- (83.8000,150.6000) -- (83.8000,150.6000) -- (83.8000,150.6000) -- (83.8000,150.6000) -- (83.8000,150.6000) -- (83.8000,150.5000) -- (83.8000,150.5000) -- (83.8000,150.5000) -- (83.8000,150.5000) -- (83.8000,150.5000) -- (83.8000,150.5000) -- (83.8000,150.5000) -- (83.8000,150.5000) -- (83.8000,150.5000) -- (83.8000,150.5000) -- (83.8000,150.5000) -- (83.8000,150.5000) -- (83.8000,150.5000) -- (83.8000,150.5000) -- (83.8000,150.5000) -- (83.8000,150.5000) -- (83.8000,150.5000) -- (83.8000,150.5000) -- (83.8000,150.5000) -- (83.8000,150.5000) -- (83.8000,150.5000) -- (83.8000,150.4000) -- (83.8000,150.4000) -- (83.8000,150.4000) -- (83.9000,150.4000) -- (83.9000,150.4000) -- (83.9000,150.4000) -- (83.9000,150.4000) -- (83.9000,150.4000) -- (83.9000,150.4000) -- (83.9000,150.4000) -- (83.9000,150.4000) -- (83.9000,150.4000) -- (83.9000,150.4000) -- (83.9000,150.4000) -- (83.9000,150.4000) -- (83.9000,150.4000) -- (83.9000,150.4000) -- (83.9000,150.4000) -- (83.9000,150.4000) -- (83.9000,150.4000) -- (83.9000,150.4000) -- (83.9000,150.4000) -- (83.9000,150.3000) -- (83.9000,150.3000) -- (83.9000,150.3000) -- (83.9000,150.3000) -- (83.9000,150.3000) -- (83.9000,150.3000) -- (83.9000,150.3000) -- (83.9000,150.3000) -- (83.9000,150.3000) -- (83.9000,150.3000) -- (83.9000,150.3000) -- (83.9000,150.3000) -- (83.9000,150.3000) -- (83.9000,150.3000) -- (83.9000,150.3000) -- (83.9000,150.3000) -- (83.9000,150.3000) -- (83.9000,150.3000) -- (83.9000,150.3000) -- (83.9000,150.3000) -- (83.9000,150.3000) -- (83.9000,150.2000) -- (83.9000,150.2000) -- (83.9000,150.2000) -- (83.9000,150.2000) -- (83.9000,150.2000) -- (83.9000,150.2000) -- (83.9000,150.2000) -- (83.9000,150.2000) -- (83.9000,150.2000) -- (83.9000,150.2000) -- (84.0000,150.2000) -- (84.0000,150.2000) -- (84.0000,150.2000) -- (84.0000,150.2000) -- (84.0000,150.2000) -- (84.0000,150.2000) -- (84.0000,150.2000) -- (84.0000,150.2000) -- (84.0000,150.2000) -- (84.0000,150.2000) -- (84.0000,150.2000) -- (84.0000,150.2000) -- (84.0000,150.1000) -- (84.0000,150.1000) -- (84.0000,150.1000) -- (84.0000,150.1000) -- (84.0000,150.1000) -- (84.0000,150.1000) -- (84.0000,150.1000) -- (84.0000,150.1000) -- (84.0000,150.1000) -- (84.0000,150.1000) -- (84.0000,150.1000) -- (84.0000,150.1000) -- (84.0000,150.1000) -- (84.0000,150.1000) -- (84.0000,150.1000) -- (84.0000,150.1000) -- (84.0000,150.1000) -- (84.0000,150.1000) -- (84.0000,150.1000) -- (84.0000,150.1000) -- (84.0000,150.1000) -- (84.0000,150.0000) -- (84.0000,150.0000) -- (84.0000,150.0000) -- (84.0000,150.0000) -- (84.0000,150.0000) -- (84.0000,150.0000) -- (84.0000,150.0000) -- (84.0000,150.0000) -- (84.0000,150.0000) -- (84.0000,150.0000) -- (84.0000,150.0000) -- (84.0000,150.0000) -- (84.0000,150.0000) -- (84.0000,150.0000) -- (84.0000,150.0000) -- (84.0000,150.0000) -- (84.1000,150.0000) -- (84.1000,150.0000) -- (84.1000,150.0000) -- (84.1000,150.0000) -- (84.1000,150.0000) -- (84.1000,150.0000) -- (84.1000,149.9000) -- (84.1000,149.9000) -- (84.1000,149.9000) -- (84.1000,149.9000) -- (84.1000,149.9000) -- (84.1000,149.9000) -- (84.1000,149.9000) -- (84.1000,149.9000) -- (84.1000,149.9000) -- (84.1000,149.9000) -- (84.1000,149.9000) -- (84.1000,149.9000) -- (84.1000,149.9000) -- (84.1000,149.9000) -- (84.1000,149.9000) -- (84.1000,149.9000) -- (84.1000,149.9000) -- (84.1000,149.9000) -- (84.1000,149.9000) -- (84.1000,149.9000) -- (84.1000,149.9000) -- (84.1000,149.8000) -- (84.1000,149.8000) -- (84.1000,149.8000) -- (84.1000,149.8000) -- (84.1000,149.8000) -- (84.1000,149.8000) -- (84.1000,149.8000) -- (84.1000,149.8000) -- (84.1000,149.8000) -- (84.1000,149.8000) -- (84.1000,149.8000) -- (84.1000,149.8000) -- (84.1000,149.8000) -- (84.1000,149.8000) -- (84.1000,149.8000) -- (84.1000,149.8000) -- (84.1000,149.8000) -- (84.1000,149.8000) -- (84.1000,149.8000) -- (84.1000,149.8000) -- (84.1000,149.8000) -- (84.1000,149.8000) -- (84.1000,149.7000) -- (84.2000,149.7000) -- (84.2000,149.7000) -- (84.2000,149.7000) -- (84.2000,149.7000) -- (84.2000,149.7000) -- (84.2000,149.7000) -- (84.2000,149.7000) -- (84.2000,149.7000) -- (84.2000,149.7000) -- (84.2000,149.7000) -- (84.2000,149.7000) -- (84.2000,149.7000) -- (84.2000,149.7000) -- (84.2000,149.7000) -- (84.2000,149.7000) -- (84.2000,149.7000) -- (84.2000,149.7000) -- (84.2000,149.7000) -- (84.2000,149.7000) -- (84.2000,149.7000) -- (84.2000,149.6000) -- (84.2000,149.6000) -- (84.2000,149.6000) -- (84.2000,149.6000) -- (84.2000,149.6000) -- (84.2000,149.6000) -- (84.2000,149.6000) -- (84.2000,149.6000) -- (84.2000,149.6000) -- (84.2000,149.6000) -- (84.2000,149.6000) -- (84.2000,149.6000) -- (84.2000,149.6000) -- (84.2000,149.6000) -- (84.2000,149.6000) -- (84.2000,149.6000) -- (84.2000,149.6000) -- (84.2000,149.6000) -- (84.2000,149.6000) -- (84.2000,149.6000) -- (84.2000,149.6000) -- (84.2000,149.6000) -- (84.2000,149.5000) -- (84.2000,149.5000) -- (84.2000,149.5000) -- (84.2000,149.5000) -- (84.2000,149.5000) -- (84.2000,149.5000) -- (84.2000,149.5000) -- (84.3000,149.5000) -- (84.3000,149.5000) -- (84.3000,149.5000) -- (84.3000,149.5000) -- (84.3000,149.5000) -- (84.3000,149.5000) -- (84.3000,149.5000) -- (84.3000,149.5000) -- (84.3000,149.5000) -- (84.3000,149.5000) -- (84.3000,149.5000) -- (84.3000,149.5000) -- (84.3000,149.5000) -- (84.3000,149.5000) -- (84.3000,149.4000) -- (84.3000,149.4000) -- (84.3000,149.4000) -- (84.3000,149.4000) -- (84.3000,149.4000) -- (84.3000,149.4000) -- (84.3000,149.4000) -- (84.3000,149.4000) -- (84.3000,149.4000) -- (84.3000,149.4000) -- (84.3000,149.4000) -- (84.3000,149.4000) -- (84.3000,149.4000) -- (84.3000,149.4000) -- (84.3000,149.4000) -- (84.3000,149.4000) -- (84.3000,149.4000) -- (84.3000,149.4000) -- (84.3000,149.4000) -- (84.3000,149.4000) -- (84.3000,149.4000) -- (84.3000,149.4000) -- (84.3000,149.3000) -- (84.3000,149.3000) -- (84.3000,149.3000) -- (84.3000,149.3000) -- (84.3000,149.3000) -- (84.3000,149.3000) -- (84.3000,149.3000) -- (84.3000,149.3000) -- (84.3000,149.3000) -- (84.3000,149.3000) -- (84.3000,149.3000) -- (84.3000,149.3000) -- (84.3000,149.3000) -- (84.3000,149.3000) -- (84.4000,149.3000) -- (84.4000,149.3000) -- (84.4000,149.3000) -- (84.4000,149.3000) -- (84.4000,149.3000) -- (84.4000,149.3000) -- (84.4000,149.3000) -- (84.4000,149.2000) -- (84.4000,149.2000) -- (84.4000,149.2000) -- (84.4000,149.2000) -- (84.4000,149.2000) -- (84.4000,149.2000) -- (84.4000,149.2000) -- (84.4000,149.2000) -- (84.4000,149.2000) -- (84.4000,149.2000) -- (84.4000,149.2000) -- (84.4000,149.2000) -- (84.4000,149.2000) -- (84.4000,149.2000) -- (84.4000,149.2000) -- (84.4000,149.2000) -- (84.4000,149.2000) -- (84.4000,149.2000) -- (84.4000,149.2000) -- (84.4000,149.2000) -- (84.4000,149.2000) -- (84.4000,149.2000) -- (84.4000,149.1000) -- (84.4000,149.1000) -- (84.4000,149.1000) -- (84.4000,149.1000) -- (84.4000,149.1000) -- (84.4000,149.1000) -- (84.4000,149.1000) -- (84.4000,149.1000) -- (84.4000,149.1000) -- (84.4000,149.1000) -- (84.4000,149.1000) -- (84.4000,149.1000) -- (84.4000,149.1000) -- (84.4000,149.1000) -- (84.4000,149.1000) -- (84.4000,149.1000) -- (84.4000,149.1000) -- (84.4000,149.1000) -- (84.4000,149.1000) -- (84.4000,149.1000) -- (84.5000,149.1000) -- (84.5000,149.0000) -- (84.5000,149.0000) -- (84.5000,149.0000) -- (84.5000,149.0000) -- (84.5000,149.0000) -- (84.5000,149.0000) -- (84.5000,149.0000) -- (84.5000,149.0000) -- (84.5000,149.0000) -- (84.5000,149.0000) -- (84.5000,149.0000) -- (84.5000,149.0000) -- (84.5000,149.0000) -- (84.5000,149.0000) -- (84.5000,149.0000) -- (84.5000,149.0000) -- (84.5000,149.0000) -- (84.5000,149.0000) -- (84.5000,149.0000) -- (84.5000,149.0000) -- (84.5000,149.0000) -- (84.5000,149.0000) -- (84.5000,148.9000) -- (84.5000,148.9000) -- (84.5000,148.9000) -- (84.5000,148.9000) -- (84.5000,148.9000) -- (84.5000,148.9000) -- (84.5000,148.9000) -- (84.5000,148.9000) -- (84.5000,148.9000) -- (84.5000,148.9000) -- (84.5000,148.9000) -- (84.5000,148.9000) -- (84.5000,148.9000) -- (84.5000,148.9000) -- (84.5000,148.9000) -- (84.5000,148.9000) -- (84.5000,148.9000) -- (84.5000,148.9000) -- (84.5000,148.9000) -- (84.5000,148.9000) -- (84.5000,148.9000) -- (84.5000,148.8000) -- (84.5000,148.8000) -- (84.5000,148.8000) -- (84.5000,148.8000) -- (84.5000,148.8000) -- (84.5000,148.8000) -- (84.6000,148.8000) -- (84.6000,148.8000) -- (84.6000,148.8000) -- (84.6000,148.8000) -- (84.6000,148.8000) -- (84.6000,148.8000) -- (84.6000,148.8000) -- (84.6000,148.8000) -- (84.6000,148.8000) -- (84.6000,148.8000) -- (84.6000,148.8000) -- (84.6000,148.8000) -- (84.6000,148.8000) -- (84.6000,148.8000) -- (84.6000,148.8000) -- (84.6000,148.8000) -- (84.6000,148.7000) -- (84.6000,148.7000) -- (84.6000,148.7000) -- (84.6000,148.7000) -- (84.6000,148.7000) -- (84.6000,148.7000) -- (84.6000,148.7000) -- (84.6000,148.7000) -- (84.6000,148.7000) -- (84.6000,148.7000) -- (84.6000,148.7000) -- (84.6000,148.7000) -- (84.6000,148.7000) -- (84.6000,148.7000) -- (84.6000,148.7000) -- (84.6000,148.7000) -- (84.6000,148.7000) -- (84.6000,148.7000) -- (84.6000,148.7000) -- (84.6000,148.7000) -- (84.6000,148.7000) -- (84.6000,148.6000) -- (84.6000,148.6000) -- (84.6000,148.6000) -- (84.6000,148.6000) -- (84.6000,148.6000) -- (84.6000,148.6000) -- (84.6000,148.6000) -- (84.6000,148.6000) -- (84.6000,148.6000) -- (84.6000,148.6000) -- (84.6000,148.6000) -- (84.6000,148.6000) -- (84.7000,148.6000) -- (84.7000,148.6000) -- (84.7000,148.6000) -- (84.7000,148.6000) -- (84.7000,148.6000) -- (84.7000,148.6000) -- (84.7000,148.6000) -- (84.7000,148.6000) -- (84.7000,148.6000) -- (84.7000,148.6000) -- (84.7000,148.5000) -- (84.7000,148.5000) -- (84.7000,148.5000) -- (84.7000,148.5000) -- (84.7000,148.5000) -- (84.7000,148.5000) -- (84.7000,148.5000) -- (84.7000,148.5000) -- (84.7000,148.5000) -- (84.7000,148.5000) -- (84.7000,148.5000) -- (84.7000,148.5000) -- (84.7000,148.5000) -- (84.7000,148.5000) -- (84.7000,148.5000) -- (84.7000,148.5000) -- (84.7000,148.5000) -- (84.7000,148.5000) -- (84.7000,148.5000) -- (84.7000,148.5000) -- (84.7000,148.5000) -- (84.7000,148.4000) -- (84.7000,148.4000) -- (84.7000,148.4000) -- (84.7000,148.4000) -- (84.7000,148.4000) -- (84.7000,148.4000) -- (84.7000,148.4000) -- (84.7000,148.4000) -- (84.7000,148.4000) -- (84.7000,148.4000) -- (84.7000,148.4000) -- (84.7000,148.4000) -- (84.7000,148.4000) -- (84.7000,148.4000) -- (84.7000,148.4000) -- (84.7000,148.4000) -- (84.7000,148.4000) -- (84.7000,148.4000) -- (84.7000,148.4000) -- (84.8000,148.4000) -- (84.8000,148.4000) -- (84.8000,148.4000) -- (84.8000,148.3000) -- (84.8000,148.3000) -- (84.8000,148.3000) -- (84.8000,148.3000) -- (84.8000,148.3000) -- (84.8000,148.3000) -- (84.8000,148.3000) -- (84.8000,148.3000) -- (84.8000,148.3000) -- (84.8000,148.3000) -- (84.8000,148.3000) -- (84.8000,148.3000) -- (84.8000,148.3000) -- (84.8000,148.3000) -- (84.8000,148.3000) -- (84.8000,148.3000) -- (84.8000,148.3000) -- (84.8000,148.3000) -- (84.8000,148.3000) -- (84.8000,148.3000) -- (84.8000,148.3000) -- (84.8000,148.2000) -- (84.8000,148.2000) -- (84.8000,148.2000) -- (84.8000,148.2000) -- (84.8000,148.2000) -- (84.8000,148.2000) -- (84.8000,148.2000) -- (84.8000,148.2000) -- (84.8000,148.2000) -- (84.8000,148.2000) -- (84.8000,148.2000) -- (84.8000,148.2000) -- (84.8000,148.2000) -- (84.8000,148.2000) -- (84.8000,148.2000) -- (84.8000,148.2000) -- (84.8000,148.2000) -- (84.8000,148.2000) -- (84.8000,148.2000) -- (84.8000,148.2000) -- (84.8000,148.2000) -- (84.8000,148.2000) -- (84.8000,148.1000) -- (84.8000,148.1000) -- (84.8000,148.1000) -- (84.9000,148.1000) -- (84.9000,148.1000) -- (84.9000,148.1000) -- (84.9000,148.1000) -- (84.9000,148.1000) -- (84.9000,148.1000) -- (84.9000,148.1000) -- (84.9000,148.1000) -- (84.9000,148.1000) -- (84.9000,148.1000) -- (84.9000,148.1000) -- (84.9000,148.1000) -- (84.9000,148.1000) -- (84.9000,148.1000) -- (84.9000,148.1000) -- (84.9000,148.1000) -- (84.9000,148.1000) -- (84.9000,148.1000) -- (84.9000,148.0000) -- (84.9000,148.0000) -- (84.9000,148.0000) -- (84.9000,148.0000) -- (84.9000,148.0000) -- (84.9000,148.0000) -- (84.9000,148.0000) -- (84.9000,148.0000) -- (84.9000,148.0000) -- (84.9000,148.0000) -- (84.9000,148.0000) -- (84.9000,148.0000) -- (84.9000,148.0000) -- (84.9000,148.0000) -- (84.9000,148.0000) -- (84.9000,148.0000) -- (84.9000,148.0000) -- (84.9000,148.0000) -- (84.9000,148.0000) -- (84.9000,148.0000) -- (84.9000,148.0000) -- (84.9000,148.0000) -- (84.9000,147.9000) -- (84.9000,147.9000) -- (84.9000,147.9000) -- (84.9000,147.9000) -- (84.9000,147.9000) -- (84.9000,147.9000) -- (84.9000,147.9000) -- (84.9000,147.9000) -- (84.9000,147.9000) -- (84.9000,147.9000) -- (85.0000,147.9000) -- (85.0000,147.9000) -- (85.0000,147.9000) -- (85.0000,147.9000) -- (85.0000,147.9000) -- (85.0000,147.9000) -- (85.0000,147.9000) -- (85.0000,147.9000) -- (85.0000,147.9000) -- (85.0000,147.9000) -- (85.0000,147.9000) -- (85.0000,147.8000) -- (85.0000,147.8000) -- (85.0000,147.8000) -- (85.0000,147.8000) -- (85.0000,147.8000) -- (85.0000,147.8000) -- (85.0000,147.8000) -- (85.0000,147.8000) -- (85.0000,147.8000) -- (85.0000,147.8000) -- (85.0000,147.8000) -- (85.0000,147.8000) -- (85.0000,147.8000) -- (85.0000,147.8000) -- (85.0000,147.8000) -- (85.0000,147.8000) -- (85.0000,147.8000) -- (85.0000,147.8000) -- (85.0000,147.8000) -- (85.0000,147.8000) -- (85.0000,147.8000) -- (85.0000,147.8000) -- (85.0000,147.7000) -- (85.0000,147.7000) -- (85.0000,147.7000) -- (85.0000,147.7000) -- (85.0000,147.7000) -- (85.0000,147.7000) -- (85.0000,147.7000) -- (85.0000,147.7000) -- (85.0000,147.7000) -- (85.0000,147.7000) -- (85.0000,147.7000) -- (85.0000,147.7000) -- (85.0000,147.7000) -- (85.0000,147.7000) -- (85.0000,147.7000) -- (85.0000,147.7000) -- (85.1000,147.7000) -- (85.1000,147.7000) -- (85.1000,147.7000) -- (85.1000,147.7000) -- (85.1000,147.7000) -- (85.1000,147.6000) -- (85.1000,147.6000) -- (85.1000,147.6000) -- (85.1000,147.6000) -- (85.1000,147.6000) -- (85.1000,147.6000) -- (85.1000,147.6000) -- (85.1000,147.6000) -- (85.1000,147.6000) -- (85.1000,147.6000) -- (85.1000,147.6000) -- (85.1000,147.6000) -- (85.1000,147.6000) -- (85.1000,147.6000) -- (85.1000,147.6000) -- (85.1000,147.6000) -- (85.1000,147.6000) -- (85.1000,147.6000) -- (85.1000,147.6000) -- (85.1000,147.6000) -- (85.1000,147.6000) -- (85.1000,147.6000) -- (85.1000,147.5000) -- (85.1000,147.5000) -- (85.1000,147.5000) -- (85.1000,147.5000) -- (85.1000,147.5000) -- (85.1000,147.5000) -- (85.1000,147.5000) -- (85.1000,147.5000) -- (85.1000,147.5000) -- (85.1000,147.5000) -- (85.1000,147.5000) -- (85.1000,147.5000) -- (85.1000,147.5000) -- (85.1000,147.5000) -- (85.1000,147.5000) -- (85.1000,147.5000) -- (85.1000,147.5000) -- (85.1000,147.5000) -- (85.1000,147.5000) -- (85.1000,147.5000) -- (85.1000,147.5000) -- (85.1000,147.4000) -- (85.1000,147.4000) -- (85.2000,147.4000) -- (85.2000,147.4000) -- (85.2000,147.4000) -- (85.2000,147.4000) -- (85.2000,147.4000) -- (85.2000,147.4000) -- (85.2000,147.4000) -- (85.2000,147.4000) -- (85.2000,147.4000) -- (85.2000,147.4000) -- (85.2000,147.4000) -- (85.2000,147.4000) -- (85.2000,147.4000) -- (85.2000,147.4000) -- (85.2000,147.4000) -- (85.2000,147.4000) -- (85.2000,147.4000) -- (85.2000,147.4000) -- (85.2000,147.4000) -- (85.2000,147.4000) -- (85.2000,147.3000) -- (85.2000,147.3000) -- (85.2000,147.3000) -- (85.2000,147.3000) -- (85.2000,147.3000) -- (85.2000,147.3000) -- (85.2000,147.3000) -- (85.2000,147.3000) -- (85.2000,147.3000) -- (85.2000,147.3000) -- (85.2000,147.3000) -- (85.2000,147.3000) -- (85.2000,147.3000) -- (85.2000,147.3000) -- (85.2000,147.3000) -- (85.2000,147.3000) -- (85.2000,147.3000) -- (85.2000,147.3000) -- (85.2000,147.3000) -- (85.2000,147.3000) -- (85.2000,147.3000) -- (85.2000,147.2000) -- (85.2000,147.2000) -- (85.2000,147.2000) -- (85.2000,147.2000) -- (85.2000,147.2000) -- (85.2000,147.2000) -- (85.2000,147.2000) -- (85.2000,147.2000) -- (85.3000,147.2000) -- (85.3000,147.2000) -- (85.3000,147.2000) -- (85.3000,147.2000) -- (85.3000,147.2000) -- (85.3000,147.2000) -- (85.3000,147.2000) -- (85.3000,147.2000) -- (85.3000,147.2000) -- (85.3000,147.2000) -- (85.3000,147.2000) -- (85.3000,147.2000) -- (85.3000,147.2000) -- (85.3000,147.2000) -- (85.3000,147.1000) -- (85.3000,147.1000) -- (85.3000,147.1000) -- (85.3000,147.1000) -- (85.3000,147.1000) -- (85.3000,147.1000) -- (85.3000,147.1000) -- (85.3000,147.1000) -- (85.3000,147.1000) -- (85.3000,147.1000) -- (85.3000,147.1000) -- (85.3000,147.1000) -- (85.3000,147.1000) -- (85.3000,147.1000) -- (85.3000,147.1000) -- (85.3000,147.1000) -- (85.3000,147.1000) -- (85.3000,147.1000) -- (85.3000,147.1000) -- (85.3000,147.1000) -- (85.3000,147.1000) -- (85.3000,147.0000) -- (85.3000,147.0000) -- (85.3000,147.0000) -- (85.3000,147.0000) -- (85.3000,147.0000) -- (85.3000,147.0000) -- (85.3000,147.0000) -- (85.3000,147.0000) -- (85.3000,147.0000) -- (85.3000,147.0000) -- (85.3000,147.0000) -- (85.3000,147.0000) -- (85.3000,147.0000) -- (85.3000,147.0000) -- (85.3000,147.0000) -- (85.4000,147.0000) -- (85.4000,147.0000) -- (85.4000,147.0000) -- (85.4000,147.0000) -- (85.4000,147.0000) -- (85.4000,147.0000) -- (85.4000,147.0000) -- (85.4000,146.9000) -- (85.4000,146.9000) -- (85.4000,146.9000) -- (85.4000,146.9000) -- (85.4000,146.9000) -- (85.4000,146.9000) -- (85.4000,146.9000) -- (85.4000,146.9000) -- (85.4000,146.9000) -- (85.4000,146.9000) -- (85.4000,146.9000) -- (85.4000,146.9000) -- (85.4000,146.9000) -- (85.4000,146.9000) -- (85.4000,146.9000) -- (85.4000,146.9000) -- (85.4000,146.9000) -- (85.4000,146.9000) -- (85.4000,146.9000) -- (85.4000,146.9000) -- (85.4000,146.9000) -- (85.4000,146.8000) -- (85.4000,146.8000) -- (85.4000,146.8000) -- (85.4000,146.8000) -- (85.4000,146.8000) -- (85.4000,146.8000) -- (85.4000,146.8000) -- (85.4000,146.8000) -- (85.4000,146.8000) -- (85.4000,146.8000) -- (85.4000,146.8000) -- (85.4000,146.8000) -- (85.4000,146.8000) -- (85.4000,146.8000) -- (85.4000,146.8000) -- (85.4000,146.8000) -- (85.4000,146.8000) -- (85.4000,146.8000) -- (85.4000,146.8000) -- (85.4000,146.8000) -- (85.4000,146.8000) -- (85.5000,146.8000) -- (85.5000,146.7000) -- (85.5000,146.7000) -- (85.5000,146.7000) -- (85.5000,146.7000) -- (85.5000,146.7000) -- (85.5000,146.7000) -- (85.5000,146.7000) -- (85.5000,146.7000) -- (85.5000,146.7000) -- (85.5000,146.7000) -- (85.5000,146.7000) -- (85.5000,146.7000) -- (85.5000,146.7000) -- (85.5000,146.7000) -- (85.5000,146.7000) -- (85.5000,146.7000) -- (85.5000,146.7000) -- (85.5000,146.7000) -- (85.5000,146.7000) -- (85.5000,146.7000) -- (85.5000,146.7000) -- (85.5000,146.6000) -- (85.5000,146.6000) -- (85.5000,146.6000) -- (85.5000,146.6000) -- (85.5000,146.6000) -- (85.5000,146.6000) -- (85.5000,146.6000) -- (85.5000,146.6000) -- (85.5000,146.6000) -- (85.5000,146.6000) -- (85.5000,146.6000) -- (85.5000,146.6000) -- (85.5000,146.6000) -- (85.5000,146.6000) -- (85.5000,146.6000) -- (85.5000,146.6000) -- (85.5000,146.6000) -- (85.5000,146.6000) -- (85.5000,146.6000) -- (85.5000,146.6000) -- (85.5000,146.6000) -- (85.5000,146.6000) -- (85.5000,146.5000) -- (85.5000,146.5000) -- (85.5000,146.5000) -- (85.5000,146.5000) -- (85.5000,146.5000) -- (85.5000,146.5000) -- (85.6000,146.5000) -- (85.6000,146.5000) -- (85.6000,146.5000) -- (85.6000,146.5000) -- (85.6000,146.5000) -- (85.6000,146.5000) -- (85.6000,146.5000) -- (85.6000,146.5000) -- (85.6000,146.5000) -- (85.6000,146.5000) -- (85.6000,146.5000) -- (85.6000,146.5000) -- (85.6000,146.5000) -- (85.6000,146.5000) -- (85.6000,146.5000) -- (85.6000,146.4000) -- (85.6000,146.4000) -- (85.6000,146.4000) -- (85.6000,146.4000) -- (85.6000,146.4000) -- (85.6000,146.4000) -- (85.6000,146.4000) -- (85.6000,146.4000) -- (85.6000,146.4000) -- (85.6000,146.4000) -- (85.6000,146.4000) -- (85.6000,146.4000) -- (85.6000,146.4000) -- (85.6000,146.4000) -- (85.6000,146.4000) -- (85.6000,146.4000) -- (85.6000,146.4000) -- (85.6000,146.4000) -- (85.6000,146.4000) -- (85.6000,146.4000) -- (85.6000,146.4000) -- (85.6000,146.4000) -- (85.6000,146.3000) -- (85.6000,146.3000) -- (85.6000,146.3000) -- (85.6000,146.3000) -- (85.6000,146.3000) -- (85.6000,146.3000) -- (85.6000,146.3000) -- (85.6000,146.3000) -- (85.6000,146.3000) -- (85.6000,146.3000) -- (85.6000,146.3000) -- (85.6000,146.3000) -- (85.7000,146.3000) -- (85.7000,146.3000) -- (85.7000,146.3000) -- (85.7000,146.3000) -- (85.7000,146.3000) -- (85.7000,146.3000) -- (85.7000,146.3000) -- (85.7000,146.3000) -- (85.7000,146.3000) -- (85.7000,146.2000) -- (85.7000,146.2000) -- (85.7000,146.2000) -- (85.7000,146.2000) -- (85.7000,146.2000) -- (85.7000,146.2000) -- (85.7000,146.2000) -- (85.7000,146.2000) -- (85.7000,146.2000) -- (85.7000,146.2000) -- (85.7000,146.2000) -- (85.7000,146.2000) -- (85.7000,146.2000) -- (85.7000,146.2000) -- (85.7000,146.2000) -- (85.7000,146.2000) -- (85.7000,146.2000) -- (85.7000,146.2000) -- (85.7000,146.2000) -- (85.7000,146.2000) -- (85.7000,146.2000) -- (85.7000,146.2000) -- (85.7000,146.1000) -- (85.7000,146.1000) -- (85.7000,146.1000) -- (85.7000,146.1000) -- (85.7000,146.1000) -- (85.7000,146.1000) -- (85.7000,146.1000) -- (85.7000,146.1000) -- (85.7000,146.1000) -- (85.7000,146.1000) -- (85.7000,146.1000) -- (85.7000,146.1000) -- (85.7000,146.1000) -- (85.7000,146.1000) -- (85.7000,146.1000) -- (85.7000,146.1000) -- (85.7000,146.1000) -- (85.7000,146.1000) -- (85.7000,146.1000) -- (85.8000,146.1000) -- (85.8000,146.1000) -- (85.8000,146.0000) -- (85.8000,146.0000) -- (85.8000,146.0000) -- (85.8000,146.0000) -- (85.8000,146.0000) -- (85.8000,146.0000) -- (85.8000,146.0000) -- (85.8000,146.0000) -- (85.8000,146.0000) -- (85.8000,146.0000) -- (85.8000,146.0000) -- (85.8000,146.0000) -- (85.8000,146.0000) -- (85.8000,146.0000) -- (85.8000,146.0000) -- (85.8000,146.0000) -- (85.8000,146.0000) -- (85.8000,146.0000) -- (85.8000,146.0000) -- (85.8000,146.0000) -- (85.8000,146.0000) -- (85.8000,146.0000) -- (85.8000,145.9000) -- (85.8000,145.9000) -- (85.8000,145.9000) -- (85.8000,145.9000) -- (85.8000,145.9000) -- (85.8000,145.9000) -- (85.8000,145.9000) -- (85.8000,145.9000) -- (85.8000,145.9000) -- (85.8000,145.9000) -- (85.8000,145.9000) -- (85.8000,145.9000) -- (85.8000,145.9000) -- (85.8000,145.9000) -- (85.8000,145.9000) -- (85.8000,145.9000) -- (85.8000,145.9000) -- (85.8000,145.9000) -- (85.8000,145.9000) -- (85.8000,145.9000) -- (85.8000,145.9000) -- (85.8000,145.8000) -- (85.8000,145.8000) -- (85.8000,145.8000) -- (85.8000,145.8000) -- (85.9000,145.8000) -- (85.9000,145.8000) -- (85.9000,145.8000) -- (85.9000,145.8000) -- (85.9000,145.8000) -- (85.9000,145.8000) -- (85.9000,145.8000) -- (85.9000,145.8000) -- (85.9000,145.8000) -- (85.9000,145.8000) -- (85.9000,145.8000) -- (85.9000,145.8000) -- (85.9000,145.8000) -- (85.9000,145.8000) -- (85.9000,145.8000) -- (85.9000,145.8000) -- (85.9000,145.8000) -- (85.9000,145.8000) -- (85.9000,145.7000) -- (85.9000,145.7000) -- (85.9000,145.7000) -- (85.9000,145.7000) -- (85.9000,145.7000) -- (85.9000,145.7000) -- (85.9000,145.7000) -- (85.9000,145.7000) -- (85.9000,145.7000) -- (85.9000,145.7000) -- (85.9000,145.7000) -- (85.9000,145.7000) -- (85.9000,145.7000) -- (85.9000,145.7000) -- (85.9000,145.7000) -- (85.9000,145.7000) -- (85.9000,145.7000) -- (85.9000,145.7000) -- (85.9000,145.7000) -- (85.9000,145.7000) -- (85.9000,145.7000) -- (85.9000,145.6000) -- (85.9000,145.6000) -- (85.9000,145.6000) -- (85.9000,145.6000) -- (85.9000,145.6000) -- (85.9000,145.6000) -- (85.9000,145.6000) -- (85.9000,145.6000) -- (85.9000,145.6000) -- (85.9000,145.6000) -- (85.9000,145.6000) -- (86.0000,145.6000) -- (86.0000,145.6000) -- (86.0000,145.6000) -- (86.0000,145.6000) -- (86.0000,145.6000) -- (86.0000,145.6000) -- (86.0000,145.6000) -- (86.0000,145.6000) -- (86.0000,145.6000) -- (86.0000,145.6000) -- (86.0000,145.6000) -- (86.0000,145.5000) -- (86.0000,145.5000) -- (86.0000,145.5000) -- (86.0000,145.5000) -- (86.0000,145.5000) -- (86.0000,145.5000) -- (86.0000,145.5000) -- (86.0000,145.5000) -- (86.0000,145.5000) -- (86.0000,145.5000) -- (86.0000,145.5000) -- (86.0000,145.5000) -- (86.0000,145.5000) -- (86.0000,145.5000) -- (86.0000,145.5000) -- (86.0000,145.5000) -- (86.0000,145.5000) -- (86.0000,145.5000) -- (86.0000,145.5000) -- (86.0000,145.5000) -- (86.0000,145.5000) -- (86.0000,145.4000) -- (86.0000,145.4000) -- (86.0000,145.4000) -- (86.0000,145.4000) -- (86.0000,145.4000) -- (86.0000,145.4000) -- (86.0000,145.4000) -- (86.0000,145.4000) -- (86.0000,145.4000) -- (86.0000,145.4000) -- (86.0000,145.4000) -- (86.0000,145.4000) -- (86.0000,145.4000) -- (86.0000,145.4000) -- (86.0000,145.4000) -- (86.0000,145.4000) -- (86.0000,145.4000) -- (86.1000,145.4000) -- (86.1000,145.4000) -- (86.1000,145.4000) -- (86.1000,145.4000) -- (86.1000,145.4000) -- (86.1000,145.3000) -- (86.1000,145.3000) -- (86.1000,145.3000) -- (86.1000,145.3000) -- (86.1000,145.3000) -- (86.1000,145.3000) -- (86.1000,145.3000) -- (86.1000,145.3000) -- (86.1000,145.3000) -- (86.1000,145.3000) -- (86.1000,145.3000) -- (86.1000,145.3000) -- (86.1000,145.3000) -- (86.1000,145.3000) -- (86.1000,145.3000) -- (86.1000,145.3000) -- (86.1000,145.3000) -- (86.1000,145.3000) -- (86.1000,145.3000) -- (86.1000,145.3000) -- (86.1000,145.3000) -- (86.1000,145.2000) -- (86.1000,145.2000) -- (86.1000,145.2000) -- (86.1000,145.2000) -- (86.1000,145.2000) -- (86.1000,145.2000) -- (86.1000,145.2000) -- (86.1000,145.2000) -- (86.1000,145.2000) -- (86.1000,145.2000) -- (86.1000,145.2000) -- (86.1000,145.2000) -- (86.1000,145.2000) -- (86.1000,145.2000) -- (86.1000,145.2000) -- (86.1000,145.2000) -- (86.1000,145.2000) -- (86.1000,145.2000) -- (86.1000,145.2000) -- (86.1000,145.2000) -- (86.1000,145.2000) -- (86.1000,145.2000) -- (86.1000,145.1000) -- (86.1000,145.1000) -- (86.2000,145.1000) -- (86.2000,145.1000) -- (86.2000,145.1000) -- (86.2000,145.1000) -- (86.2000,145.1000) -- (86.2000,145.1000) -- (86.2000,145.1000) -- (86.2000,145.1000) -- (86.2000,145.1000) -- (86.2000,145.1000) -- (86.2000,145.1000) -- (86.2000,145.1000) -- (86.2000,145.1000) -- (86.2000,145.1000) -- (86.2000,145.1000) -- (86.2000,145.1000) -- (86.2000,145.1000) -- (86.2000,145.1000) -- (86.2000,145.1000) -- (86.2000,145.0000) -- (86.2000,145.0000) -- (86.2000,145.0000) -- (86.2000,145.0000) -- (86.2000,145.0000) -- (86.2000,145.0000) -- (86.2000,145.0000) -- (86.2000,145.0000) -- (86.2000,145.0000) -- (86.2000,145.0000) -- (86.2000,145.0000) -- (86.2000,145.0000) -- (86.2000,145.0000) -- (86.2000,145.0000) -- (86.2000,145.0000) -- (86.2000,145.0000) -- (86.2000,145.0000) -- (86.2000,145.0000) -- (86.2000,145.0000) -- (86.2000,145.0000) -- (86.2000,145.0000) -- (86.2000,145.0000) -- (86.2000,144.9000) -- (86.2000,144.9000) -- (86.2000,144.9000) -- (86.2000,144.9000) -- (86.2000,144.9000) -- (86.2000,144.9000) -- (86.2000,144.9000) -- (86.2000,144.9000) -- (86.2000,144.9000) -- (86.3000,144.9000) -- (86.3000,144.9000) -- (86.3000,144.9000) -- (86.3000,144.9000) -- (86.3000,144.9000) -- (86.3000,144.9000) -- (86.3000,144.9000) -- (86.3000,144.9000) -- (86.3000,144.9000) -- (86.3000,144.9000) -- (86.3000,144.9000) -- (86.3000,144.9000) -- (86.3000,144.8000) -- (86.3000,144.8000) -- (86.3000,144.8000) -- (86.3000,144.8000) -- (86.3000,144.8000) -- (86.3000,144.8000) -- (86.3000,144.8000) -- (86.3000,144.8000) -- (86.3000,144.8000) -- (86.3000,144.8000) -- (86.3000,144.8000) -- (86.3000,144.8000) -- (86.3000,144.8000) -- (86.3000,144.8000) -- (86.3000,144.8000) -- (86.3000,144.8000) -- (86.3000,144.8000) -- (86.3000,144.8000) -- (86.3000,144.8000) -- (86.3000,144.8000) -- (86.3000,144.8000) -- (86.3000,144.8000) -- (86.3000,144.7000) -- (86.3000,144.7000) -- (86.3000,144.7000) -- (86.3000,144.7000) -- (86.3000,144.7000) -- (86.3000,144.7000) -- (86.3000,144.7000) -- (86.3000,144.7000) -- (86.3000,144.7000) -- (86.3000,144.7000) -- (86.3000,144.7000) -- (86.3000,144.7000) -- (86.3000,144.7000) -- (86.3000,144.7000) -- (86.3000,144.7000) -- (86.4000,144.7000) -- (86.4000,144.7000) -- (86.4000,144.7000) -- (86.4000,144.7000) -- (86.4000,144.7000) -- (86.4000,144.7000) -- (86.4000,144.6000) -- (86.4000,144.6000) -- (86.4000,144.6000) -- (86.4000,144.6000) -- (86.4000,144.6000) -- (86.4000,144.6000) -- (86.4000,144.6000) -- (86.4000,144.6000) -- (86.4000,144.6000) -- (86.4000,144.6000) -- (86.4000,144.6000) -- (86.4000,144.6000) -- (86.4000,144.6000) -- (86.4000,144.6000) -- (86.4000,144.6000) -- (86.4000,144.6000) -- (86.4000,144.6000) -- (86.4000,144.6000) -- (86.4000,144.6000) -- (86.4000,144.6000) -- (86.4000,144.6000) -- (86.4000,144.6000) -- (86.4000,144.5000) -- (86.4000,144.5000) -- (86.4000,144.5000) -- (86.4000,144.5000) -- (86.4000,144.5000) -- (86.4000,144.5000) -- (86.4000,144.5000) -- (86.4000,144.5000) -- (86.4000,144.5000) -- (86.4000,144.5000) -- (86.4000,144.5000) -- (86.4000,144.5000) -- (86.4000,144.5000) -- (86.4000,144.5000) -- (86.4000,144.5000) -- (86.4000,144.5000) -- (86.4000,144.5000) -- (86.4000,144.5000) -- (86.4000,144.5000) -- (86.4000,144.5000) -- (86.4000,144.5000) -- (86.4000,144.4000) -- (86.5000,144.4000) -- (86.5000,144.4000) -- (86.5000,144.4000) -- (86.5000,144.4000) -- (86.5000,144.4000) -- (86.5000,144.4000) -- (86.5000,144.4000) -- (86.5000,144.4000) -- (86.5000,144.4000) -- (86.5000,144.4000) -- (86.5000,144.4000) -- (86.5000,144.4000) -- (86.5000,144.4000) -- (86.5000,144.4000) -- (86.5000,144.4000) -- (86.5000,144.4000) -- (86.5000,144.4000) -- (86.5000,144.4000) -- (86.5000,144.4000) -- (86.5000,144.4000) -- (86.5000,144.4000) -- (86.5000,144.3000) -- (86.5000,144.3000) -- (86.5000,144.3000) -- (86.5000,144.3000) -- (86.5000,144.3000) -- (86.5000,144.3000) -- (86.5000,144.3000) -- (86.5000,144.3000) -- (86.5000,144.3000) -- (86.5000,144.3000) -- (86.5000,144.3000) -- (86.5000,144.3000) -- (86.5000,144.3000) -- (86.5000,144.3000) -- (86.5000,144.3000) -- (86.5000,144.3000) -- (86.5000,144.3000) -- (86.5000,144.3000) -- (86.5000,144.3000) -- (86.5000,144.3000) -- (86.5000,144.3000) -- (86.5000,144.2000) -- (86.5000,144.2000) -- (86.5000,144.2000) -- (86.5000,144.2000) -- (86.5000,144.2000) -- (86.5000,144.2000) -- (86.5000,144.2000) -- (86.6000,144.2000) -- (86.6000,144.2000) -- (86.6000,144.2000) -- (86.6000,144.2000) -- (86.6000,144.2000) -- (86.6000,144.2000) -- (86.6000,144.2000) -- (86.6000,144.2000) -- (86.6000,144.2000) -- (86.6000,144.2000) -- (86.6000,144.2000) -- (86.6000,144.2000) -- (86.6000,144.2000) -- (86.6000,144.2000) -- (86.6000,144.2000) -- (86.6000,144.1000) -- (86.6000,144.1000) -- (86.6000,144.1000) -- (86.6000,144.1000) -- (86.6000,144.1000) -- (86.6000,144.1000) -- (86.6000,144.1000) -- (86.6000,144.1000) -- (86.6000,144.1000) -- (86.6000,144.1000) -- (86.6000,144.1000) -- (86.6000,144.1000) -- (86.6000,144.1000) -- (86.6000,144.1000) -- (86.6000,144.1000) -- (86.6000,144.1000) -- (86.6000,144.1000) -- (86.6000,144.1000) -- (86.6000,144.1000) -- (86.6000,144.1000) -- (86.6000,144.1000) -- (86.6000,144.0000) -- (86.6000,144.0000) -- (86.6000,144.0000) -- (86.6000,144.0000) -- (86.6000,144.0000) -- (86.6000,144.0000) -- (86.6000,144.0000) -- (86.6000,144.0000) -- (86.6000,144.0000) -- (86.6000,144.0000) -- (86.6000,144.0000) -- (86.6000,144.0000) -- (86.6000,144.0000) -- (86.6000,144.0000) -- (86.7000,144.0000) -- (86.7000,144.0000) -- (86.7000,144.0000) -- (86.7000,144.0000) -- (86.7000,144.0000) -- (86.7000,144.0000) -- (86.7000,144.0000) -- (86.7000,144.0000) -- (86.7000,143.9000) -- (86.7000,143.9000) -- (86.7000,143.9000) -- (86.7000,143.9000) -- (86.7000,143.9000) -- (86.7000,143.9000) -- (86.7000,143.9000) -- (86.7000,143.9000) -- (86.7000,143.9000) -- (86.7000,143.9000) -- (86.7000,143.9000) -- (86.7000,143.9000) -- (86.7000,143.9000) -- (86.7000,143.9000) -- (86.7000,143.9000) -- (86.7000,143.9000) -- (86.7000,143.9000) -- (86.7000,143.9000) -- (86.7000,143.9000) -- (86.7000,143.9000) -- (86.7000,143.9000) -- (86.7000,143.8000) -- (86.7000,143.8000) -- (86.7000,143.8000) -- (86.7000,143.8000) -- (86.7000,143.8000) -- (86.7000,143.8000) -- (86.7000,143.8000) -- (86.7000,143.8000) -- (86.7000,143.8000) -- (86.7000,143.8000) -- (86.7000,143.8000) -- (86.7000,143.8000) -- (86.7000,143.8000) -- (86.7000,143.8000) -- (86.7000,143.8000) -- (86.7000,143.8000) -- (86.7000,143.8000) -- (86.7000,143.8000) -- (86.7000,143.8000) -- (86.7000,143.8000) -- (86.8000,143.8000) -- (86.8000,143.8000) -- (86.8000,143.7000) -- (86.8000,143.7000) -- (86.8000,143.7000) -- (86.8000,143.7000) -- (86.8000,143.7000) -- (86.8000,143.7000) -- (86.8000,143.7000) -- (86.8000,143.7000) -- (86.8000,143.7000) -- (86.8000,143.7000) -- (86.8000,143.7000) -- (86.8000,143.7000) -- (86.8000,143.7000) -- (86.8000,143.7000) -- (86.8000,143.7000) -- (86.8000,143.7000) -- (86.8000,143.7000) -- (86.8000,143.7000) -- (86.8000,143.7000) -- (86.8000,143.7000) -- (86.8000,143.7000) -- (86.8000,143.6000) -- (86.8000,143.6000) -- (86.8000,143.6000) -- (86.8000,143.6000) -- (86.8000,143.6000) -- (86.8000,143.6000) -- (86.8000,143.6000) -- (86.8000,143.6000) -- (86.8000,143.6000) -- (86.8000,143.6000) -- (86.8000,143.6000) -- (86.8000,143.6000) -- (86.8000,143.6000) -- (86.8000,143.6000) -- (86.8000,143.6000) -- (86.8000,143.6000) -- (86.8000,143.6000) -- (86.8000,143.6000) -- (86.8000,143.6000) -- (86.8000,143.6000) -- (86.8000,143.6000) -- (86.8000,143.6000) -- (86.8000,143.5000) -- (86.8000,143.5000) -- (86.8000,143.5000) -- (86.8000,143.5000) -- (86.8000,143.5000) -- (86.9000,143.5000) -- (86.9000,143.5000) -- (86.9000,143.5000) -- (86.9000,143.5000) -- (86.9000,143.5000) -- (86.9000,143.5000) -- (86.9000,143.5000) -- (86.9000,143.5000) -- (86.9000,143.5000) -- (86.9000,143.5000) -- (86.9000,143.5000) -- (86.9000,143.5000) -- (86.9000,143.5000) -- (86.9000,143.5000) -- (86.9000,143.5000) -- (86.9000,143.5000) -- (86.9000,143.4000) -- (86.9000,143.4000) -- (86.9000,143.4000) -- (86.9000,143.4000) -- (86.9000,143.4000) -- (86.9000,143.4000) -- (86.9000,143.4000) -- (86.9000,143.4000) -- (86.9000,143.4000) -- (86.9000,143.4000) -- (86.9000,143.4000) -- (86.9000,143.4000) -- (86.9000,143.4000) -- (86.9000,143.4000) -- (86.9000,143.4000) -- (86.9000,143.4000) -- (86.9000,143.4000) -- (86.9000,143.4000) -- (86.9000,143.4000) -- (86.9000,143.4000) -- (86.9000,143.4000) -- (86.9000,143.4000) -- (86.9000,143.3000) -- (86.9000,143.3000) -- (86.9000,143.3000) -- (86.9000,143.3000) -- (86.9000,143.3000) -- (86.9000,143.3000) -- (86.9000,143.3000) -- (86.9000,143.3000) -- (86.9000,143.3000) -- (86.9000,143.3000) -- (86.9000,143.3000) -- (87.0000,143.3000) -- (87.0000,143.3000) -- (87.0000,143.3000) -- (87.0000,143.3000) -- (87.0000,143.3000) -- (87.0000,143.3000) -- (87.0000,143.3000) -- (87.0000,143.3000) -- (87.0000,143.3000) -- (87.0000,143.3000) -- (87.0000,143.2000) -- (87.0000,143.2000) -- (87.0000,143.2000) -- (87.0000,143.2000) -- (87.0000,143.2000) -- (87.0000,143.2000) -- (87.0000,143.2000) -- (87.0000,143.2000) -- (87.0000,143.2000) -- (87.0000,143.2000) -- (87.0000,143.2000) -- (87.0000,143.2000) -- (87.0000,143.2000) -- (87.0000,143.2000) -- (87.0000,143.2000) -- (87.0000,143.2000) -- (87.0000,143.2000) -- (87.0000,143.2000) -- (87.0000,143.2000) -- (87.0000,143.2000) -- (87.0000,143.2000) -- (87.0000,143.2000) -- (87.0000,143.1000) -- (87.0000,143.1000) -- (87.0000,143.1000) -- (87.0000,143.1000) -- (87.0000,143.1000) -- (87.0000,143.1000) -- (87.0000,143.1000) -- (87.0000,143.1000) -- (87.0000,143.1000) -- (87.0000,143.1000) -- (87.0000,143.1000) -- (87.0000,143.1000) -- (87.0000,143.1000) -- (87.0000,143.1000) -- (87.0000,143.1000) -- (87.0000,143.1000) -- (87.0000,143.1000) -- (87.0000,143.1000) -- (87.1000,143.1000) -- (87.1000,143.1000) -- (87.1000,143.1000) -- (87.1000,143.0000) -- (87.1000,143.0000) -- (87.1000,143.0000) -- (87.1000,143.0000) -- (87.1000,143.0000) -- (87.1000,143.0000) -- (87.1000,143.0000) -- (87.1000,143.0000) -- (87.1000,143.0000) -- (87.1000,143.0000) -- (87.1000,143.0000) -- (87.1000,143.0000) -- (87.1000,143.0000) -- (87.1000,143.0000) -- (87.1000,143.0000) -- (87.1000,143.0000) -- (87.1000,143.0000) -- (87.1000,143.0000) -- (87.1000,143.0000) -- (87.1000,143.0000) -- (87.1000,143.0000) -- (87.1000,143.0000) -- (87.1000,142.9000) -- (87.1000,142.9000) -- (87.1000,142.9000) -- (87.1000,142.9000) -- (87.1000,142.9000) -- (87.1000,142.9000) -- (87.1000,142.9000) -- (87.1000,142.9000) -- (87.1000,142.9000) -- (87.1000,142.9000) -- (87.1000,142.9000) -- (87.1000,142.9000) -- (87.1000,142.9000) -- (87.1000,142.9000) -- (87.1000,142.9000) -- (87.1000,142.9000) -- (87.1000,142.9000) -- (87.1000,142.9000) -- (87.1000,142.9000) -- (87.1000,142.9000) -- (87.1000,142.9000) -- (87.1000,142.9000) -- (87.1000,142.8000) -- (87.1000,142.8000) -- (87.2000,142.8000) -- (87.2000,142.8000) -- (87.2000,142.8000) -- (87.2000,142.8000) -- (87.2000,142.8000) -- (87.2000,142.8000) -- (87.2000,142.8000) -- (87.2000,142.8000) -- (87.2000,142.8000) -- (87.2000,142.8000) -- (87.2000,142.8000) -- (87.2000,142.8000) -- (87.2000,142.8000) -- (87.2000,142.8000) -- (87.2000,142.8000) -- (87.2000,142.8000) -- (87.2000,142.8000) -- (87.2000,142.8000) -- (87.2000,142.8000) -- (87.2000,142.7000) -- (87.2000,142.7000) -- (87.2000,142.7000) -- (87.2000,142.7000) -- (87.2000,142.7000) -- (87.2000,142.7000) -- (87.2000,142.7000) -- (87.2000,142.7000) -- (87.2000,142.7000) -- (87.2000,142.7000) -- (87.2000,142.7000) -- (87.2000,142.7000) -- (87.2000,142.7000) -- (87.2000,142.7000) -- (87.2000,142.7000) -- (87.2000,142.7000) -- (87.2000,142.7000) -- (87.2000,142.7000) -- (87.2000,142.7000) -- (87.2000,142.7000) -- (87.2000,142.7000) -- (87.2000,142.7000) -- (87.2000,142.6000) -- (87.2000,142.6000) -- (87.2000,142.6000) -- (87.2000,142.6000) -- (87.2000,142.6000) -- (87.2000,142.6000) -- (87.2000,142.6000) -- (87.2000,142.6000) -- (87.2000,142.6000) -- (87.3000,142.6000) -- (87.3000,142.6000) -- (87.3000,142.6000) -- (87.3000,142.6000) -- (87.3000,142.6000) -- (87.3000,142.6000) -- (87.3000,142.6000) -- (87.3000,142.6000) -- (87.3000,142.6000) -- (87.3000,142.6000) -- (87.3000,142.6000) -- (87.3000,142.6000) -- (87.3000,142.5000) -- (87.3000,142.5000) -- (87.3000,142.5000) -- (87.3000,142.5000) -- (87.3000,142.5000) -- (87.3000,142.5000) -- (87.3000,142.5000) -- (87.3000,142.5000) -- (87.3000,142.5000) -- (87.3000,142.5000) -- (87.3000,142.5000) -- (87.3000,142.5000) -- (87.3000,142.5000) -- (87.3000,142.5000) -- (87.3000,142.5000) -- (87.3000,142.5000) -- (87.3000,142.5000) -- (87.3000,142.5000) -- (87.3000,142.5000) -- (87.3000,142.5000) -- (87.3000,142.5000) -- (87.3000,142.5000) -- (87.3000,142.4000) -- (87.3000,142.4000) -- (87.3000,142.4000) -- (87.3000,142.4000) -- (87.3000,142.4000) -- (87.3000,142.4000) -- (87.3000,142.4000) -- (87.3000,142.4000) -- (87.3000,142.4000) -- (87.3000,142.4000) -- (87.3000,142.4000) -- (87.3000,142.4000) -- (87.3000,142.4000) -- (87.3000,142.4000) -- (87.3000,142.4000) -- (87.4000,142.4000) -- (87.4000,142.4000) -- (87.4000,142.4000) -- (87.4000,142.4000) -- (87.4000,142.4000) -- (87.4000,142.4000) -- (87.4000,142.3000) -- (87.4000,142.3000) -- (87.4000,142.3000) -- (87.4000,142.3000) -- (87.4000,142.3000) -- (87.4000,142.3000) -- (87.4000,142.3000) -- (87.4000,142.3000) -- (87.4000,142.3000) -- (87.4000,142.3000) -- (87.4000,142.3000) -- (87.4000,142.3000) -- (87.4000,142.3000) -- (87.4000,142.3000) -- (87.4000,142.3000) -- (87.4000,142.3000) -- (87.4000,142.3000) -- (87.4000,142.3000) -- (87.4000,142.3000) -- (87.4000,142.3000) -- (87.4000,142.3000) -- (87.4000,142.3000) -- (87.4000,142.2000) -- (87.4000,142.2000) -- (87.4000,142.2000) -- (87.4000,142.2000) -- (87.4000,142.2000) -- (87.4000,142.2000) -- (87.4000,142.2000) -- (87.4000,142.2000) -- (87.4000,142.2000) -- (87.4000,142.2000) -- (87.4000,142.2000) -- (87.4000,142.2000) -- (87.4000,142.2000) -- (87.4000,142.2000) -- (87.4000,142.2000) -- (87.4000,142.2000) -- (87.4000,142.2000) -- (87.4000,142.2000) -- (87.4000,142.2000) -- (87.4000,142.2000) -- (87.4000,142.2000) -- (87.4000,142.1000) -- (87.5000,142.1000) -- (87.5000,142.1000) -- (87.5000,142.1000) -- (87.5000,142.1000) -- (87.5000,142.1000) -- (87.5000,142.1000) -- (87.5000,142.1000) -- (87.5000,142.1000) -- (87.5000,142.1000) -- (87.5000,142.1000) -- (87.5000,142.1000) -- (87.5000,142.1000) -- (87.5000,142.1000) -- (87.5000,142.1000) -- (87.5000,142.1000) -- (87.5000,142.1000) -- (87.5000,142.1000) -- (87.5000,142.1000) -- (87.5000,142.1000) -- (87.5000,142.1000) -- (87.5000,142.1000) -- (87.5000,142.0000) -- (87.5000,142.0000) -- (87.5000,142.0000) -- (87.5000,142.0000) -- (87.5000,142.0000) -- (87.5000,142.0000) -- (87.5000,142.0000) -- (87.5000,142.0000) -- (87.5000,142.0000) -- (87.5000,142.0000) -- (87.5000,142.0000) -- (87.5000,142.0000) -- (87.5000,142.0000) -- (87.5000,142.0000) -- (87.5000,142.0000) -- (87.5000,142.0000) -- (87.5000,142.0000) -- (87.5000,142.0000) -- (87.5000,142.0000) -- (87.5000,142.0000) -- (87.5000,142.0000) -- (87.5000,141.9000) -- (87.5000,141.9000) -- (87.5000,141.9000) -- (87.5000,141.9000) -- (87.5000,141.9000) -- (87.5000,141.9000) -- (87.5000,141.9000) -- (87.6000,141.9000) -- (87.6000,141.9000) -- (87.6000,141.9000) -- (87.6000,141.9000) -- (87.6000,141.9000) -- (87.6000,141.9000) -- (87.6000,141.9000) -- (87.6000,141.9000) -- (87.6000,141.9000) -- (87.6000,141.9000) -- (87.6000,141.9000) -- (87.6000,141.9000) -- (87.6000,141.9000) -- (87.6000,141.9000) -- (87.6000,141.9000) -- (87.6000,141.8000) -- (87.6000,141.8000) -- (87.6000,141.8000) -- (87.6000,141.8000) -- (87.6000,141.8000) -- (87.6000,141.8000) -- (87.6000,141.8000) -- (87.6000,141.8000) -- (87.6000,141.8000) -- (87.6000,141.8000) -- (87.6000,141.8000) -- (87.6000,141.8000) -- (87.6000,141.8000) -- (87.6000,141.8000) -- (87.6000,141.8000) -- (87.6000,141.8000) -- (87.6000,141.8000) -- (87.6000,141.8000) -- (87.6000,141.8000) -- (87.6000,141.8000) -- (87.6000,141.8000) -- (87.6000,141.7000) -- (87.6000,141.7000) -- (87.6000,141.7000) -- (87.6000,141.7000) -- (87.6000,141.7000) -- (87.6000,141.7000) -- (87.6000,141.7000) -- (87.6000,141.7000) -- (87.6000,141.7000) -- (87.6000,141.7000) -- (87.6000,141.7000) -- (87.6000,141.7000) -- (87.6000,141.7000) -- (87.6000,141.7000) -- (87.7000,141.7000) -- (87.7000,141.7000) -- (87.7000,141.7000) -- (87.7000,141.7000) -- (87.7000,141.7000) -- (87.7000,141.7000) -- (87.7000,141.7000) -- (87.7000,141.7000) -- (87.7000,141.6000) -- (87.7000,141.6000) -- (87.7000,141.6000) -- (87.7000,141.6000) -- (87.7000,141.6000) -- (87.7000,141.6000) -- (87.7000,141.6000) -- (87.7000,141.6000) -- (87.7000,141.6000) -- (87.7000,141.6000) -- (87.7000,141.6000) -- (87.7000,141.6000) -- (87.7000,141.6000) -- (87.7000,141.6000) -- (87.7000,141.6000) -- (87.7000,141.6000) -- (87.7000,141.6000) -- (87.7000,141.6000) -- (87.7000,141.6000) -- (87.7000,141.6000) -- (87.7000,141.6000) -- (87.7000,141.5000) -- (87.7000,141.5000) -- (87.7000,141.5000) -- (87.7000,141.5000) -- (87.7000,141.5000) -- (87.7000,141.5000) -- (87.7000,141.5000) -- (87.7000,141.5000) -- (87.7000,141.5000) -- (87.7000,141.5000) -- (87.7000,141.5000) -- (87.7000,141.5000) -- (87.7000,141.5000) -- (87.7000,141.5000) -- (87.7000,141.5000) -- (87.7000,141.5000) -- (87.7000,141.5000) -- (87.7000,141.5000) -- (87.7000,141.5000) -- (87.7000,141.5000) -- (87.8000,141.5000) -- (87.8000,141.5000) -- (87.8000,141.4000) -- (87.8000,141.4000) -- (87.8000,141.4000) -- (87.8000,141.4000) -- (87.8000,141.4000) -- (87.8000,141.4000) -- (87.8000,141.4000) -- (87.8000,141.4000) -- (87.8000,141.4000) -- (87.8000,141.4000) -- (87.8000,141.4000) -- (87.8000,141.4000) -- (87.8000,141.4000) -- (87.8000,141.4000) -- (87.8000,141.4000) -- (87.8000,141.4000) -- (87.8000,141.4000) -- (87.8000,141.4000) -- (87.8000,141.4000) -- (87.8000,141.4000) -- (87.8000,141.4000) -- (87.8000,141.3000) -- (87.8000,141.3000) -- (87.8000,141.3000) -- (87.8000,141.3000) -- (87.8000,141.3000) -- (87.8000,141.3000) -- (87.8000,141.3000) -- (87.8000,141.3000) -- (87.8000,141.3000) -- (87.8000,141.3000) -- (87.8000,141.3000) -- (87.8000,141.3000) -- (87.8000,141.3000) -- (87.8000,141.3000) -- (87.8000,141.3000) -- (87.8000,141.3000) -- (87.8000,141.3000) -- (87.8000,141.3000) -- (87.8000,141.3000) -- (87.8000,141.3000) -- (87.8000,141.3000) -- (87.8000,141.3000) -- (87.8000,141.2000) -- (87.8000,141.2000) -- (87.8000,141.2000) -- (87.8000,141.2000) -- (87.8000,141.2000) -- (87.9000,141.2000) -- (87.9000,141.2000) -- (87.9000,141.2000) -- (87.9000,141.2000) -- (87.9000,141.2000) -- (87.9000,141.2000) -- (87.9000,141.2000) -- (87.9000,141.2000) -- (87.9000,141.2000) -- (87.9000,141.2000) -- (87.9000,141.2000) -- (87.9000,141.2000) -- (87.9000,141.2000) -- (87.9000,141.2000) -- (87.9000,141.2000) -- (87.9000,141.2000) -- (87.9000,141.1000) -- (87.9000,141.1000) -- (87.9000,141.1000) -- (87.9000,141.1000) -- (87.9000,141.1000) -- (87.9000,141.1000) -- (87.9000,141.1000) -- (87.9000,141.1000) -- (87.9000,141.1000) -- (87.9000,141.1000) -- (87.9000,141.1000) -- (87.9000,141.1000) -- (87.9000,141.1000) -- (87.9000,141.1000) -- (87.9000,141.1000) -- (87.9000,141.1000) -- (87.9000,141.1000) -- (87.9000,141.1000) -- (87.9000,141.1000) -- (87.9000,141.1000) -- (87.9000,141.1000) -- (87.9000,141.1000) -- (87.9000,141.0000) -- (87.9000,141.0000) -- (87.9000,141.0000) -- (87.9000,141.0000) -- (87.9000,141.0000) -- (87.9000,141.0000) -- (87.9000,141.0000) -- (87.9000,141.0000) -- (87.9000,141.0000) -- (87.9000,141.0000) -- (87.9000,141.0000) -- (88.0000,141.0000) -- (88.0000,141.0000) -- (88.0000,141.0000) -- (88.0000,141.0000) -- (88.0000,141.0000) -- (88.0000,141.0000) -- (88.0000,141.0000) -- (88.0000,141.0000) -- (88.0000,141.0000) -- (88.0000,141.0000) -- (88.0000,140.9000) -- (88.0000,140.9000) -- (88.0000,140.9000) -- (88.0000,140.9000) -- (88.0000,140.9000) -- (88.0000,140.9000) -- (88.0000,140.9000) -- (88.0000,140.9000) -- (88.0000,140.9000) -- (88.0000,140.9000) -- (88.0000,140.9000) -- (88.0000,140.9000) -- (88.0000,140.9000) -- (88.0000,140.9000) -- (88.0000,140.9000) -- (88.0000,140.9000) -- (88.0000,140.9000) -- (88.0000,140.9000) -- (88.0000,140.9000) -- (88.0000,140.9000) -- (88.0000,140.9000) -- (88.0000,140.9000) -- (88.0000,140.8000) -- (88.0000,140.8000) -- (88.0000,140.8000) -- (88.0000,140.8000) -- (88.0000,140.8000) -- (88.0000,140.8000) -- (88.0000,140.8000) -- (88.0000,140.8000) -- (88.0000,140.8000) -- (88.0000,140.8000) -- (88.0000,140.8000) -- (88.0000,140.8000) -- (88.0000,140.8000) -- (88.0000,140.8000) -- (88.0000,140.8000) -- (88.0000,140.8000) -- (88.0000,140.8000) -- (88.0000,140.8000) -- (88.1000,140.8000) -- (88.1000,140.8000) -- (88.1000,140.8000) -- (88.1000,140.7000) -- (88.1000,140.7000) -- (88.1000,140.7000) -- (88.1000,140.7000) -- (88.1000,140.7000) -- (88.1000,140.7000) -- (88.1000,140.7000) -- (88.1000,140.7000) -- (88.1000,140.7000) -- (88.1000,140.7000) -- (88.1000,140.7000) -- (88.1000,140.7000) -- (88.1000,140.7000) -- (88.1000,140.7000) -- (88.1000,140.7000) -- (88.1000,140.7000) -- (88.1000,140.7000) -- (88.1000,140.7000) -- (88.1000,140.7000) -- (88.1000,140.7000) -- (88.1000,140.7000) -- (88.1000,140.7000) -- (88.1000,140.6000) -- (88.1000,140.6000) -- (88.1000,140.6000) -- (88.1000,140.6000) -- (88.1000,140.6000) -- (88.1000,140.6000) -- (88.1000,140.6000) -- (88.1000,140.6000) -- (88.1000,140.6000) -- (88.1000,140.6000) -- (88.1000,140.6000) -- (88.1000,140.6000) -- (88.1000,140.6000) -- (88.1000,140.6000) -- (88.1000,140.6000) -- (88.1000,140.6000) -- (88.1000,140.6000) -- (88.1000,140.6000) -- (88.1000,140.6000) -- (88.1000,140.6000) -- (88.1000,140.6000) -- (88.1000,140.5000) -- (88.1000,140.5000) -- (88.1000,140.5000) -- (88.2000,140.5000) -- (88.2000,140.5000) -- (88.2000,140.5000) -- (88.2000,140.5000) -- (88.2000,140.5000) -- (88.2000,140.5000) -- (88.2000,140.5000) -- (88.2000,140.5000) -- (88.2000,140.5000) -- (88.2000,140.5000) -- (88.2000,140.5000) -- (88.2000,140.5000) -- (88.2000,140.5000) -- (88.2000,140.5000) -- (88.2000,140.5000) -- (88.2000,140.5000) -- (88.2000,140.5000) -- (88.2000,140.5000) -- (88.2000,140.5000) -- (88.2000,140.4000) -- (88.2000,140.4000) -- (88.2000,140.4000) -- (88.2000,140.4000) -- (88.2000,140.4000) -- (88.2000,140.4000) -- (88.2000,140.4000) -- (88.2000,140.4000) -- (88.2000,140.4000) -- (88.2000,140.4000) -- (88.2000,140.4000) -- (88.2000,140.4000) -- (88.2000,140.4000) -- (88.2000,140.4000) -- (88.2000,140.4000) -- (88.2000,140.4000) -- (88.2000,140.4000) -- (88.2000,140.4000) -- (88.2000,140.4000) -- (88.2000,140.4000) -- (88.2000,140.4000) -- (88.2000,140.3000) -- (88.2000,140.3000) -- (88.2000,140.3000) -- (88.2000,140.3000) -- (88.2000,140.3000) -- (88.2000,140.3000) -- (88.2000,140.3000) -- (88.2000,140.3000) -- (88.2000,140.3000) -- (88.2000,140.3000) -- (88.3000,140.3000) -- (88.3000,140.3000) -- (88.3000,140.3000) -- (88.3000,140.3000) -- (88.3000,140.3000) -- (88.3000,140.3000) -- (88.3000,140.3000) -- (88.3000,140.3000) -- (88.3000,140.3000) -- (88.3000,140.3000) -- (88.3000,140.3000) -- (88.3000,140.3000) -- (88.3000,140.2000) -- (88.3000,140.2000) -- (88.3000,140.2000) -- (88.3000,140.2000) -- (88.3000,140.2000) -- (88.3000,140.2000) -- (88.3000,140.2000) -- (88.3000,140.2000) -- (88.3000,140.2000) -- (88.3000,140.2000) -- (88.3000,140.2000) -- (88.3000,140.2000) -- (88.3000,140.2000) -- (88.3000,140.2000) -- (88.3000,140.2000) -- (88.3000,140.2000) -- (88.3000,140.2000) -- (88.3000,140.2000) -- (88.3000,140.2000) -- (88.3000,140.2000) -- (88.3000,140.2000) -- (88.3000,140.1000) -- (88.3000,140.1000) -- (88.3000,140.1000) -- (88.3000,140.1000) -- (88.3000,140.1000) -- (88.3000,140.1000) -- (88.3000,140.1000) -- (88.3000,140.1000) -- (88.3000,140.1000) -- (88.3000,140.1000) -- (88.3000,140.1000) -- (88.3000,140.1000) -- (88.3000,140.1000) -- (88.3000,140.1000) -- (88.3000,140.1000) -- (88.3000,140.1000) -- (88.4000,140.1000) -- (88.4000,140.1000) -- (88.4000,140.1000) -- (88.4000,140.1000) -- (88.4000,140.1000) -- (88.4000,140.1000) -- (88.4000,140.0000) -- (88.4000,140.0000) -- (88.4000,140.0000) -- (88.4000,140.0000) -- (88.4000,140.0000) -- (88.4000,140.0000) -- (88.4000,140.0000) -- (88.4000,140.0000) -- (88.4000,140.0000) -- (88.4000,140.0000) -- (88.4000,140.0000) -- (88.4000,140.0000) -- (88.4000,140.0000) -- (88.4000,140.0000) -- (88.4000,140.0000) -- (88.4000,140.0000) -- (88.4000,140.0000) -- (88.4000,140.0000) -- (88.4000,140.0000) -- (88.4000,140.0000) -- (88.4000,140.0000) -- (88.4000,139.9000) -- (88.4000,139.9000) -- (88.4000,139.9000) -- (88.4000,139.9000) -- (88.4000,139.9000) -- (88.4000,139.9000) -- (88.4000,139.9000) -- (88.4000,139.9000) -- (88.4000,139.9000) -- (88.4000,139.9000) -- (88.4000,139.9000) -- (88.4000,139.9000) -- (88.4000,139.9000) -- (88.4000,139.9000) -- (88.4000,139.9000) -- (88.4000,139.9000) -- (88.4000,139.9000) -- (88.4000,139.9000) -- (88.4000,139.9000) -- (88.4000,139.9000) -- (88.4000,139.9000) -- (88.4000,139.9000) -- (88.4000,139.8000) -- (88.5000,139.8000) -- (88.5000,139.8000) -- (88.5000,139.8000) -- (88.5000,139.8000) -- (88.5000,139.8000) -- (88.5000,139.8000) -- (88.5000,139.8000) -- (88.5000,139.8000) -- (88.5000,139.8000) -- (88.5000,139.8000) -- (88.5000,139.8000) -- (88.5000,139.8000) -- (88.5000,139.8000) -- (88.5000,139.8000) -- (88.5000,139.8000) -- (88.5000,139.8000) -- (88.5000,139.8000) -- (96.0000,139.8000) -- (96.0000,139.8000) -- (96.0000,139.8000) -- (96.0000,139.7000) -- (96.0000,139.7000) -- (96.0000,139.7000) -- (96.0000,139.7000) -- (96.0000,139.7000) -- (96.0000,139.7000) -- (96.0000,139.7000) -- (96.0000,139.7000) -- (96.0000,139.7000) -- (96.0000,139.7000) -- (96.0000,139.7000) -- (96.0000,139.7000) -- (96.0000,139.7000) -- (96.0000,139.7000) -- (96.0000,139.7000) -- (96.0000,139.7000) -- (96.0000,139.7000) -- (96.0000,139.7000) -- (96.0000,139.7000) -- (96.0000,139.7000) -- (96.0000,139.7000) -- (96.0000,139.7000) -- (96.0000,139.6000) -- (96.0000,139.6000) -- (96.0000,139.6000) -- (96.0000,139.6000) -- (96.0000,139.6000) -- (96.0000,139.6000) -- (96.0000,139.6000) -- (96.0000,139.6000) -- (96.0000,139.6000) -- (96.0000,139.6000) -- (96.0000,139.6000) -- (96.0000,139.6000) -- (96.0000,139.6000) -- (96.0000,139.6000) -- (96.1000,139.6000) -- (96.1000,139.6000) -- (96.1000,139.6000) -- (96.1000,139.6000) -- (96.1000,139.6000) -- (96.1000,139.6000) -- (96.1000,139.6000) -- (96.1000,139.5000) -- (96.1000,139.5000) -- (96.1000,139.5000) -- (96.1000,139.5000) -- (96.1000,139.5000) -- (96.1000,139.5000) -- (96.1000,139.5000) -- (96.1000,139.5000) -- (96.1000,139.5000) -- (96.1000,139.5000) -- (96.1000,139.5000) -- (96.1000,139.5000) -- (96.1000,139.5000) -- (96.1000,139.5000) -- (96.1000,139.5000) -- (96.1000,139.5000) -- (96.1000,139.5000) -- (96.1000,139.5000) -- (96.1000,139.5000) -- (96.1000,139.5000) -- (96.1000,139.5000) -- (96.1000,139.5000) -- (96.1000,139.4000) -- (96.1000,139.4000) -- (96.1000,139.4000) -- (96.1000,139.4000) -- (96.1000,139.4000) -- (96.1000,139.4000) -- (96.1000,139.4000) -- (96.1000,139.4000) -- (96.1000,139.4000) -- (96.1000,139.4000) -- (96.1000,139.4000) -- (96.1000,139.4000) -- (96.1000,139.4000) -- (96.1000,139.4000) -- (96.1000,139.4000) -- (96.1000,139.4000) -- (96.1000,139.4000) -- (96.1000,139.4000) -- (96.1000,139.4000) -- (96.1000,139.4000) -- (96.2000,139.4000) -- (96.2000,139.3000) -- (96.2000,139.3000) -- (96.2000,139.3000) -- (96.2000,139.3000) -- (96.2000,139.3000) -- (96.2000,139.3000) -- (96.2000,139.3000) -- (96.2000,139.3000) -- (96.2000,139.3000) -- (96.2000,139.3000) -- (96.2000,139.3000) -- (96.2000,139.3000) -- (96.2000,139.3000) -- (96.2000,139.3000) -- (96.2000,139.3000) -- (96.2000,139.3000) -- (96.2000,139.3000) -- (96.2000,139.3000) -- (96.2000,139.3000) -- (96.2000,139.3000) -- (96.2000,139.3000) -- (96.2000,139.3000) -- (96.2000,139.2000) -- (96.2000,139.2000) -- (96.2000,139.2000) -- (96.2000,139.2000) -- (96.2000,139.2000) -- (96.2000,139.2000) -- (96.2000,139.2000) -- (96.2000,139.2000) -- (96.2000,139.2000) -- (96.2000,139.2000) -- (96.2000,139.2000) -- (96.2000,139.2000) -- (96.2000,139.2000) -- (96.2000,139.2000) -- (96.2000,139.2000) -- (96.2000,139.2000) -- (96.2000,139.2000) -- (96.2000,139.2000) -- (96.2000,139.2000) -- (96.2000,139.2000) -- (96.2000,139.2000) -- (96.2000,139.1000) -- (96.2000,139.1000) -- (96.2000,139.1000) -- (96.2000,139.1000) -- (96.2000,139.1000) -- (96.2000,139.1000) -- (96.3000,139.1000) -- (96.3000,139.1000) -- (96.3000,139.1000) -- (96.3000,139.1000) -- (96.3000,139.1000) -- (96.3000,139.1000) -- (96.3000,139.1000) -- (96.3000,139.1000) -- (96.3000,139.1000) -- (96.3000,139.1000) -- (96.3000,139.1000) -- (96.3000,139.1000) -- (96.3000,139.1000) -- (96.3000,139.1000) -- (96.3000,139.1000) -- (96.3000,139.1000) -- (96.3000,139.0000) -- (96.3000,139.0000) -- (96.3000,139.0000) -- (96.3000,139.0000) -- (96.3000,139.0000) -- (96.3000,139.0000) -- (96.3000,139.0000) -- (96.3000,139.0000) -- (96.3000,139.0000) -- (96.3000,139.0000) -- (96.3000,139.0000) -- (96.3000,139.0000) -- (96.3000,139.0000) -- (96.3000,139.0000) -- (96.3000,139.0000) -- (96.3000,139.0000) -- (96.3000,139.0000) -- (96.3000,139.0000) -- (96.3000,139.0000) -- (96.3000,139.0000) -- (96.3000,139.0000) -- (96.3000,138.9000) -- (96.3000,138.9000) -- (96.3000,138.9000) -- (96.3000,138.9000) -- (96.3000,138.9000) -- (96.3000,138.9000) -- (96.3000,138.9000) -- (96.3000,138.9000) -- (96.3000,138.9000) -- (96.3000,138.9000) -- (96.3000,138.9000) -- (96.3000,138.9000) -- (96.3000,138.9000) -- (96.4000,138.9000) -- (96.4000,138.9000) -- (96.4000,138.9000) -- (96.4000,138.9000) -- (96.4000,138.9000) -- (96.4000,138.9000) -- (96.4000,138.9000) -- (96.4000,138.9000) -- (96.4000,138.9000) -- (96.4000,138.8000) -- (96.4000,138.8000) -- (96.4000,138.8000) -- (96.4000,138.8000) -- (96.4000,138.8000) -- (96.4000,138.8000) -- (96.4000,138.8000) -- (96.4000,138.8000) -- (96.4000,138.8000) -- (96.4000,138.8000) -- (96.4000,138.8000) -- (96.4000,138.8000) -- (96.4000,138.8000) -- (96.4000,138.8000) -- (96.4000,138.8000) -- (96.4000,138.8000) -- (96.4000,138.8000) -- (96.4000,138.8000) -- (96.4000,138.8000) -- (96.4000,138.8000) -- (96.4000,138.8000) -- (96.4000,138.7000) -- (96.4000,138.7000) -- (96.4000,138.7000) -- (96.4000,138.7000) -- (96.4000,138.7000) -- (96.4000,138.7000) -- (96.4000,138.7000) -- (96.4000,138.7000) -- (96.4000,138.7000) -- (96.4000,138.7000) -- (96.4000,138.7000) -- (96.4000,138.7000) -- (96.4000,138.7000) -- (96.4000,138.7000) -- (96.4000,138.7000) -- (96.4000,138.7000) -- (96.4000,138.7000) -- (96.4000,138.7000) -- (96.4000,138.7000) -- (96.5000,138.7000) -- (96.5000,138.7000) -- (96.5000,138.7000) -- (96.5000,138.6000) -- (96.5000,138.6000) -- (96.5000,138.6000) -- (96.5000,138.6000) -- (96.5000,138.6000) -- (96.5000,138.6000) -- (96.5000,138.6000) -- (96.5000,138.6000) -- (96.5000,138.6000) -- (96.5000,138.6000) -- (96.5000,138.6000) -- (96.5000,138.6000) -- (96.5000,138.6000) -- (96.5000,138.6000) -- (96.5000,138.6000) -- (96.5000,138.6000) -- (96.5000,138.6000) -- (96.5000,138.6000) -- (96.5000,138.6000) -- (96.5000,138.6000) -- (96.5000,138.6000) -- (96.5000,138.5000) -- (96.5000,138.5000) -- (96.5000,138.5000) -- (96.5000,138.5000) -- (96.5000,138.5000) -- (96.5000,138.5000) -- (96.5000,138.5000) -- (96.5000,138.5000) -- (96.5000,138.5000) -- (96.5000,138.5000) -- (96.5000,138.5000) -- (96.5000,138.5000) -- (96.5000,138.5000) -- (96.5000,138.5000) -- (96.5000,138.5000) -- (96.5000,138.5000) -- (96.5000,138.5000) -- (96.5000,138.5000) -- (96.5000,138.5000) -- (96.5000,138.5000) -- (96.5000,138.5000) -- (96.5000,138.5000) -- (96.5000,138.4000) -- (96.5000,138.4000) -- (96.5000,138.4000) -- (96.5000,138.4000) -- (96.6000,138.4000) -- (96.6000,138.4000) -- (96.6000,138.4000) -- (96.6000,138.4000) -- (96.6000,138.4000) -- (96.6000,138.4000) -- (96.6000,138.4000) -- (96.6000,138.4000) -- (96.6000,138.4000) -- (96.6000,138.4000) -- (96.6000,138.4000) -- (96.6000,138.4000) -- (96.6000,138.4000) -- (96.6000,138.4000) -- (96.6000,138.4000) -- (96.6000,138.4000) -- (96.6000,138.4000) -- (96.6000,138.3000) -- (96.6000,138.3000) -- (96.6000,138.3000) -- (96.6000,138.3000) -- (96.6000,138.3000) -- (96.6000,138.3000) -- (96.6000,138.3000) -- (96.6000,138.3000) -- (96.6000,138.3000) -- (96.6000,138.3000) -- (96.6000,138.3000) -- (96.6000,138.3000) -- (96.6000,138.3000) -- (96.6000,138.3000) -- (96.6000,138.3000) -- (96.6000,138.3000) -- (96.6000,138.3000) -- (96.6000,138.3000) -- (96.6000,138.3000) -- (96.6000,138.3000) -- (96.6000,138.3000) -- (96.6000,138.3000) -- (96.6000,138.2000) -- (96.6000,138.2000) -- (96.6000,138.2000) -- (96.6000,138.2000) -- (96.6000,138.2000) -- (96.6000,138.2000) -- (96.6000,138.2000) -- (96.6000,138.2000) -- (96.6000,138.2000) -- (96.6000,138.2000) -- (96.7000,138.2000) -- (96.7000,138.2000) -- (96.7000,138.2000) -- (96.7000,138.2000) -- (96.7000,138.2000) -- (96.7000,138.2000) -- (96.7000,138.2000) -- (96.7000,138.2000) -- (96.7000,138.2000) -- (96.7000,138.2000) -- (96.7000,138.2000) -- (96.7000,138.1000) -- (96.7000,138.1000) -- (96.7000,138.1000) -- (96.7000,138.1000) -- (96.7000,138.1000) -- (96.7000,138.1000) -- (96.7000,138.1000) -- (96.7000,138.1000) -- (96.7000,138.1000) -- (96.7000,138.1000) -- (96.7000,138.1000) -- (96.7000,138.1000) -- (96.7000,138.1000) -- (96.7000,138.1000) -- (96.7000,138.1000) -- (96.7000,138.1000) -- (96.7000,138.1000) -- (96.7000,138.1000) -- (96.7000,138.1000) -- (96.7000,138.1000) -- (96.7000,138.1000) -- (96.7000,138.1000) -- (96.7000,138.0000) -- (96.7000,138.0000) -- (96.7000,138.0000) -- (96.7000,138.0000) -- (96.7000,138.0000) -- (96.7000,138.0000) -- (96.7000,138.0000) -- (96.7000,138.0000) -- (96.7000,138.0000) -- (96.7000,138.0000) -- (96.7000,138.0000) -- (96.7000,138.0000) -- (96.7000,138.0000) -- (96.7000,138.0000) -- (96.7000,138.0000) -- (96.7000,138.0000) -- (96.7000,138.0000) -- (96.8000,138.0000) -- (96.8000,138.0000) -- (96.8000,138.0000) -- (96.8000,138.0000) -- (96.8000,137.9000) -- (96.8000,137.9000) -- (96.8000,137.9000) -- (96.8000,137.9000) -- (96.8000,137.9000) -- (96.8000,137.9000) -- (96.8000,137.9000) -- (96.8000,137.9000) -- (96.8000,137.9000) -- (96.8000,137.9000) -- (96.8000,137.9000) -- (96.8000,137.9000) -- (96.8000,137.9000) -- (96.8000,137.9000) -- (96.8000,137.9000) -- (96.8000,137.9000) -- (96.8000,137.9000) -- (96.8000,137.9000) -- (96.8000,137.9000) -- (96.8000,137.9000) -- (96.8000,137.9000) -- (96.8000,137.9000) -- (96.8000,137.8000) -- (96.8000,137.8000) -- (96.8000,137.8000) -- (96.8000,137.8000) -- (96.8000,137.8000) -- (96.8000,137.8000) -- (96.8000,137.8000) -- (96.8000,137.8000) -- (96.8000,137.8000) -- (96.8000,137.8000) -- (96.8000,137.8000) -- (96.8000,137.8000) -- (96.8000,137.8000) -- (96.8000,137.8000) -- (96.8000,137.8000) -- (96.8000,137.8000) -- (96.8000,137.8000) -- (96.8000,137.8000) -- (96.8000,137.8000) -- (96.8000,137.8000) -- (96.8000,137.8000) -- (96.8000,137.7000) -- (96.8000,137.7000) -- (96.9000,137.7000) -- (96.9000,137.7000) -- (96.9000,137.7000) -- (96.9000,137.7000) -- (96.9000,137.7000) -- (96.9000,137.7000) -- (96.9000,137.7000) -- (96.9000,137.7000) -- (96.9000,137.7000) -- (96.9000,137.7000) -- (96.9000,137.7000) -- (96.9000,137.7000) -- (96.9000,137.7000) -- (96.9000,137.7000) -- (96.9000,137.7000) -- (96.9000,137.7000) -- (96.9000,137.7000) -- (96.9000,137.7000) -- (96.9000,137.7000) -- (96.9000,137.7000) -- (96.9000,137.6000) -- (96.9000,137.6000) -- (96.9000,137.6000) -- (96.9000,137.6000) -- (96.9000,137.6000) -- (96.9000,137.6000) -- (96.9000,137.6000) -- (96.9000,137.6000) -- (96.9000,137.6000) -- (96.9000,137.6000) -- (96.9000,137.6000) -- (96.9000,137.6000) -- (96.9000,137.6000) -- (96.9000,137.6000) -- (96.9000,137.6000) -- (96.9000,137.6000) -- (96.9000,137.6000) -- (96.9000,137.6000) -- (96.9000,137.6000) -- (96.9000,137.6000) -- (96.9000,137.6000) -- (96.9000,137.5000) -- (96.9000,137.5000) -- (96.9000,137.5000) -- (96.9000,137.5000) -- (96.9000,137.5000) -- (96.9000,137.5000) -- (96.9000,137.5000) -- (96.9000,137.5000) -- (96.9000,137.5000) -- (97.0000,137.5000) -- (97.0000,137.5000) -- (97.0000,137.5000) -- (97.0000,137.5000) -- (97.0000,137.5000) -- (97.0000,137.5000) -- (97.0000,137.5000) -- (97.0000,137.5000) -- (97.0000,137.5000) -- (97.0000,137.5000) -- (97.0000,137.5000) -- (97.0000,137.5000) -- (97.0000,137.5000) -- (97.0000,137.4000) -- (97.0000,137.4000) -- (97.0000,137.4000) -- (97.0000,137.4000) -- (97.0000,137.4000) -- (97.0000,137.4000) -- (97.0000,137.4000) -- (97.0000,137.4000) -- (97.0000,137.4000) -- (97.0000,137.4000) -- (97.0000,137.4000) -- (97.0000,137.4000) -- (97.0000,137.4000) -- (97.0000,137.4000) -- (97.0000,137.4000) -- (97.0000,137.4000) -- (97.0000,137.4000) -- (97.0000,137.4000) -- (97.0000,137.4000) -- (97.0000,137.4000) -- (97.0000,137.4000) -- (97.0000,137.3000) -- (97.0000,137.3000) -- (97.0000,137.3000) -- (97.0000,137.3000) -- (97.0000,137.3000) -- (97.0000,137.3000) -- (97.0000,137.3000) -- (97.0000,137.3000) -- (97.0000,137.3000) -- (97.0000,137.3000) -- (97.0000,137.3000) -- (97.0000,137.3000) -- (97.0000,137.3000) -- (97.0000,137.3000) -- (97.0000,137.3000) -- (97.1000,137.3000) -- (97.1000,137.3000) -- (97.1000,137.3000) -- (97.1000,137.3000) -- (97.1000,137.3000) -- (97.1000,137.3000) -- (97.1000,137.3000) -- (97.1000,137.2000) -- (97.1000,137.2000) -- (97.1000,137.2000) -- (97.1000,137.2000) -- (97.1000,137.2000) -- (97.1000,137.2000) -- (97.1000,137.2000) -- (97.1000,137.2000) -- (97.1000,137.2000) -- (97.1000,137.2000) -- (97.1000,137.2000) -- (97.1000,137.2000) -- (97.1000,137.2000) -- (97.1000,137.2000) -- (97.1000,137.2000) -- (97.1000,137.2000) -- (97.1000,137.2000) -- (97.1000,137.2000) -- (97.1000,137.2000) -- (97.1000,137.2000) -- (97.1000,137.2000) -- (97.1000,137.1000) -- (97.1000,137.1000) -- (97.1000,137.1000) -- (97.1000,137.1000) -- (97.1000,137.1000) -- (97.1000,137.1000) -- (97.1000,137.1000) -- (97.1000,137.1000) -- (97.1000,137.1000) -- (97.1000,137.1000) -- (97.1000,137.1000) -- (97.1000,137.1000) -- (97.1000,137.1000) -- (97.1000,137.1000) -- (97.1000,137.1000) -- (97.1000,137.1000) -- (97.1000,137.1000) -- (97.1000,137.1000) -- (97.1000,137.1000) -- (97.1000,137.1000) -- (97.1000,137.1000) -- (97.1000,137.1000) -- (97.2000,137.0000) -- (97.2000,137.0000) -- (97.2000,137.0000) -- (97.2000,137.0000) -- (97.2000,137.0000) -- (97.2000,137.0000) -- (97.2000,137.0000) -- (97.2000,137.0000) -- (97.2000,137.0000) -- (97.2000,137.0000) -- (97.2000,137.0000) -- (97.2000,137.0000) -- (97.2000,137.0000) -- (97.2000,137.0000) -- (97.2000,137.0000) -- (97.2000,137.0000) -- (97.2000,137.0000) -- (97.2000,137.0000) -- (97.2000,137.0000) -- (97.2000,137.0000) -- (97.2000,137.0000) -- (97.2000,136.9000) -- (97.2000,136.9000) -- (97.2000,136.9000) -- (97.2000,136.9000) -- (97.2000,136.9000) -- (97.2000,136.9000) -- (97.2000,136.9000) -- (97.2000,136.9000) -- (97.2000,136.9000) -- (97.2000,136.9000) -- (97.2000,136.9000) -- (97.2000,136.9000) -- (97.2000,136.9000) -- (97.2000,136.9000) -- (97.2000,136.9000) -- (97.2000,136.9000) -- (97.2000,136.9000) -- (97.2000,136.9000) -- (97.2000,136.9000) -- (97.2000,136.9000) -- (97.2000,136.9000) -- (97.2000,136.9000) -- (97.2000,136.8000) -- (97.2000,136.8000) -- (97.2000,136.8000) -- (97.2000,136.8000) -- (97.2000,136.8000) -- (97.2000,136.8000) -- (97.3000,136.8000) -- (97.3000,136.8000) -- (97.3000,136.8000) -- (97.3000,136.8000) -- (97.3000,136.8000) -- (97.3000,136.8000) -- (97.3000,136.8000) -- (97.3000,136.8000) -- (97.3000,136.8000) -- (97.3000,136.8000) -- (97.3000,136.8000) -- (97.3000,136.8000) -- (97.3000,136.8000) -- (97.3000,136.8000) -- (97.3000,136.8000) -- (97.3000,136.7000) -- (97.3000,136.7000) -- (97.3000,136.7000) -- (97.3000,136.7000) -- (97.3000,136.7000) -- (97.3000,136.7000) -- (97.3000,136.7000) -- (97.3000,136.7000) -- (97.3000,136.7000) -- (97.3000,136.7000) -- (97.3000,136.7000) -- (97.3000,136.7000) -- (97.3000,136.7000) -- (97.3000,136.7000) -- (97.3000,136.7000) -- (97.3000,136.7000) -- (97.3000,136.7000) -- (97.3000,136.7000) -- (97.3000,136.7000) -- (97.3000,136.7000) -- (97.3000,136.7000) -- (97.3000,136.7000) -- (97.3000,136.6000) -- (97.3000,136.6000) -- (97.3000,136.6000) -- (97.3000,136.6000) -- (97.3000,136.6000) -- (97.3000,136.6000) -- (97.3000,136.6000) -- (97.3000,136.6000) -- (97.3000,136.6000) -- (97.3000,136.6000) -- (97.3000,136.6000) -- (97.3000,136.6000) -- (97.3000,136.6000) -- (97.4000,136.6000) -- (97.4000,136.6000) -- (97.4000,136.6000) -- (97.4000,136.6000) -- (97.4000,136.6000) -- (97.4000,136.6000) -- (97.4000,136.6000) -- (97.4000,136.6000) -- (97.4000,136.5000) -- (97.4000,136.5000) -- (97.4000,136.5000) -- (97.4000,136.5000) -- (97.4000,136.5000) -- (97.4000,136.5000) -- (97.4000,136.5000) -- (97.4000,136.5000) -- (97.4000,136.5000) -- (97.4000,136.5000) -- (97.4000,136.5000) -- (97.4000,136.5000) -- (97.4000,136.5000) -- (97.4000,136.5000) -- (97.4000,136.5000) -- (97.4000,136.5000) -- (97.4000,136.5000) -- (97.4000,136.5000) -- (97.4000,136.5000) -- (97.4000,136.5000) -- (97.4000,136.5000) -- (97.4000,136.5000) -- (97.4000,136.4000) -- (97.4000,136.4000) -- (97.4000,136.4000) -- (97.4000,136.4000) -- (97.4000,136.4000) -- (97.4000,136.4000) -- (97.4000,136.4000) -- (97.4000,136.4000) -- (97.4000,136.4000) -- (97.4000,136.4000) -- (97.4000,136.4000) -- (97.4000,136.4000) -- (97.4000,136.4000) -- (97.4000,136.4000) -- (97.4000,136.4000) -- (97.4000,136.4000) -- (97.4000,136.4000) -- (97.4000,136.4000) -- (97.4000,136.4000) -- (97.5000,136.4000) -- (97.5000,136.4000) -- (97.5000,136.3000) -- (97.5000,136.3000) -- (97.5000,136.3000) -- (97.5000,136.3000) -- (97.5000,136.3000) -- (97.5000,136.3000) -- (97.5000,136.3000) -- (97.5000,136.3000) -- (97.5000,136.3000) -- (97.5000,136.3000) -- (97.5000,136.3000) -- (97.5000,136.3000) -- (97.5000,136.3000) -- (97.5000,136.3000) -- (97.5000,136.3000) -- (97.5000,136.3000) -- (97.5000,136.3000) -- (97.5000,136.3000) -- (97.5000,136.3000) -- (97.5000,136.3000) -- (97.5000,136.3000) -- (97.5000,136.3000) -- (97.5000,136.2000) -- (97.5000,136.2000) -- (97.5000,136.2000) -- (97.5000,136.2000) -- (97.5000,136.2000) -- (97.5000,136.2000) -- (97.5000,136.2000) -- (97.5000,136.2000) -- (97.5000,136.2000) -- (97.5000,136.2000) -- (97.5000,136.2000) -- (97.5000,136.2000) -- (97.5000,136.2000) -- (97.5000,136.2000) -- (97.5000,136.2000) -- (97.5000,136.2000) -- (97.5000,136.2000) -- (97.5000,136.2000) -- (97.5000,136.2000) -- (97.5000,136.2000) -- (97.5000,136.2000) -- (97.5000,136.1000) -- (97.5000,136.1000) -- (97.5000,136.1000) -- (97.5000,136.1000) -- (97.5000,136.1000) -- (97.6000,136.1000) -- (97.6000,136.1000) -- (97.6000,136.1000) -- (97.6000,136.1000) -- (97.6000,136.1000) -- (97.6000,136.1000) -- (97.6000,136.1000) -- (97.6000,136.1000) -- (97.6000,136.1000) -- (97.6000,136.1000) -- (97.6000,136.1000) -- (97.6000,136.1000) -- (97.6000,136.1000) -- (97.6000,136.1000) -- (97.6000,136.1000) -- (97.6000,136.1000) -- (97.6000,136.1000) -- (97.6000,136.0000) -- (97.6000,136.0000) -- (97.6000,136.0000) -- (97.6000,136.0000) -- (97.6000,136.0000) -- (97.6000,136.0000) -- (97.6000,136.0000) -- (97.6000,136.0000) -- (97.6000,136.0000) -- (97.6000,136.0000) -- (97.6000,136.0000) -- (97.6000,136.0000) -- (97.6000,136.0000) -- (97.6000,136.0000) -- (97.6000,136.0000) -- (97.6000,136.0000) -- (97.6000,136.0000) -- (97.6000,136.0000) -- (97.6000,136.0000) -- (97.6000,136.0000) -- (97.6000,136.0000) -- (97.6000,135.9000) -- (97.6000,135.9000) -- (97.6000,135.9000) -- (97.6000,135.9000) -- (97.6000,135.9000) -- (97.6000,135.9000) -- (97.6000,135.9000) -- (97.6000,135.9000) -- (97.6000,135.9000) -- (97.6000,135.9000) -- (97.6000,135.9000) -- (97.7000,135.9000) -- (97.7000,135.9000) -- (97.7000,135.9000) -- (97.7000,135.9000) -- (97.7000,135.9000) -- (97.7000,135.9000) -- (97.7000,135.9000) -- (97.7000,135.9000) -- (97.7000,135.9000) -- (97.7000,135.9000) -- (97.7000,135.9000) -- (97.7000,135.8000) -- (97.7000,135.8000) -- (97.7000,135.8000) -- (97.7000,135.8000) -- (97.7000,135.8000) -- (97.7000,135.8000) -- (97.7000,135.8000) -- (97.7000,135.8000) -- (97.7000,135.8000) -- (97.7000,135.8000) -- (97.7000,135.8000) -- (97.7000,135.8000) -- (97.7000,135.8000) -- (97.7000,135.8000) -- (97.7000,135.8000) -- (97.7000,135.8000) -- (97.7000,135.8000) -- (97.7000,135.8000) -- (97.7000,135.8000) -- (97.7000,135.8000) -- (97.7000,135.8000) -- (97.7000,135.7000) -- (97.7000,135.7000) -- (97.7000,135.7000) -- (97.7000,135.7000) -- (97.7000,135.7000) -- (97.7000,135.7000) -- (97.7000,135.7000) -- (97.7000,135.7000) -- (97.7000,135.7000) -- (97.7000,135.7000) -- (97.7000,135.7000) -- (97.7000,135.7000) -- (97.7000,135.7000) -- (97.7000,135.7000) -- (97.7000,135.7000) -- (97.7000,135.7000) -- (97.7000,135.7000) -- (97.7000,135.7000) -- (97.8000,135.7000) -- (97.8000,135.7000) -- (97.8000,135.7000) -- (97.8000,135.7000) -- (97.8000,135.6000) -- (97.8000,135.6000) -- (97.8000,135.6000) -- (97.8000,135.6000) -- (97.8000,135.6000) -- (97.8000,135.6000) -- (97.8000,135.6000) -- (97.8000,135.6000) -- (97.8000,135.6000) -- (97.8000,135.6000) -- (97.8000,135.6000) -- (97.8000,135.6000) -- (97.8000,135.6000) -- (97.8000,135.6000) -- (97.8000,135.6000) -- (97.8000,135.6000) -- (97.8000,135.6000) -- (97.8000,135.6000) -- (97.8000,135.6000) -- (97.8000,135.6000) -- (97.8000,135.6000) -- (97.8000,135.5000) -- (97.8000,135.5000) -- (97.8000,135.5000) -- (97.8000,135.5000) -- (97.8000,135.5000) -- (97.8000,135.5000) -- (97.8000,135.5000) -- (97.8000,135.5000) -- (97.8000,135.5000) -- (97.8000,135.5000) -- (97.8000,135.5000) -- (97.8000,135.5000) -- (97.8000,135.5000) -- (97.8000,135.5000) -- (97.8000,135.5000) -- (97.8000,135.5000) -- (97.8000,135.5000) -- (97.8000,135.5000) -- (97.8000,135.5000) -- (97.8000,135.5000) -- (97.8000,135.5000) -- (97.8000,135.5000) -- (97.8000,135.4000) -- (97.8000,135.4000) -- (97.9000,135.4000) -- (97.9000,135.4000) -- (97.9000,135.4000) -- (97.9000,135.4000) -- (97.9000,135.4000) -- (97.9000,135.4000) -- (97.9000,135.4000) -- (97.9000,135.4000) -- (97.9000,135.4000) -- (97.9000,135.4000) -- (97.9000,135.4000) -- (97.9000,135.4000) -- (97.9000,135.4000) -- (97.9000,135.4000) -- (97.9000,135.4000) -- (97.9000,135.4000) -- (97.9000,135.4000) -- (97.9000,135.4000) -- (97.9000,135.4000) -- (97.9000,135.3000) -- (97.9000,135.3000) -- (97.9000,135.3000) -- (97.9000,135.3000) -- (97.9000,135.3000) -- (97.9000,135.3000) -- (97.9000,135.3000) -- (97.9000,135.3000) -- (97.9000,135.3000) -- (97.9000,135.3000) -- (97.9000,135.3000) -- (97.9000,135.3000) -- (97.9000,135.3000) -- (107.3000,135.3000) -- (107.3000,135.3000) -- (107.3000,135.3000) -- (107.4000,135.3000) -- (121.2657,135.3000);



    \end{scope}
    \begin{scope}[cm={{1.21653,0.0,0.0,1.34548,(-346.94368,-156.28627)}},draw=blue,line cap=round,line join=round,line width=0.480pt]
      \path[draw] (81.5000,129.5000) -- (81.5000,157.5000) -- (121.5000,157.5000) -- (121.5000,129.5000) -- (81.5000,129.5000);



    \end{scope}
    \path[cm={{0.99667,0.0,0.0,1.34693,(-320.19845,-156.47287)}},draw=blue] (41.5000,88.5000) -- (41.5000,164.5000) -- (127.5000,164.5000) -- (127.5000,88.5000) -- (41.5000,88.5000);



    \path[draw=cffffff,line cap=butt,line join=miter,line width=1.312pt,miter limit=4.00] (-300.8820,-23.4840) -- (-279.6507,-38.3822);



  \end{scope}
  \begin{scope}[cm={{1.26012,0.0,0.0,1.26012,(-619.61892,-54.78689)}},draw=ca0a0a4,dash pattern=on 0.95pt off 0.95pt,line cap=round,line join=round,line width=0.238pt,miter limit=4.00]
    \path[draw,dash pattern=on 0.95pt off 0.95pt,line width=0.238pt,miter limit=4.00] (165.5000,88.5000) -- (251.5000,88.5000);



  \end{scope}
  \begin{scope}[cm={{1.26012,0.0,0.0,1.26012,(-619.61892,-54.78689)}},draw=blue,line cap=round,line join=round,line width=0.480pt]
    \path[cm={{0.9189,0.0,0.0,1.0,(13.39849,0.0)}},draw] (165.5000,88.5000) -- (168.5000,88.5000);



    \path[cm={{0.9189,0.0,0.0,1.0,(20.28216,0.0)}},draw] (251.5000,88.5000) -- (248.5000,88.5000);



  \end{scope}
  \begin{scope}[scale=1.006,draw=blue,line cap=rect,line join=bevel,line width=0.800pt]
  \end{scope}
  \begin{scope}[cm={{1.00588,0.0,0.0,1.00588,(148.871,93.5471)}},draw=blue,line cap=rect,line join=bevel,line width=0.800pt]
  \end{scope}
  \begin{scope}[cm={{1.00588,0.0,0.0,1.00588,(148.871,93.5471)}},draw=blue,line cap=rect,line join=bevel,line width=0.800pt]
  \end{scope}
  \begin{scope}[cm={{1.00588,0.0,0.0,1.00588,(148.871,93.5471)}},draw=blue,line cap=rect,line join=bevel,line width=0.800pt]
  \end{scope}
  \begin{scope}[cm={{1.00588,0.0,0.0,1.00588,(148.871,93.5471)}},draw=blue,line cap=rect,line join=bevel,line width=0.800pt]
  \end{scope}
  \begin{scope}[cm={{1.00588,0.0,0.0,1.00588,(148.871,93.5471)}},draw=blue,line cap=rect,line join=bevel,line width=0.800pt]
  \end{scope}
  \begin{scope}[cm={{1.00588,0.0,0.0,1.00588,(-425.82403,58.61615)}},draw=blue,line cap=rect,line join=bevel,line width=0.800pt]
    \path[fill=blue] (0.0000,0.0000) node[above right] (text380) {27};



  \end{scope}
  \begin{scope}[cm={{1.00588,0.0,0.0,1.00588,(148.871,93.5471)}},draw=blue,line cap=rect,line join=bevel,line width=0.800pt]
  \end{scope}
  \begin{scope}[scale=1.006,draw=blue,line cap=rect,line join=bevel,line width=0.800pt]
  \end{scope}
  \begin{scope}[cm={{1.26012,0.0,0.0,1.26012,(-619.61892,-54.78689)}},draw=ca0a0a4,dash pattern=on 0.95pt off 0.95pt,line cap=round,line join=round,line width=0.238pt,miter limit=4.00]
    \path[draw,dash pattern=on 0.95pt off 0.95pt,line width=0.238pt,miter limit=4.00] (165.5000,63.5000) -- (251.5000,63.5000);



  \end{scope}
  \begin{scope}[cm={{1.26012,0.0,0.0,1.26012,(-619.61892,-54.78689)}},draw=blue,line cap=round,line join=round,line width=0.480pt]
    \path[cm={{0.9189,0.0,0.0,1.0,(13.39849,0.0)}},draw] (165.5000,63.5000) -- (168.5000,63.5000);



    \path[cm={{0.9189,0.0,0.0,1.0,(20.28216,0.0)}},draw] (251.5000,63.5000) -- (248.5000,63.5000);



  \end{scope}
  \begin{scope}[scale=1.006,draw=blue,line cap=rect,line join=bevel,line width=0.800pt]
  \end{scope}
  \begin{scope}[cm={{1.00588,0.0,0.0,1.00588,(149.876,68.4)}},draw=blue,line cap=rect,line join=bevel,line width=0.800pt]
  \end{scope}
  \begin{scope}[cm={{1.00588,0.0,0.0,1.00588,(149.876,68.4)}},draw=blue,line cap=rect,line join=bevel,line width=0.800pt]
  \end{scope}
  \begin{scope}[cm={{1.00588,0.0,0.0,1.00588,(149.876,68.4)}},draw=blue,line cap=rect,line join=bevel,line width=0.800pt]
  \end{scope}
  \begin{scope}[cm={{1.00588,0.0,0.0,1.00588,(149.876,68.4)}},draw=blue,line cap=rect,line join=bevel,line width=0.800pt]
  \end{scope}
  \begin{scope}[cm={{1.00588,0.0,0.0,1.00588,(149.876,68.4)}},draw=blue,line cap=rect,line join=bevel,line width=0.800pt]
  \end{scope}
  \begin{scope}[cm={{1.00588,0.0,0.0,1.00588,(-425.38144,27.46905)}},draw=blue,line cap=rect,line join=bevel,line width=0.800pt]
    \path[fill=blue] (0.0000,0.0000) node[above right] (text410) {31};



  \end{scope}
  \begin{scope}[cm={{1.00588,0.0,0.0,1.00588,(149.876,68.4)}},draw=blue,line cap=rect,line join=bevel,line width=0.800pt]
  \end{scope}
  \begin{scope}[scale=1.006,draw=blue,line cap=rect,line join=bevel,line width=0.800pt]
  \end{scope}
  \begin{scope}[cm={{1.26012,0.0,0.0,1.26012,(-482.17027,-168.78691)}},draw=ca0a0a4,dash pattern=on 0.95pt off 0.95pt,line cap=round,line join=round,line width=0.238pt,miter limit=4.00]
    \path[draw,dash pattern=on 0.95pt off 0.95pt,line width=0.238pt,miter limit=4.00] (108.5000,95.5000) -- (108.5000,36.5000);



    \path[draw,dash pattern=on 0.95pt off 0.95pt,line width=0.238pt,miter limit=4.00] (108.5000,20.5000) -- (108.5000,13.5000);



  \end{scope}
  \begin{scope}[cm={{1.26012,0.0,0.0,1.26012,(-619.61892,-54.78689)}},draw=ca0a0a4,dash pattern=on 0.95pt off 0.95pt,line cap=round,line join=round,line width=0.238pt,miter limit=4.00]
    \path[draw,dash pattern=on 0.95pt off 0.95pt,line width=0.238pt,miter limit=4.00] (165.5000,38.5000) -- (251.5000,38.5000);



  \end{scope}
  \begin{scope}[cm={{1.26012,0.0,0.0,1.26012,(-619.61892,-54.78689)}},draw=blue,line cap=round,line join=round,line width=0.480pt]
    \path[cm={{0.9189,0.0,0.0,1.0,(13.39849,0.0)}},draw] (165.5000,38.5000) -- (168.5000,38.5000);



    \path[cm={{0.9189,0.0,0.0,1.0,(20.28216,0.0)}},draw] (251.5000,38.5000) -- (248.5000,38.5000);



  \end{scope}
  \begin{scope}[scale=1.006,draw=blue,line cap=rect,line join=bevel,line width=0.800pt]
  \end{scope}
  \begin{scope}[cm={{1.00588,0.0,0.0,1.00588,(149.876,43.2529)}},draw=blue,line cap=rect,line join=bevel,line width=0.800pt]
  \end{scope}
  \begin{scope}[cm={{1.00588,0.0,0.0,1.00588,(149.876,43.2529)}},draw=blue,line cap=rect,line join=bevel,line width=0.800pt]
  \end{scope}
  \begin{scope}[cm={{1.00588,0.0,0.0,1.00588,(149.876,43.2529)}},draw=blue,line cap=rect,line join=bevel,line width=0.800pt]
  \end{scope}
  \begin{scope}[cm={{1.00588,0.0,0.0,1.00588,(149.876,43.2529)}},draw=blue,line cap=rect,line join=bevel,line width=0.800pt]
  \end{scope}
  \begin{scope}[cm={{1.00588,0.0,0.0,1.00588,(149.876,43.2529)}},draw=blue,line cap=rect,line join=bevel,line width=0.800pt]
  \end{scope}
  \begin{scope}[cm={{1.00588,0.0,0.0,1.00588,(-425.73551,-5.17805)}},draw=blue,line cap=rect,line join=bevel,line width=0.800pt]
    \path[fill=blue] (0.0000,0.0000) node[above right] (text440) {35};



  \end{scope}
  \begin{scope}[cm={{1.00588,0.0,0.0,1.00588,(149.876,43.2529)}},draw=blue,line cap=rect,line join=bevel,line width=0.800pt]
  \end{scope}
  \begin{scope}[scale=1.006,draw=blue,line cap=rect,line join=bevel,line width=0.800pt]
  \end{scope}
  \begin{scope}[cm={{1.26012,0.0,0.0,1.26012,(-619.61892,-54.78689)}},draw=ca0a0a4,dash pattern=on 0.40pt off 0.80pt,line cap=round,line join=round,line width=0.400pt]
    \path[draw] (165.5000,95.5000) -- (165.5000,13.5000);



  \end{scope}
  \begin{scope}[cm={{1.26012,0.0,0.0,1.26012,(-619.61892,-54.78689)}},draw=blue,line cap=round,line join=round,line width=0.480pt]
    \path[draw] (165.5000,95.5000) -- (165.5000,92.5000);



    \path[draw] (165.5000,13.5000) -- (165.5000,16.5000);



  \end{scope}
  \begin{scope}[scale=1.006,draw=blue,line cap=rect,line join=bevel,line width=0.800pt]
  \end{scope}
  \begin{scope}[cm={{1.00588,0.0,0.0,1.00588,(162.953,110.647)}},draw=blue,line cap=rect,line join=bevel,line width=0.800pt]
  \end{scope}
  \begin{scope}[cm={{1.00588,0.0,0.0,1.00588,(162.953,110.647)}},draw=blue,line cap=rect,line join=bevel,line width=0.800pt]
  \end{scope}
  \begin{scope}[cm={{1.00588,0.0,0.0,1.00588,(162.953,110.647)}},draw=blue,line cap=rect,line join=bevel,line width=0.800pt]
  \end{scope}
  \begin{scope}[cm={{1.00588,0.0,0.0,1.00588,(162.953,110.647)}},draw=blue,line cap=rect,line join=bevel,line width=0.800pt]
  \end{scope}
  \begin{scope}[cm={{1.00588,0.0,0.0,1.00588,(162.953,110.647)}},draw=blue,line cap=rect,line join=bevel,line width=0.800pt]
  \end{scope}
  \begin{scope}[cm={{1.00588,0.0,0.0,1.00588,(-413.68137,77.03586)}},draw=blue,line cap=rect,line join=bevel,line width=0.800pt]
    \path[fill=blue] (0.0000,0.0000) node[above right] (text470) {0};



  \end{scope}
  \begin{scope}[cm={{1.00588,0.0,0.0,1.00588,(162.953,110.647)}},draw=blue,line cap=rect,line join=bevel,line width=0.800pt]
  \end{scope}
  \begin{scope}[scale=1.006,draw=blue,line cap=rect,line join=bevel,line width=0.800pt]
  \end{scope}
  \begin{scope}[cm={{1.26012,0.0,0.0,1.26012,(-619.6494,-55.17358)}},draw=ca0a0a4,dash pattern=on 1.03pt off 1.03pt,line cap=round,line join=round,line width=0.257pt,miter limit=4.00]
    \path[draw,dash pattern=on 1.03pt off 1.03pt,line width=0.257pt,miter limit=4.00] (191.5000,95.5000) -- (191.5000,13.5000);



  \end{scope}
  \begin{scope}[cm={{1.26012,0.0,0.0,1.26012,(-482.17027,-168.78691)}},draw=ca0a0a4,dash pattern=on 0.95pt off 0.95pt,line cap=round,line join=round,line width=0.238pt,miter limit=4.00]
    \path[draw,dash pattern=on 0.95pt off 0.95pt,line width=0.238pt,miter limit=4.00] (82.5000,95.5000) -- (82.5000,36.5000);



    \path[draw,dash pattern=on 0.95pt off 0.95pt,line width=0.238pt,miter limit=4.00] (82.5000,20.5000) -- (82.5000,13.5000);



  \end{scope}
  \begin{scope}[cm={{1.26012,0.0,0.0,1.26012,(-619.61892,-54.78689)}},draw=blue,line cap=round,line join=round,line width=0.480pt]
    \path[cm={{1.0,0.0,0.0,0.9189,(0.0,7.78184)}},draw] (191.5000,95.5000) -- (191.5000,92.5000);



    \path[cm={{1.0,0.0,0.0,0.9189,(0.0,1.07058)}},draw] (191.5000,13.5000) -- (191.5000,16.5000);



  \end{scope}
  \begin{scope}[scale=1.006,draw=blue,line cap=rect,line join=bevel,line width=0.800pt]
  \end{scope}
  \begin{scope}[cm={{1.00588,0.0,0.0,1.00588,(189.106,110.647)}},draw=blue,line cap=rect,line join=bevel,line width=0.800pt]
  \end{scope}
  \begin{scope}[cm={{1.00588,0.0,0.0,1.00588,(189.106,110.647)}},draw=blue,line cap=rect,line join=bevel,line width=0.800pt]
  \end{scope}
  \begin{scope}[cm={{1.00588,0.0,0.0,1.00588,(189.106,110.647)}},draw=blue,line cap=rect,line join=bevel,line width=0.800pt]
  \end{scope}
  \begin{scope}[cm={{1.00588,0.0,0.0,1.00588,(189.106,110.647)}},draw=blue,line cap=rect,line join=bevel,line width=0.800pt]
  \end{scope}
  \begin{scope}[cm={{1.00588,0.0,0.0,1.00588,(189.106,110.647)}},draw=blue,line cap=rect,line join=bevel,line width=0.800pt]
  \end{scope}
  \begin{scope}[cm={{1.00588,0.0,0.0,1.00588,(-380.02837,77.03586)}},draw=blue,line cap=rect,line join=bevel,line width=0.800pt]
    \path[fill=blue] (0.0000,0.0000) node[above right] (text500) {1};



  \end{scope}
  \begin{scope}[cm={{1.00588,0.0,0.0,1.00588,(189.106,110.647)}},draw=blue,line cap=rect,line join=bevel,line width=0.800pt]
  \end{scope}
  \begin{scope}[scale=1.006,draw=blue,line cap=rect,line join=bevel,line width=0.800pt]
  \end{scope}
  \begin{scope}[cm={{1.26012,0.0,0.0,1.26012,(-619.6494,-55.17358)}},draw=ca0a0a4,dash pattern=on 1.03pt off 1.03pt,line cap=round,line join=round,line width=0.257pt,miter limit=4.00]
    \path[draw,dash pattern=on 1.03pt off 1.03pt,line width=0.257pt,miter limit=4.00] (217.5000,95.5000) -- (217.5000,13.5000);



  \end{scope}
  \begin{scope}[cm={{1.26012,0.0,0.0,1.26012,(-619.61892,-54.78689)}},draw=blue,line cap=round,line join=round,line width=0.480pt]
    \path[cm={{1.0,0.0,0.0,0.9189,(0.0,7.78184)}},draw] (217.5000,95.5000) -- (217.5000,92.5000);



    \path[cm={{1.0,0.0,0.0,0.9189,(0.0,1.07058)}},draw] (217.5000,13.5000) -- (217.5000,16.5000);



  \end{scope}
  \begin{scope}[scale=1.006,draw=blue,line cap=rect,line join=bevel,line width=0.800pt]
  \end{scope}
  \begin{scope}[cm={{1.00588,0.0,0.0,1.00588,(215.259,110.647)}},draw=blue,line cap=rect,line join=bevel,line width=0.800pt]
  \end{scope}
  \begin{scope}[cm={{1.00588,0.0,0.0,1.00588,(215.259,110.647)}},draw=blue,line cap=rect,line join=bevel,line width=0.800pt]
  \end{scope}
  \begin{scope}[cm={{1.00588,0.0,0.0,1.00588,(215.259,110.647)}},draw=blue,line cap=rect,line join=bevel,line width=0.800pt]
  \end{scope}
  \begin{scope}[cm={{1.00588,0.0,0.0,1.00588,(215.259,110.647)}},draw=blue,line cap=rect,line join=bevel,line width=0.800pt]
  \end{scope}
  \begin{scope}[cm={{1.00588,0.0,0.0,1.00588,(215.259,110.647)}},draw=blue,line cap=rect,line join=bevel,line width=0.800pt]
  \end{scope}
  \begin{scope}[cm={{1.00588,0.0,0.0,1.00588,(-347.87533,77.03586)}},draw=blue,line cap=rect,line join=bevel,line width=0.800pt]
    \path[fill=blue] (0.0000,0.0000) node[above right] (text530) {2};



  \end{scope}
  \begin{scope}[cm={{1.00588,0.0,0.0,1.00588,(215.259,110.647)}},draw=blue,line cap=rect,line join=bevel,line width=0.800pt]
  \end{scope}
  \begin{scope}[scale=1.006,draw=blue,line cap=rect,line join=bevel,line width=0.800pt]
  \end{scope}
  \begin{scope}[cm={{1.26012,0.0,0.0,1.26012,(-619.61892,-55.16996)}},draw=ca0a0a4,dash pattern=on 0.95pt off 0.95pt,line cap=round,line join=round,line width=0.238pt,miter limit=4.00]
    \path[draw,dash pattern=on 0.95pt off 0.95pt,line width=0.238pt,miter limit=4.00] (243.5000,95.5000) -- (243.5000,13.5000);



  \end{scope}
  \begin{scope}[cm={{1.26012,0.0,0.0,1.26012,(-619.61892,-54.78689)}},draw=blue,line cap=round,line join=round,line width=0.480pt]
    \path[cm={{1.0,0.0,0.0,0.9189,(0.0,7.78184)}},draw] (243.5000,95.5000) -- (243.5000,92.5000);



    \path[cm={{1.0,0.0,0.0,0.9189,(0.0,1.07058)}},draw] (243.5000,13.5000) -- (243.5000,16.5000);



  \end{scope}
  \begin{scope}[scale=1.006,draw=blue,line cap=rect,line join=bevel,line width=0.800pt]
  \end{scope}
  \begin{scope}[cm={{1.00588,0.0,0.0,1.00588,(241.915,110.647)}},draw=blue,line cap=rect,line join=bevel,line width=0.800pt]
  \end{scope}
  \begin{scope}[cm={{1.00588,0.0,0.0,1.00588,(241.915,110.647)}},draw=blue,line cap=rect,line join=bevel,line width=0.800pt]
  \end{scope}
  \begin{scope}[cm={{1.00588,0.0,0.0,1.00588,(241.915,110.647)}},draw=blue,line cap=rect,line join=bevel,line width=0.800pt]
  \end{scope}
  \begin{scope}[cm={{1.00588,0.0,0.0,1.00588,(241.915,110.647)}},draw=blue,line cap=rect,line join=bevel,line width=0.800pt]
  \end{scope}
  \begin{scope}[cm={{1.00588,0.0,0.0,1.00588,(241.915,110.647)}},draw=blue,line cap=rect,line join=bevel,line width=0.800pt]
  \end{scope}
  \begin{scope}[cm={{1.00588,0.0,0.0,1.00588,(-315.21933,77.03586)}},draw=blue,line cap=rect,line join=bevel,line width=0.800pt]
    \path[fill=blue] (0.0000,0.0000) node[above right] (text560) {3};



  \end{scope}
  \begin{scope}[cm={{1.00588,0.0,0.0,1.00588,(241.915,110.647)}},draw=blue,line cap=rect,line join=bevel,line width=0.800pt]
  \end{scope}
  \begin{scope}[scale=1.006,draw=blue,line cap=rect,line join=bevel,line width=0.800pt]
  \end{scope}
  \begin{scope}[cm={{1.26012,0.0,0.0,1.26012,(-619.61892,-54.78689)}},draw=blue,line cap=round,line join=round,line width=0.480pt]
    \path[draw] (165.5000,13.5000) -- (165.5000,95.5000) -- (251.5000,95.5000) -- (251.5000,13.5000) -- (165.5000,13.5000);



  \end{scope}
  \begin{scope}[scale=1.006,draw=blue,line cap=rect,line join=bevel,line width=0.800pt]
  \end{scope}
  \begin{scope}[scale=1.006,draw=blue,line cap=rect,line join=bevel,line width=0.800pt]
  \end{scope}
  \begin{scope}[scale=1.006,draw=blue,line cap=rect,line join=bevel,line width=0.800pt]
  \end{scope}
  \begin{scope}[scale=1.006,draw=blue,line cap=rect,line join=bevel,line width=0.800pt]
  \end{scope}
  \begin{scope}[scale=1.006,draw=blue,line cap=rect,line join=bevel,line width=0.800pt]
  \end{scope}
  \begin{scope}[cm={{1.00588,0.0,0.0,1.00588,(235.376,29.1706)}},draw=blue,line cap=rect,line join=bevel,line width=0.800pt]
  \end{scope}
  \begin{scope}[cm={{1.00588,0.0,0.0,1.00588,(235.376,29.1706)}},draw=blue,line cap=rect,line join=bevel,line width=0.800pt]
  \end{scope}
  \begin{scope}[cm={{1.00588,0.0,0.0,1.00588,(235.376,29.1706)}},draw=blue,line cap=rect,line join=bevel,line width=0.800pt]
  \end{scope}
  \begin{scope}[cm={{1.00588,0.0,0.0,1.00588,(235.376,29.1706)}},draw=blue,line cap=rect,line join=bevel,line width=0.800pt]
  \end{scope}
  \begin{scope}[cm={{1.00588,0.0,0.0,1.00588,(235.376,29.1706)}},draw=blue,line cap=rect,line join=bevel,line width=0.800pt]
  \end{scope}
  \begin{scope}[cm={{1.00588,0.0,0.0,1.00588,(235.376,29.1706)}},draw=blue,line cap=rect,line join=bevel,line width=0.800pt]
  \end{scope}
  \begin{scope}[scale=1.006,draw=blue,line cap=rect,line join=bevel,line width=0.800pt]
  \end{scope}
  \begin{scope}[scale=1.006,draw=blue,line cap=rect,line join=bevel,line width=0.800pt]
  \end{scope}
  \begin{scope}[scale=1.006,draw=blue,line cap=rect,line join=bevel,line width=0.800pt]
  \end{scope}
  \begin{scope}[cm={{1.26012,0.0,0.0,1.26012,(-619.61892,-54.78689)}},draw=blue,line cap=round,line join=round,line width=0.480pt]
    \path[draw] (165.3000,36.9000) -- (165.3000,36.9000) -- (165.4000,41.9000) -- (165.5000,44.7000) -- (165.7000,46.2000) -- (165.8000,46.7000) -- (166.0000,46.7000) -- (166.1000,46.7000) -- (166.2000,47.2000) -- (166.4000,47.4000) -- (166.5000,47.2000) -- (166.6000,46.7000) -- (166.8000,46.1000) -- (166.9000,45.5000) -- (167.1000,44.9000) -- (167.2000,44.5000) -- (167.3000,44.2000) -- (167.5000,44.2000) -- (167.6000,44.2000) -- (167.8000,44.2000) -- (167.9000,44.3000) -- (168.0000,44.4000) -- (168.2000,44.4000) -- (168.3000,44.5000) -- (168.4000,44.5000) -- (168.6000,44.5000) -- (168.7000,44.5000) -- (168.9000,44.5000) -- (169.0000,44.5000) -- (169.1000,44.5000) -- (169.3000,44.5000) -- (169.4000,44.5000) -- (169.5000,44.5000) -- (169.7000,44.5000) -- (169.8000,44.5000) -- (170.0000,44.5000) -- (170.1000,44.5000) -- (170.2000,44.5000) -- (170.4000,44.5000) -- (170.5000,44.5000) -- (170.7000,44.5000) -- (170.8000,44.5000) -- (170.9000,44.5000) -- (171.1000,44.5000) -- (171.2000,44.5000) -- (171.3000,44.5000) -- (171.5000,44.5000) -- (171.6000,44.5000) -- (171.8000,44.6000) -- (171.9000,44.7000) -- (172.0000,44.9000) -- (172.2000,45.2000) -- (172.3000,45.5000) -- (172.5000,45.8000) -- (172.6000,46.3000) -- (172.7000,46.7000) -- (172.9000,47.3000) -- (173.0000,47.8000) -- (173.1000,48.5000) -- (173.3000,49.1000) -- (173.4000,49.9000) -- (173.6000,50.7000) -- (173.7000,51.5000) -- (173.8000,52.4000) -- (174.0000,53.4000) -- (174.1000,54.4000) -- (174.2000,55.4000) -- (174.4000,56.5000) -- (174.5000,57.7000) -- (174.7000,58.9000) -- (174.8000,60.2000) -- (174.9000,61.5000) -- (175.1000,62.8000) -- (175.2000,64.2000) -- (175.4000,65.6000) -- (175.5000,67.1000) -- (175.6000,68.5000) -- (175.8000,70.0000) -- (175.9000,71.5000) -- (176.0000,73.0000) -- (176.2000,74.5000) -- (176.3000,76.0000) -- (176.5000,77.4000) -- (176.6000,78.8000) -- (176.7000,80.2000) -- (176.9000,81.6000) -- (177.0000,83.1000) -- (177.1000,84.5000) -- (177.3000,85.7000) -- (177.4000,86.5000) -- (177.6000,87.0000) -- (177.7000,87.2000) -- (177.8000,87.2000) -- (178.0000,87.1000) -- (178.1000,86.9000) -- (178.3000,86.7000) -- (178.4000,86.6000) -- (178.5000,86.5000) -- (178.7000,86.4000) -- (178.8000,86.3000) -- (178.9000,86.3000) -- (179.1000,86.3000) -- (179.2000,86.3000) -- (179.4000,86.3000) -- (179.5000,86.3000) -- (179.6000,86.3000) -- (179.8000,86.3000) -- (179.9000,86.3000) -- (180.0000,86.3000) -- (180.2000,86.3000) -- (180.3000,86.3000) -- (180.5000,86.3000) -- (180.6000,86.3000) -- (180.7000,86.3000) -- (180.9000,86.3000) -- (181.0000,86.3000) -- (181.2000,86.3000) -- (181.3000,86.3000) -- (181.4000,86.2000) -- (181.6000,86.0000) -- (181.7000,85.8000) -- (181.8000,85.9000) -- (182.0000,86.1000) -- (182.1000,86.2000) -- (182.3000,86.0000) -- (182.4000,85.7000) -- (182.5000,85.0000) -- (182.7000,84.0000) -- (182.8000,82.8000) -- (183.0000,81.4000) -- (183.1000,79.9000) -- (183.2000,78.2000) -- (183.4000,76.4000) -- (183.5000,74.6000) -- (183.6000,72.8000) -- (183.8000,70.9000) -- (183.9000,69.1000) -- (184.1000,67.3000) -- (184.2000,65.5000) -- (184.3000,63.8000) -- (184.5000,62.1000) -- (184.6000,60.5000) -- (184.7000,59.0000) -- (184.9000,57.6000) -- (185.0000,56.2000) -- (185.2000,54.9000) -- (185.3000,53.7000) -- (185.4000,52.5000) -- (185.6000,51.5000) -- (185.7000,50.5000) -- (185.9000,49.6000) -- (186.0000,48.7000) -- (186.1000,48.0000) -- (186.3000,47.3000) -- (186.4000,46.7000) -- (186.5000,46.1000) -- (186.7000,45.6000) -- (186.8000,45.2000) -- (187.0000,44.9000) -- (187.1000,44.6000) -- (187.2000,44.3000) -- (187.4000,44.1000) -- (187.5000,44.1000) -- (187.6000,44.1000) -- (187.8000,44.2000) -- (187.9000,44.3000) -- (188.1000,44.4000) -- (188.2000,44.4000) -- (188.3000,44.5000) -- (188.5000,44.5000) -- (188.6000,44.5000) -- (188.8000,44.5000) -- (188.9000,44.5000) -- (189.0000,44.5000) -- (189.2000,44.5000) -- (189.3000,44.5000) -- (189.4000,44.5000) -- (189.6000,44.5000) -- (189.7000,44.5000) -- (189.9000,44.5000) -- (190.0000,44.5000) -- (190.1000,44.5000) -- (190.3000,44.5000) -- (190.4000,44.5000) -- (190.6000,44.5000) -- (190.7000,44.5000) -- (190.8000,44.5000) -- (191.0000,44.5000) -- (191.1000,44.5000) -- (191.2000,44.5000) -- (191.4000,44.5000) -- (191.5000,44.5000) -- (191.7000,44.5000) -- (191.8000,44.5000) -- (191.9000,44.6000) -- (192.1000,44.7000) -- (192.2000,44.9000) -- (192.3000,45.1000) -- (192.5000,45.4000) -- (192.6000,45.8000) -- (192.8000,46.2000) -- (192.9000,46.6000) -- (193.0000,47.2000) -- (193.2000,47.7000) -- (193.3000,48.3000) -- (193.5000,49.0000) -- (193.6000,49.7000) -- (193.7000,50.5000) -- (193.9000,51.3000) -- (194.0000,52.2000) -- (194.1000,53.1000) -- (194.3000,54.1000) -- (194.4000,55.1000) -- (194.6000,56.2000) -- (194.7000,57.3000) -- (194.8000,58.5000) -- (195.0000,59.8000) -- (195.1000,61.0000) -- (195.2000,62.4000) -- (195.4000,63.7000) -- (195.5000,65.1000) -- (195.7000,66.5000) -- (195.8000,68.0000) -- (195.9000,69.5000) -- (196.1000,71.0000) -- (196.2000,72.4000) -- (196.4000,73.9000) -- (196.5000,75.4000) -- (196.6000,76.8000) -- (196.8000,78.2000) -- (196.9000,79.6000) -- (197.0000,81.0000) -- (197.2000,82.5000) -- (197.3000,84.0000) -- (197.5000,85.3000) -- (197.6000,86.3000) -- (197.7000,86.9000) -- (197.9000,87.2000) -- (198.0000,87.3000) -- (198.1000,87.2000) -- (198.3000,87.0000) -- (198.4000,86.8000) -- (198.6000,86.6000) -- (198.7000,86.5000) -- (198.8000,86.4000) -- (199.0000,86.3000) -- (199.1000,86.3000) -- (199.3000,86.3000) -- (199.4000,86.3000) -- (199.5000,86.3000) -- (199.7000,86.3000) -- (199.8000,86.3000) -- (199.9000,86.3000) -- (200.1000,86.3000) -- (200.2000,86.3000) -- (200.4000,86.3000) -- (200.5000,86.3000) -- (200.6000,86.3000) -- (200.8000,86.3000) -- (200.9000,86.3000) -- (201.1000,86.3000) -- (201.2000,86.3000) -- (201.3000,86.3000) -- (201.5000,86.3000) -- (201.6000,86.3000) -- (201.7000,86.1000) -- (201.9000,85.9000) -- (202.0000,85.9000) -- (202.2000,86.0000) -- (202.3000,86.1000) -- (202.4000,86.1000) -- (202.6000,85.9000) -- (202.7000,85.3000) -- (202.8000,84.5000) -- (203.0000,83.5000) -- (203.1000,82.2000) -- (203.3000,80.7000) -- (203.4000,79.1000) -- (203.5000,77.3000) -- (203.7000,75.5000) -- (203.8000,73.7000) -- (204.0000,71.9000) -- (204.1000,70.0000) -- (204.2000,68.2000) -- (204.4000,66.4000) -- (204.5000,64.7000) -- (204.6000,63.0000) -- (204.8000,61.4000) -- (204.9000,59.8000) -- (205.1000,58.3000) -- (205.2000,56.9000) -- (205.3000,55.5000) -- (205.5000,54.3000) -- (205.6000,53.1000) -- (205.7000,52.0000) -- (205.9000,51.0000) -- (206.0000,50.0000) -- (206.2000,49.1000) -- (206.3000,48.4000) -- (206.4000,47.6000) -- (206.6000,47.0000) -- (206.7000,46.4000) -- (206.9000,45.9000) -- (207.0000,45.4000) -- (207.1000,45.1000) -- (207.3000,44.8000) -- (207.4000,44.5000) -- (207.5000,44.2000) -- (207.7000,44.1000) -- (207.8000,44.1000) -- (208.0000,44.1000) -- (208.1000,44.2000) -- (208.2000,44.3000) -- (208.4000,44.4000) -- (208.5000,44.4000) -- (208.6000,44.5000) -- (208.8000,44.5000) -- (208.9000,44.5000) -- (209.1000,44.5000) -- (209.2000,44.5000) -- (209.3000,44.5000) -- (209.5000,44.5000) -- (209.6000,44.5000) -- (209.8000,44.5000) -- (209.9000,44.5000) -- (210.0000,44.5000) -- (210.2000,44.5000) -- (210.3000,44.5000) -- (210.4000,44.5000) -- (210.6000,44.5000) -- (210.7000,44.5000) -- (210.9000,44.5000) -- (211.0000,44.5000) -- (211.1000,44.5000) -- (211.3000,44.5000) -- (211.4000,44.5000) -- (211.5000,44.5000) -- (211.7000,44.5000) -- (211.8000,44.5000) -- (212.0000,44.5000) -- (212.1000,44.6000) -- (212.2000,44.7000) -- (212.4000,44.8000) -- (212.5000,45.0000) -- (212.7000,45.3000) -- (212.8000,45.6000) -- (212.9000,45.9000) -- (213.1000,46.4000) -- (213.2000,46.9000) -- (213.3000,47.4000) -- (213.5000,48.0000) -- (213.6000,48.6000) -- (213.8000,49.3000) -- (213.9000,50.0000) -- (214.0000,50.8000) -- (214.2000,51.7000) -- (214.3000,52.6000) -- (214.5000,53.5000) -- (214.6000,54.5000) -- (214.7000,55.6000) -- (214.9000,56.7000) -- (215.0000,57.8000) -- (215.1000,59.0000) -- (215.3000,60.3000) -- (215.4000,61.6000) -- (215.6000,62.9000) -- (215.7000,64.3000) -- (215.8000,65.7000) -- (216.0000,67.1000) -- (216.1000,68.6000) -- (216.2000,70.1000) -- (216.4000,71.6000) -- (216.5000,73.0000) -- (216.7000,74.5000) -- (216.8000,76.0000) -- (216.9000,77.4000) -- (217.1000,78.8000) -- (217.2000,80.1000) -- (217.4000,81.6000) -- (217.5000,83.1000) -- (217.6000,84.6000) -- (217.8000,85.8000) -- (217.9000,86.6000) -- (218.0000,87.0000) -- (218.2000,87.2000) -- (218.3000,87.2000) -- (218.5000,87.1000) -- (218.6000,86.9000) -- (218.7000,86.7000) -- (218.9000,86.6000) -- (219.0000,86.5000) -- (219.1000,86.4000) -- (219.3000,86.3000) -- (219.4000,86.3000) -- (219.6000,86.3000) -- (219.7000,86.3000) -- (219.8000,86.3000) -- (220.0000,86.3000) -- (220.1000,86.3000) -- (220.3000,86.3000) -- (220.4000,86.3000) -- (220.5000,86.3000) -- (220.7000,86.3000) -- (220.8000,86.3000) -- (220.9000,86.3000) -- (221.1000,86.3000) -- (221.2000,86.3000) -- (221.4000,86.3000) -- (221.5000,86.3000) -- (221.6000,86.3000) -- (221.8000,86.3000) -- (221.9000,86.2000) -- (222.0000,86.0000) -- (222.2000,85.9000) -- (222.3000,85.9000) -- (222.5000,86.1000) -- (222.6000,86.2000) -- (222.7000,86.1000) -- (222.9000,85.7000) -- (223.0000,85.1000) -- (223.2000,84.1000) -- (223.3000,83.0000) -- (223.4000,81.6000) -- (223.6000,80.1000) -- (223.7000,78.4000) -- (223.8000,76.6000) -- (224.0000,74.8000) -- (224.1000,73.0000) -- (224.3000,71.2000) -- (224.4000,69.3000) -- (224.5000,67.5000) -- (224.7000,65.7000) -- (224.8000,64.0000) -- (225.0000,62.3000) -- (225.1000,60.7000) -- (225.2000,59.2000) -- (225.4000,57.7000) -- (225.5000,56.3000) -- (225.6000,55.0000) -- (225.8000,53.8000) -- (225.9000,52.7000) -- (226.1000,51.6000) -- (226.2000,50.6000) -- (226.3000,49.7000) -- (226.5000,48.8000) -- (226.6000,48.1000) -- (226.7000,47.4000) -- (226.9000,46.7000) -- (227.0000,46.2000) -- (227.2000,45.7000) -- (227.3000,45.3000) -- (227.4000,44.9000) -- (227.6000,44.7000) -- (227.7000,44.4000) -- (227.9000,44.2000) -- (228.0000,44.1000) -- (228.1000,44.1000) -- (228.3000,44.2000) -- (228.4000,44.3000) -- (228.5000,44.3000) -- (228.7000,44.4000) -- (228.8000,44.5000) -- (229.0000,44.5000) -- (229.1000,44.5000) -- (229.2000,44.5000) -- (229.4000,44.5000) -- (229.5000,44.5000) -- (229.6000,44.5000) -- (229.8000,44.5000) -- (229.9000,44.5000) -- (230.1000,44.5000) -- (230.2000,44.5000) -- (230.3000,44.5000) -- (230.5000,44.5000) -- (230.6000,44.5000) -- (230.8000,44.5000) -- (230.9000,44.5000) -- (231.0000,44.5000) -- (231.2000,44.5000) -- (231.3000,44.5000) -- (231.4000,44.5000) -- (231.6000,44.5000) -- (231.7000,44.5000) -- (231.9000,44.5000) -- (232.0000,44.5000) -- (232.1000,44.5000) -- (232.3000,44.5000) -- (232.4000,44.6000) -- (232.6000,44.7000) -- (232.7000,44.9000) -- (232.8000,45.1000) -- (233.0000,45.4000) -- (233.1000,45.7000) -- (233.2000,46.1000) -- (233.4000,46.6000) -- (233.5000,47.1000) -- (233.7000,47.6000) -- (233.8000,48.2000) -- (233.9000,48.9000) -- (234.1000,49.6000) -- (234.2000,50.3000) -- (234.3000,51.1000) -- (234.5000,52.0000) -- (234.6000,52.9000) -- (234.8000,53.9000) -- (234.9000,54.9000) -- (235.0000,56.0000) -- (235.2000,57.1000) -- (235.3000,58.3000) -- (235.5000,59.5000) -- (235.6000,60.8000) -- (235.7000,62.1000) -- (235.9000,63.4000) -- (236.0000,64.8000) -- (236.1000,66.3000) -- (236.3000,67.7000) -- (236.4000,69.2000) -- (236.6000,70.6000) -- (236.7000,72.1000) -- (236.8000,73.6000) -- (237.0000,75.1000) -- (237.1000,76.5000) -- (237.2000,77.9000) -- (237.4000,79.3000) -- (237.5000,80.7000) -- (237.7000,82.2000) -- (237.8000,83.7000) -- (237.9000,85.1000) -- (238.1000,86.1000) -- (238.2000,86.8000) -- (238.4000,87.2000) -- (238.5000,87.3000) -- (238.6000,87.2000) -- (238.8000,87.0000) -- (238.9000,86.9000) -- (239.0000,86.7000) -- (239.2000,86.5000) -- (239.3000,86.4000) -- (239.5000,86.3000) -- (239.6000,86.3000) -- (239.7000,86.3000) -- (239.9000,86.3000) -- (240.0000,86.3000) -- (240.2000,86.3000) -- (240.3000,86.3000) -- (240.4000,86.3000) -- (240.6000,86.3000) -- (240.7000,86.3000) -- (240.8000,86.3000) -- (241.0000,86.3000) -- (241.1000,86.3000) -- (241.3000,86.3000) -- (241.4000,86.3000) -- (241.5000,86.3000) -- (241.7000,86.3000) -- (241.8000,86.3000) -- (241.9000,86.3000) -- (242.1000,86.3000) -- (242.2000,86.2000) -- (242.4000,86.0000) -- (242.5000,85.9000) -- (242.6000,86.0000) -- (242.8000,86.1000) -- (242.9000,86.1000) -- (243.1000,85.9000) -- (243.2000,85.5000) -- (243.3000,84.7000) -- (243.5000,83.7000) -- (243.6000,82.5000) -- (243.7000,81.0000) -- (243.9000,79.4000) -- (244.0000,77.7000) -- (244.2000,75.9000) -- (244.3000,74.1000) -- (244.4000,72.3000) -- (244.6000,70.4000) -- (244.7000,68.6000) -- (244.8000,66.8000) -- (245.0000,65.1000) -- (245.1000,63.4000) -- (245.3000,61.7000) -- (245.4000,60.1000) -- (245.5000,58.6000) -- (245.7000,57.2000) -- (245.8000,55.8000) -- (246.0000,54.6000) -- (246.1000,53.3000) -- (246.2000,52.2000) -- (246.4000,51.2000) -- (246.5000,50.2000) -- (246.6000,49.3000) -- (246.8000,48.5000) -- (246.9000,47.8000) -- (247.1000,47.1000) -- (247.2000,46.5000) -- (247.3000,46.0000) -- (247.5000,45.5000) -- (247.6000,45.1000) -- (247.7000,44.8000) -- (247.9000,44.5000) -- (248.0000,44.3000) -- (248.2000,44.1000) -- (248.3000,44.1000) -- (248.4000,44.1000) -- (248.6000,44.2000) -- (248.7000,44.3000) -- (248.9000,44.4000) -- (249.0000,44.4000) -- (249.1000,44.5000) -- (249.3000,44.5000) -- (249.4000,44.5000) -- (249.5000,44.5000) -- (249.7000,44.5000) -- (249.8000,44.5000) -- (250.0000,44.5000) -- (250.1000,44.5000) -- (250.2000,44.5000) -- (250.4000,44.5000) -- (250.5000,44.5000) -- (250.7000,44.5000) -- (250.8000,44.5000) -- (250.9000,44.5000) -- (251.1000,44.5000) -- (251.2000,44.5000) -- (251.3000,44.5000) -- (251.5000,44.5000) -- (251.6000,44.5000) -- (251.8000,44.5000) -- (251.9000,44.5000) -- (251.9000,44.5000);



  \end{scope}
  \begin{scope}[scale=1.006,draw=blue,line cap=rect,line join=bevel,line width=0.800pt]
  \end{scope}
  \begin{scope}[scale=1.006,draw=blue,line cap=rect,line join=bevel,line width=0.800pt]
  \end{scope}
  \begin{scope}[scale=1.006,draw=blue,line cap=rect,line join=bevel,line width=0.800pt]
  \end{scope}
  \begin{scope}[scale=1.006,draw=blue,line cap=rect,line join=bevel,line width=0.800pt]
  \end{scope}
  \begin{scope}[cm={{1.26012,0.0,0.0,1.26012,(-619.61892,-54.78689)}},draw=cff0000,line cap=round,line join=round,line width=0.480pt]
    \path[draw] (165.2000,51.8000) -- (165.2000,51.8000) -- (165.3000,51.0000) -- (165.4000,50.3000) -- (165.5000,49.6000) -- (165.5000,48.9000) -- (165.6000,48.3000) -- (165.7000,47.8000) -- (165.8000,47.2000) -- (165.9000,46.8000) -- (166.0000,46.4000) -- (166.1000,46.0000) -- (166.2000,45.7000) -- (166.2000,45.4000) -- (166.3000,45.2000) -- (166.4000,45.0000) -- (166.5000,44.8000) -- (166.6000,44.7000) -- (166.7000,44.7000) -- (166.8000,44.7000) -- (166.8000,44.7000) -- (166.9000,44.7000) -- (167.0000,44.8000) -- (167.1000,44.9000) -- (167.2000,45.1000) -- (167.3000,45.3000) -- (167.4000,45.4000) -- (167.5000,45.6000) -- (167.5000,45.9000) -- (167.6000,46.1000) -- (167.7000,46.4000) -- (167.8000,46.6000) -- (167.9000,46.9000) -- (168.0000,47.1000) -- (168.1000,47.4000) -- (168.2000,47.6000) -- (168.2000,47.8000) -- (168.3000,48.0000) -- (168.4000,48.2000) -- (168.5000,48.4000) -- (168.6000,48.6000) -- (168.7000,48.8000) -- (168.8000,48.9000) -- (168.8000,49.0000) -- (168.9000,49.1000) -- (169.0000,49.1000) -- (169.1000,49.1000) -- (169.2000,49.1000) -- (169.3000,49.1000) -- (169.4000,49.0000) -- (169.5000,48.9000) -- (169.5000,48.8000) -- (169.6000,48.6000) -- (169.7000,48.5000) -- (169.8000,48.3000) -- (169.9000,48.0000) -- (170.0000,47.8000) -- (170.1000,47.5000) -- (170.1000,47.2000) -- (170.2000,46.9000) -- (170.3000,46.6000) -- (170.4000,46.2000) -- (170.5000,45.9000) -- (170.6000,45.5000) -- (170.7000,45.1000) -- (170.8000,44.8000) -- (170.8000,44.4000) -- (170.9000,44.0000) -- (171.0000,43.7000) -- (171.1000,43.4000) -- (171.2000,43.0000) -- (171.3000,42.7000) -- (171.4000,42.5000) -- (171.4000,42.2000) -- (171.5000,42.0000) -- (171.6000,41.8000) -- (171.7000,41.6000) -- (171.8000,41.5000) -- (171.9000,41.4000) -- (172.0000,41.3000) -- (172.1000,41.3000) -- (172.1000,41.3000) -- (172.2000,41.4000) -- (172.3000,41.6000) -- (172.4000,41.7000) -- (172.5000,42.0000) -- (172.6000,42.3000) -- (172.7000,42.6000) -- (172.8000,43.0000) -- (172.8000,43.4000) -- (172.9000,43.9000) -- (173.0000,44.5000) -- (173.1000,45.1000) -- (173.2000,45.7000) -- (173.3000,46.4000) -- (173.4000,47.1000) -- (173.4000,47.9000) -- (173.5000,48.8000) -- (173.6000,49.6000) -- (173.7000,50.5000) -- (173.8000,51.5000) -- (173.9000,52.5000) -- (174.0000,53.5000) -- (174.1000,54.5000) -- (174.1000,55.6000) -- (174.2000,56.6000) -- (174.3000,57.7000) -- (174.4000,58.9000) -- (174.5000,60.0000) -- (174.6000,61.1000) -- (174.7000,62.2000) -- (174.7000,63.4000) -- (174.8000,64.5000) -- (174.9000,65.6000) -- (175.0000,66.8000) -- (175.1000,67.8000) -- (175.2000,68.9000) -- (175.3000,70.0000) -- (175.4000,71.0000) -- (175.4000,72.1000) -- (175.5000,73.0000) -- (175.6000,74.0000) -- (175.7000,74.9000) -- (175.8000,75.8000) -- (175.9000,76.7000) -- (176.0000,77.5000) -- (176.0000,78.3000) -- (176.1000,79.0000) -- (176.2000,79.7000) -- (176.3000,80.3000) -- (176.4000,80.9000) -- (176.5000,81.5000) -- (176.6000,82.0000) -- (176.7000,82.5000) -- (176.7000,83.0000) -- (176.8000,83.4000) -- (176.9000,83.7000) -- (177.0000,84.0000) -- (177.1000,84.3000) -- (177.2000,84.6000) -- (177.3000,84.8000) -- (177.4000,85.0000) -- (177.4000,85.1000) -- (177.5000,85.3000) -- (177.6000,85.3000) -- (177.7000,85.4000) -- (177.8000,85.5000) -- (177.9000,85.5000) -- (178.0000,85.5000) -- (178.0000,85.5000) -- (178.1000,85.5000) -- (178.2000,85.5000) -- (178.3000,85.4000) -- (178.4000,85.4000) -- (178.5000,85.4000) -- (178.6000,85.3000) -- (178.7000,85.3000) -- (178.7000,85.2000) -- (178.8000,85.2000) -- (178.9000,85.2000) -- (179.0000,85.1000) -- (179.1000,85.1000) -- (179.2000,85.1000) -- (179.3000,85.1000) -- (179.3000,85.1000) -- (179.4000,85.1000) -- (179.5000,85.1000) -- (179.6000,85.2000) -- (179.7000,85.2000) -- (179.8000,85.3000) -- (179.9000,85.4000) -- (180.0000,85.4000) -- (180.0000,85.5000) -- (180.1000,85.6000) -- (180.2000,85.7000) -- (180.3000,85.8000) -- (180.4000,85.9000) -- (180.5000,86.0000) -- (180.6000,86.1000) -- (180.6000,86.2000) -- (180.7000,86.3000) -- (180.8000,86.4000) -- (180.9000,86.4000) -- (181.0000,86.5000) -- (181.1000,86.5000) -- (181.2000,86.6000) -- (181.3000,86.6000) -- (181.3000,86.6000) -- (181.4000,86.5000) -- (181.5000,86.4000) -- (181.6000,86.4000) -- (181.7000,86.2000) -- (181.8000,86.1000) -- (181.9000,85.9000) -- (182.0000,85.6000) -- (182.0000,85.4000) -- (182.1000,85.1000) -- (182.2000,84.7000) -- (182.3000,84.3000) -- (182.4000,83.9000) -- (182.5000,83.5000) -- (182.6000,82.9000) -- (182.6000,82.4000) -- (182.7000,81.8000) -- (182.8000,81.2000) -- (182.9000,80.5000) -- (183.0000,79.8000) -- (183.1000,79.0000) -- (183.2000,78.2000) -- (183.3000,77.4000) -- (183.3000,76.5000) -- (183.4000,75.6000) -- (183.5000,74.7000) -- (183.6000,73.7000) -- (183.7000,72.7000) -- (183.8000,71.7000) -- (183.9000,70.7000) -- (183.9000,69.6000) -- (184.0000,68.6000) -- (184.1000,67.5000) -- (184.2000,66.4000) -- (184.3000,65.3000) -- (184.4000,64.2000) -- (184.5000,63.1000) -- (184.6000,62.1000) -- (184.6000,61.0000) -- (184.7000,59.9000) -- (184.8000,58.9000) -- (184.9000,57.9000) -- (185.0000,56.9000) -- (185.1000,55.9000) -- (185.2000,54.9000) -- (185.2000,54.0000) -- (185.3000,53.1000) -- (185.4000,52.3000) -- (185.5000,51.5000) -- (185.6000,50.7000) -- (185.7000,50.0000) -- (185.8000,49.3000) -- (185.9000,48.7000) -- (185.9000,48.1000) -- (186.0000,47.5000) -- (186.1000,47.0000) -- (186.2000,46.6000) -- (186.3000,46.2000) -- (186.4000,45.8000) -- (186.5000,45.5000) -- (186.5000,45.3000) -- (186.6000,45.0000) -- (186.7000,44.9000) -- (186.8000,44.8000) -- (186.9000,44.7000) -- (187.0000,44.6000) -- (187.1000,44.6000) -- (187.2000,44.7000) -- (187.2000,44.7000) -- (187.3000,44.8000) -- (187.4000,45.0000) -- (187.5000,45.1000) -- (187.6000,45.3000) -- (187.7000,45.5000) -- (187.8000,45.7000) -- (187.9000,46.0000) -- (187.9000,46.2000) -- (188.0000,46.4000) -- (188.1000,46.7000) -- (188.2000,47.0000) -- (188.3000,47.2000) -- (188.4000,47.4000) -- (188.5000,47.7000) -- (188.5000,47.9000) -- (188.6000,48.1000) -- (188.7000,48.3000) -- (188.8000,48.5000) -- (188.9000,48.7000) -- (189.0000,48.8000) -- (189.1000,48.9000) -- (189.2000,49.0000) -- (189.2000,49.1000) -- (189.3000,49.1000) -- (189.4000,49.1000) -- (189.5000,49.1000) -- (189.6000,49.1000) -- (189.7000,49.0000) -- (189.8000,48.9000) -- (189.8000,48.8000) -- (189.9000,48.6000) -- (190.0000,48.4000) -- (190.1000,48.2000) -- (190.2000,47.9000) -- (190.3000,47.7000) -- (190.4000,47.4000) -- (190.5000,47.1000) -- (190.5000,46.8000) -- (190.6000,46.4000) -- (190.7000,46.1000) -- (190.8000,45.7000) -- (190.9000,45.4000) -- (191.0000,45.0000) -- (191.1000,44.6000) -- (191.1000,44.3000) -- (191.2000,43.9000) -- (191.3000,43.6000) -- (191.4000,43.2000) -- (191.5000,42.9000) -- (191.6000,42.6000) -- (191.7000,42.3000) -- (191.8000,42.1000) -- (191.8000,41.9000) -- (191.9000,41.7000) -- (192.0000,41.5000) -- (192.1000,41.4000) -- (192.2000,41.3000) -- (192.3000,41.3000) -- (192.4000,41.3000) -- (192.5000,41.3000) -- (192.5000,41.4000) -- (192.6000,41.6000) -- (192.7000,41.8000) -- (192.8000,42.0000) -- (192.9000,42.4000) -- (193.0000,42.7000) -- (193.1000,43.1000) -- (193.1000,43.6000) -- (193.2000,44.1000) -- (193.3000,44.7000) -- (193.4000,45.3000) -- (193.5000,45.9000) -- (193.6000,46.7000) -- (193.7000,47.4000) -- (193.8000,48.2000) -- (193.8000,49.1000) -- (193.9000,50.0000) -- (194.0000,50.9000) -- (194.1000,51.8000) -- (194.2000,52.8000) -- (194.3000,53.9000) -- (194.4000,54.9000) -- (194.4000,56.0000) -- (194.5000,57.1000) -- (194.6000,58.2000) -- (194.7000,59.3000) -- (194.8000,60.4000) -- (194.9000,61.6000) -- (195.0000,62.7000) -- (195.1000,63.8000) -- (195.1000,65.0000) -- (195.2000,66.1000) -- (195.3000,67.2000) -- (195.4000,68.3000) -- (195.5000,69.4000) -- (195.6000,70.4000) -- (195.7000,71.5000) -- (195.7000,72.5000) -- (195.8000,73.4000) -- (195.9000,74.4000) -- (196.0000,75.3000) -- (196.1000,76.2000) -- (196.2000,77.0000) -- (196.3000,77.8000) -- (196.4000,78.6000) -- (196.4000,79.3000) -- (196.5000,80.0000) -- (196.6000,80.6000) -- (196.7000,81.2000) -- (196.8000,81.7000) -- (196.9000,82.3000) -- (197.0000,82.7000) -- (197.1000,83.2000) -- (197.1000,83.5000) -- (197.2000,83.9000) -- (197.3000,84.2000) -- (197.4000,84.5000) -- (197.5000,84.7000) -- (197.6000,84.9000) -- (197.7000,85.1000) -- (197.7000,85.2000) -- (197.8000,85.3000) -- (197.9000,85.4000) -- (198.0000,85.5000) -- (198.1000,85.5000) -- (198.2000,85.5000) -- (198.3000,85.5000) -- (198.4000,85.5000) -- (198.4000,85.5000) -- (198.5000,85.5000) -- (198.6000,85.4000) -- (198.7000,85.4000) -- (198.8000,85.3000) -- (198.9000,85.3000) -- (199.0000,85.2000) -- (199.0000,85.2000) -- (199.1000,85.2000) -- (199.2000,85.1000) -- (199.3000,85.1000) -- (199.4000,85.1000) -- (199.5000,85.1000) -- (199.6000,85.1000) -- (199.7000,85.1000) -- (199.7000,85.1000) -- (199.8000,85.2000) -- (199.9000,85.2000) -- (200.0000,85.2000) -- (200.1000,85.3000) -- (200.2000,85.4000) -- (200.3000,85.5000) -- (200.3000,85.5000) -- (200.4000,85.6000) -- (200.5000,85.7000) -- (200.6000,85.8000) -- (200.7000,85.9000) -- (200.8000,86.0000) -- (200.9000,86.1000) -- (201.0000,86.2000) -- (201.0000,86.3000) -- (201.1000,86.4000) -- (201.2000,86.5000) -- (201.3000,86.5000) -- (201.4000,86.6000) -- (201.5000,86.6000) -- (201.6000,86.6000) -- (201.7000,86.6000) -- (201.7000,86.5000) -- (201.8000,86.4000) -- (201.9000,86.3000) -- (202.0000,86.2000) -- (202.1000,86.0000) -- (202.2000,85.8000) -- (202.3000,85.6000) -- (202.3000,85.3000) -- (202.4000,85.0000) -- (202.5000,84.6000) -- (202.6000,84.2000) -- (202.7000,83.8000) -- (202.8000,83.3000) -- (202.9000,82.8000) -- (203.0000,82.2000) -- (203.0000,81.6000) -- (203.1000,80.9000) -- (203.2000,80.2000) -- (203.3000,79.5000) -- (203.4000,78.7000) -- (203.5000,77.9000) -- (203.6000,77.0000) -- (203.6000,76.2000) -- (203.7000,75.2000) -- (203.8000,74.3000) -- (203.9000,73.3000) -- (204.0000,72.3000) -- (204.1000,71.3000) -- (204.2000,70.3000) -- (204.3000,69.2000) -- (204.3000,68.1000) -- (204.4000,67.1000) -- (204.5000,66.0000) -- (204.6000,64.9000) -- (204.7000,63.8000) -- (204.8000,62.7000) -- (204.9000,61.6000) -- (204.9000,60.6000) -- (205.0000,59.5000) -- (205.1000,58.5000) -- (205.2000,57.5000) -- (205.3000,56.5000) -- (205.4000,55.5000) -- (205.5000,54.6000) -- (205.6000,53.7000) -- (205.6000,52.8000) -- (205.7000,51.9000) -- (205.8000,51.1000) -- (205.9000,50.4000) -- (206.0000,49.7000) -- (206.1000,49.0000) -- (206.2000,48.4000) -- (206.3000,47.8000) -- (206.3000,47.3000) -- (206.4000,46.8000) -- (206.5000,46.4000) -- (206.6000,46.0000) -- (206.7000,45.7000) -- (206.8000,45.4000) -- (206.9000,45.1000) -- (206.9000,45.0000) -- (207.0000,44.8000) -- (207.1000,44.7000) -- (207.2000,44.6000) -- (207.3000,44.6000) -- (207.4000,44.6000) -- (207.5000,44.7000) -- (207.6000,44.8000) -- (207.6000,44.9000) -- (207.7000,45.0000) -- (207.8000,45.2000) -- (207.9000,45.4000) -- (208.0000,45.6000) -- (208.1000,45.8000) -- (208.2000,46.0000) -- (208.2000,46.3000) -- (208.3000,46.5000) -- (208.4000,46.8000) -- (208.5000,47.1000) -- (208.6000,47.3000) -- (208.7000,47.5000) -- (208.8000,47.8000) -- (208.9000,48.0000) -- (208.9000,48.2000) -- (209.0000,48.4000) -- (209.1000,48.6000) -- (209.2000,48.8000) -- (209.3000,48.9000) -- (209.4000,49.0000) -- (209.5000,49.1000) -- (209.5000,49.1000) -- (209.6000,49.2000) -- (209.7000,49.2000) -- (209.8000,49.1000) -- (209.9000,49.1000) -- (210.0000,49.0000) -- (210.1000,48.9000) -- (210.2000,48.7000) -- (210.2000,48.5000) -- (210.3000,48.3000) -- (210.4000,48.1000) -- (210.5000,47.9000) -- (210.6000,47.6000) -- (210.7000,47.3000) -- (210.8000,47.0000) -- (210.9000,46.6000) -- (210.9000,46.3000) -- (211.0000,45.9000) -- (211.1000,45.6000) -- (211.2000,45.2000) -- (211.3000,44.8000) -- (211.4000,44.5000) -- (211.5000,44.1000) -- (211.5000,43.8000) -- (211.6000,43.4000) -- (211.7000,43.1000) -- (211.8000,42.8000) -- (211.9000,42.5000) -- (212.0000,42.2000) -- (212.1000,42.0000) -- (212.2000,41.8000) -- (212.2000,41.6000) -- (212.3000,41.4000) -- (212.4000,41.3000) -- (212.5000,41.3000) -- (212.6000,41.3000) -- (212.7000,41.3000) -- (212.8000,41.4000) -- (212.8000,41.5000) -- (212.9000,41.6000) -- (213.0000,41.9000) -- (213.1000,42.1000) -- (213.2000,42.5000) -- (213.3000,42.8000) -- (213.4000,43.3000) -- (213.5000,43.7000) -- (213.5000,44.3000) -- (213.6000,44.9000) -- (213.7000,45.5000) -- (213.8000,46.2000) -- (213.9000,46.9000) -- (214.0000,47.7000) -- (214.1000,48.5000) -- (214.1000,49.4000) -- (214.2000,50.3000) -- (214.3000,51.2000) -- (214.4000,52.2000) -- (214.5000,53.2000) -- (214.6000,54.2000) -- (214.7000,55.3000) -- (214.8000,56.4000) -- (214.8000,57.5000) -- (214.9000,58.6000) -- (215.0000,59.7000) -- (215.1000,60.9000) -- (215.2000,62.0000) -- (215.3000,63.1000) -- (215.4000,64.3000) -- (215.5000,65.4000) -- (215.5000,66.5000) -- (215.6000,67.6000) -- (215.7000,68.7000) -- (215.8000,69.8000) -- (215.9000,70.8000) -- (216.0000,71.9000) -- (216.1000,72.9000) -- (216.1000,73.8000) -- (216.2000,74.8000) -- (216.3000,75.7000) -- (216.4000,76.5000) -- (216.5000,77.4000) -- (216.6000,78.1000) -- (216.7000,78.9000) -- (216.8000,79.6000) -- (216.8000,80.2000) -- (216.9000,80.9000) -- (217.0000,81.4000) -- (217.1000,82.0000) -- (217.2000,82.5000) -- (217.3000,82.9000) -- (217.4000,83.3000) -- (217.4000,83.7000) -- (217.5000,84.0000) -- (217.6000,84.3000) -- (217.7000,84.6000) -- (217.8000,84.8000) -- (217.9000,85.0000) -- (218.0000,85.1000) -- (218.1000,85.3000) -- (218.1000,85.4000) -- (218.2000,85.4000) -- (218.3000,85.5000) -- (218.4000,85.5000) -- (218.5000,85.5000) -- (218.6000,85.5000) -- (218.7000,85.5000) -- (218.7000,85.5000) -- (218.8000,85.5000) -- (218.9000,85.4000) -- (219.0000,85.4000) -- (219.1000,85.3000) -- (219.2000,85.3000) -- (219.3000,85.2000) -- (219.4000,85.2000) -- (219.4000,85.2000) -- (219.5000,85.1000) -- (219.6000,85.1000) -- (219.7000,85.1000) -- (219.8000,85.1000) -- (219.9000,85.1000) -- (220.0000,85.1000) -- (220.1000,85.1000) -- (220.1000,85.2000) -- (220.2000,85.2000) -- (220.3000,85.3000) -- (220.4000,85.3000) -- (220.5000,85.4000) -- (220.6000,85.5000) -- (220.7000,85.6000) -- (220.7000,85.7000) -- (220.8000,85.8000) -- (220.9000,85.9000) -- (221.0000,86.0000) -- (221.1000,86.1000) -- (221.2000,86.2000) -- (221.3000,86.3000) -- (221.4000,86.4000) -- (221.4000,86.4000) -- (221.5000,86.5000) -- (221.6000,86.5000) -- (221.7000,86.6000) -- (221.8000,86.6000) -- (221.9000,86.6000) -- (222.0000,86.6000) -- (222.0000,86.5000) -- (222.1000,86.4000) -- (222.2000,86.3000) -- (222.3000,86.1000) -- (222.4000,86.0000) -- (222.5000,85.7000) -- (222.6000,85.5000) -- (222.7000,85.2000) -- (222.7000,84.9000) -- (222.8000,84.5000) -- (222.9000,84.1000) -- (223.0000,83.6000) -- (223.1000,83.1000) -- (223.2000,82.6000) -- (223.3000,82.0000) -- (223.3000,81.3000) -- (223.4000,80.7000) -- (223.5000,80.0000) -- (223.6000,79.2000) -- (223.7000,78.4000) -- (223.8000,77.6000) -- (223.9000,76.7000) -- (224.0000,75.8000) -- (224.0000,74.9000) -- (224.1000,73.9000) -- (224.2000,73.0000) -- (224.3000,71.9000) -- (224.4000,70.9000) -- (224.5000,69.9000) -- (224.6000,68.8000) -- (224.7000,67.7000) -- (224.7000,66.6000) -- (224.8000,65.6000) -- (224.9000,64.5000) -- (225.0000,63.4000) -- (225.1000,62.3000) -- (225.2000,61.2000) -- (225.3000,60.1000) -- (225.3000,59.1000) -- (225.4000,58.1000) -- (225.5000,57.1000) -- (225.6000,56.1000) -- (225.7000,55.1000) -- (225.8000,54.2000) -- (225.9000,53.3000) -- (226.0000,52.4000) -- (226.0000,51.6000) -- (226.1000,50.8000) -- (226.2000,50.1000) -- (226.3000,49.4000) -- (226.4000,48.7000) -- (226.5000,48.1000) -- (226.6000,47.6000) -- (226.6000,47.1000) -- (226.7000,46.6000) -- (226.8000,46.2000) -- (226.9000,45.8000) -- (227.0000,45.5000) -- (227.1000,45.3000) -- (227.2000,45.0000) -- (227.3000,44.9000) -- (227.3000,44.7000) -- (227.4000,44.6000) -- (227.5000,44.6000) -- (227.6000,44.6000) -- (227.7000,44.6000) -- (227.8000,44.7000) -- (227.9000,44.8000) -- (227.9000,44.9000) -- (228.0000,45.1000) -- (228.1000,45.2000) -- (228.2000,45.4000) -- (228.3000,45.7000) -- (228.4000,45.9000) -- (228.5000,46.1000) -- (228.6000,46.4000) -- (228.6000,46.6000) -- (228.7000,46.9000) -- (228.8000,47.2000) -- (228.9000,47.4000) -- (229.0000,47.6000) -- (229.1000,47.9000) -- (229.2000,48.1000) -- (229.2000,48.3000) -- (229.3000,48.5000) -- (229.4000,48.7000) -- (229.5000,48.8000) -- (229.6000,48.9000) -- (229.7000,49.0000) -- (229.8000,49.1000) -- (229.9000,49.2000) -- (229.9000,49.2000) -- (230.0000,49.2000) -- (230.1000,49.1000) -- (230.2000,49.1000) -- (230.3000,49.0000) -- (230.4000,48.8000) -- (230.5000,48.7000) -- (230.6000,48.5000) -- (230.6000,48.3000) -- (230.7000,48.0000) -- (230.8000,47.8000) -- (230.9000,47.5000) -- (231.0000,47.2000) -- (231.1000,46.9000) -- (231.2000,46.5000) -- (231.2000,46.2000) -- (231.3000,45.8000) -- (231.4000,45.4000) -- (231.5000,45.1000) -- (231.6000,44.7000) -- (231.7000,44.3000) -- (231.8000,44.0000) -- (231.9000,43.6000) -- (231.9000,43.3000) -- (232.0000,43.0000) -- (232.1000,42.6000) -- (232.2000,42.4000) -- (232.3000,42.1000) -- (232.4000,41.9000) -- (232.5000,41.7000) -- (232.5000,41.5000) -- (232.6000,41.4000) -- (232.7000,41.3000) -- (232.8000,41.2000) -- (232.9000,41.2000) -- (233.0000,41.3000) -- (233.1000,41.4000) -- (233.2000,41.5000) -- (233.2000,41.7000) -- (233.3000,41.9000) -- (233.4000,42.2000) -- (233.5000,42.6000) -- (233.6000,43.0000) -- (233.7000,43.4000) -- (233.8000,43.9000) -- (233.8000,44.5000) -- (233.9000,45.1000) -- (234.0000,45.7000) -- (234.1000,46.4000) -- (234.2000,47.2000) -- (234.3000,48.0000) -- (234.4000,48.8000) -- (234.5000,49.7000) -- (234.5000,50.6000) -- (234.6000,51.6000) -- (234.7000,52.6000) -- (234.8000,53.6000) -- (234.9000,54.6000) -- (235.0000,55.7000) -- (235.1000,56.8000) -- (235.1000,57.9000) -- (235.2000,59.0000) -- (235.3000,60.2000) -- (235.4000,61.3000) -- (235.5000,62.4000) -- (235.6000,63.6000) -- (235.7000,64.7000) -- (235.8000,65.8000) -- (235.8000,67.0000) -- (235.9000,68.1000) -- (236.0000,69.2000) -- (236.1000,70.2000) -- (236.2000,71.3000) -- (236.3000,72.3000) -- (236.4000,73.3000) -- (236.5000,74.2000) -- (236.5000,75.1000) -- (236.6000,76.0000) -- (236.7000,76.9000) -- (236.8000,77.7000) -- (236.9000,78.5000) -- (237.0000,79.2000) -- (237.1000,79.9000) -- (237.1000,80.5000) -- (237.2000,81.1000) -- (237.3000,81.7000) -- (237.4000,82.2000) -- (237.5000,82.7000) -- (237.6000,83.1000) -- (237.7000,83.5000) -- (237.8000,83.9000) -- (237.8000,84.2000) -- (237.9000,84.4000) -- (238.0000,84.7000) -- (238.1000,84.9000) -- (238.2000,85.1000) -- (238.3000,85.2000) -- (238.4000,85.3000) -- (238.4000,85.4000) -- (238.5000,85.5000) -- (238.6000,85.5000) -- (238.7000,85.5000) -- (238.8000,85.5000) -- (238.9000,85.5000) -- (239.0000,85.5000) -- (239.1000,85.5000) -- (239.1000,85.4000) -- (239.2000,85.4000) -- (239.3000,85.4000) -- (239.4000,85.3000) -- (239.5000,85.3000) -- (239.6000,85.2000) -- (239.7000,85.2000) -- (239.7000,85.1000) -- (239.8000,85.1000) -- (239.9000,85.1000) -- (240.0000,85.1000) -- (240.1000,85.1000) -- (240.2000,85.1000) -- (240.3000,85.1000) -- (240.4000,85.1000) -- (240.4000,85.2000) -- (240.5000,85.2000) -- (240.6000,85.3000) -- (240.7000,85.4000) -- (240.8000,85.4000) -- (240.9000,85.5000) -- (241.0000,85.6000) -- (241.0000,85.7000) -- (241.1000,85.8000) -- (241.2000,85.9000) -- (241.3000,86.0000) -- (241.4000,86.1000) -- (241.5000,86.2000) -- (241.6000,86.3000) -- (241.7000,86.4000) -- (241.7000,86.5000) -- (241.8000,86.5000) -- (241.9000,86.6000) -- (242.0000,86.6000) -- (242.1000,86.6000) -- (242.2000,86.6000) -- (242.3000,86.5000) -- (242.4000,86.5000) -- (242.4000,86.4000) -- (242.5000,86.3000) -- (242.6000,86.1000) -- (242.7000,85.9000) -- (242.8000,85.7000) -- (242.9000,85.4000) -- (243.0000,85.1000) -- (243.0000,84.7000) -- (243.1000,84.3000) -- (243.2000,83.9000) -- (243.3000,83.4000) -- (243.4000,82.9000) -- (243.5000,82.4000) -- (243.6000,81.8000) -- (243.7000,81.1000) -- (243.7000,80.4000) -- (243.8000,79.7000) -- (243.9000,78.9000) -- (244.0000,78.1000) -- (244.1000,77.3000) -- (244.2000,76.4000) -- (244.3000,75.5000) -- (244.3000,74.5000) -- (244.4000,73.6000) -- (244.5000,72.6000) -- (244.6000,71.6000) -- (244.7000,70.5000) -- (244.8000,69.5000) -- (244.9000,68.4000) -- (245.0000,67.3000) -- (245.0000,66.2000) -- (245.1000,65.1000) -- (245.2000,64.0000) -- (245.3000,62.9000) -- (245.4000,61.9000) -- (245.5000,60.8000) -- (245.6000,59.7000) -- (245.7000,58.7000) -- (245.7000,57.7000) -- (245.8000,56.7000) -- (245.9000,55.7000) -- (246.0000,54.7000) -- (246.1000,53.8000) -- (246.2000,52.9000) -- (246.3000,52.1000) -- (246.3000,51.3000) -- (246.4000,50.5000) -- (246.5000,49.8000) -- (246.6000,49.1000) -- (246.7000,48.5000) -- (246.8000,47.9000) -- (246.9000,47.3000) -- (247.0000,46.9000) -- (247.0000,46.4000) -- (247.1000,46.0000) -- (247.2000,45.7000) -- (247.3000,45.4000) -- (247.4000,45.1000) -- (247.5000,44.9000) -- (247.6000,44.8000) -- (247.6000,44.7000) -- (247.7000,44.6000) -- (247.8000,44.6000) -- (247.9000,44.6000) -- (248.0000,44.6000) -- (248.1000,44.7000) -- (248.2000,44.8000) -- (248.3000,45.0000) -- (248.3000,45.1000) -- (248.4000,45.3000) -- (248.5000,45.5000) -- (248.6000,45.7000) -- (248.7000,46.0000) -- (248.8000,46.2000) -- (248.9000,46.5000) -- (248.9000,46.7000) -- (249.0000,47.0000) -- (249.1000,47.3000) -- (249.2000,47.5000) -- (249.3000,47.7000) -- (249.4000,48.0000) -- (249.5000,48.2000) -- (249.6000,48.4000) -- (249.6000,48.6000) -- (249.7000,48.8000) -- (249.8000,48.9000) -- (249.9000,49.0000) -- (250.0000,49.1000) -- (250.1000,49.2000) -- (250.2000,49.2000) -- (250.3000,49.2000) -- (250.3000,49.2000) -- (250.4000,49.1000) -- (250.5000,49.0000) -- (250.6000,48.9000) -- (250.7000,48.8000) -- (250.8000,48.6000) -- (250.9000,48.4000) -- (250.9000,48.2000) -- (251.0000,47.9000) -- (251.1000,47.7000) -- (251.2000,47.4000) -- (251.3000,47.1000) -- (251.4000,46.7000) -- (251.5000,46.4000) -- (251.6000,46.0000) -- (251.6000,45.7000) -- (251.7000,45.3000) -- (251.8000,44.9000) -- (251.9000,44.5000);



  \end{scope}
  \begin{scope}[scale=1.006,draw=blue,line cap=rect,line join=bevel,line width=0.800pt]
  \end{scope}
  \begin{scope}[scale=1.006,draw=blue,line cap=rect,line join=bevel,line width=0.800pt]
  \end{scope}
  \begin{scope}[cm={{1.26012,0.0,0.0,1.26012,(-619.61892,-54.78689)}},draw=blue,line cap=round,line join=round,line width=0.480pt]
    \path[draw] (165.5000,13.5000) -- (165.5000,95.5000) -- (251.5000,95.5000) -- (251.5000,13.5000) -- (165.5000,13.5000);



  \end{scope}
  \begin{scope}[scale=1.006,draw=blue,line cap=rect,line join=bevel,line width=0.800pt]
  \end{scope}
  \begin{scope}[cm={{1.00588,0.0,0.0,1.00588,(39.2294,199.165)}},draw=blue,line cap=rect,line join=bevel,line width=0.800pt]
  \end{scope}
  \begin{scope}[cm={{1.00588,0.0,0.0,1.00588,(39.2294,199.165)}},draw=blue,line cap=rect,line join=bevel,line width=0.800pt]
  \end{scope}
  \begin{scope}[cm={{1.00588,0.0,0.0,1.00588,(39.2294,199.165)}},draw=blue,line cap=rect,line join=bevel,line width=0.800pt]
  \end{scope}
  \begin{scope}[cm={{1.00588,0.0,0.0,1.00588,(39.2294,199.165)}},draw=blue,line cap=rect,line join=bevel,line width=0.800pt]
  \end{scope}
  \begin{scope}[cm={{1.00588,0.0,0.0,1.00588,(39.2294,199.165)}},draw=blue,line cap=rect,line join=bevel,line width=0.800pt]
  \end{scope}
  \begin{scope}[cm={{1.00588,0.0,0.0,1.00588,(39.2294,199.165)}},draw=blue,line cap=rect,line join=bevel,line width=0.800pt]
  \end{scope}
  \begin{scope}[scale=1.006,draw=blue,line cap=rect,line join=bevel,line width=0.800pt]
  \end{scope}
  \begin{scope}[scale=1.006,draw=blue,line cap=rect,line join=bevel,line width=0.800pt]
  \end{scope}
  \begin{scope}[cm={{1.00588,0.0,0.0,1.00588,(40.2353,177.035)}},draw=blue,line cap=rect,line join=bevel,line width=0.800pt]
  \end{scope}
  \begin{scope}[cm={{1.00588,0.0,0.0,1.00588,(40.2353,177.035)}},draw=blue,line cap=rect,line join=bevel,line width=0.800pt]
  \end{scope}
  \begin{scope}[cm={{1.00588,0.0,0.0,1.00588,(40.2353,177.035)}},draw=blue,line cap=rect,line join=bevel,line width=0.800pt]
  \end{scope}
  \begin{scope}[cm={{1.00588,0.0,0.0,1.00588,(40.2353,177.035)}},draw=blue,line cap=rect,line join=bevel,line width=0.800pt]
  \end{scope}
  \begin{scope}[cm={{1.00588,0.0,0.0,1.00588,(40.2353,177.035)}},draw=blue,line cap=rect,line join=bevel,line width=0.800pt]
  \end{scope}
  \begin{scope}[cm={{1.00588,0.0,0.0,1.00588,(40.2353,177.035)}},draw=blue,line cap=rect,line join=bevel,line width=0.800pt]
  \end{scope}
  \begin{scope}[scale=1.006,draw=blue,line cap=rect,line join=bevel,line width=0.800pt]
  \end{scope}
  \begin{scope}[scale=1.006,draw=blue,line cap=rect,line join=bevel,line width=0.800pt]
  \end{scope}
  \begin{scope}[cm={{1.00588,0.0,0.0,1.00588,(40.2353,153.9)}},draw=blue,line cap=rect,line join=bevel,line width=0.800pt]
  \end{scope}
  \begin{scope}[cm={{1.00588,0.0,0.0,1.00588,(40.2353,153.9)}},draw=blue,line cap=rect,line join=bevel,line width=0.800pt]
  \end{scope}
  \begin{scope}[cm={{1.00588,0.0,0.0,1.00588,(40.2353,153.9)}},draw=blue,line cap=rect,line join=bevel,line width=0.800pt]
  \end{scope}
  \begin{scope}[cm={{1.00588,0.0,0.0,1.00588,(40.2353,153.9)}},draw=blue,line cap=rect,line join=bevel,line width=0.800pt]
  \end{scope}
  \begin{scope}[cm={{1.00588,0.0,0.0,1.00588,(40.2353,153.9)}},draw=blue,line cap=rect,line join=bevel,line width=0.800pt]
  \end{scope}
  \begin{scope}[cm={{1.00588,0.0,0.0,1.00588,(40.2353,153.9)}},draw=blue,line cap=rect,line join=bevel,line width=0.800pt]
  \end{scope}
  \begin{scope}[scale=1.006,draw=blue,line cap=rect,line join=bevel,line width=0.800pt]
  \end{scope}
  \begin{scope}[scale=1.006,draw=blue,line cap=rect,line join=bevel,line width=0.800pt]
  \end{scope}
  \begin{scope}[cm={{1.00588,0.0,0.0,1.00588,(39.2294,131.771)}},draw=blue,line cap=rect,line join=bevel,line width=0.800pt]
  \end{scope}
  \begin{scope}[cm={{1.00588,0.0,0.0,1.00588,(39.2294,131.771)}},draw=blue,line cap=rect,line join=bevel,line width=0.800pt]
  \end{scope}
  \begin{scope}[cm={{1.00588,0.0,0.0,1.00588,(39.2294,131.771)}},draw=blue,line cap=rect,line join=bevel,line width=0.800pt]
  \end{scope}
  \begin{scope}[cm={{1.00588,0.0,0.0,1.00588,(39.2294,131.771)}},draw=blue,line cap=rect,line join=bevel,line width=0.800pt]
  \end{scope}
  \begin{scope}[cm={{1.00588,0.0,0.0,1.00588,(39.2294,131.771)}},draw=blue,line cap=rect,line join=bevel,line width=0.800pt]
  \end{scope}
  \begin{scope}[cm={{1.00588,0.0,0.0,1.00588,(39.2294,131.771)}},draw=blue,line cap=rect,line join=bevel,line width=0.800pt]
  \end{scope}
  \begin{scope}[scale=1.006,draw=blue,line cap=rect,line join=bevel,line width=0.800pt]
  \end{scope}
  \begin{scope}[scale=1.006,draw=blue,line cap=rect,line join=bevel,line width=0.800pt]
  \end{scope}
  \begin{scope}[cm={{1.00588,0.0,0.0,1.00588,(53.3118,217.271)}},draw=blue,line cap=rect,line join=bevel,line width=0.800pt]
  \end{scope}
  \begin{scope}[cm={{1.00588,0.0,0.0,1.00588,(53.3118,217.271)}},draw=blue,line cap=rect,line join=bevel,line width=0.800pt]
  \end{scope}
  \begin{scope}[cm={{1.00588,0.0,0.0,1.00588,(53.3118,217.271)}},draw=blue,line cap=rect,line join=bevel,line width=0.800pt]
  \end{scope}
  \begin{scope}[cm={{1.00588,0.0,0.0,1.00588,(53.3118,217.271)}},draw=blue,line cap=rect,line join=bevel,line width=0.800pt]
  \end{scope}
  \begin{scope}[cm={{1.00588,0.0,0.0,1.00588,(53.3118,217.271)}},draw=blue,line cap=rect,line join=bevel,line width=0.800pt]
  \end{scope}
  \begin{scope}[cm={{1.00588,0.0,0.0,1.00588,(53.3118,217.271)}},draw=blue,line cap=rect,line join=bevel,line width=0.800pt]
  \end{scope}
  \begin{scope}[scale=1.006,draw=blue,line cap=rect,line join=bevel,line width=0.800pt]
  \end{scope}
  \begin{scope}[scale=1.006,draw=blue,line cap=rect,line join=bevel,line width=0.800pt]
  \end{scope}
  \begin{scope}[cm={{1.00588,0.0,0.0,1.00588,(82.4824,217.271)}},draw=blue,line cap=rect,line join=bevel,line width=0.800pt]
  \end{scope}
  \begin{scope}[cm={{1.00588,0.0,0.0,1.00588,(82.4824,217.271)}},draw=blue,line cap=rect,line join=bevel,line width=0.800pt]
  \end{scope}
  \begin{scope}[cm={{1.00588,0.0,0.0,1.00588,(82.4824,217.271)}},draw=blue,line cap=rect,line join=bevel,line width=0.800pt]
  \end{scope}
  \begin{scope}[cm={{1.00588,0.0,0.0,1.00588,(82.4824,217.271)}},draw=blue,line cap=rect,line join=bevel,line width=0.800pt]
  \end{scope}
  \begin{scope}[cm={{1.00588,0.0,0.0,1.00588,(82.4824,217.271)}},draw=blue,line cap=rect,line join=bevel,line width=0.800pt]
  \end{scope}
  \begin{scope}[cm={{1.00588,0.0,0.0,1.00588,(82.4824,217.271)}},draw=blue,line cap=rect,line join=bevel,line width=0.800pt]
  \end{scope}
  \begin{scope}[scale=1.006,draw=blue,line cap=rect,line join=bevel,line width=0.800pt]
  \end{scope}
  \begin{scope}[scale=1.006,draw=blue,line cap=rect,line join=bevel,line width=0.800pt]
  \end{scope}
  \begin{scope}[cm={{1.00588,0.0,0.0,1.00588,(111.653,217.271)}},draw=blue,line cap=rect,line join=bevel,line width=0.800pt]
  \end{scope}
  \begin{scope}[cm={{1.00588,0.0,0.0,1.00588,(111.653,217.271)}},draw=blue,line cap=rect,line join=bevel,line width=0.800pt]
  \end{scope}
  \begin{scope}[cm={{1.00588,0.0,0.0,1.00588,(111.653,217.271)}},draw=blue,line cap=rect,line join=bevel,line width=0.800pt]
  \end{scope}
  \begin{scope}[cm={{1.00588,0.0,0.0,1.00588,(111.653,217.271)}},draw=blue,line cap=rect,line join=bevel,line width=0.800pt]
  \end{scope}
  \begin{scope}[cm={{1.00588,0.0,0.0,1.00588,(111.653,217.271)}},draw=blue,line cap=rect,line join=bevel,line width=0.800pt]
  \end{scope}
  \begin{scope}[cm={{1.00588,0.0,0.0,1.00588,(111.653,217.271)}},draw=blue,line cap=rect,line join=bevel,line width=0.800pt]
  \end{scope}
  \begin{scope}[scale=1.006,draw=blue,line cap=rect,line join=bevel,line width=0.800pt]
  \end{scope}
  \begin{scope}[scale=1.006,draw=blue,line cap=rect,line join=bevel,line width=0.800pt]
  \end{scope}
  \begin{scope}[cm={{1.00588,0.0,0.0,1.00588,(142.332,217.271)}},draw=blue,line cap=rect,line join=bevel,line width=0.800pt]
  \end{scope}
  \begin{scope}[cm={{1.00588,0.0,0.0,1.00588,(142.332,217.271)}},draw=blue,line cap=rect,line join=bevel,line width=0.800pt]
  \end{scope}
  \begin{scope}[cm={{1.00588,0.0,0.0,1.00588,(142.332,217.271)}},draw=blue,line cap=rect,line join=bevel,line width=0.800pt]
  \end{scope}
  \begin{scope}[cm={{1.00588,0.0,0.0,1.00588,(142.332,217.271)}},draw=blue,line cap=rect,line join=bevel,line width=0.800pt]
  \end{scope}
  \begin{scope}[cm={{1.00588,0.0,0.0,1.00588,(142.332,217.271)}},draw=blue,line cap=rect,line join=bevel,line width=0.800pt]
  \end{scope}
  \begin{scope}[cm={{1.00588,0.0,0.0,1.00588,(142.332,217.271)}},draw=blue,line cap=rect,line join=bevel,line width=0.800pt]
  \end{scope}
  \begin{scope}[scale=1.006,draw=blue,line cap=rect,line join=bevel,line width=0.800pt]
  \end{scope}
  \begin{scope}[scale=1.006,draw=blue,line cap=rect,line join=bevel,line width=0.800pt]
  \end{scope}
  \begin{scope}[cm={{1.00588,0.0,0.0,1.00588,(171.503,217.271)}},draw=blue,line cap=rect,line join=bevel,line width=0.800pt]
  \end{scope}
  \begin{scope}[cm={{1.00588,0.0,0.0,1.00588,(171.503,217.271)}},draw=blue,line cap=rect,line join=bevel,line width=0.800pt]
  \end{scope}
  \begin{scope}[cm={{1.00588,0.0,0.0,1.00588,(171.503,217.271)}},draw=blue,line cap=rect,line join=bevel,line width=0.800pt]
  \end{scope}
  \begin{scope}[cm={{1.00588,0.0,0.0,1.00588,(171.503,217.271)}},draw=blue,line cap=rect,line join=bevel,line width=0.800pt]
  \end{scope}
  \begin{scope}[cm={{1.00588,0.0,0.0,1.00588,(171.503,217.271)}},draw=blue,line cap=rect,line join=bevel,line width=0.800pt]
  \end{scope}
  \begin{scope}[cm={{1.00588,0.0,0.0,1.00588,(171.503,217.271)}},draw=blue,line cap=rect,line join=bevel,line width=0.800pt]
  \end{scope}
  \begin{scope}[scale=1.006,draw=blue,line cap=rect,line join=bevel,line width=0.800pt]
  \end{scope}
  \begin{scope}[scale=1.006,draw=blue,line cap=rect,line join=bevel,line width=0.800pt]
  \end{scope}
  \begin{scope}[cm={{1.00588,0.0,0.0,1.00588,(200.674,217.271)}},draw=blue,line cap=rect,line join=bevel,line width=0.800pt]
  \end{scope}
  \begin{scope}[cm={{1.00588,0.0,0.0,1.00588,(200.674,217.271)}},draw=blue,line cap=rect,line join=bevel,line width=0.800pt]
  \end{scope}
  \begin{scope}[cm={{1.00588,0.0,0.0,1.00588,(200.674,217.271)}},draw=blue,line cap=rect,line join=bevel,line width=0.800pt]
  \end{scope}
  \begin{scope}[cm={{1.00588,0.0,0.0,1.00588,(200.674,217.271)}},draw=blue,line cap=rect,line join=bevel,line width=0.800pt]
  \end{scope}
  \begin{scope}[cm={{1.00588,0.0,0.0,1.00588,(200.674,217.271)}},draw=blue,line cap=rect,line join=bevel,line width=0.800pt]
  \end{scope}
  \begin{scope}[cm={{1.00588,0.0,0.0,1.00588,(200.674,217.271)}},draw=blue,line cap=rect,line join=bevel,line width=0.800pt]
  \end{scope}
  \begin{scope}[scale=1.006,draw=blue,line cap=rect,line join=bevel,line width=0.800pt]
  \end{scope}
  \begin{scope}[scale=1.006,draw=blue,line cap=rect,line join=bevel,line width=0.800pt]
  \end{scope}
  \begin{scope}[cm={{1.00588,0.0,0.0,1.00588,(229.341,217.271)}},draw=blue,line cap=rect,line join=bevel,line width=0.800pt]
  \end{scope}
  \begin{scope}[cm={{1.00588,0.0,0.0,1.00588,(229.341,217.271)}},draw=blue,line cap=rect,line join=bevel,line width=0.800pt]
  \end{scope}
  \begin{scope}[cm={{1.00588,0.0,0.0,1.00588,(229.341,217.271)}},draw=blue,line cap=rect,line join=bevel,line width=0.800pt]
  \end{scope}
  \begin{scope}[cm={{1.00588,0.0,0.0,1.00588,(229.341,217.271)}},draw=blue,line cap=rect,line join=bevel,line width=0.800pt]
  \end{scope}
  \begin{scope}[cm={{1.00588,0.0,0.0,1.00588,(229.341,217.271)}},draw=blue,line cap=rect,line join=bevel,line width=0.800pt]
  \end{scope}
  \begin{scope}[cm={{1.00588,0.0,0.0,1.00588,(229.341,217.271)}},draw=blue,line cap=rect,line join=bevel,line width=0.800pt]
  \end{scope}
  \begin{scope}[scale=1.006,draw=blue,line cap=rect,line join=bevel,line width=0.800pt]
  \end{scope}
  \begin{scope}[scale=1.006,draw=blue,line cap=rect,line join=bevel,line width=0.800pt]
  \end{scope}
  \begin{scope}[scale=1.006,draw=blue,line cap=rect,line join=bevel,line width=0.800pt]
  \end{scope}
  \begin{scope}[scale=1.006,draw=blue,line cap=rect,line join=bevel,line width=0.800pt]
  \end{scope}
  \begin{scope}[scale=1.006,draw=blue,line cap=rect,line join=bevel,line width=0.800pt]
  \end{scope}
  \begin{scope}[scale=1.006,draw=blue,line cap=rect,line join=bevel,line width=0.800pt]
  \end{scope}
  \begin{scope}[cm={{1.00588,0.0,0.0,1.00588,(236.382,132.776)}},draw=blue,line cap=rect,line join=bevel,line width=0.800pt]
  \end{scope}
  \begin{scope}[cm={{1.00588,0.0,0.0,1.00588,(236.382,132.776)}},draw=blue,line cap=rect,line join=bevel,line width=0.800pt]
  \end{scope}
  \begin{scope}[cm={{1.00588,0.0,0.0,1.00588,(236.382,132.776)}},draw=blue,line cap=rect,line join=bevel,line width=0.800pt]
  \end{scope}
  \begin{scope}[cm={{1.00588,0.0,0.0,1.00588,(236.382,132.776)}},draw=blue,line cap=rect,line join=bevel,line width=0.800pt]
  \end{scope}
  \begin{scope}[cm={{1.00588,0.0,0.0,1.00588,(236.382,132.776)}},draw=blue,line cap=rect,line join=bevel,line width=0.800pt]
  \end{scope}
  \begin{scope}[cm={{1.00588,0.0,0.0,1.00588,(236.382,132.776)}},draw=blue,line cap=rect,line join=bevel,line width=0.800pt]
  \end{scope}
  \begin{scope}[scale=1.006,draw=blue,line cap=rect,line join=bevel,line width=0.800pt]
  \end{scope}
  \begin{scope}[scale=1.006,draw=blue,line cap=rect,line join=bevel,line width=0.800pt]
  \end{scope}
  \begin{scope}[scale=1.006,draw=blue,line cap=rect,line join=bevel,line width=0.800pt]
  \end{scope}
  \begin{scope}[scale=1.006,draw=blue,line cap=rect,line join=bevel,line width=0.800pt]
  \end{scope}
  \begin{scope}[cm={{1.00588,0.0,0.0,1.00588,(197.153,129.759)}},draw=blue,line cap=rect,line join=bevel,line width=0.800pt]
  \end{scope}
  \begin{scope}[cm={{1.00588,0.0,0.0,1.00588,(197.153,129.759)}},draw=blue,line cap=rect,line join=bevel,line width=0.800pt]
  \end{scope}
  \begin{scope}[cm={{1.00588,0.0,0.0,1.00588,(197.153,129.759)}},draw=blue,line cap=rect,line join=bevel,line width=0.800pt]
  \end{scope}
  \begin{scope}[cm={{1.00588,0.0,0.0,1.00588,(197.153,129.759)}},draw=blue,line cap=rect,line join=bevel,line width=0.800pt]
  \end{scope}
  \begin{scope}[cm={{1.00588,0.0,0.0,1.00588,(197.153,129.759)}},draw=blue,line cap=rect,line join=bevel,line width=0.800pt]
  \end{scope}
  \begin{scope}[cm={{1.00588,0.0,0.0,1.00588,(197.153,129.759)}},draw=blue,line cap=rect,line join=bevel,line width=0.800pt]
  \end{scope}
  \begin{scope}[scale=1.006,draw=blue,line cap=rect,line join=bevel,line width=0.800pt]
  \end{scope}
  \begin{scope}[scale=1.006,draw=blue,line cap=rect,line join=bevel,line width=0.800pt]
  \end{scope}
  \begin{scope}[scale=1.006,draw=blue,line cap=rect,line join=bevel,line width=0.800pt]
  \end{scope}
  \begin{scope}[scale=1.006,draw=blue,line cap=rect,line join=bevel,line width=0.800pt]
  \end{scope}
  \begin{scope}[scale=1.006,draw=blue,line cap=rect,line join=bevel,line width=0.800pt]
  \end{scope}
  \begin{scope}[scale=1.006,draw=blue,line cap=rect,line join=bevel,line width=0.800pt]
  \end{scope}
  \begin{scope}[scale=1.006,draw=blue,line cap=rect,line join=bevel,line width=0.800pt]
  \end{scope}
  \begin{scope}[scale=1.006,draw=blue,line cap=rect,line join=bevel,line width=0.800pt]
  \end{scope}
  \begin{scope}[cm={{1.00588,0.0,0.0,1.00588,(39.2294,305.788)}},draw=blue,line cap=rect,line join=bevel,line width=0.800pt]
  \end{scope}
  \begin{scope}[cm={{1.00588,0.0,0.0,1.00588,(39.2294,305.788)}},draw=blue,line cap=rect,line join=bevel,line width=0.800pt]
  \end{scope}
  \begin{scope}[cm={{1.00588,0.0,0.0,1.00588,(39.2294,305.788)}},draw=blue,line cap=rect,line join=bevel,line width=0.800pt]
  \end{scope}
  \begin{scope}[cm={{1.00588,0.0,0.0,1.00588,(39.2294,305.788)}},draw=blue,line cap=rect,line join=bevel,line width=0.800pt]
  \end{scope}
  \begin{scope}[cm={{1.00588,0.0,0.0,1.00588,(39.2294,305.788)}},draw=blue,line cap=rect,line join=bevel,line width=0.800pt]
  \end{scope}
  \begin{scope}[cm={{1.00588,0.0,0.0,1.00588,(39.2294,305.788)}},draw=blue,line cap=rect,line join=bevel,line width=0.800pt]
  \end{scope}
  \begin{scope}[scale=1.006,draw=blue,line cap=rect,line join=bevel,line width=0.800pt]
  \end{scope}
  \begin{scope}[scale=1.006,draw=blue,line cap=rect,line join=bevel,line width=0.800pt]
  \end{scope}
  \begin{scope}[cm={{1.00588,0.0,0.0,1.00588,(40.2353,282.653)}},draw=blue,line cap=rect,line join=bevel,line width=0.800pt]
  \end{scope}
  \begin{scope}[cm={{1.00588,0.0,0.0,1.00588,(40.2353,282.653)}},draw=blue,line cap=rect,line join=bevel,line width=0.800pt]
  \end{scope}
  \begin{scope}[cm={{1.00588,0.0,0.0,1.00588,(40.2353,282.653)}},draw=blue,line cap=rect,line join=bevel,line width=0.800pt]
  \end{scope}
  \begin{scope}[cm={{1.00588,0.0,0.0,1.00588,(40.2353,282.653)}},draw=blue,line cap=rect,line join=bevel,line width=0.800pt]
  \end{scope}
  \begin{scope}[cm={{1.00588,0.0,0.0,1.00588,(40.2353,282.653)}},draw=blue,line cap=rect,line join=bevel,line width=0.800pt]
  \end{scope}
  \begin{scope}[cm={{1.00588,0.0,0.0,1.00588,(40.2353,282.653)}},draw=blue,line cap=rect,line join=bevel,line width=0.800pt]
  \end{scope}
  \begin{scope}[scale=1.006,draw=blue,line cap=rect,line join=bevel,line width=0.800pt]
  \end{scope}
  \begin{scope}[scale=1.006,draw=blue,line cap=rect,line join=bevel,line width=0.800pt]
  \end{scope}
  \begin{scope}[cm={{1.00588,0.0,0.0,1.00588,(40.2353,260.524)}},draw=blue,line cap=rect,line join=bevel,line width=0.800pt]
  \end{scope}
  \begin{scope}[cm={{1.00588,0.0,0.0,1.00588,(40.2353,260.524)}},draw=blue,line cap=rect,line join=bevel,line width=0.800pt]
  \end{scope}
  \begin{scope}[cm={{1.00588,0.0,0.0,1.00588,(40.2353,260.524)}},draw=blue,line cap=rect,line join=bevel,line width=0.800pt]
  \end{scope}
  \begin{scope}[cm={{1.00588,0.0,0.0,1.00588,(40.2353,260.524)}},draw=blue,line cap=rect,line join=bevel,line width=0.800pt]
  \end{scope}
  \begin{scope}[cm={{1.00588,0.0,0.0,1.00588,(40.2353,260.524)}},draw=blue,line cap=rect,line join=bevel,line width=0.800pt]
  \end{scope}
  \begin{scope}[cm={{1.00588,0.0,0.0,1.00588,(40.2353,260.524)}},draw=blue,line cap=rect,line join=bevel,line width=0.800pt]
  \end{scope}
  \begin{scope}[scale=1.006,draw=blue,line cap=rect,line join=bevel,line width=0.800pt]
  \end{scope}
  \begin{scope}[scale=1.006,draw=blue,line cap=rect,line join=bevel,line width=0.800pt]
  \end{scope}
  \begin{scope}[cm={{1.00588,0.0,0.0,1.00588,(39.2294,237.388)}},draw=blue,line cap=rect,line join=bevel,line width=0.800pt]
  \end{scope}
  \begin{scope}[cm={{1.00588,0.0,0.0,1.00588,(39.2294,237.388)}},draw=blue,line cap=rect,line join=bevel,line width=0.800pt]
  \end{scope}
  \begin{scope}[cm={{1.00588,0.0,0.0,1.00588,(39.2294,237.388)}},draw=blue,line cap=rect,line join=bevel,line width=0.800pt]
  \end{scope}
  \begin{scope}[cm={{1.00588,0.0,0.0,1.00588,(39.2294,237.388)}},draw=blue,line cap=rect,line join=bevel,line width=0.800pt]
  \end{scope}
  \begin{scope}[cm={{1.00588,0.0,0.0,1.00588,(39.2294,237.388)}},draw=blue,line cap=rect,line join=bevel,line width=0.800pt]
  \end{scope}
  \begin{scope}[cm={{1.00588,0.0,0.0,1.00588,(39.2294,237.388)}},draw=blue,line cap=rect,line join=bevel,line width=0.800pt]
  \end{scope}
  \begin{scope}[scale=1.006,draw=blue,line cap=rect,line join=bevel,line width=0.800pt]
  \end{scope}
  \begin{scope}[scale=1.006,draw=blue,line cap=rect,line join=bevel,line width=0.800pt]
  \end{scope}
  \begin{scope}[cm={{1.00588,0.0,0.0,1.00588,(53.3118,322.888)}},draw=blue,line cap=rect,line join=bevel,line width=0.800pt]
  \end{scope}
  \begin{scope}[cm={{1.00588,0.0,0.0,1.00588,(53.3118,322.888)}},draw=blue,line cap=rect,line join=bevel,line width=0.800pt]
  \end{scope}
  \begin{scope}[cm={{1.00588,0.0,0.0,1.00588,(53.3118,322.888)}},draw=blue,line cap=rect,line join=bevel,line width=0.800pt]
  \end{scope}
  \begin{scope}[cm={{1.00588,0.0,0.0,1.00588,(53.3118,322.888)}},draw=blue,line cap=rect,line join=bevel,line width=0.800pt]
  \end{scope}
  \begin{scope}[cm={{1.00588,0.0,0.0,1.00588,(53.3118,322.888)}},draw=blue,line cap=rect,line join=bevel,line width=0.800pt]
  \end{scope}
  \begin{scope}[cm={{1.00588,0.0,0.0,1.00588,(53.3118,322.888)}},draw=blue,line cap=rect,line join=bevel,line width=0.800pt]
  \end{scope}
  \begin{scope}[scale=1.006,draw=blue,line cap=rect,line join=bevel,line width=0.800pt]
  \end{scope}
  \begin{scope}[scale=1.006,draw=blue,line cap=rect,line join=bevel,line width=0.800pt]
  \end{scope}
  \begin{scope}[cm={{1.00588,0.0,0.0,1.00588,(89.5235,322.888)}},draw=blue,line cap=rect,line join=bevel,line width=0.800pt]
  \end{scope}
  \begin{scope}[cm={{1.00588,0.0,0.0,1.00588,(89.5235,322.888)}},draw=blue,line cap=rect,line join=bevel,line width=0.800pt]
  \end{scope}
  \begin{scope}[cm={{1.00588,0.0,0.0,1.00588,(89.5235,322.888)}},draw=blue,line cap=rect,line join=bevel,line width=0.800pt]
  \end{scope}
  \begin{scope}[cm={{1.00588,0.0,0.0,1.00588,(89.5235,322.888)}},draw=blue,line cap=rect,line join=bevel,line width=0.800pt]
  \end{scope}
  \begin{scope}[cm={{1.00588,0.0,0.0,1.00588,(89.5235,322.888)}},draw=blue,line cap=rect,line join=bevel,line width=0.800pt]
  \end{scope}
  \begin{scope}[cm={{1.00588,0.0,0.0,1.00588,(89.5235,322.888)}},draw=blue,line cap=rect,line join=bevel,line width=0.800pt]
  \end{scope}
  \begin{scope}[scale=1.006,draw=blue,line cap=rect,line join=bevel,line width=0.800pt]
  \end{scope}
  \begin{scope}[scale=1.006,draw=blue,line cap=rect,line join=bevel,line width=0.800pt]
  \end{scope}
  \begin{scope}[cm={{1.00588,0.0,0.0,1.00588,(125.232,322.888)}},draw=blue,line cap=rect,line join=bevel,line width=0.800pt]
  \end{scope}
  \begin{scope}[cm={{1.00588,0.0,0.0,1.00588,(125.232,322.888)}},draw=blue,line cap=rect,line join=bevel,line width=0.800pt]
  \end{scope}
  \begin{scope}[cm={{1.00588,0.0,0.0,1.00588,(125.232,322.888)}},draw=blue,line cap=rect,line join=bevel,line width=0.800pt]
  \end{scope}
  \begin{scope}[cm={{1.00588,0.0,0.0,1.00588,(125.232,322.888)}},draw=blue,line cap=rect,line join=bevel,line width=0.800pt]
  \end{scope}
  \begin{scope}[cm={{1.00588,0.0,0.0,1.00588,(125.232,322.888)}},draw=blue,line cap=rect,line join=bevel,line width=0.800pt]
  \end{scope}
  \begin{scope}[cm={{1.00588,0.0,0.0,1.00588,(125.232,322.888)}},draw=blue,line cap=rect,line join=bevel,line width=0.800pt]
  \end{scope}
  \begin{scope}[scale=1.006,draw=blue,line cap=rect,line join=bevel,line width=0.800pt]
  \end{scope}
  \begin{scope}[scale=1.006,draw=blue,line cap=rect,line join=bevel,line width=0.800pt]
  \end{scope}
  \begin{scope}[cm={{1.00588,0.0,0.0,1.00588,(160.941,322.888)}},draw=blue,line cap=rect,line join=bevel,line width=0.800pt]
  \end{scope}
  \begin{scope}[cm={{1.00588,0.0,0.0,1.00588,(160.941,322.888)}},draw=blue,line cap=rect,line join=bevel,line width=0.800pt]
  \end{scope}
  \begin{scope}[cm={{1.00588,0.0,0.0,1.00588,(160.941,322.888)}},draw=blue,line cap=rect,line join=bevel,line width=0.800pt]
  \end{scope}
  \begin{scope}[cm={{1.00588,0.0,0.0,1.00588,(160.941,322.888)}},draw=blue,line cap=rect,line join=bevel,line width=0.800pt]
  \end{scope}
  \begin{scope}[cm={{1.00588,0.0,0.0,1.00588,(160.941,322.888)}},draw=blue,line cap=rect,line join=bevel,line width=0.800pt]
  \end{scope}
  \begin{scope}[cm={{1.00588,0.0,0.0,1.00588,(160.941,322.888)}},draw=blue,line cap=rect,line join=bevel,line width=0.800pt]
  \end{scope}
  \begin{scope}[scale=1.006,draw=blue,line cap=rect,line join=bevel,line width=0.800pt]
  \end{scope}
  \begin{scope}[scale=1.006,draw=blue,line cap=rect,line join=bevel,line width=0.800pt]
  \end{scope}
  \begin{scope}[cm={{1.00588,0.0,0.0,1.00588,(196.147,322.888)}},draw=blue,line cap=rect,line join=bevel,line width=0.800pt]
  \end{scope}
  \begin{scope}[cm={{1.00588,0.0,0.0,1.00588,(196.147,322.888)}},draw=blue,line cap=rect,line join=bevel,line width=0.800pt]
  \end{scope}
  \begin{scope}[cm={{1.00588,0.0,0.0,1.00588,(196.147,322.888)}},draw=blue,line cap=rect,line join=bevel,line width=0.800pt]
  \end{scope}
  \begin{scope}[cm={{1.00588,0.0,0.0,1.00588,(196.147,322.888)}},draw=blue,line cap=rect,line join=bevel,line width=0.800pt]
  \end{scope}
  \begin{scope}[cm={{1.00588,0.0,0.0,1.00588,(196.147,322.888)}},draw=blue,line cap=rect,line join=bevel,line width=0.800pt]
  \end{scope}
  \begin{scope}[cm={{1.00588,0.0,0.0,1.00588,(196.147,322.888)}},draw=blue,line cap=rect,line join=bevel,line width=0.800pt]
  \end{scope}
  \begin{scope}[scale=1.006,draw=blue,line cap=rect,line join=bevel,line width=0.800pt]
  \end{scope}
  \begin{scope}[scale=1.006,draw=blue,line cap=rect,line join=bevel,line width=0.800pt]
  \end{scope}
  \begin{scope}[cm={{1.00588,0.0,0.0,1.00588,(228.838,322.888)}},draw=blue,line cap=rect,line join=bevel,line width=0.800pt]
  \end{scope}
  \begin{scope}[cm={{1.00588,0.0,0.0,1.00588,(228.838,322.888)}},draw=blue,line cap=rect,line join=bevel,line width=0.800pt]
  \end{scope}
  \begin{scope}[cm={{1.00588,0.0,0.0,1.00588,(228.838,322.888)}},draw=blue,line cap=rect,line join=bevel,line width=0.800pt]
  \end{scope}
  \begin{scope}[cm={{1.00588,0.0,0.0,1.00588,(228.838,322.888)}},draw=blue,line cap=rect,line join=bevel,line width=0.800pt]
  \end{scope}
  \begin{scope}[cm={{1.00588,0.0,0.0,1.00588,(228.838,322.888)}},draw=blue,line cap=rect,line join=bevel,line width=0.800pt]
  \end{scope}
  \begin{scope}[cm={{1.00588,0.0,0.0,1.00588,(228.838,322.888)}},draw=blue,line cap=rect,line join=bevel,line width=0.800pt]
  \end{scope}
  \begin{scope}[scale=1.006,draw=blue,line cap=rect,line join=bevel,line width=0.800pt]
  \end{scope}
  \begin{scope}[scale=1.006,draw=blue,line cap=rect,line join=bevel,line width=0.800pt]
  \end{scope}
  \begin{scope}[scale=1.006,draw=blue,line cap=rect,line join=bevel,line width=0.800pt]
  \end{scope}
  \begin{scope}[scale=1.006,draw=blue,line cap=rect,line join=bevel,line width=0.800pt]
  \end{scope}
  \begin{scope}[scale=1.006,draw=blue,line cap=rect,line join=bevel,line width=0.800pt]
  \end{scope}
  \begin{scope}[scale=1.006,draw=blue,line cap=rect,line join=bevel,line width=0.800pt]
  \end{scope}
  \begin{scope}[cm={{1.00588,0.0,0.0,1.00588,(232.359,238.394)}},draw=blue,line cap=rect,line join=bevel,line width=0.800pt]
  \end{scope}
  \begin{scope}[cm={{1.00588,0.0,0.0,1.00588,(232.359,238.394)}},draw=blue,line cap=rect,line join=bevel,line width=0.800pt]
  \end{scope}
  \begin{scope}[cm={{1.00588,0.0,0.0,1.00588,(232.359,238.394)}},draw=blue,line cap=rect,line join=bevel,line width=0.800pt]
  \end{scope}
  \begin{scope}[cm={{1.00588,0.0,0.0,1.00588,(232.359,238.394)}},draw=blue,line cap=rect,line join=bevel,line width=0.800pt]
  \end{scope}
  \begin{scope}[cm={{1.00588,0.0,0.0,1.00588,(232.359,238.394)}},draw=blue,line cap=rect,line join=bevel,line width=0.800pt]
  \end{scope}
  \begin{scope}[cm={{1.00588,0.0,0.0,1.00588,(232.359,238.394)}},draw=blue,line cap=rect,line join=bevel,line width=0.800pt]
  \end{scope}
  \begin{scope}[cm={{1.00588,0.0,0.0,1.00588,(130.262,337.976)}},draw=blue,line cap=rect,line join=bevel,line width=0.800pt]
  \end{scope}
  \begin{scope}[cm={{1.00588,0.0,0.0,1.00588,(130.262,337.976)}},draw=blue,line cap=rect,line join=bevel,line width=0.800pt]
  \end{scope}
  \begin{scope}[cm={{1.00588,0.0,0.0,1.00588,(130.262,337.976)}},draw=blue,line cap=rect,line join=bevel,line width=0.800pt]
  \end{scope}
  \begin{scope}[cm={{1.00588,0.0,0.0,1.00588,(130.262,337.976)}},draw=blue,line cap=rect,line join=bevel,line width=0.800pt]
  \end{scope}
  \begin{scope}[cm={{1.00588,0.0,0.0,1.00588,(130.262,337.976)}},draw=blue,line cap=rect,line join=bevel,line width=0.800pt]
  \end{scope}
  \begin{scope}[cm={{1.00588,0.0,0.0,1.00588,(130.262,337.976)}},draw=blue,line cap=rect,line join=bevel,line width=0.800pt]
  \end{scope}
  \begin{scope}[scale=1.006,draw=blue,line cap=rect,line join=bevel,line width=0.800pt]
  \end{scope}
  \begin{scope}[scale=1.006,draw=blue,line cap=rect,line join=bevel,line width=0.800pt]
  \end{scope}
  \begin{scope}[scale=1.006,draw=blue,line cap=rect,line join=bevel,line width=0.800pt]
  \end{scope}
  \begin{scope}[scale=1.006,draw=blue,line cap=rect,line join=bevel,line width=0.800pt]
  \end{scope}
  \begin{scope}[scale=1.006,draw=blue,line cap=rect,line join=bevel,line width=0.800pt]
  \end{scope}
  \begin{scope}[scale=1.006,draw=blue,line cap=rect,line join=bevel,line width=0.800pt]
  \end{scope}
  \begin{scope}[scale=1.006,draw=blue,line cap=rect,line join=bevel,line width=0.800pt]
  \end{scope}
  \begin{scope}[scale=1.006,draw=blue,line cap=rect,line join=bevel,line width=0.800pt]
  \end{scope}
  \begin{scope}[draw=blue,line cap=rect,line join=bevel,line width=0.800pt]
  \end{scope}
  \begin{scope}[cm={{1.26012,0.0,0.0,1.26012,(-482.16108,-168.79053)}},draw=ca0a0a4,dash pattern=on 1.03pt off 1.03pt,line cap=round,line join=round,line width=0.257pt,miter limit=4.00]
    \path[draw,dash pattern=on 1.03pt off 1.03pt,line width=0.257pt,miter limit=4.00] (56.5000,88.5000) -- (142.5000,88.5000);



  \end{scope}
  \begin{scope}[cm={{1.26012,0.0,0.0,1.26012,(-482.16108,-168.79053)}},draw=blue,line cap=round,line join=round,line width=0.480pt]
    \path[cm={{0.9189,0.0,0.0,1.0,(4.55807,0.0)}},draw] (56.5000,88.5000) -- (59.5000,88.5000);



    \path[cm={{0.9189,0.0,0.0,1.0,(11.5047,0.0)}},draw] (142.5000,88.5000) -- (139.5000,88.5000);



  \end{scope}
  \begin{scope}[cm={{1.26012,0.0,0.0,1.26012,(-482.16108,-168.79053)}},draw=ca0a0a4,dash pattern=on 1.03pt off 1.03pt,line cap=round,line join=round,line width=0.257pt,miter limit=4.00]
    \path[draw,dash pattern=on 1.03pt off 1.03pt,line width=0.257pt,miter limit=4.00] (56.5000,63.5000) -- (142.5000,63.5000);



  \end{scope}
  \begin{scope}[cm={{1.26012,0.0,0.0,1.26012,(-482.16108,-168.79053)}},draw=blue,line cap=round,line join=round,line width=0.480pt]
    \path[cm={{0.9189,0.0,0.0,1.0,(4.55807,-3e-05)}},draw] (56.5000,63.5000) -- (59.5000,63.5000);



    \path[cm={{0.9189,0.0,0.0,1.0,(11.5047,0.0)}},draw] (142.5000,63.5000) -- (139.5000,63.5000);



  \end{scope}
  \begin{scope}[cm={{1.26012,0.0,0.0,1.26012,(-482.16108,-168.79053)}},draw=ca0a0a4,dash pattern=on 1.03pt off 1.03pt,line cap=round,line join=round,line width=0.257pt,miter limit=4.00]
    \path[draw,dash pattern=on 1.03pt off 1.03pt,line width=0.257pt,miter limit=4.00] (56.5000,38.5000) -- (142.5000,38.5000);



  \end{scope}
  \begin{scope}[cm={{1.26012,0.0,0.0,1.26012,(-482.16108,-168.79053)}},draw=blue,line cap=round,line join=round,line width=0.480pt]
    \path[cm={{0.9189,0.0,0.0,1.0,(4.55807,-3e-05)}},draw] (56.5000,38.5000) -- (59.5000,38.5000);



    \path[cm={{0.9189,0.0,0.0,1.0,(11.5047,0.0)}},draw] (142.5000,38.5000) -- (139.5000,38.5000);



  \end{scope}
  \begin{scope}[cm={{1.26012,0.0,0.0,1.26012,(-482.16108,-168.79053)}},draw=ca0a0a4,dash pattern=on 0.40pt off 0.80pt,line cap=round,line join=round,line width=0.400pt]
    \path[draw] (56.5000,95.5000) -- (56.5000,13.5000);



  \end{scope}
  \begin{scope}[cm={{1.26012,0.0,0.0,1.26012,(-482.16108,-168.79053)}},draw=blue,line cap=round,line join=round,line width=0.480pt]
    \path[draw] (56.5000,95.5000) -- (56.5000,92.5000);



    \path[draw] (56.5000,13.5000) -- (56.5000,16.5000);



  \end{scope}
  \begin{scope}[cm={{1.26012,0.0,0.0,1.26012,(-482.16108,-168.79053)}},draw=blue,line cap=round,line join=round,line width=0.480pt]
    \path[cm={{1.0,0.0,0.0,0.9189,(0.0,7.47783)}},draw] (82.5000,95.5000) -- (82.5000,92.5000);



    \path[cm={{1.0,0.0,0.0,0.9189,(0.0,1.07058)}},draw] (82.5000,13.5000) -- (82.5000,16.5000);



  \end{scope}
  \begin{scope}[cm={{1.26012,0.0,0.0,1.26012,(-482.16108,-168.79053)}},draw=blue,line cap=round,line join=round,line width=0.480pt]
    \path[cm={{1.0,0.0,0.0,0.9189,(0.0,7.47783)}},draw] (108.5000,95.5000) -- (108.5000,92.5000);



    \path[cm={{1.0,0.0,0.0,0.9189,(0.0,1.07058)}},draw] (108.5000,13.5000) -- (108.5000,16.5000);



  \end{scope}
  \begin{scope}[cm={{1.26012,0.0,0.0,1.26012,(-482.16108,-168.79053)}},draw=ca0a0a4,dash pattern=on 1.03pt off 1.03pt,line cap=round,line join=round,line width=0.257pt,miter limit=4.00]
    \path[draw,dash pattern=on 1.03pt off 1.03pt,line width=0.257pt,miter limit=4.00] (134.5000,95.5000) -- (134.5000,13.5000);



  \end{scope}
  \begin{scope}[cm={{1.26012,0.0,0.0,1.26012,(-482.16108,-168.79053)}},draw=blue,line cap=round,line join=round,line width=0.480pt]
    \path[cm={{1.0,0.0,0.0,0.9189,(0.0,7.47783)}},draw] (134.5000,95.5000) -- (134.5000,92.5000);



    \path[cm={{1.0,0.0,0.0,0.9189,(0.0,1.07058)}},draw] (134.5000,13.5000) -- (134.5000,16.5000);



  \end{scope}
  \begin{scope}[cm={{1.00588,0.0,0.0,1.00588,(-400.2762,-132.7966)}},draw=blue,line cap=rect,line join=bevel,line width=0.800pt]
    \path[fill=blue] (0.0000,-1.4912) node[above right] (text290) {\scriptsize $\Upsilon(t)$};



  \end{scope}
  \begin{scope}[cm={{1.26012,0.0,0.0,1.26012,(-487.73989,-169.45128)}},draw=blue,line cap=round,line join=round,line width=0.480pt]
    \path[draw,even odd rule] (86.5000,24.5000) -- (112.5000,24.5000);



  \end{scope}
  \begin{scope}[cm={{1.26012,0.0,0.0,1.26012,(-482.16108,-168.79053)}},draw=blue,line cap=round,line join=round,line width=0.480pt]
    \path[draw] (56.1000,52.3000) -- (56.1000,52.3000) -- (56.5000,69.8000) -- (56.8000,61.5000) -- (57.1000,50.5000) -- (57.4000,49.2000) -- (57.7000,52.5000) -- (58.0000,57.0000) -- (58.4000,59.6000) -- (58.7000,60.0000) -- (59.0000,59.6000) -- (59.3000,59.2000) -- (59.6000,59.1000) -- (59.9000,59.0000) -- (60.2000,59.0000) -- (60.6000,59.1000) -- (60.9000,59.1000) -- (61.2000,59.1000) -- (61.5000,59.1000) -- (61.8000,59.0000) -- (62.1000,59.5000) -- (62.4000,62.4000) -- (62.8000,66.1000) -- (63.1000,69.9000) -- (63.4000,73.4000) -- (63.7000,76.7000) -- (64.0000,79.8000) -- (64.3000,82.6000) -- (64.6000,84.9000) -- (65.0000,86.4000) -- (65.3000,87.1000) -- (65.6000,86.7000) -- (65.9000,85.4000) -- (66.2000,83.2000) -- (66.5000,80.3000) -- (66.8000,77.1000) -- (67.2000,73.7000) -- (67.5000,70.1000) -- (67.8000,65.0000) -- (68.1000,60.3000) -- (68.4000,58.1000) -- (68.7000,58.0000) -- (69.0000,58.5000) -- (69.4000,58.9000) -- (69.7000,59.1000) -- (70.0000,59.1000) -- (70.3000,59.1000) -- (70.6000,59.1000) -- (70.9000,59.1000) -- (71.2000,59.1000) -- (71.5000,59.0000) -- (71.9000,61.2000) -- (72.2000,62.2000) -- (72.5000,59.0000) -- (72.8000,55.3000) -- (73.1000,52.3000) -- (73.4000,50.0000) -- (73.7000,48.2000) -- (74.1000,46.8000) -- (74.4000,45.7000) -- (74.7000,44.9000) -- (75.0000,44.4000) -- (75.3000,44.2000) -- (75.6000,44.3000) -- (75.9000,44.7000) -- (76.3000,45.4000) -- (76.6000,46.4000) -- (76.9000,47.7000) -- (77.2000,49.3000) -- (77.5000,51.3000) -- (77.8000,53.6000) -- (78.1000,56.7000) -- (78.5000,59.2000) -- (78.8000,59.9000) -- (79.1000,59.6000) -- (79.4000,59.3000) -- (79.7000,59.1000) -- (80.0000,59.0000) -- (80.3000,59.0000) -- (80.7000,59.1000) -- (81.0000,59.1000) -- (81.3000,59.1000) -- (81.6000,59.1000) -- (81.9000,59.0000) -- (82.2000,59.1000) -- (82.5000,61.3000) -- (82.9000,65.2000) -- (83.2000,69.1000) -- (83.5000,72.7000) -- (83.8000,76.1000) -- (84.1000,79.3000) -- (84.4000,82.1000) -- (84.7000,84.6000) -- (85.1000,86.3000) -- (85.4000,87.1000) -- (85.7000,86.8000) -- (86.0000,85.5000) -- (86.3000,83.3000) -- (86.6000,80.4000) -- (86.9000,77.1000) -- (87.3000,73.6000) -- (87.6000,69.9000) -- (87.9000,65.0000) -- (88.2000,60.5000) -- (88.5000,58.3000) -- (88.8000,58.1000) -- (89.1000,58.5000) -- (89.5000,58.9000) -- (89.8000,59.1000) -- (90.1000,59.1000) -- (90.4000,59.1000) -- (90.7000,59.1000) -- (91.0000,59.1000) -- (91.3000,59.1000) -- (91.7000,59.0000) -- (92.0000,61.4000) -- (92.3000,62.0000) -- (92.6000,58.6000) -- (92.9000,55.0000) -- (93.2000,52.0000) -- (93.5000,49.7000) -- (93.9000,47.9000) -- (94.2000,46.6000) -- (94.5000,45.5000) -- (94.8000,44.8000) -- (95.1000,44.3000) -- (95.4000,44.2000) -- (95.7000,44.4000) -- (96.1000,44.8000) -- (96.4000,45.6000) -- (96.7000,46.7000) -- (97.0000,48.2000) -- (97.3000,50.0000) -- (97.6000,52.1000) -- (97.9000,54.7000) -- (98.3000,57.7000) -- (98.6000,59.4000) -- (98.9000,59.7000) -- (99.2000,59.5000) -- (99.5000,59.2000) -- (99.8000,59.1000) -- (100.1000,59.0000) -- (100.5000,59.1000) -- (100.8000,59.1000) -- (101.1000,59.1000) -- (101.4000,59.1000) -- (101.7000,59.1000) -- (102.0000,59.0000) -- (102.3000,59.5000) -- (102.7000,62.6000) -- (103.0000,66.8000) -- (103.3000,70.7000) -- (103.6000,74.2000) -- (103.9000,77.5000) -- (104.2000,80.5000) -- (104.5000,83.2000) -- (104.9000,85.3000) -- (105.2000,86.7000) -- (105.5000,87.1000) -- (105.8000,86.3000) -- (106.1000,84.5000) -- (106.4000,81.8000) -- (106.7000,78.6000) -- (107.0000,75.1000) -- (107.4000,71.6000) -- (107.7000,67.2000) -- (108.0000,62.3000) -- (108.3000,59.0000) -- (108.6000,58.0000) -- (108.9000,58.3000) -- (109.3000,58.7000) -- (109.6000,59.0000) -- (109.9000,59.1000) -- (110.2000,59.1000) -- (110.5000,59.1000) -- (110.8000,59.1000) -- (111.1000,59.1000) -- (111.4000,59.0000) -- (111.8000,60.0000) -- (112.1000,62.4000) -- (112.4000,60.2000) -- (112.7000,56.4000) -- (113.0000,53.1000) -- (113.3000,50.5000) -- (113.6000,48.6000) -- (114.0000,47.1000) -- (114.3000,45.9000) -- (114.6000,45.0000) -- (114.9000,44.5000) -- (115.2000,44.2000) -- (115.5000,44.3000) -- (115.8000,44.6000) -- (116.2000,45.3000) -- (116.5000,46.3000) -- (116.8000,47.6000) -- (117.1000,49.4000) -- (117.4000,51.4000) -- (117.7000,53.9000) -- (118.0000,56.8000) -- (118.4000,59.0000) -- (118.7000,59.7000) -- (119.0000,59.6000) -- (119.3000,59.3000) -- (119.6000,59.1000) -- (119.9000,59.1000) -- (120.2000,59.1000) -- (120.6000,59.1000) -- (120.9000,59.1000) -- (121.2000,59.1000) -- (121.5000,59.1000) -- (121.8000,59.0000) -- (122.1000,59.2000) -- (122.4000,61.3000) -- (122.8000,65.4000) -- (123.1000,69.6000) -- (123.4000,73.3000) -- (123.7000,76.6000) -- (124.0000,79.6000) -- (124.3000,82.4000) -- (124.6000,84.7000) -- (125.0000,86.4000) -- (125.3000,87.1000) -- (125.6000,86.7000) -- (125.9000,85.1000) -- (126.2000,82.5000) -- (126.5000,79.4000) -- (126.8000,76.0000) -- (127.2000,72.5000) -- (127.5000,68.4000) -- (127.8000,63.4000) -- (128.1000,59.6000) -- (128.4000,58.1000) -- (128.7000,58.2000) -- (129.0000,58.6000) -- (129.4000,59.0000) -- (129.7000,59.1000) -- (130.0000,59.1000) -- (130.3000,59.1000) -- (130.6000,59.1000) -- (130.9000,59.1000) -- (131.2000,59.0000) -- (131.6000,59.4000) -- (131.9000,62.2000) -- (132.2000,61.0000) -- (132.5000,57.3000) -- (132.8000,53.8000) -- (133.1000,51.1000) -- (133.4000,49.0000) -- (133.8000,47.4000) -- (134.1000,46.1000) -- (134.4000,45.2000) -- (134.7000,44.6000) -- (135.0000,44.2000) -- (135.3000,44.2000) -- (135.6000,44.5000) -- (136.0000,45.1000) -- (136.3000,46.0000) -- (136.6000,47.3000) -- (136.9000,49.0000) -- (137.2000,51.0000) -- (137.5000,53.3000) -- (137.8000,56.2000) -- (138.2000,58.6000) -- (138.5000,59.6000) -- (138.8000,59.6000) -- (139.1000,59.3000) -- (139.4000,59.1000) -- (139.7000,59.1000) -- (140.0000,59.1000) -- (140.4000,59.1000) -- (140.7000,59.1000) -- (141.0000,59.1000) -- (141.3000,59.1000) -- (141.6000,59.1000) -- (141.9000,59.0000) -- (142.2000,60.6000) -- (142.6000,64.5000) -- (142.7000,66.5000);



  \end{scope}
  \begin{scope}[cm={{1.00588,0.0,0.0,1.00588,(-399.05137,-123.74949)}},draw=blue,line cap=rect,line join=bevel,line width=0.800pt]
    \path[fill=blue] (0.0000,0.0000) node[above right] (text326) {\scriptsize $y(t)$};



  \end{scope}
  \begin{scope}[cm={{1.26012,0.0,0.0,1.26012,(-205.87067,-45.42881)}},draw=cff0000,line cap=round,line join=round,line width=0.480pt]
    \path[draw,even odd rule] (-137.1848,-65.9213) -- (-111.1848,-65.9213);



  \end{scope}
  \begin{scope}[cm={{1.26012,0.0,0.0,1.26012,(-482.16108,-168.79053)}},draw=cff0000,line cap=round,line join=round,line width=0.480pt]
    \path[draw] (56.0000,44.3000) -- (56.0000,44.3000) -- (56.1000,44.4000) -- (56.2000,44.6000) -- (56.3000,44.8000) -- (56.3000,45.1000) -- (56.4000,45.4000) -- (56.5000,45.7000) -- (56.6000,46.1000) -- (56.7000,46.5000) -- (56.8000,47.0000) -- (56.9000,47.4000) -- (57.0000,47.9000) -- (57.0000,48.4000) -- (57.1000,48.9000) -- (57.2000,49.5000) -- (57.3000,50.0000) -- (57.4000,50.6000) -- (57.5000,51.2000) -- (57.6000,51.8000) -- (57.6000,52.3000) -- (57.7000,52.9000) -- (57.8000,53.5000) -- (57.9000,54.0000) -- (58.0000,54.6000) -- (58.1000,55.1000) -- (58.2000,55.6000) -- (58.3000,56.1000) -- (58.3000,56.6000) -- (58.4000,57.1000) -- (58.5000,57.5000) -- (58.6000,57.9000) -- (58.7000,58.2000) -- (58.8000,58.6000) -- (58.9000,58.9000) -- (59.0000,59.1000) -- (59.0000,59.4000) -- (59.1000,59.6000) -- (59.2000,59.8000) -- (59.3000,59.9000) -- (59.4000,60.0000) -- (59.5000,60.1000) -- (59.6000,60.1000) -- (59.6000,60.1000) -- (59.7000,60.1000) -- (59.8000,60.1000) -- (59.9000,60.0000) -- (60.0000,60.0000) -- (60.1000,59.9000) -- (60.2000,59.7000) -- (60.3000,59.6000) -- (60.3000,59.5000) -- (60.4000,59.3000) -- (60.5000,59.2000) -- (60.6000,59.0000) -- (60.7000,58.9000) -- (60.8000,58.8000) -- (60.9000,58.6000) -- (60.9000,58.5000) -- (61.0000,58.4000) -- (61.1000,58.3000) -- (61.2000,58.2000) -- (61.3000,58.2000) -- (61.4000,58.2000) -- (61.5000,58.2000) -- (61.6000,58.3000) -- (61.6000,58.3000) -- (61.7000,58.5000) -- (61.8000,58.6000) -- (61.9000,58.8000) -- (62.0000,59.1000) -- (62.1000,59.3000) -- (62.2000,59.7000) -- (62.2000,60.0000) -- (62.3000,60.4000) -- (62.4000,60.9000) -- (62.5000,61.4000) -- (62.6000,61.9000) -- (62.7000,62.5000) -- (62.8000,63.1000) -- (62.9000,63.8000) -- (62.9000,64.4000) -- (63.0000,65.2000) -- (63.1000,65.9000) -- (63.2000,66.7000) -- (63.3000,67.5000) -- (63.4000,68.3000) -- (63.5000,69.2000) -- (63.6000,70.0000) -- (63.6000,70.9000) -- (63.7000,71.8000) -- (63.8000,72.7000) -- (63.9000,73.6000) -- (64.0000,74.4000) -- (64.1000,75.3000) -- (64.2000,76.2000) -- (64.2000,77.0000) -- (64.3000,77.8000) -- (64.4000,78.6000) -- (64.5000,79.4000) -- (64.6000,80.1000) -- (64.7000,80.8000) -- (64.8000,81.5000) -- (64.9000,82.1000) -- (64.9000,82.7000) -- (65.0000,83.2000) -- (65.1000,83.7000) -- (65.2000,84.1000) -- (65.3000,84.4000) -- (65.4000,84.7000) -- (65.5000,85.0000) -- (65.5000,85.2000) -- (65.6000,85.3000) -- (65.7000,85.3000) -- (65.8000,85.3000) -- (65.9000,85.3000) -- (66.0000,85.1000) -- (66.1000,84.9000) -- (66.2000,84.7000) -- (66.2000,84.3000) -- (66.3000,84.0000) -- (66.4000,83.5000) -- (66.5000,83.1000) -- (66.6000,82.5000) -- (66.7000,81.9000) -- (66.8000,81.3000) -- (66.8000,80.6000) -- (66.9000,79.9000) -- (67.0000,79.2000) -- (67.1000,78.4000) -- (67.2000,77.6000) -- (67.3000,76.7000) -- (67.4000,75.9000) -- (67.5000,75.0000) -- (67.5000,74.1000) -- (67.6000,73.2000) -- (67.7000,72.3000) -- (67.8000,71.4000) -- (67.9000,70.5000) -- (68.0000,69.6000) -- (68.1000,68.7000) -- (68.2000,67.9000) -- (68.2000,67.0000) -- (68.3000,66.2000) -- (68.4000,65.4000) -- (68.5000,64.7000) -- (68.6000,63.9000) -- (68.7000,63.2000) -- (68.8000,62.5000) -- (68.8000,61.9000) -- (68.9000,61.3000) -- (69.0000,60.8000) -- (69.1000,60.2000) -- (69.2000,59.8000) -- (69.3000,59.4000) -- (69.4000,59.0000) -- (69.5000,58.6000) -- (69.5000,58.3000) -- (69.6000,58.1000) -- (69.7000,57.8000) -- (69.8000,57.7000) -- (69.9000,57.5000) -- (70.0000,57.4000) -- (70.1000,57.3000) -- (70.1000,57.3000) -- (70.2000,57.3000) -- (70.3000,57.3000) -- (70.4000,57.3000) -- (70.5000,57.4000) -- (70.6000,57.5000) -- (70.7000,57.6000) -- (70.8000,57.7000) -- (70.8000,57.8000) -- (70.9000,57.9000) -- (71.0000,58.0000) -- (71.1000,58.2000) -- (71.2000,58.3000) -- (71.3000,58.4000) -- (71.4000,58.5000) -- (71.4000,58.6000) -- (71.5000,58.7000) -- (71.6000,58.7000) -- (71.7000,58.8000) -- (71.8000,58.8000) -- (71.9000,58.8000) -- (72.0000,58.8000) -- (72.1000,58.7000) -- (72.1000,58.6000) -- (72.2000,58.5000) -- (72.3000,58.3000) -- (72.4000,58.2000) -- (72.5000,57.9000) -- (72.6000,57.7000) -- (72.7000,57.4000) -- (72.8000,57.1000) -- (72.8000,56.8000) -- (72.9000,56.4000) -- (73.0000,56.0000) -- (73.1000,55.6000) -- (73.2000,55.2000) -- (73.3000,54.7000) -- (73.4000,54.3000) -- (73.4000,53.8000) -- (73.5000,53.3000) -- (73.6000,52.7000) -- (73.7000,52.2000) -- (73.8000,51.7000) -- (73.9000,51.1000) -- (74.0000,50.6000) -- (74.1000,50.0000) -- (74.1000,49.5000) -- (74.2000,49.0000) -- (74.3000,48.5000) -- (74.4000,48.0000) -- (74.5000,47.5000) -- (74.6000,47.1000) -- (74.7000,46.6000) -- (74.7000,46.2000) -- (74.8000,45.8000) -- (74.9000,45.5000) -- (75.0000,45.2000) -- (75.1000,44.9000) -- (75.2000,44.7000) -- (75.3000,44.5000) -- (75.4000,44.3000) -- (75.4000,44.2000) -- (75.5000,44.1000) -- (75.6000,44.1000) -- (75.7000,44.1000) -- (75.8000,44.2000) -- (75.9000,44.3000) -- (76.0000,44.4000) -- (76.0000,44.6000) -- (76.1000,44.8000) -- (76.2000,45.1000) -- (76.3000,45.4000) -- (76.4000,45.8000) -- (76.5000,46.1000) -- (76.6000,46.5000) -- (76.7000,47.0000) -- (76.7000,47.4000) -- (76.8000,47.9000) -- (76.9000,48.5000) -- (77.0000,49.0000) -- (77.1000,49.5000) -- (77.2000,50.1000) -- (77.3000,50.7000) -- (77.3000,51.2000) -- (77.4000,51.8000) -- (77.5000,52.4000) -- (77.6000,53.0000) -- (77.7000,53.5000) -- (77.8000,54.1000) -- (77.9000,54.6000) -- (78.0000,55.2000) -- (78.0000,55.7000) -- (78.1000,56.2000) -- (78.2000,56.7000) -- (78.3000,57.1000) -- (78.4000,57.5000) -- (78.5000,57.9000) -- (78.6000,58.3000) -- (78.7000,58.6000) -- (78.7000,58.9000) -- (78.8000,59.2000) -- (78.9000,59.4000) -- (79.0000,59.6000) -- (79.1000,59.8000) -- (79.2000,59.9000) -- (79.3000,60.0000) -- (79.3000,60.1000) -- (79.4000,60.1000) -- (79.5000,60.2000) -- (79.6000,60.1000) -- (79.7000,60.1000) -- (79.8000,60.0000) -- (79.9000,60.0000) -- (80.0000,59.9000) -- (80.0000,59.7000) -- (80.1000,59.6000) -- (80.2000,59.5000) -- (80.3000,59.3000) -- (80.4000,59.2000) -- (80.5000,59.0000) -- (80.6000,58.9000) -- (80.6000,58.7000) -- (80.7000,58.6000) -- (80.8000,58.5000) -- (80.9000,58.4000) -- (81.0000,58.3000) -- (81.1000,58.2000) -- (81.2000,58.2000) -- (81.3000,58.2000) -- (81.3000,58.2000) -- (81.4000,58.2000) -- (81.5000,58.3000) -- (81.6000,58.5000) -- (81.7000,58.6000) -- (81.8000,58.8000) -- (81.9000,59.1000) -- (81.9000,59.3000) -- (82.0000,59.7000) -- (82.1000,60.0000) -- (82.2000,60.4000) -- (82.3000,60.9000) -- (82.4000,61.4000) -- (82.5000,61.9000) -- (82.6000,62.5000) -- (82.6000,63.1000) -- (82.7000,63.8000) -- (82.8000,64.5000) -- (82.9000,65.2000) -- (83.0000,66.0000) -- (83.1000,66.8000) -- (83.2000,67.6000) -- (83.3000,68.4000) -- (83.3000,69.3000) -- (83.4000,70.1000) -- (83.5000,71.0000) -- (83.6000,71.9000) -- (83.7000,72.8000) -- (83.8000,73.7000) -- (83.9000,74.5000) -- (83.9000,75.4000) -- (84.0000,76.3000) -- (84.1000,77.1000) -- (84.2000,77.9000) -- (84.3000,78.7000) -- (84.4000,79.5000) -- (84.5000,80.2000) -- (84.6000,80.9000) -- (84.6000,81.6000) -- (84.7000,82.2000) -- (84.8000,82.8000) -- (84.9000,83.3000) -- (85.0000,83.7000) -- (85.1000,84.2000) -- (85.2000,84.5000) -- (85.2000,84.8000) -- (85.3000,85.0000) -- (85.4000,85.2000) -- (85.5000,85.3000) -- (85.6000,85.4000) -- (85.7000,85.4000) -- (85.8000,85.3000) -- (85.9000,85.1000) -- (85.9000,84.9000) -- (86.0000,84.7000) -- (86.1000,84.4000) -- (86.2000,84.0000) -- (86.3000,83.5000) -- (86.4000,83.0000) -- (86.5000,82.5000) -- (86.5000,81.9000) -- (86.6000,81.3000) -- (86.7000,80.6000) -- (86.8000,79.9000) -- (86.9000,79.1000) -- (87.0000,78.3000) -- (87.1000,77.5000) -- (87.2000,76.7000) -- (87.2000,75.8000) -- (87.3000,74.9000) -- (87.4000,74.0000) -- (87.5000,73.1000) -- (87.6000,72.2000) -- (87.7000,71.3000) -- (87.8000,70.4000) -- (87.9000,69.5000) -- (87.9000,68.7000) -- (88.0000,67.8000) -- (88.1000,67.0000) -- (88.2000,66.1000) -- (88.3000,65.3000) -- (88.4000,64.6000) -- (88.5000,63.8000) -- (88.5000,63.1000) -- (88.6000,62.5000) -- (88.7000,61.8000) -- (88.8000,61.2000) -- (88.9000,60.7000) -- (89.0000,60.2000) -- (89.1000,59.7000) -- (89.2000,59.3000) -- (89.2000,58.9000) -- (89.3000,58.6000) -- (89.4000,58.3000) -- (89.5000,58.0000) -- (89.6000,57.8000) -- (89.7000,57.6000) -- (89.8000,57.5000) -- (89.8000,57.4000) -- (89.9000,57.3000) -- (90.0000,57.3000) -- (90.1000,57.3000) -- (90.2000,57.3000) -- (90.3000,57.3000) -- (90.4000,57.4000) -- (90.5000,57.5000) -- (90.5000,57.6000) -- (90.6000,57.7000) -- (90.7000,57.8000) -- (90.8000,57.9000) -- (90.9000,58.1000) -- (91.0000,58.2000) -- (91.1000,58.3000) -- (91.1000,58.4000) -- (91.2000,58.5000) -- (91.3000,58.6000) -- (91.4000,58.7000) -- (91.5000,58.8000) -- (91.6000,58.8000) -- (91.7000,58.8000) -- (91.8000,58.8000) -- (91.8000,58.8000) -- (91.9000,58.7000) -- (92.0000,58.6000) -- (92.1000,58.5000) -- (92.2000,58.3000) -- (92.3000,58.2000) -- (92.4000,57.9000) -- (92.5000,57.7000) -- (92.5000,57.4000) -- (92.6000,57.1000) -- (92.7000,56.8000) -- (92.8000,56.4000) -- (92.9000,56.0000) -- (93.0000,55.6000) -- (93.1000,55.2000) -- (93.1000,54.7000) -- (93.2000,54.2000) -- (93.3000,53.7000) -- (93.4000,53.2000) -- (93.5000,52.7000) -- (93.6000,52.2000) -- (93.7000,51.6000) -- (93.8000,51.1000) -- (93.8000,50.5000) -- (93.9000,50.0000) -- (94.0000,49.5000) -- (94.1000,48.9000) -- (94.2000,48.4000) -- (94.3000,47.9000) -- (94.4000,47.5000) -- (94.4000,47.0000) -- (94.5000,46.6000) -- (94.6000,46.2000) -- (94.7000,45.8000) -- (94.8000,45.4000) -- (94.9000,45.1000) -- (95.0000,44.9000) -- (95.1000,44.6000) -- (95.1000,44.4000) -- (95.2000,44.3000) -- (95.3000,44.2000) -- (95.4000,44.1000) -- (95.5000,44.1000) -- (95.6000,44.1000) -- (95.7000,44.2000) -- (95.7000,44.3000) -- (95.8000,44.4000) -- (95.9000,44.6000) -- (96.0000,44.8000) -- (96.1000,45.1000) -- (96.2000,45.4000) -- (96.3000,45.8000) -- (96.4000,46.2000) -- (96.4000,46.6000) -- (96.5000,47.0000) -- (96.6000,47.5000) -- (96.7000,48.0000) -- (96.8000,48.5000) -- (96.9000,49.0000) -- (97.0000,49.6000) -- (97.1000,50.1000) -- (97.1000,50.7000) -- (97.2000,51.3000) -- (97.3000,51.9000) -- (97.4000,52.4000) -- (97.5000,53.0000) -- (97.6000,53.6000) -- (97.7000,54.2000) -- (97.7000,54.7000) -- (97.8000,55.2000) -- (97.9000,55.8000) -- (98.0000,56.2000) -- (98.1000,56.7000) -- (98.2000,57.2000) -- (98.3000,57.6000) -- (98.4000,58.0000) -- (98.4000,58.3000) -- (98.5000,58.7000) -- (98.6000,59.0000) -- (98.7000,59.2000) -- (98.8000,59.5000) -- (98.9000,59.7000) -- (99.0000,59.8000) -- (99.0000,60.0000) -- (99.1000,60.1000) -- (99.2000,60.1000) -- (99.3000,60.2000) -- (99.4000,60.2000) -- (99.5000,60.2000) -- (99.6000,60.1000) -- (99.7000,60.0000) -- (99.7000,60.0000) -- (99.8000,59.9000) -- (99.9000,59.7000) -- (100.0000,59.6000) -- (100.1000,59.5000) -- (100.2000,59.3000) -- (100.3000,59.2000) -- (100.3000,59.0000) -- (100.4000,58.9000) -- (100.5000,58.7000) -- (100.6000,58.6000) -- (100.7000,58.5000) -- (100.8000,58.4000) -- (100.9000,58.3000) -- (101.0000,58.2000) -- (101.0000,58.2000) -- (101.1000,58.2000) -- (101.2000,58.2000) -- (101.3000,58.2000) -- (101.4000,58.3000) -- (101.5000,58.4000) -- (101.6000,58.6000) -- (101.7000,58.8000) -- (101.7000,59.1000) -- (101.8000,59.3000) -- (101.9000,59.7000) -- (102.0000,60.0000) -- (102.1000,60.5000) -- (102.2000,60.9000) -- (102.3000,61.4000) -- (102.3000,62.0000) -- (102.4000,62.6000) -- (102.5000,63.2000) -- (102.6000,63.8000) -- (102.7000,64.5000) -- (102.8000,65.3000) -- (102.9000,66.0000) -- (103.0000,66.8000) -- (103.0000,67.6000) -- (103.1000,68.5000) -- (103.2000,69.3000) -- (103.3000,70.2000) -- (103.4000,71.1000) -- (103.5000,72.0000) -- (103.6000,72.8000) -- (103.6000,73.7000) -- (103.7000,74.6000) -- (103.8000,75.5000) -- (103.9000,76.4000) -- (104.0000,77.2000) -- (104.1000,78.0000) -- (104.2000,78.8000) -- (104.3000,79.6000) -- (104.3000,80.3000) -- (104.4000,81.0000) -- (104.5000,81.7000) -- (104.6000,82.3000) -- (104.7000,82.8000) -- (104.8000,83.4000) -- (104.9000,83.8000) -- (104.9000,84.2000) -- (105.0000,84.6000) -- (105.1000,84.9000) -- (105.2000,85.1000) -- (105.3000,85.3000) -- (105.4000,85.4000) -- (105.5000,85.4000) -- (105.6000,85.4000) -- (105.6000,85.3000) -- (105.7000,85.2000) -- (105.8000,85.0000) -- (105.9000,84.7000) -- (106.0000,84.4000) -- (106.1000,84.0000) -- (106.2000,83.5000) -- (106.3000,83.0000) -- (106.3000,82.5000) -- (106.4000,81.9000) -- (106.5000,81.2000) -- (106.6000,80.5000) -- (106.7000,79.8000) -- (106.8000,79.1000) -- (106.9000,78.3000) -- (106.9000,77.4000) -- (107.0000,76.6000) -- (107.1000,75.7000) -- (107.2000,74.8000) -- (107.3000,73.9000) -- (107.4000,73.0000) -- (107.5000,72.1000) -- (107.6000,71.2000) -- (107.6000,70.3000) -- (107.7000,69.4000) -- (107.8000,68.6000) -- (107.9000,67.7000) -- (108.0000,66.9000) -- (108.1000,66.0000) -- (108.2000,65.2000) -- (108.2000,64.5000) -- (108.3000,63.7000) -- (108.4000,63.0000) -- (108.5000,62.4000) -- (108.6000,61.8000) -- (108.7000,61.2000) -- (108.8000,60.6000) -- (108.9000,60.1000) -- (108.9000,59.6000) -- (109.0000,59.2000) -- (109.1000,58.9000) -- (109.2000,58.5000) -- (109.3000,58.2000) -- (109.4000,58.0000) -- (109.5000,57.8000) -- (109.5000,57.6000) -- (109.6000,57.5000) -- (109.7000,57.4000) -- (109.8000,57.3000) -- (109.9000,57.3000) -- (110.0000,57.3000) -- (110.1000,57.3000) -- (110.2000,57.3000) -- (110.2000,57.4000) -- (110.3000,57.5000) -- (110.4000,57.6000) -- (110.5000,57.7000) -- (110.6000,57.8000) -- (110.7000,57.9000) -- (110.8000,58.1000) -- (110.9000,58.2000) -- (110.9000,58.3000) -- (111.0000,58.4000) -- (111.1000,58.5000) -- (111.2000,58.6000) -- (111.3000,58.7000) -- (111.4000,58.8000) -- (111.5000,58.8000) -- (111.5000,58.8000) -- (111.6000,58.8000) -- (111.7000,58.8000) -- (111.8000,58.7000) -- (111.9000,58.6000) -- (112.0000,58.5000) -- (112.1000,58.3000) -- (112.2000,58.2000) -- (112.2000,57.9000) -- (112.3000,57.7000) -- (112.4000,57.4000) -- (112.5000,57.1000) -- (112.6000,56.8000) -- (112.7000,56.4000) -- (112.8000,56.0000) -- (112.8000,55.6000) -- (112.9000,55.1000) -- (113.0000,54.7000) -- (113.1000,54.2000) -- (113.2000,53.7000) -- (113.3000,53.2000) -- (113.4000,52.7000) -- (113.5000,52.1000) -- (113.5000,51.6000) -- (113.6000,51.0000) -- (113.7000,50.5000) -- (113.8000,49.9000) -- (113.9000,49.4000) -- (114.0000,48.9000) -- (114.1000,48.4000) -- (114.1000,47.9000) -- (114.2000,47.4000) -- (114.3000,46.9000) -- (114.4000,46.5000) -- (114.5000,46.1000) -- (114.6000,45.7000) -- (114.7000,45.4000) -- (114.8000,45.1000) -- (114.8000,44.8000) -- (114.9000,44.6000) -- (115.0000,44.4000) -- (115.1000,44.3000) -- (115.2000,44.1000) -- (115.3000,44.1000) -- (115.4000,44.1000) -- (115.5000,44.1000) -- (115.5000,44.1000) -- (115.6000,44.3000) -- (115.7000,44.4000) -- (115.8000,44.6000) -- (115.9000,44.8000) -- (116.0000,45.1000) -- (116.1000,45.4000) -- (116.1000,45.8000) -- (116.2000,46.2000) -- (116.3000,46.6000) -- (116.4000,47.0000) -- (116.5000,47.5000) -- (116.6000,48.0000) -- (116.7000,48.5000) -- (116.8000,49.1000) -- (116.8000,49.6000) -- (116.9000,50.2000) -- (117.0000,50.8000) -- (117.1000,51.3000) -- (117.2000,51.9000) -- (117.3000,52.5000) -- (117.4000,53.1000) -- (117.4000,53.6000) -- (117.5000,54.2000) -- (117.6000,54.8000) -- (117.7000,55.3000) -- (117.8000,55.8000) -- (117.9000,56.3000) -- (118.0000,56.8000) -- (118.1000,57.2000) -- (118.1000,57.6000) -- (118.2000,58.0000) -- (118.3000,58.4000) -- (118.4000,58.7000) -- (118.5000,59.0000) -- (118.6000,59.3000) -- (118.7000,59.5000) -- (118.7000,59.7000) -- (118.8000,59.9000) -- (118.9000,60.0000) -- (119.0000,60.1000) -- (119.1000,60.1000) -- (119.2000,60.2000) -- (119.3000,60.2000) -- (119.4000,60.2000) -- (119.4000,60.1000) -- (119.5000,60.1000) -- (119.6000,60.0000) -- (119.7000,59.9000) -- (119.8000,59.7000) -- (119.9000,59.6000) -- (120.0000,59.5000) -- (120.0000,59.3000) -- (120.1000,59.1000) -- (120.2000,59.0000) -- (120.3000,58.8000) -- (120.4000,58.7000) -- (120.5000,58.6000) -- (120.6000,58.4000) -- (120.7000,58.3000) -- (120.7000,58.2000) -- (120.8000,58.2000) -- (120.9000,58.2000) -- (121.0000,58.1000) -- (121.1000,58.2000) -- (121.2000,58.2000) -- (121.3000,58.3000) -- (121.4000,58.4000) -- (121.4000,58.6000) -- (121.5000,58.8000) -- (121.6000,59.1000) -- (121.7000,59.3000) -- (121.8000,59.7000) -- (121.9000,60.1000) -- (122.0000,60.5000) -- (122.0000,60.9000) -- (122.1000,61.5000) -- (122.2000,62.0000) -- (122.3000,62.6000) -- (122.4000,63.2000) -- (122.5000,63.9000) -- (122.6000,64.6000) -- (122.7000,65.3000) -- (122.7000,66.1000) -- (122.8000,66.9000) -- (122.9000,67.7000) -- (123.0000,68.5000) -- (123.1000,69.4000) -- (123.2000,70.3000) -- (123.3000,71.1000) -- (123.3000,72.0000) -- (123.4000,72.9000) -- (123.5000,73.8000) -- (123.6000,74.7000) -- (123.7000,75.6000) -- (123.8000,76.4000) -- (123.9000,77.3000) -- (124.0000,78.1000) -- (124.0000,78.9000) -- (124.1000,79.7000) -- (124.2000,80.4000) -- (124.3000,81.1000) -- (124.4000,81.7000) -- (124.5000,82.4000) -- (124.6000,82.9000) -- (124.6000,83.4000) -- (124.7000,83.9000) -- (124.8000,84.3000) -- (124.9000,84.6000) -- (125.0000,84.9000) -- (125.1000,85.2000) -- (125.2000,85.3000) -- (125.3000,85.4000) -- (125.3000,85.5000) -- (125.4000,85.4000) -- (125.5000,85.3000) -- (125.6000,85.2000) -- (125.7000,85.0000) -- (125.8000,84.7000) -- (125.9000,84.4000) -- (125.9000,84.0000) -- (126.0000,83.5000) -- (126.1000,83.0000) -- (126.2000,82.4000) -- (126.3000,81.8000) -- (126.4000,81.2000) -- (126.5000,80.5000) -- (126.6000,79.8000) -- (126.6000,79.0000) -- (126.7000,78.2000) -- (126.8000,77.4000) -- (126.9000,76.5000) -- (127.0000,75.7000) -- (127.1000,74.8000) -- (127.2000,73.9000) -- (127.3000,73.0000) -- (127.3000,72.1000) -- (127.4000,71.2000) -- (127.5000,70.3000) -- (127.6000,69.4000) -- (127.7000,68.5000) -- (127.8000,67.6000) -- (127.9000,66.8000) -- (127.9000,66.0000) -- (128.0000,65.2000) -- (128.1000,64.4000) -- (128.2000,63.7000) -- (128.3000,63.0000) -- (128.4000,62.3000) -- (128.5000,61.7000) -- (128.6000,61.1000) -- (128.6000,60.5000) -- (128.7000,60.0000) -- (128.8000,59.6000) -- (128.9000,59.2000) -- (129.0000,58.8000) -- (129.1000,58.5000) -- (129.2000,58.2000) -- (129.2000,57.9000) -- (129.3000,57.7000) -- (129.4000,57.6000) -- (129.5000,57.4000) -- (129.6000,57.3000) -- (129.7000,57.3000) -- (129.8000,57.2000) -- (129.9000,57.3000) -- (129.9000,57.3000) -- (130.0000,57.3000) -- (130.1000,57.4000) -- (130.2000,57.5000) -- (130.3000,57.6000) -- (130.4000,57.7000) -- (130.5000,57.8000) -- (130.5000,58.0000) -- (130.6000,58.1000) -- (130.7000,58.2000) -- (130.8000,58.3000) -- (130.9000,58.5000) -- (131.0000,58.6000) -- (131.1000,58.7000) -- (131.2000,58.7000) -- (131.2000,58.8000) -- (131.3000,58.8000) -- (131.4000,58.8000) -- (131.5000,58.8000) -- (131.6000,58.8000) -- (131.7000,58.7000) -- (131.8000,58.6000) -- (131.8000,58.5000) -- (131.9000,58.3000) -- (132.0000,58.2000) -- (132.1000,57.9000) -- (132.2000,57.7000) -- (132.3000,57.4000) -- (132.4000,57.1000) -- (132.5000,56.8000) -- (132.5000,56.4000) -- (132.6000,56.0000) -- (132.7000,55.6000) -- (132.8000,55.1000) -- (132.9000,54.6000) -- (133.0000,54.2000) -- (133.1000,53.7000) -- (133.2000,53.1000) -- (133.2000,52.6000) -- (133.3000,52.1000) -- (133.4000,51.5000) -- (133.5000,51.0000) -- (133.6000,50.4000) -- (133.7000,49.9000) -- (133.8000,49.4000) -- (133.8000,48.8000) -- (133.9000,48.3000) -- (134.0000,47.8000) -- (134.1000,47.3000) -- (134.2000,46.9000) -- (134.3000,46.5000) -- (134.4000,46.1000) -- (134.5000,45.7000) -- (134.5000,45.3000) -- (134.6000,45.0000) -- (134.7000,44.8000) -- (134.8000,44.6000) -- (134.9000,44.4000) -- (135.0000,44.2000) -- (135.1000,44.1000) -- (135.1000,44.1000) -- (135.2000,44.0000) -- (135.3000,44.1000) -- (135.4000,44.1000) -- (135.5000,44.2000) -- (135.6000,44.4000) -- (135.7000,44.6000) -- (135.8000,44.8000) -- (135.8000,45.1000) -- (135.9000,45.4000) -- (136.0000,45.8000) -- (136.1000,46.2000) -- (136.2000,46.6000) -- (136.3000,47.1000) -- (136.4000,47.5000) -- (136.5000,48.0000) -- (136.5000,48.6000) -- (136.6000,49.1000) -- (136.7000,49.7000) -- (136.8000,50.2000) -- (136.9000,50.8000) -- (137.0000,51.4000) -- (137.1000,52.0000) -- (137.1000,52.6000) -- (137.2000,53.1000) -- (137.3000,53.7000) -- (137.4000,54.3000) -- (137.5000,54.8000) -- (137.6000,55.4000) -- (137.7000,55.9000) -- (137.8000,56.4000) -- (137.8000,56.8000) -- (137.9000,57.3000) -- (138.0000,57.7000) -- (138.1000,58.1000) -- (138.2000,58.4000) -- (138.3000,58.8000) -- (138.4000,59.1000) -- (138.4000,59.3000) -- (138.5000,59.5000) -- (138.6000,59.7000) -- (138.7000,59.9000) -- (138.8000,60.0000) -- (138.9000,60.1000) -- (139.0000,60.2000) -- (139.1000,60.2000) -- (139.1000,60.2000) -- (139.2000,60.2000) -- (139.3000,60.1000) -- (139.4000,60.1000) -- (139.5000,60.0000) -- (139.6000,59.9000) -- (139.7000,59.7000) -- (139.7000,59.6000) -- (139.8000,59.4000) -- (139.9000,59.3000) -- (140.0000,59.1000) -- (140.1000,59.0000) -- (140.2000,58.8000) -- (140.3000,58.7000) -- (140.4000,58.5000) -- (140.4000,58.4000) -- (140.5000,58.3000) -- (140.6000,58.2000) -- (140.7000,58.2000) -- (140.8000,58.1000) -- (140.9000,58.1000) -- (141.0000,58.1000) -- (141.1000,58.2000) -- (141.1000,58.3000) -- (141.2000,58.4000) -- (141.3000,58.6000) -- (141.4000,58.8000) -- (141.5000,59.1000) -- (141.6000,59.4000) -- (141.7000,59.7000) -- (141.7000,60.1000) -- (141.8000,60.5000) -- (141.9000,61.0000) -- (142.0000,61.5000) -- (142.1000,62.0000) -- (142.2000,62.6000) -- (142.3000,63.3000) -- (142.4000,63.9000) -- (142.4000,64.6000) -- (142.5000,65.4000) -- (142.6000,66.1000) -- (142.7000,67.0000);



  \end{scope}
  \begin{scope}[cm={{1.26012,0.0,0.0,1.26012,(-482.16108,-168.79053)}},draw=blue,line cap=round,line join=round,line width=0.480pt]
    \path[draw] (56.5000,13.5000) -- (56.5000,95.5000) -- (142.5000,95.5000) -- (142.5000,13.5000) -- (56.5000,13.5000);



  \end{scope}
  \begin{scope}[cm={{1.15801,0.0,0.0,1.15801,(-615.96866,-167.12877)}},draw=ca0a0a4,dash pattern=on 0.40pt off 0.80pt,line cap=round,line join=round,line width=0.400pt]
    \path[draw] (44.5000,102.5000) -- (142.5000,102.5000);



  \end{scope}
  \begin{scope}[cm={{1.15801,0.0,0.0,1.15801,(-615.96866,-167.12877)}},draw=blue,line cap=round,line join=round,line width=0.480pt]
    \path[draw] (44.5000,102.5000) -- (48.5000,102.5000);



    \path[draw] (142.5000,102.5000) -- (139.5000,102.5000);



  \end{scope}
  \begin{scope}[cm={{1.00588,0.0,0.0,1.00588,(-585.77904,-46.49663)}},draw=blue,line cap=rect,line join=bevel,line width=0.800pt]
    \path[fill=blue] (0.0000,0.0000) node[above right] (text34-9) {-100};



  \end{scope}
  \begin{scope}[cm={{1.15801,0.0,0.0,1.15801,(-615.96866,-167.12877)}},draw=ca0a0a4,dash pattern=on 1.55pt off 1.55pt,line cap=round,line join=round,line width=0.259pt,miter limit=4.00]
    \path[draw,dash pattern=on 1.55pt off 1.55pt,line width=0.259pt,miter limit=4.00] (44.5000,79.5000) -- (142.5000,79.5000);



  \end{scope}
  \begin{scope}[cm={{1.15801,0.0,0.0,1.15801,(-615.96866,-167.12877)}},draw=blue,line cap=round,line join=round,line width=0.480pt]
    \path[draw] (44.5000,79.5000) -- (48.5000,79.5000);



    \path[draw] (142.5000,79.5000) -- (139.5000,79.5000);



  \end{scope}
  \begin{scope}[cm={{1.00588,0.0,0.0,1.00588,(-575.04441,-71.62653)}},draw=blue,line cap=rect,line join=bevel,line width=0.800pt]
    \path[fill=blue] (0.0000,0.0000) node[above right] (text64-6) {0};



  \end{scope}
  \begin{scope}[cm={{1.15801,0.0,0.0,1.15801,(-615.96866,-167.12877)}},draw=ca0a0a4,dash pattern=on 1.55pt off 1.55pt,line cap=round,line join=round,line width=0.259pt,miter limit=4.00]
    \path[draw,dash pattern=on 1.55pt off 1.55pt,line width=0.259pt,miter limit=4.00] (44.5000,57.5000) -- (142.5000,57.5000);



  \end{scope}
  \begin{scope}[cm={{1.15801,0.0,0.0,1.15801,(-615.96866,-167.12877)}},draw=blue,line cap=round,line join=round,line width=0.480pt]
    \path[draw] (44.5000,57.5000) -- (48.5000,57.5000);



    \path[draw] (142.5000,57.5000) -- (139.5000,57.5000);



  \end{scope}
  \begin{scope}[cm={{1.00588,0.0,0.0,1.00588,(-583.09145,-98.25592)}},draw=blue,line cap=rect,line join=bevel,line width=0.800pt]
    \path[fill=blue] (0.0000,0.0000) node[above right] (text94-7) {100};



  \end{scope}
  \begin{scope}[cm={{1.15801,0.0,0.0,1.15801,(-615.96866,-167.12877)}},draw=ca0a0a4,dash pattern=on 1.55pt off 1.55pt,line cap=round,line join=round,line width=0.259pt,miter limit=4.00]
    \path[draw,dash pattern=on 1.55pt off 1.55pt,line width=0.259pt,miter limit=4.00] (44.5000,35.5000) -- (142.5000,35.5000);



  \end{scope}
  \begin{scope}[cm={{1.15801,0.0,0.0,1.15801,(-615.96866,-167.12877)}},draw=blue,line cap=round,line join=round,line width=0.480pt]
    \path[draw] (44.5000,35.5000) -- (48.5000,35.5000);



    \path[draw] (142.5000,35.5000) -- (139.5000,35.5000);



  \end{scope}
  \begin{scope}[cm={{1.00588,0.0,0.0,1.00588,(-583.09145,-123.38532)}},draw=blue,line cap=rect,line join=bevel,line width=0.800pt]
    \path[fill=blue] (0.0000,0.0000) node[above right] (text124-9) {200};



  \end{scope}
  \begin{scope}[cm={{1.15801,0.0,0.0,1.15801,(-615.96866,-167.12877)}},draw=ca0a0a4,dash pattern=on 0.40pt off 0.80pt,line cap=round,line join=round,line width=0.400pt]
    \path[draw] (44.5000,13.5000) -- (142.5000,13.5000);



  \end{scope}
  \begin{scope}[cm={{1.15801,0.0,0.0,1.15801,(-615.96866,-167.12877)}},draw=blue,line cap=round,line join=round,line width=0.480pt]
    \path[draw] (44.5000,13.5000) -- (48.5000,13.5000);



    \path[draw] (142.5000,13.5000) -- (139.5000,13.5000);



  \end{scope}
  \begin{scope}[cm={{1.00588,0.0,0.0,1.00588,(-583.09145,-148.51472)}},draw=blue,line cap=rect,line join=bevel,line width=0.800pt]
    \path[fill=blue] (0.0000,0.0000) node[above right] (text154) {300};



  \end{scope}
  \begin{scope}[cm={{1.15801,0.0,0.0,1.15801,(-615.96866,-167.12877)}},draw=ca0a0a4,dash pattern=on 0.40pt off 0.80pt,line cap=round,line join=round,line width=0.400pt]
    \path[draw] (44.5000,102.5000) -- (44.5000,13.5000);



  \end{scope}
  \begin{scope}[cm={{1.15801,0.0,0.0,1.15801,(-615.96866,-167.12877)}},draw=blue,line cap=round,line join=round,line width=0.480pt]
    \path[draw] (44.5000,102.5000) -- (44.5000,99.5000);



    \path[draw] (44.5000,13.5000) -- (44.5000,16.5000);



  \end{scope}
  \begin{scope}[cm={{1.15801,0.0,0.0,1.15801,(-615.96866,-167.12877)}},draw=ca0a0a4,dash pattern=on 1.55pt off 1.55pt,line cap=round,line join=round,line width=0.259pt,miter limit=4.00]
    \path[draw,dash pattern=on 1.55pt off 1.55pt,line width=0.259pt,miter limit=4.00] (69.5000,102.5000) -- (69.5000,27.5000);



    \path[draw,dash pattern=on 1.55pt off 1.55pt,line width=0.259pt,miter limit=4.00] (69.5000,19.5000) -- (69.5000,13.5000);



  \end{scope}
  \begin{scope}[cm={{1.15801,0.0,0.0,1.15801,(-615.96866,-167.12877)}},draw=blue,line cap=round,line join=round,line width=0.480pt]
    \path[draw] (69.5000,102.5000) -- (69.5000,99.5000);



    \path[draw] (69.5000,13.5000) -- (69.5000,16.5000);



  \end{scope}
  \begin{scope}[cm={{1.15801,0.0,0.0,1.15801,(-615.96866,-167.12877)}},draw=ca0a0a4,dash pattern=on 1.55pt off 1.55pt,line cap=round,line join=round,line width=0.259pt,miter limit=4.00]
    \path[draw,dash pattern=on 1.55pt off 1.55pt,line width=0.259pt,miter limit=4.00] (93.5000,102.5000) -- (93.5000,27.5000);



    \path[draw,dash pattern=on 1.55pt off 1.55pt,line width=0.259pt,miter limit=4.00] (93.5000,19.5000) -- (93.5000,13.5000);



  \end{scope}
  \begin{scope}[cm={{1.15801,0.0,0.0,1.15801,(-615.96866,-167.12877)}},draw=blue,line cap=round,line join=round,line width=0.480pt]
    \path[draw] (93.5000,102.5000) -- (93.5000,99.5000);



    \path[draw] (93.5000,13.5000) -- (93.5000,16.5000);



  \end{scope}
  \begin{scope}[cm={{1.15801,0.0,0.0,1.15801,(-615.96866,-167.12877)}},draw=ca0a0a4,dash pattern=on 1.55pt off 1.55pt,line cap=round,line join=round,line width=0.259pt,miter limit=4.00]
    \path[draw,dash pattern=on 1.55pt off 1.55pt,line width=0.259pt,miter limit=4.00] (118.5000,102.5000) -- (118.5000,13.5000);



  \end{scope}
  \begin{scope}[cm={{1.15801,0.0,0.0,1.15801,(-615.96866,-167.12877)}},draw=blue,line cap=round,line join=round,line width=0.480pt]
    \path[draw] (118.5000,102.5000) -- (118.5000,99.5000);



    \path[draw] (118.5000,13.5000) -- (118.5000,16.5000);



  \end{scope}
  \begin{scope}[cm={{1.15801,0.0,0.0,1.15801,(-615.96866,-167.12877)}},draw=ca0a0a4,dash pattern=on 0.40pt off 0.80pt,line cap=round,line join=round,line width=0.400pt]
    \path[draw] (142.5000,102.5000) -- (142.5000,13.5000);



  \end{scope}
  \begin{scope}[cm={{1.15801,0.0,0.0,1.15801,(-615.96866,-167.12877)}},draw=blue,line cap=round,line join=round,line width=0.480pt]
    \path[draw] (142.5000,102.5000) -- (142.5000,99.5000);



    \path[draw] (142.5000,13.5000) -- (142.5000,16.5000);



  \end{scope}
  \begin{scope}[cm={{1.15801,0.0,0.0,1.15801,(-615.96866,-167.12877)}},draw=blue,line cap=round,line join=round,line width=0.480pt]
    \path[draw] (44.5000,13.5000) -- (44.5000,102.5000) -- (142.5000,102.5000) -- (142.5000,13.5000) -- (44.5000,13.5000);



  \end{scope}
  \begin{scope}[cm={{0.84029,0.0,0.0,0.84029,(-557.89843,-136.56808)}},draw=blue,line cap=rect,line join=bevel,line width=0.800pt]
    \path[fill=blue] (0.0000,0.0000) node[above right] (text360) {\scriptsize $\mathbf{p}(t)$};



  \end{scope}
  \begin{scope}[cm={{1.15801,0.0,0.0,1.15801,(-623.4871,-168.38184)}},draw=blue,line cap=round,line join=round,line width=0.480pt]
    \path[draw,even odd rule] (74.5000,23.5000) -- (101.5000,23.5000);



  \end{scope}
  \begin{scope}[cm={{1.15801,0.0,0.0,1.15801,(-615.96866,-167.12877)}},draw=blue,line cap=round,line join=round,line width=0.480pt]
    \path[draw] (57.5000,32.8000) -- (57.5000,32.8000) -- (58.3000,35.4000) -- (57.8000,37.2000) -- (56.3000,38.9000) -- (54.9000,41.1000) -- (53.9000,43.5000) -- (53.5000,46.1000) -- (53.4000,48.6000) -- (53.4000,51.1000) -- (53.4000,53.6000) -- (53.4000,56.1000) -- (53.4000,58.6000) -- (53.4000,61.1000) -- (53.4000,63.6000) -- (53.4000,66.1000) -- (53.4000,68.7000) -- (53.4000,71.2000) -- (53.4000,73.7000) -- (53.4000,76.2000) -- (53.4000,78.7000) -- (53.9000,81.3000) -- (54.7000,83.7000) -- (56.0000,86.0000) -- (57.6000,88.1000) -- (59.5000,89.9000) -- (61.7000,91.4000) -- (64.1000,92.6000) -- (66.6000,93.4000) -- (69.2000,93.9000) -- (71.8000,94.1000) -- (74.4000,93.9000) -- (77.0000,93.4000) -- (79.5000,92.6000) -- (81.8000,91.4000) -- (84.0000,90.0000) -- (85.9000,88.2000) -- (87.5000,86.1000) -- (88.6000,83.8000) -- (89.1000,81.3000) -- (89.2000,78.8000) -- (89.3000,76.2000) -- (89.3000,73.7000) -- (89.3000,71.1000) -- (89.3000,68.6000) -- (89.3000,66.1000) -- (89.3000,63.6000) -- (89.3000,61.0000) -- (89.3000,58.5000) -- (89.3000,56.0000) -- (89.3000,53.5000) -- (89.6000,51.0000) -- (90.0000,48.5000) -- (89.9000,46.0000) -- (89.4000,43.6000) -- (88.5000,41.3000) -- (87.2000,39.1000) -- (85.5000,37.2000) -- (83.5000,35.5000) -- (81.3000,34.1000) -- (78.9000,33.1000) -- (76.3000,32.5000) -- (73.6000,32.2000) -- (71.0000,32.3000) -- (68.3000,32.8000) -- (65.8000,33.7000) -- (63.5000,34.9000) -- (61.4000,36.5000) -- (59.6000,38.3000) -- (58.2000,40.4000) -- (57.1000,42.7000) -- (56.5000,45.2000) -- (56.3000,47.7000) -- (56.3000,50.1000) -- (56.3000,52.6000) -- (56.3000,55.1000) -- (56.3000,57.6000) -- (56.3000,60.2000) -- (56.3000,62.7000) -- (56.3000,65.2000) -- (56.3000,67.7000) -- (56.3000,70.2000) -- (56.3000,72.7000) -- (56.3000,75.3000) -- (56.3000,77.8000) -- (56.6000,80.3000) -- (57.3000,82.8000) -- (58.5000,85.1000) -- (60.1000,87.2000) -- (61.9000,89.1000) -- (64.1000,90.7000) -- (66.4000,91.9000) -- (68.9000,92.8000) -- (71.5000,93.4000) -- (74.1000,93.6000) -- (76.7000,93.5000) -- (79.3000,93.0000) -- (81.8000,92.2000) -- (84.1000,91.0000) -- (86.3000,89.5000) -- (88.2000,87.7000) -- (89.8000,85.7000) -- (90.9000,83.4000) -- (91.4000,80.9000) -- (91.6000,78.3000) -- (91.6000,75.8000) -- (91.6000,73.2000) -- (91.6000,70.7000) -- (91.6000,68.2000) -- (91.6000,65.6000) -- (91.6000,63.1000) -- (91.6000,60.6000) -- (91.6000,58.1000) -- (91.6000,55.6000) -- (91.6000,53.0000) -- (91.9000,50.5000) -- (92.3000,48.0000) -- (92.2000,45.6000) -- (91.7000,43.2000) -- (90.7000,40.9000) -- (89.4000,38.7000) -- (87.7000,36.8000) -- (85.6000,35.2000) -- (83.3000,33.9000) -- (80.9000,32.9000) -- (78.3000,32.4000) -- (75.6000,32.2000) -- (72.9000,32.4000) -- (70.3000,33.0000) -- (67.9000,33.9000) -- (65.6000,35.2000) -- (63.6000,36.9000) -- (61.9000,38.8000) -- (60.6000,41.0000) -- (59.7000,43.4000) -- (59.3000,45.9000) -- (59.2000,48.3000) -- (59.1000,50.8000) -- (59.1000,53.3000) -- (59.1000,55.8000) -- (59.1000,58.3000) -- (59.1000,60.9000) -- (59.1000,63.4000) -- (59.1000,65.9000) -- (59.1000,68.4000) -- (59.1000,70.9000) -- (59.1000,73.4000) -- (59.1000,76.0000) -- (59.2000,78.5000) -- (59.6000,81.0000) -- (60.6000,83.5000) -- (61.9000,85.7000) -- (63.6000,87.7000) -- (65.6000,89.4000) -- (67.8000,90.9000) -- (70.2000,92.0000) -- (72.7000,92.8000) -- (75.3000,93.2000) -- (78.0000,93.2000) -- (80.6000,92.9000) -- (83.1000,92.3000) -- (85.5000,91.3000) -- (87.8000,89.9000) -- (89.8000,88.3000) -- (91.6000,86.3000) -- (92.9000,84.1000) -- (93.6000,81.7000) -- (93.9000,79.2000) -- (94.0000,76.6000) -- (94.0000,74.0000) -- (94.1000,71.5000) -- (94.1000,69.0000) -- (94.1000,66.4000) -- (94.1000,63.9000) -- (94.1000,61.4000) -- (94.1000,58.9000) -- (94.1000,56.4000) -- (94.1000,53.9000) -- (94.2000,51.3000) -- (94.6000,48.8000) -- (94.7000,46.4000) -- (94.4000,43.9000) -- (93.6000,41.6000) -- (92.4000,39.3000) -- (90.8000,37.3000) -- (88.9000,35.6000) -- (86.7000,34.2000) -- (84.3000,33.1000) -- (81.7000,32.4000) -- (79.1000,32.0000) -- (76.4000,32.1000) -- (73.8000,32.5000) -- (71.3000,33.4000) -- (68.9000,34.6000) -- (66.8000,36.1000) -- (65.0000,37.9000) -- (63.6000,40.1000) -- (62.6000,42.4000) -- (62.0000,44.8000) -- (61.8000,47.3000) -- (61.8000,49.8000) -- (61.7000,52.3000) -- (61.7000,54.8000) -- (61.7000,57.3000) -- (61.7000,59.8000) -- (61.7000,62.4000) -- (61.7000,64.9000) -- (61.7000,67.4000) -- (61.7000,69.9000) -- (61.7000,72.4000) -- (61.7000,75.0000) -- (61.7000,77.5000) -- (62.0000,80.0000) -- (62.8000,82.5000) -- (64.0000,84.8000) -- (65.6000,86.9000) -- (67.5000,88.7000) -- (69.7000,90.2000) -- (72.1000,91.4000) -- (74.6000,92.3000) -- (77.1000,92.8000) -- (79.8000,93.0000) -- (82.4000,92.8000) -- (84.9000,92.2000) -- (87.4000,91.3000) -- (89.7000,90.0000) -- (91.8000,88.5000) -- (93.6000,86.6000) -- (95.1000,84.4000) -- (95.9000,82.1000) -- (96.3000,79.6000) -- (96.4000,77.0000) -- (96.5000,74.4000) -- (96.5000,71.9000) -- (96.5000,69.3000) -- (96.5000,66.8000) -- (96.5000,64.3000) -- (96.5000,61.8000) -- (96.5000,59.3000) -- (96.5000,56.7000) -- (96.5000,54.2000) -- (96.5000,51.7000) -- (97.0000,49.2000) -- (97.2000,46.7000) -- (96.9000,44.3000) -- (96.3000,41.9000) -- (95.1000,39.6000) -- (93.6000,37.6000) -- (91.8000,35.8000) -- (89.7000,34.3000) -- (87.3000,33.1000) -- (84.8000,32.3000) -- (82.1000,31.9000) -- (79.5000,31.8000) -- (76.8000,32.2000) -- (74.3000,32.9000) -- (71.9000,34.1000) -- (69.7000,35.5000) -- (67.9000,37.3000) -- (66.4000,39.4000) -- (65.2000,41.7000) -- (64.6000,44.1000) -- (64.3000,46.6000) -- (64.3000,49.1000) -- (64.2000,51.6000) -- (64.2000,54.1000) -- (64.2000,56.6000) -- (64.2000,59.1000) -- (64.2000,61.6000) -- (64.2000,64.1000) -- (64.2000,66.6000) -- (64.2000,69.2000) -- (64.2000,71.7000) -- (64.2000,74.2000) -- (64.2000,76.7000) -- (64.4000,79.3000) -- (65.1000,81.8000) -- (66.2000,84.1000) -- (67.8000,86.2000) -- (69.6000,88.1000) -- (71.7000,89.7000) -- (74.0000,91.0000) -- (76.5000,91.9000) -- (79.1000,92.5000) -- (81.7000,92.7000) -- (84.3000,92.6000) -- (86.9000,92.1000) -- (89.4000,91.3000) -- (91.7000,90.1000) -- (93.9000,88.6000) -- (95.7000,86.7000) -- (97.3000,84.6000) -- (98.2000,82.3000) -- (98.7000,79.8000) -- (98.9000,77.3000) -- (98.9000,74.7000) -- (98.9000,72.1000) -- (99.0000,69.6000) -- (99.0000,67.1000) -- (99.0000,64.6000) -- (99.0000,62.0000) -- (99.0000,59.5000) -- (99.0000,57.0000) -- (99.0000,54.5000) -- (99.0000,52.0000) -- (99.3000,49.4000) -- (99.6000,47.0000) -- (99.5000,44.5000) -- (98.9000,42.1000) -- (97.8000,39.8000) -- (96.4000,37.7000) -- (94.6000,35.9000) -- (92.5000,34.3000) -- (90.2000,33.1000) -- (87.7000,32.2000) -- (85.1000,31.7000) -- (82.4000,31.6000) -- (79.7000,31.9000) -- (77.2000,32.6000) -- (74.8000,33.6000) -- (72.6000,35.1000) -- (70.6000,36.8000) -- (69.1000,38.8000) -- (67.9000,41.1000) -- (67.2000,43.5000) -- (66.9000,46.0000) -- (66.8000,48.5000) -- (66.8000,51.0000) -- (66.7000,53.5000) -- (66.7000,56.0000) -- (66.7000,58.5000) -- (66.7000,61.0000) -- (66.7000,63.5000) -- (66.7000,66.0000) -- (66.7000,68.6000) -- (66.7000,71.1000) -- (66.7000,73.6000) -- (66.7000,76.1000) -- (66.9000,78.7000) -- (67.5000,81.2000) -- (68.6000,83.5000) -- (70.0000,85.7000) -- (71.8000,87.6000) -- (73.9000,89.2000) -- (76.2000,90.6000) -- (78.7000,91.6000) -- (81.2000,92.2000) -- (83.8000,92.5000) -- (86.5000,92.4000) -- (89.0000,92.0000) -- (91.5000,91.2000) -- (93.9000,90.0000) -- (96.1000,88.5000) -- (98.0000,86.7000) -- (99.5000,84.7000) -- (100.6000,82.4000) -- (101.1000,79.9000) -- (101.3000,77.3000) -- (101.4000,74.8000) -- (101.4000,72.2000) -- (101.4000,69.7000) -- (101.4000,67.2000) -- (101.4000,64.6000) -- (101.4000,62.1000) -- (101.4000,59.6000) -- (101.4000,57.1000) -- (101.4000,54.6000) -- (101.4000,52.0000) -- (101.7000,49.5000) -- (102.1000,47.0000) -- (102.0000,44.6000) -- (101.4000,42.2000) -- (100.4000,39.9000) -- (99.0000,37.7000) -- (97.3000,35.9000) -- (95.2000,34.2000) -- (92.9000,33.0000) -- (90.4000,32.1000) -- (87.8000,31.5000) -- (85.2000,31.4000) -- (82.5000,31.6000) -- (79.9000,32.3000) -- (77.5000,33.3000) -- (75.3000,34.6000) -- (73.3000,36.3000) -- (71.7000,38.3000) -- (70.5000,40.6000) -- (69.7000,43.0000) -- (69.3000,45.5000) -- (69.2000,47.9000) -- (69.2000,50.4000) -- (69.2000,52.9000) -- (69.2000,55.4000) -- (69.2000,57.9000) -- (69.2000,60.5000) -- (69.2000,63.0000) -- (69.2000,65.5000) -- (69.2000,68.0000) -- (69.2000,70.5000) -- (69.2000,73.1000) -- (69.2000,75.6000) -- (69.3000,78.1000) -- (69.8000,80.6000) -- (70.8000,83.0000) -- (72.3000,85.2000) -- (74.0000,87.2000) -- (76.1000,88.8000) -- (78.3000,90.2000) -- (80.8000,91.2000) -- (83.3000,91.9000) -- (85.9000,92.3000) -- (88.6000,92.2000) -- (91.2000,91.8000) -- (93.7000,91.1000) -- (96.0000,90.0000) -- (98.2000,88.5000) -- (100.2000,86.8000) -- (101.8000,84.7000) -- (102.9000,82.5000) -- (103.5000,80.0000) -- (103.7000,77.4000) -- (103.8000,74.9000) -- (103.8000,72.3000) -- (103.8000,69.8000) -- (103.8000,67.3000) -- (103.8000,64.7000) -- (103.9000,62.2000) -- (103.8000,59.7000) -- (103.8000,57.2000) -- (103.8000,54.7000) -- (103.8000,52.1000) -- (104.1000,49.6000) -- (104.5000,47.1000) -- (104.4000,44.7000) -- (103.9000,42.3000) -- (103.0000,39.9000) -- (101.7000,37.8000) -- (100.0000,35.9000) -- (98.0000,34.2000) -- (95.7000,32.9000) -- (93.2000,31.9000) -- (90.6000,31.4000) -- (88.0000,31.2000) -- (85.3000,31.3000) -- (82.7000,31.9000) -- (80.2000,32.9000) -- (78.0000,34.2000) -- (76.0000,35.9000) -- (74.3000,37.8000) -- (73.0000,40.0000) -- (72.2000,42.4000) -- (71.8000,44.9000) -- (71.7000,47.4000) -- (71.6000,49.9000) -- (71.6000,52.4000) -- (71.6000,54.9000) -- (71.6000,57.4000) -- (71.6000,59.9000) -- (71.6000,62.4000) -- (71.6000,65.0000) -- (71.6000,67.5000) -- (71.6000,70.0000) -- (71.6000,72.5000) -- (71.6000,75.0000) -- (71.7000,77.6000) -- (72.2000,80.1000) -- (73.1000,82.5000) -- (74.5000,84.7000) -- (76.2000,86.7000) -- (78.2000,88.4000) -- (80.5000,89.8000) -- (82.9000,90.9000) -- (85.4000,91.6000) -- (88.0000,92.0000) -- (90.7000,92.0000) -- (93.3000,91.7000) -- (95.8000,91.0000) -- (98.2000,89.9000) -- (100.4000,88.5000) -- (102.4000,86.8000) -- (104.1000,84.8000) -- (105.3000,82.6000) -- (105.9000,80.1000) -- (106.2000,77.6000) -- (106.2000,75.0000) -- (106.3000,72.4000) -- (106.3000,69.9000) -- (106.3000,67.4000) -- (106.3000,64.9000) -- (106.3000,62.3000) -- (106.3000,59.8000) -- (106.3000,57.3000) -- (106.3000,54.8000) -- (106.3000,52.3000) -- (106.5000,49.7000) -- (106.9000,47.2000) -- (106.9000,44.8000) -- (106.5000,42.4000) -- (105.6000,40.0000) -- (104.3000,37.8000) -- (102.7000,35.9000) -- (100.7000,34.2000) -- (98.4000,32.8000) -- (96.0000,31.8000) -- (93.4000,31.2000) -- (90.8000,30.9000) -- (88.1000,31.1000) -- (85.5000,31.6000) -- (83.0000,32.5000) -- (80.7000,33.8000) -- (78.7000,35.4000) -- (77.0000,37.4000) -- (75.6000,39.5000) -- (74.8000,41.9000) -- (74.3000,44.4000) -- (74.2000,46.9000) -- (74.1000,49.3000) -- (74.1000,51.8000) -- (74.1000,54.3000) -- (74.1000,56.9000) -- (74.1000,59.4000) -- (74.1000,61.9000) -- (74.1000,64.4000) -- (74.1000,66.9000) -- (74.1000,69.4000) -- (74.1000,72.0000) -- (74.1000,74.5000) -- (74.1000,77.0000) -- (74.5000,79.6000) -- (75.4000,82.0000) -- (76.8000,84.2000) -- (78.5000,86.2000) -- (80.4000,88.0000) -- (82.6000,89.4000) -- (85.0000,90.6000) -- (87.6000,91.3000) -- (90.2000,91.8000) -- (92.8000,91.8000) -- (95.4000,91.5000) -- (97.9000,90.9000) -- (100.3000,89.8000) -- (102.6000,88.5000) -- (104.6000,86.8000) -- (106.3000,84.8000) -- (107.6000,82.6000) -- (108.3000,80.2000) -- (108.6000,77.7000) -- (108.7000,75.1000) -- (108.7000,72.5000) -- (108.7000,70.0000) -- (108.7000,67.5000) -- (108.7000,64.9000) -- (108.7000,62.4000) -- (108.7000,59.9000) -- (108.7000,57.4000) -- (108.7000,54.9000) -- (108.7000,52.3000) -- (108.9000,49.8000) -- (109.3000,47.3000) -- (109.4000,44.9000) -- (109.0000,42.4000) -- (108.2000,40.1000) -- (106.9000,37.9000) -- (105.3000,35.9000) -- (103.4000,34.2000) -- (101.2000,32.8000) -- (98.8000,31.7000) -- (96.2000,31.0000) -- (93.5000,30.7000) -- (90.9000,30.8000) -- (88.2000,31.3000) -- (85.7000,32.2000) -- (83.4000,33.4000) -- (81.4000,35.0000) -- (79.6000,36.9000) -- (78.2000,39.0000) -- (77.3000,41.4000) -- (76.8000,43.8000) -- (76.6000,46.3000) -- (76.5000,48.8000) -- (76.5000,51.3000) -- (76.5000,53.8000) -- (76.5000,56.3000) -- (76.5000,58.8000) -- (76.5000,61.4000) -- (76.5000,63.9000) -- (76.5000,66.4000) -- (76.5000,68.9000) -- (76.5000,71.4000) -- (76.5000,73.9000) -- (76.5000,76.5000) -- (76.9000,79.0000) -- (77.7000,81.5000) -- (79.0000,83.7000) -- (80.7000,85.8000) -- (82.6000,87.6000) -- (84.8000,89.0000) -- (87.2000,90.2000) -- (89.7000,91.0000) -- (92.3000,91.5000) -- (94.9000,91.6000) -- (97.5000,91.4000) -- (100.1000,90.7000) -- (102.5000,89.8000) -- (104.8000,88.4000) -- (106.8000,86.8000) -- (108.6000,84.9000) -- (109.9000,82.7000) -- (110.7000,80.3000) -- (111.0000,77.8000) -- (111.1000,75.2000) -- (111.2000,72.6000) -- (111.2000,70.1000) -- (111.2000,67.6000) -- (111.2000,65.0000) -- (111.2000,62.5000) -- (111.2000,60.0000) -- (111.2000,57.5000) -- (111.2000,55.0000) -- (111.2000,52.4000) -- (111.3000,49.9000) -- (111.7000,47.4000) -- (111.9000,45.0000) -- (111.5000,42.5000) -- (110.8000,40.2000) -- (109.6000,37.9000) -- (108.0000,35.9000) -- (106.1000,34.1000) -- (103.9000,32.7000) -- (101.5000,31.6000) -- (99.0000,30.9000) -- (96.3000,30.5000) -- (93.6000,30.6000) -- (91.0000,31.0000) -- (88.5000,31.8000) -- (86.2000,33.0000) -- (84.1000,34.6000) -- (82.3000,36.4000) -- (80.8000,38.5000) -- (79.8000,40.8000) -- (79.3000,43.3000) -- (79.1000,45.8000) -- (79.0000,48.3000) -- (79.0000,50.8000) -- (79.0000,53.3000) -- (79.0000,55.8000) -- (79.0000,58.3000) -- (79.0000,60.8000) -- (79.0000,63.3000) -- (79.0000,65.9000) -- (79.0000,68.4000) -- (79.0000,70.9000) -- (79.0000,73.4000) -- (79.0000,75.9000) -- (79.3000,78.5000) -- (80.1000,81.0000) -- (81.3000,83.2000) -- (82.9000,85.3000) -- (84.8000,87.1000) -- (87.0000,88.6000) -- (89.3000,89.9000) -- (91.8000,90.7000) -- (94.4000,91.3000) -- (97.0000,91.4000) -- (99.6000,91.2000) -- (102.2000,90.6000) -- (104.6000,89.7000) -- (106.9000,88.4000) -- (109.0000,86.8000) -- (110.8000,84.9000) -- (112.2000,82.8000) -- (113.1000,80.4000) -- (113.4000,77.9000) -- (113.6000,75.3000) -- (113.6000,72.7000) -- (113.6000,70.2000) -- (113.6000,67.6000) -- (113.6000,65.1000) -- (113.6000,62.6000) -- (113.6000,60.1000) -- (113.6000,57.6000) -- (113.6000,55.1000) -- (113.6000,52.5000) -- (113.7000,50.0000) -- (114.1000,47.5000) -- (114.3000,45.0000) -- (114.0000,42.6000) -- (113.3000,40.2000) -- (112.2000,38.0000) -- (110.7000,35.9000) -- (108.8000,34.1000) -- (106.6000,32.6000) -- (104.3000,31.5000) -- (101.7000,30.7000) -- (99.1000,30.3000) -- (96.4000,30.3000) -- (93.8000,30.7000) -- (91.2000,31.5000) -- (88.9000,32.6000) -- (86.7000,34.1000) -- (84.9000,36.0000) -- (83.4000,38.0000) -- (82.4000,40.3000) -- (81.8000,42.8000) -- (81.6000,45.3000) -- (81.5000,47.8000) -- (81.4000,50.3000) -- (81.4000,52.8000) -- (81.4000,55.3000) -- (81.4000,57.8000) -- (81.4000,60.3000) -- (81.4000,62.8000) -- (81.4000,65.4000) -- (81.4000,67.9000) -- (81.4000,70.4000) -- (81.4000,72.9000) -- (81.4000,75.4000) -- (81.7000,78.0000) -- (82.4000,80.5000) -- (83.6000,82.8000) -- (85.2000,84.9000) -- (87.0000,86.7000) -- (89.2000,88.3000) -- (91.5000,89.5000) -- (94.0000,90.4000) -- (96.6000,91.0000) -- (99.2000,91.2000) -- (101.8000,91.0000) -- (104.4000,90.5000) -- (106.8000,89.6000) -- (109.2000,88.3000) -- (111.3000,86.8000) -- (113.1000,84.9000) -- (114.6000,82.8000) -- (115.5000,80.4000) -- (115.9000,77.9000) -- (116.0000,75.3000) -- (116.1000,72.8000) -- (116.1000,70.2000) -- (116.1000,67.7000) -- (116.1000,65.2000) -- (116.1000,62.7000) -- (116.1000,60.1000) -- (116.1000,57.6000) -- (116.1000,55.1000) -- (116.1000,52.6000) -- (116.1000,50.1000) -- (116.5000,47.5000) -- (116.8000,45.1000) -- (116.5000,42.6000) -- (115.9000,40.2000) -- (114.8000,38.0000) -- (113.3000,35.9000) -- (111.5000,34.1000) -- (109.3000,32.6000) -- (107.0000,31.4000) -- (104.5000,30.6000) -- (101.8000,30.1000) -- (99.2000,30.1000) -- (96.5000,30.4000) -- (94.0000,31.2000) -- (91.6000,32.3000) -- (89.4000,33.7000) -- (87.5000,35.5000) -- (86.0000,37.6000) -- (84.9000,39.9000) -- (84.3000,42.3000) -- (84.0000,44.8000) -- (83.9000,47.3000) -- (83.9000,49.8000) -- (83.9000,52.3000) -- (83.9000,54.8000) -- (83.9000,57.3000) -- (83.9000,59.8000) -- (83.9000,62.3000) -- (83.9000,64.8000) -- (83.9000,67.4000) -- (83.9000,69.9000) -- (83.9000,72.4000) -- (83.9000,74.9000) -- (84.0000,77.5000) -- (84.7000,80.0000) -- (85.9000,82.3000) -- (87.4000,84.4000) -- (89.2000,86.3000) -- (91.3000,87.9000) -- (93.7000,89.2000) -- (96.1000,90.1000) -- (98.7000,90.7000) -- (101.3000,91.0000) -- (103.9000,90.8000) -- (106.5000,90.3000) -- (109.0000,89.5000) -- (111.3000,88.3000) -- (113.5000,86.8000) -- (115.3000,84.9000) -- (116.9000,82.8000) -- (117.8000,80.5000) -- (118.3000,78.0000) -- (118.4000,75.4000) -- (118.5000,72.9000) -- (118.5000,70.3000) -- (118.5000,67.8000) -- (118.5000,65.3000) -- (118.5000,62.7000) -- (118.5000,60.2000) -- (118.5000,57.7000) -- (118.5000,55.2000) -- (118.5000,52.7000) -- (118.5000,50.2000) -- (118.9000,47.6000) -- (119.2000,45.1000) -- (119.0000,42.7000) -- (118.4000,40.3000) -- (117.4000,38.0000) -- (115.9000,35.9000) -- (114.1000,34.1000) -- (112.1000,32.5000) -- (109.7000,31.3000) -- (107.2000,30.4000) -- (104.6000,29.9000) -- (101.9000,29.9000) -- (99.3000,30.2000) -- (96.7000,30.8000) -- (94.3000,31.9000) -- (92.1000,33.3000) -- (90.2000,35.1000) -- (88.6000,37.1000) -- (87.5000,39.4000) -- (86.8000,41.8000) -- (86.5000,44.3000) -- (86.4000,46.8000) -- (86.3000,49.3000) -- (86.3000,51.8000) -- (86.3000,54.3000) -- (86.3000,56.8000) -- (86.3000,59.3000) -- (86.3000,61.8000) -- (86.3000,64.3000) -- (86.3000,66.8000) -- (86.3000,69.4000) -- (86.3000,71.9000) -- (86.3000,74.4000) -- (86.5000,76.9000) -- (87.1000,79.5000) -- (88.2000,81.8000) -- (89.7000,84.0000) -- (91.5000,85.9000) -- (93.6000,87.5000) -- (95.8000,88.8000) -- (98.3000,89.8000) -- (100.9000,90.5000) -- (103.5000,90.7000) -- (106.1000,90.6000) -- (108.7000,90.2000) -- (111.2000,89.4000) -- (113.5000,88.2000) -- (115.7000,86.7000) -- (117.6000,84.9000) -- (119.2000,82.9000) -- (120.2000,80.5000) -- (120.7000,78.1000) -- (120.9000,75.5000) -- (120.9000,72.9000) -- (121.0000,70.4000) -- (121.0000,67.8000) -- (121.0000,65.3000) -- (121.0000,62.8000) -- (121.0000,60.3000) -- (121.0000,57.8000) -- (121.0000,55.2000) -- (121.0000,52.7000) -- (121.0000,50.2000) -- (121.3000,47.7000) -- (121.6000,45.2000) -- (121.5000,42.8000) -- (121.0000,40.4000) -- (120.0000,38.0000) -- (118.6000,35.9000) -- (116.8000,34.0000) -- (114.8000,32.4000) -- (112.4000,31.2000) -- (110.0000,30.3000) -- (107.4000,29.7000) -- (104.7000,29.6000) -- (102.0000,29.9000) -- (99.4000,30.5000) -- (97.0000,31.5000) -- (94.8000,32.9000) -- (92.8000,34.6000) -- (91.2000,36.6000) -- (90.0000,38.8000) -- (89.2000,41.2000) -- (88.9000,43.7000) -- (88.8000,46.2000) -- (88.7000,48.7000) -- (88.7000,51.2000) -- (88.7000,53.7000) -- (88.7000,56.2000) -- (88.7000,58.7000) -- (88.7000,61.2000) -- (88.7000,63.8000) -- (88.7000,66.3000) -- (88.7000,68.8000) -- (88.7000,71.3000) -- (88.7000,73.8000) -- (88.8000,76.4000) -- (89.4000,78.9000) -- (90.4000,81.3000) -- (91.8000,83.5000) -- (93.6000,85.4000) -- (95.6000,87.1000) -- (97.9000,88.4000) -- (100.3000,89.5000) -- (102.9000,90.2000) -- (105.5000,90.5000) -- (108.1000,90.5000) -- (110.7000,90.1000) -- (113.2000,89.3000) -- (115.6000,88.2000) -- (117.8000,86.8000) -- (119.8000,85.0000) -- (121.4000,83.0000) -- (122.5000,80.7000) -- (123.1000,78.3000) -- (123.3000,75.7000) -- (123.4000,73.1000) -- (123.4000,70.6000) -- (123.4000,68.0000) -- (123.4000,65.5000) -- (123.4000,63.0000) -- (123.4000,60.5000) -- (123.4000,58.0000) -- (123.4000,55.4000) -- (123.4000,52.9000) -- (123.4000,50.4000) -- (123.7000,47.9000) -- (124.1000,45.4000) -- (124.0000,42.9000) -- (123.5000,40.5000) -- (122.6000,38.2000) -- (121.3000,36.0000) -- (119.6000,34.1000) -- (117.6000,32.5000) -- (115.3000,31.1000) -- (112.8000,30.2000) -- (110.2000,29.6000) -- (107.6000,29.4000) -- (104.9000,29.6000) -- (102.3000,30.1000) -- (99.9000,31.1000) -- (97.6000,32.4000) -- (95.6000,34.1000) -- (93.9000,36.0000) -- (92.6000,38.2000) -- (91.8000,40.6000) -- (91.4000,43.1000) -- (91.2000,45.6000) -- (91.2000,48.0000) -- (91.2000,50.5000) -- (91.2000,53.0000) -- (91.2000,55.6000) -- (91.2000,58.1000) -- (91.2000,60.6000) -- (91.2000,63.1000) -- (91.2000,65.6000) -- (91.2000,68.1000) -- (91.2000,70.7000) -- (91.2000,73.2000) -- (91.2000,75.7000) -- (91.7000,78.3000) -- (92.6000,80.7000) -- (94.0000,82.9000) -- (95.7000,84.9000) -- (97.7000,86.6000) -- (99.9000,88.0000) -- (102.4000,89.1000) -- (104.9000,89.9000) -- (107.5000,90.3000) -- (110.1000,90.3000) -- (112.7000,90.0000) -- (115.3000,89.3000) -- (117.7000,88.2000) -- (119.9000,86.8000) -- (121.9000,85.1000) -- (123.6000,83.1000) -- (124.8000,80.9000) -- (125.5000,78.5000) -- (125.7000,75.9000) -- (125.8000,73.4000) -- (125.9000,70.8000) -- (125.9000,68.2000) -- (125.9000,65.7000) -- (125.9000,63.2000) -- (125.9000,60.7000) -- (125.9000,58.2000) -- (125.9000,55.7000) -- (125.9000,53.1000) -- (125.9000,50.6000) -- (126.0000,48.1000) -- (126.5000,45.5000);



  \end{scope}
  \begin{scope}[cm={{1.15801,0.0,0.0,1.15801,(-615.96866,-167.12877)}},draw=blue,line cap=round,line join=round,line width=0.480pt]
    \path[draw] (44.5000,13.5000) -- (44.5000,102.5000) -- (142.5000,102.5000) -- (142.5000,13.5000) -- (44.5000,13.5000);



  \end{scope}
  \begin{scope}[cm={{1.15801,0.0,0.0,1.15801,(-736.40737,-53.12875)}},draw=ca0a0a4,dash pattern=on 0.40pt off 0.80pt,line cap=round,line join=round,line width=0.400pt]
    \path[draw] (148.5000,102.5000) -- (246.5000,102.5000);



  \end{scope}
  \begin{scope}[cm={{1.15801,0.0,0.0,1.15801,(-736.40737,-53.12875)}},draw=blue,line cap=round,line join=round,line width=0.480pt]
    \path[draw] (148.5000,102.5000) -- (151.5000,102.5000);



    \path[draw] (246.5000,102.5000) -- (243.5000,102.5000);



  \end{scope}
  \begin{scope}[cm={{1.15801,0.0,0.0,1.15801,(-736.40737,-53.12875)}},draw=ca0a0a4,dash pattern=on 1.55pt off 1.55pt,line cap=round,line join=round,line width=0.259pt,miter limit=4.00]
    \path[draw,dash pattern=on 1.55pt off 1.55pt,line width=0.259pt,miter limit=4.00] (148.5000,79.5000) -- (246.5000,79.5000);



  \end{scope}
  \begin{scope}[cm={{1.15801,0.0,0.0,1.15801,(-736.40737,-53.12875)}},draw=blue,line cap=round,line join=round,line width=0.480pt]
    \path[draw] (148.5000,79.5000) -- (151.5000,79.5000);



    \path[draw] (246.5000,79.5000) -- (243.5000,79.5000);



  \end{scope}
  \begin{scope}[cm={{1.15801,0.0,0.0,1.15801,(-736.40737,-53.12875)}},draw=ca0a0a4,dash pattern=on 1.55pt off 1.55pt,line cap=round,line join=round,line width=0.259pt,miter limit=4.00]
    \path[draw,dash pattern=on 1.55pt off 1.55pt,line width=0.259pt,miter limit=4.00] (148.5000,57.5000) -- (246.5000,57.5000);



  \end{scope}
  \begin{scope}[cm={{1.15801,0.0,0.0,1.15801,(-736.40737,-53.12875)}},draw=blue,line cap=round,line join=round,line width=0.480pt]
    \path[draw] (148.5000,57.5000) -- (151.5000,57.5000);



    \path[draw] (246.5000,57.5000) -- (243.5000,57.5000);



  \end{scope}
  \begin{scope}[cm={{1.15801,0.0,0.0,1.15801,(-736.40737,-53.12875)}},draw=ca0a0a4,dash pattern=on 1.55pt off 1.55pt,line cap=round,line join=round,line width=0.259pt,miter limit=4.00]
    \path[draw,dash pattern=on 1.55pt off 1.55pt,line width=0.259pt,miter limit=4.00] (148.5000,35.5000) -- (246.5000,35.5000);



  \end{scope}
  \begin{scope}[cm={{1.15801,0.0,0.0,1.15801,(-736.40737,-53.12875)}},draw=blue,line cap=round,line join=round,line width=0.480pt]
    \path[draw] (148.5000,35.5000) -- (151.5000,35.5000);



    \path[draw] (246.5000,35.5000) -- (243.5000,35.5000);



  \end{scope}
  \begin{scope}[cm={{1.15801,0.0,0.0,1.15801,(-736.40737,-53.12875)}},draw=ca0a0a4,dash pattern=on 0.40pt off 0.80pt,line cap=round,line join=round,line width=0.400pt]
    \path[draw] (148.5000,13.5000) -- (246.5000,13.5000);



  \end{scope}
  \begin{scope}[cm={{1.15801,0.0,0.0,1.15801,(-673.8747,-53.12875)}},draw=ca0a0a4,dash pattern=on 1.55pt off 1.55pt,line cap=round,line join=round,line width=0.259pt,miter limit=4.00]
    \path[draw,dash pattern=on 1.55pt off 1.55pt,line width=0.259pt,miter limit=4.00] (118.5000,102.5000) -- (118.5000,13.5000);



  \end{scope}
  \begin{scope}[cm={{1.15801,0.0,0.0,1.15801,(-736.40737,-53.12875)}},draw=blue,line cap=round,line join=round,line width=0.480pt]
    \path[draw] (148.5000,13.5000) -- (151.5000,13.5000);



    \path[draw] (246.5000,13.5000) -- (243.5000,13.5000);



  \end{scope}
  \begin{scope}[cm={{1.15801,0.0,0.0,1.15801,(-736.40737,-53.12875)}},draw=ca0a0a4,dash pattern=on 0.40pt off 0.80pt,line cap=round,line join=round,line width=0.400pt]
    \path[draw] (148.5000,102.5000) -- (148.5000,13.5000);



  \end{scope}
  \begin{scope}[cm={{1.15801,0.0,0.0,1.15801,(-736.40737,-53.12875)}},draw=blue,line cap=round,line join=round,line width=0.480pt]
    \path[draw] (148.5000,102.5000) -- (148.5000,99.5000);



    \path[draw] (148.5000,13.5000) -- (148.5000,16.5000);



  \end{scope}
  \begin{scope}[cm={{1.15801,0.0,0.0,1.15801,(-736.40737,-53.12875)}},draw=blue,line cap=round,line join=round,line width=0.480pt]
    \path[draw] (172.5000,102.5000) -- (172.5000,99.5000);



    \path[draw] (172.5000,13.5000) -- (172.5000,16.5000);



  \end{scope}
  \begin{scope}[cm={{1.15801,0.0,0.0,1.15801,(-644.92439,-53.12875)}},draw=ca0a0a4,dash pattern=on 1.55pt off 1.55pt,line cap=round,line join=round,line width=0.259pt,miter limit=4.00]
    \path[draw,dash pattern=on 1.55pt off 1.55pt,line width=0.259pt,miter limit=4.00] (118.5000,102.5000) -- (118.5000,13.5000);



  \end{scope}
  \begin{scope}[cm={{1.15801,0.0,0.0,1.15801,(-736.40737,-53.12875)}},draw=blue,line cap=round,line join=round,line width=0.480pt]
    \path[draw] (197.5000,102.5000) -- (197.5000,99.5000);



    \path[draw] (197.5000,13.5000) -- (197.5000,16.5000);



  \end{scope}
  \begin{scope}[cm={{1.15801,0.0,0.0,1.15801,(-617.13209,-53.12875)}},draw=ca0a0a4,dash pattern=on 1.55pt off 1.55pt,line cap=round,line join=round,line width=0.259pt,miter limit=4.00]
    \path[draw,dash pattern=on 1.55pt off 1.55pt,line width=0.259pt,miter limit=4.00] (118.5000,102.5000) -- (118.5000,13.5000);



  \end{scope}
  \begin{scope}[cm={{1.15801,0.0,0.0,1.15801,(-736.40737,-53.12875)}},draw=blue,line cap=round,line join=round,line width=0.480pt]
    \path[draw] (221.5000,102.5000) -- (221.5000,99.5000);



    \path[draw] (221.5000,13.5000) -- (221.5000,16.5000);



  \end{scope}
  \begin{scope}[cm={{1.15801,0.0,0.0,1.15801,(-736.40737,-53.12875)}},draw=ca0a0a4,dash pattern=on 0.40pt off 0.80pt,line cap=round,line join=round,line width=0.400pt]
    \path[draw] (246.5000,102.5000) -- (246.5000,27.5000);



    \path[draw] (246.5000,19.5000) -- (246.5000,13.5000);



  \end{scope}
  \begin{scope}[cm={{1.15801,0.0,0.0,1.15801,(-736.40737,-53.12875)}},draw=blue,line cap=round,line join=round,line width=0.480pt]
    \path[draw] (246.5000,102.5000) -- (246.5000,99.5000);



    \path[draw] (246.5000,13.5000) -- (246.5000,16.5000);



  \end{scope}
  \begin{scope}[cm={{1.15801,0.0,0.0,1.15801,(-736.40737,-53.12875)}},draw=blue,line cap=round,line join=round,line width=0.480pt]
    \path[draw] (148.5000,13.5000) -- (148.5000,102.5000) -- (246.5000,102.5000) -- (246.5000,13.5000) -- (148.5000,13.5000);



  \end{scope}
  \begin{scope}[cm={{0.84029,0.0,0.0,0.84029,(-689.6406,-75.21977)}},fill=cd9d9d9]
    \path[rounded corners=0.0000cm] (222.0000,18.0000) rectangle (238.0000,34.0000);



  \end{scope}
  \begin{scope}[cm={{1.15801,0.0,0.0,1.15801,(-736.40737,-53.12875)}},draw=blue,line cap=round,line join=round,line width=0.480pt]
    \path[draw] (160.6000,31.9000) -- (160.6000,31.9000) -- (160.7000,33.1000) -- (160.7000,34.2000) -- (160.8000,35.2000) -- (160.9000,36.2000) -- (160.9000,37.2000) -- (160.6000,38.2000) -- (159.9000,39.1000) -- (159.3000,40.0000) -- (158.7000,40.9000) -- (158.2000,41.8000) -- (157.7000,42.8000) -- (157.4000,43.8000) -- (157.2000,44.8000) -- (157.1000,45.8000) -- (157.0000,46.9000) -- (157.0000,48.0000) -- (157.0000,49.0000) -- (157.0000,50.1000) -- (157.0000,51.1000) -- (157.0000,52.2000) -- (157.0000,53.3000) -- (157.0000,54.3000) -- (157.0000,55.4000) -- (157.0000,56.4000) -- (157.0000,57.5000) -- (157.0000,58.5000) -- (157.0000,59.6000) -- (157.0000,60.6000) -- (157.0000,61.7000) -- (157.0000,62.8000) -- (157.0000,63.8000) -- (157.0000,64.9000) -- (157.0000,65.9000) -- (157.0000,67.0000) -- (157.0000,68.0000) -- (157.0000,69.1000) -- (157.0000,70.1000) -- (157.0000,71.2000) -- (157.0000,72.2000) -- (157.0000,73.3000) -- (157.0000,74.4000) -- (157.0000,75.4000) -- (157.0000,76.5000) -- (157.0000,77.5000) -- (157.0000,78.6000) -- (157.1000,79.6000) -- (157.3000,80.7000) -- (157.6000,81.7000) -- (157.9000,82.7000) -- (158.3000,83.7000) -- (158.8000,84.7000) -- (159.4000,85.6000) -- (160.0000,86.5000) -- (160.7000,87.4000) -- (161.4000,88.3000) -- (162.2000,89.0000) -- (163.1000,89.8000) -- (164.0000,90.5000) -- (165.0000,91.1000) -- (166.0000,91.7000) -- (167.1000,92.2000) -- (168.2000,92.7000) -- (169.3000,93.1000) -- (170.5000,93.4000) -- (171.7000,93.6000) -- (172.9000,93.8000) -- (174.1000,93.9000) -- (175.3000,94.0000) -- (176.6000,93.9000) -- (177.8000,93.8000) -- (179.0000,93.6000) -- (180.2000,93.4000) -- (181.4000,93.0000) -- (182.5000,92.6000) -- (183.7000,92.1000) -- (184.7000,91.6000) -- (185.8000,91.0000) -- (186.7000,90.3000) -- (187.7000,89.6000) -- (188.5000,88.8000) -- (189.4000,88.0000) -- (190.1000,87.2000) -- (190.8000,86.3000) -- (191.4000,85.3000) -- (191.9000,84.4000) -- (192.3000,83.3000) -- (192.6000,82.3000) -- (192.8000,81.3000) -- (193.0000,80.3000) -- (193.1000,79.2000) -- (193.2000,78.2000) -- (193.3000,77.2000) -- (193.3000,76.1000) -- (193.4000,75.1000) -- (193.4000,74.0000) -- (193.4000,73.0000) -- (193.4000,71.9000) -- (193.4000,70.9000) -- (193.4000,69.8000) -- (193.4000,68.8000) -- (193.5000,67.7000) -- (193.5000,66.6000) -- (193.5000,65.6000) -- (193.5000,64.5000) -- (193.5000,63.5000) -- (193.5000,62.4000) -- (193.5000,61.4000) -- (193.5000,60.3000) -- (193.5000,59.3000) -- (193.5000,58.2000) -- (193.5000,57.2000) -- (193.5000,56.1000) -- (193.5000,55.0000) -- (193.5000,54.0000) -- (193.5000,52.9000) -- (193.5000,51.9000) -- (193.5000,50.8000) -- (193.7000,49.8000) -- (193.9000,48.7000) -- (194.0000,47.7000) -- (194.0000,46.6000) -- (193.9000,45.6000) -- (193.8000,44.5000) -- (193.6000,43.5000) -- (193.2000,42.5000) -- (192.8000,41.5000) -- (192.4000,40.5000) -- (191.8000,39.6000) -- (191.2000,38.7000) -- (190.4000,37.8000) -- (189.6000,37.0000) -- (188.8000,36.3000) -- (187.8000,35.6000) -- (186.8000,34.9000) -- (185.8000,34.4000) -- (184.7000,33.9000) -- (183.5000,33.5000) -- (182.3000,33.2000) -- (181.1000,33.0000) -- (179.9000,32.8000) -- (178.7000,32.8000) -- (177.4000,32.8000) -- (176.2000,32.9000) -- (175.0000,33.1000) -- (173.8000,33.4000) -- (172.7000,33.8000) -- (171.6000,34.2000) -- (170.5000,34.8000) -- (169.5000,35.3000) -- (168.5000,36.0000) -- (167.6000,36.7000) -- (166.8000,37.5000) -- (166.0000,38.3000) -- (165.4000,39.2000) -- (164.7000,40.1000) -- (164.2000,41.1000) -- (163.8000,42.1000) -- (163.4000,43.1000) -- (163.2000,44.1000) -- (163.1000,45.2000) -- (163.0000,46.2000) -- (163.0000,47.3000) -- (163.0000,48.4000) -- (163.0000,49.4000) -- (163.0000,50.5000) -- (163.0000,51.5000) -- (163.0000,52.6000) -- (163.0000,53.6000) -- (163.0000,54.7000) -- (162.9000,55.7000) -- (162.9000,56.8000) -- (162.9000,57.9000) -- (162.9000,58.9000) -- (162.9000,60.0000) -- (162.9000,61.0000) -- (162.9000,62.1000) -- (162.9000,63.1000) -- (162.9000,64.2000) -- (162.9000,65.2000) -- (162.9000,66.3000) -- (162.9000,67.4000) -- (162.9000,68.4000) -- (162.9000,69.5000) -- (162.9000,70.5000) -- (162.9000,71.6000) -- (162.9000,72.6000) -- (162.9000,73.7000) -- (162.9000,74.7000) -- (162.9000,75.8000) -- (162.9000,76.9000) -- (163.0000,77.9000) -- (163.1000,79.0000) -- (163.3000,80.0000) -- (163.5000,81.0000) -- (163.9000,82.1000) -- (164.3000,83.1000) -- (164.7000,84.0000) -- (165.3000,85.0000) -- (165.9000,85.9000) -- (166.6000,86.8000) -- (167.3000,87.6000) -- (168.1000,88.4000) -- (168.9000,89.2000) -- (169.9000,89.9000) -- (170.8000,90.5000) -- (171.8000,91.1000) -- (172.9000,91.7000) -- (174.0000,92.1000) -- (175.1000,92.5000) -- (176.3000,92.9000) -- (177.4000,93.2000) -- (178.6000,93.4000) -- (179.9000,93.5000) -- (181.1000,93.5000) -- (182.3000,93.5000) -- (183.6000,93.4000) -- (184.8000,93.3000) -- (186.0000,93.0000) -- (187.2000,92.7000) -- (188.3000,92.3000) -- (189.5000,91.9000) -- (190.6000,91.4000) -- (191.6000,90.8000) -- (192.6000,90.1000) -- (193.6000,89.4000) -- (194.5000,88.7000) -- (195.3000,87.9000) -- (196.1000,87.1000) -- (196.8000,86.2000) -- (197.4000,85.2000) -- (197.9000,84.3000) -- (198.4000,83.3000) -- (198.7000,82.3000) -- (198.9000,81.2000) -- (199.1000,80.2000) -- (199.3000,79.2000) -- (199.4000,78.1000) -- (199.4000,77.1000) -- (199.5000,76.1000) -- (199.5000,75.0000) -- (199.6000,74.0000) -- (199.6000,72.9000) -- (199.6000,71.9000) -- (199.6000,70.8000) -- (199.6000,69.8000) -- (199.6000,68.7000) -- (199.6000,67.7000) -- (199.6000,66.6000) -- (199.6000,65.5000) -- (199.7000,64.5000) -- (199.7000,63.4000) -- (199.7000,62.4000) -- (199.7000,61.3000) -- (199.7000,60.3000) -- (199.7000,59.2000) -- (199.7000,58.2000) -- (199.7000,57.1000) -- (199.7000,56.0000) -- (199.7000,55.0000) -- (199.7000,53.9000) -- (199.7000,52.9000) -- (199.7000,51.8000) -- (199.7000,50.8000) -- (199.8000,49.7000) -- (200.0000,48.7000) -- (200.1000,47.6000) -- (200.2000,46.6000) -- (200.1000,45.5000) -- (200.0000,44.5000) -- (199.8000,43.4000) -- (199.6000,42.4000) -- (199.2000,41.4000) -- (198.8000,40.4000) -- (198.3000,39.5000) -- (197.6000,38.5000) -- (197.0000,37.7000) -- (196.2000,36.8000) -- (195.4000,36.1000) -- (194.5000,35.3000) -- (193.5000,34.7000) -- (192.5000,34.1000) -- (191.4000,33.5000) -- (190.2000,33.1000) -- (189.1000,32.7000) -- (187.9000,32.5000) -- (186.7000,32.3000) -- (185.4000,32.2000) -- (184.2000,32.2000) -- (183.0000,32.2000) -- (181.8000,32.4000) -- (180.6000,32.7000) -- (179.4000,33.0000) -- (178.3000,33.4000) -- (177.2000,33.9000) -- (176.1000,34.4000) -- (175.2000,35.1000) -- (174.2000,35.8000) -- (173.4000,36.5000) -- (172.6000,37.3000) -- (171.8000,38.2000) -- (171.2000,39.0000) -- (170.6000,40.0000) -- (170.1000,41.0000) -- (169.7000,42.0000) -- (169.4000,43.0000) -- (169.2000,44.0000) -- (169.2000,45.1000) -- (169.1000,46.1000) -- (169.1000,47.2000) -- (169.1000,48.3000) -- (169.1000,49.3000) -- (169.1000,50.4000) -- (169.1000,51.4000) -- (169.1000,52.5000) -- (169.1000,53.6000) -- (169.1000,54.6000) -- (169.1000,55.7000) -- (169.1000,56.7000) -- (169.1000,57.8000) -- (169.1000,58.8000) -- (169.1000,59.9000) -- (169.1000,60.9000) -- (169.1000,62.0000) -- (169.1000,63.1000) -- (169.1000,64.1000) -- (169.1000,65.2000) -- (169.1000,66.2000) -- (169.1000,67.3000) -- (169.1000,68.3000) -- (169.1000,69.4000) -- (169.1000,70.4000) -- (169.1000,71.5000) -- (169.1000,72.6000) -- (169.1000,73.6000) -- (169.1000,74.7000) -- (169.1000,75.7000) -- (169.1000,76.8000) -- (169.1000,77.8000) -- (169.3000,78.9000) -- (169.5000,79.9000) -- (169.8000,80.9000) -- (170.1000,82.0000) -- (170.6000,82.9000) -- (171.1000,83.9000) -- (171.7000,84.8000) -- (172.3000,85.7000) -- (173.0000,86.6000) -- (173.8000,87.4000) -- (174.6000,88.2000) -- (175.5000,88.9000) -- (176.4000,89.6000) -- (177.4000,90.2000) -- (178.4000,90.8000) -- (179.5000,91.3000) -- (180.6000,91.8000) -- (181.7000,92.2000) -- (182.9000,92.5000) -- (184.1000,92.7000) -- (185.3000,92.9000) -- (186.5000,93.0000) -- (187.8000,93.0000) -- (189.0000,93.0000) -- (190.2000,92.8000) -- (191.4000,92.6000) -- (192.6000,92.4000) -- (193.8000,92.0000) -- (195.0000,91.6000) -- (196.1000,91.1000) -- (197.2000,90.6000) -- (198.2000,90.0000) -- (199.2000,89.3000) -- (200.1000,88.6000) -- (201.0000,87.8000) -- (201.8000,87.0000) -- (202.5000,86.2000) -- (203.2000,85.3000) -- (203.8000,84.3000) -- (204.3000,83.3000) -- (204.7000,82.3000) -- (205.0000,81.3000) -- (205.2000,80.3000) -- (205.3000,79.2000) -- (205.5000,78.2000) -- (205.6000,77.2000) -- (205.6000,76.1000) -- (205.7000,75.1000) -- (205.7000,74.0000) -- (205.7000,73.0000) -- (205.8000,71.9000) -- (205.8000,70.9000) -- (205.8000,69.8000) -- (205.8000,68.8000) -- (205.8000,67.7000) -- (205.8000,66.7000) -- (205.8000,65.6000) -- (205.8000,64.6000) -- (205.8000,63.5000) -- (205.8000,62.5000) -- (205.8000,61.4000) -- (205.8000,60.3000) -- (205.8000,59.3000) -- (205.8000,58.2000) -- (205.8000,57.2000) -- (205.8000,56.1000) -- (205.8000,55.1000) -- (205.8000,54.0000) -- (205.8000,53.0000) -- (205.8000,51.9000) -- (205.8000,50.8000) -- (205.9000,49.8000) -- (206.0000,48.7000) -- (206.2000,47.7000) -- (206.3000,46.6000) -- (206.3000,45.6000) -- (206.3000,44.5000) -- (206.1000,43.5000) -- (205.9000,42.5000) -- (205.6000,41.4000) -- (205.2000,40.4000) -- (204.7000,39.5000) -- (204.2000,38.5000) -- (203.5000,37.6000) -- (202.8000,36.8000) -- (202.0000,36.0000) -- (201.2000,35.2000) -- (200.2000,34.5000) -- (199.2000,33.9000) -- (198.2000,33.3000) -- (197.1000,32.8000) -- (195.9000,32.4000) -- (194.8000,32.1000) -- (193.5000,31.8000) -- (192.3000,31.7000) -- (191.1000,31.6000) -- (189.9000,31.6000) -- (188.6000,31.7000) -- (187.4000,31.9000) -- (186.3000,32.2000) -- (185.1000,32.6000) -- (184.0000,33.0000) -- (182.9000,33.5000) -- (181.9000,34.1000) -- (180.9000,34.8000) -- (180.0000,35.5000) -- (179.2000,36.2000) -- (178.4000,37.1000) -- (177.7000,37.9000) -- (177.1000,38.9000) -- (176.6000,39.8000) -- (176.1000,40.8000) -- (175.7000,41.8000) -- (175.5000,42.8000) -- (175.3000,43.9000) -- (175.3000,44.9000) -- (175.3000,46.0000) -- (175.2000,47.1000) -- (175.2000,48.1000) -- (175.2000,49.2000) -- (175.2000,50.2000) -- (175.2000,51.3000) -- (175.2000,52.4000) -- (175.2000,53.4000) -- (175.2000,54.5000) -- (175.2000,55.5000) -- (175.2000,56.6000) -- (175.2000,57.6000) -- (175.2000,58.7000) -- (175.2000,59.7000) -- (175.2000,60.8000) -- (175.2000,61.9000) -- (175.2000,62.9000) -- (175.2000,64.0000) -- (175.2000,65.0000) -- (175.2000,66.1000) -- (175.2000,67.1000) -- (175.2000,68.2000) -- (175.2000,69.2000) -- (175.2000,70.3000) -- (175.2000,71.4000) -- (175.2000,72.4000) -- (175.2000,73.5000) -- (175.2000,74.5000) -- (175.2000,75.6000) -- (175.2000,76.6000) -- (175.3000,77.7000) -- (175.5000,78.7000) -- (175.7000,79.8000) -- (176.1000,80.8000) -- (176.5000,81.8000) -- (176.9000,82.8000) -- (177.4000,83.7000) -- (178.0000,84.6000) -- (178.7000,85.5000) -- (179.4000,86.4000) -- (180.2000,87.2000) -- (181.1000,87.9000) -- (182.0000,88.7000) -- (182.9000,89.3000) -- (183.9000,89.9000) -- (185.0000,90.5000) -- (186.0000,91.0000) -- (187.2000,91.4000) -- (188.3000,91.7000) -- (189.5000,92.0000) -- (190.7000,92.2000) -- (191.9000,92.4000) -- (193.1000,92.4000) -- (194.4000,92.4000) -- (195.6000,92.4000) -- (196.8000,92.2000) -- (198.0000,92.0000) -- (199.2000,91.7000) -- (200.4000,91.3000) -- (201.5000,90.9000) -- (202.6000,90.4000) -- (203.7000,89.8000) -- (204.7000,89.2000) -- (205.7000,88.5000) -- (206.6000,87.8000) -- (207.4000,87.0000) -- (208.2000,86.1000) -- (208.9000,85.3000) -- (209.6000,84.3000) -- (210.1000,83.4000) -- (210.6000,82.4000) -- (210.9000,81.4000) -- (211.2000,80.3000) -- (211.4000,79.3000) -- (211.5000,78.3000) -- (211.6000,77.3000) -- (211.7000,76.2000) -- (211.8000,75.2000) -- (211.8000,74.1000) -- (211.9000,73.1000) -- (211.9000,72.0000) -- (211.9000,71.0000) -- (211.9000,69.9000) -- (211.9000,68.9000) -- (211.9000,67.8000) -- (211.9000,66.8000) -- (211.9000,65.7000) -- (211.9000,64.7000) -- (211.9000,63.6000) -- (211.9000,62.5000) -- (211.9000,61.5000) -- (211.9000,60.4000) -- (211.9000,59.4000) -- (211.9000,58.3000) -- (211.9000,57.3000) -- (211.9000,56.2000) -- (211.9000,55.2000) -- (211.9000,54.1000) -- (211.9000,53.0000) -- (211.9000,52.0000) -- (211.9000,50.9000) -- (212.0000,49.9000) -- (212.1000,48.8000) -- (212.2000,47.8000) -- (212.4000,46.7000) -- (212.5000,45.7000) -- (212.4000,44.6000) -- (212.4000,43.6000) -- (212.2000,42.5000) -- (211.9000,41.5000) -- (211.6000,40.5000) -- (211.2000,39.5000) -- (210.7000,38.6000) -- (210.1000,37.6000) -- (209.4000,36.7000) -- (208.7000,35.9000) -- (207.8000,35.1000) -- (206.9000,34.4000) -- (206.0000,33.7000) -- (205.0000,33.1000) -- (203.9000,32.5000) -- (202.8000,32.1000) -- (201.6000,31.7000) -- (200.4000,31.4000) -- (199.2000,31.2000) -- (198.0000,31.1000) -- (196.8000,31.1000) -- (195.5000,31.1000) -- (194.3000,31.3000) -- (193.1000,31.5000) -- (191.9000,31.8000) -- (190.8000,32.2000) -- (189.7000,32.7000) -- (188.7000,33.2000) -- (187.7000,33.8000) -- (186.7000,34.5000) -- (185.8000,35.2000) -- (185.0000,36.0000) -- (184.3000,36.9000) -- (183.6000,37.7000) -- (183.0000,38.7000) -- (182.5000,39.6000) -- (182.1000,40.6000) -- (181.7000,41.6000) -- (181.6000,42.7000) -- (181.5000,43.7000) -- (181.4000,44.8000) -- (181.4000,45.9000) -- (181.4000,46.9000) -- (181.4000,48.0000) -- (181.4000,49.0000) -- (181.4000,50.1000) -- (181.4000,51.2000) -- (181.4000,52.2000) -- (181.4000,53.3000) -- (181.4000,54.3000) -- (181.4000,55.4000) -- (181.4000,56.4000) -- (181.4000,57.5000) -- (181.4000,58.5000) -- (181.4000,59.6000) -- (181.4000,60.7000) -- (181.4000,61.7000) -- (181.4000,62.8000) -- (181.4000,63.8000) -- (181.4000,64.9000) -- (181.4000,65.9000) -- (181.4000,67.0000) -- (181.4000,68.0000) -- (181.4000,69.1000) -- (181.4000,70.2000) -- (181.4000,71.2000) -- (181.4000,72.3000) -- (181.4000,73.3000) -- (181.4000,74.4000) -- (181.4000,75.4000) -- (181.4000,76.5000) -- (181.5000,77.5000) -- (181.7000,78.6000) -- (182.0000,79.6000) -- (182.3000,80.6000) -- (182.8000,81.6000) -- (183.3000,82.6000) -- (183.8000,83.5000) -- (184.4000,84.4000) -- (185.1000,85.3000) -- (185.9000,86.1000) -- (186.7000,86.9000) -- (187.6000,87.7000) -- (188.5000,88.4000) -- (189.4000,89.0000) -- (190.5000,89.6000) -- (191.5000,90.1000) -- (192.6000,90.6000) -- (193.8000,91.0000) -- (194.9000,91.3000) -- (196.1000,91.6000) -- (197.3000,91.7000) -- (198.5000,91.9000) -- (199.8000,91.9000) -- (201.0000,91.9000) -- (202.2000,91.8000) -- (203.5000,91.6000) -- (204.7000,91.3000) -- (205.8000,91.0000) -- (207.0000,90.6000) -- (208.1000,90.1000) -- (209.2000,89.6000) -- (210.2000,89.0000) -- (211.2000,88.4000) -- (212.2000,87.7000) -- (213.1000,86.9000) -- (213.9000,86.1000) -- (214.6000,85.2000) -- (215.3000,84.3000) -- (215.9000,83.4000) -- (216.5000,82.4000) -- (216.9000,81.4000) -- (217.2000,80.4000) -- (217.4000,79.4000) -- (217.6000,78.4000) -- (217.7000,77.3000) -- (217.8000,76.3000) -- (217.9000,75.3000) -- (217.9000,74.2000) -- (218.0000,73.2000) -- (218.0000,72.1000) -- (218.0000,71.1000) -- (218.1000,70.0000) -- (218.1000,69.0000) -- (218.1000,67.9000) -- (218.1000,66.9000) -- (218.1000,65.8000) -- (218.1000,64.7000) -- (218.1000,63.7000) -- (218.1000,62.6000) -- (218.1000,61.6000) -- (218.1000,60.5000) -- (218.1000,59.5000) -- (218.1000,58.4000) -- (218.1000,57.4000) -- (218.1000,56.3000) -- (218.1000,55.2000) -- (218.1000,54.2000) -- (218.1000,53.1000) -- (218.1000,52.1000) -- (218.1000,51.0000) -- (218.1000,50.0000) -- (218.1000,48.9000) -- (218.3000,47.9000) -- (218.4000,46.8000) -- (218.6000,45.8000) -- (218.6000,44.7000) -- (218.6000,43.7000) -- (218.4000,42.6000) -- (218.2000,41.6000) -- (217.9000,40.6000) -- (217.6000,39.6000) -- (217.1000,38.6000) -- (216.6000,37.6000) -- (216.0000,36.7000) -- (215.3000,35.9000) -- (214.5000,35.0000) -- (213.6000,34.3000) -- (212.7000,33.5000) -- (211.7000,32.9000) -- (210.7000,32.3000) -- (209.6000,31.8000) -- (208.5000,31.4000) -- (207.3000,31.0000) -- (206.1000,30.8000) -- (204.9000,30.6000) -- (203.6000,30.5000) -- (202.4000,30.5000) -- (201.2000,30.6000) -- (200.0000,30.8000) -- (198.8000,31.1000) -- (197.6000,31.4000) -- (196.5000,31.8000) -- (195.4000,32.3000) -- (194.4000,32.9000) -- (193.4000,33.5000) -- (192.5000,34.2000) -- (191.6000,35.0000) -- (190.9000,35.8000) -- (190.1000,36.6000) -- (189.5000,37.5000) -- (188.9000,38.5000) -- (188.5000,39.5000) -- (188.1000,40.5000) -- (187.8000,41.5000) -- (187.7000,42.6000) -- (187.6000,43.6000) -- (187.5000,44.7000) -- (187.5000,45.7000) -- (187.5000,46.8000) -- (187.5000,47.9000) -- (187.5000,48.9000) -- (187.5000,50.0000) -- (187.5000,51.0000) -- (187.5000,52.1000) -- (187.5000,53.1000) -- (187.5000,54.2000) -- (187.5000,55.2000) -- (187.5000,56.3000) -- (187.5000,57.4000) -- (187.5000,58.4000) -- (187.5000,59.5000) -- (187.5000,60.5000) -- (187.5000,61.6000) -- (187.5000,62.6000) -- (187.5000,63.7000) -- (187.5000,64.7000) -- (187.5000,65.8000) -- (187.5000,66.9000) -- (187.5000,67.9000) -- (187.5000,69.0000) -- (187.5000,70.0000) -- (187.5000,71.1000) -- (187.5000,72.1000) -- (187.5000,73.2000) -- (187.5000,74.2000) -- (187.5000,75.3000) -- (187.6000,76.4000) -- (187.7000,77.4000) -- (188.0000,78.4000) -- (188.3000,79.5000) -- (188.7000,80.5000) -- (189.1000,81.5000) -- (189.6000,82.4000) -- (190.2000,83.3000) -- (190.8000,84.2000) -- (191.6000,85.1000) -- (192.3000,85.9000) -- (193.2000,86.7000) -- (194.1000,87.4000) -- (195.0000,88.1000) -- (196.0000,88.7000) -- (197.0000,89.2000) -- (198.1000,89.7000) -- (199.2000,90.2000) -- (200.4000,90.6000) -- (201.5000,90.9000) -- (202.7000,91.1000) -- (204.0000,91.2000) -- (205.2000,91.3000) -- (206.4000,91.3000) -- (207.6000,91.3000) -- (208.9000,91.1000) -- (210.1000,90.9000) -- (211.3000,90.6000) -- (212.5000,90.3000) -- (213.6000,89.9000) -- (214.7000,89.4000) -- (215.8000,88.8000) -- (216.8000,88.2000) -- (217.8000,87.5000) -- (218.7000,86.8000) -- (219.5000,86.0000) -- (220.3000,85.2000) -- (221.1000,84.3000) -- (221.7000,83.4000) -- (222.3000,82.5000) -- (222.8000,81.5000) -- (223.1000,80.5000) -- (223.4000,79.4000) -- (223.6000,78.4000) -- (223.8000,77.4000) -- (223.9000,76.4000) -- (224.0000,75.3000) -- (224.1000,74.3000) -- (224.1000,73.2000) -- (224.1000,72.2000) -- (224.2000,71.1000) -- (224.2000,70.1000) -- (224.2000,69.0000) -- (224.2000,68.0000) -- (224.2000,66.9000) -- (224.2000,65.9000) -- (224.2000,64.8000) -- (224.2000,63.8000) -- (224.2000,62.7000) -- (224.2000,61.7000) -- (224.2000,60.6000) -- (224.2000,59.5000) -- (224.2000,58.5000) -- (224.2000,57.4000) -- (224.2000,56.4000) -- (224.2000,55.3000) -- (224.2000,54.3000) -- (224.2000,53.2000) -- (224.2000,52.2000) -- (224.2000,51.1000) -- (224.2000,50.0000) -- (224.2000,49.0000) -- (224.3000,47.9000) -- (224.5000,46.9000) -- (224.6000,45.8000) -- (224.7000,44.8000) -- (224.7000,43.7000) -- (224.7000,42.7000) -- (224.5000,41.6000) -- (224.3000,40.6000) -- (223.9000,39.6000) -- (223.5000,38.6000) -- (223.1000,37.6000) -- (222.5000,36.7000) -- (221.8000,35.8000) -- (221.1000,35.0000) -- (220.3000,34.2000) -- (219.4000,33.4000) -- (218.5000,32.7000) -- (217.5000,32.1000) -- (216.4000,31.5000) -- (215.3000,31.1000) -- (214.1000,30.7000) -- (213.0000,30.4000) -- (211.7000,30.1000) -- (210.5000,30.0000) -- (209.3000,29.9000) -- (208.1000,30.0000) -- (206.8000,30.1000) -- (205.6000,30.3000) -- (204.5000,30.6000) -- (203.3000,31.0000) -- (202.2000,31.5000) -- (201.2000,32.0000) -- (200.1000,32.6000) -- (199.2000,33.2000) -- (198.3000,34.0000) -- (197.5000,34.7000) -- (196.7000,35.6000) -- (196.0000,36.4000) -- (195.4000,37.4000) -- (194.9000,38.3000) -- (194.4000,39.3000) -- (194.1000,40.3000) -- (193.9000,41.4000) -- (193.8000,42.4000) -- (193.7000,43.5000) -- (193.7000,44.5000) -- (193.7000,45.6000) -- (193.7000,46.7000) -- (193.6000,47.7000) -- (193.6000,48.8000) -- (193.6000,49.8000) -- (193.6000,50.9000) -- (193.6000,51.9000) -- (193.6000,53.0000) -- (193.6000,54.1000) -- (193.6000,55.1000) -- (193.6000,56.2000) -- (193.6000,57.2000) -- (193.6000,58.3000) -- (193.6000,59.3000) -- (193.6000,60.4000) -- (193.6000,61.4000) -- (193.6000,62.5000) -- (193.6000,63.6000) -- (193.6000,64.6000) -- (193.6000,65.7000) -- (193.6000,66.7000) -- (193.6000,67.8000) -- (193.6000,68.8000) -- (193.6000,69.9000) -- (193.6000,70.9000) -- (193.6000,72.0000) -- (193.6000,73.1000) -- (193.6000,74.1000) -- (193.7000,75.2000) -- (193.8000,76.2000) -- (194.0000,77.3000) -- (194.2000,78.3000) -- (194.6000,79.3000) -- (195.0000,80.3000) -- (195.4000,81.3000) -- (196.0000,82.2000) -- (196.6000,83.1000) -- (197.3000,84.0000) -- (198.0000,84.9000) -- (198.8000,85.7000) -- (199.7000,86.4000) -- (200.6000,87.1000) -- (201.5000,87.8000) -- (202.5000,88.4000) -- (203.6000,88.9000) -- (204.7000,89.4000) -- (205.8000,89.8000) -- (207.0000,90.1000) -- (208.2000,90.4000) -- (209.4000,90.6000) -- (210.6000,90.7000) -- (211.8000,90.8000) -- (213.0000,90.8000) -- (214.3000,90.7000) -- (215.5000,90.5000) -- (216.7000,90.3000) -- (217.9000,90.0000) -- (219.0000,89.6000) -- (220.2000,89.1000) -- (221.3000,88.6000) -- (222.3000,88.0000) -- (223.3000,87.4000) -- (224.3000,86.7000) -- (225.2000,85.9000) -- (226.0000,85.2000) -- (226.8000,84.3000) -- (227.5000,83.4000) -- (228.1000,82.5000) -- (228.6000,81.5000) -- (229.1000,80.5000) -- (229.4000,79.5000) -- (229.7000,78.5000) -- (229.9000,77.5000) -- (230.0000,76.4000) -- (230.1000,75.4000) -- (230.2000,74.4000) -- (230.2000,73.3000) -- (230.3000,72.3000) -- (230.3000,71.2000) -- (230.3000,70.2000) -- (230.3000,69.1000) -- (230.3000,68.1000) -- (230.4000,67.0000) -- (230.4000,66.0000) -- (230.4000,64.9000) -- (230.4000,63.9000) -- (230.4000,62.8000) -- (230.4000,61.7000) -- (230.4000,60.7000) -- (230.4000,59.6000) -- (230.4000,58.6000) -- (230.4000,57.5000) -- (230.4000,56.5000) -- (230.4000,55.4000) -- (230.4000,54.4000) -- (230.4000,53.3000) -- (230.4000,52.2000) -- (230.4000,51.2000) -- (230.4000,50.1000) -- (230.4000,49.1000) -- (230.4000,48.0000) -- (230.5000,47.0000) -- (230.7000,45.9000) -- (230.8000,44.8000);



  \end{scope}
  \begin{scope}[cm={{1.15801,0.0,0.0,1.15801,(-736.40737,-53.12875)}},draw=blue,line cap=round,line join=round,line width=0.480pt]
    \path[draw] (148.5000,13.5000) -- (148.5000,102.5000) -- (246.5000,102.5000) -- (246.5000,13.5000) -- (148.5000,13.5000);



  \end{scope}
  \begin{scope}[cm={{0.84173,0.0,0.0,0.84173,(-601.60573,125.64086)}},fill=cffffff]
  \end{scope}
  \begin{scope}[cm={{1.00588,0.0,0.0,1.00588,(-513.7994,343.95446)}},draw=blue,line cap=rect,line join=bevel,line width=0.800pt]
    \begin{scope}[rotate around={-90.0:(-229.13106,-146.12949)},draw=blue,line cap=rect,line join=bevel,line width=0.800pt]
      \path[fill=blue] (0.0000,0.0000) node[above right] (text344-9) {\rotatebox{90}{y (m)}};



    \end{scope}
    \path[fill=blue] (-6.4779,-252.9639) node[above right] (text344-6-3) {x (m)};



    \path[fill=blue] (-76.6396,-237.2505) node[above right] (text344-6-3-1) {(a) Trajectories without re-planning};



    \path[fill=blue] (313 .5506,-237.2505) node[above right] (text344-6-3-1-1-9) {(c) State $(\alpha_0\dots\in\mathbf{q})$ evol. for {\hyperref[fig:trajs-I-static]{\color{red}I}}};



    \path[fill=blue] (100.2713,-237.2505) node[above right] (text344-6-3-1-6) {(b) Energies, detail of first instants, periods $T$};



  \end{scope}
  \begin{scope}[cm={{1.00588,0.0,0.0,1.00588,(-571.04217,77.13242)}},draw=blue,line cap=rect,line join=bevel,line width=0.800pt]
    \path[fill=blue] (0.0000,0.0000) node[above right] (text1768-1) {-150};



  \end{scope}
  \begin{scope}[cm={{1.00588,0.0,0.0,1.00588,(-543.83037,77.13242)}},draw=blue,line cap=rect,line join=bevel,line width=0.800pt]
    \path[fill=blue] (0.0000,0.0000) node[above right] (text1798-7) {-50};



  \end{scope}
  \begin{scope}[cm={{1.00588,0.0,0.0,1.00588,(-513.62717,77.13242)}},draw=blue,line cap=rect,line join=bevel,line width=0.800pt]
    \path[fill=blue] (0.0000,0.0000) node[above right] (text1828-9) {50};



  \end{scope}
  \begin{scope}[cm={{1.00588,0.0,0.0,1.00588,(-488.44517,77.13242)}},draw=blue,line cap=rect,line join=bevel,line width=0.800pt]
    \path[fill=blue] (0.0000,0.0000) node[above right] (text1858-6) {150};



  \end{scope}
  \begin{scope}[cm={{1.00588,0.0,0.0,1.00588,(-462.82756,77.13242)}},draw=blue,line cap=rect,line join=bevel,line width=0.800pt]
    \path[fill=blue] (0.0000,0.0000) node[above right] (text1858-6-6) {250};



  \end{scope}
  \begin{scope}[cm={{1.00588,0.0,0.0,1.00588,(-585.50071,67.88167)}},draw=blue,line cap=rect,line join=bevel,line width=0.800pt]
    \path[fill=blue] (0.0000,0.0000) node[above right] (text34-9-6) {-100};



  \end{scope}
  \begin{scope}[cm={{1.00588,0.0,0.0,1.00588,(-574.76608,42.75178)}},draw=blue,line cap=rect,line join=bevel,line width=0.800pt]
    \path[fill=blue] (0.0000,0.0000) node[above right] (text64-6-8) {0};



  \end{scope}
  \begin{scope}[cm={{1.00588,0.0,0.0,1.00588,(-582.81312,16.12239)}},draw=blue,line cap=rect,line join=bevel,line width=0.800pt]
    \path[fill=blue] (0.0000,0.0000) node[above right] (text94-7-9) {100};



  \end{scope}
  \begin{scope}[cm={{1.00588,0.0,0.0,1.00588,(-582.81312,-9.00701)}},draw=blue,line cap=rect,line join=bevel,line width=0.800pt]
    \path[fill=blue] (0.0000,0.0000) node[above right] (text124-9-0) {200};



  \end{scope}
  \begin{scope}[cm={{1.00588,0.0,0.0,1.00588,(-582.81312,-34.13641)}},draw=blue,line cap=rect,line join=bevel,line width=0.800pt]
    \path[fill=blue] (0.0000,0.0000) node[above right] (text154-3) {300};



  \end{scope}
  \begin{scope}[cm={{1.00769,0.0,0.0,1.00769,(-255.48702,89.55166)}},draw=blue,line cap=rect,line join=bevel,line width=0.800pt]
    \path[fill=blue] (0.0000,0.0000) node[above right] (text602-2) {\scriptsize Time (sec)};



  \end{scope}
  \begin{scope}[cm={{1.00588,0.0,0.0,1.00588,(-140.21331,89.54177)}},draw=blue,line cap=rect,line join=bevel,line width=0.800pt]
    \path[fill=blue] (0.0000,0.0000) node[above right] (text1412-73) {\scriptsize Time (min)};



  \end{scope}
  \begin{scope}[cm={{1.00588,0.0,0.0,1.00588,(-376.86267,89.54177)}},draw=blue,line cap=rect,line join=bevel,line width=0.800pt]
    \path[fill=blue] (0.0000,0.0000) node[above right] (text1412-73-1) {Time (min)};



  \end{scope}
  \begin{scope}[cm={{1.00588,0.0,0.0,1.00588,(-605.35515,-93.86128)}},draw=blue,line cap=rect,line join=bevel,line width=0.800pt]
    \path[fill=blue] (0.0000,0.0000) node[above right] (text154-9) {\Large{\label{fig:ener:static-I}\label{fig:trajs-I-static}I}};



  \end{scope}
  \begin{scope}[cm={{1.00588,0.0,0.0,1.00588,(-607.8469,20.77248)}},draw=blue,line cap=rect,line join=bevel,line width=0.800pt]
    \path[fill=blue] (2.4954,0.0000) node[above right] (text154-9-0) {\Large{\label{fig:ener:static-II}\label{fig:trajs-II-static}II}};



  \end{scope}
  \begin{scope}[cm={{0.0,-1.00588,1.00588,0.0,(154.28002,10.40301)}},draw=blue,line cap=rect,line join=bevel,line width=0.800pt]
    \path[fill=blue] (35.7896,-587.5454) node[above right] (text274-1) {\rotatebox{90}{Power (W)}};



  \end{scope}
  \begin{scope}[cm={{1.00588,0.0,0.0,1.00588,(-526.48136,104.32102)}},draw=blue,line cap=rect,line join=bevel,line width=0.800pt]
  \end{scope}
\end{scope}

\end{tikzpicture}


  \end{minipage}\hfill
  \begin{minipage}[t]{0.36\columnwidth}
    \vspace*{-53.8ex}
    \centering
    \caption{\color{black}Results for path, energy, and energy models of two boundary configurations, the lowest \hyperref[fig:trajs-I-static]{I} and the highest \hyperref[fig:trajs-II-static]{II}, without energy-aware planning-scheduling. The use case is that of coverage path planning with Zamboni-like motion and ground hazards detection scheduling. In \hyperref[fig:stat]{a} are the trajectories of the coverage. In \hyperref[fig:stat]{b} are the energy and the period evolutions for both \hyperref[fig:trajs-I-static]{I} and \hyperref[fig:trajs-II-static]{II} with different atmospheric conditions (i.e., different wind speed and direction) and initial guesses, and in \hyperref[fig:stat]{c} the states of the energy model for \hyperref[fig:trajs-I-static]{I}.}
    \label{fig:stat}
  \end{minipage}
  \vspace*{-2ex}
\end{figure*}
\begin{figure*}
  %\vspace*{-4ex}
  \centering
  \footnotesize
  \begin{minipage}[t]{0.63\columnwidth}
    
\definecolor{ca0a0a4}{RGB}{160,160,164}
\definecolor{cd9d9d9}{RGB}{235,235,235}
\definecolor{c00ff00}{RGB}{0,255,0}
\definecolor{cffffff}{RGB}{255,255,255}


\def \globalscale {1.000000}
\begin{tikzpicture}[y=0.80pt, x=0.80pt, yscale=-1.03*\globalscale, xscale=1.18*\globalscale, inner sep=0pt, outer sep=0pt]
\color{blue}
\begin{scope}[shift={(598.44106,-141.22792)},draw=blue,even odd rule,line cap=rect,line join=bevel,line width=0.800pt]
  \begin{scope}[cm={{0.84173,0.0,0.0,0.84173,(-603.85556,69.88496)}},draw=ca0a0a4,dash pattern=on 2.14pt off 2.14pt,line cap=round,line join=round,line width=0.356pt,miter limit=4.00]
    \path[draw,dash pattern=on 2.14pt off 2.14pt,line width=0.356pt,miter limit=4.00] (44.5000,181.5000) -- (179.5000,181.5000);



  \end{scope}
  \begin{scope}[cm={{0.84173,0.0,0.0,0.84173,(-603.85556,69.88496)}},draw=ca0a0a4,dash pattern=on 2.14pt off 2.14pt,line cap=round,line join=round,line width=0.356pt,miter limit=4.00]
    \path[draw,dash pattern=on 2.14pt off 2.14pt,line width=0.356pt,miter limit=4.00] (78.5000,210.5000) -- (78.5000,108.5000);



  \end{scope}
  \begin{scope}[cm={{0.84173,0.0,0.0,0.84173,(-603.85556,69.88496)}},draw=blue,line cap=round,line join=round,line width=0.480pt]
    \path[draw] (61.9000,118.7000) -- (61.9000,118.7000) -- (62.6000,120.9000) -- (63.3000,122.6000) -- (63.5000,124.1000) -- (62.7000,125.3000) -- (61.5000,126.5000) -- (60.3000,127.9000) -- (59.2000,129.5000) -- (58.3000,131.2000) -- (57.5000,133.0000) -- (57.0000,134.8000) -- (56.8000,136.7000) -- (56.7000,138.6000) -- (56.6000,140.4000) -- (56.6000,142.2000) -- (56.6000,144.1000) -- (56.6000,145.9000) -- (56.6000,147.7000) -- (56.6000,149.5000) -- (56.6000,151.4000) -- (56.6000,153.2000) -- (56.6000,155.1000) -- (56.6000,156.9000) -- (56.6000,158.8000) -- (56.6000,160.6000) -- (56.6000,162.5000) -- (56.6000,164.3000) -- (56.6000,166.2000) -- (56.6000,168.0000) -- (56.6000,169.9000) -- (56.6000,171.7000) -- (56.6000,173.6000) -- (56.6000,175.4000) -- (56.6000,177.2000) -- (56.6000,179.1000) -- (56.7000,181.0000) -- (57.1000,182.8000) -- (57.6000,184.7000) -- (58.3000,186.4000) -- (59.2000,188.2000) -- (60.2000,189.8000) -- (61.4000,191.3000) -- (62.7000,192.8000) -- (64.2000,194.1000) -- (65.7000,195.3000) -- (67.4000,196.4000) -- (69.1000,197.4000) -- (71.0000,198.2000) -- (72.9000,198.9000) -- (74.8000,199.5000) -- (76.8000,199.9000) -- (78.8000,200.2000) -- (80.8000,200.4000) -- (82.8000,200.4000) -- (84.8000,200.2000) -- (86.7000,200.0000) -- (88.7000,199.6000) -- (90.6000,199.0000) -- (92.5000,198.4000) -- (94.3000,197.5000) -- (96.1000,196.6000) -- (97.8000,195.5000) -- (99.3000,194.3000) -- (100.8000,193.0000) -- (102.1000,191.6000) -- (103.3000,190.0000) -- (104.2000,188.4000) -- (104.9000,186.7000) -- (105.3000,184.9000) -- (105.6000,183.0000) -- (105.7000,181.1000) -- (105.8000,179.2000) -- (105.8000,177.4000) -- (105.8000,175.5000) -- (105.8000,173.6000) -- (105.8000,171.7000) -- (105.8000,169.9000) -- (105.9000,168.0000) -- (105.9000,166.2000) -- (105.9000,164.3000) -- (105.9000,162.5000) -- (105.9000,160.6000) -- (105.9000,158.8000) -- (105.9000,157.0000) -- (105.9000,155.1000) -- (105.9000,153.3000) -- (105.9000,151.4000) -- (105.9000,149.6000) -- (105.9000,147.7000) -- (105.8000,145.9000) -- (106.1000,144.0000) -- (106.5000,142.1000) -- (106.8000,140.3000) -- (106.8000,138.6000) -- (106.7000,136.8000) -- (106.4000,135.0000) -- (105.9000,133.2000) -- (105.2000,131.5000) -- (104.3000,129.8000) -- (103.3000,128.3000) -- (102.1000,126.8000) -- (100.8000,125.4000) -- (99.3000,124.1000) -- (97.7000,122.9000) -- (96.0000,121.9000) -- (94.3000,121.0000) -- (92.4000,120.3000) -- (90.5000,119.7000) -- (88.5000,119.2000) -- (86.5000,118.9000) -- (84.5000,118.8000) -- (82.4000,118.8000) -- (80.4000,119.0000) -- (78.4000,119.3000) -- (76.4000,119.8000) -- (74.5000,120.4000) -- (72.6000,121.2000) -- (70.9000,122.1000) -- (69.2000,123.1000) -- (67.6000,124.3000) -- (66.2000,125.6000) -- (64.9000,127.0000) -- (63.8000,128.5000) -- (62.8000,130.1000) -- (62.0000,131.8000) -- (61.3000,133.6000) -- (60.9000,135.4000) -- (60.7000,137.2000) -- (60.6000,139.1000) -- (60.6000,140.9000) -- (60.6000,142.7000) -- (60.6000,144.5000) -- (60.6000,146.4000) -- (60.6000,148.2000) -- (60.6000,150.0000) -- (60.6000,151.9000) -- (60.6000,153.7000) -- (60.5000,155.6000) -- (60.5000,157.4000) -- (60.5000,159.3000) -- (60.5000,161.1000) -- (60.5000,163.0000) -- (60.5000,164.8000) -- (60.5000,166.7000) -- (60.5000,168.5000) -- (60.6000,170.4000) -- (60.6000,172.2000) -- (60.6000,174.1000) -- (60.6000,175.9000) -- (60.5000,177.7000) -- (60.6000,179.6000) -- (60.8000,181.5000) -- (61.2000,183.3000) -- (61.9000,185.1000) -- (62.7000,186.9000) -- (63.6000,188.6000) -- (64.7000,190.1000) -- (66.0000,191.6000) -- (67.4000,193.0000) -- (68.9000,194.3000) -- (70.6000,195.4000) -- (72.3000,196.4000) -- (74.1000,197.3000) -- (76.0000,198.1000) -- (77.9000,198.7000) -- (79.8000,199.2000) -- (81.8000,199.5000) -- (83.8000,199.7000) -- (85.8000,199.7000) -- (87.8000,199.6000) -- (89.8000,199.4000) -- (91.8000,199.0000) -- (93.7000,198.5000) -- (95.6000,197.8000) -- (97.4000,197.0000) -- (99.2000,196.1000) -- (100.8000,195.0000) -- (102.4000,193.8000) -- (103.9000,192.5000) -- (105.2000,191.1000) -- (106.4000,189.5000) -- (107.3000,187.9000) -- (108.0000,186.2000) -- (108.5000,184.4000) -- (108.7000,182.5000) -- (108.9000,180.6000) -- (109.0000,178.7000) -- (109.0000,176.9000) -- (109.0000,175.0000) -- (109.0000,173.1000) -- (109.0000,171.2000) -- (109.1000,169.4000) -- (109.1000,167.5000) -- (109.1000,165.7000) -- (109.1000,163.8000) -- (109.1000,162.0000) -- (109.1000,160.1000) -- (109.1000,158.3000) -- (109.1000,156.5000) -- (109.1000,154.6000) -- (109.1000,152.8000) -- (109.1000,150.9000) -- (109.1000,149.1000) -- (109.1000,147.2000) -- (109.1000,145.4000) -- (109.3000,143.5000) -- (109.7000,141.6000) -- (110.0000,139.8000) -- (110.0000,138.1000) -- (109.8000,136.3000) -- (109.5000,134.5000) -- (109.0000,132.7000) -- (108.3000,131.0000) -- (107.4000,129.4000) -- (106.3000,127.8000) -- (105.1000,126.3000) -- (103.7000,125.0000) -- (102.3000,123.7000) -- (100.6000,122.6000) -- (98.9000,121.6000) -- (97.1000,120.7000) -- (95.2000,120.0000) -- (93.3000,119.5000) -- (91.3000,119.1000) -- (89.3000,118.8000) -- (87.3000,118.7000) -- (85.2000,118.8000) -- (83.2000,119.0000) -- (81.2000,119.4000) -- (79.3000,119.9000) -- (77.4000,120.6000) -- (75.6000,121.5000) -- (73.9000,122.6000) -- (72.4000,123.8000) -- (70.9000,125.1000) -- (69.6000,126.5000) -- (68.5000,128.0000) -- (67.5000,129.6000) -- (66.7000,131.3000) -- (66.0000,133.0000) -- (65.7000,134.9000) -- (65.5000,136.7000) -- (65.4000,138.6000) -- (65.4000,140.4000) -- (65.4000,142.2000) -- (65.4000,144.0000) -- (65.3000,145.8000) -- (65.3000,147.7000) -- (65.3000,149.5000) -- (65.3000,151.4000) -- (65.3000,153.2000) -- (65.3000,155.1000) -- (65.3000,156.9000) -- (65.3000,158.8000) -- (65.3000,160.6000) -- (65.3000,162.4000) -- (65.3000,164.3000) -- (65.3000,166.1000) -- (65.3000,168.0000) -- (65.3000,169.8000) -- (65.3000,171.7000) -- (65.3000,173.5000) -- (65.3000,175.4000) -- (65.3000,177.2000) -- (65.4000,179.1000) -- (65.6000,181.0000) -- (66.0000,182.8000) -- (66.6000,184.6000) -- (67.4000,186.4000) -- (68.3000,188.1000) -- (69.4000,189.7000) -- (70.6000,191.2000) -- (71.9000,192.6000) -- (73.4000,193.9000) -- (75.0000,195.1000) -- (76.7000,196.2000) -- (78.5000,197.1000) -- (80.3000,198.0000) -- (82.2000,198.7000) -- (84.1000,199.2000) -- (86.1000,199.6000) -- (88.1000,199.9000) -- (90.1000,200.1000) -- (92.1000,200.2000) -- (94.1000,200.1000) -- (96.1000,199.8000) -- (98.0000,199.5000) -- (100.0000,199.0000) -- (101.9000,198.4000) -- (103.8000,197.7000) -- (105.6000,196.9000) -- (107.3000,195.9000) -- (109.0000,194.8000) -- (110.6000,193.6000) -- (112.0000,192.3000) -- (113.4000,190.9000) -- (114.6000,189.4000) -- (115.4000,187.7000) -- (115.9000,185.9000) -- (116.1000,184.1000) -- (116.2000,182.2000) -- (116.2000,180.3000) -- (116.3000,178.4000) -- (116.3000,176.5000) -- (116.3000,174.6000) -- (116.3000,172.8000) -- (116.3000,170.9000) -- (116.3000,169.1000) -- (116.3000,167.2000) -- (116.3000,165.4000) -- (116.3000,163.5000) -- (116.3000,161.7000) -- (116.3000,159.8000) -- (116.3000,158.0000) -- (116.3000,156.1000) -- (116.3000,154.3000) -- (116.3000,152.4000) -- (116.3000,150.6000) -- (116.3000,148.8000) -- (116.3000,146.9000) -- (116.3000,145.1000) -- (116.3000,143.2000) -- (116.5000,141.3000) -- (116.7000,139.5000) -- (116.7000,137.7000) -- (116.6000,135.9000) -- (116.3000,134.1000) -- (115.8000,132.3000) -- (115.1000,130.6000) -- (114.2000,129.0000) -- (113.2000,127.4000) -- (112.0000,125.9000) -- (110.6000,124.6000) -- (109.1000,123.3000) -- (107.5000,122.2000) -- (105.8000,121.2000) -- (104.0000,120.3000) -- (102.1000,119.6000) -- (100.2000,119.0000) -- (98.2000,118.6000) -- (96.2000,118.4000) -- (94.1000,118.3000) -- (92.1000,118.4000) -- (90.1000,118.6000) -- (88.1000,119.0000) -- (86.1000,119.5000) -- (84.2000,120.2000) -- (82.4000,121.1000) -- (80.7000,122.0000) -- (79.1000,123.2000) -- (77.6000,124.4000) -- (76.2000,125.8000) -- (75.0000,127.3000) -- (74.0000,128.8000) -- (73.1000,130.5000) -- (72.4000,132.2000) -- (72.0000,134.0000) -- (71.8000,135.9000) -- (71.7000,137.7000) -- (71.6000,139.5000) -- (71.6000,141.4000) -- (71.6000,143.2000) -- (71.6000,145.0000) -- (71.6000,146.8000) -- (71.6000,148.7000) -- (71.6000,150.5000) -- (71.6000,152.4000) -- (71.6000,154.2000) -- (71.6000,156.1000) -- (71.6000,157.9000) -- (71.6000,159.8000) -- (71.6000,161.6000) -- (71.6000,163.5000) -- (71.6000,165.3000) -- (71.6000,167.2000) -- (71.6000,169.0000) -- (71.6000,170.8000) -- (71.6000,172.7000) -- (71.6000,174.5000) -- (71.6000,176.4000) -- (71.6000,178.2000) -- (71.8000,180.1000) -- (72.2000,182.0000) -- (72.7000,183.8000) -- (73.5000,185.6000) -- (74.4000,187.3000) -- (75.5000,188.9000) -- (76.7000,190.4000) -- (78.0000,191.8000) -- (79.5000,193.1000) -- (81.1000,194.3000) -- (82.8000,195.4000) -- (84.6000,196.3000) -- (86.4000,197.1000) -- (88.3000,197.8000) -- (90.3000,198.3000) -- (92.2000,198.7000) -- (94.2000,199.0000) -- (96.2000,199.1000) -- (98.2000,199.1000) -- (100.2000,198.9000) -- (102.2000,198.6000) -- (104.2000,198.2000) -- (106.1000,197.6000) -- (107.9000,196.9000) -- (109.7000,196.0000) -- (111.5000,195.1000) -- (113.1000,194.0000) -- (114.7000,192.7000) -- (116.1000,191.4000) -- (117.4000,189.9000) -- (118.5000,188.4000) -- (119.4000,186.7000) -- (120.0000,185.0000) -- (120.4000,183.2000) -- (120.6000,181.3000) -- (120.7000,179.4000) -- (120.8000,177.5000) -- (120.8000,175.6000) -- (120.8000,173.8000) -- (120.8000,171.9000) -- (120.8000,170.0000) -- (120.9000,168.2000) -- (120.9000,166.3000) -- (120.9000,164.5000) -- (120.9000,162.6000) -- (120.9000,160.8000) -- (120.9000,158.9000) -- (120.9000,157.1000) -- (120.9000,155.2000) -- (120.9000,153.4000) -- (120.9000,151.5000) -- (120.9000,149.7000) -- (120.9000,147.8000) -- (120.9000,146.0000) -- (120.9000,144.1000) -- (121.2000,142.3000) -- (121.7000,140.4000) -- (121.9000,138.6000) -- (121.9000,136.9000) -- (121.8000,135.1000) -- (121.4000,133.3000) -- (120.9000,131.6000) -- (120.1000,129.9000) -- (119.2000,128.2000) -- (118.1000,126.7000) -- (116.9000,125.2000) -- (115.5000,123.9000) -- (114.0000,122.7000) -- (112.3000,121.6000) -- (110.6000,120.6000) -- (108.7000,119.8000) -- (106.8000,119.2000) -- (104.9000,118.7000) -- (102.8000,118.4000) -- (100.8000,118.2000) -- (98.8000,118.2000) -- (96.7000,118.4000) -- (94.7000,118.7000) -- (92.8000,119.2000) -- (90.9000,119.8000) -- (89.0000,120.6000) -- (87.3000,121.6000) -- (85.6000,122.6000) -- (84.1000,123.9000) -- (82.7000,125.2000) -- (81.5000,126.7000) -- (80.4000,128.2000) -- (79.5000,129.9000) -- (78.7000,131.6000) -- (78.3000,133.4000) -- (78.0000,135.2000) -- (77.9000,137.1000) -- (77.9000,138.9000) -- (77.8000,140.7000) -- (77.8000,142.5000) -- (77.8000,144.4000) -- (77.8000,146.2000) -- (77.8000,148.0000) -- (77.8000,149.9000) -- (77.8000,151.7000) -- (77.8000,153.6000) -- (77.8000,155.4000) -- (77.8000,157.3000) -- (77.8000,159.1000) -- (77.8000,161.0000) -- (77.8000,162.8000) -- (77.8000,164.6000) -- (77.8000,166.5000) -- (77.8000,168.3000) -- (77.8000,170.2000) -- (77.8000,172.0000) -- (77.8000,173.9000) -- (77.8000,175.7000) -- (77.8000,177.6000) -- (78.0000,179.5000) -- (78.4000,181.3000) -- (78.9000,183.1000) -- (79.7000,184.9000) -- (80.6000,186.6000) -- (81.7000,188.2000) -- (82.9000,189.7000) -- (84.2000,191.1000) -- (85.7000,192.5000) -- (87.3000,193.6000) -- (89.0000,194.7000) -- (90.8000,195.6000) -- (92.6000,196.4000) -- (94.5000,197.1000) -- (96.5000,197.6000) -- (98.5000,198.0000) -- (100.5000,198.2000) -- (102.5000,198.3000) -- (104.5000,198.3000) -- (106.5000,198.1000) -- (108.4000,197.7000) -- (110.4000,197.3000) -- (112.3000,196.7000) -- (114.1000,195.9000) -- (115.9000,195.0000) -- (117.6000,194.0000) -- (119.2000,192.9000) -- (120.8000,191.6000) -- (122.1000,190.2000) -- (123.4000,188.7000) -- (124.5000,187.1000) -- (125.3000,185.5000) -- (125.8000,183.7000) -- (126.1000,181.8000) -- (126.3000,180.0000) -- (126.4000,178.1000) -- (126.5000,176.2000) -- (126.5000,174.3000) -- (126.5000,172.4000) -- (126.5000,170.6000) -- (126.5000,168.7000) -- (126.5000,166.9000) -- (126.5000,165.0000) -- (126.5000,163.2000) -- (126.6000,161.3000) -- (126.6000,159.5000) -- (126.6000,157.6000) -- (126.6000,155.8000) -- (126.6000,153.9000) -- (126.6000,152.1000) -- (126.5000,150.2000) -- (126.5000,148.4000) -- (126.5000,146.5000) -- (126.5000,144.7000) -- (126.6000,142.8000) -- (127.1000,141.0000) -- (127.5000,139.2000) -- (127.6000,137.4000) -- (127.6000,135.6000) -- (127.3000,133.8000) -- (126.9000,132.0000) -- (126.3000,130.3000) -- (125.5000,128.6000) -- (124.5000,127.0000) -- (123.3000,125.5000) -- (122.0000,124.1000) -- (120.5000,122.8000) -- (118.9000,121.7000) -- (117.2000,120.7000) -- (115.4000,119.8000) -- (113.5000,119.1000) -- (111.6000,118.5000) -- (109.6000,118.1000) -- (107.6000,117.9000) -- (105.6000,117.8000) -- (103.5000,117.9000) -- (101.5000,118.2000) -- (99.5000,118.6000) -- (97.6000,119.2000) -- (95.7000,119.9000) -- (93.9000,120.8000) -- (92.2000,121.8000) -- (90.7000,123.0000) -- (89.2000,124.3000) -- (87.9000,125.7000) -- (86.8000,127.2000) -- (85.8000,128.8000) -- (85.0000,130.5000) -- (84.5000,132.3000) -- (84.1000,134.1000) -- (84.0000,136.0000) -- (83.9000,137.8000) -- (83.9000,139.6000) -- (83.9000,141.4000) -- (83.9000,143.3000) -- (83.8000,145.1000) -- (83.8000,146.9000) -- (83.8000,148.8000) -- (83.8000,150.6000) -- (83.8000,152.5000) -- (83.8000,154.3000) -- (83.8000,156.2000) -- (83.8000,158.0000) -- (83.8000,159.9000) -- (83.8000,161.7000) -- (83.8000,163.5000) -- (83.8000,165.4000) -- (83.8000,167.2000) -- (83.8000,169.1000) -- (83.8000,170.9000) -- (83.8000,172.8000) -- (83.8000,174.6000) -- (83.8000,176.5000) -- (83.9000,178.3000) -- (84.2000,180.2000) -- (84.8000,182.1000) -- (85.5000,183.8000) -- (86.3000,185.6000) -- (87.3000,187.2000) -- (88.5000,188.7000) -- (89.9000,190.2000) -- (91.3000,191.5000) -- (92.9000,192.7000) -- (94.5000,193.8000) -- (96.3000,194.8000) -- (98.1000,195.6000) -- (100.0000,196.3000) -- (102.0000,196.9000) -- (103.9000,197.3000) -- (105.9000,197.6000) -- (107.9000,197.7000) -- (109.9000,197.7000) -- (111.9000,197.5000) -- (113.9000,197.2000) -- (115.8000,196.8000) -- (117.7000,196.2000) -- (119.6000,195.5000) -- (121.4000,194.6000) -- (123.1000,193.6000) -- (124.8000,192.5000) -- (126.3000,191.3000) -- (127.7000,189.9000) -- (129.0000,188.4000) -- (130.1000,186.9000) -- (130.9000,185.2000) -- (131.5000,183.4000) -- (131.9000,181.6000) -- (132.1000,179.7000) -- (132.2000,177.8000) -- (132.3000,176.0000) -- (132.3000,174.1000) -- (132.3000,172.2000) -- (132.3000,170.3000) -- (132.3000,168.5000) -- (132.3000,166.6000) -- (132.3000,164.8000) -- (132.3000,162.9000) -- (132.3000,161.1000) -- (132.3000,159.2000) -- (132.3000,157.4000) -- (132.3000,155.5000) -- (132.3000,153.7000) -- (132.3000,151.8000) -- (132.3000,150.0000) -- (132.3000,148.1000) -- (132.3000,146.3000) -- (132.3000,144.4000) -- (132.4000,142.6000) -- (132.9000,140.7000) -- (133.2000,138.9000) -- (133.4000,137.1000) -- (133.4000,135.3000) -- (133.2000,133.5000) -- (132.8000,131.8000) -- (132.2000,130.0000) -- (131.3000,128.4000) -- (130.3000,126.8000) -- (129.1000,125.3000) -- (127.8000,123.9000) -- (126.3000,122.7000) -- (124.7000,121.6000) -- (122.9000,120.6000) -- (121.1000,119.8000) -- (119.2000,119.2000) -- (117.2000,118.7000) -- (115.2000,118.4000) -- (113.1000,118.3000) -- (111.1000,118.3000) -- (109.1000,118.6000) -- (107.1000,119.0000) -- (105.2000,119.6000) -- (103.3000,120.3000) -- (101.5000,121.2000) -- (99.9000,122.3000) -- (98.3000,123.5000) -- (96.9000,124.8000) -- (95.7000,126.3000) -- (94.6000,127.9000) -- (93.8000,129.5000) -- (93.1000,131.3000) -- (92.7000,133.1000) -- (92.6000,134.9000) -- (92.5000,136.8000) -- (92.4000,138.6000) -- (92.4000,140.4000) -- (92.4000,142.2000) -- (92.4000,144.1000) -- (92.4000,145.9000) -- (92.4000,147.7000) -- (92.4000,149.6000) -- (92.4000,151.4000) -- (92.4000,153.3000) -- (92.4000,155.1000) -- (92.4000,157.0000) -- (92.4000,158.8000) -- (92.4000,160.7000) -- (92.4000,162.5000) -- (92.4000,164.4000) -- (92.4000,166.2000) -- (92.4000,168.1000) -- (92.4000,169.9000) -- (92.4000,171.8000) -- (92.4000,173.6000) -- (92.4000,175.4000) -- (92.4000,177.3000) -- (92.7000,179.2000) -- (93.2000,181.0000) -- (93.8000,182.8000) -- (94.7000,184.5000) -- (95.7000,186.2000) -- (96.9000,187.8000) -- (98.2000,189.2000) -- (99.6000,190.6000) -- (101.1000,191.8000) -- (102.8000,192.9000) -- (104.5000,193.9000) -- (106.4000,194.8000) -- (108.2000,195.5000) -- (110.2000,196.0000) -- (112.1000,196.5000) -- (114.1000,196.8000) -- (116.1000,196.9000) -- (118.1000,196.9000) -- (120.1000,196.8000) -- (122.1000,196.5000) -- (124.1000,196.1000) -- (126.0000,195.5000) -- (127.8000,194.8000) -- (129.7000,194.0000) -- (131.4000,193.0000) -- (133.0000,191.9000) -- (134.6000,190.6000) -- (136.0000,189.3000) -- (137.3000,187.8000) -- (138.4000,186.3000) -- (139.3000,184.6000) -- (139.9000,182.8000) -- (140.3000,181.0000) -- (140.5000,179.2000) -- (140.6000,177.3000) -- (140.7000,175.4000) -- (140.7000,173.5000) -- (140.7000,171.6000) -- (140.7000,169.8000) -- (140.8000,167.9000) -- (140.8000,166.0000) -- (140.8000,164.2000) -- (140.8000,162.3000) -- (140.8000,160.5000) -- (140.8000,158.6000) -- (140.8000,156.8000) -- (140.8000,154.9000) -- (140.8000,153.1000) -- (140.8000,151.3000) -- (140.8000,149.4000) -- (140.8000,147.6000) -- (140.8000,145.7000) -- (140.8000,143.9000) -- (140.8000,142.0000) -- (141.3000,140.1000) -- (141.7000,138.3000) -- (142.0000,136.5000) -- (142.0000,134.8000) -- (141.8000,133.0000) -- (141.4000,131.2000) -- (140.8000,129.5000) -- (140.0000,127.8000) -- (139.0000,126.2000) -- (137.9000,124.7000) -- (136.5000,123.3000) -- (135.1000,122.0000) -- (133.5000,120.9000) -- (131.7000,119.9000) -- (129.9000,119.1000) -- (128.0000,118.4000) -- (126.1000,117.9000) -- (124.0000,117.6000) -- (122.0000,117.4000) -- (120.0000,117.4000) -- (117.9000,117.6000) -- (115.9000,118.0000) -- (114.0000,118.5000) -- (112.1000,119.2000) -- (110.3000,120.1000) -- (108.6000,121.1000) -- (107.0000,122.2000) -- (105.6000,123.5000) -- (104.3000,124.9000) -- (103.2000,126.5000) -- (102.2000,128.1000) -- (101.5000,129.8000) -- (101.0000,131.6000) -- (100.8000,133.5000) -- (100.6000,135.3000) -- (100.6000,137.1000) -- (100.6000,138.9000) -- (100.6000,140.8000) -- (100.5000,142.6000) -- (100.5000,144.4000) -- (100.5000,146.3000) -- (100.5000,148.1000) -- (100.5000,149.9000) -- (100.5000,151.8000) -- (100.5000,153.6000) -- (100.5000,155.5000) -- (100.5000,157.3000) -- (100.5000,159.2000) -- (100.5000,161.0000) -- (100.5000,162.9000) -- (100.5000,164.7000) -- (100.5000,166.6000) -- (100.5000,168.4000) -- (100.5000,170.3000) -- (100.5000,172.1000) -- (100.5000,174.0000) -- (100.5000,175.8000) -- (100.7000,177.7000) -- (101.1000,179.6000) -- (101.7000,181.4000) -- (102.5000,183.1000) -- (103.4000,184.8000) -- (104.5000,186.4000) -- (105.8000,187.9000) -- (107.1000,189.3000) -- (108.6000,190.6000) -- (110.2000,191.8000) -- (111.9000,192.8000) -- (113.7000,193.7000) -- (115.6000,194.5000) -- (117.5000,195.2000) -- (119.4000,195.7000) -- (121.4000,196.0000) -- (123.4000,196.2000) -- (125.4000,196.3000) -- (127.4000,196.2000) -- (129.4000,196.0000) -- (131.4000,195.7000) -- (133.3000,195.2000) -- (135.2000,194.6000) -- (137.1000,193.8000) -- (138.8000,192.9000) -- (140.5000,191.9000) -- (142.1000,190.7000) -- (143.6000,189.4000) -- (145.0000,188.0000) -- (146.2000,186.5000) -- (147.3000,184.9000) -- (148.0000,183.2000) -- (148.5000,181.4000) -- (148.8000,179.6000) -- (149.0000,177.7000) -- (149.1000,175.8000) -- (149.1000,173.9000) -- (149.2000,172.1000) -- (149.2000,170.2000) -- (149.2000,168.3000) -- (149.2000,166.5000) -- (149.2000,164.6000) -- (149.2000,162.8000) -- (149.2000,160.9000) -- (149.2000,159.1000) -- (149.2000,157.2000) -- (149.2000,155.4000) -- (149.2000,153.5000) -- (149.2000,151.7000) -- (149.2000,149.8000) -- (149.2000,148.0000) -- (149.2000,146.1000) -- (149.2000,144.3000) -- (149.2000,142.4000) -- (149.4000,140.6000) -- (149.9000,138.7000) -- (150.3000,136.9000) -- (150.4000,135.1000) -- (150.4000,133.4000) -- (150.1000,131.6000) -- (149.7000,129.8000) -- (149.0000,128.1000) -- (148.1000,126.5000) -- (147.1000,124.9000) -- (145.9000,123.4000) -- (144.5000,122.1000) -- (142.9000,120.8000) -- (141.3000,119.8000) -- (139.5000,118.8000) -- (137.7000,118.0000) -- (135.8000,117.4000) -- (133.8000,117.0000) -- (131.8000,116.7000) -- (129.7000,116.6000) -- (127.7000,116.7000) -- (125.7000,117.0000) -- (123.7000,117.4000) -- (121.7000,118.0000) -- (119.9000,118.7000) -- (118.1000,119.6000) -- (116.4000,120.7000) -- (114.9000,121.9000) -- (113.5000,123.2000) -- (112.3000,124.7000) -- (111.2000,126.3000) -- (110.3000,127.9000) -- (109.7000,129.7000) -- (109.3000,131.5000) -- (109.1000,133.3000) -- (109.0000,135.2000) -- (109.0000,137.0000) -- (109.0000,138.8000) -- (108.9000,140.6000) -- (108.9000,142.5000) -- (108.9000,144.3000) -- (108.9000,146.1000) -- (108.9000,148.0000) -- (108.9000,149.8000) -- (108.9000,151.7000) -- (108.9000,153.5000) -- (108.9000,155.4000) -- (108.9000,157.2000) -- (108.9000,159.1000) -- (108.9000,160.9000) -- (108.9000,162.8000) -- (108.9000,164.6000) -- (108.9000,166.5000) -- (108.9000,168.3000) -- (108.9000,170.1000) -- (108.9000,172.0000) -- (108.9000,173.8000) -- (109.0000,175.7000) -- (109.2000,177.6000) -- (109.7000,179.4000) -- (110.4000,181.2000) -- (111.2000,183.0000) -- (112.2000,184.6000) -- (113.3000,186.2000) -- (114.6000,187.7000) -- (116.0000,189.0000) -- (117.5000,190.3000) -- (119.2000,191.4000) -- (120.9000,192.4000) -- (122.7000,193.3000) -- (124.6000,194.0000) -- (126.5000,194.6000) -- (128.5000,195.1000) -- (130.5000,195.4000) -- (132.5000,195.6000) -- (134.5000,195.6000) -- (136.5000,195.5000) -- (138.5000,195.2000) -- (140.4000,194.8000) -- (142.3000,194.3000) -- (144.2000,193.6000) -- (146.0000,192.8000) -- (147.8000,191.9000) -- (149.5000,190.8000) -- (151.0000,189.6000) -- (152.5000,188.3000) -- (153.8000,186.9000) -- (155.0000,185.3000) -- (156.0000,183.7000) -- (156.6000,182.0000) -- (157.1000,180.2000) -- (157.3000,178.3000) -- (157.5000,176.4000) -- (157.6000,174.5000) -- (157.6000,172.7000) -- (157.6000,170.8000) -- (157.6000,168.9000) -- (157.6000,167.0000) -- (157.6000,165.2000) -- (157.6000,163.3000) -- (157.7000,161.5000) -- (157.7000,159.6000) -- (157.7000,157.8000) -- (157.7000,155.9000) -- (157.7000,154.1000) -- (157.7000,152.2000) -- (157.7000,150.4000) -- (157.7000,148.6000) -- (157.7000,146.7000) -- (157.7000,144.9000) -- (157.6000,143.0000) -- (157.6000,141.2000) -- (158.0000,139.3000) -- (158.5000,137.4000) -- (158.8000,135.6000);



  \end{scope}
  \path[fill=cd9d9d9,dash pattern=on 1.12pt off 1.12pt,even odd rule,line cap=round,line width=0.281pt,miter limit=4.00,rounded corners=0.0000cm] (-617.3423,257.0234) rectangle (-74.5817,379.8000);



  \begin{scope}[cm={{1.04177,0.0,0.0,1.04177,(-342.63646,42.1635)}},draw=ca0a0a4,dash pattern=on 1.73pt off 1.73pt,line cap=round,line join=round,line width=0.288pt,miter limit=4.00]
    \path[draw,dash pattern=on 1.73pt off 1.73pt,line width=0.288pt,miter limit=4.00] (56.5000,232.5000) -- (251.5000,232.5000);



  \end{scope}
  \begin{scope}[draw=blue,line cap=rect,line join=bevel,line width=0.800pt]
  \end{scope}
  \begin{scope}[scale=1.006,draw=blue,line cap=rect,line join=bevel,line width=0.800pt]
  \end{scope}
  \begin{scope}[scale=1.006,draw=blue,line cap=rect,line join=bevel,line width=0.800pt]
  \end{scope}
  \begin{scope}[cm={{1.00588,0.0,0.0,1.00588,(39.2294,93.5471)}},draw=blue,line cap=rect,line join=bevel,line width=0.800pt]
  \end{scope}
  \begin{scope}[cm={{1.00588,0.0,0.0,1.00588,(39.2294,93.5471)}},draw=blue,line cap=rect,line join=bevel,line width=0.800pt]
  \end{scope}
  \begin{scope}[cm={{1.00588,0.0,0.0,1.00588,(39.2294,93.5471)}},draw=blue,line cap=rect,line join=bevel,line width=0.800pt]
  \end{scope}
  \begin{scope}[cm={{1.00588,0.0,0.0,1.00588,(39.2294,93.5471)}},draw=blue,line cap=rect,line join=bevel,line width=0.800pt]
  \end{scope}
  \begin{scope}[cm={{1.00588,0.0,0.0,1.00588,(39.2294,93.5471)}},draw=blue,line cap=rect,line join=bevel,line width=0.800pt]
  \end{scope}
  \begin{scope}[cm={{1.00588,0.0,0.0,1.00588,(39.2294,93.5471)}},draw=blue,line cap=rect,line join=bevel,line width=0.800pt]
  \end{scope}
  \begin{scope}[scale=1.006,draw=blue,line cap=rect,line join=bevel,line width=0.800pt]
  \end{scope}
  \begin{scope}[scale=1.006,draw=blue,line cap=rect,line join=bevel,line width=0.800pt]
  \end{scope}
  \begin{scope}[cm={{1.00588,0.0,0.0,1.00588,(39.2294,68.4)}},draw=blue,line cap=rect,line join=bevel,line width=0.800pt]
  \end{scope}
  \begin{scope}[cm={{1.00588,0.0,0.0,1.00588,(39.2294,68.4)}},draw=blue,line cap=rect,line join=bevel,line width=0.800pt]
  \end{scope}
  \begin{scope}[cm={{1.00588,0.0,0.0,1.00588,(39.2294,68.4)}},draw=blue,line cap=rect,line join=bevel,line width=0.800pt]
  \end{scope}
  \begin{scope}[cm={{1.00588,0.0,0.0,1.00588,(39.2294,68.4)}},draw=blue,line cap=rect,line join=bevel,line width=0.800pt]
  \end{scope}
  \begin{scope}[cm={{1.00588,0.0,0.0,1.00588,(39.2294,68.4)}},draw=blue,line cap=rect,line join=bevel,line width=0.800pt]
  \end{scope}
  \begin{scope}[cm={{1.00588,0.0,0.0,1.00588,(39.2294,68.4)}},draw=blue,line cap=rect,line join=bevel,line width=0.800pt]
  \end{scope}
  \begin{scope}[scale=1.006,draw=blue,line cap=rect,line join=bevel,line width=0.800pt]
  \end{scope}
  \begin{scope}[scale=1.006,draw=blue,line cap=rect,line join=bevel,line width=0.800pt]
  \end{scope}
  \begin{scope}[cm={{1.00588,0.0,0.0,1.00588,(39.2294,43.2529)}},draw=blue,line cap=rect,line join=bevel,line width=0.800pt]
  \end{scope}
  \begin{scope}[cm={{1.00588,0.0,0.0,1.00588,(39.2294,43.2529)}},draw=blue,line cap=rect,line join=bevel,line width=0.800pt]
  \end{scope}
  \begin{scope}[cm={{1.00588,0.0,0.0,1.00588,(39.2294,43.2529)}},draw=blue,line cap=rect,line join=bevel,line width=0.800pt]
  \end{scope}
  \begin{scope}[cm={{1.00588,0.0,0.0,1.00588,(39.2294,43.2529)}},draw=blue,line cap=rect,line join=bevel,line width=0.800pt]
  \end{scope}
  \begin{scope}[cm={{1.00588,0.0,0.0,1.00588,(39.2294,43.2529)}},draw=blue,line cap=rect,line join=bevel,line width=0.800pt]
  \end{scope}
  \begin{scope}[cm={{1.00588,0.0,0.0,1.00588,(39.2294,43.2529)}},draw=blue,line cap=rect,line join=bevel,line width=0.800pt]
  \end{scope}
  \begin{scope}[scale=1.006,draw=blue,line cap=rect,line join=bevel,line width=0.800pt]
  \end{scope}
  \begin{scope}[scale=1.006,draw=blue,line cap=rect,line join=bevel,line width=0.800pt]
  \end{scope}
  \begin{scope}[cm={{1.00588,0.0,0.0,1.00588,(53.3118,110.647)}},draw=blue,line cap=rect,line join=bevel,line width=0.800pt]
  \end{scope}
  \begin{scope}[cm={{1.00588,0.0,0.0,1.00588,(53.3118,110.647)}},draw=blue,line cap=rect,line join=bevel,line width=0.800pt]
  \end{scope}
  \begin{scope}[cm={{1.00588,0.0,0.0,1.00588,(53.3118,110.647)}},draw=blue,line cap=rect,line join=bevel,line width=0.800pt]
  \end{scope}
  \begin{scope}[cm={{1.00588,0.0,0.0,1.00588,(53.3118,110.647)}},draw=blue,line cap=rect,line join=bevel,line width=0.800pt]
  \end{scope}
  \begin{scope}[cm={{1.00588,0.0,0.0,1.00588,(53.3118,110.647)}},draw=blue,line cap=rect,line join=bevel,line width=0.800pt]
  \end{scope}
  \begin{scope}[cm={{1.00588,0.0,0.0,1.00588,(53.3118,110.647)}},draw=blue,line cap=rect,line join=bevel,line width=0.800pt]
  \end{scope}
  \begin{scope}[scale=1.006,draw=blue,line cap=rect,line join=bevel,line width=0.800pt]
  \end{scope}
  \begin{scope}[scale=1.006,draw=blue,line cap=rect,line join=bevel,line width=0.800pt]
  \end{scope}
  \begin{scope}[cm={{1.00588,0.0,0.0,1.00588,(79.4647,110.647)}},draw=blue,line cap=rect,line join=bevel,line width=0.800pt]
  \end{scope}
  \begin{scope}[cm={{1.00588,0.0,0.0,1.00588,(79.4647,110.647)}},draw=blue,line cap=rect,line join=bevel,line width=0.800pt]
  \end{scope}
  \begin{scope}[cm={{1.00588,0.0,0.0,1.00588,(79.4647,110.647)}},draw=blue,line cap=rect,line join=bevel,line width=0.800pt]
  \end{scope}
  \begin{scope}[cm={{1.00588,0.0,0.0,1.00588,(79.4647,110.647)}},draw=blue,line cap=rect,line join=bevel,line width=0.800pt]
  \end{scope}
  \begin{scope}[cm={{1.00588,0.0,0.0,1.00588,(79.4647,110.647)}},draw=blue,line cap=rect,line join=bevel,line width=0.800pt]
  \end{scope}
  \begin{scope}[cm={{1.00588,0.0,0.0,1.00588,(79.4647,110.647)}},draw=blue,line cap=rect,line join=bevel,line width=0.800pt]
  \end{scope}
  \begin{scope}[scale=1.006,draw=blue,line cap=rect,line join=bevel,line width=0.800pt]
  \end{scope}
  \begin{scope}[scale=1.006,draw=blue,line cap=rect,line join=bevel,line width=0.800pt]
  \end{scope}
  \begin{scope}[cm={{1.00588,0.0,0.0,1.00588,(105.618,110.647)}},draw=blue,line cap=rect,line join=bevel,line width=0.800pt]
  \end{scope}
  \begin{scope}[cm={{1.00588,0.0,0.0,1.00588,(105.618,110.647)}},draw=blue,line cap=rect,line join=bevel,line width=0.800pt]
  \end{scope}
  \begin{scope}[cm={{1.00588,0.0,0.0,1.00588,(105.618,110.647)}},draw=blue,line cap=rect,line join=bevel,line width=0.800pt]
  \end{scope}
  \begin{scope}[cm={{1.00588,0.0,0.0,1.00588,(105.618,110.647)}},draw=blue,line cap=rect,line join=bevel,line width=0.800pt]
  \end{scope}
  \begin{scope}[cm={{1.00588,0.0,0.0,1.00588,(105.618,110.647)}},draw=blue,line cap=rect,line join=bevel,line width=0.800pt]
  \end{scope}
  \begin{scope}[cm={{1.00588,0.0,0.0,1.00588,(105.618,110.647)}},draw=blue,line cap=rect,line join=bevel,line width=0.800pt]
  \end{scope}
  \begin{scope}[scale=1.006,draw=blue,line cap=rect,line join=bevel,line width=0.800pt]
  \end{scope}
  \begin{scope}[scale=1.006,draw=blue,line cap=rect,line join=bevel,line width=0.800pt]
  \end{scope}
  \begin{scope}[cm={{1.00588,0.0,0.0,1.00588,(132.274,110.647)}},draw=blue,line cap=rect,line join=bevel,line width=0.800pt]
  \end{scope}
  \begin{scope}[cm={{1.00588,0.0,0.0,1.00588,(132.274,110.647)}},draw=blue,line cap=rect,line join=bevel,line width=0.800pt]
  \end{scope}
  \begin{scope}[cm={{1.00588,0.0,0.0,1.00588,(132.274,110.647)}},draw=blue,line cap=rect,line join=bevel,line width=0.800pt]
  \end{scope}
  \begin{scope}[cm={{1.00588,0.0,0.0,1.00588,(132.274,110.647)}},draw=blue,line cap=rect,line join=bevel,line width=0.800pt]
  \end{scope}
  \begin{scope}[cm={{1.00588,0.0,0.0,1.00588,(132.274,110.647)}},draw=blue,line cap=rect,line join=bevel,line width=0.800pt]
  \end{scope}
  \begin{scope}[cm={{1.00588,0.0,0.0,1.00588,(132.274,110.647)}},draw=blue,line cap=rect,line join=bevel,line width=0.800pt]
  \end{scope}
  \begin{scope}[scale=1.006,draw=blue,line cap=rect,line join=bevel,line width=0.800pt]
  \end{scope}
  \begin{scope}[scale=1.006,draw=blue,line cap=rect,line join=bevel,line width=0.800pt]
  \end{scope}
  \begin{scope}[scale=1.006,draw=blue,line cap=rect,line join=bevel,line width=0.800pt]
  \end{scope}
  \begin{scope}[scale=1.006,draw=blue,line cap=rect,line join=bevel,line width=0.800pt]
  \end{scope}
  \begin{scope}[scale=1.006,draw=blue,line cap=rect,line join=bevel,line width=0.800pt]
  \end{scope}
  \begin{scope}[scale=1.006,draw=blue,line cap=rect,line join=bevel,line width=0.800pt]
  \end{scope}
  \begin{scope}[cm={{1.00588,0.0,0.0,1.00588,(128.753,29.1706)}},draw=blue,line cap=rect,line join=bevel,line width=0.800pt]
  \end{scope}
  \begin{scope}[cm={{1.00588,0.0,0.0,1.00588,(128.753,29.1706)}},draw=blue,line cap=rect,line join=bevel,line width=0.800pt]
  \end{scope}
  \begin{scope}[cm={{1.00588,0.0,0.0,1.00588,(128.753,29.1706)}},draw=blue,line cap=rect,line join=bevel,line width=0.800pt]
  \end{scope}
  \begin{scope}[cm={{1.00588,0.0,0.0,1.00588,(128.753,29.1706)}},draw=blue,line cap=rect,line join=bevel,line width=0.800pt]
  \end{scope}
  \begin{scope}[cm={{1.00588,0.0,0.0,1.00588,(128.753,29.1706)}},draw=blue,line cap=rect,line join=bevel,line width=0.800pt]
  \end{scope}
  \begin{scope}[cm={{1.00588,0.0,0.0,1.00588,(128.753,29.1706)}},draw=blue,line cap=rect,line join=bevel,line width=0.800pt]
  \end{scope}
  \begin{scope}[cm={{0.0,-1.00588,1.00588,0.0,(29.1706,189.106)}},draw=blue,line cap=rect,line join=bevel,line width=0.800pt]
  \end{scope}
  \begin{scope}[cm={{0.0,-1.00588,1.00588,0.0,(29.1706,189.106)}},draw=blue,line cap=rect,line join=bevel,line width=0.800pt]
  \end{scope}
  \begin{scope}[cm={{0.0,-1.00588,1.00588,0.0,(29.1706,189.106)}},draw=blue,line cap=rect,line join=bevel,line width=0.800pt]
  \end{scope}
  \begin{scope}[cm={{0.0,-1.00588,1.00588,0.0,(29.1706,189.106)}},draw=blue,line cap=rect,line join=bevel,line width=0.800pt]
  \end{scope}
  \begin{scope}[cm={{0.0,-1.00588,1.00588,0.0,(29.1706,189.106)}},draw=blue,line cap=rect,line join=bevel,line width=0.800pt]
  \end{scope}
  \begin{scope}[cm={{0.0,-1.00588,1.00588,0.0,(281.91864,312.23446)}},draw=blue,line cap=rect,line join=bevel,line width=0.800pt]
    \path[fill=blue] (32.7896,-587.5454) node[above right] (text274) {\rotatebox{90}{Power (W)}};



  \end{scope}
  \begin{scope}[cm={{0.0,-1.00588,1.00588,0.0,(29.1706,189.106)}},draw=blue,line cap=rect,line join=bevel,line width=0.800pt]
  \end{scope}
  \begin{scope}[cm={{1.00588,0.0,0.0,1.00588,(62.3647,28.1647)}},draw=blue,line cap=rect,line join=bevel,line width=0.800pt]
  \end{scope}
  \begin{scope}[cm={{1.00588,0.0,0.0,1.00588,(62.3647,28.1647)}},draw=blue,line cap=rect,line join=bevel,line width=0.800pt]
  \end{scope}
  \begin{scope}[cm={{1.00588,0.0,0.0,1.00588,(62.3647,28.1647)}},draw=blue,line cap=rect,line join=bevel,line width=0.800pt]
  \end{scope}
  \begin{scope}[cm={{1.00588,0.0,0.0,1.00588,(62.3647,28.1647)}},draw=blue,line cap=rect,line join=bevel,line width=0.800pt]
  \end{scope}
  \begin{scope}[cm={{1.00588,0.0,0.0,1.00588,(62.3647,28.1647)}},draw=blue,line cap=rect,line join=bevel,line width=0.800pt]
  \end{scope}
  \begin{scope}[cm={{1.00588,0.0,0.0,1.00588,(62.3647,28.1647)}},draw=blue,line cap=rect,line join=bevel,line width=0.800pt]
  \end{scope}
  \begin{scope}[scale=1.006,draw=blue,line cap=rect,line join=bevel,line width=0.800pt]
  \end{scope}
  \begin{scope}[scale=1.006,draw=blue,line cap=rect,line join=bevel,line width=0.800pt]
  \end{scope}
  \begin{scope}[scale=1.006,draw=blue,line cap=rect,line join=bevel,line width=0.800pt]
  \end{scope}
  \begin{scope}[scale=1.006,draw=blue,line cap=rect,line join=bevel,line width=0.800pt]
  \end{scope}
  \begin{scope}[scale=1.006,draw=blue,line cap=rect,line join=bevel,line width=0.800pt]
  \end{scope}
  \begin{scope}[scale=1.006,draw=blue,line cap=rect,line join=bevel,line width=0.800pt]
  \end{scope}
  \begin{scope}[cm={{1.00588,0.0,0.0,1.00588,(60.3529,36.2118)}},draw=blue,line cap=rect,line join=bevel,line width=0.800pt]
  \end{scope}
  \begin{scope}[cm={{1.00588,0.0,0.0,1.00588,(60.3529,36.2118)}},draw=blue,line cap=rect,line join=bevel,line width=0.800pt]
  \end{scope}
  \begin{scope}[cm={{1.00588,0.0,0.0,1.00588,(60.3529,36.2118)}},draw=blue,line cap=rect,line join=bevel,line width=0.800pt]
  \end{scope}
  \begin{scope}[cm={{1.00588,0.0,0.0,1.00588,(60.3529,36.2118)}},draw=blue,line cap=rect,line join=bevel,line width=0.800pt]
  \end{scope}
  \begin{scope}[cm={{1.00588,0.0,0.0,1.00588,(60.3529,36.2118)}},draw=blue,line cap=rect,line join=bevel,line width=0.800pt]
  \end{scope}
  \begin{scope}[cm={{1.00588,0.0,0.0,1.00588,(60.3529,36.2118)}},draw=blue,line cap=rect,line join=bevel,line width=0.800pt]
  \end{scope}
  \begin{scope}[scale=1.006,draw=blue,line cap=rect,line join=bevel,line width=0.800pt]
  \end{scope}
  \begin{scope}[scale=1.006,draw=blue,line cap=rect,line join=bevel,line width=0.800pt]
  \end{scope}
  \begin{scope}[scale=1.006,draw=blue,line cap=rect,line join=bevel,line width=0.800pt]
  \end{scope}
  \begin{scope}[scale=1.006,draw=blue,line cap=rect,line join=bevel,line width=0.800pt]
  \end{scope}
  \begin{scope}[scale=1.006,draw=blue,line cap=rect,line join=bevel,line width=0.800pt]
  \end{scope}
  \begin{scope}[scale=1.006,draw=blue,line cap=rect,line join=bevel,line width=0.800pt]
  \end{scope}
  \begin{scope}[scale=1.006,draw=blue,line cap=rect,line join=bevel,line width=0.800pt]
  \end{scope}
  \begin{scope}[scale=1.006,draw=blue,line cap=rect,line join=bevel,line width=0.800pt]
  \end{scope}
  \begin{scope}[cm={{1.00588,0.0,0.0,1.00588,(148.871,93.5471)}},draw=blue,line cap=rect,line join=bevel,line width=0.800pt]
  \end{scope}
  \begin{scope}[cm={{1.00588,0.0,0.0,1.00588,(148.871,93.5471)}},draw=blue,line cap=rect,line join=bevel,line width=0.800pt]
  \end{scope}
  \begin{scope}[cm={{1.00588,0.0,0.0,1.00588,(148.871,93.5471)}},draw=blue,line cap=rect,line join=bevel,line width=0.800pt]
  \end{scope}
  \begin{scope}[cm={{1.00588,0.0,0.0,1.00588,(148.871,93.5471)}},draw=blue,line cap=rect,line join=bevel,line width=0.800pt]
  \end{scope}
  \begin{scope}[cm={{1.00588,0.0,0.0,1.00588,(148.871,93.5471)}},draw=blue,line cap=rect,line join=bevel,line width=0.800pt]
  \end{scope}
  \begin{scope}[cm={{1.00588,0.0,0.0,1.00588,(148.871,93.5471)}},draw=blue,line cap=rect,line join=bevel,line width=0.800pt]
  \end{scope}
  \begin{scope}[scale=1.006,draw=blue,line cap=rect,line join=bevel,line width=0.800pt]
  \end{scope}
  \begin{scope}[scale=1.006,draw=blue,line cap=rect,line join=bevel,line width=0.800pt]
  \end{scope}
  \begin{scope}[cm={{1.00588,0.0,0.0,1.00588,(149.876,68.4)}},draw=blue,line cap=rect,line join=bevel,line width=0.800pt]
  \end{scope}
  \begin{scope}[cm={{1.00588,0.0,0.0,1.00588,(149.876,68.4)}},draw=blue,line cap=rect,line join=bevel,line width=0.800pt]
  \end{scope}
  \begin{scope}[cm={{1.00588,0.0,0.0,1.00588,(149.876,68.4)}},draw=blue,line cap=rect,line join=bevel,line width=0.800pt]
  \end{scope}
  \begin{scope}[cm={{1.00588,0.0,0.0,1.00588,(149.876,68.4)}},draw=blue,line cap=rect,line join=bevel,line width=0.800pt]
  \end{scope}
  \begin{scope}[cm={{1.00588,0.0,0.0,1.00588,(149.876,68.4)}},draw=blue,line cap=rect,line join=bevel,line width=0.800pt]
  \end{scope}
  \begin{scope}[cm={{1.00588,0.0,0.0,1.00588,(149.876,68.4)}},draw=blue,line cap=rect,line join=bevel,line width=0.800pt]
  \end{scope}
  \begin{scope}[scale=1.006,draw=blue,line cap=rect,line join=bevel,line width=0.800pt]
  \end{scope}
  \begin{scope}[scale=1.006,draw=blue,line cap=rect,line join=bevel,line width=0.800pt]
  \end{scope}
  \begin{scope}[cm={{1.00588,0.0,0.0,1.00588,(149.876,43.2529)}},draw=blue,line cap=rect,line join=bevel,line width=0.800pt]
  \end{scope}
  \begin{scope}[cm={{1.00588,0.0,0.0,1.00588,(149.876,43.2529)}},draw=blue,line cap=rect,line join=bevel,line width=0.800pt]
  \end{scope}
  \begin{scope}[cm={{1.00588,0.0,0.0,1.00588,(149.876,43.2529)}},draw=blue,line cap=rect,line join=bevel,line width=0.800pt]
  \end{scope}
  \begin{scope}[cm={{1.00588,0.0,0.0,1.00588,(149.876,43.2529)}},draw=blue,line cap=rect,line join=bevel,line width=0.800pt]
  \end{scope}
  \begin{scope}[cm={{1.00588,0.0,0.0,1.00588,(149.876,43.2529)}},draw=blue,line cap=rect,line join=bevel,line width=0.800pt]
  \end{scope}
  \begin{scope}[cm={{1.00588,0.0,0.0,1.00588,(149.876,43.2529)}},draw=blue,line cap=rect,line join=bevel,line width=0.800pt]
  \end{scope}
  \begin{scope}[scale=1.006,draw=blue,line cap=rect,line join=bevel,line width=0.800pt]
  \end{scope}
  \begin{scope}[scale=1.006,draw=blue,line cap=rect,line join=bevel,line width=0.800pt]
  \end{scope}
  \begin{scope}[cm={{1.00588,0.0,0.0,1.00588,(162.953,110.647)}},draw=blue,line cap=rect,line join=bevel,line width=0.800pt]
  \end{scope}
  \begin{scope}[cm={{1.00588,0.0,0.0,1.00588,(162.953,110.647)}},draw=blue,line cap=rect,line join=bevel,line width=0.800pt]
  \end{scope}
  \begin{scope}[cm={{1.00588,0.0,0.0,1.00588,(162.953,110.647)}},draw=blue,line cap=rect,line join=bevel,line width=0.800pt]
  \end{scope}
  \begin{scope}[cm={{1.00588,0.0,0.0,1.00588,(162.953,110.647)}},draw=blue,line cap=rect,line join=bevel,line width=0.800pt]
  \end{scope}
  \begin{scope}[cm={{1.00588,0.0,0.0,1.00588,(162.953,110.647)}},draw=blue,line cap=rect,line join=bevel,line width=0.800pt]
  \end{scope}
  \begin{scope}[cm={{1.00588,0.0,0.0,1.00588,(162.953,110.647)}},draw=blue,line cap=rect,line join=bevel,line width=0.800pt]
  \end{scope}
  \begin{scope}[scale=1.006,draw=blue,line cap=rect,line join=bevel,line width=0.800pt]
  \end{scope}
  \begin{scope}[scale=1.006,draw=blue,line cap=rect,line join=bevel,line width=0.800pt]
  \end{scope}
  \begin{scope}[cm={{1.00588,0.0,0.0,1.00588,(189.106,110.647)}},draw=blue,line cap=rect,line join=bevel,line width=0.800pt]
  \end{scope}
  \begin{scope}[cm={{1.00588,0.0,0.0,1.00588,(189.106,110.647)}},draw=blue,line cap=rect,line join=bevel,line width=0.800pt]
  \end{scope}
  \begin{scope}[cm={{1.00588,0.0,0.0,1.00588,(189.106,110.647)}},draw=blue,line cap=rect,line join=bevel,line width=0.800pt]
  \end{scope}
  \begin{scope}[cm={{1.00588,0.0,0.0,1.00588,(189.106,110.647)}},draw=blue,line cap=rect,line join=bevel,line width=0.800pt]
  \end{scope}
  \begin{scope}[cm={{1.00588,0.0,0.0,1.00588,(189.106,110.647)}},draw=blue,line cap=rect,line join=bevel,line width=0.800pt]
  \end{scope}
  \begin{scope}[cm={{1.00588,0.0,0.0,1.00588,(189.106,110.647)}},draw=blue,line cap=rect,line join=bevel,line width=0.800pt]
  \end{scope}
  \begin{scope}[scale=1.006,draw=blue,line cap=rect,line join=bevel,line width=0.800pt]
  \end{scope}
  \begin{scope}[scale=1.006,draw=blue,line cap=rect,line join=bevel,line width=0.800pt]
  \end{scope}
  \begin{scope}[cm={{1.00588,0.0,0.0,1.00588,(215.259,110.647)}},draw=blue,line cap=rect,line join=bevel,line width=0.800pt]
  \end{scope}
  \begin{scope}[cm={{1.00588,0.0,0.0,1.00588,(215.259,110.647)}},draw=blue,line cap=rect,line join=bevel,line width=0.800pt]
  \end{scope}
  \begin{scope}[cm={{1.00588,0.0,0.0,1.00588,(215.259,110.647)}},draw=blue,line cap=rect,line join=bevel,line width=0.800pt]
  \end{scope}
  \begin{scope}[cm={{1.00588,0.0,0.0,1.00588,(215.259,110.647)}},draw=blue,line cap=rect,line join=bevel,line width=0.800pt]
  \end{scope}
  \begin{scope}[cm={{1.00588,0.0,0.0,1.00588,(215.259,110.647)}},draw=blue,line cap=rect,line join=bevel,line width=0.800pt]
  \end{scope}
  \begin{scope}[cm={{1.00588,0.0,0.0,1.00588,(215.259,110.647)}},draw=blue,line cap=rect,line join=bevel,line width=0.800pt]
  \end{scope}
  \begin{scope}[scale=1.006,draw=blue,line cap=rect,line join=bevel,line width=0.800pt]
  \end{scope}
  \begin{scope}[scale=1.006,draw=blue,line cap=rect,line join=bevel,line width=0.800pt]
  \end{scope}
  \begin{scope}[cm={{1.00588,0.0,0.0,1.00588,(241.915,110.647)}},draw=blue,line cap=rect,line join=bevel,line width=0.800pt]
  \end{scope}
  \begin{scope}[cm={{1.00588,0.0,0.0,1.00588,(241.915,110.647)}},draw=blue,line cap=rect,line join=bevel,line width=0.800pt]
  \end{scope}
  \begin{scope}[cm={{1.00588,0.0,0.0,1.00588,(241.915,110.647)}},draw=blue,line cap=rect,line join=bevel,line width=0.800pt]
  \end{scope}
  \begin{scope}[cm={{1.00588,0.0,0.0,1.00588,(241.915,110.647)}},draw=blue,line cap=rect,line join=bevel,line width=0.800pt]
  \end{scope}
  \begin{scope}[cm={{1.00588,0.0,0.0,1.00588,(241.915,110.647)}},draw=blue,line cap=rect,line join=bevel,line width=0.800pt]
  \end{scope}
  \begin{scope}[cm={{1.00588,0.0,0.0,1.00588,(241.915,110.647)}},draw=blue,line cap=rect,line join=bevel,line width=0.800pt]
  \end{scope}
  \begin{scope}[scale=1.006,draw=blue,line cap=rect,line join=bevel,line width=0.800pt]
  \end{scope}
  \begin{scope}[scale=1.006,draw=blue,line cap=rect,line join=bevel,line width=0.800pt]
  \end{scope}
  \begin{scope}[scale=1.006,draw=blue,line cap=rect,line join=bevel,line width=0.800pt]
  \end{scope}
  \begin{scope}[scale=1.006,draw=blue,line cap=rect,line join=bevel,line width=0.800pt]
  \end{scope}
  \begin{scope}[scale=1.006,draw=blue,line cap=rect,line join=bevel,line width=0.800pt]
  \end{scope}
  \begin{scope}[scale=1.006,draw=blue,line cap=rect,line join=bevel,line width=0.800pt]
  \end{scope}
  \begin{scope}[cm={{1.00588,0.0,0.0,1.00588,(235.376,29.1706)}},draw=blue,line cap=rect,line join=bevel,line width=0.800pt]
  \end{scope}
  \begin{scope}[cm={{1.00588,0.0,0.0,1.00588,(235.376,29.1706)}},draw=blue,line cap=rect,line join=bevel,line width=0.800pt]
  \end{scope}
  \begin{scope}[cm={{1.00588,0.0,0.0,1.00588,(235.376,29.1706)}},draw=blue,line cap=rect,line join=bevel,line width=0.800pt]
  \end{scope}
  \begin{scope}[cm={{1.00588,0.0,0.0,1.00588,(235.376,29.1706)}},draw=blue,line cap=rect,line join=bevel,line width=0.800pt]
  \end{scope}
  \begin{scope}[cm={{1.00588,0.0,0.0,1.00588,(235.376,29.1706)}},draw=blue,line cap=rect,line join=bevel,line width=0.800pt]
  \end{scope}
  \begin{scope}[cm={{1.00588,0.0,0.0,1.00588,(235.376,29.1706)}},draw=blue,line cap=rect,line join=bevel,line width=0.800pt]
  \end{scope}
  \begin{scope}[scale=1.006,draw=blue,line cap=rect,line join=bevel,line width=0.800pt]
  \end{scope}
  \begin{scope}[scale=1.006,draw=blue,line cap=rect,line join=bevel,line width=0.800pt]
  \end{scope}
  \begin{scope}[scale=1.006,draw=blue,line cap=rect,line join=bevel,line width=0.800pt]
  \end{scope}
  \begin{scope}[scale=1.006,draw=blue,line cap=rect,line join=bevel,line width=0.800pt]
  \end{scope}
  \begin{scope}[scale=1.006,draw=blue,line cap=rect,line join=bevel,line width=0.800pt]
  \end{scope}
  \begin{scope}[scale=1.006,draw=blue,line cap=rect,line join=bevel,line width=0.800pt]
  \end{scope}
  \begin{scope}[scale=1.006,draw=blue,line cap=rect,line join=bevel,line width=0.800pt]
  \end{scope}
  \begin{scope}[scale=1.006,draw=blue,line cap=rect,line join=bevel,line width=0.800pt]
  \end{scope}
  \begin{scope}[scale=1.006,draw=blue,line cap=rect,line join=bevel,line width=0.800pt]
  \end{scope}
  \begin{scope}[cm={{1.04177,0.0,0.0,1.04177,(-342.44493,31.82443)}},draw=ca0a0a4,dash pattern=on 1.73pt off 1.73pt,line cap=round,line join=round,line width=0.288pt,miter limit=4.00]
    \path[draw,dash pattern=on 1.73pt off 1.73pt,line width=0.288pt,miter limit=4.00] (56.5000,194.5000) -- (251.5000,194.5000);



  \end{scope}
  \begin{scope}[cm={{1.04177,0.0,0.0,1.04177,(-342.44493,31.82443)}},draw=blue,line cap=round,line join=round,line width=0.480pt]
    \path[cm={{1.1115,0.0,0.0,1.0,(-6.26603,0.0)}},draw] (56.5000,194.5000) -- (59.5000,194.5000);



    \path[cm={{1.1115,0.0,0.0,1.0,(-28.19312,0.0)}},draw] (251.5000,194.5000) -- (248.5000,194.5000);



  \end{scope}
  \begin{scope}[scale=1.006,draw=blue,line cap=rect,line join=bevel,line width=0.800pt]
  \end{scope}
  \begin{scope}[cm={{1.00588,0.0,0.0,1.00588,(39.2294,199.165)}},draw=blue,line cap=rect,line join=bevel,line width=0.800pt]
  \end{scope}
  \begin{scope}[cm={{1.00588,0.0,0.0,1.00588,(39.2294,199.165)}},draw=blue,line cap=rect,line join=bevel,line width=0.800pt]
  \end{scope}
  \begin{scope}[cm={{1.00588,0.0,0.0,1.00588,(39.2294,199.165)}},draw=blue,line cap=rect,line join=bevel,line width=0.800pt]
  \end{scope}
  \begin{scope}[cm={{1.00588,0.0,0.0,1.00588,(39.2294,199.165)}},draw=blue,line cap=rect,line join=bevel,line width=0.800pt]
  \end{scope}
  \begin{scope}[cm={{1.00588,0.0,0.0,1.00588,(39.2294,199.165)}},draw=blue,line cap=rect,line join=bevel,line width=0.800pt]
  \end{scope}
  \begin{scope}[cm={{1.00588,0.0,0.0,1.00588,(-298.32402,235.165)}},draw=blue,line cap=rect,line join=bevel,line width=0.800pt]
    \path[fill=blue] (0.0000,0.0000) node[above right] (text658) {27};



  \end{scope}
  \begin{scope}[cm={{1.00588,0.0,0.0,1.00588,(39.2294,199.165)}},draw=blue,line cap=rect,line join=bevel,line width=0.800pt]
  \end{scope}
  \begin{scope}[scale=1.006,draw=blue,line cap=rect,line join=bevel,line width=0.800pt]
  \end{scope}
  \begin{scope}[cm={{1.04177,0.0,0.0,1.04177,(-342.44493,31.82443)}},draw=ca0a0a4,dash pattern=on 1.73pt off 1.73pt,line cap=round,line join=round,line width=0.288pt,miter limit=4.00]
    \path[draw,dash pattern=on 1.73pt off 1.73pt,line width=0.288pt,miter limit=4.00] (56.5000,171.5000) -- (251.5000,171.5000);



  \end{scope}
  \begin{scope}[cm={{1.04177,0.0,0.0,1.04177,(-342.44493,31.82443)}},draw=blue,line cap=round,line join=round,line width=0.480pt]
    \path[cm={{1.1115,0.0,0.0,1.0,(-6.26603,0.0)}},draw] (56.5000,171.5000) -- (59.5000,171.5000);



    \path[cm={{1.1115,0.0,0.0,1.0,(-28.19312,0.0)}},draw] (251.5000,171.5000) -- (248.5000,171.5000);



  \end{scope}
  \begin{scope}[scale=1.006,draw=blue,line cap=rect,line join=bevel,line width=0.800pt]
  \end{scope}
  \begin{scope}[cm={{1.00588,0.0,0.0,1.00588,(40.2353,177.035)}},draw=blue,line cap=rect,line join=bevel,line width=0.800pt]
  \end{scope}
  \begin{scope}[cm={{1.00588,0.0,0.0,1.00588,(40.2353,177.035)}},draw=blue,line cap=rect,line join=bevel,line width=0.800pt]
  \end{scope}
  \begin{scope}[cm={{1.00588,0.0,0.0,1.00588,(40.2353,177.035)}},draw=blue,line cap=rect,line join=bevel,line width=0.800pt]
  \end{scope}
  \begin{scope}[cm={{1.00588,0.0,0.0,1.00588,(40.2353,177.035)}},draw=blue,line cap=rect,line join=bevel,line width=0.800pt]
  \end{scope}
  \begin{scope}[cm={{1.00588,0.0,0.0,1.00588,(40.2353,177.035)}},draw=blue,line cap=rect,line join=bevel,line width=0.800pt]
  \end{scope}
  \begin{scope}[cm={{1.00588,0.0,0.0,1.00588,(-298.41069,213.035)}},draw=blue,line cap=rect,line join=bevel,line width=0.800pt]
    \path[fill=blue] (0.5262,0.0000) node[above right] (text688) {31};



  \end{scope}
  \begin{scope}[cm={{1.00588,0.0,0.0,1.00588,(40.2353,177.035)}},draw=blue,line cap=rect,line join=bevel,line width=0.800pt]
  \end{scope}
  \begin{scope}[scale=1.006,draw=blue,line cap=rect,line join=bevel,line width=0.800pt]
  \end{scope}
  \begin{scope}[cm={{1.04177,0.0,0.0,1.04177,(-342.44493,31.82443)}},draw=ca0a0a4,dash pattern=on 1.73pt off 1.73pt,line cap=round,line join=round,line width=0.288pt,miter limit=4.00]
    \path[draw,dash pattern=on 1.73pt off 1.73pt,line width=0.288pt,miter limit=4.00] (56.5000,149.5000) -- (251.5000,149.5000);



  \end{scope}
  \begin{scope}[cm={{1.04177,0.0,0.0,1.04177,(-342.44493,31.82443)}},draw=blue,line cap=round,line join=round,line width=0.480pt]
    \path[cm={{1.1115,0.0,0.0,1.0,(-6.26603,0.0)}},draw] (56.5000,149.5000) -- (59.5000,149.5000);



    \path[cm={{1.1115,0.0,0.0,1.0,(-28.19312,0.0)}},draw] (251.5000,149.5000) -- (248.5000,149.5000);



  \end{scope}
  \begin{scope}[scale=1.006,draw=blue,line cap=rect,line join=bevel,line width=0.800pt]
  \end{scope}
  \begin{scope}[cm={{1.00588,0.0,0.0,1.00588,(40.2353,153.9)}},draw=blue,line cap=rect,line join=bevel,line width=0.800pt]
  \end{scope}
  \begin{scope}[cm={{1.00588,0.0,0.0,1.00588,(40.2353,153.9)}},draw=blue,line cap=rect,line join=bevel,line width=0.800pt]
  \end{scope}
  \begin{scope}[cm={{1.00588,0.0,0.0,1.00588,(40.2353,153.9)}},draw=blue,line cap=rect,line join=bevel,line width=0.800pt]
  \end{scope}
  \begin{scope}[cm={{1.00588,0.0,0.0,1.00588,(40.2353,153.9)}},draw=blue,line cap=rect,line join=bevel,line width=0.800pt]
  \end{scope}
  \begin{scope}[cm={{1.00588,0.0,0.0,1.00588,(40.2353,153.9)}},draw=blue,line cap=rect,line join=bevel,line width=0.800pt]
  \end{scope}
  \begin{scope}[cm={{1.00588,0.0,0.0,1.00588,(-298.23551,191.4)}},draw=blue,line cap=rect,line join=bevel,line width=0.800pt]
    \path[fill=blue] (0.0000,0.0000) node[above right] (text718) {35};



  \end{scope}
  \begin{scope}[cm={{1.00588,0.0,0.0,1.00588,(40.2353,153.9)}},draw=blue,line cap=rect,line join=bevel,line width=0.800pt]
  \end{scope}
  \begin{scope}[scale=1.006,draw=blue,line cap=rect,line join=bevel,line width=0.800pt]
  \end{scope}
  \begin{scope}[cm={{1.04164,0.0,0.0,1.04164,(-342.30377,31.82628)}},draw=ca0a0a4,dash pattern=on 1.73pt off 1.73pt,line cap=round,line join=round,line width=0.288pt,miter limit=4.00]
    \path[draw,dash pattern=on 1.73pt off 1.73pt,line width=0.288pt,miter limit=4.00] (56.5000,126.5000) -- (195.5000,126.5000);



    \path[draw,dash pattern=on 1.73pt off 1.73pt,line width=0.288pt,miter limit=4.00] (246.5000,126.5000) -- (251.5000,126.5000);



  \end{scope}
  \begin{scope}[cm={{1.04177,0.0,0.0,1.04177,(-342.44493,31.82443)}},draw=blue,line cap=round,line join=round,line width=0.480pt]
    \path[cm={{1.1115,0.0,0.0,1.0,(-6.26603,0.0)}},draw] (56.5000,126.5000) -- (59.5000,126.5000);



    \path[cm={{1.1115,0.0,0.0,1.0,(-28.19312,0.0)}},draw] (251.5000,126.5000) -- (248.5000,126.5000);



  \end{scope}
  \begin{scope}[scale=1.006,draw=blue,line cap=rect,line join=bevel,line width=0.800pt]
  \end{scope}
  \begin{scope}[cm={{1.00588,0.0,0.0,1.00588,(39.2294,131.771)}},draw=blue,line cap=rect,line join=bevel,line width=0.800pt]
  \end{scope}
  \begin{scope}[cm={{1.00588,0.0,0.0,1.00588,(39.2294,131.771)}},draw=blue,line cap=rect,line join=bevel,line width=0.800pt]
  \end{scope}
  \begin{scope}[cm={{1.00588,0.0,0.0,1.00588,(39.2294,131.771)}},draw=blue,line cap=rect,line join=bevel,line width=0.800pt]
  \end{scope}
  \begin{scope}[cm={{1.00588,0.0,0.0,1.00588,(39.2294,131.771)}},draw=blue,line cap=rect,line join=bevel,line width=0.800pt]
  \end{scope}
  \begin{scope}[cm={{1.00588,0.0,0.0,1.00588,(39.2294,131.771)}},draw=blue,line cap=rect,line join=bevel,line width=0.800pt]
  \end{scope}
  \begin{scope}[cm={{1.00588,0.0,0.0,1.00588,(-298.4045,166.271)}},draw=blue,line cap=rect,line join=bevel,line width=0.800pt]
    \path[fill=blue] (0.0000,0.0000) node[above right] (text750) {39};



  \end{scope}
  \begin{scope}[cm={{1.00588,0.0,0.0,1.00588,(39.2294,131.771)}},draw=blue,line cap=rect,line join=bevel,line width=0.800pt]
  \end{scope}
  \begin{scope}[scale=1.006,draw=blue,line cap=rect,line join=bevel,line width=0.800pt]
  \end{scope}
  \begin{scope}[cm={{1.04177,0.0,0.0,1.04177,(-342.44493,31.82443)}},draw=ca0a0a4,dash pattern=on 0.40pt off 0.80pt,line cap=round,line join=round,line width=0.400pt]
    \path[draw] (56.5000,200.5000) -- (56.5000,115.5000);



  \end{scope}
  \begin{scope}[cm={{1.04177,0.0,0.0,1.04177,(-342.44493,31.82443)}},draw=blue,line cap=round,line join=round,line width=0.480pt]
    \path[draw] (56.5000,200.5000) -- (56.5000,198.5000);



    \path[draw] (56.5000,115.5000) -- (56.5000,118.5000);



  \end{scope}
  \begin{scope}[scale=1.006,draw=blue,line cap=rect,line join=bevel,line width=0.800pt]
  \end{scope}
  \begin{scope}[cm={{1.00588,0.0,0.0,1.00588,(53.3118,217.271)}},draw=blue,line cap=rect,line join=bevel,line width=0.800pt]
  \end{scope}
  \begin{scope}[cm={{1.00588,0.0,0.0,1.00588,(53.3118,217.271)}},draw=blue,line cap=rect,line join=bevel,line width=0.800pt]
  \end{scope}
  \begin{scope}[cm={{1.00588,0.0,0.0,1.00588,(53.3118,217.271)}},draw=blue,line cap=rect,line join=bevel,line width=0.800pt]
  \end{scope}
  \begin{scope}[cm={{1.00588,0.0,0.0,1.00588,(53.3118,217.271)}},draw=blue,line cap=rect,line join=bevel,line width=0.800pt]
  \end{scope}
  \begin{scope}[cm={{1.00588,0.0,0.0,1.00588,(53.3118,217.271)}},draw=blue,line cap=rect,line join=bevel,line width=0.800pt]
  \end{scope}
  \begin{scope}[cm={{1.00588,0.0,0.0,1.00588,(-282.68826,252.63148)}},draw=blue,line cap=rect,line join=bevel,line width=0.800pt]
    \path[fill=blue] (0.0000,0.0000) node[above right] (text780) {0};



  \end{scope}
  \begin{scope}[cm={{1.00588,0.0,0.0,1.00588,(53.3118,217.271)}},draw=blue,line cap=rect,line join=bevel,line width=0.800pt]
  \end{scope}
  \begin{scope}[scale=1.006,draw=blue,line cap=rect,line join=bevel,line width=0.800pt]
  \end{scope}
  \begin{scope}[cm={{1.04177,0.0,0.0,1.04177,(-342.44493,31.82443)}},draw=ca0a0a4,dash pattern=on 1.73pt off 1.73pt,line cap=round,line join=round,line width=0.288pt,miter limit=4.00]
    \path[draw,dash pattern=on 1.73pt off 1.73pt,line width=0.288pt,miter limit=4.00] (85.5000,200.5000) -- (85.5000,115.5000);



  \end{scope}
  \begin{scope}[cm={{1.04177,0.0,0.0,1.04177,(-342.44493,31.82443)}},draw=blue,line cap=round,line join=round,line width=0.480pt]
    \path[cm={{1.0,0.0,0.0,1.53899,(0.0,-108.26833)}},draw] (85.5000,200.5000) -- (85.5000,198.5000);



    \path[cm={{1.0,0.0,0.0,1.1115,(0.0,-12.84424)}},draw] (85.5000,115.5000) -- (85.5000,118.5000);



  \end{scope}
  \begin{scope}[scale=1.006,draw=blue,line cap=rect,line join=bevel,line width=0.800pt]
  \end{scope}
  \begin{scope}[cm={{1.00588,0.0,0.0,1.00588,(82.4824,217.271)}},draw=blue,line cap=rect,line join=bevel,line width=0.800pt]
  \end{scope}
  \begin{scope}[cm={{1.00588,0.0,0.0,1.00588,(82.4824,217.271)}},draw=blue,line cap=rect,line join=bevel,line width=0.800pt]
  \end{scope}
  \begin{scope}[cm={{1.00588,0.0,0.0,1.00588,(82.4824,217.271)}},draw=blue,line cap=rect,line join=bevel,line width=0.800pt]
  \end{scope}
  \begin{scope}[cm={{1.00588,0.0,0.0,1.00588,(82.4824,217.271)}},draw=blue,line cap=rect,line join=bevel,line width=0.800pt]
  \end{scope}
  \begin{scope}[cm={{1.00588,0.0,0.0,1.00588,(82.4824,217.271)}},draw=blue,line cap=rect,line join=bevel,line width=0.800pt]
  \end{scope}
  \begin{scope}[cm={{1.00588,0.0,0.0,1.00588,(-253.51767,252.74414)}},draw=blue,line cap=rect,line join=bevel,line width=0.800pt]
    \path[fill=blue] (0.0000,0.0000) node[above right] (text810) {1};



  \end{scope}
  \begin{scope}[cm={{1.00588,0.0,0.0,1.00588,(82.4824,217.271)}},draw=blue,line cap=rect,line join=bevel,line width=0.800pt]
  \end{scope}
  \begin{scope}[scale=1.006,draw=blue,line cap=rect,line join=bevel,line width=0.800pt]
  \end{scope}
  \begin{scope}[cm={{1.04177,0.0,0.0,1.04177,(-342.44493,31.82443)}},draw=ca0a0a4,dash pattern=on 1.73pt off 1.73pt,line cap=round,line join=round,line width=0.288pt,miter limit=4.00]
    \path[draw,dash pattern=on 1.73pt off 1.73pt,line width=0.288pt,miter limit=4.00] (114.5000,200.5000) -- (114.5000,115.5000);



  \end{scope}
  \begin{scope}[cm={{1.04177,0.0,0.0,1.04177,(-342.44493,31.82443)}},draw=blue,line cap=round,line join=round,line width=0.480pt]
    \path[cm={{1.0,0.0,0.0,1.53899,(0.0,-108.26833)}},draw] (114.5000,200.5000) -- (114.5000,198.5000);



    \path[cm={{1.0,0.0,0.0,1.1115,(0.0,-12.84424)}},draw] (114.5000,115.5000) -- (114.5000,118.5000);



  \end{scope}
  \begin{scope}[scale=1.006,draw=blue,line cap=rect,line join=bevel,line width=0.800pt]
  \end{scope}
  \begin{scope}[cm={{1.00588,0.0,0.0,1.00588,(111.653,217.271)}},draw=blue,line cap=rect,line join=bevel,line width=0.800pt]
  \end{scope}
  \begin{scope}[cm={{1.00588,0.0,0.0,1.00588,(111.653,217.271)}},draw=blue,line cap=rect,line join=bevel,line width=0.800pt]
  \end{scope}
  \begin{scope}[cm={{1.00588,0.0,0.0,1.00588,(111.653,217.271)}},draw=blue,line cap=rect,line join=bevel,line width=0.800pt]
  \end{scope}
  \begin{scope}[cm={{1.00588,0.0,0.0,1.00588,(111.653,217.271)}},draw=blue,line cap=rect,line join=bevel,line width=0.800pt]
  \end{scope}
  \begin{scope}[cm={{1.00588,0.0,0.0,1.00588,(111.653,217.271)}},draw=blue,line cap=rect,line join=bevel,line width=0.800pt]
  \end{scope}
  \begin{scope}[cm={{1.00588,0.0,0.0,1.00588,(-224.34706,252.74414)}},draw=blue,line cap=rect,line join=bevel,line width=0.800pt]
    \path[fill=blue] (0.0000,0.0000) node[above right] (text840) {2};



  \end{scope}
  \begin{scope}[cm={{1.00588,0.0,0.0,1.00588,(111.653,217.271)}},draw=blue,line cap=rect,line join=bevel,line width=0.800pt]
  \end{scope}
  \begin{scope}[scale=1.006,draw=blue,line cap=rect,line join=bevel,line width=0.800pt]
  \end{scope}
  \begin{scope}[cm={{1.04177,0.0,0.0,1.04177,(-342.44493,31.82443)}},draw=ca0a0a4,dash pattern=on 1.73pt off 1.73pt,line cap=round,line join=round,line width=0.288pt,miter limit=4.00]
    \path[draw,dash pattern=on 1.73pt off 1.73pt,line width=0.288pt,miter limit=4.00] (143.5000,200.5000) -- (143.5000,115.5000);



  \end{scope}
  \begin{scope}[cm={{1.04177,0.0,0.0,1.04177,(-342.44493,31.82443)}},draw=blue,line cap=round,line join=round,line width=0.480pt]
    \path[cm={{1.0,0.0,0.0,1.53899,(0.0,-108.26833)}},draw] (143.5000,200.5000) -- (143.5000,198.5000);



    \path[cm={{1.0,0.0,0.0,1.1115,(0.0,-12.84424)}},draw] (143.5000,115.5000) -- (143.5000,118.5000);



  \end{scope}
  \begin{scope}[scale=1.006,draw=blue,line cap=rect,line join=bevel,line width=0.800pt]
  \end{scope}
  \begin{scope}[cm={{1.00588,0.0,0.0,1.00588,(142.332,217.271)}},draw=blue,line cap=rect,line join=bevel,line width=0.800pt]
  \end{scope}
  \begin{scope}[cm={{1.00588,0.0,0.0,1.00588,(142.332,217.271)}},draw=blue,line cap=rect,line join=bevel,line width=0.800pt]
  \end{scope}
  \begin{scope}[cm={{1.00588,0.0,0.0,1.00588,(142.332,217.271)}},draw=blue,line cap=rect,line join=bevel,line width=0.800pt]
  \end{scope}
  \begin{scope}[cm={{1.00588,0.0,0.0,1.00588,(142.332,217.271)}},draw=blue,line cap=rect,line join=bevel,line width=0.800pt]
  \end{scope}
  \begin{scope}[cm={{1.00588,0.0,0.0,1.00588,(142.332,217.271)}},draw=blue,line cap=rect,line join=bevel,line width=0.800pt]
  \end{scope}
  \begin{scope}[cm={{1.00588,0.0,0.0,1.00588,(-193.66806,252.63148)}},draw=blue,line cap=rect,line join=bevel,line width=0.800pt]
    \path[fill=blue] (0.0000,0.0000) node[above right] (text870) {3};



  \end{scope}
  \begin{scope}[cm={{1.00588,0.0,0.0,1.00588,(142.332,217.271)}},draw=blue,line cap=rect,line join=bevel,line width=0.800pt]
  \end{scope}
  \begin{scope}[scale=1.006,draw=blue,line cap=rect,line join=bevel,line width=0.800pt]
  \end{scope}
  \begin{scope}[cm={{1.04177,0.0,0.0,1.04177,(-342.44493,31.82443)}},draw=ca0a0a4,dash pattern=on 1.73pt off 1.73pt,line cap=round,line join=round,line width=0.288pt,miter limit=4.00]
    \path[draw,dash pattern=on 1.73pt off 1.73pt,line width=0.288pt,miter limit=4.00] (172.5000,200.5000) -- (172.5000,115.5000);



  \end{scope}
  \begin{scope}[cm={{1.04177,0.0,0.0,1.04177,(-342.44493,31.82443)}},draw=blue,line cap=round,line join=round,line width=0.480pt]
    \path[cm={{1.0,0.0,0.0,1.53899,(0.0,-108.26833)}},draw] (172.5000,200.5000) -- (172.5000,198.5000);



    \path[cm={{1.0,0.0,0.0,1.1115,(0.0,-12.84424)}},draw] (172.5000,115.5000) -- (172.5000,118.5000);



  \end{scope}
  \begin{scope}[scale=1.006,draw=blue,line cap=rect,line join=bevel,line width=0.800pt]
  \end{scope}
  \begin{scope}[cm={{1.00588,0.0,0.0,1.00588,(171.503,217.271)}},draw=blue,line cap=rect,line join=bevel,line width=0.800pt]
  \end{scope}
  \begin{scope}[cm={{1.00588,0.0,0.0,1.00588,(171.503,217.271)}},draw=blue,line cap=rect,line join=bevel,line width=0.800pt]
  \end{scope}
  \begin{scope}[cm={{1.00588,0.0,0.0,1.00588,(171.503,217.271)}},draw=blue,line cap=rect,line join=bevel,line width=0.800pt]
  \end{scope}
  \begin{scope}[cm={{1.00588,0.0,0.0,1.00588,(171.503,217.271)}},draw=blue,line cap=rect,line join=bevel,line width=0.800pt]
  \end{scope}
  \begin{scope}[cm={{1.00588,0.0,0.0,1.00588,(171.503,217.271)}},draw=blue,line cap=rect,line join=bevel,line width=0.800pt]
  \end{scope}
  \begin{scope}[cm={{1.00588,0.0,0.0,1.00588,(-164.49705,252.74414)}},draw=blue,line cap=rect,line join=bevel,line width=0.800pt]
    \path[fill=blue] (0.0000,0.0000) node[above right] (text900) {4};



  \end{scope}
  \begin{scope}[cm={{1.00588,0.0,0.0,1.00588,(171.503,217.271)}},draw=blue,line cap=rect,line join=bevel,line width=0.800pt]
  \end{scope}
  \begin{scope}[scale=1.006,draw=blue,line cap=rect,line join=bevel,line width=0.800pt]
  \end{scope}
  \begin{scope}[cm={{1.04164,0.0,0.0,1.04164,(-342.42871,32.01781)}},draw=ca0a0a4,dash pattern=on 1.73pt off 1.73pt,line cap=round,line join=round,line width=0.288pt,miter limit=4.00]
    \path[draw,dash pattern=on 1.73pt off 1.73pt,line width=0.288pt,miter limit=4.00] (201.5000,200.5000) -- (201.5000,129.5000);



    \path[draw,dash pattern=on 1.73pt off 1.73pt,line width=0.288pt,miter limit=4.00] (201.5000,121.5000) -- (201.5000,115.5000);



  \end{scope}
  \begin{scope}[cm={{1.04177,0.0,0.0,1.04177,(-342.51116,41.97197)}},draw=ca0a0a4,dash pattern=on 1.73pt off 1.73pt,line cap=round,line join=round,line width=0.288pt,miter limit=4.00]
    \path[draw,dash pattern=on 1.73pt off 1.73pt,line width=0.288pt,miter limit=4.00] (198.5000,306.5000) -- (198.5000,221.5000);



  \end{scope}
  \begin{scope}[cm={{1.04177,0.0,0.0,1.04177,(-342.44493,31.82445)}},draw=blue,line cap=round,line join=round,line width=0.480pt]
    \path[cm={{1.0,0.0,0.0,1.53899,(0.0,-108.26833)}},draw] (201.5000,200.5000) -- (201.5000,198.5000);



    \path[cm={{1.0,0.0,0.0,1.1115,(0.0,-12.84424)}},draw] (201.5000,115.5000) -- (201.5000,118.5000);



  \end{scope}
  \begin{scope}[scale=1.006,draw=blue,line cap=rect,line join=bevel,line width=0.800pt]
  \end{scope}
  \begin{scope}[cm={{1.00588,0.0,0.0,1.00588,(200.674,217.271)}},draw=blue,line cap=rect,line join=bevel,line width=0.800pt]
  \end{scope}
  \begin{scope}[cm={{1.00588,0.0,0.0,1.00588,(200.674,217.271)}},draw=blue,line cap=rect,line join=bevel,line width=0.800pt]
  \end{scope}
  \begin{scope}[cm={{1.00588,0.0,0.0,1.00588,(200.674,217.271)}},draw=blue,line cap=rect,line join=bevel,line width=0.800pt]
  \end{scope}
  \begin{scope}[cm={{1.00588,0.0,0.0,1.00588,(200.674,217.271)}},draw=blue,line cap=rect,line join=bevel,line width=0.800pt]
  \end{scope}
  \begin{scope}[cm={{1.00588,0.0,0.0,1.00588,(200.674,217.271)}},draw=blue,line cap=rect,line join=bevel,line width=0.800pt]
  \end{scope}
  \begin{scope}[cm={{1.00588,0.0,0.0,1.00588,(-136.82605,252.63148)}},draw=blue,line cap=rect,line join=bevel,line width=0.800pt]
    \path[fill=blue] (1.4912,0.0000) node[above right] (text932) {5};



  \end{scope}
  \begin{scope}[cm={{1.00588,0.0,0.0,1.00588,(200.674,217.271)}},draw=blue,line cap=rect,line join=bevel,line width=0.800pt]
  \end{scope}
  \begin{scope}[scale=1.006,draw=blue,line cap=rect,line join=bevel,line width=0.800pt]
  \end{scope}
  \begin{scope}[cm={{1.04164,0.0,0.0,1.04164,(-342.48662,32.01785)}},draw=ca0a0a4,dash pattern=on 1.73pt off 1.73pt,line cap=round,line join=round,line width=0.288pt,miter limit=4.00]
    \path[draw,dash pattern=on 1.73pt off 1.73pt,line width=0.288pt,miter limit=4.00] (231.5000,200.5000) -- (231.5000,129.5000);



    \path[draw,dash pattern=on 1.73pt off 1.73pt,line width=0.288pt,miter limit=4.00] (231.5000,121.5000) -- (231.5000,115.5000);



  \end{scope}
  \begin{scope}[cm={{1.04177,0.0,0.0,1.04177,(-342.45387,41.97197)}},draw=ca0a0a4,dash pattern=on 1.73pt off 1.73pt,line cap=round,line join=round,line width=0.288pt,miter limit=4.00]
    \path[draw,dash pattern=on 1.73pt off 1.73pt,line width=0.288pt,miter limit=4.00] (233.5000,306.5000) -- (233.5000,221.5000);



  \end{scope}
  \begin{scope}[cm={{1.04177,0.0,0.0,1.04177,(-342.44493,31.82443)}},draw=blue,line cap=round,line join=round,line width=0.480pt]
    \path[cm={{1.0,0.0,0.0,1.53899,(0.0,-108.26833)}},draw] (231.5000,200.5000) -- (231.5000,198.5000);



    \path[cm={{1.0,0.0,0.0,1.1115,(0.0,-12.84424)}},draw] (231.5000,115.5000) -- (231.5000,118.5000);



  \end{scope}
  \begin{scope}[scale=1.006,draw=blue,line cap=rect,line join=bevel,line width=0.800pt]
  \end{scope}
  \begin{scope}[cm={{1.00588,0.0,0.0,1.00588,(229.341,217.271)}},draw=blue,line cap=rect,line join=bevel,line width=0.800pt]
  \end{scope}
  \begin{scope}[cm={{1.00588,0.0,0.0,1.00588,(229.341,217.271)}},draw=blue,line cap=rect,line join=bevel,line width=0.800pt]
  \end{scope}
  \begin{scope}[cm={{1.00588,0.0,0.0,1.00588,(229.341,217.271)}},draw=blue,line cap=rect,line join=bevel,line width=0.800pt]
  \end{scope}
  \begin{scope}[cm={{1.00588,0.0,0.0,1.00588,(229.341,217.271)}},draw=blue,line cap=rect,line join=bevel,line width=0.800pt]
  \end{scope}
  \begin{scope}[cm={{1.00588,0.0,0.0,1.00588,(229.341,217.271)}},draw=blue,line cap=rect,line join=bevel,line width=0.800pt]
  \end{scope}
  \begin{scope}[cm={{1.00588,0.0,0.0,1.00588,(-106.65906,250.271)}},draw=blue,line cap=rect,line join=bevel,line width=0.800pt]
    \path[fill=blue] (0.0000,2.3467) node[above right] (text964) {6};



  \end{scope}
  \begin{scope}[cm={{1.00588,0.0,0.0,1.00588,(229.341,217.271)}},draw=blue,line cap=rect,line join=bevel,line width=0.800pt]
  \end{scope}
  \begin{scope}[scale=1.006,draw=blue,line cap=rect,line join=bevel,line width=0.800pt]
  \end{scope}
  \begin{scope}[cm={{1.04177,0.0,0.0,1.04177,(-342.44493,31.82443)}},draw=blue,line cap=round,line join=round,line width=0.480pt]
    \path[draw] (56.5000,115.5000) -- (56.5000,200.5000) -- (251.5000,200.5000) -- (251.5000,115.5000) -- (56.5000,115.5000);



  \end{scope}
  \begin{scope}[scale=1.006,draw=blue,line cap=rect,line join=bevel,line width=0.800pt]
  \end{scope}
  \begin{scope}[scale=1.006,draw=blue,line cap=rect,line join=bevel,line width=0.800pt]
  \end{scope}
  \begin{scope}[cm={{1.04164,0.0,0.0,1.04164,(-345.64402,156.32537)}},draw=c00ff00,line cap=round,line join=round,line width=0.480pt]
    \path[draw,even odd rule] (215.5000,6.2015) -- (241.5000,6.2015);



  \end{scope}
  \begin{scope}[scale=1.006,draw=blue,line cap=rect,line join=bevel,line width=0.800pt]
  \end{scope}
  \begin{scope}[scale=1.006,draw=blue,line cap=rect,line join=bevel,line width=0.800pt]
  \end{scope}
  \begin{scope}[scale=1.006,draw=blue,line cap=rect,line join=bevel,line width=0.800pt]
  \end{scope}
  \begin{scope}[cm={{1.00588,0.0,0.0,1.00588,(236.382,132.776)}},draw=blue,line cap=rect,line join=bevel,line width=0.800pt]
  \end{scope}
  \begin{scope}[cm={{1.00588,0.0,0.0,1.00588,(236.382,132.776)}},draw=blue,line cap=rect,line join=bevel,line width=0.800pt]
  \end{scope}
  \begin{scope}[cm={{1.00588,0.0,0.0,1.00588,(236.382,132.776)}},draw=blue,line cap=rect,line join=bevel,line width=0.800pt]
  \end{scope}
  \begin{scope}[cm={{1.00588,0.0,0.0,1.00588,(236.382,132.776)}},draw=blue,line cap=rect,line join=bevel,line width=0.800pt]
  \end{scope}
  \begin{scope}[cm={{1.00588,0.0,0.0,1.00588,(236.382,132.776)}},draw=blue,line cap=rect,line join=bevel,line width=0.800pt]
  \end{scope}
  \begin{scope}[cm={{1.00588,0.0,0.0,1.00588,(236.382,132.776)}},draw=blue,line cap=rect,line join=bevel,line width=0.800pt]
  \end{scope}
  \begin{scope}[scale=1.006,draw=blue,line cap=rect,line join=bevel,line width=0.800pt]
  \end{scope}
  \begin{scope}[scale=1.006,draw=blue,line cap=rect,line join=bevel,line width=0.800pt]
  \end{scope}
  \begin{scope}[scale=1.006,draw=blue,line cap=rect,line join=bevel,line width=0.800pt]
  \end{scope}
  \begin{scope}[cm={{1.04177,0.0,0.0,1.04177,(-342.44493,31.82443)}},draw=blue,line cap=round,line join=round,line width=0.480pt]
    \path[draw] (56.1000,127.5000) -- (56.1000,127.5000) -- (56.3000,142.6000) -- (56.5000,149.0000) -- (56.7000,148.4000) -- (56.9000,141.3000) -- (57.1000,134.7000) -- (57.3000,131.3000) -- (57.5000,130.4000) -- (57.7000,131.2000) -- (57.9000,132.8000) -- (58.1000,135.0000) -- (58.3000,137.3000) -- (58.4000,139.1000) -- (58.6000,140.1000) -- (58.8000,140.4000) -- (59.0000,140.4000) -- (59.2000,140.1000) -- (59.4000,139.9000) -- (59.6000,139.7000) -- (59.8000,139.6000) -- (60.0000,139.5000) -- (60.2000,139.5000) -- (60.4000,139.5000) -- (60.6000,139.5000) -- (60.8000,139.5000) -- (61.0000,139.5000) -- (61.2000,139.5000) -- (61.4000,139.5000) -- (61.6000,139.5000) -- (61.8000,139.5000) -- (62.0000,139.5000) -- (62.2000,139.5000) -- (62.4000,139.5000) -- (62.6000,139.5000) -- (62.8000,139.6000) -- (63.0000,140.3000) -- (63.1000,141.9000) -- (63.3000,143.7000) -- (63.5000,145.6000) -- (63.7000,147.5000) -- (63.9000,149.3000) -- (64.1000,151.1000) -- (64.3000,152.8000) -- (64.5000,154.5000) -- (64.7000,156.1000) -- (64.9000,157.6000) -- (65.1000,159.1000) -- (65.3000,160.5000) -- (65.5000,161.7000) -- (65.7000,162.7000) -- (65.9000,163.6000) -- (66.1000,164.2000) -- (66.3000,164.6000) -- (66.5000,164.7000) -- (66.7000,164.5000) -- (66.9000,164.0000) -- (67.1000,163.2000) -- (67.3000,162.2000) -- (67.5000,161.0000) -- (67.7000,159.7000) -- (67.9000,158.2000) -- (68.0000,156.6000) -- (68.2000,154.9000) -- (68.4000,153.2000) -- (68.6000,151.5000) -- (68.8000,149.7000) -- (69.0000,147.3000) -- (69.2000,144.6000) -- (69.4000,142.0000) -- (69.6000,140.1000) -- (69.8000,139.0000) -- (70.0000,138.5000) -- (70.2000,138.5000) -- (70.4000,138.7000) -- (70.6000,139.0000) -- (70.8000,139.2000) -- (71.0000,139.4000) -- (71.2000,139.5000) -- (71.4000,139.5000) -- (71.6000,139.6000) -- (71.8000,139.5000) -- (72.0000,139.5000) -- (72.2000,139.5000) -- (72.4000,139.5000) -- (72.6000,139.5000) -- (72.7000,139.5000) -- (72.9000,139.5000) -- (73.1000,139.5000) -- (73.3000,139.5000) -- (73.5000,139.4000) -- (73.7000,140.8000) -- (73.9000,142.7000) -- (74.1000,142.4000) -- (74.3000,141.0000) -- (74.5000,139.2000) -- (74.7000,137.3000) -- (74.9000,135.6000) -- (75.1000,134.1000) -- (75.3000,132.8000) -- (75.5000,131.6000) -- (75.7000,130.7000) -- (75.9000,129.8000) -- (76.1000,129.1000) -- (76.3000,128.4000) -- (76.5000,127.9000) -- (76.7000,127.4000) -- (76.9000,127.0000) -- (77.1000,126.7000) -- (77.3000,126.4000) -- (77.5000,126.3000) -- (77.6000,126.2000) -- (77.8000,126.2000) -- (78.0000,126.3000) -- (78.2000,126.5000) -- (78.4000,126.7000) -- (78.6000,127.1000) -- (78.8000,127.5000) -- (79.0000,128.0000) -- (79.2000,128.6000) -- (79.4000,129.3000) -- (79.6000,130.0000) -- (79.8000,130.9000) -- (80.0000,131.9000) -- (80.2000,132.9000) -- (80.4000,134.1000) -- (80.6000,135.3000) -- (80.8000,137.0000) -- (81.0000,138.6000) -- (81.2000,139.6000) -- (81.4000,140.1000) -- (81.6000,140.2000) -- (81.8000,140.1000) -- (82.0000,140.0000) -- (82.2000,139.8000) -- (82.4000,139.6000) -- (82.5000,139.6000) -- (82.7000,139.5000) -- (82.9000,139.5000) -- (83.1000,139.5000) -- (83.3000,139.5000) -- (83.5000,139.5000) -- (83.7000,139.5000) -- (83.9000,139.5000) -- (84.1000,139.5000) -- (84.3000,139.5000) -- (84.5000,139.5000) -- (84.7000,139.5000) -- (84.9000,139.5000) -- (85.1000,139.5000) -- (85.3000,139.5000) -- (85.5000,139.7000) -- (85.7000,140.9000) -- (85.9000,142.7000) -- (86.1000,144.6000) -- (86.3000,146.6000) -- (86.5000,148.5000) -- (86.7000,150.3000) -- (86.9000,152.1000) -- (87.1000,153.8000) -- (87.2000,155.5000) -- (87.4000,157.0000) -- (87.6000,158.6000) -- (87.8000,160.0000) -- (88.0000,161.3000) -- (88.2000,162.4000) -- (88.4000,163.3000) -- (88.6000,164.1000) -- (88.8000,164.5000) -- (89.0000,164.7000) -- (89.2000,164.6000) -- (89.4000,164.1000) -- (89.6000,163.4000) -- (89.8000,162.4000) -- (90.0000,161.2000) -- (90.2000,159.8000) -- (90.4000,158.3000) -- (90.6000,156.6000) -- (90.8000,154.9000) -- (91.0000,153.2000) -- (91.2000,151.5000) -- (91.4000,149.6000) -- (91.6000,147.3000) -- (91.8000,144.8000) -- (92.0000,142.3000) -- (92.1000,140.4000) -- (92.3000,139.2000) -- (92.5000,138.7000) -- (92.7000,138.6000) -- (92.9000,138.7000) -- (93.1000,139.0000) -- (93.3000,139.2000) -- (93.5000,139.4000) -- (93.7000,139.5000) -- (93.9000,139.5000) -- (94.1000,139.5000) -- (94.3000,139.5000) -- (94.5000,139.5000) -- (94.7000,139.5000) -- (94.9000,139.5000) -- (95.1000,139.5000) -- (95.3000,139.5000) -- (95.5000,139.5000) -- (95.7000,139.5000) -- (95.9000,139.5000) -- (96.1000,139.5000) -- (96.3000,141.0000) -- (96.5000,142.6000) -- (96.7000,142.3000) -- (96.8000,140.8000) -- (97.0000,139.0000) -- (97.2000,137.1000) -- (97.4000,135.4000) -- (97.6000,133.9000) -- (97.8000,132.6000) -- (98.0000,131.4000) -- (98.2000,130.5000) -- (98.4000,129.6000) -- (98.6000,128.9000) -- (98.8000,128.3000) -- (99.0000,127.7000) -- (99.2000,127.3000) -- (99.4000,126.9000) -- (99.6000,126.6000) -- (99.8000,126.4000) -- (100.0000,126.2000) -- (100.2000,126.2000) -- (100.4000,126.2000) -- (100.6000,126.3000) -- (100.8000,126.6000) -- (101.0000,126.8000) -- (101.2000,127.3000) -- (101.4000,127.9000) -- (101.6000,128.6000) -- (101.7000,129.3000) -- (101.9000,130.1000) -- (102.1000,130.9000) -- (102.3000,131.9000) -- (102.5000,132.9000) -- (102.7000,134.0000) -- (102.9000,135.3000) -- (103.1000,137.1000) -- (103.3000,138.7000) -- (103.5000,139.8000) -- (103.7000,140.2000) -- (103.9000,140.3000) -- (104.1000,140.2000) -- (104.3000,140.0000) -- (104.5000,139.8000) -- (104.7000,139.6000) -- (104.9000,139.5000) -- (105.1000,139.5000) -- (105.3000,139.5000) -- (105.5000,139.5000) -- (105.7000,139.5000) -- (105.9000,139.5000) -- (106.1000,139.5000) -- (106.3000,139.5000) -- (106.4000,139.5000) -- (106.6000,139.5000) -- (106.8000,139.5000) -- (107.0000,139.5000) -- (107.2000,139.5000) -- (107.4000,139.5000) -- (107.6000,139.5000) -- (107.8000,139.7000) -- (108.0000,140.8000) -- (108.2000,142.5000) -- (108.4000,144.4000) -- (108.6000,146.3000) -- (108.8000,148.1000) -- (109.0000,149.8000) -- (109.2000,151.5000) -- (109.4000,153.2000) -- (109.6000,154.8000) -- (109.8000,156.4000) -- (110.0000,157.9000) -- (110.2000,159.3000) -- (110.4000,160.6000) -- (110.6000,161.7000) -- (110.8000,162.7000) -- (111.0000,163.5000) -- (111.1000,164.1000) -- (111.3000,164.5000) -- (111.5000,164.6000) -- (111.7000,164.5000) -- (111.9000,164.1000) -- (112.1000,163.5000) -- (112.3000,162.7000) -- (112.5000,161.7000) -- (112.7000,160.6000) -- (112.9000,159.3000) -- (113.1000,157.9000) -- (113.3000,156.4000) -- (113.5000,154.9000) -- (113.7000,153.3000) -- (113.9000,151.7000) -- (114.1000,149.9000) -- (114.3000,146.7000) -- (114.5000,143.0000) -- (114.7000,140.3000) -- (114.9000,144.4000) -- (115.1000,145.2000) -- (115.3000,145.1000) -- (115.5000,145.4000) -- (115.7000,145.8000) -- (115.9000,146.2000) -- (116.0000,146.4000) -- (116.2000,146.5000) -- (116.4000,146.6000) -- (116.6000,146.6000) -- (116.8000,146.6000) -- (117.0000,146.6000) -- (117.2000,146.6000) -- (117.4000,146.6000) -- (117.6000,146.5000) -- (117.8000,146.5000) -- (118.0000,146.5000) -- (118.2000,146.5000) -- (118.4000,146.5000) -- (118.6000,146.5000) -- (118.8000,146.5000) -- (119.0000,146.7000) -- (119.2000,147.4000) -- (119.4000,147.9000) -- (119.6000,147.5000) -- (119.8000,146.2000) -- (120.0000,144.3000) -- (120.2000,146.9000) -- (120.4000,148.4000) -- (120.6000,147.9000) -- (120.8000,152.5000) -- (120.9000,151.7000) -- (121.1000,150.8000) -- (121.3000,150.0000) -- (121.5000,149.4000) -- (121.7000,148.8000) -- (121.9000,148.1000) -- (122.1000,152.5000) -- (122.3000,154.9000) -- (122.5000,154.5000) -- (122.7000,154.3000) -- (122.9000,154.3000) -- (123.1000,154.3000) -- (123.3000,154.5000) -- (123.5000,154.7000) -- (123.7000,155.0000) -- (123.9000,155.4000) -- (124.1000,155.8000) -- (124.3000,156.6000) -- (124.5000,155.2000) -- (124.7000,150.6000) -- (124.9000,151.6000) -- (125.1000,152.5000) -- (125.3000,153.6000) -- (125.5000,154.7000) -- (125.7000,156.0000) -- (125.8000,157.7000) -- (126.0000,153.9000) -- (126.2000,153.5000) -- (126.4000,154.2000) -- (126.6000,154.4000) -- (126.8000,154.4000) -- (127.0000,159.8000) -- (127.2000,161.1000) -- (127.4000,160.7000) -- (127.6000,160.6000) -- (127.8000,160.6000) -- (128.0000,160.6000) -- (128.2000,160.6000) -- (128.4000,160.6000) -- (128.6000,160.6000) -- (128.8000,160.6000) -- (129.0000,160.6000) -- (129.2000,160.6000) -- (129.4000,160.6000) -- (129.6000,160.6000) -- (129.8000,160.6000) -- (130.0000,160.6000) -- (130.2000,160.6000) -- (130.4000,160.6000) -- (130.5000,160.7000) -- (130.7000,161.7000) -- (130.9000,163.4000) -- (131.1000,165.5000) -- (131.3000,163.4000) -- (131.5000,161.8000) -- (131.7000,163.0000) -- (131.9000,158.9000) -- (132.1000,160.1000) -- (132.3000,161.9000) -- (132.5000,163.6000) -- (132.7000,164.1000) -- (132.9000,159.7000) -- (133.1000,160.7000) -- (133.3000,161.9000) -- (133.5000,162.9000) -- (133.7000,163.7000) -- (133.9000,164.3000) -- (134.1000,164.6000) -- (134.3000,164.6000) -- (134.5000,164.2000) -- (134.7000,165.2000) -- (134.9000,170.0000) -- (135.1000,169.0000) -- (135.2000,167.8000) -- (135.4000,166.2000) -- (135.6000,166.4000) -- (135.8000,170.3000) -- (136.0000,168.4000) -- (136.2000,171.9000) -- (136.4000,172.4000) -- (136.6000,170.3000) -- (136.8000,167.8000) -- (137.0000,165.0000) -- (137.2000,167.9000) -- (137.4000,168.1000) -- (137.6000,166.9000) -- (137.8000,166.6000) -- (138.0000,166.7000) -- (138.2000,166.9000) -- (138.4000,167.2000) -- (138.6000,167.4000) -- (138.8000,167.5000) -- (139.0000,167.6000) -- (139.2000,167.7000) -- (139.4000,167.7000) -- (139.6000,167.7000) -- (139.8000,167.7000) -- (140.0000,167.6000) -- (140.1000,167.6000) -- (140.3000,167.6000) -- (140.5000,167.6000) -- (140.7000,167.6000) -- (140.9000,167.6000) -- (141.1000,167.6000) -- (141.3000,167.7000) -- (141.5000,169.6000) -- (141.7000,171.2000) -- (141.9000,170.6000) -- (142.1000,169.0000) -- (142.3000,167.1000) -- (142.5000,165.1000) -- (142.7000,163.3000) -- (142.9000,161.8000) -- (143.1000,160.5000) -- (143.3000,159.3000) -- (143.5000,158.3000) -- (143.7000,157.5000) -- (143.9000,156.8000) -- (144.1000,156.1000) -- (144.3000,155.6000) -- (144.5000,155.2000) -- (144.7000,154.8000) -- (144.9000,154.6000) -- (145.0000,154.4000) -- (145.2000,154.3000) -- (145.4000,154.3000) -- (145.6000,154.4000) -- (145.8000,154.6000) -- (146.0000,154.8000) -- (146.2000,155.2000) -- (146.4000,155.7000) -- (146.6000,156.2000) -- (146.8000,156.9000) -- (147.0000,157.6000) -- (147.2000,158.4000) -- (147.4000,159.4000) -- (147.6000,160.4000) -- (147.8000,161.6000) -- (148.0000,162.8000) -- (148.2000,164.5000) -- (148.4000,166.2000) -- (148.6000,167.5000) -- (148.8000,168.2000) -- (149.0000,168.4000) -- (149.2000,168.4000) -- (149.4000,168.2000) -- (149.6000,168.0000) -- (149.8000,167.8000) -- (149.9000,167.7000) -- (150.1000,167.6000) -- (150.3000,167.6000) -- (150.5000,167.6000) -- (150.7000,167.6000) -- (150.9000,167.6000) -- (151.1000,167.6000) -- (151.3000,167.6000) -- (151.5000,167.6000) -- (151.7000,167.6000) -- (151.9000,167.6000) -- (152.1000,167.6000) -- (152.3000,167.6000) -- (152.5000,167.6000) -- (152.7000,167.6000) -- (152.9000,167.7000) -- (153.1000,168.6000) -- (153.3000,170.3000) -- (153.5000,172.2000) -- (153.7000,174.1000) -- (153.9000,176.0000) -- (154.1000,177.9000) -- (154.3000,179.6000) -- (154.5000,181.4000) -- (154.6000,183.0000) -- (154.8000,184.6000) -- (155.0000,186.2000) -- (155.2000,187.6000) -- (155.4000,189.0000) -- (155.6000,190.2000) -- (155.8000,191.2000) -- (156.0000,192.0000) -- (156.2000,192.5000) -- (156.4000,192.8000) -- (156.6000,192.8000) -- (156.8000,192.4000) -- (157.0000,191.8000) -- (157.2000,190.9000) -- (157.4000,189.8000) -- (157.6000,188.5000) -- (157.8000,187.0000) -- (158.0000,185.5000) -- (158.2000,183.8000) -- (158.4000,182.1000) -- (158.6000,180.3000) -- (158.8000,178.6000) -- (159.0000,176.5000) -- (159.2000,174.0000) -- (159.3000,171.4000) -- (159.5000,169.2000) -- (159.7000,167.7000) -- (159.9000,166.9000) -- (160.1000,166.7000) -- (160.3000,166.8000) -- (160.5000,167.0000) -- (160.7000,167.2000) -- (160.9000,167.4000) -- (161.1000,167.6000) -- (161.3000,167.6000) -- (161.5000,167.7000) -- (161.7000,167.7000) -- (161.9000,167.7000) -- (162.1000,167.6000) -- (162.3000,167.6000) -- (162.5000,167.6000) -- (162.7000,167.6000) -- (162.9000,167.6000) -- (163.1000,167.6000) -- (163.3000,167.6000) -- (163.5000,167.6000) -- (163.7000,168.2000) -- (163.9000,170.6000) -- (164.1000,171.2000) -- (164.2000,170.0000) -- (164.4000,168.2000) -- (164.6000,166.2000) -- (164.8000,164.3000) -- (165.0000,162.6000) -- (165.2000,161.2000) -- (165.4000,159.9000) -- (165.6000,158.8000) -- (165.8000,157.9000) -- (166.0000,157.1000) -- (166.2000,156.5000) -- (166.4000,155.9000) -- (166.6000,155.4000) -- (166.8000,155.0000) -- (167.0000,154.7000) -- (167.2000,154.5000) -- (167.4000,154.3000) -- (167.6000,154.3000) -- (167.8000,154.3000) -- (168.0000,154.5000) -- (168.2000,154.7000) -- (168.4000,155.0000) -- (168.6000,155.5000) -- (168.8000,156.0000) -- (169.0000,156.6000) -- (169.1000,157.3000) -- (169.3000,158.1000) -- (169.5000,159.0000) -- (169.7000,160.0000) -- (169.9000,161.2000) -- (170.1000,162.4000) -- (170.3000,163.9000) -- (170.5000,165.7000) -- (170.7000,167.1000) -- (170.9000,168.0000) -- (171.1000,168.4000) -- (171.3000,168.4000) -- (171.5000,168.2000) -- (171.7000,168.0000) -- (171.9000,167.8000) -- (172.1000,167.7000) -- (172.3000,167.7000) -- (172.5000,167.6000) -- (172.7000,167.6000) -- (172.9000,167.6000) -- (173.1000,167.6000) -- (173.3000,167.6000) -- (173.5000,167.6000) -- (173.7000,167.6000) -- (173.9000,167.6000) -- (174.0000,167.6000) -- (174.2000,167.6000) -- (174.4000,167.6000) -- (174.6000,167.6000) -- (174.8000,167.6000) -- (175.0000,167.6000) -- (175.2000,168.2000) -- (175.4000,169.6000) -- (175.6000,171.5000) -- (175.8000,173.5000) -- (176.0000,175.5000) -- (176.2000,177.4000) -- (176.4000,179.2000) -- (176.6000,180.9000) -- (176.8000,182.6000) -- (177.0000,184.2000) -- (177.2000,185.8000) -- (177.4000,187.2000) -- (177.6000,188.6000) -- (177.8000,189.8000) -- (178.0000,190.9000) -- (178.2000,191.8000) -- (178.4000,192.4000) -- (178.6000,192.7000) -- (178.7000,192.8000) -- (178.9000,192.5000) -- (179.1000,192.0000) -- (179.3000,191.2000) -- (179.5000,190.1000) -- (179.7000,188.8000) -- (179.9000,187.3000) -- (180.1000,185.7000) -- (180.3000,184.1000) -- (180.5000,182.4000) -- (180.7000,180.6000) -- (180.9000,178.9000) -- (181.1000,176.9000) -- (181.3000,174.5000) -- (181.5000,171.9000) -- (181.7000,169.6000) -- (181.9000,168.0000) -- (182.1000,167.0000) -- (182.3000,166.7000) -- (182.5000,166.8000) -- (182.7000,167.0000) -- (182.9000,167.2000) -- (183.1000,167.4000) -- (183.3000,167.5000) -- (183.4000,167.6000) -- (183.6000,167.7000) -- (183.8000,167.7000) -- (184.0000,167.7000) -- (184.2000,167.7000) -- (184.4000,167.6000) -- (184.6000,167.6000) -- (184.8000,167.6000) -- (185.0000,167.6000) -- (185.2000,167.6000) -- (185.4000,167.6000) -- (185.6000,167.6000) -- (185.8000,168.0000) -- (186.0000,170.4000) -- (186.2000,171.2000) -- (186.4000,170.3000) -- (186.6000,168.5000) -- (186.8000,166.5000) -- (187.0000,164.6000) -- (187.2000,162.8000) -- (187.4000,161.2000) -- (187.6000,159.9000) -- (187.8000,158.7000) -- (188.0000,157.7000) -- (188.2000,156.9000) -- (188.3000,156.2000) -- (188.5000,155.6000) -- (188.7000,155.1000) -- (188.9000,154.7000) -- (189.1000,154.5000) -- (189.3000,154.3000) -- (189.5000,154.2000) -- (189.7000,154.3000) -- (189.9000,154.4000) -- (190.1000,154.7000) -- (190.3000,155.0000) -- (190.5000,155.5000) -- (190.7000,156.0000) -- (190.9000,156.7000) -- (191.1000,157.5000) -- (191.3000,158.4000) -- (191.5000,159.4000) -- (191.7000,160.6000) -- (191.9000,161.8000) -- (192.1000,163.3000) -- (192.3000,165.2000) -- (192.5000,166.8000) -- (192.7000,167.8000) -- (192.9000,168.3000) -- (193.1000,168.4000) -- (193.2000,168.3000) -- (193.4000,168.1000) -- (193.6000,167.9000) -- (193.8000,167.8000) -- (194.0000,167.7000) -- (194.2000,167.6000) -- (194.4000,167.6000) -- (194.6000,167.6000) -- (194.8000,167.6000) -- (195.0000,167.6000) -- (195.2000,167.6000) -- (195.4000,167.6000) -- (195.6000,167.6000) -- (195.8000,167.6000) -- (196.0000,167.6000) -- (196.2000,167.6000) -- (196.4000,167.6000) -- (196.6000,167.6000) -- (196.8000,167.6000) -- (197.0000,168.0000) -- (197.2000,169.3000) -- (197.4000,171.2000) -- (197.6000,173.2000) -- (197.8000,175.2000) -- (197.9000,177.1000) -- (198.1000,178.9000) -- (198.3000,180.7000) -- (198.5000,182.3000) -- (198.7000,184.0000) -- (198.9000,185.5000) -- (199.1000,187.0000) -- (199.3000,188.4000) -- (199.5000,189.7000) -- (199.7000,190.8000) -- (199.9000,191.7000) -- (200.1000,192.3000) -- (200.3000,192.7000) -- (200.5000,192.8000) -- (200.7000,192.6000) -- (200.9000,192.1000) -- (201.1000,191.3000) -- (201.3000,190.2000) -- (201.5000,188.9000) -- (201.7000,187.5000) -- (201.9000,185.9000) -- (202.1000,184.3000) -- (202.3000,182.5000) -- (202.5000,180.8000) -- (202.7000,179.0000) -- (202.8000,177.1000) -- (203.0000,174.7000) -- (203.2000,172.1000) -- (203.4000,169.8000) -- (203.6000,168.1000) -- (203.8000,167.1000) -- (204.0000,166.7000) -- (204.2000,166.7000) -- (204.4000,166.9000) -- (204.6000,167.2000) -- (204.8000,167.4000) -- (205.0000,167.5000) -- (205.2000,167.6000) -- (205.4000,167.7000) -- (205.6000,167.7000) -- (205.8000,167.7000) -- (206.0000,167.7000) -- (206.2000,167.6000) -- (206.4000,167.6000) -- (206.6000,167.6000) -- (206.8000,167.6000) -- (207.0000,167.6000) -- (207.2000,167.6000) -- (207.4000,167.6000) -- (207.5000,167.8000) -- (207.7000,170.3000) -- (207.9000,171.6000) -- (208.1000,170.8000) -- (208.3000,169.0000) -- (208.5000,166.9000) -- (208.7000,164.8000) -- (208.9000,162.9000) -- (209.1000,161.3000) -- (209.3000,160.0000) -- (209.5000,158.8000) -- (209.7000,157.8000) -- (209.9000,157.0000) -- (210.1000,156.3000) -- (210.3000,155.7000) -- (210.5000,155.2000) -- (210.7000,154.8000) -- (210.9000,154.5000) -- (211.1000,154.3000) -- (211.3000,154.3000) -- (211.5000,154.3000) -- (211.7000,154.4000) -- (211.9000,154.6000) -- (212.1000,154.9000) -- (212.3000,155.3000) -- (212.4000,155.9000) -- (212.6000,156.5000) -- (212.8000,157.2000) -- (213.0000,158.1000) -- (213.2000,159.0000) -- (213.4000,160.1000) -- (213.6000,161.3000) -- (213.8000,162.7000) -- (214.0000,164.4000) -- (214.2000,166.3000) -- (214.4000,167.6000) -- (214.6000,168.3000) -- (214.8000,168.5000) -- (215.0000,168.4000) -- (215.2000,168.2000) -- (215.4000,168.0000) -- (215.6000,167.8000) -- (215.8000,167.7000) -- (216.0000,167.6000) -- (216.2000,167.6000) -- (216.4000,167.6000) -- (216.6000,167.6000) -- (216.8000,167.6000) -- (217.0000,167.6000) -- (217.1000,167.6000) -- (217.3000,167.6000) -- (217.5000,167.6000) -- (217.7000,167.6000) -- (217.9000,167.6000) -- (218.1000,167.6000) -- (218.3000,167.6000) -- (218.5000,167.6000) -- (218.7000,167.7000) -- (218.9000,168.7000) -- (219.1000,170.4000) -- (219.3000,172.4000) -- (219.5000,174.3000) -- (219.7000,176.2000) -- (219.9000,178.1000) -- (220.1000,179.8000) -- (220.3000,181.6000) -- (220.5000,183.2000) -- (220.7000,184.8000) -- (220.9000,186.4000) -- (221.1000,187.8000) -- (221.3000,189.1000) -- (221.5000,190.3000) -- (221.7000,191.3000) -- (221.9000,192.0000) -- (222.0000,192.5000) -- (222.2000,192.8000) -- (222.4000,192.7000) -- (222.6000,192.4000) -- (222.8000,191.7000) -- (223.0000,190.8000) -- (223.2000,189.7000) -- (223.4000,188.3000) -- (223.6000,186.8000) -- (223.8000,185.2000) -- (224.0000,183.6000) -- (224.2000,181.8000) -- (224.4000,180.1000) -- (224.6000,178.3000) -- (224.8000,176.2000) -- (225.0000,173.7000) -- (225.2000,171.1000) -- (225.4000,169.0000) -- (225.6000,167.6000) -- (225.8000,166.9000) -- (226.0000,166.7000) -- (226.2000,166.8000) -- (226.4000,167.0000) -- (226.6000,167.3000) -- (226.8000,167.4000) -- (226.9000,167.6000) -- (227.1000,167.6000) -- (227.3000,167.7000) -- (227.5000,167.7000) -- (227.7000,167.7000) -- (227.9000,167.6000) -- (228.1000,167.6000) -- (228.3000,167.6000) -- (228.5000,167.6000) -- (228.7000,167.6000) -- (228.9000,167.6000) -- (229.1000,167.6000) -- (229.3000,167.5000) -- (229.5000,168.5000) -- (229.7000,171.2000) -- (229.9000,171.5000) -- (230.1000,170.2000) -- (230.3000,168.2000) -- (230.5000,166.1000) -- (230.7000,164.1000) -- (230.9000,162.3000) -- (231.1000,160.8000) -- (231.3000,159.5000) -- (231.4000,158.4000) -- (231.6000,157.5000) -- (231.8000,156.7000) -- (232.0000,156.1000) -- (232.2000,155.5000) -- (232.4000,155.1000) -- (232.6000,154.7000) -- (232.8000,154.5000) -- (233.0000,154.3000) -- (233.2000,154.2000) -- (233.4000,154.3000) -- (233.6000,154.4000) -- (233.8000,154.7000) -- (234.0000,155.0000) -- (234.2000,155.5000) -- (234.4000,156.1000) -- (234.6000,156.7000) -- (234.8000,157.5000) -- (235.0000,158.4000) -- (235.2000,159.4000) -- (235.4000,160.5000) -- (235.6000,161.7000) -- (235.8000,163.2000) -- (236.0000,165.1000) -- (236.2000,166.8000) -- (236.3000,167.9000) -- (236.5000,168.4000) -- (236.7000,168.5000) -- (236.9000,168.3000) -- (237.1000,168.1000) -- (237.3000,167.9000) -- (237.5000,167.8000) -- (237.7000,167.7000) -- (237.9000,167.6000) -- (238.1000,167.6000) -- (238.3000,167.6000) -- (238.5000,167.6000) -- (238.7000,167.6000) -- (238.9000,167.6000) -- (239.1000,167.6000) -- (239.3000,167.6000) -- (239.5000,167.6000) -- (239.7000,167.6000) -- (239.9000,167.6000) -- (240.1000,167.6000) -- (240.3000,167.6000) -- (240.5000,167.6000) -- (240.7000,168.0000) -- (240.9000,169.3000) -- (241.1000,171.1000) -- (241.2000,173.0000) -- (241.4000,174.9000) -- (241.6000,176.8000) -- (241.8000,178.6000) -- (242.0000,180.4000) -- (242.2000,182.1000) -- (242.4000,183.7000) -- (242.6000,185.3000) -- (242.8000,186.8000) -- (243.0000,188.2000) -- (243.2000,189.5000) -- (243.4000,190.6000) -- (243.6000,191.5000) -- (243.8000,192.2000) -- (244.0000,192.7000) -- (244.2000,192.8000) -- (244.4000,192.6000) -- (244.6000,192.2000) -- (244.8000,191.5000) -- (245.0000,190.5000) -- (245.2000,189.3000) -- (245.4000,187.9000) -- (245.6000,186.4000) -- (245.8000,184.7000) -- (246.0000,183.0000) -- (246.1000,181.3000) -- (246.3000,179.6000) -- (246.5000,177.8000) -- (246.7000,175.5000) -- (246.9000,172.9000) -- (247.1000,170.4000) -- (247.3000,168.5000) -- (247.5000,167.3000) -- (247.7000,166.8000) -- (247.9000,166.7000) -- (248.1000,166.9000) -- (248.3000,167.1000) -- (248.5000,167.3000) -- (248.7000,167.5000) -- (248.9000,167.6000) -- (249.1000,167.6000) -- (249.3000,167.7000) -- (249.5000,167.7000) -- (249.7000,167.7000) -- (249.9000,167.6000) -- (250.1000,167.6000) -- (250.3000,167.6000) -- (250.5000,167.6000) -- (250.7000,167.6000) -- (250.9000,167.6000) -- (251.0000,167.6000) -- (251.2000,167.6000) -- (251.4000,169.4000) -- (251.6000,171.5000) -- (251.8000,171.2000);



  \end{scope}
  \begin{scope}[scale=1.006,draw=blue,line cap=rect,line join=bevel,line width=0.800pt]
  \end{scope}
  \begin{scope}[cm={{1.00588,0.0,0.0,1.00588,(197.153,129.759)}},draw=blue,line cap=rect,line join=bevel,line width=0.800pt]
  \end{scope}
  \begin{scope}[cm={{1.00588,0.0,0.0,1.00588,(197.153,129.759)}},draw=blue,line cap=rect,line join=bevel,line width=0.800pt]
  \end{scope}
  \begin{scope}[cm={{1.00588,0.0,0.0,1.00588,(197.153,129.759)}},draw=blue,line cap=rect,line join=bevel,line width=0.800pt]
  \end{scope}
  \begin{scope}[cm={{1.00588,0.0,0.0,1.00588,(197.153,129.759)}},draw=blue,line cap=rect,line join=bevel,line width=0.800pt]
  \end{scope}
  \begin{scope}[cm={{1.00588,0.0,0.0,1.00588,(197.153,129.759)}},draw=blue,line cap=rect,line join=bevel,line width=0.800pt]
  \end{scope}
  \begin{scope}[cm={{1.04164,0.0,0.0,1.04164,(-142.48887,167.27602)}},draw=blue,line cap=rect,line join=bevel,line width=0.800pt]
    \path[fill=blue] (0.0000,0.0000) node[above right] (text1032) {\scriptsize $b_0(t)$};



  \end{scope}
  \begin{scope}[cm={{1.00588,0.0,0.0,1.00588,(197.153,129.759)}},draw=blue,line cap=rect,line join=bevel,line width=0.800pt]
  \end{scope}
  \begin{scope}[scale=1.006,draw=blue,line cap=rect,line join=bevel,line width=0.800pt]
  \end{scope}
  \begin{scope}[scale=1.006,draw=blue,line cap=rect,line join=bevel,line width=0.800pt]
  \end{scope}
  \begin{scope}[scale=1.006,draw=blue,line cap=rect,line join=bevel,line width=0.800pt]
  \end{scope}
  \begin{scope}[scale=1.006,draw=blue,line cap=rect,line join=bevel,line width=0.800pt]
  \end{scope}
  \begin{scope}[scale=1.006,draw=blue,line cap=rect,line join=bevel,line width=0.800pt]
  \end{scope}
  \begin{scope}[cm={{1.04177,0.0,0.0,1.04177,(-342.44493,31.82443)}},draw=c00ff00,line cap=round,line join=round,line width=0.480pt]
    \path[draw] (101.0000,115.7000) -- (101.2000,122.5000) -- (101.7000,123.1000) -- (102.2000,123.7000) -- (102.7000,124.3000) -- (103.2000,124.9000) -- (103.6000,125.5000) -- (104.1000,126.1000) -- (104.6000,126.7000) -- (105.1000,127.3000) -- (105.6000,127.9000) -- (106.1000,128.4000) -- (106.6000,129.0000) -- (107.0000,129.6000) -- (107.5000,130.2000) -- (108.0000,130.8000) -- (108.5000,131.4000) -- (109.0000,132.0000) -- (109.5000,132.6000) -- (110.0000,133.2000) -- (110.4000,133.8000) -- (110.9000,134.3000) -- (111.4000,134.9000) -- (111.9000,135.5000) -- (112.4000,136.1000) -- (112.9000,136.7000) -- (113.4000,137.3000) -- (113.9000,137.9000) -- (114.3000,138.5000) -- (114.8000,139.1000) -- (115.3000,139.7000) -- (115.8000,140.2000) -- (116.3000,140.8000) -- (116.8000,141.4000) -- (117.3000,142.0000) -- (117.7000,142.6000) -- (118.2000,143.2000) -- (118.7000,143.8000) -- (119.2000,144.4000) -- (119.7000,145.0000) -- (120.2000,145.6000) -- (120.7000,146.1000) -- (121.1000,146.7000) -- (121.6000,147.3000) -- (122.1000,147.9000) -- (122.6000,148.5000) -- (123.1000,149.1000) -- (123.6000,149.7000) -- (124.1000,150.3000) -- (124.5000,150.9000) -- (125.0000,151.5000) -- (125.5000,152.0000) -- (126.0000,152.6000) -- (126.5000,153.2000) -- (127.0000,153.8000) -- (127.5000,154.4000) -- (128.0000,155.0000) -- (128.4000,155.6000) -- (128.9000,156.2000) -- (129.4000,156.8000) -- (129.9000,157.4000) -- (130.4000,157.9000) -- (130.9000,158.5000) -- (131.4000,159.1000) -- (131.8000,159.7000) -- (132.3000,160.3000) -- (132.8000,160.9000) -- (133.3000,161.5000) -- (133.8000,162.1000) -- (134.3000,162.7000) -- (134.8000,163.3000) -- (135.2000,163.9000) -- (135.7000,164.4000) -- (136.2000,165.0000) -- (136.7000,165.6000) -- (137.2000,166.2000) -- (137.7000,166.8000) -- (138.2000,167.4000) -- (138.6000,168.0000) -- (139.1000,168.6000) -- (139.6000,169.2000) -- (140.1000,169.7000) -- (140.6000,170.3000) -- (141.1000,170.9000) -- (141.6000,171.5000) -- (142.1000,172.1000) -- (142.5000,172.7000) -- (143.0000,173.3000) -- (143.5000,173.9000) -- (144.0000,174.5000) -- (144.5000,175.1000) -- (145.0000,175.7000) -- (145.5000,176.2000) -- (145.9000,176.8000) -- (146.4000,177.4000) -- (146.9000,178.0000) -- (147.4000,178.6000) -- (147.9000,179.2000) -- (148.4000,179.8000) -- (148.9000,180.4000) -- (149.3000,181.0000) -- (149.8000,181.6000) -- (150.3000,182.1000) -- (150.8000,182.7000) -- (151.3000,183.3000) -- (151.8000,183.9000) -- (152.3000,184.5000) -- (152.7000,185.1000) -- (153.2000,185.7000) -- (153.7000,186.3000) -- (154.2000,186.9000) -- (154.7000,187.5000) -- (155.2000,188.0000) -- (155.7000,188.6000) -- (156.2000,189.2000) -- (156.6000,189.8000) -- (157.1000,190.4000) -- (157.6000,191.0000) -- (158.1000,191.6000) -- (158.6000,192.2000) -- (159.1000,192.8000) -- (159.6000,193.4000) -- (160.0000,193.9000) -- (160.5000,194.5000) -- (161.0000,195.1000) -- (161.5000,195.7000) -- (162.0000,196.3000) -- (162.5000,196.9000) -- (163.0000,197.5000) -- (163.4000,198.1000) -- (163.9000,198.7000) -- (164.4000,199.3000) -- (164.9000,199.8000) -- (165.4000,200.4000) -- (165.5000,200.6000);



  \end{scope}
  \begin{scope}[scale=1.006,draw=blue,line cap=rect,line join=bevel,line width=0.800pt]
  \end{scope}
  \begin{scope}[scale=1.006,draw=blue,line cap=rect,line join=bevel,line width=0.800pt]
  \end{scope}
  \begin{scope}[cm={{1.04177,0.0,0.0,1.04177,(-342.44493,30.31402)}},draw=blue,line cap=round,line join=round,line width=0.480pt]
    \path[shift={(0,1.44988)},draw] (56.5000,115.5000) -- (56.5000,200.5000) -- (251.5000,200.5000) -- (251.5000,115.5000) -- (56.5000,115.5000);



  \end{scope}
  \begin{scope}[cm={{1.04164,0.0,0.0,1.04164,(-342.4953,42.14671)}},draw=ca0a0a4,dash pattern=on 1.73pt off 1.73pt,line cap=round,line join=round,line width=0.288pt,miter limit=4.00]
    \path[draw,dash pattern=on 1.73pt off 1.73pt,line width=0.288pt,miter limit=4.00] (56.5000,299.5000) -- (251.5000,299.5000);



  \end{scope}
  \begin{scope}[cm={{1.04164,0.0,0.0,1.04164,(-342.4953,42.14671)}},draw=blue,line cap=round,line join=round,line width=0.480pt]
    \path[cm={{1.11163,0.0,0.0,1.0,(-6.27372,0.0)}},draw] (56.5000,299.5000) -- (59.5000,299.5000);



    \path[cm={{1.11163,0.0,0.0,1.0,(-28.25871,0.0)}},draw] (251.5000,299.5000) -- (248.5000,299.5000);



  \end{scope}
  \begin{scope}[scale=1.006,draw=blue,line cap=rect,line join=bevel,line width=0.800pt]
  \end{scope}
  \begin{scope}[cm={{1.00588,0.0,0.0,1.00588,(39.2294,305.788)}},draw=blue,line cap=rect,line join=bevel,line width=0.800pt]
  \end{scope}
  \begin{scope}[cm={{1.00588,0.0,0.0,1.00588,(39.2294,305.788)}},draw=blue,line cap=rect,line join=bevel,line width=0.800pt]
  \end{scope}
  \begin{scope}[cm={{1.00588,0.0,0.0,1.00588,(39.2294,305.788)}},draw=blue,line cap=rect,line join=bevel,line width=0.800pt]
  \end{scope}
  \begin{scope}[cm={{1.00588,0.0,0.0,1.00588,(39.2294,305.788)}},draw=blue,line cap=rect,line join=bevel,line width=0.800pt]
  \end{scope}
  \begin{scope}[cm={{1.00588,0.0,0.0,1.00588,(39.2294,305.788)}},draw=blue,line cap=rect,line join=bevel,line width=0.800pt]
  \end{scope}
  \begin{scope}[cm={{1.00588,0.0,0.0,1.00588,(-298.32402,357.32974)}},draw=blue,line cap=rect,line join=bevel,line width=0.800pt]
    \path[fill=blue] (0.0000,0.0000) node[above right] (text1086) {27};



  \end{scope}
  \begin{scope}[cm={{1.00588,0.0,0.0,1.00588,(39.2294,305.788)}},draw=blue,line cap=rect,line join=bevel,line width=0.800pt]
  \end{scope}
  \begin{scope}[scale=1.006,draw=blue,line cap=rect,line join=bevel,line width=0.800pt]
  \end{scope}
  \begin{scope}[cm={{1.04164,0.0,0.0,1.04164,(-342.4953,42.14671)}},draw=ca0a0a4,dash pattern=on 1.73pt off 1.73pt,line cap=round,line join=round,line width=0.288pt,miter limit=4.00]
    \path[draw,dash pattern=on 1.73pt off 1.73pt,line width=0.288pt,miter limit=4.00] (56.5000,277.5000) -- (251.5000,277.5000);



  \end{scope}
  \begin{scope}[cm={{1.04164,0.0,0.0,1.04164,(-342.4953,42.14671)}},draw=blue,line cap=round,line join=round,line width=0.480pt]
    \path[cm={{1.11163,0.0,0.0,1.0,(-6.27372,0.0)}},draw] (56.5000,277.5000) -- (59.5000,277.5000);



    \path[cm={{1.11163,0.0,0.0,1.0,(-28.25871,0.0)}},draw] (251.5000,277.5000) -- (248.5000,277.5000);



  \end{scope}
  \begin{scope}[scale=1.006,draw=blue,line cap=rect,line join=bevel,line width=0.800pt]
  \end{scope}
  \begin{scope}[cm={{1.00588,0.0,0.0,1.00588,(40.2353,282.653)}},draw=blue,line cap=rect,line join=bevel,line width=0.800pt]
  \end{scope}
  \begin{scope}[cm={{1.00588,0.0,0.0,1.00588,(40.2353,282.653)}},draw=blue,line cap=rect,line join=bevel,line width=0.800pt]
  \end{scope}
  \begin{scope}[cm={{1.00588,0.0,0.0,1.00588,(40.2353,282.653)}},draw=blue,line cap=rect,line join=bevel,line width=0.800pt]
  \end{scope}
  \begin{scope}[cm={{1.00588,0.0,0.0,1.00588,(40.2353,282.653)}},draw=blue,line cap=rect,line join=bevel,line width=0.800pt]
  \end{scope}
  \begin{scope}[cm={{1.00588,0.0,0.0,1.00588,(40.2353,282.653)}},draw=blue,line cap=rect,line join=bevel,line width=0.800pt]
  \end{scope}
  \begin{scope}[cm={{1.00588,0.0,0.0,1.00588,(-298.23551,334.19474)}},draw=blue,line cap=rect,line join=bevel,line width=0.800pt]
    \path[fill=blue] (0.3520,0.0000) node[above right] (text1116) {31};



  \end{scope}
  \begin{scope}[cm={{1.00588,0.0,0.0,1.00588,(40.2353,282.653)}},draw=blue,line cap=rect,line join=bevel,line width=0.800pt]
  \end{scope}
  \begin{scope}[scale=1.006,draw=blue,line cap=rect,line join=bevel,line width=0.800pt]
  \end{scope}
  \begin{scope}[cm={{1.04164,0.0,0.0,1.04164,(-342.4953,42.14671)}},draw=ca0a0a4,dash pattern=on 1.73pt off 1.73pt,line cap=round,line join=round,line width=0.288pt,miter limit=4.00]
    \path[draw,dash pattern=on 1.73pt off 1.73pt,line width=0.288pt,miter limit=4.00] (56.5000,254.5000) -- (251.5000,254.5000);



  \end{scope}
  \begin{scope}[cm={{1.04164,0.0,0.0,1.04164,(-342.4953,42.14671)}},draw=blue,line cap=round,line join=round,line width=0.480pt]
    \path[cm={{1.11163,0.0,0.0,1.0,(-6.27372,0.0)}},draw] (56.5000,254.5000) -- (59.5000,254.5000);



    \path[cm={{1.11163,0.0,0.0,1.0,(-28.25871,0.0)}},draw] (251.5000,254.5000) -- (248.5000,254.5000);



  \end{scope}
  \begin{scope}[scale=1.006,draw=blue,line cap=rect,line join=bevel,line width=0.800pt]
  \end{scope}
  \begin{scope}[cm={{1.00588,0.0,0.0,1.00588,(40.2353,260.524)}},draw=blue,line cap=rect,line join=bevel,line width=0.800pt]
  \end{scope}
  \begin{scope}[cm={{1.00588,0.0,0.0,1.00588,(40.2353,260.524)}},draw=blue,line cap=rect,line join=bevel,line width=0.800pt]
  \end{scope}
  \begin{scope}[cm={{1.00588,0.0,0.0,1.00588,(40.2353,260.524)}},draw=blue,line cap=rect,line join=bevel,line width=0.800pt]
  \end{scope}
  \begin{scope}[cm={{1.00588,0.0,0.0,1.00588,(40.2353,260.524)}},draw=blue,line cap=rect,line join=bevel,line width=0.800pt]
  \end{scope}
  \begin{scope}[cm={{1.00588,0.0,0.0,1.00588,(40.2353,260.524)}},draw=blue,line cap=rect,line join=bevel,line width=0.800pt]
  \end{scope}
  \begin{scope}[cm={{1.00588,0.0,0.0,1.00588,(-298.23551,309.06574)}},draw=blue,line cap=rect,line join=bevel,line width=0.800pt]
    \path[fill=blue] (0.0000,0.0000) node[above right] (text1146) {35};



  \end{scope}
  \begin{scope}[cm={{1.00588,0.0,0.0,1.00588,(40.2353,260.524)}},draw=blue,line cap=rect,line join=bevel,line width=0.800pt]
  \end{scope}
  \begin{scope}[scale=1.006,draw=blue,line cap=rect,line join=bevel,line width=0.800pt]
  \end{scope}
  \begin{scope}[cm={{1.04164,0.0,0.0,1.04164,(-342.4953,42.14671)}},draw=blue,line cap=round,line join=round,line width=0.480pt]
    \path[cm={{1.11163,0.0,0.0,1.0,(-6.27372,0.0)}},draw] (56.5000,232.5000) -- (59.5000,232.5000);



    \path[cm={{1.11163,0.0,0.0,1.0,(-28.25871,0.0)}},draw] (251.5000,232.5000) -- (248.5000,232.5000);



  \end{scope}
  \begin{scope}[scale=1.006,draw=blue,line cap=rect,line join=bevel,line width=0.800pt]
  \end{scope}
  \begin{scope}[cm={{1.00588,0.0,0.0,1.00588,(39.2294,237.388)}},draw=blue,line cap=rect,line join=bevel,line width=0.800pt]
  \end{scope}
  \begin{scope}[cm={{1.00588,0.0,0.0,1.00588,(39.2294,237.388)}},draw=blue,line cap=rect,line join=bevel,line width=0.800pt]
  \end{scope}
  \begin{scope}[cm={{1.00588,0.0,0.0,1.00588,(39.2294,237.388)}},draw=blue,line cap=rect,line join=bevel,line width=0.800pt]
  \end{scope}
  \begin{scope}[cm={{1.00588,0.0,0.0,1.00588,(39.2294,237.388)}},draw=blue,line cap=rect,line join=bevel,line width=0.800pt]
  \end{scope}
  \begin{scope}[cm={{1.00588,0.0,0.0,1.00588,(39.2294,237.388)}},draw=blue,line cap=rect,line join=bevel,line width=0.800pt]
  \end{scope}
  \begin{scope}[cm={{1.00588,0.0,0.0,1.00588,(-298.4045,287.42974)}},draw=blue,line cap=rect,line join=bevel,line width=0.800pt]
    \path[fill=blue] (0.0000,0.0000) node[above right] (text1176) {39};



  \end{scope}
  \begin{scope}[cm={{1.00588,0.0,0.0,1.00588,(39.2294,237.388)}},draw=blue,line cap=rect,line join=bevel,line width=0.800pt]
  \end{scope}
  \begin{scope}[scale=1.006,draw=blue,line cap=rect,line join=bevel,line width=0.800pt]
  \end{scope}
  \begin{scope}[cm={{1.04164,0.0,0.0,1.04164,(-342.4953,42.14671)}},draw=ca0a0a4,dash pattern=on 0.40pt off 0.80pt,line cap=round,line join=round,line width=0.400pt]
    \path[draw] (56.5000,306.5000) -- (56.5000,221.5000);



  \end{scope}
  \begin{scope}[cm={{1.04164,0.0,0.0,1.04164,(-342.4953,42.14671)}},draw=blue,line cap=round,line join=round,line width=0.480pt]
    \path[draw] (56.5000,306.5000) -- (56.5000,303.5000);



    \path[draw] (56.5000,221.5000) -- (56.5000,223.5000);



  \end{scope}
  \begin{scope}[scale=1.006,draw=blue,line cap=rect,line join=bevel,line width=0.800pt]
  \end{scope}
  \begin{scope}[cm={{1.00588,0.0,0.0,1.00588,(53.3118,322.888)}},draw=blue,line cap=rect,line join=bevel,line width=0.800pt]
  \end{scope}
  \begin{scope}[cm={{1.00588,0.0,0.0,1.00588,(53.3118,322.888)}},draw=blue,line cap=rect,line join=bevel,line width=0.800pt]
  \end{scope}
  \begin{scope}[cm={{1.00588,0.0,0.0,1.00588,(53.3118,322.888)}},draw=blue,line cap=rect,line join=bevel,line width=0.800pt]
  \end{scope}
  \begin{scope}[cm={{1.00588,0.0,0.0,1.00588,(53.3118,322.888)}},draw=blue,line cap=rect,line join=bevel,line width=0.800pt]
  \end{scope}
  \begin{scope}[cm={{1.00588,0.0,0.0,1.00588,(53.3118,322.888)}},draw=blue,line cap=rect,line join=bevel,line width=0.800pt]
  \end{scope}
  \begin{scope}[cm={{1.00588,0.0,0.0,1.00588,(-285.68826,376.888)}},draw=blue,line cap=rect,line join=bevel,line width=0.800pt]
    \path[fill=blue] (0.0000,0.0000) node[above right] (text1206) {0};



  \end{scope}
  \begin{scope}[cm={{1.00588,0.0,0.0,1.00588,(53.3118,322.888)}},draw=blue,line cap=rect,line join=bevel,line width=0.800pt]
  \end{scope}
  \begin{scope}[scale=1.006,draw=blue,line cap=rect,line join=bevel,line width=0.800pt]
  \end{scope}
  \begin{scope}[cm={{1.04164,0.0,0.0,1.04164,(-342.4953,42.14671)}},draw=ca0a0a4,dash pattern=on 1.73pt off 1.73pt,line cap=round,line join=round,line width=0.288pt,miter limit=4.00]
    \path[draw,dash pattern=on 1.73pt off 1.73pt,line width=0.288pt,miter limit=4.00] (91.5000,306.5000) -- (91.5000,221.5000);



  \end{scope}
  \begin{scope}[cm={{1.04164,0.0,0.0,1.04164,(-342.4953,42.14671)}},draw=blue,line cap=round,line join=round,line width=0.480pt]
    \path[cm={{1.0,0.0,0.0,1.11163,(0.0,-34.30653)}},draw] (91.5000,306.5000) -- (91.5000,303.5000);



    \path[cm={{1.0,0.0,0.0,1.53918,(0.0,-119.26725)}},draw] (91.5000,221.5000) -- (91.5000,223.5000);



  \end{scope}
  \begin{scope}[scale=1.006,draw=blue,line cap=rect,line join=bevel,line width=0.800pt]
  \end{scope}
  \begin{scope}[cm={{1.00588,0.0,0.0,1.00588,(89.5235,322.888)}},draw=blue,line cap=rect,line join=bevel,line width=0.800pt]
  \end{scope}
  \begin{scope}[cm={{1.00588,0.0,0.0,1.00588,(89.5235,322.888)}},draw=blue,line cap=rect,line join=bevel,line width=0.800pt]
  \end{scope}
  \begin{scope}[cm={{1.00588,0.0,0.0,1.00588,(89.5235,322.888)}},draw=blue,line cap=rect,line join=bevel,line width=0.800pt]
  \end{scope}
  \begin{scope}[cm={{1.00588,0.0,0.0,1.00588,(89.5235,322.888)}},draw=blue,line cap=rect,line join=bevel,line width=0.800pt]
  \end{scope}
  \begin{scope}[cm={{1.00588,0.0,0.0,1.00588,(89.5235,322.888)}},draw=blue,line cap=rect,line join=bevel,line width=0.800pt]
  \end{scope}
  \begin{scope}[cm={{1.00588,0.0,0.0,1.00588,(-247.97657,377.00066)}},draw=blue,line cap=rect,line join=bevel,line width=0.800pt]
    \path[fill=blue] (0.0000,0.0000) node[above right] (text1236) {2};



  \end{scope}
  \begin{scope}[cm={{1.00588,0.0,0.0,1.00588,(89.5235,322.888)}},draw=blue,line cap=rect,line join=bevel,line width=0.800pt]
  \end{scope}
  \begin{scope}[scale=1.006,draw=blue,line cap=rect,line join=bevel,line width=0.800pt]
  \end{scope}
  \begin{scope}[cm={{1.04164,0.0,0.0,1.04164,(-342.4953,42.14671)}},draw=ca0a0a4,dash pattern=on 1.73pt off 1.73pt,line cap=round,line join=round,line width=0.288pt,miter limit=4.00]
    \path[draw,dash pattern=on 1.73pt off 1.73pt,line width=0.288pt,miter limit=4.00] (127.5000,306.5000) -- (127.5000,221.5000);



  \end{scope}
  \begin{scope}[cm={{1.04164,0.0,0.0,1.04164,(-342.4953,42.14671)}},draw=blue,line cap=round,line join=round,line width=0.480pt]
    \path[cm={{1.0,0.0,0.0,1.11163,(0.0,-34.30653)}},draw] (127.5000,306.5000) -- (127.5000,303.5000);



    \path[cm={{1.0,0.0,0.0,1.53918,(0.0,-119.26725)}},draw] (127.5000,221.5000) -- (127.5000,223.5000);



  \end{scope}
  \begin{scope}[scale=1.006,draw=blue,line cap=rect,line join=bevel,line width=0.800pt]
  \end{scope}
  \begin{scope}[cm={{1.00588,0.0,0.0,1.00588,(125.232,322.888)}},draw=blue,line cap=rect,line join=bevel,line width=0.800pt]
  \end{scope}
  \begin{scope}[cm={{1.00588,0.0,0.0,1.00588,(125.232,322.888)}},draw=blue,line cap=rect,line join=bevel,line width=0.800pt]
  \end{scope}
  \begin{scope}[cm={{1.00588,0.0,0.0,1.00588,(125.232,322.888)}},draw=blue,line cap=rect,line join=bevel,line width=0.800pt]
  \end{scope}
  \begin{scope}[cm={{1.00588,0.0,0.0,1.00588,(125.232,322.888)}},draw=blue,line cap=rect,line join=bevel,line width=0.800pt]
  \end{scope}
  \begin{scope}[cm={{1.00588,0.0,0.0,1.00588,(125.232,322.888)}},draw=blue,line cap=rect,line join=bevel,line width=0.800pt]
  \end{scope}
  \begin{scope}[cm={{1.00588,0.0,0.0,1.00588,(-212.26806,376.888)}},draw=blue,line cap=rect,line join=bevel,line width=0.800pt]
    \path[fill=blue] (0.0000,0.1120) node[above right] (text1266) {4};



  \end{scope}
  \begin{scope}[cm={{1.00588,0.0,0.0,1.00588,(125.232,322.888)}},draw=blue,line cap=rect,line join=bevel,line width=0.800pt]
  \end{scope}
  \begin{scope}[scale=1.006,draw=blue,line cap=rect,line join=bevel,line width=0.800pt]
  \end{scope}
  \begin{scope}[cm={{1.04164,0.0,0.0,1.04164,(-342.4953,42.14671)}},draw=ca0a0a4,dash pattern=on 1.73pt off 1.73pt,line cap=round,line join=round,line width=0.288pt,miter limit=4.00]
    \path[draw,dash pattern=on 1.73pt off 1.73pt,line width=0.288pt,miter limit=4.00] (162.5000,306.5000) -- (162.5000,221.5000);



  \end{scope}
  \begin{scope}[cm={{1.04164,0.0,0.0,1.04164,(-342.4953,42.14671)}},draw=blue,line cap=round,line join=round,line width=0.480pt]
    \path[cm={{1.0,0.0,0.0,1.11163,(0.0,-34.30653)}},draw] (162.5000,306.5000) -- (162.5000,303.5000);



    \path[cm={{1.0,0.0,0.0,1.53918,(0.0,-119.26725)}},draw] (162.5000,221.5000) -- (162.5000,223.5000);



  \end{scope}
  \begin{scope}[scale=1.006,draw=blue,line cap=rect,line join=bevel,line width=0.800pt]
  \end{scope}
  \begin{scope}[cm={{1.00588,0.0,0.0,1.00588,(160.941,322.888)}},draw=blue,line cap=rect,line join=bevel,line width=0.800pt]
  \end{scope}
  \begin{scope}[cm={{1.00588,0.0,0.0,1.00588,(160.941,322.888)}},draw=blue,line cap=rect,line join=bevel,line width=0.800pt]
  \end{scope}
  \begin{scope}[cm={{1.00588,0.0,0.0,1.00588,(160.941,322.888)}},draw=blue,line cap=rect,line join=bevel,line width=0.800pt]
  \end{scope}
  \begin{scope}[cm={{1.00588,0.0,0.0,1.00588,(160.941,322.888)}},draw=blue,line cap=rect,line join=bevel,line width=0.800pt]
  \end{scope}
  \begin{scope}[cm={{1.00588,0.0,0.0,1.00588,(160.941,322.888)}},draw=blue,line cap=rect,line join=bevel,line width=0.800pt]
  \end{scope}
  \begin{scope}[cm={{1.00588,0.0,0.0,1.00588,(-173.55905,376.888)}},draw=blue,line cap=rect,line join=bevel,line width=0.800pt]
    \path[fill=blue] (0.0000,0.0000) node[above right] (text1296) {6};



  \end{scope}
  \begin{scope}[cm={{1.00588,0.0,0.0,1.00588,(160.941,322.888)}},draw=blue,line cap=rect,line join=bevel,line width=0.800pt]
  \end{scope}
  \begin{scope}[scale=1.006,draw=blue,line cap=rect,line join=bevel,line width=0.800pt]
  \end{scope}
  \begin{scope}[cm={{1.04164,0.0,0.0,1.04164,(-342.4953,42.14671)}},draw=blue,line cap=round,line join=round,line width=0.480pt]
    \path[cm={{1.0,0.0,0.0,1.11163,(0.0,-34.30653)}},draw] (198.5000,306.5000) -- (198.5000,303.5000);



    \path[cm={{1.0,0.0,0.0,1.53918,(0.0,-119.26725)}},draw] (198.5000,221.5000) -- (198.5000,223.5000);



  \end{scope}
  \begin{scope}[scale=1.006,draw=blue,line cap=rect,line join=bevel,line width=0.800pt]
  \end{scope}
  \begin{scope}[cm={{1.00588,0.0,0.0,1.00588,(196.147,322.888)}},draw=blue,line cap=rect,line join=bevel,line width=0.800pt]
  \end{scope}
  \begin{scope}[cm={{1.00588,0.0,0.0,1.00588,(196.147,322.888)}},draw=blue,line cap=rect,line join=bevel,line width=0.800pt]
  \end{scope}
  \begin{scope}[cm={{1.00588,0.0,0.0,1.00588,(196.147,322.888)}},draw=blue,line cap=rect,line join=bevel,line width=0.800pt]
  \end{scope}
  \begin{scope}[cm={{1.00588,0.0,0.0,1.00588,(196.147,322.888)}},draw=blue,line cap=rect,line join=bevel,line width=0.800pt]
  \end{scope}
  \begin{scope}[cm={{1.00588,0.0,0.0,1.00588,(196.147,322.888)}},draw=blue,line cap=rect,line join=bevel,line width=0.800pt]
  \end{scope}
  \begin{scope}[cm={{1.00588,0.0,0.0,1.00588,(-136.85305,376.888)}},draw=blue,line cap=rect,line join=bevel,line width=0.800pt]
    \path[fill=blue] (0.0000,0.0000) node[above right] (text1326) {8};



  \end{scope}
  \begin{scope}[cm={{1.00588,0.0,0.0,1.00588,(196.147,322.888)}},draw=blue,line cap=rect,line join=bevel,line width=0.800pt]
  \end{scope}
  \begin{scope}[scale=1.006,draw=blue,line cap=rect,line join=bevel,line width=0.800pt]
  \end{scope}
  \begin{scope}[cm={{1.04164,0.0,0.0,1.04164,(-342.4953,42.14671)}},draw=blue,line cap=round,line join=round,line width=0.480pt]
    \path[cm={{1.0,0.0,0.0,1.11163,(0.0,-34.30653)}},draw] (233.5000,306.5000) -- (233.5000,303.5000);



    \path[cm={{1.0,0.0,0.0,1.53918,(0.0,-119.26725)}},draw] (233.5000,221.5000) -- (233.5000,223.5000);



  \end{scope}
  \begin{scope}[scale=1.006,draw=blue,line cap=rect,line join=bevel,line width=0.800pt]
  \end{scope}
  \begin{scope}[cm={{1.00588,0.0,0.0,1.00588,(228.838,322.888)}},draw=blue,line cap=rect,line join=bevel,line width=0.800pt]
  \end{scope}
  \begin{scope}[cm={{1.00588,0.0,0.0,1.00588,(228.838,322.888)}},draw=blue,line cap=rect,line join=bevel,line width=0.800pt]
  \end{scope}
  \begin{scope}[cm={{1.00588,0.0,0.0,1.00588,(228.838,322.888)}},draw=blue,line cap=rect,line join=bevel,line width=0.800pt]
  \end{scope}
  \begin{scope}[cm={{1.00588,0.0,0.0,1.00588,(228.838,322.888)}},draw=blue,line cap=rect,line join=bevel,line width=0.800pt]
  \end{scope}
  \begin{scope}[cm={{1.00588,0.0,0.0,1.00588,(228.838,322.888)}},draw=blue,line cap=rect,line join=bevel,line width=0.800pt]
  \end{scope}
  \begin{scope}[cm={{1.00588,0.0,0.0,1.00588,(-102.66205,376.888)}},draw=blue,line cap=rect,line join=bevel,line width=0.800pt]
    \path[fill=blue] (0.0000,0.0000) node[above right] (text1356) {10};



  \end{scope}
  \begin{scope}[cm={{1.00588,0.0,0.0,1.00588,(228.838,322.888)}},draw=blue,line cap=rect,line join=bevel,line width=0.800pt]
  \end{scope}
  \begin{scope}[scale=1.006,draw=blue,line cap=rect,line join=bevel,line width=0.800pt]
  \end{scope}
  \begin{scope}[cm={{1.04164,0.0,0.0,1.04164,(-342.4953,42.14671)}},draw=blue,line cap=round,line join=round,line width=0.480pt]
    \path[draw] (56.5000,221.5000) -- (56.5000,306.5000) -- (251.5000,306.5000) -- (251.5000,221.5000) -- (56.5000,221.5000);



  \end{scope}
  \begin{scope}[scale=1.006,draw=blue,line cap=rect,line join=bevel,line width=0.800pt]
  \end{scope}
  \begin{scope}[scale=1.006,draw=blue,line cap=rect,line join=bevel,line width=0.800pt]
  \end{scope}
  \begin{scope}[scale=1.006,draw=blue,line cap=rect,line join=bevel,line width=0.800pt]
  \end{scope}
  \begin{scope}[scale=1.006,draw=blue,line cap=rect,line join=bevel,line width=0.800pt]
  \end{scope}
  \begin{scope}[scale=1.006,draw=blue,line cap=rect,line join=bevel,line width=0.800pt]
  \end{scope}
  \begin{scope}[cm={{1.00588,0.0,0.0,1.00588,(232.359,238.394)}},draw=blue,line cap=rect,line join=bevel,line width=0.800pt]
  \end{scope}
  \begin{scope}[cm={{1.00588,0.0,0.0,1.00588,(232.359,238.394)}},draw=blue,line cap=rect,line join=bevel,line width=0.800pt]
  \end{scope}
  \begin{scope}[cm={{1.00588,0.0,0.0,1.00588,(232.359,238.394)}},draw=blue,line cap=rect,line join=bevel,line width=0.800pt]
  \end{scope}
  \begin{scope}[cm={{1.00588,0.0,0.0,1.00588,(232.359,238.394)}},draw=blue,line cap=rect,line join=bevel,line width=0.800pt]
  \end{scope}
  \begin{scope}[cm={{1.00588,0.0,0.0,1.00588,(232.359,238.394)}},draw=blue,line cap=rect,line join=bevel,line width=0.800pt]
  \end{scope}
  \begin{scope}[cm={{1.00588,0.0,0.0,1.00588,(232.359,238.394)}},draw=blue,line cap=rect,line join=bevel,line width=0.800pt]
  \end{scope}
  \begin{scope}[cm={{1.00588,0.0,0.0,1.00588,(130.262,337.976)}},draw=blue,line cap=rect,line join=bevel,line width=0.800pt]
  \end{scope}
  \begin{scope}[cm={{1.00588,0.0,0.0,1.00588,(130.262,337.976)}},draw=blue,line cap=rect,line join=bevel,line width=0.800pt]
  \end{scope}
  \begin{scope}[cm={{1.00588,0.0,0.0,1.00588,(130.262,337.976)}},draw=blue,line cap=rect,line join=bevel,line width=0.800pt]
  \end{scope}
  \begin{scope}[cm={{1.00588,0.0,0.0,1.00588,(130.262,337.976)}},draw=blue,line cap=rect,line join=bevel,line width=0.800pt]
  \end{scope}
  \begin{scope}[cm={{1.00588,0.0,0.0,1.00588,(130.262,337.976)}},draw=blue,line cap=rect,line join=bevel,line width=0.800pt]
  \end{scope}
  \begin{scope}[cm={{1.00588,0.0,0.0,1.00588,(-204.23807,390.01506)}},draw=blue,line cap=rect,line join=bevel,line width=0.800pt]
    \path[fill=blue] (0.0000,0.0000) node[above right] (text1412) {Time (min)};



  \end{scope}
  \begin{scope}[cm={{1.00588,0.0,0.0,1.00588,(130.262,337.976)}},draw=blue,line cap=rect,line join=bevel,line width=0.800pt]
  \end{scope}
  \begin{scope}[scale=1.006,draw=blue,line cap=rect,line join=bevel,line width=0.800pt]
  \end{scope}
  \begin{scope}[scale=1.006,draw=blue,line cap=rect,line join=bevel,line width=0.800pt]
  \end{scope}
  \begin{scope}[scale=1.006,draw=blue,line cap=rect,line join=bevel,line width=0.800pt]
  \end{scope}
  \begin{scope}[cm={{1.04164,0.0,0.0,1.04164,(-342.4953,42.14671)}},draw=blue,line cap=round,line join=round,line width=0.480pt]
    \path[draw] (56.1000,255.4000) -- (56.1000,255.4000) -- (56.3000,261.0000) -- (56.5000,262.0000) -- (56.7000,262.2000) -- (56.9000,262.4000) -- (57.1000,261.4000) -- (57.3000,260.3000) -- (57.5000,259.7000) -- (57.7000,259.7000) -- (57.9000,259.8000) -- (58.1000,259.9000) -- (58.3000,260.0000) -- (58.4000,260.0000) -- (58.6000,260.0000) -- (58.8000,260.0000) -- (59.0000,260.0000) -- (59.2000,260.0000) -- (59.4000,260.0000) -- (59.6000,260.0000) -- (59.8000,260.0000) -- (60.0000,260.0000) -- (60.2000,260.0000) -- (60.4000,260.0000) -- (60.6000,260.2000) -- (60.8000,260.7000) -- (61.0000,261.3000) -- (61.2000,262.2000) -- (61.4000,263.3000) -- (61.6000,264.6000) -- (61.8000,266.1000) -- (62.0000,267.8000) -- (62.2000,269.7000) -- (62.4000,271.8000) -- (62.6000,274.1000) -- (62.8000,276.6000) -- (63.0000,279.2000) -- (63.1000,281.9000) -- (63.3000,284.7000) -- (63.5000,287.5000) -- (63.7000,290.1000) -- (63.9000,292.8000) -- (64.1000,295.5000) -- (64.3000,297.5000) -- (64.5000,298.2000) -- (64.7000,298.2000) -- (64.9000,297.9000) -- (65.1000,297.6000) -- (65.3000,297.5000) -- (65.5000,297.4000) -- (65.7000,297.4000) -- (65.9000,297.4000) -- (66.1000,297.4000) -- (66.3000,297.5000) -- (66.5000,297.5000) -- (66.7000,297.5000) -- (66.9000,297.5000) -- (67.1000,297.4000) -- (67.3000,297.1000) -- (67.5000,297.2000) -- (67.7000,297.2000) -- (67.9000,296.1000) -- (68.0000,294.1000) -- (68.2000,291.2000) -- (68.4000,288.0000) -- (68.6000,284.5000) -- (68.8000,281.1000) -- (69.0000,277.8000) -- (69.2000,274.8000) -- (69.4000,272.0000) -- (69.6000,269.4000) -- (69.8000,267.2000) -- (70.0000,265.3000) -- (70.2000,263.7000) -- (70.4000,262.4000) -- (70.6000,261.3000) -- (70.8000,260.5000) -- (71.0000,260.0000) -- (71.2000,259.6000) -- (71.4000,259.7000) -- (71.6000,259.8000) -- (71.8000,259.9000) -- (72.0000,260.0000) -- (72.2000,260.0000) -- (72.4000,260.0000) -- (72.6000,260.0000) -- (72.7000,260.0000) -- (72.9000,260.0000) -- (73.1000,260.0000) -- (73.3000,260.0000) -- (73.5000,260.0000) -- (73.7000,260.0000) -- (73.9000,260.0000) -- (74.1000,260.0000) -- (74.3000,260.1000) -- (74.5000,260.5000) -- (74.7000,261.1000) -- (74.9000,261.9000) -- (75.1000,262.9000) -- (75.3000,264.1000) -- (75.5000,265.4000) -- (75.7000,267.0000) -- (75.9000,268.9000) -- (76.1000,270.9000) -- (76.3000,273.1000) -- (76.5000,275.5000) -- (76.7000,278.0000) -- (76.9000,280.7000) -- (77.1000,283.4000) -- (77.3000,286.2000) -- (77.4000,288.9000) -- (77.6000,291.5000) -- (77.8000,294.2000) -- (78.0000,296.7000) -- (78.2000,298.1000) -- (78.4000,298.3000) -- (78.6000,298.0000) -- (78.8000,297.7000) -- (79.0000,297.5000) -- (79.2000,297.4000) -- (79.4000,297.4000) -- (79.6000,297.4000) -- (79.8000,297.4000) -- (80.0000,297.4000) -- (80.2000,297.5000) -- (80.4000,297.5000) -- (80.6000,297.5000) -- (80.8000,297.5000) -- (81.0000,298.3000) -- (81.2000,280.4000) -- (81.4000,268.1000) -- (81.6000,268.7000) -- (81.8000,267.3000) -- (82.0000,265.2000) -- (82.2000,262.6000) -- (82.3000,259.7000) -- (82.5000,256.7000) -- (82.7000,253.7000) -- (82.9000,250.7000) -- (83.1000,247.9000) -- (83.3000,245.3000) -- (83.5000,242.9000) -- (83.7000,240.7000) -- (83.9000,238.8000) -- (84.1000,237.2000) -- (84.3000,235.7000) -- (84.5000,234.5000) -- (84.7000,233.5000) -- (84.9000,232.7000) -- (85.1000,232.2000) -- (85.3000,231.7000) -- (85.5000,231.6000) -- (85.7000,231.6000) -- (85.9000,231.7000) -- (86.1000,231.8000) -- (86.3000,231.9000) -- (86.5000,231.9000) -- (86.7000,231.9000) -- (86.9000,231.9000) -- (87.0000,231.9000) -- (87.2000,231.9000) -- (87.4000,231.9000) -- (87.6000,231.9000) -- (87.8000,231.9000) -- (88.0000,231.9000) -- (88.2000,231.9000) -- (88.4000,231.9000) -- (88.6000,232.1000) -- (88.8000,232.5000) -- (89.0000,233.1000) -- (89.2000,234.0000) -- (89.4000,235.0000) -- (89.6000,236.2000) -- (89.8000,237.7000) -- (90.0000,239.3000) -- (90.2000,241.2000) -- (90.4000,243.2000) -- (90.6000,245.4000) -- (90.8000,247.9000) -- (91.0000,250.4000) -- (91.2000,253.1000) -- (91.4000,255.8000) -- (91.6000,258.6000) -- (91.8000,261.3000) -- (91.9000,263.9000) -- (92.1000,266.6000) -- (92.3000,269.0000) -- (92.5000,270.1000) -- (92.7000,270.2000) -- (92.9000,269.8000) -- (93.1000,269.5000) -- (93.3000,269.4000) -- (93.5000,269.3000) -- (93.7000,269.3000) -- (93.9000,269.3000) -- (94.1000,269.3000) -- (94.3000,269.3000) -- (94.5000,269.3000) -- (94.7000,269.3000) -- (94.9000,269.3000) -- (95.1000,269.3000) -- (95.3000,269.1000) -- (95.5000,269.1000) -- (95.7000,269.2000) -- (95.9000,268.5000) -- (96.1000,266.9000) -- (96.3000,264.6000) -- (96.5000,261.9000) -- (96.6000,259.0000) -- (96.8000,255.9000) -- (97.0000,252.9000) -- (97.2000,250.0000) -- (97.4000,247.3000) -- (97.6000,244.7000) -- (97.8000,242.4000) -- (98.0000,240.3000) -- (98.2000,238.4000) -- (98.4000,236.8000) -- (98.6000,235.4000) -- (98.8000,234.2000) -- (99.0000,233.3000) -- (99.2000,232.6000) -- (99.4000,232.0000) -- (99.6000,231.7000) -- (99.8000,231.6000) -- (100.0000,231.7000) -- (100.2000,231.8000) -- (100.4000,231.8000) -- (100.6000,231.9000) -- (100.8000,231.9000) -- (101.0000,231.9000) -- (101.2000,231.9000) -- (101.3000,231.9000) -- (101.5000,231.9000) -- (101.7000,231.9000) -- (101.9000,231.9000) -- (102.1000,231.9000) -- (102.3000,231.9000) -- (102.5000,231.9000) -- (102.7000,231.9000) -- (102.9000,232.2000) -- (103.1000,232.7000) -- (103.3000,233.4000) -- (103.5000,234.3000) -- (103.7000,235.4000) -- (103.9000,236.7000) -- (104.1000,238.2000) -- (104.3000,239.9000) -- (104.5000,241.8000) -- (104.7000,244.0000) -- (104.9000,246.3000) -- (105.1000,248.7000) -- (105.3000,251.4000) -- (105.5000,254.1000) -- (105.7000,256.8000) -- (105.9000,259.6000) -- (106.0000,262.2000) -- (106.2000,264.9000) -- (106.4000,267.6000) -- (106.6000,269.5000) -- (106.8000,270.2000) -- (107.0000,270.0000) -- (107.2000,269.7000) -- (107.4000,269.5000) -- (107.6000,269.3000) -- (107.8000,269.3000) -- (108.0000,269.3000) -- (108.2000,269.3000) -- (108.4000,269.3000) -- (108.6000,269.3000) -- (108.8000,269.3000) -- (109.0000,269.3000) -- (109.2000,269.3000) -- (109.4000,269.2000) -- (109.6000,269.0000) -- (109.8000,269.2000) -- (110.0000,269.0000) -- (110.2000,268.0000) -- (110.4000,266.1000) -- (110.6000,263.6000) -- (110.8000,260.8000) -- (110.9000,257.8000) -- (111.1000,254.8000) -- (111.3000,251.8000) -- (111.5000,248.9000) -- (111.7000,246.2000) -- (111.9000,243.8000) -- (112.1000,241.5000) -- (112.3000,239.5000) -- (112.5000,237.7000) -- (112.7000,236.2000) -- (112.9000,234.9000) -- (113.1000,233.8000) -- (113.3000,233.0000) -- (113.5000,232.3000) -- (113.7000,231.9000) -- (113.9000,231.6000) -- (114.1000,231.6000) -- (114.3000,231.7000) -- (114.5000,231.8000) -- (114.7000,231.9000) -- (114.9000,231.9000) -- (115.1000,231.9000) -- (115.3000,231.9000) -- (115.5000,231.9000) -- (115.6000,231.9000) -- (115.8000,231.9000) -- (116.0000,231.9000) -- (116.2000,231.9000) -- (116.4000,231.9000) -- (116.6000,231.9000) -- (116.8000,231.9000) -- (117.0000,232.0000) -- (117.2000,232.3000) -- (117.4000,232.9000) -- (117.6000,233.7000) -- (117.8000,234.7000) -- (118.0000,235.9000) -- (118.2000,237.2000) -- (118.4000,238.8000) -- (118.6000,240.6000) -- (118.8000,242.6000) -- (119.0000,244.8000) -- (119.2000,247.2000) -- (119.4000,249.8000) -- (119.6000,252.4000) -- (119.8000,255.2000) -- (120.0000,257.9000) -- (120.2000,260.6000) -- (120.4000,263.2000) -- (120.5000,266.0000) -- (120.7000,268.5000) -- (120.9000,269.9000) -- (121.1000,270.2000) -- (121.3000,269.9000) -- (121.5000,269.6000) -- (121.7000,269.4000) -- (121.9000,269.3000) -- (122.1000,269.3000) -- (122.3000,269.3000) -- (122.5000,269.3000) -- (122.7000,269.3000) -- (122.9000,269.3000) -- (123.1000,269.3000) -- (123.3000,269.3000) -- (123.5000,269.3000) -- (123.7000,269.1000) -- (123.9000,269.1000) -- (124.1000,269.2000) -- (124.3000,268.7000) -- (124.5000,267.3000) -- (124.7000,265.2000) -- (124.9000,262.5000) -- (125.1000,259.6000) -- (125.2000,256.6000) -- (125.4000,253.6000) -- (125.6000,250.6000) -- (125.8000,247.8000) -- (126.0000,245.2000) -- (126.2000,242.8000) -- (126.4000,240.7000) -- (126.6000,238.8000) -- (126.8000,237.1000) -- (127.0000,235.7000) -- (127.2000,234.5000) -- (127.4000,233.5000) -- (127.6000,232.7000) -- (127.8000,232.1000) -- (128.0000,231.7000) -- (128.2000,231.6000) -- (128.4000,231.6000) -- (128.6000,231.8000) -- (128.8000,231.8000) -- (129.0000,231.9000) -- (129.2000,231.9000) -- (129.4000,231.9000) -- (129.6000,231.9000) -- (129.8000,231.9000) -- (129.9000,231.9000) -- (130.1000,231.9000) -- (130.3000,231.9000) -- (130.5000,231.9000) -- (130.7000,231.9000) -- (130.9000,231.9000) -- (131.1000,231.9000) -- (131.3000,232.1000) -- (131.5000,232.5000) -- (131.7000,233.2000) -- (131.9000,234.1000) -- (132.1000,235.1000) -- (132.3000,236.4000) -- (132.5000,237.9000) -- (132.7000,239.6000) -- (132.9000,241.4000) -- (133.1000,243.5000) -- (133.3000,245.8000) -- (133.5000,248.2000) -- (133.7000,250.8000) -- (133.9000,253.5000) -- (134.1000,256.3000) -- (134.3000,259.1000) -- (134.5000,261.7000) -- (134.7000,264.3000) -- (134.8000,267.1000) -- (135.0000,269.2000) -- (135.2000,270.1000) -- (135.4000,270.1000) -- (135.6000,269.8000) -- (135.8000,269.5000) -- (136.0000,269.3000) -- (136.2000,269.3000) -- (136.4000,269.3000) -- (136.6000,269.3000) -- (136.8000,269.3000) -- (137.0000,269.3000) -- (137.2000,269.3000) -- (137.4000,269.3000) -- (137.6000,269.3000) -- (137.8000,269.3000) -- (138.0000,269.0000) -- (138.2000,269.2000) -- (138.4000,269.1000) -- (138.6000,268.2000) -- (138.8000,266.5000) -- (139.0000,264.1000) -- (139.2000,261.3000) -- (139.4000,258.4000) -- (139.5000,255.3000) -- (139.7000,252.3000) -- (139.9000,249.4000) -- (140.1000,246.7000) -- (140.3000,244.2000) -- (140.5000,241.9000) -- (140.7000,239.9000) -- (140.9000,238.1000) -- (141.1000,236.5000) -- (141.3000,235.1000) -- (141.5000,234.0000) -- (141.7000,233.1000) -- (141.9000,232.4000) -- (142.1000,231.9000) -- (142.3000,231.6000) -- (142.5000,231.6000) -- (142.7000,231.7000) -- (142.9000,231.8000) -- (143.1000,231.8000) -- (143.3000,231.9000) -- (143.5000,231.9000) -- (143.7000,231.9000) -- (143.9000,231.9000) -- (144.1000,231.9000) -- (144.3000,231.9000) -- (144.4000,231.9000) -- (144.6000,231.9000) -- (144.8000,231.9000) -- (145.0000,231.9000) -- (145.2000,231.9000) -- (145.4000,232.0000) -- (145.6000,232.3000) -- (145.8000,232.8000) -- (146.0000,233.5000) -- (146.2000,234.5000) -- (146.4000,235.6000) -- (146.6000,237.0000) -- (146.8000,238.5000) -- (147.0000,240.3000) -- (147.2000,242.3000) -- (147.4000,244.4000) -- (147.6000,246.8000) -- (147.8000,249.3000) -- (148.0000,252.0000) -- (148.2000,254.7000) -- (148.4000,257.5000) -- (148.6000,260.2000) -- (148.8000,262.8000) -- (149.0000,265.5000) -- (149.1000,268.1000) -- (149.3000,269.8000) -- (149.5000,270.2000) -- (149.7000,270.0000) -- (149.9000,269.6000) -- (150.1000,269.4000) -- (150.3000,269.3000) -- (150.5000,269.3000) -- (150.7000,269.3000) -- (150.9000,269.3000) -- (151.1000,269.3000) -- (151.3000,269.3000) -- (151.5000,269.3000) -- (151.7000,269.3000) -- (151.9000,269.3000) -- (152.1000,269.2000) -- (152.3000,269.0000) -- (152.5000,269.2000) -- (152.7000,268.9000) -- (152.9000,267.6000) -- (153.1000,265.6000) -- (153.3000,263.0000) -- (153.5000,260.1000) -- (153.7000,257.1000) -- (153.8000,254.1000) -- (154.0000,251.1000) -- (154.2000,248.3000) -- (154.4000,245.7000) -- (154.6000,243.2000) -- (154.8000,241.0000) -- (155.0000,239.1000) -- (155.2000,237.1000) -- (155.4000,238.9000) -- (155.6000,242.0000) -- (155.8000,240.7000) -- (156.0000,239.8000) -- (156.2000,239.2000) -- (156.4000,238.8000) -- (156.6000,238.6000) -- (156.8000,238.6000) -- (157.0000,238.8000) -- (157.2000,238.9000) -- (157.4000,238.9000) -- (157.6000,238.9000) -- (157.8000,238.9000) -- (158.0000,238.9000) -- (158.2000,238.9000) -- (158.4000,238.9000) -- (158.6000,238.9000) -- (158.7000,238.9000) -- (158.9000,238.9000) -- (159.1000,238.9000) -- (159.3000,238.9000) -- (159.5000,238.9000) -- (159.7000,239.1000) -- (159.9000,239.5000) -- (160.1000,240.1000) -- (160.3000,240.9000) -- (160.5000,242.2000) -- (160.7000,241.5000) -- (160.9000,237.5000) -- (161.1000,239.2000) -- (161.3000,241.1000) -- (161.5000,243.1000) -- (161.7000,245.4000) -- (161.9000,247.8000) -- (162.1000,250.4000) -- (162.3000,253.0000) -- (162.5000,255.8000) -- (162.7000,258.6000) -- (162.9000,261.3000) -- (163.1000,263.8000) -- (163.3000,266.6000) -- (163.4000,268.9000) -- (163.6000,270.1000) -- (163.8000,270.1000) -- (164.0000,269.8000) -- (164.2000,269.5000) -- (164.4000,269.4000) -- (164.6000,269.3000) -- (164.8000,269.3000) -- (165.0000,269.3000) -- (165.2000,269.3000) -- (165.4000,269.3000) -- (165.6000,269.3000) -- (165.8000,269.3000) -- (166.0000,269.3000) -- (166.2000,269.3000) -- (166.4000,269.0000) -- (166.6000,269.1000) -- (166.8000,269.2000) -- (167.0000,268.5000) -- (167.2000,266.9000) -- (167.4000,264.6000) -- (167.6000,261.9000) -- (167.8000,258.9000) -- (168.0000,255.9000) -- (168.2000,252.9000) -- (168.3000,250.0000) -- (168.5000,246.9000) -- (168.7000,247.8000) -- (168.9000,249.7000) -- (169.1000,247.3000) -- (169.3000,245.3000) -- (169.5000,244.1000) -- (169.7000,248.5000) -- (169.9000,248.4000) -- (170.1000,247.3000) -- (170.3000,246.6000) -- (170.5000,246.1000) -- (170.7000,245.7000) -- (170.9000,245.6000) -- (171.1000,245.7000) -- (171.3000,245.8000) -- (171.5000,245.9000) -- (171.7000,245.9000) -- (171.9000,245.9000) -- (172.1000,245.9000) -- (172.3000,245.9000) -- (172.5000,245.9000) -- (172.7000,245.9000) -- (172.9000,245.9000) -- (173.0000,245.9000) -- (173.2000,245.9000) -- (173.4000,245.9000) -- (173.6000,245.9000) -- (173.8000,246.0000) -- (174.0000,246.2000) -- (174.2000,246.7000) -- (174.4000,247.4000) -- (174.6000,248.4000) -- (174.8000,249.6000) -- (175.0000,245.8000) -- (175.2000,245.0000) -- (175.4000,247.0000) -- (175.6000,249.1000) -- (175.8000,249.5000) -- (176.0000,246.3000) -- (176.2000,248.8000) -- (176.4000,251.4000) -- (176.6000,254.1000) -- (176.8000,256.9000) -- (177.0000,259.6000) -- (177.2000,262.3000) -- (177.4000,264.9000) -- (177.6000,267.7000) -- (177.7000,269.5000) -- (177.9000,270.2000) -- (178.1000,270.0000) -- (178.3000,269.7000) -- (178.5000,269.5000) -- (178.7000,269.3000) -- (178.9000,269.3000) -- (179.1000,269.3000) -- (179.3000,269.3000) -- (179.5000,269.3000) -- (179.7000,269.3000) -- (179.9000,269.3000) -- (180.1000,269.3000) -- (180.3000,269.3000) -- (180.5000,269.2000) -- (180.7000,269.0000) -- (180.9000,269.2000) -- (181.1000,269.0000) -- (181.3000,267.9000) -- (181.5000,266.1000) -- (181.7000,263.6000) -- (181.9000,260.7000) -- (182.1000,257.6000) -- (182.3000,255.1000) -- (182.4000,258.0000) -- (182.6000,255.9000) -- (182.8000,253.5000) -- (183.0000,256.9000) -- (183.2000,255.7000) -- (183.4000,253.5000) -- (183.6000,251.5000) -- (183.8000,253.1000) -- (184.0000,256.2000) -- (184.2000,254.9000) -- (184.4000,254.0000) -- (184.6000,253.4000) -- (184.8000,252.9000) -- (185.0000,252.7000) -- (185.2000,252.7000) -- (185.4000,252.8000) -- (185.6000,252.9000) -- (185.8000,252.9000) -- (186.0000,253.0000) -- (186.2000,252.9000) -- (186.4000,252.9000) -- (186.6000,252.9000) -- (186.8000,252.9000) -- (187.0000,252.9000) -- (187.2000,252.9000) -- (187.3000,252.9000) -- (187.5000,252.9000) -- (187.7000,252.9000) -- (187.9000,253.0000) -- (188.1000,253.1000) -- (188.3000,253.4000) -- (188.5000,254.0000) -- (188.7000,254.8000) -- (188.9000,256.0000) -- (189.1000,255.4000) -- (189.3000,251.3000) -- (189.5000,252.9000) -- (189.7000,254.8000) -- (189.9000,257.0000) -- (190.1000,254.2000) -- (190.3000,254.1000) -- (190.5000,257.1000) -- (190.7000,254.9000) -- (190.9000,255.0000) -- (191.1000,258.0000) -- (191.3000,260.7000) -- (191.5000,263.3000) -- (191.7000,266.0000) -- (191.9000,268.6000) -- (192.0000,269.9000) -- (192.2000,270.2000) -- (192.4000,269.9000) -- (192.6000,269.6000) -- (192.8000,269.4000) -- (193.0000,269.3000) -- (193.2000,269.3000) -- (193.4000,269.3000) -- (193.6000,269.3000) -- (193.8000,269.3000) -- (194.0000,269.3000) -- (194.2000,269.3000) -- (194.4000,269.3000) -- (194.6000,269.3000) -- (194.8000,269.1000) -- (195.0000,269.1000) -- (195.2000,269.2000) -- (195.4000,268.7000) -- (195.6000,267.3000) -- (195.8000,265.1000) -- (196.0000,262.4000) -- (196.2000,259.9000) -- (196.4000,262.5000) -- (196.6000,263.8000) -- (196.8000,264.9000) -- (196.9000,261.5000) -- (197.1000,262.2000) -- (197.3000,264.2000) -- (197.5000,261.7000) -- (197.7000,259.8000) -- (197.9000,258.4000) -- (198.1000,262.7000) -- (198.3000,262.7000) -- (198.5000,261.6000) -- (198.7000,260.8000) -- (198.9000,260.2000) -- (199.1000,259.8000) -- (199.3000,259.7000) -- (199.5000,259.8000) -- (199.7000,259.9000) -- (199.9000,260.0000) -- (200.1000,260.0000) -- (200.3000,260.0000) -- (200.5000,260.0000) -- (200.7000,260.0000) -- (200.9000,260.0000) -- (201.1000,260.0000) -- (201.3000,260.0000) -- (201.5000,260.0000) -- (201.6000,260.0000) -- (201.8000,260.0000) -- (202.0000,260.0000) -- (202.2000,260.0000) -- (202.4000,260.2000) -- (202.6000,260.7000) -- (202.8000,261.3000) -- (203.0000,262.2000) -- (203.2000,263.5000) -- (203.4000,259.7000) -- (203.6000,258.7000) -- (203.8000,260.7000) -- (204.0000,262.7000) -- (204.2000,263.3000) -- (204.4000,259.9000) -- (204.6000,262.5000) -- (204.8000,263.7000) -- (205.0000,260.7000) -- (205.2000,263.5000) -- (205.4000,265.0000) -- (205.6000,261.9000) -- (205.8000,264.3000) -- (206.0000,267.1000) -- (206.2000,269.3000) -- (206.3000,270.1000) -- (206.5000,270.1000) -- (206.7000,269.8000) -- (206.9000,269.5000) -- (207.1000,269.3000) -- (207.3000,269.3000) -- (207.5000,269.3000) -- (207.7000,269.3000) -- (207.9000,269.3000) -- (208.1000,269.3000) -- (208.3000,269.3000) -- (208.5000,269.3000) -- (208.7000,269.3000) -- (208.9000,269.3000) -- (209.1000,269.0000) -- (209.3000,269.2000) -- (209.5000,269.1000) -- (209.7000,268.2000) -- (209.9000,266.4000) -- (210.1000,264.5000) -- (210.3000,266.7000) -- (210.5000,274.9000) -- (210.7000,284.3000) -- (210.9000,280.4000) -- (211.1000,277.6000) -- (211.2000,274.8000) -- (211.4000,272.3000) -- (211.6000,270.0000) -- (211.8000,268.0000) -- (212.0000,266.2000) -- (212.2000,264.6000) -- (212.4000,263.3000) -- (212.6000,262.1000) -- (212.8000,261.2000) -- (213.0000,260.6000) -- (213.2000,260.1000) -- (213.4000,259.7000) -- (213.6000,259.7000) -- (213.8000,259.8000) -- (214.0000,259.9000) -- (214.2000,260.0000) -- (214.4000,260.0000) -- (214.6000,260.0000) -- (214.8000,260.0000) -- (215.0000,260.0000) -- (215.2000,260.0000) -- (215.4000,260.0000) -- (215.6000,260.0000) -- (215.8000,260.0000) -- (215.9000,260.0000) -- (216.1000,260.0000) -- (216.3000,260.0000) -- (216.5000,260.1000) -- (216.7000,260.4000) -- (216.9000,260.9000) -- (217.1000,261.7000) -- (217.3000,262.6000) -- (217.5000,263.8000) -- (217.7000,265.1000) -- (217.9000,266.7000) -- (218.1000,268.4000) -- (218.3000,270.6000) -- (218.5000,268.0000) -- (218.7000,267.6000) -- (218.9000,270.6000) -- (219.1000,268.6000) -- (219.3000,268.5000) -- (219.5000,271.8000) -- (219.7000,269.9000) -- (219.9000,269.5000) -- (220.1000,272.8000) -- (220.3000,271.0000) -- (220.5000,269.5000) -- (220.7000,270.2000) -- (220.8000,270.0000) -- (221.0000,269.6000) -- (221.2000,269.4000) -- (221.4000,269.3000) -- (221.6000,269.3000) -- (221.8000,269.3000) -- (222.0000,269.3000) -- (222.2000,269.3000) -- (222.4000,269.3000) -- (222.6000,269.2000) -- (222.8000,269.9000) -- (223.0000,275.8000) -- (223.2000,276.3000) -- (223.4000,276.1000) -- (223.6000,276.3000) -- (223.8000,275.9000) -- (224.0000,274.7000) -- (224.2000,271.8000) -- (224.4000,279.4000) -- (224.6000,289.1000) -- (224.8000,285.2000) -- (225.0000,282.2000) -- (225.2000,279.3000) -- (225.4000,276.4000) -- (225.5000,273.8000) -- (225.7000,271.4000) -- (225.9000,269.2000) -- (226.1000,267.2000) -- (226.3000,265.5000) -- (226.5000,264.0000) -- (226.7000,262.8000) -- (226.9000,261.8000) -- (227.1000,260.9000) -- (227.3000,260.3000) -- (227.5000,259.9000) -- (227.7000,259.7000) -- (227.9000,259.7000) -- (228.1000,259.9000) -- (228.3000,259.9000) -- (228.5000,260.0000) -- (228.7000,260.0000) -- (228.9000,260.0000) -- (229.1000,260.0000) -- (229.3000,260.0000) -- (229.5000,260.0000) -- (229.7000,260.0000) -- (229.9000,260.0000) -- (230.1000,260.0000) -- (230.3000,260.0000) -- (230.4000,260.0000) -- (230.6000,260.0000) -- (230.8000,260.2000) -- (231.0000,260.6000) -- (231.2000,261.2000) -- (231.4000,262.0000) -- (231.6000,263.0000) -- (231.8000,264.3000) -- (232.0000,265.7000) -- (232.2000,267.3000) -- (232.4000,269.2000) -- (232.6000,271.2000) -- (232.8000,273.5000) -- (233.0000,276.0000) -- (233.2000,277.3000) -- (233.4000,274.3000) -- (233.6000,277.0000) -- (233.8000,278.6000) -- (234.0000,275.5000) -- (234.2000,277.9000) -- (234.4000,279.7000) -- (234.6000,276.2000) -- (234.8000,277.0000) -- (234.9000,277.2000) -- (235.1000,276.9000) -- (235.3000,276.6000) -- (235.5000,276.4000) -- (235.7000,276.3000) -- (235.9000,276.3000) -- (236.1000,276.3000) -- (236.3000,276.4000) -- (236.5000,276.3000) -- (236.7000,276.9000) -- (236.9000,282.8000) -- (237.1000,283.5000) -- (237.3000,283.4000) -- (237.5000,283.1000) -- (237.7000,283.0000) -- (237.9000,284.0000) -- (238.1000,295.1000) -- (238.3000,295.2000) -- (238.5000,292.8000) -- (238.7000,290.0000) -- (238.9000,287.1000) -- (239.1000,284.0000) -- (239.3000,281.0000) -- (239.5000,278.1000) -- (239.7000,275.4000) -- (239.8000,272.8000) -- (240.0000,270.5000) -- (240.2000,268.4000) -- (240.4000,266.5000) -- (240.6000,264.9000) -- (240.8000,263.5000) -- (241.0000,262.3000) -- (241.2000,261.4000) -- (241.4000,260.7000) -- (241.6000,260.1000) -- (241.8000,259.8000) -- (242.0000,259.7000) -- (242.2000,259.8000) -- (242.4000,259.9000) -- (242.6000,260.0000) -- (242.8000,260.0000) -- (243.0000,260.0000) -- (243.2000,260.0000) -- (243.4000,260.0000) -- (243.6000,260.0000) -- (243.8000,260.0000) -- (244.0000,260.0000) -- (244.2000,260.0000) -- (244.4000,260.0000) -- (244.5000,260.0000) -- (244.7000,260.0000) -- (244.9000,260.0000) -- (245.1000,260.3000) -- (245.3000,260.8000) -- (245.5000,261.5000) -- (245.7000,262.4000) -- (245.9000,263.5000) -- (246.1000,264.8000) -- (246.3000,266.3000) -- (246.5000,268.0000) -- (246.7000,270.0000) -- (246.9000,272.1000) -- (247.1000,274.4000) -- (247.3000,276.9000) -- (247.5000,279.5000) -- (247.7000,282.2000) -- (247.9000,285.3000) -- (248.1000,283.6000) -- (248.3000,283.1000) -- (248.5000,286.3000) -- (248.7000,284.7000) -- (248.9000,283.3000) -- (249.1000,284.2000) -- (249.3000,284.1000) -- (249.4000,283.8000) -- (249.6000,283.5000) -- (249.8000,283.4000) -- (250.0000,283.4000) -- (250.2000,283.4000) -- (250.4000,283.4000) -- (250.6000,283.4000) -- (250.8000,283.1000) -- (251.0000,286.8000) -- (251.2000,290.7000) -- (251.4000,290.1000) -- (251.6000,293.6000) -- (251.8000,297.4000);



  \end{scope}
  \begin{scope}[scale=1.006,draw=blue,line cap=rect,line join=bevel,line width=0.800pt]
  \end{scope}
  \begin{scope}[scale=1.006,draw=blue,line cap=rect,line join=bevel,line width=0.800pt]
  \end{scope}
  \begin{scope}[scale=1.006,draw=blue,line cap=rect,line join=bevel,line width=0.800pt]
  \end{scope}
  \begin{scope}[scale=1.006,draw=blue,line cap=rect,line join=bevel,line width=0.800pt]
  \end{scope}
  \begin{scope}[cm={{1.04164,0.0,0.0,1.04164,(-342.4953,42.14671)}},draw=c00ff00,dash pattern=on 0.48pt off 0.48pt,line cap=round,line join=round,line width=0.480pt,miter limit=4.00]
    \path[draw,dash pattern=on 0.48pt off 0.48pt,line width=0.480pt,miter limit=4.00] (127.3000,221.1000) -- (127.6000,221.2000) -- (127.9000,221.4000) -- (128.2000,221.5000) -- (128.5000,221.7000) -- (128.8000,221.8000) -- (129.1000,222.0000) -- (129.4000,222.1000) -- (129.7000,222.3000) -- (130.0000,222.4000) -- (130.3000,222.6000) -- (130.6000,222.7000) -- (130.9000,222.9000) -- (131.2000,223.0000) -- (131.5000,223.2000) -- (131.8000,223.3000) -- (132.1000,223.5000) -- (132.3000,223.6000) -- (132.6000,223.8000) -- (132.9000,223.9000) -- (133.2000,224.1000) -- (133.5000,224.2000) -- (133.8000,224.4000) -- (134.1000,224.5000) -- (134.4000,224.7000) -- (134.7000,224.8000) -- (135.0000,225.0000) -- (135.3000,225.1000) -- (135.6000,225.2000) -- (135.9000,225.4000) -- (136.2000,225.5000) -- (136.5000,225.7000) -- (136.8000,225.8000) -- (137.1000,226.0000) -- (137.4000,226.1000) -- (137.7000,226.3000) -- (138.0000,226.4000) -- (138.3000,226.6000) -- (138.6000,226.7000) -- (138.9000,226.9000) -- (139.2000,227.0000) -- (139.4000,227.2000) -- (139.7000,227.3000) -- (140.0000,227.5000) -- (140.3000,227.6000) -- (140.6000,227.8000) -- (140.9000,227.9000) -- (141.2000,228.1000) -- (141.5000,228.2000) -- (141.8000,228.4000) -- (142.1000,228.5000) -- (142.4000,228.7000) -- (142.7000,228.8000) -- (143.0000,229.0000) -- (143.3000,229.1000) -- (143.6000,229.3000) -- (143.9000,229.4000) -- (144.2000,229.6000) -- (144.5000,229.7000) -- (144.8000,229.8000) -- (145.1000,230.0000) -- (145.4000,230.1000) -- (145.7000,230.3000) -- (146.0000,230.4000) -- (146.3000,230.6000) -- (146.6000,230.7000) -- (146.8000,230.9000) -- (147.1000,231.0000) -- (147.4000,231.2000) -- (147.7000,231.3000) -- (148.0000,231.5000) -- (148.3000,231.6000) -- (148.6000,231.8000) -- (148.9000,231.9000) -- (149.2000,232.1000) -- (149.5000,232.2000) -- (149.8000,232.4000) -- (150.1000,232.5000) -- (150.4000,232.7000) -- (150.7000,232.8000) -- (151.0000,233.0000) -- (151.3000,233.1000) -- (151.6000,233.3000) -- (151.9000,233.4000) -- (152.2000,233.6000) -- (152.5000,233.7000) -- (152.8000,233.9000) -- (153.1000,234.0000) -- (153.4000,234.1000) -- (153.7000,234.3000) -- (153.9000,234.4000) -- (154.2000,234.6000) -- (154.5000,234.7000) -- (154.8000,234.9000) -- (155.1000,235.0000) -- (155.4000,235.2000) -- (155.7000,235.3000) -- (156.0000,235.5000) -- (156.3000,235.6000) -- (156.6000,235.8000) -- (156.9000,235.9000) -- (157.2000,236.1000) -- (157.5000,236.2000) -- (157.8000,236.4000) -- (158.1000,236.5000) -- (158.4000,236.7000) -- (158.7000,236.8000) -- (159.0000,237.0000) -- (159.3000,237.1000) -- (159.6000,237.3000) -- (159.9000,237.4000) -- (160.2000,237.6000) -- (160.5000,237.7000) -- (160.8000,237.9000) -- (161.0000,238.0000) -- (161.3000,238.2000) -- (161.6000,238.3000) -- (161.9000,238.5000) -- (162.2000,238.6000) -- (162.5000,238.6000) -- (162.8000,238.7000) -- (163.1000,238.9000) -- (163.4000,239.0000) -- (163.7000,239.2000) -- (164.0000,239.3000) -- (164.3000,239.5000) -- (164.6000,239.6000) -- (164.9000,239.8000) -- (165.2000,239.9000) -- (165.5000,240.1000) -- (165.8000,240.2000) -- (166.1000,240.4000) -- (166.4000,240.5000) -- (166.7000,240.7000) -- (167.0000,240.8000) -- (167.3000,241.0000) -- (167.6000,241.1000) -- (167.9000,241.3000) -- (168.2000,241.4000) -- (168.4000,241.6000) -- (168.7000,241.7000) -- (169.0000,241.9000) -- (169.3000,242.0000) -- (169.6000,242.2000) -- (169.9000,242.3000) -- (170.2000,242.5000) -- (170.5000,242.6000) -- (170.8000,242.8000) -- (171.1000,242.9000) -- (171.4000,243.1000) -- (171.7000,243.2000) -- (172.0000,243.3000) -- (172.3000,243.5000) -- (172.6000,243.6000) -- (172.9000,243.8000) -- (173.2000,243.9000) -- (173.5000,244.1000) -- (173.8000,244.2000) -- (174.1000,244.4000) -- (174.4000,244.5000) -- (174.7000,244.7000) -- (175.0000,244.8000) -- (175.3000,245.0000) -- (175.5000,245.1000) -- (175.8000,245.3000) -- (176.1000,245.4000) -- (176.4000,245.6000) -- (176.7000,245.7000) -- (177.0000,245.9000) -- (177.3000,246.0000) -- (177.6000,246.2000) -- (177.9000,246.3000) -- (178.2000,246.5000) -- (178.5000,246.6000) -- (178.8000,246.8000) -- (179.1000,246.9000) -- (179.4000,247.1000) -- (179.7000,247.2000) -- (180.0000,247.4000) -- (180.3000,247.5000) -- (180.6000,247.6000) -- (180.9000,247.8000) -- (181.2000,247.9000) -- (181.5000,248.1000) -- (181.8000,248.2000) -- (182.1000,248.4000) -- (182.4000,248.5000) -- (182.7000,248.7000) -- (182.9000,248.8000) -- (183.2000,249.0000) -- (183.5000,249.1000) -- (183.8000,249.3000) -- (184.1000,249.4000) -- (184.4000,249.6000) -- (184.7000,249.7000) -- (185.0000,249.9000) -- (185.3000,250.0000) -- (185.6000,250.2000) -- (185.9000,250.3000) -- (186.2000,250.5000) -- (186.5000,250.6000) -- (186.8000,250.8000) -- (187.1000,250.9000) -- (187.4000,251.1000) -- (187.7000,251.2000) -- (188.0000,251.4000) -- (188.3000,251.5000) -- (188.6000,251.7000) -- (188.9000,251.8000) -- (189.2000,252.0000) -- (189.5000,252.1000) -- (189.8000,252.2000) -- (190.0000,252.4000) -- (190.3000,252.5000) -- (190.6000,252.7000) -- (190.9000,252.8000) -- (191.2000,253.0000) -- (191.5000,253.1000) -- (191.8000,253.3000) -- (192.1000,253.4000) -- (192.4000,253.6000) -- (192.7000,253.7000) -- (193.0000,253.9000) -- (193.3000,254.0000) -- (193.6000,254.2000) -- (193.9000,254.3000) -- (194.2000,254.5000) -- (194.5000,254.6000) -- (194.8000,254.8000) -- (195.1000,254.9000) -- (195.4000,255.1000) -- (195.7000,255.2000) -- (196.0000,255.4000) -- (196.3000,255.5000) -- (196.6000,255.7000) -- (196.9000,255.8000) -- (197.2000,256.0000) -- (197.4000,256.1000) -- (197.7000,256.3000) -- (198.0000,256.4000) -- (198.3000,256.5000) -- (198.6000,256.7000) -- (198.9000,256.8000) -- (199.2000,257.0000) -- (199.5000,257.1000) -- (199.8000,257.3000) -- (200.1000,257.4000) -- (200.4000,257.6000) -- (200.7000,257.7000) -- (201.0000,257.9000) -- (201.3000,258.0000) -- (201.6000,258.2000) -- (201.9000,258.3000) -- (202.2000,258.5000) -- (202.5000,258.6000) -- (202.8000,258.8000) -- (203.1000,258.9000) -- (203.4000,259.1000) -- (203.7000,259.2000) -- (204.0000,259.4000) -- (204.3000,259.5000) -- (204.5000,259.7000) -- (204.8000,259.8000) -- (205.1000,260.0000) -- (205.4000,260.1000) -- (205.7000,260.3000) -- (206.0000,260.4000) -- (206.3000,260.6000) -- (206.6000,260.7000) -- (206.9000,260.9000) -- (207.2000,261.0000) -- (207.5000,261.1000) -- (207.8000,261.3000) -- (208.1000,261.4000) -- (208.4000,261.6000) -- (208.7000,261.7000) -- (209.0000,261.9000) -- (209.3000,262.0000) -- (209.6000,262.2000) -- (209.9000,262.3000) -- (210.2000,262.5000) -- (210.5000,262.6000) -- (210.8000,262.8000) -- (211.1000,262.9000) -- (211.4000,263.1000) -- (211.7000,263.2000) -- (211.9000,263.4000) -- (212.2000,263.5000) -- (212.5000,263.7000) -- (212.8000,263.8000) -- (213.1000,264.0000) -- (213.4000,264.1000) -- (213.7000,264.3000) -- (214.0000,264.4000) -- (214.3000,264.6000) -- (214.6000,264.7000) -- (214.9000,264.9000) -- (215.2000,265.0000) -- (215.5000,265.2000) -- (215.8000,265.3000) -- (216.1000,265.5000) -- (216.4000,265.6000) -- (216.7000,265.7000) -- (217.0000,265.9000) -- (217.3000,266.0000) -- (217.6000,266.2000) -- (217.9000,266.3000) -- (218.2000,266.5000) -- (218.5000,266.6000) -- (218.8000,266.8000) -- (219.0000,266.9000) -- (219.3000,267.1000) -- (219.6000,267.2000) -- (219.9000,267.4000) -- (220.2000,267.5000) -- (220.5000,267.7000) -- (220.8000,267.8000) -- (221.1000,268.0000) -- (221.4000,268.1000) -- (221.7000,268.3000) -- (222.0000,268.4000) -- (222.3000,268.6000) -- (222.6000,268.7000) -- (222.9000,268.9000) -- (223.2000,269.0000) -- (223.5000,269.2000) -- (223.8000,269.3000) -- (224.1000,269.5000) -- (224.4000,269.6000) -- (224.7000,269.8000) -- (225.0000,269.9000) -- (225.3000,270.0000) -- (225.6000,270.2000) -- (225.9000,270.3000) -- (226.1000,270.5000) -- (226.4000,270.6000) -- (226.7000,270.8000) -- (227.0000,270.9000) -- (227.3000,271.1000) -- (227.6000,271.2000) -- (227.9000,271.4000) -- (228.2000,271.5000) -- (228.5000,271.7000) -- (228.8000,271.8000) -- (229.1000,272.0000) -- (229.4000,272.1000) -- (229.7000,272.3000) -- (230.0000,272.4000) -- (230.3000,272.6000) -- (230.6000,272.7000) -- (230.9000,272.9000) -- (231.2000,273.0000) -- (231.5000,273.2000) -- (231.8000,273.3000) -- (232.1000,273.5000) -- (232.4000,273.6000) -- (232.7000,273.8000) -- (233.0000,273.9000) -- (233.3000,274.1000) -- (233.5000,274.2000) -- (233.8000,274.4000) -- (234.1000,274.5000) -- (234.4000,274.6000) -- (234.7000,274.8000) -- (235.0000,274.9000) -- (235.3000,275.1000) -- (235.6000,275.2000) -- (235.9000,275.4000) -- (236.2000,275.5000) -- (236.5000,275.7000) -- (236.8000,275.8000) -- (237.1000,276.0000) -- (237.4000,276.1000) -- (237.7000,276.3000) -- (238.0000,276.4000) -- (238.3000,276.6000) -- (238.6000,276.7000) -- (238.9000,276.9000) -- (239.2000,277.0000) -- (239.5000,277.2000) -- (239.8000,277.3000) -- (240.1000,277.5000) -- (240.4000,277.6000) -- (240.6000,277.8000) -- (240.9000,277.9000) -- (241.2000,278.1000) -- (241.5000,278.2000) -- (241.8000,278.4000) -- (242.1000,278.5000) -- (242.4000,278.7000) -- (242.7000,278.8000) -- (243.0000,279.0000) -- (243.3000,279.1000) -- (243.6000,279.2000) -- (243.9000,279.4000) -- (244.2000,279.5000) -- (244.5000,279.7000) -- (244.8000,279.8000) -- (245.1000,280.0000) -- (245.4000,280.1000) -- (245.7000,280.3000) -- (246.0000,280.4000) -- (246.3000,280.6000) -- (246.6000,280.7000) -- (246.9000,280.9000) -- (247.2000,281.0000) -- (247.5000,281.2000) -- (247.8000,281.3000) -- (248.0000,281.5000) -- (248.3000,281.6000) -- (248.6000,281.8000) -- (248.9000,281.9000) -- (249.2000,282.1000) -- (249.5000,282.2000) -- (249.8000,282.4000) -- (250.1000,282.5000) -- (250.4000,282.7000) -- (250.7000,282.8000) -- (251.0000,283.0000) -- (251.3000,283.1000) -- (251.6000,283.3000) -- (251.9000,283.4000);



  \end{scope}
  \begin{scope}[scale=1.006,draw=blue,line cap=rect,line join=bevel,line width=0.800pt]
  \end{scope}
  \begin{scope}[draw=blue,line cap=rect,line join=bevel,line width=0.800pt]
  \end{scope}
  \begin{scope}[cm={{0.0,-1.00588,1.00588,0.0,(-600.28902,263.45539)}},draw=blue,line cap=rect,line join=bevel,line width=0.800pt]
    \path[fill=blue] (0.0000,0.0000) node[above right] (text344) {\rotatebox{90}{y (m)}};



  \end{scope}
  \begin{scope}[cm={{0.84173,0.0,0.0,0.84173,(-601.60573,125.64086)}},fill=cffffff]
  \end{scope}
  \begin{scope}[cm={{0.92047,0.0,0.0,0.92047,(-569.12952,57.84001)}},draw=ca0a0a4,dash pattern=on 1.96pt off 1.96pt,line cap=round,line join=round,line width=0.326pt,miter limit=4.00]
    \path[shift={(0,-5.96493)},draw,dash pattern=on 1.96pt off 1.96pt,line width=0.326pt,miter limit=4.00] (184.5000,151.5000) -- (246.5000,151.5000);



  \end{scope}
  \begin{scope}[cm={{0.92047,0.0,0.0,0.92047,(-569.12952,52.34945)}},draw=blue,line cap=round,line join=round,line width=0.480pt]
    \path[draw] (184.5000,151.5000) -- (186.5000,151.5000);



    \path[draw] (246.5000,151.5000) -- (245.5000,151.5000);



  \end{scope}
  \begin{scope}[cm={{0.92047,0.0,0.0,0.92047,(-569.12952,52.34945)}},draw=ca0a0a4,dash pattern=on 1.96pt off 1.96pt,line cap=round,line join=round,line width=0.326pt,miter limit=4.00]
    \path[draw,dash pattern=on 1.96pt off 1.96pt,line width=0.326pt,miter limit=4.00] (184.5000,132.5000) -- (246.5000,132.5000);



  \end{scope}
  \begin{scope}[cm={{0.92047,0.0,0.0,0.92047,(-569.12952,52.34945)}},draw=blue,line cap=round,line join=round,line width=0.480pt]
    \path[draw] (184.5000,132.5000) -- (186.5000,132.5000);



    \path[draw] (246.5000,132.5000) -- (245.5000,132.5000);



  \end{scope}
  \begin{scope}[cm={{0.92047,0.0,0.0,0.92047,(-569.12952,52.34945)}},draw=ca0a0a4,dash pattern=on 1.96pt off 1.96pt,line cap=round,line join=round,line width=0.326pt,miter limit=4.00]
    \path[draw,dash pattern=on 1.96pt off 1.96pt,line width=0.326pt,miter limit=4.00] (184.5000,113.5000) -- (246.5000,113.5000);



  \end{scope}
  \begin{scope}[cm={{0.92047,0.0,0.0,0.92047,(-569.12952,57.84001)}},draw=blue,line cap=round,line join=round,line width=0.480pt]
    \path[shift={(0,-5.96493)},draw] (184.5000,113.5000) -- (186.5000,113.5000);



    \path[shift={(0,-5.96493)},draw] (246.5000,113.5000) -- (245.5000,113.5000);



  \end{scope}
  \begin{scope}[cm={{0.92047,0.0,0.0,0.92047,(-569.12952,52.34945)}},draw=ca0a0a4,dash pattern=on 0.40pt off 0.80pt,line cap=round,line join=round,line width=0.400pt]
    \path[draw] (184.5000,156.5000) -- (184.5000,108.5000);



  \end{scope}
  \begin{scope}[cm={{0.92047,0.0,0.0,0.92047,(-569.12952,52.34945)}},draw=blue,line cap=round,line join=round,line width=0.480pt]
    \path[draw] (184.5000,156.5000) -- (184.5000,155.5000);



    \path[draw] (184.5000,108.5000) -- (184.5000,109.5000);



  \end{scope}
  \begin{scope}[cm={{0.92047,0.0,0.0,0.92047,(-569.12952,52.34945)}},draw=ca0a0a4,dash pattern=on 1.96pt off 1.96pt,line cap=round,line join=round,line width=0.326pt,miter limit=4.00]
    \path[draw,dash pattern=on 1.96pt off 1.96pt,line width=0.326pt,miter limit=4.00] (203.5000,156.5000) -- (203.5000,108.5000);



  \end{scope}
  \begin{scope}[cm={{0.92047,0.0,0.0,0.92047,(-569.12952,52.34945)}},draw=blue,line cap=round,line join=round,line width=0.480pt]
    \path[draw] (203.5000,156.5000) -- (203.5000,155.5000);



    \path[draw] (203.5000,108.5000) -- (203.5000,109.5000);



  \end{scope}
  \begin{scope}[cm={{0.92047,0.0,0.0,0.92047,(-569.12952,52.34945)}},draw=ca0a0a4,dash pattern=on 1.96pt off 1.96pt,line cap=round,line join=round,line width=0.326pt,miter limit=4.00]
    \path[draw,dash pattern=on 1.96pt off 1.96pt,line width=0.326pt,miter limit=4.00] (221.5000,156.5000) -- (221.5000,108.5000);



  \end{scope}
  \begin{scope}[cm={{0.92047,0.0,0.0,0.92047,(-569.12952,52.34945)}},draw=blue,line cap=round,line join=round,line width=0.480pt]
    \path[draw] (221.5000,156.5000) -- (221.5000,155.5000);



    \path[draw] (221.5000,108.5000) -- (221.5000,109.5000);



  \end{scope}
  \begin{scope}[cm={{0.92047,0.0,0.0,0.92047,(-569.12952,52.34945)}},draw=ca0a0a4,dash pattern=on 1.96pt off 1.96pt,line cap=round,line join=round,line width=0.326pt,miter limit=4.00]
    \path[draw,dash pattern=on 1.96pt off 1.96pt,line width=0.326pt,miter limit=4.00] (239.5000,156.5000) -- (239.5000,108.5000);



  \end{scope}
  \begin{scope}[cm={{0.92047,0.0,0.0,0.92047,(-569.12952,52.34945)}},draw=blue,line cap=round,line join=round,line width=0.480pt]
    \path[draw] (239.5000,156.5000) -- (239.5000,155.5000);



    \path[draw] (239.5000,108.5000) -- (239.5000,109.5000);



  \end{scope}
  \begin{scope}[cm={{0.92047,0.0,0.0,0.92047,(-569.12952,57.84001)}},draw=blue,line cap=round,line join=round,line width=0.480pt]
    \path[shift={(0,-5.96493)},draw] (246.5000,151.5000) -- (245.5000,151.5000);



  \end{scope}
  \begin{scope}[cm={{0.92047,0.0,0.0,0.92047,(-569.12952,57.84001)}},draw=blue,line cap=round,line join=round,line width=0.480pt]
    \path[shift={(0,-5.96493)},draw] (246.5000,132.5000) -- (245.5000,132.5000);



  \end{scope}
  \begin{scope}[cm={{0.92047,0.0,0.0,0.92047,(-569.12952,52.34945)}},draw=blue,line cap=round,line join=round,line width=0.480pt]
    \path[draw] (246.5000,113.5000) -- (245.5000,113.5000);



  \end{scope}
  \begin{scope}[cm={{0.92047,0.0,0.0,0.92047,(-569.12952,52.34945)}},draw=blue,line cap=round,line join=round,line width=0.480pt]
    \path[draw] (184.5000,108.5000) -- (184.5000,156.5000) -- (246.5000,156.5000) -- (246.5000,108.5000) -- (184.5000,108.5000);



  \end{scope}
  \begin{scope}[cm={{0.92047,0.0,0.0,0.92047,(-569.12952,52.34945)}},draw=blue,line cap=round,line join=round,line width=0.480pt]
    \path[draw] (184.8000,113.6000) -- (184.8000,113.6000) -- (184.9000,113.6000) -- (185.0000,113.6000) -- (185.0000,113.6000) -- (185.1000,113.6000) -- (185.1000,113.6000) -- (185.2000,113.6000) -- (185.3000,113.6000) -- (185.3000,113.6000) -- (185.4000,113.6000) -- (185.4000,113.6000) -- (185.5000,113.6000) -- (185.6000,113.6000) -- (185.6000,113.6000) -- (185.7000,113.6000) -- (185.8000,113.6000) -- (185.8000,113.6000) -- (185.9000,113.6000) -- (185.9000,113.6000) -- (186.0000,113.6000) -- (186.1000,113.6000) -- (186.1000,113.6000) -- (186.2000,113.6000) -- (186.2000,113.6000) -- (186.3000,113.6000) -- (186.4000,113.6000) -- (186.4000,113.6000) -- (186.5000,113.6000) -- (186.6000,113.6000) -- (186.6000,113.6000) -- (186.7000,113.6000) -- (186.7000,113.6000) -- (186.8000,113.6000) -- (186.9000,113.6000) -- (186.9000,113.6000) -- (187.0000,113.6000) -- (187.0000,113.6000) -- (187.1000,113.6000) -- (187.2000,113.6000) -- (187.2000,113.6000) -- (187.3000,113.6000) -- (187.4000,113.6000) -- (187.4000,113.6000) -- (187.5000,113.6000) -- (187.5000,113.6000) -- (187.6000,113.6000) -- (187.7000,113.6000) -- (187.7000,113.6000) -- (187.8000,113.6000) -- (187.8000,113.6000) -- (187.9000,113.6000) -- (188.0000,113.6000) -- (188.0000,113.6000) -- (188.1000,113.6000) -- (188.2000,113.6000) -- (188.2000,113.6000) -- (188.3000,113.6000) -- (188.3000,113.6000) -- (188.4000,113.6000) -- (188.5000,113.6000) -- (188.5000,113.6000) -- (188.6000,113.6000) -- (188.6000,113.6000) -- (188.7000,113.6000) -- (188.8000,113.6000) -- (188.8000,113.6000) -- (188.9000,113.6000) -- (189.0000,113.6000) -- (189.0000,113.6000) -- (189.1000,113.6000) -- (189.1000,113.6000) -- (189.2000,113.6000) -- (189.3000,113.6000) -- (189.3000,113.6000) -- (189.4000,113.6000) -- (189.4000,113.6000) -- (189.5000,113.6000) -- (189.6000,113.6000) -- (189.6000,113.6000) -- (189.7000,113.6000) -- (189.8000,113.6000) -- (189.8000,113.6000) -- (189.9000,113.6000) -- (189.9000,113.6000) -- (190.0000,113.6000) -- (190.1000,113.6000) -- (190.1000,113.6000) -- (190.2000,113.6000) -- (190.2000,113.6000) -- (190.3000,113.6000) -- (190.4000,113.6000) -- (190.4000,113.6000) -- (190.5000,113.6000) -- (190.6000,113.6000) -- (190.6000,113.6000) -- (190.7000,113.6000) -- (190.7000,113.6000) -- (190.8000,113.6000) -- (190.9000,113.6000) -- (190.9000,113.6000) -- (191.0000,113.6000) -- (191.0000,113.6000) -- (191.1000,113.6000) -- (191.2000,113.6000) -- (191.2000,113.6000) -- (191.3000,113.6000) -- (191.4000,113.6000) -- (191.4000,113.6000) -- (191.5000,113.6000) -- (191.5000,113.6000) -- (191.6000,113.6000) -- (191.7000,113.6000) -- (191.7000,113.6000) -- (191.8000,113.6000) -- (191.8000,113.6000) -- (191.9000,113.6000) -- (192.0000,113.6000) -- (192.0000,113.6000) -- (192.1000,113.6000) -- (192.2000,113.6000) -- (192.2000,113.6000) -- (192.3000,113.6000) -- (192.3000,113.6000) -- (192.4000,113.6000) -- (192.5000,113.6000) -- (192.5000,113.6000) -- (192.6000,113.6000) -- (192.6000,113.6000) -- (192.7000,113.6000) -- (192.8000,113.6000) -- (192.8000,113.6000) -- (192.9000,113.6000) -- (193.0000,113.6000) -- (193.0000,113.6000) -- (193.1000,113.6000) -- (193.1000,113.6000) -- (193.2000,113.6000) -- (193.3000,113.6000) -- (193.3000,113.6000) -- (193.4000,113.6000) -- (193.4000,113.6000) -- (193.5000,113.6000) -- (193.6000,113.6000) -- (193.6000,113.6000) -- (193.7000,113.6000) -- (193.8000,113.6000) -- (193.8000,113.6000) -- (193.9000,113.6000) -- (193.9000,113.6000) -- (194.0000,113.6000) -- (194.1000,113.6000) -- (194.1000,113.6000) -- (194.2000,113.6000) -- (194.2000,113.6000) -- (194.3000,113.6000) -- (194.4000,113.6000) -- (194.4000,113.6000) -- (194.5000,113.6000) -- (194.6000,113.6000) -- (194.6000,113.6000) -- (194.7000,113.6000) -- (194.7000,113.6000) -- (194.8000,113.6000) -- (194.9000,113.6000) -- (194.9000,113.6000) -- (195.0000,113.6000) -- (195.0000,113.6000) -- (195.1000,113.6000) -- (195.2000,113.6000) -- (195.2000,113.6000) -- (195.3000,113.6000) -- (195.4000,113.6000) -- (195.4000,113.6000) -- (195.5000,113.6000) -- (195.5000,113.6000) -- (195.6000,113.6000) -- (195.7000,113.6000) -- (195.7000,113.6000) -- (195.8000,113.6000) -- (195.8000,113.6000) -- (195.9000,113.6000) -- (196.0000,113.6000) -- (196.0000,113.6000) -- (196.1000,113.6000) -- (196.2000,113.6000) -- (196.2000,113.6000) -- (196.3000,113.6000) -- (196.3000,113.6000) -- (196.4000,113.6000) -- (196.5000,113.6000) -- (196.5000,113.6000) -- (196.6000,113.6000) -- (196.6000,113.6000) -- (196.7000,113.6000) -- (196.8000,113.6000) -- (196.8000,113.6000) -- (196.9000,113.6000) -- (196.9000,113.6000) -- (197.0000,113.6000) -- (197.1000,113.6000) -- (197.1000,113.6000) -- (197.2000,113.6000) -- (197.3000,113.6000) -- (197.3000,113.6000) -- (197.4000,113.6000) -- (197.4000,113.6000) -- (197.5000,113.6000) -- (197.6000,113.6000) -- (197.6000,113.6000) -- (197.7000,113.6000) -- (197.7000,113.6000) -- (197.8000,113.6000) -- (197.9000,113.6000) -- (197.9000,113.6000) -- (198.0000,113.6000) -- (198.1000,113.6000) -- (198.1000,113.6000) -- (198.2000,113.6000) -- (198.2000,113.6000) -- (198.3000,113.6000) -- (198.4000,113.6000) -- (198.4000,113.6000) -- (198.5000,113.6000) -- (198.5000,113.6000) -- (198.6000,113.6000) -- (198.7000,113.6000) -- (198.7000,113.6000) -- (198.8000,113.6000) -- (198.9000,113.6000) -- (198.9000,113.6000) -- (199.0000,113.6000) -- (199.0000,113.6000) -- (199.1000,113.6000) -- (199.2000,113.6000) -- (199.2000,113.6000) -- (199.3000,113.6000) -- (199.3000,113.6000) -- (199.4000,113.6000) -- (199.5000,113.6000) -- (199.5000,113.6000) -- (199.6000,113.6000) -- (199.7000,113.6000) -- (199.7000,113.6000) -- (199.8000,113.6000) -- (199.8000,113.6000) -- (199.9000,113.6000) -- (200.0000,113.6000) -- (200.0000,113.6000) -- (200.1000,113.6000) -- (200.1000,113.6000) -- (200.2000,113.6000) -- (200.3000,113.6000) -- (200.3000,113.6000) -- (200.4000,113.6000) -- (200.5000,113.6000) -- (200.5000,113.6000) -- (200.6000,113.6000) -- (200.6000,113.6000) -- (200.7000,113.6000) -- (200.8000,113.6000) -- (200.8000,113.6000) -- (200.9000,113.6000) -- (200.9000,113.6000) -- (201.0000,113.6000) -- (201.1000,113.6000) -- (201.1000,113.6000) -- (201.2000,113.6000) -- (201.3000,113.6000) -- (201.3000,113.6000) -- (201.4000,113.6000) -- (201.4000,113.6000) -- (201.5000,113.6000) -- (201.6000,113.6000) -- (201.6000,113.6000) -- (201.7000,113.6000) -- (201.7000,113.6000) -- (201.8000,113.6000) -- (201.9000,113.6000) -- (201.9000,113.6000) -- (202.0000,113.6000) -- (202.1000,113.6000) -- (202.1000,113.6000) -- (202.2000,113.6000) -- (202.2000,113.6000) -- (202.3000,113.6000) -- (202.4000,113.6000) -- (202.4000,113.6000) -- (202.5000,113.6000) -- (202.5000,113.6000) -- (202.6000,113.6000) -- (202.7000,113.6000) -- (202.7000,113.6000) -- (202.8000,113.6000) -- (202.9000,113.6000) -- (202.9000,113.6000) -- (203.0000,113.6000) -- (203.0000,113.6000) -- (203.1000,113.6000) -- (203.2000,113.6000) -- (203.2000,113.8000) -- (203.3000,121.5000) -- (203.3000,123.4000) -- (203.4000,123.1000) -- (203.5000,123.1000) -- (203.5000,123.1000) -- (203.6000,123.1000) -- (203.7000,123.1000) -- (203.7000,123.1000) -- (203.8000,123.1000) -- (203.8000,123.1000) -- (203.9000,123.1000) -- (204.0000,123.1000) -- (204.0000,123.1000) -- (204.1000,123.1000) -- (204.1000,123.1000) -- (204.2000,123.1000) -- (204.3000,123.1000) -- (204.3000,123.1000) -- (204.4000,123.1000) -- (204.5000,123.1000) -- (204.5000,123.1000) -- (204.6000,123.1000) -- (204.6000,123.1000) -- (204.7000,123.1000) -- (204.8000,123.1000) -- (204.8000,123.1000) -- (204.9000,122.8000) -- (204.9000,128.6000) -- (205.0000,132.8000) -- (205.1000,134.0000) -- (205.1000,141.8000) -- (205.2000,142.2000) -- (205.3000,142.1000) -- (205.3000,142.1000) -- (205.4000,142.1000) -- (205.4000,142.1000) -- (205.5000,141.8000) -- (205.6000,148.4000) -- (205.6000,152.0000) -- (205.7000,151.6000) -- (205.7000,151.6000) -- (205.8000,151.6000) -- (205.9000,151.6000) -- (205.9000,151.6000) -- (206.0000,151.6000) -- (206.1000,151.6000) -- (206.1000,151.6000) -- (206.2000,151.6000) -- (206.2000,151.9000) -- (206.3000,149.1000) -- (206.4000,142.0000) -- (206.4000,142.1000) -- (206.5000,142.1000) -- (206.5000,142.1000) -- (206.6000,142.1000) -- (206.7000,142.2000) -- (206.7000,142.2000) -- (206.8000,134.6000) -- (206.9000,132.4000) -- (206.9000,132.6000) -- (207.0000,132.6000) -- (207.0000,132.8000) -- (207.1000,140.5000) -- (207.2000,142.4000) -- (207.2000,142.1000) -- (207.3000,142.1000) -- (207.3000,142.1000) -- (207.4000,142.1000) -- (207.5000,142.1000) -- (207.5000,142.1000) -- (207.6000,142.1000) -- (207.7000,142.1000) -- (207.7000,142.1000) -- (207.8000,142.1000) -- (207.8000,142.1000) -- (207.9000,142.1000) -- (208.0000,142.1000) -- (208.0000,142.1000) -- (208.1000,142.1000) -- (208.1000,142.1000) -- (208.2000,142.1000) -- (208.3000,142.1000) -- (208.3000,142.1000) -- (208.4000,142.5000) -- (208.5000,137.1000) -- (208.5000,132.4000) -- (208.6000,131.6000) -- (208.6000,123.7000) -- (208.7000,123.1000) -- (208.8000,123.1000) -- (208.8000,123.3000) -- (208.9000,121.9000) -- (208.9000,114.0000) -- (209.0000,113.6000) -- (209.1000,113.6000) -- (209.1000,113.6000) -- (209.2000,113.6000) -- (209.3000,113.6000) -- (209.3000,113.6000) -- (209.4000,113.6000) -- (209.4000,113.4000) -- (209.5000,115.5000) -- (209.6000,123.1000) -- (209.6000,123.2000) -- (209.7000,123.1000) -- (209.7000,122.9000) -- (209.8000,125.3000) -- (209.9000,132.7000) -- (209.9000,132.5000) -- (210.0000,139.4000) -- (210.1000,142.5000) -- (210.1000,142.1000) -- (210.2000,142.1000) -- (210.2000,142.0000) -- (210.3000,149.3000) -- (210.4000,151.9000) -- (210.4000,151.6000) -- (210.5000,151.6000) -- (210.5000,151.6000) -- (210.6000,151.6000) -- (210.7000,151.6000) -- (210.7000,151.6000) -- (210.8000,151.6000) -- (210.9000,151.6000) -- (210.9000,151.6000) -- (211.0000,151.6000) -- (211.0000,151.6000) -- (211.1000,151.6000) -- (211.2000,151.6000) -- (211.2000,151.6000) -- (211.3000,151.6000) -- (211.3000,151.6000) -- (211.4000,151.6000) -- (211.5000,151.6000) -- (211.5000,151.6000) -- (211.6000,151.6000) -- (211.7000,151.6000) -- (211.7000,151.6000) -- (211.8000,151.6000) -- (211.8000,151.6000) -- (211.9000,151.6000) -- (212.0000,151.6000) -- (212.0000,151.6000) -- (212.1000,151.6000) -- (212.1000,151.6000) -- (212.2000,151.6000) -- (212.3000,151.6000) -- (212.3000,151.6000) -- (212.4000,151.6000) -- (212.5000,151.6000) -- (212.5000,151.6000) -- (212.6000,151.6000) -- (212.6000,151.6000) -- (212.7000,151.6000) -- (212.8000,151.6000) -- (212.8000,151.6000) -- (212.9000,151.6000) -- (212.9000,151.6000) -- (213.0000,151.6000) -- (213.1000,151.6000) -- (213.1000,151.6000) -- (213.2000,151.6000) -- (213.3000,151.6000) -- (213.3000,151.6000) -- (213.4000,151.6000) -- (213.4000,151.6000) -- (213.5000,151.6000) -- (213.6000,151.6000) -- (213.6000,151.6000) -- (213.7000,151.6000) -- (213.7000,151.6000) -- (213.8000,151.6000) -- (213.9000,151.6000) -- (213.9000,151.6000) -- (214.0000,151.6000) -- (214.1000,151.6000) -- (214.1000,151.6000) -- (214.2000,151.6000) -- (214.2000,151.6000) -- (214.3000,151.6000) -- (214.4000,151.6000) -- (214.4000,151.6000) -- (214.5000,151.6000) -- (214.5000,151.6000) -- (214.6000,151.6000) -- (214.7000,151.6000) -- (214.7000,151.6000) -- (214.8000,151.6000) -- (214.9000,151.6000) -- (214.9000,151.6000) -- (215.0000,151.6000) -- (215.0000,151.6000) -- (215.1000,151.6000) -- (215.2000,151.6000) -- (215.2000,151.6000) -- (215.3000,151.6000) -- (215.3000,151.6000) -- (215.4000,151.6000) -- (215.5000,151.6000) -- (215.5000,151.6000) -- (215.6000,151.6000) -- (215.7000,151.6000) -- (215.7000,151.6000) -- (215.8000,151.6000) -- (215.8000,151.6000) -- (215.9000,151.6000) -- (216.0000,151.6000) -- (216.0000,151.6000) -- (216.1000,151.6000) -- (216.1000,151.6000) -- (216.2000,151.6000) -- (216.3000,151.6000) -- (216.3000,151.6000) -- (216.4000,151.6000) -- (216.5000,151.6000) -- (216.5000,151.6000) -- (216.6000,151.6000) -- (216.6000,151.6000) -- (216.7000,151.6000) -- (216.8000,151.6000) -- (216.8000,151.6000) -- (216.9000,151.6000) -- (216.9000,151.6000) -- (217.0000,151.6000) -- (217.1000,151.6000) -- (217.1000,151.6000) -- (217.2000,151.6000) -- (217.3000,151.6000) -- (217.3000,151.6000) -- (217.4000,151.6000) -- (217.4000,151.6000) -- (217.5000,151.6000) -- (217.6000,151.6000) -- (217.6000,151.6000) -- (217.7000,151.6000) -- (217.7000,151.6000) -- (217.8000,151.6000) -- (217.9000,151.6000) -- (217.9000,151.6000) -- (218.0000,151.6000) -- (218.1000,151.6000) -- (218.1000,151.6000) -- (218.2000,151.6000) -- (218.2000,151.6000) -- (218.3000,151.6000) -- (218.4000,151.6000) -- (218.4000,151.6000) -- (218.5000,151.6000) -- (218.5000,151.6000) -- (218.6000,151.6000) -- (218.7000,151.6000) -- (218.7000,151.6000) -- (218.8000,151.6000) -- (218.9000,151.6000) -- (218.9000,151.6000) -- (219.0000,151.6000) -- (219.0000,151.6000) -- (219.1000,151.6000) -- (219.2000,151.6000) -- (219.2000,151.6000) -- (219.3000,151.6000) -- (219.3000,151.6000) -- (219.4000,151.6000) -- (219.5000,151.6000) -- (219.5000,151.6000) -- (219.6000,151.6000) -- (219.7000,151.6000) -- (219.7000,151.6000) -- (219.8000,151.6000) -- (219.8000,151.6000) -- (219.9000,151.6000) -- (220.0000,151.6000) -- (220.0000,151.6000) -- (220.1000,151.6000) -- (220.1000,151.6000) -- (220.2000,151.6000) -- (220.3000,151.6000) -- (220.3000,151.6000) -- (220.4000,151.6000) -- (220.5000,151.6000) -- (220.5000,151.6000) -- (220.6000,151.6000) -- (220.6000,151.6000) -- (220.7000,151.6000) -- (220.8000,151.6000) -- (220.8000,151.6000) -- (220.9000,151.6000) -- (220.9000,151.6000) -- (221.0000,151.6000) -- (221.1000,151.6000) -- (221.1000,151.6000) -- (221.2000,151.6000) -- (221.3000,151.6000) -- (221.3000,151.6000) -- (221.4000,151.6000) -- (221.4000,151.6000) -- (221.5000,151.6000) -- (221.6000,151.6000) -- (221.6000,151.6000) -- (221.7000,151.6000) -- (221.7000,151.6000) -- (221.8000,151.6000) -- (221.9000,151.6000) -- (221.9000,151.6000) -- (222.0000,151.6000) -- (222.1000,151.6000) -- (222.1000,151.6000) -- (222.2000,151.6000) -- (222.2000,151.6000) -- (222.3000,151.6000) -- (222.4000,151.6000) -- (222.4000,151.6000) -- (222.5000,151.6000) -- (222.5000,151.6000) -- (222.6000,151.6000) -- (222.7000,151.6000) -- (222.7000,151.6000) -- (222.8000,151.6000) -- (222.9000,151.6000) -- (222.9000,151.6000) -- (223.0000,151.6000) -- (223.0000,151.6000) -- (223.1000,151.6000) -- (223.2000,151.6000) -- (223.2000,151.6000) -- (223.3000,151.6000) -- (223.3000,151.6000) -- (223.4000,151.6000) -- (223.5000,151.6000) -- (223.5000,151.6000) -- (223.6000,151.6000) -- (223.7000,151.6000) -- (223.7000,151.6000) -- (223.8000,151.6000) -- (223.8000,151.6000) -- (223.9000,151.6000) -- (224.0000,151.6000) -- (224.0000,151.6000) -- (224.1000,151.6000) -- (224.1000,151.6000) -- (224.2000,151.6000) -- (224.3000,151.6000) -- (224.3000,151.6000) -- (224.4000,151.6000) -- (224.5000,151.6000) -- (224.5000,151.6000) -- (224.6000,151.6000) -- (224.6000,151.6000) -- (224.7000,151.6000) -- (224.8000,151.6000) -- (224.8000,151.6000) -- (224.9000,151.6000) -- (224.9000,151.6000) -- (225.0000,151.6000) -- (225.1000,151.6000) -- (225.1000,151.6000) -- (225.2000,151.6000) -- (225.3000,151.6000) -- (225.3000,151.6000) -- (225.4000,151.6000) -- (225.4000,151.6000) -- (225.5000,151.6000) -- (225.6000,151.6000) -- (225.6000,151.6000) -- (225.7000,151.6000) -- (225.7000,151.6000) -- (225.8000,151.6000) -- (225.9000,151.6000) -- (225.9000,151.6000) -- (226.0000,151.6000) -- (226.0000,151.6000) -- (226.1000,151.6000) -- (226.2000,151.6000) -- (226.2000,151.6000) -- (226.3000,151.6000) -- (226.4000,151.6000) -- (226.4000,151.6000) -- (226.5000,151.6000) -- (226.5000,151.6000) -- (226.6000,151.6000) -- (226.7000,151.6000) -- (226.7000,151.6000) -- (226.8000,151.6000) -- (226.8000,151.6000) -- (226.9000,151.6000) -- (227.0000,151.6000) -- (227.0000,151.6000) -- (227.1000,151.6000) -- (227.2000,151.6000) -- (227.2000,151.6000) -- (227.3000,151.6000) -- (227.3000,151.6000) -- (227.4000,151.6000) -- (227.5000,151.6000) -- (227.5000,151.6000) -- (227.6000,151.6000) -- (227.7000,151.6000) -- (227.7000,151.6000) -- (227.8000,151.6000) -- (227.8000,151.6000) -- (227.9000,151.6000) -- (228.0000,151.6000) -- (228.0000,151.6000) -- (228.1000,151.6000) -- (228.1000,151.6000) -- (228.2000,151.6000) -- (228.3000,151.6000) -- (228.3000,151.6000) -- (228.4000,151.6000) -- (228.5000,151.6000) -- (228.5000,151.6000) -- (228.6000,151.6000) -- (228.6000,151.6000) -- (228.7000,151.6000) -- (228.8000,151.6000) -- (228.8000,151.6000) -- (228.9000,151.6000) -- (228.9000,151.6000) -- (229.0000,151.6000) -- (229.1000,151.6000) -- (229.1000,151.6000) -- (229.2000,151.6000) -- (229.2000,151.6000) -- (229.3000,151.6000) -- (229.4000,151.6000) -- (229.4000,151.6000) -- (229.5000,151.6000) -- (229.6000,151.6000) -- (229.6000,151.6000) -- (229.7000,151.6000) -- (229.7000,151.6000) -- (229.8000,151.6000) -- (229.9000,151.6000) -- (229.9000,151.6000) -- (230.0000,151.6000) -- (230.0000,151.6000) -- (230.1000,151.6000) -- (230.2000,151.6000) -- (230.2000,151.6000) -- (230.3000,151.6000) -- (230.4000,151.6000) -- (230.4000,151.6000) -- (230.5000,151.6000) -- (230.5000,151.6000) -- (230.6000,151.6000) -- (230.7000,151.6000) -- (230.7000,151.6000) -- (230.8000,151.6000) -- (230.8000,151.6000) -- (230.9000,151.6000) -- (231.0000,151.6000) -- (231.0000,151.6000) -- (231.1000,151.6000) -- (231.2000,151.6000) -- (231.2000,151.6000) -- (231.3000,151.6000) -- (231.3000,151.6000) -- (231.4000,151.6000) -- (231.5000,151.6000) -- (231.5000,151.6000) -- (231.6000,151.6000) -- (231.6000,151.6000) -- (231.7000,151.6000) -- (231.8000,151.6000) -- (231.8000,151.6000) -- (231.9000,151.6000) -- (232.0000,151.6000) -- (232.0000,151.6000) -- (232.1000,151.6000) -- (232.1000,151.6000) -- (232.2000,151.6000) -- (232.3000,151.6000) -- (232.3000,151.6000) -- (232.4000,151.6000) -- (232.4000,151.6000) -- (232.5000,151.6000) -- (232.6000,151.6000) -- (232.6000,151.6000) -- (232.7000,151.6000) -- (232.8000,151.6000) -- (232.8000,151.6000) -- (232.9000,151.6000) -- (232.9000,151.6000) -- (233.0000,151.6000) -- (233.1000,151.6000) -- (233.1000,151.6000) -- (233.2000,151.6000) -- (233.2000,151.6000) -- (233.3000,151.6000) -- (233.4000,151.6000) -- (233.4000,151.6000) -- (233.5000,151.6000) -- (233.6000,151.6000) -- (233.6000,151.6000) -- (233.7000,151.6000) -- (233.7000,151.6000) -- (233.8000,151.6000) -- (233.9000,151.6000) -- (233.9000,151.6000) -- (234.0000,151.6000) -- (234.0000,151.6000) -- (234.1000,151.6000) -- (234.2000,151.6000) -- (234.2000,151.6000) -- (234.3000,151.6000) -- (234.4000,151.6000) -- (234.4000,151.6000) -- (234.5000,151.6000) -- (234.5000,151.6000) -- (234.6000,151.6000) -- (234.7000,151.6000) -- (234.7000,151.6000) -- (234.8000,151.6000) -- (234.8000,151.6000) -- (234.9000,151.6000) -- (235.0000,151.6000) -- (235.0000,151.6000) -- (235.1000,151.6000) -- (235.2000,151.6000) -- (235.2000,151.6000) -- (235.3000,151.6000) -- (235.3000,151.6000) -- (235.4000,151.6000) -- (235.5000,151.6000) -- (235.5000,151.6000) -- (235.6000,151.6000) -- (235.6000,151.6000) -- (235.7000,151.6000) -- (235.8000,151.6000) -- (235.8000,151.6000) -- (235.9000,151.6000) -- (236.0000,151.6000) -- (236.0000,151.6000) -- (236.1000,151.6000) -- (236.1000,151.6000) -- (236.2000,151.6000) -- (236.3000,151.6000) -- (236.3000,151.6000) -- (236.4000,151.6000) -- (236.4000,151.6000) -- (236.5000,151.6000) -- (236.6000,151.6000) -- (236.6000,151.6000) -- (236.7000,151.6000) -- (236.8000,151.6000) -- (236.8000,151.6000) -- (236.9000,151.6000) -- (236.9000,151.6000) -- (237.0000,151.6000) -- (237.1000,151.6000) -- (237.1000,151.6000) -- (237.2000,151.6000) -- (237.2000,151.6000) -- (237.3000,151.6000) -- (237.4000,151.6000) -- (237.4000,151.6000) -- (237.5000,151.6000) -- (237.6000,151.6000) -- (237.6000,151.6000) -- (237.7000,151.6000) -- (237.7000,151.6000) -- (237.8000,151.6000) -- (237.9000,151.6000) -- (237.9000,151.6000) -- (238.0000,151.6000) -- (238.0000,151.6000) -- (238.1000,151.6000) -- (238.2000,151.6000) -- (238.2000,151.6000) -- (238.3000,151.6000) -- (238.4000,151.6000) -- (238.4000,151.6000) -- (238.5000,151.6000) -- (238.5000,151.6000) -- (238.6000,151.6000) -- (238.7000,151.6000) -- (238.7000,151.6000) -- (238.8000,151.6000) -- (238.8000,151.6000) -- (238.9000,151.6000) -- (239.0000,151.6000) -- (239.0000,151.6000) -- (239.1000,151.6000) -- (239.2000,151.6000) -- (239.2000,151.6000) -- (239.3000,151.6000) -- (239.3000,151.6000) -- (239.4000,151.6000) -- (239.5000,151.6000) -- (239.5000,151.6000) -- (239.6000,151.6000) -- (239.6000,151.6000) -- (239.7000,151.6000) -- (239.8000,151.6000) -- (239.8000,151.6000) -- (239.9000,151.6000) -- (240.0000,151.6000) -- (240.0000,151.6000) -- (240.1000,151.6000) -- (240.1000,151.6000) -- (240.2000,151.6000) -- (240.3000,151.6000) -- (240.3000,151.6000) -- (240.4000,151.6000) -- (240.4000,151.6000) -- (240.5000,151.6000) -- (240.6000,151.6000) -- (240.6000,151.6000) -- (240.7000,151.6000) -- (240.8000,151.6000) -- (240.8000,151.6000) -- (240.9000,151.6000) -- (240.9000,151.6000) -- (241.0000,151.6000) -- (241.1000,151.6000) -- (241.1000,151.6000) -- (241.2000,151.6000) -- (241.2000,151.6000) -- (241.3000,151.6000) -- (241.4000,151.6000) -- (241.4000,151.6000) -- (241.5000,151.6000) -- (241.6000,151.6000) -- (241.6000,151.6000) -- (241.7000,151.6000) -- (241.7000,151.6000) -- (241.8000,151.6000) -- (241.9000,151.6000) -- (241.9000,151.6000) -- (242.0000,151.6000) -- (242.0000,151.6000) -- (242.1000,151.6000) -- (242.2000,151.6000) -- (242.2000,151.6000) -- (242.3000,151.6000) -- (242.4000,151.6000) -- (242.4000,151.6000) -- (242.5000,151.6000) -- (242.5000,151.6000) -- (242.6000,151.6000) -- (242.7000,151.6000) -- (242.7000,151.6000) -- (242.8000,151.6000) -- (242.8000,151.6000) -- (242.9000,151.6000) -- (243.0000,151.6000) -- (243.0000,151.6000) -- (243.1000,151.6000) -- (243.2000,151.6000) -- (243.2000,151.6000) -- (243.3000,151.6000) -- (243.3000,151.6000) -- (243.4000,151.6000) -- (243.5000,151.6000) -- (243.5000,151.6000) -- (243.6000,151.6000) -- (243.6000,151.6000) -- (243.7000,151.6000) -- (243.8000,151.6000) -- (243.8000,151.6000) -- (243.9000,151.6000) -- (244.0000,151.6000) -- (244.0000,151.6000) -- (244.1000,151.6000) -- (244.1000,151.6000) -- (244.2000,151.6000) -- (244.3000,151.6000) -- (244.3000,151.6000) -- (244.4000,151.6000) -- (244.4000,151.6000) -- (244.5000,151.6000) -- (244.6000,151.6000) -- (244.6000,151.6000) -- (244.7000,151.6000) -- (244.8000,151.6000) -- (244.8000,151.6000) -- (244.9000,151.6000) -- (244.9000,151.6000) -- (245.0000,151.6000) -- (245.1000,151.6000) -- (245.1000,151.6000) -- (245.2000,151.6000) -- (245.2000,151.6000) -- (245.3000,151.6000) -- (245.4000,151.6000) -- (245.4000,151.6000) -- (245.5000,151.6000) -- (245.6000,151.6000) -- (245.6000,151.6000) -- (245.7000,151.6000) -- (245.7000,151.6000) -- (245.8000,151.6000) -- (245.9000,151.6000) -- (245.9000,151.6000) -- (246.0000,151.6000) -- (246.0000,151.6000) -- (246.1000,151.6000) -- (246.2000,151.6000) -- (246.2000,151.6000) -- (246.3000,151.6000);



  \end{scope}
  \begin{scope}[cm={{0.92047,0.0,0.0,0.92047,(-569.12952,52.34945)}},draw=blue,line cap=round,line join=round,line width=0.480pt]
    \path[draw] (184.5000,108.5000) -- (184.5000,156.5000) -- (246.5000,156.5000) -- (246.5000,108.5000) -- (184.5000,108.5000);



  \end{scope}
  \begin{scope}[cm={{0.92047,0.0,0.0,0.92047,(-569.12952,52.34945)}},draw=ca0a0a4,dash pattern=on 1.96pt off 1.96pt,line cap=round,line join=round,line width=0.326pt,miter limit=4.00]
    \path[draw,dash pattern=on 1.96pt off 1.96pt,line width=0.326pt,miter limit=4.00] (184.5000,200.5000) -- (246.5000,200.5000);



  \end{scope}
  \begin{scope}[cm={{0.92047,0.0,0.0,0.92047,(-569.12952,52.34945)}},draw=blue,line cap=round,line join=round,line width=0.480pt]
    \path[draw] (184.5000,200.5000) -- (186.5000,200.5000);



    \path[draw] (246.5000,200.5000) -- (245.5000,200.5000);



  \end{scope}
  \begin{scope}[cm={{0.92047,0.0,0.0,0.92047,(-569.12952,52.34945)}},draw=ca0a0a4,dash pattern=on 1.96pt off 1.96pt,line cap=round,line join=round,line width=0.326pt,miter limit=4.00]
    \path[draw,dash pattern=on 1.96pt off 1.96pt,line width=0.326pt,miter limit=4.00] (184.5000,180.5000) -- (246.5000,180.5000);



  \end{scope}
  \begin{scope}[cm={{0.92047,0.0,0.0,0.92047,(-569.12952,52.34945)}},draw=blue,line cap=round,line join=round,line width=0.480pt]
    \path[draw] (184.5000,180.5000) -- (186.5000,180.5000);



    \path[draw] (246.5000,180.5000) -- (245.5000,180.5000);



  \end{scope}
  \begin{scope}[cm={{0.92047,0.0,0.0,0.92047,(-569.12952,52.34945)}},draw=ca0a0a4,dash pattern=on 1.96pt off 1.96pt,line cap=round,line join=round,line width=0.326pt,miter limit=4.00]
    \path[draw,dash pattern=on 1.96pt off 1.96pt,line width=0.326pt,miter limit=4.00] (184.5000,160.5000) -- (246.5000,160.5000);



  \end{scope}
  \begin{scope}[cm={{0.92047,0.0,0.0,0.92047,(-569.12952,52.34945)}},draw=blue,line cap=round,line join=round,line width=0.480pt]
    \path[draw] (184.5000,160.5000) -- (186.5000,160.5000);



    \path[draw] (246.5000,160.5000) -- (245.5000,160.5000);



  \end{scope}
  \begin{scope}[cm={{0.92047,0.0,0.0,0.92047,(-569.12952,52.34945)}},draw=ca0a0a4,dash pattern=on 0.40pt off 0.80pt,line cap=round,line join=round,line width=0.400pt]
    \path[draw] (184.5000,204.5000) -- (184.5000,156.5000);



  \end{scope}
  \begin{scope}[cm={{0.92047,0.0,0.0,0.92047,(-569.12952,52.34945)}},draw=blue,line cap=round,line join=round,line width=0.480pt]
    \path[draw] (184.5000,204.5000) -- (184.5000,203.5000);



    \path[draw] (184.5000,156.5000) -- (184.5000,157.5000);



  \end{scope}
  \begin{scope}[cm={{1.00588,0.0,0.0,1.00588,(-400.6796,252.66367)}},draw=blue,line cap=rect,line join=bevel,line width=0.800pt]
    \path[fill=blue] (0.0000,0.0000) node[above right] (text1416) {\scriptsize 0};



  \end{scope}
  \begin{scope}[cm={{0.92047,0.0,0.0,0.92047,(-569.12952,52.34945)}},draw=ca0a0a4,dash pattern=on 1.96pt off 1.96pt,line cap=round,line join=round,line width=0.326pt,miter limit=4.00]
    \path[draw,dash pattern=on 1.96pt off 1.96pt,line width=0.326pt,miter limit=4.00] (203.5000,204.5000) -- (203.5000,156.5000);



  \end{scope}
  \begin{scope}[cm={{0.92047,0.0,0.0,0.92047,(-569.12952,52.34945)}},draw=blue,line cap=round,line join=round,line width=0.480pt]
    \path[draw] (203.5000,204.5000) -- (203.5000,203.5000);



    \path[draw] (203.5000,156.5000) -- (203.5000,157.5000);



  \end{scope}
  \begin{scope}[cm={{1.00588,0.0,0.0,1.00588,(-384.0746,252.77633)}},draw=blue,line cap=rect,line join=bevel,line width=0.800pt]
    \path[fill=blue] (0.0000,0.0000) node[above right] (text1446) {\scriptsize 2};



  \end{scope}
  \begin{scope}[cm={{0.92047,0.0,0.0,0.92047,(-569.12952,52.34945)}},draw=ca0a0a4,dash pattern=on 1.96pt off 1.96pt,line cap=round,line join=round,line width=0.326pt,miter limit=4.00]
    \path[draw,dash pattern=on 1.96pt off 1.96pt,line width=0.326pt,miter limit=4.00] (221.5000,204.5000) -- (221.5000,156.5000);



  \end{scope}
  \begin{scope}[cm={{0.92047,0.0,0.0,0.92047,(-569.12952,52.34945)}},draw=blue,line cap=round,line join=round,line width=0.480pt]
    \path[draw] (221.5000,204.5000) -- (221.5000,203.5000);



    \path[draw] (221.5000,156.5000) -- (221.5000,157.5000);



  \end{scope}
  \begin{scope}[cm={{1.00588,0.0,0.0,1.00588,(-366.9656,252.77633)}},draw=blue,line cap=rect,line join=bevel,line width=0.800pt]
    \path[fill=blue] (0.0000,0.0000) node[above right] (text1476) {\scriptsize 4};



  \end{scope}
  \begin{scope}[cm={{0.92047,0.0,0.0,0.92047,(-569.12952,52.34945)}},draw=ca0a0a4,dash pattern=on 1.96pt off 1.96pt,line cap=round,line join=round,line width=0.326pt,miter limit=4.00]
    \path[draw,dash pattern=on 1.96pt off 1.96pt,line width=0.326pt,miter limit=4.00] (239.5000,204.5000) -- (239.5000,156.5000);



  \end{scope}
  \begin{scope}[cm={{0.92047,0.0,0.0,0.92047,(-569.12952,52.34945)}},draw=blue,line cap=round,line join=round,line width=0.480pt]
    \path[draw] (239.5000,204.5000) -- (239.5000,203.5000);



    \path[draw] (239.5000,156.5000) -- (239.5000,157.5000);



  \end{scope}
  \begin{scope}[cm={{1.00588,0.0,0.0,1.00588,(-351.3566,252.66367)}},draw=blue,line cap=rect,line join=bevel,line width=0.800pt]
    \path[fill=blue] (0.0000,0.0000) node[above right] (text1506) {\scriptsize 6};



  \end{scope}
  \begin{scope}[cm={{0.92047,0.0,0.0,0.92047,(-569.12952,52.34945)}},draw=blue,line cap=round,line join=round,line width=0.480pt]
    \path[draw] (246.5000,200.5000) -- (245.5000,200.5000);



  \end{scope}
  \begin{scope}[cm={{0.92047,0.0,0.0,0.92047,(-569.12952,52.34945)}},draw=blue,line cap=round,line join=round,line width=0.480pt]
    \path[draw] (246.5000,180.5000) -- (245.5000,180.5000);



  \end{scope}
  \begin{scope}[cm={{0.92047,0.0,0.0,0.92047,(-569.12952,52.34945)}},draw=blue,line cap=round,line join=round,line width=0.480pt]
    \path[draw] (246.5000,160.5000) -- (245.5000,160.5000);



  \end{scope}
  \begin{scope}[cm={{0.92047,0.0,0.0,0.92047,(-569.12952,52.34945)}},draw=blue,line cap=round,line join=round,line width=0.480pt]
    \path[draw] (184.5000,156.5000) -- (184.5000,204.5000) -- (246.5000,204.5000) -- (246.5000,156.5000) -- (184.5000,156.5000);



  \end{scope}
  \begin{scope}[cm={{0.92047,0.0,0.0,0.92047,(-569.12952,52.34945)}},draw=blue,line cap=round,line join=round,line width=0.480pt]
    \path[draw] (184.8000,160.5000) -- (184.8000,160.5000) -- (184.9000,160.5000) -- (185.0000,160.5000) -- (185.0000,160.5000) -- (185.1000,160.5000) -- (185.1000,160.5000) -- (185.2000,160.5000) -- (185.3000,160.5000) -- (185.3000,160.5000) -- (185.4000,160.5000) -- (185.4000,160.5000) -- (185.5000,160.5000) -- (185.6000,160.5000) -- (185.6000,160.5000) -- (185.7000,160.5000) -- (185.8000,160.5000) -- (185.8000,160.5000) -- (185.9000,160.5000) -- (185.9000,160.5000) -- (186.0000,160.5000) -- (186.1000,160.5000) -- (186.1000,160.5000) -- (186.2000,160.5000) -- (186.2000,160.5000) -- (186.3000,160.5000) -- (186.4000,160.5000) -- (186.4000,160.5000) -- (186.5000,160.5000) -- (186.6000,160.5000) -- (186.6000,160.5000) -- (186.7000,160.5000) -- (186.7000,160.5000) -- (186.8000,160.5000) -- (186.9000,160.5000) -- (186.9000,160.5000) -- (187.0000,160.5000) -- (187.0000,160.5000) -- (187.1000,160.5000) -- (187.2000,160.5000) -- (187.2000,160.5000) -- (187.3000,160.5000) -- (187.4000,160.5000) -- (187.4000,160.5000) -- (187.5000,160.5000) -- (187.5000,160.5000) -- (187.6000,160.5000) -- (187.7000,160.5000) -- (187.7000,160.5000) -- (187.8000,160.5000) -- (187.8000,160.5000) -- (187.9000,160.5000) -- (188.0000,160.5000) -- (188.0000,160.5000) -- (188.1000,160.5000) -- (188.2000,160.5000) -- (188.2000,160.5000) -- (188.3000,160.5000) -- (188.3000,160.5000) -- (188.4000,160.5000) -- (188.5000,160.5000) -- (188.5000,160.5000) -- (188.6000,160.5000) -- (188.6000,160.5000) -- (188.7000,160.5000) -- (188.8000,160.5000) -- (188.8000,160.5000) -- (188.9000,160.5000) -- (189.0000,160.5000) -- (189.0000,160.5000) -- (189.1000,160.5000) -- (189.1000,160.5000) -- (189.2000,160.5000) -- (189.3000,160.5000) -- (189.3000,160.5000) -- (189.4000,160.5000) -- (189.4000,160.5000) -- (189.5000,160.5000) -- (189.6000,160.5000) -- (189.6000,160.5000) -- (189.7000,160.5000) -- (189.8000,160.5000) -- (189.8000,160.5000) -- (189.9000,160.5000) -- (189.9000,160.5000) -- (190.0000,160.5000) -- (190.1000,160.5000) -- (190.1000,160.5000) -- (190.2000,160.5000) -- (190.2000,160.5000) -- (190.3000,160.5000) -- (190.4000,160.5000) -- (190.4000,160.5000) -- (190.5000,160.5000) -- (190.6000,160.5000) -- (190.6000,160.5000) -- (190.7000,160.5000) -- (190.7000,160.5000) -- (190.8000,160.5000) -- (190.9000,160.5000) -- (190.9000,160.5000) -- (191.0000,160.5000) -- (191.0000,160.5000) -- (191.1000,160.5000) -- (191.2000,160.5000) -- (191.2000,160.5000) -- (191.3000,160.5000) -- (191.4000,160.5000) -- (191.4000,160.5000) -- (191.5000,160.5000) -- (191.5000,160.5000) -- (191.6000,160.5000) -- (191.7000,160.5000) -- (191.7000,160.5000) -- (191.8000,160.5000) -- (191.8000,160.5000) -- (191.9000,160.5000) -- (192.0000,160.5000) -- (192.0000,160.5000) -- (192.1000,160.5000) -- (192.2000,160.5000) -- (192.2000,160.5000) -- (192.3000,160.5000) -- (192.3000,160.5000) -- (192.4000,160.5000) -- (192.5000,160.5000) -- (192.5000,160.5000) -- (192.6000,160.5000) -- (192.6000,160.5000) -- (192.7000,160.5000) -- (192.8000,160.5000) -- (192.8000,160.5000) -- (192.9000,160.5000) -- (193.0000,160.5000) -- (193.0000,160.5000) -- (193.1000,160.5000) -- (193.1000,160.5000) -- (193.2000,160.5000) -- (193.3000,160.5000) -- (193.3000,160.5000) -- (193.4000,160.5000) -- (193.4000,160.5000) -- (193.5000,160.5000) -- (193.6000,160.5000) -- (193.6000,160.5000) -- (193.7000,160.5000) -- (193.8000,160.5000) -- (193.8000,160.5000) -- (193.9000,160.5000) -- (193.9000,160.5000) -- (194.0000,160.5000) -- (194.1000,160.5000) -- (194.1000,160.5000) -- (194.2000,160.5000) -- (194.2000,160.5000) -- (194.3000,160.5000) -- (194.4000,160.5000) -- (194.4000,160.5000) -- (194.5000,160.5000) -- (194.6000,160.5000) -- (194.6000,160.5000) -- (194.7000,160.5000) -- (194.7000,160.5000) -- (194.8000,160.5000) -- (194.9000,160.5000) -- (194.9000,160.5000) -- (195.0000,160.5000) -- (195.0000,160.5000) -- (195.1000,160.5000) -- (195.2000,160.5000) -- (195.2000,160.5000) -- (195.3000,160.5000) -- (195.4000,160.5000) -- (195.4000,160.5000) -- (195.5000,160.5000) -- (195.5000,160.5000) -- (195.6000,160.5000) -- (195.7000,160.5000) -- (195.7000,160.5000) -- (195.8000,160.5000) -- (195.8000,160.5000) -- (195.9000,160.5000) -- (196.0000,160.5000) -- (196.0000,160.5000) -- (196.1000,160.5000) -- (196.2000,160.5000) -- (196.2000,160.5000) -- (196.3000,160.5000) -- (196.3000,160.5000) -- (196.4000,160.5000) -- (196.5000,160.5000) -- (196.5000,160.5000) -- (196.6000,160.5000) -- (196.6000,160.5000) -- (196.7000,160.5000) -- (196.8000,160.5000) -- (196.8000,160.5000) -- (196.9000,160.5000) -- (196.9000,160.5000) -- (197.0000,160.5000) -- (197.1000,160.5000) -- (197.1000,160.5000) -- (197.2000,160.5000) -- (197.3000,160.5000) -- (197.3000,160.5000) -- (197.4000,160.5000) -- (197.4000,160.5000) -- (197.5000,160.5000) -- (197.6000,160.5000) -- (197.6000,160.5000) -- (197.7000,160.5000) -- (197.7000,160.5000) -- (197.8000,160.5000) -- (197.9000,160.5000) -- (197.9000,160.5000) -- (198.0000,160.5000) -- (198.1000,160.5000) -- (198.1000,160.5000) -- (198.2000,160.5000) -- (198.2000,160.5000) -- (198.3000,160.5000) -- (198.4000,160.5000) -- (198.4000,160.5000) -- (198.5000,160.5000) -- (198.5000,160.5000) -- (198.6000,160.5000) -- (198.7000,160.5000) -- (198.7000,160.5000) -- (198.8000,160.5000) -- (198.9000,159.9000) -- (198.9000,165.0000) -- (199.0000,180.4000) -- (199.0000,180.3000) -- (199.1000,180.2000) -- (199.2000,180.2000) -- (199.2000,180.2000) -- (199.3000,180.2000) -- (199.3000,180.2000) -- (199.4000,180.2000) -- (199.5000,180.2000) -- (199.5000,180.2000) -- (199.6000,180.2000) -- (199.7000,180.2000) -- (199.7000,180.2000) -- (199.8000,180.2000) -- (199.8000,180.2000) -- (199.9000,180.2000) -- (200.0000,180.2000) -- (200.0000,180.2000) -- (200.1000,180.2000) -- (200.1000,180.2000) -- (200.2000,180.2000) -- (200.3000,180.2000) -- (200.3000,180.2000) -- (200.4000,180.2000) -- (200.5000,180.2000) -- (200.5000,180.2000) -- (200.6000,180.2000) -- (200.6000,180.2000) -- (200.7000,180.2000) -- (200.8000,180.2000) -- (200.8000,180.2000) -- (200.9000,180.2000) -- (200.9000,180.2000) -- (201.0000,180.2000) -- (201.1000,180.2000) -- (201.1000,180.2000) -- (201.2000,180.2000) -- (201.3000,180.2000) -- (201.3000,180.2000) -- (201.4000,180.2000) -- (201.4000,180.2000) -- (201.5000,180.2000) -- (201.6000,180.2000) -- (201.6000,180.2000) -- (201.7000,180.2000) -- (201.7000,180.2000) -- (201.8000,180.2000) -- (201.9000,180.2000) -- (201.9000,180.2000) -- (202.0000,180.2000) -- (202.1000,180.2000) -- (202.1000,180.2000) -- (202.2000,180.2000) -- (202.2000,180.2000) -- (202.3000,180.2000) -- (202.4000,180.2000) -- (202.4000,180.2000) -- (202.5000,180.2000) -- (202.5000,180.2000) -- (202.6000,180.2000) -- (202.7000,180.2000) -- (202.7000,180.2000) -- (202.8000,180.2000) -- (202.9000,180.2000) -- (202.9000,180.2000) -- (203.0000,180.2000) -- (203.0000,180.2000) -- (203.1000,180.2000) -- (203.2000,180.2000) -- (203.2000,180.2000) -- (203.3000,180.2000) -- (203.3000,180.2000) -- (203.4000,180.2000) -- (203.5000,180.2000) -- (203.5000,180.2000) -- (203.6000,180.2000) -- (203.7000,180.2000) -- (203.7000,180.2000) -- (203.8000,180.2000) -- (203.8000,180.2000) -- (203.9000,180.2000) -- (204.0000,180.2000) -- (204.0000,180.2000) -- (204.1000,180.2000) -- (204.1000,180.2000) -- (204.2000,180.2000) -- (204.3000,180.2000) -- (204.3000,180.2000) -- (204.4000,180.2000) -- (204.5000,180.2000) -- (204.5000,180.2000) -- (204.6000,180.2000) -- (204.6000,180.2000) -- (204.7000,180.2000) -- (204.8000,180.2000) -- (204.8000,180.2000) -- (204.9000,180.2000) -- (204.9000,180.2000) -- (205.0000,180.2000) -- (205.1000,180.2000) -- (205.1000,180.2000) -- (205.2000,180.2000) -- (205.3000,180.2000) -- (205.3000,180.2000) -- (205.4000,180.2000) -- (205.4000,180.2000) -- (205.5000,180.2000) -- (205.6000,180.2000) -- (205.6000,180.2000) -- (205.7000,180.2000) -- (205.7000,180.2000) -- (205.8000,180.2000) -- (205.9000,180.2000) -- (205.9000,180.2000) -- (206.0000,180.2000) -- (206.1000,180.2000) -- (206.1000,180.2000) -- (206.2000,180.2000) -- (206.2000,180.2000) -- (206.3000,180.2000) -- (206.4000,180.2000) -- (206.4000,180.2000) -- (206.5000,180.2000) -- (206.5000,180.2000) -- (206.6000,180.2000) -- (206.7000,180.2000) -- (206.7000,180.2000) -- (206.8000,180.2000) -- (206.9000,180.2000) -- (206.9000,180.2000) -- (207.0000,180.2000) -- (207.0000,180.2000) -- (207.1000,180.2000) -- (207.2000,180.2000) -- (207.2000,180.2000) -- (207.3000,180.2000) -- (207.3000,180.2000) -- (207.4000,180.2000) -- (207.5000,180.2000) -- (207.5000,180.2000) -- (207.6000,180.2000) -- (207.7000,180.2000) -- (207.7000,180.2000) -- (207.8000,180.2000) -- (207.8000,180.2000) -- (207.9000,180.2000) -- (208.0000,180.2000) -- (208.0000,180.2000) -- (208.1000,180.2000) -- (208.1000,180.2000) -- (208.2000,180.2000) -- (208.3000,180.2000) -- (208.3000,180.2000) -- (208.4000,180.2000) -- (208.5000,180.2000) -- (208.5000,180.2000) -- (208.6000,180.2000) -- (208.6000,180.2000) -- (208.7000,180.2000) -- (208.8000,180.2000) -- (208.8000,180.2000) -- (208.9000,180.2000) -- (208.9000,180.2000) -- (209.0000,180.2000) -- (209.1000,180.2000) -- (209.1000,180.2000) -- (209.2000,180.2000) -- (209.3000,180.2000) -- (209.3000,180.2000) -- (209.4000,180.2000) -- (209.4000,180.2000) -- (209.5000,180.2000) -- (209.6000,180.2000) -- (209.6000,180.2000) -- (209.7000,180.2000) -- (209.7000,180.2000) -- (209.8000,180.2000) -- (209.9000,180.2000) -- (209.9000,180.2000) -- (210.0000,180.2000) -- (210.1000,180.2000) -- (210.1000,180.2000) -- (210.2000,180.2000) -- (210.2000,180.2000) -- (210.3000,180.2000) -- (210.4000,180.2000) -- (210.4000,180.2000) -- (210.5000,180.2000) -- (210.5000,180.2000) -- (210.6000,180.2000) -- (210.7000,180.2000) -- (210.7000,180.2000) -- (210.8000,180.2000) -- (210.9000,180.2000) -- (210.9000,180.2000) -- (211.0000,180.2000) -- (211.0000,180.2000) -- (211.1000,180.2000) -- (211.2000,180.2000) -- (211.2000,180.2000) -- (211.3000,180.2000) -- (211.3000,180.2000) -- (211.4000,180.2000) -- (211.5000,180.2000) -- (211.5000,180.2000) -- (211.6000,180.2000) -- (211.7000,180.2000) -- (211.7000,180.2000) -- (211.8000,180.2000) -- (211.8000,180.2000) -- (211.9000,180.2000) -- (212.0000,180.2000) -- (212.0000,180.2000) -- (212.1000,180.2000) -- (212.1000,180.2000) -- (212.2000,180.2000) -- (212.3000,180.2000) -- (212.3000,180.2000) -- (212.4000,180.2000) -- (212.5000,180.2000) -- (212.5000,180.2000) -- (212.6000,180.2000) -- (212.6000,180.2000) -- (212.7000,180.2000) -- (212.8000,180.2000) -- (212.8000,180.2000) -- (212.9000,180.2000) -- (212.9000,180.2000) -- (213.0000,180.2000) -- (213.1000,180.2000) -- (213.1000,180.2000) -- (213.2000,180.2000) -- (213.3000,180.2000) -- (213.3000,180.2000) -- (213.4000,180.2000) -- (213.4000,180.2000) -- (213.5000,180.2000) -- (213.6000,180.2000) -- (213.6000,180.2000) -- (213.7000,180.2000) -- (213.7000,180.2000) -- (213.8000,180.2000) -- (213.9000,180.2000) -- (213.9000,180.2000) -- (214.0000,180.2000) -- (214.1000,180.2000) -- (214.1000,180.2000) -- (214.2000,180.2000) -- (214.2000,180.2000) -- (214.3000,180.2000) -- (214.4000,180.2000) -- (214.4000,180.2000) -- (214.5000,180.2000) -- (214.5000,180.2000) -- (214.6000,180.2000) -- (214.7000,180.2000) -- (214.7000,180.2000) -- (214.8000,180.2000) -- (214.9000,180.2000) -- (214.9000,180.2000) -- (215.0000,180.2000) -- (215.0000,180.2000) -- (215.1000,180.2000) -- (215.2000,180.2000) -- (215.2000,180.2000) -- (215.3000,180.2000) -- (215.3000,180.2000) -- (215.4000,180.2000) -- (215.5000,180.2000) -- (215.5000,180.2000) -- (215.6000,180.2000) -- (215.7000,180.2000) -- (215.7000,180.2000) -- (215.8000,180.2000) -- (215.8000,180.2000) -- (215.9000,180.2000) -- (216.0000,180.2000) -- (216.0000,180.2000) -- (216.1000,180.2000) -- (216.1000,180.2000) -- (216.2000,180.2000) -- (216.3000,180.2000) -- (216.3000,180.2000) -- (216.4000,180.2000) -- (216.5000,180.2000) -- (216.5000,180.2000) -- (216.6000,180.2000) -- (216.6000,180.2000) -- (216.7000,180.2000) -- (216.8000,180.2000) -- (216.8000,180.2000) -- (216.9000,180.2000) -- (216.9000,180.2000) -- (217.0000,180.2000) -- (217.1000,180.2000) -- (217.1000,180.2000) -- (217.2000,180.2000) -- (217.3000,180.2000) -- (217.3000,180.2000) -- (217.4000,180.2000) -- (217.4000,180.2000) -- (217.5000,180.2000) -- (217.6000,180.2000) -- (217.6000,180.2000) -- (217.7000,180.2000) -- (217.7000,180.2000) -- (217.8000,180.2000) -- (217.9000,180.2000) -- (217.9000,180.2000) -- (218.0000,180.2000) -- (218.1000,180.2000) -- (218.1000,180.2000) -- (218.2000,180.2000) -- (218.2000,180.2000) -- (218.3000,180.2000) -- (218.4000,180.2000) -- (218.4000,180.2000) -- (218.5000,180.2000) -- (218.5000,180.2000) -- (218.6000,180.2000) -- (218.7000,180.2000) -- (218.7000,180.2000) -- (218.8000,180.2000) -- (218.9000,180.2000) -- (218.9000,180.2000) -- (219.0000,180.2000) -- (219.0000,180.2000) -- (219.1000,180.2000) -- (219.2000,180.2000) -- (219.2000,180.2000) -- (219.3000,180.2000) -- (219.3000,180.2000) -- (219.4000,180.2000) -- (219.5000,180.2000) -- (219.5000,180.2000) -- (219.6000,180.2000) -- (219.7000,180.2000) -- (219.7000,180.2000) -- (219.8000,180.2000) -- (219.8000,180.2000) -- (219.9000,180.2000) -- (220.0000,180.2000) -- (220.0000,180.2000) -- (220.1000,180.2000) -- (220.1000,180.2000) -- (220.2000,180.2000) -- (220.3000,180.2000) -- (220.3000,180.2000) -- (220.4000,180.2000) -- (220.5000,180.2000) -- (220.5000,180.2000) -- (220.6000,180.2000) -- (220.6000,180.2000) -- (220.7000,180.2000) -- (220.8000,180.2000) -- (220.8000,180.2000) -- (220.9000,180.2000) -- (220.9000,180.2000) -- (221.0000,180.2000) -- (221.1000,180.2000) -- (221.1000,180.2000) -- (221.2000,180.2000) -- (221.3000,180.2000) -- (221.3000,180.2000) -- (221.4000,180.2000) -- (221.4000,180.2000) -- (221.5000,180.2000) -- (221.6000,180.2000) -- (221.6000,180.2000) -- (221.7000,180.2000) -- (221.7000,180.2000) -- (221.8000,180.2000) -- (221.9000,180.2000) -- (221.9000,180.2000) -- (222.0000,180.2000) -- (222.1000,180.2000) -- (222.1000,180.2000) -- (222.2000,180.2000) -- (222.2000,180.2000) -- (222.3000,180.2000) -- (222.4000,180.2000) -- (222.4000,180.2000) -- (222.5000,180.2000) -- (222.5000,180.2000) -- (222.6000,180.2000) -- (222.7000,180.2000) -- (222.7000,180.2000) -- (222.8000,180.2000) -- (222.9000,180.2000) -- (222.9000,180.2000) -- (223.0000,180.2000) -- (223.0000,180.2000) -- (223.1000,180.2000) -- (223.2000,180.2000) -- (223.2000,180.2000) -- (223.3000,180.2000) -- (223.3000,180.2000) -- (223.4000,180.2000) -- (223.5000,180.2000) -- (223.5000,180.2000) -- (223.6000,180.2000) -- (223.7000,180.2000) -- (223.7000,180.2000) -- (223.8000,180.2000) -- (223.8000,180.2000) -- (223.9000,180.2000) -- (224.0000,180.2000) -- (224.0000,180.2000) -- (224.1000,180.2000) -- (224.1000,180.2000) -- (224.2000,180.2000) -- (224.3000,180.2000) -- (224.3000,180.2000) -- (224.4000,180.2000) -- (224.5000,180.2000) -- (224.5000,180.2000) -- (224.6000,180.2000) -- (224.6000,180.2000) -- (224.7000,180.2000) -- (224.8000,180.2000) -- (224.8000,180.2000) -- (224.9000,180.2000) -- (224.9000,180.2000) -- (225.0000,180.2000) -- (225.1000,180.2000) -- (225.1000,180.2000) -- (225.2000,180.2000) -- (225.3000,180.2000) -- (225.3000,180.2000) -- (225.4000,180.2000) -- (225.4000,180.2000) -- (225.5000,180.2000) -- (225.6000,180.2000) -- (225.6000,180.2000) -- (225.7000,180.2000) -- (225.7000,180.2000) -- (225.8000,180.2000) -- (225.9000,179.4000) -- (225.9000,190.4000) -- (226.0000,200.9000) -- (226.0000,200.0000) -- (226.1000,200.0000) -- (226.2000,200.0000) -- (226.2000,200.0000) -- (226.3000,200.0000) -- (226.4000,200.0000) -- (226.4000,200.0000) -- (226.5000,200.0000) -- (226.5000,200.0000) -- (226.6000,200.0000) -- (226.7000,200.0000) -- (226.7000,200.0000) -- (226.8000,200.0000) -- (226.8000,200.0000) -- (226.9000,200.0000) -- (227.0000,200.0000) -- (227.0000,200.0000) -- (227.1000,200.0000) -- (227.2000,200.0000) -- (227.2000,200.0000) -- (227.3000,200.0000) -- (227.3000,200.0000) -- (227.4000,200.0000) -- (227.5000,200.0000) -- (227.5000,200.0000) -- (227.6000,200.0000) -- (227.7000,200.0000) -- (227.7000,200.0000) -- (227.8000,200.0000) -- (227.8000,200.0000) -- (227.9000,200.0000) -- (228.0000,200.0000) -- (228.0000,200.0000) -- (228.1000,200.0000) -- (228.1000,200.0000) -- (228.2000,200.0000) -- (228.3000,200.0000) -- (228.3000,200.0000) -- (228.4000,200.0000) -- (228.5000,200.0000) -- (228.5000,200.0000) -- (228.6000,200.0000) -- (228.6000,200.0000) -- (228.7000,200.0000) -- (228.8000,200.0000) -- (228.8000,200.0000) -- (228.9000,200.0000) -- (228.9000,200.0000) -- (229.0000,200.0000) -- (229.1000,200.0000) -- (229.1000,200.0000) -- (229.2000,200.0000) -- (229.2000,200.0000) -- (229.3000,200.0000) -- (229.4000,200.0000) -- (229.4000,200.0000) -- (229.5000,200.0000) -- (229.6000,200.0000) -- (229.6000,200.0000) -- (229.7000,200.0000) -- (229.7000,200.0000) -- (229.8000,200.0000) -- (229.9000,200.0000) -- (229.9000,200.0000) -- (230.0000,200.0000) -- (230.0000,200.0000) -- (230.1000,200.0000) -- (230.2000,200.0000) -- (230.2000,200.0000) -- (230.3000,200.0000) -- (230.4000,200.0000) -- (230.4000,200.0000) -- (230.5000,200.0000) -- (230.5000,200.0000) -- (230.6000,200.0000) -- (230.7000,200.0000) -- (230.7000,200.0000) -- (230.8000,200.0000) -- (230.8000,200.0000) -- (230.9000,200.0000) -- (231.0000,200.0000) -- (231.0000,200.0000) -- (231.1000,200.0000) -- (231.2000,200.0000) -- (231.2000,200.0000) -- (231.3000,200.0000) -- (231.3000,200.0000) -- (231.4000,200.0000) -- (231.5000,200.0000) -- (231.5000,200.0000) -- (231.6000,200.0000) -- (231.6000,200.0000) -- (231.7000,200.0000) -- (231.8000,200.0000) -- (231.8000,200.0000) -- (231.9000,200.0000) -- (232.0000,200.0000) -- (232.0000,200.0000) -- (232.1000,200.0000) -- (232.1000,200.0000) -- (232.2000,200.0000) -- (232.3000,200.0000) -- (232.3000,200.0000) -- (232.4000,200.0000) -- (232.4000,200.0000) -- (232.5000,200.0000) -- (232.6000,200.0000) -- (232.6000,200.0000) -- (232.7000,200.0000) -- (232.8000,200.0000) -- (232.8000,200.0000) -- (232.9000,200.0000) -- (232.9000,200.0000) -- (233.0000,200.0000) -- (233.1000,200.0000) -- (233.1000,200.0000) -- (233.2000,200.0000) -- (233.2000,200.0000) -- (233.3000,200.0000) -- (233.4000,200.0000) -- (233.4000,200.0000) -- (233.5000,200.0000) -- (233.6000,200.0000) -- (233.6000,200.0000) -- (233.7000,200.0000) -- (233.7000,200.0000) -- (233.8000,200.0000) -- (233.9000,200.0000) -- (233.9000,200.0000) -- (234.0000,200.0000) -- (234.0000,200.0000) -- (234.1000,200.0000) -- (234.2000,200.0000) -- (234.2000,200.0000) -- (234.3000,200.0000) -- (234.4000,200.0000) -- (234.4000,200.0000) -- (234.5000,200.0000) -- (234.5000,200.0000) -- (234.6000,200.0000) -- (234.7000,200.0000) -- (234.7000,200.0000) -- (234.8000,200.0000) -- (234.8000,200.0000) -- (234.9000,200.0000) -- (235.0000,200.0000) -- (235.0000,200.0000) -- (235.1000,200.0000) -- (235.2000,200.0000) -- (235.2000,200.0000) -- (235.3000,200.0000) -- (235.3000,200.0000) -- (235.4000,200.0000) -- (235.5000,200.0000) -- (235.5000,200.0000) -- (235.6000,200.0000) -- (235.6000,200.0000) -- (235.7000,200.0000) -- (235.8000,200.0000) -- (235.8000,200.0000) -- (235.9000,200.0000) -- (236.0000,200.0000) -- (236.0000,200.0000) -- (236.1000,200.0000) -- (236.1000,200.0000) -- (236.2000,200.0000) -- (236.3000,200.0000) -- (236.3000,200.0000) -- (236.4000,200.0000) -- (236.4000,200.0000) -- (236.5000,200.0000) -- (236.6000,200.0000) -- (236.6000,200.0000) -- (236.7000,200.0000) -- (236.8000,200.0000) -- (236.8000,200.0000) -- (236.9000,200.0000) -- (236.9000,200.0000) -- (237.0000,200.0000) -- (237.1000,200.0000) -- (237.1000,200.0000) -- (237.2000,200.0000) -- (237.2000,200.0000) -- (237.3000,200.0000) -- (237.4000,200.0000) -- (237.4000,200.0000) -- (237.5000,200.0000) -- (237.6000,200.0000) -- (237.6000,200.0000) -- (237.7000,200.0000) -- (237.7000,200.0000) -- (237.8000,200.0000) -- (237.9000,200.0000) -- (237.9000,200.0000) -- (238.0000,200.0000) -- (238.0000,200.0000) -- (238.1000,200.0000) -- (238.2000,200.0000) -- (238.2000,200.0000) -- (238.3000,200.0000) -- (238.4000,200.0000) -- (238.4000,200.0000) -- (238.5000,200.0000) -- (238.5000,200.0000) -- (238.6000,200.0000) -- (238.7000,200.0000) -- (238.7000,200.0000) -- (238.8000,200.0000) -- (238.8000,200.0000) -- (238.9000,200.0000) -- (239.0000,200.0000) -- (239.0000,200.0000) -- (239.1000,200.0000) -- (239.2000,200.0000) -- (239.2000,200.0000) -- (239.3000,200.0000) -- (239.3000,200.0000) -- (239.4000,200.0000) -- (239.5000,200.0000) -- (239.5000,200.0000) -- (239.6000,200.0000) -- (239.6000,200.0000) -- (239.7000,200.0000) -- (239.8000,200.0000) -- (239.8000,200.0000) -- (239.9000,200.0000) -- (240.0000,200.0000) -- (240.0000,200.0000) -- (240.1000,200.0000) -- (240.1000,200.0000) -- (240.2000,200.0000) -- (240.3000,200.0000) -- (240.3000,200.0000) -- (240.4000,200.0000) -- (240.4000,200.0000) -- (240.5000,200.0000) -- (240.6000,200.0000) -- (240.6000,200.0000) -- (240.7000,200.0000) -- (240.8000,200.0000) -- (240.8000,200.0000) -- (240.9000,200.0000) -- (240.9000,200.0000) -- (241.0000,200.0000) -- (241.1000,200.0000) -- (241.1000,200.0000) -- (241.2000,200.0000) -- (241.2000,200.0000) -- (241.3000,200.0000) -- (241.4000,200.0000) -- (241.4000,200.0000) -- (241.5000,200.0000) -- (241.6000,200.0000) -- (241.6000,200.0000) -- (241.7000,200.0000) -- (241.7000,200.0000) -- (241.8000,200.0000) -- (241.9000,200.0000) -- (241.9000,200.0000) -- (242.0000,200.0000) -- (242.0000,200.0000) -- (242.1000,200.0000) -- (242.2000,200.0000) -- (242.2000,200.0000) -- (242.3000,200.0000) -- (242.4000,200.0000) -- (242.4000,200.0000) -- (242.5000,200.0000) -- (242.5000,200.0000) -- (242.6000,200.0000) -- (242.7000,200.0000) -- (242.7000,200.0000) -- (242.8000,200.0000) -- (242.8000,200.0000) -- (242.9000,200.0000) -- (243.0000,200.0000) -- (243.0000,200.0000) -- (243.1000,200.0000) -- (243.2000,200.0000) -- (243.2000,200.0000) -- (243.3000,200.0000) -- (243.3000,200.0000) -- (243.4000,200.0000) -- (243.5000,200.0000) -- (243.5000,200.0000) -- (243.6000,200.0000) -- (243.6000,200.0000) -- (243.7000,200.0000) -- (243.8000,200.0000) -- (243.8000,200.0000) -- (243.9000,200.0000) -- (244.0000,200.0000) -- (244.0000,200.0000) -- (244.1000,200.0000) -- (244.1000,200.0000) -- (244.2000,200.0000) -- (244.3000,200.0000) -- (244.3000,200.0000) -- (244.4000,200.0000) -- (244.4000,200.0000) -- (244.5000,200.0000) -- (244.6000,200.0000) -- (244.6000,200.0000) -- (244.7000,200.0000) -- (244.8000,200.0000) -- (244.8000,200.0000) -- (244.9000,200.0000) -- (244.9000,200.0000) -- (245.0000,200.0000) -- (245.1000,200.0000) -- (245.1000,200.0000) -- (245.2000,200.0000) -- (245.2000,200.0000) -- (245.3000,200.0000) -- (245.4000,200.0000) -- (245.4000,200.0000) -- (245.5000,200.0000) -- (245.6000,200.0000) -- (245.6000,200.0000) -- (245.7000,200.0000) -- (245.7000,200.0000) -- (245.8000,200.0000) -- (245.9000,200.0000) -- (245.9000,200.0000) -- (246.0000,200.0000) -- (246.0000,200.0000) -- (246.1000,200.0000) -- (246.2000,200.0000) -- (246.2000,200.0000) -- (246.3000,200.0000);



  \end{scope}
  \begin{scope}[cm={{0.92047,0.0,0.0,0.92047,(-569.12952,52.34945)}},draw=blue,line cap=round,line join=round,line width=0.480pt]
    \path[draw] (184.5000,156.5000) -- (184.5000,204.5000) -- (246.5000,204.5000) -- (246.5000,156.5000) -- (184.5000,156.5000);



  \end{scope}
  \begin{scope}[cm={{0.92743,0.0,0.0,0.92743,(-570.4103,65.15216)}},draw=ca0a0a4,dash pattern=on 2.59pt off 2.59pt,line cap=round,line join=round,line width=0.323pt,miter limit=4.00]
    \path[draw,dash pattern=on 2.59pt off 2.59pt,line width=0.323pt,miter limit=4.00] (184.5000,267.5000) -- (246.5000,267.5000);



  \end{scope}
  \begin{scope}[cm={{0.92743,0.0,0.0,0.92743,(-570.4103,65.15216)}},draw=blue,line cap=round,line join=round,line width=0.480pt]
    \path[draw] (184.5000,267.5000) -- (186.5000,267.5000);



    \path[draw] (246.5000,267.5000) -- (245.5000,267.5000);



  \end{scope}
  \begin{scope}[cm={{0.92743,0.0,0.0,0.92743,(-570.4103,65.15216)}},draw=ca0a0a4,dash pattern=on 2.59pt off 2.59pt,line cap=round,line join=round,line width=0.323pt,miter limit=4.00]
    \path[draw,dash pattern=on 2.59pt off 2.59pt,line width=0.323pt,miter limit=4.00] (184.5000,248.5000) -- (246.5000,248.5000);



  \end{scope}
  \begin{scope}[cm={{0.92743,0.0,0.0,0.92743,(-570.4103,65.15216)}},draw=blue,line cap=round,line join=round,line width=0.480pt]
    \path[draw] (184.5000,248.5000) -- (186.5000,248.5000);



    \path[draw] (246.5000,248.5000) -- (245.5000,248.5000);



  \end{scope}
  \begin{scope}[cm={{0.92743,0.0,0.0,0.92743,(-570.4103,65.15216)}},draw=ca0a0a4,dash pattern=on 2.59pt off 2.59pt,line cap=round,line join=round,line width=0.323pt,miter limit=4.00]
    \path[draw,dash pattern=on 2.59pt off 2.59pt,line width=0.323pt,miter limit=4.00] (184.5000,229.5000) -- (246.5000,229.5000);



  \end{scope}
  \begin{scope}[cm={{0.92743,0.0,0.0,0.92743,(-570.4103,65.15216)}},draw=blue,line cap=round,line join=round,line width=0.480pt]
    \path[draw] (184.5000,229.5000) -- (186.5000,229.5000);



    \path[draw] (246.5000,229.5000) -- (245.5000,229.5000);



  \end{scope}
  \begin{scope}[cm={{0.92743,0.0,0.0,0.92743,(-570.4103,65.15216)}},draw=ca0a0a4,dash pattern=on 0.40pt off 0.80pt,line cap=round,line join=round,line width=0.400pt]
    \path[draw] (184.5000,272.5000) -- (184.5000,224.5000);



  \end{scope}
  \begin{scope}[cm={{0.92743,0.0,0.0,0.92743,(-570.4103,65.15216)}},draw=blue,line cap=round,line join=round,line width=0.480pt]
    \path[draw] (184.5000,272.5000) -- (184.5000,270.5000);



    \path[draw] (184.5000,224.5000) -- (184.5000,225.5000);



  \end{scope}
  \begin{scope}[cm={{0.92743,0.0,0.0,0.92743,(-570.4103,65.15216)}},draw=ca0a0a4,dash pattern=on 2.59pt off 2.59pt,line cap=round,line join=round,line width=0.323pt,miter limit=4.00]
    \path[draw,dash pattern=on 2.59pt off 2.59pt,line width=0.323pt,miter limit=4.00] (201.5000,272.5000) -- (201.5000,224.5000);



  \end{scope}
  \begin{scope}[cm={{0.92743,0.0,0.0,0.92743,(-570.4103,65.15216)}},draw=blue,line cap=round,line join=round,line width=0.480pt]
    \path[draw] (201.5000,272.5000) -- (201.5000,270.5000);



    \path[draw] (201.5000,224.5000) -- (201.5000,225.5000);



  \end{scope}
  \begin{scope}[cm={{0.92743,0.0,0.0,0.92743,(-570.4103,65.15216)}},draw=ca0a0a4,dash pattern=on 2.59pt off 2.59pt,line cap=round,line join=round,line width=0.323pt,miter limit=4.00]
    \path[draw,dash pattern=on 2.59pt off 2.59pt,line width=0.323pt,miter limit=4.00] (218.5000,272.5000) -- (218.5000,224.5000);



  \end{scope}
  \begin{scope}[cm={{0.92743,0.0,0.0,0.92743,(-570.4103,65.15216)}},draw=blue,line cap=round,line join=round,line width=0.480pt]
    \path[draw] (218.5000,272.5000) -- (218.5000,270.5000);



    \path[draw] (218.5000,224.5000) -- (218.5000,225.5000);



  \end{scope}
  \begin{scope}[cm={{0.92743,0.0,0.0,0.92743,(-570.4103,65.15216)}},draw=ca0a0a4,dash pattern=on 2.59pt off 2.59pt,line cap=round,line join=round,line width=0.323pt,miter limit=4.00]
    \path[draw,dash pattern=on 2.59pt off 2.59pt,line width=0.323pt,miter limit=4.00] (235.5000,272.5000) -- (235.5000,224.5000);



  \end{scope}
  \begin{scope}[cm={{0.92743,0.0,0.0,0.92743,(-570.4103,65.15216)}},draw=blue,line cap=round,line join=round,line width=0.480pt]
    \path[draw] (235.5000,272.5000) -- (235.5000,270.5000);



    \path[draw] (235.5000,224.5000) -- (235.5000,225.5000);



  \end{scope}
  \begin{scope}[cm={{0.92743,0.0,0.0,0.92743,(-570.4103,65.15216)}},draw=blue,line cap=round,line join=round,line width=0.480pt]
    \path[draw] (246.5000,267.5000) -- (245.5000,267.5000);



  \end{scope}
  \begin{scope}[cm={{1.00588,0.0,0.0,1.00588,(-407.95798,313.77387)}},draw=blue,line cap=rect,line join=bevel,line width=0.800pt]
    \path[fill=blue] (0.0000,0.0000) node[above right] (text2120) {\scriptsize 2};



  \end{scope}
  \begin{scope}[cm={{0.92743,0.0,0.0,0.92743,(-570.4103,65.15216)}},draw=blue,line cap=round,line join=round,line width=0.480pt]
    \path[draw] (246.5000,248.5000) -- (245.5000,248.5000);



  \end{scope}
  \begin{scope}[cm={{1.00588,0.0,0.0,1.00588,(-407.90165,297.66187)}},draw=blue,line cap=rect,line join=bevel,line width=0.800pt]
    \path[fill=blue] (0.0000,0.0000) node[above right] (text2144) {\scriptsize 6};



  \end{scope}
  \begin{scope}[cm={{0.92743,0.0,0.0,0.92743,(-570.4103,65.15216)}},draw=blue,line cap=round,line join=round,line width=0.480pt]
    \path[draw] (246.5000,229.5000) -- (245.5000,229.5000);



  \end{scope}
  \begin{scope}[cm={{1.00588,0.0,0.0,1.00588,(-411.98953,280.05087)}},draw=blue,line cap=rect,line join=bevel,line width=0.800pt]
    \path[fill=blue] (0.0000,0.0000) node[above right] (text2168) {\scriptsize 10};



  \end{scope}
  \begin{scope}[cm={{0.92743,0.0,0.0,0.92743,(-570.4103,65.15216)}},draw=blue,line cap=round,line join=round,line width=0.480pt]
    \path[draw] (184.5000,224.5000) -- (184.5000,272.5000) -- (246.5000,272.5000) -- (246.5000,224.5000) -- (184.5000,224.5000);



  \end{scope}
  \begin{scope}[cm={{0.0,-1.00588,1.00588,0.0,(-419.24335,303.67087)}},draw=blue,line cap=rect,line join=bevel,line width=0.800pt]
    \path[fill=blue] (0.0000,-5.9329) node[above right] (text2192) {\rotatebox{90}{\scriptsize $c_{i,2}$}};



  \end{scope}
  \begin{scope}[cm={{0.92743,0.0,0.0,0.92743,(-570.4103,65.15216)}},draw=blue,line cap=round,line join=round,line width=0.480pt]
    \path[draw] (184.8000,267.2000) -- (184.8000,267.2000) -- (184.9000,267.2000) -- (185.0000,267.2000) -- (185.0000,267.2000) -- (185.1000,267.2000) -- (185.1000,267.2000) -- (185.2000,267.2000) -- (185.3000,267.2000) -- (185.3000,267.2000) -- (185.4000,267.2000) -- (185.4000,267.2000) -- (185.5000,267.2000) -- (185.6000,267.2000) -- (185.6000,267.2000) -- (185.7000,267.2000) -- (185.8000,267.2000) -- (185.8000,267.2000) -- (185.9000,267.2000) -- (185.9000,267.2000) -- (186.0000,267.2000) -- (186.1000,267.2000) -- (186.1000,267.2000) -- (186.2000,267.2000) -- (186.2000,267.2000) -- (186.3000,267.2000) -- (186.4000,267.2000) -- (186.4000,267.2000) -- (186.5000,267.2000) -- (186.6000,267.2000) -- (186.6000,267.2000) -- (186.7000,267.2000) -- (186.7000,267.2000) -- (186.8000,267.2000) -- (186.9000,267.2000) -- (186.9000,267.2000) -- (187.0000,267.2000) -- (187.0000,267.2000) -- (187.1000,267.2000) -- (187.2000,267.2000) -- (187.2000,267.2000) -- (187.3000,267.2000) -- (187.4000,267.2000) -- (187.4000,267.2000) -- (187.5000,267.2000) -- (187.5000,267.2000) -- (187.6000,267.2000) -- (187.7000,267.2000) -- (187.7000,267.2000) -- (187.8000,267.2000) -- (187.8000,267.2000) -- (187.9000,267.2000) -- (188.0000,267.2000) -- (188.0000,267.2000) -- (188.1000,267.2000) -- (188.2000,267.2000) -- (188.2000,267.2000) -- (188.3000,267.2000) -- (188.3000,267.2000) -- (188.4000,267.2000) -- (188.5000,267.2000) -- (188.5000,267.2000) -- (188.6000,267.2000) -- (188.6000,267.2000) -- (188.7000,267.2000) -- (188.8000,267.2000) -- (188.8000,267.2000) -- (188.9000,267.2000) -- (189.0000,267.2000) -- (189.0000,267.2000) -- (189.1000,267.2000) -- (189.1000,267.2000) -- (189.2000,267.2000) -- (189.3000,267.2000) -- (189.3000,267.2000) -- (189.4000,267.2000) -- (189.4000,267.2000) -- (189.5000,267.2000) -- (189.6000,267.2000) -- (189.6000,267.2000) -- (189.7000,267.2000) -- (189.8000,267.2000) -- (189.8000,267.2000) -- (189.9000,267.2000) -- (189.9000,267.2000) -- (190.0000,267.2000) -- (190.1000,267.2000) -- (190.1000,267.2000) -- (190.2000,267.2000) -- (190.2000,267.2000) -- (190.3000,267.2000) -- (190.4000,267.2000) -- (190.4000,267.2000) -- (190.5000,267.2000) -- (190.5000,267.2000) -- (190.6000,267.2000) -- (190.7000,267.2000) -- (190.7000,267.2000) -- (190.8000,267.2000) -- (190.9000,267.2000) -- (190.9000,267.2000) -- (191.0000,267.2000) -- (191.0000,267.2000) -- (191.1000,267.2000) -- (191.2000,267.2000) -- (191.2000,267.2000) -- (191.3000,267.2000) -- (191.3000,267.2000) -- (191.4000,267.2000) -- (191.5000,267.2000) -- (191.5000,267.2000) -- (191.6000,267.2000) -- (191.7000,267.2000) -- (191.7000,267.2000) -- (191.8000,267.2000) -- (191.8000,267.2000) -- (191.9000,267.2000) -- (192.0000,267.2000) -- (192.0000,267.2000) -- (192.1000,267.2000) -- (192.1000,267.2000) -- (192.2000,267.2000) -- (192.3000,267.2000) -- (192.3000,267.2000) -- (192.4000,267.2000) -- (192.5000,267.2000) -- (192.5000,267.2000) -- (192.6000,267.3000) -- (192.6000,268.7000) -- (192.7000,244.7000) -- (192.8000,227.7000) -- (192.8000,229.2000) -- (192.9000,229.2000) -- (192.9000,229.2000) -- (193.0000,229.2000) -- (193.1000,229.2000) -- (193.1000,229.2000) -- (193.2000,229.2000) -- (193.3000,229.2000) -- (193.3000,229.2000) -- (193.4000,229.2000) -- (193.4000,229.2000) -- (193.5000,229.2000) -- (193.6000,229.2000) -- (193.6000,229.2000) -- (193.7000,229.2000) -- (193.7000,229.2000) -- (193.8000,229.2000) -- (193.9000,229.2000) -- (193.9000,229.2000) -- (194.0000,229.2000) -- (194.1000,229.2000) -- (194.1000,229.2000) -- (194.2000,229.2000) -- (194.2000,229.2000) -- (194.3000,229.2000) -- (194.4000,229.2000) -- (194.4000,229.2000) -- (194.5000,229.2000) -- (194.5000,229.2000) -- (194.6000,229.2000) -- (194.7000,229.2000) -- (194.7000,229.2000) -- (194.8000,229.2000) -- (194.9000,229.2000) -- (194.9000,229.2000) -- (195.0000,229.2000) -- (195.0000,229.2000) -- (195.1000,229.2000) -- (195.2000,229.2000) -- (195.2000,229.2000) -- (195.3000,229.2000) -- (195.3000,229.2000) -- (195.4000,229.2000) -- (195.5000,229.2000) -- (195.5000,229.2000) -- (195.6000,229.2000) -- (195.7000,229.2000) -- (195.7000,229.2000) -- (195.8000,229.2000) -- (195.8000,229.2000) -- (195.9000,229.2000) -- (196.0000,229.2000) -- (196.0000,229.2000) -- (196.1000,229.2000) -- (196.1000,229.2000) -- (196.2000,229.2000) -- (196.3000,229.2000) -- (196.3000,229.2000) -- (196.4000,229.2000) -- (196.5000,229.2000) -- (196.5000,229.2000) -- (196.6000,229.2000) -- (196.6000,229.2000) -- (196.7000,229.2000) -- (196.8000,229.2000) -- (196.8000,229.2000) -- (196.9000,229.2000) -- (196.9000,229.2000) -- (197.0000,229.2000) -- (197.1000,229.2000) -- (197.1000,229.2000) -- (197.2000,229.2000) -- (197.3000,229.2000) -- (197.3000,229.2000) -- (197.4000,229.2000) -- (197.4000,229.2000) -- (197.5000,229.2000) -- (197.6000,229.2000) -- (197.6000,229.2000) -- (197.7000,229.2000) -- (197.7000,229.2000) -- (197.8000,229.2000) -- (197.9000,229.2000) -- (197.9000,229.2000) -- (198.0000,229.2000) -- (198.1000,229.2000) -- (198.1000,229.2000) -- (198.2000,229.2000) -- (198.2000,229.2000) -- (198.3000,229.2000) -- (198.4000,229.2000) -- (198.4000,229.2000) -- (198.5000,229.2000) -- (198.5000,229.2000) -- (198.6000,229.2000) -- (198.7000,229.2000) -- (198.7000,229.2000) -- (198.8000,229.2000) -- (198.9000,229.2000) -- (198.9000,229.2000) -- (199.0000,229.2000) -- (199.0000,229.2000) -- (199.1000,229.2000) -- (199.2000,229.2000) -- (199.2000,229.2000) -- (199.3000,229.2000) -- (199.3000,229.2000) -- (199.4000,229.2000) -- (199.5000,229.2000) -- (199.5000,229.2000) -- (199.6000,229.2000) -- (199.7000,229.2000) -- (199.7000,229.2000) -- (199.8000,229.2000) -- (199.8000,229.2000) -- (199.9000,229.2000) -- (200.0000,229.2000) -- (200.0000,229.2000) -- (200.1000,229.2000) -- (200.1000,229.2000) -- (200.2000,229.2000) -- (200.3000,229.2000) -- (200.3000,229.2000) -- (200.4000,229.2000) -- (200.5000,229.2000) -- (200.5000,229.2000) -- (200.6000,229.2000) -- (200.6000,229.2000) -- (200.7000,229.2000) -- (200.8000,229.2000) -- (200.8000,229.2000) -- (200.9000,229.2000) -- (200.9000,229.2000) -- (201.0000,229.2000) -- (201.1000,229.2000) -- (201.1000,229.2000) -- (201.2000,229.2000) -- (201.3000,229.2000) -- (201.3000,229.2000) -- (201.4000,229.2000) -- (201.4000,229.2000) -- (201.5000,229.2000) -- (201.6000,229.2000) -- (201.6000,229.2000) -- (201.7000,229.2000) -- (201.7000,229.2000) -- (201.8000,229.2000) -- (201.9000,229.2000) -- (201.9000,229.2000) -- (202.0000,229.2000) -- (202.0000,229.2000) -- (202.1000,229.2000) -- (202.2000,229.2000) -- (202.2000,229.2000) -- (202.3000,229.2000) -- (202.4000,229.2000) -- (202.4000,229.2000) -- (202.5000,229.2000) -- (202.5000,229.2000) -- (202.6000,229.2000) -- (202.7000,229.2000) -- (202.7000,229.2000) -- (202.8000,229.2000) -- (202.8000,229.2000) -- (202.9000,229.2000) -- (203.0000,229.2000) -- (203.0000,229.2000) -- (203.1000,229.2000) -- (203.2000,229.2000) -- (203.2000,229.2000) -- (203.3000,229.2000) -- (203.3000,229.2000) -- (203.4000,229.2000) -- (203.5000,229.2000) -- (203.5000,229.2000) -- (203.6000,229.2000) -- (203.6000,229.2000) -- (203.7000,229.2000) -- (203.8000,229.2000) -- (203.8000,229.2000) -- (203.9000,229.2000) -- (204.0000,229.2000) -- (204.0000,229.2000) -- (204.1000,229.2000) -- (204.1000,229.2000) -- (204.2000,229.2000) -- (204.3000,229.2000) -- (204.3000,229.2000) -- (204.4000,229.2000) -- (204.4000,229.2000) -- (204.5000,229.2000) -- (204.6000,229.2000) -- (204.6000,229.2000) -- (204.7000,229.2000) -- (204.8000,229.2000) -- (204.8000,229.2000) -- (204.9000,229.2000) -- (204.9000,229.2000) -- (205.0000,229.2000) -- (205.1000,229.2000) -- (205.1000,229.2000) -- (205.2000,229.2000) -- (205.2000,229.2000) -- (205.3000,229.2000) -- (205.4000,229.2000) -- (205.4000,229.2000) -- (205.5000,229.2000) -- (205.6000,229.2000) -- (205.6000,229.2000) -- (205.7000,229.2000) -- (205.7000,229.2000) -- (205.8000,229.2000) -- (205.9000,229.2000) -- (205.9000,229.2000) -- (206.0000,229.2000) -- (206.0000,229.2000) -- (206.1000,229.2000) -- (206.2000,229.2000) -- (206.2000,229.2000) -- (206.3000,229.2000) -- (206.4000,229.2000) -- (206.4000,229.2000) -- (206.5000,229.2000) -- (206.5000,229.2000) -- (206.6000,229.2000) -- (206.7000,229.2000) -- (206.7000,229.2000) -- (206.8000,229.2000) -- (206.8000,229.2000) -- (206.9000,229.2000) -- (207.0000,229.2000) -- (207.0000,229.2000) -- (207.1000,229.2000) -- (207.2000,229.2000) -- (207.2000,229.2000) -- (207.3000,229.2000) -- (207.3000,229.2000) -- (207.4000,229.2000) -- (207.5000,229.2000) -- (207.5000,229.2000) -- (207.6000,229.2000) -- (207.6000,229.2000) -- (207.7000,229.2000) -- (207.8000,229.2000) -- (207.8000,229.2000) -- (207.9000,229.2000) -- (208.0000,229.2000) -- (208.0000,229.2000) -- (208.1000,229.2000) -- (208.1000,229.2000) -- (208.2000,229.2000) -- (208.3000,229.2000) -- (208.3000,229.2000) -- (208.4000,229.2000) -- (208.4000,229.2000) -- (208.5000,229.2000) -- (208.6000,229.2000) -- (208.6000,229.2000) -- (208.7000,229.2000) -- (208.8000,229.2000) -- (208.8000,229.2000) -- (208.9000,229.2000) -- (208.9000,229.2000) -- (209.0000,229.2000) -- (209.1000,229.2000) -- (209.1000,229.2000) -- (209.2000,229.2000) -- (209.2000,229.2000) -- (209.3000,229.2000) -- (209.4000,229.2000) -- (209.4000,229.2000) -- (209.5000,229.2000) -- (209.6000,229.2000) -- (209.6000,229.2000) -- (209.7000,229.2000) -- (209.7000,229.2000) -- (209.8000,229.2000) -- (209.9000,229.2000) -- (209.9000,229.2000) -- (210.0000,229.2000) -- (210.0000,229.2000) -- (210.1000,229.2000) -- (210.2000,229.2000) -- (210.2000,229.2000) -- (210.3000,229.2000) -- (210.4000,229.2000) -- (210.4000,229.2000) -- (210.5000,229.2000) -- (210.5000,229.2000) -- (210.6000,229.2000) -- (210.7000,229.2000) -- (210.7000,229.2000) -- (210.8000,229.2000) -- (210.8000,229.2000) -- (210.9000,229.2000) -- (211.0000,229.2000) -- (211.0000,229.2000) -- (211.1000,229.2000) -- (211.2000,229.2000) -- (211.2000,229.2000) -- (211.3000,229.2000) -- (211.3000,229.2000) -- (211.4000,229.2000) -- (211.5000,229.2000) -- (211.5000,229.2000) -- (211.6000,229.2000) -- (211.6000,229.2000) -- (211.7000,229.2000) -- (211.8000,229.2000) -- (211.8000,229.2000) -- (211.9000,229.2000) -- (212.0000,229.2000) -- (212.0000,229.2000) -- (212.1000,229.2000) -- (212.1000,229.2000) -- (212.2000,229.2000) -- (212.3000,229.2000) -- (212.3000,229.2000) -- (212.4000,229.2000) -- (212.4000,229.2000) -- (212.5000,229.2000) -- (212.6000,229.2000) -- (212.6000,229.2000) -- (212.7000,229.2000) -- (212.8000,229.2000) -- (212.8000,229.2000) -- (212.9000,229.2000) -- (212.9000,229.2000) -- (213.0000,229.2000) -- (213.1000,229.2000) -- (213.1000,229.2000) -- (213.2000,229.2000) -- (213.2000,229.2000) -- (213.3000,229.2000) -- (213.4000,229.2000) -- (213.4000,229.2000) -- (213.5000,229.2000) -- (213.6000,229.2000) -- (213.6000,229.2000) -- (213.7000,229.2000) -- (213.7000,229.2000) -- (213.8000,229.2000) -- (213.9000,229.2000) -- (213.9000,229.2000) -- (214.0000,229.2000) -- (214.0000,229.2000) -- (214.1000,229.2000) -- (214.2000,229.2000) -- (214.2000,229.2000) -- (214.3000,229.2000) -- (214.4000,229.2000) -- (214.4000,229.2000) -- (214.5000,229.2000) -- (214.5000,229.2000) -- (214.6000,229.2000) -- (214.7000,229.2000) -- (214.7000,229.2000) -- (214.8000,229.2000) -- (214.8000,229.2000) -- (214.9000,229.2000) -- (215.0000,229.2000) -- (215.0000,229.2000) -- (215.1000,229.2000) -- (215.1000,229.2000) -- (215.2000,229.2000) -- (215.3000,229.2000) -- (215.3000,229.2000) -- (215.4000,229.2000) -- (215.5000,229.2000) -- (215.5000,229.2000) -- (215.6000,229.2000) -- (215.6000,229.2000) -- (215.7000,229.2000) -- (215.8000,229.2000) -- (215.8000,229.2000) -- (215.9000,229.2000) -- (215.9000,228.9000) -- (216.0000,233.3000) -- (216.1000,239.1000) -- (216.1000,238.8000) -- (216.2000,238.7000) -- (216.3000,238.7000) -- (216.3000,238.7000) -- (216.4000,238.7000) -- (216.4000,238.7000) -- (216.5000,238.7000) -- (216.6000,238.7000) -- (216.6000,238.7000) -- (216.7000,238.7000) -- (216.7000,238.7000) -- (216.8000,238.7000) -- (216.9000,238.7000) -- (216.9000,238.7000) -- (217.0000,238.7000) -- (217.1000,238.7000) -- (217.1000,238.7000) -- (217.2000,238.7000) -- (217.2000,238.7000) -- (217.3000,238.7000) -- (217.4000,238.7000) -- (217.4000,238.7000) -- (217.5000,238.7000) -- (217.5000,238.7000) -- (217.6000,239.0000) -- (217.7000,236.4000) -- (217.7000,229.1000) -- (217.8000,229.2000) -- (217.9000,229.2000) -- (217.9000,229.2000) -- (218.0000,229.2000) -- (218.0000,229.2000) -- (218.1000,229.2000) -- (218.2000,229.2000) -- (218.2000,229.2000) -- (218.3000,229.2000) -- (218.3000,229.2000) -- (218.4000,229.2000) -- (218.5000,229.2000) -- (218.5000,229.2000) -- (218.6000,229.2000) -- (218.7000,229.2000) -- (218.7000,229.2000) -- (218.8000,229.2000) -- (218.8000,229.2000) -- (218.9000,229.2000) -- (219.0000,229.2000) -- (219.0000,229.2000) -- (219.1000,229.2000) -- (219.1000,229.2000) -- (219.2000,229.2000) -- (219.3000,229.2000) -- (219.3000,229.2000) -- (219.4000,229.2000) -- (219.5000,229.2000) -- (219.5000,229.2000) -- (219.6000,229.2000) -- (219.6000,229.2000) -- (219.7000,229.2000) -- (219.8000,229.2000) -- (219.8000,229.2000) -- (219.9000,229.2000) -- (219.9000,229.2000) -- (220.0000,229.2000) -- (220.1000,229.2000) -- (220.1000,228.9000) -- (220.2000,233.5000) -- (220.3000,239.1000) -- (220.3000,238.8000) -- (220.4000,238.7000) -- (220.4000,239.1000) -- (220.5000,247.0000) -- (220.6000,248.4000) -- (220.6000,248.2000) -- (220.7000,248.2000) -- (220.7000,248.2000) -- (220.8000,248.2000) -- (220.9000,248.2000) -- (220.9000,248.2000) -- (221.0000,248.2000) -- (221.1000,248.2000) -- (221.1000,248.2000) -- (221.2000,248.2000) -- (221.2000,248.2000) -- (221.3000,248.2000) -- (221.4000,248.2000) -- (221.4000,248.2000) -- (221.5000,248.2000) -- (221.5000,248.2000) -- (221.6000,248.2000) -- (221.7000,248.2000) -- (221.7000,248.2000) -- (221.8000,248.2000) -- (221.9000,248.2000) -- (221.9000,248.2000) -- (222.0000,248.2000) -- (222.0000,248.3000) -- (222.1000,248.5000) -- (222.2000,241.5000) -- (222.2000,238.4000) -- (222.3000,238.7000) -- (222.3000,239.0000) -- (222.4000,236.7000) -- (222.5000,229.2000) -- (222.5000,229.2000) -- (222.6000,229.2000) -- (222.7000,229.2000) -- (222.7000,229.2000) -- (222.8000,229.2000) -- (222.8000,229.2000) -- (222.9000,229.2000) -- (223.0000,229.2000) -- (223.0000,229.2000) -- (223.1000,229.2000) -- (223.1000,229.2000) -- (223.2000,229.2000) -- (223.3000,229.2000) -- (223.3000,229.2000) -- (223.4000,229.2000) -- (223.5000,229.2000) -- (223.5000,229.2000) -- (223.6000,229.2000) -- (223.6000,229.2000) -- (223.7000,229.2000) -- (223.8000,229.2000) -- (223.8000,229.2000) -- (223.9000,229.2000) -- (223.9000,229.2000) -- (224.0000,229.2000) -- (224.1000,229.2000) -- (224.1000,229.2000) -- (224.2000,229.2000) -- (224.3000,229.2000) -- (224.3000,229.2000) -- (224.4000,229.2000) -- (224.4000,229.8000) -- (224.5000,237.7000) -- (224.6000,238.8000) -- (224.6000,239.2000) -- (224.7000,247.1000) -- (224.7000,248.4000) -- (224.8000,248.2000) -- (224.9000,247.9000) -- (224.9000,252.1000) -- (225.0000,258.1000) -- (225.1000,257.8000) -- (225.1000,257.7000) -- (225.2000,257.7000) -- (225.2000,257.7000) -- (225.3000,257.7000) -- (225.4000,257.7000) -- (225.4000,257.7000) -- (225.5000,257.7000) -- (225.5000,257.7000) -- (225.6000,257.7000) -- (225.7000,257.7000) -- (225.7000,257.7000) -- (225.8000,257.7000) -- (225.9000,257.7000) -- (225.9000,257.7000) -- (226.0000,257.7000) -- (226.0000,257.7000) -- (226.1000,257.7000) -- (226.2000,257.7000) -- (226.2000,257.7000) -- (226.3000,257.7000) -- (226.3000,257.7000) -- (226.4000,257.7000) -- (226.5000,257.7000) -- (226.5000,258.0000) -- (226.6000,255.6000) -- (226.6000,248.2000) -- (226.7000,248.2000) -- (226.8000,248.3000) -- (226.8000,248.5000) -- (226.9000,241.8000) -- (227.0000,238.4000) -- (227.0000,239.1000) -- (227.1000,232.5000) -- (227.1000,228.9000) -- (227.2000,229.2000) -- (227.3000,229.2000) -- (227.3000,229.2000) -- (227.4000,229.2000) -- (227.4000,229.2000) -- (227.5000,229.2000) -- (227.6000,229.2000) -- (227.6000,229.2000) -- (227.7000,229.2000) -- (227.8000,229.2000) -- (227.8000,229.2000) -- (227.9000,229.2000) -- (227.9000,229.2000) -- (228.0000,229.2000) -- (228.1000,229.2000) -- (228.1000,229.2000) -- (228.2000,229.2000) -- (228.2000,229.2000) -- (228.3000,229.2000) -- (228.4000,229.2000) -- (228.4000,229.2000) -- (228.5000,229.2000) -- (228.6000,229.2000) -- (228.6000,229.2000) -- (228.7000,229.2000) -- (228.7000,229.2000) -- (228.8000,229.8000) -- (228.9000,237.3000) -- (228.9000,243.1000) -- (229.0000,248.6000) -- (229.0000,247.9000) -- (229.1000,252.3000) -- (229.2000,258.1000) -- (229.2000,257.8000) -- (229.3000,257.7000) -- (229.4000,258.0000) -- (229.4000,265.8000) -- (229.5000,267.4000) -- (229.5000,267.2000) -- (229.6000,267.2000) -- (229.7000,267.2000) -- (229.7000,267.2000) -- (229.8000,267.2000) -- (229.8000,267.2000) -- (229.9000,267.2000) -- (230.0000,267.2000) -- (230.0000,267.2000) -- (230.1000,267.2000) -- (230.2000,267.2000) -- (230.2000,267.2000) -- (230.3000,267.2000) -- (230.3000,267.2000) -- (230.4000,267.2000) -- (230.5000,267.2000) -- (230.5000,267.2000) -- (230.6000,267.2000) -- (230.6000,267.2000) -- (230.7000,267.2000) -- (230.8000,267.2000) -- (230.8000,267.2000) -- (230.9000,267.2000) -- (231.0000,267.3000) -- (231.0000,267.5000) -- (231.1000,260.7000) -- (231.1000,257.4000) -- (231.2000,257.7000) -- (231.3000,258.0000) -- (231.3000,255.9000) -- (231.4000,248.3000) -- (231.4000,248.4000) -- (231.5000,246.6000) -- (231.6000,238.9000) -- (231.6000,238.9000) -- (231.7000,237.2000) -- (231.8000,229.4000) -- (231.8000,229.2000) -- (231.9000,229.2000) -- (231.9000,229.2000) -- (232.0000,229.2000) -- (232.1000,229.2000) -- (232.1000,229.2000) -- (232.2000,229.2000) -- (232.2000,229.2000) -- (232.3000,229.2000) -- (232.4000,229.2000) -- (232.4000,229.2000) -- (232.5000,229.2000) -- (232.6000,229.2000) -- (232.6000,229.2000) -- (232.7000,229.2000) -- (232.7000,229.2000) -- (232.8000,229.2000) -- (232.9000,229.2000) -- (232.9000,229.2000) -- (233.0000,229.2000) -- (233.0000,229.2000) -- (233.1000,229.2000) -- (233.2000,229.8000) -- (233.2000,236.6000) -- (233.3000,251.6000) -- (233.4000,268.4000) -- (233.4000,267.3000) -- (233.5000,267.2000) -- (233.5000,267.2000) -- (233.6000,267.2000) -- (233.7000,267.2000) -- (233.7000,267.2000) -- (233.8000,267.2000) -- (233.8000,267.2000) -- (233.9000,267.2000) -- (234.0000,267.2000) -- (234.0000,267.2000) -- (234.1000,267.2000) -- (234.2000,267.2000) -- (234.2000,267.2000) -- (234.3000,267.2000) -- (234.3000,267.2000) -- (234.4000,267.2000) -- (234.5000,267.2000) -- (234.5000,267.2000) -- (234.6000,267.2000) -- (234.6000,267.2000) -- (234.7000,267.2000) -- (234.8000,267.2000) -- (234.8000,267.2000) -- (234.9000,267.2000) -- (235.0000,267.2000) -- (235.0000,267.2000) -- (235.1000,267.2000) -- (235.1000,267.2000) -- (235.2000,267.2000) -- (235.3000,267.2000) -- (235.3000,267.2000) -- (235.4000,267.2000) -- (235.4000,267.2000) -- (235.5000,267.2000) -- (235.6000,267.2000) -- (235.6000,267.2000) -- (235.7000,267.3000) -- (235.8000,267.6000) -- (235.8000,261.0000) -- (235.9000,257.4000) -- (235.9000,258.1000) -- (236.0000,251.7000) -- (236.1000,247.9000) -- (236.1000,248.6000) -- (236.2000,242.4000) -- (236.2000,238.4000) -- (236.3000,239.1000) -- (236.4000,233.1000) -- (236.4000,228.9000) -- (236.5000,229.2000) -- (236.6000,229.2000) -- (236.6000,229.2000) -- (236.7000,229.2000) -- (236.7000,229.2000) -- (236.8000,229.2000) -- (236.9000,229.2000) -- (236.9000,229.2000) -- (237.0000,229.2000) -- (237.0000,229.2000) -- (237.1000,229.1000) -- (237.2000,230.0000) -- (237.2000,238.0000) -- (237.3000,238.9000) -- (237.4000,238.7000) -- (237.4000,238.7000) -- (237.5000,238.7000) -- (237.5000,238.7000) -- (237.6000,237.6000) -- (237.7000,251.4000) -- (237.7000,268.4000) -- (237.8000,267.3000) -- (237.8000,267.2000) -- (237.9000,267.2000) -- (238.0000,267.2000) -- (238.0000,267.2000) -- (238.1000,267.2000) -- (238.1000,267.2000) -- (238.2000,267.2000) -- (238.3000,267.2000) -- (238.3000,267.2000) -- (238.4000,267.2000) -- (238.5000,267.2000) -- (238.5000,267.2000) -- (238.6000,267.2000) -- (238.6000,267.2000) -- (238.7000,267.2000) -- (238.8000,267.2000) -- (238.8000,267.2000) -- (238.9000,267.2000) -- (238.9000,267.2000) -- (239.0000,267.2000) -- (239.1000,267.2000) -- (239.1000,267.2000) -- (239.2000,267.2000) -- (239.3000,267.2000) -- (239.3000,267.2000) -- (239.4000,267.2000) -- (239.4000,267.2000) -- (239.5000,267.2000) -- (239.6000,267.2000) -- (239.6000,267.2000) -- (239.7000,267.2000) -- (239.7000,267.2000) -- (239.8000,267.2000) -- (239.9000,267.2000) -- (239.9000,267.2000) -- (240.0000,267.2000) -- (240.1000,267.2000) -- (240.1000,267.2000) -- (240.2000,267.2000) -- (240.2000,267.2000) -- (240.3000,267.2000) -- (240.4000,267.5000) -- (240.4000,265.7000) -- (240.5000,258.0000) -- (240.5000,257.9000) -- (240.6000,256.4000) -- (240.7000,248.5000) -- (240.7000,248.4000) -- (240.8000,247.0000) -- (240.9000,239.1000) -- (240.9000,238.7000) -- (241.0000,238.7000) -- (241.0000,238.7000) -- (241.1000,238.7000) -- (241.2000,238.7000) -- (241.2000,238.7000) -- (241.3000,238.7000) -- (241.3000,238.7000) -- (241.4000,238.7000) -- (241.5000,238.6000) -- (241.5000,239.5000) -- (241.6000,247.5000) -- (241.7000,248.4000) -- (241.7000,248.2000) -- (241.8000,248.2000) -- (241.8000,248.1000) -- (241.9000,249.3000) -- (242.0000,265.2000) -- (242.0000,267.6000) -- (242.1000,267.2000) -- (242.1000,267.2000) -- (242.2000,267.2000) -- (242.3000,267.2000) -- (242.3000,267.2000) -- (242.4000,267.2000) -- (242.5000,267.2000) -- (242.5000,267.2000) -- (242.6000,267.2000) -- (242.6000,267.2000) -- (242.7000,267.2000) -- (242.8000,267.2000) -- (242.8000,267.2000) -- (242.9000,267.2000) -- (242.9000,267.2000) -- (243.0000,267.2000) -- (243.1000,267.2000) -- (243.1000,267.2000) -- (243.2000,267.2000) -- (243.3000,267.2000) -- (243.3000,267.2000) -- (243.4000,267.2000) -- (243.4000,267.2000) -- (243.5000,267.2000) -- (243.6000,267.2000) -- (243.6000,267.2000) -- (243.7000,267.2000) -- (243.7000,267.2000) -- (243.8000,267.2000) -- (243.9000,267.2000) -- (243.9000,267.2000) -- (244.0000,267.2000) -- (244.1000,267.2000) -- (244.1000,267.2000) -- (244.2000,267.2000) -- (244.2000,267.2000) -- (244.3000,267.2000) -- (244.4000,267.2000) -- (244.4000,267.2000) -- (244.5000,267.2000) -- (244.5000,267.2000) -- (244.6000,267.2000) -- (244.7000,267.2000) -- (244.7000,267.2000) -- (244.8000,267.2000) -- (244.9000,267.2000) -- (244.9000,267.2000) -- (245.0000,267.3000) -- (245.0000,267.6000) -- (245.1000,261.6000) -- (245.2000,257.4000) -- (245.2000,258.1000) -- (245.3000,252.3000) -- (245.3000,247.9000) -- (245.4000,248.2000) -- (245.5000,248.2000) -- (245.5000,248.2000) -- (245.6000,248.2000) -- (245.7000,248.2000) -- (245.7000,248.2000) -- (245.8000,248.2000) -- (245.8000,248.2000) -- (245.9000,248.2000) -- (246.0000,247.9000) -- (246.0000,252.9000) -- (246.1000,258.1000) -- (246.1000,257.4000) -- (246.2000,262.2000) -- (246.3000,267.6000);



  \end{scope}
  \begin{scope}[cm={{0.92743,0.0,0.0,0.92743,(-570.4103,65.15216)}},draw=blue,line cap=round,line join=round,line width=0.480pt]
    \path[draw] (184.5000,224.5000) -- (184.5000,272.5000) -- (246.5000,272.5000) -- (246.5000,224.5000) -- (184.5000,224.5000);



  \end{scope}
  \begin{scope}[cm={{0.92743,0.0,0.0,0.92743,(-570.4103,65.15216)}},draw=ca0a0a4,dash pattern=on 2.59pt off 2.59pt,line cap=round,line join=round,line width=0.323pt,miter limit=4.00]
    \path[draw,dash pattern=on 2.59pt off 2.59pt,line width=0.323pt,miter limit=4.00] (184.5000,315.5000) -- (246.5000,315.5000);



  \end{scope}
  \begin{scope}[cm={{0.92743,0.0,0.0,0.92743,(-570.4103,65.15216)}},draw=blue,line cap=round,line join=round,line width=0.480pt]
    \path[draw] (184.5000,315.5000) -- (186.5000,315.5000);



    \path[draw] (246.5000,315.5000) -- (245.5000,315.5000);



  \end{scope}
  \begin{scope}[cm={{0.92743,0.0,0.0,0.92743,(-570.4103,65.15216)}},draw=ca0a0a4,dash pattern=on 2.59pt off 2.59pt,line cap=round,line join=round,line width=0.323pt,miter limit=4.00]
    \path[draw,dash pattern=on 2.59pt off 2.59pt,line width=0.323pt,miter limit=4.00] (184.5000,295.5000) -- (246.5000,295.5000);



  \end{scope}
  \begin{scope}[cm={{0.92743,0.0,0.0,0.92743,(-570.4103,65.15216)}},draw=blue,line cap=round,line join=round,line width=0.480pt]
    \path[draw] (184.5000,295.5000) -- (186.5000,295.5000);



    \path[draw] (246.5000,295.5000) -- (245.5000,295.5000);



  \end{scope}
  \begin{scope}[cm={{0.92743,0.0,0.0,0.92743,(-570.4103,65.15216)}},draw=ca0a0a4,dash pattern=on 2.59pt off 2.59pt,line cap=round,line join=round,line width=0.323pt,miter limit=4.00]
    \path[draw,dash pattern=on 2.59pt off 2.59pt,line width=0.323pt,miter limit=4.00] (184.5000,276.5000) -- (246.5000,276.5000);



  \end{scope}
  \begin{scope}[cm={{0.92743,0.0,0.0,0.92743,(-570.4103,65.15216)}},draw=blue,line cap=round,line join=round,line width=0.480pt]
    \path[draw] (184.5000,276.5000) -- (186.5000,276.5000);



    \path[draw] (246.5000,276.5000) -- (245.5000,276.5000);



  \end{scope}
  \begin{scope}[cm={{0.92743,0.0,0.0,0.92743,(-570.4103,65.15216)}},draw=ca0a0a4,dash pattern=on 0.40pt off 0.80pt,line cap=round,line join=round,line width=0.400pt]
    \path[draw] (184.5000,319.5000) -- (184.5000,272.5000);



  \end{scope}
  \begin{scope}[cm={{0.92743,0.0,0.0,0.92743,(-570.4103,65.15216)}},draw=blue,line cap=round,line join=round,line width=0.480pt]
    \path[draw] (184.5000,319.5000) -- (184.5000,318.5000);



    \path[draw] (184.5000,272.5000) -- (184.5000,273.5000);



  \end{scope}
  \begin{scope}[cm={{1.00588,0.0,0.0,1.00588,(-400.29035,376.888)}},draw=blue,line cap=rect,line join=bevel,line width=0.800pt]
    \path[fill=blue] (0.0000,0.0000) node[above right] (text2316) {\scriptsize 0};



  \end{scope}
  \begin{scope}[cm={{0.92743,0.0,0.0,0.92743,(-570.4103,65.15216)}},draw=ca0a0a4,dash pattern=on 2.59pt off 2.59pt,line cap=round,line join=round,line width=0.323pt,miter limit=4.00]
    \path[draw,dash pattern=on 2.59pt off 2.59pt,line width=0.323pt,miter limit=4.00] (201.5000,319.5000) -- (201.5000,272.5000);



  \end{scope}
  \begin{scope}[cm={{0.92743,0.0,0.0,0.92743,(-570.4103,65.15216)}},draw=blue,line cap=round,line join=round,line width=0.480pt]
    \path[draw] (201.5000,319.5000) -- (201.5000,318.5000);



    \path[draw] (201.5000,272.5000) -- (201.5000,273.5000);



  \end{scope}
  \begin{scope}[cm={{1.00588,0.0,0.0,1.00588,(-385.68735,376.888)}},draw=blue,line cap=rect,line join=bevel,line width=0.800pt]
    \path[fill=blue] (0.0000,0.0000) node[above right] (text2346) {\scriptsize 3};



  \end{scope}
  \begin{scope}[cm={{0.92743,0.0,0.0,0.92743,(-570.4103,65.15216)}},draw=ca0a0a4,dash pattern=on 2.59pt off 2.59pt,line cap=round,line join=round,line width=0.323pt,miter limit=4.00]
    \path[draw,dash pattern=on 2.59pt off 2.59pt,line width=0.323pt,miter limit=4.00] (218.5000,319.5000) -- (218.5000,272.5000);



  \end{scope}
  \begin{scope}[cm={{0.92743,0.0,0.0,0.92743,(-570.4103,65.15216)}},draw=blue,line cap=round,line join=round,line width=0.480pt]
    \path[draw] (218.5000,319.5000) -- (218.5000,318.5000);



    \path[draw] (218.5000,272.5000) -- (218.5000,273.5000);



  \end{scope}
  \begin{scope}[cm={{1.00588,0.0,0.0,1.00588,(-370.09635,376.888)}},draw=blue,line cap=rect,line join=bevel,line width=0.800pt]
    \path[fill=blue] (0.0000,0.0000) node[above right] (text2376) {\scriptsize 6};



  \end{scope}
  \begin{scope}[cm={{0.92743,0.0,0.0,0.92743,(-570.4103,65.15216)}},draw=ca0a0a4,dash pattern=on 2.59pt off 2.59pt,line cap=round,line join=round,line width=0.323pt,miter limit=4.00]
    \path[draw,dash pattern=on 2.59pt off 2.59pt,line width=0.323pt,miter limit=4.00] (235.5000,319.5000) -- (235.5000,272.5000);



  \end{scope}
  \begin{scope}[cm={{0.92743,0.0,0.0,0.92743,(-570.4103,65.15216)}},draw=blue,line cap=round,line join=round,line width=0.480pt]
    \path[draw] (235.5000,319.5000) -- (235.5000,318.5000);



    \path[draw] (235.5000,272.5000) -- (235.5000,273.5000);



  \end{scope}
  \begin{scope}[cm={{1.00588,0.0,0.0,1.00588,(-354.49635,376.82362)}},draw=blue,line cap=rect,line join=bevel,line width=0.800pt]
    \path[fill=blue] (0.0000,0.0000) node[above right] (text2406) {\scriptsize 9};



  \end{scope}
  \begin{scope}[cm={{0.92743,0.0,0.0,0.92743,(-570.4103,65.15216)}},draw=blue,line cap=round,line join=round,line width=0.480pt]
    \path[draw] (246.5000,315.5000) -- (245.5000,315.5000);



  \end{scope}
  \begin{scope}[cm={{1.00588,0.0,0.0,1.00588,(-422.71619,360.06187)}},draw=blue,line cap=rect,line join=bevel,line width=0.800pt]
    \path[fill=blue] (0.0000,0.0000) node[above right] (text2430) {\scriptsize -1000};



  \end{scope}
  \begin{scope}[cm={{0.92743,0.0,0.0,0.92743,(-570.4103,65.15216)}},draw=blue,line cap=round,line join=round,line width=0.480pt]
    \path[draw] (246.5000,295.5000) -- (245.5000,295.5000);



  \end{scope}
  \begin{scope}[cm={{1.00588,0.0,0.0,1.00588,(-418.69269,342.94487)}},draw=blue,line cap=rect,line join=bevel,line width=0.800pt]
    \path[fill=blue] (0.0000,0.0000) node[above right] (text2454) {\scriptsize -500};



  \end{scope}
  \begin{scope}[cm={{0.92743,0.0,0.0,0.92743,(-570.4103,65.15216)}},draw=blue,line cap=round,line join=round,line width=0.480pt]
    \path[draw] (246.5000,276.5000) -- (245.5000,276.5000);



  \end{scope}
  \begin{scope}[cm={{1.00588,0.0,0.0,1.00588,(-407.96603,324.32687)}},draw=blue,line cap=rect,line join=bevel,line width=0.800pt]
    \path[fill=blue] (0.0000,0.0000) node[above right] (text2478) {\scriptsize 0};



  \end{scope}
  \begin{scope}[cm={{0.92743,0.0,0.0,0.92743,(-570.4103,65.15216)}},draw=blue,line cap=round,line join=round,line width=0.480pt]
    \path[draw] (184.5000,272.5000) -- (184.5000,319.5000) -- (246.5000,319.5000) -- (246.5000,272.5000) -- (184.5000,272.5000);



  \end{scope}
  \begin{scope}[cm={{0.0,-1.00588,1.00588,0.0,(-425.21116,344.45387)}},draw=blue,line cap=rect,line join=bevel,line width=0.800pt]
    \path[fill=blue] (0.0000,0.0000) node[above right] (text2502) {\rotatebox{90}{\scriptsize $c_{i,1}$}};



  \end{scope}
  \begin{scope}[cm={{0.92743,0.0,0.0,0.92743,(-570.4103,65.15216)}},draw=blue,line cap=round,line join=round,line width=0.480pt]
    \path[draw] (184.8000,315.6000) -- (184.8000,315.6000) -- (184.9000,315.6000) -- (185.0000,315.6000) -- (185.0000,315.6000) -- (185.1000,315.6000) -- (185.1000,315.6000) -- (185.2000,315.6000) -- (185.3000,315.6000) -- (185.3000,315.6000) -- (185.4000,315.6000) -- (185.4000,315.6000) -- (185.5000,315.6000) -- (185.6000,315.6000) -- (185.6000,315.6000) -- (185.7000,315.6000) -- (185.8000,315.6000) -- (185.8000,315.6000) -- (185.9000,315.6000) -- (185.9000,315.6000) -- (186.0000,315.6000) -- (186.1000,315.6000) -- (186.1000,315.6000) -- (186.2000,315.6000) -- (186.2000,315.6000) -- (186.3000,315.6000) -- (186.4000,315.6000) -- (186.4000,315.6000) -- (186.5000,315.6000) -- (186.6000,315.6000) -- (186.6000,315.6000) -- (186.7000,315.6000) -- (186.7000,315.6000) -- (186.8000,315.6000) -- (186.9000,315.6000) -- (186.9000,315.6000) -- (187.0000,315.6000) -- (187.0000,315.6000) -- (187.1000,315.6000) -- (187.2000,315.6000) -- (187.2000,315.6000) -- (187.3000,315.6000) -- (187.4000,315.6000) -- (187.4000,315.6000) -- (187.5000,315.6000) -- (187.5000,315.6000) -- (187.6000,315.6000) -- (187.7000,315.6000) -- (187.7000,315.6000) -- (187.8000,315.6000) -- (187.8000,315.6000) -- (187.9000,315.6000) -- (188.0000,315.6000) -- (188.0000,315.6000) -- (188.1000,315.6000) -- (188.2000,315.6000) -- (188.2000,315.6000) -- (188.3000,315.6000) -- (188.3000,315.6000) -- (188.4000,315.6000) -- (188.5000,315.6000) -- (188.5000,315.6000) -- (188.6000,315.6000) -- (188.6000,315.6000) -- (188.7000,315.6000) -- (188.8000,315.6000) -- (188.8000,315.6000) -- (188.9000,315.6000) -- (189.0000,315.6000) -- (189.0000,315.6000) -- (189.1000,315.6000) -- (189.1000,315.6000) -- (189.2000,315.6000) -- (189.3000,315.6000) -- (189.3000,315.6000) -- (189.4000,315.6000) -- (189.4000,315.6000) -- (189.5000,315.6000) -- (189.6000,315.6000) -- (189.6000,315.6000) -- (189.7000,315.6000) -- (189.8000,315.6000) -- (189.8000,315.6000) -- (189.9000,315.6000) -- (189.9000,315.6000) -- (190.0000,315.6000) -- (190.1000,315.6000) -- (190.1000,315.6000) -- (190.2000,315.6000) -- (190.2000,315.6000) -- (190.3000,315.6000) -- (190.4000,315.6000) -- (190.4000,315.6000) -- (190.5000,315.6000) -- (190.5000,315.6000) -- (190.6000,315.6000) -- (190.7000,315.6000) -- (190.7000,315.6000) -- (190.8000,315.6000) -- (190.9000,315.6000) -- (190.9000,315.6000) -- (191.0000,315.6000) -- (191.0000,315.6000) -- (191.1000,315.6000) -- (191.2000,315.6000) -- (191.2000,315.6000) -- (191.3000,315.6000) -- (191.3000,315.6000) -- (191.4000,315.6000) -- (191.5000,315.6000) -- (191.5000,315.6000) -- (191.6000,315.6000) -- (191.7000,315.6000) -- (191.7000,315.6000) -- (191.8000,315.6000) -- (191.8000,315.6000) -- (191.9000,315.6000) -- (192.0000,315.6000) -- (192.0000,315.6000) -- (192.1000,315.6000) -- (192.1000,315.6000) -- (192.2000,315.6000) -- (192.3000,315.6000) -- (192.3000,315.6000) -- (192.4000,315.6000) -- (192.5000,315.6000) -- (192.5000,315.6000) -- (192.6000,315.6000) -- (192.6000,317.2000) -- (192.7000,292.4000) -- (192.8000,274.5000) -- (192.8000,276.1000) -- (192.9000,276.1000) -- (192.9000,276.1000) -- (193.0000,276.1000) -- (193.1000,276.1000) -- (193.1000,276.1000) -- (193.2000,276.1000) -- (193.3000,276.1000) -- (193.3000,276.1000) -- (193.4000,276.1000) -- (193.4000,276.1000) -- (193.5000,276.1000) -- (193.6000,276.1000) -- (193.6000,276.1000) -- (193.7000,276.1000) -- (193.7000,276.1000) -- (193.8000,276.1000) -- (193.9000,276.1000) -- (193.9000,276.1000) -- (194.0000,276.1000) -- (194.1000,276.1000) -- (194.1000,276.1000) -- (194.2000,276.1000) -- (194.2000,276.1000) -- (194.3000,276.1000) -- (194.4000,276.1000) -- (194.4000,276.1000) -- (194.5000,276.1000) -- (194.5000,276.1000) -- (194.6000,276.1000) -- (194.7000,276.1000) -- (194.7000,276.1000) -- (194.8000,276.1000) -- (194.9000,276.1000) -- (194.9000,276.1000) -- (195.0000,276.1000) -- (195.0000,276.1000) -- (195.1000,276.1000) -- (195.2000,276.1000) -- (195.2000,276.1000) -- (195.3000,276.1000) -- (195.3000,276.1000) -- (195.4000,276.1000) -- (195.5000,276.1000) -- (195.5000,276.1000) -- (195.6000,276.1000) -- (195.7000,276.1000) -- (195.7000,276.1000) -- (195.8000,276.1000) -- (195.8000,276.1000) -- (195.9000,276.1000) -- (196.0000,276.1000) -- (196.0000,276.1000) -- (196.1000,276.1000) -- (196.1000,276.1000) -- (196.2000,276.1000) -- (196.3000,276.1000) -- (196.3000,276.1000) -- (196.4000,276.1000) -- (196.5000,276.1000) -- (196.5000,276.1000) -- (196.6000,276.1000) -- (196.6000,276.1000) -- (196.7000,276.1000) -- (196.8000,276.1000) -- (196.8000,276.1000) -- (196.9000,276.1000) -- (196.9000,276.1000) -- (197.0000,276.1000) -- (197.1000,276.1000) -- (197.1000,276.1000) -- (197.2000,276.1000) -- (197.3000,276.1000) -- (197.3000,276.1000) -- (197.4000,276.1000) -- (197.4000,276.1000) -- (197.5000,276.1000) -- (197.6000,276.1000) -- (197.6000,276.1000) -- (197.7000,276.1000) -- (197.7000,276.1000) -- (197.8000,276.1000) -- (197.9000,276.1000) -- (197.9000,276.1000) -- (198.0000,276.1000) -- (198.1000,276.1000) -- (198.1000,276.1000) -- (198.2000,276.1000) -- (198.2000,276.1000) -- (198.3000,276.1000) -- (198.4000,276.1000) -- (198.4000,276.1000) -- (198.5000,276.1000) -- (198.5000,276.1000) -- (198.6000,276.1000) -- (198.7000,276.1000) -- (198.7000,276.1000) -- (198.8000,276.1000) -- (198.9000,276.1000) -- (198.9000,276.1000) -- (199.0000,276.1000) -- (199.0000,276.1000) -- (199.1000,276.1000) -- (199.2000,276.1000) -- (199.2000,276.1000) -- (199.3000,276.1000) -- (199.3000,276.1000) -- (199.4000,276.1000) -- (199.5000,276.1000) -- (199.5000,276.1000) -- (199.6000,276.1000) -- (199.7000,276.1000) -- (199.7000,276.1000) -- (199.8000,276.1000) -- (199.8000,276.1000) -- (199.9000,276.1000) -- (200.0000,276.1000) -- (200.0000,276.1000) -- (200.1000,276.1000) -- (200.1000,276.1000) -- (200.2000,276.1000) -- (200.3000,276.1000) -- (200.3000,276.1000) -- (200.4000,276.1000) -- (200.5000,276.1000) -- (200.5000,276.1000) -- (200.6000,276.1000) -- (200.6000,276.1000) -- (200.7000,276.1000) -- (200.8000,276.1000) -- (200.8000,276.1000) -- (200.9000,276.1000) -- (200.9000,276.1000) -- (201.0000,276.1000) -- (201.1000,276.1000) -- (201.1000,276.1000) -- (201.2000,276.1000) -- (201.3000,276.1000) -- (201.3000,276.1000) -- (201.4000,276.1000) -- (201.4000,276.1000) -- (201.5000,276.1000) -- (201.6000,276.1000) -- (201.6000,276.1000) -- (201.7000,276.1000) -- (201.7000,276.1000) -- (201.8000,276.1000) -- (201.9000,276.1000) -- (201.9000,276.1000) -- (202.0000,276.1000) -- (202.0000,276.1000) -- (202.1000,276.1000) -- (202.2000,276.1000) -- (202.2000,276.1000) -- (202.3000,276.1000) -- (202.4000,276.1000) -- (202.4000,276.1000) -- (202.5000,276.1000) -- (202.5000,276.1000) -- (202.6000,276.1000) -- (202.7000,276.1000) -- (202.7000,276.1000) -- (202.8000,276.1000) -- (202.8000,276.1000) -- (202.9000,276.1000) -- (203.0000,276.1000) -- (203.0000,276.1000) -- (203.1000,276.1000) -- (203.2000,276.1000) -- (203.2000,276.1000) -- (203.3000,276.1000) -- (203.3000,276.1000) -- (203.4000,276.1000) -- (203.5000,276.1000) -- (203.5000,276.1000) -- (203.6000,276.1000) -- (203.6000,276.1000) -- (203.7000,276.1000) -- (203.8000,276.1000) -- (203.8000,276.1000) -- (203.9000,276.1000) -- (204.0000,276.1000) -- (204.0000,276.1000) -- (204.1000,276.1000) -- (204.1000,276.1000) -- (204.2000,276.1000) -- (204.3000,276.1000) -- (204.3000,276.1000) -- (204.4000,276.1000) -- (204.4000,276.1000) -- (204.5000,276.1000) -- (204.6000,276.1000) -- (204.6000,276.1000) -- (204.7000,276.1000) -- (204.8000,276.1000) -- (204.8000,276.1000) -- (204.9000,276.1000) -- (204.9000,276.1000) -- (205.0000,276.1000) -- (205.1000,276.1000) -- (205.1000,276.1000) -- (205.2000,276.1000) -- (205.2000,276.1000) -- (205.3000,276.1000) -- (205.4000,276.1000) -- (205.4000,276.1000) -- (205.5000,276.1000) -- (205.6000,276.1000) -- (205.6000,276.1000) -- (205.7000,276.1000) -- (205.7000,276.1000) -- (205.8000,276.1000) -- (205.9000,276.1000) -- (205.9000,276.1000) -- (206.0000,276.1000) -- (206.0000,276.1000) -- (206.1000,276.1000) -- (206.2000,276.1000) -- (206.2000,276.1000) -- (206.3000,276.1000) -- (206.4000,276.1000) -- (206.4000,276.1000) -- (206.5000,276.1000) -- (206.5000,276.1000) -- (206.6000,276.1000) -- (206.7000,276.1000) -- (206.7000,276.1000) -- (206.8000,276.1000) -- (206.8000,276.1000) -- (206.9000,276.1000) -- (207.0000,276.1000) -- (207.0000,276.1000) -- (207.1000,276.1000) -- (207.2000,276.1000) -- (207.2000,276.1000) -- (207.3000,276.1000) -- (207.3000,276.1000) -- (207.4000,276.1000) -- (207.5000,276.1000) -- (207.5000,276.1000) -- (207.6000,276.1000) -- (207.6000,276.1000) -- (207.7000,276.1000) -- (207.8000,276.1000) -- (207.8000,276.1000) -- (207.9000,276.1000) -- (208.0000,276.1000) -- (208.0000,276.1000) -- (208.1000,276.1000) -- (208.1000,276.1000) -- (208.2000,276.1000) -- (208.3000,276.1000) -- (208.3000,276.1000) -- (208.4000,276.1000) -- (208.4000,276.1000) -- (208.5000,276.1000) -- (208.6000,276.1000) -- (208.6000,276.1000) -- (208.7000,276.1000) -- (208.8000,276.1000) -- (208.8000,276.1000) -- (208.9000,276.1000) -- (208.9000,276.1000) -- (209.0000,276.1000) -- (209.1000,276.1000) -- (209.1000,276.1000) -- (209.2000,276.1000) -- (209.2000,276.1000) -- (209.3000,276.1000) -- (209.4000,276.1000) -- (209.4000,276.1000) -- (209.5000,276.1000) -- (209.6000,276.1000) -- (209.6000,276.1000) -- (209.7000,276.1000) -- (209.7000,276.1000) -- (209.8000,276.1000) -- (209.9000,276.1000) -- (209.9000,276.1000) -- (210.0000,276.1000) -- (210.0000,276.1000) -- (210.1000,276.1000) -- (210.2000,276.1000) -- (210.2000,276.1000) -- (210.3000,276.1000) -- (210.4000,276.1000) -- (210.4000,276.1000) -- (210.5000,276.1000) -- (210.5000,276.1000) -- (210.6000,276.1000) -- (210.7000,276.1000) -- (210.7000,276.1000) -- (210.8000,276.1000) -- (210.8000,276.1000) -- (210.9000,276.1000) -- (211.0000,276.1000) -- (211.0000,276.1000) -- (211.1000,276.1000) -- (211.2000,276.1000) -- (211.2000,276.1000) -- (211.3000,276.1000) -- (211.3000,276.1000) -- (211.4000,276.1000) -- (211.5000,276.1000) -- (211.5000,276.1000) -- (211.6000,276.1000) -- (211.6000,276.1000) -- (211.7000,276.1000) -- (211.8000,276.1000) -- (211.8000,276.1000) -- (211.9000,276.1000) -- (212.0000,276.1000) -- (212.0000,276.1000) -- (212.1000,276.1000) -- (212.1000,276.1000) -- (212.2000,276.1000) -- (212.3000,276.1000) -- (212.3000,276.1000) -- (212.4000,276.1000) -- (212.4000,276.1000) -- (212.5000,276.1000) -- (212.6000,276.1000) -- (212.6000,276.1000) -- (212.7000,276.1000) -- (212.8000,276.1000) -- (212.8000,276.1000) -- (212.9000,276.1000) -- (212.9000,276.1000) -- (213.0000,276.1000) -- (213.1000,276.1000) -- (213.1000,276.1000) -- (213.2000,276.1000) -- (213.2000,276.1000) -- (213.3000,276.1000) -- (213.4000,276.1000) -- (213.4000,276.1000) -- (213.5000,276.1000) -- (213.6000,276.1000) -- (213.6000,276.1000) -- (213.7000,276.1000) -- (213.7000,276.1000) -- (213.8000,276.1000) -- (213.9000,276.1000) -- (213.9000,276.1000) -- (214.0000,276.1000) -- (214.0000,276.1000) -- (214.1000,276.1000) -- (214.2000,276.1000) -- (214.2000,276.1000) -- (214.3000,276.1000) -- (214.4000,276.1000) -- (214.4000,276.1000) -- (214.5000,276.1000) -- (214.5000,276.1000) -- (214.6000,276.1000) -- (214.7000,276.1000) -- (214.7000,276.1000) -- (214.8000,276.1000) -- (214.8000,276.1000) -- (214.9000,276.1000) -- (215.0000,276.1000) -- (215.0000,276.1000) -- (215.1000,276.1000) -- (215.1000,276.1000) -- (215.2000,276.1000) -- (215.3000,276.1000) -- (215.3000,276.1000) -- (215.4000,276.1000) -- (215.5000,276.1000) -- (215.5000,276.1000) -- (215.6000,276.1000) -- (215.6000,276.1000) -- (215.7000,276.1000) -- (215.8000,276.1000) -- (215.8000,276.1000) -- (215.9000,276.1000) -- (215.9000,276.1000) -- (216.0000,276.1000) -- (216.1000,276.1000) -- (216.1000,276.1000) -- (216.2000,276.1000) -- (216.3000,276.1000) -- (216.3000,276.1000) -- (216.4000,276.1000) -- (216.4000,276.1000) -- (216.5000,276.1000) -- (216.6000,276.1000) -- (216.6000,276.1000) -- (216.7000,276.1000) -- (216.7000,276.1000) -- (216.8000,276.1000) -- (216.9000,276.1000) -- (216.9000,276.1000) -- (217.0000,276.1000) -- (217.1000,276.1000) -- (217.1000,276.1000) -- (217.2000,276.1000) -- (217.2000,276.1000) -- (217.3000,276.1000) -- (217.4000,276.1000) -- (217.4000,276.1000) -- (217.5000,276.1000) -- (217.5000,276.1000) -- (217.6000,276.1000) -- (217.7000,276.1000) -- (217.7000,276.1000) -- (217.8000,276.1000) -- (217.9000,276.1000) -- (217.9000,276.1000) -- (218.0000,276.1000) -- (218.0000,276.1000) -- (218.1000,276.1000) -- (218.2000,276.1000) -- (218.2000,276.1000) -- (218.3000,276.1000) -- (218.3000,276.1000) -- (218.4000,276.1000) -- (218.5000,276.1000) -- (218.5000,276.1000) -- (218.6000,276.1000) -- (218.7000,276.1000) -- (218.7000,276.1000) -- (218.8000,276.1000) -- (218.8000,276.1000) -- (218.9000,276.1000) -- (219.0000,276.1000) -- (219.0000,276.1000) -- (219.1000,276.1000) -- (219.1000,276.1000) -- (219.2000,276.1000) -- (219.3000,276.1000) -- (219.3000,276.1000) -- (219.4000,276.1000) -- (219.5000,276.1000) -- (219.5000,276.1000) -- (219.6000,276.1000) -- (219.6000,276.1000) -- (219.7000,276.1000) -- (219.8000,276.1000) -- (219.8000,276.1000) -- (219.9000,276.1000) -- (219.9000,276.1000) -- (220.0000,276.1000) -- (220.1000,276.1000) -- (220.1000,276.1000) -- (220.2000,276.1000) -- (220.3000,276.1000) -- (220.3000,276.1000) -- (220.4000,276.1000) -- (220.4000,276.1000) -- (220.5000,276.1000) -- (220.6000,276.1000) -- (220.6000,276.1000) -- (220.7000,276.1000) -- (220.7000,276.1000) -- (220.8000,276.1000) -- (220.9000,276.1000) -- (220.9000,276.1000) -- (221.0000,276.1000) -- (221.1000,276.1000) -- (221.1000,276.1000) -- (221.2000,276.1000) -- (221.2000,276.1000) -- (221.3000,276.1000) -- (221.4000,276.1000) -- (221.4000,276.1000) -- (221.5000,276.1000) -- (221.5000,276.1000) -- (221.6000,276.1000) -- (221.7000,276.1000) -- (221.7000,276.1000) -- (221.8000,276.1000) -- (221.9000,276.1000) -- (221.9000,276.1000) -- (222.0000,276.1000) -- (222.0000,276.1000) -- (222.1000,276.1000) -- (222.2000,276.1000) -- (222.2000,276.1000) -- (222.3000,276.1000) -- (222.3000,276.1000) -- (222.4000,276.1000) -- (222.5000,276.1000) -- (222.5000,276.1000) -- (222.6000,276.1000) -- (222.7000,276.1000) -- (222.7000,276.1000) -- (222.8000,276.1000) -- (222.8000,276.1000) -- (222.9000,276.1000) -- (223.0000,276.1000) -- (223.0000,276.1000) -- (223.1000,276.1000) -- (223.1000,276.1000) -- (223.2000,276.1000) -- (223.3000,276.1000) -- (223.3000,276.1000) -- (223.4000,276.1000) -- (223.5000,276.1000) -- (223.5000,276.1000) -- (223.6000,276.1000) -- (223.6000,276.1000) -- (223.7000,276.1000) -- (223.8000,276.1000) -- (223.8000,276.1000) -- (223.9000,276.1000) -- (223.9000,276.1000) -- (224.0000,276.1000) -- (224.1000,276.1000) -- (224.1000,276.1000) -- (224.2000,276.1000) -- (224.3000,276.1000) -- (224.3000,276.1000) -- (224.4000,276.1000) -- (224.4000,276.1000) -- (224.5000,276.1000) -- (224.6000,276.1000) -- (224.6000,276.1000) -- (224.7000,276.1000) -- (224.7000,276.1000) -- (224.8000,276.1000) -- (224.9000,276.1000) -- (224.9000,276.1000) -- (225.0000,276.1000) -- (225.1000,276.1000) -- (225.1000,276.1000) -- (225.2000,276.1000) -- (225.2000,276.1000) -- (225.3000,276.1000) -- (225.4000,276.1000) -- (225.4000,276.1000) -- (225.5000,276.1000) -- (225.5000,276.1000) -- (225.6000,276.1000) -- (225.7000,276.1000) -- (225.7000,276.1000) -- (225.8000,276.1000) -- (225.9000,276.1000) -- (225.9000,276.1000) -- (226.0000,276.1000) -- (226.0000,276.1000) -- (226.1000,276.1000) -- (226.2000,276.1000) -- (226.2000,276.1000) -- (226.3000,276.1000) -- (226.3000,276.1000) -- (226.4000,276.1000) -- (226.5000,276.1000) -- (226.5000,276.1000) -- (226.6000,276.1000) -- (226.6000,276.1000) -- (226.7000,276.1000) -- (226.8000,276.1000) -- (226.8000,276.1000) -- (226.9000,276.1000) -- (227.0000,276.1000) -- (227.0000,276.1000) -- (227.1000,276.1000) -- (227.1000,276.1000) -- (227.2000,276.1000) -- (227.3000,276.1000) -- (227.3000,276.1000) -- (227.4000,276.1000) -- (227.4000,276.1000) -- (227.5000,276.1000) -- (227.6000,276.1000) -- (227.6000,276.1000) -- (227.7000,276.1000) -- (227.8000,276.1000) -- (227.8000,276.1000) -- (227.9000,276.1000) -- (227.9000,276.1000) -- (228.0000,276.1000) -- (228.1000,276.1000) -- (228.1000,276.1000) -- (228.2000,276.1000) -- (228.2000,276.1000) -- (228.3000,276.1000) -- (228.4000,276.1000) -- (228.4000,276.1000) -- (228.5000,276.1000) -- (228.6000,276.1000) -- (228.6000,276.1000) -- (228.7000,276.1000) -- (228.7000,276.1000) -- (228.8000,276.1000) -- (228.9000,276.1000) -- (228.9000,276.1000) -- (229.0000,276.1000) -- (229.0000,276.1000) -- (229.1000,276.1000) -- (229.2000,276.1000) -- (229.2000,276.1000) -- (229.3000,276.1000) -- (229.4000,276.1000) -- (229.4000,276.1000) -- (229.5000,276.1000) -- (229.5000,276.1000) -- (229.6000,276.1000) -- (229.7000,276.1000) -- (229.7000,276.1000) -- (229.8000,276.1000) -- (229.8000,276.1000) -- (229.9000,276.1000) -- (230.0000,276.1000) -- (230.0000,276.1000) -- (230.1000,276.1000) -- (230.2000,276.1000) -- (230.2000,276.1000) -- (230.3000,276.1000) -- (230.3000,276.1000) -- (230.4000,276.1000) -- (230.5000,276.1000) -- (230.5000,276.1000) -- (230.6000,276.1000) -- (230.6000,276.1000) -- (230.7000,276.1000) -- (230.8000,276.1000) -- (230.8000,276.1000) -- (230.9000,276.1000) -- (231.0000,276.1000) -- (231.0000,276.1000) -- (231.1000,276.1000) -- (231.1000,276.1000) -- (231.2000,276.1000) -- (231.3000,276.1000) -- (231.3000,276.1000) -- (231.4000,276.1000) -- (231.4000,276.1000) -- (231.5000,276.1000) -- (231.6000,276.1000) -- (231.6000,276.1000) -- (231.7000,276.1000) -- (231.8000,276.1000) -- (231.8000,276.1000) -- (231.9000,276.1000) -- (231.9000,276.1000) -- (232.0000,276.1000) -- (232.1000,276.1000) -- (232.1000,276.1000) -- (232.2000,276.1000) -- (232.2000,276.1000) -- (232.3000,276.1000) -- (232.4000,276.1000) -- (232.4000,276.1000) -- (232.5000,276.1000) -- (232.6000,276.1000) -- (232.6000,276.1000) -- (232.7000,276.1000) -- (232.7000,276.1000) -- (232.8000,276.1000) -- (232.9000,276.1000) -- (232.9000,276.1000) -- (233.0000,276.1000) -- (233.0000,276.1000) -- (233.1000,276.1000) -- (233.2000,276.1000) -- (233.2000,276.1000) -- (233.3000,276.1000) -- (233.4000,276.1000) -- (233.4000,276.1000) -- (233.5000,276.1000) -- (233.5000,276.1000) -- (233.6000,276.1000) -- (233.7000,276.1000) -- (233.7000,276.1000) -- (233.8000,276.1000) -- (233.8000,276.1000) -- (233.9000,276.1000) -- (234.0000,276.1000) -- (234.0000,276.1000) -- (234.1000,276.1000) -- (234.2000,276.1000) -- (234.2000,276.1000) -- (234.3000,276.1000) -- (234.3000,276.1000) -- (234.4000,276.1000) -- (234.5000,276.1000) -- (234.5000,276.1000) -- (234.6000,276.1000) -- (234.6000,276.1000) -- (234.7000,276.1000) -- (234.8000,276.1000) -- (234.8000,276.1000) -- (234.9000,276.1000) -- (235.0000,276.1000) -- (235.0000,276.1000) -- (235.1000,276.1000) -- (235.1000,276.1000) -- (235.2000,276.1000) -- (235.3000,276.1000) -- (235.3000,276.1000) -- (235.4000,276.1000) -- (235.4000,276.1000) -- (235.5000,276.1000) -- (235.6000,276.1000) -- (235.6000,276.1000) -- (235.7000,276.1000) -- (235.8000,276.1000) -- (235.8000,276.1000) -- (235.9000,276.1000) -- (235.9000,276.1000) -- (236.0000,276.1000) -- (236.1000,276.1000) -- (236.1000,276.1000) -- (236.2000,276.1000) -- (236.2000,276.1000) -- (236.3000,276.1000) -- (236.4000,276.1000) -- (236.4000,276.1000) -- (236.5000,276.1000) -- (236.6000,276.1000) -- (236.6000,276.1000) -- (236.7000,276.1000) -- (236.7000,276.1000) -- (236.8000,276.1000) -- (236.9000,276.1000) -- (236.9000,276.1000) -- (237.0000,276.1000) -- (237.0000,276.1000) -- (237.1000,276.1000) -- (237.2000,276.1000) -- (237.2000,276.1000) -- (237.3000,276.1000) -- (237.4000,276.1000) -- (237.4000,276.1000) -- (237.5000,276.1000) -- (237.5000,276.1000) -- (237.6000,276.1000) -- (237.7000,276.1000) -- (237.7000,276.1000) -- (237.8000,276.1000) -- (237.8000,276.1000) -- (237.9000,276.1000) -- (238.0000,276.1000) -- (238.0000,276.1000) -- (238.1000,276.1000) -- (238.1000,276.1000) -- (238.2000,276.1000) -- (238.3000,276.1000) -- (238.3000,276.1000) -- (238.4000,276.1000) -- (238.5000,276.1000) -- (238.5000,276.1000) -- (238.6000,276.1000) -- (238.6000,276.1000) -- (238.7000,276.1000) -- (238.8000,276.1000) -- (238.8000,276.1000) -- (238.9000,276.1000) -- (238.9000,276.1000) -- (239.0000,276.1000) -- (239.1000,276.1000) -- (239.1000,276.1000) -- (239.2000,276.1000) -- (239.3000,276.1000) -- (239.3000,276.1000) -- (239.4000,276.1000) -- (239.4000,276.1000) -- (239.5000,276.1000) -- (239.6000,276.1000) -- (239.6000,276.1000) -- (239.7000,276.1000) -- (239.7000,276.1000) -- (239.8000,276.1000) -- (239.9000,276.1000) -- (239.9000,276.1000) -- (240.0000,276.1000) -- (240.1000,276.1000) -- (240.1000,276.1000) -- (240.2000,276.1000) -- (240.2000,276.1000) -- (240.3000,276.1000) -- (240.4000,276.1000) -- (240.4000,276.1000) -- (240.5000,276.1000) -- (240.5000,276.1000) -- (240.6000,276.1000) -- (240.7000,276.1000) -- (240.7000,276.1000) -- (240.8000,276.1000) -- (240.9000,276.1000) -- (240.9000,276.1000) -- (241.0000,276.1000) -- (241.0000,276.1000) -- (241.1000,276.1000) -- (241.2000,276.1000) -- (241.2000,276.1000) -- (241.3000,276.1000) -- (241.3000,276.1000) -- (241.4000,276.1000) -- (241.5000,276.1000) -- (241.5000,276.1000) -- (241.6000,276.1000) -- (241.7000,276.1000) -- (241.7000,276.1000) -- (241.8000,276.1000) -- (241.8000,276.1000) -- (241.9000,276.1000) -- (242.0000,276.1000) -- (242.0000,276.1000) -- (242.1000,276.1000) -- (242.1000,276.1000) -- (242.2000,276.1000) -- (242.3000,276.1000) -- (242.3000,276.1000) -- (242.4000,276.1000) -- (242.5000,276.1000) -- (242.5000,276.1000) -- (242.6000,276.1000) -- (242.6000,276.1000) -- (242.7000,276.1000) -- (242.8000,276.1000) -- (242.8000,276.1000) -- (242.9000,276.1000) -- (242.9000,276.1000) -- (243.0000,276.1000) -- (243.1000,276.1000) -- (243.1000,276.1000) -- (243.2000,276.1000) -- (243.3000,276.1000) -- (243.3000,276.1000) -- (243.4000,276.1000) -- (243.4000,276.1000) -- (243.5000,276.1000) -- (243.6000,276.1000) -- (243.6000,276.1000) -- (243.7000,276.1000) -- (243.7000,276.1000) -- (243.8000,276.1000) -- (243.9000,276.1000) -- (243.9000,276.1000) -- (244.0000,276.1000) -- (244.1000,276.1000) -- (244.1000,276.1000) -- (244.2000,276.1000) -- (244.2000,276.1000) -- (244.3000,276.1000) -- (244.4000,276.1000) -- (244.4000,276.1000) -- (244.5000,276.1000) -- (244.5000,276.1000) -- (244.6000,276.1000) -- (244.7000,276.1000) -- (244.7000,276.1000) -- (244.8000,276.1000) -- (244.9000,276.1000) -- (244.9000,276.1000) -- (245.0000,276.1000) -- (245.0000,276.1000) -- (245.1000,276.1000) -- (245.2000,276.1000) -- (245.2000,276.1000) -- (245.3000,276.1000) -- (245.3000,276.1000) -- (245.4000,276.1000) -- (245.5000,276.1000) -- (245.5000,276.1000) -- (245.6000,276.1000) -- (245.7000,276.1000) -- (245.7000,276.1000) -- (245.8000,276.1000) -- (245.8000,276.1000) -- (245.9000,276.1000) -- (246.0000,276.1000) -- (246.0000,276.1000) -- (246.1000,276.1000) -- (246.1000,276.1000) -- (246.2000,276.1000) -- (246.3000,276.1000);



  \end{scope}
  \begin{scope}[cm={{1.00588,0.0,0.0,1.00588,(-510.7994,343.95446)}},draw=blue,line cap=rect,line join=bevel,line width=0.800pt]
    \path[fill=blue] (-4.4737,45.8089) node[above right] (text344-6) {x (m)};



    \path[fill=blue] (120.9636,45.7382) node[above right] (text344-6-4) {\scriptsize Time (min)};



    \path[fill=blue] (-49.9760,62.5230) node[above right] (text344-6-3-1) {(a) Re-planned trajectories};



    \path[fill=blue] (81.5138,62.5230) node[above right] (text344-6-3-1-6) {(b) Parameters ($c_{i,1},c_{i,2}$) evol.};



    \path[fill=blue] (277.1823,62.4148) node[above right] (text344-6-3-1-6-3) {(c) Energies and batteries evol.};



  \end{scope}
  \begin{scope}[cm={{1.00588,0.0,0.0,1.00588,(-407.7468,193.72978)}},draw=blue,line cap=rect,line join=bevel,line width=0.800pt]
    \path[fill=blue] (0.0000,0.0000) node[above right] (text2120-2) {\scriptsize 2};



  \end{scope}
  \begin{scope}[cm={{1.00588,0.0,0.0,1.00588,(-407.69047,177.61778)}},draw=blue,line cap=rect,line join=bevel,line width=0.800pt]
    \path[fill=blue] (0.0000,0.0000) node[above right] (text2144-6) {\scriptsize 6};



  \end{scope}
  \begin{scope}[cm={{1.00588,0.0,0.0,1.00588,(-411.77835,160.00678)}},draw=blue,line cap=rect,line join=bevel,line width=0.800pt]
    \path[fill=blue] (0.0000,0.0000) node[above right] (text2168-3) {\scriptsize 10};



  \end{scope}
  \begin{scope}[cm={{0.0,-1.00588,1.00588,0.0,(-419.03217,183.62678)}},draw=blue,line cap=rect,line join=bevel,line width=0.800pt]
    \path[fill=blue] (0.0000,-5.9329) node[above right] (text2192-1) {\rotatebox{90}{\scriptsize $c_{i,2}$}};



  \end{scope}
  \begin{scope}[cm={{1.00588,0.0,0.0,1.00588,(-422.50501,240.01778)}},draw=blue,line cap=rect,line join=bevel,line width=0.800pt]
    \path[fill=blue] (0.0000,0.0000) node[above right] (text2430-9) {\scriptsize -1000};



  \end{scope}
  \begin{scope}[cm={{1.00588,0.0,0.0,1.00588,(-418.48151,222.90078)}},draw=blue,line cap=rect,line join=bevel,line width=0.800pt]
    \path[fill=blue] (0.0000,0.0000) node[above right] (text2454-1) {\scriptsize -500};



  \end{scope}
  \begin{scope}[cm={{1.00588,0.0,0.0,1.00588,(-407.75485,204.28278)}},draw=blue,line cap=rect,line join=bevel,line width=0.800pt]
    \path[fill=blue] (0.0000,0.0000) node[above right] (text2478-9) {\scriptsize 0};



  \end{scope}
  \begin{scope}[cm={{0.0,-1.00588,1.00588,0.0,(-424.99998,224.40978)}},draw=blue,line cap=rect,line join=bevel,line width=0.800pt]
    \path[fill=blue] (0.0000,0.0000) node[above right] (text2502-0) {\rotatebox{90}{\scriptsize $c_{i,1}$}};



  \end{scope}
  \begin{scope}[cm={{1.00588,0.0,0.0,1.00588,(-605.07343,207.04185)}},draw=blue,line cap=rect,line join=bevel,line width=0.800pt]
    \path[fill=blue] (0.0000,0.0000) node[above right] (text154-9-8) {\Large i};



  \end{scope}
  \begin{scope}[cm={{1.00588,0.0,0.0,1.00588,(-607.56518,325.40947)}},draw=blue,line cap=rect,line join=bevel,line width=0.800pt]
    \path[fill=blue] (2.4954,0.0000) node[above right] (text154-9-0-0) {\Large ii};



  \end{scope}
  \begin{scope}[cm={{0.84173,0.0,0.0,0.84173,(-604.99975,95.12326)}},draw=blue,line cap=round,line join=round,line width=0.480pt]
    \path[draw] (61.6000,227.9000) -- (61.6000,227.9000) -- (61.8000,231.0000) -- (61.9000,233.8000) -- (60.9000,236.4000) -- (59.0000,238.8000) -- (57.6000,241.4000) -- (56.8000,244.1000) -- (56.6000,247.0000) -- (56.6000,249.9000) -- (56.5000,252.8000) -- (56.5000,255.7000) -- (56.5000,258.6000) -- (56.5000,261.5000) -- (56.5000,264.4000) -- (56.5000,267.3000) -- (56.5000,270.2000) -- (56.5000,273.1000) -- (56.5000,276.0000) -- (56.5000,278.9000) -- (56.5000,281.8000) -- (56.5000,284.6000) -- (56.5000,287.5000) -- (56.8000,290.4000) -- (57.5000,293.3000) -- (58.7000,296.0000) -- (60.2000,298.6000) -- (62.2000,301.0000) -- (64.5000,303.1000) -- (67.1000,305.0000) -- (70.1000,306.6000) -- (73.2000,307.7000) -- (76.6000,308.6000) -- (80.1000,309.0000) -- (83.6000,308.9000) -- (87.0000,308.5000) -- (90.4000,307.6000) -- (93.6000,306.4000) -- (96.6000,304.8000) -- (99.2000,302.8000) -- (101.6000,300.6000) -- (103.4000,298.1000) -- (104.8000,295.4000) -- (105.6000,292.6000) -- (106.1000,289.8000) -- (106.3000,286.9000) -- (106.5000,284.1000) -- (106.5000,281.2000) -- (106.6000,278.3000) -- (106.6000,275.4000) -- (106.6000,272.5000) -- (106.6000,269.6000) -- (106.6000,266.8000) -- (106.6000,263.9000) -- (106.6000,261.0000) -- (106.6000,258.1000) -- (106.6000,255.2000) -- (106.7000,252.3000) -- (107.2000,249.4000) -- (107.4000,246.5000) -- (107.1000,243.7000) -- (106.2000,240.9000) -- (104.9000,238.2000) -- (103.2000,235.7000) -- (100.9000,233.5000) -- (98.2000,231.6000) -- (95.2000,230.1000) -- (91.9000,229.0000) -- (88.5000,228.4000) -- (85.0000,228.3000) -- (81.5000,228.7000) -- (78.2000,229.6000) -- (75.1000,230.9000) -- (72.3000,232.6000) -- (69.9000,234.7000) -- (67.8000,237.1000) -- (66.3000,239.7000) -- (65.3000,242.4000) -- (64.9000,245.3000) -- (64.8000,248.2000) -- (64.8000,251.1000) -- (64.8000,254.0000) -- (64.8000,256.9000) -- (64.8000,259.8000) -- (64.8000,262.7000) -- (64.8000,265.6000) -- (64.8000,268.5000) -- (64.8000,271.4000) -- (64.8000,274.2000) -- (64.8000,277.1000) -- (64.8000,280.0000) -- (64.8000,282.9000) -- (64.7000,285.8000) -- (64.9000,288.7000) -- (65.4000,291.6000) -- (66.4000,294.3000) -- (67.8000,297.0000) -- (69.6000,299.4000) -- (71.8000,301.7000) -- (74.4000,303.7000) -- (77.2000,305.4000) -- (80.3000,306.7000) -- (83.5000,307.7000) -- (87.0000,308.3000) -- (90.5000,308.4000) -- (94.0000,308.2000) -- (97.4000,307.5000) -- (100.7000,306.4000) -- (103.8000,305.0000) -- (106.6000,303.2000) -- (109.1000,301.1000) -- (111.2000,298.7000) -- (112.8000,296.1000) -- (113.8000,293.3000) -- (114.4000,290.5000) -- (114.8000,287.7000) -- (114.9000,284.8000) -- (115.0000,282.0000) -- (115.1000,279.1000) -- (115.1000,276.2000) -- (115.1000,273.3000) -- (115.1000,270.4000) -- (115.1000,267.5000) -- (115.1000,264.6000) -- (115.1000,261.7000) -- (115.1000,258.9000) -- (115.1000,256.0000) -- (115.1000,253.1000) -- (115.4000,250.2000) -- (115.8000,247.3000) -- (115.7000,244.4000) -- (115.1000,241.6000) -- (114.1000,238.8000) -- (112.7000,236.2000) -- (110.8000,233.8000) -- (108.6000,231.5000) -- (105.9000,229.6000) -- (102.9000,228.0000) -- (99.7000,226.8000) -- (96.3000,226.0000) -- (92.8000,225.7000) -- (89.3000,225.8000) -- (85.9000,226.4000) -- (82.6000,227.4000) -- (79.5000,228.7000) -- (76.7000,230.5000) -- (74.3000,232.5000) -- (72.2000,234.8000) -- (70.4000,237.3000) -- (69.1000,240.0000) -- (68.4000,242.8000) -- (68.1000,245.7000) -- (68.1000,248.6000) -- (68.1000,251.5000) -- (68.1000,254.4000) -- (68.1000,257.3000) -- (68.1000,260.2000) -- (68.1000,263.1000) -- (68.1000,266.0000) -- (68.1000,268.9000) -- (68.1000,271.8000) -- (68.1000,274.6000) -- (68.1000,277.5000) -- (68.1000,280.4000) -- (68.1000,283.3000) -- (68.0000,286.2000) -- (68.3000,289.1000) -- (68.9000,291.9000) -- (70.0000,294.7000) -- (71.5000,297.3000) -- (73.4000,299.7000) -- (75.7000,301.9000) -- (78.3000,303.8000) -- (81.2000,305.4000) -- (84.3000,306.7000) -- (87.6000,307.6000) -- (91.1000,308.1000) -- (94.6000,308.2000) -- (98.1000,307.8000) -- (101.5000,307.1000) -- (104.7000,305.9000) -- (107.7000,304.4000) -- (110.5000,302.5000) -- (112.9000,300.3000) -- (114.9000,297.9000) -- (116.4000,295.3000) -- (117.4000,292.5000) -- (117.9000,289.7000) -- (118.2000,286.9000) -- (118.3000,284.0000) -- (118.4000,281.1000) -- (118.4000,278.3000) -- (118.5000,275.4000) -- (118.5000,272.5000) -- (118.5000,269.6000) -- (118.5000,266.7000) -- (118.5000,263.8000) -- (118.5000,260.9000) -- (118.5000,258.0000) -- (118.5000,255.1000) -- (118.5000,252.2000) -- (118.9000,249.4000) -- (119.1000,246.5000) -- (118.9000,243.6000) -- (118.3000,240.8000) -- (117.2000,238.0000) -- (115.6000,235.4000) -- (113.6000,233.0000) -- (111.3000,230.9000) -- (108.5000,229.0000) -- (105.5000,227.5000) -- (102.2000,226.4000) -- (98.8000,225.8000) -- (95.3000,225.6000) -- (91.8000,225.8000) -- (88.4000,226.5000) -- (85.2000,227.5000) -- (82.2000,229.0000) -- (79.4000,230.8000) -- (77.1000,232.9000) -- (75.0000,235.3000) -- (73.4000,237.8000) -- (72.3000,240.6000) -- (71.7000,243.4000) -- (71.5000,246.3000) -- (71.5000,249.2000) -- (71.5000,252.1000) -- (71.5000,255.0000) -- (71.5000,257.9000) -- (71.5000,260.8000) -- (71.5000,263.7000) -- (71.5000,266.6000) -- (71.5000,269.5000) -- (71.5000,272.4000) -- (71.5000,275.3000) -- (71.5000,278.1000) -- (71.5000,281.0000) -- (71.5000,283.9000) -- (71.5000,286.8000) -- (71.9000,289.7000) -- (72.7000,292.5000) -- (73.9000,295.2000) -- (75.5000,297.8000) -- (77.5000,300.1000) -- (79.9000,302.2000) -- (82.6000,304.1000) -- (85.6000,305.6000) -- (88.8000,306.7000) -- (92.2000,307.5000) -- (95.6000,307.8000) -- (99.1000,307.8000) -- (102.6000,307.3000) -- (106.0000,306.4000) -- (109.2000,305.1000) -- (112.1000,303.4000) -- (114.7000,301.4000) -- (117.0000,299.2000) -- (118.9000,296.7000) -- (120.2000,294.0000) -- (120.9000,291.2000) -- (121.3000,288.3000) -- (121.6000,285.5000) -- (121.7000,282.6000) -- (121.8000,279.8000) -- (121.8000,276.9000) -- (121.8000,274.0000) -- (121.8000,271.1000) -- (121.8000,268.2000) -- (121.8000,265.3000) -- (121.8000,262.4000) -- (121.8000,259.5000) -- (121.8000,256.6000) -- (121.8000,253.7000) -- (122.0000,250.9000) -- (122.4000,248.0000) -- (122.4000,245.1000) -- (122.1000,242.2000) -- (121.2000,239.4000) -- (120.0000,236.7000) -- (118.3000,234.2000) -- (116.1000,231.9000) -- (113.6000,229.9000) -- (110.8000,228.1000) -- (107.6000,226.8000) -- (104.3000,225.9000) -- (100.8000,225.4000) -- (97.3000,225.3000) -- (93.8000,225.7000) -- (90.5000,226.6000) -- (87.3000,227.8000) -- (84.4000,229.4000) -- (81.8000,231.3000) -- (79.6000,233.5000) -- (77.7000,236.0000) -- (76.3000,238.6000) -- (75.3000,241.4000) -- (75.0000,244.3000) -- (74.9000,247.2000) -- (74.9000,250.1000) -- (74.9000,253.0000) -- (74.9000,255.9000) -- (74.9000,258.8000) -- (74.9000,261.6000) -- (74.9000,264.5000) -- (74.9000,267.4000) -- (74.9000,270.3000) -- (74.9000,273.2000) -- (74.9000,276.1000) -- (74.9000,279.0000) -- (74.9000,281.9000) -- (74.9000,284.8000) -- (75.0000,287.7000) -- (75.5000,290.5000) -- (76.5000,293.3000) -- (77.8000,296.0000) -- (79.6000,298.4000) -- (81.8000,300.7000) -- (84.3000,302.7000) -- (87.1000,304.4000) -- (90.2000,305.8000) -- (93.5000,306.8000) -- (96.9000,307.4000) -- (100.4000,307.6000) -- (103.9000,307.3000) -- (107.3000,306.7000) -- (110.6000,305.6000) -- (113.7000,304.2000) -- (116.5000,302.4000) -- (119.0000,300.3000) -- (121.1000,297.9000) -- (122.8000,295.3000) -- (123.9000,292.6000) -- (124.5000,289.7000) -- (124.8000,286.9000) -- (125.0000,284.1000) -- (125.1000,281.2000) -- (125.1000,278.3000) -- (125.1000,275.4000) -- (125.2000,272.5000) -- (125.2000,269.6000) -- (125.2000,266.8000) -- (125.2000,263.9000) -- (125.2000,261.0000) -- (125.2000,258.1000) -- (125.2000,255.2000) -- (125.2000,252.3000) -- (125.5000,249.4000) -- (125.8000,246.5000) -- (125.7000,243.6000) -- (125.1000,240.8000) -- (124.1000,238.1000) -- (122.7000,235.4000) -- (120.8000,233.0000) -- (118.5000,230.8000) -- (115.9000,228.8000) -- (112.9000,227.3000) -- (109.6000,226.1000) -- (106.2000,225.3000) -- (102.7000,225.0000) -- (99.3000,225.2000) -- (95.8000,225.7000) -- (92.6000,226.7000) -- (89.5000,228.1000) -- (86.7000,229.8000) -- (84.2000,231.9000) -- (82.2000,234.2000) -- (80.5000,236.7000) -- (79.2000,239.4000) -- (78.5000,242.3000) -- (78.3000,245.2000) -- (78.2000,248.1000) -- (78.2000,251.0000) -- (78.2000,253.9000) -- (78.2000,256.7000) -- (78.2000,259.6000) -- (78.2000,262.5000) -- (78.2000,265.4000) -- (78.2000,268.3000) -- (78.2000,271.2000) -- (78.2000,274.1000) -- (78.2000,277.0000) -- (78.2000,279.9000) -- (78.2000,282.8000) -- (78.2000,285.7000) -- (78.5000,288.5000) -- (79.2000,291.4000) -- (80.3000,294.1000) -- (81.9000,296.7000) -- (83.8000,299.1000) -- (86.1000,301.3000) -- (88.8000,303.1000) -- (91.7000,304.7000) -- (94.9000,305.9000) -- (98.2000,306.8000) -- (101.7000,307.2000) -- (105.2000,307.2000) -- (108.7000,306.8000) -- (112.0000,306.0000) -- (115.3000,304.8000) -- (118.3000,303.2000) -- (120.9000,301.2000) -- (123.3000,299.0000) -- (125.2000,296.6000) -- (126.7000,293.9000) -- (127.5000,291.1000) -- (128.0000,288.3000) -- (128.2000,285.4000) -- (128.4000,282.6000) -- (128.5000,279.7000) -- (128.5000,276.8000) -- (128.5000,273.9000) -- (128.5000,271.0000) -- (128.5000,268.2000) -- (128.5000,265.3000) -- (128.5000,262.4000) -- (128.5000,259.5000) -- (128.5000,256.6000) -- (128.5000,253.7000) -- (128.6000,250.8000) -- (129.0000,247.9000) -- (129.2000,245.0000) -- (128.9000,242.2000) -- (128.1000,239.4000) -- (126.9000,236.7000) -- (125.3000,234.1000) -- (123.3000,231.7000) -- (120.8000,229.6000) -- (118.0000,227.9000) -- (114.9000,226.4000) -- (111.6000,225.4000) -- (108.2000,224.9000) -- (104.6000,224.7000) -- (101.2000,225.1000) -- (97.8000,225.8000) -- (94.6000,227.0000) -- (91.7000,228.5000) -- (89.0000,230.4000) -- (86.7000,232.5000) -- (84.8000,234.9000) -- (83.2000,237.5000) -- (82.2000,240.3000) -- (81.7000,243.2000) -- (81.6000,246.1000) -- (81.6000,249.0000) -- (81.6000,251.9000) -- (81.6000,254.8000) -- (81.6000,257.7000) -- (81.6000,260.6000) -- (81.6000,263.4000) -- (81.6000,266.3000) -- (81.6000,269.2000) -- (81.6000,272.1000) -- (81.6000,275.0000) -- (81.6000,277.9000) -- (81.6000,280.8000) -- (81.6000,283.7000) -- (81.6000,286.6000) -- (82.1000,289.4000) -- (83.0000,292.2000) -- (84.3000,294.9000) -- (86.0000,297.4000) -- (88.1000,299.7000) -- (90.6000,301.8000) -- (93.3000,303.5000) -- (96.4000,305.0000) -- (99.6000,306.0000) -- (103.0000,306.7000) -- (106.5000,307.0000) -- (110.0000,306.8000) -- (113.4000,306.2000) -- (116.8000,305.2000) -- (119.9000,303.8000) -- (122.8000,302.1000) -- (125.3000,300.1000) -- (127.5000,297.7000) -- (129.3000,295.2000) -- (130.4000,292.5000) -- (131.1000,289.6000) -- (131.5000,286.8000) -- (131.7000,284.0000) -- (131.8000,281.1000) -- (131.8000,278.2000) -- (131.9000,275.3000) -- (131.9000,272.4000) -- (131.9000,269.6000) -- (131.9000,266.7000) -- (131.9000,263.8000) -- (131.9000,260.9000) -- (131.9000,258.0000) -- (131.9000,255.1000) -- (131.9000,252.2000) -- (132.1000,249.3000) -- (132.5000,246.5000) -- (132.5000,243.6000) -- (132.0000,240.7000) -- (131.1000,237.9000) -- (129.7000,235.3000) -- (127.9000,232.8000) -- (125.7000,230.5000) -- (123.1000,228.6000) -- (120.1000,226.9000) -- (116.9000,225.7000) -- (113.5000,224.8000) -- (110.1000,224.5000) -- (106.6000,224.5000) -- (103.1000,225.0000) -- (99.8000,225.9000) -- (96.7000,227.3000) -- (93.9000,228.9000) -- (91.4000,230.9000) -- (89.2000,233.2000) -- (87.4000,235.7000) -- (86.1000,238.4000) -- (85.3000,241.2000) -- (85.0000,244.1000) -- (85.0000,247.0000) -- (84.9000,249.9000) -- (84.9000,252.8000) -- (84.9000,255.7000) -- (84.9000,258.6000) -- (84.9000,261.4000) -- (84.9000,264.3000) -- (84.9000,267.2000) -- (84.9000,270.1000) -- (84.9000,273.0000) -- (84.9000,275.9000) -- (84.9000,278.8000) -- (84.9000,281.7000) -- (84.9000,284.6000) -- (85.1000,287.5000) -- (85.8000,290.3000) -- (86.8000,293.1000) -- (88.3000,295.7000) -- (90.2000,298.1000) -- (92.4000,300.3000) -- (95.0000,302.3000) -- (97.9000,303.9000) -- (101.0000,305.2000) -- (104.3000,306.1000) -- (107.8000,306.6000) -- (111.3000,306.7000) -- (114.8000,306.3000) -- (118.2000,305.6000) -- (121.4000,304.4000) -- (124.5000,302.9000) -- (127.2000,301.0000) -- (129.6000,298.9000) -- (131.6000,296.5000) -- (133.2000,293.8000) -- (134.1000,291.0000) -- (134.7000,288.2000) -- (134.9000,285.4000) -- (135.1000,282.5000) -- (135.2000,279.7000) -- (135.2000,276.8000) -- (135.2000,273.9000) -- (135.2000,271.0000) -- (135.2000,268.1000) -- (135.2000,265.2000) -- (135.2000,262.3000) -- (135.2000,259.4000) -- (135.2000,256.5000) -- (135.2000,253.6000) -- (135.3000,250.8000) -- (135.7000,247.9000) -- (135.9000,245.0000) -- (135.7000,242.1000) -- (135.0000,239.3000) -- (133.9000,236.6000) -- (132.4000,234.0000) -- (130.4000,231.6000) -- (128.0000,229.4000) -- (125.3000,227.6000) -- (122.2000,226.1000) -- (118.9000,225.0000) -- (115.5000,224.3000) -- (112.0000,224.1000) -- (108.5000,224.4000) -- (105.1000,225.0000) -- (101.9000,226.1000) -- (98.9000,227.6000) -- (96.2000,229.4000) -- (93.8000,231.5000) -- (91.8000,233.9000) -- (90.2000,236.4000) -- (89.1000,239.2000) -- (88.5000,242.0000) -- (88.3000,244.9000) -- (88.3000,247.8000) -- (88.3000,250.7000) -- (88.3000,253.6000) -- (88.3000,256.5000) -- (88.3000,259.4000) -- (88.3000,262.3000) -- (88.3000,265.2000) -- (88.3000,268.1000) -- (88.3000,271.0000) -- (88.3000,273.9000) -- (88.3000,276.8000) -- (88.3000,279.7000) -- (88.3000,282.5000) -- (88.3000,285.4000) -- (88.7000,288.3000) -- (89.5000,291.1000) -- (90.7000,293.8000) -- (92.4000,296.4000) -- (94.4000,298.7000) -- (96.8000,300.8000) -- (99.5000,302.7000) -- (102.5000,304.1000) -- (105.7000,305.3000) -- (109.0000,306.0000) -- (112.5000,306.4000) -- (116.0000,306.3000) -- (119.5000,305.8000) -- (122.8000,304.9000) -- (126.0000,303.6000) -- (129.0000,301.9000) -- (131.6000,299.9000) -- (133.8000,297.6000) -- (135.7000,295.1000) -- (137.0000,292.4000) -- (137.7000,289.6000) -- (138.1000,286.8000) -- (138.4000,284.0000) -- (138.5000,281.1000) -- (138.5000,278.2000) -- (138.6000,275.3000) -- (138.6000,272.4000) -- (138.6000,269.5000) -- (138.6000,266.7000) -- (138.6000,263.8000) -- (138.6000,260.9000) -- (138.6000,258.0000) -- (138.6000,255.1000) -- (138.6000,252.2000) -- (138.8000,249.3000) -- (139.2000,246.4000) -- (139.2000,243.5000) -- (138.8000,240.7000) -- (138.0000,237.9000) -- (136.7000,235.2000) -- (135.0000,232.7000) -- (132.9000,230.4000) -- (130.3000,228.4000) -- (127.4000,226.6000) -- (124.3000,225.3000) -- (121.0000,224.4000) -- (117.5000,223.9000) -- (114.0000,223.9000) -- (110.5000,224.3000) -- (107.2000,225.1000) -- (104.0000,226.4000) -- (101.1000,228.0000) -- (98.6000,229.9000) -- (96.3000,232.1000) -- (94.5000,234.6000) -- (93.0000,237.2000) -- (92.1000,240.0000) -- (91.8000,242.9000) -- (91.7000,245.8000) -- (91.7000,248.7000) -- (91.7000,251.6000) -- (91.6000,254.5000) -- (91.6000,257.4000) -- (91.6000,260.3000) -- (91.6000,263.2000) -- (91.6000,266.1000) -- (91.6000,269.0000) -- (91.6000,271.8000) -- (91.6000,274.7000) -- (91.6000,277.6000) -- (91.6000,280.5000) -- (91.6000,283.4000) -- (91.8000,286.3000) -- (92.3000,289.2000) -- (93.3000,291.9000) -- (94.7000,294.6000) -- (96.5000,297.1000) -- (98.7000,299.3000) -- (101.2000,301.3000) -- (104.0000,303.0000) -- (107.1000,304.4000) -- (110.3000,305.3000) -- (113.8000,305.9000) -- (117.3000,306.1000) -- (120.8000,305.8000) -- (124.2000,305.2000) -- (127.5000,304.1000) -- (130.6000,302.6000) -- (133.4000,300.8000) -- (135.9000,298.7000) -- (138.0000,296.4000) -- (139.6000,293.8000) -- (140.7000,291.0000) -- (141.3000,288.2000) -- (141.6000,285.4000) -- (141.8000,282.5000) -- (141.9000,279.7000) -- (141.9000,276.8000) -- (141.9000,273.9000) -- (141.9000,271.0000) -- (142.0000,268.1000) -- (142.0000,265.2000) -- (142.0000,262.3000) -- (142.0000,259.4000) -- (142.0000,256.5000) -- (142.0000,253.6000) -- (142.0000,250.8000) -- (142.3000,247.9000) -- (142.6000,245.0000) -- (142.5000,242.1000) -- (141.9000,239.3000) -- (140.9000,236.5000) -- (139.4000,233.9000) -- (137.5000,231.5000) -- (135.2000,229.2000) -- (132.6000,227.3000) -- (129.6000,225.8000) -- (126.3000,224.6000) -- (122.9000,223.9000) -- (119.4000,223.6000) -- (115.9000,223.7000) -- (112.5000,224.3000) -- (109.3000,225.3000) -- (106.2000,226.7000) -- (103.4000,228.4000) -- (101.0000,230.5000) -- (98.9000,232.8000) -- (97.2000,235.3000) -- (96.0000,238.0000) -- (95.3000,240.9000) -- (95.1000,243.8000) -- (95.0000,246.7000) -- (95.0000,249.6000) -- (95.0000,252.5000) -- (95.0000,255.4000) -- (95.0000,258.3000) -- (95.0000,261.1000) -- (95.0000,264.0000) -- (95.0000,266.9000) -- (95.0000,269.8000) -- (95.0000,272.7000) -- (95.0000,275.6000) -- (95.0000,278.5000) -- (95.0000,281.4000) -- (95.0000,284.3000) -- (95.3000,287.1000) -- (96.0000,290.0000) -- (97.2000,292.7000) -- (98.7000,295.3000) -- (100.7000,297.7000) -- (103.0000,299.9000) -- (105.6000,301.7000) -- (108.6000,303.3000) -- (111.7000,304.5000) -- (115.1000,305.3000) -- (118.5000,305.8000) -- (122.0000,305.8000) -- (125.5000,305.3000) -- (128.9000,304.5000) -- (132.1000,303.3000) -- (135.1000,301.7000) -- (137.8000,299.7000) -- (140.1000,297.5000) -- (142.1000,295.1000) -- (143.5000,292.4000) -- (144.3000,289.6000) -- (144.8000,286.8000) -- (145.0000,283.9000) -- (145.2000,281.1000) -- (145.2000,278.2000) -- (145.3000,275.3000) -- (145.3000,272.4000) -- (145.3000,269.5000) -- (145.3000,266.7000) -- (145.3000,263.8000) -- (145.3000,260.9000) -- (145.3000,258.0000) -- (145.3000,255.1000) -- (145.3000,252.2000) -- (145.4000,249.3000) -- (145.8000,246.4000) -- (146.0000,243.6000) -- (145.7000,240.7000) -- (144.9000,237.9000) -- (143.7000,235.2000) -- (142.1000,232.6000) -- (140.0000,230.3000) -- (137.6000,228.2000) -- (134.8000,226.4000) -- (131.7000,225.0000) -- (128.4000,224.0000) -- (124.9000,223.4000) -- (121.4000,223.3000) -- (117.9000,223.6000) -- (114.6000,224.3000) -- (111.4000,225.5000) -- (108.4000,227.0000) -- (105.8000,228.9000) -- (103.5000,231.1000) -- (101.5000,233.5000) -- (100.0000,236.1000) -- (99.0000,238.9000) -- (98.5000,241.7000) -- (98.4000,244.6000) -- (98.4000,247.5000) -- (98.4000,250.4000) -- (98.4000,253.3000) -- (98.4000,256.2000) -- (98.4000,259.1000) -- (98.4000,262.0000) -- (98.4000,264.9000) -- (98.4000,267.8000) -- (98.4000,270.7000) -- (98.4000,273.6000) -- (98.4000,276.5000) -- (98.4000,279.3000) -- (98.4000,282.2000) -- (98.4000,285.1000) -- (98.9000,288.0000) -- (99.8000,290.8000) -- (101.1000,293.5000) -- (102.8000,296.0000) -- (104.9000,298.3000) -- (107.4000,300.3000) -- (110.1000,302.1000) -- (113.1000,303.5000) -- (116.4000,304.6000) -- (119.8000,305.3000) -- (123.3000,305.5000) -- (126.8000,305.3000) -- (130.2000,304.8000) -- (133.6000,303.8000) -- (136.7000,302.4000) -- (139.6000,300.7000) -- (142.1000,298.6000) -- (144.3000,296.3000) -- (146.1000,293.7000) -- (147.2000,291.0000) -- (147.9000,288.2000) -- (148.3000,285.4000) -- (148.5000,282.5000) -- (148.6000,279.7000) -- (148.6000,276.8000) -- (148.6000,273.9000) -- (148.7000,271.0000) -- (148.7000,268.1000) -- (148.7000,265.2000) -- (148.7000,262.3000) -- (148.7000,259.4000) -- (148.7000,256.5000) -- (148.7000,253.6000) -- (148.7000,250.8000) -- (148.9000,247.9000) -- (149.3000,245.0000) -- (149.3000,242.1000) -- (148.8000,239.3000) -- (147.8000,236.5000) -- (146.5000,233.8000) -- (144.7000,231.3000) -- (142.5000,229.1000) -- (139.9000,227.1000) -- (136.9000,225.5000) -- (133.7000,224.2000) -- (130.3000,223.4000) -- (126.9000,223.0000) -- (123.4000,223.1000) -- (119.9000,223.6000) -- (116.6000,224.5000) -- (113.5000,225.8000) -- (110.7000,227.5000) -- (108.2000,229.5000) -- (106.0000,231.7000) -- (104.2000,234.2000) -- (102.9000,236.9000) -- (102.1000,239.7000) -- (101.8000,242.6000) -- (101.7000,245.5000) -- (101.7000,248.4000) -- (101.7000,251.3000) -- (101.7000,254.2000) -- (101.7000,257.1000) -- (101.7000,260.0000) -- (101.7000,262.9000) -- (101.7000,265.8000) -- (101.7000,268.6000) -- (101.7000,271.5000) -- (101.7000,274.4000) -- (101.7000,277.3000) -- (101.7000,280.2000) -- (101.7000,283.1000) -- (101.9000,286.0000) -- (102.5000,288.8000) -- (103.6000,291.6000) -- (105.1000,294.2000) -- (107.0000,296.6000) -- (109.2000,298.9000) -- (111.8000,300.8000) -- (114.7000,302.4000) -- (117.8000,303.7000) -- (121.1000,304.6000) -- (124.5000,305.1000) -- (128.0000,305.2000) -- (131.5000,304.9000) -- (134.9000,304.1000) -- (138.2000,303.0000) -- (141.2000,301.4000) -- (144.0000,299.6000) -- (146.4000,297.4000) -- (148.4000,295.0000) -- (150.0000,292.4000) -- (150.9000,289.6000) -- (151.4000,286.8000) -- (151.7000,283.9000) -- (151.9000,281.1000) -- (152.0000,278.2000) -- (152.0000,275.3000) -- (152.0000,272.4000) -- (152.0000,269.5000) -- (152.0000,266.7000) -- (152.0000,263.8000) -- (152.0000,260.9000) -- (152.0000,258.0000) -- (152.0000,255.1000) -- (152.0000,252.2000) -- (152.1000,249.3000) -- (152.4000,246.4000) -- (152.7000,243.6000) -- (152.5000,240.7000) -- (151.8000,237.8000) -- (150.7000,235.1000) -- (149.2000,232.5000) -- (147.2000,230.1000) -- (144.8000,228.0000) -- (142.1000,226.1000) -- (139.0000,224.6000) -- (135.8000,223.5000) -- (132.3000,222.9000) -- (128.8000,222.7000) -- (125.3000,222.9000) -- (121.9000,223.6000) -- (118.7000,224.7000) -- (115.7000,226.1000) -- (113.0000,227.9000) -- (110.6000,230.0000) -- (108.6000,232.4000) -- (107.0000,235.0000) -- (105.8000,237.7000) -- (105.3000,240.6000) -- (105.1000,243.5000) -- (105.1000,246.4000) -- (105.1000,249.3000) -- (105.1000,252.2000) -- (105.1000,255.1000) -- (105.1000,257.9000) -- (105.1000,260.8000) -- (105.1000,263.7000) -- (105.1000,266.6000) -- (105.1000,269.5000) -- (105.1000,272.4000) -- (105.1000,275.3000) -- (105.1000,278.2000) -- (105.1000,281.1000) -- (105.1000,284.0000) -- (105.5000,286.8000) -- (106.3000,289.7000) -- (107.5000,292.4000) -- (109.1000,294.9000) -- (111.2000,297.3000) -- (113.6000,299.4000) -- (116.3000,301.2000) -- (119.2000,302.7000) -- (122.4000,303.8000) -- (125.8000,304.6000) -- (129.3000,304.9000) -- (132.8000,304.8000) -- (136.3000,304.3000) -- (139.6000,303.4000) -- (142.8000,302.1000) -- (145.7000,300.4000) -- (148.3000,298.5000) -- (150.6000,296.2000) -- (152.5000,293.7000) -- (153.8000,291.0000) -- (154.5000,288.2000) -- (154.9000,285.3000) -- (155.1000,282.5000) -- (155.3000,279.7000) -- (155.3000,276.8000) -- (155.4000,273.9000) -- (155.4000,271.0000) -- (155.4000,268.1000) -- (155.4000,265.2000) -- (155.4000,262.3000) -- (155.4000,259.4000) -- (155.4000,256.5000) -- (155.4000,253.6000) -- (155.4000,250.8000) -- (155.5000,247.9000) -- (155.9000,244.9000);



  \end{scope}
  \begin{scope}[cm={{1.00588,0.0,0.0,1.00588,(-526.48136,104.32102)}},draw=blue,line cap=rect,line join=bevel,line width=0.800pt]
  \end{scope}
\end{scope}
\begin{scope}[cm={{1.00588,0.0,0.0,1.00588,(10.36994,109.98172)}},draw=blue,line cap=rect,line join=bevel,line width=0.800pt]
  \path[fill=blue] (0.0000,0.0000) node[above right] (text34-9-1) {-100};



\end{scope}
\begin{scope}[cm={{1.15801,0.0,0.0,1.15801,(-19.81968,-10.65042)}},draw=blue,line cap=round,line join=round,line width=0.480pt]
  \path[draw] (44.5000,79.5000) -- (48.5000,79.5000);



  \path[draw] (142.5000,79.5000) -- (139.5000,79.5000);



\end{scope}
\begin{scope}[cm={{1.00588,0.0,0.0,1.00588,(21.10457,84.85183)}},draw=blue,line cap=rect,line join=bevel,line width=0.800pt]
  \path[fill=blue] (0.0000,0.0000) node[above right] (text64-6-5) {0};



\end{scope}
\begin{scope}[cm={{1.15801,0.0,0.0,1.15801,(-19.81968,-10.65042)}},draw=ca0a0a4,dash pattern=on 1.55pt off 1.55pt,line cap=round,line join=round,line width=0.259pt,miter limit=4.00]
  \path[draw,dash pattern=on 1.55pt off 1.55pt,line width=0.259pt,miter limit=4.00] (44.5000,57.5000) -- (142.5000,57.5000);



\end{scope}
\begin{scope}[cm={{1.15801,0.0,0.0,1.15801,(-19.81968,-10.65042)}},draw=blue,line cap=round,line join=round,line width=0.480pt]
  \path[draw] (44.5000,57.5000) -- (48.5000,57.5000);



  \path[draw] (142.5000,57.5000) -- (139.5000,57.5000);



\end{scope}
\begin{scope}[cm={{1.00588,0.0,0.0,1.00588,(13.05753,58.22243)}},draw=blue,line cap=rect,line join=bevel,line width=0.800pt]
  \path[fill=blue] (0.0000,0.0000) node[above right] (text94-7-3) {100};



\end{scope}
\begin{scope}[cm={{1.15801,0.0,0.0,1.15801,(-19.81968,-10.65042)}},draw=ca0a0a4,dash pattern=on 1.55pt off 1.55pt,line cap=round,line join=round,line width=0.259pt,miter limit=4.00]
  \path[draw,dash pattern=on 1.55pt off 1.55pt,line width=0.259pt,miter limit=4.00] (44.5000,35.5000) -- (142.5000,35.5000);



\end{scope}
\begin{scope}[cm={{1.15801,0.0,0.0,1.15801,(-19.81968,-10.65042)}},draw=blue,line cap=round,line join=round,line width=0.480pt]
  \path[draw] (44.5000,35.5000) -- (48.5000,35.5000);



  \path[draw] (142.5000,35.5000) -- (139.5000,35.5000);



\end{scope}
\begin{scope}[cm={{1.00588,0.0,0.0,1.00588,(13.05753,33.09303)}},draw=blue,line cap=rect,line join=bevel,line width=0.800pt]
  \path[fill=blue] (0.0000,0.0000) node[above right] (text124-9-9) {200};



\end{scope}
\begin{scope}[cm={{1.15801,0.0,0.0,1.15801,(-19.81968,-10.65042)}},draw=ca0a0a4,dash pattern=on 0.40pt off 0.80pt,line cap=round,line join=round,line width=0.400pt]
  \path[draw] (44.5000,13.5000) -- (142.5000,13.5000);



\end{scope}
\begin{scope}[cm={{1.15801,0.0,0.0,1.15801,(-19.81968,-10.65042)}},draw=blue,line cap=round,line join=round,line width=0.480pt]
  \path[draw] (44.5000,13.5000) -- (48.5000,13.5000);



  \path[draw] (142.5000,13.5000) -- (139.5000,13.5000);



\end{scope}
\begin{scope}[cm={{1.00588,0.0,0.0,1.00588,(13.05753,7.96363)}},draw=blue,line cap=rect,line join=bevel,line width=0.800pt]
  \path[fill=blue] (0.0000,0.0000) node[above right] (text154-93) {300};



\end{scope}
\begin{scope}[cm={{1.15801,0.0,0.0,1.15801,(-19.81968,-10.65042)}},draw=ca0a0a4,dash pattern=on 0.40pt off 0.80pt,line cap=round,line join=round,line width=0.400pt]
  \path[draw] (44.5000,102.5000) -- (44.5000,13.5000);



\end{scope}
\begin{scope}[cm={{1.15801,0.0,0.0,1.15801,(-19.81968,-10.65042)}},draw=blue,line cap=round,line join=round,line width=0.480pt]
  \path[draw] (44.5000,102.5000) -- (44.5000,99.5000);



  \path[draw] (44.5000,13.5000) -- (44.5000,16.5000);



\end{scope}
\begin{scope}[cm={{1.15801,0.0,0.0,1.15801,(-19.81968,-10.65042)}},draw=blue,line cap=round,line join=round,line width=0.480pt]
  \path[draw] (69.5000,102.5000) -- (69.5000,99.5000);



  \path[draw] (69.5000,13.5000) -- (69.5000,16.5000);



\end{scope}
\begin{scope}[cm={{1.15801,0.0,0.0,1.15801,(-19.81968,-10.65042)}},draw=ca0a0a4,dash pattern=on 1.55pt off 1.55pt,line cap=round,line join=round,line width=0.259pt,miter limit=4.00]
  \path[draw,dash pattern=on 1.55pt off 1.55pt,line width=0.259pt,miter limit=4.00] (93.5000,102.5000) -- (93.5000,27.5000);



  \path[draw,dash pattern=on 1.55pt off 1.55pt,line width=0.259pt,miter limit=4.00] (93.5000,19.5000) -- (93.5000,13.5000);



\end{scope}
\begin{scope}[cm={{1.15801,0.0,0.0,1.15801,(-19.81968,-10.65042)}},draw=blue,line cap=round,line join=round,line width=0.480pt]
  \path[draw] (93.5000,102.5000) -- (93.5000,99.5000);



  \path[draw] (93.5000,13.5000) -- (93.5000,16.5000);



\end{scope}
\begin{scope}[cm={{1.15801,0.0,0.0,1.15801,(-19.81968,-10.65042)}},draw=ca0a0a4,dash pattern=on 1.55pt off 1.55pt,line cap=round,line join=round,line width=0.259pt,miter limit=4.00]
  \path[draw,dash pattern=on 1.55pt off 1.55pt,line width=0.259pt,miter limit=4.00] (118.5000,102.5000) -- (118.5000,13.5000);



\end{scope}
\begin{scope}[cm={{1.15801,0.0,0.0,1.15801,(-19.81968,-10.65042)}},draw=blue,line cap=round,line join=round,line width=0.480pt]
  \path[draw] (118.5000,102.5000) -- (118.5000,99.5000);



  \path[draw] (118.5000,13.5000) -- (118.5000,16.5000);



\end{scope}
\begin{scope}[cm={{1.15801,0.0,0.0,1.15801,(-19.81968,-10.65042)}},draw=ca0a0a4,dash pattern=on 0.40pt off 0.80pt,line cap=round,line join=round,line width=0.400pt]
  \path[draw] (142.5000,102.5000) -- (142.5000,13.5000);



\end{scope}
\begin{scope}[cm={{1.15801,0.0,0.0,1.15801,(-19.81968,-10.65042)}},draw=blue,line cap=round,line join=round,line width=0.480pt]
  \path[draw] (142.5000,102.5000) -- (142.5000,99.5000);



  \path[draw] (142.5000,13.5000) -- (142.5000,16.5000);



\end{scope}
\begin{scope}[cm={{1.15801,0.0,0.0,1.15801,(-19.81968,-10.65042)}},draw=blue,line cap=round,line join=round,line width=0.480pt]
  \path[draw] (44.5000,13.5000) -- (44.5000,102.5000) -- (142.5000,102.5000) -- (142.5000,13.5000) -- (44.5000,13.5000);



\end{scope}
\begin{scope}[cm={{0.84029,0.0,0.0,0.84029,(41.25055,18.91027)}},draw=blue,line cap=rect,line join=bevel,line width=0.800pt]
  \path[fill=blue] (-0.9117,0.4559) node[above right] (text360-8) {\scriptsize $\mathbf{p}(t)$};



\end{scope}
\begin{scope}[cm={{1.15801,0.0,0.0,1.15801,(-27.33812,-11.90349)}},draw=blue,line cap=round,line join=round,line width=0.480pt]
  \path[draw,even odd rule] (74.5000,23.5000) -- (101.5000,23.5000);



\end{scope}
\begin{scope}[cm={{1.15801,0.0,0.0,1.15801,(-19.81968,-10.65042)}},draw=blue,line cap=round,line join=round,line width=0.480pt]
  \path[draw] (44.5000,13.5000) -- (44.5000,102.5000) -- (142.5000,102.5000) -- (142.5000,13.5000) -- (44.5000,13.5000);



\end{scope}
\begin{scope}[cm={{1.15801,0.0,0.0,1.15801,(-140.25839,106.3496)}},draw=ca0a0a4,dash pattern=on 0.40pt off 0.80pt,line cap=round,line join=round,line width=0.400pt]
  \path[draw] (148.5000,102.5000) -- (246.5000,102.5000);



\end{scope}
\begin{scope}[cm={{1.15801,0.0,0.0,1.15801,(-140.25839,106.3496)}},draw=blue,line cap=round,line join=round,line width=0.480pt]
  \path[draw] (148.5000,102.5000) -- (151.5000,102.5000);



  \path[draw] (246.5000,102.5000) -- (243.5000,102.5000);



\end{scope}
\begin{scope}[cm={{1.15801,0.0,0.0,1.15801,(-140.25839,106.3496)}},draw=ca0a0a4,dash pattern=on 1.55pt off 1.55pt,line cap=round,line join=round,line width=0.259pt,miter limit=4.00]
  \path[draw,dash pattern=on 1.55pt off 1.55pt,line width=0.259pt,miter limit=4.00] (148.5000,79.5000) -- (246.5000,79.5000);



\end{scope}
\begin{scope}[cm={{1.15801,0.0,0.0,1.15801,(-140.25839,106.3496)}},draw=blue,line cap=round,line join=round,line width=0.480pt]
  \path[draw] (148.5000,79.5000) -- (151.5000,79.5000);



  \path[draw] (246.5000,79.5000) -- (243.5000,79.5000);



\end{scope}
\begin{scope}[cm={{1.15801,0.0,0.0,1.15801,(-140.25839,106.3496)}},draw=ca0a0a4,dash pattern=on 1.55pt off 1.55pt,line cap=round,line join=round,line width=0.259pt,miter limit=4.00]
  \path[draw,dash pattern=on 1.55pt off 1.55pt,line width=0.259pt,miter limit=4.00] (148.5000,57.5000) -- (246.5000,57.5000);



\end{scope}
\begin{scope}[cm={{1.15801,0.0,0.0,1.15801,(-140.25839,106.3496)}},draw=blue,line cap=round,line join=round,line width=0.480pt]
  \path[draw] (148.5000,57.5000) -- (151.5000,57.5000);



  \path[draw] (246.5000,57.5000) -- (243.5000,57.5000);



\end{scope}
\begin{scope}[cm={{1.15801,0.0,0.0,1.15801,(-140.25839,106.3496)}},draw=ca0a0a4,dash pattern=on 1.55pt off 1.55pt,line cap=round,line join=round,line width=0.259pt,miter limit=4.00]
  \path[draw,dash pattern=on 1.55pt off 1.55pt,line width=0.259pt,miter limit=4.00] (148.5000,35.5000) -- (246.5000,35.5000);



\end{scope}
\begin{scope}[cm={{1.15801,0.0,0.0,1.15801,(-140.25839,106.3496)}},draw=blue,line cap=round,line join=round,line width=0.480pt]
  \path[draw] (148.5000,35.5000) -- (151.5000,35.5000);



  \path[draw] (246.5000,35.5000) -- (243.5000,35.5000);



\end{scope}
\begin{scope}[cm={{1.15801,0.0,0.0,1.15801,(-140.25839,106.3496)}},draw=ca0a0a4,dash pattern=on 0.40pt off 0.80pt,line cap=round,line join=round,line width=0.400pt]
  \path[draw] (148.5000,13.5000) -- (246.5000,13.5000);



\end{scope}
\begin{scope}[cm={{1.15801,0.0,0.0,1.15801,(-77.72572,106.3496)}},draw=ca0a0a4,dash pattern=on 1.55pt off 1.55pt,line cap=round,line join=round,line width=0.259pt,miter limit=4.00]
  \path[draw,dash pattern=on 1.55pt off 1.55pt,line width=0.259pt,miter limit=4.00] (118.5000,102.5000) -- (118.5000,13.5000);



\end{scope}
\begin{scope}[cm={{1.15801,0.0,0.0,1.15801,(-140.25839,106.3496)}},draw=blue,line cap=round,line join=round,line width=0.480pt]
  \path[draw] (148.5000,13.5000) -- (151.5000,13.5000);



  \path[draw] (246.5000,13.5000) -- (243.5000,13.5000);



\end{scope}
\begin{scope}[cm={{1.15801,0.0,0.0,1.15801,(-140.25839,106.3496)}},draw=ca0a0a4,dash pattern=on 0.40pt off 0.80pt,line cap=round,line join=round,line width=0.400pt]
  \path[draw] (148.5000,102.5000) -- (148.5000,13.5000);



\end{scope}
\begin{scope}[cm={{1.15801,0.0,0.0,1.15801,(-140.25839,106.3496)}},draw=blue,line cap=round,line join=round,line width=0.480pt]
  \path[draw] (148.5000,102.5000) -- (148.5000,99.5000);



  \path[draw] (148.5000,13.5000) -- (148.5000,16.5000);



\end{scope}
\begin{scope}[cm={{1.15801,0.0,0.0,1.15801,(-140.25839,106.3496)}},draw=blue,line cap=round,line join=round,line width=0.480pt]
  \path[draw] (172.5000,102.5000) -- (172.5000,99.5000);



  \path[draw] (172.5000,13.5000) -- (172.5000,16.5000);



\end{scope}
\begin{scope}[cm={{1.15801,0.0,0.0,1.15801,(-48.77541,106.3496)}},draw=ca0a0a4,dash pattern=on 1.55pt off 1.55pt,line cap=round,line join=round,line width=0.259pt,miter limit=4.00]
  \path[draw,dash pattern=on 1.55pt off 1.55pt,line width=0.259pt,miter limit=4.00] (118.5000,102.5000) -- (118.5000,13.5000);



\end{scope}
\begin{scope}[cm={{1.15801,0.0,0.0,1.15801,(-140.25839,106.3496)}},draw=blue,line cap=round,line join=round,line width=0.480pt]
  \path[draw] (197.5000,102.5000) -- (197.5000,99.5000);



  \path[draw] (197.5000,13.5000) -- (197.5000,16.5000);



\end{scope}
\begin{scope}[cm={{1.15801,0.0,0.0,1.15801,(-20.98311,106.3496)}},draw=ca0a0a4,dash pattern=on 1.55pt off 1.55pt,line cap=round,line join=round,line width=0.259pt,miter limit=4.00]
  \path[draw,dash pattern=on 1.55pt off 1.55pt,line width=0.259pt,miter limit=4.00] (118.5000,102.5000) -- (118.5000,13.5000);



\end{scope}
\begin{scope}[cm={{1.15801,0.0,0.0,1.15801,(-140.25839,106.3496)}},draw=blue,line cap=round,line join=round,line width=0.480pt]
  \path[draw] (221.5000,102.5000) -- (221.5000,99.5000);



  \path[draw] (221.5000,13.5000) -- (221.5000,16.5000);



\end{scope}
\begin{scope}[cm={{1.15801,0.0,0.0,1.15801,(-140.25839,106.3496)}},draw=ca0a0a4,dash pattern=on 0.40pt off 0.80pt,line cap=round,line join=round,line width=0.400pt]
  \path[draw] (246.5000,102.5000) -- (246.5000,27.5000);



  \path[draw] (246.5000,19.5000) -- (246.5000,13.5000);



\end{scope}
\begin{scope}[cm={{1.15801,0.0,0.0,1.15801,(-140.25839,106.3496)}},draw=blue,line cap=round,line join=round,line width=0.480pt]
  \path[draw] (246.5000,102.5000) -- (246.5000,99.5000);



  \path[draw] (246.5000,13.5000) -- (246.5000,16.5000);



\end{scope}
\begin{scope}[cm={{1.15801,0.0,0.0,1.15801,(-140.25839,106.3496)}},draw=blue,line cap=round,line join=round,line width=0.480pt]
  \path[draw] (148.5000,13.5000) -- (148.5000,102.5000) -- (246.5000,102.5000) -- (246.5000,13.5000) -- (148.5000,13.5000);



\end{scope}
\begin{scope}[cm={{0.84029,0.0,0.0,0.84029,(-93.49162,81.25858)}},fill=cd9d9d9]
  \path[rounded corners=0.0000cm] (222.0000,18.0000) rectangle (238.0000,34.0000);



\end{scope}
\begin{scope}[cm={{1.15801,0.0,0.0,1.15801,(-140.25839,106.3496)}},draw=blue,line cap=round,line join=round,line width=0.480pt]
  \path[draw] (148.5000,13.5000) -- (148.5000,102.5000) -- (246.5000,102.5000) -- (246.5000,13.5000) -- (148.5000,13.5000);



\end{scope}
\begin{scope}[cm={{1.00588,0.0,0.0,1.00588,(25.10681,236.06903)}},draw=blue,line cap=rect,line join=bevel,line width=0.800pt]
  \path[fill=blue] (0.0000,0.0000) node[above right] (text1768-1-4) {-150};



\end{scope}
\begin{scope}[cm={{1.00588,0.0,0.0,1.00588,(52.31861,236.06903)}},draw=blue,line cap=rect,line join=bevel,line width=0.800pt]
  \path[fill=blue] (0.0000,0.0000) node[above right] (text1798-7-9) {-50};



\end{scope}
\begin{scope}[cm={{1.00588,0.0,0.0,1.00588,(82.52181,236.06903)}},draw=blue,line cap=rect,line join=bevel,line width=0.800pt]
  \path[fill=blue] (0.0000,0.0000) node[above right] (text1828-9-59) {50};



\end{scope}
\begin{scope}[cm={{1.00588,0.0,0.0,1.00588,(107.70381,236.06903)}},draw=blue,line cap=rect,line join=bevel,line width=0.800pt]
  \path[fill=blue] (0.0000,0.0000) node[above right] (text1858-6-5) {150};



\end{scope}
\begin{scope}[cm={{1.00588,0.0,0.0,1.00588,(133.32142,236.06903)}},draw=blue,line cap=rect,line join=bevel,line width=0.800pt]
  \path[fill=blue] (0.0000,0.0000) node[above right] (text1858-6-6-08) {250};



\end{scope}
\begin{scope}[cm={{1.00588,0.0,0.0,1.00588,(10.64827,227.36002)}},draw=blue,line cap=rect,line join=bevel,line width=0.800pt]
  \path[fill=blue] (0.0000,0.0000) node[above right] (text34-9-6-9) {-100};



\end{scope}
\begin{scope}[cm={{1.00588,0.0,0.0,1.00588,(21.3829,202.23013)}},draw=blue,line cap=rect,line join=bevel,line width=0.800pt]
  \path[fill=blue] (0.0000,0.0000) node[above right] (text64-6-8-7) {0};



\end{scope}
\begin{scope}[cm={{1.00588,0.0,0.0,1.00588,(13.33586,175.60074)}},draw=blue,line cap=rect,line join=bevel,line width=0.800pt]
  \path[fill=blue] (0.0000,0.0000) node[above right] (text94-7-9-2) {100};



\end{scope}
\begin{scope}[cm={{1.00588,0.0,0.0,1.00588,(13.33586,150.47134)}},draw=blue,line cap=rect,line join=bevel,line width=0.800pt]
  \path[fill=blue] (0.0000,0.0000) node[above right] (text124-9-0-3) {200};



\end{scope}
\begin{scope}[cm={{1.00588,0.0,0.0,1.00588,(13.33586,125.34194)}},draw=blue,line cap=rect,line join=bevel,line width=0.800pt]
  \path[fill=blue] (0.0000,0.0000) node[above right] (text154-3-9) {300};



\end{scope}

\end{tikzpicture}


  %\captionsetup{labelfont={color=blue}}
  \end{minipage}\hfill
  \begin{minipage}[t]{0.36\columnwidth}
    \vspace*{-53.1ex}
    \centering
    \caption{Energy-aware plan- ning-scheduling using the lowest configuration \hyperref[fig:trajs-I-static]{I} as a starting point in \hyperref[fig:trajs-dyn-i]{i} and the highest \hyperref[fig:trajs-II-static]{II} in \hyperref[fig:trajs-dyn-ii]{ii} while varying atmospheric (same as Fig.~\ref{fig:stat}) and battery conditions. In \hyperref[fig:dyn]{a} are the re-planned trajectories, showing the re-planning in the proximity of simulated battery drops. In \hyperref[fig:dyn]{b} are the parameters evolutions, and in \hyperref[fig:dyn]{c} the energy w.r.t. the battery.}
    \label{fig:dyn}
  \end{minipage}
  \vspace*{-3ex}
\end{figure*}


%%%%%%%%%%%%%%%%%%%%%%%%%%%%%%%%%
\section{Numerical Simulations} %
\label{sec:experimental}        %
                                %
%$\subsection{Experimental setup}
\vspace*{-.2ex}

Numerical simulations of Algorithms~\ref{alg2}--\ref{alg} in this section are implemented in \textsc{Matlab}\hspace{.5ex}(R) and are extended with the computations energy model on NVIDIA\hspace{.5ex}(R) Jetson Nano\hspace{.5ex}(TM) heterogeneous computing hardware. These simulations complement early data of physical flights of a static coverage plan with the open-source Paparazzi flight controller%~\cite{papa}
.
The computing hardware carries a camera as a peripheral and is evaluated independently of the aerial robot with \powprof{} (see Sec.~\ref{sec:mod-com}). The scheduler, implemented using the Robot Operating System (ROS) middleware%~\cite{quigley2009ros}
, varies a computation parameter $c_{i,2}$ relative to the ground patterns detection rate from two to ten frames per second (FPS). The detection uses PedNet, a Convolutional Neural Network (CNN)~\cite{ullah2018pednet}, %also 
implemented in ROS. The planner varies the path parameter $c_{i,1}$ %in %Eq.~(\ref{eq:r2}) 
between zero and -1000 (i.e., the planner-scheduler is the concrete implementation of Algorithms~\ref{alg2}--\ref{alg}). The set of parameters is unaltered 
through{\color{black}out} the flight, i.e, $c_i:=\begin{bmatrix}c_{i,1}&c_{i,2}\end{bmatrix}',\forall i${\color{black}, along with $\delta_i$ %utilized 
in the greedy approach}. 

%The algorithm that we presented in this paper is motivated by a periodic behavior of empirical data on energy consumption (see the subfigures of Fig.~\ref{fig:il-abs}). 
Fig.~\ref{fig:il-abs} details the data of the physical flight  
%We collected these data flying Opterra, a fixed-wing UAV that we adapted for an agricultural scenario. The UAV was flying 
in standard atmospheric conditions. 
Fig{\color{black}s}.~\ref{fig:stat}{\color{black}--\ref{fig:dyn}} extends the flight with the computing hardware aided by a flight simulation implemented in \textsc{Matlab}\hspace{.5ex}(R). 
Upper-case roman numerals I,II indicate the plans are static (i.e., solely Algorithm~\ref{alg2}), lower-case i,ii exploit planning-scheduling. %as described in this letter.

%We later extended these UAV to carry a companion computer, NVIDIA Jetson Nano~\cite{nano}, running ROS. The companion computer has two ROS nodes; one detects hazards using PedNet, a Fully Convolutional Neural Network~\cite{ullah2018pednet}, and the other communicates with a ground station.

%\begin{figure}[t]
%  \centering
%  \footnotesize
%  
\definecolor{ca0a0a4}{RGB}{160,160,164}
\definecolor{cffffff}{RGB}{255,255,255}


\def \globalscale {1.000000}
\begin{tikzpicture}[y=0.82pt, x=0.84pt,yscale=-\globalscale, xscale=\globalscale, inner sep=0pt, outer sep=0pt]
\begin{scope}[shift={(-44.55268,-13.36975)},draw=black,line join=bevel,line cap=rect,even odd rule,line width=0.800pt]
  \begin{scope}[draw=black,line join=bevel,line cap=rect,line width=0.800pt]
  \end{scope}
  \begin{scope}[scale=1.006,draw=black,line join=bevel,line cap=rect,line width=0.800pt]
  \end{scope}
  \begin{scope}[scale=1.006,draw=ca0a0a4,dash pattern=on 0.40pt off 0.80pt,line join=round,line cap=round,line width=0.400pt]
    \path[draw] (44.5000,102.5000) -- (142.5000,102.5000);



  \end{scope}
  \begin{scope}[scale=1.006,draw=black,line join=round,line cap=round,line width=0.480pt]
    \path[draw] (44.5000,102.5000) -- (48.5000,102.5000);



    \path[draw] (142.5000,102.5000) -- (139.5000,102.5000);



  \end{scope}
  \begin{scope}[scale=1.006,draw=black,line join=bevel,line cap=rect,line width=0.800pt]
  \end{scope}
  \begin{scope}[cm={{1.00588,0.0,0.0,1.00588,(18.1059,106.624)}},draw=black,line join=bevel,line cap=rect,line width=0.800pt]
  \end{scope}
  \begin{scope}[cm={{1.00588,0.0,0.0,1.00588,(18.1059,106.624)}},draw=black,line join=bevel,line cap=rect,line width=0.800pt]
  \end{scope}
  \begin{scope}[cm={{1.00588,0.0,0.0,1.00588,(18.1059,106.624)}},draw=black,line join=bevel,line cap=rect,line width=0.800pt]
  \end{scope}
  \begin{scope}[cm={{1.00588,0.0,0.0,1.00588,(18.1059,106.624)}},draw=black,line join=bevel,line cap=rect,line width=0.800pt]
  \end{scope}
  \begin{scope}[cm={{1.00588,0.0,0.0,1.00588,(18.1059,106.624)}},draw=black,line join=bevel,line cap=rect,line width=0.800pt]
  \end{scope}
  \begin{scope}[cm={{1.00588,0.0,0.0,1.00588,(18.1059,106.624)}},draw=black,line join=bevel,line cap=rect,line width=0.800pt]
    \path[fill=black] (0.0000,0.0000) node[above right] (text34) {-100};



  \end{scope}
  \begin{scope}[cm={{1.00588,0.0,0.0,1.00588,(18.1059,106.624)}},draw=black,line join=bevel,line cap=rect,line width=0.800pt]
  \end{scope}
  \begin{scope}[scale=1.006,draw=black,line join=bevel,line cap=rect,line width=0.800pt]
  \end{scope}
  \begin{scope}[scale=1.006,draw=ca0a0a4,dash pattern=on 0.40pt off 0.80pt,line join=round,line cap=round,line width=0.400pt]
    \path[draw] (44.5000,79.5000) -- (142.5000,79.5000);



  \end{scope}
  \begin{scope}[scale=1.006,draw=black,line join=round,line cap=round,line width=0.480pt]
    \path[draw] (44.5000,79.5000) -- (48.5000,79.5000);



    \path[draw] (142.5000,79.5000) -- (139.5000,79.5000);



  \end{scope}
  \begin{scope}[scale=1.006,draw=black,line join=bevel,line cap=rect,line width=0.800pt]
  \end{scope}
  \begin{scope}[cm={{1.00588,0.0,0.0,1.00588,(34.2,84.4941)}},draw=black,line join=bevel,line cap=rect,line width=0.800pt]
  \end{scope}
  \begin{scope}[cm={{1.00588,0.0,0.0,1.00588,(34.2,84.4941)}},draw=black,line join=bevel,line cap=rect,line width=0.800pt]
  \end{scope}
  \begin{scope}[cm={{1.00588,0.0,0.0,1.00588,(34.2,84.4941)}},draw=black,line join=bevel,line cap=rect,line width=0.800pt]
  \end{scope}
  \begin{scope}[cm={{1.00588,0.0,0.0,1.00588,(34.2,84.4941)}},draw=black,line join=bevel,line cap=rect,line width=0.800pt]
  \end{scope}
  \begin{scope}[cm={{1.00588,0.0,0.0,1.00588,(34.2,84.4941)}},draw=black,line join=bevel,line cap=rect,line width=0.800pt]
  \end{scope}
  \begin{scope}[cm={{1.00588,0.0,0.0,1.00588,(34.2,84.4941)}},draw=black,line join=bevel,line cap=rect,line width=0.800pt]
    \path[fill=black] (0.0000,0.0000) node[above right] (text64) {0};



  \end{scope}
  \begin{scope}[cm={{1.00588,0.0,0.0,1.00588,(34.2,84.4941)}},draw=black,line join=bevel,line cap=rect,line width=0.800pt]
  \end{scope}
  \begin{scope}[scale=1.006,draw=black,line join=bevel,line cap=rect,line width=0.800pt]
  \end{scope}
  \begin{scope}[scale=1.006,draw=ca0a0a4,dash pattern=on 0.40pt off 0.80pt,line join=round,line cap=round,line width=0.400pt]
    \path[draw] (44.5000,57.5000) -- (142.5000,57.5000);



  \end{scope}
  \begin{scope}[scale=1.006,draw=black,line join=round,line cap=round,line width=0.480pt]
    \path[draw] (44.5000,57.5000) -- (48.5000,57.5000);



    \path[draw] (142.5000,57.5000) -- (139.5000,57.5000);



  \end{scope}
  \begin{scope}[scale=1.006,draw=black,line join=bevel,line cap=rect,line width=0.800pt]
  \end{scope}
  \begin{scope}[cm={{1.00588,0.0,0.0,1.00588,(22.1294,62.3647)}},draw=black,line join=bevel,line cap=rect,line width=0.800pt]
  \end{scope}
  \begin{scope}[cm={{1.00588,0.0,0.0,1.00588,(22.1294,62.3647)}},draw=black,line join=bevel,line cap=rect,line width=0.800pt]
  \end{scope}
  \begin{scope}[cm={{1.00588,0.0,0.0,1.00588,(22.1294,62.3647)}},draw=black,line join=bevel,line cap=rect,line width=0.800pt]
  \end{scope}
  \begin{scope}[cm={{1.00588,0.0,0.0,1.00588,(22.1294,62.3647)}},draw=black,line join=bevel,line cap=rect,line width=0.800pt]
  \end{scope}
  \begin{scope}[cm={{1.00588,0.0,0.0,1.00588,(22.1294,62.3647)}},draw=black,line join=bevel,line cap=rect,line width=0.800pt]
  \end{scope}
  \begin{scope}[cm={{1.00588,0.0,0.0,1.00588,(22.1294,62.3647)}},draw=black,line join=bevel,line cap=rect,line width=0.800pt]
    \path[fill=black] (0.0000,0.0000) node[above right] (text94) {100};



  \end{scope}
  \begin{scope}[cm={{1.00588,0.0,0.0,1.00588,(22.1294,62.3647)}},draw=black,line join=bevel,line cap=rect,line width=0.800pt]
  \end{scope}
  \begin{scope}[scale=1.006,draw=black,line join=bevel,line cap=rect,line width=0.800pt]
  \end{scope}
  \begin{scope}[scale=1.006,draw=ca0a0a4,dash pattern=on 0.40pt off 0.80pt,line join=round,line cap=round,line width=0.400pt]
    \path[draw] (44.5000,35.5000) -- (142.5000,35.5000);



  \end{scope}
  \begin{scope}[scale=1.006,draw=black,line join=round,line cap=round,line width=0.480pt]
    \path[draw] (44.5000,35.5000) -- (48.5000,35.5000);



    \path[draw] (142.5000,35.5000) -- (139.5000,35.5000);



  \end{scope}
  \begin{scope}[scale=1.006,draw=black,line join=bevel,line cap=rect,line width=0.800pt]
  \end{scope}
  \begin{scope}[cm={{1.00588,0.0,0.0,1.00588,(22.1294,40.2353)}},draw=black,line join=bevel,line cap=rect,line width=0.800pt]
  \end{scope}
  \begin{scope}[cm={{1.00588,0.0,0.0,1.00588,(22.1294,40.2353)}},draw=black,line join=bevel,line cap=rect,line width=0.800pt]
  \end{scope}
  \begin{scope}[cm={{1.00588,0.0,0.0,1.00588,(22.1294,40.2353)}},draw=black,line join=bevel,line cap=rect,line width=0.800pt]
  \end{scope}
  \begin{scope}[cm={{1.00588,0.0,0.0,1.00588,(22.1294,40.2353)}},draw=black,line join=bevel,line cap=rect,line width=0.800pt]
  \end{scope}
  \begin{scope}[cm={{1.00588,0.0,0.0,1.00588,(22.1294,40.2353)}},draw=black,line join=bevel,line cap=rect,line width=0.800pt]
  \end{scope}
  \begin{scope}[cm={{1.00588,0.0,0.0,1.00588,(22.1294,40.2353)}},draw=black,line join=bevel,line cap=rect,line width=0.800pt]
    \path[fill=black] (0.0000,0.0000) node[above right] (text124) {200};



  \end{scope}
  \begin{scope}[cm={{1.00588,0.0,0.0,1.00588,(22.1294,40.2353)}},draw=black,line join=bevel,line cap=rect,line width=0.800pt]
  \end{scope}
  \begin{scope}[scale=1.006,draw=black,line join=bevel,line cap=rect,line width=0.800pt]
  \end{scope}
  \begin{scope}[scale=1.006,draw=ca0a0a4,dash pattern=on 0.40pt off 0.80pt,line join=round,line cap=round,line width=0.400pt]
    \path[draw] (44.5000,13.5000) -- (142.5000,13.5000);



  \end{scope}
  \begin{scope}[scale=1.006,draw=black,line join=round,line cap=round,line width=0.480pt]
    \path[draw] (44.5000,13.5000) -- (48.5000,13.5000);



    \path[draw] (142.5000,13.5000) -- (139.5000,13.5000);



  \end{scope}
  \begin{scope}[scale=1.006,draw=black,line join=bevel,line cap=rect,line width=0.800pt]
  \end{scope}
  \begin{scope}[cm={{1.00588,0.0,0.0,1.00588,(22.1294,18.1059)}},draw=black,line join=bevel,line cap=rect,line width=0.800pt]
  \end{scope}
  \begin{scope}[cm={{1.00588,0.0,0.0,1.00588,(22.1294,18.1059)}},draw=black,line join=bevel,line cap=rect,line width=0.800pt]
  \end{scope}
  \begin{scope}[cm={{1.00588,0.0,0.0,1.00588,(22.1294,18.1059)}},draw=black,line join=bevel,line cap=rect,line width=0.800pt]
  \end{scope}
  \begin{scope}[cm={{1.00588,0.0,0.0,1.00588,(22.1294,18.1059)}},draw=black,line join=bevel,line cap=rect,line width=0.800pt]
  \end{scope}
  \begin{scope}[cm={{1.00588,0.0,0.0,1.00588,(22.1294,18.1059)}},draw=black,line join=bevel,line cap=rect,line width=0.800pt]
  \end{scope}
  \begin{scope}[cm={{1.00588,0.0,0.0,1.00588,(22.1294,18.1059)}},draw=black,line join=bevel,line cap=rect,line width=0.800pt]
    \path[fill=black] (0.0000,0.0000) node[above right] (text154) {300};



  \end{scope}
  \begin{scope}[cm={{1.00588,0.0,0.0,1.00588,(22.1294,18.1059)}},draw=black,line join=bevel,line cap=rect,line width=0.800pt]
  \end{scope}
  \begin{scope}[scale=1.006,draw=black,line join=bevel,line cap=rect,line width=0.800pt]
  \end{scope}
  \begin{scope}[scale=1.006,draw=ca0a0a4,dash pattern=on 0.40pt off 0.80pt,line join=round,line cap=round,line width=0.400pt]
    \path[draw] (44.5000,102.5000) -- (44.5000,13.5000);



  \end{scope}
  \begin{scope}[scale=1.006,draw=black,line join=round,line cap=round,line width=0.480pt]
    \path[draw] (44.5000,102.5000) -- (44.5000,99.5000);



    \path[draw] (44.5000,13.5000) -- (44.5000,16.5000);



  \end{scope}
  \begin{scope}[scale=1.006,draw=black,line join=bevel,line cap=rect,line width=0.800pt]
  \end{scope}
  \begin{scope}[cm={{1.00588,0.0,0.0,1.00588,(45.2647,118.694)}},draw=black,line join=bevel,line cap=rect,line width=0.800pt]
  \end{scope}
  \begin{scope}[cm={{1.00588,0.0,0.0,1.00588,(45.2647,118.694)}},draw=black,line join=bevel,line cap=rect,line width=0.800pt]
  \end{scope}
  \begin{scope}[cm={{1.00588,0.0,0.0,1.00588,(45.2647,118.694)}},draw=black,line join=bevel,line cap=rect,line width=0.800pt]
  \end{scope}
  \begin{scope}[cm={{1.00588,0.0,0.0,1.00588,(45.2647,118.694)}},draw=black,line join=bevel,line cap=rect,line width=0.800pt]
  \end{scope}
  \begin{scope}[cm={{1.00588,0.0,0.0,1.00588,(45.2647,118.694)}},draw=black,line join=bevel,line cap=rect,line width=0.800pt]
  \end{scope}
  \begin{scope}[cm={{1.00588,0.0,0.0,1.00588,(45.2647,118.694)}},draw=black,line join=bevel,line cap=rect,line width=0.800pt]
  \end{scope}
  \begin{scope}[scale=1.006,draw=black,line join=bevel,line cap=rect,line width=0.800pt]
  \end{scope}
  \begin{scope}[scale=1.006,draw=ca0a0a4,dash pattern=on 0.40pt off 0.80pt,line join=round,line cap=round,line width=0.400pt]
    \path[draw] (69.5000,102.5000) -- (69.5000,27.5000);



    \path[draw] (69.5000,19.5000) -- (69.5000,13.5000);



  \end{scope}
  \begin{scope}[scale=1.006,draw=black,line join=round,line cap=round,line width=0.480pt]
    \path[draw] (69.5000,102.5000) -- (69.5000,99.5000);



    \path[draw] (69.5000,13.5000) -- (69.5000,16.5000);



  \end{scope}
  \begin{scope}[scale=1.006,draw=black,line join=bevel,line cap=rect,line width=0.800pt]
  \end{scope}
  \begin{scope}[cm={{1.00588,0.0,0.0,1.00588,(69.4059,118.694)}},draw=black,line join=bevel,line cap=rect,line width=0.800pt]
  \end{scope}
  \begin{scope}[cm={{1.00588,0.0,0.0,1.00588,(69.4059,118.694)}},draw=black,line join=bevel,line cap=rect,line width=0.800pt]
  \end{scope}
  \begin{scope}[cm={{1.00588,0.0,0.0,1.00588,(69.4059,118.694)}},draw=black,line join=bevel,line cap=rect,line width=0.800pt]
  \end{scope}
  \begin{scope}[cm={{1.00588,0.0,0.0,1.00588,(69.4059,118.694)}},draw=black,line join=bevel,line cap=rect,line width=0.800pt]
  \end{scope}
  \begin{scope}[cm={{1.00588,0.0,0.0,1.00588,(69.4059,118.694)}},draw=black,line join=bevel,line cap=rect,line width=0.800pt]
  \end{scope}
  \begin{scope}[cm={{1.00588,0.0,0.0,1.00588,(69.4059,118.694)}},draw=black,line join=bevel,line cap=rect,line width=0.800pt]
  \end{scope}
  \begin{scope}[scale=1.006,draw=black,line join=bevel,line cap=rect,line width=0.800pt]
  \end{scope}
  \begin{scope}[scale=1.006,draw=ca0a0a4,dash pattern=on 0.40pt off 0.80pt,line join=round,line cap=round,line width=0.400pt]
    \path[draw] (93.5000,102.5000) -- (93.5000,27.5000);



    \path[draw] (93.5000,19.5000) -- (93.5000,13.5000);



  \end{scope}
  \begin{scope}[scale=1.006,draw=black,line join=round,line cap=round,line width=0.480pt]
    \path[draw] (93.5000,102.5000) -- (93.5000,99.5000);



    \path[draw] (93.5000,13.5000) -- (93.5000,16.5000);



  \end{scope}
  \begin{scope}[scale=1.006,draw=black,line join=bevel,line cap=rect,line width=0.800pt]
  \end{scope}
  \begin{scope}[cm={{1.00588,0.0,0.0,1.00588,(94.5529,118.694)}},draw=black,line join=bevel,line cap=rect,line width=0.800pt]
  \end{scope}
  \begin{scope}[cm={{1.00588,0.0,0.0,1.00588,(94.5529,118.694)}},draw=black,line join=bevel,line cap=rect,line width=0.800pt]
  \end{scope}
  \begin{scope}[cm={{1.00588,0.0,0.0,1.00588,(94.5529,118.694)}},draw=black,line join=bevel,line cap=rect,line width=0.800pt]
  \end{scope}
  \begin{scope}[cm={{1.00588,0.0,0.0,1.00588,(94.5529,118.694)}},draw=black,line join=bevel,line cap=rect,line width=0.800pt]
  \end{scope}
  \begin{scope}[cm={{1.00588,0.0,0.0,1.00588,(94.5529,118.694)}},draw=black,line join=bevel,line cap=rect,line width=0.800pt]
  \end{scope}
  \begin{scope}[cm={{1.00588,0.0,0.0,1.00588,(94.5529,118.694)}},draw=black,line join=bevel,line cap=rect,line width=0.800pt]
  \end{scope}
  \begin{scope}[scale=1.006,draw=black,line join=bevel,line cap=rect,line width=0.800pt]
  \end{scope}
  \begin{scope}[scale=1.006,draw=ca0a0a4,dash pattern=on 0.40pt off 0.80pt,line join=round,line cap=round,line width=0.400pt]
    \path[draw] (118.5000,102.5000) -- (118.5000,13.5000);



  \end{scope}
  \begin{scope}[scale=1.006,draw=black,line join=round,line cap=round,line width=0.480pt]
    \path[draw] (118.5000,102.5000) -- (118.5000,99.5000);



    \path[draw] (118.5000,13.5000) -- (118.5000,16.5000);



  \end{scope}
  \begin{scope}[scale=1.006,draw=black,line join=bevel,line cap=rect,line width=0.800pt]
  \end{scope}
  \begin{scope}[cm={{1.00588,0.0,0.0,1.00588,(118.694,118.694)}},draw=black,line join=bevel,line cap=rect,line width=0.800pt]
  \end{scope}
  \begin{scope}[cm={{1.00588,0.0,0.0,1.00588,(118.694,118.694)}},draw=black,line join=bevel,line cap=rect,line width=0.800pt]
  \end{scope}
  \begin{scope}[cm={{1.00588,0.0,0.0,1.00588,(118.694,118.694)}},draw=black,line join=bevel,line cap=rect,line width=0.800pt]
  \end{scope}
  \begin{scope}[cm={{1.00588,0.0,0.0,1.00588,(118.694,118.694)}},draw=black,line join=bevel,line cap=rect,line width=0.800pt]
  \end{scope}
  \begin{scope}[cm={{1.00588,0.0,0.0,1.00588,(118.694,118.694)}},draw=black,line join=bevel,line cap=rect,line width=0.800pt]
  \end{scope}
  \begin{scope}[cm={{1.00588,0.0,0.0,1.00588,(118.694,118.694)}},draw=black,line join=bevel,line cap=rect,line width=0.800pt]
  \end{scope}
  \begin{scope}[scale=1.006,draw=black,line join=bevel,line cap=rect,line width=0.800pt]
  \end{scope}
  \begin{scope}[scale=1.006,draw=ca0a0a4,dash pattern=on 0.40pt off 0.80pt,line join=round,line cap=round,line width=0.400pt]
    \path[draw] (142.5000,102.5000) -- (142.5000,13.5000);



  \end{scope}
  \begin{scope}[scale=1.006,draw=black,line join=round,line cap=round,line width=0.480pt]
    \path[draw] (142.5000,102.5000) -- (142.5000,99.5000);



    \path[draw] (142.5000,13.5000) -- (142.5000,16.5000);



  \end{scope}
  \begin{scope}[scale=1.006,draw=black,line join=bevel,line cap=rect,line width=0.800pt]
  \end{scope}
  \begin{scope}[cm={{1.00588,0.0,0.0,1.00588,(143.841,118.694)}},draw=black,line join=bevel,line cap=rect,line width=0.800pt]
  \end{scope}
  \begin{scope}[cm={{1.00588,0.0,0.0,1.00588,(143.841,118.694)}},draw=black,line join=bevel,line cap=rect,line width=0.800pt]
  \end{scope}
  \begin{scope}[cm={{1.00588,0.0,0.0,1.00588,(143.841,118.694)}},draw=black,line join=bevel,line cap=rect,line width=0.800pt]
  \end{scope}
  \begin{scope}[cm={{1.00588,0.0,0.0,1.00588,(143.841,118.694)}},draw=black,line join=bevel,line cap=rect,line width=0.800pt]
  \end{scope}
  \begin{scope}[cm={{1.00588,0.0,0.0,1.00588,(143.841,118.694)}},draw=black,line join=bevel,line cap=rect,line width=0.800pt]
  \end{scope}
  \begin{scope}[cm={{1.00588,0.0,0.0,1.00588,(143.841,118.694)}},draw=black,line join=bevel,line cap=rect,line width=0.800pt]
  \end{scope}
  \begin{scope}[scale=1.006,draw=black,line join=bevel,line cap=rect,line width=0.800pt]
  \end{scope}
  \begin{scope}[scale=1.006,draw=black,line join=round,line cap=round,line width=0.480pt]
    \path[draw] (44.5000,13.5000) -- (44.5000,102.5000) -- (142.5000,102.5000) -- (142.5000,13.5000) -- (44.5000,13.5000);



  \end{scope}
  \begin{scope}[scale=1.006,draw=black,line join=bevel,line cap=rect,line width=0.800pt]
  \end{scope}
  \begin{scope}[scale=1.006,draw=black,line join=bevel,line cap=rect,line width=0.800pt]
  \end{scope}
  \begin{scope}[scale=1.006,fill=cffffff]
    \path[fill,rounded corners=0.0000cm] (123.0000,18.0000) rectangle (134.0000,34.0000);



  \end{scope}
  \begin{scope}[scale=1.006,draw=black,line join=bevel,line cap=rect,line width=0.800pt]
  \end{scope}
  \begin{scope}[scale=1.006,draw=black,line join=bevel,line cap=rect,line width=0.800pt]
  \end{scope}
  \begin{scope}[scale=1.006,draw=black,line join=round,line cap=round,line width=0.800pt]
    \path[draw] (122.5000,34.5000) -- (122.5000,18.5000) -- (133.5000,18.5000) -- (133.5000,34.5000) -- (122.5000,34.5000);



  \end{scope}
  \begin{scope}[scale=1.006,draw=black,line join=bevel,line cap=rect,line width=0.800pt]
  \end{scope}
  \begin{scope}[cm={{1.00588,0.0,0.0,1.00588,(126.741,30.1765)}},draw=black,line join=bevel,line cap=rect,line width=0.800pt]
  \end{scope}
  \begin{scope}[cm={{1.00588,0.0,0.0,1.00588,(126.741,30.1765)}},draw=black,line join=bevel,line cap=rect,line width=0.800pt]
  \end{scope}
  \begin{scope}[cm={{1.00588,0.0,0.0,1.00588,(126.741,30.1765)}},draw=black,line join=bevel,line cap=rect,line width=0.800pt]
  \end{scope}
  \begin{scope}[cm={{1.00588,0.0,0.0,1.00588,(126.741,30.1765)}},draw=black,line join=bevel,line cap=rect,line width=0.800pt]
  \end{scope}
  \begin{scope}[cm={{1.00588,0.0,0.0,1.00588,(126.741,30.1765)}},draw=black,line join=bevel,line cap=rect,line width=0.800pt]
  \end{scope}
  \begin{scope}[cm={{1.00588,0.0,0.0,1.00588,(127.39293,29.95919)}},draw=black,line join=bevel,line cap=rect,line width=0.800pt]
    \path[fill=black] (0.0000,0.0000) node[above right] (text328) {\label{fig:trajs-I-static}I};



  \end{scope}
  \begin{scope}[cm={{1.00588,0.0,0.0,1.00588,(126.741,30.1765)}},draw=black,line join=bevel,line cap=rect,line width=0.800pt]
  \end{scope}
  \begin{scope}[cm={{0.0,-1.00588,1.00588,0.0,(14.0824,184.076)}},draw=black,line join=bevel,line cap=rect,line width=0.800pt]
  \end{scope}
  \begin{scope}[cm={{0.0,-1.00588,1.00588,0.0,(14.0824,184.076)}},draw=black,line join=bevel,line cap=rect,line width=0.800pt]
  \end{scope}
  \begin{scope}[cm={{0.0,-1.00588,1.00588,0.0,(14.0824,184.076)}},draw=black,line join=bevel,line cap=rect,line width=0.800pt]
  \end{scope}
  \begin{scope}[cm={{0.0,-1.00588,1.00588,0.0,(14.0824,184.076)}},draw=black,line join=bevel,line cap=rect,line width=0.800pt]
  \end{scope}
  \begin{scope}[cm={{0.0,-1.00588,1.00588,0.0,(14.0824,184.076)}},draw=black,line join=bevel,line cap=rect,line width=0.800pt]
  \end{scope}
  \begin{scope}[cm={{0.0,-1.00588,1.00588,0.0,(8.0824,184.076)}},draw=black,line join=bevel,line cap=rect,line width=0.800pt]
    \path[fill=black] (0.0000,0.0000) node[above right] (text344) {\rotatebox{90}{y (m)}};



  \end{scope}
  \begin{scope}[cm={{0.0,-1.00588,1.00588,0.0,(14.0824,184.076)}},draw=black,line join=bevel,line cap=rect,line width=0.800pt]
  \end{scope}
  \begin{scope}[cm={{1.00588,0.0,0.0,1.00588,(50.2941,27.1588)}},draw=black,line join=bevel,line cap=rect,line width=0.800pt]
  \end{scope}
  \begin{scope}[cm={{1.00588,0.0,0.0,1.00588,(50.2941,27.1588)}},draw=black,line join=bevel,line cap=rect,line width=0.800pt]
  \end{scope}
  \begin{scope}[cm={{1.00588,0.0,0.0,1.00588,(50.2941,27.1588)}},draw=black,line join=bevel,line cap=rect,line width=0.800pt]
  \end{scope}
  \begin{scope}[cm={{1.00588,0.0,0.0,1.00588,(50.2941,27.1588)}},draw=black,line join=bevel,line cap=rect,line width=0.800pt]
  \end{scope}
  \begin{scope}[cm={{1.00588,0.0,0.0,1.00588,(50.2941,27.1588)}},draw=black,line join=bevel,line cap=rect,line width=0.800pt]
  \end{scope}
  \begin{scope}[cm={{1.00588,0.0,0.0,1.00588,(50.2941,28.1647)}},draw=black,line join=bevel,line cap=rect,line width=0.800pt]
    \path[fill=black] (0.0000,0.0000) node[above right] (text360) {\scriptsize path};



  \end{scope}
  \begin{scope}[cm={{1.00588,0.0,0.0,1.00588,(50.2941,27.1588)}},draw=black,line join=bevel,line cap=rect,line width=0.800pt]
  \end{scope}
  \begin{scope}[scale=1.006,draw=black,line join=bevel,line cap=rect,line width=0.800pt]
  \end{scope}
  \begin{scope}[scale=1.006,draw=black,line join=round,line cap=round,line width=0.480pt]
    \path[draw,even odd rule] (74.5000,23.5000) -- (101.5000,23.5000);



  \end{scope}
  \begin{scope}[scale=1.006,draw=black,line join=bevel,line cap=rect,line width=0.800pt]
  \end{scope}
  \begin{scope}[scale=1.006,draw=black,line join=bevel,line cap=rect,line width=0.800pt]
  \end{scope}
  \begin{scope}[scale=1.006,draw=black,line join=bevel,line cap=rect,line width=0.800pt]
  \end{scope}
  \begin{scope}[scale=1.006,draw=black,line join=bevel,line cap=rect,line width=0.800pt]
  \end{scope}
  \begin{scope}[scale=1.006,draw=black,line join=round,line cap=round,line width=0.480pt]
    \path[draw] (57.5000,32.8000) -- (57.5000,32.8000) -- (58.3000,35.4000) -- (57.8000,37.2000) -- (56.3000,38.9000) -- (54.9000,41.1000) -- (53.9000,43.5000) -- (53.5000,46.1000) -- (53.4000,48.6000) -- (53.4000,51.1000) -- (53.4000,53.6000) -- (53.4000,56.1000) -- (53.4000,58.6000) -- (53.4000,61.1000) -- (53.4000,63.6000) -- (53.4000,66.1000) -- (53.4000,68.7000) -- (53.4000,71.2000) -- (53.4000,73.7000) -- (53.4000,76.2000) -- (53.4000,78.7000) -- (53.9000,81.3000) -- (54.7000,83.7000) -- (56.0000,86.0000) -- (57.6000,88.1000) -- (59.5000,89.9000) -- (61.7000,91.4000) -- (64.1000,92.6000) -- (66.6000,93.4000) -- (69.2000,93.9000) -- (71.8000,94.1000) -- (74.4000,93.9000) -- (77.0000,93.4000) -- (79.5000,92.6000) -- (81.8000,91.4000) -- (84.0000,90.0000) -- (85.9000,88.2000) -- (87.5000,86.1000) -- (88.6000,83.8000) -- (89.1000,81.3000) -- (89.2000,78.8000) -- (89.3000,76.2000) -- (89.3000,73.7000) -- (89.3000,71.1000) -- (89.3000,68.6000) -- (89.3000,66.1000) -- (89.3000,63.6000) -- (89.3000,61.0000) -- (89.3000,58.5000) -- (89.3000,56.0000) -- (89.3000,53.5000) -- (89.6000,51.0000) -- (90.0000,48.5000) -- (89.9000,46.0000) -- (89.4000,43.6000) -- (88.5000,41.3000) -- (87.2000,39.1000) -- (85.5000,37.2000) -- (83.5000,35.5000) -- (81.3000,34.1000) -- (78.9000,33.1000) -- (76.3000,32.5000) -- (73.6000,32.2000) -- (71.0000,32.3000) -- (68.3000,32.8000) -- (65.8000,33.7000) -- (63.5000,34.9000) -- (61.4000,36.5000) -- (59.6000,38.3000) -- (58.2000,40.4000) -- (57.1000,42.7000) -- (56.5000,45.2000) -- (56.3000,47.7000) -- (56.3000,50.1000) -- (56.3000,52.6000) -- (56.3000,55.1000) -- (56.3000,57.6000) -- (56.3000,60.2000) -- (56.3000,62.7000) -- (56.3000,65.2000) -- (56.3000,67.7000) -- (56.3000,70.2000) -- (56.3000,72.7000) -- (56.3000,75.3000) -- (56.3000,77.8000) -- (56.6000,80.3000) -- (57.3000,82.8000) -- (58.5000,85.1000) -- (60.1000,87.2000) -- (61.9000,89.1000) -- (64.1000,90.7000) -- (66.4000,91.9000) -- (68.9000,92.8000) -- (71.5000,93.4000) -- (74.1000,93.6000) -- (76.7000,93.5000) -- (79.3000,93.0000) -- (81.8000,92.2000) -- (84.1000,91.0000) -- (86.3000,89.5000) -- (88.2000,87.7000) -- (89.8000,85.7000) -- (90.9000,83.4000) -- (91.4000,80.9000) -- (91.6000,78.3000) -- (91.6000,75.8000) -- (91.6000,73.2000) -- (91.6000,70.7000) -- (91.6000,68.2000) -- (91.6000,65.6000) -- (91.6000,63.1000) -- (91.6000,60.6000) -- (91.6000,58.1000) -- (91.6000,55.6000) -- (91.6000,53.0000) -- (91.9000,50.5000) -- (92.3000,48.0000) -- (92.2000,45.6000) -- (91.7000,43.2000) -- (90.7000,40.9000) -- (89.4000,38.7000) -- (87.7000,36.8000) -- (85.6000,35.2000) -- (83.3000,33.9000) -- (80.9000,32.9000) -- (78.3000,32.4000) -- (75.6000,32.2000) -- (72.9000,32.4000) -- (70.3000,33.0000) -- (67.9000,33.9000) -- (65.6000,35.2000) -- (63.6000,36.9000) -- (61.9000,38.8000) -- (60.6000,41.0000) -- (59.7000,43.4000) -- (59.3000,45.9000) -- (59.2000,48.3000) -- (59.1000,50.8000) -- (59.1000,53.3000) -- (59.1000,55.8000) -- (59.1000,58.3000) -- (59.1000,60.9000) -- (59.1000,63.4000) -- (59.1000,65.9000) -- (59.1000,68.4000) -- (59.1000,70.9000) -- (59.1000,73.4000) -- (59.1000,76.0000) -- (59.2000,78.5000) -- (59.6000,81.0000) -- (60.6000,83.5000) -- (61.9000,85.7000) -- (63.6000,87.7000) -- (65.6000,89.4000) -- (67.8000,90.9000) -- (70.2000,92.0000) -- (72.7000,92.8000) -- (75.3000,93.2000) -- (78.0000,93.2000) -- (80.6000,92.9000) -- (83.1000,92.3000) -- (85.5000,91.3000) -- (87.8000,89.9000) -- (89.8000,88.3000) -- (91.6000,86.3000) -- (92.9000,84.1000) -- (93.6000,81.7000) -- (93.9000,79.2000) -- (94.0000,76.6000) -- (94.0000,74.0000) -- (94.1000,71.5000) -- (94.1000,69.0000) -- (94.1000,66.4000) -- (94.1000,63.9000) -- (94.1000,61.4000) -- (94.1000,58.9000) -- (94.1000,56.4000) -- (94.1000,53.9000) -- (94.2000,51.3000) -- (94.6000,48.8000) -- (94.7000,46.4000) -- (94.4000,43.9000) -- (93.6000,41.6000) -- (92.4000,39.3000) -- (90.8000,37.3000) -- (88.9000,35.6000) -- (86.7000,34.2000) -- (84.3000,33.1000) -- (81.7000,32.4000) -- (79.1000,32.0000) -- (76.4000,32.1000) -- (73.8000,32.5000) -- (71.3000,33.4000) -- (68.9000,34.6000) -- (66.8000,36.1000) -- (65.0000,37.9000) -- (63.6000,40.1000) -- (62.6000,42.4000) -- (62.0000,44.8000) -- (61.8000,47.3000) -- (61.8000,49.8000) -- (61.7000,52.3000) -- (61.7000,54.8000) -- (61.7000,57.3000) -- (61.7000,59.8000) -- (61.7000,62.4000) -- (61.7000,64.9000) -- (61.7000,67.4000) -- (61.7000,69.9000) -- (61.7000,72.4000) -- (61.7000,75.0000) -- (61.7000,77.5000) -- (62.0000,80.0000) -- (62.8000,82.5000) -- (64.0000,84.8000) -- (65.6000,86.9000) -- (67.5000,88.7000) -- (69.7000,90.2000) -- (72.1000,91.4000) -- (74.6000,92.3000) -- (77.1000,92.8000) -- (79.8000,93.0000) -- (82.4000,92.8000) -- (84.9000,92.2000) -- (87.4000,91.3000) -- (89.7000,90.0000) -- (91.8000,88.5000) -- (93.6000,86.6000) -- (95.1000,84.4000) -- (95.9000,82.1000) -- (96.3000,79.6000) -- (96.4000,77.0000) -- (96.5000,74.4000) -- (96.5000,71.9000) -- (96.5000,69.3000) -- (96.5000,66.8000) -- (96.5000,64.3000) -- (96.5000,61.8000) -- (96.5000,59.3000) -- (96.5000,56.7000) -- (96.5000,54.2000) -- (96.5000,51.7000) -- (97.0000,49.2000) -- (97.2000,46.7000) -- (96.9000,44.3000) -- (96.3000,41.9000) -- (95.1000,39.6000) -- (93.6000,37.6000) -- (91.8000,35.8000) -- (89.7000,34.3000) -- (87.3000,33.1000) -- (84.8000,32.3000) -- (82.1000,31.9000) -- (79.5000,31.8000) -- (76.8000,32.2000) -- (74.3000,32.9000) -- (71.9000,34.1000) -- (69.7000,35.5000) -- (67.9000,37.3000) -- (66.4000,39.4000) -- (65.2000,41.7000) -- (64.6000,44.1000) -- (64.3000,46.6000) -- (64.3000,49.1000) -- (64.2000,51.6000) -- (64.2000,54.1000) -- (64.2000,56.6000) -- (64.2000,59.1000) -- (64.2000,61.6000) -- (64.2000,64.1000) -- (64.2000,66.6000) -- (64.2000,69.2000) -- (64.2000,71.7000) -- (64.2000,74.2000) -- (64.2000,76.7000) -- (64.4000,79.3000) -- (65.1000,81.8000) -- (66.2000,84.1000) -- (67.8000,86.2000) -- (69.6000,88.1000) -- (71.7000,89.7000) -- (74.0000,91.0000) -- (76.5000,91.9000) -- (79.1000,92.5000) -- (81.7000,92.7000) -- (84.3000,92.6000) -- (86.9000,92.1000) -- (89.4000,91.3000) -- (91.7000,90.1000) -- (93.9000,88.6000) -- (95.7000,86.7000) -- (97.3000,84.6000) -- (98.2000,82.3000) -- (98.7000,79.8000) -- (98.9000,77.3000) -- (98.9000,74.7000) -- (98.9000,72.1000) -- (99.0000,69.6000) -- (99.0000,67.1000) -- (99.0000,64.6000) -- (99.0000,62.0000) -- (99.0000,59.5000) -- (99.0000,57.0000) -- (99.0000,54.5000) -- (99.0000,52.0000) -- (99.3000,49.4000) -- (99.6000,47.0000) -- (99.5000,44.5000) -- (98.9000,42.1000) -- (97.8000,39.8000) -- (96.4000,37.7000) -- (94.6000,35.9000) -- (92.5000,34.3000) -- (90.2000,33.1000) -- (87.7000,32.2000) -- (85.1000,31.7000) -- (82.4000,31.6000) -- (79.7000,31.9000) -- (77.2000,32.6000) -- (74.8000,33.6000) -- (72.6000,35.1000) -- (70.6000,36.8000) -- (69.1000,38.8000) -- (67.9000,41.1000) -- (67.2000,43.5000) -- (66.9000,46.0000) -- (66.8000,48.5000) -- (66.8000,51.0000) -- (66.7000,53.5000) -- (66.7000,56.0000) -- (66.7000,58.5000) -- (66.7000,61.0000) -- (66.7000,63.5000) -- (66.7000,66.0000) -- (66.7000,68.6000) -- (66.7000,71.1000) -- (66.7000,73.6000) -- (66.7000,76.1000) -- (66.9000,78.7000) -- (67.5000,81.2000) -- (68.6000,83.5000) -- (70.0000,85.7000) -- (71.8000,87.6000) -- (73.9000,89.2000) -- (76.2000,90.6000) -- (78.7000,91.6000) -- (81.2000,92.2000) -- (83.8000,92.5000) -- (86.5000,92.4000) -- (89.0000,92.0000) -- (91.5000,91.2000) -- (93.9000,90.0000) -- (96.1000,88.5000) -- (98.0000,86.7000) -- (99.5000,84.7000) -- (100.6000,82.4000) -- (101.1000,79.9000) -- (101.3000,77.3000) -- (101.4000,74.8000) -- (101.4000,72.2000) -- (101.4000,69.7000) -- (101.4000,67.2000) -- (101.4000,64.6000) -- (101.4000,62.1000) -- (101.4000,59.6000) -- (101.4000,57.1000) -- (101.4000,54.6000) -- (101.4000,52.0000) -- (101.7000,49.5000) -- (102.1000,47.0000) -- (102.0000,44.6000) -- (101.4000,42.2000) -- (100.4000,39.9000) -- (99.0000,37.7000) -- (97.3000,35.9000) -- (95.2000,34.2000) -- (92.9000,33.0000) -- (90.4000,32.1000) -- (87.8000,31.5000) -- (85.2000,31.4000) -- (82.5000,31.6000) -- (79.9000,32.3000) -- (77.5000,33.3000) -- (75.3000,34.6000) -- (73.3000,36.3000) -- (71.7000,38.3000) -- (70.5000,40.6000) -- (69.7000,43.0000) -- (69.3000,45.5000) -- (69.2000,47.9000) -- (69.2000,50.4000) -- (69.2000,52.9000) -- (69.2000,55.4000) -- (69.2000,57.9000) -- (69.2000,60.5000) -- (69.2000,63.0000) -- (69.2000,65.5000) -- (69.2000,68.0000) -- (69.2000,70.5000) -- (69.2000,73.1000) -- (69.2000,75.6000) -- (69.3000,78.1000) -- (69.8000,80.6000) -- (70.8000,83.0000) -- (72.3000,85.2000) -- (74.0000,87.2000) -- (76.1000,88.8000) -- (78.3000,90.2000) -- (80.8000,91.2000) -- (83.3000,91.9000) -- (85.9000,92.3000) -- (88.6000,92.2000) -- (91.2000,91.8000) -- (93.7000,91.1000) -- (96.0000,90.0000) -- (98.2000,88.5000) -- (100.2000,86.8000) -- (101.8000,84.7000) -- (102.9000,82.5000) -- (103.5000,80.0000) -- (103.7000,77.4000) -- (103.8000,74.9000) -- (103.8000,72.3000) -- (103.8000,69.8000) -- (103.8000,67.3000) -- (103.8000,64.7000) -- (103.9000,62.2000) -- (103.8000,59.7000) -- (103.8000,57.2000) -- (103.8000,54.7000) -- (103.8000,52.1000) -- (104.1000,49.6000) -- (104.5000,47.1000) -- (104.4000,44.7000) -- (103.9000,42.3000) -- (103.0000,39.9000) -- (101.7000,37.8000) -- (100.0000,35.9000) -- (98.0000,34.2000) -- (95.7000,32.9000) -- (93.2000,31.9000) -- (90.6000,31.4000) -- (88.0000,31.2000) -- (85.3000,31.3000) -- (82.7000,31.9000) -- (80.2000,32.9000) -- (78.0000,34.2000) -- (76.0000,35.9000) -- (74.3000,37.8000) -- (73.0000,40.0000) -- (72.2000,42.4000) -- (71.8000,44.9000) -- (71.7000,47.4000) -- (71.6000,49.9000) -- (71.6000,52.4000) -- (71.6000,54.9000) -- (71.6000,57.4000) -- (71.6000,59.9000) -- (71.6000,62.4000) -- (71.6000,65.0000) -- (71.6000,67.5000) -- (71.6000,70.0000) -- (71.6000,72.5000) -- (71.6000,75.0000) -- (71.7000,77.6000) -- (72.2000,80.1000) -- (73.1000,82.5000) -- (74.5000,84.7000) -- (76.2000,86.7000) -- (78.2000,88.4000) -- (80.5000,89.8000) -- (82.9000,90.9000) -- (85.4000,91.6000) -- (88.0000,92.0000) -- (90.7000,92.0000) -- (93.3000,91.7000) -- (95.8000,91.0000) -- (98.2000,89.9000) -- (100.4000,88.5000) -- (102.4000,86.8000) -- (104.1000,84.8000) -- (105.3000,82.6000) -- (105.9000,80.1000) -- (106.2000,77.6000) -- (106.2000,75.0000) -- (106.3000,72.4000) -- (106.3000,69.9000) -- (106.3000,67.4000) -- (106.3000,64.9000) -- (106.3000,62.3000) -- (106.3000,59.8000) -- (106.3000,57.3000) -- (106.3000,54.8000) -- (106.3000,52.3000) -- (106.5000,49.7000) -- (106.9000,47.2000) -- (106.9000,44.8000) -- (106.5000,42.4000) -- (105.6000,40.0000) -- (104.3000,37.8000) -- (102.7000,35.9000) -- (100.7000,34.2000) -- (98.4000,32.8000) -- (96.0000,31.8000) -- (93.4000,31.2000) -- (90.8000,30.9000) -- (88.1000,31.1000) -- (85.5000,31.6000) -- (83.0000,32.5000) -- (80.7000,33.8000) -- (78.7000,35.4000) -- (77.0000,37.4000) -- (75.6000,39.5000) -- (74.8000,41.9000) -- (74.3000,44.4000) -- (74.2000,46.9000) -- (74.1000,49.3000) -- (74.1000,51.8000) -- (74.1000,54.3000) -- (74.1000,56.9000) -- (74.1000,59.4000) -- (74.1000,61.9000) -- (74.1000,64.4000) -- (74.1000,66.9000) -- (74.1000,69.4000) -- (74.1000,72.0000) -- (74.1000,74.5000) -- (74.1000,77.0000) -- (74.5000,79.6000) -- (75.4000,82.0000) -- (76.8000,84.2000) -- (78.5000,86.2000) -- (80.4000,88.0000) -- (82.6000,89.4000) -- (85.0000,90.6000) -- (87.6000,91.3000) -- (90.2000,91.8000) -- (92.8000,91.8000) -- (95.4000,91.5000) -- (97.9000,90.9000) -- (100.3000,89.8000) -- (102.6000,88.5000) -- (104.6000,86.8000) -- (106.3000,84.8000) -- (107.6000,82.6000) -- (108.3000,80.2000) -- (108.6000,77.7000) -- (108.7000,75.1000) -- (108.7000,72.5000) -- (108.7000,70.0000) -- (108.7000,67.5000) -- (108.7000,64.9000) -- (108.7000,62.4000) -- (108.7000,59.9000) -- (108.7000,57.4000) -- (108.7000,54.9000) -- (108.7000,52.3000) -- (108.9000,49.8000) -- (109.3000,47.3000) -- (109.4000,44.9000) -- (109.0000,42.4000) -- (108.2000,40.1000) -- (106.9000,37.9000) -- (105.3000,35.9000) -- (103.4000,34.2000) -- (101.2000,32.8000) -- (98.8000,31.7000) -- (96.2000,31.0000) -- (93.5000,30.7000) -- (90.9000,30.8000) -- (88.2000,31.3000) -- (85.7000,32.2000) -- (83.4000,33.4000) -- (81.4000,35.0000) -- (79.6000,36.9000) -- (78.2000,39.0000) -- (77.3000,41.4000) -- (76.8000,43.8000) -- (76.6000,46.3000) -- (76.5000,48.8000) -- (76.5000,51.3000) -- (76.5000,53.8000) -- (76.5000,56.3000) -- (76.5000,58.8000) -- (76.5000,61.4000) -- (76.5000,63.9000) -- (76.5000,66.4000) -- (76.5000,68.9000) -- (76.5000,71.4000) -- (76.5000,73.9000) -- (76.5000,76.5000) -- (76.9000,79.0000) -- (77.7000,81.5000) -- (79.0000,83.7000) -- (80.7000,85.8000) -- (82.6000,87.6000) -- (84.8000,89.0000) -- (87.2000,90.2000) -- (89.7000,91.0000) -- (92.3000,91.5000) -- (94.9000,91.6000) -- (97.5000,91.4000) -- (100.1000,90.7000) -- (102.5000,89.8000) -- (104.8000,88.4000) -- (106.8000,86.8000) -- (108.6000,84.9000) -- (109.9000,82.7000) -- (110.7000,80.3000) -- (111.0000,77.8000) -- (111.1000,75.2000) -- (111.2000,72.6000) -- (111.2000,70.1000) -- (111.2000,67.6000) -- (111.2000,65.0000) -- (111.2000,62.5000) -- (111.2000,60.0000) -- (111.2000,57.5000) -- (111.2000,55.0000) -- (111.2000,52.4000) -- (111.3000,49.9000) -- (111.7000,47.4000) -- (111.9000,45.0000) -- (111.5000,42.5000) -- (110.8000,40.2000) -- (109.6000,37.9000) -- (108.0000,35.9000) -- (106.1000,34.1000) -- (103.9000,32.7000) -- (101.5000,31.6000) -- (99.0000,30.9000) -- (96.3000,30.5000) -- (93.6000,30.6000) -- (91.0000,31.0000) -- (88.5000,31.8000) -- (86.2000,33.0000) -- (84.1000,34.6000) -- (82.3000,36.4000) -- (80.8000,38.5000) -- (79.8000,40.8000) -- (79.3000,43.3000) -- (79.1000,45.8000) -- (79.0000,48.3000) -- (79.0000,50.8000) -- (79.0000,53.3000) -- (79.0000,55.8000) -- (79.0000,58.3000) -- (79.0000,60.8000) -- (79.0000,63.3000) -- (79.0000,65.9000) -- (79.0000,68.4000) -- (79.0000,70.9000) -- (79.0000,73.4000) -- (79.0000,75.9000) -- (79.3000,78.5000) -- (80.1000,81.0000) -- (81.3000,83.2000) -- (82.9000,85.3000) -- (84.8000,87.1000) -- (87.0000,88.6000) -- (89.3000,89.9000) -- (91.8000,90.7000) -- (94.4000,91.3000) -- (97.0000,91.4000) -- (99.6000,91.2000) -- (102.2000,90.6000) -- (104.6000,89.7000) -- (106.9000,88.4000) -- (109.0000,86.8000) -- (110.8000,84.9000) -- (112.2000,82.8000) -- (113.1000,80.4000) -- (113.4000,77.9000) -- (113.6000,75.3000) -- (113.6000,72.7000) -- (113.6000,70.2000) -- (113.6000,67.6000) -- (113.6000,65.1000) -- (113.6000,62.6000) -- (113.6000,60.1000) -- (113.6000,57.6000) -- (113.6000,55.1000) -- (113.6000,52.5000) -- (113.7000,50.0000) -- (114.1000,47.5000) -- (114.3000,45.0000) -- (114.0000,42.6000) -- (113.3000,40.2000) -- (112.2000,38.0000) -- (110.7000,35.9000) -- (108.8000,34.1000) -- (106.6000,32.6000) -- (104.3000,31.5000) -- (101.7000,30.7000) -- (99.1000,30.3000) -- (96.4000,30.3000) -- (93.8000,30.7000) -- (91.2000,31.5000) -- (88.9000,32.6000) -- (86.7000,34.1000) -- (84.9000,36.0000) -- (83.4000,38.0000) -- (82.4000,40.3000) -- (81.8000,42.8000) -- (81.6000,45.3000) -- (81.5000,47.8000) -- (81.4000,50.3000) -- (81.4000,52.8000) -- (81.4000,55.3000) -- (81.4000,57.8000) -- (81.4000,60.3000) -- (81.4000,62.8000) -- (81.4000,65.4000) -- (81.4000,67.9000) -- (81.4000,70.4000) -- (81.4000,72.9000) -- (81.4000,75.4000) -- (81.7000,78.0000) -- (82.4000,80.5000) -- (83.6000,82.8000) -- (85.2000,84.9000) -- (87.0000,86.7000) -- (89.2000,88.3000) -- (91.5000,89.5000) -- (94.0000,90.4000) -- (96.6000,91.0000) -- (99.2000,91.2000) -- (101.8000,91.0000) -- (104.4000,90.5000) -- (106.8000,89.6000) -- (109.2000,88.3000) -- (111.3000,86.8000) -- (113.1000,84.9000) -- (114.6000,82.8000) -- (115.5000,80.4000) -- (115.9000,77.9000) -- (116.0000,75.3000) -- (116.1000,72.8000) -- (116.1000,70.2000) -- (116.1000,67.7000) -- (116.1000,65.2000) -- (116.1000,62.7000) -- (116.1000,60.1000) -- (116.1000,57.6000) -- (116.1000,55.1000) -- (116.1000,52.6000) -- (116.1000,50.1000) -- (116.5000,47.5000) -- (116.8000,45.1000) -- (116.5000,42.6000) -- (115.9000,40.2000) -- (114.8000,38.0000) -- (113.3000,35.9000) -- (111.5000,34.1000) -- (109.3000,32.6000) -- (107.0000,31.4000) -- (104.5000,30.6000) -- (101.8000,30.1000) -- (99.2000,30.1000) -- (96.5000,30.4000) -- (94.0000,31.2000) -- (91.6000,32.3000) -- (89.4000,33.7000) -- (87.5000,35.5000) -- (86.0000,37.6000) -- (84.9000,39.9000) -- (84.3000,42.3000) -- (84.0000,44.8000) -- (83.9000,47.3000) -- (83.9000,49.8000) -- (83.9000,52.3000) -- (83.9000,54.8000) -- (83.9000,57.3000) -- (83.9000,59.8000) -- (83.9000,62.3000) -- (83.9000,64.8000) -- (83.9000,67.4000) -- (83.9000,69.9000) -- (83.9000,72.4000) -- (83.9000,74.9000) -- (84.0000,77.5000) -- (84.7000,80.0000) -- (85.9000,82.3000) -- (87.4000,84.4000) -- (89.2000,86.3000) -- (91.3000,87.9000) -- (93.7000,89.2000) -- (96.1000,90.1000) -- (98.7000,90.7000) -- (101.3000,91.0000) -- (103.9000,90.8000) -- (106.5000,90.3000) -- (109.0000,89.5000) -- (111.3000,88.3000) -- (113.5000,86.8000) -- (115.3000,84.9000) -- (116.9000,82.8000) -- (117.8000,80.5000) -- (118.3000,78.0000) -- (118.4000,75.4000) -- (118.5000,72.9000) -- (118.5000,70.3000) -- (118.5000,67.8000) -- (118.5000,65.3000) -- (118.5000,62.7000) -- (118.5000,60.2000) -- (118.5000,57.7000) -- (118.5000,55.2000) -- (118.5000,52.7000) -- (118.5000,50.2000) -- (118.9000,47.6000) -- (119.2000,45.1000) -- (119.0000,42.7000) -- (118.4000,40.3000) -- (117.4000,38.0000) -- (115.9000,35.9000) -- (114.1000,34.1000) -- (112.1000,32.5000) -- (109.7000,31.3000) -- (107.2000,30.4000) -- (104.6000,29.9000) -- (101.9000,29.9000) -- (99.3000,30.2000) -- (96.7000,30.8000) -- (94.3000,31.9000) -- (92.1000,33.3000) -- (90.2000,35.1000) -- (88.6000,37.1000) -- (87.5000,39.4000) -- (86.8000,41.8000) -- (86.5000,44.3000) -- (86.4000,46.8000) -- (86.3000,49.3000) -- (86.3000,51.8000) -- (86.3000,54.3000) -- (86.3000,56.8000) -- (86.3000,59.3000) -- (86.3000,61.8000) -- (86.3000,64.3000) -- (86.3000,66.8000) -- (86.3000,69.4000) -- (86.3000,71.9000) -- (86.3000,74.4000) -- (86.5000,76.9000) -- (87.1000,79.5000) -- (88.2000,81.8000) -- (89.7000,84.0000) -- (91.5000,85.9000) -- (93.6000,87.5000) -- (95.8000,88.8000) -- (98.3000,89.8000) -- (100.9000,90.5000) -- (103.5000,90.7000) -- (106.1000,90.6000) -- (108.7000,90.2000) -- (111.2000,89.4000) -- (113.5000,88.2000) -- (115.7000,86.7000) -- (117.6000,84.9000) -- (119.2000,82.9000) -- (120.2000,80.5000) -- (120.7000,78.1000) -- (120.9000,75.5000) -- (120.9000,72.9000) -- (121.0000,70.4000) -- (121.0000,67.8000) -- (121.0000,65.3000) -- (121.0000,62.8000) -- (121.0000,60.3000) -- (121.0000,57.8000) -- (121.0000,55.2000) -- (121.0000,52.7000) -- (121.0000,50.2000) -- (121.3000,47.7000) -- (121.6000,45.2000) -- (121.5000,42.8000) -- (121.0000,40.4000) -- (120.0000,38.0000) -- (118.6000,35.9000) -- (116.8000,34.0000) -- (114.8000,32.4000) -- (112.4000,31.2000) -- (110.0000,30.3000) -- (107.4000,29.7000) -- (104.7000,29.6000) -- (102.0000,29.9000) -- (99.4000,30.5000) -- (97.0000,31.5000) -- (94.8000,32.9000) -- (92.8000,34.6000) -- (91.2000,36.6000) -- (90.0000,38.8000) -- (89.2000,41.2000) -- (88.9000,43.7000) -- (88.8000,46.2000) -- (88.7000,48.7000) -- (88.7000,51.2000) -- (88.7000,53.7000) -- (88.7000,56.2000) -- (88.7000,58.7000) -- (88.7000,61.2000) -- (88.7000,63.8000) -- (88.7000,66.3000) -- (88.7000,68.8000) -- (88.7000,71.3000) -- (88.7000,73.8000) -- (88.8000,76.4000) -- (89.4000,78.9000) -- (90.4000,81.3000) -- (91.8000,83.5000) -- (93.6000,85.4000) -- (95.6000,87.1000) -- (97.9000,88.4000) -- (100.3000,89.5000) -- (102.9000,90.2000) -- (105.5000,90.5000) -- (108.1000,90.5000) -- (110.7000,90.1000) -- (113.2000,89.3000) -- (115.6000,88.2000) -- (117.8000,86.8000) -- (119.8000,85.0000) -- (121.4000,83.0000) -- (122.5000,80.7000) -- (123.1000,78.3000) -- (123.3000,75.7000) -- (123.4000,73.1000) -- (123.4000,70.6000) -- (123.4000,68.0000) -- (123.4000,65.5000) -- (123.4000,63.0000) -- (123.4000,60.5000) -- (123.4000,58.0000) -- (123.4000,55.4000) -- (123.4000,52.9000) -- (123.4000,50.4000) -- (123.7000,47.9000) -- (124.1000,45.4000) -- (124.0000,42.9000) -- (123.5000,40.5000) -- (122.6000,38.2000) -- (121.3000,36.0000) -- (119.6000,34.1000) -- (117.6000,32.5000) -- (115.3000,31.1000) -- (112.8000,30.2000) -- (110.2000,29.6000) -- (107.6000,29.4000) -- (104.9000,29.6000) -- (102.3000,30.1000) -- (99.9000,31.1000) -- (97.6000,32.4000) -- (95.6000,34.1000) -- (93.9000,36.0000) -- (92.6000,38.2000) -- (91.8000,40.6000) -- (91.4000,43.1000) -- (91.2000,45.6000) -- (91.2000,48.0000) -- (91.2000,50.5000) -- (91.2000,53.0000) -- (91.2000,55.6000) -- (91.2000,58.1000) -- (91.2000,60.6000) -- (91.2000,63.1000) -- (91.2000,65.6000) -- (91.2000,68.1000) -- (91.2000,70.7000) -- (91.2000,73.2000) -- (91.2000,75.7000) -- (91.7000,78.3000) -- (92.6000,80.7000) -- (94.0000,82.9000) -- (95.7000,84.9000) -- (97.7000,86.6000) -- (99.9000,88.0000) -- (102.4000,89.1000) -- (104.9000,89.9000) -- (107.5000,90.3000) -- (110.1000,90.3000) -- (112.7000,90.0000) -- (115.3000,89.3000) -- (117.7000,88.2000) -- (119.9000,86.8000) -- (121.9000,85.1000) -- (123.6000,83.1000) -- (124.8000,80.9000) -- (125.5000,78.5000) -- (125.7000,75.9000) -- (125.8000,73.4000) -- (125.9000,70.8000) -- (125.9000,68.2000) -- (125.9000,65.7000) -- (125.9000,63.2000) -- (125.9000,60.7000) -- (125.9000,58.2000) -- (125.9000,55.7000) -- (125.9000,53.1000) -- (125.9000,50.6000) -- (126.0000,48.1000) -- (126.5000,45.5000);



  \end{scope}
  \begin{scope}[scale=1.006,draw=black,line join=bevel,line cap=rect,line width=0.800pt]
  \end{scope}
  \begin{scope}[scale=1.006,draw=black,line join=bevel,line cap=rect,line width=0.800pt]
  \end{scope}
  \begin{scope}[scale=1.006,draw=black,line join=round,line cap=round,line width=0.480pt]
    \path[draw] (44.5000,13.5000) -- (44.5000,102.5000) -- (142.5000,102.5000) -- (142.5000,13.5000) -- (44.5000,13.5000);



  \end{scope}
  \begin{scope}[scale=1.006,draw=ca0a0a4,dash pattern=on 0.40pt off 0.80pt,line join=round,line cap=round,line width=0.400pt]
    \path[draw] (148.5000,102.5000) -- (246.5000,102.5000);



  \end{scope}
  \begin{scope}[scale=1.006,draw=black,line join=round,line cap=round,line width=0.480pt]
    \path[draw] (148.5000,102.5000) -- (151.5000,102.5000);



    \path[draw] (246.5000,102.5000) -- (243.5000,102.5000);



  \end{scope}
  \begin{scope}[scale=1.006,draw=black,line join=bevel,line cap=rect,line width=0.800pt]
  \end{scope}
  \begin{scope}[cm={{1.00588,0.0,0.0,1.00588,(143.841,102.6)}},draw=black,line join=bevel,line cap=rect,line width=0.800pt]
  \end{scope}
  \begin{scope}[cm={{1.00588,0.0,0.0,1.00588,(143.841,102.6)}},draw=black,line join=bevel,line cap=rect,line width=0.800pt]
  \end{scope}
  \begin{scope}[cm={{1.00588,0.0,0.0,1.00588,(143.841,102.6)}},draw=black,line join=bevel,line cap=rect,line width=0.800pt]
  \end{scope}
  \begin{scope}[cm={{1.00588,0.0,0.0,1.00588,(143.841,102.6)}},draw=black,line join=bevel,line cap=rect,line width=0.800pt]
  \end{scope}
  \begin{scope}[cm={{1.00588,0.0,0.0,1.00588,(143.841,102.6)}},draw=black,line join=bevel,line cap=rect,line width=0.800pt]
  \end{scope}
  \begin{scope}[cm={{1.00588,0.0,0.0,1.00588,(143.841,102.6)}},draw=black,line join=bevel,line cap=rect,line width=0.800pt]
  \end{scope}
  \begin{scope}[scale=1.006,draw=black,line join=bevel,line cap=rect,line width=0.800pt]
  \end{scope}
  \begin{scope}[scale=1.006,draw=ca0a0a4,dash pattern=on 0.40pt off 0.80pt,line join=round,line cap=round,line width=0.400pt]
    \path[draw] (148.5000,79.5000) -- (246.5000,79.5000);



  \end{scope}
  \begin{scope}[scale=1.006,draw=black,line join=round,line cap=round,line width=0.480pt]
    \path[draw] (148.5000,79.5000) -- (151.5000,79.5000);



    \path[draw] (246.5000,79.5000) -- (243.5000,79.5000);



  \end{scope}
  \begin{scope}[scale=1.006,draw=black,line join=bevel,line cap=rect,line width=0.800pt]
  \end{scope}
  \begin{scope}[cm={{1.00588,0.0,0.0,1.00588,(143.841,80.4706)}},draw=black,line join=bevel,line cap=rect,line width=0.800pt]
  \end{scope}
  \begin{scope}[cm={{1.00588,0.0,0.0,1.00588,(143.841,80.4706)}},draw=black,line join=bevel,line cap=rect,line width=0.800pt]
  \end{scope}
  \begin{scope}[cm={{1.00588,0.0,0.0,1.00588,(143.841,80.4706)}},draw=black,line join=bevel,line cap=rect,line width=0.800pt]
  \end{scope}
  \begin{scope}[cm={{1.00588,0.0,0.0,1.00588,(143.841,80.4706)}},draw=black,line join=bevel,line cap=rect,line width=0.800pt]
  \end{scope}
  \begin{scope}[cm={{1.00588,0.0,0.0,1.00588,(143.841,80.4706)}},draw=black,line join=bevel,line cap=rect,line width=0.800pt]
  \end{scope}
  \begin{scope}[cm={{1.00588,0.0,0.0,1.00588,(143.841,80.4706)}},draw=black,line join=bevel,line cap=rect,line width=0.800pt]
  \end{scope}
  \begin{scope}[scale=1.006,draw=black,line join=bevel,line cap=rect,line width=0.800pt]
  \end{scope}
  \begin{scope}[scale=1.006,draw=ca0a0a4,dash pattern=on 0.40pt off 0.80pt,line join=round,line cap=round,line width=0.400pt]
    \path[draw] (148.5000,57.5000) -- (246.5000,57.5000);



  \end{scope}
  \begin{scope}[scale=1.006,draw=black,line join=round,line cap=round,line width=0.480pt]
    \path[draw] (148.5000,57.5000) -- (151.5000,57.5000);



    \path[draw] (246.5000,57.5000) -- (243.5000,57.5000);



  \end{scope}
  \begin{scope}[scale=1.006,draw=black,line join=bevel,line cap=rect,line width=0.800pt]
  \end{scope}
  \begin{scope}[cm={{1.00588,0.0,0.0,1.00588,(143.841,58.3412)}},draw=black,line join=bevel,line cap=rect,line width=0.800pt]
  \end{scope}
  \begin{scope}[cm={{1.00588,0.0,0.0,1.00588,(143.841,58.3412)}},draw=black,line join=bevel,line cap=rect,line width=0.800pt]
  \end{scope}
  \begin{scope}[cm={{1.00588,0.0,0.0,1.00588,(143.841,58.3412)}},draw=black,line join=bevel,line cap=rect,line width=0.800pt]
  \end{scope}
  \begin{scope}[cm={{1.00588,0.0,0.0,1.00588,(143.841,58.3412)}},draw=black,line join=bevel,line cap=rect,line width=0.800pt]
  \end{scope}
  \begin{scope}[cm={{1.00588,0.0,0.0,1.00588,(143.841,58.3412)}},draw=black,line join=bevel,line cap=rect,line width=0.800pt]
  \end{scope}
  \begin{scope}[cm={{1.00588,0.0,0.0,1.00588,(143.841,58.3412)}},draw=black,line join=bevel,line cap=rect,line width=0.800pt]
  \end{scope}
  \begin{scope}[scale=1.006,draw=black,line join=bevel,line cap=rect,line width=0.800pt]
  \end{scope}
  \begin{scope}[scale=1.006,draw=ca0a0a4,dash pattern=on 0.40pt off 0.80pt,line join=round,line cap=round,line width=0.400pt]
    \path[draw] (148.5000,35.5000) -- (246.5000,35.5000);



  \end{scope}
  \begin{scope}[scale=1.006,draw=black,line join=round,line cap=round,line width=0.480pt]
    \path[draw] (148.5000,35.5000) -- (151.5000,35.5000);



    \path[draw] (246.5000,35.5000) -- (243.5000,35.5000);



  \end{scope}
  \begin{scope}[scale=1.006,draw=black,line join=bevel,line cap=rect,line width=0.800pt]
  \end{scope}
  \begin{scope}[cm={{1.00588,0.0,0.0,1.00588,(143.841,36.2118)}},draw=black,line join=bevel,line cap=rect,line width=0.800pt]
  \end{scope}
  \begin{scope}[cm={{1.00588,0.0,0.0,1.00588,(143.841,36.2118)}},draw=black,line join=bevel,line cap=rect,line width=0.800pt]
  \end{scope}
  \begin{scope}[cm={{1.00588,0.0,0.0,1.00588,(143.841,36.2118)}},draw=black,line join=bevel,line cap=rect,line width=0.800pt]
  \end{scope}
  \begin{scope}[cm={{1.00588,0.0,0.0,1.00588,(143.841,36.2118)}},draw=black,line join=bevel,line cap=rect,line width=0.800pt]
  \end{scope}
  \begin{scope}[cm={{1.00588,0.0,0.0,1.00588,(143.841,36.2118)}},draw=black,line join=bevel,line cap=rect,line width=0.800pt]
  \end{scope}
  \begin{scope}[cm={{1.00588,0.0,0.0,1.00588,(143.841,36.2118)}},draw=black,line join=bevel,line cap=rect,line width=0.800pt]
  \end{scope}
  \begin{scope}[scale=1.006,draw=black,line join=bevel,line cap=rect,line width=0.800pt]
  \end{scope}
  \begin{scope}[scale=1.006,draw=ca0a0a4,dash pattern=on 0.40pt off 0.80pt,line join=round,line cap=round,line width=0.400pt]
    \path[draw] (148.5000,13.5000) -- (246.5000,13.5000);



  \end{scope}
  \begin{scope}[scale=1.006,draw=black,line join=round,line cap=round,line width=0.480pt]
    \path[draw] (148.5000,13.5000) -- (151.5000,13.5000);



    \path[draw] (246.5000,13.5000) -- (243.5000,13.5000);



  \end{scope}
  \begin{scope}[scale=1.006,draw=black,line join=bevel,line cap=rect,line width=0.800pt]
  \end{scope}
  \begin{scope}[cm={{1.00588,0.0,0.0,1.00588,(143.841,14.0824)}},draw=black,line join=bevel,line cap=rect,line width=0.800pt]
  \end{scope}
  \begin{scope}[cm={{1.00588,0.0,0.0,1.00588,(143.841,14.0824)}},draw=black,line join=bevel,line cap=rect,line width=0.800pt]
  \end{scope}
  \begin{scope}[cm={{1.00588,0.0,0.0,1.00588,(143.841,14.0824)}},draw=black,line join=bevel,line cap=rect,line width=0.800pt]
  \end{scope}
  \begin{scope}[cm={{1.00588,0.0,0.0,1.00588,(143.841,14.0824)}},draw=black,line join=bevel,line cap=rect,line width=0.800pt]
  \end{scope}
  \begin{scope}[cm={{1.00588,0.0,0.0,1.00588,(143.841,14.0824)}},draw=black,line join=bevel,line cap=rect,line width=0.800pt]
  \end{scope}
  \begin{scope}[cm={{1.00588,0.0,0.0,1.00588,(143.841,14.0824)}},draw=black,line join=bevel,line cap=rect,line width=0.800pt]
  \end{scope}
  \begin{scope}[scale=1.006,draw=black,line join=bevel,line cap=rect,line width=0.800pt]
  \end{scope}
  \begin{scope}[scale=1.006,draw=ca0a0a4,dash pattern=on 0.40pt off 0.80pt,line join=round,line cap=round,line width=0.400pt]
    \path[draw] (148.5000,102.5000) -- (148.5000,13.5000);



  \end{scope}
  \begin{scope}[scale=1.006,draw=black,line join=round,line cap=round,line width=0.480pt]
    \path[draw] (148.5000,102.5000) -- (148.5000,99.5000);



    \path[draw] (148.5000,13.5000) -- (148.5000,16.5000);



  \end{scope}
  \begin{scope}[scale=1.006,draw=black,line join=bevel,line cap=rect,line width=0.800pt]
  \end{scope}
  \begin{scope}[cm={{1.00588,0.0,0.0,1.00588,(148.871,118.694)}},draw=black,line join=bevel,line cap=rect,line width=0.800pt]
  \end{scope}
  \begin{scope}[cm={{1.00588,0.0,0.0,1.00588,(148.871,118.694)}},draw=black,line join=bevel,line cap=rect,line width=0.800pt]
  \end{scope}
  \begin{scope}[cm={{1.00588,0.0,0.0,1.00588,(148.871,118.694)}},draw=black,line join=bevel,line cap=rect,line width=0.800pt]
  \end{scope}
  \begin{scope}[cm={{1.00588,0.0,0.0,1.00588,(148.871,118.694)}},draw=black,line join=bevel,line cap=rect,line width=0.800pt]
  \end{scope}
  \begin{scope}[cm={{1.00588,0.0,0.0,1.00588,(148.871,118.694)}},draw=black,line join=bevel,line cap=rect,line width=0.800pt]
  \end{scope}
  \begin{scope}[cm={{1.00588,0.0,0.0,1.00588,(148.871,118.694)}},draw=black,line join=bevel,line cap=rect,line width=0.800pt]
  \end{scope}
  \begin{scope}[scale=1.006,draw=black,line join=bevel,line cap=rect,line width=0.800pt]
  \end{scope}
  \begin{scope}[scale=1.006,draw=ca0a0a4,dash pattern=on 0.40pt off 0.80pt,line join=round,line cap=round,line width=0.400pt]
    \path[draw] (172.5000,102.5000) -- (172.5000,27.5000);



    \path[draw] (172.5000,19.5000) -- (172.5000,13.5000);



  \end{scope}
  \begin{scope}[scale=1.006,draw=black,line join=round,line cap=round,line width=0.480pt]
    \path[draw] (172.5000,102.5000) -- (172.5000,99.5000);



    \path[draw] (172.5000,13.5000) -- (172.5000,16.5000);



  \end{scope}
  \begin{scope}[scale=1.006,draw=black,line join=bevel,line cap=rect,line width=0.800pt]
  \end{scope}
  \begin{scope}[cm={{1.00588,0.0,0.0,1.00588,(174.018,118.694)}},draw=black,line join=bevel,line cap=rect,line width=0.800pt]
  \end{scope}
  \begin{scope}[cm={{1.00588,0.0,0.0,1.00588,(174.018,118.694)}},draw=black,line join=bevel,line cap=rect,line width=0.800pt]
  \end{scope}
  \begin{scope}[cm={{1.00588,0.0,0.0,1.00588,(174.018,118.694)}},draw=black,line join=bevel,line cap=rect,line width=0.800pt]
  \end{scope}
  \begin{scope}[cm={{1.00588,0.0,0.0,1.00588,(174.018,118.694)}},draw=black,line join=bevel,line cap=rect,line width=0.800pt]
  \end{scope}
  \begin{scope}[cm={{1.00588,0.0,0.0,1.00588,(174.018,118.694)}},draw=black,line join=bevel,line cap=rect,line width=0.800pt]
  \end{scope}
  \begin{scope}[cm={{1.00588,0.0,0.0,1.00588,(174.018,118.694)}},draw=black,line join=bevel,line cap=rect,line width=0.800pt]
  \end{scope}
  \begin{scope}[scale=1.006,draw=black,line join=bevel,line cap=rect,line width=0.800pt]
  \end{scope}
  \begin{scope}[scale=1.006,draw=ca0a0a4,dash pattern=on 0.40pt off 0.80pt,line join=round,line cap=round,line width=0.400pt]
    \path[draw] (197.5000,102.5000) -- (197.5000,27.5000);



    \path[draw] (197.5000,19.5000) -- (197.5000,13.5000);



  \end{scope}
  \begin{scope}[scale=1.006,draw=black,line join=round,line cap=round,line width=0.480pt]
    \path[draw] (197.5000,102.5000) -- (197.5000,99.5000);



    \path[draw] (197.5000,13.5000) -- (197.5000,16.5000);



  \end{scope}
  \begin{scope}[scale=1.006,draw=black,line join=bevel,line cap=rect,line width=0.800pt]
  \end{scope}
  \begin{scope}[cm={{1.00588,0.0,0.0,1.00588,(198.159,118.694)}},draw=black,line join=bevel,line cap=rect,line width=0.800pt]
  \end{scope}
  \begin{scope}[cm={{1.00588,0.0,0.0,1.00588,(198.159,118.694)}},draw=black,line join=bevel,line cap=rect,line width=0.800pt]
  \end{scope}
  \begin{scope}[cm={{1.00588,0.0,0.0,1.00588,(198.159,118.694)}},draw=black,line join=bevel,line cap=rect,line width=0.800pt]
  \end{scope}
  \begin{scope}[cm={{1.00588,0.0,0.0,1.00588,(198.159,118.694)}},draw=black,line join=bevel,line cap=rect,line width=0.800pt]
  \end{scope}
  \begin{scope}[cm={{1.00588,0.0,0.0,1.00588,(198.159,118.694)}},draw=black,line join=bevel,line cap=rect,line width=0.800pt]
  \end{scope}
  \begin{scope}[cm={{1.00588,0.0,0.0,1.00588,(198.159,118.694)}},draw=black,line join=bevel,line cap=rect,line width=0.800pt]
  \end{scope}
  \begin{scope}[scale=1.006,draw=black,line join=bevel,line cap=rect,line width=0.800pt]
  \end{scope}
  \begin{scope}[scale=1.006,draw=ca0a0a4,dash pattern=on 0.40pt off 0.80pt,line join=round,line cap=round,line width=0.400pt]
    \path[draw] (221.5000,102.5000) -- (221.5000,27.5000);



    \path[draw] (221.5000,19.5000) -- (221.5000,13.5000);



  \end{scope}
  \begin{scope}[scale=1.006,draw=black,line join=round,line cap=round,line width=0.480pt]
    \path[draw] (221.5000,102.5000) -- (221.5000,99.5000);



    \path[draw] (221.5000,13.5000) -- (221.5000,16.5000);



  \end{scope}
  \begin{scope}[scale=1.006,draw=black,line join=bevel,line cap=rect,line width=0.800pt]
  \end{scope}
  \begin{scope}[cm={{1.00588,0.0,0.0,1.00588,(223.306,118.694)}},draw=black,line join=bevel,line cap=rect,line width=0.800pt]
  \end{scope}
  \begin{scope}[cm={{1.00588,0.0,0.0,1.00588,(223.306,118.694)}},draw=black,line join=bevel,line cap=rect,line width=0.800pt]
  \end{scope}
  \begin{scope}[cm={{1.00588,0.0,0.0,1.00588,(223.306,118.694)}},draw=black,line join=bevel,line cap=rect,line width=0.800pt]
  \end{scope}
  \begin{scope}[cm={{1.00588,0.0,0.0,1.00588,(223.306,118.694)}},draw=black,line join=bevel,line cap=rect,line width=0.800pt]
  \end{scope}
  \begin{scope}[cm={{1.00588,0.0,0.0,1.00588,(223.306,118.694)}},draw=black,line join=bevel,line cap=rect,line width=0.800pt]
  \end{scope}
  \begin{scope}[cm={{1.00588,0.0,0.0,1.00588,(223.306,118.694)}},draw=black,line join=bevel,line cap=rect,line width=0.800pt]
  \end{scope}
  \begin{scope}[scale=1.006,draw=black,line join=bevel,line cap=rect,line width=0.800pt]
  \end{scope}
  \begin{scope}[scale=1.006,draw=ca0a0a4,dash pattern=on 0.40pt off 0.80pt,line join=round,line cap=round,line width=0.400pt]
    \path[draw] (246.5000,102.5000) -- (246.5000,27.5000);



    \path[draw] (246.5000,19.5000) -- (246.5000,13.5000);



  \end{scope}
  \begin{scope}[scale=1.006,draw=black,line join=round,line cap=round,line width=0.480pt]
    \path[draw] (246.5000,102.5000) -- (246.5000,99.5000);



    \path[draw] (246.5000,13.5000) -- (246.5000,16.5000);



  \end{scope}
  \begin{scope}[scale=1.006,draw=black,line join=bevel,line cap=rect,line width=0.800pt]
  \end{scope}
  \begin{scope}[cm={{1.00588,0.0,0.0,1.00588,(247.447,118.694)}},draw=black,line join=bevel,line cap=rect,line width=0.800pt]
  \end{scope}
  \begin{scope}[cm={{1.00588,0.0,0.0,1.00588,(247.447,118.694)}},draw=black,line join=bevel,line cap=rect,line width=0.800pt]
  \end{scope}
  \begin{scope}[cm={{1.00588,0.0,0.0,1.00588,(247.447,118.694)}},draw=black,line join=bevel,line cap=rect,line width=0.800pt]
  \end{scope}
  \begin{scope}[cm={{1.00588,0.0,0.0,1.00588,(247.447,118.694)}},draw=black,line join=bevel,line cap=rect,line width=0.800pt]
  \end{scope}
  \begin{scope}[cm={{1.00588,0.0,0.0,1.00588,(247.447,118.694)}},draw=black,line join=bevel,line cap=rect,line width=0.800pt]
  \end{scope}
  \begin{scope}[cm={{1.00588,0.0,0.0,1.00588,(247.447,118.694)}},draw=black,line join=bevel,line cap=rect,line width=0.800pt]
  \end{scope}
  \begin{scope}[scale=1.006,draw=black,line join=bevel,line cap=rect,line width=0.800pt]
  \end{scope}
  \begin{scope}[scale=1.006,draw=black,line join=round,line cap=round,line width=0.480pt]
    \path[draw] (148.5000,13.5000) -- (148.5000,102.5000) -- (246.5000,102.5000) -- (246.5000,13.5000) -- (148.5000,13.5000);



  \end{scope}
  \begin{scope}[scale=1.006,draw=black,line join=bevel,line cap=rect,line width=0.800pt]
  \end{scope}
  \begin{scope}[scale=1.006,draw=black,line join=bevel,line cap=rect,line width=0.800pt]
  \end{scope}
  \begin{scope}[scale=1.006,fill=cffffff]
    \path[fill,rounded corners=0.0000cm] (222.0000,18.0000) rectangle (238.0000,34.0000);



  \end{scope}
  \begin{scope}[scale=1.006,draw=black,line join=bevel,line cap=rect,line width=0.800pt]
  \end{scope}
  \begin{scope}[scale=1.006,draw=black,line join=bevel,line cap=rect,line width=0.800pt]
  \end{scope}
  \begin{scope}[scale=1.006,draw=black,line join=round,line cap=round,line width=0.800pt]
    \path[draw] (222.5000,34.5000) -- (222.5000,18.5000) -- (238.5000,18.5000) -- (238.5000,34.5000) -- (222.5000,34.5000);



  \end{scope}
  \begin{scope}[scale=1.006,draw=black,line join=bevel,line cap=rect,line width=0.800pt]
  \end{scope}
  \begin{scope}[cm={{1.00588,0.0,0.0,1.00588,(227.329,30.1765)}},draw=black,line join=bevel,line cap=rect,line width=0.800pt]
  \end{scope}
  \begin{scope}[cm={{1.00588,0.0,0.0,1.00588,(227.329,30.1765)}},draw=black,line join=bevel,line cap=rect,line width=0.800pt]
  \end{scope}
  \begin{scope}[cm={{1.00588,0.0,0.0,1.00588,(227.329,30.1765)}},draw=black,line join=bevel,line cap=rect,line width=0.800pt]
  \end{scope}
  \begin{scope}[cm={{1.00588,0.0,0.0,1.00588,(227.329,30.1765)}},draw=black,line join=bevel,line cap=rect,line width=0.800pt]
  \end{scope}
  \begin{scope}[cm={{1.00588,0.0,0.0,1.00588,(227.329,30.1765)}},draw=black,line join=bevel,line cap=rect,line width=0.800pt]
  \end{scope}
  \begin{scope}[cm={{1.00588,0.0,0.0,1.00588,(229.28479,29.95919)}},draw=black,line join=bevel,line cap=rect,line width=0.800pt]
    \path[fill=black] (0.0000,0.0000) node[above right] (text692) {\label{fig:trajs-II-static}II};



  \end{scope}
  \begin{scope}[cm={{1.00588,0.0,0.0,1.00588,(227.329,30.1765)}},draw=black,line join=bevel,line cap=rect,line width=0.800pt]
  \end{scope}
  \begin{scope}[scale=1.006,draw=black,line join=bevel,line cap=rect,line width=0.800pt]
  \end{scope}
  \begin{scope}[scale=1.006,draw=black,line join=bevel,line cap=rect,line width=0.800pt]
  \end{scope}
  \begin{scope}[scale=1.006,draw=black,line join=bevel,line cap=rect,line width=0.800pt]
  \end{scope}
  \begin{scope}[scale=1.006,draw=black,line join=round,line cap=round,line width=0.480pt]
    \path[draw] (160.6000,31.9000) -- (160.6000,31.9000) -- (160.7000,33.1000) -- (160.7000,34.2000) -- (160.8000,35.2000) -- (160.9000,36.2000) -- (160.9000,37.2000) -- (160.6000,38.2000) -- (159.9000,39.1000) -- (159.3000,40.0000) -- (158.7000,40.9000) -- (158.2000,41.8000) -- (157.7000,42.8000) -- (157.4000,43.8000) -- (157.2000,44.8000) -- (157.1000,45.8000) -- (157.0000,46.9000) -- (157.0000,48.0000) -- (157.0000,49.0000) -- (157.0000,50.1000) -- (157.0000,51.1000) -- (157.0000,52.2000) -- (157.0000,53.3000) -- (157.0000,54.3000) -- (157.0000,55.4000) -- (157.0000,56.4000) -- (157.0000,57.5000) -- (157.0000,58.5000) -- (157.0000,59.6000) -- (157.0000,60.6000) -- (157.0000,61.7000) -- (157.0000,62.8000) -- (157.0000,63.8000) -- (157.0000,64.9000) -- (157.0000,65.9000) -- (157.0000,67.0000) -- (157.0000,68.0000) -- (157.0000,69.1000) -- (157.0000,70.1000) -- (157.0000,71.2000) -- (157.0000,72.2000) -- (157.0000,73.3000) -- (157.0000,74.4000) -- (157.0000,75.4000) -- (157.0000,76.5000) -- (157.0000,77.5000) -- (157.0000,78.6000) -- (157.1000,79.6000) -- (157.3000,80.7000) -- (157.6000,81.7000) -- (157.9000,82.7000) -- (158.3000,83.7000) -- (158.8000,84.7000) -- (159.4000,85.6000) -- (160.0000,86.5000) -- (160.7000,87.4000) -- (161.4000,88.3000) -- (162.2000,89.0000) -- (163.1000,89.8000) -- (164.0000,90.5000) -- (165.0000,91.1000) -- (166.0000,91.7000) -- (167.1000,92.2000) -- (168.2000,92.7000) -- (169.3000,93.1000) -- (170.5000,93.4000) -- (171.7000,93.6000) -- (172.9000,93.8000) -- (174.1000,93.9000) -- (175.3000,94.0000) -- (176.6000,93.9000) -- (177.8000,93.8000) -- (179.0000,93.6000) -- (180.2000,93.4000) -- (181.4000,93.0000) -- (182.5000,92.6000) -- (183.7000,92.1000) -- (184.7000,91.6000) -- (185.8000,91.0000) -- (186.7000,90.3000) -- (187.7000,89.6000) -- (188.5000,88.8000) -- (189.4000,88.0000) -- (190.1000,87.2000) -- (190.8000,86.3000) -- (191.4000,85.3000) -- (191.9000,84.4000) -- (192.3000,83.3000) -- (192.6000,82.3000) -- (192.8000,81.3000) -- (193.0000,80.3000) -- (193.1000,79.2000) -- (193.2000,78.2000) -- (193.3000,77.2000) -- (193.3000,76.1000) -- (193.4000,75.1000) -- (193.4000,74.0000) -- (193.4000,73.0000) -- (193.4000,71.9000) -- (193.4000,70.9000) -- (193.4000,69.8000) -- (193.4000,68.8000) -- (193.5000,67.7000) -- (193.5000,66.6000) -- (193.5000,65.6000) -- (193.5000,64.5000) -- (193.5000,63.5000) -- (193.5000,62.4000) -- (193.5000,61.4000) -- (193.5000,60.3000) -- (193.5000,59.3000) -- (193.5000,58.2000) -- (193.5000,57.2000) -- (193.5000,56.1000) -- (193.5000,55.0000) -- (193.5000,54.0000) -- (193.5000,52.9000) -- (193.5000,51.9000) -- (193.5000,50.8000) -- (193.7000,49.8000) -- (193.9000,48.7000) -- (194.0000,47.7000) -- (194.0000,46.6000) -- (193.9000,45.6000) -- (193.8000,44.5000) -- (193.6000,43.5000) -- (193.2000,42.5000) -- (192.8000,41.5000) -- (192.4000,40.5000) -- (191.8000,39.6000) -- (191.2000,38.7000) -- (190.4000,37.8000) -- (189.6000,37.0000) -- (188.8000,36.3000) -- (187.8000,35.6000) -- (186.8000,34.9000) -- (185.8000,34.4000) -- (184.7000,33.9000) -- (183.5000,33.5000) -- (182.3000,33.2000) -- (181.1000,33.0000) -- (179.9000,32.8000) -- (178.7000,32.8000) -- (177.4000,32.8000) -- (176.2000,32.9000) -- (175.0000,33.1000) -- (173.8000,33.4000) -- (172.7000,33.8000) -- (171.6000,34.2000) -- (170.5000,34.8000) -- (169.5000,35.3000) -- (168.5000,36.0000) -- (167.6000,36.7000) -- (166.8000,37.5000) -- (166.0000,38.3000) -- (165.4000,39.2000) -- (164.7000,40.1000) -- (164.2000,41.1000) -- (163.8000,42.1000) -- (163.4000,43.1000) -- (163.2000,44.1000) -- (163.1000,45.2000) -- (163.0000,46.2000) -- (163.0000,47.3000) -- (163.0000,48.4000) -- (163.0000,49.4000) -- (163.0000,50.5000) -- (163.0000,51.5000) -- (163.0000,52.6000) -- (163.0000,53.6000) -- (163.0000,54.7000) -- (162.9000,55.7000) -- (162.9000,56.8000) -- (162.9000,57.9000) -- (162.9000,58.9000) -- (162.9000,60.0000) -- (162.9000,61.0000) -- (162.9000,62.1000) -- (162.9000,63.1000) -- (162.9000,64.2000) -- (162.9000,65.2000) -- (162.9000,66.3000) -- (162.9000,67.4000) -- (162.9000,68.4000) -- (162.9000,69.5000) -- (162.9000,70.5000) -- (162.9000,71.6000) -- (162.9000,72.6000) -- (162.9000,73.7000) -- (162.9000,74.7000) -- (162.9000,75.8000) -- (162.9000,76.9000) -- (163.0000,77.9000) -- (163.1000,79.0000) -- (163.3000,80.0000) -- (163.5000,81.0000) -- (163.9000,82.1000) -- (164.3000,83.1000) -- (164.7000,84.0000) -- (165.3000,85.0000) -- (165.9000,85.9000) -- (166.6000,86.8000) -- (167.3000,87.6000) -- (168.1000,88.4000) -- (168.9000,89.2000) -- (169.9000,89.9000) -- (170.8000,90.5000) -- (171.8000,91.1000) -- (172.9000,91.7000) -- (174.0000,92.1000) -- (175.1000,92.5000) -- (176.3000,92.9000) -- (177.4000,93.2000) -- (178.6000,93.4000) -- (179.9000,93.5000) -- (181.1000,93.5000) -- (182.3000,93.5000) -- (183.6000,93.4000) -- (184.8000,93.3000) -- (186.0000,93.0000) -- (187.2000,92.7000) -- (188.3000,92.3000) -- (189.5000,91.9000) -- (190.6000,91.4000) -- (191.6000,90.8000) -- (192.6000,90.1000) -- (193.6000,89.4000) -- (194.5000,88.7000) -- (195.3000,87.9000) -- (196.1000,87.1000) -- (196.8000,86.2000) -- (197.4000,85.2000) -- (197.9000,84.3000) -- (198.4000,83.3000) -- (198.7000,82.3000) -- (198.9000,81.2000) -- (199.1000,80.2000) -- (199.3000,79.2000) -- (199.4000,78.1000) -- (199.4000,77.1000) -- (199.5000,76.1000) -- (199.5000,75.0000) -- (199.6000,74.0000) -- (199.6000,72.9000) -- (199.6000,71.9000) -- (199.6000,70.8000) -- (199.6000,69.8000) -- (199.6000,68.7000) -- (199.6000,67.7000) -- (199.6000,66.6000) -- (199.6000,65.5000) -- (199.7000,64.5000) -- (199.7000,63.4000) -- (199.7000,62.4000) -- (199.7000,61.3000) -- (199.7000,60.3000) -- (199.7000,59.2000) -- (199.7000,58.2000) -- (199.7000,57.1000) -- (199.7000,56.0000) -- (199.7000,55.0000) -- (199.7000,53.9000) -- (199.7000,52.9000) -- (199.7000,51.8000) -- (199.7000,50.8000) -- (199.8000,49.7000) -- (200.0000,48.7000) -- (200.1000,47.6000) -- (200.2000,46.6000) -- (200.1000,45.5000) -- (200.0000,44.5000) -- (199.8000,43.4000) -- (199.6000,42.4000) -- (199.2000,41.4000) -- (198.8000,40.4000) -- (198.3000,39.5000) -- (197.6000,38.5000) -- (197.0000,37.7000) -- (196.2000,36.8000) -- (195.4000,36.1000) -- (194.5000,35.3000) -- (193.5000,34.7000) -- (192.5000,34.1000) -- (191.4000,33.5000) -- (190.2000,33.1000) -- (189.1000,32.7000) -- (187.9000,32.5000) -- (186.7000,32.3000) -- (185.4000,32.2000) -- (184.2000,32.2000) -- (183.0000,32.2000) -- (181.8000,32.4000) -- (180.6000,32.7000) -- (179.4000,33.0000) -- (178.3000,33.4000) -- (177.2000,33.9000) -- (176.1000,34.4000) -- (175.2000,35.1000) -- (174.2000,35.8000) -- (173.4000,36.5000) -- (172.6000,37.3000) -- (171.8000,38.2000) -- (171.2000,39.0000) -- (170.6000,40.0000) -- (170.1000,41.0000) -- (169.7000,42.0000) -- (169.4000,43.0000) -- (169.2000,44.0000) -- (169.2000,45.1000) -- (169.1000,46.1000) -- (169.1000,47.2000) -- (169.1000,48.3000) -- (169.1000,49.3000) -- (169.1000,50.4000) -- (169.1000,51.4000) -- (169.1000,52.5000) -- (169.1000,53.6000) -- (169.1000,54.6000) -- (169.1000,55.7000) -- (169.1000,56.7000) -- (169.1000,57.8000) -- (169.1000,58.8000) -- (169.1000,59.9000) -- (169.1000,60.9000) -- (169.1000,62.0000) -- (169.1000,63.1000) -- (169.1000,64.1000) -- (169.1000,65.2000) -- (169.1000,66.2000) -- (169.1000,67.3000) -- (169.1000,68.3000) -- (169.1000,69.4000) -- (169.1000,70.4000) -- (169.1000,71.5000) -- (169.1000,72.6000) -- (169.1000,73.6000) -- (169.1000,74.7000) -- (169.1000,75.7000) -- (169.1000,76.8000) -- (169.1000,77.8000) -- (169.3000,78.9000) -- (169.5000,79.9000) -- (169.8000,80.9000) -- (170.1000,82.0000) -- (170.6000,82.9000) -- (171.1000,83.9000) -- (171.7000,84.8000) -- (172.3000,85.7000) -- (173.0000,86.6000) -- (173.8000,87.4000) -- (174.6000,88.2000) -- (175.5000,88.9000) -- (176.4000,89.6000) -- (177.4000,90.2000) -- (178.4000,90.8000) -- (179.5000,91.3000) -- (180.6000,91.8000) -- (181.7000,92.2000) -- (182.9000,92.5000) -- (184.1000,92.7000) -- (185.3000,92.9000) -- (186.5000,93.0000) -- (187.8000,93.0000) -- (189.0000,93.0000) -- (190.2000,92.8000) -- (191.4000,92.6000) -- (192.6000,92.4000) -- (193.8000,92.0000) -- (195.0000,91.6000) -- (196.1000,91.1000) -- (197.2000,90.6000) -- (198.2000,90.0000) -- (199.2000,89.3000) -- (200.1000,88.6000) -- (201.0000,87.8000) -- (201.8000,87.0000) -- (202.5000,86.2000) -- (203.2000,85.3000) -- (203.8000,84.3000) -- (204.3000,83.3000) -- (204.7000,82.3000) -- (205.0000,81.3000) -- (205.2000,80.3000) -- (205.3000,79.2000) -- (205.5000,78.2000) -- (205.6000,77.2000) -- (205.6000,76.1000) -- (205.7000,75.1000) -- (205.7000,74.0000) -- (205.7000,73.0000) -- (205.8000,71.9000) -- (205.8000,70.9000) -- (205.8000,69.8000) -- (205.8000,68.8000) -- (205.8000,67.7000) -- (205.8000,66.7000) -- (205.8000,65.6000) -- (205.8000,64.6000) -- (205.8000,63.5000) -- (205.8000,62.5000) -- (205.8000,61.4000) -- (205.8000,60.3000) -- (205.8000,59.3000) -- (205.8000,58.2000) -- (205.8000,57.2000) -- (205.8000,56.1000) -- (205.8000,55.1000) -- (205.8000,54.0000) -- (205.8000,53.0000) -- (205.8000,51.9000) -- (205.8000,50.8000) -- (205.9000,49.8000) -- (206.0000,48.7000) -- (206.2000,47.7000) -- (206.3000,46.6000) -- (206.3000,45.6000) -- (206.3000,44.5000) -- (206.1000,43.5000) -- (205.9000,42.5000) -- (205.6000,41.4000) -- (205.2000,40.4000) -- (204.7000,39.5000) -- (204.2000,38.5000) -- (203.5000,37.6000) -- (202.8000,36.8000) -- (202.0000,36.0000) -- (201.2000,35.2000) -- (200.2000,34.5000) -- (199.2000,33.9000) -- (198.2000,33.3000) -- (197.1000,32.8000) -- (195.9000,32.4000) -- (194.8000,32.1000) -- (193.5000,31.8000) -- (192.3000,31.7000) -- (191.1000,31.6000) -- (189.9000,31.6000) -- (188.6000,31.7000) -- (187.4000,31.9000) -- (186.3000,32.2000) -- (185.1000,32.6000) -- (184.0000,33.0000) -- (182.9000,33.5000) -- (181.9000,34.1000) -- (180.9000,34.8000) -- (180.0000,35.5000) -- (179.2000,36.2000) -- (178.4000,37.1000) -- (177.7000,37.9000) -- (177.1000,38.9000) -- (176.6000,39.8000) -- (176.1000,40.8000) -- (175.7000,41.8000) -- (175.5000,42.8000) -- (175.3000,43.9000) -- (175.3000,44.9000) -- (175.3000,46.0000) -- (175.2000,47.1000) -- (175.2000,48.1000) -- (175.2000,49.2000) -- (175.2000,50.2000) -- (175.2000,51.3000) -- (175.2000,52.4000) -- (175.2000,53.4000) -- (175.2000,54.5000) -- (175.2000,55.5000) -- (175.2000,56.6000) -- (175.2000,57.6000) -- (175.2000,58.7000) -- (175.2000,59.7000) -- (175.2000,60.8000) -- (175.2000,61.9000) -- (175.2000,62.9000) -- (175.2000,64.0000) -- (175.2000,65.0000) -- (175.2000,66.1000) -- (175.2000,67.1000) -- (175.2000,68.2000) -- (175.2000,69.2000) -- (175.2000,70.3000) -- (175.2000,71.4000) -- (175.2000,72.4000) -- (175.2000,73.5000) -- (175.2000,74.5000) -- (175.2000,75.6000) -- (175.2000,76.6000) -- (175.3000,77.7000) -- (175.5000,78.7000) -- (175.7000,79.8000) -- (176.1000,80.8000) -- (176.5000,81.8000) -- (176.9000,82.8000) -- (177.4000,83.7000) -- (178.0000,84.6000) -- (178.7000,85.5000) -- (179.4000,86.4000) -- (180.2000,87.2000) -- (181.1000,87.9000) -- (182.0000,88.7000) -- (182.9000,89.3000) -- (183.9000,89.9000) -- (185.0000,90.5000) -- (186.0000,91.0000) -- (187.2000,91.4000) -- (188.3000,91.7000) -- (189.5000,92.0000) -- (190.7000,92.2000) -- (191.9000,92.4000) -- (193.1000,92.4000) -- (194.4000,92.4000) -- (195.6000,92.4000) -- (196.8000,92.2000) -- (198.0000,92.0000) -- (199.2000,91.7000) -- (200.4000,91.3000) -- (201.5000,90.9000) -- (202.6000,90.4000) -- (203.7000,89.8000) -- (204.7000,89.2000) -- (205.7000,88.5000) -- (206.6000,87.8000) -- (207.4000,87.0000) -- (208.2000,86.1000) -- (208.9000,85.3000) -- (209.6000,84.3000) -- (210.1000,83.4000) -- (210.6000,82.4000) -- (210.9000,81.4000) -- (211.2000,80.3000) -- (211.4000,79.3000) -- (211.5000,78.3000) -- (211.6000,77.3000) -- (211.7000,76.2000) -- (211.8000,75.2000) -- (211.8000,74.1000) -- (211.9000,73.1000) -- (211.9000,72.0000) -- (211.9000,71.0000) -- (211.9000,69.9000) -- (211.9000,68.9000) -- (211.9000,67.8000) -- (211.9000,66.8000) -- (211.9000,65.7000) -- (211.9000,64.7000) -- (211.9000,63.6000) -- (211.9000,62.5000) -- (211.9000,61.5000) -- (211.9000,60.4000) -- (211.9000,59.4000) -- (211.9000,58.3000) -- (211.9000,57.3000) -- (211.9000,56.2000) -- (211.9000,55.2000) -- (211.9000,54.1000) -- (211.9000,53.0000) -- (211.9000,52.0000) -- (211.9000,50.9000) -- (212.0000,49.9000) -- (212.1000,48.8000) -- (212.2000,47.8000) -- (212.4000,46.7000) -- (212.5000,45.7000) -- (212.4000,44.6000) -- (212.4000,43.6000) -- (212.2000,42.5000) -- (211.9000,41.5000) -- (211.6000,40.5000) -- (211.2000,39.5000) -- (210.7000,38.6000) -- (210.1000,37.6000) -- (209.4000,36.7000) -- (208.7000,35.9000) -- (207.8000,35.1000) -- (206.9000,34.4000) -- (206.0000,33.7000) -- (205.0000,33.1000) -- (203.9000,32.5000) -- (202.8000,32.1000) -- (201.6000,31.7000) -- (200.4000,31.4000) -- (199.2000,31.2000) -- (198.0000,31.1000) -- (196.8000,31.1000) -- (195.5000,31.1000) -- (194.3000,31.3000) -- (193.1000,31.5000) -- (191.9000,31.8000) -- (190.8000,32.2000) -- (189.7000,32.7000) -- (188.7000,33.2000) -- (187.7000,33.8000) -- (186.7000,34.5000) -- (185.8000,35.2000) -- (185.0000,36.0000) -- (184.3000,36.9000) -- (183.6000,37.7000) -- (183.0000,38.7000) -- (182.5000,39.6000) -- (182.1000,40.6000) -- (181.7000,41.6000) -- (181.6000,42.7000) -- (181.5000,43.7000) -- (181.4000,44.8000) -- (181.4000,45.9000) -- (181.4000,46.9000) -- (181.4000,48.0000) -- (181.4000,49.0000) -- (181.4000,50.1000) -- (181.4000,51.2000) -- (181.4000,52.2000) -- (181.4000,53.3000) -- (181.4000,54.3000) -- (181.4000,55.4000) -- (181.4000,56.4000) -- (181.4000,57.5000) -- (181.4000,58.5000) -- (181.4000,59.6000) -- (181.4000,60.7000) -- (181.4000,61.7000) -- (181.4000,62.8000) -- (181.4000,63.8000) -- (181.4000,64.9000) -- (181.4000,65.9000) -- (181.4000,67.0000) -- (181.4000,68.0000) -- (181.4000,69.1000) -- (181.4000,70.2000) -- (181.4000,71.2000) -- (181.4000,72.3000) -- (181.4000,73.3000) -- (181.4000,74.4000) -- (181.4000,75.4000) -- (181.4000,76.5000) -- (181.5000,77.5000) -- (181.7000,78.6000) -- (182.0000,79.6000) -- (182.3000,80.6000) -- (182.8000,81.6000) -- (183.3000,82.6000) -- (183.8000,83.5000) -- (184.4000,84.4000) -- (185.1000,85.3000) -- (185.9000,86.1000) -- (186.7000,86.9000) -- (187.6000,87.7000) -- (188.5000,88.4000) -- (189.4000,89.0000) -- (190.5000,89.6000) -- (191.5000,90.1000) -- (192.6000,90.6000) -- (193.8000,91.0000) -- (194.9000,91.3000) -- (196.1000,91.6000) -- (197.3000,91.7000) -- (198.5000,91.9000) -- (199.8000,91.9000) -- (201.0000,91.9000) -- (202.2000,91.8000) -- (203.5000,91.6000) -- (204.7000,91.3000) -- (205.8000,91.0000) -- (207.0000,90.6000) -- (208.1000,90.1000) -- (209.2000,89.6000) -- (210.2000,89.0000) -- (211.2000,88.4000) -- (212.2000,87.7000) -- (213.1000,86.9000) -- (213.9000,86.1000) -- (214.6000,85.2000) -- (215.3000,84.3000) -- (215.9000,83.4000) -- (216.5000,82.4000) -- (216.9000,81.4000) -- (217.2000,80.4000) -- (217.4000,79.4000) -- (217.6000,78.4000) -- (217.7000,77.3000) -- (217.8000,76.3000) -- (217.9000,75.3000) -- (217.9000,74.2000) -- (218.0000,73.2000) -- (218.0000,72.1000) -- (218.0000,71.1000) -- (218.1000,70.0000) -- (218.1000,69.0000) -- (218.1000,67.9000) -- (218.1000,66.9000) -- (218.1000,65.8000) -- (218.1000,64.7000) -- (218.1000,63.7000) -- (218.1000,62.6000) -- (218.1000,61.6000) -- (218.1000,60.5000) -- (218.1000,59.5000) -- (218.1000,58.4000) -- (218.1000,57.4000) -- (218.1000,56.3000) -- (218.1000,55.2000) -- (218.1000,54.2000) -- (218.1000,53.1000) -- (218.1000,52.1000) -- (218.1000,51.0000) -- (218.1000,50.0000) -- (218.1000,48.9000) -- (218.3000,47.9000) -- (218.4000,46.8000) -- (218.6000,45.8000) -- (218.6000,44.7000) -- (218.6000,43.7000) -- (218.4000,42.6000) -- (218.2000,41.6000) -- (217.9000,40.6000) -- (217.6000,39.6000) -- (217.1000,38.6000) -- (216.6000,37.6000) -- (216.0000,36.7000) -- (215.3000,35.9000) -- (214.5000,35.0000) -- (213.6000,34.3000) -- (212.7000,33.5000) -- (211.7000,32.9000) -- (210.7000,32.3000) -- (209.6000,31.8000) -- (208.5000,31.4000) -- (207.3000,31.0000) -- (206.1000,30.8000) -- (204.9000,30.6000) -- (203.6000,30.5000) -- (202.4000,30.5000) -- (201.2000,30.6000) -- (200.0000,30.8000) -- (198.8000,31.1000) -- (197.6000,31.4000) -- (196.5000,31.8000) -- (195.4000,32.3000) -- (194.4000,32.9000) -- (193.4000,33.5000) -- (192.5000,34.2000) -- (191.6000,35.0000) -- (190.9000,35.8000) -- (190.1000,36.6000) -- (189.5000,37.5000) -- (188.9000,38.5000) -- (188.5000,39.5000) -- (188.1000,40.5000) -- (187.8000,41.5000) -- (187.7000,42.6000) -- (187.6000,43.6000) -- (187.5000,44.7000) -- (187.5000,45.7000) -- (187.5000,46.8000) -- (187.5000,47.9000) -- (187.5000,48.9000) -- (187.5000,50.0000) -- (187.5000,51.0000) -- (187.5000,52.1000) -- (187.5000,53.1000) -- (187.5000,54.2000) -- (187.5000,55.2000) -- (187.5000,56.3000) -- (187.5000,57.4000) -- (187.5000,58.4000) -- (187.5000,59.5000) -- (187.5000,60.5000) -- (187.5000,61.6000) -- (187.5000,62.6000) -- (187.5000,63.7000) -- (187.5000,64.7000) -- (187.5000,65.8000) -- (187.5000,66.9000) -- (187.5000,67.9000) -- (187.5000,69.0000) -- (187.5000,70.0000) -- (187.5000,71.1000) -- (187.5000,72.1000) -- (187.5000,73.2000) -- (187.5000,74.2000) -- (187.5000,75.3000) -- (187.6000,76.4000) -- (187.7000,77.4000) -- (188.0000,78.4000) -- (188.3000,79.5000) -- (188.7000,80.5000) -- (189.1000,81.5000) -- (189.6000,82.4000) -- (190.2000,83.3000) -- (190.8000,84.2000) -- (191.6000,85.1000) -- (192.3000,85.9000) -- (193.2000,86.7000) -- (194.1000,87.4000) -- (195.0000,88.1000) -- (196.0000,88.7000) -- (197.0000,89.2000) -- (198.1000,89.7000) -- (199.2000,90.2000) -- (200.4000,90.6000) -- (201.5000,90.9000) -- (202.7000,91.1000) -- (204.0000,91.2000) -- (205.2000,91.3000) -- (206.4000,91.3000) -- (207.6000,91.3000) -- (208.9000,91.1000) -- (210.1000,90.9000) -- (211.3000,90.6000) -- (212.5000,90.3000) -- (213.6000,89.9000) -- (214.7000,89.4000) -- (215.8000,88.8000) -- (216.8000,88.2000) -- (217.8000,87.5000) -- (218.7000,86.8000) -- (219.5000,86.0000) -- (220.3000,85.2000) -- (221.1000,84.3000) -- (221.7000,83.4000) -- (222.3000,82.5000) -- (222.8000,81.5000) -- (223.1000,80.5000) -- (223.4000,79.4000) -- (223.6000,78.4000) -- (223.8000,77.4000) -- (223.9000,76.4000) -- (224.0000,75.3000) -- (224.1000,74.3000) -- (224.1000,73.2000) -- (224.1000,72.2000) -- (224.2000,71.1000) -- (224.2000,70.1000) -- (224.2000,69.0000) -- (224.2000,68.0000) -- (224.2000,66.9000) -- (224.2000,65.9000) -- (224.2000,64.8000) -- (224.2000,63.8000) -- (224.2000,62.7000) -- (224.2000,61.7000) -- (224.2000,60.6000) -- (224.2000,59.5000) -- (224.2000,58.5000) -- (224.2000,57.4000) -- (224.2000,56.4000) -- (224.2000,55.3000) -- (224.2000,54.3000) -- (224.2000,53.2000) -- (224.2000,52.2000) -- (224.2000,51.1000) -- (224.2000,50.0000) -- (224.2000,49.0000) -- (224.3000,47.9000) -- (224.5000,46.9000) -- (224.6000,45.8000) -- (224.7000,44.8000) -- (224.7000,43.7000) -- (224.7000,42.7000) -- (224.5000,41.6000) -- (224.3000,40.6000) -- (223.9000,39.6000) -- (223.5000,38.6000) -- (223.1000,37.6000) -- (222.5000,36.7000) -- (221.8000,35.8000) -- (221.1000,35.0000) -- (220.3000,34.2000) -- (219.4000,33.4000) -- (218.5000,32.7000) -- (217.5000,32.1000) -- (216.4000,31.5000) -- (215.3000,31.1000) -- (214.1000,30.7000) -- (213.0000,30.4000) -- (211.7000,30.1000) -- (210.5000,30.0000) -- (209.3000,29.9000) -- (208.1000,30.0000) -- (206.8000,30.1000) -- (205.6000,30.3000) -- (204.5000,30.6000) -- (203.3000,31.0000) -- (202.2000,31.5000) -- (201.2000,32.0000) -- (200.1000,32.6000) -- (199.2000,33.2000) -- (198.3000,34.0000) -- (197.5000,34.7000) -- (196.7000,35.6000) -- (196.0000,36.4000) -- (195.4000,37.4000) -- (194.9000,38.3000) -- (194.4000,39.3000) -- (194.1000,40.3000) -- (193.9000,41.4000) -- (193.8000,42.4000) -- (193.7000,43.5000) -- (193.7000,44.5000) -- (193.7000,45.6000) -- (193.7000,46.7000) -- (193.6000,47.7000) -- (193.6000,48.8000) -- (193.6000,49.8000) -- (193.6000,50.9000) -- (193.6000,51.9000) -- (193.6000,53.0000) -- (193.6000,54.1000) -- (193.6000,55.1000) -- (193.6000,56.2000) -- (193.6000,57.2000) -- (193.6000,58.3000) -- (193.6000,59.3000) -- (193.6000,60.4000) -- (193.6000,61.4000) -- (193.6000,62.5000) -- (193.6000,63.6000) -- (193.6000,64.6000) -- (193.6000,65.7000) -- (193.6000,66.7000) -- (193.6000,67.8000) -- (193.6000,68.8000) -- (193.6000,69.9000) -- (193.6000,70.9000) -- (193.6000,72.0000) -- (193.6000,73.1000) -- (193.6000,74.1000) -- (193.7000,75.2000) -- (193.8000,76.2000) -- (194.0000,77.3000) -- (194.2000,78.3000) -- (194.6000,79.3000) -- (195.0000,80.3000) -- (195.4000,81.3000) -- (196.0000,82.2000) -- (196.6000,83.1000) -- (197.3000,84.0000) -- (198.0000,84.9000) -- (198.8000,85.7000) -- (199.7000,86.4000) -- (200.6000,87.1000) -- (201.5000,87.8000) -- (202.5000,88.4000) -- (203.6000,88.9000) -- (204.7000,89.4000) -- (205.8000,89.8000) -- (207.0000,90.1000) -- (208.2000,90.4000) -- (209.4000,90.6000) -- (210.6000,90.7000) -- (211.8000,90.8000) -- (213.0000,90.8000) -- (214.3000,90.7000) -- (215.5000,90.5000) -- (216.7000,90.3000) -- (217.9000,90.0000) -- (219.0000,89.6000) -- (220.2000,89.1000) -- (221.3000,88.6000) -- (222.3000,88.0000) -- (223.3000,87.4000) -- (224.3000,86.7000) -- (225.2000,85.9000) -- (226.0000,85.2000) -- (226.8000,84.3000) -- (227.5000,83.4000) -- (228.1000,82.5000) -- (228.6000,81.5000) -- (229.1000,80.5000) -- (229.4000,79.5000) -- (229.7000,78.5000) -- (229.9000,77.5000) -- (230.0000,76.4000) -- (230.1000,75.4000) -- (230.2000,74.4000) -- (230.2000,73.3000) -- (230.3000,72.3000) -- (230.3000,71.2000) -- (230.3000,70.2000) -- (230.3000,69.1000) -- (230.3000,68.1000) -- (230.4000,67.0000) -- (230.4000,66.0000) -- (230.4000,64.9000) -- (230.4000,63.9000) -- (230.4000,62.8000) -- (230.4000,61.7000) -- (230.4000,60.7000) -- (230.4000,59.6000) -- (230.4000,58.6000) -- (230.4000,57.5000) -- (230.4000,56.5000) -- (230.4000,55.4000) -- (230.4000,54.4000) -- (230.4000,53.3000) -- (230.4000,52.2000) -- (230.4000,51.2000) -- (230.4000,50.1000) -- (230.4000,49.1000) -- (230.4000,48.0000) -- (230.5000,47.0000) -- (230.7000,45.9000) -- (230.8000,44.8000);



  \end{scope}
  \begin{scope}[scale=1.006,draw=black,line join=bevel,line cap=rect,line width=0.800pt]
  \end{scope}
  \begin{scope}[scale=1.006,draw=black,line join=bevel,line cap=rect,line width=0.800pt]
  \end{scope}
  \begin{scope}[scale=1.006,draw=black,line join=round,line cap=round,line width=0.480pt]
    \path[draw] (148.5000,13.5000) -- (148.5000,102.5000) -- (246.5000,102.5000) -- (246.5000,13.5000) -- (148.5000,13.5000);



  \end{scope}
  \begin{scope}[scale=1.006,draw=ca0a0a4,dash pattern=on 0.40pt off 0.80pt,line join=round,line cap=round,line width=0.400pt]
    \path[draw] (44.5000,210.5000) -- (179.5000,210.5000);



  \end{scope}
  \begin{scope}[scale=1.006,draw=black,line join=round,line cap=round,line width=0.480pt]
    \path[draw] (44.5000,210.5000) -- (48.5000,210.5000);



    \path[draw] (179.5000,210.5000) -- (175.5000,210.5000);



  \end{scope}
  \begin{scope}[scale=1.006,draw=black,line join=bevel,line cap=rect,line width=0.800pt]
  \end{scope}
  \begin{scope}[cm={{1.00588,0.0,0.0,1.00588,(18.1059,216.265)}},draw=black,line join=bevel,line cap=rect,line width=0.800pt]
  \end{scope}
  \begin{scope}[cm={{1.00588,0.0,0.0,1.00588,(18.1059,216.265)}},draw=black,line join=bevel,line cap=rect,line width=0.800pt]
  \end{scope}
  \begin{scope}[cm={{1.00588,0.0,0.0,1.00588,(18.1059,216.265)}},draw=black,line join=bevel,line cap=rect,line width=0.800pt]
  \end{scope}
  \begin{scope}[cm={{1.00588,0.0,0.0,1.00588,(18.1059,216.265)}},draw=black,line join=bevel,line cap=rect,line width=0.800pt]
  \end{scope}
  \begin{scope}[cm={{1.00588,0.0,0.0,1.00588,(18.1059,216.265)}},draw=black,line join=bevel,line cap=rect,line width=0.800pt]
  \end{scope}
  \begin{scope}[cm={{1.00588,0.0,0.0,1.00588,(18.1059,216.265)}},draw=black,line join=bevel,line cap=rect,line width=0.800pt]
    \path[fill=black] (0.0000,0.0000) node[above right] (text738) {-100};



  \end{scope}
  \begin{scope}[cm={{1.00588,0.0,0.0,1.00588,(18.1059,216.265)}},draw=black,line join=bevel,line cap=rect,line width=0.800pt]
  \end{scope}
  \begin{scope}[scale=1.006,draw=black,line join=bevel,line cap=rect,line width=0.800pt]
  \end{scope}
  \begin{scope}[scale=1.006,draw=ca0a0a4,dash pattern=on 0.40pt off 0.80pt,line join=round,line cap=round,line width=0.400pt]
    \path[draw] (44.5000,181.5000) -- (179.5000,181.5000);



  \end{scope}
  \begin{scope}[scale=1.006,draw=black,line join=round,line cap=round,line width=0.480pt]
    \path[draw] (44.5000,181.5000) -- (48.5000,181.5000);



    \path[draw] (179.5000,181.5000) -- (175.5000,181.5000);



  \end{scope}
  \begin{scope}[scale=1.006,draw=black,line join=bevel,line cap=rect,line width=0.800pt]
  \end{scope}
  \begin{scope}[cm={{1.00588,0.0,0.0,1.00588,(31.1824,187.094)}},draw=black,line join=bevel,line cap=rect,line width=0.800pt]
  \end{scope}
  \begin{scope}[cm={{1.00588,0.0,0.0,1.00588,(31.1824,187.094)}},draw=black,line join=bevel,line cap=rect,line width=0.800pt]
  \end{scope}
  \begin{scope}[cm={{1.00588,0.0,0.0,1.00588,(31.1824,187.094)}},draw=black,line join=bevel,line cap=rect,line width=0.800pt]
  \end{scope}
  \begin{scope}[cm={{1.00588,0.0,0.0,1.00588,(31.1824,187.094)}},draw=black,line join=bevel,line cap=rect,line width=0.800pt]
  \end{scope}
  \begin{scope}[cm={{1.00588,0.0,0.0,1.00588,(31.1824,187.094)}},draw=black,line join=bevel,line cap=rect,line width=0.800pt]
  \end{scope}
  \begin{scope}[cm={{1.00588,0.0,0.0,1.00588,(31.1824,187.094)}},draw=black,line join=bevel,line cap=rect,line width=0.800pt]
    \path[fill=black] (0.0000,0.0000) node[above right] (text768) {0};



  \end{scope}
  \begin{scope}[cm={{1.00588,0.0,0.0,1.00588,(31.1824,187.094)}},draw=black,line join=bevel,line cap=rect,line width=0.800pt]
  \end{scope}
  \begin{scope}[scale=1.006,draw=black,line join=bevel,line cap=rect,line width=0.800pt]
  \end{scope}
  \begin{scope}[scale=1.006,draw=ca0a0a4,dash pattern=on 0.40pt off 0.80pt,line join=round,line cap=round,line width=0.400pt]
    \path[draw] (44.5000,152.5000) -- (179.5000,152.5000);



  \end{scope}
  \begin{scope}[scale=1.006,draw=black,line join=round,line cap=round,line width=0.480pt]
    \path[draw] (44.5000,152.5000) -- (48.5000,152.5000);



    \path[draw] (179.5000,152.5000) -- (175.5000,152.5000);



  \end{scope}
  \begin{scope}[scale=1.006,draw=black,line join=bevel,line cap=rect,line width=0.800pt]
  \end{scope}
  \begin{scope}[cm={{1.00588,0.0,0.0,1.00588,(19.1118,157.924)}},draw=black,line join=bevel,line cap=rect,line width=0.800pt]
  \end{scope}
  \begin{scope}[cm={{1.00588,0.0,0.0,1.00588,(19.1118,157.924)}},draw=black,line join=bevel,line cap=rect,line width=0.800pt]
  \end{scope}
  \begin{scope}[cm={{1.00588,0.0,0.0,1.00588,(19.1118,157.924)}},draw=black,line join=bevel,line cap=rect,line width=0.800pt]
  \end{scope}
  \begin{scope}[cm={{1.00588,0.0,0.0,1.00588,(19.1118,157.924)}},draw=black,line join=bevel,line cap=rect,line width=0.800pt]
  \end{scope}
  \begin{scope}[cm={{1.00588,0.0,0.0,1.00588,(19.1118,157.924)}},draw=black,line join=bevel,line cap=rect,line width=0.800pt]
  \end{scope}
  \begin{scope}[cm={{1.00588,0.0,0.0,1.00588,(19.1118,157.924)}},draw=black,line join=bevel,line cap=rect,line width=0.800pt]
    \path[fill=black] (0.0000,0.0000) node[above right] (text798) {100};



  \end{scope}
  \begin{scope}[cm={{1.00588,0.0,0.0,1.00588,(19.1118,157.924)}},draw=black,line join=bevel,line cap=rect,line width=0.800pt]
  \end{scope}
  \begin{scope}[scale=1.006,draw=black,line join=bevel,line cap=rect,line width=0.800pt]
  \end{scope}
  \begin{scope}[scale=1.006,draw=ca0a0a4,dash pattern=on 0.40pt off 0.80pt,line join=round,line cap=round,line width=0.400pt]
    \path[draw] (44.5000,123.5000) -- (179.5000,123.5000);



  \end{scope}
  \begin{scope}[scale=1.006,draw=black,line join=round,line cap=round,line width=0.480pt]
    \path[draw] (44.5000,123.5000) -- (48.5000,123.5000);



    \path[draw] (179.5000,123.5000) -- (175.5000,123.5000);



  \end{scope}
  \begin{scope}[scale=1.006,draw=black,line join=bevel,line cap=rect,line width=0.800pt]
  \end{scope}
  \begin{scope}[cm={{1.00588,0.0,0.0,1.00588,(19.1118,128.753)}},draw=black,line join=bevel,line cap=rect,line width=0.800pt]
  \end{scope}
  \begin{scope}[cm={{1.00588,0.0,0.0,1.00588,(19.1118,128.753)}},draw=black,line join=bevel,line cap=rect,line width=0.800pt]
  \end{scope}
  \begin{scope}[cm={{1.00588,0.0,0.0,1.00588,(19.1118,128.753)}},draw=black,line join=bevel,line cap=rect,line width=0.800pt]
  \end{scope}
  \begin{scope}[cm={{1.00588,0.0,0.0,1.00588,(19.1118,128.753)}},draw=black,line join=bevel,line cap=rect,line width=0.800pt]
  \end{scope}
  \begin{scope}[cm={{1.00588,0.0,0.0,1.00588,(19.1118,128.753)}},draw=black,line join=bevel,line cap=rect,line width=0.800pt]
  \end{scope}
  \begin{scope}[cm={{1.00588,0.0,0.0,1.00588,(19.1118,128.753)}},draw=black,line join=bevel,line cap=rect,line width=0.800pt]
    \path[fill=black] (0.0000,0.0000) node[above right] (text828) {200};



  \end{scope}
  \begin{scope}[cm={{1.00588,0.0,0.0,1.00588,(19.1118,128.753)}},draw=black,line join=bevel,line cap=rect,line width=0.800pt]
  \end{scope}
  \begin{scope}[scale=1.006,draw=black,line join=bevel,line cap=rect,line width=0.800pt]
  \end{scope}
  \begin{scope}[scale=1.006,draw=ca0a0a4,dash pattern=on 0.40pt off 0.80pt,line join=round,line cap=round,line width=0.400pt]
    \path[draw] (44.5000,210.5000) -- (44.5000,108.5000);



  \end{scope}
  \begin{scope}[scale=1.006,draw=black,line join=round,line cap=round,line width=0.480pt]
    \path[draw] (44.5000,210.5000) -- (44.5000,208.5000);



    \path[draw] (44.5000,108.5000) -- (44.5000,111.5000);



  \end{scope}
  \begin{scope}[scale=1.006,draw=black,line join=bevel,line cap=rect,line width=0.800pt]
  \end{scope}
  \begin{scope}[cm={{1.00588,0.0,0.0,1.00588,(45.2647,228.335)}},draw=black,line join=bevel,line cap=rect,line width=0.800pt]
  \end{scope}
  \begin{scope}[cm={{1.00588,0.0,0.0,1.00588,(45.2647,228.335)}},draw=black,line join=bevel,line cap=rect,line width=0.800pt]
  \end{scope}
  \begin{scope}[cm={{1.00588,0.0,0.0,1.00588,(45.2647,228.335)}},draw=black,line join=bevel,line cap=rect,line width=0.800pt]
  \end{scope}
  \begin{scope}[cm={{1.00588,0.0,0.0,1.00588,(45.2647,228.335)}},draw=black,line join=bevel,line cap=rect,line width=0.800pt]
  \end{scope}
  \begin{scope}[cm={{1.00588,0.0,0.0,1.00588,(45.2647,228.335)}},draw=black,line join=bevel,line cap=rect,line width=0.800pt]
  \end{scope}
  \begin{scope}[cm={{1.00588,0.0,0.0,1.00588,(45.2647,228.335)}},draw=black,line join=bevel,line cap=rect,line width=0.800pt]
  \end{scope}
  \begin{scope}[scale=1.006,draw=black,line join=bevel,line cap=rect,line width=0.800pt]
  \end{scope}
  \begin{scope}[scale=1.006,draw=ca0a0a4,dash pattern=on 0.40pt off 0.80pt,line join=round,line cap=round,line width=0.400pt]
    \path[draw] (78.5000,210.5000) -- (78.5000,108.5000);



  \end{scope}
  \begin{scope}[scale=1.006,draw=black,line join=round,line cap=round,line width=0.480pt]
    \path[draw] (78.5000,210.5000) -- (78.5000,208.5000);



    \path[draw] (78.5000,108.5000) -- (78.5000,111.5000);



  \end{scope}
  \begin{scope}[scale=1.006,draw=black,line join=bevel,line cap=rect,line width=0.800pt]
  \end{scope}
  \begin{scope}[cm={{1.00588,0.0,0.0,1.00588,(78.4588,228.335)}},draw=black,line join=bevel,line cap=rect,line width=0.800pt]
  \end{scope}
  \begin{scope}[cm={{1.00588,0.0,0.0,1.00588,(78.4588,228.335)}},draw=black,line join=bevel,line cap=rect,line width=0.800pt]
  \end{scope}
  \begin{scope}[cm={{1.00588,0.0,0.0,1.00588,(78.4588,228.335)}},draw=black,line join=bevel,line cap=rect,line width=0.800pt]
  \end{scope}
  \begin{scope}[cm={{1.00588,0.0,0.0,1.00588,(78.4588,228.335)}},draw=black,line join=bevel,line cap=rect,line width=0.800pt]
  \end{scope}
  \begin{scope}[cm={{1.00588,0.0,0.0,1.00588,(78.4588,228.335)}},draw=black,line join=bevel,line cap=rect,line width=0.800pt]
  \end{scope}
  \begin{scope}[cm={{1.00588,0.0,0.0,1.00588,(78.4588,228.335)}},draw=black,line join=bevel,line cap=rect,line width=0.800pt]
  \end{scope}
  \begin{scope}[scale=1.006,draw=black,line join=bevel,line cap=rect,line width=0.800pt]
  \end{scope}
  \begin{scope}[scale=1.006,draw=ca0a0a4,dash pattern=on 0.40pt off 0.80pt,line join=round,line cap=round,line width=0.400pt]
    \path[draw] (112.5000,210.5000) -- (112.5000,108.5000);



  \end{scope}
  \begin{scope}[scale=1.006,draw=black,line join=round,line cap=round,line width=0.480pt]
    \path[draw] (112.5000,210.5000) -- (112.5000,208.5000);



    \path[draw] (112.5000,108.5000) -- (112.5000,111.5000);



  \end{scope}
  \begin{scope}[scale=1.006,draw=black,line join=bevel,line cap=rect,line width=0.800pt]
  \end{scope}
  \begin{scope}[cm={{1.00588,0.0,0.0,1.00588,(112.659,228.335)}},draw=black,line join=bevel,line cap=rect,line width=0.800pt]
  \end{scope}
  \begin{scope}[cm={{1.00588,0.0,0.0,1.00588,(112.659,228.335)}},draw=black,line join=bevel,line cap=rect,line width=0.800pt]
  \end{scope}
  \begin{scope}[cm={{1.00588,0.0,0.0,1.00588,(112.659,228.335)}},draw=black,line join=bevel,line cap=rect,line width=0.800pt]
  \end{scope}
  \begin{scope}[cm={{1.00588,0.0,0.0,1.00588,(112.659,228.335)}},draw=black,line join=bevel,line cap=rect,line width=0.800pt]
  \end{scope}
  \begin{scope}[cm={{1.00588,0.0,0.0,1.00588,(112.659,228.335)}},draw=black,line join=bevel,line cap=rect,line width=0.800pt]
  \end{scope}
  \begin{scope}[cm={{1.00588,0.0,0.0,1.00588,(112.659,228.335)}},draw=black,line join=bevel,line cap=rect,line width=0.800pt]
  \end{scope}
  \begin{scope}[scale=1.006,draw=black,line join=bevel,line cap=rect,line width=0.800pt]
  \end{scope}
  \begin{scope}[scale=1.006,draw=ca0a0a4,dash pattern=on 0.40pt off 0.80pt,line join=round,line cap=round,line width=0.400pt]
    \path[draw] (145.5000,210.5000) -- (145.5000,108.5000);



  \end{scope}
  \begin{scope}[scale=1.006,draw=black,line join=round,line cap=round,line width=0.480pt]
    \path[draw] (145.5000,210.5000) -- (145.5000,208.5000);



    \path[draw] (145.5000,108.5000) -- (145.5000,111.5000);



  \end{scope}
  \begin{scope}[scale=1.006,draw=black,line join=bevel,line cap=rect,line width=0.800pt]
  \end{scope}
  \begin{scope}[cm={{1.00588,0.0,0.0,1.00588,(146.859,228.335)}},draw=black,line join=bevel,line cap=rect,line width=0.800pt]
  \end{scope}
  \begin{scope}[cm={{1.00588,0.0,0.0,1.00588,(146.859,228.335)}},draw=black,line join=bevel,line cap=rect,line width=0.800pt]
  \end{scope}
  \begin{scope}[cm={{1.00588,0.0,0.0,1.00588,(146.859,228.335)}},draw=black,line join=bevel,line cap=rect,line width=0.800pt]
  \end{scope}
  \begin{scope}[cm={{1.00588,0.0,0.0,1.00588,(146.859,228.335)}},draw=black,line join=bevel,line cap=rect,line width=0.800pt]
  \end{scope}
  \begin{scope}[cm={{1.00588,0.0,0.0,1.00588,(146.859,228.335)}},draw=black,line join=bevel,line cap=rect,line width=0.800pt]
  \end{scope}
  \begin{scope}[cm={{1.00588,0.0,0.0,1.00588,(146.859,228.335)}},draw=black,line join=bevel,line cap=rect,line width=0.800pt]
  \end{scope}
  \begin{scope}[scale=1.006,draw=black,line join=bevel,line cap=rect,line width=0.800pt]
  \end{scope}
  \begin{scope}[scale=1.006,draw=ca0a0a4,dash pattern=on 0.40pt off 0.80pt,line join=round,line cap=round,line width=0.400pt]
    \path[draw] (179.5000,210.5000) -- (179.5000,108.5000);



  \end{scope}
  \begin{scope}[scale=1.006,draw=black,line join=round,line cap=round,line width=0.480pt]
    \path[draw] (179.5000,210.5000) -- (179.5000,208.5000);



    \path[draw] (179.5000,108.5000) -- (179.5000,111.5000);



  \end{scope}
  \begin{scope}[scale=1.006,draw=black,line join=bevel,line cap=rect,line width=0.800pt]
  \end{scope}
  \begin{scope}[cm={{1.00588,0.0,0.0,1.00588,(180.053,228.335)}},draw=black,line join=bevel,line cap=rect,line width=0.800pt]
  \end{scope}
  \begin{scope}[cm={{1.00588,0.0,0.0,1.00588,(180.053,228.335)}},draw=black,line join=bevel,line cap=rect,line width=0.800pt]
  \end{scope}
  \begin{scope}[cm={{1.00588,0.0,0.0,1.00588,(180.053,228.335)}},draw=black,line join=bevel,line cap=rect,line width=0.800pt]
  \end{scope}
  \begin{scope}[cm={{1.00588,0.0,0.0,1.00588,(180.053,228.335)}},draw=black,line join=bevel,line cap=rect,line width=0.800pt]
  \end{scope}
  \begin{scope}[cm={{1.00588,0.0,0.0,1.00588,(180.053,228.335)}},draw=black,line join=bevel,line cap=rect,line width=0.800pt]
  \end{scope}
  \begin{scope}[cm={{1.00588,0.0,0.0,1.00588,(180.053,228.335)}},draw=black,line join=bevel,line cap=rect,line width=0.800pt]
  \end{scope}
  \begin{scope}[scale=1.006,draw=black,line join=bevel,line cap=rect,line width=0.800pt]
  \end{scope}
  \begin{scope}[scale=1.006,draw=black,line join=round,line cap=round,line width=0.480pt]
    \path[draw] (44.5000,108.5000) -- (44.5000,210.5000) -- (179.5000,210.5000) -- (179.5000,108.5000) -- (44.5000,108.5000);



  \end{scope}
  \begin{scope}[scale=1.006,draw=black,line join=bevel,line cap=rect,line width=0.800pt]
  \end{scope}
  \begin{scope}[scale=1.006,draw=black,line join=bevel,line cap=rect,line width=0.800pt]
  \end{scope}
  \begin{scope}[scale=1.006,fill=cffffff]
    \path[fill,rounded corners=0.0000cm] (158.0000,115.0000) rectangle (169.0000,131.0000);



  \end{scope}
  \begin{scope}[scale=1.006,draw=black,line join=bevel,line cap=rect,line width=0.800pt]
  \end{scope}
  \begin{scope}[scale=1.006,draw=black,line join=bevel,line cap=rect,line width=0.800pt]
  \end{scope}
  \begin{scope}[scale=1.006,draw=black,line join=round,line cap=round,line width=0.800pt]
    \path[draw] (157.5000,131.5000) -- (157.5000,115.5000) -- (168.5000,115.5000) -- (168.5000,131.5000) -- (157.5000,131.5000);



  \end{scope}
  \begin{scope}[scale=1.006,draw=black,line join=bevel,line cap=rect,line width=0.800pt]
  \end{scope}
  \begin{scope}[cm={{1.00588,0.0,0.0,1.00588,(161.947,127.747)}},draw=black,line join=bevel,line cap=rect,line width=0.800pt]
  \end{scope}
  \begin{scope}[cm={{1.00588,0.0,0.0,1.00588,(161.947,127.747)}},draw=black,line join=bevel,line cap=rect,line width=0.800pt]
  \end{scope}
  \begin{scope}[cm={{1.00588,0.0,0.0,1.00588,(161.947,127.747)}},draw=black,line join=bevel,line cap=rect,line width=0.800pt]
  \end{scope}
  \begin{scope}[cm={{1.00588,0.0,0.0,1.00588,(161.947,127.747)}},draw=black,line join=bevel,line cap=rect,line width=0.800pt]
  \end{scope}
  \begin{scope}[cm={{1.00588,0.0,0.0,1.00588,(161.947,127.747)}},draw=black,line join=bevel,line cap=rect,line width=0.800pt]
  \end{scope}
  \begin{scope}[cm={{1.00588,0.0,0.0,1.00588,(161.947,127.747)}},draw=black,line join=bevel,line cap=rect,line width=0.800pt]
    \path[fill=black] (0.6481,-0.2160) node[above right] (text998) {\label{fig:trajs-dyn-i}i};



  \end{scope}
  \begin{scope}[cm={{1.00588,0.0,0.0,1.00588,(161.947,127.747)}},draw=black,line join=bevel,line cap=rect,line width=0.800pt]
  \end{scope}
  \begin{scope}[scale=1.006,draw=black,line join=bevel,line cap=rect,line width=0.800pt]
  \end{scope}
  \begin{scope}[scale=1.006,draw=black,line join=bevel,line cap=rect,line width=0.800pt]
  \end{scope}
  \begin{scope}[scale=1.006,draw=black,line join=bevel,line cap=rect,line width=0.800pt]
  \end{scope}
  \begin{scope}[scale=1.006,draw=black,line join=round,line cap=round,line width=0.480pt]
    \path[draw] (61.9000,118.7000) -- (61.9000,118.7000) -- (62.6000,120.9000) -- (63.3000,122.6000) -- (63.5000,124.1000) -- (62.7000,125.3000) -- (61.5000,126.5000) -- (60.3000,127.9000) -- (59.2000,129.5000) -- (58.3000,131.2000) -- (57.5000,133.0000) -- (57.0000,134.8000) -- (56.8000,136.7000) -- (56.7000,138.6000) -- (56.6000,140.4000) -- (56.6000,142.2000) -- (56.6000,144.1000) -- (56.6000,145.9000) -- (56.6000,147.7000) -- (56.6000,149.5000) -- (56.6000,151.4000) -- (56.6000,153.2000) -- (56.6000,155.1000) -- (56.6000,156.9000) -- (56.6000,158.8000) -- (56.6000,160.6000) -- (56.6000,162.5000) -- (56.6000,164.3000) -- (56.6000,166.2000) -- (56.6000,168.0000) -- (56.6000,169.9000) -- (56.6000,171.7000) -- (56.6000,173.6000) -- (56.6000,175.4000) -- (56.6000,177.2000) -- (56.6000,179.1000) -- (56.7000,181.0000) -- (57.1000,182.8000) -- (57.6000,184.7000) -- (58.3000,186.4000) -- (59.2000,188.2000) -- (60.2000,189.8000) -- (61.4000,191.3000) -- (62.7000,192.8000) -- (64.2000,194.1000) -- (65.7000,195.3000) -- (67.4000,196.4000) -- (69.1000,197.4000) -- (71.0000,198.2000) -- (72.9000,198.9000) -- (74.8000,199.5000) -- (76.8000,199.9000) -- (78.8000,200.2000) -- (80.8000,200.4000) -- (82.8000,200.4000) -- (84.8000,200.2000) -- (86.7000,200.0000) -- (88.7000,199.6000) -- (90.6000,199.0000) -- (92.5000,198.4000) -- (94.3000,197.5000) -- (96.1000,196.6000) -- (97.8000,195.5000) -- (99.3000,194.3000) -- (100.8000,193.0000) -- (102.1000,191.6000) -- (103.3000,190.0000) -- (104.2000,188.4000) -- (104.9000,186.7000) -- (105.3000,184.9000) -- (105.6000,183.0000) -- (105.7000,181.1000) -- (105.8000,179.2000) -- (105.8000,177.4000) -- (105.8000,175.5000) -- (105.8000,173.6000) -- (105.8000,171.7000) -- (105.8000,169.9000) -- (105.9000,168.0000) -- (105.9000,166.2000) -- (105.9000,164.3000) -- (105.9000,162.5000) -- (105.9000,160.6000) -- (105.9000,158.8000) -- (105.9000,157.0000) -- (105.9000,155.1000) -- (105.9000,153.3000) -- (105.9000,151.4000) -- (105.9000,149.6000) -- (105.9000,147.7000) -- (105.8000,145.9000) -- (106.1000,144.0000) -- (106.5000,142.1000) -- (106.8000,140.3000) -- (106.8000,138.6000) -- (106.7000,136.8000) -- (106.4000,135.0000) -- (105.9000,133.2000) -- (105.2000,131.5000) -- (104.3000,129.8000) -- (103.3000,128.3000) -- (102.1000,126.8000) -- (100.8000,125.4000) -- (99.3000,124.1000) -- (97.7000,122.9000) -- (96.0000,121.9000) -- (94.3000,121.0000) -- (92.4000,120.3000) -- (90.5000,119.7000) -- (88.5000,119.2000) -- (86.5000,118.9000) -- (84.5000,118.8000) -- (82.4000,118.8000) -- (80.4000,119.0000) -- (78.4000,119.3000) -- (76.4000,119.8000) -- (74.5000,120.4000) -- (72.6000,121.2000) -- (70.9000,122.1000) -- (69.2000,123.1000) -- (67.6000,124.3000) -- (66.2000,125.6000) -- (64.9000,127.0000) -- (63.8000,128.5000) -- (62.8000,130.1000) -- (62.0000,131.8000) -- (61.3000,133.6000) -- (60.9000,135.4000) -- (60.7000,137.2000) -- (60.6000,139.1000) -- (60.6000,140.9000) -- (60.6000,142.7000) -- (60.6000,144.5000) -- (60.6000,146.4000) -- (60.6000,148.2000) -- (60.6000,150.0000) -- (60.6000,151.9000) -- (60.6000,153.7000) -- (60.5000,155.6000) -- (60.5000,157.4000) -- (60.5000,159.3000) -- (60.5000,161.1000) -- (60.5000,163.0000) -- (60.5000,164.8000) -- (60.5000,166.7000) -- (60.5000,168.5000) -- (60.6000,170.4000) -- (60.6000,172.2000) -- (60.6000,174.1000) -- (60.6000,175.9000) -- (60.5000,177.7000) -- (60.6000,179.6000) -- (60.8000,181.5000) -- (61.2000,183.3000) -- (61.9000,185.1000) -- (62.7000,186.9000) -- (63.6000,188.6000) -- (64.7000,190.1000) -- (66.0000,191.6000) -- (67.4000,193.0000) -- (68.9000,194.3000) -- (70.6000,195.4000) -- (72.3000,196.4000) -- (74.1000,197.3000) -- (76.0000,198.1000) -- (77.9000,198.7000) -- (79.8000,199.2000) -- (81.8000,199.5000) -- (83.8000,199.7000) -- (85.8000,199.7000) -- (87.8000,199.6000) -- (89.8000,199.4000) -- (91.8000,199.0000) -- (93.7000,198.5000) -- (95.6000,197.8000) -- (97.4000,197.0000) -- (99.2000,196.1000) -- (100.8000,195.0000) -- (102.4000,193.8000) -- (103.9000,192.5000) -- (105.2000,191.1000) -- (106.4000,189.5000) -- (107.3000,187.9000) -- (108.0000,186.2000) -- (108.5000,184.4000) -- (108.7000,182.5000) -- (108.9000,180.6000) -- (109.0000,178.7000) -- (109.0000,176.9000) -- (109.0000,175.0000) -- (109.0000,173.1000) -- (109.0000,171.2000) -- (109.1000,169.4000) -- (109.1000,167.5000) -- (109.1000,165.7000) -- (109.1000,163.8000) -- (109.1000,162.0000) -- (109.1000,160.1000) -- (109.1000,158.3000) -- (109.1000,156.5000) -- (109.1000,154.6000) -- (109.1000,152.8000) -- (109.1000,150.9000) -- (109.1000,149.1000) -- (109.1000,147.2000) -- (109.1000,145.4000) -- (109.3000,143.5000) -- (109.7000,141.6000) -- (110.0000,139.8000) -- (110.0000,138.1000) -- (109.8000,136.3000) -- (109.5000,134.5000) -- (109.0000,132.7000) -- (108.3000,131.0000) -- (107.4000,129.4000) -- (106.3000,127.8000) -- (105.1000,126.3000) -- (103.7000,125.0000) -- (102.3000,123.7000) -- (100.6000,122.6000) -- (98.9000,121.6000) -- (97.1000,120.7000) -- (95.2000,120.0000) -- (93.3000,119.5000) -- (91.3000,119.1000) -- (89.3000,118.8000) -- (87.3000,118.7000) -- (85.2000,118.8000) -- (83.2000,119.0000) -- (81.2000,119.4000) -- (79.3000,119.9000) -- (77.4000,120.6000) -- (75.6000,121.5000) -- (73.9000,122.6000) -- (72.4000,123.8000) -- (70.9000,125.1000) -- (69.6000,126.5000) -- (68.5000,128.0000) -- (67.5000,129.6000) -- (66.7000,131.3000) -- (66.0000,133.0000) -- (65.7000,134.9000) -- (65.5000,136.7000) -- (65.4000,138.6000) -- (65.4000,140.4000) -- (65.4000,142.2000) -- (65.4000,144.0000) -- (65.3000,145.8000) -- (65.3000,147.7000) -- (65.3000,149.5000) -- (65.3000,151.4000) -- (65.3000,153.2000) -- (65.3000,155.1000) -- (65.3000,156.9000) -- (65.3000,158.8000) -- (65.3000,160.6000) -- (65.3000,162.4000) -- (65.3000,164.3000) -- (65.3000,166.1000) -- (65.3000,168.0000) -- (65.3000,169.8000) -- (65.3000,171.7000) -- (65.3000,173.5000) -- (65.3000,175.4000) -- (65.3000,177.2000) -- (65.4000,179.1000) -- (65.6000,181.0000) -- (66.0000,182.8000) -- (66.6000,184.6000) -- (67.4000,186.4000) -- (68.3000,188.1000) -- (69.4000,189.7000) -- (70.6000,191.2000) -- (71.9000,192.6000) -- (73.4000,193.9000) -- (75.0000,195.1000) -- (76.7000,196.2000) -- (78.5000,197.1000) -- (80.3000,198.0000) -- (82.2000,198.7000) -- (84.1000,199.2000) -- (86.1000,199.6000) -- (88.1000,199.9000) -- (90.1000,200.1000) -- (92.1000,200.2000) -- (94.1000,200.1000) -- (96.1000,199.8000) -- (98.0000,199.5000) -- (100.0000,199.0000) -- (101.9000,198.4000) -- (103.8000,197.7000) -- (105.6000,196.9000) -- (107.3000,195.9000) -- (109.0000,194.8000) -- (110.6000,193.6000) -- (112.0000,192.3000) -- (113.4000,190.9000) -- (114.6000,189.4000) -- (115.4000,187.7000) -- (115.9000,185.9000) -- (116.1000,184.1000) -- (116.2000,182.2000) -- (116.2000,180.3000) -- (116.3000,178.4000) -- (116.3000,176.5000) -- (116.3000,174.6000) -- (116.3000,172.8000) -- (116.3000,170.9000) -- (116.3000,169.1000) -- (116.3000,167.2000) -- (116.3000,165.4000) -- (116.3000,163.5000) -- (116.3000,161.7000) -- (116.3000,159.8000) -- (116.3000,158.0000) -- (116.3000,156.1000) -- (116.3000,154.3000) -- (116.3000,152.4000) -- (116.3000,150.6000) -- (116.3000,148.8000) -- (116.3000,146.9000) -- (116.3000,145.1000) -- (116.3000,143.2000) -- (116.5000,141.3000) -- (116.7000,139.5000) -- (116.7000,137.7000) -- (116.6000,135.9000) -- (116.3000,134.1000) -- (115.8000,132.3000) -- (115.1000,130.6000) -- (114.2000,129.0000) -- (113.2000,127.4000) -- (112.0000,125.9000) -- (110.6000,124.6000) -- (109.1000,123.3000) -- (107.5000,122.2000) -- (105.8000,121.2000) -- (104.0000,120.3000) -- (102.1000,119.6000) -- (100.2000,119.0000) -- (98.2000,118.6000) -- (96.2000,118.4000) -- (94.1000,118.3000) -- (92.1000,118.4000) -- (90.1000,118.6000) -- (88.1000,119.0000) -- (86.1000,119.5000) -- (84.2000,120.2000) -- (82.4000,121.1000) -- (80.7000,122.0000) -- (79.1000,123.2000) -- (77.6000,124.4000) -- (76.2000,125.8000) -- (75.0000,127.3000) -- (74.0000,128.8000) -- (73.1000,130.5000) -- (72.4000,132.2000) -- (72.0000,134.0000) -- (71.8000,135.9000) -- (71.7000,137.7000) -- (71.6000,139.5000) -- (71.6000,141.4000) -- (71.6000,143.2000) -- (71.6000,145.0000) -- (71.6000,146.8000) -- (71.6000,148.7000) -- (71.6000,150.5000) -- (71.6000,152.4000) -- (71.6000,154.2000) -- (71.6000,156.1000) -- (71.6000,157.9000) -- (71.6000,159.8000) -- (71.6000,161.6000) -- (71.6000,163.5000) -- (71.6000,165.3000) -- (71.6000,167.2000) -- (71.6000,169.0000) -- (71.6000,170.8000) -- (71.6000,172.7000) -- (71.6000,174.5000) -- (71.6000,176.4000) -- (71.6000,178.2000) -- (71.8000,180.1000) -- (72.2000,182.0000) -- (72.7000,183.8000) -- (73.5000,185.6000) -- (74.4000,187.3000) -- (75.5000,188.9000) -- (76.7000,190.4000) -- (78.0000,191.8000) -- (79.5000,193.1000) -- (81.1000,194.3000) -- (82.8000,195.4000) -- (84.6000,196.3000) -- (86.4000,197.1000) -- (88.3000,197.8000) -- (90.3000,198.3000) -- (92.2000,198.7000) -- (94.2000,199.0000) -- (96.2000,199.1000) -- (98.2000,199.1000) -- (100.2000,198.9000) -- (102.2000,198.6000) -- (104.2000,198.2000) -- (106.1000,197.6000) -- (107.9000,196.9000) -- (109.7000,196.0000) -- (111.5000,195.1000) -- (113.1000,194.0000) -- (114.7000,192.7000) -- (116.1000,191.4000) -- (117.4000,189.9000) -- (118.5000,188.4000) -- (119.4000,186.7000) -- (120.0000,185.0000) -- (120.4000,183.2000) -- (120.6000,181.3000) -- (120.7000,179.4000) -- (120.8000,177.5000) -- (120.8000,175.6000) -- (120.8000,173.8000) -- (120.8000,171.9000) -- (120.8000,170.0000) -- (120.9000,168.2000) -- (120.9000,166.3000) -- (120.9000,164.5000) -- (120.9000,162.6000) -- (120.9000,160.8000) -- (120.9000,158.9000) -- (120.9000,157.1000) -- (120.9000,155.2000) -- (120.9000,153.4000) -- (120.9000,151.5000) -- (120.9000,149.7000) -- (120.9000,147.8000) -- (120.9000,146.0000) -- (120.9000,144.1000) -- (121.2000,142.3000) -- (121.7000,140.4000) -- (121.9000,138.6000) -- (121.9000,136.9000) -- (121.8000,135.1000) -- (121.4000,133.3000) -- (120.9000,131.6000) -- (120.1000,129.9000) -- (119.2000,128.2000) -- (118.1000,126.7000) -- (116.9000,125.2000) -- (115.5000,123.9000) -- (114.0000,122.7000) -- (112.3000,121.6000) -- (110.6000,120.6000) -- (108.7000,119.8000) -- (106.8000,119.2000) -- (104.9000,118.7000) -- (102.8000,118.4000) -- (100.8000,118.2000) -- (98.8000,118.2000) -- (96.7000,118.4000) -- (94.7000,118.7000) -- (92.8000,119.2000) -- (90.9000,119.8000) -- (89.0000,120.6000) -- (87.3000,121.6000) -- (85.6000,122.6000) -- (84.1000,123.9000) -- (82.7000,125.2000) -- (81.5000,126.7000) -- (80.4000,128.2000) -- (79.5000,129.9000) -- (78.7000,131.6000) -- (78.3000,133.4000) -- (78.0000,135.2000) -- (77.9000,137.1000) -- (77.9000,138.9000) -- (77.8000,140.7000) -- (77.8000,142.5000) -- (77.8000,144.4000) -- (77.8000,146.2000) -- (77.8000,148.0000) -- (77.8000,149.9000) -- (77.8000,151.7000) -- (77.8000,153.6000) -- (77.8000,155.4000) -- (77.8000,157.3000) -- (77.8000,159.1000) -- (77.8000,161.0000) -- (77.8000,162.8000) -- (77.8000,164.6000) -- (77.8000,166.5000) -- (77.8000,168.3000) -- (77.8000,170.2000) -- (77.8000,172.0000) -- (77.8000,173.9000) -- (77.8000,175.7000) -- (77.8000,177.6000) -- (78.0000,179.5000) -- (78.4000,181.3000) -- (78.9000,183.1000) -- (79.7000,184.9000) -- (80.6000,186.6000) -- (81.7000,188.2000) -- (82.9000,189.7000) -- (84.2000,191.1000) -- (85.7000,192.5000) -- (87.3000,193.6000) -- (89.0000,194.7000) -- (90.8000,195.6000) -- (92.6000,196.4000) -- (94.5000,197.1000) -- (96.5000,197.6000) -- (98.5000,198.0000) -- (100.5000,198.2000) -- (102.5000,198.3000) -- (104.5000,198.3000) -- (106.5000,198.1000) -- (108.4000,197.7000) -- (110.4000,197.3000) -- (112.3000,196.7000) -- (114.1000,195.9000) -- (115.9000,195.0000) -- (117.6000,194.0000) -- (119.2000,192.9000) -- (120.8000,191.6000) -- (122.1000,190.2000) -- (123.4000,188.7000) -- (124.5000,187.1000) -- (125.3000,185.5000) -- (125.8000,183.7000) -- (126.1000,181.8000) -- (126.3000,180.0000) -- (126.4000,178.1000) -- (126.5000,176.2000) -- (126.5000,174.3000) -- (126.5000,172.4000) -- (126.5000,170.6000) -- (126.5000,168.7000) -- (126.5000,166.9000) -- (126.5000,165.0000) -- (126.5000,163.2000) -- (126.6000,161.3000) -- (126.6000,159.5000) -- (126.6000,157.6000) -- (126.6000,155.8000) -- (126.6000,153.9000) -- (126.6000,152.1000) -- (126.5000,150.2000) -- (126.5000,148.4000) -- (126.5000,146.5000) -- (126.5000,144.7000) -- (126.6000,142.8000) -- (127.1000,141.0000) -- (127.5000,139.2000) -- (127.6000,137.4000) -- (127.6000,135.6000) -- (127.3000,133.8000) -- (126.9000,132.0000) -- (126.3000,130.3000) -- (125.5000,128.6000) -- (124.5000,127.0000) -- (123.3000,125.5000) -- (122.0000,124.1000) -- (120.5000,122.8000) -- (118.9000,121.7000) -- (117.2000,120.7000) -- (115.4000,119.8000) -- (113.5000,119.1000) -- (111.6000,118.5000) -- (109.6000,118.1000) -- (107.6000,117.9000) -- (105.6000,117.8000) -- (103.5000,117.9000) -- (101.5000,118.2000) -- (99.5000,118.6000) -- (97.6000,119.2000) -- (95.7000,119.9000) -- (93.9000,120.8000) -- (92.2000,121.8000) -- (90.7000,123.0000) -- (89.2000,124.3000) -- (87.9000,125.7000) -- (86.8000,127.2000) -- (85.8000,128.8000) -- (85.0000,130.5000) -- (84.5000,132.3000) -- (84.1000,134.1000) -- (84.0000,136.0000) -- (83.9000,137.8000) -- (83.9000,139.6000) -- (83.9000,141.4000) -- (83.9000,143.3000) -- (83.8000,145.1000) -- (83.8000,146.9000) -- (83.8000,148.8000) -- (83.8000,150.6000) -- (83.8000,152.5000) -- (83.8000,154.3000) -- (83.8000,156.2000) -- (83.8000,158.0000) -- (83.8000,159.9000) -- (83.8000,161.7000) -- (83.8000,163.5000) -- (83.8000,165.4000) -- (83.8000,167.2000) -- (83.8000,169.1000) -- (83.8000,170.9000) -- (83.8000,172.8000) -- (83.8000,174.6000) -- (83.8000,176.5000) -- (83.9000,178.3000) -- (84.2000,180.2000) -- (84.8000,182.1000) -- (85.5000,183.8000) -- (86.3000,185.6000) -- (87.3000,187.2000) -- (88.5000,188.7000) -- (89.9000,190.2000) -- (91.3000,191.5000) -- (92.9000,192.7000) -- (94.5000,193.8000) -- (96.3000,194.8000) -- (98.1000,195.6000) -- (100.0000,196.3000) -- (102.0000,196.9000) -- (103.9000,197.3000) -- (105.9000,197.6000) -- (107.9000,197.7000) -- (109.9000,197.7000) -- (111.9000,197.5000) -- (113.9000,197.2000) -- (115.8000,196.8000) -- (117.7000,196.2000) -- (119.6000,195.5000) -- (121.4000,194.6000) -- (123.1000,193.6000) -- (124.8000,192.5000) -- (126.3000,191.3000) -- (127.7000,189.9000) -- (129.0000,188.4000) -- (130.1000,186.9000) -- (130.9000,185.2000) -- (131.5000,183.4000) -- (131.9000,181.6000) -- (132.1000,179.7000) -- (132.2000,177.8000) -- (132.3000,176.0000) -- (132.3000,174.1000) -- (132.3000,172.2000) -- (132.3000,170.3000) -- (132.3000,168.5000) -- (132.3000,166.6000) -- (132.3000,164.8000) -- (132.3000,162.9000) -- (132.3000,161.1000) -- (132.3000,159.2000) -- (132.3000,157.4000) -- (132.3000,155.5000) -- (132.3000,153.7000) -- (132.3000,151.8000) -- (132.3000,150.0000) -- (132.3000,148.1000) -- (132.3000,146.3000) -- (132.3000,144.4000) -- (132.4000,142.6000) -- (132.9000,140.7000) -- (133.2000,138.9000) -- (133.4000,137.1000) -- (133.4000,135.3000) -- (133.2000,133.5000) -- (132.8000,131.8000) -- (132.2000,130.0000) -- (131.3000,128.4000) -- (130.3000,126.8000) -- (129.1000,125.3000) -- (127.8000,123.9000) -- (126.3000,122.7000) -- (124.7000,121.6000) -- (122.9000,120.6000) -- (121.1000,119.8000) -- (119.2000,119.2000) -- (117.2000,118.7000) -- (115.2000,118.4000) -- (113.1000,118.3000) -- (111.1000,118.3000) -- (109.1000,118.6000) -- (107.1000,119.0000) -- (105.2000,119.6000) -- (103.3000,120.3000) -- (101.5000,121.2000) -- (99.9000,122.3000) -- (98.3000,123.5000) -- (96.9000,124.8000) -- (95.7000,126.3000) -- (94.6000,127.9000) -- (93.8000,129.5000) -- (93.1000,131.3000) -- (92.7000,133.1000) -- (92.6000,134.9000) -- (92.5000,136.8000) -- (92.4000,138.6000) -- (92.4000,140.4000) -- (92.4000,142.2000) -- (92.4000,144.1000) -- (92.4000,145.9000) -- (92.4000,147.7000) -- (92.4000,149.6000) -- (92.4000,151.4000) -- (92.4000,153.3000) -- (92.4000,155.1000) -- (92.4000,157.0000) -- (92.4000,158.8000) -- (92.4000,160.7000) -- (92.4000,162.5000) -- (92.4000,164.4000) -- (92.4000,166.2000) -- (92.4000,168.1000) -- (92.4000,169.9000) -- (92.4000,171.8000) -- (92.4000,173.6000) -- (92.4000,175.4000) -- (92.4000,177.3000) -- (92.7000,179.2000) -- (93.2000,181.0000) -- (93.8000,182.8000) -- (94.7000,184.5000) -- (95.7000,186.2000) -- (96.9000,187.8000) -- (98.2000,189.2000) -- (99.6000,190.6000) -- (101.1000,191.8000) -- (102.8000,192.9000) -- (104.5000,193.9000) -- (106.4000,194.8000) -- (108.2000,195.5000) -- (110.2000,196.0000) -- (112.1000,196.5000) -- (114.1000,196.8000) -- (116.1000,196.9000) -- (118.1000,196.9000) -- (120.1000,196.8000) -- (122.1000,196.5000) -- (124.1000,196.1000) -- (126.0000,195.5000) -- (127.8000,194.8000) -- (129.7000,194.0000) -- (131.4000,193.0000) -- (133.0000,191.9000) -- (134.6000,190.6000) -- (136.0000,189.3000) -- (137.3000,187.8000) -- (138.4000,186.3000) -- (139.3000,184.6000) -- (139.9000,182.8000) -- (140.3000,181.0000) -- (140.5000,179.2000) -- (140.6000,177.3000) -- (140.7000,175.4000) -- (140.7000,173.5000) -- (140.7000,171.6000) -- (140.7000,169.8000) -- (140.8000,167.9000) -- (140.8000,166.0000) -- (140.8000,164.2000) -- (140.8000,162.3000) -- (140.8000,160.5000) -- (140.8000,158.6000) -- (140.8000,156.8000) -- (140.8000,154.9000) -- (140.8000,153.1000) -- (140.8000,151.3000) -- (140.8000,149.4000) -- (140.8000,147.6000) -- (140.8000,145.7000) -- (140.8000,143.9000) -- (140.8000,142.0000) -- (141.3000,140.1000) -- (141.7000,138.3000) -- (142.0000,136.5000) -- (142.0000,134.8000) -- (141.8000,133.0000) -- (141.4000,131.2000) -- (140.8000,129.5000) -- (140.0000,127.8000) -- (139.0000,126.2000) -- (137.9000,124.7000) -- (136.5000,123.3000) -- (135.1000,122.0000) -- (133.5000,120.9000) -- (131.7000,119.9000) -- (129.9000,119.1000) -- (128.0000,118.4000) -- (126.1000,117.9000) -- (124.0000,117.6000) -- (122.0000,117.4000) -- (120.0000,117.4000) -- (117.9000,117.6000) -- (115.9000,118.0000) -- (114.0000,118.5000) -- (112.1000,119.2000) -- (110.3000,120.1000) -- (108.6000,121.1000) -- (107.0000,122.2000) -- (105.6000,123.5000) -- (104.3000,124.9000) -- (103.2000,126.5000) -- (102.2000,128.1000) -- (101.5000,129.8000) -- (101.0000,131.6000) -- (100.8000,133.5000) -- (100.6000,135.3000) -- (100.6000,137.1000) -- (100.6000,138.9000) -- (100.6000,140.8000) -- (100.5000,142.6000) -- (100.5000,144.4000) -- (100.5000,146.3000) -- (100.5000,148.1000) -- (100.5000,149.9000) -- (100.5000,151.8000) -- (100.5000,153.6000) -- (100.5000,155.5000) -- (100.5000,157.3000) -- (100.5000,159.2000) -- (100.5000,161.0000) -- (100.5000,162.9000) -- (100.5000,164.7000) -- (100.5000,166.6000) -- (100.5000,168.4000) -- (100.5000,170.3000) -- (100.5000,172.1000) -- (100.5000,174.0000) -- (100.5000,175.8000) -- (100.7000,177.7000) -- (101.1000,179.6000) -- (101.7000,181.4000) -- (102.5000,183.1000) -- (103.4000,184.8000) -- (104.5000,186.4000) -- (105.8000,187.9000) -- (107.1000,189.3000) -- (108.6000,190.6000) -- (110.2000,191.8000) -- (111.9000,192.8000) -- (113.7000,193.7000) -- (115.6000,194.5000) -- (117.5000,195.2000) -- (119.4000,195.7000) -- (121.4000,196.0000) -- (123.4000,196.2000) -- (125.4000,196.3000) -- (127.4000,196.2000) -- (129.4000,196.0000) -- (131.4000,195.7000) -- (133.3000,195.2000) -- (135.2000,194.6000) -- (137.1000,193.8000) -- (138.8000,192.9000) -- (140.5000,191.9000) -- (142.1000,190.7000) -- (143.6000,189.4000) -- (145.0000,188.0000) -- (146.2000,186.5000) -- (147.3000,184.9000) -- (148.0000,183.2000) -- (148.5000,181.4000) -- (148.8000,179.6000) -- (149.0000,177.7000) -- (149.1000,175.8000) -- (149.1000,173.9000) -- (149.2000,172.1000) -- (149.2000,170.2000) -- (149.2000,168.3000) -- (149.2000,166.5000) -- (149.2000,164.6000) -- (149.2000,162.8000) -- (149.2000,160.9000) -- (149.2000,159.1000) -- (149.2000,157.2000) -- (149.2000,155.4000) -- (149.2000,153.5000) -- (149.2000,151.7000) -- (149.2000,149.8000) -- (149.2000,148.0000) -- (149.2000,146.1000) -- (149.2000,144.3000) -- (149.2000,142.4000) -- (149.4000,140.6000) -- (149.9000,138.7000) -- (150.3000,136.9000) -- (150.4000,135.1000) -- (150.4000,133.4000) -- (150.1000,131.6000) -- (149.7000,129.8000) -- (149.0000,128.1000) -- (148.1000,126.5000) -- (147.1000,124.9000) -- (145.9000,123.4000) -- (144.5000,122.1000) -- (142.9000,120.8000) -- (141.3000,119.8000) -- (139.5000,118.8000) -- (137.7000,118.0000) -- (135.8000,117.4000) -- (133.8000,117.0000) -- (131.8000,116.7000) -- (129.7000,116.6000) -- (127.7000,116.7000) -- (125.7000,117.0000) -- (123.7000,117.4000) -- (121.7000,118.0000) -- (119.9000,118.7000) -- (118.1000,119.6000) -- (116.4000,120.7000) -- (114.9000,121.9000) -- (113.5000,123.2000) -- (112.3000,124.7000) -- (111.2000,126.3000) -- (110.3000,127.9000) -- (109.7000,129.7000) -- (109.3000,131.5000) -- (109.1000,133.3000) -- (109.0000,135.2000) -- (109.0000,137.0000) -- (109.0000,138.8000) -- (108.9000,140.6000) -- (108.9000,142.5000) -- (108.9000,144.3000) -- (108.9000,146.1000) -- (108.9000,148.0000) -- (108.9000,149.8000) -- (108.9000,151.7000) -- (108.9000,153.5000) -- (108.9000,155.4000) -- (108.9000,157.2000) -- (108.9000,159.1000) -- (108.9000,160.9000) -- (108.9000,162.8000) -- (108.9000,164.6000) -- (108.9000,166.5000) -- (108.9000,168.3000) -- (108.9000,170.1000) -- (108.9000,172.0000) -- (108.9000,173.8000) -- (109.0000,175.7000) -- (109.2000,177.6000) -- (109.7000,179.4000) -- (110.4000,181.2000) -- (111.2000,183.0000) -- (112.2000,184.6000) -- (113.3000,186.2000) -- (114.6000,187.7000) -- (116.0000,189.0000) -- (117.5000,190.3000) -- (119.2000,191.4000) -- (120.9000,192.4000) -- (122.7000,193.3000) -- (124.6000,194.0000) -- (126.5000,194.6000) -- (128.5000,195.1000) -- (130.5000,195.4000) -- (132.5000,195.6000) -- (134.5000,195.6000) -- (136.5000,195.5000) -- (138.5000,195.2000) -- (140.4000,194.8000) -- (142.3000,194.3000) -- (144.2000,193.6000) -- (146.0000,192.8000) -- (147.8000,191.9000) -- (149.5000,190.8000) -- (151.0000,189.6000) -- (152.5000,188.3000) -- (153.8000,186.9000) -- (155.0000,185.3000) -- (156.0000,183.7000) -- (156.6000,182.0000) -- (157.1000,180.2000) -- (157.3000,178.3000) -- (157.5000,176.4000) -- (157.6000,174.5000) -- (157.6000,172.7000) -- (157.6000,170.8000) -- (157.6000,168.9000) -- (157.6000,167.0000) -- (157.6000,165.2000) -- (157.6000,163.3000) -- (157.7000,161.5000) -- (157.7000,159.6000) -- (157.7000,157.8000) -- (157.7000,155.9000) -- (157.7000,154.1000) -- (157.7000,152.2000) -- (157.7000,150.4000) -- (157.7000,148.6000) -- (157.7000,146.7000) -- (157.7000,144.9000) -- (157.6000,143.0000) -- (157.6000,141.2000) -- (158.0000,139.3000) -- (158.5000,137.4000) -- (158.8000,135.6000);



  \end{scope}
  \begin{scope}[scale=1.006,draw=black,line join=bevel,line cap=rect,line width=0.800pt]
  \end{scope}
  \begin{scope}[scale=1.006,draw=black,line join=bevel,line cap=rect,line width=0.800pt]
  \end{scope}
  \begin{scope}[scale=1.006,draw=black,line join=round,line cap=round,line width=0.480pt]
    \path[draw] (44.5000,108.5000) -- (44.5000,210.5000) -- (179.5000,210.5000) -- (179.5000,108.5000) -- (44.5000,108.5000);



  \end{scope}
  \begin{scope}[scale=1.006,draw=ca0a0a4,dash pattern=on 0.40pt off 0.80pt,line join=round,line cap=round,line width=0.400pt]
    \path[draw] (184.5000,151.5000) -- (246.5000,151.5000);



  \end{scope}
  \begin{scope}[scale=1.006,draw=black,line join=round,line cap=round,line width=0.480pt]
    \path[draw] (184.5000,151.5000) -- (186.5000,151.5000);



    \path[draw] (246.5000,151.5000) -- (245.5000,151.5000);



  \end{scope}
  \begin{scope}[scale=1.006,draw=black,line join=bevel,line cap=rect,line width=0.800pt]
  \end{scope}
  \begin{scope}[cm={{1.00588,0.0,0.0,1.00588,(181.059,152.894)}},draw=black,line join=bevel,line cap=rect,line width=0.800pt]
  \end{scope}
  \begin{scope}[cm={{1.00588,0.0,0.0,1.00588,(181.059,152.894)}},draw=black,line join=bevel,line cap=rect,line width=0.800pt]
  \end{scope}
  \begin{scope}[cm={{1.00588,0.0,0.0,1.00588,(181.059,152.894)}},draw=black,line join=bevel,line cap=rect,line width=0.800pt]
  \end{scope}
  \begin{scope}[cm={{1.00588,0.0,0.0,1.00588,(181.059,152.894)}},draw=black,line join=bevel,line cap=rect,line width=0.800pt]
  \end{scope}
  \begin{scope}[cm={{1.00588,0.0,0.0,1.00588,(181.059,152.894)}},draw=black,line join=bevel,line cap=rect,line width=0.800pt]
  \end{scope}
  \begin{scope}[cm={{1.00588,0.0,0.0,1.00588,(181.059,152.894)}},draw=black,line join=bevel,line cap=rect,line width=0.800pt]
  \end{scope}
  \begin{scope}[scale=1.006,draw=black,line join=bevel,line cap=rect,line width=0.800pt]
  \end{scope}
  \begin{scope}[scale=1.006,draw=ca0a0a4,dash pattern=on 0.40pt off 0.80pt,line join=round,line cap=round,line width=0.400pt]
    \path[draw] (184.5000,132.5000) -- (246.5000,132.5000);



  \end{scope}
  \begin{scope}[scale=1.006,draw=black,line join=round,line cap=round,line width=0.480pt]
    \path[draw] (184.5000,132.5000) -- (186.5000,132.5000);



    \path[draw] (246.5000,132.5000) -- (245.5000,132.5000);



  \end{scope}
  \begin{scope}[scale=1.006,draw=black,line join=bevel,line cap=rect,line width=0.800pt]
  \end{scope}
  \begin{scope}[cm={{1.00588,0.0,0.0,1.00588,(181.059,133.782)}},draw=black,line join=bevel,line cap=rect,line width=0.800pt]
  \end{scope}
  \begin{scope}[cm={{1.00588,0.0,0.0,1.00588,(181.059,133.782)}},draw=black,line join=bevel,line cap=rect,line width=0.800pt]
  \end{scope}
  \begin{scope}[cm={{1.00588,0.0,0.0,1.00588,(181.059,133.782)}},draw=black,line join=bevel,line cap=rect,line width=0.800pt]
  \end{scope}
  \begin{scope}[cm={{1.00588,0.0,0.0,1.00588,(181.059,133.782)}},draw=black,line join=bevel,line cap=rect,line width=0.800pt]
  \end{scope}
  \begin{scope}[cm={{1.00588,0.0,0.0,1.00588,(181.059,133.782)}},draw=black,line join=bevel,line cap=rect,line width=0.800pt]
  \end{scope}
  \begin{scope}[cm={{1.00588,0.0,0.0,1.00588,(181.059,133.782)}},draw=black,line join=bevel,line cap=rect,line width=0.800pt]
  \end{scope}
  \begin{scope}[scale=1.006,draw=black,line join=bevel,line cap=rect,line width=0.800pt]
  \end{scope}
  \begin{scope}[scale=1.006,draw=ca0a0a4,dash pattern=on 0.40pt off 0.80pt,line join=round,line cap=round,line width=0.400pt]
    \path[draw] (184.5000,113.5000) -- (246.5000,113.5000);



  \end{scope}
  \begin{scope}[scale=1.006,draw=black,line join=round,line cap=round,line width=0.480pt]
    \path[draw] (184.5000,113.5000) -- (186.5000,113.5000);



    \path[draw] (246.5000,113.5000) -- (245.5000,113.5000);



  \end{scope}
  \begin{scope}[scale=1.006,draw=black,line join=bevel,line cap=rect,line width=0.800pt]
  \end{scope}
  \begin{scope}[cm={{1.00588,0.0,0.0,1.00588,(181.059,114.671)}},draw=black,line join=bevel,line cap=rect,line width=0.800pt]
  \end{scope}
  \begin{scope}[cm={{1.00588,0.0,0.0,1.00588,(181.059,114.671)}},draw=black,line join=bevel,line cap=rect,line width=0.800pt]
  \end{scope}
  \begin{scope}[cm={{1.00588,0.0,0.0,1.00588,(181.059,114.671)}},draw=black,line join=bevel,line cap=rect,line width=0.800pt]
  \end{scope}
  \begin{scope}[cm={{1.00588,0.0,0.0,1.00588,(181.059,114.671)}},draw=black,line join=bevel,line cap=rect,line width=0.800pt]
  \end{scope}
  \begin{scope}[cm={{1.00588,0.0,0.0,1.00588,(181.059,114.671)}},draw=black,line join=bevel,line cap=rect,line width=0.800pt]
  \end{scope}
  \begin{scope}[cm={{1.00588,0.0,0.0,1.00588,(181.059,114.671)}},draw=black,line join=bevel,line cap=rect,line width=0.800pt]
  \end{scope}
  \begin{scope}[scale=1.006,draw=black,line join=bevel,line cap=rect,line width=0.800pt]
  \end{scope}
  \begin{scope}[scale=1.006,draw=ca0a0a4,dash pattern=on 0.40pt off 0.80pt,line join=round,line cap=round,line width=0.400pt]
    \path[draw] (184.5000,156.5000) -- (184.5000,108.5000);



  \end{scope}
  \begin{scope}[scale=1.006,draw=black,line join=round,line cap=round,line width=0.480pt]
    \path[draw] (184.5000,156.5000) -- (184.5000,155.5000);



    \path[draw] (184.5000,108.5000) -- (184.5000,109.5000);



  \end{scope}
  \begin{scope}[scale=1.006,draw=black,line join=bevel,line cap=rect,line width=0.800pt]
  \end{scope}
  \begin{scope}[cm={{1.00588,0.0,0.0,1.00588,(186.088,173.012)}},draw=black,line join=bevel,line cap=rect,line width=0.800pt]
  \end{scope}
  \begin{scope}[cm={{1.00588,0.0,0.0,1.00588,(186.088,173.012)}},draw=black,line join=bevel,line cap=rect,line width=0.800pt]
  \end{scope}
  \begin{scope}[cm={{1.00588,0.0,0.0,1.00588,(186.088,173.012)}},draw=black,line join=bevel,line cap=rect,line width=0.800pt]
  \end{scope}
  \begin{scope}[cm={{1.00588,0.0,0.0,1.00588,(186.088,173.012)}},draw=black,line join=bevel,line cap=rect,line width=0.800pt]
  \end{scope}
  \begin{scope}[cm={{1.00588,0.0,0.0,1.00588,(186.088,173.012)}},draw=black,line join=bevel,line cap=rect,line width=0.800pt]
  \end{scope}
  \begin{scope}[cm={{1.00588,0.0,0.0,1.00588,(186.088,173.012)}},draw=black,line join=bevel,line cap=rect,line width=0.800pt]
  \end{scope}
  \begin{scope}[scale=1.006,draw=black,line join=bevel,line cap=rect,line width=0.800pt]
  \end{scope}
  \begin{scope}[scale=1.006,draw=ca0a0a4,dash pattern=on 0.40pt off 0.80pt,line join=round,line cap=round,line width=0.400pt]
    \path[draw] (203.5000,156.5000) -- (203.5000,108.5000);



  \end{scope}
  \begin{scope}[scale=1.006,draw=black,line join=round,line cap=round,line width=0.480pt]
    \path[draw] (203.5000,156.5000) -- (203.5000,155.5000);



    \path[draw] (203.5000,108.5000) -- (203.5000,109.5000);



  \end{scope}
  \begin{scope}[scale=1.006,draw=black,line join=bevel,line cap=rect,line width=0.800pt]
  \end{scope}
  \begin{scope}[cm={{1.00588,0.0,0.0,1.00588,(204.194,173.012)}},draw=black,line join=bevel,line cap=rect,line width=0.800pt]
  \end{scope}
  \begin{scope}[cm={{1.00588,0.0,0.0,1.00588,(204.194,173.012)}},draw=black,line join=bevel,line cap=rect,line width=0.800pt]
  \end{scope}
  \begin{scope}[cm={{1.00588,0.0,0.0,1.00588,(204.194,173.012)}},draw=black,line join=bevel,line cap=rect,line width=0.800pt]
  \end{scope}
  \begin{scope}[cm={{1.00588,0.0,0.0,1.00588,(204.194,173.012)}},draw=black,line join=bevel,line cap=rect,line width=0.800pt]
  \end{scope}
  \begin{scope}[cm={{1.00588,0.0,0.0,1.00588,(204.194,173.012)}},draw=black,line join=bevel,line cap=rect,line width=0.800pt]
  \end{scope}
  \begin{scope}[cm={{1.00588,0.0,0.0,1.00588,(204.194,173.012)}},draw=black,line join=bevel,line cap=rect,line width=0.800pt]
  \end{scope}
  \begin{scope}[scale=1.006,draw=black,line join=bevel,line cap=rect,line width=0.800pt]
  \end{scope}
  \begin{scope}[scale=1.006,draw=ca0a0a4,dash pattern=on 0.40pt off 0.80pt,line join=round,line cap=round,line width=0.400pt]
    \path[draw] (221.5000,156.5000) -- (221.5000,108.5000);



  \end{scope}
  \begin{scope}[scale=1.006,draw=black,line join=round,line cap=round,line width=0.480pt]
    \path[draw] (221.5000,156.5000) -- (221.5000,155.5000);



    \path[draw] (221.5000,108.5000) -- (221.5000,109.5000);



  \end{scope}
  \begin{scope}[scale=1.006,draw=black,line join=bevel,line cap=rect,line width=0.800pt]
  \end{scope}
  \begin{scope}[cm={{1.00588,0.0,0.0,1.00588,(222.3,173.012)}},draw=black,line join=bevel,line cap=rect,line width=0.800pt]
  \end{scope}
  \begin{scope}[cm={{1.00588,0.0,0.0,1.00588,(222.3,173.012)}},draw=black,line join=bevel,line cap=rect,line width=0.800pt]
  \end{scope}
  \begin{scope}[cm={{1.00588,0.0,0.0,1.00588,(222.3,173.012)}},draw=black,line join=bevel,line cap=rect,line width=0.800pt]
  \end{scope}
  \begin{scope}[cm={{1.00588,0.0,0.0,1.00588,(222.3,173.012)}},draw=black,line join=bevel,line cap=rect,line width=0.800pt]
  \end{scope}
  \begin{scope}[cm={{1.00588,0.0,0.0,1.00588,(222.3,173.012)}},draw=black,line join=bevel,line cap=rect,line width=0.800pt]
  \end{scope}
  \begin{scope}[cm={{1.00588,0.0,0.0,1.00588,(222.3,173.012)}},draw=black,line join=bevel,line cap=rect,line width=0.800pt]
  \end{scope}
  \begin{scope}[scale=1.006,draw=black,line join=bevel,line cap=rect,line width=0.800pt]
  \end{scope}
  \begin{scope}[scale=1.006,draw=ca0a0a4,dash pattern=on 0.40pt off 0.80pt,line join=round,line cap=round,line width=0.400pt]
    \path[draw] (239.5000,156.5000) -- (239.5000,108.5000);



  \end{scope}
  \begin{scope}[scale=1.006,draw=black,line join=round,line cap=round,line width=0.480pt]
    \path[draw] (239.5000,156.5000) -- (239.5000,155.5000);



    \path[draw] (239.5000,108.5000) -- (239.5000,109.5000);



  \end{scope}
  \begin{scope}[scale=1.006,draw=black,line join=bevel,line cap=rect,line width=0.800pt]
  \end{scope}
  \begin{scope}[cm={{1.00588,0.0,0.0,1.00588,(241.412,173.012)}},draw=black,line join=bevel,line cap=rect,line width=0.800pt]
  \end{scope}
  \begin{scope}[cm={{1.00588,0.0,0.0,1.00588,(241.412,173.012)}},draw=black,line join=bevel,line cap=rect,line width=0.800pt]
  \end{scope}
  \begin{scope}[cm={{1.00588,0.0,0.0,1.00588,(241.412,173.012)}},draw=black,line join=bevel,line cap=rect,line width=0.800pt]
  \end{scope}
  \begin{scope}[cm={{1.00588,0.0,0.0,1.00588,(241.412,173.012)}},draw=black,line join=bevel,line cap=rect,line width=0.800pt]
  \end{scope}
  \begin{scope}[cm={{1.00588,0.0,0.0,1.00588,(241.412,173.012)}},draw=black,line join=bevel,line cap=rect,line width=0.800pt]
  \end{scope}
  \begin{scope}[cm={{1.00588,0.0,0.0,1.00588,(241.412,173.012)}},draw=black,line join=bevel,line cap=rect,line width=0.800pt]
  \end{scope}
  \begin{scope}[scale=1.006,draw=black,line join=bevel,line cap=rect,line width=0.800pt]
  \end{scope}
  \begin{scope}[scale=1.006,draw=black,line join=round,line cap=round,line width=0.480pt]
    \path[draw] (246.5000,151.5000) -- (245.5000,151.5000);



  \end{scope}
  \begin{scope}[scale=1.006,draw=black,line join=bevel,line cap=rect,line width=0.800pt]
  \end{scope}
  \begin{scope}[cm={{1.00588,0.0,0.0,1.00588,(249.459,156.918)}},draw=black,line join=bevel,line cap=rect,line width=0.800pt]
  \end{scope}
  \begin{scope}[cm={{1.00588,0.0,0.0,1.00588,(249.459,156.918)}},draw=black,line join=bevel,line cap=rect,line width=0.800pt]
  \end{scope}
  \begin{scope}[cm={{1.00588,0.0,0.0,1.00588,(249.459,156.918)}},draw=black,line join=bevel,line cap=rect,line width=0.800pt]
  \end{scope}
  \begin{scope}[cm={{1.00588,0.0,0.0,1.00588,(249.459,156.918)}},draw=black,line join=bevel,line cap=rect,line width=0.800pt]
  \end{scope}
  \begin{scope}[cm={{1.00588,0.0,0.0,1.00588,(249.459,156.918)}},draw=black,line join=bevel,line cap=rect,line width=0.800pt]
  \end{scope}
  \begin{scope}[cm={{1.00588,0.0,0.0,1.00588,(252.459,156.918)}},draw=black,line join=bevel,line cap=rect,line width=0.800pt]
    \path[fill=black] (0.0000,0.0000) node[above right] (text1220) {\scriptsize 2};



  \end{scope}
  \begin{scope}[cm={{1.00588,0.0,0.0,1.00588,(249.459,156.918)}},draw=black,line join=bevel,line cap=rect,line width=0.800pt]
  \end{scope}
  \begin{scope}[scale=1.006,draw=black,line join=bevel,line cap=rect,line width=0.800pt]
  \end{scope}
  \begin{scope}[scale=1.006,draw=black,line join=round,line cap=round,line width=0.480pt]
    \path[draw] (246.5000,132.5000) -- (245.5000,132.5000);



  \end{scope}
  \begin{scope}[scale=1.006,draw=black,line join=bevel,line cap=rect,line width=0.800pt]
  \end{scope}
  \begin{scope}[cm={{1.00588,0.0,0.0,1.00588,(249.459,137.806)}},draw=black,line join=bevel,line cap=rect,line width=0.800pt]
  \end{scope}
  \begin{scope}[cm={{1.00588,0.0,0.0,1.00588,(249.459,137.806)}},draw=black,line join=bevel,line cap=rect,line width=0.800pt]
  \end{scope}
  \begin{scope}[cm={{1.00588,0.0,0.0,1.00588,(249.459,137.806)}},draw=black,line join=bevel,line cap=rect,line width=0.800pt]
  \end{scope}
  \begin{scope}[cm={{1.00588,0.0,0.0,1.00588,(249.459,137.806)}},draw=black,line join=bevel,line cap=rect,line width=0.800pt]
  \end{scope}
  \begin{scope}[cm={{1.00588,0.0,0.0,1.00588,(249.459,137.806)}},draw=black,line join=bevel,line cap=rect,line width=0.800pt]
  \end{scope}
  \begin{scope}[cm={{1.00588,0.0,0.0,1.00588,(252.459,137.806)}},draw=black,line join=bevel,line cap=rect,line width=0.800pt]
    \path[fill=black] (0.0000,0.0000) node[above right] (text1244) {\scriptsize 6};



  \end{scope}
  \begin{scope}[cm={{1.00588,0.0,0.0,1.00588,(249.459,137.806)}},draw=black,line join=bevel,line cap=rect,line width=0.800pt]
  \end{scope}
  \begin{scope}[scale=1.006,draw=black,line join=bevel,line cap=rect,line width=0.800pt]
  \end{scope}
  \begin{scope}[scale=1.006,draw=black,line join=round,line cap=round,line width=0.480pt]
    \path[draw] (246.5000,113.5000) -- (245.5000,113.5000);



  \end{scope}
  \begin{scope}[scale=1.006,draw=black,line join=bevel,line cap=rect,line width=0.800pt]
  \end{scope}
  \begin{scope}[cm={{1.00588,0.0,0.0,1.00588,(249.459,118.694)}},draw=black,line join=bevel,line cap=rect,line width=0.800pt]
  \end{scope}
  \begin{scope}[cm={{1.00588,0.0,0.0,1.00588,(249.459,118.694)}},draw=black,line join=bevel,line cap=rect,line width=0.800pt]
  \end{scope}
  \begin{scope}[cm={{1.00588,0.0,0.0,1.00588,(249.459,118.694)}},draw=black,line join=bevel,line cap=rect,line width=0.800pt]
  \end{scope}
  \begin{scope}[cm={{1.00588,0.0,0.0,1.00588,(249.459,118.694)}},draw=black,line join=bevel,line cap=rect,line width=0.800pt]
  \end{scope}
  \begin{scope}[cm={{1.00588,0.0,0.0,1.00588,(249.459,118.694)}},draw=black,line join=bevel,line cap=rect,line width=0.800pt]
  \end{scope}
  \begin{scope}[cm={{1.00588,0.0,0.0,1.00588,(250.959,118.694)}},draw=black,line join=bevel,line cap=rect,line width=0.800pt]
    \path[fill=black] (0.0000,0.0000) node[above right] (text1268) {\scriptsize 10};



  \end{scope}
  \begin{scope}[cm={{1.00588,0.0,0.0,1.00588,(249.459,118.694)}},draw=black,line join=bevel,line cap=rect,line width=0.800pt]
  \end{scope}
  \begin{scope}[scale=1.006,draw=black,line join=bevel,line cap=rect,line width=0.800pt]
  \end{scope}
  \begin{scope}[scale=1.006,draw=black,line join=round,line cap=round,line width=0.480pt]
    \path[draw] (184.5000,108.5000) -- (184.5000,156.5000) -- (246.5000,156.5000) -- (246.5000,108.5000) -- (184.5000,108.5000);



  \end{scope}
  \begin{scope}[scale=1.006,draw=black,line join=bevel,line cap=rect,line width=0.800pt]
  \end{scope}
  \begin{scope}[cm={{0.0,-1.00588,1.00588,0.0,(276.618,139.315)}},draw=black,line join=bevel,line cap=rect,line width=0.800pt]
  \end{scope}
  \begin{scope}[cm={{0.0,-1.00588,1.00588,0.0,(276.618,139.315)}},draw=black,line join=bevel,line cap=rect,line width=0.800pt]
  \end{scope}
  \begin{scope}[cm={{0.0,-1.00588,1.00588,0.0,(276.618,139.315)}},draw=black,line join=bevel,line cap=rect,line width=0.800pt]
  \end{scope}
  \begin{scope}[cm={{0.0,-1.00588,1.00588,0.0,(276.618,139.315)}},draw=black,line join=bevel,line cap=rect,line width=0.800pt]
  \end{scope}
  \begin{scope}[cm={{0.0,-1.00588,1.00588,0.0,(276.618,139.315)}},draw=black,line join=bevel,line cap=rect,line width=0.800pt]
  \end{scope}
  \begin{scope}[cm={{0.0,-1.00588,1.00588,0.0,(263.618,139.315)}},draw=black,line join=bevel,line cap=rect,line width=0.800pt]
    \path[fill=black] (0.0000,0.0000) node[above right] (text1292) {\rotatebox{90}{\scriptsize $c_{1,2}$}};



  \end{scope}
  \begin{scope}[cm={{0.0,-1.00588,1.00588,0.0,(276.618,139.315)}},draw=black,line join=bevel,line cap=rect,line width=0.800pt]
  \end{scope}
  \begin{scope}[scale=1.006,draw=black,line join=bevel,line cap=rect,line width=0.800pt]
  \end{scope}
  \begin{scope}[scale=1.006,draw=black,line join=bevel,line cap=rect,line width=0.800pt]
  \end{scope}
  \begin{scope}[scale=1.006,draw=black,line join=bevel,line cap=rect,line width=0.800pt]
  \end{scope}
  \begin{scope}[scale=1.006,draw=black,line join=round,line cap=round,line width=0.480pt]
    \path[draw] (184.8000,113.6000) -- (184.8000,113.6000) -- (184.9000,113.6000) -- (185.0000,113.6000) -- (185.0000,113.6000) -- (185.1000,113.6000) -- (185.1000,113.6000) -- (185.2000,113.6000) -- (185.3000,113.6000) -- (185.3000,113.6000) -- (185.4000,113.6000) -- (185.4000,113.6000) -- (185.5000,113.6000) -- (185.6000,113.6000) -- (185.6000,113.6000) -- (185.7000,113.6000) -- (185.8000,113.6000) -- (185.8000,113.6000) -- (185.9000,113.6000) -- (185.9000,113.6000) -- (186.0000,113.6000) -- (186.1000,113.6000) -- (186.1000,113.6000) -- (186.2000,113.6000) -- (186.2000,113.6000) -- (186.3000,113.6000) -- (186.4000,113.6000) -- (186.4000,113.6000) -- (186.5000,113.6000) -- (186.6000,113.6000) -- (186.6000,113.6000) -- (186.7000,113.6000) -- (186.7000,113.6000) -- (186.8000,113.6000) -- (186.9000,113.6000) -- (186.9000,113.6000) -- (187.0000,113.6000) -- (187.0000,113.6000) -- (187.1000,113.6000) -- (187.2000,113.6000) -- (187.2000,113.6000) -- (187.3000,113.6000) -- (187.4000,113.6000) -- (187.4000,113.6000) -- (187.5000,113.6000) -- (187.5000,113.6000) -- (187.6000,113.6000) -- (187.7000,113.6000) -- (187.7000,113.6000) -- (187.8000,113.6000) -- (187.8000,113.6000) -- (187.9000,113.6000) -- (188.0000,113.6000) -- (188.0000,113.6000) -- (188.1000,113.6000) -- (188.2000,113.6000) -- (188.2000,113.6000) -- (188.3000,113.6000) -- (188.3000,113.6000) -- (188.4000,113.6000) -- (188.5000,113.6000) -- (188.5000,113.6000) -- (188.6000,113.6000) -- (188.6000,113.6000) -- (188.7000,113.6000) -- (188.8000,113.6000) -- (188.8000,113.6000) -- (188.9000,113.6000) -- (189.0000,113.6000) -- (189.0000,113.6000) -- (189.1000,113.6000) -- (189.1000,113.6000) -- (189.2000,113.6000) -- (189.3000,113.6000) -- (189.3000,113.6000) -- (189.4000,113.6000) -- (189.4000,113.6000) -- (189.5000,113.6000) -- (189.6000,113.6000) -- (189.6000,113.6000) -- (189.7000,113.6000) -- (189.8000,113.6000) -- (189.8000,113.6000) -- (189.9000,113.6000) -- (189.9000,113.6000) -- (190.0000,113.6000) -- (190.1000,113.6000) -- (190.1000,113.6000) -- (190.2000,113.6000) -- (190.2000,113.6000) -- (190.3000,113.6000) -- (190.4000,113.6000) -- (190.4000,113.6000) -- (190.5000,113.6000) -- (190.6000,113.6000) -- (190.6000,113.6000) -- (190.7000,113.6000) -- (190.7000,113.6000) -- (190.8000,113.6000) -- (190.9000,113.6000) -- (190.9000,113.6000) -- (191.0000,113.6000) -- (191.0000,113.6000) -- (191.1000,113.6000) -- (191.2000,113.6000) -- (191.2000,113.6000) -- (191.3000,113.6000) -- (191.4000,113.6000) -- (191.4000,113.6000) -- (191.5000,113.6000) -- (191.5000,113.6000) -- (191.6000,113.6000) -- (191.7000,113.6000) -- (191.7000,113.6000) -- (191.8000,113.6000) -- (191.8000,113.6000) -- (191.9000,113.6000) -- (192.0000,113.6000) -- (192.0000,113.6000) -- (192.1000,113.6000) -- (192.2000,113.6000) -- (192.2000,113.6000) -- (192.3000,113.6000) -- (192.3000,113.6000) -- (192.4000,113.6000) -- (192.5000,113.6000) -- (192.5000,113.6000) -- (192.6000,113.6000) -- (192.6000,113.6000) -- (192.7000,113.6000) -- (192.8000,113.6000) -- (192.8000,113.6000) -- (192.9000,113.6000) -- (193.0000,113.6000) -- (193.0000,113.6000) -- (193.1000,113.6000) -- (193.1000,113.6000) -- (193.2000,113.6000) -- (193.3000,113.6000) -- (193.3000,113.6000) -- (193.4000,113.6000) -- (193.4000,113.6000) -- (193.5000,113.6000) -- (193.6000,113.6000) -- (193.6000,113.6000) -- (193.7000,113.6000) -- (193.8000,113.6000) -- (193.8000,113.6000) -- (193.9000,113.6000) -- (193.9000,113.6000) -- (194.0000,113.6000) -- (194.1000,113.6000) -- (194.1000,113.6000) -- (194.2000,113.6000) -- (194.2000,113.6000) -- (194.3000,113.6000) -- (194.4000,113.6000) -- (194.4000,113.6000) -- (194.5000,113.6000) -- (194.6000,113.6000) -- (194.6000,113.6000) -- (194.7000,113.6000) -- (194.7000,113.6000) -- (194.8000,113.6000) -- (194.9000,113.6000) -- (194.9000,113.6000) -- (195.0000,113.6000) -- (195.0000,113.6000) -- (195.1000,113.6000) -- (195.2000,113.6000) -- (195.2000,113.6000) -- (195.3000,113.6000) -- (195.4000,113.6000) -- (195.4000,113.6000) -- (195.5000,113.6000) -- (195.5000,113.6000) -- (195.6000,113.6000) -- (195.7000,113.6000) -- (195.7000,113.6000) -- (195.8000,113.6000) -- (195.8000,113.6000) -- (195.9000,113.6000) -- (196.0000,113.6000) -- (196.0000,113.6000) -- (196.1000,113.6000) -- (196.2000,113.6000) -- (196.2000,113.6000) -- (196.3000,113.6000) -- (196.3000,113.6000) -- (196.4000,113.6000) -- (196.5000,113.6000) -- (196.5000,113.6000) -- (196.6000,113.6000) -- (196.6000,113.6000) -- (196.7000,113.6000) -- (196.8000,113.6000) -- (196.8000,113.6000) -- (196.9000,113.6000) -- (196.9000,113.6000) -- (197.0000,113.6000) -- (197.1000,113.6000) -- (197.1000,113.6000) -- (197.2000,113.6000) -- (197.3000,113.6000) -- (197.3000,113.6000) -- (197.4000,113.6000) -- (197.4000,113.6000) -- (197.5000,113.6000) -- (197.6000,113.6000) -- (197.6000,113.6000) -- (197.7000,113.6000) -- (197.7000,113.6000) -- (197.8000,113.6000) -- (197.9000,113.6000) -- (197.9000,113.6000) -- (198.0000,113.6000) -- (198.1000,113.6000) -- (198.1000,113.6000) -- (198.2000,113.6000) -- (198.2000,113.6000) -- (198.3000,113.6000) -- (198.4000,113.6000) -- (198.4000,113.6000) -- (198.5000,113.6000) -- (198.5000,113.6000) -- (198.6000,113.6000) -- (198.7000,113.6000) -- (198.7000,113.6000) -- (198.8000,113.6000) -- (198.9000,113.6000) -- (198.9000,113.6000) -- (199.0000,113.6000) -- (199.0000,113.6000) -- (199.1000,113.6000) -- (199.2000,113.6000) -- (199.2000,113.6000) -- (199.3000,113.6000) -- (199.3000,113.6000) -- (199.4000,113.6000) -- (199.5000,113.6000) -- (199.5000,113.6000) -- (199.6000,113.6000) -- (199.7000,113.6000) -- (199.7000,113.6000) -- (199.8000,113.6000) -- (199.8000,113.6000) -- (199.9000,113.6000) -- (200.0000,113.6000) -- (200.0000,113.6000) -- (200.1000,113.6000) -- (200.1000,113.6000) -- (200.2000,113.6000) -- (200.3000,113.6000) -- (200.3000,113.6000) -- (200.4000,113.6000) -- (200.5000,113.6000) -- (200.5000,113.6000) -- (200.6000,113.6000) -- (200.6000,113.6000) -- (200.7000,113.6000) -- (200.8000,113.6000) -- (200.8000,113.6000) -- (200.9000,113.6000) -- (200.9000,113.6000) -- (201.0000,113.6000) -- (201.1000,113.6000) -- (201.1000,113.6000) -- (201.2000,113.6000) -- (201.3000,113.6000) -- (201.3000,113.6000) -- (201.4000,113.6000) -- (201.4000,113.6000) -- (201.5000,113.6000) -- (201.6000,113.6000) -- (201.6000,113.6000) -- (201.7000,113.6000) -- (201.7000,113.6000) -- (201.8000,113.6000) -- (201.9000,113.6000) -- (201.9000,113.6000) -- (202.0000,113.6000) -- (202.1000,113.6000) -- (202.1000,113.6000) -- (202.2000,113.6000) -- (202.2000,113.6000) -- (202.3000,113.6000) -- (202.4000,113.6000) -- (202.4000,113.6000) -- (202.5000,113.6000) -- (202.5000,113.6000) -- (202.6000,113.6000) -- (202.7000,113.6000) -- (202.7000,113.6000) -- (202.8000,113.6000) -- (202.9000,113.6000) -- (202.9000,113.6000) -- (203.0000,113.6000) -- (203.0000,113.6000) -- (203.1000,113.6000) -- (203.2000,113.6000) -- (203.2000,113.8000) -- (203.3000,121.5000) -- (203.3000,123.4000) -- (203.4000,123.1000) -- (203.5000,123.1000) -- (203.5000,123.1000) -- (203.6000,123.1000) -- (203.7000,123.1000) -- (203.7000,123.1000) -- (203.8000,123.1000) -- (203.8000,123.1000) -- (203.9000,123.1000) -- (204.0000,123.1000) -- (204.0000,123.1000) -- (204.1000,123.1000) -- (204.1000,123.1000) -- (204.2000,123.1000) -- (204.3000,123.1000) -- (204.3000,123.1000) -- (204.4000,123.1000) -- (204.5000,123.1000) -- (204.5000,123.1000) -- (204.6000,123.1000) -- (204.6000,123.1000) -- (204.7000,123.1000) -- (204.8000,123.1000) -- (204.8000,123.1000) -- (204.9000,122.8000) -- (204.9000,128.6000) -- (205.0000,132.8000) -- (205.1000,134.0000) -- (205.1000,141.8000) -- (205.2000,142.2000) -- (205.3000,142.1000) -- (205.3000,142.1000) -- (205.4000,142.1000) -- (205.4000,142.1000) -- (205.5000,141.8000) -- (205.6000,148.4000) -- (205.6000,152.0000) -- (205.7000,151.6000) -- (205.7000,151.6000) -- (205.8000,151.6000) -- (205.9000,151.6000) -- (205.9000,151.6000) -- (206.0000,151.6000) -- (206.1000,151.6000) -- (206.1000,151.6000) -- (206.2000,151.6000) -- (206.2000,151.9000) -- (206.3000,149.1000) -- (206.4000,142.0000) -- (206.4000,142.1000) -- (206.5000,142.1000) -- (206.5000,142.1000) -- (206.6000,142.1000) -- (206.7000,142.2000) -- (206.7000,142.2000) -- (206.8000,134.6000) -- (206.9000,132.4000) -- (206.9000,132.6000) -- (207.0000,132.6000) -- (207.0000,132.8000) -- (207.1000,140.5000) -- (207.2000,142.4000) -- (207.2000,142.1000) -- (207.3000,142.1000) -- (207.3000,142.1000) -- (207.4000,142.1000) -- (207.5000,142.1000) -- (207.5000,142.1000) -- (207.6000,142.1000) -- (207.7000,142.1000) -- (207.7000,142.1000) -- (207.8000,142.1000) -- (207.8000,142.1000) -- (207.9000,142.1000) -- (208.0000,142.1000) -- (208.0000,142.1000) -- (208.1000,142.1000) -- (208.1000,142.1000) -- (208.2000,142.1000) -- (208.3000,142.1000) -- (208.3000,142.1000) -- (208.4000,142.5000) -- (208.5000,137.1000) -- (208.5000,132.4000) -- (208.6000,131.6000) -- (208.6000,123.7000) -- (208.7000,123.1000) -- (208.8000,123.1000) -- (208.8000,123.3000) -- (208.9000,121.9000) -- (208.9000,114.0000) -- (209.0000,113.6000) -- (209.1000,113.6000) -- (209.1000,113.6000) -- (209.2000,113.6000) -- (209.3000,113.6000) -- (209.3000,113.6000) -- (209.4000,113.6000) -- (209.4000,113.4000) -- (209.5000,115.5000) -- (209.6000,123.1000) -- (209.6000,123.2000) -- (209.7000,123.1000) -- (209.7000,122.9000) -- (209.8000,125.3000) -- (209.9000,132.7000) -- (209.9000,132.5000) -- (210.0000,139.4000) -- (210.1000,142.5000) -- (210.1000,142.1000) -- (210.2000,142.1000) -- (210.2000,142.0000) -- (210.3000,149.3000) -- (210.4000,151.9000) -- (210.4000,151.6000) -- (210.5000,151.6000) -- (210.5000,151.6000) -- (210.6000,151.6000) -- (210.7000,151.6000) -- (210.7000,151.6000) -- (210.8000,151.6000) -- (210.9000,151.6000) -- (210.9000,151.6000) -- (211.0000,151.6000) -- (211.0000,151.6000) -- (211.1000,151.6000) -- (211.2000,151.6000) -- (211.2000,151.6000) -- (211.3000,151.6000) -- (211.3000,151.6000) -- (211.4000,151.6000) -- (211.5000,151.6000) -- (211.5000,151.6000) -- (211.6000,151.6000) -- (211.7000,151.6000) -- (211.7000,151.6000) -- (211.8000,151.6000) -- (211.8000,151.6000) -- (211.9000,151.6000) -- (212.0000,151.6000) -- (212.0000,151.6000) -- (212.1000,151.6000) -- (212.1000,151.6000) -- (212.2000,151.6000) -- (212.3000,151.6000) -- (212.3000,151.6000) -- (212.4000,151.6000) -- (212.5000,151.6000) -- (212.5000,151.6000) -- (212.6000,151.6000) -- (212.6000,151.6000) -- (212.7000,151.6000) -- (212.8000,151.6000) -- (212.8000,151.6000) -- (212.9000,151.6000) -- (212.9000,151.6000) -- (213.0000,151.6000) -- (213.1000,151.6000) -- (213.1000,151.6000) -- (213.2000,151.6000) -- (213.3000,151.6000) -- (213.3000,151.6000) -- (213.4000,151.6000) -- (213.4000,151.6000) -- (213.5000,151.6000) -- (213.6000,151.6000) -- (213.6000,151.6000) -- (213.7000,151.6000) -- (213.7000,151.6000) -- (213.8000,151.6000) -- (213.9000,151.6000) -- (213.9000,151.6000) -- (214.0000,151.6000) -- (214.1000,151.6000) -- (214.1000,151.6000) -- (214.2000,151.6000) -- (214.2000,151.6000) -- (214.3000,151.6000) -- (214.4000,151.6000) -- (214.4000,151.6000) -- (214.5000,151.6000) -- (214.5000,151.6000) -- (214.6000,151.6000) -- (214.7000,151.6000) -- (214.7000,151.6000) -- (214.8000,151.6000) -- (214.9000,151.6000) -- (214.9000,151.6000) -- (215.0000,151.6000) -- (215.0000,151.6000) -- (215.1000,151.6000) -- (215.2000,151.6000) -- (215.2000,151.6000) -- (215.3000,151.6000) -- (215.3000,151.6000) -- (215.4000,151.6000) -- (215.5000,151.6000) -- (215.5000,151.6000) -- (215.6000,151.6000) -- (215.7000,151.6000) -- (215.7000,151.6000) -- (215.8000,151.6000) -- (215.8000,151.6000) -- (215.9000,151.6000) -- (216.0000,151.6000) -- (216.0000,151.6000) -- (216.1000,151.6000) -- (216.1000,151.6000) -- (216.2000,151.6000) -- (216.3000,151.6000) -- (216.3000,151.6000) -- (216.4000,151.6000) -- (216.5000,151.6000) -- (216.5000,151.6000) -- (216.6000,151.6000) -- (216.6000,151.6000) -- (216.7000,151.6000) -- (216.8000,151.6000) -- (216.8000,151.6000) -- (216.9000,151.6000) -- (216.9000,151.6000) -- (217.0000,151.6000) -- (217.1000,151.6000) -- (217.1000,151.6000) -- (217.2000,151.6000) -- (217.3000,151.6000) -- (217.3000,151.6000) -- (217.4000,151.6000) -- (217.4000,151.6000) -- (217.5000,151.6000) -- (217.6000,151.6000) -- (217.6000,151.6000) -- (217.7000,151.6000) -- (217.7000,151.6000) -- (217.8000,151.6000) -- (217.9000,151.6000) -- (217.9000,151.6000) -- (218.0000,151.6000) -- (218.1000,151.6000) -- (218.1000,151.6000) -- (218.2000,151.6000) -- (218.2000,151.6000) -- (218.3000,151.6000) -- (218.4000,151.6000) -- (218.4000,151.6000) -- (218.5000,151.6000) -- (218.5000,151.6000) -- (218.6000,151.6000) -- (218.7000,151.6000) -- (218.7000,151.6000) -- (218.8000,151.6000) -- (218.9000,151.6000) -- (218.9000,151.6000) -- (219.0000,151.6000) -- (219.0000,151.6000) -- (219.1000,151.6000) -- (219.2000,151.6000) -- (219.2000,151.6000) -- (219.3000,151.6000) -- (219.3000,151.6000) -- (219.4000,151.6000) -- (219.5000,151.6000) -- (219.5000,151.6000) -- (219.6000,151.6000) -- (219.7000,151.6000) -- (219.7000,151.6000) -- (219.8000,151.6000) -- (219.8000,151.6000) -- (219.9000,151.6000) -- (220.0000,151.6000) -- (220.0000,151.6000) -- (220.1000,151.6000) -- (220.1000,151.6000) -- (220.2000,151.6000) -- (220.3000,151.6000) -- (220.3000,151.6000) -- (220.4000,151.6000) -- (220.5000,151.6000) -- (220.5000,151.6000) -- (220.6000,151.6000) -- (220.6000,151.6000) -- (220.7000,151.6000) -- (220.8000,151.6000) -- (220.8000,151.6000) -- (220.9000,151.6000) -- (220.9000,151.6000) -- (221.0000,151.6000) -- (221.1000,151.6000) -- (221.1000,151.6000) -- (221.2000,151.6000) -- (221.3000,151.6000) -- (221.3000,151.6000) -- (221.4000,151.6000) -- (221.4000,151.6000) -- (221.5000,151.6000) -- (221.6000,151.6000) -- (221.6000,151.6000) -- (221.7000,151.6000) -- (221.7000,151.6000) -- (221.8000,151.6000) -- (221.9000,151.6000) -- (221.9000,151.6000) -- (222.0000,151.6000) -- (222.1000,151.6000) -- (222.1000,151.6000) -- (222.2000,151.6000) -- (222.2000,151.6000) -- (222.3000,151.6000) -- (222.4000,151.6000) -- (222.4000,151.6000) -- (222.5000,151.6000) -- (222.5000,151.6000) -- (222.6000,151.6000) -- (222.7000,151.6000) -- (222.7000,151.6000) -- (222.8000,151.6000) -- (222.9000,151.6000) -- (222.9000,151.6000) -- (223.0000,151.6000) -- (223.0000,151.6000) -- (223.1000,151.6000) -- (223.2000,151.6000) -- (223.2000,151.6000) -- (223.3000,151.6000) -- (223.3000,151.6000) -- (223.4000,151.6000) -- (223.5000,151.6000) -- (223.5000,151.6000) -- (223.6000,151.6000) -- (223.7000,151.6000) -- (223.7000,151.6000) -- (223.8000,151.6000) -- (223.8000,151.6000) -- (223.9000,151.6000) -- (224.0000,151.6000) -- (224.0000,151.6000) -- (224.1000,151.6000) -- (224.1000,151.6000) -- (224.2000,151.6000) -- (224.3000,151.6000) -- (224.3000,151.6000) -- (224.4000,151.6000) -- (224.5000,151.6000) -- (224.5000,151.6000) -- (224.6000,151.6000) -- (224.6000,151.6000) -- (224.7000,151.6000) -- (224.8000,151.6000) -- (224.8000,151.6000) -- (224.9000,151.6000) -- (224.9000,151.6000) -- (225.0000,151.6000) -- (225.1000,151.6000) -- (225.1000,151.6000) -- (225.2000,151.6000) -- (225.3000,151.6000) -- (225.3000,151.6000) -- (225.4000,151.6000) -- (225.4000,151.6000) -- (225.5000,151.6000) -- (225.6000,151.6000) -- (225.6000,151.6000) -- (225.7000,151.6000) -- (225.7000,151.6000) -- (225.8000,151.6000) -- (225.9000,151.6000) -- (225.9000,151.6000) -- (226.0000,151.6000) -- (226.0000,151.6000) -- (226.1000,151.6000) -- (226.2000,151.6000) -- (226.2000,151.6000) -- (226.3000,151.6000) -- (226.4000,151.6000) -- (226.4000,151.6000) -- (226.5000,151.6000) -- (226.5000,151.6000) -- (226.6000,151.6000) -- (226.7000,151.6000) -- (226.7000,151.6000) -- (226.8000,151.6000) -- (226.8000,151.6000) -- (226.9000,151.6000) -- (227.0000,151.6000) -- (227.0000,151.6000) -- (227.1000,151.6000) -- (227.2000,151.6000) -- (227.2000,151.6000) -- (227.3000,151.6000) -- (227.3000,151.6000) -- (227.4000,151.6000) -- (227.5000,151.6000) -- (227.5000,151.6000) -- (227.6000,151.6000) -- (227.7000,151.6000) -- (227.7000,151.6000) -- (227.8000,151.6000) -- (227.8000,151.6000) -- (227.9000,151.6000) -- (228.0000,151.6000) -- (228.0000,151.6000) -- (228.1000,151.6000) -- (228.1000,151.6000) -- (228.2000,151.6000) -- (228.3000,151.6000) -- (228.3000,151.6000) -- (228.4000,151.6000) -- (228.5000,151.6000) -- (228.5000,151.6000) -- (228.6000,151.6000) -- (228.6000,151.6000) -- (228.7000,151.6000) -- (228.8000,151.6000) -- (228.8000,151.6000) -- (228.9000,151.6000) -- (228.9000,151.6000) -- (229.0000,151.6000) -- (229.1000,151.6000) -- (229.1000,151.6000) -- (229.2000,151.6000) -- (229.2000,151.6000) -- (229.3000,151.6000) -- (229.4000,151.6000) -- (229.4000,151.6000) -- (229.5000,151.6000) -- (229.6000,151.6000) -- (229.6000,151.6000) -- (229.7000,151.6000) -- (229.7000,151.6000) -- (229.8000,151.6000) -- (229.9000,151.6000) -- (229.9000,151.6000) -- (230.0000,151.6000) -- (230.0000,151.6000) -- (230.1000,151.6000) -- (230.2000,151.6000) -- (230.2000,151.6000) -- (230.3000,151.6000) -- (230.4000,151.6000) -- (230.4000,151.6000) -- (230.5000,151.6000) -- (230.5000,151.6000) -- (230.6000,151.6000) -- (230.7000,151.6000) -- (230.7000,151.6000) -- (230.8000,151.6000) -- (230.8000,151.6000) -- (230.9000,151.6000) -- (231.0000,151.6000) -- (231.0000,151.6000) -- (231.1000,151.6000) -- (231.2000,151.6000) -- (231.2000,151.6000) -- (231.3000,151.6000) -- (231.3000,151.6000) -- (231.4000,151.6000) -- (231.5000,151.6000) -- (231.5000,151.6000) -- (231.6000,151.6000) -- (231.6000,151.6000) -- (231.7000,151.6000) -- (231.8000,151.6000) -- (231.8000,151.6000) -- (231.9000,151.6000) -- (232.0000,151.6000) -- (232.0000,151.6000) -- (232.1000,151.6000) -- (232.1000,151.6000) -- (232.2000,151.6000) -- (232.3000,151.6000) -- (232.3000,151.6000) -- (232.4000,151.6000) -- (232.4000,151.6000) -- (232.5000,151.6000) -- (232.6000,151.6000) -- (232.6000,151.6000) -- (232.7000,151.6000) -- (232.8000,151.6000) -- (232.8000,151.6000) -- (232.9000,151.6000) -- (232.9000,151.6000) -- (233.0000,151.6000) -- (233.1000,151.6000) -- (233.1000,151.6000) -- (233.2000,151.6000) -- (233.2000,151.6000) -- (233.3000,151.6000) -- (233.4000,151.6000) -- (233.4000,151.6000) -- (233.5000,151.6000) -- (233.6000,151.6000) -- (233.6000,151.6000) -- (233.7000,151.6000) -- (233.7000,151.6000) -- (233.8000,151.6000) -- (233.9000,151.6000) -- (233.9000,151.6000) -- (234.0000,151.6000) -- (234.0000,151.6000) -- (234.1000,151.6000) -- (234.2000,151.6000) -- (234.2000,151.6000) -- (234.3000,151.6000) -- (234.4000,151.6000) -- (234.4000,151.6000) -- (234.5000,151.6000) -- (234.5000,151.6000) -- (234.6000,151.6000) -- (234.7000,151.6000) -- (234.7000,151.6000) -- (234.8000,151.6000) -- (234.8000,151.6000) -- (234.9000,151.6000) -- (235.0000,151.6000) -- (235.0000,151.6000) -- (235.1000,151.6000) -- (235.2000,151.6000) -- (235.2000,151.6000) -- (235.3000,151.6000) -- (235.3000,151.6000) -- (235.4000,151.6000) -- (235.5000,151.6000) -- (235.5000,151.6000) -- (235.6000,151.6000) -- (235.6000,151.6000) -- (235.7000,151.6000) -- (235.8000,151.6000) -- (235.8000,151.6000) -- (235.9000,151.6000) -- (236.0000,151.6000) -- (236.0000,151.6000) -- (236.1000,151.6000) -- (236.1000,151.6000) -- (236.2000,151.6000) -- (236.3000,151.6000) -- (236.3000,151.6000) -- (236.4000,151.6000) -- (236.4000,151.6000) -- (236.5000,151.6000) -- (236.6000,151.6000) -- (236.6000,151.6000) -- (236.7000,151.6000) -- (236.8000,151.6000) -- (236.8000,151.6000) -- (236.9000,151.6000) -- (236.9000,151.6000) -- (237.0000,151.6000) -- (237.1000,151.6000) -- (237.1000,151.6000) -- (237.2000,151.6000) -- (237.2000,151.6000) -- (237.3000,151.6000) -- (237.4000,151.6000) -- (237.4000,151.6000) -- (237.5000,151.6000) -- (237.6000,151.6000) -- (237.6000,151.6000) -- (237.7000,151.6000) -- (237.7000,151.6000) -- (237.8000,151.6000) -- (237.9000,151.6000) -- (237.9000,151.6000) -- (238.0000,151.6000) -- (238.0000,151.6000) -- (238.1000,151.6000) -- (238.2000,151.6000) -- (238.2000,151.6000) -- (238.3000,151.6000) -- (238.4000,151.6000) -- (238.4000,151.6000) -- (238.5000,151.6000) -- (238.5000,151.6000) -- (238.6000,151.6000) -- (238.7000,151.6000) -- (238.7000,151.6000) -- (238.8000,151.6000) -- (238.8000,151.6000) -- (238.9000,151.6000) -- (239.0000,151.6000) -- (239.0000,151.6000) -- (239.1000,151.6000) -- (239.2000,151.6000) -- (239.2000,151.6000) -- (239.3000,151.6000) -- (239.3000,151.6000) -- (239.4000,151.6000) -- (239.5000,151.6000) -- (239.5000,151.6000) -- (239.6000,151.6000) -- (239.6000,151.6000) -- (239.7000,151.6000) -- (239.8000,151.6000) -- (239.8000,151.6000) -- (239.9000,151.6000) -- (240.0000,151.6000) -- (240.0000,151.6000) -- (240.1000,151.6000) -- (240.1000,151.6000) -- (240.2000,151.6000) -- (240.3000,151.6000) -- (240.3000,151.6000) -- (240.4000,151.6000) -- (240.4000,151.6000) -- (240.5000,151.6000) -- (240.6000,151.6000) -- (240.6000,151.6000) -- (240.7000,151.6000) -- (240.8000,151.6000) -- (240.8000,151.6000) -- (240.9000,151.6000) -- (240.9000,151.6000) -- (241.0000,151.6000) -- (241.1000,151.6000) -- (241.1000,151.6000) -- (241.2000,151.6000) -- (241.2000,151.6000) -- (241.3000,151.6000) -- (241.4000,151.6000) -- (241.4000,151.6000) -- (241.5000,151.6000) -- (241.6000,151.6000) -- (241.6000,151.6000) -- (241.7000,151.6000) -- (241.7000,151.6000) -- (241.8000,151.6000) -- (241.9000,151.6000) -- (241.9000,151.6000) -- (242.0000,151.6000) -- (242.0000,151.6000) -- (242.1000,151.6000) -- (242.2000,151.6000) -- (242.2000,151.6000) -- (242.3000,151.6000) -- (242.4000,151.6000) -- (242.4000,151.6000) -- (242.5000,151.6000) -- (242.5000,151.6000) -- (242.6000,151.6000) -- (242.7000,151.6000) -- (242.7000,151.6000) -- (242.8000,151.6000) -- (242.8000,151.6000) -- (242.9000,151.6000) -- (243.0000,151.6000) -- (243.0000,151.6000) -- (243.1000,151.6000) -- (243.2000,151.6000) -- (243.2000,151.6000) -- (243.3000,151.6000) -- (243.3000,151.6000) -- (243.4000,151.6000) -- (243.5000,151.6000) -- (243.5000,151.6000) -- (243.6000,151.6000) -- (243.6000,151.6000) -- (243.7000,151.6000) -- (243.8000,151.6000) -- (243.8000,151.6000) -- (243.9000,151.6000) -- (244.0000,151.6000) -- (244.0000,151.6000) -- (244.1000,151.6000) -- (244.1000,151.6000) -- (244.2000,151.6000) -- (244.3000,151.6000) -- (244.3000,151.6000) -- (244.4000,151.6000) -- (244.4000,151.6000) -- (244.5000,151.6000) -- (244.6000,151.6000) -- (244.6000,151.6000) -- (244.7000,151.6000) -- (244.8000,151.6000) -- (244.8000,151.6000) -- (244.9000,151.6000) -- (244.9000,151.6000) -- (245.0000,151.6000) -- (245.1000,151.6000) -- (245.1000,151.6000) -- (245.2000,151.6000) -- (245.2000,151.6000) -- (245.3000,151.6000) -- (245.4000,151.6000) -- (245.4000,151.6000) -- (245.5000,151.6000) -- (245.6000,151.6000) -- (245.6000,151.6000) -- (245.7000,151.6000) -- (245.7000,151.6000) -- (245.8000,151.6000) -- (245.9000,151.6000) -- (245.9000,151.6000) -- (246.0000,151.6000) -- (246.0000,151.6000) -- (246.1000,151.6000) -- (246.2000,151.6000) -- (246.2000,151.6000) -- (246.3000,151.6000);



  \end{scope}
  \begin{scope}[scale=1.006,draw=black,line join=bevel,line cap=rect,line width=0.800pt]
  \end{scope}
  \begin{scope}[scale=1.006,draw=black,line join=bevel,line cap=rect,line width=0.800pt]
  \end{scope}
  \begin{scope}[scale=1.006,draw=black,line join=round,line cap=round,line width=0.480pt]
    \path[draw] (184.5000,108.5000) -- (184.5000,156.5000) -- (246.5000,156.5000) -- (246.5000,108.5000) -- (184.5000,108.5000);



  \end{scope}
  \begin{scope}[scale=1.006,draw=ca0a0a4,dash pattern=on 0.40pt off 0.80pt,line join=round,line cap=round,line width=0.400pt]
    \path[draw] (184.5000,200.5000) -- (246.5000,200.5000);



  \end{scope}
  \begin{scope}[scale=1.006,draw=black,line join=round,line cap=round,line width=0.480pt]
    \path[draw] (184.5000,200.5000) -- (186.5000,200.5000);



    \path[draw] (246.5000,200.5000) -- (245.5000,200.5000);



  \end{scope}
  \begin{scope}[scale=1.006,draw=black,line join=bevel,line cap=rect,line width=0.800pt]
  \end{scope}
  \begin{scope}[cm={{1.00588,0.0,0.0,1.00588,(181.059,201.176)}},draw=black,line join=bevel,line cap=rect,line width=0.800pt]
  \end{scope}
  \begin{scope}[cm={{1.00588,0.0,0.0,1.00588,(181.059,201.176)}},draw=black,line join=bevel,line cap=rect,line width=0.800pt]
  \end{scope}
  \begin{scope}[cm={{1.00588,0.0,0.0,1.00588,(181.059,201.176)}},draw=black,line join=bevel,line cap=rect,line width=0.800pt]
  \end{scope}
  \begin{scope}[cm={{1.00588,0.0,0.0,1.00588,(181.059,201.176)}},draw=black,line join=bevel,line cap=rect,line width=0.800pt]
  \end{scope}
  \begin{scope}[cm={{1.00588,0.0,0.0,1.00588,(181.059,201.176)}},draw=black,line join=bevel,line cap=rect,line width=0.800pt]
  \end{scope}
  \begin{scope}[cm={{1.00588,0.0,0.0,1.00588,(181.059,201.176)}},draw=black,line join=bevel,line cap=rect,line width=0.800pt]
  \end{scope}
  \begin{scope}[scale=1.006,draw=black,line join=bevel,line cap=rect,line width=0.800pt]
  \end{scope}
  \begin{scope}[scale=1.006,draw=ca0a0a4,dash pattern=on 0.40pt off 0.80pt,line join=round,line cap=round,line width=0.400pt]
    \path[draw] (184.5000,180.5000) -- (246.5000,180.5000);



  \end{scope}
  \begin{scope}[scale=1.006,draw=black,line join=round,line cap=round,line width=0.480pt]
    \path[draw] (184.5000,180.5000) -- (186.5000,180.5000);



    \path[draw] (246.5000,180.5000) -- (245.5000,180.5000);



  \end{scope}
  \begin{scope}[scale=1.006,draw=black,line join=bevel,line cap=rect,line width=0.800pt]
  \end{scope}
  \begin{scope}[cm={{1.00588,0.0,0.0,1.00588,(181.059,181.059)}},draw=black,line join=bevel,line cap=rect,line width=0.800pt]
  \end{scope}
  \begin{scope}[cm={{1.00588,0.0,0.0,1.00588,(181.059,181.059)}},draw=black,line join=bevel,line cap=rect,line width=0.800pt]
  \end{scope}
  \begin{scope}[cm={{1.00588,0.0,0.0,1.00588,(181.059,181.059)}},draw=black,line join=bevel,line cap=rect,line width=0.800pt]
  \end{scope}
  \begin{scope}[cm={{1.00588,0.0,0.0,1.00588,(181.059,181.059)}},draw=black,line join=bevel,line cap=rect,line width=0.800pt]
  \end{scope}
  \begin{scope}[cm={{1.00588,0.0,0.0,1.00588,(181.059,181.059)}},draw=black,line join=bevel,line cap=rect,line width=0.800pt]
  \end{scope}
  \begin{scope}[cm={{1.00588,0.0,0.0,1.00588,(181.059,181.059)}},draw=black,line join=bevel,line cap=rect,line width=0.800pt]
  \end{scope}
  \begin{scope}[scale=1.006,draw=black,line join=bevel,line cap=rect,line width=0.800pt]
  \end{scope}
  \begin{scope}[scale=1.006,draw=ca0a0a4,dash pattern=on 0.40pt off 0.80pt,line join=round,line cap=round,line width=0.400pt]
    \path[draw] (184.5000,160.5000) -- (246.5000,160.5000);



  \end{scope}
  \begin{scope}[scale=1.006,draw=black,line join=round,line cap=round,line width=0.480pt]
    \path[draw] (184.5000,160.5000) -- (186.5000,160.5000);



    \path[draw] (246.5000,160.5000) -- (245.5000,160.5000);



  \end{scope}
  \begin{scope}[scale=1.006,draw=black,line join=bevel,line cap=rect,line width=0.800pt]
  \end{scope}
  \begin{scope}[cm={{1.00588,0.0,0.0,1.00588,(181.059,161.947)}},draw=black,line join=bevel,line cap=rect,line width=0.800pt]
  \end{scope}
  \begin{scope}[cm={{1.00588,0.0,0.0,1.00588,(181.059,161.947)}},draw=black,line join=bevel,line cap=rect,line width=0.800pt]
  \end{scope}
  \begin{scope}[cm={{1.00588,0.0,0.0,1.00588,(181.059,161.947)}},draw=black,line join=bevel,line cap=rect,line width=0.800pt]
  \end{scope}
  \begin{scope}[cm={{1.00588,0.0,0.0,1.00588,(181.059,161.947)}},draw=black,line join=bevel,line cap=rect,line width=0.800pt]
  \end{scope}
  \begin{scope}[cm={{1.00588,0.0,0.0,1.00588,(181.059,161.947)}},draw=black,line join=bevel,line cap=rect,line width=0.800pt]
  \end{scope}
  \begin{scope}[cm={{1.00588,0.0,0.0,1.00588,(181.059,161.947)}},draw=black,line join=bevel,line cap=rect,line width=0.800pt]
  \end{scope}
  \begin{scope}[scale=1.006,draw=black,line join=bevel,line cap=rect,line width=0.800pt]
  \end{scope}
  \begin{scope}[scale=1.006,draw=ca0a0a4,dash pattern=on 0.40pt off 0.80pt,line join=round,line cap=round,line width=0.400pt]
    \path[draw] (184.5000,204.5000) -- (184.5000,156.5000);



  \end{scope}
  \begin{scope}[scale=1.006,draw=black,line join=round,line cap=round,line width=0.480pt]
    \path[draw] (184.5000,204.5000) -- (184.5000,203.5000);



    \path[draw] (184.5000,156.5000) -- (184.5000,157.5000);



  \end{scope}
  \begin{scope}[scale=1.006,draw=black,line join=bevel,line cap=rect,line width=0.800pt]
  \end{scope}
  \begin{scope}[cm={{1.00588,0.0,0.0,1.00588,(183.071,220.288)}},draw=black,line join=bevel,line cap=rect,line width=0.800pt]
  \end{scope}
  \begin{scope}[cm={{1.00588,0.0,0.0,1.00588,(183.071,220.288)}},draw=black,line join=bevel,line cap=rect,line width=0.800pt]
  \end{scope}
  \begin{scope}[cm={{1.00588,0.0,0.0,1.00588,(183.071,220.288)}},draw=black,line join=bevel,line cap=rect,line width=0.800pt]
  \end{scope}
  \begin{scope}[cm={{1.00588,0.0,0.0,1.00588,(183.071,220.288)}},draw=black,line join=bevel,line cap=rect,line width=0.800pt]
  \end{scope}
  \begin{scope}[cm={{1.00588,0.0,0.0,1.00588,(183.071,220.288)}},draw=black,line join=bevel,line cap=rect,line width=0.800pt]
  \end{scope}
  \begin{scope}[cm={{1.00588,0.0,0.0,1.00588,(183.071,217.288)}},draw=black,line join=bevel,line cap=rect,line width=0.800pt]
    \path[fill=black] (0.0000,0.0000) node[above right] (text1416) {\scriptsize 0};



  \end{scope}
  \begin{scope}[cm={{1.00588,0.0,0.0,1.00588,(183.071,220.288)}},draw=black,line join=bevel,line cap=rect,line width=0.800pt]
  \end{scope}
  \begin{scope}[scale=1.006,draw=black,line join=bevel,line cap=rect,line width=0.800pt]
  \end{scope}
  \begin{scope}[scale=1.006,draw=ca0a0a4,dash pattern=on 0.40pt off 0.80pt,line join=round,line cap=round,line width=0.400pt]
    \path[draw] (203.5000,204.5000) -- (203.5000,156.5000);



  \end{scope}
  \begin{scope}[scale=1.006,draw=black,line join=round,line cap=round,line width=0.480pt]
    \path[draw] (203.5000,204.5000) -- (203.5000,203.5000);



    \path[draw] (203.5000,156.5000) -- (203.5000,157.5000);



  \end{scope}
  \begin{scope}[scale=1.006,draw=black,line join=bevel,line cap=rect,line width=0.800pt]
  \end{scope}
  \begin{scope}[cm={{1.00588,0.0,0.0,1.00588,(201.176,220.288)}},draw=black,line join=bevel,line cap=rect,line width=0.800pt]
  \end{scope}
  \begin{scope}[cm={{1.00588,0.0,0.0,1.00588,(201.176,220.288)}},draw=black,line join=bevel,line cap=rect,line width=0.800pt]
  \end{scope}
  \begin{scope}[cm={{1.00588,0.0,0.0,1.00588,(201.176,220.288)}},draw=black,line join=bevel,line cap=rect,line width=0.800pt]
  \end{scope}
  \begin{scope}[cm={{1.00588,0.0,0.0,1.00588,(201.176,220.288)}},draw=black,line join=bevel,line cap=rect,line width=0.800pt]
  \end{scope}
  \begin{scope}[cm={{1.00588,0.0,0.0,1.00588,(201.176,220.288)}},draw=black,line join=bevel,line cap=rect,line width=0.800pt]
  \end{scope}
  \begin{scope}[cm={{1.00588,0.0,0.0,1.00588,(201.176,217.288)}},draw=black,line join=bevel,line cap=rect,line width=0.800pt]
    \path[fill=black] (0.0000,0.0000) node[above right] (text1446) {\scriptsize 2};



  \end{scope}
  \begin{scope}[cm={{1.00588,0.0,0.0,1.00588,(201.176,220.288)}},draw=black,line join=bevel,line cap=rect,line width=0.800pt]
  \end{scope}
  \begin{scope}[scale=1.006,draw=black,line join=bevel,line cap=rect,line width=0.800pt]
  \end{scope}
  \begin{scope}[scale=1.006,draw=ca0a0a4,dash pattern=on 0.40pt off 0.80pt,line join=round,line cap=round,line width=0.400pt]
    \path[draw] (221.5000,204.5000) -- (221.5000,156.5000);



  \end{scope}
  \begin{scope}[scale=1.006,draw=black,line join=round,line cap=round,line width=0.480pt]
    \path[draw] (221.5000,204.5000) -- (221.5000,203.5000);



    \path[draw] (221.5000,156.5000) -- (221.5000,157.5000);



  \end{scope}
  \begin{scope}[scale=1.006,draw=black,line join=bevel,line cap=rect,line width=0.800pt]
  \end{scope}
  \begin{scope}[cm={{1.00588,0.0,0.0,1.00588,(219.785,220.288)}},draw=black,line join=bevel,line cap=rect,line width=0.800pt]
  \end{scope}
  \begin{scope}[cm={{1.00588,0.0,0.0,1.00588,(219.785,220.288)}},draw=black,line join=bevel,line cap=rect,line width=0.800pt]
  \end{scope}
  \begin{scope}[cm={{1.00588,0.0,0.0,1.00588,(219.785,220.288)}},draw=black,line join=bevel,line cap=rect,line width=0.800pt]
  \end{scope}
  \begin{scope}[cm={{1.00588,0.0,0.0,1.00588,(219.785,220.288)}},draw=black,line join=bevel,line cap=rect,line width=0.800pt]
  \end{scope}
  \begin{scope}[cm={{1.00588,0.0,0.0,1.00588,(219.785,220.288)}},draw=black,line join=bevel,line cap=rect,line width=0.800pt]
  \end{scope}
  \begin{scope}[cm={{1.00588,0.0,0.0,1.00588,(219.785,217.288)}},draw=black,line join=bevel,line cap=rect,line width=0.800pt]
    \path[fill=black] (0.0000,0.0000) node[above right] (text1476) {\scriptsize 4};



  \end{scope}
  \begin{scope}[cm={{1.00588,0.0,0.0,1.00588,(219.785,220.288)}},draw=black,line join=bevel,line cap=rect,line width=0.800pt]
  \end{scope}
  \begin{scope}[scale=1.006,draw=black,line join=bevel,line cap=rect,line width=0.800pt]
  \end{scope}
  \begin{scope}[scale=1.006,draw=ca0a0a4,dash pattern=on 0.40pt off 0.80pt,line join=round,line cap=round,line width=0.400pt]
    \path[draw] (239.5000,204.5000) -- (239.5000,156.5000);



  \end{scope}
  \begin{scope}[scale=1.006,draw=black,line join=round,line cap=round,line width=0.480pt]
    \path[draw] (239.5000,204.5000) -- (239.5000,203.5000);



    \path[draw] (239.5000,156.5000) -- (239.5000,157.5000);



  \end{scope}
  \begin{scope}[scale=1.006,draw=black,line join=bevel,line cap=rect,line width=0.800pt]
  \end{scope}
  \begin{scope}[cm={{1.00588,0.0,0.0,1.00588,(238.394,220.288)}},draw=black,line join=bevel,line cap=rect,line width=0.800pt]
  \end{scope}
  \begin{scope}[cm={{1.00588,0.0,0.0,1.00588,(238.394,220.288)}},draw=black,line join=bevel,line cap=rect,line width=0.800pt]
  \end{scope}
  \begin{scope}[cm={{1.00588,0.0,0.0,1.00588,(238.394,220.288)}},draw=black,line join=bevel,line cap=rect,line width=0.800pt]
  \end{scope}
  \begin{scope}[cm={{1.00588,0.0,0.0,1.00588,(238.394,220.288)}},draw=black,line join=bevel,line cap=rect,line width=0.800pt]
  \end{scope}
  \begin{scope}[cm={{1.00588,0.0,0.0,1.00588,(238.394,220.288)}},draw=black,line join=bevel,line cap=rect,line width=0.800pt]
  \end{scope}
  \begin{scope}[cm={{1.00588,0.0,0.0,1.00588,(238.394,217.288)}},draw=black,line join=bevel,line cap=rect,line width=0.800pt]
    \path[fill=black] (0.0000,0.0000) node[above right] (text1506) {\scriptsize 6};



  \end{scope}
  \begin{scope}[cm={{1.00588,0.0,0.0,1.00588,(238.394,220.288)}},draw=black,line join=bevel,line cap=rect,line width=0.800pt]
  \end{scope}
  \begin{scope}[scale=1.006,draw=black,line join=bevel,line cap=rect,line width=0.800pt]
  \end{scope}
  \begin{scope}[scale=1.006,draw=black,line join=round,line cap=round,line width=0.480pt]
    \path[draw] (246.5000,200.5000) -- (245.5000,200.5000);



  \end{scope}
  \begin{scope}[scale=1.006,draw=black,line join=bevel,line cap=rect,line width=0.800pt]
  \end{scope}
  \begin{scope}[cm={{1.00588,0.0,0.0,1.00588,(253.482,205.2)}},draw=black,line join=bevel,line cap=rect,line width=0.800pt]
  \end{scope}
  \begin{scope}[cm={{1.00588,0.0,0.0,1.00588,(253.482,205.2)}},draw=black,line join=bevel,line cap=rect,line width=0.800pt]
  \end{scope}
  \begin{scope}[cm={{1.00588,0.0,0.0,1.00588,(253.482,205.2)}},draw=black,line join=bevel,line cap=rect,line width=0.800pt]
  \end{scope}
  \begin{scope}[cm={{1.00588,0.0,0.0,1.00588,(253.482,205.2)}},draw=black,line join=bevel,line cap=rect,line width=0.800pt]
  \end{scope}
  \begin{scope}[cm={{1.00588,0.0,0.0,1.00588,(253.482,205.2)}},draw=black,line join=bevel,line cap=rect,line width=0.800pt]
  \end{scope}
  \begin{scope}[cm={{1.00588,0.0,0.0,1.00588,(250.982,205.2)}},draw=black,line join=bevel,line cap=rect,line width=0.800pt]
    \path[fill=black] (0.0000,0.0000) node[above right] (text1530) {\scriptsize -1};



  \end{scope}
  \begin{scope}[cm={{1.00588,0.0,0.0,1.00588,(253.482,205.2)}},draw=black,line join=bevel,line cap=rect,line width=0.800pt]
  \end{scope}
  \begin{scope}[scale=1.006,draw=black,line join=bevel,line cap=rect,line width=0.800pt]
  \end{scope}
  \begin{scope}[scale=1.006,draw=black,line join=round,line cap=round,line width=0.480pt]
    \path[draw] (246.5000,180.5000) -- (245.5000,180.5000);



  \end{scope}
  \begin{scope}[scale=1.006,draw=black,line join=bevel,line cap=rect,line width=0.800pt]
  \end{scope}
  \begin{scope}[cm={{1.00588,0.0,0.0,1.00588,(253.482,185.082)}},draw=black,line join=bevel,line cap=rect,line width=0.800pt]
  \end{scope}
  \begin{scope}[cm={{1.00588,0.0,0.0,1.00588,(253.482,185.082)}},draw=black,line join=bevel,line cap=rect,line width=0.800pt]
  \end{scope}
  \begin{scope}[cm={{1.00588,0.0,0.0,1.00588,(253.482,185.082)}},draw=black,line join=bevel,line cap=rect,line width=0.800pt]
  \end{scope}
  \begin{scope}[cm={{1.00588,0.0,0.0,1.00588,(253.482,185.082)}},draw=black,line join=bevel,line cap=rect,line width=0.800pt]
  \end{scope}
  \begin{scope}[cm={{1.00588,0.0,0.0,1.00588,(253.482,185.082)}},draw=black,line join=bevel,line cap=rect,line width=0.800pt]
  \end{scope}
  \begin{scope}[cm={{1.00588,0.0,0.0,1.00588,(250.982,185.082)}},draw=black,line join=bevel,line cap=rect,line width=0.800pt]
    \path[fill=black] (0.0000,0.0000) node[above right] (text1554) {\scriptsize -.5};



  \end{scope}
  \begin{scope}[cm={{1.00588,0.0,0.0,1.00588,(253.482,185.082)}},draw=black,line join=bevel,line cap=rect,line width=0.800pt]
  \end{scope}
  \begin{scope}[scale=1.006,draw=black,line join=bevel,line cap=rect,line width=0.800pt]
  \end{scope}
  \begin{scope}[scale=1.006,draw=black,line join=round,line cap=round,line width=0.480pt]
    \path[draw] (246.5000,160.5000) -- (245.5000,160.5000);



  \end{scope}
  \begin{scope}[scale=1.006,draw=black,line join=bevel,line cap=rect,line width=0.800pt]
  \end{scope}
  \begin{scope}[cm={{1.00588,0.0,0.0,1.00588,(253.482,165.971)}},draw=black,line join=bevel,line cap=rect,line width=0.800pt]
  \end{scope}
  \begin{scope}[cm={{1.00588,0.0,0.0,1.00588,(253.482,165.971)}},draw=black,line join=bevel,line cap=rect,line width=0.800pt]
  \end{scope}
  \begin{scope}[cm={{1.00588,0.0,0.0,1.00588,(253.482,165.971)}},draw=black,line join=bevel,line cap=rect,line width=0.800pt]
  \end{scope}
  \begin{scope}[cm={{1.00588,0.0,0.0,1.00588,(253.482,165.971)}},draw=black,line join=bevel,line cap=rect,line width=0.800pt]
  \end{scope}
  \begin{scope}[cm={{1.00588,0.0,0.0,1.00588,(253.482,165.971)}},draw=black,line join=bevel,line cap=rect,line width=0.800pt]
  \end{scope}
  \begin{scope}[cm={{1.00588,0.0,0.0,1.00588,(250.982,165.971)}},draw=black,line join=bevel,line cap=rect,line width=0.800pt]
    \path[fill=black] (0.0000,0.0000) node[above right] (text1578) {\scriptsize 0};



  \end{scope}
  \begin{scope}[cm={{1.00588,0.0,0.0,1.00588,(253.482,165.971)}},draw=black,line join=bevel,line cap=rect,line width=0.800pt]
  \end{scope}
  \begin{scope}[scale=1.006,draw=black,line join=bevel,line cap=rect,line width=0.800pt]
  \end{scope}
  \begin{scope}[scale=1.006,draw=black,line join=round,line cap=round,line width=0.480pt]
    \path[draw] (184.5000,156.5000) -- (184.5000,204.5000) -- (246.5000,204.5000) -- (246.5000,156.5000) -- (184.5000,156.5000);



  \end{scope}
  \begin{scope}[scale=1.006,draw=black,line join=bevel,line cap=rect,line width=0.800pt]
  \end{scope}
  \begin{scope}[cm={{0.0,-1.00588,1.00588,0.0,(276.618,186.591)}},draw=black,line join=bevel,line cap=rect,line width=0.800pt]
  \end{scope}
  \begin{scope}[cm={{0.0,-1.00588,1.00588,0.0,(276.618,186.591)}},draw=black,line join=bevel,line cap=rect,line width=0.800pt]
  \end{scope}
  \begin{scope}[cm={{0.0,-1.00588,1.00588,0.0,(276.618,186.591)}},draw=black,line join=bevel,line cap=rect,line width=0.800pt]
  \end{scope}
  \begin{scope}[cm={{0.0,-1.00588,1.00588,0.0,(276.618,186.591)}},draw=black,line join=bevel,line cap=rect,line width=0.800pt]
  \end{scope}
  \begin{scope}[cm={{0.0,-1.00588,1.00588,0.0,(276.618,186.591)}},draw=black,line join=bevel,line cap=rect,line width=0.800pt]
  \end{scope}
  \begin{scope}[cm={{0.0,-1.00588,1.00588,0.0,(263.618,186.591)}},draw=black,line join=bevel,line cap=rect,line width=0.800pt]
    \path[fill=black] (0.0000,0.0000) node[above right] (text1602) {\rotatebox{90}{\scriptsize k$c_{1,1}$}};



  \end{scope}
  \begin{scope}[cm={{0.0,-1.00588,1.00588,0.0,(276.618,186.591)}},draw=black,line join=bevel,line cap=rect,line width=0.800pt]
  \end{scope}
  \begin{scope}[scale=1.006,draw=black,line join=bevel,line cap=rect,line width=0.800pt]
  \end{scope}
  \begin{scope}[scale=1.006,draw=black,line join=bevel,line cap=rect,line width=0.800pt]
  \end{scope}
  \begin{scope}[scale=1.006,draw=black,line join=bevel,line cap=rect,line width=0.800pt]
  \end{scope}
  \begin{scope}[scale=1.006,draw=black,line join=round,line cap=round,line width=0.480pt]
    \path[draw] (184.8000,160.5000) -- (184.8000,160.5000) -- (184.9000,160.5000) -- (185.0000,160.5000) -- (185.0000,160.5000) -- (185.1000,160.5000) -- (185.1000,160.5000) -- (185.2000,160.5000) -- (185.3000,160.5000) -- (185.3000,160.5000) -- (185.4000,160.5000) -- (185.4000,160.5000) -- (185.5000,160.5000) -- (185.6000,160.5000) -- (185.6000,160.5000) -- (185.7000,160.5000) -- (185.8000,160.5000) -- (185.8000,160.5000) -- (185.9000,160.5000) -- (185.9000,160.5000) -- (186.0000,160.5000) -- (186.1000,160.5000) -- (186.1000,160.5000) -- (186.2000,160.5000) -- (186.2000,160.5000) -- (186.3000,160.5000) -- (186.4000,160.5000) -- (186.4000,160.5000) -- (186.5000,160.5000) -- (186.6000,160.5000) -- (186.6000,160.5000) -- (186.7000,160.5000) -- (186.7000,160.5000) -- (186.8000,160.5000) -- (186.9000,160.5000) -- (186.9000,160.5000) -- (187.0000,160.5000) -- (187.0000,160.5000) -- (187.1000,160.5000) -- (187.2000,160.5000) -- (187.2000,160.5000) -- (187.3000,160.5000) -- (187.4000,160.5000) -- (187.4000,160.5000) -- (187.5000,160.5000) -- (187.5000,160.5000) -- (187.6000,160.5000) -- (187.7000,160.5000) -- (187.7000,160.5000) -- (187.8000,160.5000) -- (187.8000,160.5000) -- (187.9000,160.5000) -- (188.0000,160.5000) -- (188.0000,160.5000) -- (188.1000,160.5000) -- (188.2000,160.5000) -- (188.2000,160.5000) -- (188.3000,160.5000) -- (188.3000,160.5000) -- (188.4000,160.5000) -- (188.5000,160.5000) -- (188.5000,160.5000) -- (188.6000,160.5000) -- (188.6000,160.5000) -- (188.7000,160.5000) -- (188.8000,160.5000) -- (188.8000,160.5000) -- (188.9000,160.5000) -- (189.0000,160.5000) -- (189.0000,160.5000) -- (189.1000,160.5000) -- (189.1000,160.5000) -- (189.2000,160.5000) -- (189.3000,160.5000) -- (189.3000,160.5000) -- (189.4000,160.5000) -- (189.4000,160.5000) -- (189.5000,160.5000) -- (189.6000,160.5000) -- (189.6000,160.5000) -- (189.7000,160.5000) -- (189.8000,160.5000) -- (189.8000,160.5000) -- (189.9000,160.5000) -- (189.9000,160.5000) -- (190.0000,160.5000) -- (190.1000,160.5000) -- (190.1000,160.5000) -- (190.2000,160.5000) -- (190.2000,160.5000) -- (190.3000,160.5000) -- (190.4000,160.5000) -- (190.4000,160.5000) -- (190.5000,160.5000) -- (190.6000,160.5000) -- (190.6000,160.5000) -- (190.7000,160.5000) -- (190.7000,160.5000) -- (190.8000,160.5000) -- (190.9000,160.5000) -- (190.9000,160.5000) -- (191.0000,160.5000) -- (191.0000,160.5000) -- (191.1000,160.5000) -- (191.2000,160.5000) -- (191.2000,160.5000) -- (191.3000,160.5000) -- (191.4000,160.5000) -- (191.4000,160.5000) -- (191.5000,160.5000) -- (191.5000,160.5000) -- (191.6000,160.5000) -- (191.7000,160.5000) -- (191.7000,160.5000) -- (191.8000,160.5000) -- (191.8000,160.5000) -- (191.9000,160.5000) -- (192.0000,160.5000) -- (192.0000,160.5000) -- (192.1000,160.5000) -- (192.2000,160.5000) -- (192.2000,160.5000) -- (192.3000,160.5000) -- (192.3000,160.5000) -- (192.4000,160.5000) -- (192.5000,160.5000) -- (192.5000,160.5000) -- (192.6000,160.5000) -- (192.6000,160.5000) -- (192.7000,160.5000) -- (192.8000,160.5000) -- (192.8000,160.5000) -- (192.9000,160.5000) -- (193.0000,160.5000) -- (193.0000,160.5000) -- (193.1000,160.5000) -- (193.1000,160.5000) -- (193.2000,160.5000) -- (193.3000,160.5000) -- (193.3000,160.5000) -- (193.4000,160.5000) -- (193.4000,160.5000) -- (193.5000,160.5000) -- (193.6000,160.5000) -- (193.6000,160.5000) -- (193.7000,160.5000) -- (193.8000,160.5000) -- (193.8000,160.5000) -- (193.9000,160.5000) -- (193.9000,160.5000) -- (194.0000,160.5000) -- (194.1000,160.5000) -- (194.1000,160.5000) -- (194.2000,160.5000) -- (194.2000,160.5000) -- (194.3000,160.5000) -- (194.4000,160.5000) -- (194.4000,160.5000) -- (194.5000,160.5000) -- (194.6000,160.5000) -- (194.6000,160.5000) -- (194.7000,160.5000) -- (194.7000,160.5000) -- (194.8000,160.5000) -- (194.9000,160.5000) -- (194.9000,160.5000) -- (195.0000,160.5000) -- (195.0000,160.5000) -- (195.1000,160.5000) -- (195.2000,160.5000) -- (195.2000,160.5000) -- (195.3000,160.5000) -- (195.4000,160.5000) -- (195.4000,160.5000) -- (195.5000,160.5000) -- (195.5000,160.5000) -- (195.6000,160.5000) -- (195.7000,160.5000) -- (195.7000,160.5000) -- (195.8000,160.5000) -- (195.8000,160.5000) -- (195.9000,160.5000) -- (196.0000,160.5000) -- (196.0000,160.5000) -- (196.1000,160.5000) -- (196.2000,160.5000) -- (196.2000,160.5000) -- (196.3000,160.5000) -- (196.3000,160.5000) -- (196.4000,160.5000) -- (196.5000,160.5000) -- (196.5000,160.5000) -- (196.6000,160.5000) -- (196.6000,160.5000) -- (196.7000,160.5000) -- (196.8000,160.5000) -- (196.8000,160.5000) -- (196.9000,160.5000) -- (196.9000,160.5000) -- (197.0000,160.5000) -- (197.1000,160.5000) -- (197.1000,160.5000) -- (197.2000,160.5000) -- (197.3000,160.5000) -- (197.3000,160.5000) -- (197.4000,160.5000) -- (197.4000,160.5000) -- (197.5000,160.5000) -- (197.6000,160.5000) -- (197.6000,160.5000) -- (197.7000,160.5000) -- (197.7000,160.5000) -- (197.8000,160.5000) -- (197.9000,160.5000) -- (197.9000,160.5000) -- (198.0000,160.5000) -- (198.1000,160.5000) -- (198.1000,160.5000) -- (198.2000,160.5000) -- (198.2000,160.5000) -- (198.3000,160.5000) -- (198.4000,160.5000) -- (198.4000,160.5000) -- (198.5000,160.5000) -- (198.5000,160.5000) -- (198.6000,160.5000) -- (198.7000,160.5000) -- (198.7000,160.5000) -- (198.8000,160.5000) -- (198.9000,159.9000) -- (198.9000,165.0000) -- (199.0000,180.4000) -- (199.0000,180.3000) -- (199.1000,180.2000) -- (199.2000,180.2000) -- (199.2000,180.2000) -- (199.3000,180.2000) -- (199.3000,180.2000) -- (199.4000,180.2000) -- (199.5000,180.2000) -- (199.5000,180.2000) -- (199.6000,180.2000) -- (199.7000,180.2000) -- (199.7000,180.2000) -- (199.8000,180.2000) -- (199.8000,180.2000) -- (199.9000,180.2000) -- (200.0000,180.2000) -- (200.0000,180.2000) -- (200.1000,180.2000) -- (200.1000,180.2000) -- (200.2000,180.2000) -- (200.3000,180.2000) -- (200.3000,180.2000) -- (200.4000,180.2000) -- (200.5000,180.2000) -- (200.5000,180.2000) -- (200.6000,180.2000) -- (200.6000,180.2000) -- (200.7000,180.2000) -- (200.8000,180.2000) -- (200.8000,180.2000) -- (200.9000,180.2000) -- (200.9000,180.2000) -- (201.0000,180.2000) -- (201.1000,180.2000) -- (201.1000,180.2000) -- (201.2000,180.2000) -- (201.3000,180.2000) -- (201.3000,180.2000) -- (201.4000,180.2000) -- (201.4000,180.2000) -- (201.5000,180.2000) -- (201.6000,180.2000) -- (201.6000,180.2000) -- (201.7000,180.2000) -- (201.7000,180.2000) -- (201.8000,180.2000) -- (201.9000,180.2000) -- (201.9000,180.2000) -- (202.0000,180.2000) -- (202.1000,180.2000) -- (202.1000,180.2000) -- (202.2000,180.2000) -- (202.2000,180.2000) -- (202.3000,180.2000) -- (202.4000,180.2000) -- (202.4000,180.2000) -- (202.5000,180.2000) -- (202.5000,180.2000) -- (202.6000,180.2000) -- (202.7000,180.2000) -- (202.7000,180.2000) -- (202.8000,180.2000) -- (202.9000,180.2000) -- (202.9000,180.2000) -- (203.0000,180.2000) -- (203.0000,180.2000) -- (203.1000,180.2000) -- (203.2000,180.2000) -- (203.2000,180.2000) -- (203.3000,180.2000) -- (203.3000,180.2000) -- (203.4000,180.2000) -- (203.5000,180.2000) -- (203.5000,180.2000) -- (203.6000,180.2000) -- (203.7000,180.2000) -- (203.7000,180.2000) -- (203.8000,180.2000) -- (203.8000,180.2000) -- (203.9000,180.2000) -- (204.0000,180.2000) -- (204.0000,180.2000) -- (204.1000,180.2000) -- (204.1000,180.2000) -- (204.2000,180.2000) -- (204.3000,180.2000) -- (204.3000,180.2000) -- (204.4000,180.2000) -- (204.5000,180.2000) -- (204.5000,180.2000) -- (204.6000,180.2000) -- (204.6000,180.2000) -- (204.7000,180.2000) -- (204.8000,180.2000) -- (204.8000,180.2000) -- (204.9000,180.2000) -- (204.9000,180.2000) -- (205.0000,180.2000) -- (205.1000,180.2000) -- (205.1000,180.2000) -- (205.2000,180.2000) -- (205.3000,180.2000) -- (205.3000,180.2000) -- (205.4000,180.2000) -- (205.4000,180.2000) -- (205.5000,180.2000) -- (205.6000,180.2000) -- (205.6000,180.2000) -- (205.7000,180.2000) -- (205.7000,180.2000) -- (205.8000,180.2000) -- (205.9000,180.2000) -- (205.9000,180.2000) -- (206.0000,180.2000) -- (206.1000,180.2000) -- (206.1000,180.2000) -- (206.2000,180.2000) -- (206.2000,180.2000) -- (206.3000,180.2000) -- (206.4000,180.2000) -- (206.4000,180.2000) -- (206.5000,180.2000) -- (206.5000,180.2000) -- (206.6000,180.2000) -- (206.7000,180.2000) -- (206.7000,180.2000) -- (206.8000,180.2000) -- (206.9000,180.2000) -- (206.9000,180.2000) -- (207.0000,180.2000) -- (207.0000,180.2000) -- (207.1000,180.2000) -- (207.2000,180.2000) -- (207.2000,180.2000) -- (207.3000,180.2000) -- (207.3000,180.2000) -- (207.4000,180.2000) -- (207.5000,180.2000) -- (207.5000,180.2000) -- (207.6000,180.2000) -- (207.7000,180.2000) -- (207.7000,180.2000) -- (207.8000,180.2000) -- (207.8000,180.2000) -- (207.9000,180.2000) -- (208.0000,180.2000) -- (208.0000,180.2000) -- (208.1000,180.2000) -- (208.1000,180.2000) -- (208.2000,180.2000) -- (208.3000,180.2000) -- (208.3000,180.2000) -- (208.4000,180.2000) -- (208.5000,180.2000) -- (208.5000,180.2000) -- (208.6000,180.2000) -- (208.6000,180.2000) -- (208.7000,180.2000) -- (208.8000,180.2000) -- (208.8000,180.2000) -- (208.9000,180.2000) -- (208.9000,180.2000) -- (209.0000,180.2000) -- (209.1000,180.2000) -- (209.1000,180.2000) -- (209.2000,180.2000) -- (209.3000,180.2000) -- (209.3000,180.2000) -- (209.4000,180.2000) -- (209.4000,180.2000) -- (209.5000,180.2000) -- (209.6000,180.2000) -- (209.6000,180.2000) -- (209.7000,180.2000) -- (209.7000,180.2000) -- (209.8000,180.2000) -- (209.9000,180.2000) -- (209.9000,180.2000) -- (210.0000,180.2000) -- (210.1000,180.2000) -- (210.1000,180.2000) -- (210.2000,180.2000) -- (210.2000,180.2000) -- (210.3000,180.2000) -- (210.4000,180.2000) -- (210.4000,180.2000) -- (210.5000,180.2000) -- (210.5000,180.2000) -- (210.6000,180.2000) -- (210.7000,180.2000) -- (210.7000,180.2000) -- (210.8000,180.2000) -- (210.9000,180.2000) -- (210.9000,180.2000) -- (211.0000,180.2000) -- (211.0000,180.2000) -- (211.1000,180.2000) -- (211.2000,180.2000) -- (211.2000,180.2000) -- (211.3000,180.2000) -- (211.3000,180.2000) -- (211.4000,180.2000) -- (211.5000,180.2000) -- (211.5000,180.2000) -- (211.6000,180.2000) -- (211.7000,180.2000) -- (211.7000,180.2000) -- (211.8000,180.2000) -- (211.8000,180.2000) -- (211.9000,180.2000) -- (212.0000,180.2000) -- (212.0000,180.2000) -- (212.1000,180.2000) -- (212.1000,180.2000) -- (212.2000,180.2000) -- (212.3000,180.2000) -- (212.3000,180.2000) -- (212.4000,180.2000) -- (212.5000,180.2000) -- (212.5000,180.2000) -- (212.6000,180.2000) -- (212.6000,180.2000) -- (212.7000,180.2000) -- (212.8000,180.2000) -- (212.8000,180.2000) -- (212.9000,180.2000) -- (212.9000,180.2000) -- (213.0000,180.2000) -- (213.1000,180.2000) -- (213.1000,180.2000) -- (213.2000,180.2000) -- (213.3000,180.2000) -- (213.3000,180.2000) -- (213.4000,180.2000) -- (213.4000,180.2000) -- (213.5000,180.2000) -- (213.6000,180.2000) -- (213.6000,180.2000) -- (213.7000,180.2000) -- (213.7000,180.2000) -- (213.8000,180.2000) -- (213.9000,180.2000) -- (213.9000,180.2000) -- (214.0000,180.2000) -- (214.1000,180.2000) -- (214.1000,180.2000) -- (214.2000,180.2000) -- (214.2000,180.2000) -- (214.3000,180.2000) -- (214.4000,180.2000) -- (214.4000,180.2000) -- (214.5000,180.2000) -- (214.5000,180.2000) -- (214.6000,180.2000) -- (214.7000,180.2000) -- (214.7000,180.2000) -- (214.8000,180.2000) -- (214.9000,180.2000) -- (214.9000,180.2000) -- (215.0000,180.2000) -- (215.0000,180.2000) -- (215.1000,180.2000) -- (215.2000,180.2000) -- (215.2000,180.2000) -- (215.3000,180.2000) -- (215.3000,180.2000) -- (215.4000,180.2000) -- (215.5000,180.2000) -- (215.5000,180.2000) -- (215.6000,180.2000) -- (215.7000,180.2000) -- (215.7000,180.2000) -- (215.8000,180.2000) -- (215.8000,180.2000) -- (215.9000,180.2000) -- (216.0000,180.2000) -- (216.0000,180.2000) -- (216.1000,180.2000) -- (216.1000,180.2000) -- (216.2000,180.2000) -- (216.3000,180.2000) -- (216.3000,180.2000) -- (216.4000,180.2000) -- (216.5000,180.2000) -- (216.5000,180.2000) -- (216.6000,180.2000) -- (216.6000,180.2000) -- (216.7000,180.2000) -- (216.8000,180.2000) -- (216.8000,180.2000) -- (216.9000,180.2000) -- (216.9000,180.2000) -- (217.0000,180.2000) -- (217.1000,180.2000) -- (217.1000,180.2000) -- (217.2000,180.2000) -- (217.3000,180.2000) -- (217.3000,180.2000) -- (217.4000,180.2000) -- (217.4000,180.2000) -- (217.5000,180.2000) -- (217.6000,180.2000) -- (217.6000,180.2000) -- (217.7000,180.2000) -- (217.7000,180.2000) -- (217.8000,180.2000) -- (217.9000,180.2000) -- (217.9000,180.2000) -- (218.0000,180.2000) -- (218.1000,180.2000) -- (218.1000,180.2000) -- (218.2000,180.2000) -- (218.2000,180.2000) -- (218.3000,180.2000) -- (218.4000,180.2000) -- (218.4000,180.2000) -- (218.5000,180.2000) -- (218.5000,180.2000) -- (218.6000,180.2000) -- (218.7000,180.2000) -- (218.7000,180.2000) -- (218.8000,180.2000) -- (218.9000,180.2000) -- (218.9000,180.2000) -- (219.0000,180.2000) -- (219.0000,180.2000) -- (219.1000,180.2000) -- (219.2000,180.2000) -- (219.2000,180.2000) -- (219.3000,180.2000) -- (219.3000,180.2000) -- (219.4000,180.2000) -- (219.5000,180.2000) -- (219.5000,180.2000) -- (219.6000,180.2000) -- (219.7000,180.2000) -- (219.7000,180.2000) -- (219.8000,180.2000) -- (219.8000,180.2000) -- (219.9000,180.2000) -- (220.0000,180.2000) -- (220.0000,180.2000) -- (220.1000,180.2000) -- (220.1000,180.2000) -- (220.2000,180.2000) -- (220.3000,180.2000) -- (220.3000,180.2000) -- (220.4000,180.2000) -- (220.5000,180.2000) -- (220.5000,180.2000) -- (220.6000,180.2000) -- (220.6000,180.2000) -- (220.7000,180.2000) -- (220.8000,180.2000) -- (220.8000,180.2000) -- (220.9000,180.2000) -- (220.9000,180.2000) -- (221.0000,180.2000) -- (221.1000,180.2000) -- (221.1000,180.2000) -- (221.2000,180.2000) -- (221.3000,180.2000) -- (221.3000,180.2000) -- (221.4000,180.2000) -- (221.4000,180.2000) -- (221.5000,180.2000) -- (221.6000,180.2000) -- (221.6000,180.2000) -- (221.7000,180.2000) -- (221.7000,180.2000) -- (221.8000,180.2000) -- (221.9000,180.2000) -- (221.9000,180.2000) -- (222.0000,180.2000) -- (222.1000,180.2000) -- (222.1000,180.2000) -- (222.2000,180.2000) -- (222.2000,180.2000) -- (222.3000,180.2000) -- (222.4000,180.2000) -- (222.4000,180.2000) -- (222.5000,180.2000) -- (222.5000,180.2000) -- (222.6000,180.2000) -- (222.7000,180.2000) -- (222.7000,180.2000) -- (222.8000,180.2000) -- (222.9000,180.2000) -- (222.9000,180.2000) -- (223.0000,180.2000) -- (223.0000,180.2000) -- (223.1000,180.2000) -- (223.2000,180.2000) -- (223.2000,180.2000) -- (223.3000,180.2000) -- (223.3000,180.2000) -- (223.4000,180.2000) -- (223.5000,180.2000) -- (223.5000,180.2000) -- (223.6000,180.2000) -- (223.7000,180.2000) -- (223.7000,180.2000) -- (223.8000,180.2000) -- (223.8000,180.2000) -- (223.9000,180.2000) -- (224.0000,180.2000) -- (224.0000,180.2000) -- (224.1000,180.2000) -- (224.1000,180.2000) -- (224.2000,180.2000) -- (224.3000,180.2000) -- (224.3000,180.2000) -- (224.4000,180.2000) -- (224.5000,180.2000) -- (224.5000,180.2000) -- (224.6000,180.2000) -- (224.6000,180.2000) -- (224.7000,180.2000) -- (224.8000,180.2000) -- (224.8000,180.2000) -- (224.9000,180.2000) -- (224.9000,180.2000) -- (225.0000,180.2000) -- (225.1000,180.2000) -- (225.1000,180.2000) -- (225.2000,180.2000) -- (225.3000,180.2000) -- (225.3000,180.2000) -- (225.4000,180.2000) -- (225.4000,180.2000) -- (225.5000,180.2000) -- (225.6000,180.2000) -- (225.6000,180.2000) -- (225.7000,180.2000) -- (225.7000,180.2000) -- (225.8000,180.2000) -- (225.9000,179.4000) -- (225.9000,190.4000) -- (226.0000,200.9000) -- (226.0000,200.0000) -- (226.1000,200.0000) -- (226.2000,200.0000) -- (226.2000,200.0000) -- (226.3000,200.0000) -- (226.4000,200.0000) -- (226.4000,200.0000) -- (226.5000,200.0000) -- (226.5000,200.0000) -- (226.6000,200.0000) -- (226.7000,200.0000) -- (226.7000,200.0000) -- (226.8000,200.0000) -- (226.8000,200.0000) -- (226.9000,200.0000) -- (227.0000,200.0000) -- (227.0000,200.0000) -- (227.1000,200.0000) -- (227.2000,200.0000) -- (227.2000,200.0000) -- (227.3000,200.0000) -- (227.3000,200.0000) -- (227.4000,200.0000) -- (227.5000,200.0000) -- (227.5000,200.0000) -- (227.6000,200.0000) -- (227.7000,200.0000) -- (227.7000,200.0000) -- (227.8000,200.0000) -- (227.8000,200.0000) -- (227.9000,200.0000) -- (228.0000,200.0000) -- (228.0000,200.0000) -- (228.1000,200.0000) -- (228.1000,200.0000) -- (228.2000,200.0000) -- (228.3000,200.0000) -- (228.3000,200.0000) -- (228.4000,200.0000) -- (228.5000,200.0000) -- (228.5000,200.0000) -- (228.6000,200.0000) -- (228.6000,200.0000) -- (228.7000,200.0000) -- (228.8000,200.0000) -- (228.8000,200.0000) -- (228.9000,200.0000) -- (228.9000,200.0000) -- (229.0000,200.0000) -- (229.1000,200.0000) -- (229.1000,200.0000) -- (229.2000,200.0000) -- (229.2000,200.0000) -- (229.3000,200.0000) -- (229.4000,200.0000) -- (229.4000,200.0000) -- (229.5000,200.0000) -- (229.6000,200.0000) -- (229.6000,200.0000) -- (229.7000,200.0000) -- (229.7000,200.0000) -- (229.8000,200.0000) -- (229.9000,200.0000) -- (229.9000,200.0000) -- (230.0000,200.0000) -- (230.0000,200.0000) -- (230.1000,200.0000) -- (230.2000,200.0000) -- (230.2000,200.0000) -- (230.3000,200.0000) -- (230.4000,200.0000) -- (230.4000,200.0000) -- (230.5000,200.0000) -- (230.5000,200.0000) -- (230.6000,200.0000) -- (230.7000,200.0000) -- (230.7000,200.0000) -- (230.8000,200.0000) -- (230.8000,200.0000) -- (230.9000,200.0000) -- (231.0000,200.0000) -- (231.0000,200.0000) -- (231.1000,200.0000) -- (231.2000,200.0000) -- (231.2000,200.0000) -- (231.3000,200.0000) -- (231.3000,200.0000) -- (231.4000,200.0000) -- (231.5000,200.0000) -- (231.5000,200.0000) -- (231.6000,200.0000) -- (231.6000,200.0000) -- (231.7000,200.0000) -- (231.8000,200.0000) -- (231.8000,200.0000) -- (231.9000,200.0000) -- (232.0000,200.0000) -- (232.0000,200.0000) -- (232.1000,200.0000) -- (232.1000,200.0000) -- (232.2000,200.0000) -- (232.3000,200.0000) -- (232.3000,200.0000) -- (232.4000,200.0000) -- (232.4000,200.0000) -- (232.5000,200.0000) -- (232.6000,200.0000) -- (232.6000,200.0000) -- (232.7000,200.0000) -- (232.8000,200.0000) -- (232.8000,200.0000) -- (232.9000,200.0000) -- (232.9000,200.0000) -- (233.0000,200.0000) -- (233.1000,200.0000) -- (233.1000,200.0000) -- (233.2000,200.0000) -- (233.2000,200.0000) -- (233.3000,200.0000) -- (233.4000,200.0000) -- (233.4000,200.0000) -- (233.5000,200.0000) -- (233.6000,200.0000) -- (233.6000,200.0000) -- (233.7000,200.0000) -- (233.7000,200.0000) -- (233.8000,200.0000) -- (233.9000,200.0000) -- (233.9000,200.0000) -- (234.0000,200.0000) -- (234.0000,200.0000) -- (234.1000,200.0000) -- (234.2000,200.0000) -- (234.2000,200.0000) -- (234.3000,200.0000) -- (234.4000,200.0000) -- (234.4000,200.0000) -- (234.5000,200.0000) -- (234.5000,200.0000) -- (234.6000,200.0000) -- (234.7000,200.0000) -- (234.7000,200.0000) -- (234.8000,200.0000) -- (234.8000,200.0000) -- (234.9000,200.0000) -- (235.0000,200.0000) -- (235.0000,200.0000) -- (235.1000,200.0000) -- (235.2000,200.0000) -- (235.2000,200.0000) -- (235.3000,200.0000) -- (235.3000,200.0000) -- (235.4000,200.0000) -- (235.5000,200.0000) -- (235.5000,200.0000) -- (235.6000,200.0000) -- (235.6000,200.0000) -- (235.7000,200.0000) -- (235.8000,200.0000) -- (235.8000,200.0000) -- (235.9000,200.0000) -- (236.0000,200.0000) -- (236.0000,200.0000) -- (236.1000,200.0000) -- (236.1000,200.0000) -- (236.2000,200.0000) -- (236.3000,200.0000) -- (236.3000,200.0000) -- (236.4000,200.0000) -- (236.4000,200.0000) -- (236.5000,200.0000) -- (236.6000,200.0000) -- (236.6000,200.0000) -- (236.7000,200.0000) -- (236.8000,200.0000) -- (236.8000,200.0000) -- (236.9000,200.0000) -- (236.9000,200.0000) -- (237.0000,200.0000) -- (237.1000,200.0000) -- (237.1000,200.0000) -- (237.2000,200.0000) -- (237.2000,200.0000) -- (237.3000,200.0000) -- (237.4000,200.0000) -- (237.4000,200.0000) -- (237.5000,200.0000) -- (237.6000,200.0000) -- (237.6000,200.0000) -- (237.7000,200.0000) -- (237.7000,200.0000) -- (237.8000,200.0000) -- (237.9000,200.0000) -- (237.9000,200.0000) -- (238.0000,200.0000) -- (238.0000,200.0000) -- (238.1000,200.0000) -- (238.2000,200.0000) -- (238.2000,200.0000) -- (238.3000,200.0000) -- (238.4000,200.0000) -- (238.4000,200.0000) -- (238.5000,200.0000) -- (238.5000,200.0000) -- (238.6000,200.0000) -- (238.7000,200.0000) -- (238.7000,200.0000) -- (238.8000,200.0000) -- (238.8000,200.0000) -- (238.9000,200.0000) -- (239.0000,200.0000) -- (239.0000,200.0000) -- (239.1000,200.0000) -- (239.2000,200.0000) -- (239.2000,200.0000) -- (239.3000,200.0000) -- (239.3000,200.0000) -- (239.4000,200.0000) -- (239.5000,200.0000) -- (239.5000,200.0000) -- (239.6000,200.0000) -- (239.6000,200.0000) -- (239.7000,200.0000) -- (239.8000,200.0000) -- (239.8000,200.0000) -- (239.9000,200.0000) -- (240.0000,200.0000) -- (240.0000,200.0000) -- (240.1000,200.0000) -- (240.1000,200.0000) -- (240.2000,200.0000) -- (240.3000,200.0000) -- (240.3000,200.0000) -- (240.4000,200.0000) -- (240.4000,200.0000) -- (240.5000,200.0000) -- (240.6000,200.0000) -- (240.6000,200.0000) -- (240.7000,200.0000) -- (240.8000,200.0000) -- (240.8000,200.0000) -- (240.9000,200.0000) -- (240.9000,200.0000) -- (241.0000,200.0000) -- (241.1000,200.0000) -- (241.1000,200.0000) -- (241.2000,200.0000) -- (241.2000,200.0000) -- (241.3000,200.0000) -- (241.4000,200.0000) -- (241.4000,200.0000) -- (241.5000,200.0000) -- (241.6000,200.0000) -- (241.6000,200.0000) -- (241.7000,200.0000) -- (241.7000,200.0000) -- (241.8000,200.0000) -- (241.9000,200.0000) -- (241.9000,200.0000) -- (242.0000,200.0000) -- (242.0000,200.0000) -- (242.1000,200.0000) -- (242.2000,200.0000) -- (242.2000,200.0000) -- (242.3000,200.0000) -- (242.4000,200.0000) -- (242.4000,200.0000) -- (242.5000,200.0000) -- (242.5000,200.0000) -- (242.6000,200.0000) -- (242.7000,200.0000) -- (242.7000,200.0000) -- (242.8000,200.0000) -- (242.8000,200.0000) -- (242.9000,200.0000) -- (243.0000,200.0000) -- (243.0000,200.0000) -- (243.1000,200.0000) -- (243.2000,200.0000) -- (243.2000,200.0000) -- (243.3000,200.0000) -- (243.3000,200.0000) -- (243.4000,200.0000) -- (243.5000,200.0000) -- (243.5000,200.0000) -- (243.6000,200.0000) -- (243.6000,200.0000) -- (243.7000,200.0000) -- (243.8000,200.0000) -- (243.8000,200.0000) -- (243.9000,200.0000) -- (244.0000,200.0000) -- (244.0000,200.0000) -- (244.1000,200.0000) -- (244.1000,200.0000) -- (244.2000,200.0000) -- (244.3000,200.0000) -- (244.3000,200.0000) -- (244.4000,200.0000) -- (244.4000,200.0000) -- (244.5000,200.0000) -- (244.6000,200.0000) -- (244.6000,200.0000) -- (244.7000,200.0000) -- (244.8000,200.0000) -- (244.8000,200.0000) -- (244.9000,200.0000) -- (244.9000,200.0000) -- (245.0000,200.0000) -- (245.1000,200.0000) -- (245.1000,200.0000) -- (245.2000,200.0000) -- (245.2000,200.0000) -- (245.3000,200.0000) -- (245.4000,200.0000) -- (245.4000,200.0000) -- (245.5000,200.0000) -- (245.6000,200.0000) -- (245.6000,200.0000) -- (245.7000,200.0000) -- (245.7000,200.0000) -- (245.8000,200.0000) -- (245.9000,200.0000) -- (245.9000,200.0000) -- (246.0000,200.0000) -- (246.0000,200.0000) -- (246.1000,200.0000) -- (246.2000,200.0000) -- (246.2000,200.0000) -- (246.3000,200.0000);



  \end{scope}
  \begin{scope}[scale=1.006,draw=black,line join=bevel,line cap=rect,line width=0.800pt]
  \end{scope}
  \begin{scope}[scale=1.006,draw=black,line join=bevel,line cap=rect,line width=0.800pt]
  \end{scope}
  \begin{scope}[scale=1.006,draw=black,line join=round,line cap=round,line width=0.480pt]
    \path[draw] (184.5000,156.5000) -- (184.5000,204.5000) -- (246.5000,204.5000) -- (246.5000,156.5000) -- (184.5000,156.5000);



  \end{scope}
  \begin{scope}[scale=1.006,draw=ca0a0a4,dash pattern=on 0.40pt off 0.80pt,line join=round,line cap=round,line width=0.400pt]
    \path[draw] (44.5000,319.5000) -- (179.5000,319.5000);



  \end{scope}
  \begin{scope}[scale=1.006,draw=black,line join=round,line cap=round,line width=0.480pt]
    \path[draw] (44.5000,319.5000) -- (48.5000,319.5000);



    \path[draw] (179.5000,319.5000) -- (175.5000,319.5000);



  \end{scope}
  \begin{scope}[scale=1.006,draw=black,line join=bevel,line cap=rect,line width=0.800pt]
  \end{scope}
  \begin{scope}[cm={{1.00588,0.0,0.0,1.00588,(18.1059,325.906)}},draw=black,line join=bevel,line cap=rect,line width=0.800pt]
  \end{scope}
  \begin{scope}[cm={{1.00588,0.0,0.0,1.00588,(18.1059,325.906)}},draw=black,line join=bevel,line cap=rect,line width=0.800pt]
  \end{scope}
  \begin{scope}[cm={{1.00588,0.0,0.0,1.00588,(18.1059,325.906)}},draw=black,line join=bevel,line cap=rect,line width=0.800pt]
  \end{scope}
  \begin{scope}[cm={{1.00588,0.0,0.0,1.00588,(18.1059,325.906)}},draw=black,line join=bevel,line cap=rect,line width=0.800pt]
  \end{scope}
  \begin{scope}[cm={{1.00588,0.0,0.0,1.00588,(18.1059,325.906)}},draw=black,line join=bevel,line cap=rect,line width=0.800pt]
  \end{scope}
  \begin{scope}[cm={{1.00588,0.0,0.0,1.00588,(18.1059,325.906)}},draw=black,line join=bevel,line cap=rect,line width=0.800pt]
    \path[fill=black] (0.0000,0.0000) node[above right] (text1648) {-100};



  \end{scope}
  \begin{scope}[cm={{1.00588,0.0,0.0,1.00588,(18.1059,325.906)}},draw=black,line join=bevel,line cap=rect,line width=0.800pt]
  \end{scope}
  \begin{scope}[scale=1.006,draw=black,line join=bevel,line cap=rect,line width=0.800pt]
  \end{scope}
  \begin{scope}[scale=1.006,draw=ca0a0a4,dash pattern=on 0.40pt off 0.80pt,line join=round,line cap=round,line width=0.400pt]
    \path[draw] (44.5000,290.5000) -- (179.5000,290.5000);



  \end{scope}
  \begin{scope}[scale=1.006,draw=black,line join=round,line cap=round,line width=0.480pt]
    \path[draw] (44.5000,290.5000) -- (48.5000,290.5000);



    \path[draw] (179.5000,290.5000) -- (175.5000,290.5000);



  \end{scope}
  \begin{scope}[scale=1.006,draw=black,line join=bevel,line cap=rect,line width=0.800pt]
  \end{scope}
  \begin{scope}[cm={{1.00588,0.0,0.0,1.00588,(34.2,296.735)}},draw=black,line join=bevel,line cap=rect,line width=0.800pt]
  \end{scope}
  \begin{scope}[cm={{1.00588,0.0,0.0,1.00588,(34.2,296.735)}},draw=black,line join=bevel,line cap=rect,line width=0.800pt]
  \end{scope}
  \begin{scope}[cm={{1.00588,0.0,0.0,1.00588,(34.2,296.735)}},draw=black,line join=bevel,line cap=rect,line width=0.800pt]
  \end{scope}
  \begin{scope}[cm={{1.00588,0.0,0.0,1.00588,(34.2,296.735)}},draw=black,line join=bevel,line cap=rect,line width=0.800pt]
  \end{scope}
  \begin{scope}[cm={{1.00588,0.0,0.0,1.00588,(34.2,296.735)}},draw=black,line join=bevel,line cap=rect,line width=0.800pt]
  \end{scope}
  \begin{scope}[cm={{1.00588,0.0,0.0,1.00588,(34.2,296.735)}},draw=black,line join=bevel,line cap=rect,line width=0.800pt]
    \path[fill=black] (0.0000,0.0000) node[above right] (text1678) {0};



  \end{scope}
  \begin{scope}[cm={{1.00588,0.0,0.0,1.00588,(34.2,296.735)}},draw=black,line join=bevel,line cap=rect,line width=0.800pt]
  \end{scope}
  \begin{scope}[scale=1.006,draw=black,line join=bevel,line cap=rect,line width=0.800pt]
  \end{scope}
  \begin{scope}[scale=1.006,draw=ca0a0a4,dash pattern=on 0.40pt off 0.80pt,line join=round,line cap=round,line width=0.400pt]
    \path[draw] (44.5000,261.5000) -- (179.5000,261.5000);



  \end{scope}
  \begin{scope}[scale=1.006,draw=black,line join=round,line cap=round,line width=0.480pt]
    \path[draw] (44.5000,261.5000) -- (48.5000,261.5000);



    \path[draw] (179.5000,261.5000) -- (175.5000,261.5000);



  \end{scope}
  \begin{scope}[scale=1.006,draw=black,line join=bevel,line cap=rect,line width=0.800pt]
  \end{scope}
  \begin{scope}[cm={{1.00588,0.0,0.0,1.00588,(22.1294,266.559)}},draw=black,line join=bevel,line cap=rect,line width=0.800pt]
  \end{scope}
  \begin{scope}[cm={{1.00588,0.0,0.0,1.00588,(22.1294,266.559)}},draw=black,line join=bevel,line cap=rect,line width=0.800pt]
  \end{scope}
  \begin{scope}[cm={{1.00588,0.0,0.0,1.00588,(22.1294,266.559)}},draw=black,line join=bevel,line cap=rect,line width=0.800pt]
  \end{scope}
  \begin{scope}[cm={{1.00588,0.0,0.0,1.00588,(22.1294,266.559)}},draw=black,line join=bevel,line cap=rect,line width=0.800pt]
  \end{scope}
  \begin{scope}[cm={{1.00588,0.0,0.0,1.00588,(22.1294,266.559)}},draw=black,line join=bevel,line cap=rect,line width=0.800pt]
  \end{scope}
  \begin{scope}[cm={{1.00588,0.0,0.0,1.00588,(22.1294,266.559)}},draw=black,line join=bevel,line cap=rect,line width=0.800pt]
    \path[fill=black] (0.0000,0.0000) node[above right] (text1708) {100};



  \end{scope}
  \begin{scope}[cm={{1.00588,0.0,0.0,1.00588,(22.1294,266.559)}},draw=black,line join=bevel,line cap=rect,line width=0.800pt]
  \end{scope}
  \begin{scope}[scale=1.006,draw=black,line join=bevel,line cap=rect,line width=0.800pt]
  \end{scope}
  \begin{scope}[scale=1.006,draw=ca0a0a4,dash pattern=on 0.40pt off 0.80pt,line join=round,line cap=round,line width=0.400pt]
    \path[draw] (44.5000,232.5000) -- (179.5000,232.5000);



  \end{scope}
  \begin{scope}[scale=1.006,draw=black,line join=round,line cap=round,line width=0.480pt]
    \path[draw] (44.5000,232.5000) -- (48.5000,232.5000);



    \path[draw] (179.5000,232.5000) -- (175.5000,232.5000);



  \end{scope}
  \begin{scope}[scale=1.006,draw=black,line join=bevel,line cap=rect,line width=0.800pt]
  \end{scope}
  \begin{scope}[cm={{1.00588,0.0,0.0,1.00588,(22.1294,237.388)}},draw=black,line join=bevel,line cap=rect,line width=0.800pt]
  \end{scope}
  \begin{scope}[cm={{1.00588,0.0,0.0,1.00588,(22.1294,237.388)}},draw=black,line join=bevel,line cap=rect,line width=0.800pt]
  \end{scope}
  \begin{scope}[cm={{1.00588,0.0,0.0,1.00588,(22.1294,237.388)}},draw=black,line join=bevel,line cap=rect,line width=0.800pt]
  \end{scope}
  \begin{scope}[cm={{1.00588,0.0,0.0,1.00588,(22.1294,237.388)}},draw=black,line join=bevel,line cap=rect,line width=0.800pt]
  \end{scope}
  \begin{scope}[cm={{1.00588,0.0,0.0,1.00588,(22.1294,237.388)}},draw=black,line join=bevel,line cap=rect,line width=0.800pt]
  \end{scope}
  \begin{scope}[cm={{1.00588,0.0,0.0,1.00588,(22.1294,237.388)}},draw=black,line join=bevel,line cap=rect,line width=0.800pt]
    \path[fill=black] (0.0000,0.0000) node[above right] (text1738) {200};



  \end{scope}
  \begin{scope}[cm={{1.00588,0.0,0.0,1.00588,(22.1294,237.388)}},draw=black,line join=bevel,line cap=rect,line width=0.800pt]
  \end{scope}
  \begin{scope}[scale=1.006,draw=black,line join=bevel,line cap=rect,line width=0.800pt]
  \end{scope}
  \begin{scope}[scale=1.006,draw=ca0a0a4,dash pattern=on 0.40pt off 0.80pt,line join=round,line cap=round,line width=0.400pt]
    \path[draw] (44.5000,319.5000) -- (44.5000,217.5000);



  \end{scope}
  \begin{scope}[scale=1.006,draw=black,line join=round,line cap=round,line width=0.480pt]
    \path[draw] (44.5000,319.5000) -- (44.5000,316.5000);



    \path[draw] (44.5000,217.5000) -- (44.5000,220.5000);



  \end{scope}
  \begin{scope}[scale=1.006,draw=black,line join=bevel,line cap=rect,line width=0.800pt]
  \end{scope}
  \begin{scope}[cm={{1.00588,0.0,0.0,1.00588,(34.2,336.971)}},draw=black,line join=bevel,line cap=rect,line width=0.800pt]
  \end{scope}
  \begin{scope}[cm={{1.00588,0.0,0.0,1.00588,(34.2,336.971)}},draw=black,line join=bevel,line cap=rect,line width=0.800pt]
  \end{scope}
  \begin{scope}[cm={{1.00588,0.0,0.0,1.00588,(34.2,336.971)}},draw=black,line join=bevel,line cap=rect,line width=0.800pt]
  \end{scope}
  \begin{scope}[cm={{1.00588,0.0,0.0,1.00588,(34.2,336.971)}},draw=black,line join=bevel,line cap=rect,line width=0.800pt]
  \end{scope}
  \begin{scope}[cm={{1.00588,0.0,0.0,1.00588,(34.2,336.971)}},draw=black,line join=bevel,line cap=rect,line width=0.800pt]
  \end{scope}
  \begin{scope}[cm={{1.00588,0.0,0.0,1.00588,(34.2,336.971)}},draw=black,line join=bevel,line cap=rect,line width=0.800pt]
    \path[fill=black] (0.0000,0.0000) node[above right] (text1768) {-150};



  \end{scope}
  \begin{scope}[cm={{1.00588,0.0,0.0,1.00588,(34.2,336.971)}},draw=black,line join=bevel,line cap=rect,line width=0.800pt]
  \end{scope}
  \begin{scope}[scale=1.006,draw=black,line join=bevel,line cap=rect,line width=0.800pt]
  \end{scope}
  \begin{scope}[scale=1.006,draw=ca0a0a4,dash pattern=on 0.40pt off 0.80pt,line join=round,line cap=round,line width=0.400pt]
    \path[draw] (78.5000,319.5000) -- (78.5000,217.5000);



  \end{scope}
  \begin{scope}[scale=1.006,draw=black,line join=round,line cap=round,line width=0.480pt]
    \path[draw] (78.5000,319.5000) -- (78.5000,316.5000);



    \path[draw] (78.5000,217.5000) -- (78.5000,220.5000);



  \end{scope}
  \begin{scope}[scale=1.006,draw=black,line join=bevel,line cap=rect,line width=0.800pt]
  \end{scope}
  \begin{scope}[cm={{1.00588,0.0,0.0,1.00588,(70.4118,336.971)}},draw=black,line join=bevel,line cap=rect,line width=0.800pt]
  \end{scope}
  \begin{scope}[cm={{1.00588,0.0,0.0,1.00588,(70.4118,336.971)}},draw=black,line join=bevel,line cap=rect,line width=0.800pt]
  \end{scope}
  \begin{scope}[cm={{1.00588,0.0,0.0,1.00588,(70.4118,336.971)}},draw=black,line join=bevel,line cap=rect,line width=0.800pt]
  \end{scope}
  \begin{scope}[cm={{1.00588,0.0,0.0,1.00588,(70.4118,336.971)}},draw=black,line join=bevel,line cap=rect,line width=0.800pt]
  \end{scope}
  \begin{scope}[cm={{1.00588,0.0,0.0,1.00588,(70.4118,336.971)}},draw=black,line join=bevel,line cap=rect,line width=0.800pt]
  \end{scope}
  \begin{scope}[cm={{1.00588,0.0,0.0,1.00588,(70.4118,336.971)}},draw=black,line join=bevel,line cap=rect,line width=0.800pt]
    \path[fill=black] (0.0000,0.0000) node[above right] (text1798) {-50};



  \end{scope}
  \begin{scope}[cm={{1.00588,0.0,0.0,1.00588,(70.4118,336.971)}},draw=black,line join=bevel,line cap=rect,line width=0.800pt]
  \end{scope}
  \begin{scope}[scale=1.006,draw=black,line join=bevel,line cap=rect,line width=0.800pt]
  \end{scope}
  \begin{scope}[scale=1.006,draw=ca0a0a4,dash pattern=on 0.40pt off 0.80pt,line join=round,line cap=round,line width=0.400pt]
    \path[draw] (112.5000,319.5000) -- (112.5000,217.5000);



  \end{scope}
  \begin{scope}[scale=1.006,draw=black,line join=round,line cap=round,line width=0.480pt]
    \path[draw] (112.5000,319.5000) -- (112.5000,316.5000);



    \path[draw] (112.5000,217.5000) -- (112.5000,220.5000);



  \end{scope}
  \begin{scope}[scale=1.006,draw=black,line join=bevel,line cap=rect,line width=0.800pt]
  \end{scope}
  \begin{scope}[cm={{1.00588,0.0,0.0,1.00588,(105.115,336.971)}},draw=black,line join=bevel,line cap=rect,line width=0.800pt]
  \end{scope}
  \begin{scope}[cm={{1.00588,0.0,0.0,1.00588,(105.115,336.971)}},draw=black,line join=bevel,line cap=rect,line width=0.800pt]
  \end{scope}
  \begin{scope}[cm={{1.00588,0.0,0.0,1.00588,(105.115,336.971)}},draw=black,line join=bevel,line cap=rect,line width=0.800pt]
  \end{scope}
  \begin{scope}[cm={{1.00588,0.0,0.0,1.00588,(105.115,336.971)}},draw=black,line join=bevel,line cap=rect,line width=0.800pt]
  \end{scope}
  \begin{scope}[cm={{1.00588,0.0,0.0,1.00588,(105.115,336.971)}},draw=black,line join=bevel,line cap=rect,line width=0.800pt]
  \end{scope}
  \begin{scope}[cm={{1.00588,0.0,0.0,1.00588,(105.115,336.971)}},draw=black,line join=bevel,line cap=rect,line width=0.800pt]
    \path[fill=black] (0.0000,0.0000) node[above right] (text1828) {50};



  \end{scope}
  \begin{scope}[cm={{1.00588,0.0,0.0,1.00588,(105.115,336.971)}},draw=black,line join=bevel,line cap=rect,line width=0.800pt]
  \end{scope}
  \begin{scope}[scale=1.006,draw=black,line join=bevel,line cap=rect,line width=0.800pt]
  \end{scope}
  \begin{scope}[scale=1.006,draw=ca0a0a4,dash pattern=on 0.40pt off 0.80pt,line join=round,line cap=round,line width=0.400pt]
    \path[draw] (145.5000,319.5000) -- (145.5000,217.5000);



  \end{scope}
  \begin{scope}[scale=1.006,draw=black,line join=round,line cap=round,line width=0.480pt]
    \path[draw] (145.5000,319.5000) -- (145.5000,316.5000);



    \path[draw] (145.5000,217.5000) -- (145.5000,220.5000);



  \end{scope}
  \begin{scope}[scale=1.006,draw=black,line join=bevel,line cap=rect,line width=0.800pt]
  \end{scope}
  \begin{scope}[cm={{1.00588,0.0,0.0,1.00588,(136.297,336.971)}},draw=black,line join=bevel,line cap=rect,line width=0.800pt]
  \end{scope}
  \begin{scope}[cm={{1.00588,0.0,0.0,1.00588,(136.297,336.971)}},draw=black,line join=bevel,line cap=rect,line width=0.800pt]
  \end{scope}
  \begin{scope}[cm={{1.00588,0.0,0.0,1.00588,(136.297,336.971)}},draw=black,line join=bevel,line cap=rect,line width=0.800pt]
  \end{scope}
  \begin{scope}[cm={{1.00588,0.0,0.0,1.00588,(136.297,336.971)}},draw=black,line join=bevel,line cap=rect,line width=0.800pt]
  \end{scope}
  \begin{scope}[cm={{1.00588,0.0,0.0,1.00588,(136.297,336.971)}},draw=black,line join=bevel,line cap=rect,line width=0.800pt]
  \end{scope}
  \begin{scope}[cm={{1.00588,0.0,0.0,1.00588,(136.297,336.971)}},draw=black,line join=bevel,line cap=rect,line width=0.800pt]
    \path[fill=black] (0.0000,0.0000) node[above right] (text1858) {150};



  \end{scope}
  \begin{scope}[cm={{1.00588,0.0,0.0,1.00588,(136.297,336.971)}},draw=black,line join=bevel,line cap=rect,line width=0.800pt]
  \end{scope}
  \begin{scope}[scale=1.006,draw=black,line join=bevel,line cap=rect,line width=0.800pt]
  \end{scope}
  \begin{scope}[scale=1.006,draw=black,line join=round,line cap=round,line width=0.480pt]
    \path[draw] (44.5000,217.5000) -- (44.5000,319.5000) -- (179.5000,319.5000) -- (179.5000,217.5000) -- (44.5000,217.5000);



  \end{scope}
  \begin{scope}[scale=1.006,draw=black,line join=bevel,line cap=rect,line width=0.800pt]
  \end{scope}
  \begin{scope}[scale=1.006,draw=black,line join=bevel,line cap=rect,line width=0.800pt]
  \end{scope}
  \begin{scope}[scale=1.006,fill=cffffff]
    \path[fill,rounded corners=0.0000cm] (156.0000,224.0000) rectangle (172.0000,240.0000);



  \end{scope}
  \begin{scope}[scale=1.006,draw=black,line join=bevel,line cap=rect,line width=0.800pt]
  \end{scope}
  \begin{scope}[scale=1.006,draw=black,line join=bevel,line cap=rect,line width=0.800pt]
  \end{scope}
  \begin{scope}[scale=1.006,draw=black,line join=round,line cap=round,line width=0.800pt]
    \path[draw] (156.5000,240.5000) -- (156.5000,224.5000) -- (172.5000,224.5000) -- (172.5000,240.5000) -- (156.5000,240.5000);



  \end{scope}
  \begin{scope}[scale=1.006,draw=black,line join=bevel,line cap=rect,line width=0.800pt]
  \end{scope}
  \begin{scope}[cm={{1.00588,0.0,0.0,1.00588,(161.947,237.388)}},draw=black,line join=bevel,line cap=rect,line width=0.800pt]
  \end{scope}
  \begin{scope}[cm={{1.00588,0.0,0.0,1.00588,(161.947,237.388)}},draw=black,line join=bevel,line cap=rect,line width=0.800pt]
  \end{scope}
  \begin{scope}[cm={{1.00588,0.0,0.0,1.00588,(161.947,237.388)}},draw=black,line join=bevel,line cap=rect,line width=0.800pt]
  \end{scope}
  \begin{scope}[cm={{1.00588,0.0,0.0,1.00588,(161.947,237.388)}},draw=black,line join=bevel,line cap=rect,line width=0.800pt]
  \end{scope}
  \begin{scope}[cm={{1.00588,0.0,0.0,1.00588,(161.947,237.388)}},draw=black,line join=bevel,line cap=rect,line width=0.800pt]
  \end{scope}
  \begin{scope}[cm={{1.00588,0.0,0.0,1.00588,(163.03355,237.17069)}},draw=black,line join=bevel,line cap=rect,line width=0.800pt]
    \path[fill=black] (0.0000,0.0000) node[above right] (text1898) {\label{fig:trajs-dyn-ii}ii};



  \end{scope}
  \begin{scope}[cm={{1.00588,0.0,0.0,1.00588,(161.947,237.388)}},draw=black,line join=bevel,line cap=rect,line width=0.800pt]
  \end{scope}
  \begin{scope}[scale=1.006,draw=black,line join=bevel,line cap=rect,line width=0.800pt]
  \end{scope}
  \begin{scope}[scale=1.006,draw=black,line join=bevel,line cap=rect,line width=0.800pt]
  \end{scope}
  \begin{scope}[scale=1.006,draw=black,line join=bevel,line cap=rect,line width=0.800pt]
  \end{scope}
  \begin{scope}[scale=1.006,draw=black,line join=round,line cap=round,line width=0.480pt]
    \path[draw] (61.6000,227.9000) -- (61.6000,227.9000) -- (61.8000,231.0000) -- (61.9000,233.8000) -- (60.9000,236.4000) -- (59.0000,238.8000) -- (57.6000,241.4000) -- (56.8000,244.1000) -- (56.6000,247.0000) -- (56.6000,249.9000) -- (56.5000,252.8000) -- (56.5000,255.7000) -- (56.5000,258.6000) -- (56.5000,261.5000) -- (56.5000,264.4000) -- (56.5000,267.3000) -- (56.5000,270.2000) -- (56.5000,273.1000) -- (56.5000,276.0000) -- (56.5000,278.9000) -- (56.5000,281.8000) -- (56.5000,284.6000) -- (56.5000,287.5000) -- (56.8000,290.4000) -- (57.5000,293.3000) -- (58.7000,296.0000) -- (60.2000,298.6000) -- (62.2000,301.0000) -- (64.5000,303.1000) -- (67.1000,305.0000) -- (70.1000,306.6000) -- (73.2000,307.7000) -- (76.6000,308.6000) -- (80.1000,309.0000) -- (83.6000,308.9000) -- (87.0000,308.5000) -- (90.4000,307.6000) -- (93.6000,306.4000) -- (96.6000,304.8000) -- (99.2000,302.8000) -- (101.6000,300.6000) -- (103.4000,298.1000) -- (104.8000,295.4000) -- (105.6000,292.6000) -- (106.1000,289.8000) -- (106.3000,286.9000) -- (106.5000,284.1000) -- (106.5000,281.2000) -- (106.6000,278.3000) -- (106.6000,275.4000) -- (106.6000,272.5000) -- (106.6000,269.6000) -- (106.6000,266.8000) -- (106.6000,263.9000) -- (106.6000,261.0000) -- (106.6000,258.1000) -- (106.6000,255.2000) -- (106.7000,252.3000) -- (107.2000,249.4000) -- (107.4000,246.5000) -- (107.1000,243.7000) -- (106.2000,240.9000) -- (104.9000,238.2000) -- (103.2000,235.7000) -- (100.9000,233.5000) -- (98.2000,231.6000) -- (95.2000,230.1000) -- (91.9000,229.0000) -- (88.5000,228.4000) -- (85.0000,228.3000) -- (81.5000,228.7000) -- (78.2000,229.6000) -- (75.1000,230.9000) -- (72.3000,232.6000) -- (69.9000,234.7000) -- (67.8000,237.1000) -- (66.3000,239.7000) -- (65.3000,242.4000) -- (64.9000,245.3000) -- (64.8000,248.2000) -- (64.8000,251.1000) -- (64.8000,254.0000) -- (64.8000,256.9000) -- (64.8000,259.8000) -- (64.8000,262.7000) -- (64.8000,265.6000) -- (64.8000,268.5000) -- (64.8000,271.4000) -- (64.8000,274.2000) -- (64.8000,277.1000) -- (64.8000,280.0000) -- (64.8000,282.9000) -- (64.7000,285.8000) -- (64.9000,288.7000) -- (65.4000,291.6000) -- (66.4000,294.3000) -- (67.8000,297.0000) -- (69.6000,299.4000) -- (71.8000,301.7000) -- (74.4000,303.7000) -- (77.2000,305.4000) -- (80.3000,306.7000) -- (83.5000,307.7000) -- (87.0000,308.3000) -- (90.5000,308.4000) -- (94.0000,308.2000) -- (97.4000,307.5000) -- (100.7000,306.4000) -- (103.8000,305.0000) -- (106.6000,303.2000) -- (109.1000,301.1000) -- (111.2000,298.7000) -- (112.8000,296.1000) -- (113.8000,293.3000) -- (114.4000,290.5000) -- (114.8000,287.7000) -- (114.9000,284.8000) -- (115.0000,282.0000) -- (115.1000,279.1000) -- (115.1000,276.2000) -- (115.1000,273.3000) -- (115.1000,270.4000) -- (115.1000,267.5000) -- (115.1000,264.6000) -- (115.1000,261.7000) -- (115.1000,258.9000) -- (115.1000,256.0000) -- (115.1000,253.1000) -- (115.4000,250.2000) -- (115.8000,247.3000) -- (115.7000,244.4000) -- (115.1000,241.6000) -- (114.1000,238.8000) -- (112.7000,236.2000) -- (110.8000,233.8000) -- (108.6000,231.5000) -- (105.9000,229.6000) -- (102.9000,228.0000) -- (99.7000,226.8000) -- (96.3000,226.0000) -- (92.8000,225.7000) -- (89.3000,225.8000) -- (85.9000,226.4000) -- (82.6000,227.4000) -- (79.5000,228.7000) -- (76.7000,230.5000) -- (74.3000,232.5000) -- (72.2000,234.8000) -- (70.4000,237.3000) -- (69.1000,240.0000) -- (68.4000,242.8000) -- (68.1000,245.7000) -- (68.1000,248.6000) -- (68.1000,251.5000) -- (68.1000,254.4000) -- (68.1000,257.3000) -- (68.1000,260.2000) -- (68.1000,263.1000) -- (68.1000,266.0000) -- (68.1000,268.9000) -- (68.1000,271.8000) -- (68.1000,274.6000) -- (68.1000,277.5000) -- (68.1000,280.4000) -- (68.1000,283.3000) -- (68.0000,286.2000) -- (68.3000,289.1000) -- (68.9000,291.9000) -- (70.0000,294.7000) -- (71.5000,297.3000) -- (73.4000,299.7000) -- (75.7000,301.9000) -- (78.3000,303.8000) -- (81.2000,305.4000) -- (84.3000,306.7000) -- (87.6000,307.6000) -- (91.1000,308.1000) -- (94.6000,308.2000) -- (98.1000,307.8000) -- (101.5000,307.1000) -- (104.7000,305.9000) -- (107.7000,304.4000) -- (110.5000,302.5000) -- (112.9000,300.3000) -- (114.9000,297.9000) -- (116.4000,295.3000) -- (117.4000,292.5000) -- (117.9000,289.7000) -- (118.2000,286.9000) -- (118.3000,284.0000) -- (118.4000,281.1000) -- (118.4000,278.3000) -- (118.5000,275.4000) -- (118.5000,272.5000) -- (118.5000,269.6000) -- (118.5000,266.7000) -- (118.5000,263.8000) -- (118.5000,260.9000) -- (118.5000,258.0000) -- (118.5000,255.1000) -- (118.5000,252.2000) -- (118.9000,249.4000) -- (119.1000,246.5000) -- (118.9000,243.6000) -- (118.3000,240.8000) -- (117.2000,238.0000) -- (115.6000,235.4000) -- (113.6000,233.0000) -- (111.3000,230.9000) -- (108.5000,229.0000) -- (105.5000,227.5000) -- (102.2000,226.4000) -- (98.8000,225.8000) -- (95.3000,225.6000) -- (91.8000,225.8000) -- (88.4000,226.5000) -- (85.2000,227.5000) -- (82.2000,229.0000) -- (79.4000,230.8000) -- (77.1000,232.9000) -- (75.0000,235.3000) -- (73.4000,237.8000) -- (72.3000,240.6000) -- (71.7000,243.4000) -- (71.5000,246.3000) -- (71.5000,249.2000) -- (71.5000,252.1000) -- (71.5000,255.0000) -- (71.5000,257.9000) -- (71.5000,260.8000) -- (71.5000,263.7000) -- (71.5000,266.6000) -- (71.5000,269.5000) -- (71.5000,272.4000) -- (71.5000,275.3000) -- (71.5000,278.1000) -- (71.5000,281.0000) -- (71.5000,283.9000) -- (71.5000,286.8000) -- (71.9000,289.7000) -- (72.7000,292.5000) -- (73.9000,295.2000) -- (75.5000,297.8000) -- (77.5000,300.1000) -- (79.9000,302.2000) -- (82.6000,304.1000) -- (85.6000,305.6000) -- (88.8000,306.7000) -- (92.2000,307.5000) -- (95.6000,307.8000) -- (99.1000,307.8000) -- (102.6000,307.3000) -- (106.0000,306.4000) -- (109.2000,305.1000) -- (112.1000,303.4000) -- (114.7000,301.4000) -- (117.0000,299.2000) -- (118.9000,296.7000) -- (120.2000,294.0000) -- (120.9000,291.2000) -- (121.3000,288.3000) -- (121.6000,285.5000) -- (121.7000,282.6000) -- (121.8000,279.8000) -- (121.8000,276.9000) -- (121.8000,274.0000) -- (121.8000,271.1000) -- (121.8000,268.2000) -- (121.8000,265.3000) -- (121.8000,262.4000) -- (121.8000,259.5000) -- (121.8000,256.6000) -- (121.8000,253.7000) -- (122.0000,250.9000) -- (122.4000,248.0000) -- (122.4000,245.1000) -- (122.1000,242.2000) -- (121.2000,239.4000) -- (120.0000,236.7000) -- (118.3000,234.2000) -- (116.1000,231.9000) -- (113.6000,229.9000) -- (110.8000,228.1000) -- (107.6000,226.8000) -- (104.3000,225.9000) -- (100.8000,225.4000) -- (97.3000,225.3000) -- (93.8000,225.7000) -- (90.5000,226.6000) -- (87.3000,227.8000) -- (84.4000,229.4000) -- (81.8000,231.3000) -- (79.6000,233.5000) -- (77.7000,236.0000) -- (76.3000,238.6000) -- (75.3000,241.4000) -- (75.0000,244.3000) -- (74.9000,247.2000) -- (74.9000,250.1000) -- (74.9000,253.0000) -- (74.9000,255.9000) -- (74.9000,258.8000) -- (74.9000,261.6000) -- (74.9000,264.5000) -- (74.9000,267.4000) -- (74.9000,270.3000) -- (74.9000,273.2000) -- (74.9000,276.1000) -- (74.9000,279.0000) -- (74.9000,281.9000) -- (74.9000,284.8000) -- (75.0000,287.7000) -- (75.5000,290.5000) -- (76.5000,293.3000) -- (77.8000,296.0000) -- (79.6000,298.4000) -- (81.8000,300.7000) -- (84.3000,302.7000) -- (87.1000,304.4000) -- (90.2000,305.8000) -- (93.5000,306.8000) -- (96.9000,307.4000) -- (100.4000,307.6000) -- (103.9000,307.3000) -- (107.3000,306.7000) -- (110.6000,305.6000) -- (113.7000,304.2000) -- (116.5000,302.4000) -- (119.0000,300.3000) -- (121.1000,297.9000) -- (122.8000,295.3000) -- (123.9000,292.6000) -- (124.5000,289.7000) -- (124.8000,286.9000) -- (125.0000,284.1000) -- (125.1000,281.2000) -- (125.1000,278.3000) -- (125.1000,275.4000) -- (125.2000,272.5000) -- (125.2000,269.6000) -- (125.2000,266.8000) -- (125.2000,263.9000) -- (125.2000,261.0000) -- (125.2000,258.1000) -- (125.2000,255.2000) -- (125.2000,252.3000) -- (125.5000,249.4000) -- (125.8000,246.5000) -- (125.7000,243.6000) -- (125.1000,240.8000) -- (124.1000,238.1000) -- (122.7000,235.4000) -- (120.8000,233.0000) -- (118.5000,230.8000) -- (115.9000,228.8000) -- (112.9000,227.3000) -- (109.6000,226.1000) -- (106.2000,225.3000) -- (102.7000,225.0000) -- (99.3000,225.2000) -- (95.8000,225.7000) -- (92.6000,226.7000) -- (89.5000,228.1000) -- (86.7000,229.8000) -- (84.2000,231.9000) -- (82.2000,234.2000) -- (80.5000,236.7000) -- (79.2000,239.4000) -- (78.5000,242.3000) -- (78.3000,245.2000) -- (78.2000,248.1000) -- (78.2000,251.0000) -- (78.2000,253.9000) -- (78.2000,256.7000) -- (78.2000,259.6000) -- (78.2000,262.5000) -- (78.2000,265.4000) -- (78.2000,268.3000) -- (78.2000,271.2000) -- (78.2000,274.1000) -- (78.2000,277.0000) -- (78.2000,279.9000) -- (78.2000,282.8000) -- (78.2000,285.7000) -- (78.5000,288.5000) -- (79.2000,291.4000) -- (80.3000,294.1000) -- (81.9000,296.7000) -- (83.8000,299.1000) -- (86.1000,301.3000) -- (88.8000,303.1000) -- (91.7000,304.7000) -- (94.9000,305.9000) -- (98.2000,306.8000) -- (101.7000,307.2000) -- (105.2000,307.2000) -- (108.7000,306.8000) -- (112.0000,306.0000) -- (115.3000,304.8000) -- (118.3000,303.2000) -- (120.9000,301.2000) -- (123.3000,299.0000) -- (125.2000,296.6000) -- (126.7000,293.9000) -- (127.5000,291.1000) -- (128.0000,288.3000) -- (128.2000,285.4000) -- (128.4000,282.6000) -- (128.5000,279.7000) -- (128.5000,276.8000) -- (128.5000,273.9000) -- (128.5000,271.0000) -- (128.5000,268.2000) -- (128.5000,265.3000) -- (128.5000,262.4000) -- (128.5000,259.5000) -- (128.5000,256.6000) -- (128.5000,253.7000) -- (128.6000,250.8000) -- (129.0000,247.9000) -- (129.2000,245.0000) -- (128.9000,242.2000) -- (128.1000,239.4000) -- (126.9000,236.7000) -- (125.3000,234.1000) -- (123.3000,231.7000) -- (120.8000,229.6000) -- (118.0000,227.9000) -- (114.9000,226.4000) -- (111.6000,225.4000) -- (108.2000,224.9000) -- (104.6000,224.7000) -- (101.2000,225.1000) -- (97.8000,225.8000) -- (94.6000,227.0000) -- (91.7000,228.5000) -- (89.0000,230.4000) -- (86.7000,232.5000) -- (84.8000,234.9000) -- (83.2000,237.5000) -- (82.2000,240.3000) -- (81.7000,243.2000) -- (81.6000,246.1000) -- (81.6000,249.0000) -- (81.6000,251.9000) -- (81.6000,254.8000) -- (81.6000,257.7000) -- (81.6000,260.6000) -- (81.6000,263.4000) -- (81.6000,266.3000) -- (81.6000,269.2000) -- (81.6000,272.1000) -- (81.6000,275.0000) -- (81.6000,277.9000) -- (81.6000,280.8000) -- (81.6000,283.7000) -- (81.6000,286.6000) -- (82.1000,289.4000) -- (83.0000,292.2000) -- (84.3000,294.9000) -- (86.0000,297.4000) -- (88.1000,299.7000) -- (90.6000,301.8000) -- (93.3000,303.5000) -- (96.4000,305.0000) -- (99.6000,306.0000) -- (103.0000,306.7000) -- (106.5000,307.0000) -- (110.0000,306.8000) -- (113.4000,306.2000) -- (116.8000,305.2000) -- (119.9000,303.8000) -- (122.8000,302.1000) -- (125.3000,300.1000) -- (127.5000,297.7000) -- (129.3000,295.2000) -- (130.4000,292.5000) -- (131.1000,289.6000) -- (131.5000,286.8000) -- (131.7000,284.0000) -- (131.8000,281.1000) -- (131.8000,278.2000) -- (131.9000,275.3000) -- (131.9000,272.4000) -- (131.9000,269.6000) -- (131.9000,266.7000) -- (131.9000,263.8000) -- (131.9000,260.9000) -- (131.9000,258.0000) -- (131.9000,255.1000) -- (131.9000,252.2000) -- (132.1000,249.3000) -- (132.5000,246.5000) -- (132.5000,243.6000) -- (132.0000,240.7000) -- (131.1000,237.9000) -- (129.7000,235.3000) -- (127.9000,232.8000) -- (125.7000,230.5000) -- (123.1000,228.6000) -- (120.1000,226.9000) -- (116.9000,225.7000) -- (113.5000,224.8000) -- (110.1000,224.5000) -- (106.6000,224.5000) -- (103.1000,225.0000) -- (99.8000,225.9000) -- (96.7000,227.3000) -- (93.9000,228.9000) -- (91.4000,230.9000) -- (89.2000,233.2000) -- (87.4000,235.7000) -- (86.1000,238.4000) -- (85.3000,241.2000) -- (85.0000,244.1000) -- (85.0000,247.0000) -- (84.9000,249.9000) -- (84.9000,252.8000) -- (84.9000,255.7000) -- (84.9000,258.6000) -- (84.9000,261.4000) -- (84.9000,264.3000) -- (84.9000,267.2000) -- (84.9000,270.1000) -- (84.9000,273.0000) -- (84.9000,275.9000) -- (84.9000,278.8000) -- (84.9000,281.7000) -- (84.9000,284.6000) -- (85.1000,287.5000) -- (85.8000,290.3000) -- (86.8000,293.1000) -- (88.3000,295.7000) -- (90.2000,298.1000) -- (92.4000,300.3000) -- (95.0000,302.3000) -- (97.9000,303.9000) -- (101.0000,305.2000) -- (104.3000,306.1000) -- (107.8000,306.6000) -- (111.3000,306.7000) -- (114.8000,306.3000) -- (118.2000,305.6000) -- (121.4000,304.4000) -- (124.5000,302.9000) -- (127.2000,301.0000) -- (129.6000,298.9000) -- (131.6000,296.5000) -- (133.2000,293.8000) -- (134.1000,291.0000) -- (134.7000,288.2000) -- (134.9000,285.4000) -- (135.1000,282.5000) -- (135.2000,279.7000) -- (135.2000,276.8000) -- (135.2000,273.9000) -- (135.2000,271.0000) -- (135.2000,268.1000) -- (135.2000,265.2000) -- (135.2000,262.3000) -- (135.2000,259.4000) -- (135.2000,256.5000) -- (135.2000,253.6000) -- (135.3000,250.8000) -- (135.7000,247.9000) -- (135.9000,245.0000) -- (135.7000,242.1000) -- (135.0000,239.3000) -- (133.9000,236.6000) -- (132.4000,234.0000) -- (130.4000,231.6000) -- (128.0000,229.4000) -- (125.3000,227.6000) -- (122.2000,226.1000) -- (118.9000,225.0000) -- (115.5000,224.3000) -- (112.0000,224.1000) -- (108.5000,224.4000) -- (105.1000,225.0000) -- (101.9000,226.1000) -- (98.9000,227.6000) -- (96.2000,229.4000) -- (93.8000,231.5000) -- (91.8000,233.9000) -- (90.2000,236.4000) -- (89.1000,239.2000) -- (88.5000,242.0000) -- (88.3000,244.9000) -- (88.3000,247.8000) -- (88.3000,250.7000) -- (88.3000,253.6000) -- (88.3000,256.5000) -- (88.3000,259.4000) -- (88.3000,262.3000) -- (88.3000,265.2000) -- (88.3000,268.1000) -- (88.3000,271.0000) -- (88.3000,273.9000) -- (88.3000,276.8000) -- (88.3000,279.7000) -- (88.3000,282.5000) -- (88.3000,285.4000) -- (88.7000,288.3000) -- (89.5000,291.1000) -- (90.7000,293.8000) -- (92.4000,296.4000) -- (94.4000,298.7000) -- (96.8000,300.8000) -- (99.5000,302.7000) -- (102.5000,304.1000) -- (105.7000,305.3000) -- (109.0000,306.0000) -- (112.5000,306.4000) -- (116.0000,306.3000) -- (119.5000,305.8000) -- (122.8000,304.9000) -- (126.0000,303.6000) -- (129.0000,301.9000) -- (131.6000,299.9000) -- (133.8000,297.6000) -- (135.7000,295.1000) -- (137.0000,292.4000) -- (137.7000,289.6000) -- (138.1000,286.8000) -- (138.4000,284.0000) -- (138.5000,281.1000) -- (138.5000,278.2000) -- (138.6000,275.3000) -- (138.6000,272.4000) -- (138.6000,269.5000) -- (138.6000,266.7000) -- (138.6000,263.8000) -- (138.6000,260.9000) -- (138.6000,258.0000) -- (138.6000,255.1000) -- (138.6000,252.2000) -- (138.8000,249.3000) -- (139.2000,246.4000) -- (139.2000,243.5000) -- (138.8000,240.7000) -- (138.0000,237.9000) -- (136.7000,235.2000) -- (135.0000,232.7000) -- (132.9000,230.4000) -- (130.3000,228.4000) -- (127.4000,226.6000) -- (124.3000,225.3000) -- (121.0000,224.4000) -- (117.5000,223.9000) -- (114.0000,223.9000) -- (110.5000,224.3000) -- (107.2000,225.1000) -- (104.0000,226.4000) -- (101.1000,228.0000) -- (98.6000,229.9000) -- (96.3000,232.1000) -- (94.5000,234.6000) -- (93.0000,237.2000) -- (92.1000,240.0000) -- (91.8000,242.9000) -- (91.7000,245.8000) -- (91.7000,248.7000) -- (91.7000,251.6000) -- (91.6000,254.5000) -- (91.6000,257.4000) -- (91.6000,260.3000) -- (91.6000,263.2000) -- (91.6000,266.1000) -- (91.6000,269.0000) -- (91.6000,271.8000) -- (91.6000,274.7000) -- (91.6000,277.6000) -- (91.6000,280.5000) -- (91.6000,283.4000) -- (91.8000,286.3000) -- (92.3000,289.2000) -- (93.3000,291.9000) -- (94.7000,294.6000) -- (96.5000,297.1000) -- (98.7000,299.3000) -- (101.2000,301.3000) -- (104.0000,303.0000) -- (107.1000,304.4000) -- (110.3000,305.3000) -- (113.8000,305.9000) -- (117.3000,306.1000) -- (120.8000,305.8000) -- (124.2000,305.2000) -- (127.5000,304.1000) -- (130.6000,302.6000) -- (133.4000,300.8000) -- (135.9000,298.7000) -- (138.0000,296.4000) -- (139.6000,293.8000) -- (140.7000,291.0000) -- (141.3000,288.2000) -- (141.6000,285.4000) -- (141.8000,282.5000) -- (141.9000,279.7000) -- (141.9000,276.8000) -- (141.9000,273.9000) -- (141.9000,271.0000) -- (142.0000,268.1000) -- (142.0000,265.2000) -- (142.0000,262.3000) -- (142.0000,259.4000) -- (142.0000,256.5000) -- (142.0000,253.6000) -- (142.0000,250.8000) -- (142.3000,247.9000) -- (142.6000,245.0000) -- (142.5000,242.1000) -- (141.9000,239.3000) -- (140.9000,236.5000) -- (139.4000,233.9000) -- (137.5000,231.5000) -- (135.2000,229.2000) -- (132.6000,227.3000) -- (129.6000,225.8000) -- (126.3000,224.6000) -- (122.9000,223.9000) -- (119.4000,223.6000) -- (115.9000,223.7000) -- (112.5000,224.3000) -- (109.3000,225.3000) -- (106.2000,226.7000) -- (103.4000,228.4000) -- (101.0000,230.5000) -- (98.9000,232.8000) -- (97.2000,235.3000) -- (96.0000,238.0000) -- (95.3000,240.9000) -- (95.1000,243.8000) -- (95.0000,246.7000) -- (95.0000,249.6000) -- (95.0000,252.5000) -- (95.0000,255.4000) -- (95.0000,258.3000) -- (95.0000,261.1000) -- (95.0000,264.0000) -- (95.0000,266.9000) -- (95.0000,269.8000) -- (95.0000,272.7000) -- (95.0000,275.6000) -- (95.0000,278.5000) -- (95.0000,281.4000) -- (95.0000,284.3000) -- (95.3000,287.1000) -- (96.0000,290.0000) -- (97.2000,292.7000) -- (98.7000,295.3000) -- (100.7000,297.7000) -- (103.0000,299.9000) -- (105.6000,301.7000) -- (108.6000,303.3000) -- (111.7000,304.5000) -- (115.1000,305.3000) -- (118.5000,305.8000) -- (122.0000,305.8000) -- (125.5000,305.3000) -- (128.9000,304.5000) -- (132.1000,303.3000) -- (135.1000,301.7000) -- (137.8000,299.7000) -- (140.1000,297.5000) -- (142.1000,295.1000) -- (143.5000,292.4000) -- (144.3000,289.6000) -- (144.8000,286.8000) -- (145.0000,283.9000) -- (145.2000,281.1000) -- (145.2000,278.2000) -- (145.3000,275.3000) -- (145.3000,272.4000) -- (145.3000,269.5000) -- (145.3000,266.7000) -- (145.3000,263.8000) -- (145.3000,260.9000) -- (145.3000,258.0000) -- (145.3000,255.1000) -- (145.3000,252.2000) -- (145.4000,249.3000) -- (145.8000,246.4000) -- (146.0000,243.6000) -- (145.7000,240.7000) -- (144.9000,237.9000) -- (143.7000,235.2000) -- (142.1000,232.6000) -- (140.0000,230.3000) -- (137.6000,228.2000) -- (134.8000,226.4000) -- (131.7000,225.0000) -- (128.4000,224.0000) -- (124.9000,223.4000) -- (121.4000,223.3000) -- (117.9000,223.6000) -- (114.6000,224.3000) -- (111.4000,225.5000) -- (108.4000,227.0000) -- (105.8000,228.9000) -- (103.5000,231.1000) -- (101.5000,233.5000) -- (100.0000,236.1000) -- (99.0000,238.9000) -- (98.5000,241.7000) -- (98.4000,244.6000) -- (98.4000,247.5000) -- (98.4000,250.4000) -- (98.4000,253.3000) -- (98.4000,256.2000) -- (98.4000,259.1000) -- (98.4000,262.0000) -- (98.4000,264.9000) -- (98.4000,267.8000) -- (98.4000,270.7000) -- (98.4000,273.6000) -- (98.4000,276.5000) -- (98.4000,279.3000) -- (98.4000,282.2000) -- (98.4000,285.1000) -- (98.9000,288.0000) -- (99.8000,290.8000) -- (101.1000,293.5000) -- (102.8000,296.0000) -- (104.9000,298.3000) -- (107.4000,300.3000) -- (110.1000,302.1000) -- (113.1000,303.5000) -- (116.4000,304.6000) -- (119.8000,305.3000) -- (123.3000,305.5000) -- (126.8000,305.3000) -- (130.2000,304.8000) -- (133.6000,303.8000) -- (136.7000,302.4000) -- (139.6000,300.7000) -- (142.1000,298.6000) -- (144.3000,296.3000) -- (146.1000,293.7000) -- (147.2000,291.0000) -- (147.9000,288.2000) -- (148.3000,285.4000) -- (148.5000,282.5000) -- (148.6000,279.7000) -- (148.6000,276.8000) -- (148.6000,273.9000) -- (148.7000,271.0000) -- (148.7000,268.1000) -- (148.7000,265.2000) -- (148.7000,262.3000) -- (148.7000,259.4000) -- (148.7000,256.5000) -- (148.7000,253.6000) -- (148.7000,250.8000) -- (148.9000,247.9000) -- (149.3000,245.0000) -- (149.3000,242.1000) -- (148.8000,239.3000) -- (147.8000,236.5000) -- (146.5000,233.8000) -- (144.7000,231.3000) -- (142.5000,229.1000) -- (139.9000,227.1000) -- (136.9000,225.5000) -- (133.7000,224.2000) -- (130.3000,223.4000) -- (126.9000,223.0000) -- (123.4000,223.1000) -- (119.9000,223.6000) -- (116.6000,224.5000) -- (113.5000,225.8000) -- (110.7000,227.5000) -- (108.2000,229.5000) -- (106.0000,231.7000) -- (104.2000,234.2000) -- (102.9000,236.9000) -- (102.1000,239.7000) -- (101.8000,242.6000) -- (101.7000,245.5000) -- (101.7000,248.4000) -- (101.7000,251.3000) -- (101.7000,254.2000) -- (101.7000,257.1000) -- (101.7000,260.0000) -- (101.7000,262.9000) -- (101.7000,265.8000) -- (101.7000,268.6000) -- (101.7000,271.5000) -- (101.7000,274.4000) -- (101.7000,277.3000) -- (101.7000,280.2000) -- (101.7000,283.1000) -- (101.9000,286.0000) -- (102.5000,288.8000) -- (103.6000,291.6000) -- (105.1000,294.2000) -- (107.0000,296.6000) -- (109.2000,298.9000) -- (111.8000,300.8000) -- (114.7000,302.4000) -- (117.8000,303.7000) -- (121.1000,304.6000) -- (124.5000,305.1000) -- (128.0000,305.2000) -- (131.5000,304.9000) -- (134.9000,304.1000) -- (138.2000,303.0000) -- (141.2000,301.4000) -- (144.0000,299.6000) -- (146.4000,297.4000) -- (148.4000,295.0000) -- (150.0000,292.4000) -- (150.9000,289.6000) -- (151.4000,286.8000) -- (151.7000,283.9000) -- (151.9000,281.1000) -- (152.0000,278.2000) -- (152.0000,275.3000) -- (152.0000,272.4000) -- (152.0000,269.5000) -- (152.0000,266.7000) -- (152.0000,263.8000) -- (152.0000,260.9000) -- (152.0000,258.0000) -- (152.0000,255.1000) -- (152.0000,252.2000) -- (152.1000,249.3000) -- (152.4000,246.4000) -- (152.7000,243.6000) -- (152.5000,240.7000) -- (151.8000,237.8000) -- (150.7000,235.1000) -- (149.2000,232.5000) -- (147.2000,230.1000) -- (144.8000,228.0000) -- (142.1000,226.1000) -- (139.0000,224.6000) -- (135.8000,223.5000) -- (132.3000,222.9000) -- (128.8000,222.7000) -- (125.3000,222.9000) -- (121.9000,223.6000) -- (118.7000,224.7000) -- (115.7000,226.1000) -- (113.0000,227.9000) -- (110.6000,230.0000) -- (108.6000,232.4000) -- (107.0000,235.0000) -- (105.8000,237.7000) -- (105.3000,240.6000) -- (105.1000,243.5000) -- (105.1000,246.4000) -- (105.1000,249.3000) -- (105.1000,252.2000) -- (105.1000,255.1000) -- (105.1000,257.9000) -- (105.1000,260.8000) -- (105.1000,263.7000) -- (105.1000,266.6000) -- (105.1000,269.5000) -- (105.1000,272.4000) -- (105.1000,275.3000) -- (105.1000,278.2000) -- (105.1000,281.1000) -- (105.1000,284.0000) -- (105.5000,286.8000) -- (106.3000,289.7000) -- (107.5000,292.4000) -- (109.1000,294.9000) -- (111.2000,297.3000) -- (113.6000,299.4000) -- (116.3000,301.2000) -- (119.2000,302.7000) -- (122.4000,303.8000) -- (125.8000,304.6000) -- (129.3000,304.9000) -- (132.8000,304.8000) -- (136.3000,304.3000) -- (139.6000,303.4000) -- (142.8000,302.1000) -- (145.7000,300.4000) -- (148.3000,298.5000) -- (150.6000,296.2000) -- (152.5000,293.7000) -- (153.8000,291.0000) -- (154.5000,288.2000) -- (154.9000,285.3000) -- (155.1000,282.5000) -- (155.3000,279.7000) -- (155.3000,276.8000) -- (155.4000,273.9000) -- (155.4000,271.0000) -- (155.4000,268.1000) -- (155.4000,265.2000) -- (155.4000,262.3000) -- (155.4000,259.4000) -- (155.4000,256.5000) -- (155.4000,253.6000) -- (155.4000,250.8000) -- (155.5000,247.9000) -- (155.9000,244.9000);



  \end{scope}
  \begin{scope}[scale=1.006,draw=black,line join=bevel,line cap=rect,line width=0.800pt]
  \end{scope}
  \begin{scope}[scale=1.006,draw=black,line join=bevel,line cap=rect,line width=0.800pt]
  \end{scope}
  \begin{scope}[scale=1.006,draw=black,line join=round,line cap=round,line width=0.480pt]
    \path[draw] (44.5000,217.5000) -- (44.5000,319.5000) -- (179.5000,319.5000) -- (179.5000,217.5000) -- (44.5000,217.5000);



  \end{scope}
  \begin{scope}[scale=1.006,draw=ca0a0a4,dash pattern=on 0.40pt off 0.80pt,line join=round,line cap=round,line width=0.400pt]
    \path[draw] (184.5000,267.5000) -- (246.5000,267.5000);



  \end{scope}
  \begin{scope}[scale=1.006,draw=black,line join=round,line cap=round,line width=0.480pt]
    \path[draw] (184.5000,267.5000) -- (186.5000,267.5000);



    \path[draw] (246.5000,267.5000) -- (245.5000,267.5000);



  \end{scope}
  \begin{scope}[scale=1.006,draw=black,line join=bevel,line cap=rect,line width=0.800pt]
  \end{scope}
  \begin{scope}[cm={{1.00588,0.0,0.0,1.00588,(181.059,268.571)}},draw=black,line join=bevel,line cap=rect,line width=0.800pt]
  \end{scope}
  \begin{scope}[cm={{1.00588,0.0,0.0,1.00588,(181.059,268.571)}},draw=black,line join=bevel,line cap=rect,line width=0.800pt]
  \end{scope}
  \begin{scope}[cm={{1.00588,0.0,0.0,1.00588,(181.059,268.571)}},draw=black,line join=bevel,line cap=rect,line width=0.800pt]
  \end{scope}
  \begin{scope}[cm={{1.00588,0.0,0.0,1.00588,(181.059,268.571)}},draw=black,line join=bevel,line cap=rect,line width=0.800pt]
  \end{scope}
  \begin{scope}[cm={{1.00588,0.0,0.0,1.00588,(181.059,268.571)}},draw=black,line join=bevel,line cap=rect,line width=0.800pt]
  \end{scope}
  \begin{scope}[cm={{1.00588,0.0,0.0,1.00588,(181.059,268.571)}},draw=black,line join=bevel,line cap=rect,line width=0.800pt]
  \end{scope}
  \begin{scope}[scale=1.006,draw=black,line join=bevel,line cap=rect,line width=0.800pt]
  \end{scope}
  \begin{scope}[scale=1.006,draw=ca0a0a4,dash pattern=on 0.40pt off 0.80pt,line join=round,line cap=round,line width=0.400pt]
    \path[draw] (184.5000,248.5000) -- (246.5000,248.5000);



  \end{scope}
  \begin{scope}[scale=1.006,draw=black,line join=round,line cap=round,line width=0.480pt]
    \path[draw] (184.5000,248.5000) -- (186.5000,248.5000);



    \path[draw] (246.5000,248.5000) -- (245.5000,248.5000);



  \end{scope}
  \begin{scope}[scale=1.006,draw=black,line join=bevel,line cap=rect,line width=0.800pt]
  \end{scope}
  \begin{scope}[cm={{1.00588,0.0,0.0,1.00588,(181.059,249.459)}},draw=black,line join=bevel,line cap=rect,line width=0.800pt]
  \end{scope}
  \begin{scope}[cm={{1.00588,0.0,0.0,1.00588,(181.059,249.459)}},draw=black,line join=bevel,line cap=rect,line width=0.800pt]
  \end{scope}
  \begin{scope}[cm={{1.00588,0.0,0.0,1.00588,(181.059,249.459)}},draw=black,line join=bevel,line cap=rect,line width=0.800pt]
  \end{scope}
  \begin{scope}[cm={{1.00588,0.0,0.0,1.00588,(181.059,249.459)}},draw=black,line join=bevel,line cap=rect,line width=0.800pt]
  \end{scope}
  \begin{scope}[cm={{1.00588,0.0,0.0,1.00588,(181.059,249.459)}},draw=black,line join=bevel,line cap=rect,line width=0.800pt]
  \end{scope}
  \begin{scope}[cm={{1.00588,0.0,0.0,1.00588,(181.059,249.459)}},draw=black,line join=bevel,line cap=rect,line width=0.800pt]
  \end{scope}
  \begin{scope}[scale=1.006,draw=black,line join=bevel,line cap=rect,line width=0.800pt]
  \end{scope}
  \begin{scope}[scale=1.006,draw=ca0a0a4,dash pattern=on 0.40pt off 0.80pt,line join=round,line cap=round,line width=0.400pt]
    \path[draw] (184.5000,229.5000) -- (246.5000,229.5000);



  \end{scope}
  \begin{scope}[scale=1.006,draw=black,line join=round,line cap=round,line width=0.480pt]
    \path[draw] (184.5000,229.5000) -- (186.5000,229.5000);



    \path[draw] (246.5000,229.5000) -- (245.5000,229.5000);



  \end{scope}
  \begin{scope}[scale=1.006,draw=black,line join=bevel,line cap=rect,line width=0.800pt]
  \end{scope}
  \begin{scope}[cm={{1.00588,0.0,0.0,1.00588,(181.059,230.347)}},draw=black,line join=bevel,line cap=rect,line width=0.800pt]
  \end{scope}
  \begin{scope}[cm={{1.00588,0.0,0.0,1.00588,(181.059,230.347)}},draw=black,line join=bevel,line cap=rect,line width=0.800pt]
  \end{scope}
  \begin{scope}[cm={{1.00588,0.0,0.0,1.00588,(181.059,230.347)}},draw=black,line join=bevel,line cap=rect,line width=0.800pt]
  \end{scope}
  \begin{scope}[cm={{1.00588,0.0,0.0,1.00588,(181.059,230.347)}},draw=black,line join=bevel,line cap=rect,line width=0.800pt]
  \end{scope}
  \begin{scope}[cm={{1.00588,0.0,0.0,1.00588,(181.059,230.347)}},draw=black,line join=bevel,line cap=rect,line width=0.800pt]
  \end{scope}
  \begin{scope}[cm={{1.00588,0.0,0.0,1.00588,(181.059,230.347)}},draw=black,line join=bevel,line cap=rect,line width=0.800pt]
  \end{scope}
  \begin{scope}[scale=1.006,draw=black,line join=bevel,line cap=rect,line width=0.800pt]
  \end{scope}
  \begin{scope}[scale=1.006,draw=ca0a0a4,dash pattern=on 0.40pt off 0.80pt,line join=round,line cap=round,line width=0.400pt]
    \path[draw] (184.5000,272.5000) -- (184.5000,224.5000);



  \end{scope}
  \begin{scope}[scale=1.006,draw=black,line join=round,line cap=round,line width=0.480pt]
    \path[draw] (184.5000,272.5000) -- (184.5000,270.5000);



    \path[draw] (184.5000,224.5000) -- (184.5000,225.5000);



  \end{scope}
  \begin{scope}[scale=1.006,draw=black,line join=bevel,line cap=rect,line width=0.800pt]
  \end{scope}
  \begin{scope}[cm={{1.00588,0.0,0.0,1.00588,(186.088,284.665)}},draw=black,line join=bevel,line cap=rect,line width=0.800pt]
  \end{scope}
  \begin{scope}[cm={{1.00588,0.0,0.0,1.00588,(186.088,284.665)}},draw=black,line join=bevel,line cap=rect,line width=0.800pt]
  \end{scope}
  \begin{scope}[cm={{1.00588,0.0,0.0,1.00588,(186.088,284.665)}},draw=black,line join=bevel,line cap=rect,line width=0.800pt]
  \end{scope}
  \begin{scope}[cm={{1.00588,0.0,0.0,1.00588,(186.088,284.665)}},draw=black,line join=bevel,line cap=rect,line width=0.800pt]
  \end{scope}
  \begin{scope}[cm={{1.00588,0.0,0.0,1.00588,(186.088,284.665)}},draw=black,line join=bevel,line cap=rect,line width=0.800pt]
  \end{scope}
  \begin{scope}[cm={{1.00588,0.0,0.0,1.00588,(186.088,284.665)}},draw=black,line join=bevel,line cap=rect,line width=0.800pt]
  \end{scope}
  \begin{scope}[scale=1.006,draw=black,line join=bevel,line cap=rect,line width=0.800pt]
  \end{scope}
  \begin{scope}[scale=1.006,draw=ca0a0a4,dash pattern=on 0.40pt off 0.80pt,line join=round,line cap=round,line width=0.400pt]
    \path[draw] (201.5000,272.5000) -- (201.5000,224.5000);



  \end{scope}
  \begin{scope}[scale=1.006,draw=black,line join=round,line cap=round,line width=0.480pt]
    \path[draw] (201.5000,272.5000) -- (201.5000,270.5000);



    \path[draw] (201.5000,224.5000) -- (201.5000,225.5000);



  \end{scope}
  \begin{scope}[scale=1.006,draw=black,line join=bevel,line cap=rect,line width=0.800pt]
  \end{scope}
  \begin{scope}[cm={{1.00588,0.0,0.0,1.00588,(203.188,284.665)}},draw=black,line join=bevel,line cap=rect,line width=0.800pt]
  \end{scope}
  \begin{scope}[cm={{1.00588,0.0,0.0,1.00588,(203.188,284.665)}},draw=black,line join=bevel,line cap=rect,line width=0.800pt]
  \end{scope}
  \begin{scope}[cm={{1.00588,0.0,0.0,1.00588,(203.188,284.665)}},draw=black,line join=bevel,line cap=rect,line width=0.800pt]
  \end{scope}
  \begin{scope}[cm={{1.00588,0.0,0.0,1.00588,(203.188,284.665)}},draw=black,line join=bevel,line cap=rect,line width=0.800pt]
  \end{scope}
  \begin{scope}[cm={{1.00588,0.0,0.0,1.00588,(203.188,284.665)}},draw=black,line join=bevel,line cap=rect,line width=0.800pt]
  \end{scope}
  \begin{scope}[cm={{1.00588,0.0,0.0,1.00588,(203.188,284.665)}},draw=black,line join=bevel,line cap=rect,line width=0.800pt]
  \end{scope}
  \begin{scope}[scale=1.006,draw=black,line join=bevel,line cap=rect,line width=0.800pt]
  \end{scope}
  \begin{scope}[scale=1.006,draw=ca0a0a4,dash pattern=on 0.40pt off 0.80pt,line join=round,line cap=round,line width=0.400pt]
    \path[draw] (218.5000,272.5000) -- (218.5000,224.5000);



  \end{scope}
  \begin{scope}[scale=1.006,draw=black,line join=round,line cap=round,line width=0.480pt]
    \path[draw] (218.5000,272.5000) -- (218.5000,270.5000);



    \path[draw] (218.5000,224.5000) -- (218.5000,225.5000);



  \end{scope}
  \begin{scope}[scale=1.006,draw=black,line join=bevel,line cap=rect,line width=0.800pt]
  \end{scope}
  \begin{scope}[cm={{1.00588,0.0,0.0,1.00588,(219.282,284.665)}},draw=black,line join=bevel,line cap=rect,line width=0.800pt]
  \end{scope}
  \begin{scope}[cm={{1.00588,0.0,0.0,1.00588,(219.282,284.665)}},draw=black,line join=bevel,line cap=rect,line width=0.800pt]
  \end{scope}
  \begin{scope}[cm={{1.00588,0.0,0.0,1.00588,(219.282,284.665)}},draw=black,line join=bevel,line cap=rect,line width=0.800pt]
  \end{scope}
  \begin{scope}[cm={{1.00588,0.0,0.0,1.00588,(219.282,284.665)}},draw=black,line join=bevel,line cap=rect,line width=0.800pt]
  \end{scope}
  \begin{scope}[cm={{1.00588,0.0,0.0,1.00588,(219.282,284.665)}},draw=black,line join=bevel,line cap=rect,line width=0.800pt]
  \end{scope}
  \begin{scope}[cm={{1.00588,0.0,0.0,1.00588,(219.282,284.665)}},draw=black,line join=bevel,line cap=rect,line width=0.800pt]
  \end{scope}
  \begin{scope}[scale=1.006,draw=black,line join=bevel,line cap=rect,line width=0.800pt]
  \end{scope}
  \begin{scope}[scale=1.006,draw=ca0a0a4,dash pattern=on 0.40pt off 0.80pt,line join=round,line cap=round,line width=0.400pt]
    \path[draw] (235.5000,272.5000) -- (235.5000,224.5000);



  \end{scope}
  \begin{scope}[scale=1.006,draw=black,line join=round,line cap=round,line width=0.480pt]
    \path[draw] (235.5000,272.5000) -- (235.5000,270.5000);



    \path[draw] (235.5000,224.5000) -- (235.5000,225.5000);



  \end{scope}
  \begin{scope}[scale=1.006,draw=black,line join=bevel,line cap=rect,line width=0.800pt]
  \end{scope}
  \begin{scope}[cm={{1.00588,0.0,0.0,1.00588,(236.382,284.665)}},draw=black,line join=bevel,line cap=rect,line width=0.800pt]
  \end{scope}
  \begin{scope}[cm={{1.00588,0.0,0.0,1.00588,(236.382,284.665)}},draw=black,line join=bevel,line cap=rect,line width=0.800pt]
  \end{scope}
  \begin{scope}[cm={{1.00588,0.0,0.0,1.00588,(236.382,284.665)}},draw=black,line join=bevel,line cap=rect,line width=0.800pt]
  \end{scope}
  \begin{scope}[cm={{1.00588,0.0,0.0,1.00588,(236.382,284.665)}},draw=black,line join=bevel,line cap=rect,line width=0.800pt]
  \end{scope}
  \begin{scope}[cm={{1.00588,0.0,0.0,1.00588,(236.382,284.665)}},draw=black,line join=bevel,line cap=rect,line width=0.800pt]
  \end{scope}
  \begin{scope}[cm={{1.00588,0.0,0.0,1.00588,(236.382,284.665)}},draw=black,line join=bevel,line cap=rect,line width=0.800pt]
  \end{scope}
  \begin{scope}[scale=1.006,draw=black,line join=bevel,line cap=rect,line width=0.800pt]
  \end{scope}
  \begin{scope}[scale=1.006,draw=black,line join=round,line cap=round,line width=0.480pt]
    \path[draw] (246.5000,267.5000) -- (245.5000,267.5000);



  \end{scope}
  \begin{scope}[scale=1.006,draw=black,line join=bevel,line cap=rect,line width=0.800pt]
  \end{scope}
  \begin{scope}[cm={{1.00588,0.0,0.0,1.00588,(249.459,272.594)}},draw=black,line join=bevel,line cap=rect,line width=0.800pt]
  \end{scope}
  \begin{scope}[cm={{1.00588,0.0,0.0,1.00588,(249.459,272.594)}},draw=black,line join=bevel,line cap=rect,line width=0.800pt]
  \end{scope}
  \begin{scope}[cm={{1.00588,0.0,0.0,1.00588,(249.459,272.594)}},draw=black,line join=bevel,line cap=rect,line width=0.800pt]
  \end{scope}
  \begin{scope}[cm={{1.00588,0.0,0.0,1.00588,(249.459,272.594)}},draw=black,line join=bevel,line cap=rect,line width=0.800pt]
  \end{scope}
  \begin{scope}[cm={{1.00588,0.0,0.0,1.00588,(249.459,272.594)}},draw=black,line join=bevel,line cap=rect,line width=0.800pt]
  \end{scope}
  \begin{scope}[cm={{1.00588,0.0,0.0,1.00588,(252.459,272.594)}},draw=black,line join=bevel,line cap=rect,line width=0.800pt]
    \path[fill=black] (0.0000,0.0000) node[above right] (text2120) {\scriptsize 2};



  \end{scope}
  \begin{scope}[cm={{1.00588,0.0,0.0,1.00588,(249.459,272.594)}},draw=black,line join=bevel,line cap=rect,line width=0.800pt]
  \end{scope}
  \begin{scope}[scale=1.006,draw=black,line join=bevel,line cap=rect,line width=0.800pt]
  \end{scope}
  \begin{scope}[scale=1.006,draw=black,line join=round,line cap=round,line width=0.480pt]
    \path[draw] (246.5000,248.5000) -- (245.5000,248.5000);



  \end{scope}
  \begin{scope}[scale=1.006,draw=black,line join=bevel,line cap=rect,line width=0.800pt]
  \end{scope}
  \begin{scope}[cm={{1.00588,0.0,0.0,1.00588,(249.459,253.482)}},draw=black,line join=bevel,line cap=rect,line width=0.800pt]
  \end{scope}
  \begin{scope}[cm={{1.00588,0.0,0.0,1.00588,(249.459,253.482)}},draw=black,line join=bevel,line cap=rect,line width=0.800pt]
  \end{scope}
  \begin{scope}[cm={{1.00588,0.0,0.0,1.00588,(249.459,253.482)}},draw=black,line join=bevel,line cap=rect,line width=0.800pt]
  \end{scope}
  \begin{scope}[cm={{1.00588,0.0,0.0,1.00588,(249.459,253.482)}},draw=black,line join=bevel,line cap=rect,line width=0.800pt]
  \end{scope}
  \begin{scope}[cm={{1.00588,0.0,0.0,1.00588,(249.459,253.482)}},draw=black,line join=bevel,line cap=rect,line width=0.800pt]
  \end{scope}
  \begin{scope}[cm={{1.00588,0.0,0.0,1.00588,(252.459,253.482)}},draw=black,line join=bevel,line cap=rect,line width=0.800pt]
    \path[fill=black] (0.0000,0.0000) node[above right] (text2144) {\scriptsize 6};



  \end{scope}
  \begin{scope}[cm={{1.00588,0.0,0.0,1.00588,(249.459,253.482)}},draw=black,line join=bevel,line cap=rect,line width=0.800pt]
  \end{scope}
  \begin{scope}[scale=1.006,draw=black,line join=bevel,line cap=rect,line width=0.800pt]
  \end{scope}
  \begin{scope}[scale=1.006,draw=black,line join=round,line cap=round,line width=0.480pt]
    \path[draw] (246.5000,229.5000) -- (245.5000,229.5000);



  \end{scope}
  \begin{scope}[scale=1.006,draw=black,line join=bevel,line cap=rect,line width=0.800pt]
  \end{scope}
  \begin{scope}[cm={{1.00588,0.0,0.0,1.00588,(249.459,234.371)}},draw=black,line join=bevel,line cap=rect,line width=0.800pt]
  \end{scope}
  \begin{scope}[cm={{1.00588,0.0,0.0,1.00588,(249.459,234.371)}},draw=black,line join=bevel,line cap=rect,line width=0.800pt]
  \end{scope}
  \begin{scope}[cm={{1.00588,0.0,0.0,1.00588,(249.459,234.371)}},draw=black,line join=bevel,line cap=rect,line width=0.800pt]
  \end{scope}
  \begin{scope}[cm={{1.00588,0.0,0.0,1.00588,(249.459,234.371)}},draw=black,line join=bevel,line cap=rect,line width=0.800pt]
  \end{scope}
  \begin{scope}[cm={{1.00588,0.0,0.0,1.00588,(249.459,234.371)}},draw=black,line join=bevel,line cap=rect,line width=0.800pt]
  \end{scope}
  \begin{scope}[cm={{1.00588,0.0,0.0,1.00588,(250.959,234.371)}},draw=black,line join=bevel,line cap=rect,line width=0.800pt]
    \path[fill=black] (0.0000,0.0000) node[above right] (text2168) {\scriptsize 10};



  \end{scope}
  \begin{scope}[cm={{1.00588,0.0,0.0,1.00588,(249.459,234.371)}},draw=black,line join=bevel,line cap=rect,line width=0.800pt]
  \end{scope}
  \begin{scope}[scale=1.006,draw=black,line join=bevel,line cap=rect,line width=0.800pt]
  \end{scope}
  \begin{scope}[scale=1.006,draw=black,line join=round,line cap=round,line width=0.480pt]
    \path[draw] (184.5000,224.5000) -- (184.5000,272.5000) -- (246.5000,272.5000) -- (246.5000,224.5000) -- (184.5000,224.5000);



  \end{scope}
  \begin{scope}[scale=1.006,draw=black,line join=bevel,line cap=rect,line width=0.800pt]
  \end{scope}
  \begin{scope}[cm={{0.0,-1.00588,1.00588,0.0,(276.618,254.991)}},draw=black,line join=bevel,line cap=rect,line width=0.800pt]
  \end{scope}
  \begin{scope}[cm={{0.0,-1.00588,1.00588,0.0,(276.618,254.991)}},draw=black,line join=bevel,line cap=rect,line width=0.800pt]
  \end{scope}
  \begin{scope}[cm={{0.0,-1.00588,1.00588,0.0,(276.618,254.991)}},draw=black,line join=bevel,line cap=rect,line width=0.800pt]
  \end{scope}
  \begin{scope}[cm={{0.0,-1.00588,1.00588,0.0,(276.618,254.991)}},draw=black,line join=bevel,line cap=rect,line width=0.800pt]
  \end{scope}
  \begin{scope}[cm={{0.0,-1.00588,1.00588,0.0,(276.618,254.991)}},draw=black,line join=bevel,line cap=rect,line width=0.800pt]
  \end{scope}
  \begin{scope}[cm={{0.0,-1.00588,1.00588,0.0,(263.618,254.991)}},draw=black,line join=bevel,line cap=rect,line width=0.800pt]
    \path[fill=black] (0.0000,0.0000) node[above right] (text2192) {\rotatebox{90}{\scriptsize $c_{1,2}$}};



  \end{scope}
  \begin{scope}[cm={{0.0,-1.00588,1.00588,0.0,(276.618,254.991)}},draw=black,line join=bevel,line cap=rect,line width=0.800pt]
  \end{scope}
  \begin{scope}[scale=1.006,draw=black,line join=bevel,line cap=rect,line width=0.800pt]
  \end{scope}
  \begin{scope}[scale=1.006,draw=black,line join=bevel,line cap=rect,line width=0.800pt]
  \end{scope}
  \begin{scope}[scale=1.006,draw=black,line join=bevel,line cap=rect,line width=0.800pt]
  \end{scope}
  \begin{scope}[scale=1.006,draw=black,line join=round,line cap=round,line width=0.480pt]
    \path[draw] (184.8000,267.2000) -- (184.8000,267.2000) -- (184.9000,267.2000) -- (185.0000,267.2000) -- (185.0000,267.2000) -- (185.1000,267.2000) -- (185.1000,267.2000) -- (185.2000,267.2000) -- (185.3000,267.2000) -- (185.3000,267.2000) -- (185.4000,267.2000) -- (185.4000,267.2000) -- (185.5000,267.2000) -- (185.6000,267.2000) -- (185.6000,267.2000) -- (185.7000,267.2000) -- (185.8000,267.2000) -- (185.8000,267.2000) -- (185.9000,267.2000) -- (185.9000,267.2000) -- (186.0000,267.2000) -- (186.1000,267.2000) -- (186.1000,267.2000) -- (186.2000,267.2000) -- (186.2000,267.2000) -- (186.3000,267.2000) -- (186.4000,267.2000) -- (186.4000,267.2000) -- (186.5000,267.2000) -- (186.6000,267.2000) -- (186.6000,267.2000) -- (186.7000,267.2000) -- (186.7000,267.2000) -- (186.8000,267.2000) -- (186.9000,267.2000) -- (186.9000,267.2000) -- (187.0000,267.2000) -- (187.0000,267.2000) -- (187.1000,267.2000) -- (187.2000,267.2000) -- (187.2000,267.2000) -- (187.3000,267.2000) -- (187.4000,267.2000) -- (187.4000,267.2000) -- (187.5000,267.2000) -- (187.5000,267.2000) -- (187.6000,267.2000) -- (187.7000,267.2000) -- (187.7000,267.2000) -- (187.8000,267.2000) -- (187.8000,267.2000) -- (187.9000,267.2000) -- (188.0000,267.2000) -- (188.0000,267.2000) -- (188.1000,267.2000) -- (188.2000,267.2000) -- (188.2000,267.2000) -- (188.3000,267.2000) -- (188.3000,267.2000) -- (188.4000,267.2000) -- (188.5000,267.2000) -- (188.5000,267.2000) -- (188.6000,267.2000) -- (188.6000,267.2000) -- (188.7000,267.2000) -- (188.8000,267.2000) -- (188.8000,267.2000) -- (188.9000,267.2000) -- (189.0000,267.2000) -- (189.0000,267.2000) -- (189.1000,267.2000) -- (189.1000,267.2000) -- (189.2000,267.2000) -- (189.3000,267.2000) -- (189.3000,267.2000) -- (189.4000,267.2000) -- (189.4000,267.2000) -- (189.5000,267.2000) -- (189.6000,267.2000) -- (189.6000,267.2000) -- (189.7000,267.2000) -- (189.8000,267.2000) -- (189.8000,267.2000) -- (189.9000,267.2000) -- (189.9000,267.2000) -- (190.0000,267.2000) -- (190.1000,267.2000) -- (190.1000,267.2000) -- (190.2000,267.2000) -- (190.2000,267.2000) -- (190.3000,267.2000) -- (190.4000,267.2000) -- (190.4000,267.2000) -- (190.5000,267.2000) -- (190.5000,267.2000) -- (190.6000,267.2000) -- (190.7000,267.2000) -- (190.7000,267.2000) -- (190.8000,267.2000) -- (190.9000,267.2000) -- (190.9000,267.2000) -- (191.0000,267.2000) -- (191.0000,267.2000) -- (191.1000,267.2000) -- (191.2000,267.2000) -- (191.2000,267.2000) -- (191.3000,267.2000) -- (191.3000,267.2000) -- (191.4000,267.2000) -- (191.5000,267.2000) -- (191.5000,267.2000) -- (191.6000,267.2000) -- (191.7000,267.2000) -- (191.7000,267.2000) -- (191.8000,267.2000) -- (191.8000,267.2000) -- (191.9000,267.2000) -- (192.0000,267.2000) -- (192.0000,267.2000) -- (192.1000,267.2000) -- (192.1000,267.2000) -- (192.2000,267.2000) -- (192.3000,267.2000) -- (192.3000,267.2000) -- (192.4000,267.2000) -- (192.5000,267.2000) -- (192.5000,267.2000) -- (192.6000,267.3000) -- (192.6000,268.7000) -- (192.7000,244.7000) -- (192.8000,227.7000) -- (192.8000,229.2000) -- (192.9000,229.2000) -- (192.9000,229.2000) -- (193.0000,229.2000) -- (193.1000,229.2000) -- (193.1000,229.2000) -- (193.2000,229.2000) -- (193.3000,229.2000) -- (193.3000,229.2000) -- (193.4000,229.2000) -- (193.4000,229.2000) -- (193.5000,229.2000) -- (193.6000,229.2000) -- (193.6000,229.2000) -- (193.7000,229.2000) -- (193.7000,229.2000) -- (193.8000,229.2000) -- (193.9000,229.2000) -- (193.9000,229.2000) -- (194.0000,229.2000) -- (194.1000,229.2000) -- (194.1000,229.2000) -- (194.2000,229.2000) -- (194.2000,229.2000) -- (194.3000,229.2000) -- (194.4000,229.2000) -- (194.4000,229.2000) -- (194.5000,229.2000) -- (194.5000,229.2000) -- (194.6000,229.2000) -- (194.7000,229.2000) -- (194.7000,229.2000) -- (194.8000,229.2000) -- (194.9000,229.2000) -- (194.9000,229.2000) -- (195.0000,229.2000) -- (195.0000,229.2000) -- (195.1000,229.2000) -- (195.2000,229.2000) -- (195.2000,229.2000) -- (195.3000,229.2000) -- (195.3000,229.2000) -- (195.4000,229.2000) -- (195.5000,229.2000) -- (195.5000,229.2000) -- (195.6000,229.2000) -- (195.7000,229.2000) -- (195.7000,229.2000) -- (195.8000,229.2000) -- (195.8000,229.2000) -- (195.9000,229.2000) -- (196.0000,229.2000) -- (196.0000,229.2000) -- (196.1000,229.2000) -- (196.1000,229.2000) -- (196.2000,229.2000) -- (196.3000,229.2000) -- (196.3000,229.2000) -- (196.4000,229.2000) -- (196.5000,229.2000) -- (196.5000,229.2000) -- (196.6000,229.2000) -- (196.6000,229.2000) -- (196.7000,229.2000) -- (196.8000,229.2000) -- (196.8000,229.2000) -- (196.9000,229.2000) -- (196.9000,229.2000) -- (197.0000,229.2000) -- (197.1000,229.2000) -- (197.1000,229.2000) -- (197.2000,229.2000) -- (197.3000,229.2000) -- (197.3000,229.2000) -- (197.4000,229.2000) -- (197.4000,229.2000) -- (197.5000,229.2000) -- (197.6000,229.2000) -- (197.6000,229.2000) -- (197.7000,229.2000) -- (197.7000,229.2000) -- (197.8000,229.2000) -- (197.9000,229.2000) -- (197.9000,229.2000) -- (198.0000,229.2000) -- (198.1000,229.2000) -- (198.1000,229.2000) -- (198.2000,229.2000) -- (198.2000,229.2000) -- (198.3000,229.2000) -- (198.4000,229.2000) -- (198.4000,229.2000) -- (198.5000,229.2000) -- (198.5000,229.2000) -- (198.6000,229.2000) -- (198.7000,229.2000) -- (198.7000,229.2000) -- (198.8000,229.2000) -- (198.9000,229.2000) -- (198.9000,229.2000) -- (199.0000,229.2000) -- (199.0000,229.2000) -- (199.1000,229.2000) -- (199.2000,229.2000) -- (199.2000,229.2000) -- (199.3000,229.2000) -- (199.3000,229.2000) -- (199.4000,229.2000) -- (199.5000,229.2000) -- (199.5000,229.2000) -- (199.6000,229.2000) -- (199.7000,229.2000) -- (199.7000,229.2000) -- (199.8000,229.2000) -- (199.8000,229.2000) -- (199.9000,229.2000) -- (200.0000,229.2000) -- (200.0000,229.2000) -- (200.1000,229.2000) -- (200.1000,229.2000) -- (200.2000,229.2000) -- (200.3000,229.2000) -- (200.3000,229.2000) -- (200.4000,229.2000) -- (200.5000,229.2000) -- (200.5000,229.2000) -- (200.6000,229.2000) -- (200.6000,229.2000) -- (200.7000,229.2000) -- (200.8000,229.2000) -- (200.8000,229.2000) -- (200.9000,229.2000) -- (200.9000,229.2000) -- (201.0000,229.2000) -- (201.1000,229.2000) -- (201.1000,229.2000) -- (201.2000,229.2000) -- (201.3000,229.2000) -- (201.3000,229.2000) -- (201.4000,229.2000) -- (201.4000,229.2000) -- (201.5000,229.2000) -- (201.6000,229.2000) -- (201.6000,229.2000) -- (201.7000,229.2000) -- (201.7000,229.2000) -- (201.8000,229.2000) -- (201.9000,229.2000) -- (201.9000,229.2000) -- (202.0000,229.2000) -- (202.0000,229.2000) -- (202.1000,229.2000) -- (202.2000,229.2000) -- (202.2000,229.2000) -- (202.3000,229.2000) -- (202.4000,229.2000) -- (202.4000,229.2000) -- (202.5000,229.2000) -- (202.5000,229.2000) -- (202.6000,229.2000) -- (202.7000,229.2000) -- (202.7000,229.2000) -- (202.8000,229.2000) -- (202.8000,229.2000) -- (202.9000,229.2000) -- (203.0000,229.2000) -- (203.0000,229.2000) -- (203.1000,229.2000) -- (203.2000,229.2000) -- (203.2000,229.2000) -- (203.3000,229.2000) -- (203.3000,229.2000) -- (203.4000,229.2000) -- (203.5000,229.2000) -- (203.5000,229.2000) -- (203.6000,229.2000) -- (203.6000,229.2000) -- (203.7000,229.2000) -- (203.8000,229.2000) -- (203.8000,229.2000) -- (203.9000,229.2000) -- (204.0000,229.2000) -- (204.0000,229.2000) -- (204.1000,229.2000) -- (204.1000,229.2000) -- (204.2000,229.2000) -- (204.3000,229.2000) -- (204.3000,229.2000) -- (204.4000,229.2000) -- (204.4000,229.2000) -- (204.5000,229.2000) -- (204.6000,229.2000) -- (204.6000,229.2000) -- (204.7000,229.2000) -- (204.8000,229.2000) -- (204.8000,229.2000) -- (204.9000,229.2000) -- (204.9000,229.2000) -- (205.0000,229.2000) -- (205.1000,229.2000) -- (205.1000,229.2000) -- (205.2000,229.2000) -- (205.2000,229.2000) -- (205.3000,229.2000) -- (205.4000,229.2000) -- (205.4000,229.2000) -- (205.5000,229.2000) -- (205.6000,229.2000) -- (205.6000,229.2000) -- (205.7000,229.2000) -- (205.7000,229.2000) -- (205.8000,229.2000) -- (205.9000,229.2000) -- (205.9000,229.2000) -- (206.0000,229.2000) -- (206.0000,229.2000) -- (206.1000,229.2000) -- (206.2000,229.2000) -- (206.2000,229.2000) -- (206.3000,229.2000) -- (206.4000,229.2000) -- (206.4000,229.2000) -- (206.5000,229.2000) -- (206.5000,229.2000) -- (206.6000,229.2000) -- (206.7000,229.2000) -- (206.7000,229.2000) -- (206.8000,229.2000) -- (206.8000,229.2000) -- (206.9000,229.2000) -- (207.0000,229.2000) -- (207.0000,229.2000) -- (207.1000,229.2000) -- (207.2000,229.2000) -- (207.2000,229.2000) -- (207.3000,229.2000) -- (207.3000,229.2000) -- (207.4000,229.2000) -- (207.5000,229.2000) -- (207.5000,229.2000) -- (207.6000,229.2000) -- (207.6000,229.2000) -- (207.7000,229.2000) -- (207.8000,229.2000) -- (207.8000,229.2000) -- (207.9000,229.2000) -- (208.0000,229.2000) -- (208.0000,229.2000) -- (208.1000,229.2000) -- (208.1000,229.2000) -- (208.2000,229.2000) -- (208.3000,229.2000) -- (208.3000,229.2000) -- (208.4000,229.2000) -- (208.4000,229.2000) -- (208.5000,229.2000) -- (208.6000,229.2000) -- (208.6000,229.2000) -- (208.7000,229.2000) -- (208.8000,229.2000) -- (208.8000,229.2000) -- (208.9000,229.2000) -- (208.9000,229.2000) -- (209.0000,229.2000) -- (209.1000,229.2000) -- (209.1000,229.2000) -- (209.2000,229.2000) -- (209.2000,229.2000) -- (209.3000,229.2000) -- (209.4000,229.2000) -- (209.4000,229.2000) -- (209.5000,229.2000) -- (209.6000,229.2000) -- (209.6000,229.2000) -- (209.7000,229.2000) -- (209.7000,229.2000) -- (209.8000,229.2000) -- (209.9000,229.2000) -- (209.9000,229.2000) -- (210.0000,229.2000) -- (210.0000,229.2000) -- (210.1000,229.2000) -- (210.2000,229.2000) -- (210.2000,229.2000) -- (210.3000,229.2000) -- (210.4000,229.2000) -- (210.4000,229.2000) -- (210.5000,229.2000) -- (210.5000,229.2000) -- (210.6000,229.2000) -- (210.7000,229.2000) -- (210.7000,229.2000) -- (210.8000,229.2000) -- (210.8000,229.2000) -- (210.9000,229.2000) -- (211.0000,229.2000) -- (211.0000,229.2000) -- (211.1000,229.2000) -- (211.2000,229.2000) -- (211.2000,229.2000) -- (211.3000,229.2000) -- (211.3000,229.2000) -- (211.4000,229.2000) -- (211.5000,229.2000) -- (211.5000,229.2000) -- (211.6000,229.2000) -- (211.6000,229.2000) -- (211.7000,229.2000) -- (211.8000,229.2000) -- (211.8000,229.2000) -- (211.9000,229.2000) -- (212.0000,229.2000) -- (212.0000,229.2000) -- (212.1000,229.2000) -- (212.1000,229.2000) -- (212.2000,229.2000) -- (212.3000,229.2000) -- (212.3000,229.2000) -- (212.4000,229.2000) -- (212.4000,229.2000) -- (212.5000,229.2000) -- (212.6000,229.2000) -- (212.6000,229.2000) -- (212.7000,229.2000) -- (212.8000,229.2000) -- (212.8000,229.2000) -- (212.9000,229.2000) -- (212.9000,229.2000) -- (213.0000,229.2000) -- (213.1000,229.2000) -- (213.1000,229.2000) -- (213.2000,229.2000) -- (213.2000,229.2000) -- (213.3000,229.2000) -- (213.4000,229.2000) -- (213.4000,229.2000) -- (213.5000,229.2000) -- (213.6000,229.2000) -- (213.6000,229.2000) -- (213.7000,229.2000) -- (213.7000,229.2000) -- (213.8000,229.2000) -- (213.9000,229.2000) -- (213.9000,229.2000) -- (214.0000,229.2000) -- (214.0000,229.2000) -- (214.1000,229.2000) -- (214.2000,229.2000) -- (214.2000,229.2000) -- (214.3000,229.2000) -- (214.4000,229.2000) -- (214.4000,229.2000) -- (214.5000,229.2000) -- (214.5000,229.2000) -- (214.6000,229.2000) -- (214.7000,229.2000) -- (214.7000,229.2000) -- (214.8000,229.2000) -- (214.8000,229.2000) -- (214.9000,229.2000) -- (215.0000,229.2000) -- (215.0000,229.2000) -- (215.1000,229.2000) -- (215.1000,229.2000) -- (215.2000,229.2000) -- (215.3000,229.2000) -- (215.3000,229.2000) -- (215.4000,229.2000) -- (215.5000,229.2000) -- (215.5000,229.2000) -- (215.6000,229.2000) -- (215.6000,229.2000) -- (215.7000,229.2000) -- (215.8000,229.2000) -- (215.8000,229.2000) -- (215.9000,229.2000) -- (215.9000,228.9000) -- (216.0000,233.3000) -- (216.1000,239.1000) -- (216.1000,238.8000) -- (216.2000,238.7000) -- (216.3000,238.7000) -- (216.3000,238.7000) -- (216.4000,238.7000) -- (216.4000,238.7000) -- (216.5000,238.7000) -- (216.6000,238.7000) -- (216.6000,238.7000) -- (216.7000,238.7000) -- (216.7000,238.7000) -- (216.8000,238.7000) -- (216.9000,238.7000) -- (216.9000,238.7000) -- (217.0000,238.7000) -- (217.1000,238.7000) -- (217.1000,238.7000) -- (217.2000,238.7000) -- (217.2000,238.7000) -- (217.3000,238.7000) -- (217.4000,238.7000) -- (217.4000,238.7000) -- (217.5000,238.7000) -- (217.5000,238.7000) -- (217.6000,239.0000) -- (217.7000,236.4000) -- (217.7000,229.1000) -- (217.8000,229.2000) -- (217.9000,229.2000) -- (217.9000,229.2000) -- (218.0000,229.2000) -- (218.0000,229.2000) -- (218.1000,229.2000) -- (218.2000,229.2000) -- (218.2000,229.2000) -- (218.3000,229.2000) -- (218.3000,229.2000) -- (218.4000,229.2000) -- (218.5000,229.2000) -- (218.5000,229.2000) -- (218.6000,229.2000) -- (218.7000,229.2000) -- (218.7000,229.2000) -- (218.8000,229.2000) -- (218.8000,229.2000) -- (218.9000,229.2000) -- (219.0000,229.2000) -- (219.0000,229.2000) -- (219.1000,229.2000) -- (219.1000,229.2000) -- (219.2000,229.2000) -- (219.3000,229.2000) -- (219.3000,229.2000) -- (219.4000,229.2000) -- (219.5000,229.2000) -- (219.5000,229.2000) -- (219.6000,229.2000) -- (219.6000,229.2000) -- (219.7000,229.2000) -- (219.8000,229.2000) -- (219.8000,229.2000) -- (219.9000,229.2000) -- (219.9000,229.2000) -- (220.0000,229.2000) -- (220.1000,229.2000) -- (220.1000,228.9000) -- (220.2000,233.5000) -- (220.3000,239.1000) -- (220.3000,238.8000) -- (220.4000,238.7000) -- (220.4000,239.1000) -- (220.5000,247.0000) -- (220.6000,248.4000) -- (220.6000,248.2000) -- (220.7000,248.2000) -- (220.7000,248.2000) -- (220.8000,248.2000) -- (220.9000,248.2000) -- (220.9000,248.2000) -- (221.0000,248.2000) -- (221.1000,248.2000) -- (221.1000,248.2000) -- (221.2000,248.2000) -- (221.2000,248.2000) -- (221.3000,248.2000) -- (221.4000,248.2000) -- (221.4000,248.2000) -- (221.5000,248.2000) -- (221.5000,248.2000) -- (221.6000,248.2000) -- (221.7000,248.2000) -- (221.7000,248.2000) -- (221.8000,248.2000) -- (221.9000,248.2000) -- (221.9000,248.2000) -- (222.0000,248.2000) -- (222.0000,248.3000) -- (222.1000,248.5000) -- (222.2000,241.5000) -- (222.2000,238.4000) -- (222.3000,238.7000) -- (222.3000,239.0000) -- (222.4000,236.7000) -- (222.5000,229.2000) -- (222.5000,229.2000) -- (222.6000,229.2000) -- (222.7000,229.2000) -- (222.7000,229.2000) -- (222.8000,229.2000) -- (222.8000,229.2000) -- (222.9000,229.2000) -- (223.0000,229.2000) -- (223.0000,229.2000) -- (223.1000,229.2000) -- (223.1000,229.2000) -- (223.2000,229.2000) -- (223.3000,229.2000) -- (223.3000,229.2000) -- (223.4000,229.2000) -- (223.5000,229.2000) -- (223.5000,229.2000) -- (223.6000,229.2000) -- (223.6000,229.2000) -- (223.7000,229.2000) -- (223.8000,229.2000) -- (223.8000,229.2000) -- (223.9000,229.2000) -- (223.9000,229.2000) -- (224.0000,229.2000) -- (224.1000,229.2000) -- (224.1000,229.2000) -- (224.2000,229.2000) -- (224.3000,229.2000) -- (224.3000,229.2000) -- (224.4000,229.2000) -- (224.4000,229.8000) -- (224.5000,237.7000) -- (224.6000,238.8000) -- (224.6000,239.2000) -- (224.7000,247.1000) -- (224.7000,248.4000) -- (224.8000,248.2000) -- (224.9000,247.9000) -- (224.9000,252.1000) -- (225.0000,258.1000) -- (225.1000,257.8000) -- (225.1000,257.7000) -- (225.2000,257.7000) -- (225.2000,257.7000) -- (225.3000,257.7000) -- (225.4000,257.7000) -- (225.4000,257.7000) -- (225.5000,257.7000) -- (225.5000,257.7000) -- (225.6000,257.7000) -- (225.7000,257.7000) -- (225.7000,257.7000) -- (225.8000,257.7000) -- (225.9000,257.7000) -- (225.9000,257.7000) -- (226.0000,257.7000) -- (226.0000,257.7000) -- (226.1000,257.7000) -- (226.2000,257.7000) -- (226.2000,257.7000) -- (226.3000,257.7000) -- (226.3000,257.7000) -- (226.4000,257.7000) -- (226.5000,257.7000) -- (226.5000,258.0000) -- (226.6000,255.6000) -- (226.6000,248.2000) -- (226.7000,248.2000) -- (226.8000,248.3000) -- (226.8000,248.5000) -- (226.9000,241.8000) -- (227.0000,238.4000) -- (227.0000,239.1000) -- (227.1000,232.5000) -- (227.1000,228.9000) -- (227.2000,229.2000) -- (227.3000,229.2000) -- (227.3000,229.2000) -- (227.4000,229.2000) -- (227.4000,229.2000) -- (227.5000,229.2000) -- (227.6000,229.2000) -- (227.6000,229.2000) -- (227.7000,229.2000) -- (227.8000,229.2000) -- (227.8000,229.2000) -- (227.9000,229.2000) -- (227.9000,229.2000) -- (228.0000,229.2000) -- (228.1000,229.2000) -- (228.1000,229.2000) -- (228.2000,229.2000) -- (228.2000,229.2000) -- (228.3000,229.2000) -- (228.4000,229.2000) -- (228.4000,229.2000) -- (228.5000,229.2000) -- (228.6000,229.2000) -- (228.6000,229.2000) -- (228.7000,229.2000) -- (228.7000,229.2000) -- (228.8000,229.8000) -- (228.9000,237.3000) -- (228.9000,243.1000) -- (229.0000,248.6000) -- (229.0000,247.9000) -- (229.1000,252.3000) -- (229.2000,258.1000) -- (229.2000,257.8000) -- (229.3000,257.7000) -- (229.4000,258.0000) -- (229.4000,265.8000) -- (229.5000,267.4000) -- (229.5000,267.2000) -- (229.6000,267.2000) -- (229.7000,267.2000) -- (229.7000,267.2000) -- (229.8000,267.2000) -- (229.8000,267.2000) -- (229.9000,267.2000) -- (230.0000,267.2000) -- (230.0000,267.2000) -- (230.1000,267.2000) -- (230.2000,267.2000) -- (230.2000,267.2000) -- (230.3000,267.2000) -- (230.3000,267.2000) -- (230.4000,267.2000) -- (230.5000,267.2000) -- (230.5000,267.2000) -- (230.6000,267.2000) -- (230.6000,267.2000) -- (230.7000,267.2000) -- (230.8000,267.2000) -- (230.8000,267.2000) -- (230.9000,267.2000) -- (231.0000,267.3000) -- (231.0000,267.5000) -- (231.1000,260.7000) -- (231.1000,257.4000) -- (231.2000,257.7000) -- (231.3000,258.0000) -- (231.3000,255.9000) -- (231.4000,248.3000) -- (231.4000,248.4000) -- (231.5000,246.6000) -- (231.6000,238.9000) -- (231.6000,238.9000) -- (231.7000,237.2000) -- (231.8000,229.4000) -- (231.8000,229.2000) -- (231.9000,229.2000) -- (231.9000,229.2000) -- (232.0000,229.2000) -- (232.1000,229.2000) -- (232.1000,229.2000) -- (232.2000,229.2000) -- (232.2000,229.2000) -- (232.3000,229.2000) -- (232.4000,229.2000) -- (232.4000,229.2000) -- (232.5000,229.2000) -- (232.6000,229.2000) -- (232.6000,229.2000) -- (232.7000,229.2000) -- (232.7000,229.2000) -- (232.8000,229.2000) -- (232.9000,229.2000) -- (232.9000,229.2000) -- (233.0000,229.2000) -- (233.0000,229.2000) -- (233.1000,229.2000) -- (233.2000,229.8000) -- (233.2000,236.6000) -- (233.3000,251.6000) -- (233.4000,268.4000) -- (233.4000,267.3000) -- (233.5000,267.2000) -- (233.5000,267.2000) -- (233.6000,267.2000) -- (233.7000,267.2000) -- (233.7000,267.2000) -- (233.8000,267.2000) -- (233.8000,267.2000) -- (233.9000,267.2000) -- (234.0000,267.2000) -- (234.0000,267.2000) -- (234.1000,267.2000) -- (234.2000,267.2000) -- (234.2000,267.2000) -- (234.3000,267.2000) -- (234.3000,267.2000) -- (234.4000,267.2000) -- (234.5000,267.2000) -- (234.5000,267.2000) -- (234.6000,267.2000) -- (234.6000,267.2000) -- (234.7000,267.2000) -- (234.8000,267.2000) -- (234.8000,267.2000) -- (234.9000,267.2000) -- (235.0000,267.2000) -- (235.0000,267.2000) -- (235.1000,267.2000) -- (235.1000,267.2000) -- (235.2000,267.2000) -- (235.3000,267.2000) -- (235.3000,267.2000) -- (235.4000,267.2000) -- (235.4000,267.2000) -- (235.5000,267.2000) -- (235.6000,267.2000) -- (235.6000,267.2000) -- (235.7000,267.3000) -- (235.8000,267.6000) -- (235.8000,261.0000) -- (235.9000,257.4000) -- (235.9000,258.1000) -- (236.0000,251.7000) -- (236.1000,247.9000) -- (236.1000,248.6000) -- (236.2000,242.4000) -- (236.2000,238.4000) -- (236.3000,239.1000) -- (236.4000,233.1000) -- (236.4000,228.9000) -- (236.5000,229.2000) -- (236.6000,229.2000) -- (236.6000,229.2000) -- (236.7000,229.2000) -- (236.7000,229.2000) -- (236.8000,229.2000) -- (236.9000,229.2000) -- (236.9000,229.2000) -- (237.0000,229.2000) -- (237.0000,229.2000) -- (237.1000,229.1000) -- (237.2000,230.0000) -- (237.2000,238.0000) -- (237.3000,238.9000) -- (237.4000,238.7000) -- (237.4000,238.7000) -- (237.5000,238.7000) -- (237.5000,238.7000) -- (237.6000,237.6000) -- (237.7000,251.4000) -- (237.7000,268.4000) -- (237.8000,267.3000) -- (237.8000,267.2000) -- (237.9000,267.2000) -- (238.0000,267.2000) -- (238.0000,267.2000) -- (238.1000,267.2000) -- (238.1000,267.2000) -- (238.2000,267.2000) -- (238.3000,267.2000) -- (238.3000,267.2000) -- (238.4000,267.2000) -- (238.5000,267.2000) -- (238.5000,267.2000) -- (238.6000,267.2000) -- (238.6000,267.2000) -- (238.7000,267.2000) -- (238.8000,267.2000) -- (238.8000,267.2000) -- (238.9000,267.2000) -- (238.9000,267.2000) -- (239.0000,267.2000) -- (239.1000,267.2000) -- (239.1000,267.2000) -- (239.2000,267.2000) -- (239.3000,267.2000) -- (239.3000,267.2000) -- (239.4000,267.2000) -- (239.4000,267.2000) -- (239.5000,267.2000) -- (239.6000,267.2000) -- (239.6000,267.2000) -- (239.7000,267.2000) -- (239.7000,267.2000) -- (239.8000,267.2000) -- (239.9000,267.2000) -- (239.9000,267.2000) -- (240.0000,267.2000) -- (240.1000,267.2000) -- (240.1000,267.2000) -- (240.2000,267.2000) -- (240.2000,267.2000) -- (240.3000,267.2000) -- (240.4000,267.5000) -- (240.4000,265.7000) -- (240.5000,258.0000) -- (240.5000,257.9000) -- (240.6000,256.4000) -- (240.7000,248.5000) -- (240.7000,248.4000) -- (240.8000,247.0000) -- (240.9000,239.1000) -- (240.9000,238.7000) -- (241.0000,238.7000) -- (241.0000,238.7000) -- (241.1000,238.7000) -- (241.2000,238.7000) -- (241.2000,238.7000) -- (241.3000,238.7000) -- (241.3000,238.7000) -- (241.4000,238.7000) -- (241.5000,238.6000) -- (241.5000,239.5000) -- (241.6000,247.5000) -- (241.7000,248.4000) -- (241.7000,248.2000) -- (241.8000,248.2000) -- (241.8000,248.1000) -- (241.9000,249.3000) -- (242.0000,265.2000) -- (242.0000,267.6000) -- (242.1000,267.2000) -- (242.1000,267.2000) -- (242.2000,267.2000) -- (242.3000,267.2000) -- (242.3000,267.2000) -- (242.4000,267.2000) -- (242.5000,267.2000) -- (242.5000,267.2000) -- (242.6000,267.2000) -- (242.6000,267.2000) -- (242.7000,267.2000) -- (242.8000,267.2000) -- (242.8000,267.2000) -- (242.9000,267.2000) -- (242.9000,267.2000) -- (243.0000,267.2000) -- (243.1000,267.2000) -- (243.1000,267.2000) -- (243.2000,267.2000) -- (243.3000,267.2000) -- (243.3000,267.2000) -- (243.4000,267.2000) -- (243.4000,267.2000) -- (243.5000,267.2000) -- (243.6000,267.2000) -- (243.6000,267.2000) -- (243.7000,267.2000) -- (243.7000,267.2000) -- (243.8000,267.2000) -- (243.9000,267.2000) -- (243.9000,267.2000) -- (244.0000,267.2000) -- (244.1000,267.2000) -- (244.1000,267.2000) -- (244.2000,267.2000) -- (244.2000,267.2000) -- (244.3000,267.2000) -- (244.4000,267.2000) -- (244.4000,267.2000) -- (244.5000,267.2000) -- (244.5000,267.2000) -- (244.6000,267.2000) -- (244.7000,267.2000) -- (244.7000,267.2000) -- (244.8000,267.2000) -- (244.9000,267.2000) -- (244.9000,267.2000) -- (245.0000,267.3000) -- (245.0000,267.6000) -- (245.1000,261.6000) -- (245.2000,257.4000) -- (245.2000,258.1000) -- (245.3000,252.3000) -- (245.3000,247.9000) -- (245.4000,248.2000) -- (245.5000,248.2000) -- (245.5000,248.2000) -- (245.6000,248.2000) -- (245.7000,248.2000) -- (245.7000,248.2000) -- (245.8000,248.2000) -- (245.8000,248.2000) -- (245.9000,248.2000) -- (246.0000,247.9000) -- (246.0000,252.9000) -- (246.1000,258.1000) -- (246.1000,257.4000) -- (246.2000,262.2000) -- (246.3000,267.6000);



  \end{scope}
  \begin{scope}[scale=1.006,draw=black,line join=bevel,line cap=rect,line width=0.800pt]
  \end{scope}
  \begin{scope}[scale=1.006,draw=black,line join=bevel,line cap=rect,line width=0.800pt]
  \end{scope}
  \begin{scope}[scale=1.006,draw=black,line join=round,line cap=round,line width=0.480pt]
    \path[draw] (184.5000,224.5000) -- (184.5000,272.5000) -- (246.5000,272.5000) -- (246.5000,224.5000) -- (184.5000,224.5000);



  \end{scope}
  \begin{scope}[scale=1.006,draw=ca0a0a4,dash pattern=on 0.40pt off 0.80pt,line join=round,line cap=round,line width=0.400pt]
    \path[draw] (184.5000,315.5000) -- (246.5000,315.5000);



  \end{scope}
  \begin{scope}[scale=1.006,draw=black,line join=round,line cap=round,line width=0.480pt]
    \path[draw] (184.5000,315.5000) -- (186.5000,315.5000);



    \path[draw] (246.5000,315.5000) -- (245.5000,315.5000);



  \end{scope}
  \begin{scope}[scale=1.006,draw=black,line join=bevel,line cap=rect,line width=0.800pt]
  \end{scope}
  \begin{scope}[cm={{1.00588,0.0,0.0,1.00588,(181.059,317.859)}},draw=black,line join=bevel,line cap=rect,line width=0.800pt]
  \end{scope}
  \begin{scope}[cm={{1.00588,0.0,0.0,1.00588,(181.059,317.859)}},draw=black,line join=bevel,line cap=rect,line width=0.800pt]
  \end{scope}
  \begin{scope}[cm={{1.00588,0.0,0.0,1.00588,(181.059,317.859)}},draw=black,line join=bevel,line cap=rect,line width=0.800pt]
  \end{scope}
  \begin{scope}[cm={{1.00588,0.0,0.0,1.00588,(181.059,317.859)}},draw=black,line join=bevel,line cap=rect,line width=0.800pt]
  \end{scope}
  \begin{scope}[cm={{1.00588,0.0,0.0,1.00588,(181.059,317.859)}},draw=black,line join=bevel,line cap=rect,line width=0.800pt]
  \end{scope}
  \begin{scope}[cm={{1.00588,0.0,0.0,1.00588,(181.059,317.859)}},draw=black,line join=bevel,line cap=rect,line width=0.800pt]
  \end{scope}
  \begin{scope}[scale=1.006,draw=black,line join=bevel,line cap=rect,line width=0.800pt]
  \end{scope}
  \begin{scope}[scale=1.006,draw=ca0a0a4,dash pattern=on 0.40pt off 0.80pt,line join=round,line cap=round,line width=0.400pt]
    \path[draw] (184.5000,295.5000) -- (246.5000,295.5000);



  \end{scope}
  \begin{scope}[scale=1.006,draw=black,line join=round,line cap=round,line width=0.480pt]
    \path[draw] (184.5000,295.5000) -- (186.5000,295.5000);



    \path[draw] (246.5000,295.5000) -- (245.5000,295.5000);



  \end{scope}
  \begin{scope}[scale=1.006,draw=black,line join=bevel,line cap=rect,line width=0.800pt]
  \end{scope}
  \begin{scope}[cm={{1.00588,0.0,0.0,1.00588,(181.059,297.741)}},draw=black,line join=bevel,line cap=rect,line width=0.800pt]
  \end{scope}
  \begin{scope}[cm={{1.00588,0.0,0.0,1.00588,(181.059,297.741)}},draw=black,line join=bevel,line cap=rect,line width=0.800pt]
  \end{scope}
  \begin{scope}[cm={{1.00588,0.0,0.0,1.00588,(181.059,297.741)}},draw=black,line join=bevel,line cap=rect,line width=0.800pt]
  \end{scope}
  \begin{scope}[cm={{1.00588,0.0,0.0,1.00588,(181.059,297.741)}},draw=black,line join=bevel,line cap=rect,line width=0.800pt]
  \end{scope}
  \begin{scope}[cm={{1.00588,0.0,0.0,1.00588,(181.059,297.741)}},draw=black,line join=bevel,line cap=rect,line width=0.800pt]
  \end{scope}
  \begin{scope}[cm={{1.00588,0.0,0.0,1.00588,(181.059,297.741)}},draw=black,line join=bevel,line cap=rect,line width=0.800pt]
  \end{scope}
  \begin{scope}[scale=1.006,draw=black,line join=bevel,line cap=rect,line width=0.800pt]
  \end{scope}
  \begin{scope}[scale=1.006,draw=ca0a0a4,dash pattern=on 0.40pt off 0.80pt,line join=round,line cap=round,line width=0.400pt]
    \path[draw] (184.5000,276.5000) -- (246.5000,276.5000);



  \end{scope}
  \begin{scope}[scale=1.006,draw=black,line join=round,line cap=round,line width=0.480pt]
    \path[draw] (184.5000,276.5000) -- (186.5000,276.5000);



    \path[draw] (246.5000,276.5000) -- (245.5000,276.5000);



  \end{scope}
  \begin{scope}[scale=1.006,draw=black,line join=bevel,line cap=rect,line width=0.800pt]
  \end{scope}
  \begin{scope}[cm={{1.00588,0.0,0.0,1.00588,(181.059,277.624)}},draw=black,line join=bevel,line cap=rect,line width=0.800pt]
  \end{scope}
  \begin{scope}[cm={{1.00588,0.0,0.0,1.00588,(181.059,277.624)}},draw=black,line join=bevel,line cap=rect,line width=0.800pt]
  \end{scope}
  \begin{scope}[cm={{1.00588,0.0,0.0,1.00588,(181.059,277.624)}},draw=black,line join=bevel,line cap=rect,line width=0.800pt]
  \end{scope}
  \begin{scope}[cm={{1.00588,0.0,0.0,1.00588,(181.059,277.624)}},draw=black,line join=bevel,line cap=rect,line width=0.800pt]
  \end{scope}
  \begin{scope}[cm={{1.00588,0.0,0.0,1.00588,(181.059,277.624)}},draw=black,line join=bevel,line cap=rect,line width=0.800pt]
  \end{scope}
  \begin{scope}[cm={{1.00588,0.0,0.0,1.00588,(181.059,277.624)}},draw=black,line join=bevel,line cap=rect,line width=0.800pt]
  \end{scope}
  \begin{scope}[scale=1.006,draw=black,line join=bevel,line cap=rect,line width=0.800pt]
  \end{scope}
  \begin{scope}[scale=1.006,draw=ca0a0a4,dash pattern=on 0.40pt off 0.80pt,line join=round,line cap=round,line width=0.400pt]
    \path[draw] (184.5000,319.5000) -- (184.5000,272.5000);



  \end{scope}
  \begin{scope}[scale=1.006,draw=black,line join=round,line cap=round,line width=0.480pt]
    \path[draw] (184.5000,319.5000) -- (184.5000,318.5000);



    \path[draw] (184.5000,272.5000) -- (184.5000,273.5000);



  \end{scope}
  \begin{scope}[scale=1.006,draw=black,line join=bevel,line cap=rect,line width=0.800pt]
  \end{scope}
  \begin{scope}[cm={{1.00588,0.0,0.0,1.00588,(183.071,336.971)}},draw=black,line join=bevel,line cap=rect,line width=0.800pt]
  \end{scope}
  \begin{scope}[cm={{1.00588,0.0,0.0,1.00588,(183.071,336.971)}},draw=black,line join=bevel,line cap=rect,line width=0.800pt]
  \end{scope}
  \begin{scope}[cm={{1.00588,0.0,0.0,1.00588,(183.071,336.971)}},draw=black,line join=bevel,line cap=rect,line width=0.800pt]
  \end{scope}
  \begin{scope}[cm={{1.00588,0.0,0.0,1.00588,(183.071,336.971)}},draw=black,line join=bevel,line cap=rect,line width=0.800pt]
  \end{scope}
  \begin{scope}[cm={{1.00588,0.0,0.0,1.00588,(183.071,336.971)}},draw=black,line join=bevel,line cap=rect,line width=0.800pt]
  \end{scope}
  \begin{scope}[cm={{1.00588,0.0,0.0,1.00588,(183.071,336.971)}},draw=black,line join=bevel,line cap=rect,line width=0.800pt]
    \path[fill=black] (0.0000,0.0000) node[above right] (text2316) {\scriptsize 0};



  \end{scope}
  \begin{scope}[cm={{1.00588,0.0,0.0,1.00588,(183.071,336.971)}},draw=black,line join=bevel,line cap=rect,line width=0.800pt]
  \end{scope}
  \begin{scope}[scale=1.006,draw=black,line join=bevel,line cap=rect,line width=0.800pt]
  \end{scope}
  \begin{scope}[scale=1.006,draw=ca0a0a4,dash pattern=on 0.40pt off 0.80pt,line join=round,line cap=round,line width=0.400pt]
    \path[draw] (201.5000,319.5000) -- (201.5000,272.5000);



  \end{scope}
  \begin{scope}[scale=1.006,draw=black,line join=round,line cap=round,line width=0.480pt]
    \path[draw] (201.5000,319.5000) -- (201.5000,318.5000);



    \path[draw] (201.5000,272.5000) -- (201.5000,273.5000);



  \end{scope}
  \begin{scope}[scale=1.006,draw=black,line join=bevel,line cap=rect,line width=0.800pt]
  \end{scope}
  \begin{scope}[cm={{1.00588,0.0,0.0,1.00588,(200.674,336.971)}},draw=black,line join=bevel,line cap=rect,line width=0.800pt]
  \end{scope}
  \begin{scope}[cm={{1.00588,0.0,0.0,1.00588,(200.674,336.971)}},draw=black,line join=bevel,line cap=rect,line width=0.800pt]
  \end{scope}
  \begin{scope}[cm={{1.00588,0.0,0.0,1.00588,(200.674,336.971)}},draw=black,line join=bevel,line cap=rect,line width=0.800pt]
  \end{scope}
  \begin{scope}[cm={{1.00588,0.0,0.0,1.00588,(200.674,336.971)}},draw=black,line join=bevel,line cap=rect,line width=0.800pt]
  \end{scope}
  \begin{scope}[cm={{1.00588,0.0,0.0,1.00588,(200.674,336.971)}},draw=black,line join=bevel,line cap=rect,line width=0.800pt]
  \end{scope}
  \begin{scope}[cm={{1.00588,0.0,0.0,1.00588,(200.674,336.971)}},draw=black,line join=bevel,line cap=rect,line width=0.800pt]
    \path[fill=black] (0.0000,0.0000) node[above right] (text2346) {\scriptsize 3};



  \end{scope}
  \begin{scope}[cm={{1.00588,0.0,0.0,1.00588,(200.674,336.971)}},draw=black,line join=bevel,line cap=rect,line width=0.800pt]
  \end{scope}
  \begin{scope}[scale=1.006,draw=black,line join=bevel,line cap=rect,line width=0.800pt]
  \end{scope}
  \begin{scope}[scale=1.006,draw=ca0a0a4,dash pattern=on 0.40pt off 0.80pt,line join=round,line cap=round,line width=0.400pt]
    \path[draw] (218.5000,319.5000) -- (218.5000,272.5000);



  \end{scope}
  \begin{scope}[scale=1.006,draw=black,line join=round,line cap=round,line width=0.480pt]
    \path[draw] (218.5000,319.5000) -- (218.5000,318.5000);



    \path[draw] (218.5000,272.5000) -- (218.5000,273.5000);



  \end{scope}
  \begin{scope}[scale=1.006,draw=black,line join=bevel,line cap=rect,line width=0.800pt]
  \end{scope}
  \begin{scope}[cm={{1.00588,0.0,0.0,1.00588,(216.265,336.971)}},draw=black,line join=bevel,line cap=rect,line width=0.800pt]
  \end{scope}
  \begin{scope}[cm={{1.00588,0.0,0.0,1.00588,(216.265,336.971)}},draw=black,line join=bevel,line cap=rect,line width=0.800pt]
  \end{scope}
  \begin{scope}[cm={{1.00588,0.0,0.0,1.00588,(216.265,336.971)}},draw=black,line join=bevel,line cap=rect,line width=0.800pt]
  \end{scope}
  \begin{scope}[cm={{1.00588,0.0,0.0,1.00588,(216.265,336.971)}},draw=black,line join=bevel,line cap=rect,line width=0.800pt]
  \end{scope}
  \begin{scope}[cm={{1.00588,0.0,0.0,1.00588,(216.265,336.971)}},draw=black,line join=bevel,line cap=rect,line width=0.800pt]
  \end{scope}
  \begin{scope}[cm={{1.00588,0.0,0.0,1.00588,(216.265,336.971)}},draw=black,line join=bevel,line cap=rect,line width=0.800pt]
    \path[fill=black] (0.0000,0.0000) node[above right] (text2376) {\scriptsize 6};



  \end{scope}
  \begin{scope}[cm={{1.00588,0.0,0.0,1.00588,(216.265,336.971)}},draw=black,line join=bevel,line cap=rect,line width=0.800pt]
  \end{scope}
  \begin{scope}[scale=1.006,draw=black,line join=bevel,line cap=rect,line width=0.800pt]
  \end{scope}
  \begin{scope}[scale=1.006,draw=ca0a0a4,dash pattern=on 0.40pt off 0.80pt,line join=round,line cap=round,line width=0.400pt]
    \path[draw] (235.5000,319.5000) -- (235.5000,272.5000);



  \end{scope}
  \begin{scope}[scale=1.006,draw=black,line join=round,line cap=round,line width=0.480pt]
    \path[draw] (235.5000,319.5000) -- (235.5000,318.5000);



    \path[draw] (235.5000,272.5000) -- (235.5000,273.5000);



  \end{scope}
  \begin{scope}[scale=1.006,draw=black,line join=bevel,line cap=rect,line width=0.800pt]
  \end{scope}
  \begin{scope}[cm={{1.00588,0.0,0.0,1.00588,(233.365,336.971)}},draw=black,line join=bevel,line cap=rect,line width=0.800pt]
  \end{scope}
  \begin{scope}[cm={{1.00588,0.0,0.0,1.00588,(233.365,336.971)}},draw=black,line join=bevel,line cap=rect,line width=0.800pt]
  \end{scope}
  \begin{scope}[cm={{1.00588,0.0,0.0,1.00588,(233.365,336.971)}},draw=black,line join=bevel,line cap=rect,line width=0.800pt]
  \end{scope}
  \begin{scope}[cm={{1.00588,0.0,0.0,1.00588,(233.365,336.971)}},draw=black,line join=bevel,line cap=rect,line width=0.800pt]
  \end{scope}
  \begin{scope}[cm={{1.00588,0.0,0.0,1.00588,(233.365,336.971)}},draw=black,line join=bevel,line cap=rect,line width=0.800pt]
  \end{scope}
  \begin{scope}[cm={{1.00588,0.0,0.0,1.00588,(233.365,336.971)}},draw=black,line join=bevel,line cap=rect,line width=0.800pt]
    \path[fill=black] (0.0000,0.0000) node[above right] (text2406) {\scriptsize 9};



  \end{scope}
  \begin{scope}[cm={{1.00588,0.0,0.0,1.00588,(233.365,336.971)}},draw=black,line join=bevel,line cap=rect,line width=0.800pt]
  \end{scope}
  \begin{scope}[scale=1.006,draw=black,line join=bevel,line cap=rect,line width=0.800pt]
  \end{scope}
  \begin{scope}[scale=1.006,draw=black,line join=round,line cap=round,line width=0.480pt]
    \path[draw] (246.5000,315.5000) -- (245.5000,315.5000);



  \end{scope}
  \begin{scope}[scale=1.006,draw=black,line join=bevel,line cap=rect,line width=0.800pt]
  \end{scope}
  \begin{scope}[cm={{1.00588,0.0,0.0,1.00588,(253.482,321.882)}},draw=black,line join=bevel,line cap=rect,line width=0.800pt]
  \end{scope}
  \begin{scope}[cm={{1.00588,0.0,0.0,1.00588,(253.482,321.882)}},draw=black,line join=bevel,line cap=rect,line width=0.800pt]
  \end{scope}
  \begin{scope}[cm={{1.00588,0.0,0.0,1.00588,(253.482,321.882)}},draw=black,line join=bevel,line cap=rect,line width=0.800pt]
  \end{scope}
  \begin{scope}[cm={{1.00588,0.0,0.0,1.00588,(253.482,321.882)}},draw=black,line join=bevel,line cap=rect,line width=0.800pt]
  \end{scope}
  \begin{scope}[cm={{1.00588,0.0,0.0,1.00588,(253.482,321.882)}},draw=black,line join=bevel,line cap=rect,line width=0.800pt]
  \end{scope}
  \begin{scope}[cm={{1.00588,0.0,0.0,1.00588,(250.982,321.882)}},draw=black,line join=bevel,line cap=rect,line width=0.800pt]
    \path[fill=black] (0.0000,0.0000) node[above right] (text2430) {\scriptsize -1};



  \end{scope}
  \begin{scope}[cm={{1.00588,0.0,0.0,1.00588,(253.482,321.882)}},draw=black,line join=bevel,line cap=rect,line width=0.800pt]
  \end{scope}
  \begin{scope}[scale=1.006,draw=black,line join=bevel,line cap=rect,line width=0.800pt]
  \end{scope}
  \begin{scope}[scale=1.006,draw=black,line join=round,line cap=round,line width=0.480pt]
    \path[draw] (246.5000,295.5000) -- (245.5000,295.5000);



  \end{scope}
  \begin{scope}[scale=1.006,draw=black,line join=bevel,line cap=rect,line width=0.800pt]
  \end{scope}
  \begin{scope}[cm={{1.00588,0.0,0.0,1.00588,(253.482,301.765)}},draw=black,line join=bevel,line cap=rect,line width=0.800pt]
  \end{scope}
  \begin{scope}[cm={{1.00588,0.0,0.0,1.00588,(253.482,301.765)}},draw=black,line join=bevel,line cap=rect,line width=0.800pt]
  \end{scope}
  \begin{scope}[cm={{1.00588,0.0,0.0,1.00588,(253.482,301.765)}},draw=black,line join=bevel,line cap=rect,line width=0.800pt]
  \end{scope}
  \begin{scope}[cm={{1.00588,0.0,0.0,1.00588,(253.482,301.765)}},draw=black,line join=bevel,line cap=rect,line width=0.800pt]
  \end{scope}
  \begin{scope}[cm={{1.00588,0.0,0.0,1.00588,(253.482,301.765)}},draw=black,line join=bevel,line cap=rect,line width=0.800pt]
  \end{scope}
  \begin{scope}[cm={{1.00588,0.0,0.0,1.00588,(250.982,301.765)}},draw=black,line join=bevel,line cap=rect,line width=0.800pt]
    \path[fill=black] (0.0000,0.0000) node[above right] (text2454) {\scriptsize -.5};



  \end{scope}
  \begin{scope}[cm={{1.00588,0.0,0.0,1.00588,(253.482,301.765)}},draw=black,line join=bevel,line cap=rect,line width=0.800pt]
  \end{scope}
  \begin{scope}[scale=1.006,draw=black,line join=bevel,line cap=rect,line width=0.800pt]
  \end{scope}
  \begin{scope}[scale=1.006,draw=black,line join=round,line cap=round,line width=0.480pt]
    \path[draw] (246.5000,276.5000) -- (245.5000,276.5000);



  \end{scope}
  \begin{scope}[scale=1.006,draw=black,line join=bevel,line cap=rect,line width=0.800pt]
  \end{scope}
  \begin{scope}[cm={{1.00588,0.0,0.0,1.00588,(253.482,281.647)}},draw=black,line join=bevel,line cap=rect,line width=0.800pt]
  \end{scope}
  \begin{scope}[cm={{1.00588,0.0,0.0,1.00588,(253.482,281.647)}},draw=black,line join=bevel,line cap=rect,line width=0.800pt]
  \end{scope}
  \begin{scope}[cm={{1.00588,0.0,0.0,1.00588,(253.482,281.647)}},draw=black,line join=bevel,line cap=rect,line width=0.800pt]
  \end{scope}
  \begin{scope}[cm={{1.00588,0.0,0.0,1.00588,(253.482,281.647)}},draw=black,line join=bevel,line cap=rect,line width=0.800pt]
  \end{scope}
  \begin{scope}[cm={{1.00588,0.0,0.0,1.00588,(253.482,281.647)}},draw=black,line join=bevel,line cap=rect,line width=0.800pt]
  \end{scope}
  \begin{scope}[cm={{1.00588,0.0,0.0,1.00588,(250.982,281.647)}},draw=black,line join=bevel,line cap=rect,line width=0.800pt]
    \path[fill=black] (0.0000,0.0000) node[above right] (text2478) {\scriptsize 0};



  \end{scope}
  \begin{scope}[cm={{1.00588,0.0,0.0,1.00588,(253.482,281.647)}},draw=black,line join=bevel,line cap=rect,line width=0.800pt]
  \end{scope}
  \begin{scope}[scale=1.006,draw=black,line join=bevel,line cap=rect,line width=0.800pt]
  \end{scope}
  \begin{scope}[scale=1.006,draw=black,line join=round,line cap=round,line width=0.480pt]
    \path[draw] (184.5000,272.5000) -- (184.5000,319.5000) -- (246.5000,319.5000) -- (246.5000,272.5000) -- (184.5000,272.5000);



  \end{scope}
  \begin{scope}[scale=1.006,draw=black,line join=bevel,line cap=rect,line width=0.800pt]
  \end{scope}
  \begin{scope}[cm={{0.0,-1.00588,1.00588,0.0,(276.618,303.274)}},draw=black,line join=bevel,line cap=rect,line width=0.800pt]
  \end{scope}
  \begin{scope}[cm={{0.0,-1.00588,1.00588,0.0,(276.618,303.274)}},draw=black,line join=bevel,line cap=rect,line width=0.800pt]
  \end{scope}
  \begin{scope}[cm={{0.0,-1.00588,1.00588,0.0,(276.618,303.274)}},draw=black,line join=bevel,line cap=rect,line width=0.800pt]
  \end{scope}
  \begin{scope}[cm={{0.0,-1.00588,1.00588,0.0,(276.618,303.274)}},draw=black,line join=bevel,line cap=rect,line width=0.800pt]
  \end{scope}
  \begin{scope}[cm={{0.0,-1.00588,1.00588,0.0,(276.618,303.274)}},draw=black,line join=bevel,line cap=rect,line width=0.800pt]
  \end{scope}
  \begin{scope}[cm={{0.0,-1.00588,1.00588,0.0,(263.618,303.274)}},draw=black,line join=bevel,line cap=rect,line width=0.800pt]
    \path[fill=black] (0.0000,0.0000) node[above right] (text2502) {\rotatebox{90}{\scriptsize k$c_{1,1}$}};



  \end{scope}
  \begin{scope}[cm={{0.0,-1.00588,1.00588,0.0,(276.618,303.274)}},draw=black,line join=bevel,line cap=rect,line width=0.800pt]
  \end{scope}
  \begin{scope}[scale=1.006,draw=black,line join=bevel,line cap=rect,line width=0.800pt]
  \end{scope}
  \begin{scope}[scale=1.006,draw=black,line join=bevel,line cap=rect,line width=0.800pt]
  \end{scope}
  \begin{scope}[scale=1.006,draw=black,line join=bevel,line cap=rect,line width=0.800pt]
  \end{scope}
  \begin{scope}[scale=1.006,draw=black,line join=round,line cap=round,line width=0.480pt]
    \path[draw] (184.8000,315.6000) -- (184.8000,315.6000) -- (184.9000,315.6000) -- (185.0000,315.6000) -- (185.0000,315.6000) -- (185.1000,315.6000) -- (185.1000,315.6000) -- (185.2000,315.6000) -- (185.3000,315.6000) -- (185.3000,315.6000) -- (185.4000,315.6000) -- (185.4000,315.6000) -- (185.5000,315.6000) -- (185.6000,315.6000) -- (185.6000,315.6000) -- (185.7000,315.6000) -- (185.8000,315.6000) -- (185.8000,315.6000) -- (185.9000,315.6000) -- (185.9000,315.6000) -- (186.0000,315.6000) -- (186.1000,315.6000) -- (186.1000,315.6000) -- (186.2000,315.6000) -- (186.2000,315.6000) -- (186.3000,315.6000) -- (186.4000,315.6000) -- (186.4000,315.6000) -- (186.5000,315.6000) -- (186.6000,315.6000) -- (186.6000,315.6000) -- (186.7000,315.6000) -- (186.7000,315.6000) -- (186.8000,315.6000) -- (186.9000,315.6000) -- (186.9000,315.6000) -- (187.0000,315.6000) -- (187.0000,315.6000) -- (187.1000,315.6000) -- (187.2000,315.6000) -- (187.2000,315.6000) -- (187.3000,315.6000) -- (187.4000,315.6000) -- (187.4000,315.6000) -- (187.5000,315.6000) -- (187.5000,315.6000) -- (187.6000,315.6000) -- (187.7000,315.6000) -- (187.7000,315.6000) -- (187.8000,315.6000) -- (187.8000,315.6000) -- (187.9000,315.6000) -- (188.0000,315.6000) -- (188.0000,315.6000) -- (188.1000,315.6000) -- (188.2000,315.6000) -- (188.2000,315.6000) -- (188.3000,315.6000) -- (188.3000,315.6000) -- (188.4000,315.6000) -- (188.5000,315.6000) -- (188.5000,315.6000) -- (188.6000,315.6000) -- (188.6000,315.6000) -- (188.7000,315.6000) -- (188.8000,315.6000) -- (188.8000,315.6000) -- (188.9000,315.6000) -- (189.0000,315.6000) -- (189.0000,315.6000) -- (189.1000,315.6000) -- (189.1000,315.6000) -- (189.2000,315.6000) -- (189.3000,315.6000) -- (189.3000,315.6000) -- (189.4000,315.6000) -- (189.4000,315.6000) -- (189.5000,315.6000) -- (189.6000,315.6000) -- (189.6000,315.6000) -- (189.7000,315.6000) -- (189.8000,315.6000) -- (189.8000,315.6000) -- (189.9000,315.6000) -- (189.9000,315.6000) -- (190.0000,315.6000) -- (190.1000,315.6000) -- (190.1000,315.6000) -- (190.2000,315.6000) -- (190.2000,315.6000) -- (190.3000,315.6000) -- (190.4000,315.6000) -- (190.4000,315.6000) -- (190.5000,315.6000) -- (190.5000,315.6000) -- (190.6000,315.6000) -- (190.7000,315.6000) -- (190.7000,315.6000) -- (190.8000,315.6000) -- (190.9000,315.6000) -- (190.9000,315.6000) -- (191.0000,315.6000) -- (191.0000,315.6000) -- (191.1000,315.6000) -- (191.2000,315.6000) -- (191.2000,315.6000) -- (191.3000,315.6000) -- (191.3000,315.6000) -- (191.4000,315.6000) -- (191.5000,315.6000) -- (191.5000,315.6000) -- (191.6000,315.6000) -- (191.7000,315.6000) -- (191.7000,315.6000) -- (191.8000,315.6000) -- (191.8000,315.6000) -- (191.9000,315.6000) -- (192.0000,315.6000) -- (192.0000,315.6000) -- (192.1000,315.6000) -- (192.1000,315.6000) -- (192.2000,315.6000) -- (192.3000,315.6000) -- (192.3000,315.6000) -- (192.4000,315.6000) -- (192.5000,315.6000) -- (192.5000,315.6000) -- (192.6000,315.6000) -- (192.6000,317.2000) -- (192.7000,292.4000) -- (192.8000,274.5000) -- (192.8000,276.1000) -- (192.9000,276.1000) -- (192.9000,276.1000) -- (193.0000,276.1000) -- (193.1000,276.1000) -- (193.1000,276.1000) -- (193.2000,276.1000) -- (193.3000,276.1000) -- (193.3000,276.1000) -- (193.4000,276.1000) -- (193.4000,276.1000) -- (193.5000,276.1000) -- (193.6000,276.1000) -- (193.6000,276.1000) -- (193.7000,276.1000) -- (193.7000,276.1000) -- (193.8000,276.1000) -- (193.9000,276.1000) -- (193.9000,276.1000) -- (194.0000,276.1000) -- (194.1000,276.1000) -- (194.1000,276.1000) -- (194.2000,276.1000) -- (194.2000,276.1000) -- (194.3000,276.1000) -- (194.4000,276.1000) -- (194.4000,276.1000) -- (194.5000,276.1000) -- (194.5000,276.1000) -- (194.6000,276.1000) -- (194.7000,276.1000) -- (194.7000,276.1000) -- (194.8000,276.1000) -- (194.9000,276.1000) -- (194.9000,276.1000) -- (195.0000,276.1000) -- (195.0000,276.1000) -- (195.1000,276.1000) -- (195.2000,276.1000) -- (195.2000,276.1000) -- (195.3000,276.1000) -- (195.3000,276.1000) -- (195.4000,276.1000) -- (195.5000,276.1000) -- (195.5000,276.1000) -- (195.6000,276.1000) -- (195.7000,276.1000) -- (195.7000,276.1000) -- (195.8000,276.1000) -- (195.8000,276.1000) -- (195.9000,276.1000) -- (196.0000,276.1000) -- (196.0000,276.1000) -- (196.1000,276.1000) -- (196.1000,276.1000) -- (196.2000,276.1000) -- (196.3000,276.1000) -- (196.3000,276.1000) -- (196.4000,276.1000) -- (196.5000,276.1000) -- (196.5000,276.1000) -- (196.6000,276.1000) -- (196.6000,276.1000) -- (196.7000,276.1000) -- (196.8000,276.1000) -- (196.8000,276.1000) -- (196.9000,276.1000) -- (196.9000,276.1000) -- (197.0000,276.1000) -- (197.1000,276.1000) -- (197.1000,276.1000) -- (197.2000,276.1000) -- (197.3000,276.1000) -- (197.3000,276.1000) -- (197.4000,276.1000) -- (197.4000,276.1000) -- (197.5000,276.1000) -- (197.6000,276.1000) -- (197.6000,276.1000) -- (197.7000,276.1000) -- (197.7000,276.1000) -- (197.8000,276.1000) -- (197.9000,276.1000) -- (197.9000,276.1000) -- (198.0000,276.1000) -- (198.1000,276.1000) -- (198.1000,276.1000) -- (198.2000,276.1000) -- (198.2000,276.1000) -- (198.3000,276.1000) -- (198.4000,276.1000) -- (198.4000,276.1000) -- (198.5000,276.1000) -- (198.5000,276.1000) -- (198.6000,276.1000) -- (198.7000,276.1000) -- (198.7000,276.1000) -- (198.8000,276.1000) -- (198.9000,276.1000) -- (198.9000,276.1000) -- (199.0000,276.1000) -- (199.0000,276.1000) -- (199.1000,276.1000) -- (199.2000,276.1000) -- (199.2000,276.1000) -- (199.3000,276.1000) -- (199.3000,276.1000) -- (199.4000,276.1000) -- (199.5000,276.1000) -- (199.5000,276.1000) -- (199.6000,276.1000) -- (199.7000,276.1000) -- (199.7000,276.1000) -- (199.8000,276.1000) -- (199.8000,276.1000) -- (199.9000,276.1000) -- (200.0000,276.1000) -- (200.0000,276.1000) -- (200.1000,276.1000) -- (200.1000,276.1000) -- (200.2000,276.1000) -- (200.3000,276.1000) -- (200.3000,276.1000) -- (200.4000,276.1000) -- (200.5000,276.1000) -- (200.5000,276.1000) -- (200.6000,276.1000) -- (200.6000,276.1000) -- (200.7000,276.1000) -- (200.8000,276.1000) -- (200.8000,276.1000) -- (200.9000,276.1000) -- (200.9000,276.1000) -- (201.0000,276.1000) -- (201.1000,276.1000) -- (201.1000,276.1000) -- (201.2000,276.1000) -- (201.3000,276.1000) -- (201.3000,276.1000) -- (201.4000,276.1000) -- (201.4000,276.1000) -- (201.5000,276.1000) -- (201.6000,276.1000) -- (201.6000,276.1000) -- (201.7000,276.1000) -- (201.7000,276.1000) -- (201.8000,276.1000) -- (201.9000,276.1000) -- (201.9000,276.1000) -- (202.0000,276.1000) -- (202.0000,276.1000) -- (202.1000,276.1000) -- (202.2000,276.1000) -- (202.2000,276.1000) -- (202.3000,276.1000) -- (202.4000,276.1000) -- (202.4000,276.1000) -- (202.5000,276.1000) -- (202.5000,276.1000) -- (202.6000,276.1000) -- (202.7000,276.1000) -- (202.7000,276.1000) -- (202.8000,276.1000) -- (202.8000,276.1000) -- (202.9000,276.1000) -- (203.0000,276.1000) -- (203.0000,276.1000) -- (203.1000,276.1000) -- (203.2000,276.1000) -- (203.2000,276.1000) -- (203.3000,276.1000) -- (203.3000,276.1000) -- (203.4000,276.1000) -- (203.5000,276.1000) -- (203.5000,276.1000) -- (203.6000,276.1000) -- (203.6000,276.1000) -- (203.7000,276.1000) -- (203.8000,276.1000) -- (203.8000,276.1000) -- (203.9000,276.1000) -- (204.0000,276.1000) -- (204.0000,276.1000) -- (204.1000,276.1000) -- (204.1000,276.1000) -- (204.2000,276.1000) -- (204.3000,276.1000) -- (204.3000,276.1000) -- (204.4000,276.1000) -- (204.4000,276.1000) -- (204.5000,276.1000) -- (204.6000,276.1000) -- (204.6000,276.1000) -- (204.7000,276.1000) -- (204.8000,276.1000) -- (204.8000,276.1000) -- (204.9000,276.1000) -- (204.9000,276.1000) -- (205.0000,276.1000) -- (205.1000,276.1000) -- (205.1000,276.1000) -- (205.2000,276.1000) -- (205.2000,276.1000) -- (205.3000,276.1000) -- (205.4000,276.1000) -- (205.4000,276.1000) -- (205.5000,276.1000) -- (205.6000,276.1000) -- (205.6000,276.1000) -- (205.7000,276.1000) -- (205.7000,276.1000) -- (205.8000,276.1000) -- (205.9000,276.1000) -- (205.9000,276.1000) -- (206.0000,276.1000) -- (206.0000,276.1000) -- (206.1000,276.1000) -- (206.2000,276.1000) -- (206.2000,276.1000) -- (206.3000,276.1000) -- (206.4000,276.1000) -- (206.4000,276.1000) -- (206.5000,276.1000) -- (206.5000,276.1000) -- (206.6000,276.1000) -- (206.7000,276.1000) -- (206.7000,276.1000) -- (206.8000,276.1000) -- (206.8000,276.1000) -- (206.9000,276.1000) -- (207.0000,276.1000) -- (207.0000,276.1000) -- (207.1000,276.1000) -- (207.2000,276.1000) -- (207.2000,276.1000) -- (207.3000,276.1000) -- (207.3000,276.1000) -- (207.4000,276.1000) -- (207.5000,276.1000) -- (207.5000,276.1000) -- (207.6000,276.1000) -- (207.6000,276.1000) -- (207.7000,276.1000) -- (207.8000,276.1000) -- (207.8000,276.1000) -- (207.9000,276.1000) -- (208.0000,276.1000) -- (208.0000,276.1000) -- (208.1000,276.1000) -- (208.1000,276.1000) -- (208.2000,276.1000) -- (208.3000,276.1000) -- (208.3000,276.1000) -- (208.4000,276.1000) -- (208.4000,276.1000) -- (208.5000,276.1000) -- (208.6000,276.1000) -- (208.6000,276.1000) -- (208.7000,276.1000) -- (208.8000,276.1000) -- (208.8000,276.1000) -- (208.9000,276.1000) -- (208.9000,276.1000) -- (209.0000,276.1000) -- (209.1000,276.1000) -- (209.1000,276.1000) -- (209.2000,276.1000) -- (209.2000,276.1000) -- (209.3000,276.1000) -- (209.4000,276.1000) -- (209.4000,276.1000) -- (209.5000,276.1000) -- (209.6000,276.1000) -- (209.6000,276.1000) -- (209.7000,276.1000) -- (209.7000,276.1000) -- (209.8000,276.1000) -- (209.9000,276.1000) -- (209.9000,276.1000) -- (210.0000,276.1000) -- (210.0000,276.1000) -- (210.1000,276.1000) -- (210.2000,276.1000) -- (210.2000,276.1000) -- (210.3000,276.1000) -- (210.4000,276.1000) -- (210.4000,276.1000) -- (210.5000,276.1000) -- (210.5000,276.1000) -- (210.6000,276.1000) -- (210.7000,276.1000) -- (210.7000,276.1000) -- (210.8000,276.1000) -- (210.8000,276.1000) -- (210.9000,276.1000) -- (211.0000,276.1000) -- (211.0000,276.1000) -- (211.1000,276.1000) -- (211.2000,276.1000) -- (211.2000,276.1000) -- (211.3000,276.1000) -- (211.3000,276.1000) -- (211.4000,276.1000) -- (211.5000,276.1000) -- (211.5000,276.1000) -- (211.6000,276.1000) -- (211.6000,276.1000) -- (211.7000,276.1000) -- (211.8000,276.1000) -- (211.8000,276.1000) -- (211.9000,276.1000) -- (212.0000,276.1000) -- (212.0000,276.1000) -- (212.1000,276.1000) -- (212.1000,276.1000) -- (212.2000,276.1000) -- (212.3000,276.1000) -- (212.3000,276.1000) -- (212.4000,276.1000) -- (212.4000,276.1000) -- (212.5000,276.1000) -- (212.6000,276.1000) -- (212.6000,276.1000) -- (212.7000,276.1000) -- (212.8000,276.1000) -- (212.8000,276.1000) -- (212.9000,276.1000) -- (212.9000,276.1000) -- (213.0000,276.1000) -- (213.1000,276.1000) -- (213.1000,276.1000) -- (213.2000,276.1000) -- (213.2000,276.1000) -- (213.3000,276.1000) -- (213.4000,276.1000) -- (213.4000,276.1000) -- (213.5000,276.1000) -- (213.6000,276.1000) -- (213.6000,276.1000) -- (213.7000,276.1000) -- (213.7000,276.1000) -- (213.8000,276.1000) -- (213.9000,276.1000) -- (213.9000,276.1000) -- (214.0000,276.1000) -- (214.0000,276.1000) -- (214.1000,276.1000) -- (214.2000,276.1000) -- (214.2000,276.1000) -- (214.3000,276.1000) -- (214.4000,276.1000) -- (214.4000,276.1000) -- (214.5000,276.1000) -- (214.5000,276.1000) -- (214.6000,276.1000) -- (214.7000,276.1000) -- (214.7000,276.1000) -- (214.8000,276.1000) -- (214.8000,276.1000) -- (214.9000,276.1000) -- (215.0000,276.1000) -- (215.0000,276.1000) -- (215.1000,276.1000) -- (215.1000,276.1000) -- (215.2000,276.1000) -- (215.3000,276.1000) -- (215.3000,276.1000) -- (215.4000,276.1000) -- (215.5000,276.1000) -- (215.5000,276.1000) -- (215.6000,276.1000) -- (215.6000,276.1000) -- (215.7000,276.1000) -- (215.8000,276.1000) -- (215.8000,276.1000) -- (215.9000,276.1000) -- (215.9000,276.1000) -- (216.0000,276.1000) -- (216.1000,276.1000) -- (216.1000,276.1000) -- (216.2000,276.1000) -- (216.3000,276.1000) -- (216.3000,276.1000) -- (216.4000,276.1000) -- (216.4000,276.1000) -- (216.5000,276.1000) -- (216.6000,276.1000) -- (216.6000,276.1000) -- (216.7000,276.1000) -- (216.7000,276.1000) -- (216.8000,276.1000) -- (216.9000,276.1000) -- (216.9000,276.1000) -- (217.0000,276.1000) -- (217.1000,276.1000) -- (217.1000,276.1000) -- (217.2000,276.1000) -- (217.2000,276.1000) -- (217.3000,276.1000) -- (217.4000,276.1000) -- (217.4000,276.1000) -- (217.5000,276.1000) -- (217.5000,276.1000) -- (217.6000,276.1000) -- (217.7000,276.1000) -- (217.7000,276.1000) -- (217.8000,276.1000) -- (217.9000,276.1000) -- (217.9000,276.1000) -- (218.0000,276.1000) -- (218.0000,276.1000) -- (218.1000,276.1000) -- (218.2000,276.1000) -- (218.2000,276.1000) -- (218.3000,276.1000) -- (218.3000,276.1000) -- (218.4000,276.1000) -- (218.5000,276.1000) -- (218.5000,276.1000) -- (218.6000,276.1000) -- (218.7000,276.1000) -- (218.7000,276.1000) -- (218.8000,276.1000) -- (218.8000,276.1000) -- (218.9000,276.1000) -- (219.0000,276.1000) -- (219.0000,276.1000) -- (219.1000,276.1000) -- (219.1000,276.1000) -- (219.2000,276.1000) -- (219.3000,276.1000) -- (219.3000,276.1000) -- (219.4000,276.1000) -- (219.5000,276.1000) -- (219.5000,276.1000) -- (219.6000,276.1000) -- (219.6000,276.1000) -- (219.7000,276.1000) -- (219.8000,276.1000) -- (219.8000,276.1000) -- (219.9000,276.1000) -- (219.9000,276.1000) -- (220.0000,276.1000) -- (220.1000,276.1000) -- (220.1000,276.1000) -- (220.2000,276.1000) -- (220.3000,276.1000) -- (220.3000,276.1000) -- (220.4000,276.1000) -- (220.4000,276.1000) -- (220.5000,276.1000) -- (220.6000,276.1000) -- (220.6000,276.1000) -- (220.7000,276.1000) -- (220.7000,276.1000) -- (220.8000,276.1000) -- (220.9000,276.1000) -- (220.9000,276.1000) -- (221.0000,276.1000) -- (221.1000,276.1000) -- (221.1000,276.1000) -- (221.2000,276.1000) -- (221.2000,276.1000) -- (221.3000,276.1000) -- (221.4000,276.1000) -- (221.4000,276.1000) -- (221.5000,276.1000) -- (221.5000,276.1000) -- (221.6000,276.1000) -- (221.7000,276.1000) -- (221.7000,276.1000) -- (221.8000,276.1000) -- (221.9000,276.1000) -- (221.9000,276.1000) -- (222.0000,276.1000) -- (222.0000,276.1000) -- (222.1000,276.1000) -- (222.2000,276.1000) -- (222.2000,276.1000) -- (222.3000,276.1000) -- (222.3000,276.1000) -- (222.4000,276.1000) -- (222.5000,276.1000) -- (222.5000,276.1000) -- (222.6000,276.1000) -- (222.7000,276.1000) -- (222.7000,276.1000) -- (222.8000,276.1000) -- (222.8000,276.1000) -- (222.9000,276.1000) -- (223.0000,276.1000) -- (223.0000,276.1000) -- (223.1000,276.1000) -- (223.1000,276.1000) -- (223.2000,276.1000) -- (223.3000,276.1000) -- (223.3000,276.1000) -- (223.4000,276.1000) -- (223.5000,276.1000) -- (223.5000,276.1000) -- (223.6000,276.1000) -- (223.6000,276.1000) -- (223.7000,276.1000) -- (223.8000,276.1000) -- (223.8000,276.1000) -- (223.9000,276.1000) -- (223.9000,276.1000) -- (224.0000,276.1000) -- (224.1000,276.1000) -- (224.1000,276.1000) -- (224.2000,276.1000) -- (224.3000,276.1000) -- (224.3000,276.1000) -- (224.4000,276.1000) -- (224.4000,276.1000) -- (224.5000,276.1000) -- (224.6000,276.1000) -- (224.6000,276.1000) -- (224.7000,276.1000) -- (224.7000,276.1000) -- (224.8000,276.1000) -- (224.9000,276.1000) -- (224.9000,276.1000) -- (225.0000,276.1000) -- (225.1000,276.1000) -- (225.1000,276.1000) -- (225.2000,276.1000) -- (225.2000,276.1000) -- (225.3000,276.1000) -- (225.4000,276.1000) -- (225.4000,276.1000) -- (225.5000,276.1000) -- (225.5000,276.1000) -- (225.6000,276.1000) -- (225.7000,276.1000) -- (225.7000,276.1000) -- (225.8000,276.1000) -- (225.9000,276.1000) -- (225.9000,276.1000) -- (226.0000,276.1000) -- (226.0000,276.1000) -- (226.1000,276.1000) -- (226.2000,276.1000) -- (226.2000,276.1000) -- (226.3000,276.1000) -- (226.3000,276.1000) -- (226.4000,276.1000) -- (226.5000,276.1000) -- (226.5000,276.1000) -- (226.6000,276.1000) -- (226.6000,276.1000) -- (226.7000,276.1000) -- (226.8000,276.1000) -- (226.8000,276.1000) -- (226.9000,276.1000) -- (227.0000,276.1000) -- (227.0000,276.1000) -- (227.1000,276.1000) -- (227.1000,276.1000) -- (227.2000,276.1000) -- (227.3000,276.1000) -- (227.3000,276.1000) -- (227.4000,276.1000) -- (227.4000,276.1000) -- (227.5000,276.1000) -- (227.6000,276.1000) -- (227.6000,276.1000) -- (227.7000,276.1000) -- (227.8000,276.1000) -- (227.8000,276.1000) -- (227.9000,276.1000) -- (227.9000,276.1000) -- (228.0000,276.1000) -- (228.1000,276.1000) -- (228.1000,276.1000) -- (228.2000,276.1000) -- (228.2000,276.1000) -- (228.3000,276.1000) -- (228.4000,276.1000) -- (228.4000,276.1000) -- (228.5000,276.1000) -- (228.6000,276.1000) -- (228.6000,276.1000) -- (228.7000,276.1000) -- (228.7000,276.1000) -- (228.8000,276.1000) -- (228.9000,276.1000) -- (228.9000,276.1000) -- (229.0000,276.1000) -- (229.0000,276.1000) -- (229.1000,276.1000) -- (229.2000,276.1000) -- (229.2000,276.1000) -- (229.3000,276.1000) -- (229.4000,276.1000) -- (229.4000,276.1000) -- (229.5000,276.1000) -- (229.5000,276.1000) -- (229.6000,276.1000) -- (229.7000,276.1000) -- (229.7000,276.1000) -- (229.8000,276.1000) -- (229.8000,276.1000) -- (229.9000,276.1000) -- (230.0000,276.1000) -- (230.0000,276.1000) -- (230.1000,276.1000) -- (230.2000,276.1000) -- (230.2000,276.1000) -- (230.3000,276.1000) -- (230.3000,276.1000) -- (230.4000,276.1000) -- (230.5000,276.1000) -- (230.5000,276.1000) -- (230.6000,276.1000) -- (230.6000,276.1000) -- (230.7000,276.1000) -- (230.8000,276.1000) -- (230.8000,276.1000) -- (230.9000,276.1000) -- (231.0000,276.1000) -- (231.0000,276.1000) -- (231.1000,276.1000) -- (231.1000,276.1000) -- (231.2000,276.1000) -- (231.3000,276.1000) -- (231.3000,276.1000) -- (231.4000,276.1000) -- (231.4000,276.1000) -- (231.5000,276.1000) -- (231.6000,276.1000) -- (231.6000,276.1000) -- (231.7000,276.1000) -- (231.8000,276.1000) -- (231.8000,276.1000) -- (231.9000,276.1000) -- (231.9000,276.1000) -- (232.0000,276.1000) -- (232.1000,276.1000) -- (232.1000,276.1000) -- (232.2000,276.1000) -- (232.2000,276.1000) -- (232.3000,276.1000) -- (232.4000,276.1000) -- (232.4000,276.1000) -- (232.5000,276.1000) -- (232.6000,276.1000) -- (232.6000,276.1000) -- (232.7000,276.1000) -- (232.7000,276.1000) -- (232.8000,276.1000) -- (232.9000,276.1000) -- (232.9000,276.1000) -- (233.0000,276.1000) -- (233.0000,276.1000) -- (233.1000,276.1000) -- (233.2000,276.1000) -- (233.2000,276.1000) -- (233.3000,276.1000) -- (233.4000,276.1000) -- (233.4000,276.1000) -- (233.5000,276.1000) -- (233.5000,276.1000) -- (233.6000,276.1000) -- (233.7000,276.1000) -- (233.7000,276.1000) -- (233.8000,276.1000) -- (233.8000,276.1000) -- (233.9000,276.1000) -- (234.0000,276.1000) -- (234.0000,276.1000) -- (234.1000,276.1000) -- (234.2000,276.1000) -- (234.2000,276.1000) -- (234.3000,276.1000) -- (234.3000,276.1000) -- (234.4000,276.1000) -- (234.5000,276.1000) -- (234.5000,276.1000) -- (234.6000,276.1000) -- (234.6000,276.1000) -- (234.7000,276.1000) -- (234.8000,276.1000) -- (234.8000,276.1000) -- (234.9000,276.1000) -- (235.0000,276.1000) -- (235.0000,276.1000) -- (235.1000,276.1000) -- (235.1000,276.1000) -- (235.2000,276.1000) -- (235.3000,276.1000) -- (235.3000,276.1000) -- (235.4000,276.1000) -- (235.4000,276.1000) -- (235.5000,276.1000) -- (235.6000,276.1000) -- (235.6000,276.1000) -- (235.7000,276.1000) -- (235.8000,276.1000) -- (235.8000,276.1000) -- (235.9000,276.1000) -- (235.9000,276.1000) -- (236.0000,276.1000) -- (236.1000,276.1000) -- (236.1000,276.1000) -- (236.2000,276.1000) -- (236.2000,276.1000) -- (236.3000,276.1000) -- (236.4000,276.1000) -- (236.4000,276.1000) -- (236.5000,276.1000) -- (236.6000,276.1000) -- (236.6000,276.1000) -- (236.7000,276.1000) -- (236.7000,276.1000) -- (236.8000,276.1000) -- (236.9000,276.1000) -- (236.9000,276.1000) -- (237.0000,276.1000) -- (237.0000,276.1000) -- (237.1000,276.1000) -- (237.2000,276.1000) -- (237.2000,276.1000) -- (237.3000,276.1000) -- (237.4000,276.1000) -- (237.4000,276.1000) -- (237.5000,276.1000) -- (237.5000,276.1000) -- (237.6000,276.1000) -- (237.7000,276.1000) -- (237.7000,276.1000) -- (237.8000,276.1000) -- (237.8000,276.1000) -- (237.9000,276.1000) -- (238.0000,276.1000) -- (238.0000,276.1000) -- (238.1000,276.1000) -- (238.1000,276.1000) -- (238.2000,276.1000) -- (238.3000,276.1000) -- (238.3000,276.1000) -- (238.4000,276.1000) -- (238.5000,276.1000) -- (238.5000,276.1000) -- (238.6000,276.1000) -- (238.6000,276.1000) -- (238.7000,276.1000) -- (238.8000,276.1000) -- (238.8000,276.1000) -- (238.9000,276.1000) -- (238.9000,276.1000) -- (239.0000,276.1000) -- (239.1000,276.1000) -- (239.1000,276.1000) -- (239.2000,276.1000) -- (239.3000,276.1000) -- (239.3000,276.1000) -- (239.4000,276.1000) -- (239.4000,276.1000) -- (239.5000,276.1000) -- (239.6000,276.1000) -- (239.6000,276.1000) -- (239.7000,276.1000) -- (239.7000,276.1000) -- (239.8000,276.1000) -- (239.9000,276.1000) -- (239.9000,276.1000) -- (240.0000,276.1000) -- (240.1000,276.1000) -- (240.1000,276.1000) -- (240.2000,276.1000) -- (240.2000,276.1000) -- (240.3000,276.1000) -- (240.4000,276.1000) -- (240.4000,276.1000) -- (240.5000,276.1000) -- (240.5000,276.1000) -- (240.6000,276.1000) -- (240.7000,276.1000) -- (240.7000,276.1000) -- (240.8000,276.1000) -- (240.9000,276.1000) -- (240.9000,276.1000) -- (241.0000,276.1000) -- (241.0000,276.1000) -- (241.1000,276.1000) -- (241.2000,276.1000) -- (241.2000,276.1000) -- (241.3000,276.1000) -- (241.3000,276.1000) -- (241.4000,276.1000) -- (241.5000,276.1000) -- (241.5000,276.1000) -- (241.6000,276.1000) -- (241.7000,276.1000) -- (241.7000,276.1000) -- (241.8000,276.1000) -- (241.8000,276.1000) -- (241.9000,276.1000) -- (242.0000,276.1000) -- (242.0000,276.1000) -- (242.1000,276.1000) -- (242.1000,276.1000) -- (242.2000,276.1000) -- (242.3000,276.1000) -- (242.3000,276.1000) -- (242.4000,276.1000) -- (242.5000,276.1000) -- (242.5000,276.1000) -- (242.6000,276.1000) -- (242.6000,276.1000) -- (242.7000,276.1000) -- (242.8000,276.1000) -- (242.8000,276.1000) -- (242.9000,276.1000) -- (242.9000,276.1000) -- (243.0000,276.1000) -- (243.1000,276.1000) -- (243.1000,276.1000) -- (243.2000,276.1000) -- (243.3000,276.1000) -- (243.3000,276.1000) -- (243.4000,276.1000) -- (243.4000,276.1000) -- (243.5000,276.1000) -- (243.6000,276.1000) -- (243.6000,276.1000) -- (243.7000,276.1000) -- (243.7000,276.1000) -- (243.8000,276.1000) -- (243.9000,276.1000) -- (243.9000,276.1000) -- (244.0000,276.1000) -- (244.1000,276.1000) -- (244.1000,276.1000) -- (244.2000,276.1000) -- (244.2000,276.1000) -- (244.3000,276.1000) -- (244.4000,276.1000) -- (244.4000,276.1000) -- (244.5000,276.1000) -- (244.5000,276.1000) -- (244.6000,276.1000) -- (244.7000,276.1000) -- (244.7000,276.1000) -- (244.8000,276.1000) -- (244.9000,276.1000) -- (244.9000,276.1000) -- (245.0000,276.1000) -- (245.0000,276.1000) -- (245.1000,276.1000) -- (245.2000,276.1000) -- (245.2000,276.1000) -- (245.3000,276.1000) -- (245.3000,276.1000) -- (245.4000,276.1000) -- (245.5000,276.1000) -- (245.5000,276.1000) -- (245.6000,276.1000) -- (245.7000,276.1000) -- (245.7000,276.1000) -- (245.8000,276.1000) -- (245.8000,276.1000) -- (245.9000,276.1000) -- (246.0000,276.1000) -- (246.0000,276.1000) -- (246.1000,276.1000) -- (246.1000,276.1000) -- (246.2000,276.1000) -- (246.3000,276.1000);



  \end{scope}
  \begin{scope}[scale=1.006,draw=black,line join=bevel,line cap=rect,line width=0.800pt]
  \end{scope}
  \begin{scope}[draw=black,line join=bevel,line cap=rect,line width=0.800pt]
  \end{scope}

  \begin{scope}[cm={{1.00588,0.0,0.0,1.00588,(93.22185,353.00124)}},draw=black,line join=bevel,line cap=rect,line width=0.800pt]
    \path[fill=black] (8.9474,0.0000) node[above right] (text1828-5) {x (m)};



    \path[fill=black,line width=0.800pt] (105.4224,0.1883) node[above right] (text1828-5-3) {\scriptsize Time (min)};



  \end{scope}

\end{scope}

\end{tikzpicture}


%  \caption[Path of a static and dynamic plan]{Path simulations with variations of wind speed and direction. In \hyperref[fig:trajs-I-static]{5.I} and \hyperref[fig:trajs-II-static]{5.II} the path is static. It is dynamically replanned with the algorithm in \hyperref[fig:trajs-dyn-i]{5.i} and \hyperref[fig:trajs-dyn-ii]{5.ii}. The algorithm adapts path parameter radius of the circle $c_{1,1}$ and computation parameter fps rate $c_{1,2}$.}
%  \label{fig:trajs}
%\end{figure}
%\begin{figure}[h]
%  \centering
%  \footnotesize
%  
\definecolor{ca0a0a4}{RGB}{160,160,164}
\definecolor{cffffff}{RGB}{255,255,255}
\definecolor{cff0000}{RGB}{255,0,0}
\definecolor{c00ff00}{RGB}{0,255,0}


\def \globalscale {1.000000}
\begin{tikzpicture}[y=0.8pt, x=0.9pt, yscale=-\globalscale, xscale=\globalscale, inner sep=0pt, outer sep=0pt]
\begin{scope}[shift={(-55.85969,-13.57938)},draw=black,line join=bevel,line cap=rect,even odd rule,line width=0.800pt]
  \begin{scope}[draw=black,line join=bevel,line cap=rect,line width=0.800pt]
  \end{scope}
  \begin{scope}[scale=1.006,draw=black,line join=bevel,line cap=rect,line width=0.800pt]
  \end{scope}
  \begin{scope}[scale=1.006,draw=ca0a0a4,dash pattern=on 0.40pt off 0.80pt,line join=round,line cap=round,line width=0.400pt]
    \path[draw] (56.5000,88.5000) -- (142.5000,88.5000);



  \end{scope}
  \begin{scope}[scale=1.006,draw=black,line join=round,line cap=round,line width=0.480pt]
    \path[draw] (56.5000,88.5000) -- (59.5000,88.5000);



    \path[draw] (142.5000,88.5000) -- (139.5000,88.5000);



  \end{scope}
  \begin{scope}[scale=1.006,draw=black,line join=bevel,line cap=rect,line width=0.800pt]
  \end{scope}
  \begin{scope}[cm={{1.00588,0.0,0.0,1.00588,(39.2294,93.5471)}},draw=black,line join=bevel,line cap=rect,line width=0.800pt]
  \end{scope}
  \begin{scope}[cm={{1.00588,0.0,0.0,1.00588,(39.2294,93.5471)}},draw=black,line join=bevel,line cap=rect,line width=0.800pt]
  \end{scope}
  \begin{scope}[cm={{1.00588,0.0,0.0,1.00588,(39.2294,93.5471)}},draw=black,line join=bevel,line cap=rect,line width=0.800pt]
  \end{scope}
  \begin{scope}[cm={{1.00588,0.0,0.0,1.00588,(39.2294,93.5471)}},draw=black,line join=bevel,line cap=rect,line width=0.800pt]
  \end{scope}
  \begin{scope}[cm={{1.00588,0.0,0.0,1.00588,(39.2294,93.5471)}},draw=black,line join=bevel,line cap=rect,line width=0.800pt]
  \end{scope}
  \begin{scope}[cm={{1.00588,0.0,0.0,1.00588,(39.2294,93.5471)}},draw=black,line join=bevel,line cap=rect,line width=0.800pt]
    \path[fill=black] (0.0000,0.0000) node[above right] (text34) {32};



  \end{scope}
  \begin{scope}[cm={{1.00588,0.0,0.0,1.00588,(39.2294,93.5471)}},draw=black,line join=bevel,line cap=rect,line width=0.800pt]
  \end{scope}
  \begin{scope}[scale=1.006,draw=black,line join=bevel,line cap=rect,line width=0.800pt]
  \end{scope}
  \begin{scope}[scale=1.006,draw=ca0a0a4,dash pattern=on 0.40pt off 0.80pt,line join=round,line cap=round,line width=0.400pt]
    \path[draw] (56.5000,63.5000) -- (142.5000,63.5000);



  \end{scope}
  \begin{scope}[scale=1.006,draw=black,line join=round,line cap=round,line width=0.480pt]
    \path[draw] (56.5000,63.5000) -- (59.5000,63.5000);



    \path[draw] (142.5000,63.5000) -- (139.5000,63.5000);



  \end{scope}
  \begin{scope}[scale=1.006,draw=black,line join=bevel,line cap=rect,line width=0.800pt]
  \end{scope}
  \begin{scope}[cm={{1.00588,0.0,0.0,1.00588,(39.2294,68.4)}},draw=black,line join=bevel,line cap=rect,line width=0.800pt]
  \end{scope}
  \begin{scope}[cm={{1.00588,0.0,0.0,1.00588,(39.2294,68.4)}},draw=black,line join=bevel,line cap=rect,line width=0.800pt]
  \end{scope}
  \begin{scope}[cm={{1.00588,0.0,0.0,1.00588,(39.2294,68.4)}},draw=black,line join=bevel,line cap=rect,line width=0.800pt]
  \end{scope}
  \begin{scope}[cm={{1.00588,0.0,0.0,1.00588,(39.2294,68.4)}},draw=black,line join=bevel,line cap=rect,line width=0.800pt]
  \end{scope}
  \begin{scope}[cm={{1.00588,0.0,0.0,1.00588,(39.2294,68.4)}},draw=black,line join=bevel,line cap=rect,line width=0.800pt]
  \end{scope}
  \begin{scope}[cm={{1.00588,0.0,0.0,1.00588,(39.2294,68.4)}},draw=black,line join=bevel,line cap=rect,line width=0.800pt]
    \path[fill=black] (0.0000,0.0000) node[above right] (text64) {36};



  \end{scope}
  \begin{scope}[cm={{1.00588,0.0,0.0,1.00588,(39.2294,68.4)}},draw=black,line join=bevel,line cap=rect,line width=0.800pt]
  \end{scope}
  \begin{scope}[scale=1.006,draw=black,line join=bevel,line cap=rect,line width=0.800pt]
  \end{scope}
  \begin{scope}[scale=1.006,draw=ca0a0a4,dash pattern=on 0.40pt off 0.80pt,line join=round,line cap=round,line width=0.400pt]
    \path[draw] (56.5000,38.5000) -- (142.5000,38.5000);



  \end{scope}
  \begin{scope}[scale=1.006,draw=black,line join=round,line cap=round,line width=0.480pt]
    \path[draw] (56.5000,38.5000) -- (59.5000,38.5000);



    \path[draw] (142.5000,38.5000) -- (139.5000,38.5000);



  \end{scope}
  \begin{scope}[scale=1.006,draw=black,line join=bevel,line cap=rect,line width=0.800pt]
  \end{scope}
  \begin{scope}[cm={{1.00588,0.0,0.0,1.00588,(39.2294,43.2529)}},draw=black,line join=bevel,line cap=rect,line width=0.800pt]
  \end{scope}
  \begin{scope}[cm={{1.00588,0.0,0.0,1.00588,(39.2294,43.2529)}},draw=black,line join=bevel,line cap=rect,line width=0.800pt]
  \end{scope}
  \begin{scope}[cm={{1.00588,0.0,0.0,1.00588,(39.2294,43.2529)}},draw=black,line join=bevel,line cap=rect,line width=0.800pt]
  \end{scope}
  \begin{scope}[cm={{1.00588,0.0,0.0,1.00588,(39.2294,43.2529)}},draw=black,line join=bevel,line cap=rect,line width=0.800pt]
  \end{scope}
  \begin{scope}[cm={{1.00588,0.0,0.0,1.00588,(39.2294,43.2529)}},draw=black,line join=bevel,line cap=rect,line width=0.800pt]
  \end{scope}
  \begin{scope}[cm={{1.00588,0.0,0.0,1.00588,(39.2294,43.2529)}},draw=black,line join=bevel,line cap=rect,line width=0.800pt]
    \path[fill=black] (0.0000,0.0000) node[above right] (text94) {40};



  \end{scope}
  \begin{scope}[cm={{1.00588,0.0,0.0,1.00588,(39.2294,43.2529)}},draw=black,line join=bevel,line cap=rect,line width=0.800pt]
  \end{scope}
  \begin{scope}[scale=1.006,draw=black,line join=bevel,line cap=rect,line width=0.800pt]
  \end{scope}
  \begin{scope}[scale=1.006,draw=ca0a0a4,dash pattern=on 0.40pt off 0.80pt,line join=round,line cap=round,line width=0.400pt]
    \path[draw] (56.5000,95.5000) -- (56.5000,13.5000);



  \end{scope}
  \begin{scope}[scale=1.006,draw=black,line join=round,line cap=round,line width=0.480pt]
    \path[draw] (56.5000,95.5000) -- (56.5000,92.5000);



    \path[draw] (56.5000,13.5000) -- (56.5000,16.5000);



  \end{scope}
  \begin{scope}[scale=1.006,draw=black,line join=bevel,line cap=rect,line width=0.800pt]
  \end{scope}
  \begin{scope}[cm={{1.00588,0.0,0.0,1.00588,(53.3118,110.647)}},draw=black,line join=bevel,line cap=rect,line width=0.800pt]
  \end{scope}
  \begin{scope}[cm={{1.00588,0.0,0.0,1.00588,(53.3118,110.647)}},draw=black,line join=bevel,line cap=rect,line width=0.800pt]
  \end{scope}
  \begin{scope}[cm={{1.00588,0.0,0.0,1.00588,(53.3118,110.647)}},draw=black,line join=bevel,line cap=rect,line width=0.800pt]
  \end{scope}
  \begin{scope}[cm={{1.00588,0.0,0.0,1.00588,(53.3118,110.647)}},draw=black,line join=bevel,line cap=rect,line width=0.800pt]
  \end{scope}
  \begin{scope}[cm={{1.00588,0.0,0.0,1.00588,(53.3118,110.647)}},draw=black,line join=bevel,line cap=rect,line width=0.800pt]
  \end{scope}
  \begin{scope}[cm={{1.00588,0.0,0.0,1.00588,(53.3118,110.647)}},draw=black,line join=bevel,line cap=rect,line width=0.800pt]
    \path[fill=black] (0.0000,0.0000) node[above right] (text124) {0};



  \end{scope}
  \begin{scope}[cm={{1.00588,0.0,0.0,1.00588,(53.3118,110.647)}},draw=black,line join=bevel,line cap=rect,line width=0.800pt]
  \end{scope}
  \begin{scope}[scale=1.006,draw=black,line join=bevel,line cap=rect,line width=0.800pt]
  \end{scope}
  \begin{scope}[scale=1.006,draw=ca0a0a4,dash pattern=on 0.40pt off 0.80pt,line join=round,line cap=round,line width=0.400pt]
    \path[draw] (82.5000,95.5000) -- (82.5000,36.5000);



    \path[draw] (82.5000,20.5000) -- (82.5000,13.5000);



  \end{scope}
  \begin{scope}[scale=1.006,draw=black,line join=round,line cap=round,line width=0.480pt]
    \path[draw] (82.5000,95.5000) -- (82.5000,92.5000);



    \path[draw] (82.5000,13.5000) -- (82.5000,16.5000);



  \end{scope}
  \begin{scope}[scale=1.006,draw=black,line join=bevel,line cap=rect,line width=0.800pt]
  \end{scope}
  \begin{scope}[cm={{1.00588,0.0,0.0,1.00588,(79.4647,110.647)}},draw=black,line join=bevel,line cap=rect,line width=0.800pt]
  \end{scope}
  \begin{scope}[cm={{1.00588,0.0,0.0,1.00588,(79.4647,110.647)}},draw=black,line join=bevel,line cap=rect,line width=0.800pt]
  \end{scope}
  \begin{scope}[cm={{1.00588,0.0,0.0,1.00588,(79.4647,110.647)}},draw=black,line join=bevel,line cap=rect,line width=0.800pt]
  \end{scope}
  \begin{scope}[cm={{1.00588,0.0,0.0,1.00588,(79.4647,110.647)}},draw=black,line join=bevel,line cap=rect,line width=0.800pt]
  \end{scope}
  \begin{scope}[cm={{1.00588,0.0,0.0,1.00588,(79.4647,110.647)}},draw=black,line join=bevel,line cap=rect,line width=0.800pt]
  \end{scope}
  \begin{scope}[cm={{1.00588,0.0,0.0,1.00588,(79.4647,110.647)}},draw=black,line join=bevel,line cap=rect,line width=0.800pt]
    \path[fill=black] (0.0000,0.0000) node[above right] (text156) {1};



  \end{scope}
  \begin{scope}[cm={{1.00588,0.0,0.0,1.00588,(79.4647,110.647)}},draw=black,line join=bevel,line cap=rect,line width=0.800pt]
  \end{scope}
  \begin{scope}[scale=1.006,draw=black,line join=bevel,line cap=rect,line width=0.800pt]
  \end{scope}
  \begin{scope}[scale=1.006,draw=ca0a0a4,dash pattern=on 0.40pt off 0.80pt,line join=round,line cap=round,line width=0.400pt]
    \path[draw] (108.5000,95.5000) -- (108.5000,36.5000);



    \path[draw] (108.5000,20.5000) -- (108.5000,13.5000);



  \end{scope}
  \begin{scope}[scale=1.006,draw=black,line join=round,line cap=round,line width=0.480pt]
    \path[draw] (108.5000,95.5000) -- (108.5000,92.5000);



    \path[draw] (108.5000,13.5000) -- (108.5000,16.5000);



  \end{scope}
  \begin{scope}[scale=1.006,draw=black,line join=bevel,line cap=rect,line width=0.800pt]
  \end{scope}
  \begin{scope}[cm={{1.00588,0.0,0.0,1.00588,(105.618,110.647)}},draw=black,line join=bevel,line cap=rect,line width=0.800pt]
  \end{scope}
  \begin{scope}[cm={{1.00588,0.0,0.0,1.00588,(105.618,110.647)}},draw=black,line join=bevel,line cap=rect,line width=0.800pt]
  \end{scope}
  \begin{scope}[cm={{1.00588,0.0,0.0,1.00588,(105.618,110.647)}},draw=black,line join=bevel,line cap=rect,line width=0.800pt]
  \end{scope}
  \begin{scope}[cm={{1.00588,0.0,0.0,1.00588,(105.618,110.647)}},draw=black,line join=bevel,line cap=rect,line width=0.800pt]
  \end{scope}
  \begin{scope}[cm={{1.00588,0.0,0.0,1.00588,(105.618,110.647)}},draw=black,line join=bevel,line cap=rect,line width=0.800pt]
  \end{scope}
  \begin{scope}[cm={{1.00588,0.0,0.0,1.00588,(105.618,110.647)}},draw=black,line join=bevel,line cap=rect,line width=0.800pt]
    \path[fill=black] (0.0000,0.0000) node[above right] (text188) {2};



  \end{scope}
  \begin{scope}[cm={{1.00588,0.0,0.0,1.00588,(105.618,110.647)}},draw=black,line join=bevel,line cap=rect,line width=0.800pt]
  \end{scope}
  \begin{scope}[scale=1.006,draw=black,line join=bevel,line cap=rect,line width=0.800pt]
  \end{scope}
  \begin{scope}[scale=1.006,draw=ca0a0a4,dash pattern=on 0.40pt off 0.80pt,line join=round,line cap=round,line width=0.400pt]
    \path[draw] (134.5000,95.5000) -- (134.5000,13.5000);



  \end{scope}
  \begin{scope}[scale=1.006,draw=black,line join=round,line cap=round,line width=0.480pt]
    \path[draw] (134.5000,95.5000) -- (134.5000,92.5000);



    \path[draw] (134.5000,13.5000) -- (134.5000,16.5000);



  \end{scope}
  \begin{scope}[scale=1.006,draw=black,line join=bevel,line cap=rect,line width=0.800pt]
  \end{scope}
  \begin{scope}[cm={{1.00588,0.0,0.0,1.00588,(132.274,110.647)}},draw=black,line join=bevel,line cap=rect,line width=0.800pt]
  \end{scope}
  \begin{scope}[cm={{1.00588,0.0,0.0,1.00588,(132.274,110.647)}},draw=black,line join=bevel,line cap=rect,line width=0.800pt]
  \end{scope}
  \begin{scope}[cm={{1.00588,0.0,0.0,1.00588,(132.274,110.647)}},draw=black,line join=bevel,line cap=rect,line width=0.800pt]
  \end{scope}
  \begin{scope}[cm={{1.00588,0.0,0.0,1.00588,(132.274,110.647)}},draw=black,line join=bevel,line cap=rect,line width=0.800pt]
  \end{scope}
  \begin{scope}[cm={{1.00588,0.0,0.0,1.00588,(132.274,110.647)}},draw=black,line join=bevel,line cap=rect,line width=0.800pt]
  \end{scope}
  \begin{scope}[cm={{1.00588,0.0,0.0,1.00588,(132.274,110.647)}},draw=black,line join=bevel,line cap=rect,line width=0.800pt]
    \path[fill=black] (0.0000,0.0000) node[above right] (text218) {3};



  \end{scope}
  \begin{scope}[cm={{1.00588,0.0,0.0,1.00588,(132.274,110.647)}},draw=black,line join=bevel,line cap=rect,line width=0.800pt]
  \end{scope}
  \begin{scope}[scale=1.006,draw=black,line join=bevel,line cap=rect,line width=0.800pt]
  \end{scope}
  \begin{scope}[scale=1.006,draw=black,line join=round,line cap=round,line width=0.480pt]
    \path[draw] (56.5000,13.5000) -- (56.5000,95.5000) -- (142.5000,95.5000) -- (142.5000,13.5000) -- (56.5000,13.5000);



  \end{scope}
  \begin{scope}[scale=1.006,draw=black,line join=bevel,line cap=rect,line width=0.800pt]
  \end{scope}
  \begin{scope}[scale=1.006,draw=black,line join=bevel,line cap=rect,line width=0.800pt]
  \end{scope}
  \begin{scope}[scale=1.006,fill=cffffff]
    \path[fill,rounded corners=0.0000cm] (124.0000,17.0000) rectangle (135.0000,33.0000);



  \end{scope}
  \begin{scope}[scale=1.006,draw=black,line join=bevel,line cap=rect,line width=0.800pt]
  \end{scope}
  \begin{scope}[scale=1.006,draw=black,line join=bevel,line cap=rect,line width=0.800pt]
  \end{scope}
  \begin{scope}[scale=1.006,draw=black,line join=round,line cap=round,line width=0.800pt]
    \path[draw] (123.5000,33.5000) -- (123.5000,17.5000) -- (134.5000,17.5000) -- (134.5000,33.5000) -- (123.5000,33.5000);



  \end{scope}
  \begin{scope}[scale=1.006,draw=black,line join=bevel,line cap=rect,line width=0.800pt]
  \end{scope}
  \begin{scope}[cm={{1.00588,0.0,0.0,1.00588,(128.753,29.1706)}},draw=black,line join=bevel,line cap=rect,line width=0.800pt]
  \end{scope}
  \begin{scope}[cm={{1.00588,0.0,0.0,1.00588,(128.753,29.1706)}},draw=black,line join=bevel,line cap=rect,line width=0.800pt]
  \end{scope}
  \begin{scope}[cm={{1.00588,0.0,0.0,1.00588,(128.753,29.1706)}},draw=black,line join=bevel,line cap=rect,line width=0.800pt]
  \end{scope}
  \begin{scope}[cm={{1.00588,0.0,0.0,1.00588,(128.753,29.1706)}},draw=black,line join=bevel,line cap=rect,line width=0.800pt]
  \end{scope}
  \begin{scope}[cm={{1.00588,0.0,0.0,1.00588,(128.753,29.1706)}},draw=black,line join=bevel,line cap=rect,line width=0.800pt]
  \end{scope}
  \begin{scope}[cm={{1.00588,0.0,0.0,1.00588,(128.61515,28.75705)}},draw=black,line join=bevel,line cap=rect,line width=0.800pt]
    \path[fill=black] (0.0000,0.0000) node[above right] (text258) {\label{fig:ener:static-I}I};



  \end{scope}
  \begin{scope}[cm={{1.00588,0.0,0.0,1.00588,(128.753,29.1706)}},draw=black,line join=bevel,line cap=rect,line width=0.800pt]
  \end{scope}
  \begin{scope}[cm={{0.0,-1.00588,1.00588,0.0,(29.1706,189.106)}},draw=black,line join=bevel,line cap=rect,line width=0.800pt]
  \end{scope}
  \begin{scope}[cm={{0.0,-1.00588,1.00588,0.0,(29.1706,189.106)}},draw=black,line join=bevel,line cap=rect,line width=0.800pt]
  \end{scope}
  \begin{scope}[cm={{0.0,-1.00588,1.00588,0.0,(29.1706,189.106)}},draw=black,line join=bevel,line cap=rect,line width=0.800pt]
  \end{scope}
  \begin{scope}[cm={{0.0,-1.00588,1.00588,0.0,(29.1706,189.106)}},draw=black,line join=bevel,line cap=rect,line width=0.800pt]
  \end{scope}
  \begin{scope}[cm={{0.0,-1.00588,1.00588,0.0,(29.1706,189.106)}},draw=black,line join=bevel,line cap=rect,line width=0.800pt]
  \end{scope}
  \begin{scope}[cm={{0.0,-1.00588,1.00588,0.0,(26.1706,189.106)}},draw=black,line join=bevel,line cap=rect,line width=0.800pt]
    \path[fill=black] (0.0000,0.0000) node[above right] (text274) {\rotatebox{90}{Power (W)}};



  \end{scope}
  \begin{scope}[cm={{0.0,-1.00588,1.00588,0.0,(29.1706,189.106)}},draw=black,line join=bevel,line cap=rect,line width=0.800pt]
  \end{scope}
  \begin{scope}[cm={{1.00588,0.0,0.0,1.00588,(62.3647,28.1647)}},draw=black,line join=bevel,line cap=rect,line width=0.800pt]
  \end{scope}
  \begin{scope}[cm={{1.00588,0.0,0.0,1.00588,(62.3647,28.1647)}},draw=black,line join=bevel,line cap=rect,line width=0.800pt]
  \end{scope}
  \begin{scope}[cm={{1.00588,0.0,0.0,1.00588,(62.3647,28.1647)}},draw=black,line join=bevel,line cap=rect,line width=0.800pt]
  \end{scope}
  \begin{scope}[cm={{1.00588,0.0,0.0,1.00588,(62.3647,28.1647)}},draw=black,line join=bevel,line cap=rect,line width=0.800pt]
  \end{scope}
  \begin{scope}[cm={{1.00588,0.0,0.0,1.00588,(62.3647,28.1647)}},draw=black,line join=bevel,line cap=rect,line width=0.800pt]
  \end{scope}
  \begin{scope}[cm={{1.00588,0.0,0.0,1.00588,(66.8647,26.1647)}},draw=black,line join=bevel,line cap=rect,line width=0.800pt]
    \path[fill=black] (0.0000,0.0000) node[above right] (text290) {\scriptsize data};



  \end{scope}
  \begin{scope}[cm={{1.00588,0.0,0.0,1.00588,(62.3647,28.1647)}},draw=black,line join=bevel,line cap=rect,line width=0.800pt]
  \end{scope}
  \begin{scope}[scale=1.006,draw=black,line join=bevel,line cap=rect,line width=0.800pt]
  \end{scope}
  \begin{scope}[scale=1.006,draw=black,line join=round,line cap=round,line width=0.480pt]
    \path[draw,even odd rule] (86.5000,24.5000) -- (112.5000,24.5000);



  \end{scope}
  \begin{scope}[scale=1.006,draw=black,line join=bevel,line cap=rect,line width=0.800pt]
  \end{scope}
  \begin{scope}[scale=1.006,draw=black,line join=bevel,line cap=rect,line width=0.800pt]
  \end{scope}
  \begin{scope}[scale=1.006,draw=black,line join=bevel,line cap=rect,line width=0.800pt]
  \end{scope}
  \begin{scope}[scale=1.006,draw=black,line join=bevel,line cap=rect,line width=0.800pt]
  \end{scope}
  \begin{scope}[scale=1.006,draw=black,line join=round,line cap=round,line width=0.480pt]
    \path[draw] (56.1000,52.3000) -- (56.1000,52.3000) -- (56.5000,69.8000) -- (56.8000,61.5000) -- (57.1000,50.5000) -- (57.4000,49.2000) -- (57.7000,52.5000) -- (58.0000,57.0000) -- (58.4000,59.6000) -- (58.7000,60.0000) -- (59.0000,59.6000) -- (59.3000,59.2000) -- (59.6000,59.1000) -- (59.9000,59.0000) -- (60.2000,59.0000) -- (60.6000,59.1000) -- (60.9000,59.1000) -- (61.2000,59.1000) -- (61.5000,59.1000) -- (61.8000,59.0000) -- (62.1000,59.5000) -- (62.4000,62.4000) -- (62.8000,66.1000) -- (63.1000,69.9000) -- (63.4000,73.4000) -- (63.7000,76.7000) -- (64.0000,79.8000) -- (64.3000,82.6000) -- (64.6000,84.9000) -- (65.0000,86.4000) -- (65.3000,87.1000) -- (65.6000,86.7000) -- (65.9000,85.4000) -- (66.2000,83.2000) -- (66.5000,80.3000) -- (66.8000,77.1000) -- (67.2000,73.7000) -- (67.5000,70.1000) -- (67.8000,65.0000) -- (68.1000,60.3000) -- (68.4000,58.1000) -- (68.7000,58.0000) -- (69.0000,58.5000) -- (69.4000,58.9000) -- (69.7000,59.1000) -- (70.0000,59.1000) -- (70.3000,59.1000) -- (70.6000,59.1000) -- (70.9000,59.1000) -- (71.2000,59.1000) -- (71.5000,59.0000) -- (71.9000,61.2000) -- (72.2000,62.2000) -- (72.5000,59.0000) -- (72.8000,55.3000) -- (73.1000,52.3000) -- (73.4000,50.0000) -- (73.7000,48.2000) -- (74.1000,46.8000) -- (74.4000,45.7000) -- (74.7000,44.9000) -- (75.0000,44.4000) -- (75.3000,44.2000) -- (75.6000,44.3000) -- (75.9000,44.7000) -- (76.3000,45.4000) -- (76.6000,46.4000) -- (76.9000,47.7000) -- (77.2000,49.3000) -- (77.5000,51.3000) -- (77.8000,53.6000) -- (78.1000,56.7000) -- (78.5000,59.2000) -- (78.8000,59.9000) -- (79.1000,59.6000) -- (79.4000,59.3000) -- (79.7000,59.1000) -- (80.0000,59.0000) -- (80.3000,59.0000) -- (80.7000,59.1000) -- (81.0000,59.1000) -- (81.3000,59.1000) -- (81.6000,59.1000) -- (81.9000,59.0000) -- (82.2000,59.1000) -- (82.5000,61.3000) -- (82.9000,65.2000) -- (83.2000,69.1000) -- (83.5000,72.7000) -- (83.8000,76.1000) -- (84.1000,79.3000) -- (84.4000,82.1000) -- (84.7000,84.6000) -- (85.1000,86.3000) -- (85.4000,87.1000) -- (85.7000,86.8000) -- (86.0000,85.5000) -- (86.3000,83.3000) -- (86.6000,80.4000) -- (86.9000,77.1000) -- (87.3000,73.6000) -- (87.6000,69.9000) -- (87.9000,65.0000) -- (88.2000,60.5000) -- (88.5000,58.3000) -- (88.8000,58.1000) -- (89.1000,58.5000) -- (89.5000,58.9000) -- (89.8000,59.1000) -- (90.1000,59.1000) -- (90.4000,59.1000) -- (90.7000,59.1000) -- (91.0000,59.1000) -- (91.3000,59.1000) -- (91.7000,59.0000) -- (92.0000,61.4000) -- (92.3000,62.0000) -- (92.6000,58.6000) -- (92.9000,55.0000) -- (93.2000,52.0000) -- (93.5000,49.7000) -- (93.9000,47.9000) -- (94.2000,46.6000) -- (94.5000,45.5000) -- (94.8000,44.8000) -- (95.1000,44.3000) -- (95.4000,44.2000) -- (95.7000,44.4000) -- (96.1000,44.8000) -- (96.4000,45.6000) -- (96.7000,46.7000) -- (97.0000,48.2000) -- (97.3000,50.0000) -- (97.6000,52.1000) -- (97.9000,54.7000) -- (98.3000,57.7000) -- (98.6000,59.4000) -- (98.9000,59.7000) -- (99.2000,59.5000) -- (99.5000,59.2000) -- (99.8000,59.1000) -- (100.1000,59.0000) -- (100.5000,59.1000) -- (100.8000,59.1000) -- (101.1000,59.1000) -- (101.4000,59.1000) -- (101.7000,59.1000) -- (102.0000,59.0000) -- (102.3000,59.5000) -- (102.7000,62.6000) -- (103.0000,66.8000) -- (103.3000,70.7000) -- (103.6000,74.2000) -- (103.9000,77.5000) -- (104.2000,80.5000) -- (104.5000,83.2000) -- (104.9000,85.3000) -- (105.2000,86.7000) -- (105.5000,87.1000) -- (105.8000,86.3000) -- (106.1000,84.5000) -- (106.4000,81.8000) -- (106.7000,78.6000) -- (107.0000,75.1000) -- (107.4000,71.6000) -- (107.7000,67.2000) -- (108.0000,62.3000) -- (108.3000,59.0000) -- (108.6000,58.0000) -- (108.9000,58.3000) -- (109.3000,58.7000) -- (109.6000,59.0000) -- (109.9000,59.1000) -- (110.2000,59.1000) -- (110.5000,59.1000) -- (110.8000,59.1000) -- (111.1000,59.1000) -- (111.4000,59.0000) -- (111.8000,60.0000) -- (112.1000,62.4000) -- (112.4000,60.2000) -- (112.7000,56.4000) -- (113.0000,53.1000) -- (113.3000,50.5000) -- (113.6000,48.6000) -- (114.0000,47.1000) -- (114.3000,45.9000) -- (114.6000,45.0000) -- (114.9000,44.5000) -- (115.2000,44.2000) -- (115.5000,44.3000) -- (115.8000,44.6000) -- (116.2000,45.3000) -- (116.5000,46.3000) -- (116.8000,47.6000) -- (117.1000,49.4000) -- (117.4000,51.4000) -- (117.7000,53.9000) -- (118.0000,56.8000) -- (118.4000,59.0000) -- (118.7000,59.7000) -- (119.0000,59.6000) -- (119.3000,59.3000) -- (119.6000,59.1000) -- (119.9000,59.1000) -- (120.2000,59.1000) -- (120.6000,59.1000) -- (120.9000,59.1000) -- (121.2000,59.1000) -- (121.5000,59.1000) -- (121.8000,59.0000) -- (122.1000,59.2000) -- (122.4000,61.3000) -- (122.8000,65.4000) -- (123.1000,69.6000) -- (123.4000,73.3000) -- (123.7000,76.6000) -- (124.0000,79.6000) -- (124.3000,82.4000) -- (124.6000,84.7000) -- (125.0000,86.4000) -- (125.3000,87.1000) -- (125.6000,86.7000) -- (125.9000,85.1000) -- (126.2000,82.5000) -- (126.5000,79.4000) -- (126.8000,76.0000) -- (127.2000,72.5000) -- (127.5000,68.4000) -- (127.8000,63.4000) -- (128.1000,59.6000) -- (128.4000,58.1000) -- (128.7000,58.2000) -- (129.0000,58.6000) -- (129.4000,59.0000) -- (129.7000,59.1000) -- (130.0000,59.1000) -- (130.3000,59.1000) -- (130.6000,59.1000) -- (130.9000,59.1000) -- (131.2000,59.0000) -- (131.6000,59.4000) -- (131.9000,62.2000) -- (132.2000,61.0000) -- (132.5000,57.3000) -- (132.8000,53.8000) -- (133.1000,51.1000) -- (133.4000,49.0000) -- (133.8000,47.4000) -- (134.1000,46.1000) -- (134.4000,45.2000) -- (134.7000,44.6000) -- (135.0000,44.2000) -- (135.3000,44.2000) -- (135.6000,44.5000) -- (136.0000,45.1000) -- (136.3000,46.0000) -- (136.6000,47.3000) -- (136.9000,49.0000) -- (137.2000,51.0000) -- (137.5000,53.3000) -- (137.8000,56.2000) -- (138.2000,58.6000) -- (138.5000,59.6000) -- (138.8000,59.6000) -- (139.1000,59.3000) -- (139.4000,59.1000) -- (139.7000,59.1000) -- (140.0000,59.1000) -- (140.4000,59.1000) -- (140.7000,59.1000) -- (141.0000,59.1000) -- (141.3000,59.1000) -- (141.6000,59.1000) -- (141.9000,59.0000) -- (142.2000,60.6000) -- (142.6000,64.5000) -- (142.7000,66.5000);



  \end{scope}
  \begin{scope}[scale=1.006,draw=black,line join=bevel,line cap=rect,line width=0.800pt]
  \end{scope}
  \begin{scope}[cm={{1.00588,0.0,0.0,1.00588,(60.3529,36.2118)}},draw=black,line join=bevel,line cap=rect,line width=0.800pt]
  \end{scope}
  \begin{scope}[cm={{1.00588,0.0,0.0,1.00588,(60.3529,36.2118)}},draw=black,line join=bevel,line cap=rect,line width=0.800pt]
  \end{scope}
  \begin{scope}[cm={{1.00588,0.0,0.0,1.00588,(60.3529,36.2118)}},draw=black,line join=bevel,line cap=rect,line width=0.800pt]
  \end{scope}
  \begin{scope}[cm={{1.00588,0.0,0.0,1.00588,(60.3529,36.2118)}},draw=black,line join=bevel,line cap=rect,line width=0.800pt]
  \end{scope}
  \begin{scope}[cm={{1.00588,0.0,0.0,1.00588,(60.3529,36.2118)}},draw=black,line join=bevel,line cap=rect,line width=0.800pt]
  \end{scope}
  \begin{scope}[cm={{1.00588,0.0,0.0,1.00588,(60.3529,36.2118)}},draw=black,line join=bevel,line cap=rect,line width=0.800pt]
    \path[fill=black] (0.0000,0.0000) node[above right] (text326) {\scriptsize model};



  \end{scope}
  \begin{scope}[cm={{1.00588,0.0,0.0,1.00588,(60.3529,36.2118)}},draw=black,line join=bevel,line cap=rect,line width=0.800pt]
  \end{scope}
  \begin{scope}[scale=1.006,draw=black,line join=bevel,line cap=rect,line width=0.800pt]
  \end{scope}
  \begin{scope}[scale=1.006,draw=cff0000,line join=round,line cap=round,line width=0.480pt]
    \path[draw,even odd rule] (86.5000,32.5000) -- (112.5000,32.5000);



  \end{scope}
  \begin{scope}[scale=1.006,draw=black,line join=bevel,line cap=rect,line width=0.800pt]
  \end{scope}
  \begin{scope}[scale=1.006,draw=black,line join=bevel,line cap=rect,line width=0.800pt]
  \end{scope}
  \begin{scope}[scale=1.006,draw=black,line join=bevel,line cap=rect,line width=0.800pt]
  \end{scope}
  \begin{scope}[scale=1.006,draw=black,line join=bevel,line cap=rect,line width=0.800pt]
  \end{scope}
  \begin{scope}[scale=1.006,draw=cff0000,line join=round,line cap=round,line width=0.480pt]
    \path[draw] (56.0000,44.3000) -- (56.0000,44.3000) -- (56.1000,44.4000) -- (56.2000,44.6000) -- (56.3000,44.8000) -- (56.3000,45.1000) -- (56.4000,45.4000) -- (56.5000,45.7000) -- (56.6000,46.1000) -- (56.7000,46.5000) -- (56.8000,47.0000) -- (56.9000,47.4000) -- (57.0000,47.9000) -- (57.0000,48.4000) -- (57.1000,48.9000) -- (57.2000,49.5000) -- (57.3000,50.0000) -- (57.4000,50.6000) -- (57.5000,51.2000) -- (57.6000,51.8000) -- (57.6000,52.3000) -- (57.7000,52.9000) -- (57.8000,53.5000) -- (57.9000,54.0000) -- (58.0000,54.6000) -- (58.1000,55.1000) -- (58.2000,55.6000) -- (58.3000,56.1000) -- (58.3000,56.6000) -- (58.4000,57.1000) -- (58.5000,57.5000) -- (58.6000,57.9000) -- (58.7000,58.2000) -- (58.8000,58.6000) -- (58.9000,58.9000) -- (59.0000,59.1000) -- (59.0000,59.4000) -- (59.1000,59.6000) -- (59.2000,59.8000) -- (59.3000,59.9000) -- (59.4000,60.0000) -- (59.5000,60.1000) -- (59.6000,60.1000) -- (59.6000,60.1000) -- (59.7000,60.1000) -- (59.8000,60.1000) -- (59.9000,60.0000) -- (60.0000,60.0000) -- (60.1000,59.9000) -- (60.2000,59.7000) -- (60.3000,59.6000) -- (60.3000,59.5000) -- (60.4000,59.3000) -- (60.5000,59.2000) -- (60.6000,59.0000) -- (60.7000,58.9000) -- (60.8000,58.8000) -- (60.9000,58.6000) -- (60.9000,58.5000) -- (61.0000,58.4000) -- (61.1000,58.3000) -- (61.2000,58.2000) -- (61.3000,58.2000) -- (61.4000,58.2000) -- (61.5000,58.2000) -- (61.6000,58.3000) -- (61.6000,58.3000) -- (61.7000,58.5000) -- (61.8000,58.6000) -- (61.9000,58.8000) -- (62.0000,59.1000) -- (62.1000,59.3000) -- (62.2000,59.7000) -- (62.2000,60.0000) -- (62.3000,60.4000) -- (62.4000,60.9000) -- (62.5000,61.4000) -- (62.6000,61.9000) -- (62.7000,62.5000) -- (62.8000,63.1000) -- (62.9000,63.8000) -- (62.9000,64.4000) -- (63.0000,65.2000) -- (63.1000,65.9000) -- (63.2000,66.7000) -- (63.3000,67.5000) -- (63.4000,68.3000) -- (63.5000,69.2000) -- (63.6000,70.0000) -- (63.6000,70.9000) -- (63.7000,71.8000) -- (63.8000,72.7000) -- (63.9000,73.6000) -- (64.0000,74.4000) -- (64.1000,75.3000) -- (64.2000,76.2000) -- (64.2000,77.0000) -- (64.3000,77.8000) -- (64.4000,78.6000) -- (64.5000,79.4000) -- (64.6000,80.1000) -- (64.7000,80.8000) -- (64.8000,81.5000) -- (64.9000,82.1000) -- (64.9000,82.7000) -- (65.0000,83.2000) -- (65.1000,83.7000) -- (65.2000,84.1000) -- (65.3000,84.4000) -- (65.4000,84.7000) -- (65.5000,85.0000) -- (65.5000,85.2000) -- (65.6000,85.3000) -- (65.7000,85.3000) -- (65.8000,85.3000) -- (65.9000,85.3000) -- (66.0000,85.1000) -- (66.1000,84.9000) -- (66.2000,84.7000) -- (66.2000,84.3000) -- (66.3000,84.0000) -- (66.4000,83.5000) -- (66.5000,83.1000) -- (66.6000,82.5000) -- (66.7000,81.9000) -- (66.8000,81.3000) -- (66.8000,80.6000) -- (66.9000,79.9000) -- (67.0000,79.2000) -- (67.1000,78.4000) -- (67.2000,77.6000) -- (67.3000,76.7000) -- (67.4000,75.9000) -- (67.5000,75.0000) -- (67.5000,74.1000) -- (67.6000,73.2000) -- (67.7000,72.3000) -- (67.8000,71.4000) -- (67.9000,70.5000) -- (68.0000,69.6000) -- (68.1000,68.7000) -- (68.2000,67.9000) -- (68.2000,67.0000) -- (68.3000,66.2000) -- (68.4000,65.4000) -- (68.5000,64.7000) -- (68.6000,63.9000) -- (68.7000,63.2000) -- (68.8000,62.5000) -- (68.8000,61.9000) -- (68.9000,61.3000) -- (69.0000,60.8000) -- (69.1000,60.2000) -- (69.2000,59.8000) -- (69.3000,59.4000) -- (69.4000,59.0000) -- (69.5000,58.6000) -- (69.5000,58.3000) -- (69.6000,58.1000) -- (69.7000,57.8000) -- (69.8000,57.7000) -- (69.9000,57.5000) -- (70.0000,57.4000) -- (70.1000,57.3000) -- (70.1000,57.3000) -- (70.2000,57.3000) -- (70.3000,57.3000) -- (70.4000,57.3000) -- (70.5000,57.4000) -- (70.6000,57.5000) -- (70.7000,57.6000) -- (70.8000,57.7000) -- (70.8000,57.8000) -- (70.9000,57.9000) -- (71.0000,58.0000) -- (71.1000,58.2000) -- (71.2000,58.3000) -- (71.3000,58.4000) -- (71.4000,58.5000) -- (71.4000,58.6000) -- (71.5000,58.7000) -- (71.6000,58.7000) -- (71.7000,58.8000) -- (71.8000,58.8000) -- (71.9000,58.8000) -- (72.0000,58.8000) -- (72.1000,58.7000) -- (72.1000,58.6000) -- (72.2000,58.5000) -- (72.3000,58.3000) -- (72.4000,58.2000) -- (72.5000,57.9000) -- (72.6000,57.7000) -- (72.7000,57.4000) -- (72.8000,57.1000) -- (72.8000,56.8000) -- (72.9000,56.4000) -- (73.0000,56.0000) -- (73.1000,55.6000) -- (73.2000,55.2000) -- (73.3000,54.7000) -- (73.4000,54.3000) -- (73.4000,53.8000) -- (73.5000,53.3000) -- (73.6000,52.7000) -- (73.7000,52.2000) -- (73.8000,51.7000) -- (73.9000,51.1000) -- (74.0000,50.6000) -- (74.1000,50.0000) -- (74.1000,49.5000) -- (74.2000,49.0000) -- (74.3000,48.5000) -- (74.4000,48.0000) -- (74.5000,47.5000) -- (74.6000,47.1000) -- (74.7000,46.6000) -- (74.7000,46.2000) -- (74.8000,45.8000) -- (74.9000,45.5000) -- (75.0000,45.2000) -- (75.1000,44.9000) -- (75.2000,44.7000) -- (75.3000,44.5000) -- (75.4000,44.3000) -- (75.4000,44.2000) -- (75.5000,44.1000) -- (75.6000,44.1000) -- (75.7000,44.1000) -- (75.8000,44.2000) -- (75.9000,44.3000) -- (76.0000,44.4000) -- (76.0000,44.6000) -- (76.1000,44.8000) -- (76.2000,45.1000) -- (76.3000,45.4000) -- (76.4000,45.8000) -- (76.5000,46.1000) -- (76.6000,46.5000) -- (76.7000,47.0000) -- (76.7000,47.4000) -- (76.8000,47.9000) -- (76.9000,48.5000) -- (77.0000,49.0000) -- (77.1000,49.5000) -- (77.2000,50.1000) -- (77.3000,50.7000) -- (77.3000,51.2000) -- (77.4000,51.8000) -- (77.5000,52.4000) -- (77.6000,53.0000) -- (77.7000,53.5000) -- (77.8000,54.1000) -- (77.9000,54.6000) -- (78.0000,55.2000) -- (78.0000,55.7000) -- (78.1000,56.2000) -- (78.2000,56.7000) -- (78.3000,57.1000) -- (78.4000,57.5000) -- (78.5000,57.9000) -- (78.6000,58.3000) -- (78.7000,58.6000) -- (78.7000,58.9000) -- (78.8000,59.2000) -- (78.9000,59.4000) -- (79.0000,59.6000) -- (79.1000,59.8000) -- (79.2000,59.9000) -- (79.3000,60.0000) -- (79.3000,60.1000) -- (79.4000,60.1000) -- (79.5000,60.2000) -- (79.6000,60.1000) -- (79.7000,60.1000) -- (79.8000,60.0000) -- (79.9000,60.0000) -- (80.0000,59.9000) -- (80.0000,59.7000) -- (80.1000,59.6000) -- (80.2000,59.5000) -- (80.3000,59.3000) -- (80.4000,59.2000) -- (80.5000,59.0000) -- (80.6000,58.9000) -- (80.6000,58.7000) -- (80.7000,58.6000) -- (80.8000,58.5000) -- (80.9000,58.4000) -- (81.0000,58.3000) -- (81.1000,58.2000) -- (81.2000,58.2000) -- (81.3000,58.2000) -- (81.3000,58.2000) -- (81.4000,58.2000) -- (81.5000,58.3000) -- (81.6000,58.5000) -- (81.7000,58.6000) -- (81.8000,58.8000) -- (81.9000,59.1000) -- (81.9000,59.3000) -- (82.0000,59.7000) -- (82.1000,60.0000) -- (82.2000,60.4000) -- (82.3000,60.9000) -- (82.4000,61.4000) -- (82.5000,61.9000) -- (82.6000,62.5000) -- (82.6000,63.1000) -- (82.7000,63.8000) -- (82.8000,64.5000) -- (82.9000,65.2000) -- (83.0000,66.0000) -- (83.1000,66.8000) -- (83.2000,67.6000) -- (83.3000,68.4000) -- (83.3000,69.3000) -- (83.4000,70.1000) -- (83.5000,71.0000) -- (83.6000,71.9000) -- (83.7000,72.8000) -- (83.8000,73.7000) -- (83.9000,74.5000) -- (83.9000,75.4000) -- (84.0000,76.3000) -- (84.1000,77.1000) -- (84.2000,77.9000) -- (84.3000,78.7000) -- (84.4000,79.5000) -- (84.5000,80.2000) -- (84.6000,80.9000) -- (84.6000,81.6000) -- (84.7000,82.2000) -- (84.8000,82.8000) -- (84.9000,83.3000) -- (85.0000,83.7000) -- (85.1000,84.2000) -- (85.2000,84.5000) -- (85.2000,84.8000) -- (85.3000,85.0000) -- (85.4000,85.2000) -- (85.5000,85.3000) -- (85.6000,85.4000) -- (85.7000,85.4000) -- (85.8000,85.3000) -- (85.9000,85.1000) -- (85.9000,84.9000) -- (86.0000,84.7000) -- (86.1000,84.4000) -- (86.2000,84.0000) -- (86.3000,83.5000) -- (86.4000,83.0000) -- (86.5000,82.5000) -- (86.5000,81.9000) -- (86.6000,81.3000) -- (86.7000,80.6000) -- (86.8000,79.9000) -- (86.9000,79.1000) -- (87.0000,78.3000) -- (87.1000,77.5000) -- (87.2000,76.7000) -- (87.2000,75.8000) -- (87.3000,74.9000) -- (87.4000,74.0000) -- (87.5000,73.1000) -- (87.6000,72.2000) -- (87.7000,71.3000) -- (87.8000,70.4000) -- (87.9000,69.5000) -- (87.9000,68.7000) -- (88.0000,67.8000) -- (88.1000,67.0000) -- (88.2000,66.1000) -- (88.3000,65.3000) -- (88.4000,64.6000) -- (88.5000,63.8000) -- (88.5000,63.1000) -- (88.6000,62.5000) -- (88.7000,61.8000) -- (88.8000,61.2000) -- (88.9000,60.7000) -- (89.0000,60.2000) -- (89.1000,59.7000) -- (89.2000,59.3000) -- (89.2000,58.9000) -- (89.3000,58.6000) -- (89.4000,58.3000) -- (89.5000,58.0000) -- (89.6000,57.8000) -- (89.7000,57.6000) -- (89.8000,57.5000) -- (89.8000,57.4000) -- (89.9000,57.3000) -- (90.0000,57.3000) -- (90.1000,57.3000) -- (90.2000,57.3000) -- (90.3000,57.3000) -- (90.4000,57.4000) -- (90.5000,57.5000) -- (90.5000,57.6000) -- (90.6000,57.7000) -- (90.7000,57.8000) -- (90.8000,57.9000) -- (90.9000,58.1000) -- (91.0000,58.2000) -- (91.1000,58.3000) -- (91.1000,58.4000) -- (91.2000,58.5000) -- (91.3000,58.6000) -- (91.4000,58.7000) -- (91.5000,58.8000) -- (91.6000,58.8000) -- (91.7000,58.8000) -- (91.8000,58.8000) -- (91.8000,58.8000) -- (91.9000,58.7000) -- (92.0000,58.6000) -- (92.1000,58.5000) -- (92.2000,58.3000) -- (92.3000,58.2000) -- (92.4000,57.9000) -- (92.5000,57.7000) -- (92.5000,57.4000) -- (92.6000,57.1000) -- (92.7000,56.8000) -- (92.8000,56.4000) -- (92.9000,56.0000) -- (93.0000,55.6000) -- (93.1000,55.2000) -- (93.1000,54.7000) -- (93.2000,54.2000) -- (93.3000,53.7000) -- (93.4000,53.2000) -- (93.5000,52.7000) -- (93.6000,52.2000) -- (93.7000,51.6000) -- (93.8000,51.1000) -- (93.8000,50.5000) -- (93.9000,50.0000) -- (94.0000,49.5000) -- (94.1000,48.9000) -- (94.2000,48.4000) -- (94.3000,47.9000) -- (94.4000,47.5000) -- (94.4000,47.0000) -- (94.5000,46.6000) -- (94.6000,46.2000) -- (94.7000,45.8000) -- (94.8000,45.4000) -- (94.9000,45.1000) -- (95.0000,44.9000) -- (95.1000,44.6000) -- (95.1000,44.4000) -- (95.2000,44.3000) -- (95.3000,44.2000) -- (95.4000,44.1000) -- (95.5000,44.1000) -- (95.6000,44.1000) -- (95.7000,44.2000) -- (95.7000,44.3000) -- (95.8000,44.4000) -- (95.9000,44.6000) -- (96.0000,44.8000) -- (96.1000,45.1000) -- (96.2000,45.4000) -- (96.3000,45.8000) -- (96.4000,46.2000) -- (96.4000,46.6000) -- (96.5000,47.0000) -- (96.6000,47.5000) -- (96.7000,48.0000) -- (96.8000,48.5000) -- (96.9000,49.0000) -- (97.0000,49.6000) -- (97.1000,50.1000) -- (97.1000,50.7000) -- (97.2000,51.3000) -- (97.3000,51.9000) -- (97.4000,52.4000) -- (97.5000,53.0000) -- (97.6000,53.6000) -- (97.7000,54.2000) -- (97.7000,54.7000) -- (97.8000,55.2000) -- (97.9000,55.8000) -- (98.0000,56.2000) -- (98.1000,56.7000) -- (98.2000,57.2000) -- (98.3000,57.6000) -- (98.4000,58.0000) -- (98.4000,58.3000) -- (98.5000,58.7000) -- (98.6000,59.0000) -- (98.7000,59.2000) -- (98.8000,59.5000) -- (98.9000,59.7000) -- (99.0000,59.8000) -- (99.0000,60.0000) -- (99.1000,60.1000) -- (99.2000,60.1000) -- (99.3000,60.2000) -- (99.4000,60.2000) -- (99.5000,60.2000) -- (99.6000,60.1000) -- (99.7000,60.0000) -- (99.7000,60.0000) -- (99.8000,59.9000) -- (99.9000,59.7000) -- (100.0000,59.6000) -- (100.1000,59.5000) -- (100.2000,59.3000) -- (100.3000,59.2000) -- (100.3000,59.0000) -- (100.4000,58.9000) -- (100.5000,58.7000) -- (100.6000,58.6000) -- (100.7000,58.5000) -- (100.8000,58.4000) -- (100.9000,58.3000) -- (101.0000,58.2000) -- (101.0000,58.2000) -- (101.1000,58.2000) -- (101.2000,58.2000) -- (101.3000,58.2000) -- (101.4000,58.3000) -- (101.5000,58.4000) -- (101.6000,58.6000) -- (101.7000,58.8000) -- (101.7000,59.1000) -- (101.8000,59.3000) -- (101.9000,59.7000) -- (102.0000,60.0000) -- (102.1000,60.5000) -- (102.2000,60.9000) -- (102.3000,61.4000) -- (102.3000,62.0000) -- (102.4000,62.6000) -- (102.5000,63.2000) -- (102.6000,63.8000) -- (102.7000,64.5000) -- (102.8000,65.3000) -- (102.9000,66.0000) -- (103.0000,66.8000) -- (103.0000,67.6000) -- (103.1000,68.5000) -- (103.2000,69.3000) -- (103.3000,70.2000) -- (103.4000,71.1000) -- (103.5000,72.0000) -- (103.6000,72.8000) -- (103.6000,73.7000) -- (103.7000,74.6000) -- (103.8000,75.5000) -- (103.9000,76.4000) -- (104.0000,77.2000) -- (104.1000,78.0000) -- (104.2000,78.8000) -- (104.3000,79.6000) -- (104.3000,80.3000) -- (104.4000,81.0000) -- (104.5000,81.7000) -- (104.6000,82.3000) -- (104.7000,82.8000) -- (104.8000,83.4000) -- (104.9000,83.8000) -- (104.9000,84.2000) -- (105.0000,84.6000) -- (105.1000,84.9000) -- (105.2000,85.1000) -- (105.3000,85.3000) -- (105.4000,85.4000) -- (105.5000,85.4000) -- (105.6000,85.4000) -- (105.6000,85.3000) -- (105.7000,85.2000) -- (105.8000,85.0000) -- (105.9000,84.7000) -- (106.0000,84.4000) -- (106.1000,84.0000) -- (106.2000,83.5000) -- (106.3000,83.0000) -- (106.3000,82.5000) -- (106.4000,81.9000) -- (106.5000,81.2000) -- (106.6000,80.5000) -- (106.7000,79.8000) -- (106.8000,79.1000) -- (106.9000,78.3000) -- (106.9000,77.4000) -- (107.0000,76.6000) -- (107.1000,75.7000) -- (107.2000,74.8000) -- (107.3000,73.9000) -- (107.4000,73.0000) -- (107.5000,72.1000) -- (107.6000,71.2000) -- (107.6000,70.3000) -- (107.7000,69.4000) -- (107.8000,68.6000) -- (107.9000,67.7000) -- (108.0000,66.9000) -- (108.1000,66.0000) -- (108.2000,65.2000) -- (108.2000,64.5000) -- (108.3000,63.7000) -- (108.4000,63.0000) -- (108.5000,62.4000) -- (108.6000,61.8000) -- (108.7000,61.2000) -- (108.8000,60.6000) -- (108.9000,60.1000) -- (108.9000,59.6000) -- (109.0000,59.2000) -- (109.1000,58.9000) -- (109.2000,58.5000) -- (109.3000,58.2000) -- (109.4000,58.0000) -- (109.5000,57.8000) -- (109.5000,57.6000) -- (109.6000,57.5000) -- (109.7000,57.4000) -- (109.8000,57.3000) -- (109.9000,57.3000) -- (110.0000,57.3000) -- (110.1000,57.3000) -- (110.2000,57.3000) -- (110.2000,57.4000) -- (110.3000,57.5000) -- (110.4000,57.6000) -- (110.5000,57.7000) -- (110.6000,57.8000) -- (110.7000,57.9000) -- (110.8000,58.1000) -- (110.9000,58.2000) -- (110.9000,58.3000) -- (111.0000,58.4000) -- (111.1000,58.5000) -- (111.2000,58.6000) -- (111.3000,58.7000) -- (111.4000,58.8000) -- (111.5000,58.8000) -- (111.5000,58.8000) -- (111.6000,58.8000) -- (111.7000,58.8000) -- (111.8000,58.7000) -- (111.9000,58.6000) -- (112.0000,58.5000) -- (112.1000,58.3000) -- (112.2000,58.2000) -- (112.2000,57.9000) -- (112.3000,57.7000) -- (112.4000,57.4000) -- (112.5000,57.1000) -- (112.6000,56.8000) -- (112.7000,56.4000) -- (112.8000,56.0000) -- (112.8000,55.6000) -- (112.9000,55.1000) -- (113.0000,54.7000) -- (113.1000,54.2000) -- (113.2000,53.7000) -- (113.3000,53.2000) -- (113.4000,52.7000) -- (113.5000,52.1000) -- (113.5000,51.6000) -- (113.6000,51.0000) -- (113.7000,50.5000) -- (113.8000,49.9000) -- (113.9000,49.4000) -- (114.0000,48.9000) -- (114.1000,48.4000) -- (114.1000,47.9000) -- (114.2000,47.4000) -- (114.3000,46.9000) -- (114.4000,46.5000) -- (114.5000,46.1000) -- (114.6000,45.7000) -- (114.7000,45.4000) -- (114.8000,45.1000) -- (114.8000,44.8000) -- (114.9000,44.6000) -- (115.0000,44.4000) -- (115.1000,44.3000) -- (115.2000,44.1000) -- (115.3000,44.1000) -- (115.4000,44.1000) -- (115.5000,44.1000) -- (115.5000,44.1000) -- (115.6000,44.3000) -- (115.7000,44.4000) -- (115.8000,44.6000) -- (115.9000,44.8000) -- (116.0000,45.1000) -- (116.1000,45.4000) -- (116.1000,45.8000) -- (116.2000,46.2000) -- (116.3000,46.6000) -- (116.4000,47.0000) -- (116.5000,47.5000) -- (116.6000,48.0000) -- (116.7000,48.5000) -- (116.8000,49.1000) -- (116.8000,49.6000) -- (116.9000,50.2000) -- (117.0000,50.8000) -- (117.1000,51.3000) -- (117.2000,51.9000) -- (117.3000,52.5000) -- (117.4000,53.1000) -- (117.4000,53.6000) -- (117.5000,54.2000) -- (117.6000,54.8000) -- (117.7000,55.3000) -- (117.8000,55.8000) -- (117.9000,56.3000) -- (118.0000,56.8000) -- (118.1000,57.2000) -- (118.1000,57.6000) -- (118.2000,58.0000) -- (118.3000,58.4000) -- (118.4000,58.7000) -- (118.5000,59.0000) -- (118.6000,59.3000) -- (118.7000,59.5000) -- (118.7000,59.7000) -- (118.8000,59.9000) -- (118.9000,60.0000) -- (119.0000,60.1000) -- (119.1000,60.1000) -- (119.2000,60.2000) -- (119.3000,60.2000) -- (119.4000,60.2000) -- (119.4000,60.1000) -- (119.5000,60.1000) -- (119.6000,60.0000) -- (119.7000,59.9000) -- (119.8000,59.7000) -- (119.9000,59.6000) -- (120.0000,59.5000) -- (120.0000,59.3000) -- (120.1000,59.1000) -- (120.2000,59.0000) -- (120.3000,58.8000) -- (120.4000,58.7000) -- (120.5000,58.6000) -- (120.6000,58.4000) -- (120.7000,58.3000) -- (120.7000,58.2000) -- (120.8000,58.2000) -- (120.9000,58.2000) -- (121.0000,58.1000) -- (121.1000,58.2000) -- (121.2000,58.2000) -- (121.3000,58.3000) -- (121.4000,58.4000) -- (121.4000,58.6000) -- (121.5000,58.8000) -- (121.6000,59.1000) -- (121.7000,59.3000) -- (121.8000,59.7000) -- (121.9000,60.1000) -- (122.0000,60.5000) -- (122.0000,60.9000) -- (122.1000,61.5000) -- (122.2000,62.0000) -- (122.3000,62.6000) -- (122.4000,63.2000) -- (122.5000,63.9000) -- (122.6000,64.6000) -- (122.7000,65.3000) -- (122.7000,66.1000) -- (122.8000,66.9000) -- (122.9000,67.7000) -- (123.0000,68.5000) -- (123.1000,69.4000) -- (123.2000,70.3000) -- (123.3000,71.1000) -- (123.3000,72.0000) -- (123.4000,72.9000) -- (123.5000,73.8000) -- (123.6000,74.7000) -- (123.7000,75.6000) -- (123.8000,76.4000) -- (123.9000,77.3000) -- (124.0000,78.1000) -- (124.0000,78.9000) -- (124.1000,79.7000) -- (124.2000,80.4000) -- (124.3000,81.1000) -- (124.4000,81.7000) -- (124.5000,82.4000) -- (124.6000,82.9000) -- (124.6000,83.4000) -- (124.7000,83.9000) -- (124.8000,84.3000) -- (124.9000,84.6000) -- (125.0000,84.9000) -- (125.1000,85.2000) -- (125.2000,85.3000) -- (125.3000,85.4000) -- (125.3000,85.5000) -- (125.4000,85.4000) -- (125.5000,85.3000) -- (125.6000,85.2000) -- (125.7000,85.0000) -- (125.8000,84.7000) -- (125.9000,84.4000) -- (125.9000,84.0000) -- (126.0000,83.5000) -- (126.1000,83.0000) -- (126.2000,82.4000) -- (126.3000,81.8000) -- (126.4000,81.2000) -- (126.5000,80.5000) -- (126.6000,79.8000) -- (126.6000,79.0000) -- (126.7000,78.2000) -- (126.8000,77.4000) -- (126.9000,76.5000) -- (127.0000,75.7000) -- (127.1000,74.8000) -- (127.2000,73.9000) -- (127.3000,73.0000) -- (127.3000,72.1000) -- (127.4000,71.2000) -- (127.5000,70.3000) -- (127.6000,69.4000) -- (127.7000,68.5000) -- (127.8000,67.6000) -- (127.9000,66.8000) -- (127.9000,66.0000) -- (128.0000,65.2000) -- (128.1000,64.4000) -- (128.2000,63.7000) -- (128.3000,63.0000) -- (128.4000,62.3000) -- (128.5000,61.7000) -- (128.6000,61.1000) -- (128.6000,60.5000) -- (128.7000,60.0000) -- (128.8000,59.6000) -- (128.9000,59.2000) -- (129.0000,58.8000) -- (129.1000,58.5000) -- (129.2000,58.2000) -- (129.2000,57.9000) -- (129.3000,57.7000) -- (129.4000,57.6000) -- (129.5000,57.4000) -- (129.6000,57.3000) -- (129.7000,57.3000) -- (129.8000,57.2000) -- (129.9000,57.3000) -- (129.9000,57.3000) -- (130.0000,57.3000) -- (130.1000,57.4000) -- (130.2000,57.5000) -- (130.3000,57.6000) -- (130.4000,57.7000) -- (130.5000,57.8000) -- (130.5000,58.0000) -- (130.6000,58.1000) -- (130.7000,58.2000) -- (130.8000,58.3000) -- (130.9000,58.5000) -- (131.0000,58.6000) -- (131.1000,58.7000) -- (131.2000,58.7000) -- (131.2000,58.8000) -- (131.3000,58.8000) -- (131.4000,58.8000) -- (131.5000,58.8000) -- (131.6000,58.8000) -- (131.7000,58.7000) -- (131.8000,58.6000) -- (131.8000,58.5000) -- (131.9000,58.3000) -- (132.0000,58.2000) -- (132.1000,57.9000) -- (132.2000,57.7000) -- (132.3000,57.4000) -- (132.4000,57.1000) -- (132.5000,56.8000) -- (132.5000,56.4000) -- (132.6000,56.0000) -- (132.7000,55.6000) -- (132.8000,55.1000) -- (132.9000,54.6000) -- (133.0000,54.2000) -- (133.1000,53.7000) -- (133.2000,53.1000) -- (133.2000,52.6000) -- (133.3000,52.1000) -- (133.4000,51.5000) -- (133.5000,51.0000) -- (133.6000,50.4000) -- (133.7000,49.9000) -- (133.8000,49.4000) -- (133.8000,48.8000) -- (133.9000,48.3000) -- (134.0000,47.8000) -- (134.1000,47.3000) -- (134.2000,46.9000) -- (134.3000,46.5000) -- (134.4000,46.1000) -- (134.5000,45.7000) -- (134.5000,45.3000) -- (134.6000,45.0000) -- (134.7000,44.8000) -- (134.8000,44.6000) -- (134.9000,44.4000) -- (135.0000,44.2000) -- (135.1000,44.1000) -- (135.1000,44.1000) -- (135.2000,44.0000) -- (135.3000,44.1000) -- (135.4000,44.1000) -- (135.5000,44.2000) -- (135.6000,44.4000) -- (135.7000,44.6000) -- (135.8000,44.8000) -- (135.8000,45.1000) -- (135.9000,45.4000) -- (136.0000,45.8000) -- (136.1000,46.2000) -- (136.2000,46.6000) -- (136.3000,47.1000) -- (136.4000,47.5000) -- (136.5000,48.0000) -- (136.5000,48.6000) -- (136.6000,49.1000) -- (136.7000,49.7000) -- (136.8000,50.2000) -- (136.9000,50.8000) -- (137.0000,51.4000) -- (137.1000,52.0000) -- (137.1000,52.6000) -- (137.2000,53.1000) -- (137.3000,53.7000) -- (137.4000,54.3000) -- (137.5000,54.8000) -- (137.6000,55.4000) -- (137.7000,55.9000) -- (137.8000,56.4000) -- (137.8000,56.8000) -- (137.9000,57.3000) -- (138.0000,57.7000) -- (138.1000,58.1000) -- (138.2000,58.4000) -- (138.3000,58.8000) -- (138.4000,59.1000) -- (138.4000,59.3000) -- (138.5000,59.5000) -- (138.6000,59.7000) -- (138.7000,59.9000) -- (138.8000,60.0000) -- (138.9000,60.1000) -- (139.0000,60.2000) -- (139.1000,60.2000) -- (139.1000,60.2000) -- (139.2000,60.2000) -- (139.3000,60.1000) -- (139.4000,60.1000) -- (139.5000,60.0000) -- (139.6000,59.9000) -- (139.7000,59.7000) -- (139.7000,59.6000) -- (139.8000,59.4000) -- (139.9000,59.3000) -- (140.0000,59.1000) -- (140.1000,59.0000) -- (140.2000,58.8000) -- (140.3000,58.7000) -- (140.4000,58.5000) -- (140.4000,58.4000) -- (140.5000,58.3000) -- (140.6000,58.2000) -- (140.7000,58.2000) -- (140.8000,58.1000) -- (140.9000,58.1000) -- (141.0000,58.1000) -- (141.1000,58.2000) -- (141.1000,58.3000) -- (141.2000,58.4000) -- (141.3000,58.6000) -- (141.4000,58.8000) -- (141.5000,59.1000) -- (141.6000,59.4000) -- (141.7000,59.7000) -- (141.7000,60.1000) -- (141.8000,60.5000) -- (141.9000,61.0000) -- (142.0000,61.5000) -- (142.1000,62.0000) -- (142.2000,62.6000) -- (142.3000,63.3000) -- (142.4000,63.9000) -- (142.4000,64.6000) -- (142.5000,65.4000) -- (142.6000,66.1000) -- (142.7000,67.0000);



  \end{scope}
  \begin{scope}[scale=1.006,draw=black,line join=bevel,line cap=rect,line width=0.800pt]
  \end{scope}
  \begin{scope}[scale=1.006,draw=black,line join=bevel,line cap=rect,line width=0.800pt]
  \end{scope}
  \begin{scope}[scale=1.006,draw=black,line join=round,line cap=round,line width=0.480pt]
    \path[draw] (56.5000,13.5000) -- (56.5000,95.5000) -- (142.5000,95.5000) -- (142.5000,13.5000) -- (56.5000,13.5000);



  \end{scope}
  \begin{scope}[scale=1.006,draw=ca0a0a4,dash pattern=on 0.40pt off 0.80pt,line join=round,line cap=round,line width=0.400pt]
    \path[draw] (165.5000,88.5000) -- (251.5000,88.5000);



  \end{scope}
  \begin{scope}[scale=1.006,draw=black,line join=round,line cap=round,line width=0.480pt]
    \path[draw] (165.5000,88.5000) -- (168.5000,88.5000);



    \path[draw] (251.5000,88.5000) -- (248.5000,88.5000);



  \end{scope}
  \begin{scope}[scale=1.006,draw=black,line join=bevel,line cap=rect,line width=0.800pt]
  \end{scope}
  \begin{scope}[cm={{1.00588,0.0,0.0,1.00588,(148.871,93.5471)}},draw=black,line join=bevel,line cap=rect,line width=0.800pt]
  \end{scope}
  \begin{scope}[cm={{1.00588,0.0,0.0,1.00588,(148.871,93.5471)}},draw=black,line join=bevel,line cap=rect,line width=0.800pt]
  \end{scope}
  \begin{scope}[cm={{1.00588,0.0,0.0,1.00588,(148.871,93.5471)}},draw=black,line join=bevel,line cap=rect,line width=0.800pt]
  \end{scope}
  \begin{scope}[cm={{1.00588,0.0,0.0,1.00588,(148.871,93.5471)}},draw=black,line join=bevel,line cap=rect,line width=0.800pt]
  \end{scope}
  \begin{scope}[cm={{1.00588,0.0,0.0,1.00588,(148.871,93.5471)}},draw=black,line join=bevel,line cap=rect,line width=0.800pt]
  \end{scope}
  \begin{scope}[cm={{1.00588,0.0,0.0,1.00588,(148.871,93.5471)}},draw=black,line join=bevel,line cap=rect,line width=0.800pt]
    \path[fill=black] (0.0000,0.0000) node[above right] (text380) {27};



  \end{scope}
  \begin{scope}[cm={{1.00588,0.0,0.0,1.00588,(148.871,93.5471)}},draw=black,line join=bevel,line cap=rect,line width=0.800pt]
  \end{scope}
  \begin{scope}[scale=1.006,draw=black,line join=bevel,line cap=rect,line width=0.800pt]
  \end{scope}
  \begin{scope}[scale=1.006,draw=ca0a0a4,dash pattern=on 0.40pt off 0.80pt,line join=round,line cap=round,line width=0.400pt]
    \path[draw] (165.5000,63.5000) -- (251.5000,63.5000);



  \end{scope}
  \begin{scope}[scale=1.006,draw=black,line join=round,line cap=round,line width=0.480pt]
    \path[draw] (165.5000,63.5000) -- (168.5000,63.5000);



    \path[draw] (251.5000,63.5000) -- (248.5000,63.5000);



  \end{scope}
  \begin{scope}[scale=1.006,draw=black,line join=bevel,line cap=rect,line width=0.800pt]
  \end{scope}
  \begin{scope}[cm={{1.00588,0.0,0.0,1.00588,(149.876,68.4)}},draw=black,line join=bevel,line cap=rect,line width=0.800pt]
  \end{scope}
  \begin{scope}[cm={{1.00588,0.0,0.0,1.00588,(149.876,68.4)}},draw=black,line join=bevel,line cap=rect,line width=0.800pt]
  \end{scope}
  \begin{scope}[cm={{1.00588,0.0,0.0,1.00588,(149.876,68.4)}},draw=black,line join=bevel,line cap=rect,line width=0.800pt]
  \end{scope}
  \begin{scope}[cm={{1.00588,0.0,0.0,1.00588,(149.876,68.4)}},draw=black,line join=bevel,line cap=rect,line width=0.800pt]
  \end{scope}
  \begin{scope}[cm={{1.00588,0.0,0.0,1.00588,(149.876,68.4)}},draw=black,line join=bevel,line cap=rect,line width=0.800pt]
  \end{scope}
  \begin{scope}[cm={{1.00588,0.0,0.0,1.00588,(149.876,68.4)}},draw=black,line join=bevel,line cap=rect,line width=0.800pt]
    \path[fill=black] (0.0000,0.0000) node[above right] (text410) {31};



  \end{scope}
  \begin{scope}[cm={{1.00588,0.0,0.0,1.00588,(149.876,68.4)}},draw=black,line join=bevel,line cap=rect,line width=0.800pt]
  \end{scope}
  \begin{scope}[scale=1.006,draw=black,line join=bevel,line cap=rect,line width=0.800pt]
  \end{scope}
  \begin{scope}[scale=1.006,draw=ca0a0a4,dash pattern=on 0.40pt off 0.80pt,line join=round,line cap=round,line width=0.400pt]
    \path[draw] (165.5000,38.5000) -- (251.5000,38.5000);



  \end{scope}
  \begin{scope}[scale=1.006,draw=black,line join=round,line cap=round,line width=0.480pt]
    \path[draw] (165.5000,38.5000) -- (168.5000,38.5000);



    \path[draw] (251.5000,38.5000) -- (248.5000,38.5000);



  \end{scope}
  \begin{scope}[scale=1.006,draw=black,line join=bevel,line cap=rect,line width=0.800pt]
  \end{scope}
  \begin{scope}[cm={{1.00588,0.0,0.0,1.00588,(149.876,43.2529)}},draw=black,line join=bevel,line cap=rect,line width=0.800pt]
  \end{scope}
  \begin{scope}[cm={{1.00588,0.0,0.0,1.00588,(149.876,43.2529)}},draw=black,line join=bevel,line cap=rect,line width=0.800pt]
  \end{scope}
  \begin{scope}[cm={{1.00588,0.0,0.0,1.00588,(149.876,43.2529)}},draw=black,line join=bevel,line cap=rect,line width=0.800pt]
  \end{scope}
  \begin{scope}[cm={{1.00588,0.0,0.0,1.00588,(149.876,43.2529)}},draw=black,line join=bevel,line cap=rect,line width=0.800pt]
  \end{scope}
  \begin{scope}[cm={{1.00588,0.0,0.0,1.00588,(149.876,43.2529)}},draw=black,line join=bevel,line cap=rect,line width=0.800pt]
  \end{scope}
  \begin{scope}[cm={{1.00588,0.0,0.0,1.00588,(149.876,43.2529)}},draw=black,line join=bevel,line cap=rect,line width=0.800pt]
    \path[fill=black] (0.0000,0.0000) node[above right] (text440) {35};



  \end{scope}
  \begin{scope}[cm={{1.00588,0.0,0.0,1.00588,(149.876,43.2529)}},draw=black,line join=bevel,line cap=rect,line width=0.800pt]
  \end{scope}
  \begin{scope}[scale=1.006,draw=black,line join=bevel,line cap=rect,line width=0.800pt]
  \end{scope}
  \begin{scope}[scale=1.006,draw=ca0a0a4,dash pattern=on 0.40pt off 0.80pt,line join=round,line cap=round,line width=0.400pt]
    \path[draw] (165.5000,95.5000) -- (165.5000,13.5000);



  \end{scope}
  \begin{scope}[scale=1.006,draw=black,line join=round,line cap=round,line width=0.480pt]
    \path[draw] (165.5000,95.5000) -- (165.5000,92.5000);



    \path[draw] (165.5000,13.5000) -- (165.5000,16.5000);



  \end{scope}
  \begin{scope}[scale=1.006,draw=black,line join=bevel,line cap=rect,line width=0.800pt]
  \end{scope}
  \begin{scope}[cm={{1.00588,0.0,0.0,1.00588,(162.953,110.647)}},draw=black,line join=bevel,line cap=rect,line width=0.800pt]
  \end{scope}
  \begin{scope}[cm={{1.00588,0.0,0.0,1.00588,(162.953,110.647)}},draw=black,line join=bevel,line cap=rect,line width=0.800pt]
  \end{scope}
  \begin{scope}[cm={{1.00588,0.0,0.0,1.00588,(162.953,110.647)}},draw=black,line join=bevel,line cap=rect,line width=0.800pt]
  \end{scope}
  \begin{scope}[cm={{1.00588,0.0,0.0,1.00588,(162.953,110.647)}},draw=black,line join=bevel,line cap=rect,line width=0.800pt]
  \end{scope}
  \begin{scope}[cm={{1.00588,0.0,0.0,1.00588,(162.953,110.647)}},draw=black,line join=bevel,line cap=rect,line width=0.800pt]
  \end{scope}
  \begin{scope}[cm={{1.00588,0.0,0.0,1.00588,(162.953,110.647)}},draw=black,line join=bevel,line cap=rect,line width=0.800pt]
    \path[fill=black] (0.0000,0.0000) node[above right] (text470) {0};



  \end{scope}
  \begin{scope}[cm={{1.00588,0.0,0.0,1.00588,(162.953,110.647)}},draw=black,line join=bevel,line cap=rect,line width=0.800pt]
  \end{scope}
  \begin{scope}[scale=1.006,draw=black,line join=bevel,line cap=rect,line width=0.800pt]
  \end{scope}
  \begin{scope}[scale=1.006,draw=ca0a0a4,dash pattern=on 0.40pt off 0.80pt,line join=round,line cap=round,line width=0.400pt]
    \path[draw] (191.5000,95.5000) -- (191.5000,13.5000);



  \end{scope}
  \begin{scope}[scale=1.006,draw=black,line join=round,line cap=round,line width=0.480pt]
    \path[draw] (191.5000,95.5000) -- (191.5000,92.5000);



    \path[draw] (191.5000,13.5000) -- (191.5000,16.5000);



  \end{scope}
  \begin{scope}[scale=1.006,draw=black,line join=bevel,line cap=rect,line width=0.800pt]
  \end{scope}
  \begin{scope}[cm={{1.00588,0.0,0.0,1.00588,(189.106,110.647)}},draw=black,line join=bevel,line cap=rect,line width=0.800pt]
  \end{scope}
  \begin{scope}[cm={{1.00588,0.0,0.0,1.00588,(189.106,110.647)}},draw=black,line join=bevel,line cap=rect,line width=0.800pt]
  \end{scope}
  \begin{scope}[cm={{1.00588,0.0,0.0,1.00588,(189.106,110.647)}},draw=black,line join=bevel,line cap=rect,line width=0.800pt]
  \end{scope}
  \begin{scope}[cm={{1.00588,0.0,0.0,1.00588,(189.106,110.647)}},draw=black,line join=bevel,line cap=rect,line width=0.800pt]
  \end{scope}
  \begin{scope}[cm={{1.00588,0.0,0.0,1.00588,(189.106,110.647)}},draw=black,line join=bevel,line cap=rect,line width=0.800pt]
  \end{scope}
  \begin{scope}[cm={{1.00588,0.0,0.0,1.00588,(189.106,110.647)}},draw=black,line join=bevel,line cap=rect,line width=0.800pt]
    \path[fill=black] (0.0000,0.0000) node[above right] (text500) {1};



  \end{scope}
  \begin{scope}[cm={{1.00588,0.0,0.0,1.00588,(189.106,110.647)}},draw=black,line join=bevel,line cap=rect,line width=0.800pt]
  \end{scope}
  \begin{scope}[scale=1.006,draw=black,line join=bevel,line cap=rect,line width=0.800pt]
  \end{scope}
  \begin{scope}[scale=1.006,draw=ca0a0a4,dash pattern=on 0.40pt off 0.80pt,line join=round,line cap=round,line width=0.400pt]
    \path[draw] (217.5000,95.5000) -- (217.5000,13.5000);



  \end{scope}
  \begin{scope}[scale=1.006,draw=black,line join=round,line cap=round,line width=0.480pt]
    \path[draw] (217.5000,95.5000) -- (217.5000,92.5000);



    \path[draw] (217.5000,13.5000) -- (217.5000,16.5000);



  \end{scope}
  \begin{scope}[scale=1.006,draw=black,line join=bevel,line cap=rect,line width=0.800pt]
  \end{scope}
  \begin{scope}[cm={{1.00588,0.0,0.0,1.00588,(215.259,110.647)}},draw=black,line join=bevel,line cap=rect,line width=0.800pt]
  \end{scope}
  \begin{scope}[cm={{1.00588,0.0,0.0,1.00588,(215.259,110.647)}},draw=black,line join=bevel,line cap=rect,line width=0.800pt]
  \end{scope}
  \begin{scope}[cm={{1.00588,0.0,0.0,1.00588,(215.259,110.647)}},draw=black,line join=bevel,line cap=rect,line width=0.800pt]
  \end{scope}
  \begin{scope}[cm={{1.00588,0.0,0.0,1.00588,(215.259,110.647)}},draw=black,line join=bevel,line cap=rect,line width=0.800pt]
  \end{scope}
  \begin{scope}[cm={{1.00588,0.0,0.0,1.00588,(215.259,110.647)}},draw=black,line join=bevel,line cap=rect,line width=0.800pt]
  \end{scope}
  \begin{scope}[cm={{1.00588,0.0,0.0,1.00588,(215.259,110.647)}},draw=black,line join=bevel,line cap=rect,line width=0.800pt]
    \path[fill=black] (0.0000,0.0000) node[above right] (text530) {2};



  \end{scope}
  \begin{scope}[cm={{1.00588,0.0,0.0,1.00588,(215.259,110.647)}},draw=black,line join=bevel,line cap=rect,line width=0.800pt]
  \end{scope}
  \begin{scope}[scale=1.006,draw=black,line join=bevel,line cap=rect,line width=0.800pt]
  \end{scope}
  \begin{scope}[scale=1.006,draw=ca0a0a4,dash pattern=on 0.40pt off 0.80pt,line join=round,line cap=round,line width=0.400pt]
    \path[draw] (243.5000,95.5000) -- (243.5000,13.5000);



  \end{scope}
  \begin{scope}[scale=1.006,draw=black,line join=round,line cap=round,line width=0.480pt]
    \path[draw] (243.5000,95.5000) -- (243.5000,92.5000);



    \path[draw] (243.5000,13.5000) -- (243.5000,16.5000);



  \end{scope}
  \begin{scope}[scale=1.006,draw=black,line join=bevel,line cap=rect,line width=0.800pt]
  \end{scope}
  \begin{scope}[cm={{1.00588,0.0,0.0,1.00588,(241.915,110.647)}},draw=black,line join=bevel,line cap=rect,line width=0.800pt]
  \end{scope}
  \begin{scope}[cm={{1.00588,0.0,0.0,1.00588,(241.915,110.647)}},draw=black,line join=bevel,line cap=rect,line width=0.800pt]
  \end{scope}
  \begin{scope}[cm={{1.00588,0.0,0.0,1.00588,(241.915,110.647)}},draw=black,line join=bevel,line cap=rect,line width=0.800pt]
  \end{scope}
  \begin{scope}[cm={{1.00588,0.0,0.0,1.00588,(241.915,110.647)}},draw=black,line join=bevel,line cap=rect,line width=0.800pt]
  \end{scope}
  \begin{scope}[cm={{1.00588,0.0,0.0,1.00588,(241.915,110.647)}},draw=black,line join=bevel,line cap=rect,line width=0.800pt]
  \end{scope}
  \begin{scope}[cm={{1.00588,0.0,0.0,1.00588,(241.915,110.647)}},draw=black,line join=bevel,line cap=rect,line width=0.800pt]
    \path[fill=black] (0.0000,0.0000) node[above right] (text560) {3};



  \end{scope}
  \begin{scope}[cm={{1.00588,0.0,0.0,1.00588,(241.915,110.647)}},draw=black,line join=bevel,line cap=rect,line width=0.800pt]
  \end{scope}
  \begin{scope}[scale=1.006,draw=black,line join=bevel,line cap=rect,line width=0.800pt]
  \end{scope}
  \begin{scope}[scale=1.006,draw=black,line join=round,line cap=round,line width=0.480pt]
    \path[draw] (165.5000,13.5000) -- (165.5000,95.5000) -- (251.5000,95.5000) -- (251.5000,13.5000) -- (165.5000,13.5000);



  \end{scope}
  \begin{scope}[scale=1.006,draw=black,line join=bevel,line cap=rect,line width=0.800pt]
  \end{scope}
  \begin{scope}[scale=1.006,draw=black,line join=bevel,line cap=rect,line width=0.800pt]
  \end{scope}
  \begin{scope}[scale=1.006,fill=cffffff]
    \path[fill,rounded corners=0.0000cm] (229.0000,17.0000) rectangle (245.0000,33.0000);



  \end{scope}
  \begin{scope}[scale=1.006,draw=black,line join=bevel,line cap=rect,line width=0.800pt]
  \end{scope}
  \begin{scope}[scale=1.006,draw=black,line join=bevel,line cap=rect,line width=0.800pt]
  \end{scope}
  \begin{scope}[scale=1.006,draw=black,line join=round,line cap=round,line width=0.800pt]
    \path[draw] (229.5000,33.5000) -- (229.5000,17.5000) -- (245.5000,17.5000) -- (245.5000,33.5000) -- (229.5000,33.5000);



  \end{scope}
  \begin{scope}[scale=1.006,draw=black,line join=bevel,line cap=rect,line width=0.800pt]
  \end{scope}
  \begin{scope}[cm={{1.00588,0.0,0.0,1.00588,(235.376,29.1706)}},draw=black,line join=bevel,line cap=rect,line width=0.800pt]
  \end{scope}
  \begin{scope}[cm={{1.00588,0.0,0.0,1.00588,(235.376,29.1706)}},draw=black,line join=bevel,line cap=rect,line width=0.800pt]
  \end{scope}
  \begin{scope}[cm={{1.00588,0.0,0.0,1.00588,(235.376,29.1706)}},draw=black,line join=bevel,line cap=rect,line width=0.800pt]
  \end{scope}
  \begin{scope}[cm={{1.00588,0.0,0.0,1.00588,(235.376,29.1706)}},draw=black,line join=bevel,line cap=rect,line width=0.800pt]
  \end{scope}
  \begin{scope}[cm={{1.00588,0.0,0.0,1.00588,(235.376,29.1706)}},draw=black,line join=bevel,line cap=rect,line width=0.800pt]
  \end{scope}
  \begin{scope}[cm={{1.00588,0.0,0.0,1.00588,(236.2031,28.6192)}},draw=black,line join=bevel,line cap=rect,line width=0.800pt]
    \path[fill=black] (0.0000,0.0000) node[above right] (text600) {\label{fig:ener:static-II}II};



  \end{scope}
  \begin{scope}[cm={{1.00588,0.0,0.0,1.00588,(235.376,29.1706)}},draw=black,line join=bevel,line cap=rect,line width=0.800pt]
  \end{scope}
  \begin{scope}[scale=1.006,draw=black,line join=bevel,line cap=rect,line width=0.800pt]
  \end{scope}
  \begin{scope}[scale=1.006,draw=black,line join=bevel,line cap=rect,line width=0.800pt]
  \end{scope}
  \begin{scope}[scale=1.006,draw=black,line join=bevel,line cap=rect,line width=0.800pt]
  \end{scope}
  \begin{scope}[scale=1.006,draw=black,line join=round,line cap=round,line width=0.480pt]
    \path[draw] (165.3000,36.9000) -- (165.3000,36.9000) -- (165.4000,41.9000) -- (165.5000,44.7000) -- (165.7000,46.2000) -- (165.8000,46.7000) -- (166.0000,46.7000) -- (166.1000,46.7000) -- (166.2000,47.2000) -- (166.4000,47.4000) -- (166.5000,47.2000) -- (166.6000,46.7000) -- (166.8000,46.1000) -- (166.9000,45.5000) -- (167.1000,44.9000) -- (167.2000,44.5000) -- (167.3000,44.2000) -- (167.5000,44.2000) -- (167.6000,44.2000) -- (167.8000,44.2000) -- (167.9000,44.3000) -- (168.0000,44.4000) -- (168.2000,44.4000) -- (168.3000,44.5000) -- (168.4000,44.5000) -- (168.6000,44.5000) -- (168.7000,44.5000) -- (168.9000,44.5000) -- (169.0000,44.5000) -- (169.1000,44.5000) -- (169.3000,44.5000) -- (169.4000,44.5000) -- (169.5000,44.5000) -- (169.7000,44.5000) -- (169.8000,44.5000) -- (170.0000,44.5000) -- (170.1000,44.5000) -- (170.2000,44.5000) -- (170.4000,44.5000) -- (170.5000,44.5000) -- (170.7000,44.5000) -- (170.8000,44.5000) -- (170.9000,44.5000) -- (171.1000,44.5000) -- (171.2000,44.5000) -- (171.3000,44.5000) -- (171.5000,44.5000) -- (171.6000,44.5000) -- (171.8000,44.6000) -- (171.9000,44.7000) -- (172.0000,44.9000) -- (172.2000,45.2000) -- (172.3000,45.5000) -- (172.5000,45.8000) -- (172.6000,46.3000) -- (172.7000,46.7000) -- (172.9000,47.3000) -- (173.0000,47.8000) -- (173.1000,48.5000) -- (173.3000,49.1000) -- (173.4000,49.9000) -- (173.6000,50.7000) -- (173.7000,51.5000) -- (173.8000,52.4000) -- (174.0000,53.4000) -- (174.1000,54.4000) -- (174.2000,55.4000) -- (174.4000,56.5000) -- (174.5000,57.7000) -- (174.7000,58.9000) -- (174.8000,60.2000) -- (174.9000,61.5000) -- (175.1000,62.8000) -- (175.2000,64.2000) -- (175.4000,65.6000) -- (175.5000,67.1000) -- (175.6000,68.5000) -- (175.8000,70.0000) -- (175.9000,71.5000) -- (176.0000,73.0000) -- (176.2000,74.5000) -- (176.3000,76.0000) -- (176.5000,77.4000) -- (176.6000,78.8000) -- (176.7000,80.2000) -- (176.9000,81.6000) -- (177.0000,83.1000) -- (177.1000,84.5000) -- (177.3000,85.7000) -- (177.4000,86.5000) -- (177.6000,87.0000) -- (177.7000,87.2000) -- (177.8000,87.2000) -- (178.0000,87.1000) -- (178.1000,86.9000) -- (178.3000,86.7000) -- (178.4000,86.6000) -- (178.5000,86.5000) -- (178.7000,86.4000) -- (178.8000,86.3000) -- (178.9000,86.3000) -- (179.1000,86.3000) -- (179.2000,86.3000) -- (179.4000,86.3000) -- (179.5000,86.3000) -- (179.6000,86.3000) -- (179.8000,86.3000) -- (179.9000,86.3000) -- (180.0000,86.3000) -- (180.2000,86.3000) -- (180.3000,86.3000) -- (180.5000,86.3000) -- (180.6000,86.3000) -- (180.7000,86.3000) -- (180.9000,86.3000) -- (181.0000,86.3000) -- (181.2000,86.3000) -- (181.3000,86.3000) -- (181.4000,86.2000) -- (181.6000,86.0000) -- (181.7000,85.8000) -- (181.8000,85.9000) -- (182.0000,86.1000) -- (182.1000,86.2000) -- (182.3000,86.0000) -- (182.4000,85.7000) -- (182.5000,85.0000) -- (182.7000,84.0000) -- (182.8000,82.8000) -- (183.0000,81.4000) -- (183.1000,79.9000) -- (183.2000,78.2000) -- (183.4000,76.4000) -- (183.5000,74.6000) -- (183.6000,72.8000) -- (183.8000,70.9000) -- (183.9000,69.1000) -- (184.1000,67.3000) -- (184.2000,65.5000) -- (184.3000,63.8000) -- (184.5000,62.1000) -- (184.6000,60.5000) -- (184.7000,59.0000) -- (184.9000,57.6000) -- (185.0000,56.2000) -- (185.2000,54.9000) -- (185.3000,53.7000) -- (185.4000,52.5000) -- (185.6000,51.5000) -- (185.7000,50.5000) -- (185.9000,49.6000) -- (186.0000,48.7000) -- (186.1000,48.0000) -- (186.3000,47.3000) -- (186.4000,46.7000) -- (186.5000,46.1000) -- (186.7000,45.6000) -- (186.8000,45.2000) -- (187.0000,44.9000) -- (187.1000,44.6000) -- (187.2000,44.3000) -- (187.4000,44.1000) -- (187.5000,44.1000) -- (187.6000,44.1000) -- (187.8000,44.2000) -- (187.9000,44.3000) -- (188.1000,44.4000) -- (188.2000,44.4000) -- (188.3000,44.5000) -- (188.5000,44.5000) -- (188.6000,44.5000) -- (188.8000,44.5000) -- (188.9000,44.5000) -- (189.0000,44.5000) -- (189.2000,44.5000) -- (189.3000,44.5000) -- (189.4000,44.5000) -- (189.6000,44.5000) -- (189.7000,44.5000) -- (189.9000,44.5000) -- (190.0000,44.5000) -- (190.1000,44.5000) -- (190.3000,44.5000) -- (190.4000,44.5000) -- (190.6000,44.5000) -- (190.7000,44.5000) -- (190.8000,44.5000) -- (191.0000,44.5000) -- (191.1000,44.5000) -- (191.2000,44.5000) -- (191.4000,44.5000) -- (191.5000,44.5000) -- (191.7000,44.5000) -- (191.8000,44.5000) -- (191.9000,44.6000) -- (192.1000,44.7000) -- (192.2000,44.9000) -- (192.3000,45.1000) -- (192.5000,45.4000) -- (192.6000,45.8000) -- (192.8000,46.2000) -- (192.9000,46.6000) -- (193.0000,47.2000) -- (193.2000,47.7000) -- (193.3000,48.3000) -- (193.5000,49.0000) -- (193.6000,49.7000) -- (193.7000,50.5000) -- (193.9000,51.3000) -- (194.0000,52.2000) -- (194.1000,53.1000) -- (194.3000,54.1000) -- (194.4000,55.1000) -- (194.6000,56.2000) -- (194.7000,57.3000) -- (194.8000,58.5000) -- (195.0000,59.8000) -- (195.1000,61.0000) -- (195.2000,62.4000) -- (195.4000,63.7000) -- (195.5000,65.1000) -- (195.7000,66.5000) -- (195.8000,68.0000) -- (195.9000,69.5000) -- (196.1000,71.0000) -- (196.2000,72.4000) -- (196.4000,73.9000) -- (196.5000,75.4000) -- (196.6000,76.8000) -- (196.8000,78.2000) -- (196.9000,79.6000) -- (197.0000,81.0000) -- (197.2000,82.5000) -- (197.3000,84.0000) -- (197.5000,85.3000) -- (197.6000,86.3000) -- (197.7000,86.9000) -- (197.9000,87.2000) -- (198.0000,87.3000) -- (198.1000,87.2000) -- (198.3000,87.0000) -- (198.4000,86.8000) -- (198.6000,86.6000) -- (198.7000,86.5000) -- (198.8000,86.4000) -- (199.0000,86.3000) -- (199.1000,86.3000) -- (199.3000,86.3000) -- (199.4000,86.3000) -- (199.5000,86.3000) -- (199.7000,86.3000) -- (199.8000,86.3000) -- (199.9000,86.3000) -- (200.1000,86.3000) -- (200.2000,86.3000) -- (200.4000,86.3000) -- (200.5000,86.3000) -- (200.6000,86.3000) -- (200.8000,86.3000) -- (200.9000,86.3000) -- (201.1000,86.3000) -- (201.2000,86.3000) -- (201.3000,86.3000) -- (201.5000,86.3000) -- (201.6000,86.3000) -- (201.7000,86.1000) -- (201.9000,85.9000) -- (202.0000,85.9000) -- (202.2000,86.0000) -- (202.3000,86.1000) -- (202.4000,86.1000) -- (202.6000,85.9000) -- (202.7000,85.3000) -- (202.8000,84.5000) -- (203.0000,83.5000) -- (203.1000,82.2000) -- (203.3000,80.7000) -- (203.4000,79.1000) -- (203.5000,77.3000) -- (203.7000,75.5000) -- (203.8000,73.7000) -- (204.0000,71.9000) -- (204.1000,70.0000) -- (204.2000,68.2000) -- (204.4000,66.4000) -- (204.5000,64.7000) -- (204.6000,63.0000) -- (204.8000,61.4000) -- (204.9000,59.8000) -- (205.1000,58.3000) -- (205.2000,56.9000) -- (205.3000,55.5000) -- (205.5000,54.3000) -- (205.6000,53.1000) -- (205.7000,52.0000) -- (205.9000,51.0000) -- (206.0000,50.0000) -- (206.2000,49.1000) -- (206.3000,48.4000) -- (206.4000,47.6000) -- (206.6000,47.0000) -- (206.7000,46.4000) -- (206.9000,45.9000) -- (207.0000,45.4000) -- (207.1000,45.1000) -- (207.3000,44.8000) -- (207.4000,44.5000) -- (207.5000,44.2000) -- (207.7000,44.1000) -- (207.8000,44.1000) -- (208.0000,44.1000) -- (208.1000,44.2000) -- (208.2000,44.3000) -- (208.4000,44.4000) -- (208.5000,44.4000) -- (208.6000,44.5000) -- (208.8000,44.5000) -- (208.9000,44.5000) -- (209.1000,44.5000) -- (209.2000,44.5000) -- (209.3000,44.5000) -- (209.5000,44.5000) -- (209.6000,44.5000) -- (209.8000,44.5000) -- (209.9000,44.5000) -- (210.0000,44.5000) -- (210.2000,44.5000) -- (210.3000,44.5000) -- (210.4000,44.5000) -- (210.6000,44.5000) -- (210.7000,44.5000) -- (210.9000,44.5000) -- (211.0000,44.5000) -- (211.1000,44.5000) -- (211.3000,44.5000) -- (211.4000,44.5000) -- (211.5000,44.5000) -- (211.7000,44.5000) -- (211.8000,44.5000) -- (212.0000,44.5000) -- (212.1000,44.6000) -- (212.2000,44.7000) -- (212.4000,44.8000) -- (212.5000,45.0000) -- (212.7000,45.3000) -- (212.8000,45.6000) -- (212.9000,45.9000) -- (213.1000,46.4000) -- (213.2000,46.9000) -- (213.3000,47.4000) -- (213.5000,48.0000) -- (213.6000,48.6000) -- (213.8000,49.3000) -- (213.9000,50.0000) -- (214.0000,50.8000) -- (214.2000,51.7000) -- (214.3000,52.6000) -- (214.5000,53.5000) -- (214.6000,54.5000) -- (214.7000,55.6000) -- (214.9000,56.7000) -- (215.0000,57.8000) -- (215.1000,59.0000) -- (215.3000,60.3000) -- (215.4000,61.6000) -- (215.6000,62.9000) -- (215.7000,64.3000) -- (215.8000,65.7000) -- (216.0000,67.1000) -- (216.1000,68.6000) -- (216.2000,70.1000) -- (216.4000,71.6000) -- (216.5000,73.0000) -- (216.7000,74.5000) -- (216.8000,76.0000) -- (216.9000,77.4000) -- (217.1000,78.8000) -- (217.2000,80.1000) -- (217.4000,81.6000) -- (217.5000,83.1000) -- (217.6000,84.6000) -- (217.8000,85.8000) -- (217.9000,86.6000) -- (218.0000,87.0000) -- (218.2000,87.2000) -- (218.3000,87.2000) -- (218.5000,87.1000) -- (218.6000,86.9000) -- (218.7000,86.7000) -- (218.9000,86.6000) -- (219.0000,86.5000) -- (219.1000,86.4000) -- (219.3000,86.3000) -- (219.4000,86.3000) -- (219.6000,86.3000) -- (219.7000,86.3000) -- (219.8000,86.3000) -- (220.0000,86.3000) -- (220.1000,86.3000) -- (220.3000,86.3000) -- (220.4000,86.3000) -- (220.5000,86.3000) -- (220.7000,86.3000) -- (220.8000,86.3000) -- (220.9000,86.3000) -- (221.1000,86.3000) -- (221.2000,86.3000) -- (221.4000,86.3000) -- (221.5000,86.3000) -- (221.6000,86.3000) -- (221.8000,86.3000) -- (221.9000,86.2000) -- (222.0000,86.0000) -- (222.2000,85.9000) -- (222.3000,85.9000) -- (222.5000,86.1000) -- (222.6000,86.2000) -- (222.7000,86.1000) -- (222.9000,85.7000) -- (223.0000,85.1000) -- (223.2000,84.1000) -- (223.3000,83.0000) -- (223.4000,81.6000) -- (223.6000,80.1000) -- (223.7000,78.4000) -- (223.8000,76.6000) -- (224.0000,74.8000) -- (224.1000,73.0000) -- (224.3000,71.2000) -- (224.4000,69.3000) -- (224.5000,67.5000) -- (224.7000,65.7000) -- (224.8000,64.0000) -- (225.0000,62.3000) -- (225.1000,60.7000) -- (225.2000,59.2000) -- (225.4000,57.7000) -- (225.5000,56.3000) -- (225.6000,55.0000) -- (225.8000,53.8000) -- (225.9000,52.7000) -- (226.1000,51.6000) -- (226.2000,50.6000) -- (226.3000,49.7000) -- (226.5000,48.8000) -- (226.6000,48.1000) -- (226.7000,47.4000) -- (226.9000,46.7000) -- (227.0000,46.2000) -- (227.2000,45.7000) -- (227.3000,45.3000) -- (227.4000,44.9000) -- (227.6000,44.7000) -- (227.7000,44.4000) -- (227.9000,44.2000) -- (228.0000,44.1000) -- (228.1000,44.1000) -- (228.3000,44.2000) -- (228.4000,44.3000) -- (228.5000,44.3000) -- (228.7000,44.4000) -- (228.8000,44.5000) -- (229.0000,44.5000) -- (229.1000,44.5000) -- (229.2000,44.5000) -- (229.4000,44.5000) -- (229.5000,44.5000) -- (229.6000,44.5000) -- (229.8000,44.5000) -- (229.9000,44.5000) -- (230.1000,44.5000) -- (230.2000,44.5000) -- (230.3000,44.5000) -- (230.5000,44.5000) -- (230.6000,44.5000) -- (230.8000,44.5000) -- (230.9000,44.5000) -- (231.0000,44.5000) -- (231.2000,44.5000) -- (231.3000,44.5000) -- (231.4000,44.5000) -- (231.6000,44.5000) -- (231.7000,44.5000) -- (231.9000,44.5000) -- (232.0000,44.5000) -- (232.1000,44.5000) -- (232.3000,44.5000) -- (232.4000,44.6000) -- (232.6000,44.7000) -- (232.7000,44.9000) -- (232.8000,45.1000) -- (233.0000,45.4000) -- (233.1000,45.7000) -- (233.2000,46.1000) -- (233.4000,46.6000) -- (233.5000,47.1000) -- (233.7000,47.6000) -- (233.8000,48.2000) -- (233.9000,48.9000) -- (234.1000,49.6000) -- (234.2000,50.3000) -- (234.3000,51.1000) -- (234.5000,52.0000) -- (234.6000,52.9000) -- (234.8000,53.9000) -- (234.9000,54.9000) -- (235.0000,56.0000) -- (235.2000,57.1000) -- (235.3000,58.3000) -- (235.5000,59.5000) -- (235.6000,60.8000) -- (235.7000,62.1000) -- (235.9000,63.4000) -- (236.0000,64.8000) -- (236.1000,66.3000) -- (236.3000,67.7000) -- (236.4000,69.2000) -- (236.6000,70.6000) -- (236.7000,72.1000) -- (236.8000,73.6000) -- (237.0000,75.1000) -- (237.1000,76.5000) -- (237.2000,77.9000) -- (237.4000,79.3000) -- (237.5000,80.7000) -- (237.7000,82.2000) -- (237.8000,83.7000) -- (237.9000,85.1000) -- (238.1000,86.1000) -- (238.2000,86.8000) -- (238.4000,87.2000) -- (238.5000,87.3000) -- (238.6000,87.2000) -- (238.8000,87.0000) -- (238.9000,86.9000) -- (239.0000,86.7000) -- (239.2000,86.5000) -- (239.3000,86.4000) -- (239.5000,86.3000) -- (239.6000,86.3000) -- (239.7000,86.3000) -- (239.9000,86.3000) -- (240.0000,86.3000) -- (240.2000,86.3000) -- (240.3000,86.3000) -- (240.4000,86.3000) -- (240.6000,86.3000) -- (240.7000,86.3000) -- (240.8000,86.3000) -- (241.0000,86.3000) -- (241.1000,86.3000) -- (241.3000,86.3000) -- (241.4000,86.3000) -- (241.5000,86.3000) -- (241.7000,86.3000) -- (241.8000,86.3000) -- (241.9000,86.3000) -- (242.1000,86.3000) -- (242.2000,86.2000) -- (242.4000,86.0000) -- (242.5000,85.9000) -- (242.6000,86.0000) -- (242.8000,86.1000) -- (242.9000,86.1000) -- (243.1000,85.9000) -- (243.2000,85.5000) -- (243.3000,84.7000) -- (243.5000,83.7000) -- (243.6000,82.5000) -- (243.7000,81.0000) -- (243.9000,79.4000) -- (244.0000,77.7000) -- (244.2000,75.9000) -- (244.3000,74.1000) -- (244.4000,72.3000) -- (244.6000,70.4000) -- (244.7000,68.6000) -- (244.8000,66.8000) -- (245.0000,65.1000) -- (245.1000,63.4000) -- (245.3000,61.7000) -- (245.4000,60.1000) -- (245.5000,58.6000) -- (245.7000,57.2000) -- (245.8000,55.8000) -- (246.0000,54.6000) -- (246.1000,53.3000) -- (246.2000,52.2000) -- (246.4000,51.2000) -- (246.5000,50.2000) -- (246.6000,49.3000) -- (246.8000,48.5000) -- (246.9000,47.8000) -- (247.1000,47.1000) -- (247.2000,46.5000) -- (247.3000,46.0000) -- (247.5000,45.5000) -- (247.6000,45.1000) -- (247.7000,44.8000) -- (247.9000,44.5000) -- (248.0000,44.3000) -- (248.2000,44.1000) -- (248.3000,44.1000) -- (248.4000,44.1000) -- (248.6000,44.2000) -- (248.7000,44.3000) -- (248.9000,44.4000) -- (249.0000,44.4000) -- (249.1000,44.5000) -- (249.3000,44.5000) -- (249.4000,44.5000) -- (249.5000,44.5000) -- (249.7000,44.5000) -- (249.8000,44.5000) -- (250.0000,44.5000) -- (250.1000,44.5000) -- (250.2000,44.5000) -- (250.4000,44.5000) -- (250.5000,44.5000) -- (250.7000,44.5000) -- (250.8000,44.5000) -- (250.9000,44.5000) -- (251.1000,44.5000) -- (251.2000,44.5000) -- (251.3000,44.5000) -- (251.5000,44.5000) -- (251.6000,44.5000) -- (251.8000,44.5000) -- (251.9000,44.5000) -- (251.9000,44.5000);



  \end{scope}
  \begin{scope}[scale=1.006,draw=black,line join=bevel,line cap=rect,line width=0.800pt]
  \end{scope}
  \begin{scope}[scale=1.006,draw=black,line join=bevel,line cap=rect,line width=0.800pt]
  \end{scope}
  \begin{scope}[scale=1.006,draw=black,line join=bevel,line cap=rect,line width=0.800pt]
  \end{scope}
  \begin{scope}[scale=1.006,draw=black,line join=bevel,line cap=rect,line width=0.800pt]
  \end{scope}
  \begin{scope}[scale=1.006,draw=cff0000,line join=round,line cap=round,line width=0.480pt]
    \path[draw] (165.2000,51.8000) -- (165.2000,51.8000) -- (165.3000,51.0000) -- (165.4000,50.3000) -- (165.5000,49.6000) -- (165.5000,48.9000) -- (165.6000,48.3000) -- (165.7000,47.8000) -- (165.8000,47.2000) -- (165.9000,46.8000) -- (166.0000,46.4000) -- (166.1000,46.0000) -- (166.2000,45.7000) -- (166.2000,45.4000) -- (166.3000,45.2000) -- (166.4000,45.0000) -- (166.5000,44.8000) -- (166.6000,44.7000) -- (166.7000,44.7000) -- (166.8000,44.7000) -- (166.8000,44.7000) -- (166.9000,44.7000) -- (167.0000,44.8000) -- (167.1000,44.9000) -- (167.2000,45.1000) -- (167.3000,45.3000) -- (167.4000,45.4000) -- (167.5000,45.6000) -- (167.5000,45.9000) -- (167.6000,46.1000) -- (167.7000,46.4000) -- (167.8000,46.6000) -- (167.9000,46.9000) -- (168.0000,47.1000) -- (168.1000,47.4000) -- (168.2000,47.6000) -- (168.2000,47.8000) -- (168.3000,48.0000) -- (168.4000,48.2000) -- (168.5000,48.4000) -- (168.6000,48.6000) -- (168.7000,48.8000) -- (168.8000,48.9000) -- (168.8000,49.0000) -- (168.9000,49.1000) -- (169.0000,49.1000) -- (169.1000,49.1000) -- (169.2000,49.1000) -- (169.3000,49.1000) -- (169.4000,49.0000) -- (169.5000,48.9000) -- (169.5000,48.8000) -- (169.6000,48.6000) -- (169.7000,48.5000) -- (169.8000,48.3000) -- (169.9000,48.0000) -- (170.0000,47.8000) -- (170.1000,47.5000) -- (170.1000,47.2000) -- (170.2000,46.9000) -- (170.3000,46.6000) -- (170.4000,46.2000) -- (170.5000,45.9000) -- (170.6000,45.5000) -- (170.7000,45.1000) -- (170.8000,44.8000) -- (170.8000,44.4000) -- (170.9000,44.0000) -- (171.0000,43.7000) -- (171.1000,43.4000) -- (171.2000,43.0000) -- (171.3000,42.7000) -- (171.4000,42.5000) -- (171.4000,42.2000) -- (171.5000,42.0000) -- (171.6000,41.8000) -- (171.7000,41.6000) -- (171.8000,41.5000) -- (171.9000,41.4000) -- (172.0000,41.3000) -- (172.1000,41.3000) -- (172.1000,41.3000) -- (172.2000,41.4000) -- (172.3000,41.6000) -- (172.4000,41.7000) -- (172.5000,42.0000) -- (172.6000,42.3000) -- (172.7000,42.6000) -- (172.8000,43.0000) -- (172.8000,43.4000) -- (172.9000,43.9000) -- (173.0000,44.5000) -- (173.1000,45.1000) -- (173.2000,45.7000) -- (173.3000,46.4000) -- (173.4000,47.1000) -- (173.4000,47.9000) -- (173.5000,48.8000) -- (173.6000,49.6000) -- (173.7000,50.5000) -- (173.8000,51.5000) -- (173.9000,52.5000) -- (174.0000,53.5000) -- (174.1000,54.5000) -- (174.1000,55.6000) -- (174.2000,56.6000) -- (174.3000,57.7000) -- (174.4000,58.9000) -- (174.5000,60.0000) -- (174.6000,61.1000) -- (174.7000,62.2000) -- (174.7000,63.4000) -- (174.8000,64.5000) -- (174.9000,65.6000) -- (175.0000,66.8000) -- (175.1000,67.8000) -- (175.2000,68.9000) -- (175.3000,70.0000) -- (175.4000,71.0000) -- (175.4000,72.1000) -- (175.5000,73.0000) -- (175.6000,74.0000) -- (175.7000,74.9000) -- (175.8000,75.8000) -- (175.9000,76.7000) -- (176.0000,77.5000) -- (176.0000,78.3000) -- (176.1000,79.0000) -- (176.2000,79.7000) -- (176.3000,80.3000) -- (176.4000,80.9000) -- (176.5000,81.5000) -- (176.6000,82.0000) -- (176.7000,82.5000) -- (176.7000,83.0000) -- (176.8000,83.4000) -- (176.9000,83.7000) -- (177.0000,84.0000) -- (177.1000,84.3000) -- (177.2000,84.6000) -- (177.3000,84.8000) -- (177.4000,85.0000) -- (177.4000,85.1000) -- (177.5000,85.3000) -- (177.6000,85.3000) -- (177.7000,85.4000) -- (177.8000,85.5000) -- (177.9000,85.5000) -- (178.0000,85.5000) -- (178.0000,85.5000) -- (178.1000,85.5000) -- (178.2000,85.5000) -- (178.3000,85.4000) -- (178.4000,85.4000) -- (178.5000,85.4000) -- (178.6000,85.3000) -- (178.7000,85.3000) -- (178.7000,85.2000) -- (178.8000,85.2000) -- (178.9000,85.2000) -- (179.0000,85.1000) -- (179.1000,85.1000) -- (179.2000,85.1000) -- (179.3000,85.1000) -- (179.3000,85.1000) -- (179.4000,85.1000) -- (179.5000,85.1000) -- (179.6000,85.2000) -- (179.7000,85.2000) -- (179.8000,85.3000) -- (179.9000,85.4000) -- (180.0000,85.4000) -- (180.0000,85.5000) -- (180.1000,85.6000) -- (180.2000,85.7000) -- (180.3000,85.8000) -- (180.4000,85.9000) -- (180.5000,86.0000) -- (180.6000,86.1000) -- (180.6000,86.2000) -- (180.7000,86.3000) -- (180.8000,86.4000) -- (180.9000,86.4000) -- (181.0000,86.5000) -- (181.1000,86.5000) -- (181.2000,86.6000) -- (181.3000,86.6000) -- (181.3000,86.6000) -- (181.4000,86.5000) -- (181.5000,86.4000) -- (181.6000,86.4000) -- (181.7000,86.2000) -- (181.8000,86.1000) -- (181.9000,85.9000) -- (182.0000,85.6000) -- (182.0000,85.4000) -- (182.1000,85.1000) -- (182.2000,84.7000) -- (182.3000,84.3000) -- (182.4000,83.9000) -- (182.5000,83.5000) -- (182.6000,82.9000) -- (182.6000,82.4000) -- (182.7000,81.8000) -- (182.8000,81.2000) -- (182.9000,80.5000) -- (183.0000,79.8000) -- (183.1000,79.0000) -- (183.2000,78.2000) -- (183.3000,77.4000) -- (183.3000,76.5000) -- (183.4000,75.6000) -- (183.5000,74.7000) -- (183.6000,73.7000) -- (183.7000,72.7000) -- (183.8000,71.7000) -- (183.9000,70.7000) -- (183.9000,69.6000) -- (184.0000,68.6000) -- (184.1000,67.5000) -- (184.2000,66.4000) -- (184.3000,65.3000) -- (184.4000,64.2000) -- (184.5000,63.1000) -- (184.6000,62.1000) -- (184.6000,61.0000) -- (184.7000,59.9000) -- (184.8000,58.9000) -- (184.9000,57.9000) -- (185.0000,56.9000) -- (185.1000,55.9000) -- (185.2000,54.9000) -- (185.2000,54.0000) -- (185.3000,53.1000) -- (185.4000,52.3000) -- (185.5000,51.5000) -- (185.6000,50.7000) -- (185.7000,50.0000) -- (185.8000,49.3000) -- (185.9000,48.7000) -- (185.9000,48.1000) -- (186.0000,47.5000) -- (186.1000,47.0000) -- (186.2000,46.6000) -- (186.3000,46.2000) -- (186.4000,45.8000) -- (186.5000,45.5000) -- (186.5000,45.3000) -- (186.6000,45.0000) -- (186.7000,44.9000) -- (186.8000,44.8000) -- (186.9000,44.7000) -- (187.0000,44.6000) -- (187.1000,44.6000) -- (187.2000,44.7000) -- (187.2000,44.7000) -- (187.3000,44.8000) -- (187.4000,45.0000) -- (187.5000,45.1000) -- (187.6000,45.3000) -- (187.7000,45.5000) -- (187.8000,45.7000) -- (187.9000,46.0000) -- (187.9000,46.2000) -- (188.0000,46.4000) -- (188.1000,46.7000) -- (188.2000,47.0000) -- (188.3000,47.2000) -- (188.4000,47.4000) -- (188.5000,47.7000) -- (188.5000,47.9000) -- (188.6000,48.1000) -- (188.7000,48.3000) -- (188.8000,48.5000) -- (188.9000,48.7000) -- (189.0000,48.8000) -- (189.1000,48.9000) -- (189.2000,49.0000) -- (189.2000,49.1000) -- (189.3000,49.1000) -- (189.4000,49.1000) -- (189.5000,49.1000) -- (189.6000,49.1000) -- (189.7000,49.0000) -- (189.8000,48.9000) -- (189.8000,48.8000) -- (189.9000,48.6000) -- (190.0000,48.4000) -- (190.1000,48.2000) -- (190.2000,47.9000) -- (190.3000,47.7000) -- (190.4000,47.4000) -- (190.5000,47.1000) -- (190.5000,46.8000) -- (190.6000,46.4000) -- (190.7000,46.1000) -- (190.8000,45.7000) -- (190.9000,45.4000) -- (191.0000,45.0000) -- (191.1000,44.6000) -- (191.1000,44.3000) -- (191.2000,43.9000) -- (191.3000,43.6000) -- (191.4000,43.2000) -- (191.5000,42.9000) -- (191.6000,42.6000) -- (191.7000,42.3000) -- (191.8000,42.1000) -- (191.8000,41.9000) -- (191.9000,41.7000) -- (192.0000,41.5000) -- (192.1000,41.4000) -- (192.2000,41.3000) -- (192.3000,41.3000) -- (192.4000,41.3000) -- (192.5000,41.3000) -- (192.5000,41.4000) -- (192.6000,41.6000) -- (192.7000,41.8000) -- (192.8000,42.0000) -- (192.9000,42.4000) -- (193.0000,42.7000) -- (193.1000,43.1000) -- (193.1000,43.6000) -- (193.2000,44.1000) -- (193.3000,44.7000) -- (193.4000,45.3000) -- (193.5000,45.9000) -- (193.6000,46.7000) -- (193.7000,47.4000) -- (193.8000,48.2000) -- (193.8000,49.1000) -- (193.9000,50.0000) -- (194.0000,50.9000) -- (194.1000,51.8000) -- (194.2000,52.8000) -- (194.3000,53.9000) -- (194.4000,54.9000) -- (194.4000,56.0000) -- (194.5000,57.1000) -- (194.6000,58.2000) -- (194.7000,59.3000) -- (194.8000,60.4000) -- (194.9000,61.6000) -- (195.0000,62.7000) -- (195.1000,63.8000) -- (195.1000,65.0000) -- (195.2000,66.1000) -- (195.3000,67.2000) -- (195.4000,68.3000) -- (195.5000,69.4000) -- (195.6000,70.4000) -- (195.7000,71.5000) -- (195.7000,72.5000) -- (195.8000,73.4000) -- (195.9000,74.4000) -- (196.0000,75.3000) -- (196.1000,76.2000) -- (196.2000,77.0000) -- (196.3000,77.8000) -- (196.4000,78.6000) -- (196.4000,79.3000) -- (196.5000,80.0000) -- (196.6000,80.6000) -- (196.7000,81.2000) -- (196.8000,81.7000) -- (196.9000,82.3000) -- (197.0000,82.7000) -- (197.1000,83.2000) -- (197.1000,83.5000) -- (197.2000,83.9000) -- (197.3000,84.2000) -- (197.4000,84.5000) -- (197.5000,84.7000) -- (197.6000,84.9000) -- (197.7000,85.1000) -- (197.7000,85.2000) -- (197.8000,85.3000) -- (197.9000,85.4000) -- (198.0000,85.5000) -- (198.1000,85.5000) -- (198.2000,85.5000) -- (198.3000,85.5000) -- (198.4000,85.5000) -- (198.4000,85.5000) -- (198.5000,85.5000) -- (198.6000,85.4000) -- (198.7000,85.4000) -- (198.8000,85.3000) -- (198.9000,85.3000) -- (199.0000,85.2000) -- (199.0000,85.2000) -- (199.1000,85.2000) -- (199.2000,85.1000) -- (199.3000,85.1000) -- (199.4000,85.1000) -- (199.5000,85.1000) -- (199.6000,85.1000) -- (199.7000,85.1000) -- (199.7000,85.1000) -- (199.8000,85.2000) -- (199.9000,85.2000) -- (200.0000,85.2000) -- (200.1000,85.3000) -- (200.2000,85.4000) -- (200.3000,85.5000) -- (200.3000,85.5000) -- (200.4000,85.6000) -- (200.5000,85.7000) -- (200.6000,85.8000) -- (200.7000,85.9000) -- (200.8000,86.0000) -- (200.9000,86.1000) -- (201.0000,86.2000) -- (201.0000,86.3000) -- (201.1000,86.4000) -- (201.2000,86.5000) -- (201.3000,86.5000) -- (201.4000,86.6000) -- (201.5000,86.6000) -- (201.6000,86.6000) -- (201.7000,86.6000) -- (201.7000,86.5000) -- (201.8000,86.4000) -- (201.9000,86.3000) -- (202.0000,86.2000) -- (202.1000,86.0000) -- (202.2000,85.8000) -- (202.3000,85.6000) -- (202.3000,85.3000) -- (202.4000,85.0000) -- (202.5000,84.6000) -- (202.6000,84.2000) -- (202.7000,83.8000) -- (202.8000,83.3000) -- (202.9000,82.8000) -- (203.0000,82.2000) -- (203.0000,81.6000) -- (203.1000,80.9000) -- (203.2000,80.2000) -- (203.3000,79.5000) -- (203.4000,78.7000) -- (203.5000,77.9000) -- (203.6000,77.0000) -- (203.6000,76.2000) -- (203.7000,75.2000) -- (203.8000,74.3000) -- (203.9000,73.3000) -- (204.0000,72.3000) -- (204.1000,71.3000) -- (204.2000,70.3000) -- (204.3000,69.2000) -- (204.3000,68.1000) -- (204.4000,67.1000) -- (204.5000,66.0000) -- (204.6000,64.9000) -- (204.7000,63.8000) -- (204.8000,62.7000) -- (204.9000,61.6000) -- (204.9000,60.6000) -- (205.0000,59.5000) -- (205.1000,58.5000) -- (205.2000,57.5000) -- (205.3000,56.5000) -- (205.4000,55.5000) -- (205.5000,54.6000) -- (205.6000,53.7000) -- (205.6000,52.8000) -- (205.7000,51.9000) -- (205.8000,51.1000) -- (205.9000,50.4000) -- (206.0000,49.7000) -- (206.1000,49.0000) -- (206.2000,48.4000) -- (206.3000,47.8000) -- (206.3000,47.3000) -- (206.4000,46.8000) -- (206.5000,46.4000) -- (206.6000,46.0000) -- (206.7000,45.7000) -- (206.8000,45.4000) -- (206.9000,45.1000) -- (206.9000,45.0000) -- (207.0000,44.8000) -- (207.1000,44.7000) -- (207.2000,44.6000) -- (207.3000,44.6000) -- (207.4000,44.6000) -- (207.5000,44.7000) -- (207.6000,44.8000) -- (207.6000,44.9000) -- (207.7000,45.0000) -- (207.8000,45.2000) -- (207.9000,45.4000) -- (208.0000,45.6000) -- (208.1000,45.8000) -- (208.2000,46.0000) -- (208.2000,46.3000) -- (208.3000,46.5000) -- (208.4000,46.8000) -- (208.5000,47.1000) -- (208.6000,47.3000) -- (208.7000,47.5000) -- (208.8000,47.8000) -- (208.9000,48.0000) -- (208.9000,48.2000) -- (209.0000,48.4000) -- (209.1000,48.6000) -- (209.2000,48.8000) -- (209.3000,48.9000) -- (209.4000,49.0000) -- (209.5000,49.1000) -- (209.5000,49.1000) -- (209.6000,49.2000) -- (209.7000,49.2000) -- (209.8000,49.1000) -- (209.9000,49.1000) -- (210.0000,49.0000) -- (210.1000,48.9000) -- (210.2000,48.7000) -- (210.2000,48.5000) -- (210.3000,48.3000) -- (210.4000,48.1000) -- (210.5000,47.9000) -- (210.6000,47.6000) -- (210.7000,47.3000) -- (210.8000,47.0000) -- (210.9000,46.6000) -- (210.9000,46.3000) -- (211.0000,45.9000) -- (211.1000,45.6000) -- (211.2000,45.2000) -- (211.3000,44.8000) -- (211.4000,44.5000) -- (211.5000,44.1000) -- (211.5000,43.8000) -- (211.6000,43.4000) -- (211.7000,43.1000) -- (211.8000,42.8000) -- (211.9000,42.5000) -- (212.0000,42.2000) -- (212.1000,42.0000) -- (212.2000,41.8000) -- (212.2000,41.6000) -- (212.3000,41.4000) -- (212.4000,41.3000) -- (212.5000,41.3000) -- (212.6000,41.3000) -- (212.7000,41.3000) -- (212.8000,41.4000) -- (212.8000,41.5000) -- (212.9000,41.6000) -- (213.0000,41.9000) -- (213.1000,42.1000) -- (213.2000,42.5000) -- (213.3000,42.8000) -- (213.4000,43.3000) -- (213.5000,43.7000) -- (213.5000,44.3000) -- (213.6000,44.9000) -- (213.7000,45.5000) -- (213.8000,46.2000) -- (213.9000,46.9000) -- (214.0000,47.7000) -- (214.1000,48.5000) -- (214.1000,49.4000) -- (214.2000,50.3000) -- (214.3000,51.2000) -- (214.4000,52.2000) -- (214.5000,53.2000) -- (214.6000,54.2000) -- (214.7000,55.3000) -- (214.8000,56.4000) -- (214.8000,57.5000) -- (214.9000,58.6000) -- (215.0000,59.7000) -- (215.1000,60.9000) -- (215.2000,62.0000) -- (215.3000,63.1000) -- (215.4000,64.3000) -- (215.5000,65.4000) -- (215.5000,66.5000) -- (215.6000,67.6000) -- (215.7000,68.7000) -- (215.8000,69.8000) -- (215.9000,70.8000) -- (216.0000,71.9000) -- (216.1000,72.9000) -- (216.1000,73.8000) -- (216.2000,74.8000) -- (216.3000,75.7000) -- (216.4000,76.5000) -- (216.5000,77.4000) -- (216.6000,78.1000) -- (216.7000,78.9000) -- (216.8000,79.6000) -- (216.8000,80.2000) -- (216.9000,80.9000) -- (217.0000,81.4000) -- (217.1000,82.0000) -- (217.2000,82.5000) -- (217.3000,82.9000) -- (217.4000,83.3000) -- (217.4000,83.7000) -- (217.5000,84.0000) -- (217.6000,84.3000) -- (217.7000,84.6000) -- (217.8000,84.8000) -- (217.9000,85.0000) -- (218.0000,85.1000) -- (218.1000,85.3000) -- (218.1000,85.4000) -- (218.2000,85.4000) -- (218.3000,85.5000) -- (218.4000,85.5000) -- (218.5000,85.5000) -- (218.6000,85.5000) -- (218.7000,85.5000) -- (218.7000,85.5000) -- (218.8000,85.5000) -- (218.9000,85.4000) -- (219.0000,85.4000) -- (219.1000,85.3000) -- (219.2000,85.3000) -- (219.3000,85.2000) -- (219.4000,85.2000) -- (219.4000,85.2000) -- (219.5000,85.1000) -- (219.6000,85.1000) -- (219.7000,85.1000) -- (219.8000,85.1000) -- (219.9000,85.1000) -- (220.0000,85.1000) -- (220.1000,85.1000) -- (220.1000,85.2000) -- (220.2000,85.2000) -- (220.3000,85.3000) -- (220.4000,85.3000) -- (220.5000,85.4000) -- (220.6000,85.5000) -- (220.7000,85.6000) -- (220.7000,85.7000) -- (220.8000,85.8000) -- (220.9000,85.9000) -- (221.0000,86.0000) -- (221.1000,86.1000) -- (221.2000,86.2000) -- (221.3000,86.3000) -- (221.4000,86.4000) -- (221.4000,86.4000) -- (221.5000,86.5000) -- (221.6000,86.5000) -- (221.7000,86.6000) -- (221.8000,86.6000) -- (221.9000,86.6000) -- (222.0000,86.6000) -- (222.0000,86.5000) -- (222.1000,86.4000) -- (222.2000,86.3000) -- (222.3000,86.1000) -- (222.4000,86.0000) -- (222.5000,85.7000) -- (222.6000,85.5000) -- (222.7000,85.2000) -- (222.7000,84.9000) -- (222.8000,84.5000) -- (222.9000,84.1000) -- (223.0000,83.6000) -- (223.1000,83.1000) -- (223.2000,82.6000) -- (223.3000,82.0000) -- (223.3000,81.3000) -- (223.4000,80.7000) -- (223.5000,80.0000) -- (223.6000,79.2000) -- (223.7000,78.4000) -- (223.8000,77.6000) -- (223.9000,76.7000) -- (224.0000,75.8000) -- (224.0000,74.9000) -- (224.1000,73.9000) -- (224.2000,73.0000) -- (224.3000,71.9000) -- (224.4000,70.9000) -- (224.5000,69.9000) -- (224.6000,68.8000) -- (224.7000,67.7000) -- (224.7000,66.6000) -- (224.8000,65.6000) -- (224.9000,64.5000) -- (225.0000,63.4000) -- (225.1000,62.3000) -- (225.2000,61.2000) -- (225.3000,60.1000) -- (225.3000,59.1000) -- (225.4000,58.1000) -- (225.5000,57.1000) -- (225.6000,56.1000) -- (225.7000,55.1000) -- (225.8000,54.2000) -- (225.9000,53.3000) -- (226.0000,52.4000) -- (226.0000,51.6000) -- (226.1000,50.8000) -- (226.2000,50.1000) -- (226.3000,49.4000) -- (226.4000,48.7000) -- (226.5000,48.1000) -- (226.6000,47.6000) -- (226.6000,47.1000) -- (226.7000,46.6000) -- (226.8000,46.2000) -- (226.9000,45.8000) -- (227.0000,45.5000) -- (227.1000,45.3000) -- (227.2000,45.0000) -- (227.3000,44.9000) -- (227.3000,44.7000) -- (227.4000,44.6000) -- (227.5000,44.6000) -- (227.6000,44.6000) -- (227.7000,44.6000) -- (227.8000,44.7000) -- (227.9000,44.8000) -- (227.9000,44.9000) -- (228.0000,45.1000) -- (228.1000,45.2000) -- (228.2000,45.4000) -- (228.3000,45.7000) -- (228.4000,45.9000) -- (228.5000,46.1000) -- (228.6000,46.4000) -- (228.6000,46.6000) -- (228.7000,46.9000) -- (228.8000,47.2000) -- (228.9000,47.4000) -- (229.0000,47.6000) -- (229.1000,47.9000) -- (229.2000,48.1000) -- (229.2000,48.3000) -- (229.3000,48.5000) -- (229.4000,48.7000) -- (229.5000,48.8000) -- (229.6000,48.9000) -- (229.7000,49.0000) -- (229.8000,49.1000) -- (229.9000,49.2000) -- (229.9000,49.2000) -- (230.0000,49.2000) -- (230.1000,49.1000) -- (230.2000,49.1000) -- (230.3000,49.0000) -- (230.4000,48.8000) -- (230.5000,48.7000) -- (230.6000,48.5000) -- (230.6000,48.3000) -- (230.7000,48.0000) -- (230.8000,47.8000) -- (230.9000,47.5000) -- (231.0000,47.2000) -- (231.1000,46.9000) -- (231.2000,46.5000) -- (231.2000,46.2000) -- (231.3000,45.8000) -- (231.4000,45.4000) -- (231.5000,45.1000) -- (231.6000,44.7000) -- (231.7000,44.3000) -- (231.8000,44.0000) -- (231.9000,43.6000) -- (231.9000,43.3000) -- (232.0000,43.0000) -- (232.1000,42.6000) -- (232.2000,42.4000) -- (232.3000,42.1000) -- (232.4000,41.9000) -- (232.5000,41.7000) -- (232.5000,41.5000) -- (232.6000,41.4000) -- (232.7000,41.3000) -- (232.8000,41.2000) -- (232.9000,41.2000) -- (233.0000,41.3000) -- (233.1000,41.4000) -- (233.2000,41.5000) -- (233.2000,41.7000) -- (233.3000,41.9000) -- (233.4000,42.2000) -- (233.5000,42.6000) -- (233.6000,43.0000) -- (233.7000,43.4000) -- (233.8000,43.9000) -- (233.8000,44.5000) -- (233.9000,45.1000) -- (234.0000,45.7000) -- (234.1000,46.4000) -- (234.2000,47.2000) -- (234.3000,48.0000) -- (234.4000,48.8000) -- (234.5000,49.7000) -- (234.5000,50.6000) -- (234.6000,51.6000) -- (234.7000,52.6000) -- (234.8000,53.6000) -- (234.9000,54.6000) -- (235.0000,55.7000) -- (235.1000,56.8000) -- (235.1000,57.9000) -- (235.2000,59.0000) -- (235.3000,60.2000) -- (235.4000,61.3000) -- (235.5000,62.4000) -- (235.6000,63.6000) -- (235.7000,64.7000) -- (235.8000,65.8000) -- (235.8000,67.0000) -- (235.9000,68.1000) -- (236.0000,69.2000) -- (236.1000,70.2000) -- (236.2000,71.3000) -- (236.3000,72.3000) -- (236.4000,73.3000) -- (236.5000,74.2000) -- (236.5000,75.1000) -- (236.6000,76.0000) -- (236.7000,76.9000) -- (236.8000,77.7000) -- (236.9000,78.5000) -- (237.0000,79.2000) -- (237.1000,79.9000) -- (237.1000,80.5000) -- (237.2000,81.1000) -- (237.3000,81.7000) -- (237.4000,82.2000) -- (237.5000,82.7000) -- (237.6000,83.1000) -- (237.7000,83.5000) -- (237.8000,83.9000) -- (237.8000,84.2000) -- (237.9000,84.4000) -- (238.0000,84.7000) -- (238.1000,84.9000) -- (238.2000,85.1000) -- (238.3000,85.2000) -- (238.4000,85.3000) -- (238.4000,85.4000) -- (238.5000,85.5000) -- (238.6000,85.5000) -- (238.7000,85.5000) -- (238.8000,85.5000) -- (238.9000,85.5000) -- (239.0000,85.5000) -- (239.1000,85.5000) -- (239.1000,85.4000) -- (239.2000,85.4000) -- (239.3000,85.4000) -- (239.4000,85.3000) -- (239.5000,85.3000) -- (239.6000,85.2000) -- (239.7000,85.2000) -- (239.7000,85.1000) -- (239.8000,85.1000) -- (239.9000,85.1000) -- (240.0000,85.1000) -- (240.1000,85.1000) -- (240.2000,85.1000) -- (240.3000,85.1000) -- (240.4000,85.1000) -- (240.4000,85.2000) -- (240.5000,85.2000) -- (240.6000,85.3000) -- (240.7000,85.4000) -- (240.8000,85.4000) -- (240.9000,85.5000) -- (241.0000,85.6000) -- (241.0000,85.7000) -- (241.1000,85.8000) -- (241.2000,85.9000) -- (241.3000,86.0000) -- (241.4000,86.1000) -- (241.5000,86.2000) -- (241.6000,86.3000) -- (241.7000,86.4000) -- (241.7000,86.5000) -- (241.8000,86.5000) -- (241.9000,86.6000) -- (242.0000,86.6000) -- (242.1000,86.6000) -- (242.2000,86.6000) -- (242.3000,86.5000) -- (242.4000,86.5000) -- (242.4000,86.4000) -- (242.5000,86.3000) -- (242.6000,86.1000) -- (242.7000,85.9000) -- (242.8000,85.7000) -- (242.9000,85.4000) -- (243.0000,85.1000) -- (243.0000,84.7000) -- (243.1000,84.3000) -- (243.2000,83.9000) -- (243.3000,83.4000) -- (243.4000,82.9000) -- (243.5000,82.4000) -- (243.6000,81.8000) -- (243.7000,81.1000) -- (243.7000,80.4000) -- (243.8000,79.7000) -- (243.9000,78.9000) -- (244.0000,78.1000) -- (244.1000,77.3000) -- (244.2000,76.4000) -- (244.3000,75.5000) -- (244.3000,74.5000) -- (244.4000,73.6000) -- (244.5000,72.6000) -- (244.6000,71.6000) -- (244.7000,70.5000) -- (244.8000,69.5000) -- (244.9000,68.4000) -- (245.0000,67.3000) -- (245.0000,66.2000) -- (245.1000,65.1000) -- (245.2000,64.0000) -- (245.3000,62.9000) -- (245.4000,61.9000) -- (245.5000,60.8000) -- (245.6000,59.7000) -- (245.7000,58.7000) -- (245.7000,57.7000) -- (245.8000,56.7000) -- (245.9000,55.7000) -- (246.0000,54.7000) -- (246.1000,53.8000) -- (246.2000,52.9000) -- (246.3000,52.1000) -- (246.3000,51.3000) -- (246.4000,50.5000) -- (246.5000,49.8000) -- (246.6000,49.1000) -- (246.7000,48.5000) -- (246.8000,47.9000) -- (246.9000,47.3000) -- (247.0000,46.9000) -- (247.0000,46.4000) -- (247.1000,46.0000) -- (247.2000,45.7000) -- (247.3000,45.4000) -- (247.4000,45.1000) -- (247.5000,44.9000) -- (247.6000,44.8000) -- (247.6000,44.7000) -- (247.7000,44.6000) -- (247.8000,44.6000) -- (247.9000,44.6000) -- (248.0000,44.6000) -- (248.1000,44.7000) -- (248.2000,44.8000) -- (248.3000,45.0000) -- (248.3000,45.1000) -- (248.4000,45.3000) -- (248.5000,45.5000) -- (248.6000,45.7000) -- (248.7000,46.0000) -- (248.8000,46.2000) -- (248.9000,46.5000) -- (248.9000,46.7000) -- (249.0000,47.0000) -- (249.1000,47.3000) -- (249.2000,47.5000) -- (249.3000,47.7000) -- (249.4000,48.0000) -- (249.5000,48.2000) -- (249.6000,48.4000) -- (249.6000,48.6000) -- (249.7000,48.8000) -- (249.8000,48.9000) -- (249.9000,49.0000) -- (250.0000,49.1000) -- (250.1000,49.2000) -- (250.2000,49.2000) -- (250.3000,49.2000) -- (250.3000,49.2000) -- (250.4000,49.1000) -- (250.5000,49.0000) -- (250.6000,48.9000) -- (250.7000,48.8000) -- (250.8000,48.6000) -- (250.9000,48.4000) -- (250.9000,48.2000) -- (251.0000,47.9000) -- (251.1000,47.7000) -- (251.2000,47.4000) -- (251.3000,47.1000) -- (251.4000,46.7000) -- (251.5000,46.4000) -- (251.6000,46.0000) -- (251.6000,45.7000) -- (251.7000,45.3000) -- (251.8000,44.9000) -- (251.9000,44.5000);



  \end{scope}
  \begin{scope}[scale=1.006,draw=black,line join=bevel,line cap=rect,line width=0.800pt]
  \end{scope}
  \begin{scope}[scale=1.006,draw=black,line join=bevel,line cap=rect,line width=0.800pt]
  \end{scope}
  \begin{scope}[scale=1.006,draw=black,line join=round,line cap=round,line width=0.480pt]
    \path[draw] (165.5000,13.5000) -- (165.5000,95.5000) -- (251.5000,95.5000) -- (251.5000,13.5000) -- (165.5000,13.5000);



  \end{scope}
  \begin{scope}[scale=1.006,draw=ca0a0a4,dash pattern=on 0.40pt off 0.80pt,line join=round,line cap=round,line width=0.400pt]
    \path[draw] (56.5000,194.5000) -- (251.5000,194.5000);



  \end{scope}
  \begin{scope}[scale=1.006,draw=black,line join=round,line cap=round,line width=0.480pt]
    \path[draw] (56.5000,194.5000) -- (59.5000,194.5000);



    \path[draw] (251.5000,194.5000) -- (248.5000,194.5000);



  \end{scope}
  \begin{scope}[scale=1.006,draw=black,line join=bevel,line cap=rect,line width=0.800pt]
  \end{scope}
  \begin{scope}[cm={{1.00588,0.0,0.0,1.00588,(39.2294,199.165)}},draw=black,line join=bevel,line cap=rect,line width=0.800pt]
  \end{scope}
  \begin{scope}[cm={{1.00588,0.0,0.0,1.00588,(39.2294,199.165)}},draw=black,line join=bevel,line cap=rect,line width=0.800pt]
  \end{scope}
  \begin{scope}[cm={{1.00588,0.0,0.0,1.00588,(39.2294,199.165)}},draw=black,line join=bevel,line cap=rect,line width=0.800pt]
  \end{scope}
  \begin{scope}[cm={{1.00588,0.0,0.0,1.00588,(39.2294,199.165)}},draw=black,line join=bevel,line cap=rect,line width=0.800pt]
  \end{scope}
  \begin{scope}[cm={{1.00588,0.0,0.0,1.00588,(39.2294,199.165)}},draw=black,line join=bevel,line cap=rect,line width=0.800pt]
  \end{scope}
  \begin{scope}[cm={{1.00588,0.0,0.0,1.00588,(39.2294,199.165)}},draw=black,line join=bevel,line cap=rect,line width=0.800pt]
    \path[fill=black] (0.0000,0.0000) node[above right] (text658) {27};



  \end{scope}
  \begin{scope}[cm={{1.00588,0.0,0.0,1.00588,(39.2294,199.165)}},draw=black,line join=bevel,line cap=rect,line width=0.800pt]
  \end{scope}
  \begin{scope}[scale=1.006,draw=black,line join=bevel,line cap=rect,line width=0.800pt]
  \end{scope}
  \begin{scope}[scale=1.006,draw=ca0a0a4,dash pattern=on 0.40pt off 0.80pt,line join=round,line cap=round,line width=0.400pt]
    \path[draw] (56.5000,171.5000) -- (251.5000,171.5000);



  \end{scope}
  \begin{scope}[scale=1.006,draw=black,line join=round,line cap=round,line width=0.480pt]
    \path[draw] (56.5000,171.5000) -- (59.5000,171.5000);



    \path[draw] (251.5000,171.5000) -- (248.5000,171.5000);



  \end{scope}
  \begin{scope}[scale=1.006,draw=black,line join=bevel,line cap=rect,line width=0.800pt]
  \end{scope}
  \begin{scope}[cm={{1.00588,0.0,0.0,1.00588,(40.2353,177.035)}},draw=black,line join=bevel,line cap=rect,line width=0.800pt]
  \end{scope}
  \begin{scope}[cm={{1.00588,0.0,0.0,1.00588,(40.2353,177.035)}},draw=black,line join=bevel,line cap=rect,line width=0.800pt]
  \end{scope}
  \begin{scope}[cm={{1.00588,0.0,0.0,1.00588,(40.2353,177.035)}},draw=black,line join=bevel,line cap=rect,line width=0.800pt]
  \end{scope}
  \begin{scope}[cm={{1.00588,0.0,0.0,1.00588,(40.2353,177.035)}},draw=black,line join=bevel,line cap=rect,line width=0.800pt]
  \end{scope}
  \begin{scope}[cm={{1.00588,0.0,0.0,1.00588,(40.2353,177.035)}},draw=black,line join=bevel,line cap=rect,line width=0.800pt]
  \end{scope}
  \begin{scope}[cm={{1.00588,0.0,0.0,1.00588,(40.2353,177.035)}},draw=black,line join=bevel,line cap=rect,line width=0.800pt]
    \path[fill=black] (0.0000,0.0000) node[above right] (text688) {31};



  \end{scope}
  \begin{scope}[cm={{1.00588,0.0,0.0,1.00588,(40.2353,177.035)}},draw=black,line join=bevel,line cap=rect,line width=0.800pt]
  \end{scope}
  \begin{scope}[scale=1.006,draw=black,line join=bevel,line cap=rect,line width=0.800pt]
  \end{scope}
  \begin{scope}[scale=1.006,draw=ca0a0a4,dash pattern=on 0.40pt off 0.80pt,line join=round,line cap=round,line width=0.400pt]
    \path[draw] (56.5000,149.5000) -- (251.5000,149.5000);



  \end{scope}
  \begin{scope}[scale=1.006,draw=black,line join=round,line cap=round,line width=0.480pt]
    \path[draw] (56.5000,149.5000) -- (59.5000,149.5000);



    \path[draw] (251.5000,149.5000) -- (248.5000,149.5000);



  \end{scope}
  \begin{scope}[scale=1.006,draw=black,line join=bevel,line cap=rect,line width=0.800pt]
  \end{scope}
  \begin{scope}[cm={{1.00588,0.0,0.0,1.00588,(40.2353,153.9)}},draw=black,line join=bevel,line cap=rect,line width=0.800pt]
  \end{scope}
  \begin{scope}[cm={{1.00588,0.0,0.0,1.00588,(40.2353,153.9)}},draw=black,line join=bevel,line cap=rect,line width=0.800pt]
  \end{scope}
  \begin{scope}[cm={{1.00588,0.0,0.0,1.00588,(40.2353,153.9)}},draw=black,line join=bevel,line cap=rect,line width=0.800pt]
  \end{scope}
  \begin{scope}[cm={{1.00588,0.0,0.0,1.00588,(40.2353,153.9)}},draw=black,line join=bevel,line cap=rect,line width=0.800pt]
  \end{scope}
  \begin{scope}[cm={{1.00588,0.0,0.0,1.00588,(40.2353,153.9)}},draw=black,line join=bevel,line cap=rect,line width=0.800pt]
  \end{scope}
  \begin{scope}[cm={{1.00588,0.0,0.0,1.00588,(40.2353,153.9)}},draw=black,line join=bevel,line cap=rect,line width=0.800pt]
    \path[fill=black] (0.0000,0.0000) node[above right] (text718) {35};



  \end{scope}
  \begin{scope}[cm={{1.00588,0.0,0.0,1.00588,(40.2353,153.9)}},draw=black,line join=bevel,line cap=rect,line width=0.800pt]
  \end{scope}
  \begin{scope}[scale=1.006,draw=black,line join=bevel,line cap=rect,line width=0.800pt]
  \end{scope}
  \begin{scope}[scale=1.006,draw=ca0a0a4,dash pattern=on 0.40pt off 0.80pt,line join=round,line cap=round,line width=0.400pt]
    \path[draw] (56.5000,126.5000) -- (195.5000,126.5000);



    \path[draw] (246.5000,126.5000) -- (251.5000,126.5000);



  \end{scope}
  \begin{scope}[scale=1.006,draw=black,line join=round,line cap=round,line width=0.480pt]
    \path[draw] (56.5000,126.5000) -- (59.5000,126.5000);



    \path[draw] (251.5000,126.5000) -- (248.5000,126.5000);



  \end{scope}
  \begin{scope}[scale=1.006,draw=black,line join=bevel,line cap=rect,line width=0.800pt]
  \end{scope}
  \begin{scope}[cm={{1.00588,0.0,0.0,1.00588,(39.2294,131.771)}},draw=black,line join=bevel,line cap=rect,line width=0.800pt]
  \end{scope}
  \begin{scope}[cm={{1.00588,0.0,0.0,1.00588,(39.2294,131.771)}},draw=black,line join=bevel,line cap=rect,line width=0.800pt]
  \end{scope}
  \begin{scope}[cm={{1.00588,0.0,0.0,1.00588,(39.2294,131.771)}},draw=black,line join=bevel,line cap=rect,line width=0.800pt]
  \end{scope}
  \begin{scope}[cm={{1.00588,0.0,0.0,1.00588,(39.2294,131.771)}},draw=black,line join=bevel,line cap=rect,line width=0.800pt]
  \end{scope}
  \begin{scope}[cm={{1.00588,0.0,0.0,1.00588,(39.2294,131.771)}},draw=black,line join=bevel,line cap=rect,line width=0.800pt]
  \end{scope}
  \begin{scope}[cm={{1.00588,0.0,0.0,1.00588,(39.2294,131.771)}},draw=black,line join=bevel,line cap=rect,line width=0.800pt]
    \path[fill=black] (0.0000,0.0000) node[above right] (text750) {39};



  \end{scope}
  \begin{scope}[cm={{1.00588,0.0,0.0,1.00588,(39.2294,131.771)}},draw=black,line join=bevel,line cap=rect,line width=0.800pt]
  \end{scope}
  \begin{scope}[scale=1.006,draw=black,line join=bevel,line cap=rect,line width=0.800pt]
  \end{scope}
  \begin{scope}[scale=1.006,draw=ca0a0a4,dash pattern=on 0.40pt off 0.80pt,line join=round,line cap=round,line width=0.400pt]
    \path[draw] (56.5000,200.5000) -- (56.5000,115.5000);



  \end{scope}
  \begin{scope}[scale=1.006,draw=black,line join=round,line cap=round,line width=0.480pt]
    \path[draw] (56.5000,200.5000) -- (56.5000,198.5000);



    \path[draw] (56.5000,115.5000) -- (56.5000,118.5000);



  \end{scope}
  \begin{scope}[scale=1.006,draw=black,line join=bevel,line cap=rect,line width=0.800pt]
  \end{scope}
  \begin{scope}[cm={{1.00588,0.0,0.0,1.00588,(53.3118,217.271)}},draw=black,line join=bevel,line cap=rect,line width=0.800pt]
  \end{scope}
  \begin{scope}[cm={{1.00588,0.0,0.0,1.00588,(53.3118,217.271)}},draw=black,line join=bevel,line cap=rect,line width=0.800pt]
  \end{scope}
  \begin{scope}[cm={{1.00588,0.0,0.0,1.00588,(53.3118,217.271)}},draw=black,line join=bevel,line cap=rect,line width=0.800pt]
  \end{scope}
  \begin{scope}[cm={{1.00588,0.0,0.0,1.00588,(53.3118,217.271)}},draw=black,line join=bevel,line cap=rect,line width=0.800pt]
  \end{scope}
  \begin{scope}[cm={{1.00588,0.0,0.0,1.00588,(53.3118,217.271)}},draw=black,line join=bevel,line cap=rect,line width=0.800pt]
  \end{scope}
  \begin{scope}[cm={{1.00588,0.0,0.0,1.00588,(53.3118,217.271)}},draw=black,line join=bevel,line cap=rect,line width=0.800pt]
    \path[fill=black] (0.0000,0.0000) node[above right] (text780) {0};



  \end{scope}
  \begin{scope}[cm={{1.00588,0.0,0.0,1.00588,(53.3118,217.271)}},draw=black,line join=bevel,line cap=rect,line width=0.800pt]
  \end{scope}
  \begin{scope}[scale=1.006,draw=black,line join=bevel,line cap=rect,line width=0.800pt]
  \end{scope}
  \begin{scope}[scale=1.006,draw=ca0a0a4,dash pattern=on 0.40pt off 0.80pt,line join=round,line cap=round,line width=0.400pt]
    \path[draw] (85.5000,200.5000) -- (85.5000,115.5000);



  \end{scope}
  \begin{scope}[scale=1.006,draw=black,line join=round,line cap=round,line width=0.480pt]
    \path[draw] (85.5000,200.5000) -- (85.5000,198.5000);



    \path[draw] (85.5000,115.5000) -- (85.5000,118.5000);



  \end{scope}
  \begin{scope}[scale=1.006,draw=black,line join=bevel,line cap=rect,line width=0.800pt]
  \end{scope}
  \begin{scope}[cm={{1.00588,0.0,0.0,1.00588,(82.4824,217.271)}},draw=black,line join=bevel,line cap=rect,line width=0.800pt]
  \end{scope}
  \begin{scope}[cm={{1.00588,0.0,0.0,1.00588,(82.4824,217.271)}},draw=black,line join=bevel,line cap=rect,line width=0.800pt]
  \end{scope}
  \begin{scope}[cm={{1.00588,0.0,0.0,1.00588,(82.4824,217.271)}},draw=black,line join=bevel,line cap=rect,line width=0.800pt]
  \end{scope}
  \begin{scope}[cm={{1.00588,0.0,0.0,1.00588,(82.4824,217.271)}},draw=black,line join=bevel,line cap=rect,line width=0.800pt]
  \end{scope}
  \begin{scope}[cm={{1.00588,0.0,0.0,1.00588,(82.4824,217.271)}},draw=black,line join=bevel,line cap=rect,line width=0.800pt]
  \end{scope}
  \begin{scope}[cm={{1.00588,0.0,0.0,1.00588,(82.4824,217.271)}},draw=black,line join=bevel,line cap=rect,line width=0.800pt]
    \path[fill=black] (0.0000,0.0000) node[above right] (text810) {1};



  \end{scope}
  \begin{scope}[cm={{1.00588,0.0,0.0,1.00588,(82.4824,217.271)}},draw=black,line join=bevel,line cap=rect,line width=0.800pt]
  \end{scope}
  \begin{scope}[scale=1.006,draw=black,line join=bevel,line cap=rect,line width=0.800pt]
  \end{scope}
  \begin{scope}[scale=1.006,draw=ca0a0a4,dash pattern=on 0.40pt off 0.80pt,line join=round,line cap=round,line width=0.400pt]
    \path[draw] (114.5000,200.5000) -- (114.5000,115.5000);



  \end{scope}
  \begin{scope}[scale=1.006,draw=black,line join=round,line cap=round,line width=0.480pt]
    \path[draw] (114.5000,200.5000) -- (114.5000,198.5000);



    \path[draw] (114.5000,115.5000) -- (114.5000,118.5000);



  \end{scope}
  \begin{scope}[scale=1.006,draw=black,line join=bevel,line cap=rect,line width=0.800pt]
  \end{scope}
  \begin{scope}[cm={{1.00588,0.0,0.0,1.00588,(111.653,217.271)}},draw=black,line join=bevel,line cap=rect,line width=0.800pt]
  \end{scope}
  \begin{scope}[cm={{1.00588,0.0,0.0,1.00588,(111.653,217.271)}},draw=black,line join=bevel,line cap=rect,line width=0.800pt]
  \end{scope}
  \begin{scope}[cm={{1.00588,0.0,0.0,1.00588,(111.653,217.271)}},draw=black,line join=bevel,line cap=rect,line width=0.800pt]
  \end{scope}
  \begin{scope}[cm={{1.00588,0.0,0.0,1.00588,(111.653,217.271)}},draw=black,line join=bevel,line cap=rect,line width=0.800pt]
  \end{scope}
  \begin{scope}[cm={{1.00588,0.0,0.0,1.00588,(111.653,217.271)}},draw=black,line join=bevel,line cap=rect,line width=0.800pt]
  \end{scope}
  \begin{scope}[cm={{1.00588,0.0,0.0,1.00588,(111.653,217.271)}},draw=black,line join=bevel,line cap=rect,line width=0.800pt]
    \path[fill=black] (0.0000,0.0000) node[above right] (text840) {2};



  \end{scope}
  \begin{scope}[cm={{1.00588,0.0,0.0,1.00588,(111.653,217.271)}},draw=black,line join=bevel,line cap=rect,line width=0.800pt]
  \end{scope}
  \begin{scope}[scale=1.006,draw=black,line join=bevel,line cap=rect,line width=0.800pt]
  \end{scope}
  \begin{scope}[scale=1.006,draw=ca0a0a4,dash pattern=on 0.40pt off 0.80pt,line join=round,line cap=round,line width=0.400pt]
    \path[draw] (143.5000,200.5000) -- (143.5000,115.5000);



  \end{scope}
  \begin{scope}[scale=1.006,draw=black,line join=round,line cap=round,line width=0.480pt]
    \path[draw] (143.5000,200.5000) -- (143.5000,198.5000);



    \path[draw] (143.5000,115.5000) -- (143.5000,118.5000);



  \end{scope}
  \begin{scope}[scale=1.006,draw=black,line join=bevel,line cap=rect,line width=0.800pt]
  \end{scope}
  \begin{scope}[cm={{1.00588,0.0,0.0,1.00588,(142.332,217.271)}},draw=black,line join=bevel,line cap=rect,line width=0.800pt]
  \end{scope}
  \begin{scope}[cm={{1.00588,0.0,0.0,1.00588,(142.332,217.271)}},draw=black,line join=bevel,line cap=rect,line width=0.800pt]
  \end{scope}
  \begin{scope}[cm={{1.00588,0.0,0.0,1.00588,(142.332,217.271)}},draw=black,line join=bevel,line cap=rect,line width=0.800pt]
  \end{scope}
  \begin{scope}[cm={{1.00588,0.0,0.0,1.00588,(142.332,217.271)}},draw=black,line join=bevel,line cap=rect,line width=0.800pt]
  \end{scope}
  \begin{scope}[cm={{1.00588,0.0,0.0,1.00588,(142.332,217.271)}},draw=black,line join=bevel,line cap=rect,line width=0.800pt]
  \end{scope}
  \begin{scope}[cm={{1.00588,0.0,0.0,1.00588,(142.332,217.271)}},draw=black,line join=bevel,line cap=rect,line width=0.800pt]
    \path[fill=black] (0.0000,0.0000) node[above right] (text870) {3};



  \end{scope}
  \begin{scope}[cm={{1.00588,0.0,0.0,1.00588,(142.332,217.271)}},draw=black,line join=bevel,line cap=rect,line width=0.800pt]
  \end{scope}
  \begin{scope}[scale=1.006,draw=black,line join=bevel,line cap=rect,line width=0.800pt]
  \end{scope}
  \begin{scope}[scale=1.006,draw=ca0a0a4,dash pattern=on 0.40pt off 0.80pt,line join=round,line cap=round,line width=0.400pt]
    \path[draw] (172.5000,200.5000) -- (172.5000,115.5000);



  \end{scope}
  \begin{scope}[scale=1.006,draw=black,line join=round,line cap=round,line width=0.480pt]
    \path[draw] (172.5000,200.5000) -- (172.5000,198.5000);



    \path[draw] (172.5000,115.5000) -- (172.5000,118.5000);



  \end{scope}
  \begin{scope}[scale=1.006,draw=black,line join=bevel,line cap=rect,line width=0.800pt]
  \end{scope}
  \begin{scope}[cm={{1.00588,0.0,0.0,1.00588,(171.503,217.271)}},draw=black,line join=bevel,line cap=rect,line width=0.800pt]
  \end{scope}
  \begin{scope}[cm={{1.00588,0.0,0.0,1.00588,(171.503,217.271)}},draw=black,line join=bevel,line cap=rect,line width=0.800pt]
  \end{scope}
  \begin{scope}[cm={{1.00588,0.0,0.0,1.00588,(171.503,217.271)}},draw=black,line join=bevel,line cap=rect,line width=0.800pt]
  \end{scope}
  \begin{scope}[cm={{1.00588,0.0,0.0,1.00588,(171.503,217.271)}},draw=black,line join=bevel,line cap=rect,line width=0.800pt]
  \end{scope}
  \begin{scope}[cm={{1.00588,0.0,0.0,1.00588,(171.503,217.271)}},draw=black,line join=bevel,line cap=rect,line width=0.800pt]
  \end{scope}
  \begin{scope}[cm={{1.00588,0.0,0.0,1.00588,(171.503,217.271)}},draw=black,line join=bevel,line cap=rect,line width=0.800pt]
    \path[fill=black] (0.0000,0.0000) node[above right] (text900) {4};



  \end{scope}
  \begin{scope}[cm={{1.00588,0.0,0.0,1.00588,(171.503,217.271)}},draw=black,line join=bevel,line cap=rect,line width=0.800pt]
  \end{scope}
  \begin{scope}[scale=1.006,draw=black,line join=bevel,line cap=rect,line width=0.800pt]
  \end{scope}
  \begin{scope}[scale=1.006,draw=ca0a0a4,dash pattern=on 0.40pt off 0.80pt,line join=round,line cap=round,line width=0.400pt]
    \path[draw] (201.5000,200.5000) -- (201.5000,129.5000);



    \path[draw] (201.5000,121.5000) -- (201.5000,115.5000);



  \end{scope}
  \begin{scope}[scale=1.006,draw=black,line join=round,line cap=round,line width=0.480pt]
    \path[draw] (201.5000,200.5000) -- (201.5000,198.5000);



    \path[draw] (201.5000,115.5000) -- (201.5000,118.5000);



  \end{scope}
  \begin{scope}[scale=1.006,draw=black,line join=bevel,line cap=rect,line width=0.800pt]
  \end{scope}
  \begin{scope}[cm={{1.00588,0.0,0.0,1.00588,(200.674,217.271)}},draw=black,line join=bevel,line cap=rect,line width=0.800pt]
  \end{scope}
  \begin{scope}[cm={{1.00588,0.0,0.0,1.00588,(200.674,217.271)}},draw=black,line join=bevel,line cap=rect,line width=0.800pt]
  \end{scope}
  \begin{scope}[cm={{1.00588,0.0,0.0,1.00588,(200.674,217.271)}},draw=black,line join=bevel,line cap=rect,line width=0.800pt]
  \end{scope}
  \begin{scope}[cm={{1.00588,0.0,0.0,1.00588,(200.674,217.271)}},draw=black,line join=bevel,line cap=rect,line width=0.800pt]
  \end{scope}
  \begin{scope}[cm={{1.00588,0.0,0.0,1.00588,(200.674,217.271)}},draw=black,line join=bevel,line cap=rect,line width=0.800pt]
  \end{scope}
  \begin{scope}[cm={{1.00588,0.0,0.0,1.00588,(200.674,217.271)}},draw=black,line join=bevel,line cap=rect,line width=0.800pt]
    \path[fill=black] (0.0000,0.0000) node[above right] (text932) {5};



  \end{scope}
  \begin{scope}[cm={{1.00588,0.0,0.0,1.00588,(200.674,217.271)}},draw=black,line join=bevel,line cap=rect,line width=0.800pt]
  \end{scope}
  \begin{scope}[scale=1.006,draw=black,line join=bevel,line cap=rect,line width=0.800pt]
  \end{scope}
  \begin{scope}[scale=1.006,draw=ca0a0a4,dash pattern=on 0.40pt off 0.80pt,line join=round,line cap=round,line width=0.400pt]
    \path[draw] (231.5000,200.5000) -- (231.5000,129.5000);



    \path[draw] (231.5000,121.5000) -- (231.5000,115.5000);



  \end{scope}
  \begin{scope}[scale=1.006,draw=black,line join=round,line cap=round,line width=0.480pt]
    \path[draw] (231.5000,200.5000) -- (231.5000,198.5000);



    \path[draw] (231.5000,115.5000) -- (231.5000,118.5000);



  \end{scope}
  \begin{scope}[scale=1.006,draw=black,line join=bevel,line cap=rect,line width=0.800pt]
  \end{scope}
  \begin{scope}[cm={{1.00588,0.0,0.0,1.00588,(229.341,217.271)}},draw=black,line join=bevel,line cap=rect,line width=0.800pt]
  \end{scope}
  \begin{scope}[cm={{1.00588,0.0,0.0,1.00588,(229.341,217.271)}},draw=black,line join=bevel,line cap=rect,line width=0.800pt]
  \end{scope}
  \begin{scope}[cm={{1.00588,0.0,0.0,1.00588,(229.341,217.271)}},draw=black,line join=bevel,line cap=rect,line width=0.800pt]
  \end{scope}
  \begin{scope}[cm={{1.00588,0.0,0.0,1.00588,(229.341,217.271)}},draw=black,line join=bevel,line cap=rect,line width=0.800pt]
  \end{scope}
  \begin{scope}[cm={{1.00588,0.0,0.0,1.00588,(229.341,217.271)}},draw=black,line join=bevel,line cap=rect,line width=0.800pt]
  \end{scope}
  \begin{scope}[cm={{1.00588,0.0,0.0,1.00588,(229.341,217.271)}},draw=black,line join=bevel,line cap=rect,line width=0.800pt]
    \path[fill=black] (0.0000,0.0000) node[above right] (text964) {6};



  \end{scope}
  \begin{scope}[cm={{1.00588,0.0,0.0,1.00588,(229.341,217.271)}},draw=black,line join=bevel,line cap=rect,line width=0.800pt]
  \end{scope}
  \begin{scope}[scale=1.006,draw=black,line join=bevel,line cap=rect,line width=0.800pt]
  \end{scope}
  \begin{scope}[scale=1.006,draw=black,line join=round,line cap=round,line width=0.480pt]
    \path[draw] (56.5000,115.5000) -- (56.5000,200.5000) -- (251.5000,200.5000) -- (251.5000,115.5000) -- (56.5000,115.5000);



  \end{scope}
  \begin{scope}[scale=1.006,draw=black,line join=bevel,line cap=rect,line width=0.800pt]
  \end{scope}
  \begin{scope}[scale=1.006,draw=black,line join=bevel,line cap=rect,line width=0.800pt]
  \end{scope}
  \begin{scope}[scale=1.006,draw=c00ff00,line join=round,line cap=round,line width=0.480pt]
    \path[draw,even odd rule] (215.5000,125.5000) -- (241.5000,125.5000);



  \end{scope}
  \begin{scope}[scale=1.006,fill=cffffff]
    \path[fill,rounded corners=0.0000cm] (231.0000,120.0000) rectangle (242.0000,136.0000);



  \end{scope}
  \begin{scope}[scale=1.006,draw=black,line join=bevel,line cap=rect,line width=0.800pt]
  \end{scope}
  \begin{scope}[scale=1.006,draw=black,line join=bevel,line cap=rect,line width=0.800pt]
  \end{scope}
  \begin{scope}[scale=1.006,draw=black,line join=round,line cap=round,line width=0.800pt]
    \path[draw] (231.5000,135.5000) -- (231.5000,119.5000) -- (242.5000,119.5000) -- (242.5000,135.5000) -- (231.5000,135.5000);



  \end{scope}
  \begin{scope}[scale=1.006,draw=black,line join=bevel,line cap=rect,line width=0.800pt]
  \end{scope}
  \begin{scope}[cm={{1.00588,0.0,0.0,1.00588,(236.382,132.776)}},draw=black,line join=bevel,line cap=rect,line width=0.800pt]
  \end{scope}
  \begin{scope}[cm={{1.00588,0.0,0.0,1.00588,(236.382,132.776)}},draw=black,line join=bevel,line cap=rect,line width=0.800pt]
  \end{scope}
  \begin{scope}[cm={{1.00588,0.0,0.0,1.00588,(236.382,132.776)}},draw=black,line join=bevel,line cap=rect,line width=0.800pt]
  \end{scope}
  \begin{scope}[cm={{1.00588,0.0,0.0,1.00588,(236.382,132.776)}},draw=black,line join=bevel,line cap=rect,line width=0.800pt]
  \end{scope}
  \begin{scope}[cm={{1.00588,0.0,0.0,1.00588,(236.382,132.776)}},draw=black,line join=bevel,line cap=rect,line width=0.800pt]
  \end{scope}
  \begin{scope}[cm={{1.00588,0.0,0.0,1.00588,(236.79555,131.53535)}},draw=black,line join=bevel,line cap=rect,line width=0.800pt]
    \path[fill=black] (0.0000,0.0000) node[above right] (text1004) {\label{fig:ener-dyn-i}i};



  \end{scope}
  \begin{scope}[cm={{1.00588,0.0,0.0,1.00588,(236.382,132.776)}},draw=black,line join=bevel,line cap=rect,line width=0.800pt]
  \end{scope}
  \begin{scope}[scale=1.006,draw=black,line join=bevel,line cap=rect,line width=0.800pt]
  \end{scope}
  \begin{scope}[scale=1.006,draw=black,line join=bevel,line cap=rect,line width=0.800pt]
  \end{scope}
  \begin{scope}[scale=1.006,draw=black,line join=bevel,line cap=rect,line width=0.800pt]
  \end{scope}
  \begin{scope}[scale=1.006,draw=black,line join=round,line cap=round,line width=0.480pt]
    \path[draw] (56.1000,127.5000) -- (56.1000,127.5000) -- (56.3000,142.6000) -- (56.5000,149.0000) -- (56.7000,148.4000) -- (56.9000,141.3000) -- (57.1000,134.7000) -- (57.3000,131.3000) -- (57.5000,130.4000) -- (57.7000,131.2000) -- (57.9000,132.8000) -- (58.1000,135.0000) -- (58.3000,137.3000) -- (58.4000,139.1000) -- (58.6000,140.1000) -- (58.8000,140.4000) -- (59.0000,140.4000) -- (59.2000,140.1000) -- (59.4000,139.9000) -- (59.6000,139.7000) -- (59.8000,139.6000) -- (60.0000,139.5000) -- (60.2000,139.5000) -- (60.4000,139.5000) -- (60.6000,139.5000) -- (60.8000,139.5000) -- (61.0000,139.5000) -- (61.2000,139.5000) -- (61.4000,139.5000) -- (61.6000,139.5000) -- (61.8000,139.5000) -- (62.0000,139.5000) -- (62.2000,139.5000) -- (62.4000,139.5000) -- (62.6000,139.5000) -- (62.8000,139.6000) -- (63.0000,140.3000) -- (63.1000,141.9000) -- (63.3000,143.7000) -- (63.5000,145.6000) -- (63.7000,147.5000) -- (63.9000,149.3000) -- (64.1000,151.1000) -- (64.3000,152.8000) -- (64.5000,154.5000) -- (64.7000,156.1000) -- (64.9000,157.6000) -- (65.1000,159.1000) -- (65.3000,160.5000) -- (65.5000,161.7000) -- (65.7000,162.7000) -- (65.9000,163.6000) -- (66.1000,164.2000) -- (66.3000,164.6000) -- (66.5000,164.7000) -- (66.7000,164.5000) -- (66.9000,164.0000) -- (67.1000,163.2000) -- (67.3000,162.2000) -- (67.5000,161.0000) -- (67.7000,159.7000) -- (67.9000,158.2000) -- (68.0000,156.6000) -- (68.2000,154.9000) -- (68.4000,153.2000) -- (68.6000,151.5000) -- (68.8000,149.7000) -- (69.0000,147.3000) -- (69.2000,144.6000) -- (69.4000,142.0000) -- (69.6000,140.1000) -- (69.8000,139.0000) -- (70.0000,138.5000) -- (70.2000,138.5000) -- (70.4000,138.7000) -- (70.6000,139.0000) -- (70.8000,139.2000) -- (71.0000,139.4000) -- (71.2000,139.5000) -- (71.4000,139.5000) -- (71.6000,139.6000) -- (71.8000,139.5000) -- (72.0000,139.5000) -- (72.2000,139.5000) -- (72.4000,139.5000) -- (72.6000,139.5000) -- (72.7000,139.5000) -- (72.9000,139.5000) -- (73.1000,139.5000) -- (73.3000,139.5000) -- (73.5000,139.4000) -- (73.7000,140.8000) -- (73.9000,142.7000) -- (74.1000,142.4000) -- (74.3000,141.0000) -- (74.5000,139.2000) -- (74.7000,137.3000) -- (74.9000,135.6000) -- (75.1000,134.1000) -- (75.3000,132.8000) -- (75.5000,131.6000) -- (75.7000,130.7000) -- (75.9000,129.8000) -- (76.1000,129.1000) -- (76.3000,128.4000) -- (76.5000,127.9000) -- (76.7000,127.4000) -- (76.9000,127.0000) -- (77.1000,126.7000) -- (77.3000,126.4000) -- (77.5000,126.3000) -- (77.6000,126.2000) -- (77.8000,126.2000) -- (78.0000,126.3000) -- (78.2000,126.5000) -- (78.4000,126.7000) -- (78.6000,127.1000) -- (78.8000,127.5000) -- (79.0000,128.0000) -- (79.2000,128.6000) -- (79.4000,129.3000) -- (79.6000,130.0000) -- (79.8000,130.9000) -- (80.0000,131.9000) -- (80.2000,132.9000) -- (80.4000,134.1000) -- (80.6000,135.3000) -- (80.8000,137.0000) -- (81.0000,138.6000) -- (81.2000,139.6000) -- (81.4000,140.1000) -- (81.6000,140.2000) -- (81.8000,140.1000) -- (82.0000,140.0000) -- (82.2000,139.8000) -- (82.4000,139.6000) -- (82.5000,139.6000) -- (82.7000,139.5000) -- (82.9000,139.5000) -- (83.1000,139.5000) -- (83.3000,139.5000) -- (83.5000,139.5000) -- (83.7000,139.5000) -- (83.9000,139.5000) -- (84.1000,139.5000) -- (84.3000,139.5000) -- (84.5000,139.5000) -- (84.7000,139.5000) -- (84.9000,139.5000) -- (85.1000,139.5000) -- (85.3000,139.5000) -- (85.5000,139.7000) -- (85.7000,140.9000) -- (85.9000,142.7000) -- (86.1000,144.6000) -- (86.3000,146.6000) -- (86.5000,148.5000) -- (86.7000,150.3000) -- (86.9000,152.1000) -- (87.1000,153.8000) -- (87.2000,155.5000) -- (87.4000,157.0000) -- (87.6000,158.6000) -- (87.8000,160.0000) -- (88.0000,161.3000) -- (88.2000,162.4000) -- (88.4000,163.3000) -- (88.6000,164.1000) -- (88.8000,164.5000) -- (89.0000,164.7000) -- (89.2000,164.6000) -- (89.4000,164.1000) -- (89.6000,163.4000) -- (89.8000,162.4000) -- (90.0000,161.2000) -- (90.2000,159.8000) -- (90.4000,158.3000) -- (90.6000,156.6000) -- (90.8000,154.9000) -- (91.0000,153.2000) -- (91.2000,151.5000) -- (91.4000,149.6000) -- (91.6000,147.3000) -- (91.8000,144.8000) -- (92.0000,142.3000) -- (92.1000,140.4000) -- (92.3000,139.2000) -- (92.5000,138.7000) -- (92.7000,138.6000) -- (92.9000,138.7000) -- (93.1000,139.0000) -- (93.3000,139.2000) -- (93.5000,139.4000) -- (93.7000,139.5000) -- (93.9000,139.5000) -- (94.1000,139.5000) -- (94.3000,139.5000) -- (94.5000,139.5000) -- (94.7000,139.5000) -- (94.9000,139.5000) -- (95.1000,139.5000) -- (95.3000,139.5000) -- (95.5000,139.5000) -- (95.7000,139.5000) -- (95.9000,139.5000) -- (96.1000,139.5000) -- (96.3000,141.0000) -- (96.5000,142.6000) -- (96.7000,142.3000) -- (96.8000,140.8000) -- (97.0000,139.0000) -- (97.2000,137.1000) -- (97.4000,135.4000) -- (97.6000,133.9000) -- (97.8000,132.6000) -- (98.0000,131.4000) -- (98.2000,130.5000) -- (98.4000,129.6000) -- (98.6000,128.9000) -- (98.8000,128.3000) -- (99.0000,127.7000) -- (99.2000,127.3000) -- (99.4000,126.9000) -- (99.6000,126.6000) -- (99.8000,126.4000) -- (100.0000,126.2000) -- (100.2000,126.2000) -- (100.4000,126.2000) -- (100.6000,126.3000) -- (100.8000,126.6000) -- (101.0000,126.8000) -- (101.2000,127.3000) -- (101.4000,127.9000) -- (101.6000,128.6000) -- (101.7000,129.3000) -- (101.9000,130.1000) -- (102.1000,130.9000) -- (102.3000,131.9000) -- (102.5000,132.9000) -- (102.7000,134.0000) -- (102.9000,135.3000) -- (103.1000,137.1000) -- (103.3000,138.7000) -- (103.5000,139.8000) -- (103.7000,140.2000) -- (103.9000,140.3000) -- (104.1000,140.2000) -- (104.3000,140.0000) -- (104.5000,139.8000) -- (104.7000,139.6000) -- (104.9000,139.5000) -- (105.1000,139.5000) -- (105.3000,139.5000) -- (105.5000,139.5000) -- (105.7000,139.5000) -- (105.9000,139.5000) -- (106.1000,139.5000) -- (106.3000,139.5000) -- (106.4000,139.5000) -- (106.6000,139.5000) -- (106.8000,139.5000) -- (107.0000,139.5000) -- (107.2000,139.5000) -- (107.4000,139.5000) -- (107.6000,139.5000) -- (107.8000,139.7000) -- (108.0000,140.8000) -- (108.2000,142.5000) -- (108.4000,144.4000) -- (108.6000,146.3000) -- (108.8000,148.1000) -- (109.0000,149.8000) -- (109.2000,151.5000) -- (109.4000,153.2000) -- (109.6000,154.8000) -- (109.8000,156.4000) -- (110.0000,157.9000) -- (110.2000,159.3000) -- (110.4000,160.6000) -- (110.6000,161.7000) -- (110.8000,162.7000) -- (111.0000,163.5000) -- (111.1000,164.1000) -- (111.3000,164.5000) -- (111.5000,164.6000) -- (111.7000,164.5000) -- (111.9000,164.1000) -- (112.1000,163.5000) -- (112.3000,162.7000) -- (112.5000,161.7000) -- (112.7000,160.6000) -- (112.9000,159.3000) -- (113.1000,157.9000) -- (113.3000,156.4000) -- (113.5000,154.9000) -- (113.7000,153.3000) -- (113.9000,151.7000) -- (114.1000,149.9000) -- (114.3000,146.7000) -- (114.5000,143.0000) -- (114.7000,140.3000) -- (114.9000,144.4000) -- (115.1000,145.2000) -- (115.3000,145.1000) -- (115.5000,145.4000) -- (115.7000,145.8000) -- (115.9000,146.2000) -- (116.0000,146.4000) -- (116.2000,146.5000) -- (116.4000,146.6000) -- (116.6000,146.6000) -- (116.8000,146.6000) -- (117.0000,146.6000) -- (117.2000,146.6000) -- (117.4000,146.6000) -- (117.6000,146.5000) -- (117.8000,146.5000) -- (118.0000,146.5000) -- (118.2000,146.5000) -- (118.4000,146.5000) -- (118.6000,146.5000) -- (118.8000,146.5000) -- (119.0000,146.7000) -- (119.2000,147.4000) -- (119.4000,147.9000) -- (119.6000,147.5000) -- (119.8000,146.2000) -- (120.0000,144.3000) -- (120.2000,146.9000) -- (120.4000,148.4000) -- (120.6000,147.9000) -- (120.8000,152.5000) -- (120.9000,151.7000) -- (121.1000,150.8000) -- (121.3000,150.0000) -- (121.5000,149.4000) -- (121.7000,148.8000) -- (121.9000,148.1000) -- (122.1000,152.5000) -- (122.3000,154.9000) -- (122.5000,154.5000) -- (122.7000,154.3000) -- (122.9000,154.3000) -- (123.1000,154.3000) -- (123.3000,154.5000) -- (123.5000,154.7000) -- (123.7000,155.0000) -- (123.9000,155.4000) -- (124.1000,155.8000) -- (124.3000,156.6000) -- (124.5000,155.2000) -- (124.7000,150.6000) -- (124.9000,151.6000) -- (125.1000,152.5000) -- (125.3000,153.6000) -- (125.5000,154.7000) -- (125.7000,156.0000) -- (125.8000,157.7000) -- (126.0000,153.9000) -- (126.2000,153.5000) -- (126.4000,154.2000) -- (126.6000,154.4000) -- (126.8000,154.4000) -- (127.0000,159.8000) -- (127.2000,161.1000) -- (127.4000,160.7000) -- (127.6000,160.6000) -- (127.8000,160.6000) -- (128.0000,160.6000) -- (128.2000,160.6000) -- (128.4000,160.6000) -- (128.6000,160.6000) -- (128.8000,160.6000) -- (129.0000,160.6000) -- (129.2000,160.6000) -- (129.4000,160.6000) -- (129.6000,160.6000) -- (129.8000,160.6000) -- (130.0000,160.6000) -- (130.2000,160.6000) -- (130.4000,160.6000) -- (130.5000,160.7000) -- (130.7000,161.7000) -- (130.9000,163.4000) -- (131.1000,165.5000) -- (131.3000,163.4000) -- (131.5000,161.8000) -- (131.7000,163.0000) -- (131.9000,158.9000) -- (132.1000,160.1000) -- (132.3000,161.9000) -- (132.5000,163.6000) -- (132.7000,164.1000) -- (132.9000,159.7000) -- (133.1000,160.7000) -- (133.3000,161.9000) -- (133.5000,162.9000) -- (133.7000,163.7000) -- (133.9000,164.3000) -- (134.1000,164.6000) -- (134.3000,164.6000) -- (134.5000,164.2000) -- (134.7000,165.2000) -- (134.9000,170.0000) -- (135.1000,169.0000) -- (135.2000,167.8000) -- (135.4000,166.2000) -- (135.6000,166.4000) -- (135.8000,170.3000) -- (136.0000,168.4000) -- (136.2000,171.9000) -- (136.4000,172.4000) -- (136.6000,170.3000) -- (136.8000,167.8000) -- (137.0000,165.0000) -- (137.2000,167.9000) -- (137.4000,168.1000) -- (137.6000,166.9000) -- (137.8000,166.6000) -- (138.0000,166.7000) -- (138.2000,166.9000) -- (138.4000,167.2000) -- (138.6000,167.4000) -- (138.8000,167.5000) -- (139.0000,167.6000) -- (139.2000,167.7000) -- (139.4000,167.7000) -- (139.6000,167.7000) -- (139.8000,167.7000) -- (140.0000,167.6000) -- (140.1000,167.6000) -- (140.3000,167.6000) -- (140.5000,167.6000) -- (140.7000,167.6000) -- (140.9000,167.6000) -- (141.1000,167.6000) -- (141.3000,167.7000) -- (141.5000,169.6000) -- (141.7000,171.2000) -- (141.9000,170.6000) -- (142.1000,169.0000) -- (142.3000,167.1000) -- (142.5000,165.1000) -- (142.7000,163.3000) -- (142.9000,161.8000) -- (143.1000,160.5000) -- (143.3000,159.3000) -- (143.5000,158.3000) -- (143.7000,157.5000) -- (143.9000,156.8000) -- (144.1000,156.1000) -- (144.3000,155.6000) -- (144.5000,155.2000) -- (144.7000,154.8000) -- (144.9000,154.6000) -- (145.0000,154.4000) -- (145.2000,154.3000) -- (145.4000,154.3000) -- (145.6000,154.4000) -- (145.8000,154.6000) -- (146.0000,154.8000) -- (146.2000,155.2000) -- (146.4000,155.7000) -- (146.6000,156.2000) -- (146.8000,156.9000) -- (147.0000,157.6000) -- (147.2000,158.4000) -- (147.4000,159.4000) -- (147.6000,160.4000) -- (147.8000,161.6000) -- (148.0000,162.8000) -- (148.2000,164.5000) -- (148.4000,166.2000) -- (148.6000,167.5000) -- (148.8000,168.2000) -- (149.0000,168.4000) -- (149.2000,168.4000) -- (149.4000,168.2000) -- (149.6000,168.0000) -- (149.8000,167.8000) -- (149.9000,167.7000) -- (150.1000,167.6000) -- (150.3000,167.6000) -- (150.5000,167.6000) -- (150.7000,167.6000) -- (150.9000,167.6000) -- (151.1000,167.6000) -- (151.3000,167.6000) -- (151.5000,167.6000) -- (151.7000,167.6000) -- (151.9000,167.6000) -- (152.1000,167.6000) -- (152.3000,167.6000) -- (152.5000,167.6000) -- (152.7000,167.6000) -- (152.9000,167.7000) -- (153.1000,168.6000) -- (153.3000,170.3000) -- (153.5000,172.2000) -- (153.7000,174.1000) -- (153.9000,176.0000) -- (154.1000,177.9000) -- (154.3000,179.6000) -- (154.5000,181.4000) -- (154.6000,183.0000) -- (154.8000,184.6000) -- (155.0000,186.2000) -- (155.2000,187.6000) -- (155.4000,189.0000) -- (155.6000,190.2000) -- (155.8000,191.2000) -- (156.0000,192.0000) -- (156.2000,192.5000) -- (156.4000,192.8000) -- (156.6000,192.8000) -- (156.8000,192.4000) -- (157.0000,191.8000) -- (157.2000,190.9000) -- (157.4000,189.8000) -- (157.6000,188.5000) -- (157.8000,187.0000) -- (158.0000,185.5000) -- (158.2000,183.8000) -- (158.4000,182.1000) -- (158.6000,180.3000) -- (158.8000,178.6000) -- (159.0000,176.5000) -- (159.2000,174.0000) -- (159.3000,171.4000) -- (159.5000,169.2000) -- (159.7000,167.7000) -- (159.9000,166.9000) -- (160.1000,166.7000) -- (160.3000,166.8000) -- (160.5000,167.0000) -- (160.7000,167.2000) -- (160.9000,167.4000) -- (161.1000,167.6000) -- (161.3000,167.6000) -- (161.5000,167.7000) -- (161.7000,167.7000) -- (161.9000,167.7000) -- (162.1000,167.6000) -- (162.3000,167.6000) -- (162.5000,167.6000) -- (162.7000,167.6000) -- (162.9000,167.6000) -- (163.1000,167.6000) -- (163.3000,167.6000) -- (163.5000,167.6000) -- (163.7000,168.2000) -- (163.9000,170.6000) -- (164.1000,171.2000) -- (164.2000,170.0000) -- (164.4000,168.2000) -- (164.6000,166.2000) -- (164.8000,164.3000) -- (165.0000,162.6000) -- (165.2000,161.2000) -- (165.4000,159.9000) -- (165.6000,158.8000) -- (165.8000,157.9000) -- (166.0000,157.1000) -- (166.2000,156.5000) -- (166.4000,155.9000) -- (166.6000,155.4000) -- (166.8000,155.0000) -- (167.0000,154.7000) -- (167.2000,154.5000) -- (167.4000,154.3000) -- (167.6000,154.3000) -- (167.8000,154.3000) -- (168.0000,154.5000) -- (168.2000,154.7000) -- (168.4000,155.0000) -- (168.6000,155.5000) -- (168.8000,156.0000) -- (169.0000,156.6000) -- (169.1000,157.3000) -- (169.3000,158.1000) -- (169.5000,159.0000) -- (169.7000,160.0000) -- (169.9000,161.2000) -- (170.1000,162.4000) -- (170.3000,163.9000) -- (170.5000,165.7000) -- (170.7000,167.1000) -- (170.9000,168.0000) -- (171.1000,168.4000) -- (171.3000,168.4000) -- (171.5000,168.2000) -- (171.7000,168.0000) -- (171.9000,167.8000) -- (172.1000,167.7000) -- (172.3000,167.7000) -- (172.5000,167.6000) -- (172.7000,167.6000) -- (172.9000,167.6000) -- (173.1000,167.6000) -- (173.3000,167.6000) -- (173.5000,167.6000) -- (173.7000,167.6000) -- (173.9000,167.6000) -- (174.0000,167.6000) -- (174.2000,167.6000) -- (174.4000,167.6000) -- (174.6000,167.6000) -- (174.8000,167.6000) -- (175.0000,167.6000) -- (175.2000,168.2000) -- (175.4000,169.6000) -- (175.6000,171.5000) -- (175.8000,173.5000) -- (176.0000,175.5000) -- (176.2000,177.4000) -- (176.4000,179.2000) -- (176.6000,180.9000) -- (176.8000,182.6000) -- (177.0000,184.2000) -- (177.2000,185.8000) -- (177.4000,187.2000) -- (177.6000,188.6000) -- (177.8000,189.8000) -- (178.0000,190.9000) -- (178.2000,191.8000) -- (178.4000,192.4000) -- (178.6000,192.7000) -- (178.7000,192.8000) -- (178.9000,192.5000) -- (179.1000,192.0000) -- (179.3000,191.2000) -- (179.5000,190.1000) -- (179.7000,188.8000) -- (179.9000,187.3000) -- (180.1000,185.7000) -- (180.3000,184.1000) -- (180.5000,182.4000) -- (180.7000,180.6000) -- (180.9000,178.9000) -- (181.1000,176.9000) -- (181.3000,174.5000) -- (181.5000,171.9000) -- (181.7000,169.6000) -- (181.9000,168.0000) -- (182.1000,167.0000) -- (182.3000,166.7000) -- (182.5000,166.8000) -- (182.7000,167.0000) -- (182.9000,167.2000) -- (183.1000,167.4000) -- (183.3000,167.5000) -- (183.4000,167.6000) -- (183.6000,167.7000) -- (183.8000,167.7000) -- (184.0000,167.7000) -- (184.2000,167.7000) -- (184.4000,167.6000) -- (184.6000,167.6000) -- (184.8000,167.6000) -- (185.0000,167.6000) -- (185.2000,167.6000) -- (185.4000,167.6000) -- (185.6000,167.6000) -- (185.8000,168.0000) -- (186.0000,170.4000) -- (186.2000,171.2000) -- (186.4000,170.3000) -- (186.6000,168.5000) -- (186.8000,166.5000) -- (187.0000,164.6000) -- (187.2000,162.8000) -- (187.4000,161.2000) -- (187.6000,159.9000) -- (187.8000,158.7000) -- (188.0000,157.7000) -- (188.2000,156.9000) -- (188.3000,156.2000) -- (188.5000,155.6000) -- (188.7000,155.1000) -- (188.9000,154.7000) -- (189.1000,154.5000) -- (189.3000,154.3000) -- (189.5000,154.2000) -- (189.7000,154.3000) -- (189.9000,154.4000) -- (190.1000,154.7000) -- (190.3000,155.0000) -- (190.5000,155.5000) -- (190.7000,156.0000) -- (190.9000,156.7000) -- (191.1000,157.5000) -- (191.3000,158.4000) -- (191.5000,159.4000) -- (191.7000,160.6000) -- (191.9000,161.8000) -- (192.1000,163.3000) -- (192.3000,165.2000) -- (192.5000,166.8000) -- (192.7000,167.8000) -- (192.9000,168.3000) -- (193.1000,168.4000) -- (193.2000,168.3000) -- (193.4000,168.1000) -- (193.6000,167.9000) -- (193.8000,167.8000) -- (194.0000,167.7000) -- (194.2000,167.6000) -- (194.4000,167.6000) -- (194.6000,167.6000) -- (194.8000,167.6000) -- (195.0000,167.6000) -- (195.2000,167.6000) -- (195.4000,167.6000) -- (195.6000,167.6000) -- (195.8000,167.6000) -- (196.0000,167.6000) -- (196.2000,167.6000) -- (196.4000,167.6000) -- (196.6000,167.6000) -- (196.8000,167.6000) -- (197.0000,168.0000) -- (197.2000,169.3000) -- (197.4000,171.2000) -- (197.6000,173.2000) -- (197.8000,175.2000) -- (197.9000,177.1000) -- (198.1000,178.9000) -- (198.3000,180.7000) -- (198.5000,182.3000) -- (198.7000,184.0000) -- (198.9000,185.5000) -- (199.1000,187.0000) -- (199.3000,188.4000) -- (199.5000,189.7000) -- (199.7000,190.8000) -- (199.9000,191.7000) -- (200.1000,192.3000) -- (200.3000,192.7000) -- (200.5000,192.8000) -- (200.7000,192.6000) -- (200.9000,192.1000) -- (201.1000,191.3000) -- (201.3000,190.2000) -- (201.5000,188.9000) -- (201.7000,187.5000) -- (201.9000,185.9000) -- (202.1000,184.3000) -- (202.3000,182.5000) -- (202.5000,180.8000) -- (202.7000,179.0000) -- (202.8000,177.1000) -- (203.0000,174.7000) -- (203.2000,172.1000) -- (203.4000,169.8000) -- (203.6000,168.1000) -- (203.8000,167.1000) -- (204.0000,166.7000) -- (204.2000,166.7000) -- (204.4000,166.9000) -- (204.6000,167.2000) -- (204.8000,167.4000) -- (205.0000,167.5000) -- (205.2000,167.6000) -- (205.4000,167.7000) -- (205.6000,167.7000) -- (205.8000,167.7000) -- (206.0000,167.7000) -- (206.2000,167.6000) -- (206.4000,167.6000) -- (206.6000,167.6000) -- (206.8000,167.6000) -- (207.0000,167.6000) -- (207.2000,167.6000) -- (207.4000,167.6000) -- (207.5000,167.8000) -- (207.7000,170.3000) -- (207.9000,171.6000) -- (208.1000,170.8000) -- (208.3000,169.0000) -- (208.5000,166.9000) -- (208.7000,164.8000) -- (208.9000,162.9000) -- (209.1000,161.3000) -- (209.3000,160.0000) -- (209.5000,158.8000) -- (209.7000,157.8000) -- (209.9000,157.0000) -- (210.1000,156.3000) -- (210.3000,155.7000) -- (210.5000,155.2000) -- (210.7000,154.8000) -- (210.9000,154.5000) -- (211.1000,154.3000) -- (211.3000,154.3000) -- (211.5000,154.3000) -- (211.7000,154.4000) -- (211.9000,154.6000) -- (212.1000,154.9000) -- (212.3000,155.3000) -- (212.4000,155.9000) -- (212.6000,156.5000) -- (212.8000,157.2000) -- (213.0000,158.1000) -- (213.2000,159.0000) -- (213.4000,160.1000) -- (213.6000,161.3000) -- (213.8000,162.7000) -- (214.0000,164.4000) -- (214.2000,166.3000) -- (214.4000,167.6000) -- (214.6000,168.3000) -- (214.8000,168.5000) -- (215.0000,168.4000) -- (215.2000,168.2000) -- (215.4000,168.0000) -- (215.6000,167.8000) -- (215.8000,167.7000) -- (216.0000,167.6000) -- (216.2000,167.6000) -- (216.4000,167.6000) -- (216.6000,167.6000) -- (216.8000,167.6000) -- (217.0000,167.6000) -- (217.1000,167.6000) -- (217.3000,167.6000) -- (217.5000,167.6000) -- (217.7000,167.6000) -- (217.9000,167.6000) -- (218.1000,167.6000) -- (218.3000,167.6000) -- (218.5000,167.6000) -- (218.7000,167.7000) -- (218.9000,168.7000) -- (219.1000,170.4000) -- (219.3000,172.4000) -- (219.5000,174.3000) -- (219.7000,176.2000) -- (219.9000,178.1000) -- (220.1000,179.8000) -- (220.3000,181.6000) -- (220.5000,183.2000) -- (220.7000,184.8000) -- (220.9000,186.4000) -- (221.1000,187.8000) -- (221.3000,189.1000) -- (221.5000,190.3000) -- (221.7000,191.3000) -- (221.9000,192.0000) -- (222.0000,192.5000) -- (222.2000,192.8000) -- (222.4000,192.7000) -- (222.6000,192.4000) -- (222.8000,191.7000) -- (223.0000,190.8000) -- (223.2000,189.7000) -- (223.4000,188.3000) -- (223.6000,186.8000) -- (223.8000,185.2000) -- (224.0000,183.6000) -- (224.2000,181.8000) -- (224.4000,180.1000) -- (224.6000,178.3000) -- (224.8000,176.2000) -- (225.0000,173.7000) -- (225.2000,171.1000) -- (225.4000,169.0000) -- (225.6000,167.6000) -- (225.8000,166.9000) -- (226.0000,166.7000) -- (226.2000,166.8000) -- (226.4000,167.0000) -- (226.6000,167.3000) -- (226.8000,167.4000) -- (226.9000,167.6000) -- (227.1000,167.6000) -- (227.3000,167.7000) -- (227.5000,167.7000) -- (227.7000,167.7000) -- (227.9000,167.6000) -- (228.1000,167.6000) -- (228.3000,167.6000) -- (228.5000,167.6000) -- (228.7000,167.6000) -- (228.9000,167.6000) -- (229.1000,167.6000) -- (229.3000,167.5000) -- (229.5000,168.5000) -- (229.7000,171.2000) -- (229.9000,171.5000) -- (230.1000,170.2000) -- (230.3000,168.2000) -- (230.5000,166.1000) -- (230.7000,164.1000) -- (230.9000,162.3000) -- (231.1000,160.8000) -- (231.3000,159.5000) -- (231.4000,158.4000) -- (231.6000,157.5000) -- (231.8000,156.7000) -- (232.0000,156.1000) -- (232.2000,155.5000) -- (232.4000,155.1000) -- (232.6000,154.7000) -- (232.8000,154.5000) -- (233.0000,154.3000) -- (233.2000,154.2000) -- (233.4000,154.3000) -- (233.6000,154.4000) -- (233.8000,154.7000) -- (234.0000,155.0000) -- (234.2000,155.5000) -- (234.4000,156.1000) -- (234.6000,156.7000) -- (234.8000,157.5000) -- (235.0000,158.4000) -- (235.2000,159.4000) -- (235.4000,160.5000) -- (235.6000,161.7000) -- (235.8000,163.2000) -- (236.0000,165.1000) -- (236.2000,166.8000) -- (236.3000,167.9000) -- (236.5000,168.4000) -- (236.7000,168.5000) -- (236.9000,168.3000) -- (237.1000,168.1000) -- (237.3000,167.9000) -- (237.5000,167.8000) -- (237.7000,167.7000) -- (237.9000,167.6000) -- (238.1000,167.6000) -- (238.3000,167.6000) -- (238.5000,167.6000) -- (238.7000,167.6000) -- (238.9000,167.6000) -- (239.1000,167.6000) -- (239.3000,167.6000) -- (239.5000,167.6000) -- (239.7000,167.6000) -- (239.9000,167.6000) -- (240.1000,167.6000) -- (240.3000,167.6000) -- (240.5000,167.6000) -- (240.7000,168.0000) -- (240.9000,169.3000) -- (241.1000,171.1000) -- (241.2000,173.0000) -- (241.4000,174.9000) -- (241.6000,176.8000) -- (241.8000,178.6000) -- (242.0000,180.4000) -- (242.2000,182.1000) -- (242.4000,183.7000) -- (242.6000,185.3000) -- (242.8000,186.8000) -- (243.0000,188.2000) -- (243.2000,189.5000) -- (243.4000,190.6000) -- (243.6000,191.5000) -- (243.8000,192.2000) -- (244.0000,192.7000) -- (244.2000,192.8000) -- (244.4000,192.6000) -- (244.6000,192.2000) -- (244.8000,191.5000) -- (245.0000,190.5000) -- (245.2000,189.3000) -- (245.4000,187.9000) -- (245.6000,186.4000) -- (245.8000,184.7000) -- (246.0000,183.0000) -- (246.1000,181.3000) -- (246.3000,179.6000) -- (246.5000,177.8000) -- (246.7000,175.5000) -- (246.9000,172.9000) -- (247.1000,170.4000) -- (247.3000,168.5000) -- (247.5000,167.3000) -- (247.7000,166.8000) -- (247.9000,166.7000) -- (248.1000,166.9000) -- (248.3000,167.1000) -- (248.5000,167.3000) -- (248.7000,167.5000) -- (248.9000,167.6000) -- (249.1000,167.6000) -- (249.3000,167.7000) -- (249.5000,167.7000) -- (249.7000,167.7000) -- (249.9000,167.6000) -- (250.1000,167.6000) -- (250.3000,167.6000) -- (250.5000,167.6000) -- (250.7000,167.6000) -- (250.9000,167.6000) -- (251.0000,167.6000) -- (251.2000,167.6000) -- (251.4000,169.4000) -- (251.6000,171.5000) -- (251.8000,171.2000);



  \end{scope}
  \begin{scope}[scale=1.006,draw=black,line join=bevel,line cap=rect,line width=0.800pt]
  \end{scope}
  \begin{scope}[cm={{1.00588,0.0,0.0,1.00588,(197.153,129.759)}},draw=black,line join=bevel,line cap=rect,line width=0.800pt]
  \end{scope}
  \begin{scope}[cm={{1.00588,0.0,0.0,1.00588,(197.153,129.759)}},draw=black,line join=bevel,line cap=rect,line width=0.800pt]
  \end{scope}
  \begin{scope}[cm={{1.00588,0.0,0.0,1.00588,(197.153,129.759)}},draw=black,line join=bevel,line cap=rect,line width=0.800pt]
  \end{scope}
  \begin{scope}[cm={{1.00588,0.0,0.0,1.00588,(197.153,129.759)}},draw=black,line join=bevel,line cap=rect,line width=0.800pt]
  \end{scope}
  \begin{scope}[cm={{1.00588,0.0,0.0,1.00588,(197.153,129.759)}},draw=black,line join=bevel,line cap=rect,line width=0.800pt]
  \end{scope}
  \begin{scope}[cm={{1.00588,0.0,0.0,1.00588,(190.653,130.76399)}},draw=black,line join=bevel,line cap=rect,line width=0.800pt]
    \path[fill=black] (0.0000,0.0000) node[above right] (text1032) {\scriptsize $b_k Q_c V$};



  \end{scope}
  \begin{scope}[cm={{1.00588,0.0,0.0,1.00588,(197.153,129.759)}},draw=black,line join=bevel,line cap=rect,line width=0.800pt]
  \end{scope}
  \begin{scope}[scale=1.006,draw=black,line join=bevel,line cap=rect,line width=0.800pt]
  \end{scope}
  \begin{scope}[scale=1.006,draw=black,line join=bevel,line cap=rect,line width=0.800pt]
  \end{scope}
  \begin{scope}[scale=1.006,draw=black,line join=bevel,line cap=rect,line width=0.800pt]
  \end{scope}
  \begin{scope}[scale=1.006,draw=black,line join=bevel,line cap=rect,line width=0.800pt]
  \end{scope}
  \begin{scope}[scale=1.006,draw=black,line join=bevel,line cap=rect,line width=0.800pt]
  \end{scope}
  \begin{scope}[scale=1.006,draw=c00ff00,line join=round,line cap=round,line width=0.480pt]
    \path[draw] (101.0000,115.7000) -- (101.2000,122.5000) -- (101.7000,123.1000) -- (102.2000,123.7000) -- (102.7000,124.3000) -- (103.2000,124.9000) -- (103.6000,125.5000) -- (104.1000,126.1000) -- (104.6000,126.7000) -- (105.1000,127.3000) -- (105.6000,127.9000) -- (106.1000,128.4000) -- (106.6000,129.0000) -- (107.0000,129.6000) -- (107.5000,130.2000) -- (108.0000,130.8000) -- (108.5000,131.4000) -- (109.0000,132.0000) -- (109.5000,132.6000) -- (110.0000,133.2000) -- (110.4000,133.8000) -- (110.9000,134.3000) -- (111.4000,134.9000) -- (111.9000,135.5000) -- (112.4000,136.1000) -- (112.9000,136.7000) -- (113.4000,137.3000) -- (113.9000,137.9000) -- (114.3000,138.5000) -- (114.8000,139.1000) -- (115.3000,139.7000) -- (115.8000,140.2000) -- (116.3000,140.8000) -- (116.8000,141.4000) -- (117.3000,142.0000) -- (117.7000,142.6000) -- (118.2000,143.2000) -- (118.7000,143.8000) -- (119.2000,144.4000) -- (119.7000,145.0000) -- (120.2000,145.6000) -- (120.7000,146.1000) -- (121.1000,146.7000) -- (121.6000,147.3000) -- (122.1000,147.9000) -- (122.6000,148.5000) -- (123.1000,149.1000) -- (123.6000,149.7000) -- (124.1000,150.3000) -- (124.5000,150.9000) -- (125.0000,151.5000) -- (125.5000,152.0000) -- (126.0000,152.6000) -- (126.5000,153.2000) -- (127.0000,153.8000) -- (127.5000,154.4000) -- (128.0000,155.0000) -- (128.4000,155.6000) -- (128.9000,156.2000) -- (129.4000,156.8000) -- (129.9000,157.4000) -- (130.4000,157.9000) -- (130.9000,158.5000) -- (131.4000,159.1000) -- (131.8000,159.7000) -- (132.3000,160.3000) -- (132.8000,160.9000) -- (133.3000,161.5000) -- (133.8000,162.1000) -- (134.3000,162.7000) -- (134.8000,163.3000) -- (135.2000,163.9000) -- (135.7000,164.4000) -- (136.2000,165.0000) -- (136.7000,165.6000) -- (137.2000,166.2000) -- (137.7000,166.8000) -- (138.2000,167.4000) -- (138.6000,168.0000) -- (139.1000,168.6000) -- (139.6000,169.2000) -- (140.1000,169.7000) -- (140.6000,170.3000) -- (141.1000,170.9000) -- (141.6000,171.5000) -- (142.1000,172.1000) -- (142.5000,172.7000) -- (143.0000,173.3000) -- (143.5000,173.9000) -- (144.0000,174.5000) -- (144.5000,175.1000) -- (145.0000,175.7000) -- (145.5000,176.2000) -- (145.9000,176.8000) -- (146.4000,177.4000) -- (146.9000,178.0000) -- (147.4000,178.6000) -- (147.9000,179.2000) -- (148.4000,179.8000) -- (148.9000,180.4000) -- (149.3000,181.0000) -- (149.8000,181.6000) -- (150.3000,182.1000) -- (150.8000,182.7000) -- (151.3000,183.3000) -- (151.8000,183.9000) -- (152.3000,184.5000) -- (152.7000,185.1000) -- (153.2000,185.7000) -- (153.7000,186.3000) -- (154.2000,186.9000) -- (154.7000,187.5000) -- (155.2000,188.0000) -- (155.7000,188.6000) -- (156.2000,189.2000) -- (156.6000,189.8000) -- (157.1000,190.4000) -- (157.6000,191.0000) -- (158.1000,191.6000) -- (158.6000,192.2000) -- (159.1000,192.8000) -- (159.6000,193.4000) -- (160.0000,193.9000) -- (160.5000,194.5000) -- (161.0000,195.1000) -- (161.5000,195.7000) -- (162.0000,196.3000) -- (162.5000,196.9000) -- (163.0000,197.5000) -- (163.4000,198.1000) -- (163.9000,198.7000) -- (164.4000,199.3000) -- (164.9000,199.8000) -- (165.4000,200.4000) -- (165.5000,200.6000);



  \end{scope}
  \begin{scope}[scale=1.006,draw=black,line join=bevel,line cap=rect,line width=0.800pt]
  \end{scope}
  \begin{scope}[scale=1.006,draw=black,line join=bevel,line cap=rect,line width=0.800pt]
  \end{scope}
  \begin{scope}[scale=1.006,draw=black,line join=round,line cap=round,line width=0.480pt]
    \path[draw] (56.5000,115.5000) -- (56.5000,200.5000) -- (251.5000,200.5000) -- (251.5000,115.5000) -- (56.5000,115.5000);



  \end{scope}
  \begin{scope}[scale=1.006,draw=ca0a0a4,dash pattern=on 0.40pt off 0.80pt,line join=round,line cap=round,line width=0.400pt]
    \path[draw] (56.5000,299.5000) -- (251.5000,299.5000);



  \end{scope}
  \begin{scope}[scale=1.006,draw=black,line join=round,line cap=round,line width=0.480pt]
    \path[draw] (56.5000,299.5000) -- (59.5000,299.5000);



    \path[draw] (251.5000,299.5000) -- (248.5000,299.5000);



  \end{scope}
  \begin{scope}[scale=1.006,draw=black,line join=bevel,line cap=rect,line width=0.800pt]
  \end{scope}
  \begin{scope}[cm={{1.00588,0.0,0.0,1.00588,(39.2294,305.788)}},draw=black,line join=bevel,line cap=rect,line width=0.800pt]
  \end{scope}
  \begin{scope}[cm={{1.00588,0.0,0.0,1.00588,(39.2294,305.788)}},draw=black,line join=bevel,line cap=rect,line width=0.800pt]
  \end{scope}
  \begin{scope}[cm={{1.00588,0.0,0.0,1.00588,(39.2294,305.788)}},draw=black,line join=bevel,line cap=rect,line width=0.800pt]
  \end{scope}
  \begin{scope}[cm={{1.00588,0.0,0.0,1.00588,(39.2294,305.788)}},draw=black,line join=bevel,line cap=rect,line width=0.800pt]
  \end{scope}
  \begin{scope}[cm={{1.00588,0.0,0.0,1.00588,(39.2294,305.788)}},draw=black,line join=bevel,line cap=rect,line width=0.800pt]
  \end{scope}
  \begin{scope}[cm={{1.00588,0.0,0.0,1.00588,(39.2294,305.788)}},draw=black,line join=bevel,line cap=rect,line width=0.800pt]
    \path[fill=black] (0.0000,0.0000) node[above right] (text1086) {27};



  \end{scope}
  \begin{scope}[cm={{1.00588,0.0,0.0,1.00588,(39.2294,305.788)}},draw=black,line join=bevel,line cap=rect,line width=0.800pt]
  \end{scope}
  \begin{scope}[scale=1.006,draw=black,line join=bevel,line cap=rect,line width=0.800pt]
  \end{scope}
  \begin{scope}[scale=1.006,draw=ca0a0a4,dash pattern=on 0.40pt off 0.80pt,line join=round,line cap=round,line width=0.400pt]
    \path[draw] (56.5000,277.5000) -- (251.5000,277.5000);



  \end{scope}
  \begin{scope}[scale=1.006,draw=black,line join=round,line cap=round,line width=0.480pt]
    \path[draw] (56.5000,277.5000) -- (59.5000,277.5000);



    \path[draw] (251.5000,277.5000) -- (248.5000,277.5000);



  \end{scope}
  \begin{scope}[scale=1.006,draw=black,line join=bevel,line cap=rect,line width=0.800pt]
  \end{scope}
  \begin{scope}[cm={{1.00588,0.0,0.0,1.00588,(40.2353,282.653)}},draw=black,line join=bevel,line cap=rect,line width=0.800pt]
  \end{scope}
  \begin{scope}[cm={{1.00588,0.0,0.0,1.00588,(40.2353,282.653)}},draw=black,line join=bevel,line cap=rect,line width=0.800pt]
  \end{scope}
  \begin{scope}[cm={{1.00588,0.0,0.0,1.00588,(40.2353,282.653)}},draw=black,line join=bevel,line cap=rect,line width=0.800pt]
  \end{scope}
  \begin{scope}[cm={{1.00588,0.0,0.0,1.00588,(40.2353,282.653)}},draw=black,line join=bevel,line cap=rect,line width=0.800pt]
  \end{scope}
  \begin{scope}[cm={{1.00588,0.0,0.0,1.00588,(40.2353,282.653)}},draw=black,line join=bevel,line cap=rect,line width=0.800pt]
  \end{scope}
  \begin{scope}[cm={{1.00588,0.0,0.0,1.00588,(40.2353,282.653)}},draw=black,line join=bevel,line cap=rect,line width=0.800pt]
    \path[fill=black] (0.0000,0.0000) node[above right] (text1116) {31};



  \end{scope}
  \begin{scope}[cm={{1.00588,0.0,0.0,1.00588,(40.2353,282.653)}},draw=black,line join=bevel,line cap=rect,line width=0.800pt]
  \end{scope}
  \begin{scope}[scale=1.006,draw=black,line join=bevel,line cap=rect,line width=0.800pt]
  \end{scope}
  \begin{scope}[scale=1.006,draw=ca0a0a4,dash pattern=on 0.40pt off 0.80pt,line join=round,line cap=round,line width=0.400pt]
    \path[draw] (56.5000,254.5000) -- (251.5000,254.5000);



  \end{scope}
  \begin{scope}[scale=1.006,draw=black,line join=round,line cap=round,line width=0.480pt]
    \path[draw] (56.5000,254.5000) -- (59.5000,254.5000);



    \path[draw] (251.5000,254.5000) -- (248.5000,254.5000);



  \end{scope}
  \begin{scope}[scale=1.006,draw=black,line join=bevel,line cap=rect,line width=0.800pt]
  \end{scope}
  \begin{scope}[cm={{1.00588,0.0,0.0,1.00588,(40.2353,260.524)}},draw=black,line join=bevel,line cap=rect,line width=0.800pt]
  \end{scope}
  \begin{scope}[cm={{1.00588,0.0,0.0,1.00588,(40.2353,260.524)}},draw=black,line join=bevel,line cap=rect,line width=0.800pt]
  \end{scope}
  \begin{scope}[cm={{1.00588,0.0,0.0,1.00588,(40.2353,260.524)}},draw=black,line join=bevel,line cap=rect,line width=0.800pt]
  \end{scope}
  \begin{scope}[cm={{1.00588,0.0,0.0,1.00588,(40.2353,260.524)}},draw=black,line join=bevel,line cap=rect,line width=0.800pt]
  \end{scope}
  \begin{scope}[cm={{1.00588,0.0,0.0,1.00588,(40.2353,260.524)}},draw=black,line join=bevel,line cap=rect,line width=0.800pt]
  \end{scope}
  \begin{scope}[cm={{1.00588,0.0,0.0,1.00588,(40.2353,260.524)}},draw=black,line join=bevel,line cap=rect,line width=0.800pt]
    \path[fill=black] (0.0000,0.0000) node[above right] (text1146) {35};



  \end{scope}
  \begin{scope}[cm={{1.00588,0.0,0.0,1.00588,(40.2353,260.524)}},draw=black,line join=bevel,line cap=rect,line width=0.800pt]
  \end{scope}
  \begin{scope}[scale=1.006,draw=black,line join=bevel,line cap=rect,line width=0.800pt]
  \end{scope}
  \begin{scope}[scale=1.006,draw=ca0a0a4,dash pattern=on 0.40pt off 0.80pt,line join=round,line cap=round,line width=0.400pt]
    \path[draw] (56.5000,232.5000) -- (251.5000,232.5000);



  \end{scope}
  \begin{scope}[scale=1.006,draw=black,line join=round,line cap=round,line width=0.480pt]
    \path[draw] (56.5000,232.5000) -- (59.5000,232.5000);



    \path[draw] (251.5000,232.5000) -- (248.5000,232.5000);



  \end{scope}
  \begin{scope}[scale=1.006,draw=black,line join=bevel,line cap=rect,line width=0.800pt]
  \end{scope}
  \begin{scope}[cm={{1.00588,0.0,0.0,1.00588,(39.2294,237.388)}},draw=black,line join=bevel,line cap=rect,line width=0.800pt]
  \end{scope}
  \begin{scope}[cm={{1.00588,0.0,0.0,1.00588,(39.2294,237.388)}},draw=black,line join=bevel,line cap=rect,line width=0.800pt]
  \end{scope}
  \begin{scope}[cm={{1.00588,0.0,0.0,1.00588,(39.2294,237.388)}},draw=black,line join=bevel,line cap=rect,line width=0.800pt]
  \end{scope}
  \begin{scope}[cm={{1.00588,0.0,0.0,1.00588,(39.2294,237.388)}},draw=black,line join=bevel,line cap=rect,line width=0.800pt]
  \end{scope}
  \begin{scope}[cm={{1.00588,0.0,0.0,1.00588,(39.2294,237.388)}},draw=black,line join=bevel,line cap=rect,line width=0.800pt]
  \end{scope}
  \begin{scope}[cm={{1.00588,0.0,0.0,1.00588,(39.2294,237.388)}},draw=black,line join=bevel,line cap=rect,line width=0.800pt]
    \path[fill=black] (0.0000,0.0000) node[above right] (text1176) {39};



  \end{scope}
  \begin{scope}[cm={{1.00588,0.0,0.0,1.00588,(39.2294,237.388)}},draw=black,line join=bevel,line cap=rect,line width=0.800pt]
  \end{scope}
  \begin{scope}[scale=1.006,draw=black,line join=bevel,line cap=rect,line width=0.800pt]
  \end{scope}
  \begin{scope}[scale=1.006,draw=ca0a0a4,dash pattern=on 0.40pt off 0.80pt,line join=round,line cap=round,line width=0.400pt]
    \path[draw] (56.5000,306.5000) -- (56.5000,221.5000);



  \end{scope}
  \begin{scope}[scale=1.006,draw=black,line join=round,line cap=round,line width=0.480pt]
    \path[draw] (56.5000,306.5000) -- (56.5000,303.5000);



    \path[draw] (56.5000,221.5000) -- (56.5000,223.5000);



  \end{scope}
  \begin{scope}[scale=1.006,draw=black,line join=bevel,line cap=rect,line width=0.800pt]
  \end{scope}
  \begin{scope}[cm={{1.00588,0.0,0.0,1.00588,(53.3118,322.888)}},draw=black,line join=bevel,line cap=rect,line width=0.800pt]
  \end{scope}
  \begin{scope}[cm={{1.00588,0.0,0.0,1.00588,(53.3118,322.888)}},draw=black,line join=bevel,line cap=rect,line width=0.800pt]
  \end{scope}
  \begin{scope}[cm={{1.00588,0.0,0.0,1.00588,(53.3118,322.888)}},draw=black,line join=bevel,line cap=rect,line width=0.800pt]
  \end{scope}
  \begin{scope}[cm={{1.00588,0.0,0.0,1.00588,(53.3118,322.888)}},draw=black,line join=bevel,line cap=rect,line width=0.800pt]
  \end{scope}
  \begin{scope}[cm={{1.00588,0.0,0.0,1.00588,(53.3118,322.888)}},draw=black,line join=bevel,line cap=rect,line width=0.800pt]
  \end{scope}
  \begin{scope}[cm={{1.00588,0.0,0.0,1.00588,(53.3118,322.888)}},draw=black,line join=bevel,line cap=rect,line width=0.800pt]
    \path[fill=black] (0.0000,0.0000) node[above right] (text1206) {0};



  \end{scope}
  \begin{scope}[cm={{1.00588,0.0,0.0,1.00588,(53.3118,322.888)}},draw=black,line join=bevel,line cap=rect,line width=0.800pt]
  \end{scope}
  \begin{scope}[scale=1.006,draw=black,line join=bevel,line cap=rect,line width=0.800pt]
  \end{scope}
  \begin{scope}[scale=1.006,draw=ca0a0a4,dash pattern=on 0.40pt off 0.80pt,line join=round,line cap=round,line width=0.400pt]
    \path[draw] (91.5000,306.5000) -- (91.5000,221.5000);



  \end{scope}
  \begin{scope}[scale=1.006,draw=black,line join=round,line cap=round,line width=0.480pt]
    \path[draw] (91.5000,306.5000) -- (91.5000,303.5000);



    \path[draw] (91.5000,221.5000) -- (91.5000,223.5000);



  \end{scope}
  \begin{scope}[scale=1.006,draw=black,line join=bevel,line cap=rect,line width=0.800pt]
  \end{scope}
  \begin{scope}[cm={{1.00588,0.0,0.0,1.00588,(89.5235,322.888)}},draw=black,line join=bevel,line cap=rect,line width=0.800pt]
  \end{scope}
  \begin{scope}[cm={{1.00588,0.0,0.0,1.00588,(89.5235,322.888)}},draw=black,line join=bevel,line cap=rect,line width=0.800pt]
  \end{scope}
  \begin{scope}[cm={{1.00588,0.0,0.0,1.00588,(89.5235,322.888)}},draw=black,line join=bevel,line cap=rect,line width=0.800pt]
  \end{scope}
  \begin{scope}[cm={{1.00588,0.0,0.0,1.00588,(89.5235,322.888)}},draw=black,line join=bevel,line cap=rect,line width=0.800pt]
  \end{scope}
  \begin{scope}[cm={{1.00588,0.0,0.0,1.00588,(89.5235,322.888)}},draw=black,line join=bevel,line cap=rect,line width=0.800pt]
  \end{scope}
  \begin{scope}[cm={{1.00588,0.0,0.0,1.00588,(89.5235,322.888)}},draw=black,line join=bevel,line cap=rect,line width=0.800pt]
    \path[fill=black] (0.0000,0.0000) node[above right] (text1236) {2};



  \end{scope}
  \begin{scope}[cm={{1.00588,0.0,0.0,1.00588,(89.5235,322.888)}},draw=black,line join=bevel,line cap=rect,line width=0.800pt]
  \end{scope}
  \begin{scope}[scale=1.006,draw=black,line join=bevel,line cap=rect,line width=0.800pt]
  \end{scope}
  \begin{scope}[scale=1.006,draw=ca0a0a4,dash pattern=on 0.40pt off 0.80pt,line join=round,line cap=round,line width=0.400pt]
    \path[draw] (127.5000,306.5000) -- (127.5000,221.5000);



  \end{scope}
  \begin{scope}[scale=1.006,draw=black,line join=round,line cap=round,line width=0.480pt]
    \path[draw] (127.5000,306.5000) -- (127.5000,303.5000);



    \path[draw] (127.5000,221.5000) -- (127.5000,223.5000);



  \end{scope}
  \begin{scope}[scale=1.006,draw=black,line join=bevel,line cap=rect,line width=0.800pt]
  \end{scope}
  \begin{scope}[cm={{1.00588,0.0,0.0,1.00588,(125.232,322.888)}},draw=black,line join=bevel,line cap=rect,line width=0.800pt]
  \end{scope}
  \begin{scope}[cm={{1.00588,0.0,0.0,1.00588,(125.232,322.888)}},draw=black,line join=bevel,line cap=rect,line width=0.800pt]
  \end{scope}
  \begin{scope}[cm={{1.00588,0.0,0.0,1.00588,(125.232,322.888)}},draw=black,line join=bevel,line cap=rect,line width=0.800pt]
  \end{scope}
  \begin{scope}[cm={{1.00588,0.0,0.0,1.00588,(125.232,322.888)}},draw=black,line join=bevel,line cap=rect,line width=0.800pt]
  \end{scope}
  \begin{scope}[cm={{1.00588,0.0,0.0,1.00588,(125.232,322.888)}},draw=black,line join=bevel,line cap=rect,line width=0.800pt]
  \end{scope}
  \begin{scope}[cm={{1.00588,0.0,0.0,1.00588,(125.232,322.888)}},draw=black,line join=bevel,line cap=rect,line width=0.800pt]
    \path[fill=black] (0.0000,0.0000) node[above right] (text1266) {4};



  \end{scope}
  \begin{scope}[cm={{1.00588,0.0,0.0,1.00588,(125.232,322.888)}},draw=black,line join=bevel,line cap=rect,line width=0.800pt]
  \end{scope}
  \begin{scope}[scale=1.006,draw=black,line join=bevel,line cap=rect,line width=0.800pt]
  \end{scope}
  \begin{scope}[scale=1.006,draw=ca0a0a4,dash pattern=on 0.40pt off 0.80pt,line join=round,line cap=round,line width=0.400pt]
    \path[draw] (162.5000,306.5000) -- (162.5000,221.5000);



  \end{scope}
  \begin{scope}[scale=1.006,draw=black,line join=round,line cap=round,line width=0.480pt]
    \path[draw] (162.5000,306.5000) -- (162.5000,303.5000);



    \path[draw] (162.5000,221.5000) -- (162.5000,223.5000);



  \end{scope}
  \begin{scope}[scale=1.006,draw=black,line join=bevel,line cap=rect,line width=0.800pt]
  \end{scope}
  \begin{scope}[cm={{1.00588,0.0,0.0,1.00588,(160.941,322.888)}},draw=black,line join=bevel,line cap=rect,line width=0.800pt]
  \end{scope}
  \begin{scope}[cm={{1.00588,0.0,0.0,1.00588,(160.941,322.888)}},draw=black,line join=bevel,line cap=rect,line width=0.800pt]
  \end{scope}
  \begin{scope}[cm={{1.00588,0.0,0.0,1.00588,(160.941,322.888)}},draw=black,line join=bevel,line cap=rect,line width=0.800pt]
  \end{scope}
  \begin{scope}[cm={{1.00588,0.0,0.0,1.00588,(160.941,322.888)}},draw=black,line join=bevel,line cap=rect,line width=0.800pt]
  \end{scope}
  \begin{scope}[cm={{1.00588,0.0,0.0,1.00588,(160.941,322.888)}},draw=black,line join=bevel,line cap=rect,line width=0.800pt]
  \end{scope}
  \begin{scope}[cm={{1.00588,0.0,0.0,1.00588,(160.941,322.888)}},draw=black,line join=bevel,line cap=rect,line width=0.800pt]
    \path[fill=black] (0.0000,0.0000) node[above right] (text1296) {6};



  \end{scope}
  \begin{scope}[cm={{1.00588,0.0,0.0,1.00588,(160.941,322.888)}},draw=black,line join=bevel,line cap=rect,line width=0.800pt]
  \end{scope}
  \begin{scope}[scale=1.006,draw=black,line join=bevel,line cap=rect,line width=0.800pt]
  \end{scope}
  \begin{scope}[scale=1.006,draw=ca0a0a4,dash pattern=on 0.40pt off 0.80pt,line join=round,line cap=round,line width=0.400pt]
    \path[draw] (198.5000,306.5000) -- (198.5000,221.5000);



  \end{scope}
  \begin{scope}[scale=1.006,draw=black,line join=round,line cap=round,line width=0.480pt]
    \path[draw] (198.5000,306.5000) -- (198.5000,303.5000);



    \path[draw] (198.5000,221.5000) -- (198.5000,223.5000);



  \end{scope}
  \begin{scope}[scale=1.006,draw=black,line join=bevel,line cap=rect,line width=0.800pt]
  \end{scope}
  \begin{scope}[cm={{1.00588,0.0,0.0,1.00588,(196.147,322.888)}},draw=black,line join=bevel,line cap=rect,line width=0.800pt]
  \end{scope}
  \begin{scope}[cm={{1.00588,0.0,0.0,1.00588,(196.147,322.888)}},draw=black,line join=bevel,line cap=rect,line width=0.800pt]
  \end{scope}
  \begin{scope}[cm={{1.00588,0.0,0.0,1.00588,(196.147,322.888)}},draw=black,line join=bevel,line cap=rect,line width=0.800pt]
  \end{scope}
  \begin{scope}[cm={{1.00588,0.0,0.0,1.00588,(196.147,322.888)}},draw=black,line join=bevel,line cap=rect,line width=0.800pt]
  \end{scope}
  \begin{scope}[cm={{1.00588,0.0,0.0,1.00588,(196.147,322.888)}},draw=black,line join=bevel,line cap=rect,line width=0.800pt]
  \end{scope}
  \begin{scope}[cm={{1.00588,0.0,0.0,1.00588,(196.147,322.888)}},draw=black,line join=bevel,line cap=rect,line width=0.800pt]
    \path[fill=black] (0.0000,0.0000) node[above right] (text1326) {8};



  \end{scope}
  \begin{scope}[cm={{1.00588,0.0,0.0,1.00588,(196.147,322.888)}},draw=black,line join=bevel,line cap=rect,line width=0.800pt]
  \end{scope}
  \begin{scope}[scale=1.006,draw=black,line join=bevel,line cap=rect,line width=0.800pt]
  \end{scope}
  \begin{scope}[scale=1.006,draw=ca0a0a4,dash pattern=on 0.40pt off 0.80pt,line join=round,line cap=round,line width=0.400pt]
    \path[draw] (233.5000,306.5000) -- (233.5000,221.5000);



  \end{scope}
  \begin{scope}[scale=1.006,draw=black,line join=round,line cap=round,line width=0.480pt]
    \path[draw] (233.5000,306.5000) -- (233.5000,303.5000);



    \path[draw] (233.5000,221.5000) -- (233.5000,223.5000);



  \end{scope}
  \begin{scope}[scale=1.006,draw=black,line join=bevel,line cap=rect,line width=0.800pt]
  \end{scope}
  \begin{scope}[cm={{1.00588,0.0,0.0,1.00588,(228.838,322.888)}},draw=black,line join=bevel,line cap=rect,line width=0.800pt]
  \end{scope}
  \begin{scope}[cm={{1.00588,0.0,0.0,1.00588,(228.838,322.888)}},draw=black,line join=bevel,line cap=rect,line width=0.800pt]
  \end{scope}
  \begin{scope}[cm={{1.00588,0.0,0.0,1.00588,(228.838,322.888)}},draw=black,line join=bevel,line cap=rect,line width=0.800pt]
  \end{scope}
  \begin{scope}[cm={{1.00588,0.0,0.0,1.00588,(228.838,322.888)}},draw=black,line join=bevel,line cap=rect,line width=0.800pt]
  \end{scope}
  \begin{scope}[cm={{1.00588,0.0,0.0,1.00588,(228.838,322.888)}},draw=black,line join=bevel,line cap=rect,line width=0.800pt]
  \end{scope}
  \begin{scope}[cm={{1.00588,0.0,0.0,1.00588,(228.838,322.888)}},draw=black,line join=bevel,line cap=rect,line width=0.800pt]
    \path[fill=black] (0.0000,0.0000) node[above right] (text1356) {10};



  \end{scope}
  \begin{scope}[cm={{1.00588,0.0,0.0,1.00588,(228.838,322.888)}},draw=black,line join=bevel,line cap=rect,line width=0.800pt]
  \end{scope}
  \begin{scope}[scale=1.006,draw=black,line join=bevel,line cap=rect,line width=0.800pt]
  \end{scope}
  \begin{scope}[scale=1.006,draw=black,line join=round,line cap=round,line width=0.480pt]
    \path[draw] (56.5000,221.5000) -- (56.5000,306.5000) -- (251.5000,306.5000) -- (251.5000,221.5000) -- (56.5000,221.5000);



  \end{scope}
  \begin{scope}[scale=1.006,draw=black,line join=bevel,line cap=rect,line width=0.800pt]
  \end{scope}
  \begin{scope}[scale=1.006,draw=black,line join=bevel,line cap=rect,line width=0.800pt]
  \end{scope}
  \begin{scope}[scale=1.006,fill=cffffff]
    \path[fill,rounded corners=0.0000cm] (226.0000,225.0000) rectangle (242.0000,241.0000);



  \end{scope}
  \begin{scope}[scale=1.006,draw=black,line join=bevel,line cap=rect,line width=0.800pt]
  \end{scope}
  \begin{scope}[scale=1.006,draw=black,line join=bevel,line cap=rect,line width=0.800pt]
  \end{scope}
  \begin{scope}[scale=1.006,draw=black,line join=round,line cap=round,line width=0.800pt]
    \path[draw] (225.5000,241.5000) -- (225.5000,225.5000) -- (241.5000,225.5000) -- (241.5000,241.5000) -- (225.5000,241.5000);



  \end{scope}
  \begin{scope}[scale=1.006,draw=black,line join=bevel,line cap=rect,line width=0.800pt]
  \end{scope}
  \begin{scope}[cm={{1.00588,0.0,0.0,1.00588,(232.359,238.394)}},draw=black,line join=bevel,line cap=rect,line width=0.800pt]
  \end{scope}
  \begin{scope}[cm={{1.00588,0.0,0.0,1.00588,(232.359,238.394)}},draw=black,line join=bevel,line cap=rect,line width=0.800pt]
  \end{scope}
  \begin{scope}[cm={{1.00588,0.0,0.0,1.00588,(232.359,238.394)}},draw=black,line join=bevel,line cap=rect,line width=0.800pt]
  \end{scope}
  \begin{scope}[cm={{1.00588,0.0,0.0,1.00588,(232.359,238.394)}},draw=black,line join=bevel,line cap=rect,line width=0.800pt]
  \end{scope}
  \begin{scope}[cm={{1.00588,0.0,0.0,1.00588,(232.359,238.394)}},draw=black,line join=bevel,line cap=rect,line width=0.800pt]
  \end{scope}
  \begin{scope}[cm={{1.00588,0.0,0.0,1.00588,(232.359,237.8426)}},draw=black,line join=bevel,line cap=rect,line width=0.800pt]
    \path[fill=black] (0.0000,0.0000) node[above right] (text1396) {\label{fig:ener-dyn-ii}ii};



  \end{scope}
  \begin{scope}[cm={{1.00588,0.0,0.0,1.00588,(232.359,238.394)}},draw=black,line join=bevel,line cap=rect,line width=0.800pt]
  \end{scope}
  \begin{scope}[cm={{1.00588,0.0,0.0,1.00588,(130.262,337.976)}},draw=black,line join=bevel,line cap=rect,line width=0.800pt]
  \end{scope}
  \begin{scope}[cm={{1.00588,0.0,0.0,1.00588,(130.262,337.976)}},draw=black,line join=bevel,line cap=rect,line width=0.800pt]
  \end{scope}
  \begin{scope}[cm={{1.00588,0.0,0.0,1.00588,(130.262,337.976)}},draw=black,line join=bevel,line cap=rect,line width=0.800pt]
  \end{scope}
  \begin{scope}[cm={{1.00588,0.0,0.0,1.00588,(130.262,337.976)}},draw=black,line join=bevel,line cap=rect,line width=0.800pt]
  \end{scope}
  \begin{scope}[cm={{1.00588,0.0,0.0,1.00588,(130.262,337.976)}},draw=black,line join=bevel,line cap=rect,line width=0.800pt]
  \end{scope}
  \begin{scope}[cm={{1.00588,0.0,0.0,1.00588,(130.262,337.976)}},draw=black,line join=bevel,line cap=rect,line width=0.800pt]
    \path[fill=black] (0.0000,0.0000) node[above right] (text1412) {Time (min)};



  \end{scope}
  \begin{scope}[cm={{1.00588,0.0,0.0,1.00588,(130.262,337.976)}},draw=black,line join=bevel,line cap=rect,line width=0.800pt]
  \end{scope}
  \begin{scope}[scale=1.006,draw=black,line join=bevel,line cap=rect,line width=0.800pt]
  \end{scope}
  \begin{scope}[scale=1.006,draw=black,line join=bevel,line cap=rect,line width=0.800pt]
  \end{scope}
  \begin{scope}[scale=1.006,draw=black,line join=bevel,line cap=rect,line width=0.800pt]
  \end{scope}
  \begin{scope}[scale=1.006,draw=black,line join=round,line cap=round,line width=0.480pt]
    \path[draw] (56.1000,255.4000) -- (56.1000,255.4000) -- (56.3000,261.0000) -- (56.5000,262.0000) -- (56.7000,262.2000) -- (56.9000,262.4000) -- (57.1000,261.4000) -- (57.3000,260.3000) -- (57.5000,259.7000) -- (57.7000,259.7000) -- (57.9000,259.8000) -- (58.1000,259.9000) -- (58.3000,260.0000) -- (58.4000,260.0000) -- (58.6000,260.0000) -- (58.8000,260.0000) -- (59.0000,260.0000) -- (59.2000,260.0000) -- (59.4000,260.0000) -- (59.6000,260.0000) -- (59.8000,260.0000) -- (60.0000,260.0000) -- (60.2000,260.0000) -- (60.4000,260.0000) -- (60.6000,260.2000) -- (60.8000,260.7000) -- (61.0000,261.3000) -- (61.2000,262.2000) -- (61.4000,263.3000) -- (61.6000,264.6000) -- (61.8000,266.1000) -- (62.0000,267.8000) -- (62.2000,269.7000) -- (62.4000,271.8000) -- (62.6000,274.1000) -- (62.8000,276.6000) -- (63.0000,279.2000) -- (63.1000,281.9000) -- (63.3000,284.7000) -- (63.5000,287.5000) -- (63.7000,290.1000) -- (63.9000,292.8000) -- (64.1000,295.5000) -- (64.3000,297.5000) -- (64.5000,298.2000) -- (64.7000,298.2000) -- (64.9000,297.9000) -- (65.1000,297.6000) -- (65.3000,297.5000) -- (65.5000,297.4000) -- (65.7000,297.4000) -- (65.9000,297.4000) -- (66.1000,297.4000) -- (66.3000,297.5000) -- (66.5000,297.5000) -- (66.7000,297.5000) -- (66.9000,297.5000) -- (67.1000,297.4000) -- (67.3000,297.1000) -- (67.5000,297.2000) -- (67.7000,297.2000) -- (67.9000,296.1000) -- (68.0000,294.1000) -- (68.2000,291.2000) -- (68.4000,288.0000) -- (68.6000,284.5000) -- (68.8000,281.1000) -- (69.0000,277.8000) -- (69.2000,274.8000) -- (69.4000,272.0000) -- (69.6000,269.4000) -- (69.8000,267.2000) -- (70.0000,265.3000) -- (70.2000,263.7000) -- (70.4000,262.4000) -- (70.6000,261.3000) -- (70.8000,260.5000) -- (71.0000,260.0000) -- (71.2000,259.6000) -- (71.4000,259.7000) -- (71.6000,259.8000) -- (71.8000,259.9000) -- (72.0000,260.0000) -- (72.2000,260.0000) -- (72.4000,260.0000) -- (72.6000,260.0000) -- (72.7000,260.0000) -- (72.9000,260.0000) -- (73.1000,260.0000) -- (73.3000,260.0000) -- (73.5000,260.0000) -- (73.7000,260.0000) -- (73.9000,260.0000) -- (74.1000,260.0000) -- (74.3000,260.1000) -- (74.5000,260.5000) -- (74.7000,261.1000) -- (74.9000,261.9000) -- (75.1000,262.9000) -- (75.3000,264.1000) -- (75.5000,265.4000) -- (75.7000,267.0000) -- (75.9000,268.9000) -- (76.1000,270.9000) -- (76.3000,273.1000) -- (76.5000,275.5000) -- (76.7000,278.0000) -- (76.9000,280.7000) -- (77.1000,283.4000) -- (77.3000,286.2000) -- (77.4000,288.9000) -- (77.6000,291.5000) -- (77.8000,294.2000) -- (78.0000,296.7000) -- (78.2000,298.1000) -- (78.4000,298.3000) -- (78.6000,298.0000) -- (78.8000,297.7000) -- (79.0000,297.5000) -- (79.2000,297.4000) -- (79.4000,297.4000) -- (79.6000,297.4000) -- (79.8000,297.4000) -- (80.0000,297.4000) -- (80.2000,297.5000) -- (80.4000,297.5000) -- (80.6000,297.5000) -- (80.8000,297.5000) -- (81.0000,298.3000) -- (81.2000,280.4000) -- (81.4000,268.1000) -- (81.6000,268.7000) -- (81.8000,267.3000) -- (82.0000,265.2000) -- (82.2000,262.6000) -- (82.3000,259.7000) -- (82.5000,256.7000) -- (82.7000,253.7000) -- (82.9000,250.7000) -- (83.1000,247.9000) -- (83.3000,245.3000) -- (83.5000,242.9000) -- (83.7000,240.7000) -- (83.9000,238.8000) -- (84.1000,237.2000) -- (84.3000,235.7000) -- (84.5000,234.5000) -- (84.7000,233.5000) -- (84.9000,232.7000) -- (85.1000,232.2000) -- (85.3000,231.7000) -- (85.5000,231.6000) -- (85.7000,231.6000) -- (85.9000,231.7000) -- (86.1000,231.8000) -- (86.3000,231.9000) -- (86.5000,231.9000) -- (86.7000,231.9000) -- (86.9000,231.9000) -- (87.0000,231.9000) -- (87.2000,231.9000) -- (87.4000,231.9000) -- (87.6000,231.9000) -- (87.8000,231.9000) -- (88.0000,231.9000) -- (88.2000,231.9000) -- (88.4000,231.9000) -- (88.6000,232.1000) -- (88.8000,232.5000) -- (89.0000,233.1000) -- (89.2000,234.0000) -- (89.4000,235.0000) -- (89.6000,236.2000) -- (89.8000,237.7000) -- (90.0000,239.3000) -- (90.2000,241.2000) -- (90.4000,243.2000) -- (90.6000,245.4000) -- (90.8000,247.9000) -- (91.0000,250.4000) -- (91.2000,253.1000) -- (91.4000,255.8000) -- (91.6000,258.6000) -- (91.8000,261.3000) -- (91.9000,263.9000) -- (92.1000,266.6000) -- (92.3000,269.0000) -- (92.5000,270.1000) -- (92.7000,270.2000) -- (92.9000,269.8000) -- (93.1000,269.5000) -- (93.3000,269.4000) -- (93.5000,269.3000) -- (93.7000,269.3000) -- (93.9000,269.3000) -- (94.1000,269.3000) -- (94.3000,269.3000) -- (94.5000,269.3000) -- (94.7000,269.3000) -- (94.9000,269.3000) -- (95.1000,269.3000) -- (95.3000,269.1000) -- (95.5000,269.1000) -- (95.7000,269.2000) -- (95.9000,268.5000) -- (96.1000,266.9000) -- (96.3000,264.6000) -- (96.5000,261.9000) -- (96.6000,259.0000) -- (96.8000,255.9000) -- (97.0000,252.9000) -- (97.2000,250.0000) -- (97.4000,247.3000) -- (97.6000,244.7000) -- (97.8000,242.4000) -- (98.0000,240.3000) -- (98.2000,238.4000) -- (98.4000,236.8000) -- (98.6000,235.4000) -- (98.8000,234.2000) -- (99.0000,233.3000) -- (99.2000,232.6000) -- (99.4000,232.0000) -- (99.6000,231.7000) -- (99.8000,231.6000) -- (100.0000,231.7000) -- (100.2000,231.8000) -- (100.4000,231.8000) -- (100.6000,231.9000) -- (100.8000,231.9000) -- (101.0000,231.9000) -- (101.2000,231.9000) -- (101.3000,231.9000) -- (101.5000,231.9000) -- (101.7000,231.9000) -- (101.9000,231.9000) -- (102.1000,231.9000) -- (102.3000,231.9000) -- (102.5000,231.9000) -- (102.7000,231.9000) -- (102.9000,232.2000) -- (103.1000,232.7000) -- (103.3000,233.4000) -- (103.5000,234.3000) -- (103.7000,235.4000) -- (103.9000,236.7000) -- (104.1000,238.2000) -- (104.3000,239.9000) -- (104.5000,241.8000) -- (104.7000,244.0000) -- (104.9000,246.3000) -- (105.1000,248.7000) -- (105.3000,251.4000) -- (105.5000,254.1000) -- (105.7000,256.8000) -- (105.9000,259.6000) -- (106.0000,262.2000) -- (106.2000,264.9000) -- (106.4000,267.6000) -- (106.6000,269.5000) -- (106.8000,270.2000) -- (107.0000,270.0000) -- (107.2000,269.7000) -- (107.4000,269.5000) -- (107.6000,269.3000) -- (107.8000,269.3000) -- (108.0000,269.3000) -- (108.2000,269.3000) -- (108.4000,269.3000) -- (108.6000,269.3000) -- (108.8000,269.3000) -- (109.0000,269.3000) -- (109.2000,269.3000) -- (109.4000,269.2000) -- (109.6000,269.0000) -- (109.8000,269.2000) -- (110.0000,269.0000) -- (110.2000,268.0000) -- (110.4000,266.1000) -- (110.6000,263.6000) -- (110.8000,260.8000) -- (110.9000,257.8000) -- (111.1000,254.8000) -- (111.3000,251.8000) -- (111.5000,248.9000) -- (111.7000,246.2000) -- (111.9000,243.8000) -- (112.1000,241.5000) -- (112.3000,239.5000) -- (112.5000,237.7000) -- (112.7000,236.2000) -- (112.9000,234.9000) -- (113.1000,233.8000) -- (113.3000,233.0000) -- (113.5000,232.3000) -- (113.7000,231.9000) -- (113.9000,231.6000) -- (114.1000,231.6000) -- (114.3000,231.7000) -- (114.5000,231.8000) -- (114.7000,231.9000) -- (114.9000,231.9000) -- (115.1000,231.9000) -- (115.3000,231.9000) -- (115.5000,231.9000) -- (115.6000,231.9000) -- (115.8000,231.9000) -- (116.0000,231.9000) -- (116.2000,231.9000) -- (116.4000,231.9000) -- (116.6000,231.9000) -- (116.8000,231.9000) -- (117.0000,232.0000) -- (117.2000,232.3000) -- (117.4000,232.9000) -- (117.6000,233.7000) -- (117.8000,234.7000) -- (118.0000,235.9000) -- (118.2000,237.2000) -- (118.4000,238.8000) -- (118.6000,240.6000) -- (118.8000,242.6000) -- (119.0000,244.8000) -- (119.2000,247.2000) -- (119.4000,249.8000) -- (119.6000,252.4000) -- (119.8000,255.2000) -- (120.0000,257.9000) -- (120.2000,260.6000) -- (120.4000,263.2000) -- (120.5000,266.0000) -- (120.7000,268.5000) -- (120.9000,269.9000) -- (121.1000,270.2000) -- (121.3000,269.9000) -- (121.5000,269.6000) -- (121.7000,269.4000) -- (121.9000,269.3000) -- (122.1000,269.3000) -- (122.3000,269.3000) -- (122.5000,269.3000) -- (122.7000,269.3000) -- (122.9000,269.3000) -- (123.1000,269.3000) -- (123.3000,269.3000) -- (123.5000,269.3000) -- (123.7000,269.1000) -- (123.9000,269.1000) -- (124.1000,269.2000) -- (124.3000,268.7000) -- (124.5000,267.3000) -- (124.7000,265.2000) -- (124.9000,262.5000) -- (125.1000,259.6000) -- (125.2000,256.6000) -- (125.4000,253.6000) -- (125.6000,250.6000) -- (125.8000,247.8000) -- (126.0000,245.2000) -- (126.2000,242.8000) -- (126.4000,240.7000) -- (126.6000,238.8000) -- (126.8000,237.1000) -- (127.0000,235.7000) -- (127.2000,234.5000) -- (127.4000,233.5000) -- (127.6000,232.7000) -- (127.8000,232.1000) -- (128.0000,231.7000) -- (128.2000,231.6000) -- (128.4000,231.6000) -- (128.6000,231.8000) -- (128.8000,231.8000) -- (129.0000,231.9000) -- (129.2000,231.9000) -- (129.4000,231.9000) -- (129.6000,231.9000) -- (129.8000,231.9000) -- (129.9000,231.9000) -- (130.1000,231.9000) -- (130.3000,231.9000) -- (130.5000,231.9000) -- (130.7000,231.9000) -- (130.9000,231.9000) -- (131.1000,231.9000) -- (131.3000,232.1000) -- (131.5000,232.5000) -- (131.7000,233.2000) -- (131.9000,234.1000) -- (132.1000,235.1000) -- (132.3000,236.4000) -- (132.5000,237.9000) -- (132.7000,239.6000) -- (132.9000,241.4000) -- (133.1000,243.5000) -- (133.3000,245.8000) -- (133.5000,248.2000) -- (133.7000,250.8000) -- (133.9000,253.5000) -- (134.1000,256.3000) -- (134.3000,259.1000) -- (134.5000,261.7000) -- (134.7000,264.3000) -- (134.8000,267.1000) -- (135.0000,269.2000) -- (135.2000,270.1000) -- (135.4000,270.1000) -- (135.6000,269.8000) -- (135.8000,269.5000) -- (136.0000,269.3000) -- (136.2000,269.3000) -- (136.4000,269.3000) -- (136.6000,269.3000) -- (136.8000,269.3000) -- (137.0000,269.3000) -- (137.2000,269.3000) -- (137.4000,269.3000) -- (137.6000,269.3000) -- (137.8000,269.3000) -- (138.0000,269.0000) -- (138.2000,269.2000) -- (138.4000,269.1000) -- (138.6000,268.2000) -- (138.8000,266.5000) -- (139.0000,264.1000) -- (139.2000,261.3000) -- (139.4000,258.4000) -- (139.5000,255.3000) -- (139.7000,252.3000) -- (139.9000,249.4000) -- (140.1000,246.7000) -- (140.3000,244.2000) -- (140.5000,241.9000) -- (140.7000,239.9000) -- (140.9000,238.1000) -- (141.1000,236.5000) -- (141.3000,235.1000) -- (141.5000,234.0000) -- (141.7000,233.1000) -- (141.9000,232.4000) -- (142.1000,231.9000) -- (142.3000,231.6000) -- (142.5000,231.6000) -- (142.7000,231.7000) -- (142.9000,231.8000) -- (143.1000,231.8000) -- (143.3000,231.9000) -- (143.5000,231.9000) -- (143.7000,231.9000) -- (143.9000,231.9000) -- (144.1000,231.9000) -- (144.3000,231.9000) -- (144.4000,231.9000) -- (144.6000,231.9000) -- (144.8000,231.9000) -- (145.0000,231.9000) -- (145.2000,231.9000) -- (145.4000,232.0000) -- (145.6000,232.3000) -- (145.8000,232.8000) -- (146.0000,233.5000) -- (146.2000,234.5000) -- (146.4000,235.6000) -- (146.6000,237.0000) -- (146.8000,238.5000) -- (147.0000,240.3000) -- (147.2000,242.3000) -- (147.4000,244.4000) -- (147.6000,246.8000) -- (147.8000,249.3000) -- (148.0000,252.0000) -- (148.2000,254.7000) -- (148.4000,257.5000) -- (148.6000,260.2000) -- (148.8000,262.8000) -- (149.0000,265.5000) -- (149.1000,268.1000) -- (149.3000,269.8000) -- (149.5000,270.2000) -- (149.7000,270.0000) -- (149.9000,269.6000) -- (150.1000,269.4000) -- (150.3000,269.3000) -- (150.5000,269.3000) -- (150.7000,269.3000) -- (150.9000,269.3000) -- (151.1000,269.3000) -- (151.3000,269.3000) -- (151.5000,269.3000) -- (151.7000,269.3000) -- (151.9000,269.3000) -- (152.1000,269.2000) -- (152.3000,269.0000) -- (152.5000,269.2000) -- (152.7000,268.9000) -- (152.9000,267.6000) -- (153.1000,265.6000) -- (153.3000,263.0000) -- (153.5000,260.1000) -- (153.7000,257.1000) -- (153.8000,254.1000) -- (154.0000,251.1000) -- (154.2000,248.3000) -- (154.4000,245.7000) -- (154.6000,243.2000) -- (154.8000,241.0000) -- (155.0000,239.1000) -- (155.2000,237.1000) -- (155.4000,238.9000) -- (155.6000,242.0000) -- (155.8000,240.7000) -- (156.0000,239.8000) -- (156.2000,239.2000) -- (156.4000,238.8000) -- (156.6000,238.6000) -- (156.8000,238.6000) -- (157.0000,238.8000) -- (157.2000,238.9000) -- (157.4000,238.9000) -- (157.6000,238.9000) -- (157.8000,238.9000) -- (158.0000,238.9000) -- (158.2000,238.9000) -- (158.4000,238.9000) -- (158.6000,238.9000) -- (158.7000,238.9000) -- (158.9000,238.9000) -- (159.1000,238.9000) -- (159.3000,238.9000) -- (159.5000,238.9000) -- (159.7000,239.1000) -- (159.9000,239.5000) -- (160.1000,240.1000) -- (160.3000,240.9000) -- (160.5000,242.2000) -- (160.7000,241.5000) -- (160.9000,237.5000) -- (161.1000,239.2000) -- (161.3000,241.1000) -- (161.5000,243.1000) -- (161.7000,245.4000) -- (161.9000,247.8000) -- (162.1000,250.4000) -- (162.3000,253.0000) -- (162.5000,255.8000) -- (162.7000,258.6000) -- (162.9000,261.3000) -- (163.1000,263.8000) -- (163.3000,266.6000) -- (163.4000,268.9000) -- (163.6000,270.1000) -- (163.8000,270.1000) -- (164.0000,269.8000) -- (164.2000,269.5000) -- (164.4000,269.4000) -- (164.6000,269.3000) -- (164.8000,269.3000) -- (165.0000,269.3000) -- (165.2000,269.3000) -- (165.4000,269.3000) -- (165.6000,269.3000) -- (165.8000,269.3000) -- (166.0000,269.3000) -- (166.2000,269.3000) -- (166.4000,269.0000) -- (166.6000,269.1000) -- (166.8000,269.2000) -- (167.0000,268.5000) -- (167.2000,266.9000) -- (167.4000,264.6000) -- (167.6000,261.9000) -- (167.8000,258.9000) -- (168.0000,255.9000) -- (168.2000,252.9000) -- (168.3000,250.0000) -- (168.5000,246.9000) -- (168.7000,247.8000) -- (168.9000,249.7000) -- (169.1000,247.3000) -- (169.3000,245.3000) -- (169.5000,244.1000) -- (169.7000,248.5000) -- (169.9000,248.4000) -- (170.1000,247.3000) -- (170.3000,246.6000) -- (170.5000,246.1000) -- (170.7000,245.7000) -- (170.9000,245.6000) -- (171.1000,245.7000) -- (171.3000,245.8000) -- (171.5000,245.9000) -- (171.7000,245.9000) -- (171.9000,245.9000) -- (172.1000,245.9000) -- (172.3000,245.9000) -- (172.5000,245.9000) -- (172.7000,245.9000) -- (172.9000,245.9000) -- (173.0000,245.9000) -- (173.2000,245.9000) -- (173.4000,245.9000) -- (173.6000,245.9000) -- (173.8000,246.0000) -- (174.0000,246.2000) -- (174.2000,246.7000) -- (174.4000,247.4000) -- (174.6000,248.4000) -- (174.8000,249.6000) -- (175.0000,245.8000) -- (175.2000,245.0000) -- (175.4000,247.0000) -- (175.6000,249.1000) -- (175.8000,249.5000) -- (176.0000,246.3000) -- (176.2000,248.8000) -- (176.4000,251.4000) -- (176.6000,254.1000) -- (176.8000,256.9000) -- (177.0000,259.6000) -- (177.2000,262.3000) -- (177.4000,264.9000) -- (177.6000,267.7000) -- (177.7000,269.5000) -- (177.9000,270.2000) -- (178.1000,270.0000) -- (178.3000,269.7000) -- (178.5000,269.5000) -- (178.7000,269.3000) -- (178.9000,269.3000) -- (179.1000,269.3000) -- (179.3000,269.3000) -- (179.5000,269.3000) -- (179.7000,269.3000) -- (179.9000,269.3000) -- (180.1000,269.3000) -- (180.3000,269.3000) -- (180.5000,269.2000) -- (180.7000,269.0000) -- (180.9000,269.2000) -- (181.1000,269.0000) -- (181.3000,267.9000) -- (181.5000,266.1000) -- (181.7000,263.6000) -- (181.9000,260.7000) -- (182.1000,257.6000) -- (182.3000,255.1000) -- (182.4000,258.0000) -- (182.6000,255.9000) -- (182.8000,253.5000) -- (183.0000,256.9000) -- (183.2000,255.7000) -- (183.4000,253.5000) -- (183.6000,251.5000) -- (183.8000,253.1000) -- (184.0000,256.2000) -- (184.2000,254.9000) -- (184.4000,254.0000) -- (184.6000,253.4000) -- (184.8000,252.9000) -- (185.0000,252.7000) -- (185.2000,252.7000) -- (185.4000,252.8000) -- (185.6000,252.9000) -- (185.8000,252.9000) -- (186.0000,253.0000) -- (186.2000,252.9000) -- (186.4000,252.9000) -- (186.6000,252.9000) -- (186.8000,252.9000) -- (187.0000,252.9000) -- (187.2000,252.9000) -- (187.3000,252.9000) -- (187.5000,252.9000) -- (187.7000,252.9000) -- (187.9000,253.0000) -- (188.1000,253.1000) -- (188.3000,253.4000) -- (188.5000,254.0000) -- (188.7000,254.8000) -- (188.9000,256.0000) -- (189.1000,255.4000) -- (189.3000,251.3000) -- (189.5000,252.9000) -- (189.7000,254.8000) -- (189.9000,257.0000) -- (190.1000,254.2000) -- (190.3000,254.1000) -- (190.5000,257.1000) -- (190.7000,254.9000) -- (190.9000,255.0000) -- (191.1000,258.0000) -- (191.3000,260.7000) -- (191.5000,263.3000) -- (191.7000,266.0000) -- (191.9000,268.6000) -- (192.0000,269.9000) -- (192.2000,270.2000) -- (192.4000,269.9000) -- (192.6000,269.6000) -- (192.8000,269.4000) -- (193.0000,269.3000) -- (193.2000,269.3000) -- (193.4000,269.3000) -- (193.6000,269.3000) -- (193.8000,269.3000) -- (194.0000,269.3000) -- (194.2000,269.3000) -- (194.4000,269.3000) -- (194.6000,269.3000) -- (194.8000,269.1000) -- (195.0000,269.1000) -- (195.2000,269.2000) -- (195.4000,268.7000) -- (195.6000,267.3000) -- (195.8000,265.1000) -- (196.0000,262.4000) -- (196.2000,259.9000) -- (196.4000,262.5000) -- (196.6000,263.8000) -- (196.8000,264.9000) -- (196.9000,261.5000) -- (197.1000,262.2000) -- (197.3000,264.2000) -- (197.5000,261.7000) -- (197.7000,259.8000) -- (197.9000,258.4000) -- (198.1000,262.7000) -- (198.3000,262.7000) -- (198.5000,261.6000) -- (198.7000,260.8000) -- (198.9000,260.2000) -- (199.1000,259.8000) -- (199.3000,259.7000) -- (199.5000,259.8000) -- (199.7000,259.9000) -- (199.9000,260.0000) -- (200.1000,260.0000) -- (200.3000,260.0000) -- (200.5000,260.0000) -- (200.7000,260.0000) -- (200.9000,260.0000) -- (201.1000,260.0000) -- (201.3000,260.0000) -- (201.5000,260.0000) -- (201.6000,260.0000) -- (201.8000,260.0000) -- (202.0000,260.0000) -- (202.2000,260.0000) -- (202.4000,260.2000) -- (202.6000,260.7000) -- (202.8000,261.3000) -- (203.0000,262.2000) -- (203.2000,263.5000) -- (203.4000,259.7000) -- (203.6000,258.7000) -- (203.8000,260.7000) -- (204.0000,262.7000) -- (204.2000,263.3000) -- (204.4000,259.9000) -- (204.6000,262.5000) -- (204.8000,263.7000) -- (205.0000,260.7000) -- (205.2000,263.5000) -- (205.4000,265.0000) -- (205.6000,261.9000) -- (205.8000,264.3000) -- (206.0000,267.1000) -- (206.2000,269.3000) -- (206.3000,270.1000) -- (206.5000,270.1000) -- (206.7000,269.8000) -- (206.9000,269.5000) -- (207.1000,269.3000) -- (207.3000,269.3000) -- (207.5000,269.3000) -- (207.7000,269.3000) -- (207.9000,269.3000) -- (208.1000,269.3000) -- (208.3000,269.3000) -- (208.5000,269.3000) -- (208.7000,269.3000) -- (208.9000,269.3000) -- (209.1000,269.0000) -- (209.3000,269.2000) -- (209.5000,269.1000) -- (209.7000,268.2000) -- (209.9000,266.4000) -- (210.1000,264.5000) -- (210.3000,266.7000) -- (210.5000,274.9000) -- (210.7000,284.3000) -- (210.9000,280.4000) -- (211.1000,277.6000) -- (211.2000,274.8000) -- (211.4000,272.3000) -- (211.6000,270.0000) -- (211.8000,268.0000) -- (212.0000,266.2000) -- (212.2000,264.6000) -- (212.4000,263.3000) -- (212.6000,262.1000) -- (212.8000,261.2000) -- (213.0000,260.6000) -- (213.2000,260.1000) -- (213.4000,259.7000) -- (213.6000,259.7000) -- (213.8000,259.8000) -- (214.0000,259.9000) -- (214.2000,260.0000) -- (214.4000,260.0000) -- (214.6000,260.0000) -- (214.8000,260.0000) -- (215.0000,260.0000) -- (215.2000,260.0000) -- (215.4000,260.0000) -- (215.6000,260.0000) -- (215.8000,260.0000) -- (215.9000,260.0000) -- (216.1000,260.0000) -- (216.3000,260.0000) -- (216.5000,260.1000) -- (216.7000,260.4000) -- (216.9000,260.9000) -- (217.1000,261.7000) -- (217.3000,262.6000) -- (217.5000,263.8000) -- (217.7000,265.1000) -- (217.9000,266.7000) -- (218.1000,268.4000) -- (218.3000,270.6000) -- (218.5000,268.0000) -- (218.7000,267.6000) -- (218.9000,270.6000) -- (219.1000,268.6000) -- (219.3000,268.5000) -- (219.5000,271.8000) -- (219.7000,269.9000) -- (219.9000,269.5000) -- (220.1000,272.8000) -- (220.3000,271.0000) -- (220.5000,269.5000) -- (220.7000,270.2000) -- (220.8000,270.0000) -- (221.0000,269.6000) -- (221.2000,269.4000) -- (221.4000,269.3000) -- (221.6000,269.3000) -- (221.8000,269.3000) -- (222.0000,269.3000) -- (222.2000,269.3000) -- (222.4000,269.3000) -- (222.6000,269.2000) -- (222.8000,269.9000) -- (223.0000,275.8000) -- (223.2000,276.3000) -- (223.4000,276.1000) -- (223.6000,276.3000) -- (223.8000,275.9000) -- (224.0000,274.7000) -- (224.2000,271.8000) -- (224.4000,279.4000) -- (224.6000,289.1000) -- (224.8000,285.2000) -- (225.0000,282.2000) -- (225.2000,279.3000) -- (225.4000,276.4000) -- (225.5000,273.8000) -- (225.7000,271.4000) -- (225.9000,269.2000) -- (226.1000,267.2000) -- (226.3000,265.5000) -- (226.5000,264.0000) -- (226.7000,262.8000) -- (226.9000,261.8000) -- (227.1000,260.9000) -- (227.3000,260.3000) -- (227.5000,259.9000) -- (227.7000,259.7000) -- (227.9000,259.7000) -- (228.1000,259.9000) -- (228.3000,259.9000) -- (228.5000,260.0000) -- (228.7000,260.0000) -- (228.9000,260.0000) -- (229.1000,260.0000) -- (229.3000,260.0000) -- (229.5000,260.0000) -- (229.7000,260.0000) -- (229.9000,260.0000) -- (230.1000,260.0000) -- (230.3000,260.0000) -- (230.4000,260.0000) -- (230.6000,260.0000) -- (230.8000,260.2000) -- (231.0000,260.6000) -- (231.2000,261.2000) -- (231.4000,262.0000) -- (231.6000,263.0000) -- (231.8000,264.3000) -- (232.0000,265.7000) -- (232.2000,267.3000) -- (232.4000,269.2000) -- (232.6000,271.2000) -- (232.8000,273.5000) -- (233.0000,276.0000) -- (233.2000,277.3000) -- (233.4000,274.3000) -- (233.6000,277.0000) -- (233.8000,278.6000) -- (234.0000,275.5000) -- (234.2000,277.9000) -- (234.4000,279.7000) -- (234.6000,276.2000) -- (234.8000,277.0000) -- (234.9000,277.2000) -- (235.1000,276.9000) -- (235.3000,276.6000) -- (235.5000,276.4000) -- (235.7000,276.3000) -- (235.9000,276.3000) -- (236.1000,276.3000) -- (236.3000,276.4000) -- (236.5000,276.3000) -- (236.7000,276.9000) -- (236.9000,282.8000) -- (237.1000,283.5000) -- (237.3000,283.4000) -- (237.5000,283.1000) -- (237.7000,283.0000) -- (237.9000,284.0000) -- (238.1000,295.1000) -- (238.3000,295.2000) -- (238.5000,292.8000) -- (238.7000,290.0000) -- (238.9000,287.1000) -- (239.1000,284.0000) -- (239.3000,281.0000) -- (239.5000,278.1000) -- (239.7000,275.4000) -- (239.8000,272.8000) -- (240.0000,270.5000) -- (240.2000,268.4000) -- (240.4000,266.5000) -- (240.6000,264.9000) -- (240.8000,263.5000) -- (241.0000,262.3000) -- (241.2000,261.4000) -- (241.4000,260.7000) -- (241.6000,260.1000) -- (241.8000,259.8000) -- (242.0000,259.7000) -- (242.2000,259.8000) -- (242.4000,259.9000) -- (242.6000,260.0000) -- (242.8000,260.0000) -- (243.0000,260.0000) -- (243.2000,260.0000) -- (243.4000,260.0000) -- (243.6000,260.0000) -- (243.8000,260.0000) -- (244.0000,260.0000) -- (244.2000,260.0000) -- (244.4000,260.0000) -- (244.5000,260.0000) -- (244.7000,260.0000) -- (244.9000,260.0000) -- (245.1000,260.3000) -- (245.3000,260.8000) -- (245.5000,261.5000) -- (245.7000,262.4000) -- (245.9000,263.5000) -- (246.1000,264.8000) -- (246.3000,266.3000) -- (246.5000,268.0000) -- (246.7000,270.0000) -- (246.9000,272.1000) -- (247.1000,274.4000) -- (247.3000,276.9000) -- (247.5000,279.5000) -- (247.7000,282.2000) -- (247.9000,285.3000) -- (248.1000,283.6000) -- (248.3000,283.1000) -- (248.5000,286.3000) -- (248.7000,284.7000) -- (248.9000,283.3000) -- (249.1000,284.2000) -- (249.3000,284.1000) -- (249.4000,283.8000) -- (249.6000,283.5000) -- (249.8000,283.4000) -- (250.0000,283.4000) -- (250.2000,283.4000) -- (250.4000,283.4000) -- (250.6000,283.4000) -- (250.8000,283.1000) -- (251.0000,286.8000) -- (251.2000,290.7000) -- (251.4000,290.1000) -- (251.6000,293.6000) -- (251.8000,297.4000);



  \end{scope}
  \begin{scope}[scale=1.006,draw=black,line join=bevel,line cap=rect,line width=0.800pt]
  \end{scope}
  \begin{scope}[scale=1.006,draw=black,line join=bevel,line cap=rect,line width=0.800pt]
  \end{scope}
  \begin{scope}[scale=1.006,draw=black,line join=bevel,line cap=rect,line width=0.800pt]
  \end{scope}
  \begin{scope}[scale=1.006,draw=black,line join=bevel,line cap=rect,line width=0.800pt]
  \end{scope}
  \begin{scope}[scale=1.006,draw=c00ff00,line join=round,line cap=round,line width=0.480pt]
    \path[draw] (127.3000,221.1000) -- (127.6000,221.2000) -- (127.9000,221.4000) -- (128.2000,221.5000) -- (128.5000,221.7000) -- (128.8000,221.8000) -- (129.1000,222.0000) -- (129.4000,222.1000) -- (129.7000,222.3000) -- (130.0000,222.4000) -- (130.3000,222.6000) -- (130.6000,222.7000) -- (130.9000,222.9000) -- (131.2000,223.0000) -- (131.5000,223.2000) -- (131.8000,223.3000) -- (132.1000,223.5000) -- (132.3000,223.6000) -- (132.6000,223.8000) -- (132.9000,223.9000) -- (133.2000,224.1000) -- (133.5000,224.2000) -- (133.8000,224.4000) -- (134.1000,224.5000) -- (134.4000,224.7000) -- (134.7000,224.8000) -- (135.0000,225.0000) -- (135.3000,225.1000) -- (135.6000,225.2000) -- (135.9000,225.4000) -- (136.2000,225.5000) -- (136.5000,225.7000) -- (136.8000,225.8000) -- (137.1000,226.0000) -- (137.4000,226.1000) -- (137.7000,226.3000) -- (138.0000,226.4000) -- (138.3000,226.6000) -- (138.6000,226.7000) -- (138.9000,226.9000) -- (139.2000,227.0000) -- (139.4000,227.2000) -- (139.7000,227.3000) -- (140.0000,227.5000) -- (140.3000,227.6000) -- (140.6000,227.8000) -- (140.9000,227.9000) -- (141.2000,228.1000) -- (141.5000,228.2000) -- (141.8000,228.4000) -- (142.1000,228.5000) -- (142.4000,228.7000) -- (142.7000,228.8000) -- (143.0000,229.0000) -- (143.3000,229.1000) -- (143.6000,229.3000) -- (143.9000,229.4000) -- (144.2000,229.6000) -- (144.5000,229.7000) -- (144.8000,229.8000) -- (145.1000,230.0000) -- (145.4000,230.1000) -- (145.7000,230.3000) -- (146.0000,230.4000) -- (146.3000,230.6000) -- (146.6000,230.7000) -- (146.8000,230.9000) -- (147.1000,231.0000) -- (147.4000,231.2000) -- (147.7000,231.3000) -- (148.0000,231.5000) -- (148.3000,231.6000) -- (148.6000,231.8000) -- (148.9000,231.9000) -- (149.2000,232.1000) -- (149.5000,232.2000) -- (149.8000,232.4000) -- (150.1000,232.5000) -- (150.4000,232.7000) -- (150.7000,232.8000) -- (151.0000,233.0000) -- (151.3000,233.1000) -- (151.6000,233.3000) -- (151.9000,233.4000) -- (152.2000,233.6000) -- (152.5000,233.7000) -- (152.8000,233.9000) -- (153.1000,234.0000) -- (153.4000,234.1000) -- (153.7000,234.3000) -- (153.9000,234.4000) -- (154.2000,234.6000) -- (154.5000,234.7000) -- (154.8000,234.9000) -- (155.1000,235.0000) -- (155.4000,235.2000) -- (155.7000,235.3000) -- (156.0000,235.5000) -- (156.3000,235.6000) -- (156.6000,235.8000) -- (156.9000,235.9000) -- (157.2000,236.1000) -- (157.5000,236.2000) -- (157.8000,236.4000) -- (158.1000,236.5000) -- (158.4000,236.7000) -- (158.7000,236.8000) -- (159.0000,237.0000) -- (159.3000,237.1000) -- (159.6000,237.3000) -- (159.9000,237.4000) -- (160.2000,237.6000) -- (160.5000,237.7000) -- (160.8000,237.9000) -- (161.0000,238.0000) -- (161.3000,238.2000) -- (161.6000,238.3000) -- (161.9000,238.5000) -- (162.2000,238.6000) -- (162.5000,238.6000) -- (162.8000,238.7000) -- (163.1000,238.9000) -- (163.4000,239.0000) -- (163.7000,239.2000) -- (164.0000,239.3000) -- (164.3000,239.5000) -- (164.6000,239.6000) -- (164.9000,239.8000) -- (165.2000,239.9000) -- (165.5000,240.1000) -- (165.8000,240.2000) -- (166.1000,240.4000) -- (166.4000,240.5000) -- (166.7000,240.7000) -- (167.0000,240.8000) -- (167.3000,241.0000) -- (167.6000,241.1000) -- (167.9000,241.3000) -- (168.2000,241.4000) -- (168.4000,241.6000) -- (168.7000,241.7000) -- (169.0000,241.9000) -- (169.3000,242.0000) -- (169.6000,242.2000) -- (169.9000,242.3000) -- (170.2000,242.5000) -- (170.5000,242.6000) -- (170.8000,242.8000) -- (171.1000,242.9000) -- (171.4000,243.1000) -- (171.7000,243.2000) -- (172.0000,243.3000) -- (172.3000,243.5000) -- (172.6000,243.6000) -- (172.9000,243.8000) -- (173.2000,243.9000) -- (173.5000,244.1000) -- (173.8000,244.2000) -- (174.1000,244.4000) -- (174.4000,244.5000) -- (174.7000,244.7000) -- (175.0000,244.8000) -- (175.3000,245.0000) -- (175.5000,245.1000) -- (175.8000,245.3000) -- (176.1000,245.4000) -- (176.4000,245.6000) -- (176.7000,245.7000) -- (177.0000,245.9000) -- (177.3000,246.0000) -- (177.6000,246.2000) -- (177.9000,246.3000) -- (178.2000,246.5000) -- (178.5000,246.6000) -- (178.8000,246.8000) -- (179.1000,246.9000) -- (179.4000,247.1000) -- (179.7000,247.2000) -- (180.0000,247.4000) -- (180.3000,247.5000) -- (180.6000,247.6000) -- (180.9000,247.8000) -- (181.2000,247.9000) -- (181.5000,248.1000) -- (181.8000,248.2000) -- (182.1000,248.4000) -- (182.4000,248.5000) -- (182.7000,248.7000) -- (182.9000,248.8000) -- (183.2000,249.0000) -- (183.5000,249.1000) -- (183.8000,249.3000) -- (184.1000,249.4000) -- (184.4000,249.6000) -- (184.7000,249.7000) -- (185.0000,249.9000) -- (185.3000,250.0000) -- (185.6000,250.2000) -- (185.9000,250.3000) -- (186.2000,250.5000) -- (186.5000,250.6000) -- (186.8000,250.8000) -- (187.1000,250.9000) -- (187.4000,251.1000) -- (187.7000,251.2000) -- (188.0000,251.4000) -- (188.3000,251.5000) -- (188.6000,251.7000) -- (188.9000,251.8000) -- (189.2000,252.0000) -- (189.5000,252.1000) -- (189.8000,252.2000) -- (190.0000,252.4000) -- (190.3000,252.5000) -- (190.6000,252.7000) -- (190.9000,252.8000) -- (191.2000,253.0000) -- (191.5000,253.1000) -- (191.8000,253.3000) -- (192.1000,253.4000) -- (192.4000,253.6000) -- (192.7000,253.7000) -- (193.0000,253.9000) -- (193.3000,254.0000) -- (193.6000,254.2000) -- (193.9000,254.3000) -- (194.2000,254.5000) -- (194.5000,254.6000) -- (194.8000,254.8000) -- (195.1000,254.9000) -- (195.4000,255.1000) -- (195.7000,255.2000) -- (196.0000,255.4000) -- (196.3000,255.5000) -- (196.6000,255.7000) -- (196.9000,255.8000) -- (197.2000,256.0000) -- (197.4000,256.1000) -- (197.7000,256.3000) -- (198.0000,256.4000) -- (198.3000,256.5000) -- (198.6000,256.7000) -- (198.9000,256.8000) -- (199.2000,257.0000) -- (199.5000,257.1000) -- (199.8000,257.3000) -- (200.1000,257.4000) -- (200.4000,257.6000) -- (200.7000,257.7000) -- (201.0000,257.9000) -- (201.3000,258.0000) -- (201.6000,258.2000) -- (201.9000,258.3000) -- (202.2000,258.5000) -- (202.5000,258.6000) -- (202.8000,258.8000) -- (203.1000,258.9000) -- (203.4000,259.1000) -- (203.7000,259.2000) -- (204.0000,259.4000) -- (204.3000,259.5000) -- (204.5000,259.7000) -- (204.8000,259.8000) -- (205.1000,260.0000) -- (205.4000,260.1000) -- (205.7000,260.3000) -- (206.0000,260.4000) -- (206.3000,260.6000) -- (206.6000,260.7000) -- (206.9000,260.9000) -- (207.2000,261.0000) -- (207.5000,261.1000) -- (207.8000,261.3000) -- (208.1000,261.4000) -- (208.4000,261.6000) -- (208.7000,261.7000) -- (209.0000,261.9000) -- (209.3000,262.0000) -- (209.6000,262.2000) -- (209.9000,262.3000) -- (210.2000,262.5000) -- (210.5000,262.6000) -- (210.8000,262.8000) -- (211.1000,262.9000) -- (211.4000,263.1000) -- (211.7000,263.2000) -- (211.9000,263.4000) -- (212.2000,263.5000) -- (212.5000,263.7000) -- (212.8000,263.8000) -- (213.1000,264.0000) -- (213.4000,264.1000) -- (213.7000,264.3000) -- (214.0000,264.4000) -- (214.3000,264.6000) -- (214.6000,264.7000) -- (214.9000,264.9000) -- (215.2000,265.0000) -- (215.5000,265.2000) -- (215.8000,265.3000) -- (216.1000,265.5000) -- (216.4000,265.6000) -- (216.7000,265.7000) -- (217.0000,265.9000) -- (217.3000,266.0000) -- (217.6000,266.2000) -- (217.9000,266.3000) -- (218.2000,266.5000) -- (218.5000,266.6000) -- (218.8000,266.8000) -- (219.0000,266.9000) -- (219.3000,267.1000) -- (219.6000,267.2000) -- (219.9000,267.4000) -- (220.2000,267.5000) -- (220.5000,267.7000) -- (220.8000,267.8000) -- (221.1000,268.0000) -- (221.4000,268.1000) -- (221.7000,268.3000) -- (222.0000,268.4000) -- (222.3000,268.6000) -- (222.6000,268.7000) -- (222.9000,268.9000) -- (223.2000,269.0000) -- (223.5000,269.2000) -- (223.8000,269.3000) -- (224.1000,269.5000) -- (224.4000,269.6000) -- (224.7000,269.8000) -- (225.0000,269.9000) -- (225.3000,270.0000) -- (225.6000,270.2000) -- (225.9000,270.3000) -- (226.1000,270.5000) -- (226.4000,270.6000) -- (226.7000,270.8000) -- (227.0000,270.9000) -- (227.3000,271.1000) -- (227.6000,271.2000) -- (227.9000,271.4000) -- (228.2000,271.5000) -- (228.5000,271.7000) -- (228.8000,271.8000) -- (229.1000,272.0000) -- (229.4000,272.1000) -- (229.7000,272.3000) -- (230.0000,272.4000) -- (230.3000,272.6000) -- (230.6000,272.7000) -- (230.9000,272.9000) -- (231.2000,273.0000) -- (231.5000,273.2000) -- (231.8000,273.3000) -- (232.1000,273.5000) -- (232.4000,273.6000) -- (232.7000,273.8000) -- (233.0000,273.9000) -- (233.3000,274.1000) -- (233.5000,274.2000) -- (233.8000,274.4000) -- (234.1000,274.5000) -- (234.4000,274.6000) -- (234.7000,274.8000) -- (235.0000,274.9000) -- (235.3000,275.1000) -- (235.6000,275.2000) -- (235.9000,275.4000) -- (236.2000,275.5000) -- (236.5000,275.7000) -- (236.8000,275.8000) -- (237.1000,276.0000) -- (237.4000,276.1000) -- (237.7000,276.3000) -- (238.0000,276.4000) -- (238.3000,276.6000) -- (238.6000,276.7000) -- (238.9000,276.9000) -- (239.2000,277.0000) -- (239.5000,277.2000) -- (239.8000,277.3000) -- (240.1000,277.5000) -- (240.4000,277.6000) -- (240.6000,277.8000) -- (240.9000,277.9000) -- (241.2000,278.1000) -- (241.5000,278.2000) -- (241.8000,278.4000) -- (242.1000,278.5000) -- (242.4000,278.7000) -- (242.7000,278.8000) -- (243.0000,279.0000) -- (243.3000,279.1000) -- (243.6000,279.2000) -- (243.9000,279.4000) -- (244.2000,279.5000) -- (244.5000,279.7000) -- (244.8000,279.8000) -- (245.1000,280.0000) -- (245.4000,280.1000) -- (245.7000,280.3000) -- (246.0000,280.4000) -- (246.3000,280.6000) -- (246.6000,280.7000) -- (246.9000,280.9000) -- (247.2000,281.0000) -- (247.5000,281.2000) -- (247.8000,281.3000) -- (248.0000,281.5000) -- (248.3000,281.6000) -- (248.6000,281.8000) -- (248.9000,281.9000) -- (249.2000,282.1000) -- (249.5000,282.2000) -- (249.8000,282.4000) -- (250.1000,282.5000) -- (250.4000,282.7000) -- (250.7000,282.8000) -- (251.0000,283.0000) -- (251.3000,283.1000) -- (251.6000,283.3000) -- (251.9000,283.4000);



  \end{scope}
  \begin{scope}[scale=1.006,draw=black,line join=bevel,line cap=rect,line width=0.800pt]
  \end{scope}
  \begin{scope}[draw=black,line join=bevel,line cap=rect,line width=0.800pt]
  \end{scope}
  \begin{scope}[cm={{1.00588,0.0,0.0,1.00588,(100.44074,197.47563)}},draw=black,line join=bevel,line cap=rect,line width=0.800pt]
    \path[fill=black] (0.0000,2.0000) node[above right] (text1032-6) {\scriptsize $\downarrow\,\overline{c}_{1,1}-\delta$};



    \path[fill=black,line width=0.800pt] (87.7767,3.9) node[above right] (text1032-6-7) {\hspace*{-.4ex}\scriptsize $\downarrow\,\underline{c}_{1,1}$};



    \path[fill=black,line width=0.800pt] (-17.7540,106.2897) node[above right] (text1032-6-9) {\hspace{-.4ex}\scriptsize $\downarrow\,\overline{c}_{1,1}$};



  \end{scope}
\end{scope}

\end{tikzpicture}


%  \caption[Energy models of different static and dynamic plans] {The energy models of the paths from Fig.~\ref{fig:trajs} for 200 seconds against the simulated data (\hyperref[fig:ener:static-I]{6.I} and \hyperref[fig:ener:static-II]{6.II}). Below are the energy evolutions from the algorithm (\hyperref[fig:ener-dyn-i]{6.i} and \hyperref[fig:ener-dyn-ii]{6.ii}). It replans the path when the final time and battery time do not match, and the computation when the battery is discharged.}
%  \label{fig:ener}
%\end{figure}
%\begin{figure}[h!]
%  \centering
%  \footnotesize
%  
\definecolor{ca0a0a4}{RGB}{160,160,164}
\definecolor{cffffff}{RGB}{255,255,255}
\definecolor{cff0000}{RGB}{255,0,0}


\def \globalscale {1.000000}
\begin{tikzpicture}[y=0.80pt, x=0.80pt, yscale=-\globalscale, xscale=\globalscale, inner sep=0pt, outer sep=0pt]
\begin{scope}[shift={(-18.54884,-0.65233)},draw=black,line join=bevel,line cap=rect,even odd rule,line width=0.800pt]
  \begin{scope}[draw=black,line join=bevel,line cap=rect,line width=0.800pt]
  \end{scope}
  \begin{scope}[scale=1.008,draw=black,line join=bevel,line cap=rect,line width=0.800pt]
  \end{scope}
  \begin{scope}[cm={{1.00769,0.0,0.0,1.00769,(-21.0,-7.5)}},draw=ca0a0a4,dash pattern=on 0.40pt off 0.80pt,line join=round,line cap=round,line width=0.400pt]
    \path[draw] (41.5000,74.5000) -- (127.5000,74.5000);



  \end{scope}
  \begin{scope}[cm={{1.00769,0.0,0.0,1.00769,(-21.0,-7.5)}},draw=black,line join=round,line cap=round,line width=0.480pt]
    \path[draw] (41.5000,74.5000) -- (44.5000,74.5000);



    \path[draw] (127.5000,74.5000) -- (124.5000,74.5000);



  \end{scope}
  \begin{scope}[scale=1.008,draw=black,line join=bevel,line cap=rect,line width=0.800pt]
  \end{scope}
  \begin{scope}[cm={{1.00769,0.0,0.0,1.00769,(25.1923,79.6077)}},draw=black,line join=bevel,line cap=rect,line width=0.800pt]
  \end{scope}
  \begin{scope}[cm={{1.00769,0.0,0.0,1.00769,(25.1923,79.6077)}},draw=black,line join=bevel,line cap=rect,line width=0.800pt]
  \end{scope}
  \begin{scope}[cm={{1.00769,0.0,0.0,1.00769,(25.1923,79.6077)}},draw=black,line join=bevel,line cap=rect,line width=0.800pt]
  \end{scope}
  \begin{scope}[cm={{1.00769,0.0,0.0,1.00769,(25.1923,79.6077)}},draw=black,line join=bevel,line cap=rect,line width=0.800pt]
  \end{scope}
  \begin{scope}[cm={{1.00769,0.0,0.0,1.00769,(25.1923,79.6077)}},draw=black,line join=bevel,line cap=rect,line width=0.800pt]
  \end{scope}
  \begin{scope}[cm={{1.00769,0.0,0.0,1.00769,(4.1923,72.1077)}},draw=black,line join=bevel,line cap=rect,line width=0.800pt]
    \path[fill=black] (0.0000,0.0000) node[above right] (text34) {32};



  \end{scope}
  \begin{scope}[cm={{1.00769,0.0,0.0,1.00769,(25.1923,79.6077)}},draw=black,line join=bevel,line cap=rect,line width=0.800pt]
  \end{scope}
  \begin{scope}[scale=1.008,draw=black,line join=bevel,line cap=rect,line width=0.800pt]
  \end{scope}
  \begin{scope}[cm={{1.00769,0.0,0.0,1.00769,(-21.0,-7.5)}},draw=ca0a0a4,dash pattern=on 0.40pt off 0.80pt,line join=round,line cap=round,line width=0.400pt]
    \path[draw] (41.5000,48.5000) -- (127.5000,48.5000);



  \end{scope}
  \begin{scope}[cm={{1.00769,0.0,0.0,1.00769,(-21.0,-7.5)}},draw=black,line join=round,line cap=round,line width=0.480pt]
    \path[draw] (41.5000,48.5000) -- (44.5000,48.5000);



    \path[draw] (127.5000,48.5000) -- (124.5000,48.5000);



  \end{scope}
  \begin{scope}[scale=1.008,draw=black,line join=bevel,line cap=rect,line width=0.800pt]
  \end{scope}
  \begin{scope}[cm={{1.00769,0.0,0.0,1.00769,(25.1923,52.4)}},draw=black,line join=bevel,line cap=rect,line width=0.800pt]
  \end{scope}
  \begin{scope}[cm={{1.00769,0.0,0.0,1.00769,(25.1923,52.4)}},draw=black,line join=bevel,line cap=rect,line width=0.800pt]
  \end{scope}
  \begin{scope}[cm={{1.00769,0.0,0.0,1.00769,(25.1923,52.4)}},draw=black,line join=bevel,line cap=rect,line width=0.800pt]
  \end{scope}
  \begin{scope}[cm={{1.00769,0.0,0.0,1.00769,(25.1923,52.4)}},draw=black,line join=bevel,line cap=rect,line width=0.800pt]
  \end{scope}
  \begin{scope}[cm={{1.00769,0.0,0.0,1.00769,(25.1923,52.4)}},draw=black,line join=bevel,line cap=rect,line width=0.800pt]
  \end{scope}
  \begin{scope}[cm={{1.00769,0.0,0.0,1.00769,(4.1923,44.9)}},draw=black,line join=bevel,line cap=rect,line width=0.800pt]
    \path[fill=black] (0.0000,0.0000) node[above right] (text64) {36};



  \end{scope}
  \begin{scope}[cm={{1.00769,0.0,0.0,1.00769,(25.1923,52.4)}},draw=black,line join=bevel,line cap=rect,line width=0.800pt]
  \end{scope}
  \begin{scope}[scale=1.008,draw=black,line join=bevel,line cap=rect,line width=0.800pt]
  \end{scope}
  \begin{scope}[cm={{1.00769,0.0,0.0,1.00769,(-21.0,-7.5)}},draw=ca0a0a4,dash pattern=on 0.40pt off 0.80pt,line join=round,line cap=round,line width=0.400pt]
    \path[draw] (41.5000,22.5000) -- (46.5000,22.5000);



    \path[draw] (103.5000,22.5000) -- (127.5000,22.5000);



  \end{scope}
  \begin{scope}[cm={{1.00769,0.0,0.0,1.00769,(-21.0,-7.5)}},draw=black,line join=round,line cap=round,line width=0.480pt]
    \path[draw] (41.5000,22.5000) -- (44.5000,22.5000);



    \path[draw] (127.5000,22.5000) -- (124.5000,22.5000);



  \end{scope}
  \begin{scope}[scale=1.008,draw=black,line join=bevel,line cap=rect,line width=0.800pt]
  \end{scope}
  \begin{scope}[cm={{1.00769,0.0,0.0,1.00769,(25.1923,26.2)}},draw=black,line join=bevel,line cap=rect,line width=0.800pt]
  \end{scope}
  \begin{scope}[cm={{1.00769,0.0,0.0,1.00769,(25.1923,26.2)}},draw=black,line join=bevel,line cap=rect,line width=0.800pt]
  \end{scope}
  \begin{scope}[cm={{1.00769,0.0,0.0,1.00769,(25.1923,26.2)}},draw=black,line join=bevel,line cap=rect,line width=0.800pt]
  \end{scope}
  \begin{scope}[cm={{1.00769,0.0,0.0,1.00769,(25.1923,26.2)}},draw=black,line join=bevel,line cap=rect,line width=0.800pt]
  \end{scope}
  \begin{scope}[cm={{1.00769,0.0,0.0,1.00769,(25.1923,26.2)}},draw=black,line join=bevel,line cap=rect,line width=0.800pt]
  \end{scope}
  \begin{scope}[cm={{1.00769,0.0,0.0,1.00769,(4.1923,18.7)}},draw=black,line join=bevel,line cap=rect,line width=0.800pt]
    \path[fill=black] (0.0000,0.0000) node[above right] (text96) {40};



  \end{scope}
  \begin{scope}[cm={{1.00769,0.0,0.0,1.00769,(25.1923,26.2)}},draw=black,line join=bevel,line cap=rect,line width=0.800pt]
  \end{scope}
  \begin{scope}[scale=1.008,draw=black,line join=bevel,line cap=rect,line width=0.800pt]
  \end{scope}
  \begin{scope}[cm={{1.00769,0.0,0.0,1.00769,(-21.0,-7.5)}},draw=ca0a0a4,dash pattern=on 0.40pt off 0.80pt,line join=round,line cap=round,line width=0.400pt]
    \path[draw] (41.5000,84.5000) -- (41.5000,8.5000);



  \end{scope}
  \begin{scope}[cm={{1.00769,0.0,0.0,1.00769,(-21.0,-7.5)}},draw=black,line join=round,line cap=round,line width=0.480pt]
    \path[draw] (41.5000,84.5000) -- (41.5000,80.5000);



    \path[draw] (41.5000,8.5000) -- (41.5000,11.5000);



  \end{scope}
  \begin{scope}[scale=1.008,draw=black,line join=bevel,line cap=rect,line width=0.800pt]
  \end{scope}
  \begin{scope}[cm={{1.00769,0.0,0.0,1.00769,(42.3231,100.769)}},draw=black,line join=bevel,line cap=rect,line width=0.800pt]
  \end{scope}
  \begin{scope}[cm={{1.00769,0.0,0.0,1.00769,(42.3231,100.769)}},draw=black,line join=bevel,line cap=rect,line width=0.800pt]
  \end{scope}
  \begin{scope}[cm={{1.00769,0.0,0.0,1.00769,(42.3231,100.769)}},draw=black,line join=bevel,line cap=rect,line width=0.800pt]
  \end{scope}
  \begin{scope}[cm={{1.00769,0.0,0.0,1.00769,(42.3231,100.769)}},draw=black,line join=bevel,line cap=rect,line width=0.800pt]
  \end{scope}
  \begin{scope}[cm={{1.00769,0.0,0.0,1.00769,(42.3231,100.769)}},draw=black,line join=bevel,line cap=rect,line width=0.800pt]
  \end{scope}
  \begin{scope}[cm={{1.00769,0.0,0.0,1.00769,(42.3231,100.769)}},draw=black,line join=bevel,line cap=rect,line width=0.800pt]
  \end{scope}
  \begin{scope}[scale=1.008,draw=black,line join=bevel,line cap=rect,line width=0.800pt]
  \end{scope}
  \begin{scope}[cm={{1.00769,0.0,0.0,1.00769,(-21.0,-7.5)}},draw=ca0a0a4,dash pattern=on 0.40pt off 0.80pt,line join=round,line cap=round,line width=0.400pt]
    \path[draw] (70.5000,84.5000) -- (70.5000,30.5000);



    \path[draw] (70.5000,14.5000) -- (70.5000,8.5000);



  \end{scope}
  \begin{scope}[cm={{1.00769,0.0,0.0,1.00769,(-21.0,-7.5)}},draw=black,line join=round,line cap=round,line width=0.480pt]
    \path[draw] (70.5000,84.5000) -- (70.5000,80.5000);



    \path[draw] (70.5000,8.5000) -- (70.5000,11.5000);



  \end{scope}
  \begin{scope}[scale=1.008,draw=black,line join=bevel,line cap=rect,line width=0.800pt]
  \end{scope}
  \begin{scope}[cm={{1.00769,0.0,0.0,1.00769,(70.5385,100.769)}},draw=black,line join=bevel,line cap=rect,line width=0.800pt]
  \end{scope}
  \begin{scope}[cm={{1.00769,0.0,0.0,1.00769,(70.5385,100.769)}},draw=black,line join=bevel,line cap=rect,line width=0.800pt]
  \end{scope}
  \begin{scope}[cm={{1.00769,0.0,0.0,1.00769,(70.5385,100.769)}},draw=black,line join=bevel,line cap=rect,line width=0.800pt]
  \end{scope}
  \begin{scope}[cm={{1.00769,0.0,0.0,1.00769,(70.5385,100.769)}},draw=black,line join=bevel,line cap=rect,line width=0.800pt]
  \end{scope}
  \begin{scope}[cm={{1.00769,0.0,0.0,1.00769,(70.5385,100.769)}},draw=black,line join=bevel,line cap=rect,line width=0.800pt]
  \end{scope}
  \begin{scope}[cm={{1.00769,0.0,0.0,1.00769,(70.5385,100.769)}},draw=black,line join=bevel,line cap=rect,line width=0.800pt]
  \end{scope}
  \begin{scope}[scale=1.008,draw=black,line join=bevel,line cap=rect,line width=0.800pt]
  \end{scope}
  \begin{scope}[cm={{1.00769,0.0,0.0,1.00769,(-21.0,-7.5)}},draw=ca0a0a4,dash pattern=on 0.40pt off 0.80pt,line join=round,line cap=round,line width=0.400pt]
    \path[draw] (98.5000,84.5000) -- (98.5000,30.5000);



    \path[draw] (98.5000,14.5000) -- (98.5000,8.5000);



  \end{scope}
  \begin{scope}[cm={{1.00769,0.0,0.0,1.00769,(-21.0,-7.5)}},draw=black,line join=round,line cap=round,line width=0.480pt]
    \path[draw] (98.5000,84.5000) -- (98.5000,80.5000);



    \path[draw] (98.5000,8.5000) -- (98.5000,11.5000);



  \end{scope}
  \begin{scope}[scale=1.008,draw=black,line join=bevel,line cap=rect,line width=0.800pt]
  \end{scope}
  \begin{scope}[cm={{1.00769,0.0,0.0,1.00769,(99.7615,100.769)}},draw=black,line join=bevel,line cap=rect,line width=0.800pt]
  \end{scope}
  \begin{scope}[cm={{1.00769,0.0,0.0,1.00769,(99.7615,100.769)}},draw=black,line join=bevel,line cap=rect,line width=0.800pt]
  \end{scope}
  \begin{scope}[cm={{1.00769,0.0,0.0,1.00769,(99.7615,100.769)}},draw=black,line join=bevel,line cap=rect,line width=0.800pt]
  \end{scope}
  \begin{scope}[cm={{1.00769,0.0,0.0,1.00769,(99.7615,100.769)}},draw=black,line join=bevel,line cap=rect,line width=0.800pt]
  \end{scope}
  \begin{scope}[cm={{1.00769,0.0,0.0,1.00769,(99.7615,100.769)}},draw=black,line join=bevel,line cap=rect,line width=0.800pt]
  \end{scope}
  \begin{scope}[cm={{1.00769,0.0,0.0,1.00769,(99.7615,100.769)}},draw=black,line join=bevel,line cap=rect,line width=0.800pt]
  \end{scope}
  \begin{scope}[scale=1.008,draw=black,line join=bevel,line cap=rect,line width=0.800pt]
  \end{scope}
  \begin{scope}[cm={{1.00769,0.0,0.0,1.00769,(-21.0,-7.5)}},draw=ca0a0a4,dash pattern=on 0.40pt off 0.80pt,line join=round,line cap=round,line width=0.400pt]
    \path[draw] (127.5000,84.5000) -- (127.5000,8.5000);



  \end{scope}
  \begin{scope}[cm={{1.00769,0.0,0.0,1.00769,(-21.0,-7.5)}},draw=black,line join=round,line cap=round,line width=0.480pt]
    \path[draw] (127.5000,84.5000) -- (127.5000,80.5000);



    \path[draw] (127.5000,8.5000) -- (127.5000,11.5000);



  \end{scope}
  \begin{scope}[scale=1.008,draw=black,line join=bevel,line cap=rect,line width=0.800pt]
  \end{scope}
  \begin{scope}[cm={{1.00769,0.0,0.0,1.00769,(127.977,100.769)}},draw=black,line join=bevel,line cap=rect,line width=0.800pt]
  \end{scope}
  \begin{scope}[cm={{1.00769,0.0,0.0,1.00769,(127.977,100.769)}},draw=black,line join=bevel,line cap=rect,line width=0.800pt]
  \end{scope}
  \begin{scope}[cm={{1.00769,0.0,0.0,1.00769,(127.977,100.769)}},draw=black,line join=bevel,line cap=rect,line width=0.800pt]
  \end{scope}
  \begin{scope}[cm={{1.00769,0.0,0.0,1.00769,(127.977,100.769)}},draw=black,line join=bevel,line cap=rect,line width=0.800pt]
  \end{scope}
  \begin{scope}[cm={{1.00769,0.0,0.0,1.00769,(127.977,100.769)}},draw=black,line join=bevel,line cap=rect,line width=0.800pt]
  \end{scope}
  \begin{scope}[cm={{1.00769,0.0,0.0,1.00769,(127.977,100.769)}},draw=black,line join=bevel,line cap=rect,line width=0.800pt]
  \end{scope}
  \begin{scope}[scale=1.008,draw=black,line join=bevel,line cap=rect,line width=0.800pt]
  \end{scope}
  \begin{scope}[cm={{1.00769,0.0,0.0,1.00769,(-21.0,-7.5)}},draw=black,line join=round,line cap=round,line width=0.480pt]
    \path[draw] (41.5000,8.5000) -- (41.5000,84.5000) -- (127.5000,84.5000) -- (127.5000,8.5000) -- (41.5000,8.5000);



  \end{scope}
  \begin{scope}[scale=1.008,draw=black,line join=bevel,line cap=rect,line width=0.800pt]
  \end{scope}
  \begin{scope}[scale=1.008,draw=black,line join=bevel,line cap=rect,line width=0.800pt]
  \end{scope}
  \begin{scope}[cm={{1.00769,0.0,0.0,1.00769,(-21.0,-7.5)}},fill=cffffff]
    \path[fill,rounded corners=0.0000cm] (108.0000,12.0000) rectangle (119.0000,28.0000);



  \end{scope}
  \begin{scope}[scale=1.008,draw=black,line join=bevel,line cap=rect,line width=0.800pt]
  \end{scope}
  \begin{scope}[scale=1.008,draw=black,line join=bevel,line cap=rect,line width=0.800pt]
  \end{scope}
  \begin{scope}[cm={{1.00769,0.0,0.0,1.00769,(-21.0,-7.5)}},draw=black,line join=round,line cap=round,line width=0.800pt]
    \path[draw] (108.5000,28.5000) -- (108.5000,12.5000) -- (119.5000,12.5000) -- (119.5000,28.5000) -- (108.5000,28.5000);



  \end{scope}
  \begin{scope}[scale=1.008,draw=black,line join=bevel,line cap=rect,line width=0.800pt]
  \end{scope}
  \begin{scope}[cm={{1.00769,0.0,0.0,1.00769,(112.862,24.1846)}},draw=black,line join=bevel,line cap=rect,line width=0.800pt]
  \end{scope}
  \begin{scope}[cm={{1.00769,0.0,0.0,1.00769,(112.862,24.1846)}},draw=black,line join=bevel,line cap=rect,line width=0.800pt]
  \end{scope}
  \begin{scope}[cm={{1.00769,0.0,0.0,1.00769,(112.862,24.1846)}},draw=black,line join=bevel,line cap=rect,line width=0.800pt]
  \end{scope}
  \begin{scope}[cm={{1.00769,0.0,0.0,1.00769,(112.862,24.1846)}},draw=black,line join=bevel,line cap=rect,line width=0.800pt]
  \end{scope}
  \begin{scope}[cm={{1.00769,0.0,0.0,1.00769,(112.862,24.1846)}},draw=black,line join=bevel,line cap=rect,line width=0.800pt]
  \end{scope}
  \begin{scope}[cm={{1.00769,0.0,0.0,1.00769,(91.862,16.6846)}},draw=black,line join=bevel,line cap=rect,line width=0.800pt]
    \path[fill=black] (0.0000,0.0000) node[above right] (text244) {I};



  \end{scope}
  \begin{scope}[cm={{1.00769,0.0,0.0,1.00769,(112.862,24.1846)}},draw=black,line join=bevel,line cap=rect,line width=0.800pt]
  \end{scope}
  \begin{scope}[cm={{0.0,-1.00769,1.00769,0.0,(15.1154,113.869)}},draw=black,line join=bevel,line cap=rect,line width=0.800pt]
  \end{scope}
  \begin{scope}[cm={{0.0,-1.00769,1.00769,0.0,(15.1154,113.869)}},draw=black,line join=bevel,line cap=rect,line width=0.800pt]
  \end{scope}
  \begin{scope}[cm={{0.0,-1.00769,1.00769,0.0,(15.1154,113.869)}},draw=black,line join=bevel,line cap=rect,line width=0.800pt]
  \end{scope}
  \begin{scope}[cm={{0.0,-1.00769,1.00769,0.0,(15.1154,113.869)}},draw=black,line join=bevel,line cap=rect,line width=0.800pt]
  \end{scope}
  \begin{scope}[cm={{0.0,-1.00769,1.00769,0.0,(15.1154,113.869)}},draw=black,line join=bevel,line cap=rect,line width=0.800pt]
  \end{scope}
  \begin{scope}[cm={{0.0,-1.00769,1.00769,0.0,(-8.8846,106.369)}},draw=black,line join=bevel,line cap=rect,line width=0.800pt]
    \path[fill=black] (0.0000,0.0000) node[above right] (text260) {\rotatebox{90}{Power (W)}};



  \end{scope}
  \begin{scope}[cm={{0.0,-1.00769,1.00769,0.0,(15.1154,113.869)}},draw=black,line join=bevel,line cap=rect,line width=0.800pt]
  \end{scope}
  \begin{scope}[cm={{1.00769,0.0,0.0,1.00769,(48.3692,23.1769)}},draw=black,line join=bevel,line cap=rect,line width=0.800pt]
  \end{scope}
  \begin{scope}[cm={{1.00769,0.0,0.0,1.00769,(48.3692,23.1769)}},draw=black,line join=bevel,line cap=rect,line width=0.800pt]
  \end{scope}
  \begin{scope}[cm={{1.00769,0.0,0.0,1.00769,(48.3692,23.1769)}},draw=black,line join=bevel,line cap=rect,line width=0.800pt]
  \end{scope}
  \begin{scope}[cm={{1.00769,0.0,0.0,1.00769,(48.3692,23.1769)}},draw=black,line join=bevel,line cap=rect,line width=0.800pt]
  \end{scope}
  \begin{scope}[cm={{1.00769,0.0,0.0,1.00769,(48.3692,23.1769)}},draw=black,line join=bevel,line cap=rect,line width=0.800pt]
  \end{scope}
  \begin{scope}[cm={{1.00769,0.0,0.0,1.00769,(30.3692,13.6769)}},draw=black,line join=bevel,line cap=rect,line width=0.800pt]
    \path[fill=black] (0.0000,0.0000) node[above right] (text276) {\scriptsize data};



  \end{scope}
  \begin{scope}[cm={{1.00769,0.0,0.0,1.00769,(48.3692,23.1769)}},draw=black,line join=bevel,line cap=rect,line width=0.800pt]
  \end{scope}
  \begin{scope}[scale=1.008,draw=black,line join=bevel,line cap=rect,line width=0.800pt]
  \end{scope}
  \begin{scope}[cm={{1.00769,0.0,0.0,1.00769,(-21.0,-7.5)}},draw=black,line join=round,line cap=round,line width=0.480pt]
    \path[draw,even odd rule] (71.5000,18.5000) -- (98.5000,18.5000);



  \end{scope}
  \begin{scope}[scale=1.008,draw=black,line join=bevel,line cap=rect,line width=0.800pt]
  \end{scope}
  \begin{scope}[scale=1.008,draw=black,line join=bevel,line cap=rect,line width=0.800pt]
  \end{scope}
  \begin{scope}[scale=1.008,draw=black,line join=bevel,line cap=rect,line width=0.800pt]
  \end{scope}
  \begin{scope}[scale=1.008,draw=black,line join=bevel,line cap=rect,line width=0.800pt]
  \end{scope}
  \begin{scope}[cm={{1.00769,0.0,0.0,1.00769,(-21.0,-7.5)}},draw=black,line join=round,line cap=round,line width=0.480pt]
    \path[draw] (41.6000,17.0000) -- (41.6000,17.0000) -- (41.7000,17.8000) -- (41.9000,18.6000) -- (42.0000,19.4000) -- (42.2000,20.2000) -- (42.3000,20.9000) -- (42.5000,21.6000) -- (42.6000,22.4000) -- (42.7000,23.1000) -- (42.9000,23.8000) -- (43.0000,24.5000) -- (43.2000,25.2000) -- (43.3000,25.8000) -- (43.5000,26.5000) -- (43.6000,27.1000) -- (43.7000,27.7000) -- (43.9000,28.4000) -- (44.0000,29.0000) -- (44.2000,29.6000) -- (44.3000,30.1000) -- (44.5000,30.7000) -- (44.6000,31.3000) -- (44.7000,31.8000) -- (44.9000,32.4000) -- (45.0000,32.9000) -- (45.2000,33.4000) -- (45.3000,34.0000) -- (45.5000,34.5000) -- (45.6000,35.0000) -- (45.7000,35.5000) -- (45.9000,35.9000) -- (46.0000,36.4000) -- (46.2000,36.9000) -- (46.3000,37.3000) -- (46.5000,37.8000) -- (46.6000,38.2000) -- (46.7000,38.7000) -- (46.9000,39.1000) -- (47.0000,39.5000) -- (47.2000,39.9000) -- (47.3000,40.3000) -- (47.5000,40.7000) -- (47.6000,41.1000) -- (47.7000,41.5000) -- (47.9000,41.9000) -- (48.0000,42.2000) -- (48.2000,42.6000) -- (48.3000,43.0000) -- (48.5000,43.3000) -- (48.6000,43.6000) -- (48.7000,44.0000) -- (48.9000,44.3000) -- (49.0000,44.6000) -- (49.2000,45.0000) -- (49.3000,45.3000) -- (49.5000,45.6000) -- (49.6000,45.9000) -- (49.7000,46.2000) -- (49.9000,46.5000) -- (50.0000,46.8000) -- (50.2000,47.0000) -- (50.3000,47.3000) -- (50.5000,47.6000) -- (50.6000,47.8000) -- (50.7000,48.1000) -- (50.9000,48.4000) -- (51.0000,48.6000) -- (51.2000,48.8000) -- (51.3000,49.1000) -- (51.5000,49.3000) -- (51.6000,49.5000) -- (51.7000,49.8000) -- (51.9000,50.0000) -- (52.0000,50.2000) -- (52.2000,50.4000) -- (52.3000,50.6000) -- (52.5000,50.8000) -- (52.6000,51.0000) -- (52.7000,51.2000) -- (52.9000,51.4000) -- (53.0000,51.6000) -- (53.2000,51.7000) -- (53.3000,51.9000) -- (53.5000,52.1000) -- (53.6000,52.3000) -- (53.7000,52.4000) -- (53.9000,52.6000) -- (54.0000,52.7000) -- (54.2000,52.9000) -- (54.3000,53.0000) -- (54.5000,53.1000) -- (54.6000,53.3000) -- (54.7000,53.4000) -- (54.9000,53.5000) -- (55.0000,53.7000) -- (55.2000,53.8000) -- (55.3000,53.9000) -- (55.5000,54.0000) -- (55.6000,54.1000) -- (55.7000,54.2000) -- (55.9000,54.3000) -- (56.0000,54.4000) -- (56.2000,54.5000) -- (56.3000,54.6000) -- (56.5000,54.6000) -- (56.6000,54.7000) -- (56.7000,54.8000) -- (56.9000,54.9000) -- (57.0000,54.9000) -- (57.2000,55.0000) -- (57.3000,55.0000) -- (57.5000,55.1000) -- (57.6000,55.1000) -- (57.7000,55.2000) -- (57.9000,55.2000) -- (58.0000,55.2000) -- (58.2000,55.3000) -- (58.3000,55.3000) -- (58.5000,55.3000) -- (58.6000,55.3000) -- (58.7000,55.3000) -- (58.9000,55.3000) -- (59.0000,55.3000) -- (59.2000,55.3000) -- (59.3000,55.3000) -- (59.5000,55.3000) -- (59.6000,55.2000) -- (59.7000,55.2000) -- (59.9000,55.2000) -- (60.0000,55.1000) -- (60.2000,55.1000) -- (60.3000,55.0000) -- (60.5000,54.9000) -- (60.6000,54.9000) -- (60.7000,54.8000) -- (60.9000,54.7000) -- (61.0000,54.6000) -- (61.2000,54.5000) -- (61.3000,54.4000) -- (61.5000,54.2000) -- (61.6000,54.1000) -- (61.7000,54.0000) -- (61.9000,53.8000) -- (62.0000,53.7000) -- (62.2000,53.5000) -- (62.3000,53.3000) -- (62.5000,53.1000) -- (62.6000,53.0000) -- (62.7000,52.8000) -- (62.9000,52.6000) -- (63.0000,52.4000) -- (63.2000,52.2000) -- (63.3000,52.0000) -- (63.5000,51.7000) -- (63.6000,51.5000) -- (63.7000,51.3000) -- (63.9000,51.1000) -- (64.0000,50.8000) -- (64.2000,50.6000) -- (64.3000,50.4000) -- (64.5000,50.1000) -- (64.6000,49.9000) -- (64.7000,49.7000) -- (64.9000,49.4000) -- (65.0000,49.2000) -- (65.2000,49.0000) -- (65.3000,48.7000) -- (65.5000,48.5000) -- (65.6000,48.2000) -- (65.7000,48.0000) -- (65.9000,47.8000) -- (66.0000,47.5000) -- (66.2000,47.3000) -- (66.3000,47.1000) -- (66.5000,46.8000) -- (66.6000,46.6000) -- (66.7000,46.4000) -- (66.9000,46.1000) -- (67.0000,45.9000) -- (67.2000,45.7000) -- (67.3000,45.4000) -- (67.5000,45.2000) -- (67.6000,45.0000) -- (67.7000,44.8000) -- (67.9000,44.5000) -- (68.0000,44.3000) -- (68.2000,44.1000) -- (68.3000,43.9000) -- (68.5000,43.7000) -- (68.6000,43.5000) -- (68.7000,43.2000) -- (68.9000,43.0000) -- (69.0000,42.8000) -- (69.2000,42.6000) -- (69.3000,42.4000) -- (69.5000,42.2000) -- (69.6000,42.0000) -- (69.7000,41.8000) -- (69.9000,41.6000) -- (70.0000,41.4000) -- (70.2000,41.2000) -- (70.3000,41.0000) -- (70.5000,40.9000) -- (70.6000,40.7000) -- (70.7000,40.5000) -- (70.9000,40.3000) -- (71.0000,40.1000) -- (71.2000,40.0000) -- (71.3000,39.8000) -- (71.5000,39.6000) -- (71.6000,39.4000) -- (71.7000,39.3000) -- (71.9000,39.1000) -- (72.0000,38.9000) -- (72.2000,38.8000) -- (72.3000,38.6000) -- (72.5000,38.5000) -- (72.6000,38.3000) -- (72.7000,38.1000) -- (72.9000,38.0000) -- (73.0000,37.9000) -- (73.2000,37.7000) -- (73.3000,37.6000) -- (73.5000,37.4000) -- (73.6000,37.3000) -- (73.7000,37.1000) -- (73.9000,37.0000) -- (74.0000,36.9000) -- (74.2000,36.8000) -- (74.3000,36.6000) -- (74.5000,36.5000) -- (74.6000,36.4000) -- (74.7000,36.2000) -- (74.9000,36.1000) -- (75.0000,36.0000) -- (75.2000,35.9000) -- (75.3000,35.8000) -- (75.5000,35.7000) -- (75.6000,35.6000) -- (75.7000,35.5000) -- (75.9000,35.4000) -- (76.0000,35.3000) -- (76.2000,35.2000) -- (76.3000,35.1000) -- (76.5000,35.0000) -- (76.6000,34.9000) -- (76.7000,34.8000) -- (76.9000,34.7000) -- (77.0000,34.6000) -- (77.2000,34.5000) -- (77.3000,34.4000) -- (77.5000,34.4000) -- (77.6000,34.3000) -- (77.7000,34.2000) -- (77.9000,34.1000) -- (78.0000,34.1000) -- (78.2000,34.0000) -- (78.3000,33.9000) -- (78.5000,33.9000) -- (78.6000,33.8000) -- (78.7000,33.7000) -- (78.9000,33.7000) -- (79.0000,33.6000) -- (79.2000,33.6000) -- (79.3000,33.5000) -- (79.5000,33.5000) -- (79.6000,33.4000) -- (79.7000,33.4000) -- (79.9000,33.3000) -- (80.0000,33.3000) -- (80.2000,33.2000) -- (80.3000,33.2000) -- (80.5000,33.1000) -- (80.6000,33.1000) -- (80.7000,33.1000) -- (80.9000,33.0000) -- (81.0000,33.0000) -- (81.2000,33.0000) -- (81.3000,32.9000) -- (81.5000,32.9000) -- (81.6000,32.9000) -- (81.7000,32.9000) -- (81.9000,32.8000) -- (82.0000,32.8000) -- (82.2000,32.8000) -- (82.3000,32.8000) -- (82.5000,32.8000) -- (82.6000,32.7000) -- (82.7000,32.7000) -- (82.9000,32.7000) -- (83.0000,32.7000) -- (83.2000,32.7000) -- (83.3000,32.7000) -- (83.5000,32.7000) -- (83.6000,32.7000) -- (83.7000,32.7000) -- (83.9000,32.6000) -- (84.0000,32.6000) -- (84.2000,32.6000) -- (84.3000,32.6000) -- (84.5000,32.6000) -- (84.6000,32.6000) -- (84.7000,32.6000) -- (84.9000,32.7000) -- (85.0000,32.7000) -- (85.2000,32.7000) -- (85.3000,32.7000) -- (85.4000,32.7000) -- (85.6000,32.7000) -- (85.7000,32.7000) -- (85.9000,32.7000) -- (86.0000,32.7000) -- (86.2000,32.7000) -- (86.3000,32.8000) -- (86.4000,32.8000) -- (86.6000,32.8000) -- (86.7000,32.8000) -- (86.9000,32.8000) -- (87.0000,32.8000) -- (87.2000,32.9000) -- (87.3000,32.9000) -- (87.4000,32.9000) -- (87.6000,32.9000) -- (87.7000,33.0000) -- (87.9000,33.0000) -- (88.0000,33.0000) -- (88.2000,33.0000) -- (88.3000,33.1000) -- (88.4000,33.1000) -- (88.6000,33.1000) -- (88.7000,33.1000) -- (88.9000,33.2000) -- (89.0000,33.2000) -- (89.2000,33.2000) -- (89.3000,33.3000) -- (89.4000,33.3000) -- (89.6000,33.3000) -- (89.7000,33.4000) -- (89.9000,33.4000) -- (90.0000,33.4000) -- (90.2000,33.5000) -- (90.3000,33.5000) -- (90.4000,33.6000) -- (90.6000,33.6000) -- (90.7000,33.6000) -- (90.9000,33.7000) -- (91.0000,33.7000) -- (91.2000,33.8000) -- (91.3000,33.8000) -- (91.4000,33.8000) -- (91.6000,33.9000) -- (91.7000,33.9000) -- (91.9000,34.0000) -- (92.0000,34.0000) -- (92.2000,34.1000) -- (92.3000,34.1000) -- (92.4000,34.1000) -- (92.6000,34.2000) -- (92.7000,34.2000) -- (92.9000,34.3000) -- (93.0000,34.3000) -- (93.2000,34.4000) -- (93.3000,34.4000) -- (93.4000,34.5000) -- (93.6000,34.5000) -- (93.7000,34.6000) -- (93.9000,34.6000) -- (94.0000,34.7000) -- (94.2000,34.7000) -- (94.3000,34.8000) -- (94.4000,34.8000) -- (94.6000,34.9000) -- (94.7000,34.9000) -- (94.9000,35.0000) -- (95.0000,35.0000) -- (95.2000,35.1000) -- (95.3000,35.1000) -- (95.4000,35.2000) -- (95.6000,35.2000) -- (95.7000,35.3000) -- (95.9000,35.3000) -- (96.0000,35.4000) -- (96.2000,35.4000) -- (96.3000,35.5000) -- (96.4000,35.5000) -- (96.6000,35.6000) -- (96.7000,35.6000) -- (96.9000,35.7000) -- (97.0000,35.7000) -- (97.2000,35.8000) -- (97.3000,35.9000) -- (97.4000,35.9000) -- (97.6000,36.0000) -- (97.7000,36.0000) -- (97.9000,36.1000) -- (98.0000,36.1000) -- (98.2000,36.2000) -- (98.3000,36.2000) -- (98.4000,36.3000) -- (98.6000,36.4000) -- (98.7000,36.4000) -- (98.9000,36.5000) -- (99.0000,36.6000) -- (99.2000,36.6000) -- (99.3000,36.7000) -- (99.4000,36.8000) -- (99.6000,36.8000) -- (99.7000,36.9000) -- (99.9000,37.0000) -- (100.0000,37.0000) -- (100.2000,37.1000) -- (100.3000,37.2000) -- (100.4000,37.2000) -- (100.6000,37.3000) -- (100.7000,37.4000) -- (100.9000,37.4000) -- (101.0000,37.5000) -- (101.2000,37.6000) -- (101.3000,37.6000) -- (101.4000,37.7000) -- (101.6000,37.8000) -- (101.7000,37.9000) -- (101.9000,37.9000) -- (102.0000,38.0000) -- (102.2000,38.1000) -- (102.3000,38.2000) -- (102.4000,38.2000) -- (102.6000,38.3000) -- (102.7000,38.4000) -- (102.9000,38.4000) -- (103.0000,38.5000) -- (103.2000,38.6000) -- (103.3000,38.7000) -- (103.4000,38.7000) -- (103.6000,38.8000) -- (103.7000,38.9000) -- (103.9000,38.9000) -- (104.0000,39.0000) -- (104.2000,39.1000) -- (104.3000,39.2000) -- (104.4000,39.2000) -- (104.6000,39.3000) -- (104.7000,39.4000) -- (104.9000,39.4000) -- (105.0000,39.5000) -- (105.2000,39.6000) -- (105.3000,39.7000) -- (105.4000,39.7000) -- (105.6000,39.8000) -- (105.7000,39.9000) -- (105.9000,39.9000) -- (106.0000,40.0000) -- (106.2000,40.1000) -- (106.3000,40.1000) -- (106.4000,40.2000) -- (106.6000,40.3000) -- (106.7000,40.3000) -- (106.9000,40.4000) -- (107.0000,40.5000) -- (107.2000,40.5000) -- (107.3000,40.6000) -- (107.4000,40.6000) -- (107.6000,40.7000) -- (107.7000,40.8000) -- (107.9000,40.8000) -- (108.0000,40.9000) -- (108.2000,41.0000) -- (108.3000,41.0000) -- (108.4000,41.1000) -- (108.6000,41.1000) -- (108.7000,41.2000) -- (108.9000,41.3000) -- (109.0000,41.3000) -- (109.2000,41.4000) -- (109.3000,41.4000) -- (109.4000,41.5000) -- (109.6000,41.5000) -- (109.7000,41.6000) -- (109.9000,41.6000) -- (110.0000,41.7000) -- (110.2000,41.7000) -- (110.3000,41.8000) -- (110.4000,41.9000) -- (110.6000,41.9000) -- (110.7000,42.0000) -- (110.9000,42.0000) -- (111.0000,42.1000) -- (111.2000,42.1000) -- (111.3000,42.1000) -- (111.4000,42.2000) -- (111.6000,42.2000) -- (111.7000,42.3000) -- (111.9000,42.3000) -- (112.0000,42.4000) -- (112.2000,42.4000) -- (112.3000,42.5000) -- (112.4000,42.5000) -- (112.6000,42.6000) -- (112.7000,42.6000) -- (112.9000,42.6000) -- (113.0000,42.7000) -- (113.2000,42.7000) -- (113.3000,42.8000) -- (113.4000,42.8000) -- (113.6000,42.8000) -- (113.7000,42.9000) -- (113.9000,42.9000) -- (114.0000,43.0000) -- (114.2000,43.0000) -- (114.3000,43.0000) -- (114.4000,43.1000) -- (114.6000,43.1000) -- (114.7000,43.1000) -- (114.9000,43.2000) -- (115.0000,43.2000) -- (115.2000,43.2000) -- (115.3000,43.3000) -- (115.4000,43.3000) -- (115.6000,43.3000) -- (115.7000,43.4000) -- (115.9000,43.4000) -- (116.0000,43.4000) -- (116.2000,43.4000) -- (116.3000,43.5000) -- (116.4000,43.5000) -- (116.6000,43.5000) -- (116.7000,43.6000) -- (116.9000,43.6000) -- (117.0000,43.6000) -- (117.2000,43.6000) -- (117.3000,43.7000) -- (117.4000,43.7000) -- (117.6000,43.7000) -- (117.7000,43.7000) -- (117.9000,43.7000) -- (118.0000,43.8000) -- (118.2000,43.8000) -- (118.3000,43.8000) -- (118.4000,43.8000) -- (118.6000,43.9000) -- (118.7000,43.9000) -- (118.9000,43.9000) -- (119.0000,43.9000) -- (119.2000,43.9000) -- (119.3000,44.0000) -- (119.4000,44.0000) -- (119.6000,44.0000) -- (119.7000,44.0000) -- (119.9000,44.0000) -- (120.0000,44.0000) -- (120.2000,44.0000) -- (120.3000,44.1000) -- (120.4000,44.1000) -- (120.6000,44.1000) -- (120.7000,44.1000) -- (120.9000,44.1000) -- (121.0000,44.1000) -- (121.2000,44.1000) -- (121.3000,44.2000) -- (121.4000,44.2000) -- (121.6000,44.2000) -- (121.7000,44.2000) -- (121.9000,44.2000) -- (122.0000,44.2000) -- (122.2000,44.2000) -- (122.3000,44.2000) -- (122.4000,44.2000) -- (122.6000,44.2000) -- (122.7000,44.3000) -- (122.9000,44.3000) -- (123.0000,44.3000) -- (123.2000,44.3000) -- (123.3000,44.3000) -- (123.4000,44.3000) -- (123.6000,44.3000) -- (123.7000,44.3000) -- (123.9000,44.3000) -- (124.0000,44.3000) -- (124.2000,44.3000) -- (124.3000,44.3000) -- (124.4000,44.3000) -- (124.6000,44.3000) -- (124.7000,44.3000) -- (124.9000,44.3000) -- (125.0000,44.3000) -- (125.2000,44.3000) -- (125.3000,44.3000) -- (125.4000,44.4000) -- (125.6000,44.4000) -- (125.7000,44.4000) -- (125.9000,44.4000) -- (126.0000,44.4000) -- (126.2000,44.4000) -- (126.3000,44.4000) -- (126.4000,44.4000) -- (126.6000,44.4000) -- (126.7000,44.4000) -- (126.9000,44.4000) -- (127.0000,44.4000) -- (127.2000,44.4000) -- (127.3000,44.4000);



  \end{scope}
  \begin{scope}[scale=1.008,draw=black,line join=bevel,line cap=rect,line width=0.800pt]
  \end{scope}
  \begin{scope}[cm={{1.00769,0.0,0.0,1.00769,(52.4,31.2385)}},draw=black,line join=bevel,line cap=rect,line width=0.800pt]
  \end{scope}
  \begin{scope}[cm={{1.00769,0.0,0.0,1.00769,(52.4,31.2385)}},draw=black,line join=bevel,line cap=rect,line width=0.800pt]
  \end{scope}
  \begin{scope}[cm={{1.00769,0.0,0.0,1.00769,(52.4,31.2385)}},draw=black,line join=bevel,line cap=rect,line width=0.800pt]
  \end{scope}
  \begin{scope}[cm={{1.00769,0.0,0.0,1.00769,(52.4,31.2385)}},draw=black,line join=bevel,line cap=rect,line width=0.800pt]
  \end{scope}
  \begin{scope}[cm={{1.00769,0.0,0.0,1.00769,(52.4,31.2385)}},draw=black,line join=bevel,line cap=rect,line width=0.800pt]
  \end{scope}
  \begin{scope}[cm={{1.00769,0.0,0.0,1.00769,(34.4,23.7385)}},draw=black,line join=bevel,line cap=rect,line width=0.800pt]
    \path[fill=black] (0.0000,0.0000) node[above right] (text312) {\scriptsize KF};



  \end{scope}
  \begin{scope}[cm={{1.00769,0.0,0.0,1.00769,(52.4,31.2385)}},draw=black,line join=bevel,line cap=rect,line width=0.800pt]
  \end{scope}
  \begin{scope}[scale=1.008,draw=black,line join=bevel,line cap=rect,line width=0.800pt]
  \end{scope}
  \begin{scope}[cm={{1.00769,0.0,0.0,1.00769,(-21.0,-7.5)}},draw=cff0000,line join=round,line cap=round,line width=0.480pt]
    \path[draw,even odd rule] (71.5000,26.5000) -- (98.5000,26.5000);



  \end{scope}
  \begin{scope}[scale=1.008,draw=black,line join=bevel,line cap=rect,line width=0.800pt]
  \end{scope}
  \begin{scope}[scale=1.008,draw=black,line join=bevel,line cap=rect,line width=0.800pt]
  \end{scope}
  \begin{scope}[scale=1.008,draw=black,line join=bevel,line cap=rect,line width=0.800pt]
  \end{scope}
  \begin{scope}[scale=1.008,draw=black,line join=bevel,line cap=rect,line width=0.800pt]
  \end{scope}
  \begin{scope}[cm={{1.00769,0.0,0.0,1.00769,(-21.0,-7.5)}},draw=cff0000,line join=round,line cap=round,line width=0.480pt]
    \path[draw] (41.6000,21.8000) -- (41.6000,21.8000) -- (41.7000,17.5000) -- (41.9000,18.2000) -- (42.0000,19.1000) -- (42.2000,20.0000) -- (42.3000,20.9000) -- (42.5000,21.7000) -- (42.6000,22.5000) -- (42.7000,23.3000) -- (42.9000,24.1000) -- (43.0000,24.8000) -- (43.2000,25.5000) -- (43.3000,26.2000) -- (43.5000,26.9000) -- (43.6000,27.6000) -- (43.7000,28.2000) -- (43.9000,28.8000) -- (44.0000,29.4000) -- (44.2000,30.0000) -- (44.3000,30.5000) -- (44.5000,31.1000) -- (44.6000,31.6000) -- (44.7000,32.1000) -- (44.9000,32.7000) -- (45.0000,33.2000) -- (45.2000,33.7000) -- (45.3000,34.2000) -- (45.5000,34.6000) -- (45.6000,35.1000) -- (45.7000,35.6000) -- (45.9000,36.1000) -- (46.0000,36.5000) -- (46.2000,37.0000) -- (46.3000,37.5000) -- (46.5000,37.9000) -- (46.6000,38.4000) -- (46.7000,38.8000) -- (46.9000,39.3000) -- (47.0000,39.7000) -- (47.2000,40.1000) -- (47.3000,40.6000) -- (47.5000,41.0000) -- (47.6000,41.4000) -- (47.7000,41.8000) -- (47.9000,42.2000) -- (48.0000,42.5000) -- (48.2000,42.9000) -- (48.3000,43.3000) -- (48.5000,43.6000) -- (48.6000,43.9000) -- (48.7000,44.3000) -- (48.9000,44.6000) -- (49.0000,44.9000) -- (49.2000,45.2000) -- (49.3000,45.5000) -- (49.5000,45.8000) -- (49.6000,46.0000) -- (49.7000,46.3000) -- (49.9000,46.6000) -- (50.0000,46.8000) -- (50.2000,47.1000) -- (50.3000,47.4000) -- (50.5000,47.6000) -- (50.6000,47.9000) -- (50.7000,48.1000) -- (50.9000,48.4000) -- (51.0000,48.7000) -- (51.2000,48.9000) -- (51.3000,49.2000) -- (51.5000,49.4000) -- (51.6000,49.7000) -- (51.7000,50.0000) -- (51.9000,50.2000) -- (52.0000,50.5000) -- (52.2000,50.8000) -- (52.3000,51.0000) -- (52.5000,51.3000) -- (52.6000,51.5000) -- (52.7000,51.7000) -- (52.9000,51.9000) -- (53.0000,52.1000) -- (53.2000,52.3000) -- (53.3000,52.5000) -- (53.5000,52.7000) -- (53.6000,52.8000) -- (53.7000,52.9000) -- (53.9000,53.1000) -- (54.0000,53.2000) -- (54.2000,53.3000) -- (54.3000,53.3000) -- (54.5000,53.4000) -- (54.6000,53.5000) -- (54.7000,53.5000) -- (54.9000,53.6000) -- (55.0000,53.6000) -- (55.2000,53.6000) -- (55.3000,53.7000) -- (55.5000,53.7000) -- (55.6000,53.7000) -- (55.7000,53.8000) -- (55.9000,53.8000) -- (56.0000,53.8000) -- (56.2000,53.9000) -- (56.3000,54.0000) -- (56.5000,54.0000) -- (56.6000,54.1000) -- (56.7000,54.2000) -- (56.9000,54.3000) -- (57.0000,54.4000) -- (57.2000,54.5000) -- (57.3000,54.6000) -- (57.5000,54.7000) -- (57.6000,54.8000) -- (57.7000,54.9000) -- (57.9000,55.0000) -- (58.0000,55.1000) -- (58.2000,55.2000) -- (58.3000,55.3000) -- (58.5000,55.3000) -- (58.6000,55.4000) -- (58.7000,55.4000) -- (58.9000,55.5000) -- (59.0000,55.5000) -- (59.2000,55.5000) -- (59.3000,55.5000) -- (59.5000,55.5000) -- (59.6000,55.5000) -- (59.7000,55.5000) -- (59.9000,55.4000) -- (60.0000,55.3000) -- (60.2000,55.3000) -- (60.3000,55.2000) -- (60.5000,55.1000) -- (60.6000,55.0000) -- (60.7000,54.9000) -- (60.9000,54.8000) -- (61.0000,54.6000) -- (61.2000,54.5000) -- (61.3000,54.4000) -- (61.5000,54.2000) -- (61.6000,54.0000) -- (61.7000,53.9000) -- (61.9000,53.7000) -- (62.0000,53.5000) -- (62.2000,53.3000) -- (62.3000,53.1000) -- (62.5000,52.9000) -- (62.6000,52.7000) -- (62.7000,52.5000) -- (62.9000,52.3000) -- (63.0000,52.1000) -- (63.2000,51.9000) -- (63.3000,51.7000) -- (63.5000,51.5000) -- (63.6000,51.3000) -- (63.7000,51.1000) -- (63.9000,50.9000) -- (64.0000,50.7000) -- (64.2000,50.5000) -- (64.3000,50.3000) -- (64.5000,50.0000) -- (64.6000,49.8000) -- (64.7000,49.6000) -- (64.9000,49.4000) -- (65.0000,49.2000) -- (65.2000,49.0000) -- (65.3000,48.8000) -- (65.5000,48.6000) -- (65.6000,48.3000) -- (65.7000,48.1000) -- (65.9000,47.9000) -- (66.0000,47.7000) -- (66.2000,47.5000) -- (66.3000,47.3000) -- (66.5000,47.0000) -- (66.6000,46.8000) -- (66.7000,46.6000) -- (66.9000,46.4000) -- (67.0000,46.2000) -- (67.2000,45.9000) -- (67.3000,45.7000) -- (67.5000,45.5000) -- (67.6000,45.3000) -- (67.7000,45.0000) -- (67.9000,44.8000) -- (68.0000,44.6000) -- (68.2000,44.3000) -- (68.3000,44.1000) -- (68.5000,43.9000) -- (68.6000,43.6000) -- (68.7000,43.4000) -- (68.9000,43.2000) -- (69.0000,43.0000) -- (69.2000,42.7000) -- (69.3000,42.5000) -- (69.5000,42.3000) -- (69.6000,42.1000) -- (69.7000,41.8000) -- (69.9000,41.6000) -- (70.0000,41.4000) -- (70.2000,41.2000) -- (70.3000,41.0000) -- (70.5000,40.8000) -- (70.6000,40.6000) -- (70.7000,40.4000) -- (70.9000,40.2000) -- (71.0000,40.0000) -- (71.2000,39.8000) -- (71.3000,39.6000) -- (71.5000,39.4000) -- (71.6000,39.2000) -- (71.7000,39.0000) -- (71.9000,38.9000) -- (72.0000,38.7000) -- (72.2000,38.5000) -- (72.3000,38.3000) -- (72.5000,38.2000) -- (72.6000,38.0000) -- (72.7000,37.9000) -- (72.9000,37.7000) -- (73.0000,37.6000) -- (73.2000,37.4000) -- (73.3000,37.3000) -- (73.5000,37.2000) -- (73.6000,37.0000) -- (73.7000,36.9000) -- (73.9000,36.8000) -- (74.0000,36.7000) -- (74.2000,36.5000) -- (74.3000,36.4000) -- (74.5000,36.3000) -- (74.6000,36.2000) -- (74.7000,36.1000) -- (74.9000,36.0000) -- (75.0000,35.9000) -- (75.2000,35.8000) -- (75.3000,35.7000) -- (75.5000,35.6000) -- (75.6000,35.5000) -- (75.7000,35.4000) -- (75.9000,35.3000) -- (76.0000,35.3000) -- (76.2000,35.2000) -- (76.3000,35.1000) -- (76.5000,35.0000) -- (76.6000,34.9000) -- (76.7000,34.9000) -- (76.9000,34.8000) -- (77.0000,34.7000) -- (77.2000,34.6000) -- (77.3000,34.6000) -- (77.5000,34.5000) -- (77.6000,34.4000) -- (77.7000,34.4000) -- (77.9000,34.3000) -- (78.0000,34.3000) -- (78.2000,34.2000) -- (78.3000,34.1000) -- (78.5000,34.1000) -- (78.6000,34.0000) -- (78.7000,34.0000) -- (78.9000,33.9000) -- (79.0000,33.9000) -- (79.2000,33.8000) -- (79.3000,33.7000) -- (79.5000,33.7000) -- (79.6000,33.6000) -- (79.7000,33.6000) -- (79.9000,33.5000) -- (80.0000,33.5000) -- (80.2000,33.5000) -- (80.3000,33.4000) -- (80.5000,33.4000) -- (80.6000,33.3000) -- (80.7000,33.3000) -- (80.9000,33.2000) -- (81.0000,33.2000) -- (81.2000,33.2000) -- (81.3000,33.1000) -- (81.5000,33.1000) -- (81.6000,33.0000) -- (81.7000,33.0000) -- (81.9000,33.0000) -- (82.0000,32.9000) -- (82.2000,32.9000) -- (82.3000,32.9000) -- (82.5000,32.9000) -- (82.6000,32.8000) -- (82.7000,32.8000) -- (82.9000,32.8000) -- (83.0000,32.8000) -- (83.2000,32.7000) -- (83.3000,32.7000) -- (83.5000,32.7000) -- (83.6000,32.7000) -- (83.7000,32.7000) -- (83.9000,32.7000) -- (84.0000,32.6000) -- (84.2000,32.6000) -- (84.3000,32.6000) -- (84.5000,32.6000) -- (84.6000,32.6000) -- (84.7000,32.6000) -- (84.9000,32.6000) -- (85.0000,32.6000) -- (85.2000,32.6000) -- (85.3000,32.6000) -- (85.4000,32.6000) -- (85.6000,32.6000) -- (85.7000,32.6000) -- (85.9000,32.6000) -- (86.0000,32.6000) -- (86.2000,32.6000) -- (86.3000,32.6000) -- (86.4000,32.7000) -- (86.6000,32.7000) -- (86.7000,32.7000) -- (86.9000,32.7000) -- (87.0000,32.7000) -- (87.2000,32.7000) -- (87.3000,32.8000) -- (87.4000,32.8000) -- (87.6000,32.8000) -- (87.7000,32.8000) -- (87.9000,32.8000) -- (88.0000,32.9000) -- (88.2000,32.9000) -- (88.3000,32.9000) -- (88.4000,33.0000) -- (88.6000,33.0000) -- (88.7000,33.0000) -- (88.9000,33.1000) -- (89.0000,33.1000) -- (89.2000,33.1000) -- (89.3000,33.2000) -- (89.4000,33.2000) -- (89.6000,33.2000) -- (89.7000,33.3000) -- (89.9000,33.3000) -- (90.0000,33.4000) -- (90.2000,33.4000) -- (90.3000,33.5000) -- (90.4000,33.5000) -- (90.6000,33.5000) -- (90.7000,33.6000) -- (90.9000,33.6000) -- (91.0000,33.7000) -- (91.2000,33.7000) -- (91.3000,33.8000) -- (91.4000,33.8000) -- (91.6000,33.9000) -- (91.7000,33.9000) -- (91.9000,34.0000) -- (92.0000,34.0000) -- (92.2000,34.1000) -- (92.3000,34.1000) -- (92.4000,34.2000) -- (92.6000,34.2000) -- (92.7000,34.3000) -- (92.9000,34.3000) -- (93.0000,34.4000) -- (93.2000,34.4000) -- (93.3000,34.5000) -- (93.4000,34.5000) -- (93.6000,34.6000) -- (93.7000,34.6000) -- (93.9000,34.7000) -- (94.0000,34.7000) -- (94.2000,34.8000) -- (94.3000,34.8000) -- (94.4000,34.9000) -- (94.6000,34.9000) -- (94.7000,35.0000) -- (94.9000,35.1000) -- (95.0000,35.1000) -- (95.2000,35.2000) -- (95.3000,35.2000) -- (95.4000,35.3000) -- (95.6000,35.3000) -- (95.7000,35.4000) -- (95.9000,35.4000) -- (96.0000,35.5000) -- (96.2000,35.5000) -- (96.3000,35.6000) -- (96.4000,35.6000) -- (96.6000,35.7000) -- (96.7000,35.7000) -- (96.9000,35.8000) -- (97.0000,35.9000) -- (97.2000,35.9000) -- (97.3000,36.0000) -- (97.4000,36.0000) -- (97.6000,36.1000) -- (97.7000,36.1000) -- (97.9000,36.2000) -- (98.0000,36.2000) -- (98.2000,36.3000) -- (98.3000,36.3000) -- (98.4000,36.4000) -- (98.6000,36.5000) -- (98.7000,36.5000) -- (98.9000,36.6000) -- (99.0000,36.6000) -- (99.2000,36.7000) -- (99.3000,36.8000) -- (99.4000,36.8000) -- (99.6000,36.9000) -- (99.7000,37.0000) -- (99.9000,37.0000) -- (100.0000,37.1000) -- (100.2000,37.1000) -- (100.3000,37.2000) -- (100.4000,37.3000) -- (100.6000,37.3000) -- (100.7000,37.4000) -- (100.9000,37.5000) -- (101.0000,37.5000) -- (101.2000,37.6000) -- (101.3000,37.7000) -- (101.4000,37.7000) -- (101.6000,37.8000) -- (101.7000,37.9000) -- (101.9000,37.9000) -- (102.0000,38.0000) -- (102.2000,38.1000) -- (102.3000,38.1000) -- (102.4000,38.2000) -- (102.6000,38.3000) -- (102.7000,38.4000) -- (102.9000,38.4000) -- (103.0000,38.5000) -- (103.2000,38.6000) -- (103.3000,38.6000) -- (103.4000,38.7000) -- (103.6000,38.8000) -- (103.7000,38.8000) -- (103.9000,38.9000) -- (104.0000,39.0000) -- (104.2000,39.0000) -- (104.3000,39.1000) -- (104.4000,39.2000) -- (104.6000,39.2000) -- (104.7000,39.3000) -- (104.9000,39.4000) -- (105.0000,39.4000) -- (105.2000,39.5000) -- (105.3000,39.6000) -- (105.4000,39.6000) -- (105.6000,39.7000) -- (105.7000,39.8000) -- (105.9000,39.8000) -- (106.0000,39.9000) -- (106.2000,40.0000) -- (106.3000,40.0000) -- (106.4000,40.1000) -- (106.6000,40.2000) -- (106.7000,40.2000) -- (106.9000,40.3000) -- (107.0000,40.4000) -- (107.2000,40.4000) -- (107.3000,40.5000) -- (107.4000,40.5000) -- (107.6000,40.6000) -- (107.7000,40.7000) -- (107.9000,40.7000) -- (108.0000,40.8000) -- (108.2000,40.9000) -- (108.3000,40.9000) -- (108.4000,41.0000) -- (108.6000,41.0000) -- (108.7000,41.1000) -- (108.9000,41.2000) -- (109.0000,41.2000) -- (109.2000,41.3000) -- (109.3000,41.3000) -- (109.4000,41.4000) -- (109.6000,41.4000) -- (109.7000,41.5000) -- (109.9000,41.5000) -- (110.0000,41.6000) -- (110.2000,41.7000) -- (110.3000,41.7000) -- (110.4000,41.8000) -- (110.6000,41.8000) -- (110.7000,41.9000) -- (110.9000,41.9000) -- (111.0000,42.0000) -- (111.2000,42.0000) -- (111.3000,42.1000) -- (111.4000,42.1000) -- (111.6000,42.2000) -- (111.7000,42.2000) -- (111.9000,42.3000) -- (112.0000,42.3000) -- (112.2000,42.4000) -- (112.3000,42.4000) -- (112.4000,42.5000) -- (112.6000,42.5000) -- (112.7000,42.6000) -- (112.9000,42.6000) -- (113.0000,42.6000) -- (113.2000,42.7000) -- (113.3000,42.7000) -- (113.4000,42.8000) -- (113.6000,42.8000) -- (113.7000,42.9000) -- (113.9000,42.9000) -- (114.0000,42.9000) -- (114.2000,43.0000) -- (114.3000,43.0000) -- (114.4000,43.1000) -- (114.6000,43.1000) -- (114.7000,43.1000) -- (114.9000,43.2000) -- (115.0000,43.2000) -- (115.2000,43.2000) -- (115.3000,43.3000) -- (115.4000,43.3000) -- (115.6000,43.3000) -- (115.7000,43.4000) -- (115.9000,43.4000) -- (116.0000,43.4000) -- (116.2000,43.5000) -- (116.3000,43.5000) -- (116.4000,43.5000) -- (116.6000,43.6000) -- (116.7000,43.6000) -- (116.9000,43.6000) -- (117.0000,43.6000) -- (117.2000,43.7000) -- (117.3000,43.7000) -- (117.4000,43.7000) -- (117.6000,43.7000) -- (117.7000,43.8000) -- (117.9000,43.8000) -- (118.0000,43.8000) -- (118.2000,43.8000) -- (118.3000,43.9000) -- (118.4000,43.9000) -- (118.6000,43.9000) -- (118.7000,43.9000) -- (118.9000,44.0000) -- (119.0000,44.0000) -- (119.2000,44.0000) -- (119.3000,44.0000) -- (119.4000,44.0000) -- (119.6000,44.1000) -- (119.7000,44.1000) -- (119.9000,44.1000) -- (120.0000,44.1000) -- (120.2000,44.1000) -- (120.3000,44.1000) -- (120.4000,44.1000) -- (120.6000,44.2000) -- (120.7000,44.2000) -- (120.9000,44.2000) -- (121.0000,44.2000) -- (121.2000,44.2000) -- (121.3000,44.2000) -- (121.4000,44.2000) -- (121.6000,44.3000) -- (121.7000,44.3000) -- (121.9000,44.3000) -- (122.0000,44.3000) -- (122.2000,44.3000) -- (122.3000,44.3000) -- (122.4000,44.3000) -- (122.6000,44.3000) -- (122.7000,44.3000) -- (122.9000,44.3000) -- (123.0000,44.3000) -- (123.2000,44.4000) -- (123.3000,44.4000) -- (123.4000,44.4000) -- (123.6000,44.4000) -- (123.7000,44.4000) -- (123.9000,44.4000) -- (124.0000,44.4000) -- (124.2000,44.4000) -- (124.3000,44.4000) -- (124.4000,44.4000) -- (124.6000,44.4000) -- (124.7000,44.4000) -- (124.9000,44.4000) -- (125.0000,44.4000) -- (125.2000,44.4000) -- (125.3000,44.4000) -- (125.4000,44.4000) -- (125.6000,44.4000) -- (125.7000,44.4000) -- (125.9000,44.4000) -- (126.0000,44.4000) -- (126.2000,44.4000) -- (126.3000,44.4000) -- (126.4000,44.4000) -- (126.6000,44.4000) -- (126.7000,44.4000) -- (126.9000,44.4000) -- (127.0000,44.4000) -- (127.2000,44.4000) -- (127.3000,44.4000);



  \end{scope}
  \begin{scope}[scale=1.008,draw=black,line join=bevel,line cap=rect,line width=0.800pt]
  \end{scope}
  \begin{scope}[scale=1.008,draw=black,line join=bevel,line cap=rect,line width=0.800pt]
  \end{scope}
  \begin{scope}[cm={{1.00769,0.0,0.0,1.00769,(-21.0,-7.5)}},draw=black,line join=round,line cap=round,line width=0.480pt]
    \path[draw] (41.5000,8.5000) -- (41.5000,84.5000) -- (127.5000,84.5000) -- (127.5000,8.5000) -- (41.5000,8.5000);



  \end{scope}
  \begin{scope}[cm={{1.00769,0.0,0.0,1.00769,(-21.0,-3.0)}},draw=ca0a0a4,dash pattern=on 0.40pt off 0.80pt,line join=round,line cap=round,line width=0.400pt]
    \path[draw] (41.5000,153.5000) -- (127.5000,153.5000);



  \end{scope}
  \begin{scope}[cm={{1.00769,0.0,0.0,1.00769,(-21.0,-3.0)}},draw=black,line join=round,line cap=round,line width=0.480pt]
    \path[draw] (41.5000,153.5000) -- (44.5000,153.5000);



    \path[draw] (127.5000,153.5000) -- (124.5000,153.5000);



  \end{scope}
  \begin{scope}[scale=1.008,draw=black,line join=bevel,line cap=rect,line width=0.800pt]
  \end{scope}
  \begin{scope}[cm={{1.00769,0.0,0.0,1.00769,(25.1923,159.215)}},draw=black,line join=bevel,line cap=rect,line width=0.800pt]
  \end{scope}
  \begin{scope}[cm={{1.00769,0.0,0.0,1.00769,(25.1923,159.215)}},draw=black,line join=bevel,line cap=rect,line width=0.800pt]
  \end{scope}
  \begin{scope}[cm={{1.00769,0.0,0.0,1.00769,(25.1923,159.215)}},draw=black,line join=bevel,line cap=rect,line width=0.800pt]
  \end{scope}
  \begin{scope}[cm={{1.00769,0.0,0.0,1.00769,(25.1923,159.215)}},draw=black,line join=bevel,line cap=rect,line width=0.800pt]
  \end{scope}
  \begin{scope}[cm={{1.00769,0.0,0.0,1.00769,(25.1923,159.215)}},draw=black,line join=bevel,line cap=rect,line width=0.800pt]
  \end{scope}
  \begin{scope}[cm={{1.00769,0.0,0.0,1.00769,(4.1923,156.215)}},draw=black,line join=bevel,line cap=rect,line width=0.800pt]
    \path[fill=black] (0.0000,0.0000) node[above right] (text366) {30};



  \end{scope}
  \begin{scope}[cm={{1.00769,0.0,0.0,1.00769,(25.1923,159.215)}},draw=black,line join=bevel,line cap=rect,line width=0.800pt]
  \end{scope}
  \begin{scope}[scale=1.008,draw=black,line join=bevel,line cap=rect,line width=0.800pt]
  \end{scope}
  \begin{scope}[cm={{1.00769,0.0,0.0,1.00769,(-21.0,-3.0)}},draw=ca0a0a4,dash pattern=on 0.40pt off 0.80pt,line join=round,line cap=round,line width=0.400pt]
    \path[draw] (41.5000,127.5000) -- (127.5000,127.5000);



  \end{scope}
  \begin{scope}[cm={{1.00769,0.0,0.0,1.00769,(-21.0,-3.0)}},draw=black,line join=round,line cap=round,line width=0.480pt]
    \path[draw] (41.5000,127.5000) -- (44.5000,127.5000);



    \path[draw] (127.5000,127.5000) -- (124.5000,127.5000);



  \end{scope}
  \begin{scope}[scale=1.008,draw=black,line join=bevel,line cap=rect,line width=0.800pt]
  \end{scope}
  \begin{scope}[cm={{1.00769,0.0,0.0,1.00769,(26.2,132.008)}},draw=black,line join=bevel,line cap=rect,line width=0.800pt]
  \end{scope}
  \begin{scope}[cm={{1.00769,0.0,0.0,1.00769,(26.2,132.008)}},draw=black,line join=bevel,line cap=rect,line width=0.800pt]
  \end{scope}
  \begin{scope}[cm={{1.00769,0.0,0.0,1.00769,(26.2,132.008)}},draw=black,line join=bevel,line cap=rect,line width=0.800pt]
  \end{scope}
  \begin{scope}[cm={{1.00769,0.0,0.0,1.00769,(26.2,132.008)}},draw=black,line join=bevel,line cap=rect,line width=0.800pt]
  \end{scope}
  \begin{scope}[cm={{1.00769,0.0,0.0,1.00769,(26.2,132.008)}},draw=black,line join=bevel,line cap=rect,line width=0.800pt]
  \end{scope}
  \begin{scope}[cm={{1.00769,0.0,0.0,1.00769,(5.2,129.008)}},draw=black,line join=bevel,line cap=rect,line width=0.800pt]
    \path[fill=black] (0.0000,0.0000) node[above right] (text396) {33};



  \end{scope}
  \begin{scope}[cm={{1.00769,0.0,0.0,1.00769,(26.2,132.008)}},draw=black,line join=bevel,line cap=rect,line width=0.800pt]
  \end{scope}
  \begin{scope}[scale=1.008,draw=black,line join=bevel,line cap=rect,line width=0.800pt]
  \end{scope}
  \begin{scope}[cm={{1.00769,0.0,0.0,1.00769,(-21.0,-3.0)}},draw=ca0a0a4,dash pattern=on 0.40pt off 0.80pt,line join=round,line cap=round,line width=0.400pt]
    \path[draw] (41.5000,101.5000) -- (127.5000,101.5000);



  \end{scope}
  \begin{scope}[cm={{1.00769,0.0,0.0,1.00769,(-21.0,-3.0)}},draw=black,line join=round,line cap=round,line width=0.480pt]
    \path[draw] (41.5000,101.5000) -- (44.5000,101.5000);



    \path[draw] (127.5000,101.5000) -- (124.5000,101.5000);



  \end{scope}
  \begin{scope}[scale=1.008,draw=black,line join=bevel,line cap=rect,line width=0.800pt]
  \end{scope}
  \begin{scope}[cm={{1.00769,0.0,0.0,1.00769,(25.1923,105.808)}},draw=black,line join=bevel,line cap=rect,line width=0.800pt]
  \end{scope}
  \begin{scope}[cm={{1.00769,0.0,0.0,1.00769,(25.1923,105.808)}},draw=black,line join=bevel,line cap=rect,line width=0.800pt]
  \end{scope}
  \begin{scope}[cm={{1.00769,0.0,0.0,1.00769,(25.1923,105.808)}},draw=black,line join=bevel,line cap=rect,line width=0.800pt]
  \end{scope}
  \begin{scope}[cm={{1.00769,0.0,0.0,1.00769,(25.1923,105.808)}},draw=black,line join=bevel,line cap=rect,line width=0.800pt]
  \end{scope}
  \begin{scope}[cm={{1.00769,0.0,0.0,1.00769,(25.1923,105.808)}},draw=black,line join=bevel,line cap=rect,line width=0.800pt]
  \end{scope}
  \begin{scope}[cm={{1.00769,0.0,0.0,1.00769,(4.1923,102.808)}},draw=black,line join=bevel,line cap=rect,line width=0.800pt]
    \path[fill=black] (0.0000,0.0000) node[above right] (text426) {36};



  \end{scope}
  \begin{scope}[cm={{1.00769,0.0,0.0,1.00769,(25.1923,105.808)}},draw=black,line join=bevel,line cap=rect,line width=0.800pt]
  \end{scope}
  \begin{scope}[scale=1.008,draw=black,line join=bevel,line cap=rect,line width=0.800pt]
  \end{scope}
  \begin{scope}[cm={{1.00769,0.0,0.0,1.00769,(-21.0,-3.0)}},draw=ca0a0a4,dash pattern=on 0.40pt off 0.80pt,line join=round,line cap=round,line width=0.400pt]
    \path[draw] (41.5000,164.5000) -- (41.5000,88.5000);



  \end{scope}
  \begin{scope}[cm={{1.00769,0.0,0.0,1.00769,(-21.0,-3.0)}},draw=black,line join=round,line cap=round,line width=0.480pt]
    \path[draw] (41.5000,164.5000) -- (41.5000,159.5000);



    \path[draw] (41.5000,88.5000) -- (41.5000,92.5000);



  \end{scope}
  \begin{scope}[scale=1.008,draw=black,line join=bevel,line cap=rect,line width=0.800pt]
  \end{scope}
  \begin{scope}[cm={{1.00769,0.0,0.0,1.00769,(39.3,180.377)}},draw=black,line join=bevel,line cap=rect,line width=0.800pt]
  \end{scope}
  \begin{scope}[cm={{1.00769,0.0,0.0,1.00769,(39.3,180.377)}},draw=black,line join=bevel,line cap=rect,line width=0.800pt]
  \end{scope}
  \begin{scope}[cm={{1.00769,0.0,0.0,1.00769,(39.3,180.377)}},draw=black,line join=bevel,line cap=rect,line width=0.800pt]
  \end{scope}
  \begin{scope}[cm={{1.00769,0.0,0.0,1.00769,(39.3,180.377)}},draw=black,line join=bevel,line cap=rect,line width=0.800pt]
  \end{scope}
  \begin{scope}[cm={{1.00769,0.0,0.0,1.00769,(39.3,180.377)}},draw=black,line join=bevel,line cap=rect,line width=0.800pt]
  \end{scope}
  \begin{scope}[cm={{1.00769,0.0,0.0,1.00769,(18.3,177.377)}},draw=black,line join=bevel,line cap=rect,line width=0.800pt]
    \path[fill=black] (0.0000,0.0000) node[above right] (text456) {0};



  \end{scope}
  \begin{scope}[cm={{1.00769,0.0,0.0,1.00769,(39.3,180.377)}},draw=black,line join=bevel,line cap=rect,line width=0.800pt]
  \end{scope}
  \begin{scope}[scale=1.008,draw=black,line join=bevel,line cap=rect,line width=0.800pt]
  \end{scope}
  \begin{scope}[cm={{1.00769,0.0,0.0,1.00769,(-21.0,-3.0)}},draw=ca0a0a4,dash pattern=on 0.40pt off 0.80pt,line join=round,line cap=round,line width=0.400pt]
    \path[draw] (70.5000,164.5000) -- (70.5000,88.5000);



  \end{scope}
  \begin{scope}[cm={{1.00769,0.0,0.0,1.00769,(-21.0,-3.0)}},draw=black,line join=round,line cap=round,line width=0.480pt]
    \path[draw] (70.5000,164.5000) -- (70.5000,159.5000);



    \path[draw] (70.5000,88.5000) -- (70.5000,92.5000);



  \end{scope}
  \begin{scope}[scale=1.008,draw=black,line join=bevel,line cap=rect,line width=0.800pt]
  \end{scope}
  \begin{scope}[cm={{1.00769,0.0,0.0,1.00769,(67.5154,180.377)}},draw=black,line join=bevel,line cap=rect,line width=0.800pt]
  \end{scope}
  \begin{scope}[cm={{1.00769,0.0,0.0,1.00769,(67.5154,180.377)}},draw=black,line join=bevel,line cap=rect,line width=0.800pt]
  \end{scope}
  \begin{scope}[cm={{1.00769,0.0,0.0,1.00769,(67.5154,180.377)}},draw=black,line join=bevel,line cap=rect,line width=0.800pt]
  \end{scope}
  \begin{scope}[cm={{1.00769,0.0,0.0,1.00769,(67.5154,180.377)}},draw=black,line join=bevel,line cap=rect,line width=0.800pt]
  \end{scope}
  \begin{scope}[cm={{1.00769,0.0,0.0,1.00769,(67.5154,180.377)}},draw=black,line join=bevel,line cap=rect,line width=0.800pt]
  \end{scope}
  \begin{scope}[cm={{1.00769,0.0,0.0,1.00769,(46.5154,177.377)}},draw=black,line join=bevel,line cap=rect,line width=0.800pt]
    \path[fill=black] (0.0000,0.0000) node[above right] (text486) {2};



  \end{scope}
  \begin{scope}[cm={{1.00769,0.0,0.0,1.00769,(67.5154,180.377)}},draw=black,line join=bevel,line cap=rect,line width=0.800pt]
  \end{scope}
  \begin{scope}[scale=1.008,draw=black,line join=bevel,line cap=rect,line width=0.800pt]
  \end{scope}
  \begin{scope}[cm={{1.00769,0.0,0.0,1.00769,(-21.0,-3.0)}},draw=ca0a0a4,dash pattern=on 0.40pt off 0.80pt,line join=round,line cap=round,line width=0.400pt]
    \path[draw] (98.5000,164.5000) -- (98.5000,88.5000);



  \end{scope}
  \begin{scope}[cm={{1.00769,0.0,0.0,1.00769,(-21.0,-3.0)}},draw=black,line join=round,line cap=round,line width=0.480pt]
    \path[draw] (98.5000,164.5000) -- (98.5000,159.5000);



    \path[draw] (98.5000,88.5000) -- (98.5000,92.5000);



  \end{scope}
  \begin{scope}[scale=1.008,draw=black,line join=bevel,line cap=rect,line width=0.800pt]
  \end{scope}
  \begin{scope}[cm={{1.00769,0.0,0.0,1.00769,(97.2423,180.377)}},draw=black,line join=bevel,line cap=rect,line width=0.800pt]
  \end{scope}
  \begin{scope}[cm={{1.00769,0.0,0.0,1.00769,(97.2423,180.377)}},draw=black,line join=bevel,line cap=rect,line width=0.800pt]
  \end{scope}
  \begin{scope}[cm={{1.00769,0.0,0.0,1.00769,(97.2423,180.377)}},draw=black,line join=bevel,line cap=rect,line width=0.800pt]
  \end{scope}
  \begin{scope}[cm={{1.00769,0.0,0.0,1.00769,(97.2423,180.377)}},draw=black,line join=bevel,line cap=rect,line width=0.800pt]
  \end{scope}
  \begin{scope}[cm={{1.00769,0.0,0.0,1.00769,(97.2423,180.377)}},draw=black,line join=bevel,line cap=rect,line width=0.800pt]
  \end{scope}
  \begin{scope}[cm={{1.00769,0.0,0.0,1.00769,(76.2423,177.377)}},draw=black,line join=bevel,line cap=rect,line width=0.800pt]
    \path[fill=black] (0.0000,0.0000) node[above right] (text516) {4};



  \end{scope}
  \begin{scope}[cm={{1.00769,0.0,0.0,1.00769,(97.2423,180.377)}},draw=black,line join=bevel,line cap=rect,line width=0.800pt]
  \end{scope}
  \begin{scope}[scale=1.008,draw=black,line join=bevel,line cap=rect,line width=0.800pt]
  \end{scope}
  \begin{scope}[cm={{1.00769,0.0,0.0,1.00769,(-21.0,-3.0)}},draw=ca0a0a4,dash pattern=on 0.40pt off 0.80pt,line join=round,line cap=round,line width=0.400pt]
    \path[draw] (127.5000,164.5000) -- (127.5000,88.5000);



  \end{scope}
  \begin{scope}[cm={{1.00769,0.0,0.0,1.00769,(-21.0,-3.0)}},draw=black,line join=round,line cap=round,line width=0.480pt]
    \path[draw] (127.5000,164.5000) -- (127.5000,159.5000);



    \path[draw] (127.5000,88.5000) -- (127.5000,92.5000);



  \end{scope}
  \begin{scope}[scale=1.008,draw=black,line join=bevel,line cap=rect,line width=0.800pt]
  \end{scope}
  \begin{scope}[cm={{1.00769,0.0,0.0,1.00769,(124.954,180.377)}},draw=black,line join=bevel,line cap=rect,line width=0.800pt]
  \end{scope}
  \begin{scope}[cm={{1.00769,0.0,0.0,1.00769,(124.954,180.377)}},draw=black,line join=bevel,line cap=rect,line width=0.800pt]
  \end{scope}
  \begin{scope}[cm={{1.00769,0.0,0.0,1.00769,(124.954,180.377)}},draw=black,line join=bevel,line cap=rect,line width=0.800pt]
  \end{scope}
  \begin{scope}[cm={{1.00769,0.0,0.0,1.00769,(124.954,180.377)}},draw=black,line join=bevel,line cap=rect,line width=0.800pt]
  \end{scope}
  \begin{scope}[cm={{1.00769,0.0,0.0,1.00769,(124.954,180.377)}},draw=black,line join=bevel,line cap=rect,line width=0.800pt]
  \end{scope}
  \begin{scope}[cm={{1.00769,0.0,0.0,1.00769,(103.954,177.377)}},draw=black,line join=bevel,line cap=rect,line width=0.800pt]
    \path[fill=black] (0.0000,0.0000) node[above right] (text546) {6};



  \end{scope}
  \begin{scope}[cm={{1.00769,0.0,0.0,1.00769,(124.954,180.377)}},draw=black,line join=bevel,line cap=rect,line width=0.800pt]
  \end{scope}
  \begin{scope}[scale=1.008,draw=black,line join=bevel,line cap=rect,line width=0.800pt]
  \end{scope}
  \begin{scope}[cm={{1.00769,0.0,0.0,1.00769,(-21.0,-3.0)}},draw=black,line join=round,line cap=round,line width=0.480pt]
    \path[draw] (41.5000,88.5000) -- (41.5000,164.5000) -- (127.5000,164.5000) -- (127.5000,88.5000) -- (41.5000,88.5000);



  \end{scope}
  \begin{scope}[scale=1.008,draw=black,line join=bevel,line cap=rect,line width=0.800pt]
  \end{scope}
  \begin{scope}[scale=1.008,draw=black,line join=bevel,line cap=rect,line width=0.800pt]
  \end{scope}
  \begin{scope}[cm={{1.00769,0.0,0.0,1.00769,(-21.0,-3.0)}},fill=cffffff]
    \path[fill,rounded corners=0.0000cm] (105.0000,92.0000) rectangle (121.0000,108.0000);



  \end{scope}
  \begin{scope}[scale=1.008,draw=black,line join=bevel,line cap=rect,line width=0.800pt]
  \end{scope}
  \begin{scope}[scale=1.008,draw=black,line join=bevel,line cap=rect,line width=0.800pt]
  \end{scope}
  \begin{scope}[cm={{1.00769,0.0,0.0,1.00769,(-21.0,-3.0)}},draw=black,line join=round,line cap=round,line width=0.800pt]
    \path[draw] (104.5000,108.5000) -- (104.5000,92.5000) -- (120.5000,92.5000) -- (120.5000,108.5000) -- (104.5000,108.5000);



  \end{scope}
  \begin{scope}[scale=1.008,draw=black,line join=bevel,line cap=rect,line width=0.800pt]
  \end{scope}
  \begin{scope}[cm={{1.00769,0.0,0.0,1.00769,(109.838,104.8)}},draw=black,line join=bevel,line cap=rect,line width=0.800pt]
  \end{scope}
  \begin{scope}[cm={{1.00769,0.0,0.0,1.00769,(109.838,104.8)}},draw=black,line join=bevel,line cap=rect,line width=0.800pt]
  \end{scope}
  \begin{scope}[cm={{1.00769,0.0,0.0,1.00769,(109.838,104.8)}},draw=black,line join=bevel,line cap=rect,line width=0.800pt]
  \end{scope}
  \begin{scope}[cm={{1.00769,0.0,0.0,1.00769,(109.838,104.8)}},draw=black,line join=bevel,line cap=rect,line width=0.800pt]
  \end{scope}
  \begin{scope}[cm={{1.00769,0.0,0.0,1.00769,(109.838,104.8)}},draw=black,line join=bevel,line cap=rect,line width=0.800pt]
  \end{scope}
  \begin{scope}[cm={{1.00769,0.0,0.0,1.00769,(88.838,101.8)}},draw=black,line join=bevel,line cap=rect,line width=0.800pt]
    \path[fill=black] (0.0000,0.0000) node[above right] (text586) {II};



  \end{scope}
  \begin{scope}[cm={{1.00769,0.0,0.0,1.00769,(109.838,104.8)}},draw=black,line join=bevel,line cap=rect,line width=0.800pt]
  \end{scope}
  \begin{scope}[cm={{1.00769,0.0,0.0,1.00769,(61.9731,195.492)}},draw=black,line join=bevel,line cap=rect,line width=0.800pt]
  \end{scope}
  \begin{scope}[cm={{1.00769,0.0,0.0,1.00769,(61.9731,195.492)}},draw=black,line join=bevel,line cap=rect,line width=0.800pt]
  \end{scope}
  \begin{scope}[cm={{1.00769,0.0,0.0,1.00769,(61.9731,195.492)}},draw=black,line join=bevel,line cap=rect,line width=0.800pt]
  \end{scope}
  \begin{scope}[cm={{1.00769,0.0,0.0,1.00769,(61.9731,195.492)}},draw=black,line join=bevel,line cap=rect,line width=0.800pt]
  \end{scope}
  \begin{scope}[cm={{1.00769,0.0,0.0,1.00769,(61.9731,195.492)}},draw=black,line join=bevel,line cap=rect,line width=0.800pt]
  \end{scope}
  \begin{scope}[cm={{1.00769,0.0,0.0,1.00769,(40.9731,192.492)}},draw=black,line join=bevel,line cap=rect,line width=0.800pt]
    \path[fill=black] (0.0000,0.0000) node[above right] (text602) {Time (sec)};



  \end{scope}
  \begin{scope}[cm={{1.00769,0.0,0.0,1.00769,(61.9731,195.492)}},draw=black,line join=bevel,line cap=rect,line width=0.800pt]
  \end{scope}
  \begin{scope}[scale=1.008,draw=black,line join=bevel,line cap=rect,line width=0.800pt]
  \end{scope}
  \begin{scope}[scale=1.008,draw=black,line join=bevel,line cap=rect,line width=0.800pt]
  \end{scope}
  \begin{scope}[scale=1.008,draw=black,line join=bevel,line cap=rect,line width=0.800pt]
  \end{scope}
  \begin{scope}[cm={{1.00769,0.0,0.0,1.00769,(-21.0,-3.0)}},draw=black,line join=round,line cap=round,line width=0.480pt]
    \path[draw] (41.6000,103.3000) -- (41.6000,103.3000) -- (41.7000,103.6000) -- (41.9000,104.0000) -- (42.0000,104.3000) -- (42.2000,104.6000) -- (42.3000,104.9000) -- (42.5000,105.1000) -- (42.6000,105.4000) -- (42.7000,105.7000) -- (42.9000,106.0000) -- (43.0000,106.3000) -- (43.2000,106.5000) -- (43.3000,106.8000) -- (43.5000,107.1000) -- (43.6000,107.3000) -- (43.7000,107.6000) -- (43.9000,107.9000) -- (44.0000,108.1000) -- (44.2000,108.4000) -- (44.3000,108.6000) -- (44.5000,108.8000) -- (44.6000,109.1000) -- (44.7000,109.3000) -- (44.9000,109.5000) -- (45.0000,109.8000) -- (45.2000,110.0000) -- (45.3000,110.2000) -- (45.5000,110.4000) -- (45.6000,110.7000) -- (45.7000,110.9000) -- (45.9000,111.1000) -- (46.0000,111.3000) -- (46.2000,111.5000) -- (46.3000,111.7000) -- (46.5000,111.9000) -- (46.6000,112.1000) -- (46.7000,112.3000) -- (46.9000,112.4000) -- (47.0000,112.6000) -- (47.2000,112.8000) -- (47.3000,113.0000) -- (47.5000,113.2000) -- (47.6000,113.3000) -- (47.7000,113.5000) -- (47.9000,113.7000) -- (48.0000,113.9000) -- (48.2000,114.0000) -- (48.3000,114.2000) -- (48.5000,114.3000) -- (48.6000,114.5000) -- (48.7000,114.7000) -- (48.9000,114.8000) -- (49.0000,115.0000) -- (49.2000,115.1000) -- (49.3000,115.2000) -- (49.5000,115.4000) -- (49.6000,115.5000) -- (49.7000,115.7000) -- (49.9000,115.8000) -- (50.0000,115.9000) -- (50.2000,116.1000) -- (50.3000,116.2000) -- (50.5000,116.3000) -- (50.6000,116.4000) -- (50.7000,116.6000) -- (50.9000,116.7000) -- (51.0000,116.8000) -- (51.2000,116.9000) -- (51.3000,117.0000) -- (51.5000,117.1000) -- (51.6000,117.2000) -- (51.7000,117.4000) -- (51.9000,117.5000) -- (52.0000,117.6000) -- (52.2000,117.7000) -- (52.3000,117.8000) -- (52.5000,117.9000) -- (52.6000,118.0000) -- (52.7000,118.0000) -- (52.9000,118.1000) -- (53.0000,118.2000) -- (53.2000,118.3000) -- (53.3000,118.4000) -- (53.5000,118.5000) -- (53.6000,118.6000) -- (53.7000,118.7000) -- (53.9000,118.7000) -- (54.0000,118.8000) -- (54.2000,118.9000) -- (54.3000,119.0000) -- (54.5000,119.0000) -- (54.6000,119.1000) -- (54.7000,119.2000) -- (54.9000,119.2000) -- (55.0000,119.3000) -- (55.2000,119.4000) -- (55.3000,119.4000) -- (55.5000,119.5000) -- (55.6000,119.6000) -- (55.7000,119.6000) -- (55.9000,119.7000) -- (56.0000,119.7000) -- (56.2000,119.8000) -- (56.3000,119.8000) -- (56.5000,119.9000) -- (56.6000,119.9000) -- (56.7000,120.0000) -- (56.9000,120.0000) -- (57.0000,120.1000) -- (57.2000,120.1000) -- (57.3000,120.2000) -- (57.5000,120.2000) -- (57.6000,120.2000) -- (57.7000,120.3000) -- (57.9000,120.3000) -- (58.0000,120.4000) -- (58.2000,120.4000) -- (58.3000,120.4000) -- (58.5000,120.5000) -- (58.6000,120.5000) -- (58.7000,120.5000) -- (58.9000,120.6000) -- (59.0000,120.6000) -- (59.2000,120.6000) -- (59.3000,120.6000) -- (59.5000,120.7000) -- (59.6000,120.7000) -- (59.7000,120.7000) -- (59.9000,120.7000) -- (60.0000,120.8000) -- (60.2000,120.8000) -- (60.3000,120.8000) -- (60.5000,120.8000) -- (60.6000,120.8000) -- (60.7000,120.9000) -- (60.9000,120.9000) -- (61.0000,120.9000) -- (61.2000,120.9000) -- (61.3000,120.9000) -- (61.5000,120.9000) -- (61.6000,120.9000) -- (61.7000,121.0000) -- (61.9000,121.0000) -- (62.0000,121.0000) -- (62.2000,121.0000) -- (62.3000,121.0000) -- (62.5000,121.0000) -- (62.6000,121.0000) -- (62.7000,121.0000) -- (62.9000,121.0000) -- (63.0000,121.0000) -- (63.2000,121.0000) -- (63.3000,121.0000) -- (63.5000,121.0000) -- (63.6000,121.0000) -- (63.7000,121.0000) -- (63.9000,121.0000) -- (64.0000,121.0000) -- (64.2000,121.0000) -- (64.3000,121.0000) -- (64.5000,121.0000) -- (64.6000,121.0000) -- (64.7000,121.0000) -- (64.9000,121.0000) -- (65.0000,121.0000) -- (65.2000,121.0000) -- (65.3000,121.0000) -- (65.5000,121.0000) -- (65.6000,121.0000) -- (65.7000,121.0000) -- (65.9000,121.0000) -- (66.0000,121.0000) -- (66.2000,121.0000) -- (66.3000,121.0000) -- (66.5000,120.9000) -- (66.6000,120.9000) -- (66.7000,120.9000) -- (66.9000,120.9000) -- (67.0000,120.9000) -- (67.2000,120.9000) -- (67.3000,120.9000) -- (67.5000,120.9000) -- (67.6000,120.9000) -- (67.7000,120.9000) -- (67.9000,120.9000) -- (68.0000,120.9000) -- (68.2000,120.8000) -- (68.3000,120.8000) -- (68.5000,120.8000) -- (68.6000,120.8000) -- (68.7000,120.8000) -- (68.9000,120.8000) -- (69.0000,120.8000) -- (69.2000,120.8000) -- (69.3000,120.8000) -- (69.5000,120.8000) -- (69.6000,120.8000) -- (69.7000,120.8000) -- (69.9000,120.8000) -- (70.0000,120.8000) -- (70.2000,120.8000) -- (70.3000,120.8000) -- (70.5000,120.8000) -- (70.6000,120.8000) -- (70.7000,120.8000) -- (70.9000,120.8000) -- (71.0000,120.8000) -- (71.2000,120.8000) -- (71.3000,120.8000) -- (71.5000,120.9000) -- (71.6000,120.9000) -- (71.7000,120.9000) -- (71.9000,120.9000) -- (72.0000,120.9000) -- (72.2000,120.9000) -- (72.3000,121.0000) -- (72.5000,121.0000) -- (72.6000,121.0000) -- (72.7000,121.0000) -- (72.9000,121.1000) -- (73.0000,121.1000) -- (73.2000,121.1000) -- (73.3000,121.2000) -- (73.5000,121.2000) -- (73.6000,121.2000) -- (73.7000,121.2000) -- (73.9000,121.3000) -- (74.0000,121.3000) -- (74.2000,121.3000) -- (74.3000,121.3000) -- (74.5000,121.4000) -- (74.6000,121.4000) -- (74.7000,121.4000) -- (74.9000,121.5000) -- (75.0000,121.5000) -- (75.2000,121.5000) -- (75.3000,121.5000) -- (75.5000,121.6000) -- (75.6000,121.6000) -- (75.7000,121.6000) -- (75.9000,121.6000) -- (76.0000,121.7000) -- (76.2000,121.7000) -- (76.3000,121.7000) -- (76.5000,121.7000) -- (76.6000,121.7000) -- (76.7000,121.7000) -- (76.9000,121.8000) -- (77.0000,121.8000) -- (77.2000,121.8000) -- (77.3000,121.8000) -- (77.5000,121.8000) -- (77.6000,121.8000) -- (77.7000,121.9000) -- (77.9000,121.9000) -- (78.0000,121.9000) -- (78.2000,121.9000) -- (78.3000,121.9000) -- (78.5000,121.9000) -- (78.6000,121.9000) -- (78.7000,121.9000) -- (78.9000,121.9000) -- (79.0000,121.9000) -- (79.2000,121.9000) -- (79.3000,121.9000) -- (79.5000,121.9000) -- (79.6000,121.9000) -- (79.7000,122.0000) -- (79.9000,122.0000) -- (80.0000,122.0000) -- (80.2000,122.0000) -- (80.3000,122.0000) -- (80.5000,121.9000) -- (80.6000,121.9000) -- (80.7000,121.9000) -- (80.9000,121.9000) -- (81.0000,121.9000) -- (81.2000,121.9000) -- (81.3000,121.9000) -- (81.5000,121.9000) -- (81.6000,121.9000) -- (81.7000,121.9000) -- (81.9000,121.9000) -- (82.0000,121.9000) -- (82.2000,121.9000) -- (82.3000,121.9000) -- (82.5000,121.9000) -- (82.6000,121.9000) -- (82.7000,121.8000) -- (82.9000,121.8000) -- (83.0000,121.8000) -- (83.2000,121.8000) -- (83.3000,121.8000) -- (83.5000,121.8000) -- (83.6000,121.8000) -- (83.7000,121.8000) -- (83.9000,121.7000) -- (84.0000,121.7000) -- (84.2000,121.7000) -- (84.3000,121.7000) -- (84.5000,121.7000) -- (84.6000,121.7000) -- (84.7000,121.6000) -- (84.9000,121.6000) -- (85.0000,121.6000) -- (85.2000,121.6000) -- (85.3000,121.6000) -- (85.5000,121.5000) -- (85.6000,121.5000) -- (85.7000,121.5000) -- (85.9000,121.5000) -- (86.0000,121.5000) -- (86.2000,121.5000) -- (86.3000,121.4000) -- (86.5000,121.4000) -- (86.6000,121.4000) -- (86.7000,121.4000) -- (86.9000,121.3000) -- (87.0000,121.3000) -- (87.2000,121.3000) -- (87.3000,121.3000) -- (87.5000,121.3000) -- (87.6000,121.2000) -- (87.7000,121.2000) -- (87.9000,121.2000) -- (88.0000,121.2000) -- (88.2000,121.1000) -- (88.3000,121.1000) -- (88.5000,121.1000) -- (88.6000,121.1000) -- (88.7000,121.0000) -- (88.9000,121.0000) -- (89.0000,121.0000) -- (89.2000,121.0000) -- (89.3000,120.9000) -- (89.5000,120.9000) -- (89.6000,120.9000) -- (89.7000,120.9000) -- (89.9000,120.8000) -- (90.0000,120.8000) -- (90.2000,120.8000) -- (90.3000,120.8000) -- (90.5000,120.7000) -- (90.6000,120.7000) -- (90.7000,120.7000) -- (90.9000,120.6000) -- (91.0000,120.6000) -- (91.2000,120.6000) -- (91.3000,120.6000) -- (91.5000,120.5000) -- (91.6000,120.5000) -- (91.7000,120.5000) -- (91.9000,120.5000) -- (92.0000,120.4000) -- (92.2000,120.4000) -- (92.3000,120.4000) -- (92.5000,120.4000) -- (92.6000,120.3000) -- (92.7000,120.3000) -- (92.9000,120.3000) -- (93.0000,120.2000) -- (93.2000,120.2000) -- (93.3000,120.2000) -- (93.4000,120.2000) -- (93.6000,120.1000) -- (93.7000,120.1000) -- (93.9000,120.1000) -- (94.0000,120.1000) -- (94.2000,120.0000) -- (94.3000,120.0000) -- (94.4000,120.0000) -- (94.6000,120.0000) -- (94.7000,119.9000) -- (94.9000,119.9000) -- (95.0000,119.9000) -- (95.2000,119.8000) -- (95.3000,119.8000) -- (95.4000,119.8000) -- (95.6000,119.8000) -- (95.7000,119.7000) -- (95.9000,119.7000) -- (96.0000,119.7000) -- (96.2000,119.7000) -- (96.3000,119.6000) -- (96.4000,119.6000) -- (96.6000,119.6000) -- (96.7000,119.6000) -- (96.9000,119.5000) -- (97.0000,119.5000) -- (97.2000,119.5000) -- (97.3000,119.5000) -- (97.4000,119.4000) -- (97.6000,119.4000) -- (97.7000,119.4000) -- (97.9000,119.3000) -- (98.0000,119.3000) -- (98.2000,119.3000) -- (98.3000,119.3000) -- (98.4000,119.2000) -- (98.6000,119.2000) -- (98.7000,119.2000) -- (98.9000,119.1000) -- (99.0000,119.1000) -- (99.2000,119.1000) -- (99.3000,119.1000) -- (99.4000,119.0000) -- (99.6000,119.0000) -- (99.7000,119.0000) -- (99.9000,119.0000) -- (100.0000,118.9000) -- (100.2000,118.9000) -- (100.3000,118.9000) -- (100.4000,118.9000) -- (100.6000,118.8000) -- (100.7000,118.8000) -- (100.9000,118.8000) -- (101.0000,118.8000) -- (101.2000,118.7000) -- (101.3000,118.7000) -- (101.4000,118.7000) -- (101.6000,118.7000) -- (101.7000,118.6000) -- (101.9000,118.6000) -- (102.0000,118.6000) -- (102.2000,118.6000) -- (102.3000,118.5000) -- (102.4000,118.5000) -- (102.6000,118.5000) -- (102.7000,118.5000) -- (102.9000,118.4000) -- (103.0000,118.4000) -- (103.2000,118.4000) -- (103.3000,118.4000) -- (103.4000,118.4000) -- (103.6000,118.3000) -- (103.7000,118.3000) -- (103.9000,118.3000) -- (104.0000,118.3000) -- (104.2000,118.3000) -- (104.3000,118.2000) -- (104.4000,118.2000) -- (104.6000,118.2000) -- (104.7000,118.2000) -- (104.9000,118.2000) -- (105.0000,118.1000) -- (105.2000,118.1000) -- (105.3000,118.1000) -- (105.4000,118.1000) -- (105.6000,118.1000) -- (105.7000,118.0000) -- (105.9000,118.0000) -- (106.0000,118.0000) -- (106.2000,118.0000) -- (106.3000,118.0000) -- (106.4000,118.0000) -- (106.6000,117.9000) -- (106.7000,117.9000) -- (106.9000,117.9000) -- (107.0000,117.9000) -- (107.2000,117.9000) -- (107.3000,117.9000) -- (107.4000,117.9000) -- (107.6000,117.8000) -- (107.7000,117.8000) -- (107.9000,117.8000) -- (108.0000,117.8000) -- (108.2000,117.8000) -- (108.3000,117.8000) -- (108.4000,117.8000) -- (108.6000,117.8000) -- (108.7000,117.7000) -- (108.9000,117.7000) -- (109.0000,117.7000) -- (109.2000,117.7000) -- (109.3000,117.7000) -- (109.4000,117.7000) -- (109.6000,117.7000) -- (109.7000,117.7000) -- (109.9000,117.7000) -- (110.0000,117.6000) -- (110.2000,117.6000) -- (110.3000,117.6000) -- (110.4000,117.6000) -- (110.6000,117.6000) -- (110.7000,117.6000) -- (110.9000,117.6000) -- (111.0000,117.6000) -- (111.2000,117.6000) -- (111.3000,117.6000) -- (111.4000,117.6000) -- (111.6000,117.6000) -- (111.7000,117.5000) -- (111.9000,117.5000) -- (112.0000,117.5000) -- (112.2000,117.5000) -- (112.3000,117.5000) -- (112.4000,117.5000) -- (112.6000,117.5000) -- (112.7000,117.5000) -- (112.9000,117.5000) -- (113.0000,117.5000) -- (113.2000,117.5000) -- (113.3000,117.5000) -- (113.4000,117.5000) -- (113.6000,117.5000) -- (113.7000,117.5000) -- (113.9000,117.5000) -- (114.0000,117.5000) -- (114.2000,117.5000) -- (114.3000,117.4000) -- (114.4000,117.4000) -- (114.6000,117.4000) -- (114.7000,117.4000) -- (114.9000,117.4000) -- (115.0000,117.4000) -- (115.2000,117.4000) -- (115.3000,117.4000) -- (115.4000,117.4000) -- (115.6000,117.4000) -- (115.7000,117.4000) -- (115.9000,117.4000) -- (116.0000,117.4000) -- (116.2000,117.4000) -- (116.3000,117.4000) -- (116.4000,117.4000) -- (116.6000,117.4000) -- (116.7000,117.4000) -- (116.9000,117.4000) -- (117.0000,117.4000) -- (117.2000,117.4000) -- (117.3000,117.4000) -- (117.4000,117.4000) -- (117.6000,117.4000) -- (117.7000,117.4000) -- (117.9000,117.4000) -- (118.0000,117.4000) -- (118.2000,117.4000) -- (118.3000,117.4000) -- (118.4000,117.4000) -- (118.6000,117.4000) -- (118.7000,117.4000) -- (118.9000,117.4000) -- (119.0000,117.4000) -- (119.2000,117.4000) -- (119.3000,117.4000) -- (119.4000,117.4000) -- (119.6000,117.4000) -- (119.7000,117.4000) -- (119.9000,117.4000) -- (120.0000,117.4000) -- (120.2000,117.4000) -- (120.3000,117.4000) -- (120.4000,117.4000) -- (120.6000,117.4000) -- (120.7000,117.4000) -- (120.9000,117.4000) -- (121.0000,117.4000) -- (121.2000,117.4000) -- (121.3000,117.4000) -- (121.4000,117.4000) -- (121.6000,117.4000) -- (121.7000,117.4000) -- (121.9000,117.4000) -- (122.0000,117.4000) -- (122.2000,117.4000) -- (122.3000,117.4000) -- (122.4000,117.4000) -- (122.6000,117.4000) -- (122.7000,117.4000) -- (122.9000,117.4000) -- (123.0000,117.4000) -- (123.2000,117.4000) -- (123.3000,117.4000) -- (123.4000,117.4000) -- (123.6000,117.4000) -- (123.7000,117.4000) -- (123.9000,117.4000) -- (124.0000,117.4000) -- (124.2000,117.4000) -- (124.3000,117.4000) -- (124.4000,117.5000) -- (124.6000,117.5000) -- (124.7000,117.5000) -- (124.9000,117.5000) -- (125.0000,117.5000) -- (125.2000,117.5000) -- (125.3000,117.5000) -- (125.4000,117.5000) -- (125.6000,117.5000) -- (125.7000,117.5000) -- (125.9000,117.5000) -- (126.0000,117.5000) -- (126.2000,117.5000) -- (126.3000,117.5000) -- (126.4000,117.5000) -- (126.6000,117.5000) -- (126.7000,117.5000) -- (126.9000,117.5000) -- (127.0000,117.5000) -- (127.2000,117.5000) -- (127.3000,117.5000);



  \end{scope}
  \begin{scope}[scale=1.008,draw=black,line join=bevel,line cap=rect,line width=0.800pt]
  \end{scope}
  \begin{scope}[scale=1.008,draw=black,line join=bevel,line cap=rect,line width=0.800pt]
  \end{scope}
  \begin{scope}[scale=1.008,draw=black,line join=bevel,line cap=rect,line width=0.800pt]
  \end{scope}
  \begin{scope}[scale=1.008,draw=black,line join=bevel,line cap=rect,line width=0.800pt]
  \end{scope}
  \begin{scope}[cm={{1.00769,0.0,0.0,1.00769,(-21.0,-3.0)}},draw=cff0000,line join=round,line cap=round,line width=0.480pt]
    \path[draw] (41.6000,109.8000) -- (41.6000,109.8000) -- (41.7000,103.2000) -- (41.9000,103.5000) -- (42.0000,103.9000) -- (42.2000,104.4000) -- (42.3000,104.8000) -- (42.5000,105.2000) -- (42.6000,105.6000) -- (42.7000,106.0000) -- (42.9000,106.4000) -- (43.0000,106.7000) -- (43.2000,107.1000) -- (43.3000,107.4000) -- (43.5000,107.7000) -- (43.6000,107.9000) -- (43.7000,108.2000) -- (43.9000,108.5000) -- (44.0000,108.7000) -- (44.2000,108.9000) -- (44.3000,109.1000) -- (44.5000,109.4000) -- (44.6000,109.6000) -- (44.7000,109.7000) -- (44.9000,109.9000) -- (45.0000,110.1000) -- (45.2000,110.3000) -- (45.3000,110.5000) -- (45.5000,110.7000) -- (45.6000,110.9000) -- (45.7000,111.1000) -- (45.9000,111.3000) -- (46.0000,111.5000) -- (46.2000,111.7000) -- (46.3000,111.9000) -- (46.5000,112.1000) -- (46.6000,112.3000) -- (46.7000,112.5000) -- (46.9000,112.7000) -- (47.0000,112.9000) -- (47.2000,113.1000) -- (47.3000,113.3000) -- (47.5000,113.5000) -- (47.6000,113.7000) -- (47.7000,113.9000) -- (47.9000,114.1000) -- (48.0000,114.3000) -- (48.2000,114.5000) -- (48.3000,114.6000) -- (48.5000,114.8000) -- (48.6000,114.9000) -- (48.7000,115.1000) -- (48.9000,115.2000) -- (49.0000,115.3000) -- (49.2000,115.4000) -- (49.3000,115.6000) -- (49.5000,115.7000) -- (49.6000,115.8000) -- (49.7000,115.9000) -- (49.9000,116.0000) -- (50.0000,116.1000) -- (50.2000,116.2000) -- (50.3000,116.3000) -- (50.5000,116.4000) -- (50.6000,116.5000) -- (50.7000,116.6000) -- (50.9000,116.8000) -- (51.0000,116.9000) -- (51.2000,117.0000) -- (51.3000,117.2000) -- (51.5000,117.3000) -- (51.6000,117.5000) -- (51.7000,117.7000) -- (51.9000,117.8000) -- (52.0000,118.0000) -- (52.2000,118.2000) -- (52.3000,118.3000) -- (52.5000,118.5000) -- (52.6000,118.6000) -- (52.7000,118.8000) -- (52.9000,118.9000) -- (53.0000,119.0000) -- (53.2000,119.1000) -- (53.3000,119.2000) -- (53.5000,119.3000) -- (53.6000,119.4000) -- (53.7000,119.4000) -- (53.9000,119.4000) -- (54.0000,119.5000) -- (54.2000,119.5000) -- (54.3000,119.5000) -- (54.5000,119.4000) -- (54.6000,119.4000) -- (54.7000,119.4000) -- (54.9000,119.3000) -- (55.0000,119.3000) -- (55.2000,119.2000) -- (55.3000,119.2000) -- (55.5000,119.1000) -- (55.6000,119.1000) -- (55.7000,119.1000) -- (55.9000,119.0000) -- (56.0000,119.0000) -- (56.2000,119.0000) -- (56.3000,119.1000) -- (56.5000,119.1000) -- (56.6000,119.2000) -- (56.7000,119.2000) -- (56.9000,119.3000) -- (57.0000,119.4000) -- (57.2000,119.5000) -- (57.3000,119.6000) -- (57.5000,119.7000) -- (57.6000,119.8000) -- (57.7000,119.9000) -- (57.9000,120.1000) -- (58.0000,120.2000) -- (58.2000,120.3000) -- (58.3000,120.4000) -- (58.5000,120.5000) -- (58.6000,120.6000) -- (58.7000,120.7000) -- (58.9000,120.8000) -- (59.0000,120.8000) -- (59.2000,120.9000) -- (59.3000,121.0000) -- (59.5000,121.0000) -- (59.6000,121.0000) -- (59.7000,121.1000) -- (59.9000,121.1000) -- (60.0000,121.1000) -- (60.2000,121.1000) -- (60.3000,121.1000) -- (60.5000,121.1000) -- (60.6000,121.1000) -- (60.7000,121.0000) -- (60.9000,121.0000) -- (61.0000,121.0000) -- (61.2000,121.0000) -- (61.3000,120.9000) -- (61.5000,120.9000) -- (61.6000,120.9000) -- (61.7000,120.8000) -- (61.9000,120.8000) -- (62.0000,120.8000) -- (62.2000,120.8000) -- (62.3000,120.7000) -- (62.5000,120.7000) -- (62.6000,120.7000) -- (62.7000,120.7000) -- (62.9000,120.7000) -- (63.0000,120.7000) -- (63.2000,120.7000) -- (63.3000,120.7000) -- (63.5000,120.7000) -- (63.6000,120.8000) -- (63.7000,120.8000) -- (63.9000,120.8000) -- (64.0000,120.8000) -- (64.2000,120.8000) -- (64.3000,120.9000) -- (64.5000,120.9000) -- (64.6000,120.9000) -- (64.7000,121.0000) -- (64.9000,121.0000) -- (65.0000,121.0000) -- (65.2000,121.0000) -- (65.3000,121.1000) -- (65.5000,121.1000) -- (65.6000,121.1000) -- (65.7000,121.1000) -- (65.9000,121.2000) -- (66.0000,121.2000) -- (66.2000,121.2000) -- (66.3000,121.2000) -- (66.5000,121.2000) -- (66.6000,121.2000) -- (66.7000,121.2000) -- (66.9000,121.2000) -- (67.0000,121.2000) -- (67.2000,121.2000) -- (67.3000,121.2000) -- (67.5000,121.2000) -- (67.6000,121.2000) -- (67.7000,121.2000) -- (67.9000,121.2000) -- (68.0000,121.2000) -- (68.2000,121.1000) -- (68.3000,121.1000) -- (68.5000,121.1000) -- (68.6000,121.1000) -- (68.7000,121.0000) -- (68.9000,121.0000) -- (69.0000,121.0000) -- (69.2000,120.9000) -- (69.3000,120.9000) -- (69.5000,120.9000) -- (69.6000,120.8000) -- (69.7000,120.8000) -- (69.9000,120.8000) -- (70.0000,120.7000) -- (70.2000,120.7000) -- (70.3000,120.7000) -- (70.5000,120.7000) -- (70.6000,120.6000) -- (70.7000,120.6000) -- (70.9000,120.6000) -- (71.0000,120.6000) -- (71.2000,120.6000) -- (71.3000,120.6000) -- (71.5000,120.6000) -- (71.6000,120.6000) -- (71.7000,120.6000) -- (71.9000,120.6000) -- (72.0000,120.6000) -- (72.2000,120.6000) -- (72.3000,120.6000) -- (72.5000,120.6000) -- (72.6000,120.6000) -- (72.7000,120.7000) -- (72.9000,120.7000) -- (73.0000,120.7000) -- (73.2000,120.8000) -- (73.3000,120.8000) -- (73.5000,120.8000) -- (73.6000,120.9000) -- (73.7000,120.9000) -- (73.9000,120.9000) -- (74.0000,121.0000) -- (74.2000,121.0000) -- (74.3000,121.1000) -- (74.5000,121.1000) -- (74.6000,121.2000) -- (74.7000,121.2000) -- (74.9000,121.3000) -- (75.0000,121.3000) -- (75.2000,121.3000) -- (75.3000,121.4000) -- (75.5000,121.4000) -- (75.6000,121.5000) -- (75.7000,121.5000) -- (75.9000,121.6000) -- (76.0000,121.6000) -- (76.2000,121.7000) -- (76.3000,121.7000) -- (76.5000,121.7000) -- (76.6000,121.8000) -- (76.7000,121.8000) -- (76.9000,121.9000) -- (77.0000,121.9000) -- (77.2000,121.9000) -- (77.3000,122.0000) -- (77.5000,122.0000) -- (77.6000,122.0000) -- (77.7000,122.1000) -- (77.9000,122.1000) -- (78.0000,122.1000) -- (78.2000,122.1000) -- (78.3000,122.1000) -- (78.5000,122.2000) -- (78.6000,122.2000) -- (78.7000,122.2000) -- (78.9000,122.2000) -- (79.0000,122.2000) -- (79.2000,122.2000) -- (79.3000,122.2000) -- (79.5000,122.2000) -- (79.6000,122.2000) -- (79.7000,122.2000) -- (79.9000,122.2000) -- (80.0000,122.2000) -- (80.2000,122.2000) -- (80.3000,122.2000) -- (80.5000,122.2000) -- (80.6000,122.2000) -- (80.7000,122.2000) -- (80.9000,122.2000) -- (81.0000,122.2000) -- (81.2000,122.2000) -- (81.3000,122.2000) -- (81.5000,122.1000) -- (81.6000,122.1000) -- (81.7000,122.1000) -- (81.9000,122.1000) -- (82.0000,122.1000) -- (82.2000,122.0000) -- (82.3000,122.0000) -- (82.5000,122.0000) -- (82.6000,122.0000) -- (82.7000,121.9000) -- (82.9000,121.9000) -- (83.0000,121.9000) -- (83.2000,121.9000) -- (83.3000,121.8000) -- (83.5000,121.8000) -- (83.6000,121.8000) -- (83.7000,121.8000) -- (83.9000,121.7000) -- (84.0000,121.7000) -- (84.2000,121.7000) -- (84.3000,121.7000) -- (84.5000,121.6000) -- (84.6000,121.6000) -- (84.7000,121.6000) -- (84.9000,121.5000) -- (85.0000,121.5000) -- (85.2000,121.5000) -- (85.3000,121.5000) -- (85.5000,121.4000) -- (85.6000,121.4000) -- (85.7000,121.4000) -- (85.9000,121.3000) -- (86.0000,121.3000) -- (86.2000,121.3000) -- (86.3000,121.3000) -- (86.5000,121.2000) -- (86.6000,121.2000) -- (86.7000,121.2000) -- (86.9000,121.2000) -- (87.0000,121.1000) -- (87.2000,121.1000) -- (87.3000,121.1000) -- (87.5000,121.1000) -- (87.6000,121.1000) -- (87.7000,121.0000) -- (87.9000,121.0000) -- (88.0000,121.0000) -- (88.2000,121.0000) -- (88.3000,120.9000) -- (88.5000,120.9000) -- (88.6000,120.9000) -- (88.7000,120.9000) -- (88.9000,120.9000) -- (89.0000,120.8000) -- (89.2000,120.8000) -- (89.3000,120.8000) -- (89.5000,120.8000) -- (89.6000,120.8000) -- (89.7000,120.7000) -- (89.9000,120.7000) -- (90.0000,120.7000) -- (90.2000,120.7000) -- (90.3000,120.7000) -- (90.5000,120.6000) -- (90.6000,120.6000) -- (90.7000,120.6000) -- (90.9000,120.6000) -- (91.0000,120.6000) -- (91.2000,120.5000) -- (91.3000,120.5000) -- (91.5000,120.5000) -- (91.6000,120.5000) -- (91.7000,120.5000) -- (91.9000,120.4000) -- (92.0000,120.4000) -- (92.2000,120.4000) -- (92.3000,120.4000) -- (92.5000,120.4000) -- (92.6000,120.4000) -- (92.7000,120.3000) -- (92.9000,120.3000) -- (93.0000,120.3000) -- (93.2000,120.3000) -- (93.3000,120.3000) -- (93.4000,120.2000) -- (93.6000,120.2000) -- (93.7000,120.2000) -- (93.9000,120.2000) -- (94.0000,120.2000) -- (94.2000,120.1000) -- (94.3000,120.1000) -- (94.4000,120.1000) -- (94.6000,120.1000) -- (94.7000,120.0000) -- (94.9000,120.0000) -- (95.0000,120.0000) -- (95.2000,120.0000) -- (95.3000,120.0000) -- (95.4000,119.9000) -- (95.6000,119.9000) -- (95.7000,119.9000) -- (95.9000,119.9000) -- (96.0000,119.8000) -- (96.2000,119.8000) -- (96.3000,119.8000) -- (96.4000,119.8000) -- (96.6000,119.7000) -- (96.7000,119.7000) -- (96.9000,119.7000) -- (97.0000,119.7000) -- (97.2000,119.6000) -- (97.3000,119.6000) -- (97.4000,119.6000) -- (97.6000,119.5000) -- (97.7000,119.5000) -- (97.9000,119.5000) -- (98.0000,119.5000) -- (98.2000,119.4000) -- (98.3000,119.4000) -- (98.4000,119.4000) -- (98.6000,119.3000) -- (98.7000,119.3000) -- (98.9000,119.3000) -- (99.0000,119.2000) -- (99.2000,119.2000) -- (99.3000,119.2000) -- (99.4000,119.1000) -- (99.6000,119.1000) -- (99.7000,119.1000) -- (99.9000,119.1000) -- (100.0000,119.0000) -- (100.2000,119.0000) -- (100.3000,119.0000) -- (100.4000,118.9000) -- (100.6000,118.9000) -- (100.7000,118.9000) -- (100.9000,118.8000) -- (101.0000,118.8000) -- (101.2000,118.8000) -- (101.3000,118.7000) -- (101.4000,118.7000) -- (101.6000,118.7000) -- (101.7000,118.6000) -- (101.9000,118.6000) -- (102.0000,118.6000) -- (102.2000,118.6000) -- (102.3000,118.5000) -- (102.4000,118.5000) -- (102.6000,118.5000) -- (102.7000,118.4000) -- (102.9000,118.4000) -- (103.0000,118.4000) -- (103.2000,118.4000) -- (103.3000,118.3000) -- (103.4000,118.3000) -- (103.6000,118.3000) -- (103.7000,118.3000) -- (103.9000,118.2000) -- (104.0000,118.2000) -- (104.2000,118.2000) -- (104.3000,118.2000) -- (104.4000,118.1000) -- (104.6000,118.1000) -- (104.7000,118.1000) -- (104.9000,118.1000) -- (105.0000,118.0000) -- (105.2000,118.0000) -- (105.3000,118.0000) -- (105.4000,118.0000) -- (105.6000,118.0000) -- (105.7000,117.9000) -- (105.9000,117.9000) -- (106.0000,117.9000) -- (106.2000,117.9000) -- (106.3000,117.9000) -- (106.4000,117.8000) -- (106.6000,117.8000) -- (106.7000,117.8000) -- (106.9000,117.8000) -- (107.0000,117.8000) -- (107.2000,117.8000) -- (107.3000,117.7000) -- (107.4000,117.7000) -- (107.6000,117.7000) -- (107.7000,117.7000) -- (107.9000,117.7000) -- (108.0000,117.7000) -- (108.2000,117.7000) -- (108.3000,117.6000) -- (108.4000,117.6000) -- (108.6000,117.6000) -- (108.7000,117.6000) -- (108.9000,117.6000) -- (109.0000,117.6000) -- (109.2000,117.6000) -- (109.3000,117.6000) -- (109.4000,117.6000) -- (109.6000,117.6000) -- (109.7000,117.5000) -- (109.9000,117.5000) -- (110.0000,117.5000) -- (110.2000,117.5000) -- (110.3000,117.5000) -- (110.4000,117.5000) -- (110.6000,117.5000) -- (110.7000,117.5000) -- (110.9000,117.5000) -- (111.0000,117.5000) -- (111.2000,117.5000) -- (111.3000,117.5000) -- (111.4000,117.5000) -- (111.6000,117.5000) -- (111.7000,117.5000) -- (111.9000,117.5000) -- (112.0000,117.5000) -- (112.2000,117.5000) -- (112.3000,117.4000) -- (112.4000,117.4000) -- (112.6000,117.4000) -- (112.7000,117.4000) -- (112.9000,117.4000) -- (113.0000,117.4000) -- (113.2000,117.4000) -- (113.3000,117.4000) -- (113.4000,117.4000) -- (113.6000,117.4000) -- (113.7000,117.4000) -- (113.9000,117.4000) -- (114.0000,117.4000) -- (114.2000,117.4000) -- (114.3000,117.4000) -- (114.4000,117.4000) -- (114.6000,117.4000) -- (114.7000,117.4000) -- (114.9000,117.4000) -- (115.0000,117.4000) -- (115.2000,117.4000) -- (115.3000,117.4000) -- (115.4000,117.4000) -- (115.6000,117.4000) -- (115.7000,117.4000) -- (115.9000,117.4000) -- (116.0000,117.4000) -- (116.2000,117.4000) -- (116.3000,117.4000) -- (116.4000,117.4000) -- (116.6000,117.4000) -- (116.7000,117.4000) -- (116.9000,117.5000) -- (117.0000,117.5000) -- (117.2000,117.5000) -- (117.3000,117.5000) -- (117.4000,117.5000) -- (117.6000,117.5000) -- (117.7000,117.5000) -- (117.9000,117.5000) -- (118.0000,117.5000) -- (118.2000,117.5000) -- (118.3000,117.5000) -- (118.4000,117.5000) -- (118.6000,117.5000) -- (118.7000,117.5000) -- (118.9000,117.5000) -- (119.0000,117.5000) -- (119.2000,117.5000) -- (119.3000,117.5000) -- (119.4000,117.5000) -- (119.6000,117.5000) -- (119.7000,117.5000) -- (119.9000,117.5000) -- (120.0000,117.5000) -- (120.2000,117.5000) -- (120.3000,117.5000) -- (120.4000,117.5000) -- (120.6000,117.5000) -- (120.7000,117.5000) -- (120.9000,117.5000) -- (121.0000,117.5000) -- (121.2000,117.5000) -- (121.3000,117.5000) -- (121.4000,117.5000) -- (121.6000,117.5000) -- (121.7000,117.5000) -- (121.9000,117.5000) -- (122.0000,117.5000) -- (122.2000,117.5000) -- (122.3000,117.5000) -- (122.4000,117.5000) -- (122.6000,117.5000) -- (122.7000,117.5000) -- (122.9000,117.5000) -- (123.0000,117.5000) -- (123.2000,117.5000) -- (123.3000,117.5000) -- (123.4000,117.5000) -- (123.6000,117.5000) -- (123.7000,117.5000) -- (123.9000,117.6000) -- (124.0000,117.6000) -- (124.2000,117.6000) -- (124.3000,117.6000) -- (124.4000,117.6000) -- (124.6000,117.6000) -- (124.7000,117.6000) -- (124.9000,117.6000) -- (125.0000,117.6000) -- (125.2000,117.6000) -- (125.3000,117.6000) -- (125.4000,117.6000) -- (125.6000,117.6000) -- (125.7000,117.6000) -- (125.9000,117.6000) -- (126.0000,117.6000) -- (126.2000,117.6000) -- (126.3000,117.6000) -- (126.4000,117.6000) -- (126.6000,117.6000) -- (126.7000,117.6000) -- (126.9000,117.6000) -- (127.0000,117.6000) -- (127.2000,117.6000) -- (127.3000,117.6000);



  \end{scope}
  \begin{scope}[scale=1.008,draw=black,line join=bevel,line cap=rect,line width=0.800pt]
  \end{scope}
  \begin{scope}[scale=1.008,draw=black,line join=bevel,line cap=rect,line width=0.800pt]
  \end{scope}
  \begin{scope}[cm={{1.00769,0.0,0.0,1.00769,(-21.0,-3.0)}},draw=black,line join=round,line cap=round,line width=0.480pt]
    \path[draw] (41.5000,88.5000) -- (41.5000,164.5000) -- (127.5000,164.5000) -- (127.5000,88.5000) -- (41.5000,88.5000);



  \end{scope}
  \begin{scope}[scale=1.008,draw=black,line join=bevel,line cap=rect,line width=0.800pt]
  \end{scope}
  \begin{scope}[scale=1.008,draw=black,line join=bevel,line cap=rect,line width=0.800pt]
  \end{scope}
  \begin{scope}[cm={{1.00769,0.0,0.0,1.00769,(-21.0,-7.5)}},fill=cffffff]
    \path[fill,rounded corners=0.0000cm] (81.0000,50.0000) rectangle (121.0000,78.0000);



  \end{scope}
  \begin{scope}[scale=1.008,draw=black,line join=bevel,line cap=rect,line width=0.800pt]
  \end{scope}
  \begin{scope}[scale=1.008,draw=black,line join=bevel,line cap=rect,line width=0.800pt]
  \end{scope}
  \begin{scope}[cm={{1.00769,0.0,0.0,1.00769,(-21.0,-7.5)}},draw=ca0a0a4,dash pattern=on 0.40pt off 0.80pt,line join=round,line cap=round,line width=0.400pt]
    \path[draw] (81.5000,55.5000) -- (121.5000,55.5000);



  \end{scope}
  \begin{scope}[cm={{1.00769,0.0,0.0,1.00769,(-21.0,-7.5)}},draw=black,line join=round,line cap=round,line width=0.480pt]
    \path[draw] (81.5000,55.5000) -- (82.6777,55.5000);



    \path[draw] (121.5000,55.5000) -- (120.1220,55.5000);



  \end{scope}
  \begin{scope}[scale=1.008,draw=black,line join=bevel,line cap=rect,line width=0.800pt]
  \end{scope}
  \begin{scope}[cm={{1.00769,0.0,0.0,1.00769,(64.4923,60.4615)}},draw=black,line join=bevel,line cap=rect,line width=0.800pt]
  \end{scope}
  \begin{scope}[cm={{1.00769,0.0,0.0,1.00769,(64.4923,60.4615)}},draw=black,line join=bevel,line cap=rect,line width=0.800pt]
  \end{scope}
  \begin{scope}[cm={{1.00769,0.0,0.0,1.00769,(64.4923,60.4615)}},draw=black,line join=bevel,line cap=rect,line width=0.800pt]
  \end{scope}
  \begin{scope}[cm={{1.00769,0.0,0.0,1.00769,(64.4923,60.4615)}},draw=black,line join=bevel,line cap=rect,line width=0.800pt]
  \end{scope}
  \begin{scope}[cm={{1.00769,0.0,0.0,1.00769,(64.4923,60.4615)}},draw=black,line join=bevel,line cap=rect,line width=0.800pt]
  \end{scope}
  \begin{scope}[cm={{1.00769,0.0,0.0,1.00769,(33.4923,52.9615)}},draw=black,line join=bevel,line cap=rect,line width=0.800pt]
    \path[fill=black] (0.0000,0.0000) node[above right] (text672) {\scriptsize $T$= 46};



  \end{scope}
  \begin{scope}[cm={{1.00769,0.0,0.0,1.00769,(64.4923,60.4615)}},draw=black,line join=bevel,line cap=rect,line width=0.800pt]
  \end{scope}
  \begin{scope}[scale=1.008,draw=black,line join=bevel,line cap=rect,line width=0.800pt]
  \end{scope}
  \begin{scope}[cm={{1.00769,0.0,0.0,1.00769,(-21.0,-7.5)}},draw=ca0a0a4,dash pattern=on 0.40pt off 0.80pt,line join=round,line cap=round,line width=0.400pt]
    \path[draw] (97.5000,78.5000) -- (97.5000,50.5000);



  \end{scope}
  \begin{scope}[cm={{1.00769,0.0,0.0,1.00769,(-21.0,-7.5)}},draw=black,line join=round,line cap=round,line width=0.480pt]
    \path[draw] (97.5000,50.5000) -- (97.5000,50.5000) -- (97.5000,51.6564);



  \end{scope}
  \begin{scope}[scale=1.008,draw=black,line join=bevel,line cap=rect,line width=0.800pt]
  \end{scope}
  \begin{scope}[cm={{1.00769,0.0,0.0,1.00769,(93.2115,46.3538)}},draw=black,line join=bevel,line cap=rect,line width=0.800pt]
  \end{scope}
  \begin{scope}[cm={{1.00769,0.0,0.0,1.00769,(93.2115,46.3538)}},draw=black,line join=bevel,line cap=rect,line width=0.800pt]
  \end{scope}
  \begin{scope}[cm={{1.00769,0.0,0.0,1.00769,(93.2115,46.3538)}},draw=black,line join=bevel,line cap=rect,line width=0.800pt]
  \end{scope}
  \begin{scope}[cm={{1.00769,0.0,0.0,1.00769,(93.2115,46.3538)}},draw=black,line join=bevel,line cap=rect,line width=0.800pt]
  \end{scope}
  \begin{scope}[cm={{1.00769,0.0,0.0,1.00769,(93.2115,46.3538)}},draw=black,line join=bevel,line cap=rect,line width=0.800pt]
  \end{scope}
  \begin{scope}[cm={{1.00769,0.0,0.0,1.00769,(72.2115,40.8538)}},draw=black,line join=bevel,line cap=rect,line width=0.800pt]
    \path[fill=black] (0.0000,0.0000) node[above right] (text700) {\scriptsize 83};



  \end{scope}
  \begin{scope}[cm={{1.00769,0.0,0.0,1.00769,(93.2115,46.3538)}},draw=black,line join=bevel,line cap=rect,line width=0.800pt]
  \end{scope}
  \begin{scope}[scale=1.008,draw=black,line join=bevel,line cap=rect,line width=0.800pt]
  \end{scope}
  \begin{scope}[cm={{1.00769,0.0,0.0,1.00769,(-21.0,-7.5)}},draw=black,line join=round,line cap=round,line width=0.480pt]
    \path[draw] (81.5000,50.5000) -- (81.5000,78.5000) -- (121.5000,78.5000) -- (121.5000,50.5000) -- (81.5000,50.5000);



  \end{scope}
  \begin{scope}[scale=1.008,draw=black,line join=bevel,line cap=rect,line width=0.800pt]
  \end{scope}
  \begin{scope}[scale=1.008,draw=black,line join=bevel,line cap=rect,line width=0.800pt]
  \end{scope}
  \begin{scope}[scale=1.008,draw=black,line join=bevel,line cap=rect,line width=0.800pt]
  \end{scope}
  \begin{scope}[scale=1.008,draw=black,line join=bevel,line cap=rect,line width=0.800pt]
  \end{scope}
  \begin{scope}[cm={{1.00769,0.0,0.0,1.00769,(-21.0,-7.5)}},draw=black,line join=round,line cap=round,line width=0.480pt]
    \path[draw] (81.2000,77.5000) -- (81.2000,77.5000) -- (81.2000,77.5000) -- (81.2000,77.5000) -- (81.2000,77.5000) -- (81.2000,77.5000) -- (81.2000,77.5000) -- (81.2000,77.5000) -- (81.2000,77.5000) -- (81.2000,77.5000) -- (81.2000,77.5000) -- (81.2000,77.5000) -- (81.2000,77.5000) -- (81.2000,77.5000) -- (81.2000,77.5000) -- (81.2000,77.5000) -- (81.2000,77.4000) -- (81.2000,77.4000) -- (81.2000,77.4000) -- (81.2000,77.4000) -- (81.2000,77.4000) -- (81.2000,77.4000) -- (81.2000,77.4000) -- (81.2000,77.4000) -- (81.2000,77.4000) -- (81.2000,77.4000) -- (81.3000,77.4000) -- (81.3000,77.4000) -- (81.3000,77.4000) -- (81.3000,77.4000) -- (81.3000,77.4000) -- (81.3000,77.4000) -- (81.3000,77.4000) -- (81.3000,77.4000) -- (81.3000,77.4000) -- (81.3000,77.4000) -- (81.3000,77.4000) -- (81.3000,77.3000) -- (81.3000,77.3000) -- (81.3000,77.3000) -- (81.3000,77.3000) -- (81.3000,77.3000) -- (81.3000,77.3000) -- (81.3000,77.3000) -- (81.3000,77.3000) -- (81.3000,77.3000) -- (81.3000,77.3000) -- (81.3000,77.3000) -- (81.3000,77.3000) -- (81.3000,77.3000) -- (81.3000,77.3000) -- (81.3000,77.3000) -- (81.3000,77.3000) -- (81.3000,77.3000) -- (81.3000,77.3000) -- (81.3000,77.3000) -- (81.3000,77.3000) -- (81.3000,77.2000) -- (81.3000,77.2000) -- (81.3000,77.2000) -- (81.3000,77.2000) -- (81.3000,77.2000) -- (81.3000,77.2000) -- (81.3000,77.2000) -- (81.3000,77.2000) -- (81.3000,77.2000) -- (81.3000,77.2000) -- (81.3000,77.2000) -- (81.3000,77.2000) -- (81.3000,77.2000) -- (81.3000,77.2000) -- (81.3000,77.2000) -- (81.3000,77.2000) -- (81.3000,77.2000) -- (81.3000,77.2000) -- (81.3000,77.2000) -- (81.4000,77.2000) -- (81.4000,77.2000) -- (81.4000,77.1000) -- (81.4000,77.1000) -- (81.4000,77.1000) -- (81.4000,77.1000) -- (81.4000,77.1000) -- (81.4000,77.1000) -- (81.4000,77.1000) -- (81.4000,77.1000) -- (81.4000,77.1000) -- (81.4000,77.1000) -- (81.4000,77.1000) -- (81.4000,77.1000) -- (81.4000,77.1000) -- (81.4000,77.1000) -- (81.4000,77.1000) -- (81.4000,77.1000) -- (81.4000,77.1000) -- (81.4000,77.1000) -- (81.4000,77.1000) -- (81.4000,77.1000) -- (81.4000,77.1000) -- (81.4000,77.0000) -- (81.4000,77.0000) -- (81.4000,77.0000) -- (81.4000,77.0000) -- (81.4000,77.0000) -- (81.4000,77.0000) -- (81.4000,77.0000) -- (81.4000,77.0000) -- (81.4000,77.0000) -- (81.4000,77.0000) -- (81.4000,77.0000) -- (81.4000,77.0000) -- (81.4000,77.0000) -- (81.4000,77.0000) -- (81.4000,77.0000) -- (81.4000,77.0000) -- (81.4000,77.0000) -- (81.4000,77.0000) -- (81.4000,77.0000) -- (81.4000,77.0000) -- (81.4000,77.0000) -- (81.4000,76.9000) -- (81.4000,76.9000) -- (81.4000,76.9000) -- (81.4000,76.9000) -- (81.4000,76.9000) -- (81.5000,76.9000) -- (81.5000,76.9000) -- (81.5000,76.9000) -- (81.5000,76.9000) -- (81.5000,76.9000) -- (81.5000,76.9000) -- (81.5000,76.9000) -- (81.5000,76.9000) -- (81.5000,76.9000) -- (81.5000,76.9000) -- (81.5000,76.9000) -- (81.5000,76.9000) -- (81.5000,76.9000) -- (81.5000,76.9000) -- (81.5000,76.9000) -- (81.5000,76.9000) -- (81.5000,76.8000) -- (81.5000,76.8000) -- (81.5000,76.8000) -- (81.5000,76.8000) -- (81.5000,76.8000) -- (81.5000,76.8000) -- (81.5000,76.8000) -- (81.5000,76.8000) -- (81.5000,76.8000) -- (81.5000,76.8000) -- (81.5000,76.8000) -- (81.5000,76.8000) -- (81.5000,76.8000) -- (81.5000,76.8000) -- (81.5000,76.8000) -- (81.5000,76.8000) -- (81.5000,76.8000) -- (81.5000,76.8000) -- (81.5000,76.8000) -- (81.5000,76.8000) -- (81.5000,76.7000) -- (81.5000,76.7000) -- (81.5000,76.7000) -- (81.5000,76.7000) -- (81.5000,76.7000) -- (81.5000,76.7000) -- (81.5000,76.7000) -- (81.5000,76.7000) -- (81.5000,76.7000) -- (81.5000,76.7000) -- (81.5000,76.7000) -- (81.5000,76.7000) -- (81.5000,76.7000) -- (81.5000,76.7000) -- (81.6000,76.7000) -- (81.6000,76.7000) -- (81.6000,76.7000) -- (81.6000,76.7000) -- (81.6000,76.7000) -- (81.6000,76.7000) -- (81.6000,76.7000) -- (81.6000,76.6000) -- (81.6000,76.6000) -- (81.6000,76.6000) -- (81.6000,76.6000) -- (81.6000,76.6000) -- (81.6000,76.6000) -- (81.6000,76.6000) -- (81.6000,76.6000) -- (81.6000,76.6000) -- (81.6000,76.6000) -- (81.6000,76.6000) -- (81.6000,76.6000) -- (81.6000,76.6000) -- (81.6000,76.6000) -- (81.6000,76.6000) -- (81.6000,76.6000) -- (81.6000,76.6000) -- (81.6000,76.6000) -- (81.6000,76.6000) -- (81.6000,76.6000) -- (81.6000,76.6000) -- (81.6000,76.5000) -- (81.6000,76.5000) -- (81.6000,76.5000) -- (81.6000,76.5000) -- (81.6000,76.5000) -- (81.6000,76.5000) -- (81.6000,76.5000) -- (81.6000,76.5000) -- (81.6000,76.5000) -- (81.6000,76.5000) -- (81.6000,76.5000) -- (81.6000,76.5000) -- (81.6000,76.5000) -- (81.6000,76.5000) -- (81.6000,76.5000) -- (81.6000,76.5000) -- (81.6000,76.5000) -- (81.6000,76.5000) -- (81.6000,76.5000) -- (81.6000,76.5000) -- (81.6000,76.5000) -- (81.7000,76.4000) -- (81.7000,76.4000) -- (81.7000,76.4000) -- (81.7000,76.4000) -- (81.7000,76.4000) -- (81.7000,76.4000) -- (81.7000,76.4000) -- (81.7000,76.4000) -- (81.7000,76.4000) -- (81.7000,76.4000) -- (81.7000,76.4000) -- (81.7000,76.4000) -- (81.7000,76.4000) -- (81.7000,76.4000) -- (81.7000,76.4000) -- (81.7000,76.4000) -- (81.7000,76.4000) -- (81.7000,76.4000) -- (81.7000,76.4000) -- (81.7000,76.4000) -- (81.7000,76.4000) -- (81.7000,76.3000) -- (81.7000,76.3000) -- (81.7000,76.3000) -- (81.7000,76.3000) -- (81.7000,76.3000) -- (81.7000,76.3000) -- (81.7000,76.3000) -- (81.7000,76.3000) -- (81.7000,76.3000) -- (81.7000,76.3000) -- (81.7000,76.3000) -- (81.7000,76.3000) -- (81.7000,76.3000) -- (81.7000,76.3000) -- (81.7000,76.3000) -- (81.7000,76.3000) -- (81.7000,76.3000) -- (81.7000,76.3000) -- (81.7000,76.3000) -- (81.7000,76.3000) -- (81.7000,76.2000) -- (81.7000,76.2000) -- (81.7000,76.2000) -- (81.7000,76.2000) -- (81.7000,76.2000) -- (81.7000,76.2000) -- (81.7000,76.2000) -- (81.7000,76.2000) -- (81.7000,76.2000) -- (81.8000,76.2000) -- (81.8000,76.2000) -- (81.8000,76.2000) -- (81.8000,76.2000) -- (81.8000,76.2000) -- (81.8000,76.2000) -- (81.8000,76.2000) -- (81.8000,76.2000) -- (81.8000,76.2000) -- (81.8000,76.2000) -- (81.8000,76.2000) -- (81.8000,76.2000) -- (81.8000,76.1000) -- (81.8000,76.1000) -- (81.8000,76.1000) -- (81.8000,76.1000) -- (81.8000,76.1000) -- (81.8000,76.1000) -- (81.8000,76.1000) -- (81.8000,76.1000) -- (81.8000,76.1000) -- (81.8000,76.1000) -- (81.8000,76.1000) -- (81.8000,76.1000) -- (81.8000,76.1000) -- (81.8000,76.1000) -- (81.8000,76.1000) -- (81.8000,76.1000) -- (81.8000,76.1000) -- (81.8000,76.1000) -- (81.8000,76.1000) -- (81.8000,76.1000) -- (81.8000,76.1000) -- (81.8000,76.0000) -- (81.8000,76.0000) -- (81.8000,76.0000) -- (81.8000,76.0000) -- (81.8000,76.0000) -- (81.8000,76.0000) -- (81.8000,76.0000) -- (81.8000,76.0000) -- (81.8000,76.0000) -- (81.8000,76.0000) -- (81.8000,76.0000) -- (81.8000,76.0000) -- (81.8000,76.0000) -- (81.8000,76.0000) -- (81.8000,76.0000) -- (81.8000,76.0000) -- (81.9000,76.0000) -- (81.9000,76.0000) -- (81.9000,76.0000) -- (81.9000,76.0000) -- (81.9000,76.0000) -- (81.9000,75.9000) -- (81.9000,75.9000) -- (81.9000,75.9000) -- (81.9000,75.9000) -- (81.9000,75.9000) -- (81.9000,75.9000) -- (81.9000,75.9000) -- (81.9000,75.9000) -- (81.9000,75.9000) -- (81.9000,75.9000) -- (81.9000,75.9000) -- (81.9000,75.9000) -- (81.9000,75.9000) -- (81.9000,75.9000) -- (81.9000,75.9000) -- (81.9000,75.9000) -- (81.9000,75.9000) -- (81.9000,75.9000) -- (81.9000,75.9000) -- (81.9000,75.9000) -- (81.9000,75.8000) -- (81.9000,75.8000) -- (81.9000,75.8000) -- (81.9000,75.8000) -- (81.9000,75.8000) -- (81.9000,75.8000) -- (81.9000,75.8000) -- (81.9000,75.8000) -- (81.9000,75.8000) -- (81.9000,75.8000) -- (81.9000,75.8000) -- (81.9000,75.8000) -- (81.9000,75.8000) -- (81.9000,75.8000) -- (81.9000,75.8000) -- (81.9000,75.8000) -- (81.9000,75.8000) -- (81.9000,75.8000) -- (81.9000,75.8000) -- (81.9000,75.8000) -- (81.9000,75.8000) -- (81.9000,75.7000) -- (81.9000,75.7000) -- (81.9000,75.7000) -- (81.9000,75.7000) -- (82.0000,75.7000) -- (82.0000,75.7000) -- (82.0000,75.7000) -- (82.0000,75.7000) -- (82.0000,75.7000) -- (82.0000,75.7000) -- (82.0000,75.7000) -- (82.0000,75.7000) -- (82.0000,75.7000) -- (82.0000,75.7000) -- (82.0000,75.7000) -- (82.0000,75.7000) -- (82.0000,75.7000) -- (82.0000,75.7000) -- (82.0000,75.7000) -- (82.0000,75.7000) -- (82.0000,75.7000) -- (82.0000,75.6000) -- (82.0000,75.6000) -- (82.0000,75.6000) -- (82.0000,75.6000) -- (82.0000,75.6000) -- (82.0000,75.6000) -- (82.0000,75.6000) -- (82.0000,75.6000) -- (82.0000,75.6000) -- (82.0000,75.6000) -- (82.0000,75.6000) -- (82.0000,75.6000) -- (82.0000,75.6000) -- (82.0000,75.6000) -- (82.0000,75.6000) -- (82.0000,75.6000) -- (82.0000,75.6000) -- (82.0000,75.6000) -- (82.0000,75.6000) -- (82.0000,75.6000) -- (82.0000,75.6000) -- (82.0000,75.5000) -- (82.0000,75.5000) -- (82.0000,75.5000) -- (82.0000,75.5000) -- (82.0000,75.5000) -- (82.0000,75.5000) -- (82.0000,75.5000) -- (82.0000,75.5000) -- (82.0000,75.5000) -- (82.0000,75.5000) -- (82.0000,75.5000) -- (82.1000,75.5000) -- (82.1000,75.5000) -- (82.1000,75.5000) -- (82.1000,75.5000) -- (82.1000,75.5000) -- (82.1000,75.5000) -- (82.1000,75.5000) -- (82.1000,75.5000) -- (82.1000,75.5000) -- (82.1000,75.5000) -- (82.1000,75.4000) -- (82.1000,75.4000) -- (82.1000,75.4000) -- (82.1000,75.4000) -- (82.1000,75.4000) -- (82.1000,75.4000) -- (82.1000,75.4000) -- (82.1000,75.4000) -- (82.1000,75.4000) -- (82.1000,75.4000) -- (82.1000,75.4000) -- (82.1000,75.4000) -- (82.1000,75.4000) -- (82.1000,75.4000) -- (82.1000,75.4000) -- (82.1000,75.4000) -- (82.1000,75.4000) -- (82.1000,75.4000) -- (82.1000,75.4000) -- (82.1000,75.4000) -- (82.1000,75.3000) -- (82.1000,75.3000) -- (82.1000,75.3000) -- (82.1000,75.3000) -- (82.1000,75.3000) -- (82.1000,75.3000) -- (82.1000,75.3000) -- (82.1000,75.3000) -- (82.1000,75.3000) -- (82.1000,75.3000) -- (82.1000,75.3000) -- (82.1000,75.3000) -- (82.1000,75.3000) -- (82.1000,75.3000) -- (82.1000,75.3000) -- (82.1000,75.3000) -- (82.1000,75.3000) -- (82.1000,75.3000) -- (82.1000,75.3000) -- (82.1000,75.3000) -- (82.2000,75.3000) -- (82.2000,75.2000) -- (82.2000,75.2000) -- (82.2000,75.2000) -- (82.2000,75.2000) -- (82.2000,75.2000) -- (82.2000,75.2000) -- (82.2000,75.2000) -- (82.2000,75.2000) -- (82.2000,75.2000) -- (82.2000,75.2000) -- (82.2000,75.2000) -- (82.2000,75.2000) -- (82.2000,75.2000) -- (82.2000,75.2000) -- (82.2000,75.2000) -- (82.2000,75.2000) -- (82.2000,75.2000) -- (82.2000,75.2000) -- (82.2000,75.2000) -- (82.2000,75.2000) -- (82.2000,75.2000) -- (82.2000,75.1000) -- (82.2000,75.1000) -- (82.2000,75.1000) -- (82.2000,75.1000) -- (82.2000,75.1000) -- (82.2000,75.1000) -- (82.2000,75.1000) -- (82.2000,75.1000) -- (82.2000,75.1000) -- (82.2000,75.1000) -- (82.2000,75.1000) -- (82.2000,75.1000) -- (82.2000,75.1000) -- (82.2000,75.1000) -- (82.2000,75.1000) -- (82.2000,75.1000) -- (82.2000,75.1000) -- (82.2000,75.1000) -- (82.2000,75.1000) -- (82.2000,75.1000) -- (82.2000,75.1000) -- (82.2000,75.0000) -- (82.2000,75.0000) -- (82.2000,75.0000) -- (82.2000,75.0000) -- (82.2000,75.0000) -- (82.2000,75.0000) -- (82.3000,75.0000) -- (82.3000,75.0000) -- (82.3000,75.0000) -- (82.3000,75.0000) -- (82.3000,75.0000) -- (82.3000,75.0000) -- (82.3000,75.0000) -- (82.3000,75.0000) -- (82.3000,75.0000) -- (82.3000,75.0000) -- (82.3000,75.0000) -- (82.3000,75.0000) -- (82.3000,75.0000) -- (82.3000,75.0000) -- (82.3000,75.0000) -- (82.3000,74.9000) -- (82.3000,74.9000) -- (82.3000,74.9000) -- (82.3000,74.9000) -- (82.3000,74.9000) -- (82.3000,74.9000) -- (82.3000,74.9000) -- (82.3000,74.9000) -- (82.3000,74.9000) -- (82.3000,74.9000) -- (82.3000,74.9000) -- (82.3000,74.9000) -- (82.3000,74.9000) -- (82.3000,74.9000) -- (82.3000,74.9000) -- (82.3000,74.9000) -- (82.3000,74.9000) -- (82.3000,74.9000) -- (82.3000,74.9000) -- (82.3000,74.9000) -- (82.3000,74.8000) -- (82.3000,74.8000) -- (82.3000,74.8000) -- (82.3000,74.8000) -- (82.3000,74.8000) -- (82.3000,74.8000) -- (82.3000,74.8000) -- (82.3000,74.8000) -- (82.3000,74.8000) -- (82.3000,74.8000) -- (82.3000,74.8000) -- (82.3000,74.8000) -- (82.3000,74.8000) -- (82.3000,74.8000) -- (82.3000,74.8000) -- (82.4000,74.8000) -- (82.4000,74.8000) -- (82.4000,74.8000) -- (82.4000,74.8000) -- (82.4000,74.8000) -- (82.4000,74.8000) -- (82.4000,74.7000) -- (82.4000,74.7000) -- (82.4000,74.7000) -- (82.4000,74.7000) -- (82.4000,74.7000) -- (82.4000,74.7000) -- (82.4000,74.7000) -- (82.4000,74.7000) -- (82.4000,74.7000) -- (82.4000,74.7000) -- (82.4000,74.7000) -- (82.4000,74.7000) -- (82.4000,74.7000) -- (82.4000,74.7000) -- (82.4000,74.7000) -- (82.4000,74.7000) -- (82.4000,74.7000) -- (82.4000,74.7000) -- (82.4000,74.7000) -- (82.4000,74.7000) -- (82.4000,74.7000) -- (82.4000,74.6000) -- (82.4000,74.6000) -- (82.4000,74.6000) -- (82.4000,74.6000) -- (82.4000,74.6000) -- (82.4000,74.6000) -- (82.4000,74.6000) -- (82.4000,74.6000) -- (82.4000,74.6000) -- (82.4000,74.6000) -- (82.4000,74.6000) -- (82.4000,74.6000) -- (82.4000,74.6000) -- (82.4000,74.6000) -- (82.4000,74.6000) -- (82.4000,74.6000) -- (82.4000,74.6000) -- (82.4000,74.6000) -- (82.4000,74.6000) -- (82.4000,74.6000) -- (82.4000,74.6000) -- (82.4000,74.5000) -- (82.5000,74.5000) -- (82.5000,74.5000) -- (82.5000,74.5000) -- (82.5000,74.5000) -- (82.5000,74.5000) -- (82.5000,74.5000) -- (82.5000,74.5000) -- (82.5000,74.5000) -- (82.5000,74.5000) -- (82.5000,74.5000) -- (82.5000,74.5000) -- (82.5000,74.5000) -- (82.5000,74.5000) -- (82.5000,74.5000) -- (82.5000,74.5000) -- (82.5000,74.5000) -- (82.5000,74.5000) -- (82.5000,74.5000) -- (82.5000,74.5000) -- (82.5000,74.4000) -- (82.5000,74.4000) -- (82.5000,74.4000) -- (82.5000,74.4000) -- (82.5000,74.4000) -- (82.5000,74.4000) -- (82.5000,74.4000) -- (82.5000,74.4000) -- (82.5000,74.4000) -- (82.5000,74.4000) -- (82.5000,74.4000) -- (82.5000,74.4000) -- (82.5000,74.4000) -- (82.5000,74.4000) -- (82.5000,74.4000) -- (82.5000,74.4000) -- (82.5000,74.4000) -- (82.5000,74.4000) -- (82.5000,74.4000) -- (82.5000,74.4000) -- (82.5000,74.4000) -- (82.5000,74.3000) -- (82.5000,74.3000) -- (82.5000,74.3000) -- (82.5000,74.3000) -- (82.5000,74.3000) -- (82.5000,74.3000) -- (82.5000,74.3000) -- (82.5000,74.3000) -- (82.5000,74.3000) -- (82.5000,74.3000) -- (82.6000,74.3000) -- (82.6000,74.3000) -- (82.6000,74.3000) -- (82.6000,74.3000) -- (82.6000,74.3000) -- (82.6000,74.3000) -- (82.6000,74.3000) -- (82.6000,74.3000) -- (82.6000,74.3000) -- (82.6000,74.3000) -- (82.6000,74.3000) -- (82.6000,74.2000) -- (82.6000,74.2000) -- (82.6000,74.2000) -- (82.6000,74.2000) -- (82.6000,74.2000) -- (82.6000,74.2000) -- (82.6000,74.2000) -- (82.6000,74.2000) -- (82.6000,74.2000) -- (82.6000,74.2000) -- (82.6000,74.2000) -- (82.6000,74.2000) -- (82.6000,74.2000) -- (82.6000,74.2000) -- (82.6000,74.2000) -- (82.6000,74.2000) -- (82.6000,74.2000) -- (82.6000,74.2000) -- (82.6000,74.2000) -- (82.6000,74.2000) -- (82.6000,74.2000) -- (82.6000,74.1000) -- (82.6000,74.1000) -- (82.6000,74.1000) -- (82.6000,74.1000) -- (82.6000,74.1000) -- (82.6000,74.1000) -- (82.6000,74.1000) -- (82.6000,74.1000) -- (82.6000,74.1000) -- (82.6000,74.1000) -- (82.6000,74.1000) -- (82.6000,74.1000) -- (82.6000,74.1000) -- (82.6000,74.1000) -- (82.6000,74.1000) -- (82.6000,74.1000) -- (82.6000,74.1000) -- (82.7000,74.1000) -- (82.7000,74.1000) -- (82.7000,74.1000) -- (82.7000,74.1000) -- (82.7000,74.0000) -- (82.7000,74.0000) -- (82.7000,74.0000) -- (82.7000,74.0000) -- (82.7000,74.0000) -- (82.7000,74.0000) -- (82.7000,74.0000) -- (82.7000,74.0000) -- (82.7000,74.0000) -- (82.7000,74.0000) -- (82.7000,74.0000) -- (82.7000,74.0000) -- (82.7000,74.0000) -- (82.7000,74.0000) -- (82.7000,74.0000) -- (82.7000,74.0000) -- (82.7000,74.0000) -- (82.7000,74.0000) -- (82.7000,74.0000) -- (82.7000,74.0000) -- (82.7000,73.9000) -- (82.7000,73.9000) -- (82.7000,73.9000) -- (82.7000,73.9000) -- (82.7000,73.9000) -- (82.7000,73.9000) -- (82.7000,73.9000) -- (82.7000,73.9000) -- (82.7000,73.9000) -- (82.7000,73.9000) -- (82.7000,73.9000) -- (82.7000,73.9000) -- (82.7000,73.9000) -- (82.7000,73.9000) -- (82.7000,73.9000) -- (82.7000,73.9000) -- (82.7000,73.9000) -- (82.7000,73.9000) -- (82.7000,73.9000) -- (82.7000,73.9000) -- (82.7000,73.9000) -- (82.7000,73.8000) -- (82.7000,73.8000) -- (82.7000,73.8000) -- (82.7000,73.8000) -- (82.7000,73.8000) -- (82.8000,73.8000) -- (82.8000,73.8000) -- (82.8000,73.8000) -- (82.8000,73.8000) -- (82.8000,73.8000) -- (82.8000,73.8000) -- (82.8000,73.8000) -- (82.8000,73.8000) -- (82.8000,73.8000) -- (82.8000,73.8000) -- (82.8000,73.8000) -- (82.8000,73.8000) -- (82.8000,73.8000) -- (82.8000,73.8000) -- (82.8000,73.8000) -- (82.8000,73.8000) -- (82.8000,73.7000) -- (82.8000,73.7000) -- (82.8000,73.7000) -- (82.8000,73.7000) -- (82.8000,73.7000) -- (82.8000,73.7000) -- (82.8000,73.7000) -- (82.8000,73.7000) -- (82.8000,73.7000) -- (82.8000,73.7000) -- (82.8000,73.7000) -- (82.8000,73.7000) -- (82.8000,73.7000) -- (82.8000,73.7000) -- (82.8000,73.7000) -- (82.8000,73.7000) -- (82.8000,73.7000) -- (82.8000,73.7000) -- (82.8000,73.7000) -- (82.8000,73.7000) -- (82.8000,73.7000) -- (82.8000,73.6000) -- (82.8000,73.6000) -- (82.8000,73.6000) -- (82.8000,73.6000) -- (82.8000,73.6000) -- (82.8000,73.6000) -- (82.8000,73.6000) -- (82.8000,73.6000) -- (82.8000,73.6000) -- (82.8000,73.6000) -- (82.8000,73.6000) -- (82.8000,73.6000) -- (82.9000,73.6000) -- (82.9000,73.6000) -- (82.9000,73.6000) -- (82.9000,73.6000) -- (82.9000,73.6000) -- (82.9000,73.6000) -- (82.9000,73.6000) -- (82.9000,73.6000) -- (82.9000,73.6000) -- (82.9000,73.5000) -- (82.9000,73.5000) -- (82.9000,73.5000) -- (82.9000,73.5000) -- (82.9000,73.5000) -- (82.9000,73.5000) -- (82.9000,73.5000) -- (82.9000,73.5000) -- (82.9000,73.5000) -- (82.9000,73.5000) -- (82.9000,73.5000) -- (82.9000,73.5000) -- (82.9000,73.5000) -- (82.9000,73.5000) -- (82.9000,73.5000) -- (82.9000,73.5000) -- (82.9000,73.5000) -- (82.9000,73.5000) -- (82.9000,73.5000) -- (82.9000,73.5000) -- (82.9000,73.4000) -- (82.9000,73.4000) -- (82.9000,73.4000) -- (82.9000,73.4000) -- (82.9000,73.4000) -- (82.9000,73.4000) -- (82.9000,73.4000) -- (82.9000,73.4000) -- (82.9000,73.4000) -- (82.9000,73.4000) -- (82.9000,73.4000) -- (82.9000,73.4000) -- (82.9000,73.4000) -- (82.9000,73.4000) -- (82.9000,73.4000) -- (82.9000,73.4000) -- (82.9000,73.4000) -- (82.9000,73.4000) -- (82.9000,73.4000) -- (82.9000,73.4000) -- (82.9000,73.4000) -- (83.0000,73.3000) -- (83.0000,73.3000) -- (83.0000,73.3000) -- (83.0000,73.3000) -- (83.0000,73.3000) -- (83.0000,73.3000) -- (83.0000,73.3000) -- (83.0000,73.3000) -- (83.0000,73.3000) -- (83.0000,73.3000) -- (83.0000,73.3000) -- (83.0000,73.3000) -- (83.0000,73.3000) -- (83.0000,73.3000) -- (83.0000,73.3000) -- (83.0000,73.3000) -- (83.0000,73.3000) -- (83.0000,73.3000) -- (83.0000,73.3000) -- (83.0000,73.3000) -- (83.0000,73.3000) -- (83.0000,73.2000) -- (83.0000,73.2000) -- (83.0000,73.2000) -- (83.0000,73.2000) -- (83.0000,73.2000) -- (83.0000,73.2000) -- (83.0000,73.2000) -- (83.0000,73.2000) -- (83.0000,73.2000) -- (83.0000,73.2000) -- (83.0000,73.2000) -- (83.0000,73.2000) -- (83.0000,73.2000) -- (83.0000,73.2000) -- (83.0000,73.2000) -- (83.0000,73.2000) -- (83.0000,73.2000) -- (83.0000,73.2000) -- (83.0000,73.2000) -- (83.0000,73.2000) -- (83.0000,73.2000) -- (83.0000,73.1000) -- (83.0000,73.1000) -- (83.0000,73.1000) -- (83.0000,73.1000) -- (83.0000,73.1000) -- (83.0000,73.1000) -- (83.0000,73.1000) -- (83.1000,73.1000) -- (83.1000,73.1000) -- (83.1000,73.1000) -- (83.1000,73.1000) -- (83.1000,73.1000) -- (83.1000,73.1000) -- (83.1000,73.1000) -- (83.1000,73.1000) -- (83.1000,73.1000) -- (83.1000,73.1000) -- (83.1000,73.1000) -- (83.1000,73.1000) -- (83.1000,73.1000) -- (83.1000,73.1000) -- (83.1000,73.0000) -- (83.1000,73.0000) -- (83.1000,73.0000) -- (83.1000,73.0000) -- (83.1000,73.0000) -- (83.1000,73.0000) -- (83.1000,73.0000) -- (83.1000,73.0000) -- (83.1000,73.0000) -- (83.1000,73.0000) -- (83.1000,73.0000) -- (83.1000,73.0000) -- (83.1000,73.0000) -- (83.1000,73.0000) -- (83.1000,73.0000) -- (83.1000,73.0000) -- (83.1000,73.0000) -- (83.1000,73.0000) -- (83.1000,73.0000) -- (83.1000,73.0000) -- (83.1000,72.9000) -- (83.1000,72.9000) -- (83.1000,72.9000) -- (83.1000,72.9000) -- (83.1000,72.9000) -- (83.1000,72.9000) -- (83.1000,72.9000) -- (83.1000,72.9000) -- (83.1000,72.9000) -- (83.1000,72.9000) -- (83.1000,72.9000) -- (83.1000,72.9000) -- (83.1000,72.9000) -- (83.1000,72.9000) -- (83.1000,72.9000) -- (83.1000,72.9000) -- (83.2000,72.9000) -- (83.2000,72.9000) -- (83.2000,72.9000) -- (83.2000,72.9000) -- (83.2000,72.9000) -- (83.2000,72.8000) -- (83.2000,72.8000) -- (83.2000,72.8000) -- (83.2000,72.8000) -- (83.2000,72.8000) -- (83.2000,72.8000) -- (83.2000,72.8000) -- (83.2000,72.8000) -- (83.2000,72.8000) -- (83.2000,72.8000) -- (83.2000,72.8000) -- (83.2000,72.8000) -- (83.2000,72.8000) -- (83.2000,72.8000) -- (83.2000,72.8000) -- (83.2000,72.8000) -- (83.2000,72.8000) -- (83.2000,72.8000) -- (83.2000,72.8000) -- (83.2000,72.8000) -- (83.2000,72.8000) -- (83.2000,72.7000) -- (83.2000,72.7000) -- (83.2000,72.7000) -- (83.2000,72.7000) -- (83.2000,72.7000) -- (83.2000,72.7000) -- (83.2000,72.7000) -- (83.2000,72.7000) -- (83.2000,72.7000) -- (83.2000,72.7000) -- (83.2000,72.7000) -- (83.2000,72.7000) -- (83.2000,72.7000) -- (83.2000,72.7000) -- (83.2000,72.7000) -- (83.2000,72.7000) -- (83.2000,72.7000) -- (83.2000,72.7000) -- (83.2000,72.7000) -- (83.2000,72.7000) -- (83.2000,72.7000) -- (83.2000,72.6000) -- (83.2000,72.6000) -- (83.3000,72.6000) -- (83.3000,72.6000) -- (83.3000,72.6000) -- (83.3000,72.6000) -- (83.3000,72.6000) -- (83.3000,72.6000) -- (83.3000,72.6000) -- (83.3000,72.6000) -- (83.3000,72.6000) -- (83.3000,72.6000) -- (83.3000,72.6000) -- (83.3000,72.6000) -- (83.3000,72.6000) -- (83.3000,72.6000) -- (83.3000,72.6000) -- (83.3000,72.6000) -- (83.3000,72.6000) -- (83.3000,72.6000) -- (83.3000,72.5000) -- (83.3000,72.5000) -- (83.3000,72.5000) -- (83.3000,72.5000) -- (83.3000,72.5000) -- (83.3000,72.5000) -- (83.3000,72.5000) -- (83.3000,72.5000) -- (83.3000,72.5000) -- (83.3000,72.5000) -- (83.3000,72.5000) -- (83.3000,72.5000) -- (83.3000,72.5000) -- (83.3000,72.5000) -- (83.3000,72.5000) -- (83.3000,72.5000) -- (83.3000,72.5000) -- (83.3000,72.5000) -- (83.3000,72.5000) -- (83.3000,72.5000) -- (83.3000,72.5000) -- (83.3000,72.4000) -- (83.3000,72.4000) -- (83.3000,72.4000) -- (83.3000,72.4000) -- (83.3000,72.4000) -- (83.3000,72.4000) -- (83.3000,72.4000) -- (83.3000,72.4000) -- (83.3000,72.4000) -- (83.3000,72.4000) -- (83.3000,72.4000) -- (83.4000,72.4000) -- (83.4000,72.4000) -- (83.4000,72.4000) -- (83.4000,72.4000) -- (83.4000,72.4000) -- (83.4000,72.4000) -- (83.4000,72.4000) -- (83.4000,72.4000) -- (83.4000,72.4000) -- (83.4000,72.4000) -- (83.4000,72.3000) -- (83.4000,72.3000) -- (83.4000,72.3000) -- (83.4000,72.3000) -- (83.4000,72.3000) -- (83.4000,72.3000) -- (83.4000,72.3000) -- (83.4000,72.3000) -- (83.4000,72.3000) -- (83.4000,72.3000) -- (83.4000,72.3000) -- (83.4000,72.3000) -- (83.4000,72.3000) -- (83.4000,72.3000) -- (83.4000,72.3000) -- (83.4000,72.3000) -- (83.4000,72.3000) -- (83.4000,72.3000) -- (83.4000,72.3000) -- (83.4000,72.3000) -- (83.4000,72.3000) -- (83.4000,72.2000) -- (83.4000,72.2000) -- (83.4000,72.2000) -- (83.4000,72.2000) -- (83.4000,72.2000) -- (83.4000,72.2000) -- (83.4000,72.2000) -- (83.4000,72.2000) -- (83.4000,72.2000) -- (83.4000,72.2000) -- (83.4000,72.2000) -- (83.4000,72.2000) -- (83.4000,72.2000) -- (83.4000,72.2000) -- (83.4000,72.2000) -- (83.4000,72.2000) -- (83.4000,72.2000) -- (83.4000,72.2000) -- (83.5000,72.2000) -- (83.5000,72.2000) -- (83.5000,72.2000) -- (83.5000,72.1000) -- (83.5000,72.1000) -- (83.5000,72.1000) -- (83.5000,72.1000) -- (83.5000,72.1000) -- (83.5000,72.1000) -- (83.5000,72.1000) -- (83.5000,72.1000) -- (83.5000,72.1000) -- (83.5000,72.1000) -- (83.5000,72.1000) -- (83.5000,72.1000) -- (83.5000,72.1000) -- (83.5000,72.1000) -- (83.5000,72.1000) -- (83.5000,72.1000) -- (83.5000,72.1000) -- (83.5000,72.1000) -- (83.5000,72.1000) -- (83.5000,72.1000) -- (83.5000,72.0000) -- (83.5000,72.0000) -- (83.5000,72.0000) -- (83.5000,72.0000) -- (83.5000,72.0000) -- (83.5000,72.0000) -- (83.5000,72.0000) -- (83.5000,72.0000) -- (83.5000,72.0000) -- (83.5000,72.0000) -- (83.5000,72.0000) -- (83.5000,72.0000) -- (83.5000,72.0000) -- (83.5000,72.0000) -- (83.5000,72.0000) -- (83.5000,72.0000) -- (83.5000,72.0000) -- (83.5000,72.0000) -- (83.5000,72.0000) -- (83.5000,72.0000) -- (83.5000,72.0000) -- (83.5000,71.9000) -- (83.5000,71.9000) -- (83.5000,71.9000) -- (83.5000,71.9000) -- (83.5000,71.9000) -- (83.5000,71.9000) -- (83.6000,71.9000) -- (83.6000,71.9000) -- (83.6000,71.9000) -- (83.6000,71.9000) -- (83.6000,71.9000) -- (83.6000,71.9000) -- (83.6000,71.9000) -- (83.6000,71.9000) -- (83.6000,71.9000) -- (83.6000,71.9000) -- (83.6000,71.9000) -- (83.6000,71.9000) -- (83.6000,71.9000) -- (83.6000,71.9000) -- (83.6000,71.9000) -- (83.6000,71.8000) -- (83.6000,71.8000) -- (83.6000,71.8000) -- (83.6000,71.8000) -- (83.6000,71.8000) -- (83.6000,71.8000) -- (83.6000,71.8000) -- (83.6000,71.8000) -- (83.6000,71.8000) -- (83.6000,71.8000) -- (83.6000,71.8000) -- (83.6000,71.8000) -- (83.6000,71.8000) -- (83.6000,71.8000) -- (83.6000,71.8000) -- (83.6000,71.8000) -- (83.6000,71.8000) -- (83.6000,71.8000) -- (83.6000,71.8000) -- (83.6000,71.8000) -- (83.6000,71.8000) -- (83.6000,71.7000) -- (83.6000,71.7000) -- (83.6000,71.7000) -- (83.6000,71.7000) -- (83.6000,71.7000) -- (83.6000,71.7000) -- (83.6000,71.7000) -- (83.6000,71.7000) -- (83.6000,71.7000) -- (83.6000,71.7000) -- (83.6000,71.7000) -- (83.6000,71.7000) -- (83.6000,71.7000) -- (83.7000,71.7000) -- (83.7000,71.7000) -- (83.7000,71.7000) -- (83.7000,71.7000) -- (83.7000,71.7000) -- (83.7000,71.7000) -- (83.7000,71.7000) -- (83.7000,71.7000) -- (83.7000,71.6000) -- (83.7000,71.6000) -- (83.7000,71.6000) -- (83.7000,71.6000) -- (83.7000,71.6000) -- (83.7000,71.6000) -- (83.7000,71.6000) -- (83.7000,71.6000) -- (83.7000,71.6000) -- (83.7000,71.6000) -- (83.7000,71.6000) -- (83.7000,71.6000) -- (83.7000,71.6000) -- (83.7000,71.6000) -- (83.7000,71.6000) -- (83.7000,71.6000) -- (83.7000,71.6000) -- (83.7000,71.6000) -- (83.7000,71.6000) -- (83.7000,71.6000) -- (83.7000,71.5000) -- (83.7000,71.5000) -- (83.7000,71.5000) -- (83.7000,71.5000) -- (83.7000,71.5000) -- (83.7000,71.5000) -- (83.7000,71.5000) -- (83.7000,71.5000) -- (83.7000,71.5000) -- (83.7000,71.5000) -- (83.7000,71.5000) -- (83.7000,71.5000) -- (83.7000,71.5000) -- (83.7000,71.5000) -- (83.7000,71.5000) -- (83.7000,71.5000) -- (83.7000,71.5000) -- (83.7000,71.5000) -- (83.7000,71.5000) -- (83.7000,71.5000) -- (83.7000,71.5000) -- (83.7000,71.4000) -- (83.8000,71.4000) -- (83.8000,71.4000) -- (83.8000,71.4000) -- (83.8000,71.4000) -- (83.8000,71.4000) -- (83.8000,71.4000) -- (83.8000,71.4000) -- (83.8000,71.4000) -- (83.8000,71.4000) -- (83.8000,71.4000) -- (83.8000,71.4000) -- (83.8000,71.4000) -- (83.8000,71.4000) -- (83.8000,71.4000) -- (83.8000,71.4000) -- (83.8000,71.4000) -- (83.8000,71.4000) -- (83.8000,71.4000) -- (83.8000,71.4000) -- (83.8000,71.4000) -- (83.8000,71.3000) -- (83.8000,71.3000) -- (83.8000,71.3000) -- (83.8000,71.3000) -- (83.8000,71.3000) -- (83.8000,71.3000) -- (83.8000,71.3000) -- (83.8000,71.3000) -- (83.8000,71.3000) -- (83.8000,71.3000) -- (83.8000,71.3000) -- (83.8000,71.3000) -- (83.8000,71.3000) -- (83.8000,71.3000) -- (83.8000,71.3000) -- (83.8000,71.3000) -- (83.8000,71.3000) -- (83.8000,71.3000) -- (83.8000,71.3000) -- (83.8000,71.3000) -- (83.8000,71.3000) -- (83.8000,71.2000) -- (83.8000,71.2000) -- (83.8000,71.2000) -- (83.8000,71.2000) -- (83.8000,71.2000) -- (83.8000,71.2000) -- (83.8000,71.2000) -- (83.8000,71.2000) -- (83.9000,71.2000) -- (83.9000,71.2000) -- (83.9000,71.2000) -- (83.9000,71.2000) -- (83.9000,71.2000) -- (83.9000,71.2000) -- (83.9000,71.2000) -- (83.9000,71.2000) -- (83.9000,71.2000) -- (83.9000,71.2000) -- (83.9000,71.2000) -- (83.9000,71.2000) -- (83.9000,71.2000) -- (83.9000,71.1000) -- (83.9000,71.1000) -- (83.9000,71.1000) -- (83.9000,71.1000) -- (83.9000,71.1000) -- (83.9000,71.1000) -- (83.9000,71.1000) -- (83.9000,71.1000) -- (83.9000,71.1000) -- (83.9000,71.1000) -- (83.9000,71.1000) -- (83.9000,71.1000) -- (83.9000,71.1000) -- (83.9000,71.1000) -- (83.9000,71.1000) -- (83.9000,71.1000) -- (83.9000,71.1000) -- (83.9000,71.1000) -- (83.9000,71.1000) -- (83.9000,71.1000) -- (83.9000,71.0000) -- (83.9000,71.0000) -- (83.9000,71.0000) -- (83.9000,71.0000) -- (83.9000,71.0000) -- (83.9000,71.0000) -- (83.9000,71.0000) -- (83.9000,71.0000) -- (83.9000,71.0000) -- (83.9000,71.0000) -- (83.9000,71.0000) -- (83.9000,71.0000) -- (83.9000,71.0000) -- (83.9000,71.0000) -- (83.9000,71.0000) -- (83.9000,71.0000) -- (83.9000,71.0000) -- (84.0000,71.0000) -- (84.0000,71.0000) -- (84.0000,71.0000) -- (84.0000,71.0000) -- (84.0000,70.9000) -- (84.0000,70.9000) -- (84.0000,70.9000) -- (84.0000,70.9000) -- (84.0000,70.9000) -- (84.0000,70.9000) -- (84.0000,70.9000) -- (84.0000,70.9000) -- (84.0000,70.9000) -- (84.0000,70.9000) -- (84.0000,70.9000) -- (84.0000,70.9000) -- (84.0000,70.9000) -- (84.0000,70.9000) -- (84.0000,70.9000) -- (84.0000,70.9000) -- (84.0000,70.9000) -- (84.0000,70.9000) -- (84.0000,70.9000) -- (84.0000,70.9000) -- (84.0000,70.9000) -- (84.0000,70.8000) -- (84.0000,70.8000) -- (84.0000,70.8000) -- (84.0000,70.8000) -- (84.0000,70.8000) -- (84.0000,70.8000) -- (84.0000,70.8000) -- (84.0000,70.8000) -- (84.0000,70.8000) -- (84.0000,70.8000) -- (84.0000,70.8000) -- (84.0000,70.8000) -- (84.0000,70.8000) -- (84.0000,70.8000) -- (84.0000,70.8000) -- (84.0000,70.8000) -- (84.0000,70.8000) -- (84.0000,70.8000) -- (84.0000,70.8000) -- (84.0000,70.8000) -- (84.0000,70.8000) -- (84.0000,70.7000) -- (84.0000,70.7000) -- (84.0000,70.7000) -- (84.1000,70.7000) -- (84.1000,70.7000) -- (84.1000,70.7000) -- (84.1000,70.7000) -- (84.1000,70.7000) -- (84.1000,70.7000) -- (84.1000,70.7000) -- (84.1000,70.7000) -- (84.1000,70.7000) -- (84.1000,70.7000) -- (84.1000,70.7000) -- (84.1000,70.7000) -- (84.1000,70.7000) -- (84.1000,70.7000) -- (84.1000,70.7000) -- (84.1000,70.7000) -- (84.1000,70.7000) -- (84.1000,70.6000) -- (84.1000,70.6000) -- (84.1000,70.6000) -- (84.1000,70.6000) -- (84.1000,70.6000) -- (84.1000,70.6000) -- (84.1000,70.6000) -- (84.1000,70.6000) -- (84.1000,70.6000) -- (84.1000,70.6000) -- (84.1000,70.6000) -- (84.1000,70.6000) -- (84.1000,70.6000) -- (84.1000,70.6000) -- (84.1000,70.6000) -- (84.1000,70.6000) -- (84.1000,70.6000) -- (84.1000,70.6000) -- (84.1000,70.6000) -- (84.1000,70.6000) -- (84.1000,70.6000) -- (84.1000,70.5000) -- (84.1000,70.5000) -- (84.1000,70.5000) -- (84.1000,70.5000) -- (84.1000,70.5000) -- (84.1000,70.5000) -- (84.1000,70.5000) -- (84.1000,70.5000) -- (84.1000,70.5000) -- (84.1000,70.5000) -- (84.1000,70.5000) -- (84.1000,70.5000) -- (84.2000,70.5000) -- (84.2000,70.5000) -- (84.2000,70.5000) -- (84.2000,70.5000) -- (84.2000,70.5000) -- (84.2000,70.5000) -- (84.2000,70.5000) -- (84.2000,70.5000) -- (84.2000,70.5000) -- (84.2000,70.4000) -- (84.2000,70.4000) -- (84.2000,70.4000) -- (84.2000,70.4000) -- (84.2000,70.4000) -- (84.2000,70.4000) -- (84.2000,70.4000) -- (84.2000,70.4000) -- (84.2000,70.4000) -- (84.2000,70.4000) -- (84.2000,70.4000) -- (84.2000,70.4000) -- (84.2000,70.4000) -- (84.2000,70.4000) -- (84.2000,70.4000) -- (84.2000,70.4000) -- (84.2000,70.4000) -- (84.2000,70.4000) -- (84.2000,70.4000) -- (84.2000,70.4000) -- (84.2000,70.4000) -- (84.2000,70.3000) -- (84.2000,70.3000) -- (84.2000,70.3000) -- (84.2000,70.3000) -- (84.2000,70.3000) -- (84.2000,70.3000) -- (84.2000,70.3000) -- (84.2000,70.3000) -- (84.2000,70.3000) -- (84.2000,70.3000) -- (84.2000,70.3000) -- (84.2000,70.3000) -- (84.2000,70.3000) -- (84.2000,70.3000) -- (84.2000,70.3000) -- (84.2000,70.3000) -- (84.2000,70.3000) -- (84.2000,70.3000) -- (84.2000,70.3000) -- (84.3000,70.3000) -- (84.3000,70.3000) -- (84.3000,70.2000) -- (84.3000,70.2000) -- (84.3000,70.2000) -- (84.3000,70.2000) -- (84.3000,70.2000) -- (84.3000,70.2000) -- (84.3000,70.2000) -- (84.3000,70.2000) -- (84.3000,70.2000) -- (84.3000,70.2000) -- (84.3000,70.2000) -- (84.3000,70.2000) -- (84.3000,70.2000) -- (84.3000,70.2000) -- (84.3000,70.2000) -- (84.3000,70.2000) -- (84.3000,70.2000) -- (84.3000,70.2000) -- (84.3000,70.2000) -- (84.3000,70.2000) -- (84.3000,70.1000) -- (84.3000,70.1000) -- (84.3000,70.1000) -- (84.3000,70.1000) -- (84.3000,70.1000) -- (84.3000,70.1000) -- (84.3000,70.1000) -- (84.3000,70.1000) -- (84.3000,70.1000) -- (84.3000,70.1000) -- (84.3000,70.1000) -- (84.3000,70.1000) -- (84.3000,70.1000) -- (84.3000,70.1000) -- (84.3000,70.1000) -- (84.3000,70.1000) -- (84.3000,70.1000) -- (84.3000,70.1000) -- (84.3000,70.1000) -- (84.3000,70.1000) -- (84.3000,70.1000) -- (84.3000,70.0000) -- (84.3000,70.0000) -- (84.3000,70.0000) -- (84.3000,70.0000) -- (84.3000,70.0000) -- (84.3000,70.0000) -- (84.3000,70.0000) -- (84.4000,70.0000) -- (84.4000,70.0000) -- (84.4000,70.0000) -- (84.4000,70.0000) -- (84.4000,70.0000) -- (84.4000,70.0000) -- (84.4000,70.0000) -- (84.4000,70.0000) -- (84.4000,70.0000) -- (84.4000,70.0000) -- (84.4000,70.0000) -- (84.4000,70.0000) -- (84.4000,70.0000) -- (84.4000,70.0000) -- (84.4000,69.9000) -- (84.4000,69.9000) -- (84.4000,69.9000) -- (84.4000,69.9000) -- (84.4000,69.9000) -- (84.4000,69.9000) -- (84.4000,69.9000) -- (84.4000,69.9000) -- (84.4000,69.9000) -- (84.4000,69.9000) -- (84.4000,69.9000) -- (84.4000,69.9000) -- (84.4000,69.9000) -- (84.4000,69.9000) -- (84.4000,69.9000) -- (84.4000,69.9000) -- (84.4000,69.9000) -- (84.4000,69.9000) -- (84.4000,69.9000) -- (84.4000,69.9000) -- (84.4000,69.9000) -- (84.4000,69.8000) -- (84.4000,69.8000) -- (84.4000,69.8000) -- (84.4000,69.8000) -- (84.4000,69.8000) -- (84.4000,69.8000) -- (84.4000,69.8000) -- (84.4000,69.8000) -- (84.4000,69.8000) -- (84.4000,69.8000) -- (84.4000,69.8000) -- (84.4000,69.8000) -- (84.4000,69.8000) -- (84.4000,69.8000) -- (84.5000,69.8000) -- (84.5000,69.8000) -- (84.5000,69.8000) -- (84.5000,69.8000) -- (84.5000,69.8000) -- (84.5000,69.8000) -- (84.5000,69.8000) -- (84.5000,69.7000) -- (84.5000,69.7000) -- (84.5000,69.7000) -- (84.5000,69.7000) -- (84.5000,69.7000) -- (84.5000,69.7000) -- (84.5000,69.7000) -- (84.5000,69.7000) -- (84.5000,69.7000) -- (84.5000,69.7000) -- (84.5000,69.7000) -- (84.5000,69.7000) -- (84.5000,69.7000) -- (84.5000,69.7000) -- (84.5000,69.7000) -- (84.5000,69.7000) -- (84.5000,69.7000) -- (84.5000,69.7000) -- (84.5000,69.7000) -- (84.5000,69.7000) -- (84.5000,69.6000) -- (84.5000,69.6000) -- (84.5000,69.6000) -- (84.5000,69.6000) -- (84.5000,69.6000) -- (84.5000,69.6000) -- (84.5000,69.6000) -- (84.5000,69.6000) -- (84.5000,69.6000) -- (84.5000,69.6000) -- (84.5000,69.6000) -- (84.5000,69.6000) -- (84.5000,69.6000) -- (84.5000,69.6000) -- (84.5000,69.6000) -- (84.5000,69.6000) -- (84.5000,69.6000) -- (84.5000,69.6000) -- (84.5000,69.6000) -- (84.5000,69.6000) -- (84.5000,69.6000) -- (84.5000,69.5000) -- (84.5000,69.5000) -- (84.6000,69.5000) -- (84.6000,69.5000) -- (84.6000,69.5000) -- (84.6000,69.5000) -- (84.6000,69.5000) -- (84.6000,69.5000) -- (84.6000,69.5000) -- (84.6000,69.5000) -- (84.6000,69.5000) -- (84.6000,69.5000) -- (84.6000,69.5000) -- (84.6000,69.5000) -- (84.6000,69.5000) -- (84.6000,69.5000) -- (84.6000,69.5000) -- (84.6000,69.5000) -- (84.6000,69.5000) -- (84.6000,69.5000) -- (84.6000,69.5000) -- (84.6000,69.4000) -- (84.6000,69.4000) -- (84.6000,69.4000) -- (84.6000,69.4000) -- (84.6000,69.4000) -- (84.6000,69.4000) -- (84.6000,69.4000) -- (84.6000,69.4000) -- (84.6000,69.4000) -- (84.6000,69.4000) -- (84.6000,69.4000) -- (84.6000,69.4000) -- (84.6000,69.4000) -- (84.6000,69.4000) -- (84.6000,69.4000) -- (84.6000,69.4000) -- (84.6000,69.4000) -- (84.6000,69.4000) -- (84.6000,69.4000) -- (84.6000,69.4000) -- (84.6000,69.4000) -- (84.6000,69.3000) -- (84.6000,69.3000) -- (84.6000,69.3000) -- (84.6000,69.3000) -- (84.6000,69.3000) -- (84.6000,69.3000) -- (84.6000,69.3000) -- (84.6000,69.3000) -- (84.6000,69.3000) -- (84.7000,69.3000) -- (84.7000,69.3000) -- (84.7000,69.3000) -- (84.7000,69.3000) -- (84.7000,69.3000) -- (84.7000,69.3000) -- (84.7000,69.3000) -- (84.7000,69.3000) -- (84.7000,69.3000) -- (84.7000,69.3000) -- (84.7000,69.3000) -- (84.7000,69.2000) -- (84.7000,69.2000) -- (84.7000,69.2000) -- (84.7000,69.2000) -- (84.7000,69.2000) -- (84.7000,69.2000) -- (84.7000,69.2000) -- (84.7000,69.2000) -- (84.7000,69.2000) -- (84.7000,69.2000) -- (84.7000,69.2000) -- (84.7000,69.2000) -- (84.7000,69.2000) -- (84.7000,69.2000) -- (84.7000,69.2000) -- (84.7000,69.2000) -- (84.7000,69.2000) -- (84.7000,69.2000) -- (84.7000,69.2000) -- (84.7000,69.2000) -- (84.7000,69.2000) -- (84.7000,69.1000) -- (84.7000,69.1000) -- (84.7000,69.1000) -- (84.7000,69.1000) -- (84.7000,69.1000) -- (84.7000,69.1000) -- (84.7000,69.1000) -- (84.7000,69.1000) -- (84.7000,69.1000) -- (84.7000,69.1000) -- (84.7000,69.1000) -- (84.7000,69.1000) -- (84.7000,69.1000) -- (84.7000,69.1000) -- (84.7000,69.1000) -- (84.7000,69.1000) -- (84.7000,69.1000) -- (84.7000,69.1000) -- (84.8000,69.1000) -- (84.8000,69.1000) -- (84.8000,69.1000) -- (84.8000,69.0000) -- (84.8000,69.0000) -- (84.8000,69.0000) -- (84.8000,69.0000) -- (84.8000,69.0000) -- (84.8000,69.0000) -- (84.8000,69.0000) -- (84.8000,69.0000) -- (84.8000,69.0000) -- (84.8000,69.0000) -- (84.8000,69.0000) -- (84.8000,69.0000) -- (84.8000,69.0000) -- (84.8000,69.0000) -- (84.8000,69.0000) -- (84.8000,69.0000) -- (84.8000,69.0000) -- (84.8000,69.0000) -- (84.8000,69.0000) -- (84.8000,69.0000) -- (84.8000,69.0000) -- (84.8000,68.9000) -- (84.8000,68.9000) -- (84.8000,68.9000) -- (84.8000,68.9000) -- (84.8000,68.9000) -- (84.8000,68.9000) -- (84.8000,68.9000) -- (84.8000,68.9000) -- (84.8000,68.9000) -- (84.8000,68.9000) -- (84.8000,68.9000) -- (84.8000,68.9000) -- (84.8000,68.9000) -- (84.8000,68.9000) -- (84.8000,68.9000) -- (84.8000,68.9000) -- (84.8000,68.9000) -- (84.8000,68.9000) -- (84.8000,68.9000) -- (84.8000,68.9000) -- (84.8000,68.9000) -- (84.8000,68.8000) -- (84.8000,68.8000) -- (84.8000,68.8000) -- (84.8000,68.8000) -- (84.9000,68.8000) -- (84.9000,68.8000) -- (84.9000,68.8000) -- (84.9000,68.8000) -- (84.9000,68.8000) -- (84.9000,68.8000) -- (84.9000,68.8000) -- (84.9000,68.8000) -- (84.9000,68.8000) -- (84.9000,68.8000) -- (84.9000,68.8000) -- (84.9000,68.8000) -- (84.9000,68.8000) -- (84.9000,68.8000) -- (84.9000,68.8000) -- (84.9000,68.8000) -- (84.9000,68.7000) -- (84.9000,68.7000) -- (84.9000,68.7000) -- (84.9000,68.7000) -- (84.9000,68.7000) -- (84.9000,68.7000) -- (84.9000,68.7000) -- (84.9000,68.7000) -- (84.9000,68.7000) -- (84.9000,68.7000) -- (84.9000,68.7000) -- (84.9000,68.7000) -- (84.9000,68.7000) -- (84.9000,68.7000) -- (84.9000,68.7000) -- (84.9000,68.7000) -- (84.9000,68.7000) -- (84.9000,68.7000) -- (84.9000,68.7000) -- (84.9000,68.7000) -- (84.9000,68.7000) -- (84.9000,68.6000) -- (84.9000,68.6000) -- (84.9000,68.6000) -- (84.9000,68.6000) -- (84.9000,68.6000) -- (84.9000,68.6000) -- (84.9000,68.6000) -- (84.9000,68.6000) -- (84.9000,68.6000) -- (84.9000,68.6000) -- (84.9000,68.6000) -- (84.9000,68.6000) -- (84.9000,68.6000) -- (85.0000,68.6000) -- (85.0000,68.6000) -- (85.0000,68.6000) -- (85.0000,68.6000) -- (85.0000,68.6000) -- (85.0000,68.6000) -- (85.0000,68.6000) -- (85.0000,68.6000) -- (85.0000,68.5000) -- (85.0000,68.5000) -- (85.0000,68.5000) -- (85.0000,68.5000) -- (85.0000,68.5000) -- (85.0000,68.5000) -- (85.0000,68.5000) -- (85.0000,68.5000) -- (85.0000,68.5000) -- (85.0000,68.5000) -- (85.0000,68.5000) -- (85.0000,68.5000) -- (85.0000,68.5000) -- (85.0000,68.5000) -- (85.0000,68.5000) -- (85.0000,68.5000) -- (85.0000,68.5000) -- (85.0000,68.5000) -- (85.0000,68.5000) -- (85.0000,68.5000) -- (85.0000,68.5000) -- (85.0000,68.4000) -- (85.0000,68.4000) -- (85.0000,68.4000) -- (85.0000,68.4000) -- (85.0000,68.4000) -- (85.0000,68.4000) -- (85.0000,68.4000) -- (85.0000,68.4000) -- (85.0000,68.4000) -- (85.0000,68.4000) -- (85.0000,68.4000) -- (85.0000,68.4000) -- (85.0000,68.4000) -- (85.0000,68.4000) -- (85.0000,68.4000) -- (85.0000,68.4000) -- (85.0000,68.4000) -- (85.0000,68.4000) -- (85.0000,68.4000) -- (85.0000,68.4000) -- (85.1000,68.4000) -- (85.1000,68.3000) -- (85.1000,68.3000) -- (85.1000,68.3000) -- (85.1000,68.3000) -- (85.1000,68.3000) -- (85.1000,68.3000) -- (85.1000,68.3000) -- (85.1000,68.3000) -- (85.1000,68.3000) -- (85.1000,68.3000) -- (85.1000,68.3000) -- (85.1000,68.3000) -- (85.1000,68.3000) -- (85.1000,68.3000) -- (85.1000,68.3000) -- (85.1000,68.3000) -- (85.1000,68.3000) -- (85.1000,68.3000) -- (85.1000,68.3000) -- (85.1000,68.3000) -- (85.1000,68.2000) -- (85.1000,68.2000) -- (85.1000,68.2000) -- (85.1000,68.2000) -- (85.1000,68.2000) -- (85.1000,68.2000) -- (85.1000,68.2000) -- (85.1000,68.2000) -- (85.1000,68.2000) -- (85.1000,68.2000) -- (85.1000,68.2000) -- (85.1000,68.2000) -- (85.1000,68.2000) -- (85.1000,68.2000) -- (85.1000,68.2000) -- (85.1000,68.2000) -- (85.1000,68.2000) -- (85.1000,68.2000) -- (85.1000,68.2000) -- (85.1000,68.2000) -- (85.1000,68.2000) -- (85.1000,68.1000) -- (85.1000,68.1000) -- (85.1000,68.1000) -- (85.1000,68.1000) -- (85.1000,68.1000) -- (85.1000,68.1000) -- (85.1000,68.1000) -- (85.1000,68.1000) -- (85.2000,68.1000) -- (85.2000,68.1000) -- (85.2000,68.1000) -- (85.2000,68.1000) -- (85.2000,68.1000) -- (85.2000,68.1000) -- (85.2000,68.1000) -- (85.2000,68.1000) -- (85.2000,68.1000) -- (85.2000,68.1000) -- (85.2000,68.1000) -- (85.2000,68.1000) -- (85.2000,68.1000) -- (85.2000,68.0000) -- (85.2000,68.0000) -- (85.2000,68.0000) -- (85.2000,68.0000) -- (85.2000,68.0000) -- (85.2000,68.0000) -- (85.2000,68.0000) -- (85.2000,68.0000) -- (85.2000,68.0000) -- (85.2000,68.0000) -- (85.2000,68.0000) -- (85.2000,68.0000) -- (85.2000,68.0000) -- (85.2000,68.0000) -- (85.2000,68.0000) -- (85.2000,68.0000) -- (85.2000,68.0000) -- (85.2000,68.0000) -- (85.2000,68.0000) -- (85.2000,68.0000) -- (85.2000,68.0000) -- (85.2000,67.9000) -- (85.2000,67.9000) -- (85.2000,67.9000) -- (85.2000,67.9000) -- (85.2000,67.9000) -- (85.2000,67.9000) -- (85.2000,67.9000) -- (85.2000,67.9000) -- (85.2000,67.9000) -- (85.2000,67.9000) -- (85.2000,67.9000) -- (85.2000,67.9000) -- (85.2000,67.9000) -- (85.2000,67.9000) -- (85.2000,67.9000) -- (85.3000,67.9000) -- (85.3000,67.9000) -- (85.3000,67.9000) -- (85.3000,67.9000) -- (85.3000,67.9000) -- (85.3000,67.9000) -- (85.3000,67.8000) -- (85.3000,67.8000) -- (85.3000,67.8000) -- (85.3000,67.8000) -- (85.3000,67.8000) -- (85.3000,67.8000) -- (85.3000,67.8000) -- (85.3000,67.8000) -- (85.3000,67.8000) -- (85.3000,67.8000) -- (85.3000,67.8000) -- (85.3000,67.8000) -- (85.3000,67.8000) -- (85.3000,67.8000) -- (85.3000,67.8000) -- (85.3000,67.8000) -- (85.3000,67.8000) -- (85.3000,67.8000) -- (85.3000,67.8000) -- (85.3000,67.8000) -- (85.3000,67.7000) -- (85.3000,67.7000) -- (85.3000,67.7000) -- (85.3000,67.7000) -- (85.3000,67.7000) -- (85.3000,67.7000) -- (85.3000,67.7000) -- (85.3000,67.7000) -- (85.3000,67.7000) -- (85.3000,67.7000) -- (85.3000,67.7000) -- (85.3000,67.7000) -- (85.3000,67.7000) -- (85.3000,67.7000) -- (85.3000,67.7000) -- (85.3000,67.7000) -- (85.3000,67.7000) -- (85.3000,67.7000) -- (85.3000,67.7000) -- (85.3000,67.7000) -- (85.3000,67.7000) -- (85.3000,67.6000) -- (85.3000,67.6000) -- (85.3000,67.6000) -- (85.4000,67.6000) -- (85.4000,67.6000) -- (85.4000,67.6000) -- (85.4000,67.6000) -- (85.4000,67.6000) -- (85.4000,67.6000) -- (85.4000,67.6000) -- (85.4000,67.6000) -- (85.4000,67.6000) -- (85.4000,67.6000) -- (85.4000,67.6000) -- (85.4000,67.6000) -- (85.4000,67.6000) -- (85.4000,67.6000) -- (85.4000,67.6000) -- (85.4000,67.6000) -- (85.4000,67.6000) -- (85.4000,67.6000) -- (85.4000,67.5000) -- (85.4000,67.5000) -- (85.4000,67.5000) -- (85.4000,67.5000) -- (85.4000,67.5000) -- (85.4000,67.5000) -- (85.4000,67.5000) -- (85.4000,67.5000) -- (85.4000,67.5000) -- (85.4000,67.5000) -- (85.4000,67.5000) -- (85.4000,67.5000) -- (85.4000,67.5000) -- (85.4000,67.5000) -- (85.4000,67.5000) -- (85.4000,67.5000) -- (85.4000,67.5000) -- (85.4000,67.5000) -- (85.4000,67.5000) -- (85.4000,67.5000) -- (85.4000,67.5000) -- (85.4000,67.4000) -- (85.4000,67.4000) -- (85.4000,67.4000) -- (85.4000,67.4000) -- (85.4000,67.4000) -- (85.4000,67.4000) -- (85.4000,67.4000) -- (85.4000,67.4000) -- (85.4000,67.4000) -- (85.4000,67.4000) -- (85.5000,67.4000) -- (85.5000,67.4000) -- (85.5000,67.4000) -- (85.5000,67.4000) -- (85.5000,67.4000) -- (85.5000,67.4000) -- (85.5000,67.4000) -- (85.5000,67.4000) -- (85.5000,67.4000) -- (85.5000,67.4000) -- (85.5000,67.3000) -- (85.5000,67.3000) -- (85.5000,67.3000) -- (85.5000,67.3000) -- (85.5000,67.3000) -- (85.5000,67.3000) -- (85.5000,67.3000) -- (85.5000,67.3000) -- (85.5000,67.3000) -- (85.5000,67.3000) -- (85.5000,67.3000) -- (85.5000,67.3000) -- (85.5000,67.3000) -- (85.5000,67.3000) -- (85.5000,67.3000) -- (85.5000,67.3000) -- (85.5000,67.3000) -- (85.5000,67.3000) -- (85.5000,67.3000) -- (85.5000,67.3000) -- (85.5000,67.3000) -- (85.5000,67.2000) -- (85.5000,67.2000) -- (85.5000,67.2000) -- (85.5000,67.2000) -- (85.5000,67.2000) -- (85.5000,67.2000) -- (85.5000,67.2000) -- (85.5000,67.2000) -- (85.5000,67.2000) -- (85.5000,67.2000) -- (85.5000,67.2000) -- (85.5000,67.2000) -- (85.5000,67.2000) -- (85.5000,67.2000) -- (85.5000,67.2000) -- (85.5000,67.2000) -- (85.5000,67.2000) -- (85.5000,67.2000) -- (85.5000,67.2000) -- (85.6000,67.2000) -- (85.6000,67.2000) -- (85.6000,67.1000) -- (85.6000,67.1000) -- (85.6000,67.1000) -- (85.6000,67.1000) -- (85.6000,67.1000) -- (85.6000,67.1000) -- (85.6000,67.1000) -- (85.6000,67.1000) -- (85.6000,67.1000) -- (85.6000,67.1000) -- (85.6000,67.1000) -- (85.6000,67.1000) -- (85.6000,67.1000) -- (85.6000,67.1000) -- (85.6000,67.1000) -- (85.6000,67.1000) -- (85.6000,67.1000) -- (85.6000,67.1000) -- (85.6000,67.1000) -- (85.6000,67.1000) -- (85.6000,67.1000) -- (85.6000,67.0000) -- (85.6000,67.0000) -- (85.6000,67.0000) -- (85.6000,67.0000) -- (85.6000,67.0000) -- (85.6000,67.0000) -- (85.6000,67.0000) -- (85.6000,67.0000) -- (85.6000,67.0000) -- (85.6000,67.0000) -- (85.6000,67.0000) -- (85.6000,67.0000) -- (85.6000,67.0000) -- (85.6000,67.0000) -- (85.6000,67.0000) -- (85.6000,67.0000) -- (85.6000,67.0000) -- (85.6000,67.0000) -- (85.6000,67.0000) -- (85.6000,67.0000) -- (85.6000,67.0000) -- (85.6000,66.9000) -- (85.6000,66.9000) -- (85.6000,66.9000) -- (85.6000,66.9000) -- (85.6000,66.9000) -- (85.7000,66.9000) -- (85.7000,66.9000) -- (85.7000,66.9000) -- (85.7000,66.9000) -- (85.7000,66.9000) -- (85.7000,66.9000) -- (85.7000,66.9000) -- (85.7000,66.9000) -- (85.7000,66.9000) -- (85.7000,66.9000) -- (85.7000,66.9000) -- (85.7000,66.9000) -- (85.7000,66.9000) -- (85.7000,66.9000) -- (85.7000,66.9000) -- (85.7000,66.8000) -- (85.7000,66.8000) -- (85.7000,66.8000) -- (85.7000,66.8000) -- (85.7000,66.8000) -- (85.7000,66.8000) -- (85.7000,66.8000) -- (85.7000,66.8000) -- (85.7000,66.8000) -- (85.7000,66.8000) -- (85.7000,66.8000) -- (85.7000,66.8000) -- (85.7000,66.8000) -- (85.7000,66.8000) -- (85.7000,66.8000) -- (85.7000,66.8000) -- (85.7000,66.8000) -- (85.7000,66.8000) -- (85.7000,66.8000) -- (85.7000,66.8000) -- (85.7000,66.8000) -- (85.7000,66.7000) -- (85.7000,66.7000) -- (85.7000,66.7000) -- (85.7000,66.7000) -- (85.7000,66.7000) -- (85.7000,66.7000) -- (85.7000,66.7000) -- (85.7000,66.7000) -- (85.7000,66.7000) -- (85.7000,66.7000) -- (85.7000,66.7000) -- (85.7000,66.7000) -- (85.7000,66.7000) -- (85.7000,66.7000) -- (85.8000,66.7000) -- (85.8000,66.7000) -- (85.8000,66.7000) -- (85.8000,66.7000) -- (85.8000,66.7000) -- (85.8000,66.7000) -- (85.8000,66.7000) -- (85.8000,66.6000) -- (85.8000,66.6000) -- (85.8000,66.6000) -- (85.8000,66.6000) -- (85.8000,66.6000) -- (85.8000,66.6000) -- (85.8000,66.6000) -- (85.8000,66.6000) -- (85.8000,66.6000) -- (85.8000,66.6000) -- (85.8000,66.6000) -- (85.8000,66.6000) -- (85.8000,66.6000) -- (85.8000,66.6000) -- (85.8000,66.6000) -- (85.8000,66.6000) -- (85.8000,66.6000) -- (85.8000,66.6000) -- (85.8000,66.6000) -- (85.8000,66.6000) -- (85.8000,66.6000) -- (85.8000,66.5000) -- (85.8000,66.5000) -- (85.8000,66.5000) -- (85.8000,66.5000) -- (85.8000,66.5000) -- (85.8000,66.5000) -- (85.8000,66.5000) -- (85.8000,66.5000) -- (85.8000,66.5000) -- (85.8000,66.5000) -- (85.8000,66.5000) -- (85.8000,66.5000) -- (85.8000,66.5000) -- (85.8000,66.5000) -- (85.8000,66.5000) -- (85.8000,66.5000) -- (85.8000,66.5000) -- (85.8000,66.5000) -- (85.8000,66.5000) -- (85.8000,66.5000) -- (85.8000,66.5000) -- (85.9000,66.4000) -- (85.9000,66.4000) -- (85.9000,66.4000) -- (85.9000,66.4000) -- (85.9000,66.4000) -- (85.9000,66.4000) -- (85.9000,66.4000) -- (85.9000,66.4000) -- (85.9000,66.4000) -- (85.9000,66.4000) -- (85.9000,66.4000) -- (85.9000,66.4000) -- (85.9000,66.4000) -- (85.9000,66.4000) -- (85.9000,66.4000) -- (85.9000,66.4000) -- (85.9000,66.4000) -- (85.9000,66.4000) -- (85.9000,66.4000) -- (85.9000,66.4000) -- (85.9000,66.3000) -- (85.9000,66.3000) -- (85.9000,66.3000) -- (85.9000,66.3000) -- (85.9000,66.3000) -- (85.9000,66.3000) -- (85.9000,66.3000) -- (85.9000,66.3000) -- (85.9000,66.3000) -- (85.9000,66.3000) -- (85.9000,66.3000) -- (85.9000,66.3000) -- (85.9000,66.3000) -- (85.9000,66.3000) -- (85.9000,66.3000) -- (85.9000,66.3000) -- (85.9000,66.3000) -- (85.9000,66.3000) -- (85.9000,66.3000) -- (85.9000,66.3000) -- (85.9000,66.3000) -- (85.9000,66.2000) -- (85.9000,66.2000) -- (85.9000,66.2000) -- (85.9000,66.2000) -- (85.9000,66.2000) -- (85.9000,66.2000) -- (85.9000,66.2000) -- (85.9000,66.2000) -- (85.9000,66.2000) -- (86.0000,66.2000) -- (86.0000,66.2000) -- (86.0000,66.2000) -- (86.0000,66.2000) -- (86.0000,66.2000) -- (86.0000,66.2000) -- (86.0000,66.2000) -- (86.0000,66.2000) -- (86.0000,66.2000) -- (86.0000,66.2000) -- (86.0000,66.2000) -- (86.0000,66.2000) -- (86.0000,66.1000) -- (86.0000,66.1000) -- (86.0000,66.1000) -- (86.0000,66.1000) -- (86.0000,66.1000) -- (86.0000,66.1000) -- (86.0000,66.1000) -- (86.0000,66.1000) -- (86.0000,66.1000) -- (86.0000,66.1000) -- (86.0000,66.1000) -- (86.0000,66.1000) -- (86.0000,66.1000) -- (86.0000,66.1000) -- (86.0000,66.1000) -- (86.0000,66.1000) -- (86.0000,66.1000) -- (86.0000,66.1000) -- (86.0000,66.1000) -- (86.0000,66.1000) -- (86.0000,66.1000) -- (86.0000,66.0000) -- (86.0000,66.0000) -- (86.0000,66.0000) -- (86.0000,66.0000) -- (86.0000,66.0000) -- (86.0000,66.0000) -- (86.0000,66.0000) -- (86.0000,66.0000) -- (86.0000,66.0000) -- (86.0000,66.0000) -- (86.0000,66.0000) -- (86.0000,66.0000) -- (86.0000,66.0000) -- (86.0000,66.0000) -- (86.0000,66.0000) -- (86.0000,66.0000) -- (86.1000,66.0000) -- (86.1000,66.0000) -- (86.1000,66.0000) -- (86.1000,66.0000) -- (86.1000,66.0000) -- (86.1000,65.9000) -- (86.1000,65.9000) -- (86.1000,65.9000) -- (86.1000,65.9000) -- (86.1000,65.9000) -- (86.1000,65.9000) -- (86.1000,65.9000) -- (86.1000,65.9000) -- (86.1000,65.9000) -- (86.1000,65.9000) -- (86.1000,65.9000) -- (86.1000,65.9000) -- (86.1000,65.9000) -- (86.1000,65.9000) -- (86.1000,65.9000) -- (86.1000,65.9000) -- (86.1000,65.9000) -- (86.1000,65.9000) -- (86.1000,65.9000) -- (86.1000,65.9000) -- (86.1000,65.8000) -- (86.1000,65.8000) -- (86.1000,65.8000) -- (86.1000,65.8000) -- (86.1000,65.8000) -- (86.1000,65.8000) -- (86.1000,65.8000) -- (86.1000,65.8000) -- (86.1000,65.8000) -- (86.1000,65.8000) -- (86.1000,65.8000) -- (86.1000,65.8000) -- (86.1000,65.8000) -- (86.1000,65.8000) -- (86.1000,65.8000) -- (86.1000,65.8000) -- (86.1000,65.8000) -- (86.1000,65.8000) -- (86.1000,65.8000) -- (86.1000,65.8000) -- (86.1000,65.8000) -- (86.1000,65.7000) -- (86.1000,65.7000) -- (86.1000,65.7000) -- (86.1000,65.7000) -- (86.2000,65.7000) -- (86.2000,65.7000) -- (86.2000,65.7000) -- (86.2000,65.7000) -- (86.2000,65.7000) -- (86.2000,65.7000) -- (86.2000,65.7000) -- (86.2000,65.7000) -- (86.2000,65.7000) -- (86.2000,65.7000) -- (86.2000,65.7000) -- (86.2000,65.7000) -- (86.2000,65.7000) -- (86.2000,65.7000) -- (86.2000,65.7000) -- (86.2000,65.7000) -- (86.2000,65.7000) -- (86.2000,65.6000) -- (86.2000,65.6000) -- (86.2000,65.6000) -- (86.2000,65.6000) -- (86.2000,65.6000) -- (86.2000,65.6000) -- (86.2000,65.6000) -- (86.2000,65.6000) -- (86.2000,65.6000) -- (86.2000,65.6000) -- (86.2000,65.6000) -- (86.2000,65.6000) -- (86.2000,65.6000) -- (86.2000,65.6000) -- (86.2000,65.6000) -- (86.2000,65.6000) -- (86.2000,65.6000) -- (86.2000,65.6000) -- (86.2000,65.6000) -- (86.2000,65.6000) -- (86.2000,65.6000) -- (86.2000,65.5000) -- (86.2000,65.5000) -- (86.2000,65.5000) -- (86.2000,65.5000) -- (86.2000,65.5000) -- (86.2000,65.5000) -- (86.2000,65.5000) -- (86.2000,65.5000) -- (86.2000,65.5000) -- (86.2000,65.5000) -- (86.2000,65.5000) -- (86.2000,65.5000) -- (86.3000,65.5000) -- (86.3000,65.5000) -- (86.3000,65.5000) -- (86.3000,65.5000) -- (86.3000,65.5000) -- (86.3000,65.5000) -- (86.3000,65.5000) -- (86.3000,65.5000) -- (86.3000,65.4000) -- (86.3000,65.4000) -- (86.3000,65.4000) -- (86.3000,65.4000) -- (86.3000,65.4000) -- (86.3000,65.4000) -- (86.3000,65.4000) -- (86.3000,65.4000) -- (86.3000,65.4000) -- (86.3000,65.4000) -- (86.3000,65.4000) -- (86.3000,65.4000) -- (86.3000,65.4000) -- (86.3000,65.4000) -- (86.3000,65.4000) -- (86.3000,65.4000) -- (86.3000,65.4000) -- (86.3000,65.4000) -- (86.3000,65.4000) -- (86.3000,65.4000) -- (86.3000,65.4000) -- (86.3000,65.3000) -- (86.3000,65.3000) -- (86.3000,65.3000) -- (86.3000,65.3000) -- (86.3000,65.3000) -- (86.3000,65.3000) -- (86.3000,65.3000) -- (86.3000,65.3000) -- (86.3000,65.3000) -- (86.3000,65.3000) -- (86.3000,65.3000) -- (86.3000,65.3000) -- (86.3000,65.3000) -- (86.3000,65.3000) -- (86.3000,65.3000) -- (86.3000,65.3000) -- (86.3000,65.3000) -- (86.3000,65.3000) -- (86.3000,65.3000) -- (86.3000,65.3000) -- (86.4000,65.3000) -- (86.4000,65.2000) -- (86.4000,65.2000) -- (86.4000,65.2000) -- (86.4000,65.2000) -- (86.4000,65.2000) -- (86.4000,65.2000) -- (86.4000,65.2000) -- (86.4000,65.2000) -- (86.4000,65.2000) -- (86.4000,65.2000) -- (86.4000,65.2000) -- (86.4000,65.2000) -- (86.4000,65.2000) -- (86.4000,65.2000) -- (86.4000,65.2000) -- (86.4000,65.2000) -- (86.4000,65.2000) -- (86.4000,65.2000) -- (86.4000,65.2000) -- (86.4000,65.2000) -- (86.4000,65.2000) -- (86.4000,65.1000) -- (86.4000,65.1000) -- (86.4000,65.1000) -- (86.4000,65.1000) -- (86.4000,65.1000) -- (86.4000,65.1000) -- (86.4000,65.1000) -- (86.4000,65.1000) -- (86.4000,65.1000) -- (86.4000,65.1000) -- (86.4000,65.1000) -- (86.4000,65.1000) -- (86.4000,65.1000) -- (86.4000,65.1000) -- (86.4000,65.1000) -- (86.4000,65.1000) -- (86.4000,65.1000) -- (86.4000,65.1000) -- (86.4000,65.1000) -- (86.4000,65.1000) -- (86.4000,65.1000) -- (86.4000,65.0000) -- (86.4000,65.0000) -- (86.4000,65.0000) -- (86.4000,65.0000) -- (86.4000,65.0000) -- (86.4000,65.0000) -- (86.4000,65.0000) -- (86.5000,65.0000) -- (86.5000,65.0000) -- (86.5000,65.0000) -- (86.5000,65.0000) -- (86.5000,65.0000) -- (86.5000,65.0000) -- (86.5000,65.0000) -- (86.5000,65.0000) -- (86.5000,65.0000) -- (86.5000,65.0000) -- (86.5000,65.0000) -- (86.5000,65.0000) -- (86.5000,65.0000) -- (86.5000,64.9000) -- (86.5000,64.9000) -- (86.5000,64.9000) -- (86.5000,64.9000) -- (86.5000,64.9000) -- (86.5000,64.9000) -- (86.5000,64.9000) -- (86.5000,64.9000) -- (86.5000,64.9000) -- (86.5000,64.9000) -- (86.5000,64.9000) -- (86.5000,64.9000) -- (86.5000,64.9000) -- (86.5000,64.9000) -- (86.5000,64.9000) -- (86.5000,64.9000) -- (86.5000,64.9000) -- (86.5000,64.9000) -- (86.5000,64.9000) -- (86.5000,64.9000) -- (86.5000,64.9000) -- (86.5000,64.8000) -- (86.5000,64.8000) -- (86.5000,64.8000) -- (86.5000,64.8000) -- (86.5000,64.8000) -- (86.5000,64.8000) -- (86.5000,64.8000) -- (86.5000,64.8000) -- (86.5000,64.8000) -- (86.5000,64.8000) -- (86.5000,64.8000) -- (86.5000,64.8000) -- (86.5000,64.8000) -- (86.5000,64.8000) -- (86.5000,64.8000) -- (86.6000,64.8000) -- (86.6000,64.8000) -- (86.6000,64.8000) -- (86.6000,64.8000) -- (86.6000,64.8000) -- (86.6000,64.8000) -- (86.6000,64.7000) -- (86.6000,64.7000) -- (86.6000,64.7000) -- (86.6000,64.7000) -- (86.6000,64.7000) -- (86.6000,64.7000) -- (86.6000,64.7000) -- (86.6000,64.7000) -- (86.6000,64.7000) -- (86.6000,64.7000) -- (86.6000,64.7000) -- (86.6000,64.7000) -- (86.6000,64.7000) -- (86.6000,64.7000) -- (86.6000,64.7000) -- (86.6000,64.7000) -- (86.6000,64.7000) -- (86.6000,64.7000) -- (86.6000,64.7000) -- (86.6000,64.7000) -- (86.6000,64.7000) -- (86.6000,64.6000) -- (86.6000,64.6000) -- (86.6000,64.6000) -- (86.6000,64.6000) -- (86.6000,64.6000) -- (86.6000,64.6000) -- (86.6000,64.6000) -- (86.6000,64.6000) -- (86.6000,64.6000) -- (86.6000,64.6000) -- (86.6000,64.6000) -- (86.6000,64.6000) -- (86.6000,64.6000) -- (86.6000,64.6000) -- (86.6000,64.6000) -- (86.6000,64.6000) -- (86.6000,64.6000) -- (86.6000,64.6000) -- (86.6000,64.6000) -- (86.6000,64.6000) -- (86.6000,64.6000) -- (86.6000,64.5000) -- (86.6000,64.5000) -- (86.7000,64.5000) -- (86.7000,64.5000) -- (86.7000,64.5000) -- (86.7000,64.5000) -- (86.7000,64.5000) -- (86.7000,64.5000) -- (86.7000,64.5000) -- (86.7000,64.5000) -- (86.7000,64.5000) -- (86.7000,64.5000) -- (86.7000,64.5000) -- (86.7000,64.5000) -- (86.7000,64.5000) -- (86.7000,64.5000) -- (86.7000,64.5000) -- (86.7000,64.5000) -- (86.7000,64.5000) -- (86.7000,64.5000) -- (86.7000,64.4000) -- (86.7000,64.4000) -- (86.7000,64.4000) -- (86.7000,64.4000) -- (86.7000,64.4000) -- (86.7000,64.4000) -- (86.7000,64.4000) -- (86.7000,64.4000) -- (86.7000,64.4000) -- (86.7000,64.4000) -- (86.7000,64.4000) -- (86.7000,64.4000) -- (86.7000,64.4000) -- (86.7000,64.4000) -- (86.7000,64.4000) -- (86.7000,64.4000) -- (86.7000,64.4000) -- (86.7000,64.4000) -- (86.7000,64.4000) -- (86.7000,64.4000) -- (86.7000,64.4000) -- (86.7000,64.3000) -- (86.7000,64.3000) -- (86.7000,64.3000) -- (86.7000,64.3000) -- (86.7000,64.3000) -- (86.7000,64.3000) -- (86.7000,64.3000) -- (86.7000,64.3000) -- (86.7000,64.3000) -- (86.7000,64.3000) -- (86.8000,64.3000) -- (86.8000,64.3000) -- (86.8000,64.3000) -- (86.8000,64.3000) -- (86.8000,64.3000) -- (86.8000,64.3000) -- (86.8000,64.3000) -- (86.8000,64.3000) -- (86.8000,64.3000) -- (86.8000,64.3000) -- (86.8000,64.3000) -- (86.8000,64.2000) -- (86.8000,64.2000) -- (86.8000,64.2000) -- (86.8000,64.2000) -- (86.8000,64.2000) -- (86.8000,64.2000) -- (86.8000,64.2000) -- (86.8000,64.2000) -- (86.8000,64.2000) -- (86.8000,64.2000) -- (86.8000,64.2000) -- (86.8000,64.2000) -- (86.8000,64.2000) -- (86.8000,64.2000) -- (86.8000,64.2000) -- (86.8000,64.2000) -- (86.8000,64.2000) -- (86.8000,64.2000) -- (86.8000,64.2000) -- (86.8000,64.2000) -- (86.8000,64.2000) -- (86.8000,64.1000) -- (86.8000,64.1000) -- (86.8000,64.1000) -- (86.8000,64.1000) -- (86.8000,64.1000) -- (86.8000,64.1000) -- (86.8000,64.1000) -- (86.8000,64.1000) -- (86.8000,64.1000) -- (86.8000,64.1000) -- (86.8000,64.1000) -- (86.8000,64.1000) -- (86.8000,64.1000) -- (86.8000,64.1000) -- (86.8000,64.1000) -- (86.8000,64.1000) -- (86.8000,64.1000) -- (86.8000,64.1000) -- (86.9000,64.1000) -- (86.9000,64.1000) -- (86.9000,64.0000) -- (86.9000,64.0000) -- (86.9000,64.0000) -- (86.9000,64.0000) -- (86.9000,64.0000) -- (86.9000,64.0000) -- (86.9000,64.0000) -- (86.9000,64.0000) -- (86.9000,64.0000) -- (86.9000,64.0000) -- (86.9000,64.0000) -- (86.9000,64.0000) -- (86.9000,64.0000) -- (86.9000,64.0000) -- (86.9000,64.0000) -- (86.9000,64.0000) -- (86.9000,64.0000) -- (86.9000,64.0000) -- (86.9000,64.0000) -- (86.9000,64.0000) -- (86.9000,64.0000) -- (86.9000,63.9000) -- (86.9000,63.9000) -- (86.9000,63.9000) -- (86.9000,63.9000) -- (86.9000,63.9000) -- (86.9000,63.9000) -- (86.9000,63.9000) -- (86.9000,63.9000) -- (86.9000,63.9000) -- (86.9000,63.9000) -- (86.9000,63.9000) -- (86.9000,63.9000) -- (86.9000,63.9000) -- (86.9000,63.9000) -- (86.9000,63.9000) -- (86.9000,63.9000) -- (86.9000,63.9000) -- (86.9000,63.9000) -- (86.9000,63.9000) -- (86.9000,63.9000) -- (86.9000,63.9000) -- (86.9000,63.8000) -- (86.9000,63.8000) -- (86.9000,63.8000) -- (86.9000,63.8000) -- (86.9000,63.8000) -- (87.0000,63.8000) -- (87.0000,63.8000) -- (87.0000,63.8000) -- (87.0000,63.8000) -- (87.0000,63.8000) -- (87.0000,63.8000) -- (87.0000,63.8000) -- (87.0000,63.8000) -- (87.0000,63.8000) -- (87.0000,63.8000) -- (87.0000,63.8000) -- (87.0000,63.8000) -- (87.0000,63.8000) -- (87.0000,63.8000) -- (87.0000,63.8000) -- (87.0000,63.8000) -- (87.0000,63.7000) -- (87.0000,63.7000) -- (87.0000,63.7000) -- (87.0000,63.7000) -- (87.0000,63.7000) -- (87.0000,63.7000) -- (87.0000,63.7000) -- (87.0000,63.7000) -- (87.0000,63.7000) -- (87.0000,63.7000) -- (87.0000,63.7000) -- (87.0000,63.7000) -- (87.0000,63.7000) -- (87.0000,63.7000) -- (87.0000,63.7000) -- (87.0000,63.7000) -- (87.0000,63.7000) -- (87.0000,63.7000) -- (87.0000,63.7000) -- (87.0000,63.7000) -- (87.0000,63.7000) -- (87.0000,63.6000) -- (87.0000,63.6000) -- (87.0000,63.6000) -- (87.0000,63.6000) -- (87.0000,63.6000) -- (87.0000,63.6000) -- (87.0000,63.6000) -- (87.0000,63.6000) -- (87.0000,63.6000) -- (87.0000,63.6000) -- (87.0000,63.6000) -- (87.0000,63.6000) -- (87.0000,63.6000) -- (87.1000,63.6000) -- (87.1000,63.6000) -- (87.1000,63.6000) -- (87.1000,63.6000) -- (87.1000,63.6000) -- (87.1000,63.6000) -- (87.1000,63.6000) -- (87.1000,63.5000) -- (87.1000,63.5000) -- (87.1000,63.5000) -- (87.1000,63.5000) -- (87.1000,63.5000) -- (87.1000,63.5000) -- (87.1000,63.5000) -- (87.1000,63.5000) -- (87.1000,63.5000) -- (87.1000,63.5000) -- (87.1000,63.5000) -- (87.1000,63.5000) -- (87.1000,63.5000) -- (87.1000,63.5000) -- (87.1000,63.5000) -- (87.1000,63.5000) -- (87.1000,63.5000) -- (87.1000,63.5000) -- (87.1000,63.5000) -- (87.1000,63.5000) -- (87.1000,63.5000) -- (87.1000,63.4000) -- (87.1000,63.4000) -- (87.1000,63.4000) -- (87.1000,63.4000) -- (87.1000,63.4000) -- (87.1000,63.4000) -- (87.1000,63.4000) -- (87.1000,63.4000) -- (87.1000,63.4000) -- (87.1000,63.4000) -- (87.1000,63.4000) -- (87.1000,63.4000) -- (87.1000,63.4000) -- (87.1000,63.4000) -- (87.1000,63.4000) -- (87.1000,63.4000) -- (87.1000,63.4000) -- (87.1000,63.4000) -- (87.1000,63.4000) -- (87.1000,63.4000) -- (87.1000,63.4000) -- (87.2000,63.3000) -- (87.2000,63.3000) -- (87.2000,63.3000) -- (87.2000,63.3000) -- (87.2000,63.3000) -- (87.2000,63.3000) -- (87.2000,63.3000) -- (87.2000,63.3000) -- (87.2000,63.3000) -- (87.2000,63.3000) -- (87.2000,63.3000) -- (87.2000,63.3000) -- (87.2000,63.3000) -- (87.2000,63.3000) -- (87.2000,63.3000) -- (87.2000,63.3000) -- (87.2000,63.3000) -- (87.2000,63.3000) -- (87.2000,63.3000) -- (87.2000,63.3000) -- (87.2000,63.3000) -- (87.2000,63.2000) -- (87.2000,63.2000) -- (87.2000,63.2000) -- (87.2000,63.2000) -- (87.2000,63.2000) -- (87.2000,63.2000) -- (87.2000,63.2000) -- (87.2000,63.2000) -- (87.2000,63.2000) -- (87.2000,63.2000) -- (87.2000,63.2000) -- (87.2000,63.2000) -- (87.2000,63.2000) -- (87.2000,63.2000) -- (87.2000,63.2000) -- (87.2000,63.2000) -- (87.2000,63.2000) -- (87.2000,63.2000) -- (87.2000,63.2000) -- (87.2000,63.2000) -- (87.2000,63.2000) -- (87.2000,63.1000) -- (87.2000,63.1000) -- (87.2000,63.1000) -- (87.2000,63.1000) -- (87.2000,63.1000) -- (87.2000,63.1000) -- (87.2000,63.1000) -- (87.2000,63.1000) -- (87.3000,63.1000) -- (87.3000,63.1000) -- (87.3000,63.1000) -- (87.3000,63.1000) -- (87.3000,63.1000) -- (87.3000,63.1000) -- (87.3000,63.1000) -- (87.3000,63.1000) -- (87.3000,63.1000) -- (87.3000,63.1000) -- (87.3000,63.1000) -- (87.3000,63.1000) -- (87.3000,63.0000) -- (87.3000,63.0000) -- (87.3000,63.0000) -- (87.3000,63.0000) -- (87.3000,63.0000) -- (87.3000,63.0000) -- (87.3000,63.0000) -- (87.3000,63.0000) -- (87.3000,63.0000) -- (87.3000,63.0000) -- (87.3000,63.0000) -- (87.3000,63.0000) -- (87.3000,63.0000) -- (87.3000,63.0000) -- (87.3000,63.0000) -- (87.3000,63.0000) -- (87.3000,63.0000) -- (87.3000,63.0000) -- (87.3000,63.0000) -- (87.3000,63.0000) -- (87.3000,63.0000) -- (87.3000,62.9000) -- (87.3000,62.9000) -- (87.3000,62.9000) -- (87.3000,62.9000) -- (87.3000,62.9000) -- (87.3000,62.9000) -- (87.3000,62.9000) -- (87.3000,62.9000) -- (87.3000,62.9000) -- (87.3000,62.9000) -- (87.3000,62.9000) -- (87.3000,62.9000) -- (87.3000,62.9000) -- (87.3000,62.9000) -- (87.3000,62.9000) -- (87.3000,62.9000) -- (87.4000,62.9000) -- (87.4000,62.9000) -- (87.4000,62.9000) -- (87.4000,62.9000) -- (87.4000,62.9000) -- (87.4000,62.8000) -- (87.4000,62.8000) -- (87.4000,62.8000) -- (87.4000,62.8000) -- (87.4000,62.8000) -- (87.4000,62.8000) -- (87.4000,62.8000) -- (87.4000,62.8000) -- (87.4000,62.8000) -- (87.4000,62.8000) -- (87.4000,62.8000) -- (87.4000,62.8000) -- (87.4000,62.8000) -- (87.4000,62.8000) -- (87.4000,62.8000) -- (87.4000,62.8000) -- (87.4000,62.8000) -- (87.4000,62.8000) -- (87.4000,62.8000) -- (87.4000,62.8000) -- (87.4000,62.8000) -- (87.4000,62.7000) -- (87.4000,62.7000) -- (87.4000,62.7000) -- (87.4000,62.7000) -- (87.4000,62.7000) -- (87.4000,62.7000) -- (87.4000,62.7000) -- (87.4000,62.7000) -- (87.4000,62.7000) -- (87.4000,62.7000) -- (87.4000,62.7000) -- (87.4000,62.7000) -- (87.4000,62.7000) -- (87.4000,62.7000) -- (87.4000,62.7000) -- (87.4000,62.7000) -- (87.4000,62.7000) -- (87.4000,62.7000) -- (87.4000,62.7000) -- (87.4000,62.7000) -- (87.4000,62.7000) -- (87.4000,62.6000) -- (87.4000,62.6000) -- (87.4000,62.6000) -- (87.5000,62.6000) -- (87.5000,62.6000) -- (87.5000,62.6000) -- (87.5000,62.6000) -- (87.5000,62.6000) -- (87.5000,62.6000) -- (87.5000,62.6000) -- (87.5000,62.6000) -- (87.5000,62.6000) -- (87.5000,62.6000) -- (87.5000,62.6000) -- (87.5000,62.6000) -- (87.5000,62.6000) -- (87.5000,62.6000) -- (87.5000,62.6000) -- (87.5000,62.6000) -- (87.5000,62.6000) -- (87.5000,62.5000) -- (87.5000,62.5000) -- (87.5000,62.5000) -- (87.5000,62.5000) -- (87.5000,62.5000) -- (87.5000,62.5000) -- (87.5000,62.5000) -- (87.5000,62.5000) -- (87.5000,62.5000) -- (87.5000,62.5000) -- (87.5000,62.5000) -- (87.5000,62.5000) -- (87.5000,62.5000) -- (87.5000,62.5000) -- (87.5000,62.5000) -- (87.5000,62.5000) -- (87.5000,62.5000) -- (87.5000,62.5000) -- (87.5000,62.5000) -- (87.5000,62.5000) -- (87.5000,62.5000) -- (87.5000,62.4000) -- (87.5000,62.4000) -- (87.5000,62.4000) -- (87.5000,62.4000) -- (87.5000,62.4000) -- (87.5000,62.4000) -- (87.5000,62.4000) -- (87.5000,62.4000) -- (87.5000,62.4000) -- (87.5000,62.4000) -- (87.5000,62.4000) -- (87.6000,62.4000) -- (87.6000,62.4000) -- (87.6000,62.4000) -- (87.6000,62.4000) -- (87.6000,62.4000) -- (87.6000,62.4000) -- (87.6000,62.4000) -- (87.6000,62.4000) -- (87.6000,62.4000) -- (87.6000,62.4000) -- (87.6000,62.3000) -- (87.6000,62.3000) -- (87.6000,62.3000) -- (87.6000,62.3000) -- (87.6000,62.3000) -- (87.6000,62.3000) -- (87.6000,62.3000) -- (87.6000,62.3000) -- (87.6000,62.3000) -- (87.6000,62.3000) -- (87.6000,62.3000) -- (87.6000,62.3000) -- (87.6000,62.3000) -- (87.6000,62.3000) -- (87.6000,62.3000) -- (87.6000,62.3000) -- (87.6000,62.3000) -- (87.6000,62.3000) -- (87.6000,62.3000) -- (87.6000,62.3000) -- (87.6000,62.3000) -- (87.6000,62.2000) -- (87.6000,62.2000) -- (87.6000,62.2000) -- (87.6000,62.2000) -- (87.6000,62.2000) -- (87.6000,62.2000) -- (87.6000,62.2000) -- (87.6000,62.2000) -- (87.6000,62.2000) -- (87.6000,62.2000) -- (87.6000,62.2000) -- (87.6000,62.2000) -- (87.6000,62.2000) -- (87.6000,62.2000) -- (87.6000,62.2000) -- (87.6000,62.2000) -- (87.6000,62.2000) -- (87.6000,62.2000) -- (87.6000,62.2000) -- (87.7000,62.2000) -- (87.7000,62.1000) -- (87.7000,62.1000) -- (87.7000,62.1000) -- (87.7000,62.1000) -- (87.7000,62.1000) -- (87.7000,62.1000) -- (87.7000,62.1000) -- (87.7000,62.1000) -- (87.7000,62.1000) -- (87.7000,62.1000) -- (87.7000,62.1000) -- (87.7000,62.1000) -- (87.7000,62.1000) -- (87.7000,62.1000) -- (87.7000,62.1000) -- (87.7000,62.1000) -- (87.7000,62.1000) -- (87.7000,62.1000) -- (87.7000,62.1000) -- (87.7000,62.1000) -- (87.7000,62.1000) -- (87.7000,62.0000) -- (87.7000,62.0000) -- (87.7000,62.0000) -- (87.7000,62.0000) -- (87.7000,62.0000) -- (87.7000,62.0000) -- (87.7000,62.0000) -- (87.7000,62.0000) -- (87.7000,62.0000) -- (87.7000,62.0000) -- (87.7000,62.0000) -- (87.7000,62.0000) -- (87.7000,62.0000) -- (87.7000,62.0000) -- (87.7000,62.0000) -- (87.7000,62.0000) -- (87.7000,62.0000) -- (87.7000,62.0000) -- (87.7000,62.0000) -- (87.7000,62.0000) -- (87.7000,62.0000) -- (87.7000,61.9000) -- (87.7000,61.9000) -- (87.7000,61.9000) -- (87.7000,61.9000) -- (87.7000,61.9000) -- (87.7000,61.9000) -- (87.8000,61.9000) -- (87.8000,61.9000) -- (87.8000,61.9000) -- (87.8000,61.9000) -- (87.8000,61.9000) -- (87.8000,61.9000) -- (87.8000,61.9000) -- (87.8000,61.9000) -- (87.8000,61.9000) -- (87.8000,61.9000) -- (87.8000,61.9000) -- (87.8000,61.9000) -- (87.8000,61.9000) -- (87.8000,61.9000) -- (87.8000,61.9000) -- (87.8000,61.8000) -- (87.8000,61.8000) -- (87.8000,61.8000) -- (87.8000,61.8000) -- (87.8000,61.8000) -- (87.8000,61.8000) -- (87.8000,61.8000) -- (87.8000,61.8000) -- (87.8000,61.8000) -- (87.8000,61.8000) -- (87.8000,61.8000) -- (87.8000,61.8000) -- (87.8000,61.8000) -- (87.8000,61.8000) -- (87.8000,61.8000) -- (87.8000,61.8000) -- (87.8000,61.8000) -- (87.8000,61.8000) -- (87.8000,61.8000) -- (87.8000,61.8000) -- (87.8000,61.8000) -- (87.8000,61.7000) -- (87.8000,61.7000) -- (87.8000,61.7000) -- (87.8000,61.7000) -- (87.8000,61.7000) -- (87.8000,61.7000) -- (87.8000,61.7000) -- (87.8000,61.7000) -- (87.8000,61.7000) -- (87.8000,61.7000) -- (87.8000,61.7000) -- (87.8000,61.7000) -- (87.8000,61.7000) -- (87.8000,61.7000) -- (87.9000,61.7000) -- (87.9000,61.7000) -- (87.9000,61.7000) -- (87.9000,61.7000) -- (87.9000,61.7000) -- (87.9000,61.7000) -- (87.9000,61.6000) -- (87.9000,61.6000) -- (87.9000,61.6000) -- (87.9000,61.6000) -- (87.9000,61.6000) -- (87.9000,61.6000) -- (87.9000,61.6000) -- (87.9000,61.6000) -- (87.9000,61.6000) -- (87.9000,61.6000) -- (87.9000,61.6000) -- (87.9000,61.6000) -- (87.9000,61.6000) -- (87.9000,61.6000) -- (87.9000,61.6000) -- (87.9000,61.6000) -- (87.9000,61.6000) -- (87.9000,61.6000) -- (87.9000,61.6000) -- (87.9000,61.6000) -- (87.9000,61.6000) -- (87.9000,61.5000) -- (87.9000,61.5000) -- (87.9000,61.5000) -- (87.9000,61.5000) -- (87.9000,61.5000) -- (87.9000,61.5000) -- (87.9000,61.5000) -- (87.9000,61.5000) -- (87.9000,61.5000) -- (87.9000,61.5000) -- (87.9000,61.5000) -- (87.9000,61.5000) -- (87.9000,61.5000) -- (87.9000,61.5000) -- (87.9000,61.5000) -- (87.9000,61.5000) -- (87.9000,61.5000) -- (87.9000,61.5000) -- (87.9000,61.5000) -- (87.9000,61.5000) -- (87.9000,61.5000) -- (87.9000,61.4000) -- (88.0000,61.4000) -- (88.0000,61.4000) -- (88.0000,61.4000) -- (88.0000,61.4000) -- (88.0000,61.4000) -- (88.0000,61.4000) -- (88.0000,61.4000) -- (88.0000,61.4000) -- (88.0000,61.4000) -- (88.0000,61.4000) -- (88.0000,61.4000) -- (88.0000,61.4000) -- (88.0000,61.4000) -- (88.0000,61.4000) -- (88.0000,61.4000) -- (88.0000,61.4000) -- (88.0000,61.4000) -- (88.0000,61.4000) -- (88.0000,61.4000) -- (88.0000,61.4000) -- (88.0000,61.3000) -- (88.0000,61.3000) -- (88.0000,61.3000) -- (88.0000,61.3000) -- (88.0000,61.3000) -- (88.0000,61.3000) -- (88.0000,61.3000) -- (88.0000,61.3000) -- (88.0000,61.3000) -- (88.0000,61.3000) -- (88.0000,61.3000) -- (88.0000,61.3000) -- (88.0000,61.3000) -- (88.0000,61.3000) -- (88.0000,61.3000) -- (88.0000,61.3000) -- (88.0000,61.3000) -- (88.0000,61.3000) -- (88.0000,61.3000) -- (88.0000,61.3000) -- (88.0000,61.3000) -- (88.0000,61.2000) -- (88.0000,61.2000) -- (88.0000,61.2000) -- (88.0000,61.2000) -- (88.0000,61.2000) -- (88.0000,61.2000) -- (88.0000,61.2000) -- (88.0000,61.2000) -- (88.0000,61.2000) -- (88.1000,61.2000) -- (88.1000,61.2000) -- (88.1000,61.2000) -- (88.1000,61.2000) -- (88.1000,61.2000) -- (88.1000,61.2000) -- (88.1000,61.2000) -- (88.1000,61.2000) -- (88.1000,61.2000) -- (88.1000,61.2000) -- (88.1000,61.2000) -- (88.1000,61.1000) -- (88.1000,61.1000) -- (88.1000,61.1000) -- (88.1000,61.1000) -- (88.1000,61.1000) -- (88.1000,61.1000) -- (88.1000,61.1000) -- (88.1000,61.1000) -- (88.1000,61.1000) -- (88.1000,61.1000) -- (88.1000,61.1000) -- (88.1000,61.1000) -- (88.1000,61.1000) -- (88.1000,61.1000) -- (88.1000,61.1000) -- (88.1000,61.1000) -- (88.1000,61.1000) -- (88.1000,61.1000) -- (88.1000,61.1000) -- (88.1000,61.1000) -- (88.1000,61.1000) -- (88.1000,61.0000) -- (88.1000,61.0000) -- (88.1000,61.0000) -- (88.1000,61.0000) -- (88.1000,61.0000) -- (88.1000,61.0000) -- (88.1000,61.0000) -- (88.1000,61.0000) -- (88.1000,61.0000) -- (88.1000,61.0000) -- (88.1000,61.0000) -- (88.1000,61.0000) -- (88.1000,61.0000) -- (88.1000,61.0000) -- (88.1000,61.0000) -- (88.1000,61.0000) -- (88.1000,61.0000) -- (88.2000,61.0000) -- (88.2000,61.0000) -- (88.2000,61.0000) -- (88.2000,61.0000) -- (88.2000,60.9000) -- (88.2000,60.9000) -- (88.2000,60.9000) -- (88.2000,60.9000) -- (88.2000,60.9000) -- (88.2000,60.9000) -- (88.2000,60.9000) -- (88.2000,60.9000) -- (88.2000,60.9000) -- (88.2000,60.9000) -- (88.2000,60.9000) -- (88.2000,60.9000) -- (88.2000,60.9000) -- (88.2000,60.9000) -- (88.2000,60.9000) -- (88.2000,60.9000) -- (88.2000,60.9000) -- (88.2000,60.9000) -- (88.2000,60.9000) -- (88.2000,60.9000) -- (88.2000,60.9000) -- (88.2000,60.8000) -- (88.2000,60.8000) -- (88.2000,60.8000) -- (88.2000,60.8000) -- (88.2000,60.8000) -- (88.2000,60.8000) -- (88.2000,60.8000) -- (88.2000,60.8000) -- (88.2000,60.8000) -- (88.2000,60.8000) -- (88.2000,60.8000) -- (88.2000,60.8000) -- (88.2000,60.8000) -- (88.2000,60.8000) -- (88.2000,60.8000) -- (88.2000,60.8000) -- (88.2000,60.8000) -- (88.2000,60.8000) -- (88.2000,60.8000) -- (88.2000,60.8000) -- (88.2000,60.8000) -- (88.2000,60.7000) -- (88.2000,60.7000) -- (88.2000,60.7000) -- (88.2000,60.7000) -- (88.3000,60.7000) -- (88.3000,60.7000) -- (88.3000,60.7000) -- (88.3000,60.7000) -- (88.3000,60.7000) -- (88.3000,60.7000) -- (88.3000,60.7000) -- (88.3000,60.7000) -- (88.3000,60.7000) -- (88.3000,60.7000) -- (88.3000,60.7000) -- (88.3000,60.7000) -- (88.3000,60.7000) -- (88.3000,60.7000) -- (88.3000,60.7000) -- (88.3000,60.7000) -- (88.3000,60.6000) -- (88.3000,60.6000) -- (88.3000,60.6000) -- (88.3000,60.6000) -- (88.3000,60.6000) -- (88.3000,60.6000) -- (88.3000,60.6000) -- (88.3000,60.6000) -- (88.3000,60.6000) -- (88.3000,60.6000) -- (88.3000,60.6000) -- (88.3000,60.6000) -- (88.3000,60.6000) -- (88.3000,60.6000) -- (88.3000,60.6000) -- (88.3000,60.6000) -- (88.3000,60.6000) -- (88.3000,60.6000) -- (88.3000,60.6000) -- (88.3000,60.6000) -- (88.3000,60.6000) -- (95.7000,60.5000) -- (95.7000,60.5000) -- (95.7000,60.5000) -- (95.7000,60.5000) -- (95.7000,60.5000) -- (95.7000,60.5000) -- (95.7000,60.5000) -- (95.7000,60.5000) -- (95.7000,60.5000) -- (95.7000,60.5000) -- (95.7000,60.5000) -- (95.7000,60.5000) -- (95.7000,60.5000) -- (95.7000,60.5000) -- (95.7000,60.5000) -- (95.7000,60.5000) -- (95.7000,60.5000) -- (95.7000,60.5000) -- (95.7000,60.5000) -- (95.7000,60.5000) -- (95.7000,60.5000) -- (95.7000,60.4000) -- (95.7000,60.4000) -- (95.7000,60.4000) -- (95.7000,60.4000) -- (95.7000,60.4000) -- (95.7000,60.4000) -- (95.7000,60.4000) -- (95.7000,60.4000) -- (95.7000,60.4000) -- (95.7000,60.4000) -- (95.7000,60.4000) -- (95.7000,60.4000) -- (95.7000,60.4000) -- (95.7000,60.4000) -- (95.7000,60.4000) -- (95.7000,60.4000) -- (95.7000,60.4000) -- (95.7000,60.4000) -- (95.7000,60.4000) -- (95.7000,60.4000) -- (95.7000,60.4000) -- (95.7000,60.3000) -- (95.7000,60.3000) -- (95.7000,60.3000) -- (95.7000,60.3000) -- (95.7000,60.3000) -- (95.7000,60.3000) -- (95.8000,60.3000) -- (95.8000,60.3000) -- (95.8000,60.3000) -- (95.8000,60.3000) -- (95.8000,60.3000) -- (95.8000,60.3000) -- (95.8000,60.3000) -- (95.8000,60.3000) -- (95.8000,60.3000) -- (95.8000,60.3000) -- (95.8000,60.3000) -- (95.8000,60.3000) -- (95.8000,60.3000) -- (95.8000,60.3000) -- (95.8000,60.2000) -- (95.8000,60.2000) -- (95.8000,60.2000) -- (95.8000,60.2000) -- (95.8000,60.2000) -- (95.8000,60.2000) -- (95.8000,60.2000) -- (95.8000,60.2000) -- (95.8000,60.2000) -- (95.8000,60.2000) -- (95.8000,60.2000) -- (95.8000,60.2000) -- (95.8000,60.2000) -- (95.8000,60.2000) -- (95.8000,60.2000) -- (95.8000,60.2000) -- (95.8000,60.2000) -- (95.8000,60.2000) -- (95.8000,60.2000) -- (95.8000,60.2000) -- (95.8000,60.2000) -- (95.8000,60.1000) -- (95.8000,60.1000) -- (95.8000,60.1000) -- (95.8000,60.1000) -- (95.8000,60.1000) -- (95.8000,60.1000) -- (95.8000,60.1000) -- (95.8000,60.1000) -- (95.8000,60.1000) -- (95.8000,60.1000) -- (95.8000,60.1000) -- (95.8000,60.1000) -- (95.8000,60.1000) -- (95.8000,60.1000) -- (95.8000,60.1000) -- (95.9000,60.1000) -- (95.9000,60.1000) -- (95.9000,60.1000) -- (95.9000,60.1000) -- (95.9000,60.1000) -- (95.9000,60.1000) -- (95.9000,60.0000) -- (95.9000,60.0000) -- (95.9000,60.0000) -- (95.9000,60.0000) -- (95.9000,60.0000) -- (95.9000,60.0000) -- (95.9000,60.0000) -- (95.9000,60.0000) -- (95.9000,60.0000) -- (95.9000,60.0000) -- (95.9000,60.0000) -- (95.9000,60.0000) -- (95.9000,60.0000) -- (95.9000,60.0000) -- (95.9000,60.0000) -- (95.9000,60.0000) -- (95.9000,60.0000) -- (95.9000,60.0000) -- (95.9000,60.0000) -- (95.9000,60.0000) -- (95.9000,60.0000) -- (95.9000,59.9000) -- (95.9000,59.9000) -- (95.9000,59.9000) -- (95.9000,59.9000) -- (95.9000,59.9000) -- (95.9000,59.9000) -- (95.9000,59.9000) -- (95.9000,59.9000) -- (95.9000,59.9000) -- (95.9000,59.9000) -- (95.9000,59.9000) -- (95.9000,59.9000) -- (95.9000,59.9000) -- (95.9000,59.9000) -- (95.9000,59.9000) -- (95.9000,59.9000) -- (95.9000,59.9000) -- (95.9000,59.9000) -- (95.9000,59.9000) -- (95.9000,59.9000) -- (95.9000,59.9000) -- (95.9000,59.8000) -- (96.0000,59.8000) -- (96.0000,59.8000) -- (96.0000,59.8000) -- (96.0000,59.8000) -- (96.0000,59.8000) -- (96.0000,59.8000) -- (96.0000,59.8000) -- (96.0000,59.8000) -- (96.0000,59.8000) -- (96.0000,59.8000) -- (96.0000,59.8000) -- (96.0000,59.8000) -- (96.0000,59.8000) -- (96.0000,59.8000) -- (96.0000,59.8000) -- (96.0000,59.8000) -- (96.0000,59.8000) -- (96.0000,59.8000) -- (96.0000,59.8000) -- (96.0000,59.7000) -- (96.0000,59.7000) -- (96.0000,59.7000) -- (96.0000,59.7000) -- (96.0000,59.7000) -- (96.0000,59.7000) -- (96.0000,59.7000) -- (96.0000,59.7000) -- (96.0000,59.7000) -- (96.0000,59.7000) -- (96.0000,59.7000) -- (96.0000,59.7000) -- (96.0000,59.7000) -- (96.0000,59.7000) -- (96.0000,59.7000) -- (96.0000,59.7000) -- (96.0000,59.7000) -- (96.0000,59.7000) -- (96.0000,59.7000) -- (96.0000,59.7000) -- (96.0000,59.7000) -- (96.0000,59.6000) -- (96.0000,59.6000) -- (96.0000,59.6000) -- (96.0000,59.6000) -- (96.0000,59.6000) -- (96.0000,59.6000) -- (96.0000,59.6000) -- (96.0000,59.6000) -- (96.0000,59.6000) -- (96.0000,59.6000) -- (96.1000,59.6000) -- (96.1000,59.6000) -- (96.1000,59.6000) -- (96.1000,59.6000) -- (96.1000,59.6000) -- (96.1000,59.6000) -- (96.1000,59.6000) -- (96.1000,59.6000) -- (96.1000,59.6000) -- (96.1000,59.6000) -- (96.1000,59.6000) -- (96.1000,59.5000) -- (96.1000,59.5000) -- (96.1000,59.5000) -- (96.1000,59.5000) -- (96.1000,59.5000) -- (96.1000,59.5000) -- (96.1000,59.5000) -- (96.1000,59.5000) -- (96.1000,59.5000) -- (96.1000,59.5000) -- (96.1000,59.5000) -- (96.1000,59.5000) -- (96.1000,59.5000) -- (96.1000,59.5000) -- (96.1000,59.5000) -- (96.1000,59.5000) -- (96.1000,59.5000) -- (96.1000,59.5000) -- (96.1000,59.5000) -- (96.1000,59.5000) -- (96.1000,59.5000) -- (96.1000,59.4000) -- (96.1000,59.4000) -- (96.1000,59.4000) -- (96.1000,59.4000) -- (96.1000,59.4000) -- (96.1000,59.4000) -- (96.1000,59.4000) -- (96.1000,59.4000) -- (96.1000,59.4000) -- (96.1000,59.4000) -- (96.1000,59.4000) -- (96.1000,59.4000) -- (96.1000,59.4000) -- (96.1000,59.4000) -- (96.1000,59.4000) -- (96.1000,59.4000) -- (96.1000,59.4000) -- (96.2000,59.4000) -- (96.2000,59.4000) -- (96.2000,59.4000) -- (96.2000,59.4000) -- (96.2000,59.3000) -- (96.2000,59.3000) -- (96.2000,59.3000) -- (96.2000,59.3000) -- (96.2000,59.3000) -- (96.2000,59.3000) -- (96.2000,59.3000) -- (96.2000,59.3000) -- (96.2000,59.3000) -- (96.2000,59.3000) -- (96.2000,59.3000) -- (96.2000,59.3000) -- (96.2000,59.3000) -- (96.2000,59.3000) -- (96.2000,59.3000) -- (96.2000,59.3000) -- (96.2000,59.3000) -- (96.2000,59.3000) -- (96.2000,59.3000) -- (96.2000,59.3000) -- (96.2000,59.2000) -- (96.2000,59.2000) -- (96.2000,59.2000) -- (96.2000,59.2000) -- (96.2000,59.2000) -- (96.2000,59.2000) -- (96.2000,59.2000) -- (96.2000,59.2000) -- (96.2000,59.2000) -- (96.2000,59.2000) -- (96.2000,59.2000) -- (96.2000,59.2000) -- (96.2000,59.2000) -- (96.2000,59.2000) -- (96.2000,59.2000) -- (96.2000,59.2000) -- (96.2000,59.2000) -- (96.2000,59.2000) -- (96.2000,59.2000) -- (96.2000,59.2000) -- (96.2000,59.2000) -- (96.2000,59.1000) -- (96.2000,59.1000) -- (96.2000,59.1000) -- (96.2000,59.1000) -- (96.2000,59.1000) -- (96.3000,59.1000) -- (96.3000,59.1000) -- (96.3000,59.1000) -- (96.3000,59.1000) -- (96.3000,59.1000) -- (96.3000,59.1000) -- (96.3000,59.1000) -- (96.3000,59.1000) -- (96.3000,59.1000) -- (96.3000,59.1000) -- (96.3000,59.1000) -- (96.3000,59.1000) -- (96.3000,59.1000) -- (96.3000,59.1000) -- (96.3000,59.1000) -- (96.3000,59.1000) -- (96.3000,59.0000) -- (96.3000,59.0000) -- (96.3000,59.0000) -- (96.3000,59.0000) -- (96.3000,59.0000) -- (96.3000,59.0000) -- (96.3000,59.0000) -- (96.3000,59.0000) -- (96.3000,59.0000) -- (96.3000,59.0000) -- (96.3000,59.0000) -- (96.3000,59.0000) -- (96.3000,59.0000) -- (96.3000,59.0000) -- (96.3000,59.0000) -- (96.3000,59.0000) -- (96.3000,59.0000) -- (96.3000,59.0000) -- (96.3000,59.0000) -- (96.3000,59.0000) -- (96.3000,59.0000) -- (96.3000,58.9000) -- (96.3000,58.9000) -- (96.3000,58.9000) -- (96.3000,58.9000) -- (96.3000,58.9000) -- (96.3000,58.9000) -- (96.3000,58.9000) -- (96.3000,58.9000) -- (96.3000,58.9000) -- (96.3000,58.9000) -- (96.3000,58.9000) -- (96.3000,58.9000) -- (96.3000,58.9000) -- (96.4000,58.9000) -- (96.4000,58.9000) -- (96.4000,58.9000) -- (96.4000,58.9000) -- (96.4000,58.9000) -- (96.4000,58.9000) -- (96.4000,58.9000) -- (96.4000,58.9000) -- (96.4000,58.8000) -- (96.4000,58.8000) -- (96.4000,58.8000) -- (96.4000,58.8000) -- (96.4000,58.8000) -- (96.4000,58.8000) -- (96.4000,58.8000) -- (96.4000,58.8000) -- (96.4000,58.8000) -- (96.4000,58.8000) -- (96.4000,58.8000) -- (96.4000,58.8000) -- (96.4000,58.8000) -- (96.4000,58.8000) -- (96.4000,58.8000) -- (96.4000,58.8000) -- (96.4000,58.8000) -- (96.4000,58.8000) -- (96.4000,58.8000) -- (96.4000,58.8000) -- (96.4000,58.7000) -- (96.4000,58.7000) -- (96.4000,58.7000) -- (96.4000,58.7000) -- (96.4000,58.7000) -- (96.4000,58.7000) -- (96.4000,58.7000) -- (96.4000,58.7000) -- (96.4000,58.7000) -- (96.4000,58.7000) -- (96.4000,58.7000) -- (96.4000,58.7000) -- (96.4000,58.7000) -- (96.4000,58.7000) -- (96.4000,58.7000) -- (96.4000,58.7000) -- (96.4000,58.7000) -- (96.4000,58.7000) -- (96.4000,58.7000) -- (96.4000,58.7000) -- (96.4000,58.7000) -- (96.5000,58.6000) -- (96.5000,58.6000) -- (96.5000,58.6000) -- (96.5000,58.6000) -- (96.5000,58.6000) -- (96.5000,58.6000) -- (96.5000,58.6000) -- (96.5000,58.6000) -- (96.5000,58.6000) -- (96.5000,58.6000) -- (96.5000,58.6000) -- (96.5000,58.6000) -- (96.5000,58.6000) -- (96.5000,58.6000) -- (96.5000,58.6000) -- (96.5000,58.6000) -- (96.5000,58.6000) -- (96.5000,58.6000) -- (96.5000,58.6000) -- (96.5000,58.6000) -- (96.5000,58.6000) -- (96.5000,58.5000) -- (96.5000,58.5000) -- (96.5000,58.5000) -- (96.5000,58.5000) -- (96.5000,58.5000) -- (96.5000,58.5000) -- (96.5000,58.5000) -- (96.5000,58.5000) -- (96.5000,58.5000) -- (96.5000,58.5000) -- (96.5000,58.5000) -- (96.5000,58.5000) -- (96.5000,58.5000) -- (96.5000,58.5000) -- (96.5000,58.5000) -- (96.5000,58.5000) -- (96.5000,58.5000) -- (96.5000,58.5000) -- (96.5000,58.5000) -- (96.5000,58.5000) -- (96.5000,58.5000) -- (96.5000,58.4000) -- (96.5000,58.4000) -- (96.5000,58.4000) -- (96.5000,58.4000) -- (96.5000,58.4000) -- (96.5000,58.4000) -- (96.5000,58.4000) -- (96.5000,58.4000) -- (96.6000,58.4000) -- (96.6000,58.4000) -- (96.6000,58.4000) -- (96.6000,58.4000) -- (96.6000,58.4000) -- (96.6000,58.4000) -- (96.6000,58.4000) -- (96.6000,58.4000) -- (96.6000,58.4000) -- (96.6000,58.4000) -- (96.6000,58.4000) -- (96.6000,58.4000) -- (96.6000,58.3000) -- (96.6000,58.3000) -- (96.6000,58.3000) -- (96.6000,58.3000) -- (96.6000,58.3000) -- (96.6000,58.3000) -- (96.6000,58.3000) -- (96.6000,58.3000) -- (96.6000,58.3000) -- (96.6000,58.3000) -- (96.6000,58.3000) -- (96.6000,58.3000) -- (96.6000,58.3000) -- (96.6000,58.3000) -- (96.6000,58.3000) -- (96.6000,58.3000) -- (96.6000,58.3000) -- (96.6000,58.3000) -- (96.6000,58.3000) -- (96.6000,58.3000) -- (96.6000,58.3000) -- (96.6000,58.2000) -- (96.6000,58.2000) -- (96.6000,58.2000) -- (96.6000,58.2000) -- (96.6000,58.2000) -- (96.6000,58.2000) -- (96.6000,58.2000) -- (96.6000,58.2000) -- (96.6000,58.2000) -- (96.6000,58.2000) -- (96.6000,58.2000) -- (96.6000,58.2000) -- (96.6000,58.2000) -- (96.6000,58.2000) -- (96.6000,58.2000) -- (96.6000,58.2000) -- (96.7000,58.2000) -- (96.7000,58.2000) -- (96.7000,58.2000) -- (96.7000,58.2000) -- (96.7000,58.2000) -- (96.7000,58.1000) -- (96.7000,58.1000) -- (96.7000,58.1000) -- (96.7000,58.1000) -- (96.7000,58.1000) -- (96.7000,58.1000) -- (96.7000,58.1000) -- (96.7000,58.1000) -- (96.7000,58.1000) -- (96.7000,58.1000) -- (96.7000,58.1000) -- (96.7000,58.1000) -- (96.7000,58.1000) -- (96.7000,58.1000) -- (96.7000,58.1000) -- (96.7000,58.1000) -- (96.7000,58.1000) -- (96.7000,58.1000) -- (96.7000,58.1000) -- (96.7000,58.1000) -- (96.7000,58.1000) -- (96.7000,58.0000) -- (96.7000,58.0000) -- (96.7000,58.0000) -- (96.7000,58.0000) -- (96.7000,58.0000) -- (96.7000,58.0000) -- (96.7000,58.0000) -- (96.7000,58.0000) -- (96.7000,58.0000) -- (96.7000,58.0000) -- (96.7000,58.0000) -- (96.7000,58.0000) -- (96.7000,58.0000) -- (96.7000,58.0000) -- (96.7000,58.0000) -- (96.7000,58.0000) -- (96.7000,58.0000) -- (96.7000,58.0000) -- (96.7000,58.0000) -- (96.7000,58.0000) -- (96.7000,58.0000) -- (96.7000,57.9000) -- (96.7000,57.9000) -- (96.7000,57.9000) -- (96.8000,57.9000) -- (96.8000,57.9000) -- (96.8000,57.9000) -- (96.8000,57.9000) -- (96.8000,57.9000) -- (96.8000,57.9000) -- (96.8000,57.9000) -- (96.8000,57.9000) -- (96.8000,57.9000) -- (96.8000,57.9000) -- (96.8000,57.9000) -- (96.8000,57.9000) -- (96.8000,57.9000) -- (96.8000,57.9000) -- (96.8000,57.9000) -- (96.8000,57.9000) -- (96.8000,57.9000) -- (96.8000,57.8000) -- (96.8000,57.8000) -- (96.8000,57.8000) -- (96.8000,57.8000) -- (96.8000,57.8000) -- (96.8000,57.8000) -- (96.8000,57.8000) -- (96.8000,57.8000) -- (96.8000,57.8000) -- (96.8000,57.8000) -- (96.8000,57.8000) -- (96.8000,57.8000) -- (96.8000,57.8000) -- (96.8000,57.8000) -- (96.8000,57.8000) -- (96.8000,57.8000) -- (96.8000,57.8000) -- (96.8000,57.8000) -- (96.8000,57.8000) -- (96.8000,57.8000) -- (96.8000,57.8000) -- (96.8000,57.7000) -- (96.8000,57.7000) -- (96.8000,57.7000) -- (96.8000,57.7000) -- (96.8000,57.7000) -- (96.8000,57.7000) -- (96.8000,57.7000) -- (96.8000,57.7000) -- (96.8000,57.7000) -- (96.8000,57.7000) -- (96.8000,57.7000) -- (96.9000,57.7000) -- (96.9000,57.7000) -- (96.9000,57.7000) -- (96.9000,57.7000) -- (96.9000,57.7000) -- (96.9000,57.7000) -- (96.9000,57.7000) -- (96.9000,57.7000) -- (96.9000,57.7000) -- (96.9000,57.7000) -- (96.9000,57.6000) -- (96.9000,57.6000) -- (96.9000,57.6000) -- (96.9000,57.6000) -- (96.9000,57.6000) -- (96.9000,57.6000) -- (96.9000,57.6000) -- (96.9000,57.6000) -- (96.9000,57.6000) -- (96.9000,57.6000) -- (96.9000,57.6000) -- (96.9000,57.6000) -- (96.9000,57.6000) -- (96.9000,57.6000) -- (96.9000,57.6000) -- (96.9000,57.6000) -- (96.9000,57.6000) -- (96.9000,57.6000) -- (96.9000,57.6000) -- (96.9000,57.6000) -- (96.9000,57.6000) -- (96.9000,57.5000) -- (96.9000,57.5000) -- (96.9000,57.5000) -- (96.9000,57.5000) -- (96.9000,57.5000) -- (96.9000,57.5000) -- (96.9000,57.5000) -- (96.9000,57.5000) -- (96.9000,57.5000) -- (96.9000,57.5000) -- (96.9000,57.5000) -- (96.9000,57.5000) -- (96.9000,57.5000) -- (96.9000,57.5000) -- (96.9000,57.5000) -- (96.9000,57.5000) -- (96.9000,57.5000) -- (96.9000,57.5000) -- (96.9000,57.5000) -- (97.0000,57.5000) -- (97.0000,57.5000) -- (97.0000,57.4000) -- (97.0000,57.4000) -- (97.0000,57.4000) -- (97.0000,57.4000) -- (97.0000,57.4000) -- (97.0000,57.4000) -- (97.0000,57.4000) -- (97.0000,57.4000) -- (97.0000,57.4000) -- (97.0000,57.4000) -- (97.0000,57.4000) -- (97.0000,57.4000) -- (97.0000,57.4000) -- (97.0000,57.4000) -- (97.0000,57.4000) -- (97.0000,57.4000) -- (97.0000,57.4000) -- (97.0000,57.4000) -- (97.0000,57.4000) -- (97.0000,57.4000) -- (97.0000,57.3000) -- (97.0000,57.3000) -- (97.0000,57.3000) -- (97.0000,57.3000) -- (97.0000,57.3000) -- (97.0000,57.3000) -- (97.0000,57.3000) -- (97.0000,57.3000) -- (97.0000,57.3000) -- (97.0000,57.3000) -- (97.0000,57.3000) -- (97.0000,57.3000) -- (97.0000,57.3000) -- (97.0000,57.3000) -- (97.0000,57.3000) -- (97.0000,57.3000) -- (97.0000,57.3000) -- (97.0000,57.3000) -- (97.0000,57.3000) -- (97.0000,57.3000) -- (97.0000,57.3000) -- (97.0000,57.2000) -- (97.0000,57.2000) -- (97.0000,57.2000) -- (97.0000,57.2000) -- (97.0000,57.2000) -- (97.0000,57.2000) -- (97.1000,57.2000) -- (97.1000,57.2000) -- (97.1000,57.2000) -- (97.1000,57.2000) -- (97.1000,57.2000) -- (97.1000,57.2000) -- (97.1000,57.2000) -- (97.1000,57.2000) -- (97.1000,57.2000) -- (97.1000,57.2000) -- (97.1000,57.2000) -- (97.1000,57.2000) -- (97.1000,57.2000) -- (97.1000,57.2000) -- (97.1000,57.2000) -- (97.1000,57.1000) -- (97.1000,57.1000) -- (97.1000,57.1000) -- (97.1000,57.1000) -- (97.1000,57.1000) -- (97.1000,57.1000) -- (97.1000,57.1000) -- (97.1000,57.1000) -- (97.1000,57.1000) -- (97.1000,57.1000) -- (97.1000,57.1000) -- (97.1000,57.1000) -- (97.1000,57.1000) -- (97.1000,57.1000) -- (97.1000,57.1000) -- (97.1000,57.1000) -- (97.1000,57.1000) -- (97.1000,57.1000) -- (97.1000,57.1000) -- (97.1000,57.1000) -- (97.1000,57.1000) -- (97.1000,57.0000) -- (97.1000,57.0000) -- (97.1000,57.0000) -- (97.1000,57.0000) -- (97.1000,57.0000) -- (97.1000,57.0000) -- (97.1000,57.0000) -- (97.1000,57.0000) -- (97.1000,57.0000) -- (97.1000,57.0000) -- (97.1000,57.0000) -- (97.1000,57.0000) -- (97.1000,57.0000) -- (97.1000,57.0000) -- (97.2000,57.0000) -- (97.2000,57.0000) -- (97.2000,57.0000) -- (97.2000,57.0000) -- (97.2000,57.0000) -- (97.2000,57.0000) -- (97.2000,56.9000) -- (97.2000,56.9000) -- (97.2000,56.9000) -- (97.2000,56.9000) -- (97.2000,56.9000) -- (97.2000,56.9000) -- (97.2000,56.9000) -- (97.2000,56.9000) -- (97.2000,56.9000) -- (97.2000,56.9000) -- (97.2000,56.9000) -- (97.2000,56.9000) -- (97.2000,56.9000) -- (97.2000,56.9000) -- (97.2000,56.9000) -- (97.2000,56.9000) -- (97.2000,56.9000) -- (97.2000,56.9000) -- (97.2000,56.9000) -- (97.2000,56.9000) -- (97.2000,56.9000) -- (97.2000,56.8000) -- (97.2000,56.8000) -- (97.2000,56.8000) -- (97.2000,56.8000) -- (97.2000,56.8000) -- (97.2000,56.8000) -- (97.2000,56.8000) -- (97.2000,56.8000) -- (97.2000,56.8000) -- (97.2000,56.8000) -- (97.2000,56.8000) -- (97.2000,56.8000) -- (97.2000,56.8000) -- (97.2000,56.8000) -- (97.2000,56.8000) -- (97.2000,56.8000) -- (97.2000,56.8000) -- (97.2000,56.8000) -- (97.2000,56.8000) -- (97.2000,56.8000) -- (97.2000,56.8000) -- (97.2000,56.7000) -- (97.3000,56.7000) -- (97.3000,56.7000) -- (97.3000,56.7000) -- (97.3000,56.7000) -- (97.3000,56.7000) -- (97.3000,56.7000) -- (97.3000,56.7000) -- (97.3000,56.7000) -- (97.3000,56.7000) -- (97.3000,56.7000) -- (97.3000,56.7000) -- (97.3000,56.7000) -- (97.3000,56.7000) -- (97.3000,56.7000) -- (97.3000,56.7000) -- (97.3000,56.7000) -- (97.3000,56.7000) -- (97.3000,56.7000) -- (97.3000,56.7000) -- (97.3000,56.7000) -- (97.3000,56.6000) -- (97.3000,56.6000) -- (97.3000,56.6000) -- (97.3000,56.6000) -- (97.3000,56.6000) -- (97.3000,56.6000) -- (97.3000,56.6000) -- (97.3000,56.6000) -- (97.3000,56.6000) -- (97.3000,56.6000) -- (97.3000,56.6000) -- (97.3000,56.6000) -- (97.3000,56.6000) -- (97.3000,56.6000) -- (97.3000,56.6000) -- (97.3000,56.6000) -- (97.3000,56.6000) -- (97.3000,56.6000) -- (97.3000,56.6000) -- (97.3000,56.6000) -- (97.3000,56.6000) -- (97.3000,56.5000) -- (97.3000,56.5000) -- (97.3000,56.5000) -- (97.3000,56.5000) -- (97.3000,56.5000) -- (97.3000,56.5000) -- (97.3000,56.5000) -- (97.3000,56.5000) -- (97.3000,56.5000) -- (97.4000,56.5000) -- (97.4000,56.5000) -- (97.4000,56.5000) -- (97.4000,56.5000) -- (97.4000,56.5000) -- (97.4000,56.5000) -- (97.4000,56.5000) -- (97.4000,56.5000) -- (97.4000,56.5000) -- (97.4000,56.5000) -- (97.4000,56.5000) -- (97.4000,56.4000) -- (97.4000,56.4000) -- (97.4000,56.4000) -- (97.4000,56.4000) -- (97.4000,56.4000) -- (97.4000,56.4000) -- (97.4000,56.4000) -- (97.4000,56.4000) -- (97.4000,56.4000) -- (97.4000,56.4000) -- (97.4000,56.4000) -- (97.4000,56.4000) -- (97.4000,56.4000) -- (97.4000,56.4000) -- (97.4000,56.4000) -- (97.4000,56.4000) -- (97.4000,56.4000) -- (97.4000,56.4000) -- (97.4000,56.4000) -- (97.4000,56.4000) -- (97.4000,56.4000) -- (97.4000,56.3000) -- (97.4000,56.3000) -- (97.4000,56.3000) -- (97.4000,56.3000) -- (97.4000,56.3000) -- (97.4000,56.3000) -- (97.4000,56.3000) -- (97.4000,56.3000) -- (97.4000,56.3000) -- (97.4000,56.3000) -- (97.4000,56.3000) -- (97.4000,56.3000) -- (97.4000,56.3000) -- (97.4000,56.3000) -- (97.4000,56.3000) -- (97.4000,56.3000) -- (97.4000,56.3000) -- (97.5000,56.3000) -- (97.5000,56.3000) -- (97.5000,56.3000) -- (97.5000,56.3000) -- (97.5000,56.2000) -- (97.5000,56.2000) -- (97.5000,56.2000) -- (97.5000,56.2000) -- (97.5000,56.2000) -- (97.5000,56.2000) -- (97.5000,56.2000) -- (97.5000,56.2000) -- (97.5000,56.2000) -- (97.5000,56.2000) -- (97.5000,56.2000) -- (97.5000,56.2000) -- (97.5000,56.2000) -- (97.5000,56.2000) -- (97.5000,56.2000) -- (97.5000,56.2000) -- (97.5000,56.2000) -- (97.5000,56.2000) -- (97.5000,56.2000) -- (97.5000,56.2000) -- (97.5000,56.2000) -- (97.5000,56.1000) -- (97.5000,56.1000) -- (97.5000,56.1000) -- (97.5000,56.1000) -- (97.5000,56.1000) -- (97.5000,56.1000) -- (97.5000,56.1000) -- (97.5000,56.1000) -- (97.5000,56.1000) -- (97.5000,56.1000) -- (97.5000,56.1000) -- (97.5000,56.1000) -- (97.5000,56.1000) -- (97.5000,56.1000) -- (97.5000,56.1000) -- (97.5000,56.1000) -- (97.5000,56.1000) -- (97.5000,56.1000) -- (97.5000,56.1000) -- (97.5000,56.1000) -- (97.5000,56.1000) -- (97.5000,56.0000) -- (97.5000,56.0000) -- (97.5000,56.0000) -- (97.5000,56.0000) -- (97.6000,56.0000) -- (97.6000,56.0000) -- (97.6000,56.0000) -- (97.6000,56.0000) -- (97.6000,56.0000) -- (97.6000,56.0000) -- (97.6000,56.0000) -- (97.6000,56.0000) -- (97.6000,56.0000) -- (97.6000,56.0000) -- (97.6000,56.0000) -- (97.6000,56.0000) -- (97.6000,56.0000) -- (97.6000,56.0000) -- (97.6000,56.0000) -- (97.6000,56.0000) -- (97.6000,55.9000) -- (97.6000,55.9000) -- (97.6000,55.9000) -- (97.6000,55.9000) -- (97.6000,55.9000) -- (97.6000,55.9000) -- (97.6000,55.9000) -- (97.6000,55.9000) -- (97.6000,55.9000) -- (97.6000,55.9000) -- (97.6000,55.9000) -- (97.6000,55.9000) -- (97.6000,55.9000) -- (97.6000,55.9000) -- (97.6000,55.9000) -- (97.6000,55.9000) -- (97.6000,55.9000) -- (97.6000,55.9000) -- (97.6000,55.9000) -- (97.6000,55.9000) -- (97.6000,55.9000) -- (97.6000,55.8000) -- (97.6000,55.8000) -- (97.6000,55.8000) -- (97.6000,55.8000) -- (97.6000,55.8000) -- (97.6000,55.8000) -- (97.6000,55.8000) -- (97.6000,55.8000) -- (97.6000,55.8000) -- (97.6000,55.8000) -- (97.6000,55.8000) -- (97.6000,55.8000) -- (97.7000,55.8000) -- (97.7000,55.8000) -- (97.7000,55.8000) -- (97.7000,55.8000) -- (97.7000,55.8000) -- (97.7000,55.8000) -- (97.7000,55.8000) -- (97.7000,55.8000) -- (97.7000,55.8000) -- (97.7000,55.7000) -- (97.7000,55.7000) -- (97.7000,55.7000) -- (97.7000,55.7000) -- (97.7000,55.7000) -- (97.7000,55.7000) -- (97.7000,55.7000) -- (121.4000,55.9000);



  \end{scope}
  \begin{scope}[scale=1.008,draw=black,line join=bevel,line cap=rect,line width=0.800pt]
  \end{scope}
  \begin{scope}[scale=1.008,draw=black,line join=bevel,line cap=rect,line width=0.800pt]
  \end{scope}
  \begin{scope}[cm={{1.00769,0.0,0.0,1.00769,(-21.0,-7.5)}},draw=black,line join=round,line cap=round,line width=0.480pt]
    \path[draw] (81.5000,50.5000) -- (81.5000,78.5000) -- (121.5000,78.5000) -- (121.5000,50.5000) -- (81.5000,50.5000);



  \end{scope}
  \begin{scope}[scale=1.008,draw=black,line join=bevel,line cap=rect,line width=0.800pt]
  \end{scope}
  \begin{scope}[scale=1.008,draw=black,line join=bevel,line cap=rect,line width=0.800pt]
  \end{scope}
  \begin{scope}[cm={{1.00769,0.0,0.0,1.00769,(-21.0,-3.0)}},fill=cffffff]
    \path[fill,rounded corners=0.0000cm] (81.0000,129.0000) rectangle (121.0000,157.0000);



  \end{scope}
  \begin{scope}[scale=1.008,draw=black,line join=bevel,line cap=rect,line width=0.800pt]
  \end{scope}
  \begin{scope}[scale=1.008,draw=black,line join=bevel,line cap=rect,line width=0.800pt]
  \end{scope}
  \begin{scope}[cm={{1.00769,0.0,0.0,1.00769,(-21.0,-3.0)}},draw=ca0a0a4,dash pattern=on 0.40pt off 0.80pt,line join=round,line cap=round,line width=0.400pt]
    \path[draw] (81.5000,135.5000) -- (121.5000,135.5000);



  \end{scope}
  \begin{scope}[cm={{1.00769,0.0,0.0,1.00769,(-21.0,-3.0)}},draw=black,line join=round,line cap=round,line width=0.480pt]
    \path[draw] (81.5000,135.5000) -- (82.6460,135.5000);



    \path[draw] (121.5000,135.5000) -- (120.1220,135.5000);



  \end{scope}
  \begin{scope}[scale=1.008,draw=black,line join=bevel,line cap=rect,line width=0.800pt]
  \end{scope}
  \begin{scope}[cm={{1.00769,0.0,0.0,1.00769,(64.4923,140.069)}},draw=black,line join=bevel,line cap=rect,line width=0.800pt]
  \end{scope}
  \begin{scope}[cm={{1.00769,0.0,0.0,1.00769,(64.4923,140.069)}},draw=black,line join=bevel,line cap=rect,line width=0.800pt]
  \end{scope}
  \begin{scope}[cm={{1.00769,0.0,0.0,1.00769,(64.4923,140.069)}},draw=black,line join=bevel,line cap=rect,line width=0.800pt]
  \end{scope}
  \begin{scope}[cm={{1.00769,0.0,0.0,1.00769,(64.4923,140.069)}},draw=black,line join=bevel,line cap=rect,line width=0.800pt]
  \end{scope}
  \begin{scope}[cm={{1.00769,0.0,0.0,1.00769,(64.4923,140.069)}},draw=black,line join=bevel,line cap=rect,line width=0.800pt]
  \end{scope}
  \begin{scope}[cm={{1.00769,0.0,0.0,1.00769,(33.4923,137.069)}},draw=black,line join=bevel,line cap=rect,line width=0.800pt]
    \path[fill=black] (0.0000,0.0000) node[above right] (text766) {\scriptsize $T$= 47};



  \end{scope}
  \begin{scope}[cm={{1.00769,0.0,0.0,1.00769,(64.4923,140.069)}},draw=black,line join=bevel,line cap=rect,line width=0.800pt]
  \end{scope}
  \begin{scope}[scale=1.008,draw=black,line join=bevel,line cap=rect,line width=0.800pt]
  \end{scope}
  \begin{scope}[cm={{1.00769,0.0,0.0,1.00769,(-21.0,-3.0)}},draw=ca0a0a4,dash pattern=on 0.40pt off 0.80pt,line join=round,line cap=round,line width=0.400pt]
    \path[draw] (97.5000,157.5000) -- (97.5000,129.5000);



  \end{scope}
  \begin{scope}[cm={{1.00769,0.0,0.0,1.00769,(-21.0,-3.0)}},draw=black,line join=round,line cap=round,line width=0.480pt]
    \path[draw] (97.5000,129.5000) -- (97.5000,129.5000) -- (97.5000,130.6564);



  \end{scope}
  \begin{scope}[scale=1.008,draw=black,line join=bevel,line cap=rect,line width=0.800pt]
  \end{scope}
  \begin{scope}[cm={{1.00769,0.0,0.0,1.00769,(92.7077,125.962)}},draw=black,line join=bevel,line cap=rect,line width=0.800pt]
  \end{scope}
  \begin{scope}[cm={{1.00769,0.0,0.0,1.00769,(92.7077,125.962)}},draw=black,line join=bevel,line cap=rect,line width=0.800pt]
  \end{scope}
  \begin{scope}[cm={{1.00769,0.0,0.0,1.00769,(92.7077,125.962)}},draw=black,line join=bevel,line cap=rect,line width=0.800pt]
  \end{scope}
  \begin{scope}[cm={{1.00769,0.0,0.0,1.00769,(92.7077,125.962)}},draw=black,line join=bevel,line cap=rect,line width=0.800pt]
  \end{scope}
  \begin{scope}[cm={{1.00769,0.0,0.0,1.00769,(92.7077,125.962)}},draw=black,line join=bevel,line cap=rect,line width=0.800pt]
  \end{scope}
  \begin{scope}[cm={{1.00769,0.0,0.0,1.00769,(71.7077,124.962)}},draw=black,line join=bevel,line cap=rect,line width=0.800pt]
    \path[fill=black] (0.0000,0.0000) node[above right] (text794) {\scriptsize 84};



  \end{scope}
  \begin{scope}[cm={{1.00769,0.0,0.0,1.00769,(92.7077,125.962)}},draw=black,line join=bevel,line cap=rect,line width=0.800pt]
  \end{scope}
  \begin{scope}[scale=1.008,draw=black,line join=bevel,line cap=rect,line width=0.800pt]
  \end{scope}
  \begin{scope}[cm={{1.00769,0.0,0.0,1.00769,(-21.0,-3.0)}},draw=black,line join=round,line cap=round,line width=0.480pt]
    \path[draw] (81.5000,129.5000) -- (81.5000,157.5000) -- (121.5000,157.5000) -- (121.5000,129.5000) -- (81.5000,129.5000);



  \end{scope}
  \begin{scope}[scale=1.008,draw=black,line join=bevel,line cap=rect,line width=0.800pt]
  \end{scope}
  \begin{scope}[scale=1.008,draw=black,line join=bevel,line cap=rect,line width=0.800pt]
  \end{scope}
  \begin{scope}[scale=1.008,draw=black,line join=bevel,line cap=rect,line width=0.800pt]
  \end{scope}
  \begin{scope}[scale=1.008,draw=black,line join=bevel,line cap=rect,line width=0.800pt]
  \end{scope}
  \begin{scope}[cm={{1.00769,0.0,0.0,1.00769,(-21.0,-3.0)}},draw=black,line join=round,line cap=round,line width=0.480pt]
    \path[draw] (81.2000,156.5000) -- (81.2000,156.5000) -- (81.2000,156.5000) -- (81.2000,156.5000) -- (81.2000,156.5000) -- (81.2000,156.5000) -- (81.2000,156.5000) -- (81.2000,156.5000) -- (81.2000,156.5000) -- (81.2000,156.5000) -- (81.2000,156.5000) -- (81.2000,156.5000) -- (81.2000,156.5000) -- (81.2000,156.5000) -- (81.2000,156.5000) -- (81.2000,156.5000) -- (81.2000,156.5000) -- (81.2000,156.5000) -- (81.2000,156.5000) -- (81.2000,156.5000) -- (81.2000,156.4000) -- (81.2000,156.4000) -- (81.2000,156.4000) -- (81.2000,156.4000) -- (81.2000,156.4000) -- (81.2000,156.4000) -- (81.3000,156.4000) -- (81.3000,156.4000) -- (81.3000,156.4000) -- (81.3000,156.4000) -- (81.3000,156.4000) -- (81.3000,156.4000) -- (81.3000,156.4000) -- (81.3000,156.4000) -- (81.3000,156.4000) -- (81.3000,156.4000) -- (81.3000,156.4000) -- (81.3000,156.4000) -- (81.3000,156.4000) -- (81.3000,156.4000) -- (81.3000,156.4000) -- (81.3000,156.3000) -- (81.3000,156.3000) -- (81.3000,156.3000) -- (81.3000,156.3000) -- (81.3000,156.3000) -- (81.3000,156.3000) -- (81.3000,156.3000) -- (81.3000,156.3000) -- (81.3000,156.3000) -- (81.3000,156.3000) -- (81.3000,156.3000) -- (81.3000,156.3000) -- (81.3000,156.3000) -- (81.3000,156.3000) -- (81.3000,156.3000) -- (81.3000,156.3000) -- (81.3000,156.3000) -- (81.3000,156.3000) -- (81.3000,156.3000) -- (81.3000,156.3000) -- (81.3000,156.3000) -- (81.3000,156.3000) -- (81.3000,156.2000) -- (81.3000,156.2000) -- (81.3000,156.2000) -- (81.3000,156.2000) -- (81.3000,156.2000) -- (81.3000,156.2000) -- (81.3000,156.2000) -- (81.3000,156.2000) -- (81.3000,156.2000) -- (81.3000,156.2000) -- (81.3000,156.2000) -- (81.3000,156.2000) -- (81.3000,156.2000) -- (81.4000,156.2000) -- (81.4000,156.2000) -- (81.4000,156.2000) -- (81.4000,156.2000) -- (81.4000,156.2000) -- (81.4000,156.2000) -- (81.4000,156.2000) -- (81.4000,156.2000) -- (81.4000,156.1000) -- (81.4000,156.1000) -- (81.4000,156.1000) -- (81.4000,156.1000) -- (81.4000,156.1000) -- (81.4000,156.1000) -- (81.4000,156.1000) -- (81.4000,156.1000) -- (81.4000,156.1000) -- (81.4000,156.1000) -- (81.4000,156.1000) -- (81.4000,156.1000) -- (81.4000,156.1000) -- (81.4000,156.1000) -- (81.4000,156.1000) -- (81.4000,156.1000) -- (81.4000,156.1000) -- (81.4000,156.1000) -- (81.4000,156.1000) -- (81.4000,156.1000) -- (81.4000,156.1000) -- (81.4000,156.1000) -- (81.4000,156.0000) -- (81.4000,156.0000) -- (81.4000,156.0000) -- (81.4000,156.0000) -- (81.4000,156.0000) -- (81.4000,156.0000) -- (81.4000,156.0000) -- (81.4000,156.0000) -- (81.4000,156.0000) -- (81.4000,156.0000) -- (81.4000,156.0000) -- (81.4000,156.0000) -- (81.4000,156.0000) -- (81.4000,156.0000) -- (81.4000,156.0000) -- (81.4000,156.0000) -- (81.4000,156.0000) -- (81.4000,156.0000) -- (81.4000,156.0000) -- (81.5000,156.0000) -- (81.5000,156.0000) -- (81.5000,155.9000) -- (81.5000,155.9000) -- (81.5000,155.9000) -- (81.5000,155.9000) -- (81.5000,155.9000) -- (81.5000,155.9000) -- (81.5000,155.9000) -- (81.5000,155.9000) -- (81.5000,155.9000) -- (81.5000,155.9000) -- (81.5000,155.9000) -- (81.5000,155.9000) -- (81.5000,155.9000) -- (81.5000,155.9000) -- (81.5000,155.9000) -- (81.5000,155.9000) -- (81.5000,155.9000) -- (81.5000,155.9000) -- (81.5000,155.9000) -- (81.5000,155.9000) -- (81.5000,155.9000) -- (81.5000,155.9000) -- (81.5000,155.8000) -- (81.5000,155.8000) -- (81.5000,155.8000) -- (81.5000,155.8000) -- (81.5000,155.8000) -- (81.5000,155.8000) -- (81.5000,155.8000) -- (81.5000,155.8000) -- (81.5000,155.8000) -- (81.5000,155.8000) -- (81.5000,155.8000) -- (81.5000,155.8000) -- (81.5000,155.8000) -- (81.5000,155.8000) -- (81.5000,155.8000) -- (81.5000,155.8000) -- (81.5000,155.8000) -- (81.5000,155.8000) -- (81.5000,155.8000) -- (81.5000,155.8000) -- (81.5000,155.8000) -- (81.5000,155.7000) -- (81.5000,155.7000) -- (81.5000,155.7000) -- (81.5000,155.7000) -- (81.5000,155.7000) -- (81.6000,155.7000) -- (81.6000,155.7000) -- (81.6000,155.7000) -- (81.6000,155.7000) -- (81.6000,155.7000) -- (81.6000,155.7000) -- (81.6000,155.7000) -- (81.6000,155.7000) -- (81.6000,155.7000) -- (81.6000,155.7000) -- (81.6000,155.7000) -- (81.6000,155.7000) -- (81.6000,155.7000) -- (81.6000,155.7000) -- (81.6000,155.7000) -- (81.6000,155.7000) -- (81.6000,155.7000) -- (81.6000,155.6000) -- (81.6000,155.6000) -- (81.6000,155.6000) -- (81.6000,155.6000) -- (81.6000,155.6000) -- (81.6000,155.6000) -- (81.6000,155.6000) -- (81.6000,155.6000) -- (81.6000,155.6000) -- (81.6000,155.6000) -- (81.6000,155.6000) -- (81.6000,155.6000) -- (81.6000,155.6000) -- (81.6000,155.6000) -- (81.6000,155.6000) -- (81.6000,155.6000) -- (81.6000,155.6000) -- (81.6000,155.6000) -- (81.6000,155.6000) -- (81.6000,155.6000) -- (81.6000,155.6000) -- (81.6000,155.5000) -- (81.6000,155.5000) -- (81.6000,155.5000) -- (81.6000,155.5000) -- (81.6000,155.5000) -- (81.6000,155.5000) -- (81.6000,155.5000) -- (81.6000,155.5000) -- (81.6000,155.5000) -- (81.6000,155.5000) -- (81.6000,155.5000) -- (81.7000,155.5000) -- (81.7000,155.5000) -- (81.7000,155.5000) -- (81.7000,155.5000) -- (81.7000,155.5000) -- (81.7000,155.5000) -- (81.7000,155.5000) -- (81.7000,155.5000) -- (81.7000,155.5000) -- (81.7000,155.5000) -- (81.7000,155.5000) -- (81.7000,155.4000) -- (81.7000,155.4000) -- (81.7000,155.4000) -- (81.7000,155.4000) -- (81.7000,155.4000) -- (81.7000,155.4000) -- (81.7000,155.4000) -- (81.7000,155.4000) -- (81.7000,155.4000) -- (81.7000,155.4000) -- (81.7000,155.4000) -- (81.7000,155.4000) -- (81.7000,155.4000) -- (81.7000,155.4000) -- (81.7000,155.4000) -- (81.7000,155.4000) -- (81.7000,155.4000) -- (81.7000,155.4000) -- (81.7000,155.4000) -- (81.7000,155.4000) -- (81.7000,155.4000) -- (81.7000,155.3000) -- (81.7000,155.3000) -- (81.7000,155.3000) -- (81.7000,155.3000) -- (81.7000,155.3000) -- (81.7000,155.3000) -- (81.7000,155.3000) -- (81.7000,155.3000) -- (81.7000,155.3000) -- (81.7000,155.3000) -- (81.7000,155.3000) -- (81.7000,155.3000) -- (81.7000,155.3000) -- (81.7000,155.3000) -- (81.7000,155.3000) -- (81.7000,155.3000) -- (81.7000,155.3000) -- (81.7000,155.3000) -- (81.8000,155.3000) -- (81.8000,155.3000) -- (81.8000,155.3000) -- (81.8000,155.3000) -- (81.8000,155.2000) -- (81.8000,155.2000) -- (81.8000,155.2000) -- (81.8000,155.2000) -- (81.8000,155.2000) -- (81.8000,155.2000) -- (81.8000,155.2000) -- (81.8000,155.2000) -- (81.8000,155.2000) -- (81.8000,155.2000) -- (81.8000,155.2000) -- (81.8000,155.2000) -- (81.8000,155.2000) -- (81.8000,155.2000) -- (81.8000,155.2000) -- (81.8000,155.2000) -- (81.8000,155.2000) -- (81.8000,155.2000) -- (81.8000,155.2000) -- (81.8000,155.2000) -- (81.8000,155.2000) -- (81.8000,155.1000) -- (81.8000,155.1000) -- (81.8000,155.1000) -- (81.8000,155.1000) -- (81.8000,155.1000) -- (81.8000,155.1000) -- (81.8000,155.1000) -- (81.8000,155.1000) -- (81.8000,155.1000) -- (81.8000,155.1000) -- (81.8000,155.1000) -- (81.8000,155.1000) -- (81.8000,155.1000) -- (81.8000,155.1000) -- (81.8000,155.1000) -- (81.8000,155.1000) -- (81.8000,155.1000) -- (81.8000,155.1000) -- (81.8000,155.1000) -- (81.8000,155.1000) -- (81.8000,155.1000) -- (81.8000,155.1000) -- (81.8000,155.0000) -- (81.8000,155.0000) -- (81.9000,155.0000) -- (81.9000,155.0000) -- (81.9000,155.0000) -- (81.9000,155.0000) -- (81.9000,155.0000) -- (81.9000,155.0000) -- (81.9000,155.0000) -- (81.9000,155.0000) -- (81.9000,155.0000) -- (81.9000,155.0000) -- (81.9000,155.0000) -- (81.9000,155.0000) -- (81.9000,155.0000) -- (81.9000,155.0000) -- (81.9000,155.0000) -- (81.9000,155.0000) -- (81.9000,155.0000) -- (81.9000,155.0000) -- (81.9000,155.0000) -- (81.9000,154.9000) -- (81.9000,154.9000) -- (81.9000,154.9000) -- (81.9000,154.9000) -- (81.9000,154.9000) -- (81.9000,154.9000) -- (81.9000,154.9000) -- (81.9000,154.9000) -- (81.9000,154.9000) -- (81.9000,154.9000) -- (81.9000,154.9000) -- (81.9000,154.9000) -- (81.9000,154.9000) -- (81.9000,154.9000) -- (81.9000,154.9000) -- (81.9000,154.9000) -- (81.9000,154.9000) -- (81.9000,154.9000) -- (81.9000,154.9000) -- (81.9000,154.9000) -- (81.9000,154.9000) -- (81.9000,154.9000) -- (81.9000,154.8000) -- (81.9000,154.8000) -- (81.9000,154.8000) -- (81.9000,154.8000) -- (81.9000,154.8000) -- (81.9000,154.8000) -- (81.9000,154.8000) -- (81.9000,154.8000) -- (81.9000,154.8000) -- (82.0000,154.8000) -- (82.0000,154.8000) -- (82.0000,154.8000) -- (82.0000,154.8000) -- (82.0000,154.8000) -- (82.0000,154.8000) -- (82.0000,154.8000) -- (82.0000,154.8000) -- (82.0000,154.8000) -- (82.0000,154.8000) -- (82.0000,154.8000) -- (82.0000,154.8000) -- (82.0000,154.7000) -- (82.0000,154.7000) -- (82.0000,154.7000) -- (82.0000,154.7000) -- (82.0000,154.7000) -- (82.0000,154.7000) -- (82.0000,154.7000) -- (82.0000,154.7000) -- (82.0000,154.7000) -- (82.0000,154.7000) -- (82.0000,154.7000) -- (82.0000,154.7000) -- (82.0000,154.7000) -- (82.0000,154.7000) -- (82.0000,154.7000) -- (82.0000,154.7000) -- (82.0000,154.7000) -- (82.0000,154.7000) -- (82.0000,154.7000) -- (82.0000,154.7000) -- (82.0000,154.7000) -- (82.0000,154.7000) -- (82.0000,154.6000) -- (82.0000,154.6000) -- (82.0000,154.6000) -- (82.0000,154.6000) -- (82.0000,154.6000) -- (82.0000,154.6000) -- (82.0000,154.6000) -- (82.0000,154.6000) -- (82.0000,154.6000) -- (82.0000,154.6000) -- (82.0000,154.6000) -- (82.0000,154.6000) -- (82.0000,154.6000) -- (82.0000,154.6000) -- (82.0000,154.6000) -- (82.1000,154.6000) -- (82.1000,154.6000) -- (82.1000,154.6000) -- (82.1000,154.6000) -- (82.1000,154.6000) -- (82.1000,154.6000) -- (82.1000,154.5000) -- (82.1000,154.5000) -- (82.1000,154.5000) -- (82.1000,154.5000) -- (82.1000,154.5000) -- (82.1000,154.5000) -- (82.1000,154.5000) -- (82.1000,154.5000) -- (82.1000,154.5000) -- (82.1000,154.5000) -- (82.1000,154.5000) -- (82.1000,154.5000) -- (82.1000,154.5000) -- (82.1000,154.5000) -- (82.1000,154.5000) -- (82.1000,154.5000) -- (82.1000,154.5000) -- (82.1000,154.5000) -- (82.1000,154.5000) -- (82.1000,154.5000) -- (82.1000,154.5000) -- (82.1000,154.5000) -- (82.1000,154.4000) -- (82.1000,154.4000) -- (82.1000,154.4000) -- (82.1000,154.4000) -- (82.1000,154.4000) -- (82.1000,154.4000) -- (82.1000,154.4000) -- (82.1000,154.4000) -- (82.1000,154.4000) -- (82.1000,154.4000) -- (82.1000,154.4000) -- (82.1000,154.4000) -- (82.1000,154.4000) -- (82.1000,154.4000) -- (82.1000,154.4000) -- (82.1000,154.4000) -- (82.1000,154.4000) -- (82.1000,154.4000) -- (82.1000,154.4000) -- (82.1000,154.4000) -- (82.1000,154.4000) -- (82.1000,154.3000) -- (82.2000,154.3000) -- (82.2000,154.3000) -- (82.2000,154.3000) -- (82.2000,154.3000) -- (82.2000,154.3000) -- (82.2000,154.3000) -- (82.2000,154.3000) -- (82.2000,154.3000) -- (82.2000,154.3000) -- (82.2000,154.3000) -- (82.2000,154.3000) -- (82.2000,154.3000) -- (82.2000,154.3000) -- (82.2000,154.3000) -- (82.2000,154.3000) -- (82.2000,154.3000) -- (82.2000,154.3000) -- (82.2000,154.3000) -- (82.2000,154.3000) -- (82.2000,154.3000) -- (82.2000,154.3000) -- (82.2000,154.2000) -- (82.2000,154.2000) -- (82.2000,154.2000) -- (82.2000,154.2000) -- (82.2000,154.2000) -- (82.2000,154.2000) -- (82.2000,154.2000) -- (82.2000,154.2000) -- (82.2000,154.2000) -- (82.2000,154.2000) -- (82.2000,154.2000) -- (82.2000,154.2000) -- (82.2000,154.2000) -- (82.2000,154.2000) -- (82.2000,154.2000) -- (82.2000,154.2000) -- (82.2000,154.2000) -- (82.2000,154.2000) -- (82.2000,154.2000) -- (82.2000,154.2000) -- (82.2000,154.2000) -- (82.2000,154.1000) -- (82.2000,154.1000) -- (82.2000,154.1000) -- (82.2000,154.1000) -- (82.2000,154.1000) -- (82.2000,154.1000) -- (82.2000,154.1000) -- (82.3000,154.1000) -- (82.3000,154.1000) -- (82.3000,154.1000) -- (82.3000,154.1000) -- (82.3000,154.1000) -- (82.3000,154.1000) -- (82.3000,154.1000) -- (82.3000,154.1000) -- (82.3000,154.1000) -- (82.3000,154.1000) -- (82.3000,154.1000) -- (82.3000,154.1000) -- (82.3000,154.1000) -- (82.3000,154.1000) -- (82.3000,154.1000) -- (82.3000,154.0000) -- (82.3000,154.0000) -- (82.3000,154.0000) -- (82.3000,154.0000) -- (82.3000,154.0000) -- (82.3000,154.0000) -- (82.3000,154.0000) -- (82.3000,154.0000) -- (82.3000,154.0000) -- (82.3000,154.0000) -- (82.3000,154.0000) -- (82.3000,154.0000) -- (82.3000,154.0000) -- (82.3000,154.0000) -- (82.3000,154.0000) -- (82.3000,154.0000) -- (82.3000,154.0000) -- (82.3000,154.0000) -- (82.3000,154.0000) -- (82.3000,154.0000) -- (82.3000,154.0000) -- (82.3000,153.9000) -- (82.3000,153.9000) -- (82.3000,153.9000) -- (82.3000,153.9000) -- (82.3000,153.9000) -- (82.3000,153.9000) -- (82.3000,153.9000) -- (82.3000,153.9000) -- (82.3000,153.9000) -- (82.3000,153.9000) -- (82.3000,153.9000) -- (82.3000,153.9000) -- (82.3000,153.9000) -- (82.3000,153.9000) -- (82.4000,153.9000) -- (82.4000,153.9000) -- (82.4000,153.9000) -- (82.4000,153.9000) -- (82.4000,153.9000) -- (82.4000,153.9000) -- (82.4000,153.9000) -- (82.4000,153.9000) -- (82.4000,153.8000) -- (82.4000,153.8000) -- (82.4000,153.8000) -- (82.4000,153.8000) -- (82.4000,153.8000) -- (82.4000,153.8000) -- (82.4000,153.8000) -- (82.4000,153.8000) -- (82.4000,153.8000) -- (82.4000,153.8000) -- (82.4000,153.8000) -- (82.4000,153.8000) -- (82.4000,153.8000) -- (82.4000,153.8000) -- (82.4000,153.8000) -- (82.4000,153.8000) -- (82.4000,153.8000) -- (82.4000,153.8000) -- (82.4000,153.8000) -- (82.4000,153.8000) -- (82.4000,153.8000) -- (82.4000,153.7000) -- (82.4000,153.7000) -- (82.4000,153.7000) -- (82.4000,153.7000) -- (82.4000,153.7000) -- (82.4000,153.7000) -- (82.4000,153.7000) -- (82.4000,153.7000) -- (82.4000,153.7000) -- (82.4000,153.7000) -- (82.4000,153.7000) -- (82.4000,153.7000) -- (82.4000,153.7000) -- (82.4000,153.7000) -- (82.4000,153.7000) -- (82.4000,153.7000) -- (82.4000,153.7000) -- (82.4000,153.7000) -- (82.4000,153.7000) -- (82.4000,153.7000) -- (82.5000,153.7000) -- (82.5000,153.7000) -- (82.5000,153.6000) -- (82.5000,153.6000) -- (82.5000,153.6000) -- (82.5000,153.6000) -- (82.5000,153.6000) -- (82.5000,153.6000) -- (82.5000,153.6000) -- (82.5000,153.6000) -- (82.5000,153.6000) -- (82.5000,153.6000) -- (82.5000,153.6000) -- (82.5000,153.6000) -- (82.5000,153.6000) -- (82.5000,153.6000) -- (82.5000,153.6000) -- (82.5000,153.6000) -- (82.5000,153.6000) -- (82.5000,153.6000) -- (82.5000,153.6000) -- (82.5000,153.6000) -- (82.5000,153.6000) -- (82.5000,153.5000) -- (82.5000,153.5000) -- (82.5000,153.5000) -- (82.5000,153.5000) -- (82.5000,153.5000) -- (82.5000,153.5000) -- (82.5000,153.5000) -- (82.5000,153.5000) -- (82.5000,153.5000) -- (82.5000,153.5000) -- (82.5000,153.5000) -- (82.5000,153.5000) -- (82.5000,153.5000) -- (82.5000,153.5000) -- (82.5000,153.5000) -- (82.5000,153.5000) -- (82.5000,153.5000) -- (82.5000,153.5000) -- (82.5000,153.5000) -- (82.5000,153.5000) -- (82.5000,153.5000) -- (82.5000,153.5000) -- (82.5000,153.4000) -- (82.5000,153.4000) -- (82.5000,153.4000) -- (82.5000,153.4000) -- (82.5000,153.4000) -- (82.6000,153.4000) -- (82.6000,153.4000) -- (82.6000,153.4000) -- (82.6000,153.4000) -- (82.6000,153.4000) -- (82.6000,153.4000) -- (82.6000,153.4000) -- (82.6000,153.4000) -- (82.6000,153.4000) -- (82.6000,153.4000) -- (82.6000,153.4000) -- (82.6000,153.4000) -- (82.6000,153.4000) -- (82.6000,153.4000) -- (82.6000,153.4000) -- (82.6000,153.4000) -- (82.6000,153.3000) -- (82.6000,153.3000) -- (82.6000,153.3000) -- (82.6000,153.3000) -- (82.6000,153.3000) -- (82.6000,153.3000) -- (82.6000,153.3000) -- (82.6000,153.3000) -- (82.6000,153.3000) -- (82.6000,153.3000) -- (82.6000,153.3000) -- (82.6000,153.3000) -- (82.6000,153.3000) -- (82.6000,153.3000) -- (82.6000,153.3000) -- (82.6000,153.3000) -- (82.6000,153.3000) -- (82.6000,153.3000) -- (82.6000,153.3000) -- (82.6000,153.3000) -- (82.6000,153.3000) -- (82.6000,153.3000) -- (82.6000,153.2000) -- (82.6000,153.2000) -- (82.6000,153.2000) -- (82.6000,153.2000) -- (82.6000,153.2000) -- (82.6000,153.2000) -- (82.6000,153.2000) -- (82.6000,153.2000) -- (82.6000,153.2000) -- (82.6000,153.2000) -- (82.6000,153.2000) -- (82.7000,153.2000) -- (82.7000,153.2000) -- (82.7000,153.2000) -- (82.7000,153.2000) -- (82.7000,153.2000) -- (82.7000,153.2000) -- (82.7000,153.2000) -- (82.7000,153.2000) -- (82.7000,153.2000) -- (82.7000,153.2000) -- (82.7000,153.1000) -- (82.7000,153.1000) -- (82.7000,153.1000) -- (82.7000,153.1000) -- (82.7000,153.1000) -- (82.7000,153.1000) -- (82.7000,153.1000) -- (82.7000,153.1000) -- (82.7000,153.1000) -- (82.7000,153.1000) -- (82.7000,153.1000) -- (82.7000,153.1000) -- (82.7000,153.1000) -- (82.7000,153.1000) -- (82.7000,153.1000) -- (82.7000,153.1000) -- (82.7000,153.1000) -- (82.7000,153.1000) -- (82.7000,153.1000) -- (82.7000,153.1000) -- (82.7000,153.1000) -- (82.7000,153.1000) -- (82.7000,153.0000) -- (82.7000,153.0000) -- (82.7000,153.0000) -- (82.7000,153.0000) -- (82.7000,153.0000) -- (82.7000,153.0000) -- (82.7000,153.0000) -- (82.7000,153.0000) -- (82.7000,153.0000) -- (82.7000,153.0000) -- (82.7000,153.0000) -- (82.7000,153.0000) -- (82.7000,153.0000) -- (82.7000,153.0000) -- (82.7000,153.0000) -- (82.7000,153.0000) -- (82.7000,153.0000) -- (82.7000,153.0000) -- (82.8000,153.0000) -- (82.8000,153.0000) -- (82.8000,153.0000) -- (82.8000,152.9000) -- (82.8000,152.9000) -- (82.8000,152.9000) -- (82.8000,152.9000) -- (82.8000,152.9000) -- (82.8000,152.9000) -- (82.8000,152.9000) -- (82.8000,152.9000) -- (82.8000,152.9000) -- (82.8000,152.9000) -- (82.8000,152.9000) -- (82.8000,152.9000) -- (82.8000,152.9000) -- (82.8000,152.9000) -- (82.8000,152.9000) -- (82.8000,152.9000) -- (82.8000,152.9000) -- (82.8000,152.9000) -- (82.8000,152.9000) -- (82.8000,152.9000) -- (82.8000,152.9000) -- (82.8000,152.9000) -- (82.8000,152.8000) -- (82.8000,152.8000) -- (82.8000,152.8000) -- (82.8000,152.8000) -- (82.8000,152.8000) -- (82.8000,152.8000) -- (82.8000,152.8000) -- (82.8000,152.8000) -- (82.8000,152.8000) -- (82.8000,152.8000) -- (82.8000,152.8000) -- (82.8000,152.8000) -- (82.8000,152.8000) -- (82.8000,152.8000) -- (82.8000,152.8000) -- (82.8000,152.8000) -- (82.8000,152.8000) -- (82.8000,152.8000) -- (82.8000,152.8000) -- (82.8000,152.8000) -- (82.8000,152.8000) -- (82.8000,152.7000) -- (82.8000,152.7000) -- (82.8000,152.7000) -- (82.9000,152.7000) -- (82.9000,152.7000) -- (82.9000,152.7000) -- (82.9000,152.7000) -- (82.9000,152.7000) -- (82.9000,152.7000) -- (82.9000,152.7000) -- (82.9000,152.7000) -- (82.9000,152.7000) -- (82.9000,152.7000) -- (82.9000,152.7000) -- (82.9000,152.7000) -- (82.9000,152.7000) -- (82.9000,152.7000) -- (82.9000,152.7000) -- (82.9000,152.7000) -- (82.9000,152.7000) -- (82.9000,152.7000) -- (82.9000,152.7000) -- (82.9000,152.6000) -- (82.9000,152.6000) -- (82.9000,152.6000) -- (82.9000,152.6000) -- (82.9000,152.6000) -- (82.9000,152.6000) -- (82.9000,152.6000) -- (82.9000,152.6000) -- (82.9000,152.6000) -- (82.9000,152.6000) -- (82.9000,152.6000) -- (82.9000,152.6000) -- (82.9000,152.6000) -- (82.9000,152.6000) -- (82.9000,152.6000) -- (82.9000,152.6000) -- (82.9000,152.6000) -- (82.9000,152.6000) -- (82.9000,152.6000) -- (82.9000,152.6000) -- (82.9000,152.6000) -- (82.9000,152.5000) -- (82.9000,152.5000) -- (82.9000,152.5000) -- (82.9000,152.5000) -- (82.9000,152.5000) -- (82.9000,152.5000) -- (82.9000,152.5000) -- (82.9000,152.5000) -- (82.9000,152.5000) -- (82.9000,152.5000) -- (83.0000,152.5000) -- (83.0000,152.5000) -- (83.0000,152.5000) -- (83.0000,152.5000) -- (83.0000,152.5000) -- (83.0000,152.5000) -- (83.0000,152.5000) -- (83.0000,152.5000) -- (83.0000,152.5000) -- (83.0000,152.5000) -- (83.0000,152.5000) -- (83.0000,152.5000) -- (83.0000,152.4000) -- (83.0000,152.4000) -- (83.0000,152.4000) -- (83.0000,152.4000) -- (83.0000,152.4000) -- (83.0000,152.4000) -- (83.0000,152.4000) -- (83.0000,152.4000) -- (83.0000,152.4000) -- (83.0000,152.4000) -- (83.0000,152.4000) -- (83.0000,152.4000) -- (83.0000,152.4000) -- (83.0000,152.4000) -- (83.0000,152.4000) -- (83.0000,152.4000) -- (83.0000,152.4000) -- (83.0000,152.4000) -- (83.0000,152.4000) -- (83.0000,152.4000) -- (83.0000,152.4000) -- (83.0000,152.3000) -- (83.0000,152.3000) -- (83.0000,152.3000) -- (83.0000,152.3000) -- (83.0000,152.3000) -- (83.0000,152.3000) -- (83.0000,152.3000) -- (83.0000,152.3000) -- (83.0000,152.3000) -- (83.0000,152.3000) -- (83.0000,152.3000) -- (83.0000,152.3000) -- (83.0000,152.3000) -- (83.0000,152.3000) -- (83.0000,152.3000) -- (83.0000,152.3000) -- (83.1000,152.3000) -- (83.1000,152.3000) -- (83.1000,152.3000) -- (83.1000,152.3000) -- (83.1000,152.3000) -- (83.1000,152.3000) -- (83.1000,152.2000) -- (83.1000,152.2000) -- (83.1000,152.2000) -- (83.1000,152.2000) -- (83.1000,152.2000) -- (83.1000,152.2000) -- (83.1000,152.2000) -- (83.1000,152.2000) -- (83.1000,152.2000) -- (83.1000,152.2000) -- (83.1000,152.2000) -- (83.1000,152.2000) -- (83.1000,152.2000) -- (83.1000,152.2000) -- (83.1000,152.2000) -- (83.1000,152.2000) -- (83.1000,152.2000) -- (83.1000,152.2000) -- (83.1000,152.2000) -- (83.1000,152.2000) -- (83.1000,152.2000) -- (83.1000,152.2000) -- (83.1000,152.1000) -- (83.1000,152.1000) -- (83.1000,152.1000) -- (83.1000,152.1000) -- (83.1000,152.1000) -- (83.1000,152.1000) -- (83.1000,152.1000) -- (83.1000,152.1000) -- (83.1000,152.1000) -- (83.1000,152.1000) -- (83.1000,152.1000) -- (83.1000,152.1000) -- (83.1000,152.1000) -- (83.1000,152.1000) -- (83.1000,152.1000) -- (83.1000,152.1000) -- (83.1000,152.1000) -- (83.1000,152.1000) -- (83.1000,152.1000) -- (83.1000,152.1000) -- (83.1000,152.1000) -- (83.1000,152.0000) -- (83.2000,152.0000) -- (83.2000,152.0000) -- (83.2000,152.0000) -- (83.2000,152.0000) -- (83.2000,152.0000) -- (83.2000,152.0000) -- (83.2000,152.0000) -- (83.2000,152.0000) -- (83.2000,152.0000) -- (83.2000,152.0000) -- (83.2000,152.0000) -- (83.2000,152.0000) -- (83.2000,152.0000) -- (83.2000,152.0000) -- (83.2000,152.0000) -- (83.2000,152.0000) -- (83.2000,152.0000) -- (83.2000,152.0000) -- (83.2000,152.0000) -- (83.2000,152.0000) -- (83.2000,152.0000) -- (83.2000,151.9000) -- (83.2000,151.9000) -- (83.2000,151.9000) -- (83.2000,151.9000) -- (83.2000,151.9000) -- (83.2000,151.9000) -- (83.2000,151.9000) -- (83.2000,151.9000) -- (83.2000,151.9000) -- (83.2000,151.9000) -- (83.2000,151.9000) -- (83.2000,151.9000) -- (83.2000,151.9000) -- (83.2000,151.9000) -- (83.2000,151.9000) -- (83.2000,151.9000) -- (83.2000,151.9000) -- (83.2000,151.9000) -- (83.2000,151.9000) -- (83.2000,151.9000) -- (83.2000,151.9000) -- (83.2000,151.8000) -- (83.2000,151.8000) -- (83.2000,151.8000) -- (83.2000,151.8000) -- (83.2000,151.8000) -- (83.2000,151.8000) -- (83.2000,151.8000) -- (83.3000,151.8000) -- (83.3000,151.8000) -- (83.3000,151.8000) -- (83.3000,151.8000) -- (83.3000,151.8000) -- (83.3000,151.8000) -- (83.3000,151.8000) -- (83.3000,151.8000) -- (83.3000,151.8000) -- (83.3000,151.8000) -- (83.3000,151.8000) -- (83.3000,151.8000) -- (83.3000,151.8000) -- (83.3000,151.8000) -- (83.3000,151.8000) -- (83.3000,151.7000) -- (83.3000,151.7000) -- (83.3000,151.7000) -- (83.3000,151.7000) -- (83.3000,151.7000) -- (83.3000,151.7000) -- (83.3000,151.7000) -- (83.3000,151.7000) -- (83.3000,151.7000) -- (83.3000,151.7000) -- (83.3000,151.7000) -- (83.3000,151.7000) -- (83.3000,151.7000) -- (83.3000,151.7000) -- (83.3000,151.7000) -- (83.3000,151.7000) -- (83.3000,151.7000) -- (83.3000,151.7000) -- (83.3000,151.7000) -- (83.3000,151.7000) -- (83.3000,151.7000) -- (83.3000,151.6000) -- (83.3000,151.6000) -- (83.3000,151.6000) -- (83.3000,151.6000) -- (83.3000,151.6000) -- (83.3000,151.6000) -- (83.3000,151.6000) -- (83.3000,151.6000) -- (83.3000,151.6000) -- (83.3000,151.6000) -- (83.3000,151.6000) -- (83.3000,151.6000) -- (83.3000,151.6000) -- (83.3000,151.6000) -- (83.4000,151.6000) -- (83.4000,151.6000) -- (83.4000,151.6000) -- (83.4000,151.6000) -- (83.4000,151.6000) -- (83.4000,151.6000) -- (83.4000,151.6000) -- (83.4000,151.6000) -- (83.4000,151.5000) -- (83.4000,151.5000) -- (83.4000,151.5000) -- (83.4000,151.5000) -- (83.4000,151.5000) -- (83.4000,151.5000) -- (83.4000,151.5000) -- (83.4000,151.5000) -- (83.4000,151.5000) -- (83.4000,151.5000) -- (83.4000,151.5000) -- (83.4000,151.5000) -- (83.4000,151.5000) -- (83.4000,151.5000) -- (83.4000,151.5000) -- (83.4000,151.5000) -- (83.4000,151.5000) -- (83.4000,151.5000) -- (83.4000,151.5000) -- (83.4000,151.5000) -- (83.4000,151.5000) -- (83.4000,151.4000) -- (83.4000,151.4000) -- (83.4000,151.4000) -- (83.4000,151.4000) -- (83.4000,151.4000) -- (83.4000,151.4000) -- (83.4000,151.4000) -- (83.4000,151.4000) -- (83.4000,151.4000) -- (83.4000,151.4000) -- (83.4000,151.4000) -- (83.4000,151.4000) -- (83.4000,151.4000) -- (83.4000,151.4000) -- (83.4000,151.4000) -- (83.4000,151.4000) -- (83.4000,151.4000) -- (83.4000,151.4000) -- (83.4000,151.4000) -- (83.4000,151.4000) -- (83.5000,151.4000) -- (83.5000,151.4000) -- (83.5000,151.3000) -- (83.5000,151.3000) -- (83.5000,151.3000) -- (83.5000,151.3000) -- (83.5000,151.3000) -- (83.5000,151.3000) -- (83.5000,151.3000) -- (83.5000,151.3000) -- (83.5000,151.3000) -- (83.5000,151.3000) -- (83.5000,151.3000) -- (83.5000,151.3000) -- (83.5000,151.3000) -- (83.5000,151.3000) -- (83.5000,151.3000) -- (83.5000,151.3000) -- (83.5000,151.3000) -- (83.5000,151.3000) -- (83.5000,151.3000) -- (83.5000,151.3000) -- (83.5000,151.3000) -- (83.5000,151.2000) -- (83.5000,151.2000) -- (83.5000,151.2000) -- (83.5000,151.2000) -- (83.5000,151.2000) -- (83.5000,151.2000) -- (83.5000,151.2000) -- (83.5000,151.2000) -- (83.5000,151.2000) -- (83.5000,151.2000) -- (83.5000,151.2000) -- (83.5000,151.2000) -- (83.5000,151.2000) -- (83.5000,151.2000) -- (83.5000,151.2000) -- (83.5000,151.2000) -- (83.5000,151.2000) -- (83.5000,151.2000) -- (83.5000,151.2000) -- (83.5000,151.2000) -- (83.5000,151.2000) -- (83.5000,151.2000) -- (83.5000,151.1000) -- (83.5000,151.1000) -- (83.5000,151.1000) -- (83.5000,151.1000) -- (83.5000,151.1000) -- (83.6000,151.1000) -- (83.6000,151.1000) -- (83.6000,151.1000) -- (83.6000,151.1000) -- (83.6000,151.1000) -- (83.6000,151.1000) -- (83.6000,151.1000) -- (83.6000,151.1000) -- (83.6000,151.1000) -- (83.6000,151.1000) -- (83.6000,151.1000) -- (83.6000,151.1000) -- (83.6000,151.1000) -- (83.6000,151.1000) -- (83.6000,151.1000) -- (83.6000,151.1000) -- (83.6000,151.0000) -- (83.6000,151.0000) -- (83.6000,151.0000) -- (83.6000,151.0000) -- (83.6000,151.0000) -- (83.6000,151.0000) -- (83.6000,151.0000) -- (83.6000,151.0000) -- (83.6000,151.0000) -- (83.6000,151.0000) -- (83.6000,151.0000) -- (83.6000,151.0000) -- (83.6000,151.0000) -- (83.6000,151.0000) -- (83.6000,151.0000) -- (83.6000,151.0000) -- (83.6000,151.0000) -- (83.6000,151.0000) -- (83.6000,151.0000) -- (83.6000,151.0000) -- (83.6000,151.0000) -- (83.6000,151.0000) -- (83.6000,150.9000) -- (83.6000,150.9000) -- (83.6000,150.9000) -- (83.6000,150.9000) -- (83.6000,150.9000) -- (83.6000,150.9000) -- (83.6000,150.9000) -- (83.6000,150.9000) -- (83.6000,150.9000) -- (83.6000,150.9000) -- (83.6000,150.9000) -- (83.7000,150.9000) -- (83.7000,150.9000) -- (83.7000,150.9000) -- (83.7000,150.9000) -- (83.7000,150.9000) -- (83.7000,150.9000) -- (83.7000,150.9000) -- (83.7000,150.9000) -- (83.7000,150.9000) -- (83.7000,150.9000) -- (83.7000,150.8000) -- (83.7000,150.8000) -- (83.7000,150.8000) -- (83.7000,150.8000) -- (83.7000,150.8000) -- (83.7000,150.8000) -- (83.7000,150.8000) -- (83.7000,150.8000) -- (83.7000,150.8000) -- (83.7000,150.8000) -- (83.7000,150.8000) -- (83.7000,150.8000) -- (83.7000,150.8000) -- (83.7000,150.8000) -- (83.7000,150.8000) -- (83.7000,150.8000) -- (83.7000,150.8000) -- (83.7000,150.8000) -- (83.7000,150.8000) -- (83.7000,150.8000) -- (83.7000,150.8000) -- (83.7000,150.8000) -- (83.7000,150.7000) -- (83.7000,150.7000) -- (83.7000,150.7000) -- (83.7000,150.7000) -- (83.7000,150.7000) -- (83.7000,150.7000) -- (83.7000,150.7000) -- (83.7000,150.7000) -- (83.7000,150.7000) -- (83.7000,150.7000) -- (83.7000,150.7000) -- (83.7000,150.7000) -- (83.7000,150.7000) -- (83.7000,150.7000) -- (83.7000,150.7000) -- (83.7000,150.7000) -- (83.7000,150.7000) -- (83.7000,150.7000) -- (83.8000,150.7000) -- (83.8000,150.7000) -- (83.8000,150.7000) -- (83.8000,150.6000) -- (83.8000,150.6000) -- (83.8000,150.6000) -- (83.8000,150.6000) -- (83.8000,150.6000) -- (83.8000,150.6000) -- (83.8000,150.6000) -- (83.8000,150.6000) -- (83.8000,150.6000) -- (83.8000,150.6000) -- (83.8000,150.6000) -- (83.8000,150.6000) -- (83.8000,150.6000) -- (83.8000,150.6000) -- (83.8000,150.6000) -- (83.8000,150.6000) -- (83.8000,150.6000) -- (83.8000,150.6000) -- (83.8000,150.6000) -- (83.8000,150.6000) -- (83.8000,150.6000) -- (83.8000,150.6000) -- (83.8000,150.5000) -- (83.8000,150.5000) -- (83.8000,150.5000) -- (83.8000,150.5000) -- (83.8000,150.5000) -- (83.8000,150.5000) -- (83.8000,150.5000) -- (83.8000,150.5000) -- (83.8000,150.5000) -- (83.8000,150.5000) -- (83.8000,150.5000) -- (83.8000,150.5000) -- (83.8000,150.5000) -- (83.8000,150.5000) -- (83.8000,150.5000) -- (83.8000,150.5000) -- (83.8000,150.5000) -- (83.8000,150.5000) -- (83.8000,150.5000) -- (83.8000,150.5000) -- (83.8000,150.5000) -- (83.8000,150.4000) -- (83.8000,150.4000) -- (83.8000,150.4000) -- (83.9000,150.4000) -- (83.9000,150.4000) -- (83.9000,150.4000) -- (83.9000,150.4000) -- (83.9000,150.4000) -- (83.9000,150.4000) -- (83.9000,150.4000) -- (83.9000,150.4000) -- (83.9000,150.4000) -- (83.9000,150.4000) -- (83.9000,150.4000) -- (83.9000,150.4000) -- (83.9000,150.4000) -- (83.9000,150.4000) -- (83.9000,150.4000) -- (83.9000,150.4000) -- (83.9000,150.4000) -- (83.9000,150.4000) -- (83.9000,150.4000) -- (83.9000,150.3000) -- (83.9000,150.3000) -- (83.9000,150.3000) -- (83.9000,150.3000) -- (83.9000,150.3000) -- (83.9000,150.3000) -- (83.9000,150.3000) -- (83.9000,150.3000) -- (83.9000,150.3000) -- (83.9000,150.3000) -- (83.9000,150.3000) -- (83.9000,150.3000) -- (83.9000,150.3000) -- (83.9000,150.3000) -- (83.9000,150.3000) -- (83.9000,150.3000) -- (83.9000,150.3000) -- (83.9000,150.3000) -- (83.9000,150.3000) -- (83.9000,150.3000) -- (83.9000,150.3000) -- (83.9000,150.2000) -- (83.9000,150.2000) -- (83.9000,150.2000) -- (83.9000,150.2000) -- (83.9000,150.2000) -- (83.9000,150.2000) -- (83.9000,150.2000) -- (83.9000,150.2000) -- (83.9000,150.2000) -- (83.9000,150.2000) -- (84.0000,150.2000) -- (84.0000,150.2000) -- (84.0000,150.2000) -- (84.0000,150.2000) -- (84.0000,150.2000) -- (84.0000,150.2000) -- (84.0000,150.2000) -- (84.0000,150.2000) -- (84.0000,150.2000) -- (84.0000,150.2000) -- (84.0000,150.2000) -- (84.0000,150.2000) -- (84.0000,150.1000) -- (84.0000,150.1000) -- (84.0000,150.1000) -- (84.0000,150.1000) -- (84.0000,150.1000) -- (84.0000,150.1000) -- (84.0000,150.1000) -- (84.0000,150.1000) -- (84.0000,150.1000) -- (84.0000,150.1000) -- (84.0000,150.1000) -- (84.0000,150.1000) -- (84.0000,150.1000) -- (84.0000,150.1000) -- (84.0000,150.1000) -- (84.0000,150.1000) -- (84.0000,150.1000) -- (84.0000,150.1000) -- (84.0000,150.1000) -- (84.0000,150.1000) -- (84.0000,150.1000) -- (84.0000,150.0000) -- (84.0000,150.0000) -- (84.0000,150.0000) -- (84.0000,150.0000) -- (84.0000,150.0000) -- (84.0000,150.0000) -- (84.0000,150.0000) -- (84.0000,150.0000) -- (84.0000,150.0000) -- (84.0000,150.0000) -- (84.0000,150.0000) -- (84.0000,150.0000) -- (84.0000,150.0000) -- (84.0000,150.0000) -- (84.0000,150.0000) -- (84.0000,150.0000) -- (84.1000,150.0000) -- (84.1000,150.0000) -- (84.1000,150.0000) -- (84.1000,150.0000) -- (84.1000,150.0000) -- (84.1000,150.0000) -- (84.1000,149.9000) -- (84.1000,149.9000) -- (84.1000,149.9000) -- (84.1000,149.9000) -- (84.1000,149.9000) -- (84.1000,149.9000) -- (84.1000,149.9000) -- (84.1000,149.9000) -- (84.1000,149.9000) -- (84.1000,149.9000) -- (84.1000,149.9000) -- (84.1000,149.9000) -- (84.1000,149.9000) -- (84.1000,149.9000) -- (84.1000,149.9000) -- (84.1000,149.9000) -- (84.1000,149.9000) -- (84.1000,149.9000) -- (84.1000,149.9000) -- (84.1000,149.9000) -- (84.1000,149.9000) -- (84.1000,149.8000) -- (84.1000,149.8000) -- (84.1000,149.8000) -- (84.1000,149.8000) -- (84.1000,149.8000) -- (84.1000,149.8000) -- (84.1000,149.8000) -- (84.1000,149.8000) -- (84.1000,149.8000) -- (84.1000,149.8000) -- (84.1000,149.8000) -- (84.1000,149.8000) -- (84.1000,149.8000) -- (84.1000,149.8000) -- (84.1000,149.8000) -- (84.1000,149.8000) -- (84.1000,149.8000) -- (84.1000,149.8000) -- (84.1000,149.8000) -- (84.1000,149.8000) -- (84.1000,149.8000) -- (84.1000,149.8000) -- (84.1000,149.7000) -- (84.2000,149.7000) -- (84.2000,149.7000) -- (84.2000,149.7000) -- (84.2000,149.7000) -- (84.2000,149.7000) -- (84.2000,149.7000) -- (84.2000,149.7000) -- (84.2000,149.7000) -- (84.2000,149.7000) -- (84.2000,149.7000) -- (84.2000,149.7000) -- (84.2000,149.7000) -- (84.2000,149.7000) -- (84.2000,149.7000) -- (84.2000,149.7000) -- (84.2000,149.7000) -- (84.2000,149.7000) -- (84.2000,149.7000) -- (84.2000,149.7000) -- (84.2000,149.7000) -- (84.2000,149.6000) -- (84.2000,149.6000) -- (84.2000,149.6000) -- (84.2000,149.6000) -- (84.2000,149.6000) -- (84.2000,149.6000) -- (84.2000,149.6000) -- (84.2000,149.6000) -- (84.2000,149.6000) -- (84.2000,149.6000) -- (84.2000,149.6000) -- (84.2000,149.6000) -- (84.2000,149.6000) -- (84.2000,149.6000) -- (84.2000,149.6000) -- (84.2000,149.6000) -- (84.2000,149.6000) -- (84.2000,149.6000) -- (84.2000,149.6000) -- (84.2000,149.6000) -- (84.2000,149.6000) -- (84.2000,149.6000) -- (84.2000,149.5000) -- (84.2000,149.5000) -- (84.2000,149.5000) -- (84.2000,149.5000) -- (84.2000,149.5000) -- (84.2000,149.5000) -- (84.2000,149.5000) -- (84.3000,149.5000) -- (84.3000,149.5000) -- (84.3000,149.5000) -- (84.3000,149.5000) -- (84.3000,149.5000) -- (84.3000,149.5000) -- (84.3000,149.5000) -- (84.3000,149.5000) -- (84.3000,149.5000) -- (84.3000,149.5000) -- (84.3000,149.5000) -- (84.3000,149.5000) -- (84.3000,149.5000) -- (84.3000,149.5000) -- (84.3000,149.4000) -- (84.3000,149.4000) -- (84.3000,149.4000) -- (84.3000,149.4000) -- (84.3000,149.4000) -- (84.3000,149.4000) -- (84.3000,149.4000) -- (84.3000,149.4000) -- (84.3000,149.4000) -- (84.3000,149.4000) -- (84.3000,149.4000) -- (84.3000,149.4000) -- (84.3000,149.4000) -- (84.3000,149.4000) -- (84.3000,149.4000) -- (84.3000,149.4000) -- (84.3000,149.4000) -- (84.3000,149.4000) -- (84.3000,149.4000) -- (84.3000,149.4000) -- (84.3000,149.4000) -- (84.3000,149.4000) -- (84.3000,149.3000) -- (84.3000,149.3000) -- (84.3000,149.3000) -- (84.3000,149.3000) -- (84.3000,149.3000) -- (84.3000,149.3000) -- (84.3000,149.3000) -- (84.3000,149.3000) -- (84.3000,149.3000) -- (84.3000,149.3000) -- (84.3000,149.3000) -- (84.3000,149.3000) -- (84.3000,149.3000) -- (84.3000,149.3000) -- (84.4000,149.3000) -- (84.4000,149.3000) -- (84.4000,149.3000) -- (84.4000,149.3000) -- (84.4000,149.3000) -- (84.4000,149.3000) -- (84.4000,149.3000) -- (84.4000,149.2000) -- (84.4000,149.2000) -- (84.4000,149.2000) -- (84.4000,149.2000) -- (84.4000,149.2000) -- (84.4000,149.2000) -- (84.4000,149.2000) -- (84.4000,149.2000) -- (84.4000,149.2000) -- (84.4000,149.2000) -- (84.4000,149.2000) -- (84.4000,149.2000) -- (84.4000,149.2000) -- (84.4000,149.2000) -- (84.4000,149.2000) -- (84.4000,149.2000) -- (84.4000,149.2000) -- (84.4000,149.2000) -- (84.4000,149.2000) -- (84.4000,149.2000) -- (84.4000,149.2000) -- (84.4000,149.2000) -- (84.4000,149.1000) -- (84.4000,149.1000) -- (84.4000,149.1000) -- (84.4000,149.1000) -- (84.4000,149.1000) -- (84.4000,149.1000) -- (84.4000,149.1000) -- (84.4000,149.1000) -- (84.4000,149.1000) -- (84.4000,149.1000) -- (84.4000,149.1000) -- (84.4000,149.1000) -- (84.4000,149.1000) -- (84.4000,149.1000) -- (84.4000,149.1000) -- (84.4000,149.1000) -- (84.4000,149.1000) -- (84.4000,149.1000) -- (84.4000,149.1000) -- (84.4000,149.1000) -- (84.5000,149.1000) -- (84.5000,149.0000) -- (84.5000,149.0000) -- (84.5000,149.0000) -- (84.5000,149.0000) -- (84.5000,149.0000) -- (84.5000,149.0000) -- (84.5000,149.0000) -- (84.5000,149.0000) -- (84.5000,149.0000) -- (84.5000,149.0000) -- (84.5000,149.0000) -- (84.5000,149.0000) -- (84.5000,149.0000) -- (84.5000,149.0000) -- (84.5000,149.0000) -- (84.5000,149.0000) -- (84.5000,149.0000) -- (84.5000,149.0000) -- (84.5000,149.0000) -- (84.5000,149.0000) -- (84.5000,149.0000) -- (84.5000,149.0000) -- (84.5000,148.9000) -- (84.5000,148.9000) -- (84.5000,148.9000) -- (84.5000,148.9000) -- (84.5000,148.9000) -- (84.5000,148.9000) -- (84.5000,148.9000) -- (84.5000,148.9000) -- (84.5000,148.9000) -- (84.5000,148.9000) -- (84.5000,148.9000) -- (84.5000,148.9000) -- (84.5000,148.9000) -- (84.5000,148.9000) -- (84.5000,148.9000) -- (84.5000,148.9000) -- (84.5000,148.9000) -- (84.5000,148.9000) -- (84.5000,148.9000) -- (84.5000,148.9000) -- (84.5000,148.9000) -- (84.5000,148.8000) -- (84.5000,148.8000) -- (84.5000,148.8000) -- (84.5000,148.8000) -- (84.5000,148.8000) -- (84.5000,148.8000) -- (84.6000,148.8000) -- (84.6000,148.8000) -- (84.6000,148.8000) -- (84.6000,148.8000) -- (84.6000,148.8000) -- (84.6000,148.8000) -- (84.6000,148.8000) -- (84.6000,148.8000) -- (84.6000,148.8000) -- (84.6000,148.8000) -- (84.6000,148.8000) -- (84.6000,148.8000) -- (84.6000,148.8000) -- (84.6000,148.8000) -- (84.6000,148.8000) -- (84.6000,148.8000) -- (84.6000,148.7000) -- (84.6000,148.7000) -- (84.6000,148.7000) -- (84.6000,148.7000) -- (84.6000,148.7000) -- (84.6000,148.7000) -- (84.6000,148.7000) -- (84.6000,148.7000) -- (84.6000,148.7000) -- (84.6000,148.7000) -- (84.6000,148.7000) -- (84.6000,148.7000) -- (84.6000,148.7000) -- (84.6000,148.7000) -- (84.6000,148.7000) -- (84.6000,148.7000) -- (84.6000,148.7000) -- (84.6000,148.7000) -- (84.6000,148.7000) -- (84.6000,148.7000) -- (84.6000,148.7000) -- (84.6000,148.6000) -- (84.6000,148.6000) -- (84.6000,148.6000) -- (84.6000,148.6000) -- (84.6000,148.6000) -- (84.6000,148.6000) -- (84.6000,148.6000) -- (84.6000,148.6000) -- (84.6000,148.6000) -- (84.6000,148.6000) -- (84.6000,148.6000) -- (84.6000,148.6000) -- (84.7000,148.6000) -- (84.7000,148.6000) -- (84.7000,148.6000) -- (84.7000,148.6000) -- (84.7000,148.6000) -- (84.7000,148.6000) -- (84.7000,148.6000) -- (84.7000,148.6000) -- (84.7000,148.6000) -- (84.7000,148.6000) -- (84.7000,148.5000) -- (84.7000,148.5000) -- (84.7000,148.5000) -- (84.7000,148.5000) -- (84.7000,148.5000) -- (84.7000,148.5000) -- (84.7000,148.5000) -- (84.7000,148.5000) -- (84.7000,148.5000) -- (84.7000,148.5000) -- (84.7000,148.5000) -- (84.7000,148.5000) -- (84.7000,148.5000) -- (84.7000,148.5000) -- (84.7000,148.5000) -- (84.7000,148.5000) -- (84.7000,148.5000) -- (84.7000,148.5000) -- (84.7000,148.5000) -- (84.7000,148.5000) -- (84.7000,148.5000) -- (84.7000,148.4000) -- (84.7000,148.4000) -- (84.7000,148.4000) -- (84.7000,148.4000) -- (84.7000,148.4000) -- (84.7000,148.4000) -- (84.7000,148.4000) -- (84.7000,148.4000) -- (84.7000,148.4000) -- (84.7000,148.4000) -- (84.7000,148.4000) -- (84.7000,148.4000) -- (84.7000,148.4000) -- (84.7000,148.4000) -- (84.7000,148.4000) -- (84.7000,148.4000) -- (84.7000,148.4000) -- (84.7000,148.4000) -- (84.7000,148.4000) -- (84.8000,148.4000) -- (84.8000,148.4000) -- (84.8000,148.4000) -- (84.8000,148.3000) -- (84.8000,148.3000) -- (84.8000,148.3000) -- (84.8000,148.3000) -- (84.8000,148.3000) -- (84.8000,148.3000) -- (84.8000,148.3000) -- (84.8000,148.3000) -- (84.8000,148.3000) -- (84.8000,148.3000) -- (84.8000,148.3000) -- (84.8000,148.3000) -- (84.8000,148.3000) -- (84.8000,148.3000) -- (84.8000,148.3000) -- (84.8000,148.3000) -- (84.8000,148.3000) -- (84.8000,148.3000) -- (84.8000,148.3000) -- (84.8000,148.3000) -- (84.8000,148.3000) -- (84.8000,148.2000) -- (84.8000,148.2000) -- (84.8000,148.2000) -- (84.8000,148.2000) -- (84.8000,148.2000) -- (84.8000,148.2000) -- (84.8000,148.2000) -- (84.8000,148.2000) -- (84.8000,148.2000) -- (84.8000,148.2000) -- (84.8000,148.2000) -- (84.8000,148.2000) -- (84.8000,148.2000) -- (84.8000,148.2000) -- (84.8000,148.2000) -- (84.8000,148.2000) -- (84.8000,148.2000) -- (84.8000,148.2000) -- (84.8000,148.2000) -- (84.8000,148.2000) -- (84.8000,148.2000) -- (84.8000,148.2000) -- (84.8000,148.1000) -- (84.8000,148.1000) -- (84.8000,148.1000) -- (84.9000,148.1000) -- (84.9000,148.1000) -- (84.9000,148.1000) -- (84.9000,148.1000) -- (84.9000,148.1000) -- (84.9000,148.1000) -- (84.9000,148.1000) -- (84.9000,148.1000) -- (84.9000,148.1000) -- (84.9000,148.1000) -- (84.9000,148.1000) -- (84.9000,148.1000) -- (84.9000,148.1000) -- (84.9000,148.1000) -- (84.9000,148.1000) -- (84.9000,148.1000) -- (84.9000,148.1000) -- (84.9000,148.1000) -- (84.9000,148.0000) -- (84.9000,148.0000) -- (84.9000,148.0000) -- (84.9000,148.0000) -- (84.9000,148.0000) -- (84.9000,148.0000) -- (84.9000,148.0000) -- (84.9000,148.0000) -- (84.9000,148.0000) -- (84.9000,148.0000) -- (84.9000,148.0000) -- (84.9000,148.0000) -- (84.9000,148.0000) -- (84.9000,148.0000) -- (84.9000,148.0000) -- (84.9000,148.0000) -- (84.9000,148.0000) -- (84.9000,148.0000) -- (84.9000,148.0000) -- (84.9000,148.0000) -- (84.9000,148.0000) -- (84.9000,148.0000) -- (84.9000,147.9000) -- (84.9000,147.9000) -- (84.9000,147.9000) -- (84.9000,147.9000) -- (84.9000,147.9000) -- (84.9000,147.9000) -- (84.9000,147.9000) -- (84.9000,147.9000) -- (84.9000,147.9000) -- (84.9000,147.9000) -- (85.0000,147.9000) -- (85.0000,147.9000) -- (85.0000,147.9000) -- (85.0000,147.9000) -- (85.0000,147.9000) -- (85.0000,147.9000) -- (85.0000,147.9000) -- (85.0000,147.9000) -- (85.0000,147.9000) -- (85.0000,147.9000) -- (85.0000,147.9000) -- (85.0000,147.8000) -- (85.0000,147.8000) -- (85.0000,147.8000) -- (85.0000,147.8000) -- (85.0000,147.8000) -- (85.0000,147.8000) -- (85.0000,147.8000) -- (85.0000,147.8000) -- (85.0000,147.8000) -- (85.0000,147.8000) -- (85.0000,147.8000) -- (85.0000,147.8000) -- (85.0000,147.8000) -- (85.0000,147.8000) -- (85.0000,147.8000) -- (85.0000,147.8000) -- (85.0000,147.8000) -- (85.0000,147.8000) -- (85.0000,147.8000) -- (85.0000,147.8000) -- (85.0000,147.8000) -- (85.0000,147.8000) -- (85.0000,147.7000) -- (85.0000,147.7000) -- (85.0000,147.7000) -- (85.0000,147.7000) -- (85.0000,147.7000) -- (85.0000,147.7000) -- (85.0000,147.7000) -- (85.0000,147.7000) -- (85.0000,147.7000) -- (85.0000,147.7000) -- (85.0000,147.7000) -- (85.0000,147.7000) -- (85.0000,147.7000) -- (85.0000,147.7000) -- (85.0000,147.7000) -- (85.0000,147.7000) -- (85.1000,147.7000) -- (85.1000,147.7000) -- (85.1000,147.7000) -- (85.1000,147.7000) -- (85.1000,147.7000) -- (85.1000,147.6000) -- (85.1000,147.6000) -- (85.1000,147.6000) -- (85.1000,147.6000) -- (85.1000,147.6000) -- (85.1000,147.6000) -- (85.1000,147.6000) -- (85.1000,147.6000) -- (85.1000,147.6000) -- (85.1000,147.6000) -- (85.1000,147.6000) -- (85.1000,147.6000) -- (85.1000,147.6000) -- (85.1000,147.6000) -- (85.1000,147.6000) -- (85.1000,147.6000) -- (85.1000,147.6000) -- (85.1000,147.6000) -- (85.1000,147.6000) -- (85.1000,147.6000) -- (85.1000,147.6000) -- (85.1000,147.6000) -- (85.1000,147.5000) -- (85.1000,147.5000) -- (85.1000,147.5000) -- (85.1000,147.5000) -- (85.1000,147.5000) -- (85.1000,147.5000) -- (85.1000,147.5000) -- (85.1000,147.5000) -- (85.1000,147.5000) -- (85.1000,147.5000) -- (85.1000,147.5000) -- (85.1000,147.5000) -- (85.1000,147.5000) -- (85.1000,147.5000) -- (85.1000,147.5000) -- (85.1000,147.5000) -- (85.1000,147.5000) -- (85.1000,147.5000) -- (85.1000,147.5000) -- (85.1000,147.5000) -- (85.1000,147.5000) -- (85.1000,147.4000) -- (85.1000,147.4000) -- (85.2000,147.4000) -- (85.2000,147.4000) -- (85.2000,147.4000) -- (85.2000,147.4000) -- (85.2000,147.4000) -- (85.2000,147.4000) -- (85.2000,147.4000) -- (85.2000,147.4000) -- (85.2000,147.4000) -- (85.2000,147.4000) -- (85.2000,147.4000) -- (85.2000,147.4000) -- (85.2000,147.4000) -- (85.2000,147.4000) -- (85.2000,147.4000) -- (85.2000,147.4000) -- (85.2000,147.4000) -- (85.2000,147.4000) -- (85.2000,147.4000) -- (85.2000,147.4000) -- (85.2000,147.3000) -- (85.2000,147.3000) -- (85.2000,147.3000) -- (85.2000,147.3000) -- (85.2000,147.3000) -- (85.2000,147.3000) -- (85.2000,147.3000) -- (85.2000,147.3000) -- (85.2000,147.3000) -- (85.2000,147.3000) -- (85.2000,147.3000) -- (85.2000,147.3000) -- (85.2000,147.3000) -- (85.2000,147.3000) -- (85.2000,147.3000) -- (85.2000,147.3000) -- (85.2000,147.3000) -- (85.2000,147.3000) -- (85.2000,147.3000) -- (85.2000,147.3000) -- (85.2000,147.3000) -- (85.2000,147.2000) -- (85.2000,147.2000) -- (85.2000,147.2000) -- (85.2000,147.2000) -- (85.2000,147.2000) -- (85.2000,147.2000) -- (85.2000,147.2000) -- (85.2000,147.2000) -- (85.3000,147.2000) -- (85.3000,147.2000) -- (85.3000,147.2000) -- (85.3000,147.2000) -- (85.3000,147.2000) -- (85.3000,147.2000) -- (85.3000,147.2000) -- (85.3000,147.2000) -- (85.3000,147.2000) -- (85.3000,147.2000) -- (85.3000,147.2000) -- (85.3000,147.2000) -- (85.3000,147.2000) -- (85.3000,147.2000) -- (85.3000,147.1000) -- (85.3000,147.1000) -- (85.3000,147.1000) -- (85.3000,147.1000) -- (85.3000,147.1000) -- (85.3000,147.1000) -- (85.3000,147.1000) -- (85.3000,147.1000) -- (85.3000,147.1000) -- (85.3000,147.1000) -- (85.3000,147.1000) -- (85.3000,147.1000) -- (85.3000,147.1000) -- (85.3000,147.1000) -- (85.3000,147.1000) -- (85.3000,147.1000) -- (85.3000,147.1000) -- (85.3000,147.1000) -- (85.3000,147.1000) -- (85.3000,147.1000) -- (85.3000,147.1000) -- (85.3000,147.0000) -- (85.3000,147.0000) -- (85.3000,147.0000) -- (85.3000,147.0000) -- (85.3000,147.0000) -- (85.3000,147.0000) -- (85.3000,147.0000) -- (85.3000,147.0000) -- (85.3000,147.0000) -- (85.3000,147.0000) -- (85.3000,147.0000) -- (85.3000,147.0000) -- (85.3000,147.0000) -- (85.3000,147.0000) -- (85.3000,147.0000) -- (85.4000,147.0000) -- (85.4000,147.0000) -- (85.4000,147.0000) -- (85.4000,147.0000) -- (85.4000,147.0000) -- (85.4000,147.0000) -- (85.4000,147.0000) -- (85.4000,146.9000) -- (85.4000,146.9000) -- (85.4000,146.9000) -- (85.4000,146.9000) -- (85.4000,146.9000) -- (85.4000,146.9000) -- (85.4000,146.9000) -- (85.4000,146.9000) -- (85.4000,146.9000) -- (85.4000,146.9000) -- (85.4000,146.9000) -- (85.4000,146.9000) -- (85.4000,146.9000) -- (85.4000,146.9000) -- (85.4000,146.9000) -- (85.4000,146.9000) -- (85.4000,146.9000) -- (85.4000,146.9000) -- (85.4000,146.9000) -- (85.4000,146.9000) -- (85.4000,146.9000) -- (85.4000,146.8000) -- (85.4000,146.8000) -- (85.4000,146.8000) -- (85.4000,146.8000) -- (85.4000,146.8000) -- (85.4000,146.8000) -- (85.4000,146.8000) -- (85.4000,146.8000) -- (85.4000,146.8000) -- (85.4000,146.8000) -- (85.4000,146.8000) -- (85.4000,146.8000) -- (85.4000,146.8000) -- (85.4000,146.8000) -- (85.4000,146.8000) -- (85.4000,146.8000) -- (85.4000,146.8000) -- (85.4000,146.8000) -- (85.4000,146.8000) -- (85.4000,146.8000) -- (85.4000,146.8000) -- (85.5000,146.8000) -- (85.5000,146.7000) -- (85.5000,146.7000) -- (85.5000,146.7000) -- (85.5000,146.7000) -- (85.5000,146.7000) -- (85.5000,146.7000) -- (85.5000,146.7000) -- (85.5000,146.7000) -- (85.5000,146.7000) -- (85.5000,146.7000) -- (85.5000,146.7000) -- (85.5000,146.7000) -- (85.5000,146.7000) -- (85.5000,146.7000) -- (85.5000,146.7000) -- (85.5000,146.7000) -- (85.5000,146.7000) -- (85.5000,146.7000) -- (85.5000,146.7000) -- (85.5000,146.7000) -- (85.5000,146.7000) -- (85.5000,146.6000) -- (85.5000,146.6000) -- (85.5000,146.6000) -- (85.5000,146.6000) -- (85.5000,146.6000) -- (85.5000,146.6000) -- (85.5000,146.6000) -- (85.5000,146.6000) -- (85.5000,146.6000) -- (85.5000,146.6000) -- (85.5000,146.6000) -- (85.5000,146.6000) -- (85.5000,146.6000) -- (85.5000,146.6000) -- (85.5000,146.6000) -- (85.5000,146.6000) -- (85.5000,146.6000) -- (85.5000,146.6000) -- (85.5000,146.6000) -- (85.5000,146.6000) -- (85.5000,146.6000) -- (85.5000,146.6000) -- (85.5000,146.5000) -- (85.5000,146.5000) -- (85.5000,146.5000) -- (85.5000,146.5000) -- (85.5000,146.5000) -- (85.5000,146.5000) -- (85.6000,146.5000) -- (85.6000,146.5000) -- (85.6000,146.5000) -- (85.6000,146.5000) -- (85.6000,146.5000) -- (85.6000,146.5000) -- (85.6000,146.5000) -- (85.6000,146.5000) -- (85.6000,146.5000) -- (85.6000,146.5000) -- (85.6000,146.5000) -- (85.6000,146.5000) -- (85.6000,146.5000) -- (85.6000,146.5000) -- (85.6000,146.5000) -- (85.6000,146.4000) -- (85.6000,146.4000) -- (85.6000,146.4000) -- (85.6000,146.4000) -- (85.6000,146.4000) -- (85.6000,146.4000) -- (85.6000,146.4000) -- (85.6000,146.4000) -- (85.6000,146.4000) -- (85.6000,146.4000) -- (85.6000,146.4000) -- (85.6000,146.4000) -- (85.6000,146.4000) -- (85.6000,146.4000) -- (85.6000,146.4000) -- (85.6000,146.4000) -- (85.6000,146.4000) -- (85.6000,146.4000) -- (85.6000,146.4000) -- (85.6000,146.4000) -- (85.6000,146.4000) -- (85.6000,146.4000) -- (85.6000,146.3000) -- (85.6000,146.3000) -- (85.6000,146.3000) -- (85.6000,146.3000) -- (85.6000,146.3000) -- (85.6000,146.3000) -- (85.6000,146.3000) -- (85.6000,146.3000) -- (85.6000,146.3000) -- (85.6000,146.3000) -- (85.6000,146.3000) -- (85.6000,146.3000) -- (85.7000,146.3000) -- (85.7000,146.3000) -- (85.7000,146.3000) -- (85.7000,146.3000) -- (85.7000,146.3000) -- (85.7000,146.3000) -- (85.7000,146.3000) -- (85.7000,146.3000) -- (85.7000,146.3000) -- (85.7000,146.2000) -- (85.7000,146.2000) -- (85.7000,146.2000) -- (85.7000,146.2000) -- (85.7000,146.2000) -- (85.7000,146.2000) -- (85.7000,146.2000) -- (85.7000,146.2000) -- (85.7000,146.2000) -- (85.7000,146.2000) -- (85.7000,146.2000) -- (85.7000,146.2000) -- (85.7000,146.2000) -- (85.7000,146.2000) -- (85.7000,146.2000) -- (85.7000,146.2000) -- (85.7000,146.2000) -- (85.7000,146.2000) -- (85.7000,146.2000) -- (85.7000,146.2000) -- (85.7000,146.2000) -- (85.7000,146.2000) -- (85.7000,146.1000) -- (85.7000,146.1000) -- (85.7000,146.1000) -- (85.7000,146.1000) -- (85.7000,146.1000) -- (85.7000,146.1000) -- (85.7000,146.1000) -- (85.7000,146.1000) -- (85.7000,146.1000) -- (85.7000,146.1000) -- (85.7000,146.1000) -- (85.7000,146.1000) -- (85.7000,146.1000) -- (85.7000,146.1000) -- (85.7000,146.1000) -- (85.7000,146.1000) -- (85.7000,146.1000) -- (85.7000,146.1000) -- (85.7000,146.1000) -- (85.8000,146.1000) -- (85.8000,146.1000) -- (85.8000,146.0000) -- (85.8000,146.0000) -- (85.8000,146.0000) -- (85.8000,146.0000) -- (85.8000,146.0000) -- (85.8000,146.0000) -- (85.8000,146.0000) -- (85.8000,146.0000) -- (85.8000,146.0000) -- (85.8000,146.0000) -- (85.8000,146.0000) -- (85.8000,146.0000) -- (85.8000,146.0000) -- (85.8000,146.0000) -- (85.8000,146.0000) -- (85.8000,146.0000) -- (85.8000,146.0000) -- (85.8000,146.0000) -- (85.8000,146.0000) -- (85.8000,146.0000) -- (85.8000,146.0000) -- (85.8000,146.0000) -- (85.8000,145.9000) -- (85.8000,145.9000) -- (85.8000,145.9000) -- (85.8000,145.9000) -- (85.8000,145.9000) -- (85.8000,145.9000) -- (85.8000,145.9000) -- (85.8000,145.9000) -- (85.8000,145.9000) -- (85.8000,145.9000) -- (85.8000,145.9000) -- (85.8000,145.9000) -- (85.8000,145.9000) -- (85.8000,145.9000) -- (85.8000,145.9000) -- (85.8000,145.9000) -- (85.8000,145.9000) -- (85.8000,145.9000) -- (85.8000,145.9000) -- (85.8000,145.9000) -- (85.8000,145.9000) -- (85.8000,145.8000) -- (85.8000,145.8000) -- (85.8000,145.8000) -- (85.8000,145.8000) -- (85.9000,145.8000) -- (85.9000,145.8000) -- (85.9000,145.8000) -- (85.9000,145.8000) -- (85.9000,145.8000) -- (85.9000,145.8000) -- (85.9000,145.8000) -- (85.9000,145.8000) -- (85.9000,145.8000) -- (85.9000,145.8000) -- (85.9000,145.8000) -- (85.9000,145.8000) -- (85.9000,145.8000) -- (85.9000,145.8000) -- (85.9000,145.8000) -- (85.9000,145.8000) -- (85.9000,145.8000) -- (85.9000,145.8000) -- (85.9000,145.7000) -- (85.9000,145.7000) -- (85.9000,145.7000) -- (85.9000,145.7000) -- (85.9000,145.7000) -- (85.9000,145.7000) -- (85.9000,145.7000) -- (85.9000,145.7000) -- (85.9000,145.7000) -- (85.9000,145.7000) -- (85.9000,145.7000) -- (85.9000,145.7000) -- (85.9000,145.7000) -- (85.9000,145.7000) -- (85.9000,145.7000) -- (85.9000,145.7000) -- (85.9000,145.7000) -- (85.9000,145.7000) -- (85.9000,145.7000) -- (85.9000,145.7000) -- (85.9000,145.7000) -- (85.9000,145.6000) -- (85.9000,145.6000) -- (85.9000,145.6000) -- (85.9000,145.6000) -- (85.9000,145.6000) -- (85.9000,145.6000) -- (85.9000,145.6000) -- (85.9000,145.6000) -- (85.9000,145.6000) -- (85.9000,145.6000) -- (85.9000,145.6000) -- (86.0000,145.6000) -- (86.0000,145.6000) -- (86.0000,145.6000) -- (86.0000,145.6000) -- (86.0000,145.6000) -- (86.0000,145.6000) -- (86.0000,145.6000) -- (86.0000,145.6000) -- (86.0000,145.6000) -- (86.0000,145.6000) -- (86.0000,145.6000) -- (86.0000,145.5000) -- (86.0000,145.5000) -- (86.0000,145.5000) -- (86.0000,145.5000) -- (86.0000,145.5000) -- (86.0000,145.5000) -- (86.0000,145.5000) -- (86.0000,145.5000) -- (86.0000,145.5000) -- (86.0000,145.5000) -- (86.0000,145.5000) -- (86.0000,145.5000) -- (86.0000,145.5000) -- (86.0000,145.5000) -- (86.0000,145.5000) -- (86.0000,145.5000) -- (86.0000,145.5000) -- (86.0000,145.5000) -- (86.0000,145.5000) -- (86.0000,145.5000) -- (86.0000,145.5000) -- (86.0000,145.4000) -- (86.0000,145.4000) -- (86.0000,145.4000) -- (86.0000,145.4000) -- (86.0000,145.4000) -- (86.0000,145.4000) -- (86.0000,145.4000) -- (86.0000,145.4000) -- (86.0000,145.4000) -- (86.0000,145.4000) -- (86.0000,145.4000) -- (86.0000,145.4000) -- (86.0000,145.4000) -- (86.0000,145.4000) -- (86.0000,145.4000) -- (86.0000,145.4000) -- (86.0000,145.4000) -- (86.1000,145.4000) -- (86.1000,145.4000) -- (86.1000,145.4000) -- (86.1000,145.4000) -- (86.1000,145.4000) -- (86.1000,145.3000) -- (86.1000,145.3000) -- (86.1000,145.3000) -- (86.1000,145.3000) -- (86.1000,145.3000) -- (86.1000,145.3000) -- (86.1000,145.3000) -- (86.1000,145.3000) -- (86.1000,145.3000) -- (86.1000,145.3000) -- (86.1000,145.3000) -- (86.1000,145.3000) -- (86.1000,145.3000) -- (86.1000,145.3000) -- (86.1000,145.3000) -- (86.1000,145.3000) -- (86.1000,145.3000) -- (86.1000,145.3000) -- (86.1000,145.3000) -- (86.1000,145.3000) -- (86.1000,145.3000) -- (86.1000,145.2000) -- (86.1000,145.2000) -- (86.1000,145.2000) -- (86.1000,145.2000) -- (86.1000,145.2000) -- (86.1000,145.2000) -- (86.1000,145.2000) -- (86.1000,145.2000) -- (86.1000,145.2000) -- (86.1000,145.2000) -- (86.1000,145.2000) -- (86.1000,145.2000) -- (86.1000,145.2000) -- (86.1000,145.2000) -- (86.1000,145.2000) -- (86.1000,145.2000) -- (86.1000,145.2000) -- (86.1000,145.2000) -- (86.1000,145.2000) -- (86.1000,145.2000) -- (86.1000,145.2000) -- (86.1000,145.2000) -- (86.1000,145.1000) -- (86.1000,145.1000) -- (86.2000,145.1000) -- (86.2000,145.1000) -- (86.2000,145.1000) -- (86.2000,145.1000) -- (86.2000,145.1000) -- (86.2000,145.1000) -- (86.2000,145.1000) -- (86.2000,145.1000) -- (86.2000,145.1000) -- (86.2000,145.1000) -- (86.2000,145.1000) -- (86.2000,145.1000) -- (86.2000,145.1000) -- (86.2000,145.1000) -- (86.2000,145.1000) -- (86.2000,145.1000) -- (86.2000,145.1000) -- (86.2000,145.1000) -- (86.2000,145.1000) -- (86.2000,145.0000) -- (86.2000,145.0000) -- (86.2000,145.0000) -- (86.2000,145.0000) -- (86.2000,145.0000) -- (86.2000,145.0000) -- (86.2000,145.0000) -- (86.2000,145.0000) -- (86.2000,145.0000) -- (86.2000,145.0000) -- (86.2000,145.0000) -- (86.2000,145.0000) -- (86.2000,145.0000) -- (86.2000,145.0000) -- (86.2000,145.0000) -- (86.2000,145.0000) -- (86.2000,145.0000) -- (86.2000,145.0000) -- (86.2000,145.0000) -- (86.2000,145.0000) -- (86.2000,145.0000) -- (86.2000,145.0000) -- (86.2000,144.9000) -- (86.2000,144.9000) -- (86.2000,144.9000) -- (86.2000,144.9000) -- (86.2000,144.9000) -- (86.2000,144.9000) -- (86.2000,144.9000) -- (86.2000,144.9000) -- (86.2000,144.9000) -- (86.3000,144.9000) -- (86.3000,144.9000) -- (86.3000,144.9000) -- (86.3000,144.9000) -- (86.3000,144.9000) -- (86.3000,144.9000) -- (86.3000,144.9000) -- (86.3000,144.9000) -- (86.3000,144.9000) -- (86.3000,144.9000) -- (86.3000,144.9000) -- (86.3000,144.9000) -- (86.3000,144.8000) -- (86.3000,144.8000) -- (86.3000,144.8000) -- (86.3000,144.8000) -- (86.3000,144.8000) -- (86.3000,144.8000) -- (86.3000,144.8000) -- (86.3000,144.8000) -- (86.3000,144.8000) -- (86.3000,144.8000) -- (86.3000,144.8000) -- (86.3000,144.8000) -- (86.3000,144.8000) -- (86.3000,144.8000) -- (86.3000,144.8000) -- (86.3000,144.8000) -- (86.3000,144.8000) -- (86.3000,144.8000) -- (86.3000,144.8000) -- (86.3000,144.8000) -- (86.3000,144.8000) -- (86.3000,144.8000) -- (86.3000,144.7000) -- (86.3000,144.7000) -- (86.3000,144.7000) -- (86.3000,144.7000) -- (86.3000,144.7000) -- (86.3000,144.7000) -- (86.3000,144.7000) -- (86.3000,144.7000) -- (86.3000,144.7000) -- (86.3000,144.7000) -- (86.3000,144.7000) -- (86.3000,144.7000) -- (86.3000,144.7000) -- (86.3000,144.7000) -- (86.3000,144.7000) -- (86.4000,144.7000) -- (86.4000,144.7000) -- (86.4000,144.7000) -- (86.4000,144.7000) -- (86.4000,144.7000) -- (86.4000,144.7000) -- (86.4000,144.6000) -- (86.4000,144.6000) -- (86.4000,144.6000) -- (86.4000,144.6000) -- (86.4000,144.6000) -- (86.4000,144.6000) -- (86.4000,144.6000) -- (86.4000,144.6000) -- (86.4000,144.6000) -- (86.4000,144.6000) -- (86.4000,144.6000) -- (86.4000,144.6000) -- (86.4000,144.6000) -- (86.4000,144.6000) -- (86.4000,144.6000) -- (86.4000,144.6000) -- (86.4000,144.6000) -- (86.4000,144.6000) -- (86.4000,144.6000) -- (86.4000,144.6000) -- (86.4000,144.6000) -- (86.4000,144.6000) -- (86.4000,144.5000) -- (86.4000,144.5000) -- (86.4000,144.5000) -- (86.4000,144.5000) -- (86.4000,144.5000) -- (86.4000,144.5000) -- (86.4000,144.5000) -- (86.4000,144.5000) -- (86.4000,144.5000) -- (86.4000,144.5000) -- (86.4000,144.5000) -- (86.4000,144.5000) -- (86.4000,144.5000) -- (86.4000,144.5000) -- (86.4000,144.5000) -- (86.4000,144.5000) -- (86.4000,144.5000) -- (86.4000,144.5000) -- (86.4000,144.5000) -- (86.4000,144.5000) -- (86.4000,144.5000) -- (86.4000,144.4000) -- (86.5000,144.4000) -- (86.5000,144.4000) -- (86.5000,144.4000) -- (86.5000,144.4000) -- (86.5000,144.4000) -- (86.5000,144.4000) -- (86.5000,144.4000) -- (86.5000,144.4000) -- (86.5000,144.4000) -- (86.5000,144.4000) -- (86.5000,144.4000) -- (86.5000,144.4000) -- (86.5000,144.4000) -- (86.5000,144.4000) -- (86.5000,144.4000) -- (86.5000,144.4000) -- (86.5000,144.4000) -- (86.5000,144.4000) -- (86.5000,144.4000) -- (86.5000,144.4000) -- (86.5000,144.4000) -- (86.5000,144.3000) -- (86.5000,144.3000) -- (86.5000,144.3000) -- (86.5000,144.3000) -- (86.5000,144.3000) -- (86.5000,144.3000) -- (86.5000,144.3000) -- (86.5000,144.3000) -- (86.5000,144.3000) -- (86.5000,144.3000) -- (86.5000,144.3000) -- (86.5000,144.3000) -- (86.5000,144.3000) -- (86.5000,144.3000) -- (86.5000,144.3000) -- (86.5000,144.3000) -- (86.5000,144.3000) -- (86.5000,144.3000) -- (86.5000,144.3000) -- (86.5000,144.3000) -- (86.5000,144.3000) -- (86.5000,144.2000) -- (86.5000,144.2000) -- (86.5000,144.2000) -- (86.5000,144.2000) -- (86.5000,144.2000) -- (86.5000,144.2000) -- (86.5000,144.2000) -- (86.6000,144.2000) -- (86.6000,144.2000) -- (86.6000,144.2000) -- (86.6000,144.2000) -- (86.6000,144.2000) -- (86.6000,144.2000) -- (86.6000,144.2000) -- (86.6000,144.2000) -- (86.6000,144.2000) -- (86.6000,144.2000) -- (86.6000,144.2000) -- (86.6000,144.2000) -- (86.6000,144.2000) -- (86.6000,144.2000) -- (86.6000,144.2000) -- (86.6000,144.1000) -- (86.6000,144.1000) -- (86.6000,144.1000) -- (86.6000,144.1000) -- (86.6000,144.1000) -- (86.6000,144.1000) -- (86.6000,144.1000) -- (86.6000,144.1000) -- (86.6000,144.1000) -- (86.6000,144.1000) -- (86.6000,144.1000) -- (86.6000,144.1000) -- (86.6000,144.1000) -- (86.6000,144.1000) -- (86.6000,144.1000) -- (86.6000,144.1000) -- (86.6000,144.1000) -- (86.6000,144.1000) -- (86.6000,144.1000) -- (86.6000,144.1000) -- (86.6000,144.1000) -- (86.6000,144.0000) -- (86.6000,144.0000) -- (86.6000,144.0000) -- (86.6000,144.0000) -- (86.6000,144.0000) -- (86.6000,144.0000) -- (86.6000,144.0000) -- (86.6000,144.0000) -- (86.6000,144.0000) -- (86.6000,144.0000) -- (86.6000,144.0000) -- (86.6000,144.0000) -- (86.6000,144.0000) -- (86.6000,144.0000) -- (86.7000,144.0000) -- (86.7000,144.0000) -- (86.7000,144.0000) -- (86.7000,144.0000) -- (86.7000,144.0000) -- (86.7000,144.0000) -- (86.7000,144.0000) -- (86.7000,144.0000) -- (86.7000,143.9000) -- (86.7000,143.9000) -- (86.7000,143.9000) -- (86.7000,143.9000) -- (86.7000,143.9000) -- (86.7000,143.9000) -- (86.7000,143.9000) -- (86.7000,143.9000) -- (86.7000,143.9000) -- (86.7000,143.9000) -- (86.7000,143.9000) -- (86.7000,143.9000) -- (86.7000,143.9000) -- (86.7000,143.9000) -- (86.7000,143.9000) -- (86.7000,143.9000) -- (86.7000,143.9000) -- (86.7000,143.9000) -- (86.7000,143.9000) -- (86.7000,143.9000) -- (86.7000,143.9000) -- (86.7000,143.8000) -- (86.7000,143.8000) -- (86.7000,143.8000) -- (86.7000,143.8000) -- (86.7000,143.8000) -- (86.7000,143.8000) -- (86.7000,143.8000) -- (86.7000,143.8000) -- (86.7000,143.8000) -- (86.7000,143.8000) -- (86.7000,143.8000) -- (86.7000,143.8000) -- (86.7000,143.8000) -- (86.7000,143.8000) -- (86.7000,143.8000) -- (86.7000,143.8000) -- (86.7000,143.8000) -- (86.7000,143.8000) -- (86.7000,143.8000) -- (86.7000,143.8000) -- (86.8000,143.8000) -- (86.8000,143.8000) -- (86.8000,143.7000) -- (86.8000,143.7000) -- (86.8000,143.7000) -- (86.8000,143.7000) -- (86.8000,143.7000) -- (86.8000,143.7000) -- (86.8000,143.7000) -- (86.8000,143.7000) -- (86.8000,143.7000) -- (86.8000,143.7000) -- (86.8000,143.7000) -- (86.8000,143.7000) -- (86.8000,143.7000) -- (86.8000,143.7000) -- (86.8000,143.7000) -- (86.8000,143.7000) -- (86.8000,143.7000) -- (86.8000,143.7000) -- (86.8000,143.7000) -- (86.8000,143.7000) -- (86.8000,143.7000) -- (86.8000,143.6000) -- (86.8000,143.6000) -- (86.8000,143.6000) -- (86.8000,143.6000) -- (86.8000,143.6000) -- (86.8000,143.6000) -- (86.8000,143.6000) -- (86.8000,143.6000) -- (86.8000,143.6000) -- (86.8000,143.6000) -- (86.8000,143.6000) -- (86.8000,143.6000) -- (86.8000,143.6000) -- (86.8000,143.6000) -- (86.8000,143.6000) -- (86.8000,143.6000) -- (86.8000,143.6000) -- (86.8000,143.6000) -- (86.8000,143.6000) -- (86.8000,143.6000) -- (86.8000,143.6000) -- (86.8000,143.6000) -- (86.8000,143.5000) -- (86.8000,143.5000) -- (86.8000,143.5000) -- (86.8000,143.5000) -- (86.8000,143.5000) -- (86.9000,143.5000) -- (86.9000,143.5000) -- (86.9000,143.5000) -- (86.9000,143.5000) -- (86.9000,143.5000) -- (86.9000,143.5000) -- (86.9000,143.5000) -- (86.9000,143.5000) -- (86.9000,143.5000) -- (86.9000,143.5000) -- (86.9000,143.5000) -- (86.9000,143.5000) -- (86.9000,143.5000) -- (86.9000,143.5000) -- (86.9000,143.5000) -- (86.9000,143.5000) -- (86.9000,143.4000) -- (86.9000,143.4000) -- (86.9000,143.4000) -- (86.9000,143.4000) -- (86.9000,143.4000) -- (86.9000,143.4000) -- (86.9000,143.4000) -- (86.9000,143.4000) -- (86.9000,143.4000) -- (86.9000,143.4000) -- (86.9000,143.4000) -- (86.9000,143.4000) -- (86.9000,143.4000) -- (86.9000,143.4000) -- (86.9000,143.4000) -- (86.9000,143.4000) -- (86.9000,143.4000) -- (86.9000,143.4000) -- (86.9000,143.4000) -- (86.9000,143.4000) -- (86.9000,143.4000) -- (86.9000,143.4000) -- (86.9000,143.3000) -- (86.9000,143.3000) -- (86.9000,143.3000) -- (86.9000,143.3000) -- (86.9000,143.3000) -- (86.9000,143.3000) -- (86.9000,143.3000) -- (86.9000,143.3000) -- (86.9000,143.3000) -- (86.9000,143.3000) -- (86.9000,143.3000) -- (87.0000,143.3000) -- (87.0000,143.3000) -- (87.0000,143.3000) -- (87.0000,143.3000) -- (87.0000,143.3000) -- (87.0000,143.3000) -- (87.0000,143.3000) -- (87.0000,143.3000) -- (87.0000,143.3000) -- (87.0000,143.3000) -- (87.0000,143.2000) -- (87.0000,143.2000) -- (87.0000,143.2000) -- (87.0000,143.2000) -- (87.0000,143.2000) -- (87.0000,143.2000) -- (87.0000,143.2000) -- (87.0000,143.2000) -- (87.0000,143.2000) -- (87.0000,143.2000) -- (87.0000,143.2000) -- (87.0000,143.2000) -- (87.0000,143.2000) -- (87.0000,143.2000) -- (87.0000,143.2000) -- (87.0000,143.2000) -- (87.0000,143.2000) -- (87.0000,143.2000) -- (87.0000,143.2000) -- (87.0000,143.2000) -- (87.0000,143.2000) -- (87.0000,143.2000) -- (87.0000,143.1000) -- (87.0000,143.1000) -- (87.0000,143.1000) -- (87.0000,143.1000) -- (87.0000,143.1000) -- (87.0000,143.1000) -- (87.0000,143.1000) -- (87.0000,143.1000) -- (87.0000,143.1000) -- (87.0000,143.1000) -- (87.0000,143.1000) -- (87.0000,143.1000) -- (87.0000,143.1000) -- (87.0000,143.1000) -- (87.0000,143.1000) -- (87.0000,143.1000) -- (87.0000,143.1000) -- (87.0000,143.1000) -- (87.1000,143.1000) -- (87.1000,143.1000) -- (87.1000,143.1000) -- (87.1000,143.0000) -- (87.1000,143.0000) -- (87.1000,143.0000) -- (87.1000,143.0000) -- (87.1000,143.0000) -- (87.1000,143.0000) -- (87.1000,143.0000) -- (87.1000,143.0000) -- (87.1000,143.0000) -- (87.1000,143.0000) -- (87.1000,143.0000) -- (87.1000,143.0000) -- (87.1000,143.0000) -- (87.1000,143.0000) -- (87.1000,143.0000) -- (87.1000,143.0000) -- (87.1000,143.0000) -- (87.1000,143.0000) -- (87.1000,143.0000) -- (87.1000,143.0000) -- (87.1000,143.0000) -- (87.1000,143.0000) -- (87.1000,142.9000) -- (87.1000,142.9000) -- (87.1000,142.9000) -- (87.1000,142.9000) -- (87.1000,142.9000) -- (87.1000,142.9000) -- (87.1000,142.9000) -- (87.1000,142.9000) -- (87.1000,142.9000) -- (87.1000,142.9000) -- (87.1000,142.9000) -- (87.1000,142.9000) -- (87.1000,142.9000) -- (87.1000,142.9000) -- (87.1000,142.9000) -- (87.1000,142.9000) -- (87.1000,142.9000) -- (87.1000,142.9000) -- (87.1000,142.9000) -- (87.1000,142.9000) -- (87.1000,142.9000) -- (87.1000,142.9000) -- (87.1000,142.8000) -- (87.1000,142.8000) -- (87.2000,142.8000) -- (87.2000,142.8000) -- (87.2000,142.8000) -- (87.2000,142.8000) -- (87.2000,142.8000) -- (87.2000,142.8000) -- (87.2000,142.8000) -- (87.2000,142.8000) -- (87.2000,142.8000) -- (87.2000,142.8000) -- (87.2000,142.8000) -- (87.2000,142.8000) -- (87.2000,142.8000) -- (87.2000,142.8000) -- (87.2000,142.8000) -- (87.2000,142.8000) -- (87.2000,142.8000) -- (87.2000,142.8000) -- (87.2000,142.8000) -- (87.2000,142.7000) -- (87.2000,142.7000) -- (87.2000,142.7000) -- (87.2000,142.7000) -- (87.2000,142.7000) -- (87.2000,142.7000) -- (87.2000,142.7000) -- (87.2000,142.7000) -- (87.2000,142.7000) -- (87.2000,142.7000) -- (87.2000,142.7000) -- (87.2000,142.7000) -- (87.2000,142.7000) -- (87.2000,142.7000) -- (87.2000,142.7000) -- (87.2000,142.7000) -- (87.2000,142.7000) -- (87.2000,142.7000) -- (87.2000,142.7000) -- (87.2000,142.7000) -- (87.2000,142.7000) -- (87.2000,142.7000) -- (87.2000,142.6000) -- (87.2000,142.6000) -- (87.2000,142.6000) -- (87.2000,142.6000) -- (87.2000,142.6000) -- (87.2000,142.6000) -- (87.2000,142.6000) -- (87.2000,142.6000) -- (87.2000,142.6000) -- (87.3000,142.6000) -- (87.3000,142.6000) -- (87.3000,142.6000) -- (87.3000,142.6000) -- (87.3000,142.6000) -- (87.3000,142.6000) -- (87.3000,142.6000) -- (87.3000,142.6000) -- (87.3000,142.6000) -- (87.3000,142.6000) -- (87.3000,142.6000) -- (87.3000,142.6000) -- (87.3000,142.5000) -- (87.3000,142.5000) -- (87.3000,142.5000) -- (87.3000,142.5000) -- (87.3000,142.5000) -- (87.3000,142.5000) -- (87.3000,142.5000) -- (87.3000,142.5000) -- (87.3000,142.5000) -- (87.3000,142.5000) -- (87.3000,142.5000) -- (87.3000,142.5000) -- (87.3000,142.5000) -- (87.3000,142.5000) -- (87.3000,142.5000) -- (87.3000,142.5000) -- (87.3000,142.5000) -- (87.3000,142.5000) -- (87.3000,142.5000) -- (87.3000,142.5000) -- (87.3000,142.5000) -- (87.3000,142.5000) -- (87.3000,142.4000) -- (87.3000,142.4000) -- (87.3000,142.4000) -- (87.3000,142.4000) -- (87.3000,142.4000) -- (87.3000,142.4000) -- (87.3000,142.4000) -- (87.3000,142.4000) -- (87.3000,142.4000) -- (87.3000,142.4000) -- (87.3000,142.4000) -- (87.3000,142.4000) -- (87.3000,142.4000) -- (87.3000,142.4000) -- (87.3000,142.4000) -- (87.4000,142.4000) -- (87.4000,142.4000) -- (87.4000,142.4000) -- (87.4000,142.4000) -- (87.4000,142.4000) -- (87.4000,142.4000) -- (87.4000,142.3000) -- (87.4000,142.3000) -- (87.4000,142.3000) -- (87.4000,142.3000) -- (87.4000,142.3000) -- (87.4000,142.3000) -- (87.4000,142.3000) -- (87.4000,142.3000) -- (87.4000,142.3000) -- (87.4000,142.3000) -- (87.4000,142.3000) -- (87.4000,142.3000) -- (87.4000,142.3000) -- (87.4000,142.3000) -- (87.4000,142.3000) -- (87.4000,142.3000) -- (87.4000,142.3000) -- (87.4000,142.3000) -- (87.4000,142.3000) -- (87.4000,142.3000) -- (87.4000,142.3000) -- (87.4000,142.3000) -- (87.4000,142.2000) -- (87.4000,142.2000) -- (87.4000,142.2000) -- (87.4000,142.2000) -- (87.4000,142.2000) -- (87.4000,142.2000) -- (87.4000,142.2000) -- (87.4000,142.2000) -- (87.4000,142.2000) -- (87.4000,142.2000) -- (87.4000,142.2000) -- (87.4000,142.2000) -- (87.4000,142.2000) -- (87.4000,142.2000) -- (87.4000,142.2000) -- (87.4000,142.2000) -- (87.4000,142.2000) -- (87.4000,142.2000) -- (87.4000,142.2000) -- (87.4000,142.2000) -- (87.4000,142.2000) -- (87.4000,142.1000) -- (87.5000,142.1000) -- (87.5000,142.1000) -- (87.5000,142.1000) -- (87.5000,142.1000) -- (87.5000,142.1000) -- (87.5000,142.1000) -- (87.5000,142.1000) -- (87.5000,142.1000) -- (87.5000,142.1000) -- (87.5000,142.1000) -- (87.5000,142.1000) -- (87.5000,142.1000) -- (87.5000,142.1000) -- (87.5000,142.1000) -- (87.5000,142.1000) -- (87.5000,142.1000) -- (87.5000,142.1000) -- (87.5000,142.1000) -- (87.5000,142.1000) -- (87.5000,142.1000) -- (87.5000,142.1000) -- (87.5000,142.0000) -- (87.5000,142.0000) -- (87.5000,142.0000) -- (87.5000,142.0000) -- (87.5000,142.0000) -- (87.5000,142.0000) -- (87.5000,142.0000) -- (87.5000,142.0000) -- (87.5000,142.0000) -- (87.5000,142.0000) -- (87.5000,142.0000) -- (87.5000,142.0000) -- (87.5000,142.0000) -- (87.5000,142.0000) -- (87.5000,142.0000) -- (87.5000,142.0000) -- (87.5000,142.0000) -- (87.5000,142.0000) -- (87.5000,142.0000) -- (87.5000,142.0000) -- (87.5000,142.0000) -- (87.5000,141.9000) -- (87.5000,141.9000) -- (87.5000,141.9000) -- (87.5000,141.9000) -- (87.5000,141.9000) -- (87.5000,141.9000) -- (87.5000,141.9000) -- (87.6000,141.9000) -- (87.6000,141.9000) -- (87.6000,141.9000) -- (87.6000,141.9000) -- (87.6000,141.9000) -- (87.6000,141.9000) -- (87.6000,141.9000) -- (87.6000,141.9000) -- (87.6000,141.9000) -- (87.6000,141.9000) -- (87.6000,141.9000) -- (87.6000,141.9000) -- (87.6000,141.9000) -- (87.6000,141.9000) -- (87.6000,141.9000) -- (87.6000,141.8000) -- (87.6000,141.8000) -- (87.6000,141.8000) -- (87.6000,141.8000) -- (87.6000,141.8000) -- (87.6000,141.8000) -- (87.6000,141.8000) -- (87.6000,141.8000) -- (87.6000,141.8000) -- (87.6000,141.8000) -- (87.6000,141.8000) -- (87.6000,141.8000) -- (87.6000,141.8000) -- (87.6000,141.8000) -- (87.6000,141.8000) -- (87.6000,141.8000) -- (87.6000,141.8000) -- (87.6000,141.8000) -- (87.6000,141.8000) -- (87.6000,141.8000) -- (87.6000,141.8000) -- (87.6000,141.7000) -- (87.6000,141.7000) -- (87.6000,141.7000) -- (87.6000,141.7000) -- (87.6000,141.7000) -- (87.6000,141.7000) -- (87.6000,141.7000) -- (87.6000,141.7000) -- (87.6000,141.7000) -- (87.6000,141.7000) -- (87.6000,141.7000) -- (87.6000,141.7000) -- (87.6000,141.7000) -- (87.6000,141.7000) -- (87.7000,141.7000) -- (87.7000,141.7000) -- (87.7000,141.7000) -- (87.7000,141.7000) -- (87.7000,141.7000) -- (87.7000,141.7000) -- (87.7000,141.7000) -- (87.7000,141.7000) -- (87.7000,141.6000) -- (87.7000,141.6000) -- (87.7000,141.6000) -- (87.7000,141.6000) -- (87.7000,141.6000) -- (87.7000,141.6000) -- (87.7000,141.6000) -- (87.7000,141.6000) -- (87.7000,141.6000) -- (87.7000,141.6000) -- (87.7000,141.6000) -- (87.7000,141.6000) -- (87.7000,141.6000) -- (87.7000,141.6000) -- (87.7000,141.6000) -- (87.7000,141.6000) -- (87.7000,141.6000) -- (87.7000,141.6000) -- (87.7000,141.6000) -- (87.7000,141.6000) -- (87.7000,141.6000) -- (87.7000,141.5000) -- (87.7000,141.5000) -- (87.7000,141.5000) -- (87.7000,141.5000) -- (87.7000,141.5000) -- (87.7000,141.5000) -- (87.7000,141.5000) -- (87.7000,141.5000) -- (87.7000,141.5000) -- (87.7000,141.5000) -- (87.7000,141.5000) -- (87.7000,141.5000) -- (87.7000,141.5000) -- (87.7000,141.5000) -- (87.7000,141.5000) -- (87.7000,141.5000) -- (87.7000,141.5000) -- (87.7000,141.5000) -- (87.7000,141.5000) -- (87.7000,141.5000) -- (87.8000,141.5000) -- (87.8000,141.5000) -- (87.8000,141.4000) -- (87.8000,141.4000) -- (87.8000,141.4000) -- (87.8000,141.4000) -- (87.8000,141.4000) -- (87.8000,141.4000) -- (87.8000,141.4000) -- (87.8000,141.4000) -- (87.8000,141.4000) -- (87.8000,141.4000) -- (87.8000,141.4000) -- (87.8000,141.4000) -- (87.8000,141.4000) -- (87.8000,141.4000) -- (87.8000,141.4000) -- (87.8000,141.4000) -- (87.8000,141.4000) -- (87.8000,141.4000) -- (87.8000,141.4000) -- (87.8000,141.4000) -- (87.8000,141.4000) -- (87.8000,141.3000) -- (87.8000,141.3000) -- (87.8000,141.3000) -- (87.8000,141.3000) -- (87.8000,141.3000) -- (87.8000,141.3000) -- (87.8000,141.3000) -- (87.8000,141.3000) -- (87.8000,141.3000) -- (87.8000,141.3000) -- (87.8000,141.3000) -- (87.8000,141.3000) -- (87.8000,141.3000) -- (87.8000,141.3000) -- (87.8000,141.3000) -- (87.8000,141.3000) -- (87.8000,141.3000) -- (87.8000,141.3000) -- (87.8000,141.3000) -- (87.8000,141.3000) -- (87.8000,141.3000) -- (87.8000,141.3000) -- (87.8000,141.2000) -- (87.8000,141.2000) -- (87.8000,141.2000) -- (87.8000,141.2000) -- (87.8000,141.2000) -- (87.9000,141.2000) -- (87.9000,141.2000) -- (87.9000,141.2000) -- (87.9000,141.2000) -- (87.9000,141.2000) -- (87.9000,141.2000) -- (87.9000,141.2000) -- (87.9000,141.2000) -- (87.9000,141.2000) -- (87.9000,141.2000) -- (87.9000,141.2000) -- (87.9000,141.2000) -- (87.9000,141.2000) -- (87.9000,141.2000) -- (87.9000,141.2000) -- (87.9000,141.2000) -- (87.9000,141.1000) -- (87.9000,141.1000) -- (87.9000,141.1000) -- (87.9000,141.1000) -- (87.9000,141.1000) -- (87.9000,141.1000) -- (87.9000,141.1000) -- (87.9000,141.1000) -- (87.9000,141.1000) -- (87.9000,141.1000) -- (87.9000,141.1000) -- (87.9000,141.1000) -- (87.9000,141.1000) -- (87.9000,141.1000) -- (87.9000,141.1000) -- (87.9000,141.1000) -- (87.9000,141.1000) -- (87.9000,141.1000) -- (87.9000,141.1000) -- (87.9000,141.1000) -- (87.9000,141.1000) -- (87.9000,141.1000) -- (87.9000,141.0000) -- (87.9000,141.0000) -- (87.9000,141.0000) -- (87.9000,141.0000) -- (87.9000,141.0000) -- (87.9000,141.0000) -- (87.9000,141.0000) -- (87.9000,141.0000) -- (87.9000,141.0000) -- (87.9000,141.0000) -- (87.9000,141.0000) -- (88.0000,141.0000) -- (88.0000,141.0000) -- (88.0000,141.0000) -- (88.0000,141.0000) -- (88.0000,141.0000) -- (88.0000,141.0000) -- (88.0000,141.0000) -- (88.0000,141.0000) -- (88.0000,141.0000) -- (88.0000,141.0000) -- (88.0000,140.9000) -- (88.0000,140.9000) -- (88.0000,140.9000) -- (88.0000,140.9000) -- (88.0000,140.9000) -- (88.0000,140.9000) -- (88.0000,140.9000) -- (88.0000,140.9000) -- (88.0000,140.9000) -- (88.0000,140.9000) -- (88.0000,140.9000) -- (88.0000,140.9000) -- (88.0000,140.9000) -- (88.0000,140.9000) -- (88.0000,140.9000) -- (88.0000,140.9000) -- (88.0000,140.9000) -- (88.0000,140.9000) -- (88.0000,140.9000) -- (88.0000,140.9000) -- (88.0000,140.9000) -- (88.0000,140.9000) -- (88.0000,140.8000) -- (88.0000,140.8000) -- (88.0000,140.8000) -- (88.0000,140.8000) -- (88.0000,140.8000) -- (88.0000,140.8000) -- (88.0000,140.8000) -- (88.0000,140.8000) -- (88.0000,140.8000) -- (88.0000,140.8000) -- (88.0000,140.8000) -- (88.0000,140.8000) -- (88.0000,140.8000) -- (88.0000,140.8000) -- (88.0000,140.8000) -- (88.0000,140.8000) -- (88.0000,140.8000) -- (88.0000,140.8000) -- (88.1000,140.8000) -- (88.1000,140.8000) -- (88.1000,140.8000) -- (88.1000,140.7000) -- (88.1000,140.7000) -- (88.1000,140.7000) -- (88.1000,140.7000) -- (88.1000,140.7000) -- (88.1000,140.7000) -- (88.1000,140.7000) -- (88.1000,140.7000) -- (88.1000,140.7000) -- (88.1000,140.7000) -- (88.1000,140.7000) -- (88.1000,140.7000) -- (88.1000,140.7000) -- (88.1000,140.7000) -- (88.1000,140.7000) -- (88.1000,140.7000) -- (88.1000,140.7000) -- (88.1000,140.7000) -- (88.1000,140.7000) -- (88.1000,140.7000) -- (88.1000,140.7000) -- (88.1000,140.7000) -- (88.1000,140.6000) -- (88.1000,140.6000) -- (88.1000,140.6000) -- (88.1000,140.6000) -- (88.1000,140.6000) -- (88.1000,140.6000) -- (88.1000,140.6000) -- (88.1000,140.6000) -- (88.1000,140.6000) -- (88.1000,140.6000) -- (88.1000,140.6000) -- (88.1000,140.6000) -- (88.1000,140.6000) -- (88.1000,140.6000) -- (88.1000,140.6000) -- (88.1000,140.6000) -- (88.1000,140.6000) -- (88.1000,140.6000) -- (88.1000,140.6000) -- (88.1000,140.6000) -- (88.1000,140.6000) -- (88.1000,140.5000) -- (88.1000,140.5000) -- (88.1000,140.5000) -- (88.2000,140.5000) -- (88.2000,140.5000) -- (88.2000,140.5000) -- (88.2000,140.5000) -- (88.2000,140.5000) -- (88.2000,140.5000) -- (88.2000,140.5000) -- (88.2000,140.5000) -- (88.2000,140.5000) -- (88.2000,140.5000) -- (88.2000,140.5000) -- (88.2000,140.5000) -- (88.2000,140.5000) -- (88.2000,140.5000) -- (88.2000,140.5000) -- (88.2000,140.5000) -- (88.2000,140.5000) -- (88.2000,140.5000) -- (88.2000,140.5000) -- (88.2000,140.4000) -- (88.2000,140.4000) -- (88.2000,140.4000) -- (88.2000,140.4000) -- (88.2000,140.4000) -- (88.2000,140.4000) -- (88.2000,140.4000) -- (88.2000,140.4000) -- (88.2000,140.4000) -- (88.2000,140.4000) -- (88.2000,140.4000) -- (88.2000,140.4000) -- (88.2000,140.4000) -- (88.2000,140.4000) -- (88.2000,140.4000) -- (88.2000,140.4000) -- (88.2000,140.4000) -- (88.2000,140.4000) -- (88.2000,140.4000) -- (88.2000,140.4000) -- (88.2000,140.4000) -- (88.2000,140.3000) -- (88.2000,140.3000) -- (88.2000,140.3000) -- (88.2000,140.3000) -- (88.2000,140.3000) -- (88.2000,140.3000) -- (88.2000,140.3000) -- (88.2000,140.3000) -- (88.2000,140.3000) -- (88.2000,140.3000) -- (88.3000,140.3000) -- (88.3000,140.3000) -- (88.3000,140.3000) -- (88.3000,140.3000) -- (88.3000,140.3000) -- (88.3000,140.3000) -- (88.3000,140.3000) -- (88.3000,140.3000) -- (88.3000,140.3000) -- (88.3000,140.3000) -- (88.3000,140.3000) -- (88.3000,140.3000) -- (88.3000,140.2000) -- (88.3000,140.2000) -- (88.3000,140.2000) -- (88.3000,140.2000) -- (88.3000,140.2000) -- (88.3000,140.2000) -- (88.3000,140.2000) -- (88.3000,140.2000) -- (88.3000,140.2000) -- (88.3000,140.2000) -- (88.3000,140.2000) -- (88.3000,140.2000) -- (88.3000,140.2000) -- (88.3000,140.2000) -- (88.3000,140.2000) -- (88.3000,140.2000) -- (88.3000,140.2000) -- (88.3000,140.2000) -- (88.3000,140.2000) -- (88.3000,140.2000) -- (88.3000,140.2000) -- (88.3000,140.1000) -- (88.3000,140.1000) -- (88.3000,140.1000) -- (88.3000,140.1000) -- (88.3000,140.1000) -- (88.3000,140.1000) -- (88.3000,140.1000) -- (88.3000,140.1000) -- (88.3000,140.1000) -- (88.3000,140.1000) -- (88.3000,140.1000) -- (88.3000,140.1000) -- (88.3000,140.1000) -- (88.3000,140.1000) -- (88.3000,140.1000) -- (88.3000,140.1000) -- (88.4000,140.1000) -- (88.4000,140.1000) -- (88.4000,140.1000) -- (88.4000,140.1000) -- (88.4000,140.1000) -- (88.4000,140.1000) -- (88.4000,140.0000) -- (88.4000,140.0000) -- (88.4000,140.0000) -- (88.4000,140.0000) -- (88.4000,140.0000) -- (88.4000,140.0000) -- (88.4000,140.0000) -- (88.4000,140.0000) -- (88.4000,140.0000) -- (88.4000,140.0000) -- (88.4000,140.0000) -- (88.4000,140.0000) -- (88.4000,140.0000) -- (88.4000,140.0000) -- (88.4000,140.0000) -- (88.4000,140.0000) -- (88.4000,140.0000) -- (88.4000,140.0000) -- (88.4000,140.0000) -- (88.4000,140.0000) -- (88.4000,140.0000) -- (88.4000,139.9000) -- (88.4000,139.9000) -- (88.4000,139.9000) -- (88.4000,139.9000) -- (88.4000,139.9000) -- (88.4000,139.9000) -- (88.4000,139.9000) -- (88.4000,139.9000) -- (88.4000,139.9000) -- (88.4000,139.9000) -- (88.4000,139.9000) -- (88.4000,139.9000) -- (88.4000,139.9000) -- (88.4000,139.9000) -- (88.4000,139.9000) -- (88.4000,139.9000) -- (88.4000,139.9000) -- (88.4000,139.9000) -- (88.4000,139.9000) -- (88.4000,139.9000) -- (88.4000,139.9000) -- (88.4000,139.9000) -- (88.4000,139.8000) -- (88.5000,139.8000) -- (88.5000,139.8000) -- (88.5000,139.8000) -- (88.5000,139.8000) -- (88.5000,139.8000) -- (88.5000,139.8000) -- (88.5000,139.8000) -- (88.5000,139.8000) -- (88.5000,139.8000) -- (88.5000,139.8000) -- (88.5000,139.8000) -- (88.5000,139.8000) -- (88.5000,139.8000) -- (88.5000,139.8000) -- (88.5000,139.8000) -- (88.5000,139.8000) -- (88.5000,139.8000) -- (96.0000,139.8000) -- (96.0000,139.8000) -- (96.0000,139.8000) -- (96.0000,139.7000) -- (96.0000,139.7000) -- (96.0000,139.7000) -- (96.0000,139.7000) -- (96.0000,139.7000) -- (96.0000,139.7000) -- (96.0000,139.7000) -- (96.0000,139.7000) -- (96.0000,139.7000) -- (96.0000,139.7000) -- (96.0000,139.7000) -- (96.0000,139.7000) -- (96.0000,139.7000) -- (96.0000,139.7000) -- (96.0000,139.7000) -- (96.0000,139.7000) -- (96.0000,139.7000) -- (96.0000,139.7000) -- (96.0000,139.7000) -- (96.0000,139.7000) -- (96.0000,139.7000) -- (96.0000,139.7000) -- (96.0000,139.6000) -- (96.0000,139.6000) -- (96.0000,139.6000) -- (96.0000,139.6000) -- (96.0000,139.6000) -- (96.0000,139.6000) -- (96.0000,139.6000) -- (96.0000,139.6000) -- (96.0000,139.6000) -- (96.0000,139.6000) -- (96.0000,139.6000) -- (96.0000,139.6000) -- (96.0000,139.6000) -- (96.0000,139.6000) -- (96.1000,139.6000) -- (96.1000,139.6000) -- (96.1000,139.6000) -- (96.1000,139.6000) -- (96.1000,139.6000) -- (96.1000,139.6000) -- (96.1000,139.6000) -- (96.1000,139.5000) -- (96.1000,139.5000) -- (96.1000,139.5000) -- (96.1000,139.5000) -- (96.1000,139.5000) -- (96.1000,139.5000) -- (96.1000,139.5000) -- (96.1000,139.5000) -- (96.1000,139.5000) -- (96.1000,139.5000) -- (96.1000,139.5000) -- (96.1000,139.5000) -- (96.1000,139.5000) -- (96.1000,139.5000) -- (96.1000,139.5000) -- (96.1000,139.5000) -- (96.1000,139.5000) -- (96.1000,139.5000) -- (96.1000,139.5000) -- (96.1000,139.5000) -- (96.1000,139.5000) -- (96.1000,139.5000) -- (96.1000,139.4000) -- (96.1000,139.4000) -- (96.1000,139.4000) -- (96.1000,139.4000) -- (96.1000,139.4000) -- (96.1000,139.4000) -- (96.1000,139.4000) -- (96.1000,139.4000) -- (96.1000,139.4000) -- (96.1000,139.4000) -- (96.1000,139.4000) -- (96.1000,139.4000) -- (96.1000,139.4000) -- (96.1000,139.4000) -- (96.1000,139.4000) -- (96.1000,139.4000) -- (96.1000,139.4000) -- (96.1000,139.4000) -- (96.1000,139.4000) -- (96.1000,139.4000) -- (96.2000,139.4000) -- (96.2000,139.3000) -- (96.2000,139.3000) -- (96.2000,139.3000) -- (96.2000,139.3000) -- (96.2000,139.3000) -- (96.2000,139.3000) -- (96.2000,139.3000) -- (96.2000,139.3000) -- (96.2000,139.3000) -- (96.2000,139.3000) -- (96.2000,139.3000) -- (96.2000,139.3000) -- (96.2000,139.3000) -- (96.2000,139.3000) -- (96.2000,139.3000) -- (96.2000,139.3000) -- (96.2000,139.3000) -- (96.2000,139.3000) -- (96.2000,139.3000) -- (96.2000,139.3000) -- (96.2000,139.3000) -- (96.2000,139.3000) -- (96.2000,139.2000) -- (96.2000,139.2000) -- (96.2000,139.2000) -- (96.2000,139.2000) -- (96.2000,139.2000) -- (96.2000,139.2000) -- (96.2000,139.2000) -- (96.2000,139.2000) -- (96.2000,139.2000) -- (96.2000,139.2000) -- (96.2000,139.2000) -- (96.2000,139.2000) -- (96.2000,139.2000) -- (96.2000,139.2000) -- (96.2000,139.2000) -- (96.2000,139.2000) -- (96.2000,139.2000) -- (96.2000,139.2000) -- (96.2000,139.2000) -- (96.2000,139.2000) -- (96.2000,139.2000) -- (96.2000,139.1000) -- (96.2000,139.1000) -- (96.2000,139.1000) -- (96.2000,139.1000) -- (96.2000,139.1000) -- (96.2000,139.1000) -- (96.3000,139.1000) -- (96.3000,139.1000) -- (96.3000,139.1000) -- (96.3000,139.1000) -- (96.3000,139.1000) -- (96.3000,139.1000) -- (96.3000,139.1000) -- (96.3000,139.1000) -- (96.3000,139.1000) -- (96.3000,139.1000) -- (96.3000,139.1000) -- (96.3000,139.1000) -- (96.3000,139.1000) -- (96.3000,139.1000) -- (96.3000,139.1000) -- (96.3000,139.1000) -- (96.3000,139.0000) -- (96.3000,139.0000) -- (96.3000,139.0000) -- (96.3000,139.0000) -- (96.3000,139.0000) -- (96.3000,139.0000) -- (96.3000,139.0000) -- (96.3000,139.0000) -- (96.3000,139.0000) -- (96.3000,139.0000) -- (96.3000,139.0000) -- (96.3000,139.0000) -- (96.3000,139.0000) -- (96.3000,139.0000) -- (96.3000,139.0000) -- (96.3000,139.0000) -- (96.3000,139.0000) -- (96.3000,139.0000) -- (96.3000,139.0000) -- (96.3000,139.0000) -- (96.3000,139.0000) -- (96.3000,138.9000) -- (96.3000,138.9000) -- (96.3000,138.9000) -- (96.3000,138.9000) -- (96.3000,138.9000) -- (96.3000,138.9000) -- (96.3000,138.9000) -- (96.3000,138.9000) -- (96.3000,138.9000) -- (96.3000,138.9000) -- (96.3000,138.9000) -- (96.3000,138.9000) -- (96.3000,138.9000) -- (96.4000,138.9000) -- (96.4000,138.9000) -- (96.4000,138.9000) -- (96.4000,138.9000) -- (96.4000,138.9000) -- (96.4000,138.9000) -- (96.4000,138.9000) -- (96.4000,138.9000) -- (96.4000,138.9000) -- (96.4000,138.8000) -- (96.4000,138.8000) -- (96.4000,138.8000) -- (96.4000,138.8000) -- (96.4000,138.8000) -- (96.4000,138.8000) -- (96.4000,138.8000) -- (96.4000,138.8000) -- (96.4000,138.8000) -- (96.4000,138.8000) -- (96.4000,138.8000) -- (96.4000,138.8000) -- (96.4000,138.8000) -- (96.4000,138.8000) -- (96.4000,138.8000) -- (96.4000,138.8000) -- (96.4000,138.8000) -- (96.4000,138.8000) -- (96.4000,138.8000) -- (96.4000,138.8000) -- (96.4000,138.8000) -- (96.4000,138.7000) -- (96.4000,138.7000) -- (96.4000,138.7000) -- (96.4000,138.7000) -- (96.4000,138.7000) -- (96.4000,138.7000) -- (96.4000,138.7000) -- (96.4000,138.7000) -- (96.4000,138.7000) -- (96.4000,138.7000) -- (96.4000,138.7000) -- (96.4000,138.7000) -- (96.4000,138.7000) -- (96.4000,138.7000) -- (96.4000,138.7000) -- (96.4000,138.7000) -- (96.4000,138.7000) -- (96.4000,138.7000) -- (96.4000,138.7000) -- (96.5000,138.7000) -- (96.5000,138.7000) -- (96.5000,138.7000) -- (96.5000,138.6000) -- (96.5000,138.6000) -- (96.5000,138.6000) -- (96.5000,138.6000) -- (96.5000,138.6000) -- (96.5000,138.6000) -- (96.5000,138.6000) -- (96.5000,138.6000) -- (96.5000,138.6000) -- (96.5000,138.6000) -- (96.5000,138.6000) -- (96.5000,138.6000) -- (96.5000,138.6000) -- (96.5000,138.6000) -- (96.5000,138.6000) -- (96.5000,138.6000) -- (96.5000,138.6000) -- (96.5000,138.6000) -- (96.5000,138.6000) -- (96.5000,138.6000) -- (96.5000,138.6000) -- (96.5000,138.5000) -- (96.5000,138.5000) -- (96.5000,138.5000) -- (96.5000,138.5000) -- (96.5000,138.5000) -- (96.5000,138.5000) -- (96.5000,138.5000) -- (96.5000,138.5000) -- (96.5000,138.5000) -- (96.5000,138.5000) -- (96.5000,138.5000) -- (96.5000,138.5000) -- (96.5000,138.5000) -- (96.5000,138.5000) -- (96.5000,138.5000) -- (96.5000,138.5000) -- (96.5000,138.5000) -- (96.5000,138.5000) -- (96.5000,138.5000) -- (96.5000,138.5000) -- (96.5000,138.5000) -- (96.5000,138.5000) -- (96.5000,138.4000) -- (96.5000,138.4000) -- (96.5000,138.4000) -- (96.5000,138.4000) -- (96.6000,138.4000) -- (96.6000,138.4000) -- (96.6000,138.4000) -- (96.6000,138.4000) -- (96.6000,138.4000) -- (96.6000,138.4000) -- (96.6000,138.4000) -- (96.6000,138.4000) -- (96.6000,138.4000) -- (96.6000,138.4000) -- (96.6000,138.4000) -- (96.6000,138.4000) -- (96.6000,138.4000) -- (96.6000,138.4000) -- (96.6000,138.4000) -- (96.6000,138.4000) -- (96.6000,138.4000) -- (96.6000,138.3000) -- (96.6000,138.3000) -- (96.6000,138.3000) -- (96.6000,138.3000) -- (96.6000,138.3000) -- (96.6000,138.3000) -- (96.6000,138.3000) -- (96.6000,138.3000) -- (96.6000,138.3000) -- (96.6000,138.3000) -- (96.6000,138.3000) -- (96.6000,138.3000) -- (96.6000,138.3000) -- (96.6000,138.3000) -- (96.6000,138.3000) -- (96.6000,138.3000) -- (96.6000,138.3000) -- (96.6000,138.3000) -- (96.6000,138.3000) -- (96.6000,138.3000) -- (96.6000,138.3000) -- (96.6000,138.3000) -- (96.6000,138.2000) -- (96.6000,138.2000) -- (96.6000,138.2000) -- (96.6000,138.2000) -- (96.6000,138.2000) -- (96.6000,138.2000) -- (96.6000,138.2000) -- (96.6000,138.2000) -- (96.6000,138.2000) -- (96.6000,138.2000) -- (96.7000,138.2000) -- (96.7000,138.2000) -- (96.7000,138.2000) -- (96.7000,138.2000) -- (96.7000,138.2000) -- (96.7000,138.2000) -- (96.7000,138.2000) -- (96.7000,138.2000) -- (96.7000,138.2000) -- (96.7000,138.2000) -- (96.7000,138.2000) -- (96.7000,138.1000) -- (96.7000,138.1000) -- (96.7000,138.1000) -- (96.7000,138.1000) -- (96.7000,138.1000) -- (96.7000,138.1000) -- (96.7000,138.1000) -- (96.7000,138.1000) -- (96.7000,138.1000) -- (96.7000,138.1000) -- (96.7000,138.1000) -- (96.7000,138.1000) -- (96.7000,138.1000) -- (96.7000,138.1000) -- (96.7000,138.1000) -- (96.7000,138.1000) -- (96.7000,138.1000) -- (96.7000,138.1000) -- (96.7000,138.1000) -- (96.7000,138.1000) -- (96.7000,138.1000) -- (96.7000,138.1000) -- (96.7000,138.0000) -- (96.7000,138.0000) -- (96.7000,138.0000) -- (96.7000,138.0000) -- (96.7000,138.0000) -- (96.7000,138.0000) -- (96.7000,138.0000) -- (96.7000,138.0000) -- (96.7000,138.0000) -- (96.7000,138.0000) -- (96.7000,138.0000) -- (96.7000,138.0000) -- (96.7000,138.0000) -- (96.7000,138.0000) -- (96.7000,138.0000) -- (96.7000,138.0000) -- (96.7000,138.0000) -- (96.8000,138.0000) -- (96.8000,138.0000) -- (96.8000,138.0000) -- (96.8000,138.0000) -- (96.8000,137.9000) -- (96.8000,137.9000) -- (96.8000,137.9000) -- (96.8000,137.9000) -- (96.8000,137.9000) -- (96.8000,137.9000) -- (96.8000,137.9000) -- (96.8000,137.9000) -- (96.8000,137.9000) -- (96.8000,137.9000) -- (96.8000,137.9000) -- (96.8000,137.9000) -- (96.8000,137.9000) -- (96.8000,137.9000) -- (96.8000,137.9000) -- (96.8000,137.9000) -- (96.8000,137.9000) -- (96.8000,137.9000) -- (96.8000,137.9000) -- (96.8000,137.9000) -- (96.8000,137.9000) -- (96.8000,137.9000) -- (96.8000,137.8000) -- (96.8000,137.8000) -- (96.8000,137.8000) -- (96.8000,137.8000) -- (96.8000,137.8000) -- (96.8000,137.8000) -- (96.8000,137.8000) -- (96.8000,137.8000) -- (96.8000,137.8000) -- (96.8000,137.8000) -- (96.8000,137.8000) -- (96.8000,137.8000) -- (96.8000,137.8000) -- (96.8000,137.8000) -- (96.8000,137.8000) -- (96.8000,137.8000) -- (96.8000,137.8000) -- (96.8000,137.8000) -- (96.8000,137.8000) -- (96.8000,137.8000) -- (96.8000,137.8000) -- (96.8000,137.7000) -- (96.8000,137.7000) -- (96.9000,137.7000) -- (96.9000,137.7000) -- (96.9000,137.7000) -- (96.9000,137.7000) -- (96.9000,137.7000) -- (96.9000,137.7000) -- (96.9000,137.7000) -- (96.9000,137.7000) -- (96.9000,137.7000) -- (96.9000,137.7000) -- (96.9000,137.7000) -- (96.9000,137.7000) -- (96.9000,137.7000) -- (96.9000,137.7000) -- (96.9000,137.7000) -- (96.9000,137.7000) -- (96.9000,137.7000) -- (96.9000,137.7000) -- (96.9000,137.7000) -- (96.9000,137.7000) -- (96.9000,137.6000) -- (96.9000,137.6000) -- (96.9000,137.6000) -- (96.9000,137.6000) -- (96.9000,137.6000) -- (96.9000,137.6000) -- (96.9000,137.6000) -- (96.9000,137.6000) -- (96.9000,137.6000) -- (96.9000,137.6000) -- (96.9000,137.6000) -- (96.9000,137.6000) -- (96.9000,137.6000) -- (96.9000,137.6000) -- (96.9000,137.6000) -- (96.9000,137.6000) -- (96.9000,137.6000) -- (96.9000,137.6000) -- (96.9000,137.6000) -- (96.9000,137.6000) -- (96.9000,137.6000) -- (96.9000,137.5000) -- (96.9000,137.5000) -- (96.9000,137.5000) -- (96.9000,137.5000) -- (96.9000,137.5000) -- (96.9000,137.5000) -- (96.9000,137.5000) -- (96.9000,137.5000) -- (96.9000,137.5000) -- (97.0000,137.5000) -- (97.0000,137.5000) -- (97.0000,137.5000) -- (97.0000,137.5000) -- (97.0000,137.5000) -- (97.0000,137.5000) -- (97.0000,137.5000) -- (97.0000,137.5000) -- (97.0000,137.5000) -- (97.0000,137.5000) -- (97.0000,137.5000) -- (97.0000,137.5000) -- (97.0000,137.5000) -- (97.0000,137.4000) -- (97.0000,137.4000) -- (97.0000,137.4000) -- (97.0000,137.4000) -- (97.0000,137.4000) -- (97.0000,137.4000) -- (97.0000,137.4000) -- (97.0000,137.4000) -- (97.0000,137.4000) -- (97.0000,137.4000) -- (97.0000,137.4000) -- (97.0000,137.4000) -- (97.0000,137.4000) -- (97.0000,137.4000) -- (97.0000,137.4000) -- (97.0000,137.4000) -- (97.0000,137.4000) -- (97.0000,137.4000) -- (97.0000,137.4000) -- (97.0000,137.4000) -- (97.0000,137.4000) -- (97.0000,137.3000) -- (97.0000,137.3000) -- (97.0000,137.3000) -- (97.0000,137.3000) -- (97.0000,137.3000) -- (97.0000,137.3000) -- (97.0000,137.3000) -- (97.0000,137.3000) -- (97.0000,137.3000) -- (97.0000,137.3000) -- (97.0000,137.3000) -- (97.0000,137.3000) -- (97.0000,137.3000) -- (97.0000,137.3000) -- (97.0000,137.3000) -- (97.1000,137.3000) -- (97.1000,137.3000) -- (97.1000,137.3000) -- (97.1000,137.3000) -- (97.1000,137.3000) -- (97.1000,137.3000) -- (97.1000,137.3000) -- (97.1000,137.2000) -- (97.1000,137.2000) -- (97.1000,137.2000) -- (97.1000,137.2000) -- (97.1000,137.2000) -- (97.1000,137.2000) -- (97.1000,137.2000) -- (97.1000,137.2000) -- (97.1000,137.2000) -- (97.1000,137.2000) -- (97.1000,137.2000) -- (97.1000,137.2000) -- (97.1000,137.2000) -- (97.1000,137.2000) -- (97.1000,137.2000) -- (97.1000,137.2000) -- (97.1000,137.2000) -- (97.1000,137.2000) -- (97.1000,137.2000) -- (97.1000,137.2000) -- (97.1000,137.2000) -- (97.1000,137.1000) -- (97.1000,137.1000) -- (97.1000,137.1000) -- (97.1000,137.1000) -- (97.1000,137.1000) -- (97.1000,137.1000) -- (97.1000,137.1000) -- (97.1000,137.1000) -- (97.1000,137.1000) -- (97.1000,137.1000) -- (97.1000,137.1000) -- (97.1000,137.1000) -- (97.1000,137.1000) -- (97.1000,137.1000) -- (97.1000,137.1000) -- (97.1000,137.1000) -- (97.1000,137.1000) -- (97.1000,137.1000) -- (97.1000,137.1000) -- (97.1000,137.1000) -- (97.1000,137.1000) -- (97.1000,137.1000) -- (97.2000,137.0000) -- (97.2000,137.0000) -- (97.2000,137.0000) -- (97.2000,137.0000) -- (97.2000,137.0000) -- (97.2000,137.0000) -- (97.2000,137.0000) -- (97.2000,137.0000) -- (97.2000,137.0000) -- (97.2000,137.0000) -- (97.2000,137.0000) -- (97.2000,137.0000) -- (97.2000,137.0000) -- (97.2000,137.0000) -- (97.2000,137.0000) -- (97.2000,137.0000) -- (97.2000,137.0000) -- (97.2000,137.0000) -- (97.2000,137.0000) -- (97.2000,137.0000) -- (97.2000,137.0000) -- (97.2000,136.9000) -- (97.2000,136.9000) -- (97.2000,136.9000) -- (97.2000,136.9000) -- (97.2000,136.9000) -- (97.2000,136.9000) -- (97.2000,136.9000) -- (97.2000,136.9000) -- (97.2000,136.9000) -- (97.2000,136.9000) -- (97.2000,136.9000) -- (97.2000,136.9000) -- (97.2000,136.9000) -- (97.2000,136.9000) -- (97.2000,136.9000) -- (97.2000,136.9000) -- (97.2000,136.9000) -- (97.2000,136.9000) -- (97.2000,136.9000) -- (97.2000,136.9000) -- (97.2000,136.9000) -- (97.2000,136.9000) -- (97.2000,136.8000) -- (97.2000,136.8000) -- (97.2000,136.8000) -- (97.2000,136.8000) -- (97.2000,136.8000) -- (97.2000,136.8000) -- (97.3000,136.8000) -- (97.3000,136.8000) -- (97.3000,136.8000) -- (97.3000,136.8000) -- (97.3000,136.8000) -- (97.3000,136.8000) -- (97.3000,136.8000) -- (97.3000,136.8000) -- (97.3000,136.8000) -- (97.3000,136.8000) -- (97.3000,136.8000) -- (97.3000,136.8000) -- (97.3000,136.8000) -- (97.3000,136.8000) -- (97.3000,136.8000) -- (97.3000,136.7000) -- (97.3000,136.7000) -- (97.3000,136.7000) -- (97.3000,136.7000) -- (97.3000,136.7000) -- (97.3000,136.7000) -- (97.3000,136.7000) -- (97.3000,136.7000) -- (97.3000,136.7000) -- (97.3000,136.7000) -- (97.3000,136.7000) -- (97.3000,136.7000) -- (97.3000,136.7000) -- (97.3000,136.7000) -- (97.3000,136.7000) -- (97.3000,136.7000) -- (97.3000,136.7000) -- (97.3000,136.7000) -- (97.3000,136.7000) -- (97.3000,136.7000) -- (97.3000,136.7000) -- (97.3000,136.7000) -- (97.3000,136.6000) -- (97.3000,136.6000) -- (97.3000,136.6000) -- (97.3000,136.6000) -- (97.3000,136.6000) -- (97.3000,136.6000) -- (97.3000,136.6000) -- (97.3000,136.6000) -- (97.3000,136.6000) -- (97.3000,136.6000) -- (97.3000,136.6000) -- (97.3000,136.6000) -- (97.3000,136.6000) -- (97.4000,136.6000) -- (97.4000,136.6000) -- (97.4000,136.6000) -- (97.4000,136.6000) -- (97.4000,136.6000) -- (97.4000,136.6000) -- (97.4000,136.6000) -- (97.4000,136.6000) -- (97.4000,136.5000) -- (97.4000,136.5000) -- (97.4000,136.5000) -- (97.4000,136.5000) -- (97.4000,136.5000) -- (97.4000,136.5000) -- (97.4000,136.5000) -- (97.4000,136.5000) -- (97.4000,136.5000) -- (97.4000,136.5000) -- (97.4000,136.5000) -- (97.4000,136.5000) -- (97.4000,136.5000) -- (97.4000,136.5000) -- (97.4000,136.5000) -- (97.4000,136.5000) -- (97.4000,136.5000) -- (97.4000,136.5000) -- (97.4000,136.5000) -- (97.4000,136.5000) -- (97.4000,136.5000) -- (97.4000,136.5000) -- (97.4000,136.4000) -- (97.4000,136.4000) -- (97.4000,136.4000) -- (97.4000,136.4000) -- (97.4000,136.4000) -- (97.4000,136.4000) -- (97.4000,136.4000) -- (97.4000,136.4000) -- (97.4000,136.4000) -- (97.4000,136.4000) -- (97.4000,136.4000) -- (97.4000,136.4000) -- (97.4000,136.4000) -- (97.4000,136.4000) -- (97.4000,136.4000) -- (97.4000,136.4000) -- (97.4000,136.4000) -- (97.4000,136.4000) -- (97.4000,136.4000) -- (97.5000,136.4000) -- (97.5000,136.4000) -- (97.5000,136.3000) -- (97.5000,136.3000) -- (97.5000,136.3000) -- (97.5000,136.3000) -- (97.5000,136.3000) -- (97.5000,136.3000) -- (97.5000,136.3000) -- (97.5000,136.3000) -- (97.5000,136.3000) -- (97.5000,136.3000) -- (97.5000,136.3000) -- (97.5000,136.3000) -- (97.5000,136.3000) -- (97.5000,136.3000) -- (97.5000,136.3000) -- (97.5000,136.3000) -- (97.5000,136.3000) -- (97.5000,136.3000) -- (97.5000,136.3000) -- (97.5000,136.3000) -- (97.5000,136.3000) -- (97.5000,136.3000) -- (97.5000,136.2000) -- (97.5000,136.2000) -- (97.5000,136.2000) -- (97.5000,136.2000) -- (97.5000,136.2000) -- (97.5000,136.2000) -- (97.5000,136.2000) -- (97.5000,136.2000) -- (97.5000,136.2000) -- (97.5000,136.2000) -- (97.5000,136.2000) -- (97.5000,136.2000) -- (97.5000,136.2000) -- (97.5000,136.2000) -- (97.5000,136.2000) -- (97.5000,136.2000) -- (97.5000,136.2000) -- (97.5000,136.2000) -- (97.5000,136.2000) -- (97.5000,136.2000) -- (97.5000,136.2000) -- (97.5000,136.1000) -- (97.5000,136.1000) -- (97.5000,136.1000) -- (97.5000,136.1000) -- (97.5000,136.1000) -- (97.6000,136.1000) -- (97.6000,136.1000) -- (97.6000,136.1000) -- (97.6000,136.1000) -- (97.6000,136.1000) -- (97.6000,136.1000) -- (97.6000,136.1000) -- (97.6000,136.1000) -- (97.6000,136.1000) -- (97.6000,136.1000) -- (97.6000,136.1000) -- (97.6000,136.1000) -- (97.6000,136.1000) -- (97.6000,136.1000) -- (97.6000,136.1000) -- (97.6000,136.1000) -- (97.6000,136.1000) -- (97.6000,136.0000) -- (97.6000,136.0000) -- (97.6000,136.0000) -- (97.6000,136.0000) -- (97.6000,136.0000) -- (97.6000,136.0000) -- (97.6000,136.0000) -- (97.6000,136.0000) -- (97.6000,136.0000) -- (97.6000,136.0000) -- (97.6000,136.0000) -- (97.6000,136.0000) -- (97.6000,136.0000) -- (97.6000,136.0000) -- (97.6000,136.0000) -- (97.6000,136.0000) -- (97.6000,136.0000) -- (97.6000,136.0000) -- (97.6000,136.0000) -- (97.6000,136.0000) -- (97.6000,136.0000) -- (97.6000,135.9000) -- (97.6000,135.9000) -- (97.6000,135.9000) -- (97.6000,135.9000) -- (97.6000,135.9000) -- (97.6000,135.9000) -- (97.6000,135.9000) -- (97.6000,135.9000) -- (97.6000,135.9000) -- (97.6000,135.9000) -- (97.6000,135.9000) -- (97.7000,135.9000) -- (97.7000,135.9000) -- (97.7000,135.9000) -- (97.7000,135.9000) -- (97.7000,135.9000) -- (97.7000,135.9000) -- (97.7000,135.9000) -- (97.7000,135.9000) -- (97.7000,135.9000) -- (97.7000,135.9000) -- (97.7000,135.9000) -- (97.7000,135.8000) -- (97.7000,135.8000) -- (97.7000,135.8000) -- (97.7000,135.8000) -- (97.7000,135.8000) -- (97.7000,135.8000) -- (97.7000,135.8000) -- (97.7000,135.8000) -- (97.7000,135.8000) -- (97.7000,135.8000) -- (97.7000,135.8000) -- (97.7000,135.8000) -- (97.7000,135.8000) -- (97.7000,135.8000) -- (97.7000,135.8000) -- (97.7000,135.8000) -- (97.7000,135.8000) -- (97.7000,135.8000) -- (97.7000,135.8000) -- (97.7000,135.8000) -- (97.7000,135.8000) -- (97.7000,135.7000) -- (97.7000,135.7000) -- (97.7000,135.7000) -- (97.7000,135.7000) -- (97.7000,135.7000) -- (97.7000,135.7000) -- (97.7000,135.7000) -- (97.7000,135.7000) -- (97.7000,135.7000) -- (97.7000,135.7000) -- (97.7000,135.7000) -- (97.7000,135.7000) -- (97.7000,135.7000) -- (97.7000,135.7000) -- (97.7000,135.7000) -- (97.7000,135.7000) -- (97.7000,135.7000) -- (97.7000,135.7000) -- (97.8000,135.7000) -- (97.8000,135.7000) -- (97.8000,135.7000) -- (97.8000,135.7000) -- (97.8000,135.6000) -- (97.8000,135.6000) -- (97.8000,135.6000) -- (97.8000,135.6000) -- (97.8000,135.6000) -- (97.8000,135.6000) -- (97.8000,135.6000) -- (97.8000,135.6000) -- (97.8000,135.6000) -- (97.8000,135.6000) -- (97.8000,135.6000) -- (97.8000,135.6000) -- (97.8000,135.6000) -- (97.8000,135.6000) -- (97.8000,135.6000) -- (97.8000,135.6000) -- (97.8000,135.6000) -- (97.8000,135.6000) -- (97.8000,135.6000) -- (97.8000,135.6000) -- (97.8000,135.6000) -- (97.8000,135.5000) -- (97.8000,135.5000) -- (97.8000,135.5000) -- (97.8000,135.5000) -- (97.8000,135.5000) -- (97.8000,135.5000) -- (97.8000,135.5000) -- (97.8000,135.5000) -- (97.8000,135.5000) -- (97.8000,135.5000) -- (97.8000,135.5000) -- (97.8000,135.5000) -- (97.8000,135.5000) -- (97.8000,135.5000) -- (97.8000,135.5000) -- (97.8000,135.5000) -- (97.8000,135.5000) -- (97.8000,135.5000) -- (97.8000,135.5000) -- (97.8000,135.5000) -- (97.8000,135.5000) -- (97.8000,135.5000) -- (97.8000,135.4000) -- (97.8000,135.4000) -- (97.9000,135.4000) -- (97.9000,135.4000) -- (97.9000,135.4000) -- (97.9000,135.4000) -- (97.9000,135.4000) -- (97.9000,135.4000) -- (97.9000,135.4000) -- (97.9000,135.4000) -- (97.9000,135.4000) -- (97.9000,135.4000) -- (97.9000,135.4000) -- (97.9000,135.4000) -- (97.9000,135.4000) -- (97.9000,135.4000) -- (97.9000,135.4000) -- (97.9000,135.4000) -- (97.9000,135.4000) -- (97.9000,135.4000) -- (97.9000,135.4000) -- (97.9000,135.3000) -- (97.9000,135.3000) -- (97.9000,135.3000) -- (97.9000,135.3000) -- (97.9000,135.3000) -- (97.9000,135.3000) -- (97.9000,135.3000) -- (97.9000,135.3000) -- (97.9000,135.3000) -- (97.9000,135.3000) -- (97.9000,135.3000) -- (97.9000,135.3000) -- (97.9000,135.3000) -- (107.3000,135.3000) -- (107.3000,135.3000) -- (107.3000,135.3000) -- (107.4000,135.3000) -- (121.2657,135.3000);



  \end{scope}
  \begin{scope}[scale=1.008,draw=black,line join=bevel,line cap=rect,line width=0.800pt]
  \end{scope}
  \begin{scope}[scale=1.008,draw=black,line join=bevel,line cap=rect,line width=0.800pt]
  \end{scope}
  \begin{scope}[cm={{1.00769,0.0,0.0,1.00769,(-21.0,-3.0)}},draw=black,line join=round,line cap=round,line width=0.480pt]
    \path[draw] (81.5000,129.5000) -- (81.5000,157.5000) -- (121.5000,157.5000) -- (121.5000,129.5000) -- (81.5000,129.5000);



  \end{scope}
  \begin{scope}[scale=1.008,draw=black,line join=bevel,line cap=rect,line width=0.800pt]
  \end{scope}
  \begin{scope}[draw=black,line join=bevel,line cap=rect,line width=0.800pt]
  \end{scope}
  \begin{scope}[cm={{0.0,-1.00769,1.00769,0.0,(118.95021,99.53908)}},draw=black,line join=bevel,line cap=rect,line width=0.800pt]
    \path[fill=black] (4.4657,1.4886) node[above right] (text260-6) {\rotatebox{90}{Value}};



  \end{scope}
  \begin{scope}[cm={{1.00769,0.0,0.0,1.00769,(182.07818,192.50503)}},draw=black,line join=bevel,line cap=rect,line width=0.800pt]
    \path[fill=black] (2.9771,0.0000) node[above right] (text602-9) {Time (min)};



  \end{scope}
\end{scope}
\begin{scope}[cm={{0.95,0.0,0.0,0.95,(99.80481,-7.61988)}},draw=black,line join=bevel,line cap=rect,even odd rule,line width=0.800pt]
  \begin{scope}[draw=black,line join=bevel,line cap=rect,line width=0.800pt]
  \end{scope}
  \begin{scope}[scale=1.010,draw=black,line join=bevel,line cap=rect,line width=0.800pt]
  \end{scope}
  \begin{scope}[cm={{1.01,0.0,0.0,1.01,(6.3158,0.0)}},draw=ca0a0a4,dash pattern=on 0.40pt off 0.80pt,line join=round,line cap=round,line width=0.400pt]
    \path[draw] (25.5000,26.5000) -- (108.5000,26.5000);



    \path[draw] (137.5000,26.5000) -- (142.5000,26.5000);



  \end{scope}
  \begin{scope}[cm={{1.01,0.0,0.0,1.01,(6.3158,0.0)}},draw=black,line join=round,line cap=round,line width=0.480pt]
    \path[draw] (25.5000,26.5000) -- (28.5000,26.5000);



    \path[draw] (142.5000,26.5000) -- (139.5000,26.5000);



  \end{scope}
  \begin{scope}[scale=1.010,draw=black,line join=bevel,line cap=rect,line width=0.800pt]
  \end{scope}
  \begin{scope}[cm={{1.01,0.0,0.0,1.01,(9.09,30.3)}},draw=black,line join=bevel,line cap=rect,line width=0.800pt]
  \end{scope}
  \begin{scope}[cm={{1.01,0.0,0.0,1.01,(9.09,30.3)}},draw=black,line join=bevel,line cap=rect,line width=0.800pt]
  \end{scope}
  \begin{scope}[cm={{1.01,0.0,0.0,1.01,(9.09,30.3)}},draw=black,line join=bevel,line cap=rect,line width=0.800pt]
  \end{scope}
  \begin{scope}[cm={{1.01,0.0,0.0,1.01,(9.09,30.3)}},draw=black,line join=bevel,line cap=rect,line width=0.800pt]
  \end{scope}
  \begin{scope}[cm={{1.01,0.0,0.0,1.01,(9.09,30.3)}},draw=black,line join=bevel,line cap=rect,line width=0.800pt]
  \end{scope}
  \begin{scope}[cm={{1.01,0.0,0.0,1.01,(15.4058,30.3)}},draw=black,line join=bevel,line cap=rect,line width=0.800pt]
    \path[fill=black] (0.0000,0.0000) node[above right] (text36) {30};



  \end{scope}
  \begin{scope}[cm={{1.01,0.0,0.0,1.01,(9.09,30.3)}},draw=black,line join=bevel,line cap=rect,line width=0.800pt]
  \end{scope}
  \begin{scope}[scale=1.010,draw=black,line join=bevel,line cap=rect,line width=0.800pt]
  \end{scope}
  \begin{scope}[cm={{1.01,0.0,0.0,1.01,(6.3158,0.0)}},draw=ca0a0a4,dash pattern=on 0.40pt off 0.80pt,line join=round,line cap=round,line width=0.400pt]
    \path[draw] (25.5000,11.5000) -- (142.5000,11.5000);



  \end{scope}
  \begin{scope}[cm={{1.01,0.0,0.0,1.01,(6.3158,0.0)}},draw=black,line join=round,line cap=round,line width=0.480pt]
    \path[draw] (25.5000,11.5000) -- (28.5000,11.5000);



    \path[draw] (142.5000,11.5000) -- (139.5000,11.5000);



  \end{scope}
  \begin{scope}[scale=1.010,draw=black,line join=bevel,line cap=rect,line width=0.800pt]
  \end{scope}
  \begin{scope}[cm={{1.01,0.0,0.0,1.01,(9.09,15.15)}},draw=black,line join=bevel,line cap=rect,line width=0.800pt]
  \end{scope}
  \begin{scope}[cm={{1.01,0.0,0.0,1.01,(9.09,15.15)}},draw=black,line join=bevel,line cap=rect,line width=0.800pt]
  \end{scope}
  \begin{scope}[cm={{1.01,0.0,0.0,1.01,(9.09,15.15)}},draw=black,line join=bevel,line cap=rect,line width=0.800pt]
  \end{scope}
  \begin{scope}[cm={{1.01,0.0,0.0,1.01,(9.09,15.15)}},draw=black,line join=bevel,line cap=rect,line width=0.800pt]
  \end{scope}
  \begin{scope}[cm={{1.01,0.0,0.0,1.01,(9.09,15.15)}},draw=black,line join=bevel,line cap=rect,line width=0.800pt]
  \end{scope}
  \begin{scope}[cm={{1.01,0.0,0.0,1.01,(15.4058,15.15)}},draw=black,line join=bevel,line cap=rect,line width=0.800pt]
    \path[fill=black] (0.0000,0.0000) node[above right] (text66) {40};



  \end{scope}
  \begin{scope}[cm={{1.01,0.0,0.0,1.01,(9.09,15.15)}},draw=black,line join=bevel,line cap=rect,line width=0.800pt]
  \end{scope}
  \begin{scope}[scale=1.010,draw=black,line join=bevel,line cap=rect,line width=0.800pt]
  \end{scope}
  \begin{scope}[cm={{1.01,0.0,0.0,1.01,(6.3158,0.0)}},draw=ca0a0a4,dash pattern=on 0.40pt off 0.80pt,line join=round,line cap=round,line width=0.400pt]
    \path[draw] (25.5000,32.5000) -- (25.5000,8.5000);



  \end{scope}
  \begin{scope}[cm={{1.01,0.0,0.0,1.01,(6.3158,0.0)}},draw=black,line join=round,line cap=round,line width=0.480pt]
    \path[draw] (25.5000,32.5000) -- (25.5000,28.5000);



    \path[draw] (25.5000,8.5000) -- (25.5000,11.5000);



  \end{scope}
  \begin{scope}[scale=1.010,draw=black,line join=bevel,line cap=rect,line width=0.800pt]
  \end{scope}
  \begin{scope}[cm={{1.01,0.0,0.0,1.01,(26.26,48.48)}},draw=black,line join=bevel,line cap=rect,line width=0.800pt]
  \end{scope}
  \begin{scope}[cm={{1.01,0.0,0.0,1.01,(26.26,48.48)}},draw=black,line join=bevel,line cap=rect,line width=0.800pt]
  \end{scope}
  \begin{scope}[cm={{1.01,0.0,0.0,1.01,(26.26,48.48)}},draw=black,line join=bevel,line cap=rect,line width=0.800pt]
  \end{scope}
  \begin{scope}[cm={{1.01,0.0,0.0,1.01,(26.26,48.48)}},draw=black,line join=bevel,line cap=rect,line width=0.800pt]
  \end{scope}
  \begin{scope}[cm={{1.01,0.0,0.0,1.01,(26.26,48.48)}},draw=black,line join=bevel,line cap=rect,line width=0.800pt]
  \end{scope}
  \begin{scope}[cm={{1.01,0.0,0.0,1.01,(26.26,48.48)}},draw=black,line join=bevel,line cap=rect,line width=0.800pt]
  \end{scope}
  \begin{scope}[scale=1.010,draw=black,line join=bevel,line cap=rect,line width=0.800pt]
  \end{scope}
  \begin{scope}[cm={{1.01,0.0,0.0,1.01,(6.3158,0.0)}},draw=ca0a0a4,dash pattern=on 0.40pt off 0.80pt,line join=round,line cap=round,line width=0.400pt]
    \path[draw] (60.5000,32.5000) -- (60.5000,8.5000);



  \end{scope}
  \begin{scope}[cm={{1.01,0.0,0.0,1.01,(6.3158,0.0)}},draw=black,line join=round,line cap=round,line width=0.480pt]
    \path[draw] (60.5000,32.5000) -- (60.5000,28.5000);



    \path[draw] (60.5000,8.5000) -- (60.5000,11.5000);



  \end{scope}
  \begin{scope}[scale=1.010,draw=black,line join=bevel,line cap=rect,line width=0.800pt]
  \end{scope}
  \begin{scope}[cm={{1.01,0.0,0.0,1.01,(61.61,48.48)}},draw=black,line join=bevel,line cap=rect,line width=0.800pt]
  \end{scope}
  \begin{scope}[cm={{1.01,0.0,0.0,1.01,(61.61,48.48)}},draw=black,line join=bevel,line cap=rect,line width=0.800pt]
  \end{scope}
  \begin{scope}[cm={{1.01,0.0,0.0,1.01,(61.61,48.48)}},draw=black,line join=bevel,line cap=rect,line width=0.800pt]
  \end{scope}
  \begin{scope}[cm={{1.01,0.0,0.0,1.01,(61.61,48.48)}},draw=black,line join=bevel,line cap=rect,line width=0.800pt]
  \end{scope}
  \begin{scope}[cm={{1.01,0.0,0.0,1.01,(61.61,48.48)}},draw=black,line join=bevel,line cap=rect,line width=0.800pt]
  \end{scope}
  \begin{scope}[cm={{1.01,0.0,0.0,1.01,(61.61,48.48)}},draw=black,line join=bevel,line cap=rect,line width=0.800pt]
  \end{scope}
  \begin{scope}[scale=1.010,draw=black,line join=bevel,line cap=rect,line width=0.800pt]
  \end{scope}
  \begin{scope}[cm={{1.01,0.0,0.0,1.01,(6.3158,0.0)}},draw=ca0a0a4,dash pattern=on 0.40pt off 0.80pt,line join=round,line cap=round,line width=0.400pt]
    \path[draw] (95.5000,32.5000) -- (95.5000,8.5000);



  \end{scope}
  \begin{scope}[cm={{1.01,0.0,0.0,1.01,(6.3158,0.0)}},draw=black,line join=round,line cap=round,line width=0.480pt]
    \path[draw] (95.5000,32.5000) -- (95.5000,28.5000);



    \path[draw] (95.5000,8.5000) -- (95.5000,11.5000);



  \end{scope}
  \begin{scope}[scale=1.010,draw=black,line join=bevel,line cap=rect,line width=0.800pt]
  \end{scope}
  \begin{scope}[cm={{1.01,0.0,0.0,1.01,(96.96,48.48)}},draw=black,line join=bevel,line cap=rect,line width=0.800pt]
  \end{scope}
  \begin{scope}[cm={{1.01,0.0,0.0,1.01,(96.96,48.48)}},draw=black,line join=bevel,line cap=rect,line width=0.800pt]
  \end{scope}
  \begin{scope}[cm={{1.01,0.0,0.0,1.01,(96.96,48.48)}},draw=black,line join=bevel,line cap=rect,line width=0.800pt]
  \end{scope}
  \begin{scope}[cm={{1.01,0.0,0.0,1.01,(96.96,48.48)}},draw=black,line join=bevel,line cap=rect,line width=0.800pt]
  \end{scope}
  \begin{scope}[cm={{1.01,0.0,0.0,1.01,(96.96,48.48)}},draw=black,line join=bevel,line cap=rect,line width=0.800pt]
  \end{scope}
  \begin{scope}[cm={{1.01,0.0,0.0,1.01,(96.96,48.48)}},draw=black,line join=bevel,line cap=rect,line width=0.800pt]
  \end{scope}
  \begin{scope}[scale=1.010,draw=black,line join=bevel,line cap=rect,line width=0.800pt]
  \end{scope}
  \begin{scope}[cm={{1.01,0.0,0.0,1.01,(6.3158,0.0)}},draw=ca0a0a4,dash pattern=on 0.40pt off 0.80pt,line join=round,line cap=round,line width=0.400pt]
    \path[draw] (130.5000,20.5000) -- (130.5000,8.5000);



  \end{scope}
  \begin{scope}[cm={{1.01,0.0,0.0,1.01,(6.3158,0.0)}},draw=black,line join=round,line cap=round,line width=0.480pt]
    \path[draw] (130.5000,32.5000) -- (130.5000,28.5000);



    \path[draw] (130.5000,8.5000) -- (130.5000,11.5000);



  \end{scope}
  \begin{scope}[scale=1.010,draw=black,line join=bevel,line cap=rect,line width=0.800pt]
  \end{scope}
  \begin{scope}[cm={{1.01,0.0,0.0,1.01,(132.31,48.48)}},draw=black,line join=bevel,line cap=rect,line width=0.800pt]
  \end{scope}
  \begin{scope}[cm={{1.01,0.0,0.0,1.01,(132.31,48.48)}},draw=black,line join=bevel,line cap=rect,line width=0.800pt]
  \end{scope}
  \begin{scope}[cm={{1.01,0.0,0.0,1.01,(132.31,48.48)}},draw=black,line join=bevel,line cap=rect,line width=0.800pt]
  \end{scope}
  \begin{scope}[cm={{1.01,0.0,0.0,1.01,(132.31,48.48)}},draw=black,line join=bevel,line cap=rect,line width=0.800pt]
  \end{scope}
  \begin{scope}[cm={{1.01,0.0,0.0,1.01,(132.31,48.48)}},draw=black,line join=bevel,line cap=rect,line width=0.800pt]
  \end{scope}
  \begin{scope}[cm={{1.01,0.0,0.0,1.01,(132.31,48.48)}},draw=black,line join=bevel,line cap=rect,line width=0.800pt]
  \end{scope}
  \begin{scope}[scale=1.010,draw=black,line join=bevel,line cap=rect,line width=0.800pt]
  \end{scope}
  \begin{scope}[cm={{1.01,0.0,0.0,1.01,(6.3158,0.0)}},draw=black,line join=round,line cap=round,line width=0.480pt]
    \path[draw] (25.5000,8.5000) -- (25.5000,32.5000) -- (142.5000,32.5000) -- (142.5000,8.5000) -- (25.5000,8.5000);



  \end{scope}
  \begin{scope}[scale=1.010,draw=black,line join=bevel,line cap=rect,line width=0.800pt]
  \end{scope}
  \begin{scope}[cm={{1.01,0.0,0.0,1.01,(109.08,32.32)}},draw=black,line join=bevel,line cap=rect,line width=0.800pt]
  \end{scope}
  \begin{scope}[cm={{1.01,0.0,0.0,1.01,(109.08,32.32)}},draw=black,line join=bevel,line cap=rect,line width=0.800pt]
  \end{scope}
  \begin{scope}[cm={{1.01,0.0,0.0,1.01,(109.08,32.32)}},draw=black,line join=bevel,line cap=rect,line width=0.800pt]
  \end{scope}
  \begin{scope}[cm={{1.01,0.0,0.0,1.01,(109.08,32.32)}},draw=black,line join=bevel,line cap=rect,line width=0.800pt]
  \end{scope}
  \begin{scope}[cm={{1.01,0.0,0.0,1.01,(109.08,32.32)}},draw=black,line join=bevel,line cap=rect,line width=0.800pt]
  \end{scope}
  \begin{scope}[cm={{1.01,0.0,0.0,1.01,(119.30192,30.69248)}},draw=black,line join=bevel,line cap=rect,line width=0.800pt]
    \path[fill=black] (0.0000,0.0000) node[above right] (text194) {\scriptsize $\alpha_0$};



  \end{scope}
  \begin{scope}[cm={{1.01,0.0,0.0,1.01,(109.08,32.32)}},draw=black,line join=bevel,line cap=rect,line width=0.800pt]
  \end{scope}
  \begin{scope}[scale=1.010,draw=black,line join=bevel,line cap=rect,line width=0.800pt]
  \end{scope}
  \begin{scope}[cm={{1.01,0.0,0.0,1.01,(6.3158,0.0)}},draw=black,line join=round,line cap=round,line width=0.480pt]
    \path[draw,even odd rule] (123.5000,28.5000) -- (132.5000,28.5000);



  \end{scope}
  \begin{scope}[scale=1.010,draw=black,line join=bevel,line cap=rect,line width=0.800pt]
  \end{scope}
  \begin{scope}[scale=1.010,draw=black,line join=bevel,line cap=rect,line width=0.800pt]
  \end{scope}
  \begin{scope}[scale=1.010,draw=black,line join=bevel,line cap=rect,line width=0.800pt]
  \end{scope}
  \begin{scope}[scale=1.010,draw=black,line join=bevel,line cap=rect,line width=0.800pt]
  \end{scope}
  \begin{scope}[cm={{1.01,0.0,0.0,1.01,(6.3158,0.0)}},draw=black,line join=round,line cap=round,line width=0.480pt]
    \path[draw] (25.8000,32.0000) -- (26.2000,14.3000) -- (26.7000,17.2000) -- (27.1000,18.1000) -- (27.5000,14.6000) -- (27.9000,12.6000) -- (28.3000,14.9000) -- (28.8000,17.3000) -- (29.2000,10.3000) -- (29.6000,14.5000) -- (30.0000,19.1000) -- (30.5000,15.7000) -- (30.9000,12.7000) -- (31.3000,15.0000) -- (31.7000,18.7000) -- (32.2000,19.3000) -- (32.6000,16.6000) -- (33.0000,13.7000) -- (33.4000,12.8000) -- (33.8000,14.3000) -- (34.3000,16.9000) -- (34.7000,18.9000) -- (35.1000,19.4000) -- (35.5000,18.6000) -- (36.0000,17.2000) -- (36.4000,15.9000) -- (36.8000,15.5000) -- (37.2000,15.9000) -- (37.6000,17.1000) -- (38.1000,18.6000) -- (38.5000,20.0000) -- (38.9000,21.0000) -- (39.3000,21.3000) -- (39.8000,21.0000) -- (40.2000,20.3000) -- (40.6000,19.3000) -- (41.0000,18.3000) -- (41.5000,17.4000) -- (41.9000,16.9000) -- (42.3000,16.8000) -- (42.7000,17.1000) -- (43.1000,17.6000) -- (43.6000,18.3000) -- (44.0000,19.0000) -- (44.4000,19.7000) -- (44.8000,20.1000) -- (45.3000,20.4000) -- (45.7000,20.5000) -- (46.1000,20.4000) -- (46.5000,20.1000) -- (47.0000,19.8000) -- (47.4000,19.3000) -- (47.8000,18.5000) -- (48.2000,17.8000) -- (48.6000,17.2000) -- (49.1000,16.7000) -- (49.5000,16.5000) -- (49.9000,16.4000) -- (50.3000,16.4000) -- (50.8000,16.5000) -- (51.2000,16.7000) -- (51.6000,17.0000) -- (52.0000,17.2000) -- (52.4000,17.4000) -- (52.9000,17.5000) -- (53.3000,17.5000) -- (53.7000,17.3000) -- (54.1000,17.0000) -- (54.6000,16.6000) -- (55.0000,16.1000) -- (55.4000,15.6000) -- (55.8000,15.0000) -- (56.3000,14.4000) -- (56.7000,13.9000) -- (57.1000,13.5000) -- (57.5000,13.2000) -- (57.9000,13.2000) -- (58.4000,13.3000) -- (58.8000,13.5000) -- (59.2000,13.9000) -- (59.6000,14.3000) -- (60.1000,14.7000) -- (60.5000,15.0000) -- (60.9000,15.4000) -- (61.3000,15.7000) -- (61.8000,16.0000) -- (62.2000,16.2000) -- (62.6000,16.3000) -- (63.0000,16.2000) -- (63.4000,16.2000) -- (63.9000,16.1000) -- (64.3000,16.0000) -- (64.7000,15.9000) -- (65.1000,15.9000) -- (65.6000,15.9000) -- (66.0000,16.0000) -- (66.4000,16.2000) -- (66.8000,16.4000) -- (67.3000,16.7000) -- (67.7000,16.9000) -- (68.1000,17.2000) -- (68.5000,17.3000) -- (68.9000,17.4000) -- (69.4000,17.5000) -- (69.8000,17.7000) -- (70.2000,17.7000) -- (70.6000,17.7000) -- (71.1000,17.6000) -- (71.5000,17.5000) -- (71.9000,17.3000) -- (72.3000,17.1000) -- (72.7000,17.0000) -- (73.2000,16.9000) -- (73.6000,16.8000) -- (74.0000,16.9000) -- (74.4000,17.0000) -- (74.9000,16.9000) -- (75.3000,16.8000) -- (75.7000,16.8000) -- (76.1000,16.8000) -- (76.6000,16.9000) -- (77.0000,17.1000) -- (77.4000,17.2000) -- (77.8000,17.3000) -- (78.2000,17.4000) -- (78.7000,17.4000) -- (79.1000,17.4000) -- (79.5000,17.4000) -- (79.9000,17.4000) -- (80.4000,17.3000) -- (80.8000,17.2000) -- (81.2000,17.1000) -- (81.6000,17.1000) -- (82.1000,17.0000) -- (82.5000,17.0000) -- (82.9000,16.9000) -- (83.3000,16.7000) -- (83.7000,16.6000) -- (84.2000,16.4000) -- (84.6000,16.4000) -- (85.0000,16.3000) -- (85.4000,16.3000) -- (85.9000,16.3000) -- (86.3000,16.3000) -- (86.7000,16.2000) -- (87.1000,16.2000) -- (87.6000,16.1000) -- (88.0000,16.1000) -- (88.4000,16.2000) -- (88.8000,16.3000) -- (89.2000,16.3000) -- (89.7000,16.3000) -- (90.1000,16.3000) -- (90.5000,16.3000) -- (90.9000,16.3000) -- (91.4000,16.3000) -- (91.8000,16.3000) -- (92.2000,16.3000) -- (92.6000,16.3000) -- (93.0000,16.2000) -- (93.5000,16.2000) -- (93.9000,16.2000) -- (94.3000,16.2000) -- (94.7000,16.2000) -- (95.2000,16.1000) -- (95.6000,16.1000) -- (96.0000,16.1000) -- (96.4000,16.2000) -- (96.9000,16.4000) -- (97.3000,16.5000) -- (97.7000,16.5000) -- (98.1000,16.6000) -- (98.5000,16.6000) -- (99.0000,16.6000) -- (99.4000,16.6000) -- (99.8000,16.5000) -- (100.2000,16.5000) -- (100.7000,16.6000) -- (101.1000,16.7000) -- (101.5000,16.6000) -- (101.9000,16.5000) -- (102.3000,16.3000) -- (102.8000,16.3000) -- (103.2000,16.3000) -- (103.6000,16.3000) -- (104.0000,16.3000) -- (104.5000,16.3000) -- (104.9000,16.3000) -- (105.3000,16.3000) -- (105.7000,16.2000) -- (106.2000,16.2000) -- (106.6000,16.2000) -- (107.0000,16.2000) -- (107.4000,16.2000) -- (107.8000,16.2000) -- (108.3000,16.3000) -- (108.7000,16.3000) -- (109.1000,16.4000) -- (109.5000,16.5000) -- (110.0000,16.5000) -- (110.4000,16.5000) -- (110.8000,16.5000) -- (111.2000,16.5000) -- (111.7000,16.5000) -- (112.1000,16.6000) -- (112.5000,16.6000) -- (112.9000,16.6000) -- (113.3000,16.5000) -- (113.8000,16.5000) -- (114.2000,16.4000) -- (114.6000,16.3000) -- (115.0000,16.3000) -- (115.5000,16.4000) -- (115.9000,16.5000) -- (116.3000,16.5000) -- (116.7000,16.5000) -- (117.2000,16.5000) -- (117.6000,16.5000) -- (118.0000,16.5000) -- (118.4000,16.5000) -- (118.8000,16.6000) -- (119.3000,16.6000) -- (119.7000,16.6000) -- (120.1000,16.5000) -- (120.5000,16.5000) -- (121.0000,16.5000) -- (121.4000,16.5000) -- (121.8000,16.5000) -- (122.2000,16.3000) -- (122.6000,16.3000) -- (123.1000,16.3000) -- (123.5000,16.4000) -- (123.9000,16.4000) -- (124.3000,16.4000) -- (124.8000,16.4000) -- (125.2000,16.4000) -- (125.6000,16.3000) -- (126.0000,16.3000) -- (126.5000,16.2000) -- (126.9000,16.2000) -- (127.3000,16.3000) -- (127.7000,16.5000) -- (128.1000,16.5000) -- (128.6000,16.4000) -- (129.0000,16.3000) -- (129.4000,16.3000) -- (129.8000,16.4000) -- (130.3000,16.5000) -- (130.7000,16.5000) -- (131.1000,16.6000) -- (131.5000,16.6000) -- (131.9000,16.6000) -- (132.4000,16.6000) -- (132.8000,16.5000) -- (133.2000,16.5000) -- (133.6000,16.4000) -- (134.1000,16.4000) -- (134.5000,16.3000) -- (134.9000,16.3000) -- (135.3000,16.3000) -- (135.8000,16.4000) -- (136.2000,16.4000) -- (136.6000,16.4000) -- (137.0000,16.4000) -- (137.4000,16.4000) -- (137.9000,16.4000) -- (138.3000,16.4000) -- (138.7000,16.5000) -- (139.1000,16.5000) -- (139.6000,16.5000) -- (140.0000,16.5000) -- (140.4000,16.5000) -- (140.8000,16.4000) -- (141.3000,16.4000) -- (141.7000,16.3000) -- (142.1000,16.4000) -- (142.3000,16.4000);



  \end{scope}
  \begin{scope}[scale=1.010,draw=black,line join=bevel,line cap=rect,line width=0.800pt]
  \end{scope}
  \begin{scope}[scale=1.010,draw=black,line join=bevel,line cap=rect,line width=0.800pt]
  \end{scope}
  \begin{scope}[cm={{1.01,0.0,0.0,1.01,(6.3158,0.0)}},draw=black,line join=round,line cap=round,line width=0.480pt]
    \path[draw] (25.5000,8.5000) -- (25.5000,32.5000) -- (142.5000,32.5000) -- (142.5000,8.5000) -- (25.5000,8.5000);



  \end{scope}
  \begin{scope}[cm={{1.01,0.0,0.0,1.01,(6.3158,0.0)}},draw=ca0a0a4,dash pattern=on 0.40pt off 0.80pt,line join=round,line cap=round,line width=0.400pt]
    \path[draw] (25.5000,46.5000) -- (108.5000,46.5000);



    \path[draw] (137.5000,46.5000) -- (142.5000,46.5000);



  \end{scope}
  \begin{scope}[cm={{1.01,0.0,0.0,1.01,(6.3158,0.0)}},draw=black,line join=round,line cap=round,line width=0.480pt]
    \path[draw] (25.5000,46.5000) -- (28.5000,46.5000);



    \path[draw] (142.5000,46.5000) -- (139.5000,46.5000);



  \end{scope}
  \begin{scope}[scale=1.010,draw=black,line join=bevel,line cap=rect,line width=0.800pt]
  \end{scope}
  \begin{scope}[cm={{1.01,0.0,0.0,1.01,(5.05,50.5)}},draw=black,line join=bevel,line cap=rect,line width=0.800pt]
  \end{scope}
  \begin{scope}[cm={{1.01,0.0,0.0,1.01,(5.05,50.5)}},draw=black,line join=bevel,line cap=rect,line width=0.800pt]
  \end{scope}
  \begin{scope}[cm={{1.01,0.0,0.0,1.01,(5.05,50.5)}},draw=black,line join=bevel,line cap=rect,line width=0.800pt]
  \end{scope}
  \begin{scope}[cm={{1.01,0.0,0.0,1.01,(5.05,50.5)}},draw=black,line join=bevel,line cap=rect,line width=0.800pt]
  \end{scope}
  \begin{scope}[cm={{1.01,0.0,0.0,1.01,(5.05,50.5)}},draw=black,line join=bevel,line cap=rect,line width=0.800pt]
  \end{scope}
  \begin{scope}[cm={{1.01,0.0,0.0,1.01,(11.3658,50.5)}},draw=black,line join=bevel,line cap=rect,line width=0.800pt]
    \path[fill=black] (0.0000,0.0000) node[above right] (text250) {-10};



  \end{scope}
  \begin{scope}[cm={{1.01,0.0,0.0,1.01,(5.05,50.5)}},draw=black,line join=bevel,line cap=rect,line width=0.800pt]
  \end{scope}
  \begin{scope}[scale=1.010,draw=black,line join=bevel,line cap=rect,line width=0.800pt]
  \end{scope}
  \begin{scope}[cm={{1.01,0.0,0.0,1.01,(6.3158,0.0)}},draw=ca0a0a4,dash pattern=on 0.40pt off 0.80pt,line join=round,line cap=round,line width=0.400pt]
    \path[draw] (25.5000,34.5000) -- (142.5000,34.5000);



  \end{scope}
  \begin{scope}[cm={{1.01,0.0,0.0,1.01,(6.3158,0.0)}},draw=black,line join=round,line cap=round,line width=0.480pt]
    \path[draw] (25.5000,34.5000) -- (28.5000,34.5000);



    \path[draw] (142.5000,34.5000) -- (139.5000,34.5000);



  \end{scope}
  \begin{scope}[scale=1.010,draw=black,line join=bevel,line cap=rect,line width=0.800pt]
  \end{scope}
  \begin{scope}[cm={{1.01,0.0,0.0,1.01,(9.09,39.39)}},draw=black,line join=bevel,line cap=rect,line width=0.800pt]
  \end{scope}
  \begin{scope}[cm={{1.01,0.0,0.0,1.01,(9.09,39.39)}},draw=black,line join=bevel,line cap=rect,line width=0.800pt]
  \end{scope}
  \begin{scope}[cm={{1.01,0.0,0.0,1.01,(9.09,39.39)}},draw=black,line join=bevel,line cap=rect,line width=0.800pt]
  \end{scope}
  \begin{scope}[cm={{1.01,0.0,0.0,1.01,(9.09,39.39)}},draw=black,line join=bevel,line cap=rect,line width=0.800pt]
  \end{scope}
  \begin{scope}[cm={{1.01,0.0,0.0,1.01,(9.09,39.39)}},draw=black,line join=bevel,line cap=rect,line width=0.800pt]
  \end{scope}
  \begin{scope}[cm={{1.01,0.0,0.0,1.01,(15.4058,39.39)}},draw=black,line join=bevel,line cap=rect,line width=0.800pt]
    \path[fill=black] (0.0000,0.0000) node[above right] (text280) {10};



  \end{scope}
  \begin{scope}[cm={{1.01,0.0,0.0,1.01,(9.09,39.39)}},draw=black,line join=bevel,line cap=rect,line width=0.800pt]
  \end{scope}
  \begin{scope}[scale=1.010,draw=black,line join=bevel,line cap=rect,line width=0.800pt]
  \end{scope}
  \begin{scope}[cm={{1.01,0.0,0.0,1.01,(6.3158,0.0)}},draw=ca0a0a4,dash pattern=on 0.40pt off 0.80pt,line join=round,line cap=round,line width=0.400pt]
    \path[draw] (25.5000,56.5000) -- (25.5000,32.5000);



  \end{scope}
  \begin{scope}[cm={{1.01,0.0,0.0,1.01,(6.3158,0.0)}},draw=black,line join=round,line cap=round,line width=0.480pt]
    \path[draw] (25.5000,56.5000) -- (25.5000,51.5000);



    \path[draw] (25.5000,32.5000) -- (25.5000,36.5000);



  \end{scope}
  \begin{scope}[scale=1.010,draw=black,line join=bevel,line cap=rect,line width=0.800pt]
  \end{scope}
  \begin{scope}[cm={{1.01,0.0,0.0,1.01,(26.26,72.72)}},draw=black,line join=bevel,line cap=rect,line width=0.800pt]
  \end{scope}
  \begin{scope}[cm={{1.01,0.0,0.0,1.01,(26.26,72.72)}},draw=black,line join=bevel,line cap=rect,line width=0.800pt]
  \end{scope}
  \begin{scope}[cm={{1.01,0.0,0.0,1.01,(26.26,72.72)}},draw=black,line join=bevel,line cap=rect,line width=0.800pt]
  \end{scope}
  \begin{scope}[cm={{1.01,0.0,0.0,1.01,(26.26,72.72)}},draw=black,line join=bevel,line cap=rect,line width=0.800pt]
  \end{scope}
  \begin{scope}[cm={{1.01,0.0,0.0,1.01,(26.26,72.72)}},draw=black,line join=bevel,line cap=rect,line width=0.800pt]
  \end{scope}
  \begin{scope}[cm={{1.01,0.0,0.0,1.01,(26.26,72.72)}},draw=black,line join=bevel,line cap=rect,line width=0.800pt]
  \end{scope}
  \begin{scope}[scale=1.010,draw=black,line join=bevel,line cap=rect,line width=0.800pt]
  \end{scope}
  \begin{scope}[cm={{1.01,0.0,0.0,1.01,(6.3158,0.0)}},draw=ca0a0a4,dash pattern=on 0.40pt off 0.80pt,line join=round,line cap=round,line width=0.400pt]
    \path[draw] (60.5000,56.5000) -- (60.5000,32.5000);



  \end{scope}
  \begin{scope}[cm={{1.01,0.0,0.0,1.01,(6.3158,0.0)}},draw=black,line join=round,line cap=round,line width=0.480pt]
    \path[draw] (60.5000,56.5000) -- (60.5000,51.5000);



    \path[draw] (60.5000,32.5000) -- (60.5000,36.5000);



  \end{scope}
  \begin{scope}[scale=1.010,draw=black,line join=bevel,line cap=rect,line width=0.800pt]
  \end{scope}
  \begin{scope}[cm={{1.01,0.0,0.0,1.01,(61.61,72.72)}},draw=black,line join=bevel,line cap=rect,line width=0.800pt]
  \end{scope}
  \begin{scope}[cm={{1.01,0.0,0.0,1.01,(61.61,72.72)}},draw=black,line join=bevel,line cap=rect,line width=0.800pt]
  \end{scope}
  \begin{scope}[cm={{1.01,0.0,0.0,1.01,(61.61,72.72)}},draw=black,line join=bevel,line cap=rect,line width=0.800pt]
  \end{scope}
  \begin{scope}[cm={{1.01,0.0,0.0,1.01,(61.61,72.72)}},draw=black,line join=bevel,line cap=rect,line width=0.800pt]
  \end{scope}
  \begin{scope}[cm={{1.01,0.0,0.0,1.01,(61.61,72.72)}},draw=black,line join=bevel,line cap=rect,line width=0.800pt]
  \end{scope}
  \begin{scope}[cm={{1.01,0.0,0.0,1.01,(61.61,72.72)}},draw=black,line join=bevel,line cap=rect,line width=0.800pt]
  \end{scope}
  \begin{scope}[scale=1.010,draw=black,line join=bevel,line cap=rect,line width=0.800pt]
  \end{scope}
  \begin{scope}[cm={{1.01,0.0,0.0,1.01,(6.3158,0.0)}},draw=ca0a0a4,dash pattern=on 0.40pt off 0.80pt,line join=round,line cap=round,line width=0.400pt]
    \path[draw] (95.5000,56.5000) -- (95.5000,32.5000);



  \end{scope}
  \begin{scope}[cm={{1.01,0.0,0.0,1.01,(6.3158,0.0)}},draw=black,line join=round,line cap=round,line width=0.480pt]
    \path[draw] (95.5000,56.5000) -- (95.5000,51.5000);



    \path[draw] (95.5000,32.5000) -- (95.5000,36.5000);



  \end{scope}
  \begin{scope}[scale=1.010,draw=black,line join=bevel,line cap=rect,line width=0.800pt]
  \end{scope}
  \begin{scope}[cm={{1.01,0.0,0.0,1.01,(96.96,72.72)}},draw=black,line join=bevel,line cap=rect,line width=0.800pt]
  \end{scope}
  \begin{scope}[cm={{1.01,0.0,0.0,1.01,(96.96,72.72)}},draw=black,line join=bevel,line cap=rect,line width=0.800pt]
  \end{scope}
  \begin{scope}[cm={{1.01,0.0,0.0,1.01,(96.96,72.72)}},draw=black,line join=bevel,line cap=rect,line width=0.800pt]
  \end{scope}
  \begin{scope}[cm={{1.01,0.0,0.0,1.01,(96.96,72.72)}},draw=black,line join=bevel,line cap=rect,line width=0.800pt]
  \end{scope}
  \begin{scope}[cm={{1.01,0.0,0.0,1.01,(96.96,72.72)}},draw=black,line join=bevel,line cap=rect,line width=0.800pt]
  \end{scope}
  \begin{scope}[cm={{1.01,0.0,0.0,1.01,(96.96,72.72)}},draw=black,line join=bevel,line cap=rect,line width=0.800pt]
  \end{scope}
  \begin{scope}[scale=1.010,draw=black,line join=bevel,line cap=rect,line width=0.800pt]
  \end{scope}
  \begin{scope}[cm={{1.01,0.0,0.0,1.01,(6.3158,0.0)}},draw=ca0a0a4,dash pattern=on 0.40pt off 0.80pt,line join=round,line cap=round,line width=0.400pt]
    \path[draw] (130.5000,44.5000) -- (130.5000,32.5000);



  \end{scope}
  \begin{scope}[cm={{1.01,0.0,0.0,1.01,(6.3158,0.0)}},draw=black,line join=round,line cap=round,line width=0.480pt]
    \path[draw] (130.5000,56.5000) -- (130.5000,51.5000);



    \path[draw] (130.5000,32.5000) -- (130.5000,36.5000);



  \end{scope}
  \begin{scope}[scale=1.010,draw=black,line join=bevel,line cap=rect,line width=0.800pt]
  \end{scope}
  \begin{scope}[cm={{1.01,0.0,0.0,1.01,(132.31,72.72)}},draw=black,line join=bevel,line cap=rect,line width=0.800pt]
  \end{scope}
  \begin{scope}[cm={{1.01,0.0,0.0,1.01,(132.31,72.72)}},draw=black,line join=bevel,line cap=rect,line width=0.800pt]
  \end{scope}
  \begin{scope}[cm={{1.01,0.0,0.0,1.01,(132.31,72.72)}},draw=black,line join=bevel,line cap=rect,line width=0.800pt]
  \end{scope}
  \begin{scope}[cm={{1.01,0.0,0.0,1.01,(132.31,72.72)}},draw=black,line join=bevel,line cap=rect,line width=0.800pt]
  \end{scope}
  \begin{scope}[cm={{1.01,0.0,0.0,1.01,(132.31,72.72)}},draw=black,line join=bevel,line cap=rect,line width=0.800pt]
  \end{scope}
  \begin{scope}[cm={{1.01,0.0,0.0,1.01,(132.31,72.72)}},draw=black,line join=bevel,line cap=rect,line width=0.800pt]
  \end{scope}
  \begin{scope}[scale=1.010,draw=black,line join=bevel,line cap=rect,line width=0.800pt]
  \end{scope}
  \begin{scope}[cm={{1.01,0.0,0.0,1.01,(6.3158,0.0)}},draw=black,line join=round,line cap=round,line width=0.480pt]
    \path[draw] (25.5000,32.5000) -- (25.5000,56.5000) -- (142.5000,56.5000) -- (142.5000,32.5000) -- (25.5000,32.5000);



  \end{scope}
  \begin{scope}[scale=1.010,draw=black,line join=bevel,line cap=rect,line width=0.800pt]
  \end{scope}
  \begin{scope}[cm={{1.01,0.0,0.0,1.01,(110.09,56.56)}},draw=black,line join=bevel,line cap=rect,line width=0.800pt]
  \end{scope}
  \begin{scope}[cm={{1.01,0.0,0.0,1.01,(110.09,56.56)}},draw=black,line join=bevel,line cap=rect,line width=0.800pt]
  \end{scope}
  \begin{scope}[cm={{1.01,0.0,0.0,1.01,(110.09,56.56)}},draw=black,line join=bevel,line cap=rect,line width=0.800pt]
  \end{scope}
  \begin{scope}[cm={{1.01,0.0,0.0,1.01,(110.09,56.56)}},draw=black,line join=bevel,line cap=rect,line width=0.800pt]
  \end{scope}
  \begin{scope}[cm={{1.01,0.0,0.0,1.01,(110.09,56.56)}},draw=black,line join=bevel,line cap=rect,line width=0.800pt]
  \end{scope}
  \begin{scope}[cm={{1.01,0.0,0.0,1.01,(120.31192,54.93247)}},draw=black,line join=bevel,line cap=rect,line width=0.800pt]
    \path[fill=black] (0.0000,0.0000) node[above right] (text408) {\scriptsize $\alpha_1$};



  \end{scope}
  \begin{scope}[cm={{1.01,0.0,0.0,1.01,(110.09,56.56)}},draw=black,line join=bevel,line cap=rect,line width=0.800pt]
  \end{scope}
  \begin{scope}[scale=1.010,draw=black,line join=bevel,line cap=rect,line width=0.800pt]
  \end{scope}
  \begin{scope}[cm={{1.01,0.0,0.0,1.01,(6.3158,0.0)}},draw=black,line join=round,line cap=round,line width=0.480pt]
    \path[draw,even odd rule] (123.5000,52.5000) -- (132.5000,52.5000);



  \end{scope}
  \begin{scope}[scale=1.010,draw=black,line join=bevel,line cap=rect,line width=0.800pt]
  \end{scope}
  \begin{scope}[scale=1.010,draw=black,line join=bevel,line cap=rect,line width=0.800pt]
  \end{scope}
  \begin{scope}[scale=1.010,draw=black,line join=bevel,line cap=rect,line width=0.800pt]
  \end{scope}
  \begin{scope}[scale=1.010,draw=black,line join=bevel,line cap=rect,line width=0.800pt]
  \end{scope}
  \begin{scope}[cm={{1.01,0.0,0.0,1.01,(6.3158,0.0)}},draw=black,line join=round,line cap=round,line width=0.480pt]
    \path[draw] (25.8000,36.4000) -- (25.8000,36.4000) -- (26.2000,41.9000) -- (26.7000,39.4000) -- (27.1000,40.5000) -- (27.5000,40.1000) -- (27.9000,39.9000) -- (28.3000,40.6000) -- (28.8000,43.1000) -- (29.2000,39.4000) -- (29.6000,39.6000) -- (30.0000,41.8000) -- (30.5000,40.8000) -- (30.9000,39.0000) -- (31.3000,39.5000) -- (31.7000,41.2000) -- (32.2000,42.0000) -- (32.6000,41.2000) -- (33.0000,39.7000) -- (33.4000,38.9000) -- (33.8000,39.1000) -- (34.3000,40.2000) -- (34.7000,41.3000) -- (35.1000,41.9000) -- (35.5000,41.9000) -- (36.0000,41.5000) -- (36.4000,40.9000) -- (36.8000,40.4000) -- (37.2000,40.4000) -- (37.6000,40.7000) -- (38.1000,41.2000) -- (38.5000,41.8000) -- (38.9000,42.2000) -- (39.3000,42.3000) -- (39.8000,42.2000) -- (40.2000,41.8000) -- (40.6000,41.3000) -- (41.0000,40.6000) -- (41.5000,39.9000) -- (41.9000,39.3000) -- (42.3000,38.9000) -- (42.7000,38.7000) -- (43.1000,38.7000) -- (43.6000,38.8000) -- (44.0000,39.0000) -- (44.4000,39.2000) -- (44.8000,39.4000) -- (45.3000,39.6000) -- (45.7000,39.8000) -- (46.1000,39.8000) -- (46.5000,39.8000) -- (47.0000,39.8000) -- (47.4000,39.7000) -- (47.8000,39.5000) -- (48.2000,39.3000) -- (48.6000,39.1000) -- (49.1000,39.0000) -- (49.5000,38.9000) -- (49.9000,38.9000) -- (50.3000,39.0000) -- (50.8000,39.2000) -- (51.2000,39.4000) -- (51.6000,39.7000) -- (52.0000,39.9000) -- (52.4000,40.2000) -- (52.9000,40.5000) -- (53.3000,40.7000) -- (53.7000,40.8000) -- (54.1000,40.9000) -- (54.6000,40.9000) -- (55.0000,40.9000) -- (55.4000,40.8000) -- (55.8000,40.6000) -- (56.3000,40.5000) -- (56.7000,40.3000) -- (57.1000,40.1000) -- (57.5000,40.0000) -- (57.9000,40.0000) -- (58.4000,40.0000) -- (58.8000,40.1000) -- (59.2000,40.3000) -- (59.6000,40.5000) -- (60.1000,40.7000) -- (60.5000,40.9000) -- (60.9000,41.1000) -- (61.3000,41.4000) -- (61.8000,41.6000) -- (62.2000,41.7000) -- (62.6000,41.8000) -- (63.0000,41.9000) -- (63.4000,41.9000) -- (63.9000,41.8000) -- (64.3000,41.8000) -- (64.7000,41.7000) -- (65.1000,41.6000) -- (65.6000,41.6000) -- (66.0000,41.5000) -- (66.4000,41.4000) -- (66.8000,41.4000) -- (67.3000,41.4000) -- (67.7000,41.3000) -- (68.1000,41.3000) -- (68.5000,41.2000) -- (68.9000,41.1000) -- (69.4000,41.0000) -- (69.8000,40.9000) -- (70.2000,40.8000) -- (70.6000,40.7000) -- (71.1000,40.5000) -- (71.5000,40.3000) -- (71.9000,40.1000) -- (72.3000,39.9000) -- (72.7000,39.7000) -- (73.2000,39.5000) -- (73.6000,39.4000) -- (74.0000,39.3000) -- (74.4000,39.2000) -- (74.9000,39.0000) -- (75.3000,38.9000) -- (75.7000,38.9000) -- (76.1000,38.8000) -- (76.6000,38.8000) -- (77.0000,38.9000) -- (77.4000,38.9000) -- (77.8000,39.0000) -- (78.2000,39.0000) -- (78.7000,39.1000) -- (79.1000,39.2000) -- (79.5000,39.3000) -- (79.9000,39.3000) -- (80.4000,39.4000) -- (80.8000,39.5000) -- (81.2000,39.6000) -- (81.6000,39.7000) -- (82.1000,39.8000) -- (82.5000,39.9000) -- (82.9000,40.0000) -- (83.3000,40.1000) -- (83.7000,40.1000) -- (84.2000,40.2000) -- (84.6000,40.3000) -- (85.0000,40.4000) -- (85.4000,40.5000) -- (85.9000,40.7000) -- (86.3000,40.8000) -- (86.7000,40.9000) -- (87.1000,41.0000) -- (87.6000,41.1000) -- (88.0000,41.1000) -- (88.4000,41.3000) -- (88.8000,41.4000) -- (89.2000,41.5000) -- (89.7000,41.6000) -- (90.1000,41.6000) -- (90.5000,41.7000) -- (90.9000,41.7000) -- (91.4000,41.7000) -- (91.8000,41.7000) -- (92.2000,41.7000) -- (92.6000,41.7000) -- (93.0000,41.6000) -- (93.5000,41.6000) -- (93.9000,41.5000) -- (94.3000,41.5000) -- (94.7000,41.4000) -- (95.2000,41.3000) -- (95.6000,41.2000) -- (96.0000,41.1000) -- (96.4000,41.0000) -- (96.9000,41.0000) -- (97.3000,40.9000) -- (97.7000,40.8000) -- (98.1000,40.7000) -- (98.5000,40.6000) -- (99.0000,40.5000) -- (99.4000,40.4000) -- (99.8000,40.3000) -- (100.2000,40.1000) -- (100.7000,40.1000) -- (101.1000,40.0000) -- (101.5000,39.9000) -- (101.9000,39.7000) -- (102.3000,39.6000) -- (102.8000,39.5000) -- (103.2000,39.4000) -- (103.6000,39.3000) -- (104.0000,39.2000) -- (104.5000,39.2000) -- (104.9000,39.1000) -- (105.3000,39.1000) -- (105.7000,39.1000) -- (106.2000,39.1000) -- (106.6000,39.1000) -- (107.0000,39.1000) -- (107.4000,39.2000) -- (107.8000,39.2000) -- (108.3000,39.3000) -- (108.7000,39.4000) -- (109.1000,39.5000) -- (109.5000,39.7000) -- (110.0000,39.8000) -- (110.4000,39.9000) -- (110.8000,40.0000) -- (111.2000,40.1000) -- (111.7000,40.3000) -- (112.1000,40.4000) -- (112.5000,40.6000) -- (112.9000,40.7000) -- (113.3000,40.8000) -- (113.8000,40.9000) -- (114.2000,41.0000) -- (114.6000,41.1000) -- (115.0000,41.2000) -- (115.5000,41.3000) -- (115.9000,41.4000) -- (116.3000,41.5000) -- (116.7000,41.6000) -- (117.2000,41.6000) -- (117.6000,41.7000) -- (118.0000,41.7000) -- (118.4000,41.8000) -- (118.8000,41.8000) -- (119.3000,41.8000) -- (119.7000,41.8000) -- (120.1000,41.7000) -- (120.5000,41.7000) -- (121.0000,41.6000) -- (121.4000,41.6000) -- (121.8000,41.5000) -- (122.2000,41.3000) -- (122.6000,41.2000) -- (123.1000,41.1000) -- (123.5000,41.0000) -- (123.9000,40.9000) -- (124.3000,40.8000) -- (124.8000,40.7000) -- (125.2000,40.5000) -- (125.6000,40.4000) -- (126.0000,40.2000) -- (126.5000,40.1000) -- (126.9000,39.9000) -- (127.3000,39.8000) -- (127.7000,39.8000) -- (128.1000,39.7000) -- (128.6000,39.5000) -- (129.0000,39.4000) -- (129.4000,39.4000) -- (129.8000,39.3000) -- (130.3000,39.3000) -- (130.7000,39.2000) -- (131.1000,39.2000) -- (131.5000,39.2000) -- (131.9000,39.2000) -- (132.4000,39.2000) -- (132.8000,39.2000) -- (133.2000,39.2000) -- (133.6000,39.2000) -- (134.1000,39.3000) -- (134.5000,39.3000) -- (134.9000,39.4000) -- (135.3000,39.5000) -- (135.8000,39.6000) -- (136.2000,39.7000) -- (136.6000,39.8000) -- (137.0000,39.9000) -- (137.4000,40.0000) -- (137.9000,40.1000) -- (138.3000,40.3000) -- (138.7000,40.4000) -- (139.1000,40.6000) -- (139.6000,40.7000) -- (140.0000,40.8000) -- (140.4000,40.9000) -- (140.8000,41.0000) -- (141.3000,41.1000) -- (141.7000,41.2000) -- (142.1000,41.3000) -- (142.3000,41.4000);



  \end{scope}
  \begin{scope}[scale=1.010,draw=black,line join=bevel,line cap=rect,line width=0.800pt]
  \end{scope}
  \begin{scope}[scale=1.010,draw=black,line join=bevel,line cap=rect,line width=0.800pt]
  \end{scope}
  \begin{scope}[cm={{1.01,0.0,0.0,1.01,(6.3158,0.0)}},draw=black,line join=round,line cap=round,line width=0.480pt]
    \path[draw] (25.5000,32.5000) -- (25.5000,56.5000) -- (142.5000,56.5000) -- (142.5000,32.5000) -- (25.5000,32.5000);



  \end{scope}
  \begin{scope}[cm={{1.01,0.0,0.0,1.01,(6.3158,0.0)}},draw=ca0a0a4,dash pattern=on 0.40pt off 0.80pt,line join=round,line cap=round,line width=0.400pt]
    \path[draw] (25.5000,70.5000) -- (108.5000,70.5000);



    \path[draw] (137.5000,70.5000) -- (142.5000,70.5000);



  \end{scope}
  \begin{scope}[cm={{1.01,0.0,0.0,1.01,(6.3158,0.0)}},draw=black,line join=round,line cap=round,line width=0.480pt]
    \path[draw] (25.5000,70.5000) -- (28.5000,70.5000);



    \path[draw] (142.5000,70.5000) -- (139.5000,70.5000);



  \end{scope}
  \begin{scope}[scale=1.010,draw=black,line join=bevel,line cap=rect,line width=0.800pt]
  \end{scope}
  \begin{scope}[cm={{1.01,0.0,0.0,1.01,(5.05,74.74)}},draw=black,line join=bevel,line cap=rect,line width=0.800pt]
  \end{scope}
  \begin{scope}[cm={{1.01,0.0,0.0,1.01,(5.05,74.74)}},draw=black,line join=bevel,line cap=rect,line width=0.800pt]
  \end{scope}
  \begin{scope}[cm={{1.01,0.0,0.0,1.01,(5.05,74.74)}},draw=black,line join=bevel,line cap=rect,line width=0.800pt]
  \end{scope}
  \begin{scope}[cm={{1.01,0.0,0.0,1.01,(5.05,74.74)}},draw=black,line join=bevel,line cap=rect,line width=0.800pt]
  \end{scope}
  \begin{scope}[cm={{1.01,0.0,0.0,1.01,(5.05,74.74)}},draw=black,line join=bevel,line cap=rect,line width=0.800pt]
  \end{scope}
  \begin{scope}[cm={{1.01,0.0,0.0,1.01,(11.3658,74.74)}},draw=black,line join=bevel,line cap=rect,line width=0.800pt]
    \path[fill=black] (0.0000,0.0000) node[above right] (text464) {-10};



  \end{scope}
  \begin{scope}[cm={{1.01,0.0,0.0,1.01,(5.05,74.74)}},draw=black,line join=bevel,line cap=rect,line width=0.800pt]
  \end{scope}
  \begin{scope}[scale=1.010,draw=black,line join=bevel,line cap=rect,line width=0.800pt]
  \end{scope}
  \begin{scope}[cm={{1.01,0.0,0.0,1.01,(6.3158,0.0)}},draw=ca0a0a4,dash pattern=on 0.40pt off 0.80pt,line join=round,line cap=round,line width=0.400pt]
    \path[draw] (25.5000,58.5000) -- (142.5000,58.5000);



  \end{scope}
  \begin{scope}[cm={{1.01,0.0,0.0,1.01,(6.3158,0.0)}},draw=black,line join=round,line cap=round,line width=0.480pt]
    \path[draw] (25.5000,58.5000) -- (28.5000,58.5000);



    \path[draw] (142.5000,58.5000) -- (139.5000,58.5000);



  \end{scope}
  \begin{scope}[scale=1.010,draw=black,line join=bevel,line cap=rect,line width=0.800pt]
  \end{scope}
  \begin{scope}[cm={{1.01,0.0,0.0,1.01,(9.09,63.63)}},draw=black,line join=bevel,line cap=rect,line width=0.800pt]
  \end{scope}
  \begin{scope}[cm={{1.01,0.0,0.0,1.01,(9.09,63.63)}},draw=black,line join=bevel,line cap=rect,line width=0.800pt]
  \end{scope}
  \begin{scope}[cm={{1.01,0.0,0.0,1.01,(9.09,63.63)}},draw=black,line join=bevel,line cap=rect,line width=0.800pt]
  \end{scope}
  \begin{scope}[cm={{1.01,0.0,0.0,1.01,(9.09,63.63)}},draw=black,line join=bevel,line cap=rect,line width=0.800pt]
  \end{scope}
  \begin{scope}[cm={{1.01,0.0,0.0,1.01,(9.09,63.63)}},draw=black,line join=bevel,line cap=rect,line width=0.800pt]
  \end{scope}
  \begin{scope}[cm={{1.01,0.0,0.0,1.01,(15.4058,63.63)}},draw=black,line join=bevel,line cap=rect,line width=0.800pt]
    \path[fill=black] (0.0000,0.0000) node[above right] (text494) {10};



  \end{scope}
  \begin{scope}[cm={{1.01,0.0,0.0,1.01,(9.09,63.63)}},draw=black,line join=bevel,line cap=rect,line width=0.800pt]
  \end{scope}
  \begin{scope}[scale=1.010,draw=black,line join=bevel,line cap=rect,line width=0.800pt]
  \end{scope}
  \begin{scope}[cm={{1.01,0.0,0.0,1.01,(6.3158,0.0)}},draw=ca0a0a4,dash pattern=on 0.40pt off 0.80pt,line join=round,line cap=round,line width=0.400pt]
    \path[draw] (25.5000,80.5000) -- (25.5000,56.5000);



  \end{scope}
  \begin{scope}[cm={{1.01,0.0,0.0,1.01,(6.3158,0.0)}},draw=black,line join=round,line cap=round,line width=0.480pt]
    \path[draw] (25.5000,80.5000) -- (25.5000,75.5000);



    \path[draw] (25.5000,56.5000) -- (25.5000,60.5000);



  \end{scope}
  \begin{scope}[scale=1.010,draw=black,line join=bevel,line cap=rect,line width=0.800pt]
  \end{scope}
  \begin{scope}[cm={{1.01,0.0,0.0,1.01,(26.26,96.96)}},draw=black,line join=bevel,line cap=rect,line width=0.800pt]
  \end{scope}
  \begin{scope}[cm={{1.01,0.0,0.0,1.01,(26.26,96.96)}},draw=black,line join=bevel,line cap=rect,line width=0.800pt]
  \end{scope}
  \begin{scope}[cm={{1.01,0.0,0.0,1.01,(26.26,96.96)}},draw=black,line join=bevel,line cap=rect,line width=0.800pt]
  \end{scope}
  \begin{scope}[cm={{1.01,0.0,0.0,1.01,(26.26,96.96)}},draw=black,line join=bevel,line cap=rect,line width=0.800pt]
  \end{scope}
  \begin{scope}[cm={{1.01,0.0,0.0,1.01,(26.26,96.96)}},draw=black,line join=bevel,line cap=rect,line width=0.800pt]
  \end{scope}
  \begin{scope}[cm={{1.01,0.0,0.0,1.01,(26.26,96.96)}},draw=black,line join=bevel,line cap=rect,line width=0.800pt]
  \end{scope}
  \begin{scope}[scale=1.010,draw=black,line join=bevel,line cap=rect,line width=0.800pt]
  \end{scope}
  \begin{scope}[cm={{1.01,0.0,0.0,1.01,(6.3158,0.0)}},draw=ca0a0a4,dash pattern=on 0.40pt off 0.80pt,line join=round,line cap=round,line width=0.400pt]
    \path[draw] (60.5000,80.5000) -- (60.5000,56.5000);



  \end{scope}
  \begin{scope}[cm={{1.01,0.0,0.0,1.01,(6.3158,0.0)}},draw=black,line join=round,line cap=round,line width=0.480pt]
    \path[draw] (60.5000,80.5000) -- (60.5000,75.5000);



    \path[draw] (60.5000,56.5000) -- (60.5000,60.5000);



  \end{scope}
  \begin{scope}[scale=1.010,draw=black,line join=bevel,line cap=rect,line width=0.800pt]
  \end{scope}
  \begin{scope}[cm={{1.01,0.0,0.0,1.01,(61.61,96.96)}},draw=black,line join=bevel,line cap=rect,line width=0.800pt]
  \end{scope}
  \begin{scope}[cm={{1.01,0.0,0.0,1.01,(61.61,96.96)}},draw=black,line join=bevel,line cap=rect,line width=0.800pt]
  \end{scope}
  \begin{scope}[cm={{1.01,0.0,0.0,1.01,(61.61,96.96)}},draw=black,line join=bevel,line cap=rect,line width=0.800pt]
  \end{scope}
  \begin{scope}[cm={{1.01,0.0,0.0,1.01,(61.61,96.96)}},draw=black,line join=bevel,line cap=rect,line width=0.800pt]
  \end{scope}
  \begin{scope}[cm={{1.01,0.0,0.0,1.01,(61.61,96.96)}},draw=black,line join=bevel,line cap=rect,line width=0.800pt]
  \end{scope}
  \begin{scope}[cm={{1.01,0.0,0.0,1.01,(61.61,96.96)}},draw=black,line join=bevel,line cap=rect,line width=0.800pt]
  \end{scope}
  \begin{scope}[scale=1.010,draw=black,line join=bevel,line cap=rect,line width=0.800pt]
  \end{scope}
  \begin{scope}[cm={{1.01,0.0,0.0,1.01,(6.3158,0.0)}},draw=ca0a0a4,dash pattern=on 0.40pt off 0.80pt,line join=round,line cap=round,line width=0.400pt]
    \path[draw] (95.5000,80.5000) -- (95.5000,56.5000);



  \end{scope}
  \begin{scope}[cm={{1.01,0.0,0.0,1.01,(6.3158,0.0)}},draw=black,line join=round,line cap=round,line width=0.480pt]
    \path[draw] (95.5000,80.5000) -- (95.5000,75.5000);



    \path[draw] (95.5000,56.5000) -- (95.5000,60.5000);



  \end{scope}
  \begin{scope}[scale=1.010,draw=black,line join=bevel,line cap=rect,line width=0.800pt]
  \end{scope}
  \begin{scope}[cm={{1.01,0.0,0.0,1.01,(96.96,96.96)}},draw=black,line join=bevel,line cap=rect,line width=0.800pt]
  \end{scope}
  \begin{scope}[cm={{1.01,0.0,0.0,1.01,(96.96,96.96)}},draw=black,line join=bevel,line cap=rect,line width=0.800pt]
  \end{scope}
  \begin{scope}[cm={{1.01,0.0,0.0,1.01,(96.96,96.96)}},draw=black,line join=bevel,line cap=rect,line width=0.800pt]
  \end{scope}
  \begin{scope}[cm={{1.01,0.0,0.0,1.01,(96.96,96.96)}},draw=black,line join=bevel,line cap=rect,line width=0.800pt]
  \end{scope}
  \begin{scope}[cm={{1.01,0.0,0.0,1.01,(96.96,96.96)}},draw=black,line join=bevel,line cap=rect,line width=0.800pt]
  \end{scope}
  \begin{scope}[cm={{1.01,0.0,0.0,1.01,(96.96,96.96)}},draw=black,line join=bevel,line cap=rect,line width=0.800pt]
  \end{scope}
  \begin{scope}[scale=1.010,draw=black,line join=bevel,line cap=rect,line width=0.800pt]
  \end{scope}
  \begin{scope}[cm={{1.01,0.0,0.0,1.01,(6.3158,0.0)}},draw=ca0a0a4,dash pattern=on 0.40pt off 0.80pt,line join=round,line cap=round,line width=0.400pt]
    \path[draw] (130.5000,68.5000) -- (130.5000,56.5000);



  \end{scope}
  \begin{scope}[cm={{1.01,0.0,0.0,1.01,(6.3158,0.0)}},draw=black,line join=round,line cap=round,line width=0.480pt]
    \path[draw] (130.5000,80.5000) -- (130.5000,75.5000);



    \path[draw] (130.5000,56.5000) -- (130.5000,60.5000);



  \end{scope}
  \begin{scope}[scale=1.010,draw=black,line join=bevel,line cap=rect,line width=0.800pt]
  \end{scope}
  \begin{scope}[cm={{1.01,0.0,0.0,1.01,(132.31,96.96)}},draw=black,line join=bevel,line cap=rect,line width=0.800pt]
  \end{scope}
  \begin{scope}[cm={{1.01,0.0,0.0,1.01,(132.31,96.96)}},draw=black,line join=bevel,line cap=rect,line width=0.800pt]
  \end{scope}
  \begin{scope}[cm={{1.01,0.0,0.0,1.01,(132.31,96.96)}},draw=black,line join=bevel,line cap=rect,line width=0.800pt]
  \end{scope}
  \begin{scope}[cm={{1.01,0.0,0.0,1.01,(132.31,96.96)}},draw=black,line join=bevel,line cap=rect,line width=0.800pt]
  \end{scope}
  \begin{scope}[cm={{1.01,0.0,0.0,1.01,(132.31,96.96)}},draw=black,line join=bevel,line cap=rect,line width=0.800pt]
  \end{scope}
  \begin{scope}[cm={{1.01,0.0,0.0,1.01,(132.31,96.96)}},draw=black,line join=bevel,line cap=rect,line width=0.800pt]
  \end{scope}
  \begin{scope}[scale=1.010,draw=black,line join=bevel,line cap=rect,line width=0.800pt]
  \end{scope}
  \begin{scope}[cm={{1.01,0.0,0.0,1.01,(6.3158,0.0)}},draw=black,line join=round,line cap=round,line width=0.480pt]
    \path[draw] (25.5000,56.5000) -- (25.5000,80.5000) -- (142.5000,80.5000) -- (142.5000,56.5000) -- (25.5000,56.5000);



  \end{scope}
  \begin{scope}[scale=1.010,draw=black,line join=bevel,line cap=rect,line width=0.800pt]
  \end{scope}
  \begin{scope}[cm={{1.01,0.0,0.0,1.01,(109.08,80.8)}},draw=black,line join=bevel,line cap=rect,line width=0.800pt]
  \end{scope}
  \begin{scope}[cm={{1.01,0.0,0.0,1.01,(109.08,80.8)}},draw=black,line join=bevel,line cap=rect,line width=0.800pt]
  \end{scope}
  \begin{scope}[cm={{1.01,0.0,0.0,1.01,(109.08,80.8)}},draw=black,line join=bevel,line cap=rect,line width=0.800pt]
  \end{scope}
  \begin{scope}[cm={{1.01,0.0,0.0,1.01,(109.08,80.8)}},draw=black,line join=bevel,line cap=rect,line width=0.800pt]
  \end{scope}
  \begin{scope}[cm={{1.01,0.0,0.0,1.01,(109.08,80.8)}},draw=black,line join=bevel,line cap=rect,line width=0.800pt]
  \end{scope}
  \begin{scope}[cm={{1.01,0.0,0.0,1.01,(119.30192,79.17248)}},draw=black,line join=bevel,line cap=rect,line width=0.800pt]
    \path[fill=black] (0.0000,0.0000) node[above right] (text622) {\scriptsize $\beta_1$};



  \end{scope}
  \begin{scope}[cm={{1.01,0.0,0.0,1.01,(109.08,80.8)}},draw=black,line join=bevel,line cap=rect,line width=0.800pt]
  \end{scope}
  \begin{scope}[scale=1.010,draw=black,line join=bevel,line cap=rect,line width=0.800pt]
  \end{scope}
  \begin{scope}[cm={{1.01,0.0,0.0,1.01,(6.3158,0.0)}},draw=black,line join=round,line cap=round,line width=0.480pt]
    \path[draw,even odd rule] (123.5000,76.5000) -- (132.5000,76.5000);



  \end{scope}
  \begin{scope}[scale=1.010,draw=black,line join=bevel,line cap=rect,line width=0.800pt]
  \end{scope}
  \begin{scope}[scale=1.010,draw=black,line join=bevel,line cap=rect,line width=0.800pt]
  \end{scope}
  \begin{scope}[scale=1.010,draw=black,line join=bevel,line cap=rect,line width=0.800pt]
  \end{scope}
  \begin{scope}[scale=1.010,draw=black,line join=bevel,line cap=rect,line width=0.800pt]
  \end{scope}
  \begin{scope}[cm={{1.01,0.0,0.0,1.01,(6.3158,0.0)}},draw=black,line join=round,line cap=round,line width=0.480pt]
    \path[draw] (25.8000,67.9000) -- (25.8000,67.9000) -- (26.2000,64.4000) -- (26.7000,63.4000) -- (27.1000,65.9000) -- (27.5000,64.8000) -- (27.9000,64.0000) -- (28.3000,65.9000) -- (28.8000,65.0000) -- (29.2000,61.7000) -- (29.6000,64.2000) -- (30.0000,65.6000) -- (30.5000,63.7000) -- (30.9000,62.9000) -- (31.3000,64.5000) -- (31.7000,65.9000) -- (32.2000,65.6000) -- (32.6000,64.1000) -- (33.0000,62.9000) -- (33.4000,63.0000) -- (33.8000,64.1000) -- (34.3000,65.3000) -- (34.7000,66.0000) -- (35.1000,65.8000) -- (35.5000,65.1000) -- (36.0000,64.3000) -- (36.4000,63.6000) -- (36.8000,63.4000) -- (37.2000,63.6000) -- (37.6000,64.0000) -- (38.1000,64.5000) -- (38.5000,64.8000) -- (38.9000,64.8000) -- (39.3000,64.6000) -- (39.8000,64.1000) -- (40.2000,63.5000) -- (40.6000,63.0000) -- (41.0000,62.5000) -- (41.5000,62.2000) -- (41.9000,62.1000) -- (42.3000,62.3000) -- (42.7000,62.7000) -- (43.1000,63.2000) -- (43.6000,63.7000) -- (44.0000,64.1000) -- (44.4000,64.6000) -- (44.8000,64.9000) -- (45.3000,65.1000) -- (45.7000,65.3000) -- (46.1000,65.3000) -- (46.5000,65.3000) -- (47.0000,65.2000) -- (47.4000,65.1000) -- (47.8000,64.9000) -- (48.2000,64.8000) -- (48.6000,64.7000) -- (49.1000,64.8000) -- (49.5000,64.8000) -- (49.9000,65.0000) -- (50.3000,65.2000) -- (50.8000,65.4000) -- (51.2000,65.6000) -- (51.6000,65.8000) -- (52.0000,66.0000) -- (52.4000,66.1000) -- (52.9000,66.2000) -- (53.3000,66.1000) -- (53.7000,66.0000) -- (54.1000,65.9000) -- (54.6000,65.7000) -- (55.0000,65.4000) -- (55.4000,65.2000) -- (55.8000,64.9000) -- (56.3000,64.7000) -- (56.7000,64.5000) -- (57.1000,64.4000) -- (57.5000,64.3000) -- (57.9000,64.4000) -- (58.4000,64.5000) -- (58.8000,64.6000) -- (59.2000,64.8000) -- (59.6000,64.9000) -- (60.1000,65.0000) -- (60.5000,65.1000) -- (60.9000,65.2000) -- (61.3000,65.2000) -- (61.8000,65.2000) -- (62.2000,65.1000) -- (62.6000,65.0000) -- (63.0000,64.8000) -- (63.4000,64.6000) -- (63.9000,64.3000) -- (64.3000,64.1000) -- (64.7000,63.9000) -- (65.1000,63.8000) -- (65.6000,63.7000) -- (66.0000,63.6000) -- (66.4000,63.5000) -- (66.8000,63.4000) -- (67.3000,63.4000) -- (67.7000,63.4000) -- (68.1000,63.4000) -- (68.5000,63.3000) -- (68.9000,63.3000) -- (69.4000,63.3000) -- (69.8000,63.3000) -- (70.2000,63.2000) -- (70.6000,63.2000) -- (71.1000,63.1000) -- (71.5000,63.1000) -- (71.9000,63.0000) -- (72.3000,63.0000) -- (72.7000,63.0000) -- (73.2000,63.1000) -- (73.6000,63.2000) -- (74.0000,63.3000) -- (74.4000,63.4000) -- (74.9000,63.5000) -- (75.3000,63.6000) -- (75.7000,63.8000) -- (76.1000,64.0000) -- (76.6000,64.2000) -- (77.0000,64.4000) -- (77.4000,64.6000) -- (77.8000,64.8000) -- (78.2000,64.9000) -- (78.7000,65.1000) -- (79.1000,65.2000) -- (79.5000,65.3000) -- (79.9000,65.4000) -- (80.4000,65.5000) -- (80.8000,65.6000) -- (81.2000,65.6000) -- (81.6000,65.7000) -- (82.1000,65.7000) -- (82.5000,65.8000) -- (82.9000,65.8000) -- (83.3000,65.8000) -- (83.7000,65.8000) -- (84.2000,65.7000) -- (84.6000,65.7000) -- (85.0000,65.7000) -- (85.4000,65.7000) -- (85.9000,65.7000) -- (86.3000,65.6000) -- (86.7000,65.6000) -- (87.1000,65.5000) -- (87.6000,65.5000) -- (88.0000,65.4000) -- (88.4000,65.3000) -- (88.8000,65.3000) -- (89.2000,65.2000) -- (89.7000,65.1000) -- (90.1000,65.0000) -- (90.5000,64.9000) -- (90.9000,64.7000) -- (91.4000,64.6000) -- (91.8000,64.5000) -- (92.2000,64.4000) -- (92.6000,64.2000) -- (93.0000,64.1000) -- (93.5000,64.0000) -- (93.9000,63.8000) -- (94.3000,63.7000) -- (94.7000,63.6000) -- (95.2000,63.5000) -- (95.6000,63.4000) -- (96.0000,63.4000) -- (96.4000,63.4000) -- (96.9000,63.4000) -- (97.3000,63.4000) -- (97.7000,63.3000) -- (98.1000,63.3000) -- (98.5000,63.3000) -- (99.0000,63.3000) -- (99.4000,63.3000) -- (99.8000,63.3000) -- (100.2000,63.3000) -- (100.7000,63.3000) -- (101.1000,63.4000) -- (101.5000,63.4000) -- (101.9000,63.4000) -- (102.3000,63.5000) -- (102.8000,63.6000) -- (103.2000,63.7000) -- (103.6000,63.8000) -- (104.0000,63.9000) -- (104.5000,64.0000) -- (104.9000,64.1000) -- (105.3000,64.3000) -- (105.7000,64.4000) -- (106.2000,64.5000) -- (106.6000,64.6000) -- (107.0000,64.8000) -- (107.4000,64.9000) -- (107.8000,65.0000) -- (108.3000,65.2000) -- (108.7000,65.3000) -- (109.1000,65.4000) -- (109.5000,65.5000) -- (110.0000,65.6000) -- (110.4000,65.7000) -- (110.8000,65.7000) -- (111.2000,65.7000) -- (111.7000,65.8000) -- (112.1000,65.8000) -- (112.5000,65.8000) -- (112.9000,65.8000) -- (113.3000,65.7000) -- (113.8000,65.7000) -- (114.2000,65.6000) -- (114.6000,65.5000) -- (115.0000,65.4000) -- (115.5000,65.4000) -- (115.9000,65.3000) -- (116.3000,65.2000) -- (116.7000,65.1000) -- (117.2000,65.0000) -- (117.6000,64.9000) -- (118.0000,64.7000) -- (118.4000,64.6000) -- (118.8000,64.5000) -- (119.3000,64.4000) -- (119.7000,64.2000) -- (120.1000,64.1000) -- (120.5000,64.0000) -- (121.0000,63.8000) -- (121.4000,63.7000) -- (121.8000,63.6000) -- (122.2000,63.4000) -- (122.6000,63.3000) -- (123.1000,63.3000) -- (123.5000,63.2000) -- (123.9000,63.2000) -- (124.3000,63.2000) -- (124.8000,63.1000) -- (125.2000,63.1000) -- (125.6000,63.1000) -- (126.0000,63.1000) -- (126.5000,63.1000) -- (126.9000,63.1000) -- (127.3000,63.2000) -- (127.7000,63.4000) -- (128.1000,63.4000) -- (128.6000,63.5000) -- (129.0000,63.5000) -- (129.4000,63.6000) -- (129.8000,63.8000) -- (130.3000,63.9000) -- (130.7000,64.1000) -- (131.1000,64.2000) -- (131.5000,64.3000) -- (131.9000,64.4000) -- (132.4000,64.6000) -- (132.8000,64.7000) -- (133.2000,64.8000) -- (133.6000,64.9000) -- (134.1000,65.0000) -- (134.5000,65.1000) -- (134.9000,65.2000) -- (135.3000,65.3000) -- (135.8000,65.4000) -- (136.2000,65.5000) -- (136.6000,65.6000) -- (137.0000,65.6000) -- (137.4000,65.6000) -- (137.9000,65.7000) -- (138.3000,65.7000) -- (138.7000,65.8000) -- (139.1000,65.8000) -- (139.6000,65.8000) -- (140.0000,65.7000) -- (140.4000,65.7000) -- (140.8000,65.6000) -- (141.3000,65.5000) -- (141.7000,65.4000) -- (142.1000,65.4000) -- (142.3000,65.3000);



  \end{scope}
  \begin{scope}[scale=1.010,draw=black,line join=bevel,line cap=rect,line width=0.800pt]
  \end{scope}
  \begin{scope}[scale=1.010,draw=black,line join=bevel,line cap=rect,line width=0.800pt]
  \end{scope}
  \begin{scope}[cm={{1.01,0.0,0.0,1.01,(6.3158,0.0)}},draw=black,line join=round,line cap=round,line width=0.480pt]
    \path[draw] (25.5000,56.5000) -- (25.5000,80.5000) -- (142.5000,80.5000) -- (142.5000,56.5000) -- (25.5000,56.5000);



  \end{scope}
  \begin{scope}[cm={{1.01,0.0,0.0,1.01,(6.3158,0.0)}},draw=ca0a0a4,dash pattern=on 0.40pt off 0.80pt,line join=round,line cap=round,line width=0.400pt]
    \path[draw] (25.5000,94.5000) -- (108.5000,94.5000);



    \path[draw] (137.5000,94.5000) -- (142.5000,94.5000);



  \end{scope}
  \begin{scope}[cm={{1.01,0.0,0.0,1.01,(6.3158,0.0)}},draw=black,line join=round,line cap=round,line width=0.480pt]
    \path[draw] (25.5000,94.5000) -- (28.5000,94.5000);



    \path[draw] (142.5000,94.5000) -- (139.5000,94.5000);



  \end{scope}
  \begin{scope}[scale=1.010,draw=black,line join=bevel,line cap=rect,line width=0.800pt]
  \end{scope}
  \begin{scope}[cm={{1.01,0.0,0.0,1.01,(5.05,98.98)}},draw=black,line join=bevel,line cap=rect,line width=0.800pt]
  \end{scope}
  \begin{scope}[cm={{1.01,0.0,0.0,1.01,(5.05,98.98)}},draw=black,line join=bevel,line cap=rect,line width=0.800pt]
  \end{scope}
  \begin{scope}[cm={{1.01,0.0,0.0,1.01,(5.05,98.98)}},draw=black,line join=bevel,line cap=rect,line width=0.800pt]
  \end{scope}
  \begin{scope}[cm={{1.01,0.0,0.0,1.01,(5.05,98.98)}},draw=black,line join=bevel,line cap=rect,line width=0.800pt]
  \end{scope}
  \begin{scope}[cm={{1.01,0.0,0.0,1.01,(5.05,98.98)}},draw=black,line join=bevel,line cap=rect,line width=0.800pt]
  \end{scope}
  \begin{scope}[cm={{1.01,0.0,0.0,1.01,(11.3658,98.98)}},draw=black,line join=bevel,line cap=rect,line width=0.800pt]
    \path[fill=black] (0.0000,0.0000) node[above right] (text678) {-10};



  \end{scope}
  \begin{scope}[cm={{1.01,0.0,0.0,1.01,(5.05,98.98)}},draw=black,line join=bevel,line cap=rect,line width=0.800pt]
  \end{scope}
  \begin{scope}[scale=1.010,draw=black,line join=bevel,line cap=rect,line width=0.800pt]
  \end{scope}
  \begin{scope}[cm={{1.01,0.0,0.0,1.01,(6.3158,0.0)}},draw=ca0a0a4,dash pattern=on 0.40pt off 0.80pt,line join=round,line cap=round,line width=0.400pt]
    \path[draw] (25.5000,82.5000) -- (142.5000,82.5000);



  \end{scope}
  \begin{scope}[cm={{1.01,0.0,0.0,1.01,(6.3158,0.0)}},draw=black,line join=round,line cap=round,line width=0.480pt]
    \path[draw] (25.5000,82.5000) -- (28.5000,82.5000);



    \path[draw] (142.5000,82.5000) -- (139.5000,82.5000);



  \end{scope}
  \begin{scope}[scale=1.010,draw=black,line join=bevel,line cap=rect,line width=0.800pt]
  \end{scope}
  \begin{scope}[cm={{1.01,0.0,0.0,1.01,(9.09,87.87)}},draw=black,line join=bevel,line cap=rect,line width=0.800pt]
  \end{scope}
  \begin{scope}[cm={{1.01,0.0,0.0,1.01,(9.09,87.87)}},draw=black,line join=bevel,line cap=rect,line width=0.800pt]
  \end{scope}
  \begin{scope}[cm={{1.01,0.0,0.0,1.01,(9.09,87.87)}},draw=black,line join=bevel,line cap=rect,line width=0.800pt]
  \end{scope}
  \begin{scope}[cm={{1.01,0.0,0.0,1.01,(9.09,87.87)}},draw=black,line join=bevel,line cap=rect,line width=0.800pt]
  \end{scope}
  \begin{scope}[cm={{1.01,0.0,0.0,1.01,(9.09,87.87)}},draw=black,line join=bevel,line cap=rect,line width=0.800pt]
  \end{scope}
  \begin{scope}[cm={{1.01,0.0,0.0,1.01,(15.4058,87.87)}},draw=black,line join=bevel,line cap=rect,line width=0.800pt]
    \path[fill=black] (0.0000,0.0000) node[above right] (text708) {10};



  \end{scope}
  \begin{scope}[cm={{1.01,0.0,0.0,1.01,(9.09,87.87)}},draw=black,line join=bevel,line cap=rect,line width=0.800pt]
  \end{scope}
  \begin{scope}[scale=1.010,draw=black,line join=bevel,line cap=rect,line width=0.800pt]
  \end{scope}
  \begin{scope}[cm={{1.01,0.0,0.0,1.01,(6.3158,0.0)}},draw=ca0a0a4,dash pattern=on 0.40pt off 0.80pt,line join=round,line cap=round,line width=0.400pt]
    \path[draw] (25.5000,104.5000) -- (25.5000,80.5000);



  \end{scope}
  \begin{scope}[cm={{1.01,0.0,0.0,1.01,(6.3158,0.0)}},draw=black,line join=round,line cap=round,line width=0.480pt]
    \path[draw] (25.5000,104.5000) -- (25.5000,99.5000);



    \path[draw] (25.5000,80.5000) -- (25.5000,84.5000);



  \end{scope}
  \begin{scope}[scale=1.010,draw=black,line join=bevel,line cap=rect,line width=0.800pt]
  \end{scope}
  \begin{scope}[cm={{1.01,0.0,0.0,1.01,(26.26,121.2)}},draw=black,line join=bevel,line cap=rect,line width=0.800pt]
  \end{scope}
  \begin{scope}[cm={{1.01,0.0,0.0,1.01,(26.26,121.2)}},draw=black,line join=bevel,line cap=rect,line width=0.800pt]
  \end{scope}
  \begin{scope}[cm={{1.01,0.0,0.0,1.01,(26.26,121.2)}},draw=black,line join=bevel,line cap=rect,line width=0.800pt]
  \end{scope}
  \begin{scope}[cm={{1.01,0.0,0.0,1.01,(26.26,121.2)}},draw=black,line join=bevel,line cap=rect,line width=0.800pt]
  \end{scope}
  \begin{scope}[cm={{1.01,0.0,0.0,1.01,(26.26,121.2)}},draw=black,line join=bevel,line cap=rect,line width=0.800pt]
  \end{scope}
  \begin{scope}[cm={{1.01,0.0,0.0,1.01,(26.26,121.2)}},draw=black,line join=bevel,line cap=rect,line width=0.800pt]
  \end{scope}
  \begin{scope}[scale=1.010,draw=black,line join=bevel,line cap=rect,line width=0.800pt]
  \end{scope}
  \begin{scope}[cm={{1.01,0.0,0.0,1.01,(6.3158,0.0)}},draw=ca0a0a4,dash pattern=on 0.40pt off 0.80pt,line join=round,line cap=round,line width=0.400pt]
    \path[draw] (60.5000,104.5000) -- (60.5000,80.5000);



  \end{scope}
  \begin{scope}[cm={{1.01,0.0,0.0,1.01,(6.3158,0.0)}},draw=black,line join=round,line cap=round,line width=0.480pt]
    \path[draw] (60.5000,104.5000) -- (60.5000,99.5000);



    \path[draw] (60.5000,80.5000) -- (60.5000,84.5000);



  \end{scope}
  \begin{scope}[scale=1.010,draw=black,line join=bevel,line cap=rect,line width=0.800pt]
  \end{scope}
  \begin{scope}[cm={{1.01,0.0,0.0,1.01,(61.61,121.2)}},draw=black,line join=bevel,line cap=rect,line width=0.800pt]
  \end{scope}
  \begin{scope}[cm={{1.01,0.0,0.0,1.01,(61.61,121.2)}},draw=black,line join=bevel,line cap=rect,line width=0.800pt]
  \end{scope}
  \begin{scope}[cm={{1.01,0.0,0.0,1.01,(61.61,121.2)}},draw=black,line join=bevel,line cap=rect,line width=0.800pt]
  \end{scope}
  \begin{scope}[cm={{1.01,0.0,0.0,1.01,(61.61,121.2)}},draw=black,line join=bevel,line cap=rect,line width=0.800pt]
  \end{scope}
  \begin{scope}[cm={{1.01,0.0,0.0,1.01,(61.61,121.2)}},draw=black,line join=bevel,line cap=rect,line width=0.800pt]
  \end{scope}
  \begin{scope}[cm={{1.01,0.0,0.0,1.01,(61.61,121.2)}},draw=black,line join=bevel,line cap=rect,line width=0.800pt]
  \end{scope}
  \begin{scope}[scale=1.010,draw=black,line join=bevel,line cap=rect,line width=0.800pt]
  \end{scope}
  \begin{scope}[cm={{1.01,0.0,0.0,1.01,(6.3158,0.0)}},draw=ca0a0a4,dash pattern=on 0.40pt off 0.80pt,line join=round,line cap=round,line width=0.400pt]
    \path[draw] (95.5000,104.5000) -- (95.5000,80.5000);



  \end{scope}
  \begin{scope}[cm={{1.01,0.0,0.0,1.01,(6.3158,0.0)}},draw=black,line join=round,line cap=round,line width=0.480pt]
    \path[draw] (95.5000,104.5000) -- (95.5000,99.5000);



    \path[draw] (95.5000,80.5000) -- (95.5000,84.5000);



  \end{scope}
  \begin{scope}[scale=1.010,draw=black,line join=bevel,line cap=rect,line width=0.800pt]
  \end{scope}
  \begin{scope}[cm={{1.01,0.0,0.0,1.01,(96.96,121.2)}},draw=black,line join=bevel,line cap=rect,line width=0.800pt]
  \end{scope}
  \begin{scope}[cm={{1.01,0.0,0.0,1.01,(96.96,121.2)}},draw=black,line join=bevel,line cap=rect,line width=0.800pt]
  \end{scope}
  \begin{scope}[cm={{1.01,0.0,0.0,1.01,(96.96,121.2)}},draw=black,line join=bevel,line cap=rect,line width=0.800pt]
  \end{scope}
  \begin{scope}[cm={{1.01,0.0,0.0,1.01,(96.96,121.2)}},draw=black,line join=bevel,line cap=rect,line width=0.800pt]
  \end{scope}
  \begin{scope}[cm={{1.01,0.0,0.0,1.01,(96.96,121.2)}},draw=black,line join=bevel,line cap=rect,line width=0.800pt]
  \end{scope}
  \begin{scope}[cm={{1.01,0.0,0.0,1.01,(96.96,121.2)}},draw=black,line join=bevel,line cap=rect,line width=0.800pt]
  \end{scope}
  \begin{scope}[scale=1.010,draw=black,line join=bevel,line cap=rect,line width=0.800pt]
  \end{scope}
  \begin{scope}[cm={{1.01,0.0,0.0,1.01,(6.3158,0.0)}},draw=ca0a0a4,dash pattern=on 0.40pt off 0.80pt,line join=round,line cap=round,line width=0.400pt]
    \path[draw] (130.5000,92.5000) -- (130.5000,80.5000);



  \end{scope}
  \begin{scope}[cm={{1.01,0.0,0.0,1.01,(6.3158,0.0)}},draw=black,line join=round,line cap=round,line width=0.480pt]
    \path[draw] (130.5000,104.5000) -- (130.5000,99.5000);



    \path[draw] (130.5000,80.5000) -- (130.5000,84.5000);



  \end{scope}
  \begin{scope}[scale=1.010,draw=black,line join=bevel,line cap=rect,line width=0.800pt]
  \end{scope}
  \begin{scope}[cm={{1.01,0.0,0.0,1.01,(132.31,121.2)}},draw=black,line join=bevel,line cap=rect,line width=0.800pt]
  \end{scope}
  \begin{scope}[cm={{1.01,0.0,0.0,1.01,(132.31,121.2)}},draw=black,line join=bevel,line cap=rect,line width=0.800pt]
  \end{scope}
  \begin{scope}[cm={{1.01,0.0,0.0,1.01,(132.31,121.2)}},draw=black,line join=bevel,line cap=rect,line width=0.800pt]
  \end{scope}
  \begin{scope}[cm={{1.01,0.0,0.0,1.01,(132.31,121.2)}},draw=black,line join=bevel,line cap=rect,line width=0.800pt]
  \end{scope}
  \begin{scope}[cm={{1.01,0.0,0.0,1.01,(132.31,121.2)}},draw=black,line join=bevel,line cap=rect,line width=0.800pt]
  \end{scope}
  \begin{scope}[cm={{1.01,0.0,0.0,1.01,(132.31,121.2)}},draw=black,line join=bevel,line cap=rect,line width=0.800pt]
  \end{scope}
  \begin{scope}[scale=1.010,draw=black,line join=bevel,line cap=rect,line width=0.800pt]
  \end{scope}
  \begin{scope}[cm={{1.01,0.0,0.0,1.01,(6.3158,0.0)}},draw=black,line join=round,line cap=round,line width=0.480pt]
    \path[draw] (25.5000,80.5000) -- (25.5000,104.5000) -- (142.5000,104.5000) -- (142.5000,80.5000) -- (25.5000,80.5000);



  \end{scope}
  \begin{scope}[scale=1.010,draw=black,line join=bevel,line cap=rect,line width=0.800pt]
  \end{scope}
  \begin{scope}[cm={{1.01,0.0,0.0,1.01,(109.08,105.04)}},draw=black,line join=bevel,line cap=rect,line width=0.800pt]
  \end{scope}
  \begin{scope}[cm={{1.01,0.0,0.0,1.01,(109.08,105.04)}},draw=black,line join=bevel,line cap=rect,line width=0.800pt]
  \end{scope}
  \begin{scope}[cm={{1.01,0.0,0.0,1.01,(109.08,105.04)}},draw=black,line join=bevel,line cap=rect,line width=0.800pt]
  \end{scope}
  \begin{scope}[cm={{1.01,0.0,0.0,1.01,(109.08,105.04)}},draw=black,line join=bevel,line cap=rect,line width=0.800pt]
  \end{scope}
  \begin{scope}[cm={{1.01,0.0,0.0,1.01,(109.08,105.04)}},draw=black,line join=bevel,line cap=rect,line width=0.800pt]
  \end{scope}
  \begin{scope}[cm={{1.01,0.0,0.0,1.01,(119.30192,103.41245)}},draw=black,line join=bevel,line cap=rect,line width=0.800pt]
    \path[fill=black] (0.0000,0.0000) node[above right] (text836) {\scriptsize $\alpha_2$};



  \end{scope}
  \begin{scope}[cm={{1.01,0.0,0.0,1.01,(109.08,105.04)}},draw=black,line join=bevel,line cap=rect,line width=0.800pt]
  \end{scope}
  \begin{scope}[scale=1.010,draw=black,line join=bevel,line cap=rect,line width=0.800pt]
  \end{scope}
  \begin{scope}[cm={{1.01,0.0,0.0,1.01,(6.3158,0.0)}},draw=black,line join=round,line cap=round,line width=0.480pt]
    \path[draw,even odd rule] (123.5000,100.5000) -- (132.5000,100.5000);



  \end{scope}
  \begin{scope}[scale=1.010,draw=black,line join=bevel,line cap=rect,line width=0.800pt]
  \end{scope}
  \begin{scope}[scale=1.010,draw=black,line join=bevel,line cap=rect,line width=0.800pt]
  \end{scope}
  \begin{scope}[scale=1.010,draw=black,line join=bevel,line cap=rect,line width=0.800pt]
  \end{scope}
  \begin{scope}[scale=1.010,draw=black,line join=bevel,line cap=rect,line width=0.800pt]
  \end{scope}
  \begin{scope}[cm={{1.01,0.0,0.0,1.01,(6.3158,0.0)}},draw=black,line join=round,line cap=round,line width=0.480pt]
    \path[draw] (25.8000,86.8000) -- (25.8000,86.8000) -- (26.2000,88.5000) -- (26.7000,88.7000) -- (27.1000,88.0000) -- (27.5000,88.0000) -- (27.9000,90.0000) -- (28.3000,85.4000) -- (28.8000,92.2000) -- (29.2000,86.5000) -- (29.6000,86.2000) -- (30.0000,92.2000) -- (30.5000,91.0000) -- (30.9000,85.7000) -- (31.3000,84.6000) -- (31.7000,88.3000) -- (32.2000,91.6000) -- (32.6000,91.4000) -- (33.0000,88.6000) -- (33.4000,86.0000) -- (33.8000,85.5000) -- (34.3000,87.1000) -- (34.7000,89.4000) -- (35.1000,91.2000) -- (35.5000,91.5000) -- (36.0000,90.5000) -- (36.4000,88.8000) -- (36.8000,87.2000) -- (37.2000,86.2000) -- (37.6000,86.2000) -- (38.1000,86.9000) -- (38.5000,88.0000) -- (38.9000,89.2000) -- (39.3000,90.2000) -- (39.8000,90.6000) -- (40.2000,90.5000) -- (40.6000,89.9000) -- (41.0000,89.0000) -- (41.5000,88.0000) -- (41.9000,87.1000) -- (42.3000,86.4000) -- (42.7000,86.1000) -- (43.1000,86.2000) -- (43.6000,86.6000) -- (44.0000,87.3000) -- (44.4000,88.1000) -- (44.8000,88.9000) -- (45.3000,89.6000) -- (45.7000,90.1000) -- (46.1000,90.5000) -- (46.5000,90.6000) -- (47.0000,90.6000) -- (47.4000,90.3000) -- (47.8000,89.7000) -- (48.2000,89.0000) -- (48.6000,88.3000) -- (49.1000,87.6000) -- (49.5000,86.9000) -- (49.9000,86.4000) -- (50.3000,86.0000) -- (50.8000,85.9000) -- (51.2000,86.0000) -- (51.6000,86.2000) -- (52.0000,86.6000) -- (52.4000,87.2000) -- (52.9000,87.8000) -- (53.3000,88.3000) -- (53.7000,88.9000) -- (54.1000,89.3000) -- (54.6000,89.6000) -- (55.0000,89.8000) -- (55.4000,89.8000) -- (55.8000,89.6000) -- (56.3000,89.4000) -- (56.7000,89.0000) -- (57.1000,88.6000) -- (57.5000,88.2000) -- (57.9000,87.9000) -- (58.4000,87.7000) -- (58.8000,87.6000) -- (59.2000,87.6000) -- (59.6000,87.7000) -- (60.1000,87.9000) -- (60.5000,88.2000) -- (60.9000,88.5000) -- (61.3000,88.8000) -- (61.8000,89.1000) -- (62.2000,89.4000) -- (62.6000,89.5000) -- (63.0000,89.6000) -- (63.4000,89.6000) -- (63.9000,89.4000) -- (64.3000,89.2000) -- (64.7000,89.0000) -- (65.1000,88.7000) -- (65.6000,88.5000) -- (66.0000,88.2000) -- (66.4000,88.1000) -- (66.8000,87.9000) -- (67.3000,87.9000) -- (67.7000,87.9000) -- (68.1000,87.9000) -- (68.5000,88.0000) -- (68.9000,88.1000) -- (69.4000,88.2000) -- (69.8000,88.4000) -- (70.2000,88.5000) -- (70.6000,88.6000) -- (71.1000,88.7000) -- (71.5000,88.7000) -- (71.9000,88.7000) -- (72.3000,88.6000) -- (72.7000,88.6000) -- (73.2000,88.5000) -- (73.6000,88.5000) -- (74.0000,88.5000) -- (74.4000,88.5000) -- (74.9000,88.4000) -- (75.3000,88.4000) -- (75.7000,88.3000) -- (76.1000,88.3000) -- (76.6000,88.4000) -- (77.0000,88.4000) -- (77.4000,88.4000) -- (77.8000,88.5000) -- (78.2000,88.5000) -- (78.7000,88.6000) -- (79.1000,88.6000) -- (79.5000,88.6000) -- (79.9000,88.6000) -- (80.4000,88.6000) -- (80.8000,88.5000) -- (81.2000,88.5000) -- (81.6000,88.4000) -- (82.1000,88.3000) -- (82.5000,88.3000) -- (82.9000,88.2000) -- (83.3000,88.1000) -- (83.7000,88.0000) -- (84.2000,88.0000) -- (84.6000,87.9000) -- (85.0000,87.9000) -- (85.4000,87.9000) -- (85.9000,87.9000) -- (86.3000,88.0000) -- (86.7000,88.0000) -- (87.1000,88.1000) -- (87.6000,88.2000) -- (88.0000,88.3000) -- (88.4000,88.4000) -- (88.8000,88.5000) -- (89.2000,88.7000) -- (89.7000,88.8000) -- (90.1000,88.9000) -- (90.5000,89.0000) -- (90.9000,89.0000) -- (91.4000,89.1000) -- (91.8000,89.1000) -- (92.2000,89.0000) -- (92.6000,89.0000) -- (93.0000,88.9000) -- (93.5000,88.8000) -- (93.9000,88.7000) -- (94.3000,88.6000) -- (94.7000,88.5000) -- (95.2000,88.3000) -- (95.6000,88.2000) -- (96.0000,88.1000) -- (96.4000,88.0000) -- (96.9000,88.0000) -- (97.3000,88.0000) -- (97.7000,88.0000) -- (98.1000,88.0000) -- (98.5000,88.0000) -- (99.0000,88.1000) -- (99.4000,88.1000) -- (99.8000,88.2000) -- (100.2000,88.2000) -- (100.7000,88.3000) -- (101.1000,88.5000) -- (101.5000,88.5000) -- (101.9000,88.5000) -- (102.3000,88.5000) -- (102.8000,88.6000) -- (103.2000,88.6000) -- (103.6000,88.6000) -- (104.0000,88.7000) -- (104.5000,88.7000) -- (104.9000,88.7000) -- (105.3000,88.7000) -- (105.7000,88.7000) -- (106.2000,88.6000) -- (106.6000,88.6000) -- (107.0000,88.6000) -- (107.4000,88.5000) -- (107.8000,88.5000) -- (108.3000,88.4000) -- (108.7000,88.4000) -- (109.1000,88.4000) -- (109.5000,88.4000) -- (110.0000,88.4000) -- (110.4000,88.4000) -- (110.8000,88.3000) -- (111.2000,88.3000) -- (111.7000,88.3000) -- (112.1000,88.3000) -- (112.5000,88.3000) -- (112.9000,88.3000) -- (113.3000,88.3000) -- (113.8000,88.3000) -- (114.2000,88.3000) -- (114.6000,88.3000) -- (115.0000,88.3000) -- (115.5000,88.3000) -- (115.9000,88.4000) -- (116.3000,88.4000) -- (116.7000,88.4000) -- (117.2000,88.5000) -- (117.6000,88.5000) -- (118.0000,88.6000) -- (118.4000,88.6000) -- (118.8000,88.6000) -- (119.3000,88.7000) -- (119.7000,88.7000) -- (120.1000,88.7000) -- (120.5000,88.6000) -- (121.0000,88.6000) -- (121.4000,88.6000) -- (121.8000,88.5000) -- (122.2000,88.4000) -- (122.6000,88.4000) -- (123.1000,88.3000) -- (123.5000,88.3000) -- (123.9000,88.3000) -- (124.3000,88.3000) -- (124.8000,88.2000) -- (125.2000,88.2000) -- (125.6000,88.2000) -- (126.0000,88.1000) -- (126.5000,88.1000) -- (126.9000,88.2000) -- (127.3000,88.2000) -- (127.7000,88.3000) -- (128.1000,88.4000) -- (128.6000,88.4000) -- (129.0000,88.4000) -- (129.4000,88.5000) -- (129.8000,88.6000) -- (130.3000,88.6000) -- (130.7000,88.7000) -- (131.1000,88.8000) -- (131.5000,88.8000) -- (131.9000,88.8000) -- (132.4000,88.8000) -- (132.8000,88.8000) -- (133.2000,88.7000) -- (133.6000,88.7000) -- (134.1000,88.6000) -- (134.5000,88.5000) -- (134.9000,88.4000) -- (135.3000,88.4000) -- (135.8000,88.3000) -- (136.2000,88.3000) -- (136.6000,88.2000) -- (137.0000,88.2000) -- (137.4000,88.1000) -- (137.9000,88.1000) -- (138.3000,88.1000) -- (138.7000,88.1000) -- (139.1000,88.2000) -- (139.6000,88.2000) -- (140.0000,88.3000) -- (140.4000,88.3000) -- (140.8000,88.3000) -- (141.3000,88.4000) -- (141.7000,88.4000) -- (142.1000,88.5000) -- (142.3000,88.5000);



  \end{scope}
  \begin{scope}[scale=1.010,draw=black,line join=bevel,line cap=rect,line width=0.800pt]
  \end{scope}
  \begin{scope}[scale=1.010,draw=black,line join=bevel,line cap=rect,line width=0.800pt]
  \end{scope}
  \begin{scope}[cm={{1.01,0.0,0.0,1.01,(6.3158,0.0)}},draw=black,line join=round,line cap=round,line width=0.480pt]
    \path[draw] (25.5000,80.5000) -- (25.5000,104.5000) -- (142.5000,104.5000) -- (142.5000,80.5000) -- (25.5000,80.5000);



  \end{scope}
  \begin{scope}[cm={{1.01,0.0,0.0,1.01,(6.3158,0.0)}},draw=ca0a0a4,dash pattern=on 0.40pt off 0.80pt,line join=round,line cap=round,line width=0.400pt]
    \path[draw] (25.5000,118.5000) -- (108.5000,118.5000);



    \path[draw] (137.5000,118.5000) -- (142.5000,118.5000);



  \end{scope}
  \begin{scope}[cm={{1.01,0.0,0.0,1.01,(6.3158,0.0)}},draw=black,line join=round,line cap=round,line width=0.480pt]
    \path[draw] (25.5000,118.5000) -- (28.5000,118.5000);



    \path[draw] (142.5000,118.5000) -- (139.5000,118.5000);



  \end{scope}
  \begin{scope}[scale=1.010,draw=black,line join=bevel,line cap=rect,line width=0.800pt]
  \end{scope}
  \begin{scope}[cm={{1.01,0.0,0.0,1.01,(5.05,123.22)}},draw=black,line join=bevel,line cap=rect,line width=0.800pt]
  \end{scope}
  \begin{scope}[cm={{1.01,0.0,0.0,1.01,(5.05,123.22)}},draw=black,line join=bevel,line cap=rect,line width=0.800pt]
  \end{scope}
  \begin{scope}[cm={{1.01,0.0,0.0,1.01,(5.05,123.22)}},draw=black,line join=bevel,line cap=rect,line width=0.800pt]
  \end{scope}
  \begin{scope}[cm={{1.01,0.0,0.0,1.01,(5.05,123.22)}},draw=black,line join=bevel,line cap=rect,line width=0.800pt]
  \end{scope}
  \begin{scope}[cm={{1.01,0.0,0.0,1.01,(5.05,123.22)}},draw=black,line join=bevel,line cap=rect,line width=0.800pt]
  \end{scope}
  \begin{scope}[cm={{1.01,0.0,0.0,1.01,(11.3658,123.22)}},draw=black,line join=bevel,line cap=rect,line width=0.800pt]
    \path[fill=black] (0.0000,0.0000) node[above right] (text892) {-10};



  \end{scope}
  \begin{scope}[cm={{1.01,0.0,0.0,1.01,(5.05,123.22)}},draw=black,line join=bevel,line cap=rect,line width=0.800pt]
  \end{scope}
  \begin{scope}[scale=1.010,draw=black,line join=bevel,line cap=rect,line width=0.800pt]
  \end{scope}
  \begin{scope}[cm={{1.01,0.0,0.0,1.01,(6.3158,0.0)}},draw=ca0a0a4,dash pattern=on 0.40pt off 0.80pt,line join=round,line cap=round,line width=0.400pt]
    \path[draw] (25.5000,106.5000) -- (142.5000,106.5000);



  \end{scope}
  \begin{scope}[cm={{1.01,0.0,0.0,1.01,(6.3158,0.0)}},draw=black,line join=round,line cap=round,line width=0.480pt]
    \path[draw] (25.5000,106.5000) -- (28.5000,106.5000);



    \path[draw] (142.5000,106.5000) -- (139.5000,106.5000);



  \end{scope}
  \begin{scope}[scale=1.010,draw=black,line join=bevel,line cap=rect,line width=0.800pt]
  \end{scope}
  \begin{scope}[cm={{1.01,0.0,0.0,1.01,(9.09,112.11)}},draw=black,line join=bevel,line cap=rect,line width=0.800pt]
  \end{scope}
  \begin{scope}[cm={{1.01,0.0,0.0,1.01,(9.09,112.11)}},draw=black,line join=bevel,line cap=rect,line width=0.800pt]
  \end{scope}
  \begin{scope}[cm={{1.01,0.0,0.0,1.01,(9.09,112.11)}},draw=black,line join=bevel,line cap=rect,line width=0.800pt]
  \end{scope}
  \begin{scope}[cm={{1.01,0.0,0.0,1.01,(9.09,112.11)}},draw=black,line join=bevel,line cap=rect,line width=0.800pt]
  \end{scope}
  \begin{scope}[cm={{1.01,0.0,0.0,1.01,(9.09,112.11)}},draw=black,line join=bevel,line cap=rect,line width=0.800pt]
  \end{scope}
  \begin{scope}[cm={{1.01,0.0,0.0,1.01,(15.4058,112.11)}},draw=black,line join=bevel,line cap=rect,line width=0.800pt]
    \path[fill=black] (0.0000,0.0000) node[above right] (text922) {10};



  \end{scope}
  \begin{scope}[cm={{1.01,0.0,0.0,1.01,(9.09,112.11)}},draw=black,line join=bevel,line cap=rect,line width=0.800pt]
  \end{scope}
  \begin{scope}[scale=1.010,draw=black,line join=bevel,line cap=rect,line width=0.800pt]
  \end{scope}
  \begin{scope}[cm={{1.01,0.0,0.0,1.01,(6.3158,0.0)}},draw=ca0a0a4,dash pattern=on 0.40pt off 0.80pt,line join=round,line cap=round,line width=0.400pt]
    \path[draw] (25.5000,128.5000) -- (25.5000,104.5000);



  \end{scope}
  \begin{scope}[cm={{1.01,0.0,0.0,1.01,(6.3158,0.0)}},draw=black,line join=round,line cap=round,line width=0.480pt]
    \path[draw] (25.5000,128.5000) -- (25.5000,123.5000);



    \path[draw] (25.5000,104.5000) -- (25.5000,108.5000);



  \end{scope}
  \begin{scope}[scale=1.010,draw=black,line join=bevel,line cap=rect,line width=0.800pt]
  \end{scope}
  \begin{scope}[cm={{1.01,0.0,0.0,1.01,(26.26,145.44)}},draw=black,line join=bevel,line cap=rect,line width=0.800pt]
  \end{scope}
  \begin{scope}[cm={{1.01,0.0,0.0,1.01,(26.26,145.44)}},draw=black,line join=bevel,line cap=rect,line width=0.800pt]
  \end{scope}
  \begin{scope}[cm={{1.01,0.0,0.0,1.01,(26.26,145.44)}},draw=black,line join=bevel,line cap=rect,line width=0.800pt]
  \end{scope}
  \begin{scope}[cm={{1.01,0.0,0.0,1.01,(26.26,145.44)}},draw=black,line join=bevel,line cap=rect,line width=0.800pt]
  \end{scope}
  \begin{scope}[cm={{1.01,0.0,0.0,1.01,(26.26,145.44)}},draw=black,line join=bevel,line cap=rect,line width=0.800pt]
  \end{scope}
  \begin{scope}[cm={{1.01,0.0,0.0,1.01,(26.26,145.44)}},draw=black,line join=bevel,line cap=rect,line width=0.800pt]
  \end{scope}
  \begin{scope}[scale=1.010,draw=black,line join=bevel,line cap=rect,line width=0.800pt]
  \end{scope}
  \begin{scope}[cm={{1.01,0.0,0.0,1.01,(6.3158,0.0)}},draw=ca0a0a4,dash pattern=on 0.40pt off 0.80pt,line join=round,line cap=round,line width=0.400pt]
    \path[draw] (60.5000,128.5000) -- (60.5000,104.5000);



  \end{scope}
  \begin{scope}[cm={{1.01,0.0,0.0,1.01,(6.3158,0.0)}},draw=black,line join=round,line cap=round,line width=0.480pt]
    \path[draw] (60.5000,128.5000) -- (60.5000,123.5000);



    \path[draw] (60.5000,104.5000) -- (60.5000,108.5000);



  \end{scope}
  \begin{scope}[scale=1.010,draw=black,line join=bevel,line cap=rect,line width=0.800pt]
  \end{scope}
  \begin{scope}[cm={{1.01,0.0,0.0,1.01,(61.61,145.44)}},draw=black,line join=bevel,line cap=rect,line width=0.800pt]
  \end{scope}
  \begin{scope}[cm={{1.01,0.0,0.0,1.01,(61.61,145.44)}},draw=black,line join=bevel,line cap=rect,line width=0.800pt]
  \end{scope}
  \begin{scope}[cm={{1.01,0.0,0.0,1.01,(61.61,145.44)}},draw=black,line join=bevel,line cap=rect,line width=0.800pt]
  \end{scope}
  \begin{scope}[cm={{1.01,0.0,0.0,1.01,(61.61,145.44)}},draw=black,line join=bevel,line cap=rect,line width=0.800pt]
  \end{scope}
  \begin{scope}[cm={{1.01,0.0,0.0,1.01,(61.61,145.44)}},draw=black,line join=bevel,line cap=rect,line width=0.800pt]
  \end{scope}
  \begin{scope}[cm={{1.01,0.0,0.0,1.01,(61.61,145.44)}},draw=black,line join=bevel,line cap=rect,line width=0.800pt]
  \end{scope}
  \begin{scope}[scale=1.010,draw=black,line join=bevel,line cap=rect,line width=0.800pt]
  \end{scope}
  \begin{scope}[cm={{1.01,0.0,0.0,1.01,(6.3158,0.0)}},draw=ca0a0a4,dash pattern=on 0.40pt off 0.80pt,line join=round,line cap=round,line width=0.400pt]
    \path[draw] (95.5000,128.5000) -- (95.5000,104.5000);



  \end{scope}
  \begin{scope}[cm={{1.01,0.0,0.0,1.01,(6.3158,0.0)}},draw=black,line join=round,line cap=round,line width=0.480pt]
    \path[draw] (95.5000,128.5000) -- (95.5000,123.5000);



    \path[draw] (95.5000,104.5000) -- (95.5000,108.5000);



  \end{scope}
  \begin{scope}[scale=1.010,draw=black,line join=bevel,line cap=rect,line width=0.800pt]
  \end{scope}
  \begin{scope}[cm={{1.01,0.0,0.0,1.01,(96.96,145.44)}},draw=black,line join=bevel,line cap=rect,line width=0.800pt]
  \end{scope}
  \begin{scope}[cm={{1.01,0.0,0.0,1.01,(96.96,145.44)}},draw=black,line join=bevel,line cap=rect,line width=0.800pt]
  \end{scope}
  \begin{scope}[cm={{1.01,0.0,0.0,1.01,(96.96,145.44)}},draw=black,line join=bevel,line cap=rect,line width=0.800pt]
  \end{scope}
  \begin{scope}[cm={{1.01,0.0,0.0,1.01,(96.96,145.44)}},draw=black,line join=bevel,line cap=rect,line width=0.800pt]
  \end{scope}
  \begin{scope}[cm={{1.01,0.0,0.0,1.01,(96.96,145.44)}},draw=black,line join=bevel,line cap=rect,line width=0.800pt]
  \end{scope}
  \begin{scope}[cm={{1.01,0.0,0.0,1.01,(96.96,145.44)}},draw=black,line join=bevel,line cap=rect,line width=0.800pt]
  \end{scope}
  \begin{scope}[scale=1.010,draw=black,line join=bevel,line cap=rect,line width=0.800pt]
  \end{scope}
  \begin{scope}[cm={{1.01,0.0,0.0,1.01,(6.3158,0.0)}},draw=ca0a0a4,dash pattern=on 0.40pt off 0.80pt,line join=round,line cap=round,line width=0.400pt]
    \path[draw] (130.5000,116.5000) -- (130.5000,104.5000);



  \end{scope}
  \begin{scope}[cm={{1.01,0.0,0.0,1.01,(6.3158,0.0)}},draw=black,line join=round,line cap=round,line width=0.480pt]
    \path[draw] (130.5000,128.5000) -- (130.5000,123.5000);



    \path[draw] (130.5000,104.5000) -- (130.5000,108.5000);



  \end{scope}
  \begin{scope}[scale=1.010,draw=black,line join=bevel,line cap=rect,line width=0.800pt]
  \end{scope}
  \begin{scope}[cm={{1.01,0.0,0.0,1.01,(132.31,145.44)}},draw=black,line join=bevel,line cap=rect,line width=0.800pt]
  \end{scope}
  \begin{scope}[cm={{1.01,0.0,0.0,1.01,(132.31,145.44)}},draw=black,line join=bevel,line cap=rect,line width=0.800pt]
  \end{scope}
  \begin{scope}[cm={{1.01,0.0,0.0,1.01,(132.31,145.44)}},draw=black,line join=bevel,line cap=rect,line width=0.800pt]
  \end{scope}
  \begin{scope}[cm={{1.01,0.0,0.0,1.01,(132.31,145.44)}},draw=black,line join=bevel,line cap=rect,line width=0.800pt]
  \end{scope}
  \begin{scope}[cm={{1.01,0.0,0.0,1.01,(132.31,145.44)}},draw=black,line join=bevel,line cap=rect,line width=0.800pt]
  \end{scope}
  \begin{scope}[cm={{1.01,0.0,0.0,1.01,(132.31,145.44)}},draw=black,line join=bevel,line cap=rect,line width=0.800pt]
  \end{scope}
  \begin{scope}[scale=1.010,draw=black,line join=bevel,line cap=rect,line width=0.800pt]
  \end{scope}
  \begin{scope}[cm={{1.01,0.0,0.0,1.01,(6.3158,0.0)}},draw=black,line join=round,line cap=round,line width=0.480pt]
    \path[draw] (25.5000,104.5000) -- (25.5000,128.5000) -- (142.5000,128.5000) -- (142.5000,104.5000) -- (25.5000,104.5000);



  \end{scope}
  \begin{scope}[scale=1.010,draw=black,line join=bevel,line cap=rect,line width=0.800pt]
  \end{scope}
  \begin{scope}[cm={{1.01,0.0,0.0,1.01,(108.07,129.28)}},draw=black,line join=bevel,line cap=rect,line width=0.800pt]
  \end{scope}
  \begin{scope}[cm={{1.01,0.0,0.0,1.01,(108.07,129.28)}},draw=black,line join=bevel,line cap=rect,line width=0.800pt]
  \end{scope}
  \begin{scope}[cm={{1.01,0.0,0.0,1.01,(108.07,129.28)}},draw=black,line join=bevel,line cap=rect,line width=0.800pt]
  \end{scope}
  \begin{scope}[cm={{1.01,0.0,0.0,1.01,(108.07,129.28)}},draw=black,line join=bevel,line cap=rect,line width=0.800pt]
  \end{scope}
  \begin{scope}[cm={{1.01,0.0,0.0,1.01,(108.07,129.28)}},draw=black,line join=bevel,line cap=rect,line width=0.800pt]
  \end{scope}
  \begin{scope}[cm={{1.01,0.0,0.0,1.01,(118.29192,127.65245)}},draw=black,line join=bevel,line cap=rect,line width=0.800pt]
    \path[fill=black] (0.0000,0.0000) node[above right] (text1050) {\scriptsize $\beta_2$};



  \end{scope}
  \begin{scope}[cm={{1.01,0.0,0.0,1.01,(108.07,129.28)}},draw=black,line join=bevel,line cap=rect,line width=0.800pt]
  \end{scope}
  \begin{scope}[scale=1.010,draw=black,line join=bevel,line cap=rect,line width=0.800pt]
  \end{scope}
  \begin{scope}[cm={{1.01,0.0,0.0,1.01,(6.3158,0.0)}},draw=black,line join=round,line cap=round,line width=0.480pt]
    \path[draw,even odd rule] (123.5000,124.5000) -- (132.5000,124.5000);



  \end{scope}
  \begin{scope}[scale=1.010,draw=black,line join=bevel,line cap=rect,line width=0.800pt]
  \end{scope}
  \begin{scope}[scale=1.010,draw=black,line join=bevel,line cap=rect,line width=0.800pt]
  \end{scope}
  \begin{scope}[scale=1.010,draw=black,line join=bevel,line cap=rect,line width=0.800pt]
  \end{scope}
  \begin{scope}[scale=1.010,draw=black,line join=bevel,line cap=rect,line width=0.800pt]
  \end{scope}
  \begin{scope}[cm={{1.01,0.0,0.0,1.01,(6.3158,0.0)}},draw=black,line join=round,line cap=round,line width=0.480pt]
    \path[draw] (25.8000,114.2000) -- (25.8000,114.2000) -- (26.2000,112.5000) -- (26.7000,112.2000) -- (27.1000,112.0000) -- (27.5000,115.1000) -- (27.9000,109.2000) -- (28.3000,114.1000) -- (28.8000,113.9000) -- (29.2000,108.7000) -- (29.6000,114.6000) -- (30.0000,115.3000) -- (30.5000,109.5000) -- (30.9000,108.5000) -- (31.3000,113.0000) -- (31.7000,116.4000) -- (32.2000,115.2000) -- (32.6000,111.5000) -- (33.0000,109.2000) -- (33.4000,109.7000) -- (33.8000,112.1000) -- (34.3000,114.6000) -- (34.7000,115.4000) -- (35.1000,114.5000) -- (35.5000,112.6000) -- (36.0000,110.6000) -- (36.4000,109.5000) -- (36.8000,109.5000) -- (37.2000,110.5000) -- (37.6000,112.0000) -- (38.1000,113.4000) -- (38.5000,114.4000) -- (38.9000,114.7000) -- (39.3000,114.3000) -- (39.8000,113.4000) -- (40.2000,112.3000) -- (40.6000,111.3000) -- (41.0000,110.6000) -- (41.5000,110.2000) -- (41.9000,110.3000) -- (42.3000,110.8000) -- (42.7000,111.6000) -- (43.1000,112.5000) -- (43.6000,113.4000) -- (44.0000,114.1000) -- (44.4000,114.5000) -- (44.8000,114.7000) -- (45.3000,114.6000) -- (45.7000,114.3000) -- (46.1000,113.7000) -- (46.5000,113.0000) -- (47.0000,112.4000) -- (47.4000,111.7000) -- (47.8000,111.0000) -- (48.2000,110.5000) -- (48.6000,110.2000) -- (49.1000,110.2000) -- (49.5000,110.4000) -- (49.9000,110.8000) -- (50.3000,111.4000) -- (50.8000,112.0000) -- (51.2000,112.7000) -- (51.6000,113.4000) -- (52.0000,114.0000) -- (52.4000,114.5000) -- (52.9000,114.8000) -- (53.3000,114.8000) -- (53.7000,114.7000) -- (54.1000,114.5000) -- (54.6000,114.0000) -- (55.0000,113.5000) -- (55.4000,113.0000) -- (55.8000,112.5000) -- (56.3000,112.0000) -- (56.7000,111.6000) -- (57.1000,111.4000) -- (57.5000,111.3000) -- (57.9000,111.3000) -- (58.4000,111.5000) -- (58.8000,111.8000) -- (59.2000,112.2000) -- (59.6000,112.5000) -- (60.1000,112.8000) -- (60.5000,113.0000) -- (60.9000,113.2000) -- (61.3000,113.2000) -- (61.8000,113.2000) -- (62.2000,113.1000) -- (62.6000,112.8000) -- (63.0000,112.6000) -- (63.4000,112.2000) -- (63.9000,111.9000) -- (64.3000,111.7000) -- (64.7000,111.5000) -- (65.1000,111.4000) -- (65.6000,111.3000) -- (66.0000,111.4000) -- (66.4000,111.5000) -- (66.8000,111.7000) -- (67.3000,112.0000) -- (67.7000,112.2000) -- (68.1000,112.4000) -- (68.5000,112.6000) -- (68.9000,112.7000) -- (69.4000,112.8000) -- (69.8000,112.9000) -- (70.2000,112.9000) -- (70.6000,112.9000) -- (71.1000,112.8000) -- (71.5000,112.7000) -- (71.9000,112.6000) -- (72.3000,112.5000) -- (72.7000,112.4000) -- (73.2000,112.3000) -- (73.6000,112.3000) -- (74.0000,112.3000) -- (74.4000,112.3000) -- (74.9000,112.3000) -- (75.3000,112.3000) -- (75.7000,112.3000) -- (76.1000,112.3000) -- (76.6000,112.4000) -- (77.0000,112.5000) -- (77.4000,112.5000) -- (77.8000,112.5000) -- (78.2000,112.5000) -- (78.7000,112.5000) -- (79.1000,112.5000) -- (79.5000,112.5000) -- (79.9000,112.4000) -- (80.4000,112.4000) -- (80.8000,112.3000) -- (81.2000,112.3000) -- (81.6000,112.3000) -- (82.1000,112.2000) -- (82.5000,112.3000) -- (82.9000,112.3000) -- (83.3000,112.3000) -- (83.7000,112.3000) -- (84.2000,112.3000) -- (84.6000,112.4000) -- (85.0000,112.5000) -- (85.4000,112.6000) -- (85.9000,112.7000) -- (86.3000,112.8000) -- (86.7000,112.8000) -- (87.1000,112.9000) -- (87.6000,112.9000) -- (88.0000,113.0000) -- (88.4000,113.0000) -- (88.8000,113.0000) -- (89.2000,113.0000) -- (89.7000,113.0000) -- (90.1000,112.9000) -- (90.5000,112.8000) -- (90.9000,112.7000) -- (91.4000,112.6000) -- (91.8000,112.4000) -- (92.2000,112.3000) -- (92.6000,112.2000) -- (93.0000,112.0000) -- (93.5000,111.9000) -- (93.9000,111.9000) -- (94.3000,111.8000) -- (94.7000,111.8000) -- (95.2000,111.8000) -- (95.6000,111.8000) -- (96.0000,111.9000) -- (96.4000,112.0000) -- (96.9000,112.2000) -- (97.3000,112.3000) -- (97.7000,112.4000) -- (98.1000,112.5000) -- (98.5000,112.6000) -- (99.0000,112.6000) -- (99.4000,112.7000) -- (99.8000,112.7000) -- (100.2000,112.8000) -- (100.7000,112.8000) -- (101.1000,112.9000) -- (101.5000,112.9000) -- (101.9000,112.8000) -- (102.3000,112.7000) -- (102.8000,112.7000) -- (103.2000,112.6000) -- (103.6000,112.6000) -- (104.0000,112.6000) -- (104.5000,112.5000) -- (104.9000,112.5000) -- (105.3000,112.4000) -- (105.7000,112.3000) -- (106.2000,112.3000) -- (106.6000,112.3000) -- (107.0000,112.2000) -- (107.4000,112.2000) -- (107.8000,112.2000) -- (108.3000,112.2000) -- (108.7000,112.2000) -- (109.1000,112.3000) -- (109.5000,112.3000) -- (110.0000,112.3000) -- (110.4000,112.3000) -- (110.8000,112.3000) -- (111.2000,112.3000) -- (111.7000,112.4000) -- (112.1000,112.4000) -- (112.5000,112.4000) -- (112.9000,112.5000) -- (113.3000,112.5000) -- (113.8000,112.5000) -- (114.2000,112.5000) -- (114.6000,112.5000) -- (115.0000,112.5000) -- (115.5000,112.6000) -- (115.9000,112.6000) -- (116.3000,112.6000) -- (116.7000,112.6000) -- (117.2000,112.6000) -- (117.6000,112.6000) -- (118.0000,112.6000) -- (118.4000,112.6000) -- (118.8000,112.6000) -- (119.3000,112.5000) -- (119.7000,112.5000) -- (120.1000,112.4000) -- (120.5000,112.4000) -- (121.0000,112.3000) -- (121.4000,112.3000) -- (121.8000,112.2000) -- (122.2000,112.2000) -- (122.6000,112.2000) -- (123.1000,112.2000) -- (123.5000,112.2000) -- (123.9000,112.3000) -- (124.3000,112.3000) -- (124.8000,112.3000) -- (125.2000,112.4000) -- (125.6000,112.4000) -- (126.0000,112.4000) -- (126.5000,112.5000) -- (126.9000,112.5000) -- (127.3000,112.6000) -- (127.7000,112.7000) -- (128.1000,112.7000) -- (128.6000,112.7000) -- (129.0000,112.7000) -- (129.4000,112.7000) -- (129.8000,112.7000) -- (130.3000,112.7000) -- (130.7000,112.7000) -- (131.1000,112.6000) -- (131.5000,112.6000) -- (131.9000,112.5000) -- (132.4000,112.4000) -- (132.8000,112.3000) -- (133.2000,112.2000) -- (133.6000,112.2000) -- (134.1000,112.1000) -- (134.5000,112.1000) -- (134.9000,112.1000) -- (135.3000,112.1000) -- (135.8000,112.1000) -- (136.2000,112.2000) -- (136.6000,112.2000) -- (137.0000,112.2000) -- (137.4000,112.3000) -- (137.9000,112.4000) -- (138.3000,112.4000) -- (138.7000,112.5000) -- (139.1000,112.6000) -- (139.6000,112.7000) -- (140.0000,112.7000) -- (140.4000,112.7000) -- (140.8000,112.7000) -- (141.3000,112.7000) -- (141.7000,112.7000) -- (142.1000,112.7000) -- (142.3000,112.7000);



  \end{scope}
  \begin{scope}[scale=1.010,draw=black,line join=bevel,line cap=rect,line width=0.800pt]
  \end{scope}
  \begin{scope}[scale=1.010,draw=black,line join=bevel,line cap=rect,line width=0.800pt]
  \end{scope}
  \begin{scope}[cm={{1.01,0.0,0.0,1.01,(6.3158,0.0)}},draw=black,line join=round,line cap=round,line width=0.480pt]
    \path[draw] (25.5000,104.5000) -- (25.5000,128.5000) -- (142.5000,128.5000) -- (142.5000,104.5000) -- (25.5000,104.5000);



  \end{scope}
  \begin{scope}[cm={{1.01,0.0,0.0,1.01,(6.3158,0.0)}},draw=ca0a0a4,dash pattern=on 0.40pt off 0.80pt,line join=round,line cap=round,line width=0.400pt]
    \path[draw] (25.5000,144.5000) -- (108.5000,144.5000);



    \path[draw] (137.5000,144.5000) -- (142.5000,144.5000);



  \end{scope}
  \begin{scope}[cm={{1.01,0.0,0.0,1.01,(6.3158,0.0)}},draw=black,line join=round,line cap=round,line width=0.480pt]
    \path[draw] (25.5000,144.5000) -- (28.5000,144.5000);



    \path[draw] (142.5000,144.5000) -- (139.5000,144.5000);



  \end{scope}
  \begin{scope}[scale=1.010,draw=black,line join=bevel,line cap=rect,line width=0.800pt]
  \end{scope}
  \begin{scope}[cm={{1.01,0.0,0.0,1.01,(5.05,150.49)}},draw=black,line join=bevel,line cap=rect,line width=0.800pt]
  \end{scope}
  \begin{scope}[cm={{1.01,0.0,0.0,1.01,(5.05,150.49)}},draw=black,line join=bevel,line cap=rect,line width=0.800pt]
  \end{scope}
  \begin{scope}[cm={{1.01,0.0,0.0,1.01,(5.05,150.49)}},draw=black,line join=bevel,line cap=rect,line width=0.800pt]
  \end{scope}
  \begin{scope}[cm={{1.01,0.0,0.0,1.01,(5.05,150.49)}},draw=black,line join=bevel,line cap=rect,line width=0.800pt]
  \end{scope}
  \begin{scope}[cm={{1.01,0.0,0.0,1.01,(5.05,150.49)}},draw=black,line join=bevel,line cap=rect,line width=0.800pt]
  \end{scope}
  \begin{scope}[cm={{1.01,0.0,0.0,1.01,(11.3658,150.49)}},draw=black,line join=bevel,line cap=rect,line width=0.800pt]
    \path[fill=black] (0.0000,0.0000) node[above right] (text1106) {-20};



  \end{scope}
  \begin{scope}[cm={{1.01,0.0,0.0,1.01,(5.05,150.49)}},draw=black,line join=bevel,line cap=rect,line width=0.800pt]
  \end{scope}
  \begin{scope}[scale=1.010,draw=black,line join=bevel,line cap=rect,line width=0.800pt]
  \end{scope}
  \begin{scope}[cm={{1.01,0.0,0.0,1.01,(6.3158,0.0)}},draw=ca0a0a4,dash pattern=on 0.40pt off 0.80pt,line join=round,line cap=round,line width=0.400pt]
    \path[draw] (25.5000,129.5000) -- (142.5000,129.5000);



  \end{scope}
  \begin{scope}[cm={{1.01,0.0,0.0,1.01,(6.3158,0.0)}},draw=black,line join=round,line cap=round,line width=0.480pt]
    \path[draw] (25.5000,129.5000) -- (28.5000,129.5000);



    \path[draw] (142.5000,129.5000) -- (139.5000,129.5000);



  \end{scope}
  \begin{scope}[scale=1.010,draw=black,line join=bevel,line cap=rect,line width=0.800pt]
  \end{scope}
  \begin{scope}[cm={{1.01,0.0,0.0,1.01,(9.09,135.34)}},draw=black,line join=bevel,line cap=rect,line width=0.800pt]
  \end{scope}
  \begin{scope}[cm={{1.01,0.0,0.0,1.01,(9.09,135.34)}},draw=black,line join=bevel,line cap=rect,line width=0.800pt]
  \end{scope}
  \begin{scope}[cm={{1.01,0.0,0.0,1.01,(9.09,135.34)}},draw=black,line join=bevel,line cap=rect,line width=0.800pt]
  \end{scope}
  \begin{scope}[cm={{1.01,0.0,0.0,1.01,(9.09,135.34)}},draw=black,line join=bevel,line cap=rect,line width=0.800pt]
  \end{scope}
  \begin{scope}[cm={{1.01,0.0,0.0,1.01,(9.09,135.34)}},draw=black,line join=bevel,line cap=rect,line width=0.800pt]
  \end{scope}
  \begin{scope}[cm={{1.01,0.0,0.0,1.01,(15.4058,135.34)}},draw=black,line join=bevel,line cap=rect,line width=0.800pt]
    \path[fill=black] (0.0000,0.0000) node[above right] (text1136) {20};



  \end{scope}
  \begin{scope}[cm={{1.01,0.0,0.0,1.01,(9.09,135.34)}},draw=black,line join=bevel,line cap=rect,line width=0.800pt]
  \end{scope}
  \begin{scope}[scale=1.010,draw=black,line join=bevel,line cap=rect,line width=0.800pt]
  \end{scope}
  \begin{scope}[cm={{1.01,0.0,0.0,1.01,(6.3158,0.0)}},draw=ca0a0a4,dash pattern=on 0.40pt off 0.80pt,line join=round,line cap=round,line width=0.400pt]
    \path[draw] (25.5000,152.5000) -- (25.5000,128.5000);



  \end{scope}
  \begin{scope}[cm={{1.01,0.0,0.0,1.01,(6.3158,0.0)}},draw=black,line join=round,line cap=round,line width=0.480pt]
    \path[draw] (25.5000,152.5000) -- (25.5000,147.5000);



    \path[draw] (25.5000,128.5000) -- (25.5000,132.5000);



  \end{scope}
  \begin{scope}[scale=1.010,draw=black,line join=bevel,line cap=rect,line width=0.800pt]
  \end{scope}
  \begin{scope}[cm={{1.01,0.0,0.0,1.01,(26.26,169.68)}},draw=black,line join=bevel,line cap=rect,line width=0.800pt]
  \end{scope}
  \begin{scope}[cm={{1.01,0.0,0.0,1.01,(26.26,169.68)}},draw=black,line join=bevel,line cap=rect,line width=0.800pt]
  \end{scope}
  \begin{scope}[cm={{1.01,0.0,0.0,1.01,(26.26,169.68)}},draw=black,line join=bevel,line cap=rect,line width=0.800pt]
  \end{scope}
  \begin{scope}[cm={{1.01,0.0,0.0,1.01,(26.26,169.68)}},draw=black,line join=bevel,line cap=rect,line width=0.800pt]
  \end{scope}
  \begin{scope}[cm={{1.01,0.0,0.0,1.01,(26.26,169.68)}},draw=black,line join=bevel,line cap=rect,line width=0.800pt]
  \end{scope}
  \begin{scope}[cm={{1.01,0.0,0.0,1.01,(26.26,169.68)}},draw=black,line join=bevel,line cap=rect,line width=0.800pt]
  \end{scope}
  \begin{scope}[scale=1.010,draw=black,line join=bevel,line cap=rect,line width=0.800pt]
  \end{scope}
  \begin{scope}[cm={{1.01,0.0,0.0,1.01,(6.3158,0.0)}},draw=ca0a0a4,dash pattern=on 0.40pt off 0.80pt,line join=round,line cap=round,line width=0.400pt]
    \path[draw] (60.5000,152.5000) -- (60.5000,128.5000);



  \end{scope}
  \begin{scope}[cm={{1.01,0.0,0.0,1.01,(6.3158,0.0)}},draw=black,line join=round,line cap=round,line width=0.480pt]
    \path[draw] (60.5000,152.5000) -- (60.5000,147.5000);



    \path[draw] (60.5000,128.5000) -- (60.5000,132.5000);



  \end{scope}
  \begin{scope}[scale=1.010,draw=black,line join=bevel,line cap=rect,line width=0.800pt]
  \end{scope}
  \begin{scope}[cm={{1.01,0.0,0.0,1.01,(61.61,169.68)}},draw=black,line join=bevel,line cap=rect,line width=0.800pt]
  \end{scope}
  \begin{scope}[cm={{1.01,0.0,0.0,1.01,(61.61,169.68)}},draw=black,line join=bevel,line cap=rect,line width=0.800pt]
  \end{scope}
  \begin{scope}[cm={{1.01,0.0,0.0,1.01,(61.61,169.68)}},draw=black,line join=bevel,line cap=rect,line width=0.800pt]
  \end{scope}
  \begin{scope}[cm={{1.01,0.0,0.0,1.01,(61.61,169.68)}},draw=black,line join=bevel,line cap=rect,line width=0.800pt]
  \end{scope}
  \begin{scope}[cm={{1.01,0.0,0.0,1.01,(61.61,169.68)}},draw=black,line join=bevel,line cap=rect,line width=0.800pt]
  \end{scope}
  \begin{scope}[cm={{1.01,0.0,0.0,1.01,(61.61,169.68)}},draw=black,line join=bevel,line cap=rect,line width=0.800pt]
  \end{scope}
  \begin{scope}[scale=1.010,draw=black,line join=bevel,line cap=rect,line width=0.800pt]
  \end{scope}
  \begin{scope}[cm={{1.01,0.0,0.0,1.01,(6.3158,0.0)}},draw=ca0a0a4,dash pattern=on 0.40pt off 0.80pt,line join=round,line cap=round,line width=0.400pt]
    \path[draw] (95.5000,152.5000) -- (95.5000,128.5000);



  \end{scope}
  \begin{scope}[cm={{1.01,0.0,0.0,1.01,(6.3158,0.0)}},draw=black,line join=round,line cap=round,line width=0.480pt]
    \path[draw] (95.5000,152.5000) -- (95.5000,147.5000);



    \path[draw] (95.5000,128.5000) -- (95.5000,132.5000);



  \end{scope}
  \begin{scope}[scale=1.010,draw=black,line join=bevel,line cap=rect,line width=0.800pt]
  \end{scope}
  \begin{scope}[cm={{1.01,0.0,0.0,1.01,(96.96,169.68)}},draw=black,line join=bevel,line cap=rect,line width=0.800pt]
  \end{scope}
  \begin{scope}[cm={{1.01,0.0,0.0,1.01,(96.96,169.68)}},draw=black,line join=bevel,line cap=rect,line width=0.800pt]
  \end{scope}
  \begin{scope}[cm={{1.01,0.0,0.0,1.01,(96.96,169.68)}},draw=black,line join=bevel,line cap=rect,line width=0.800pt]
  \end{scope}
  \begin{scope}[cm={{1.01,0.0,0.0,1.01,(96.96,169.68)}},draw=black,line join=bevel,line cap=rect,line width=0.800pt]
  \end{scope}
  \begin{scope}[cm={{1.01,0.0,0.0,1.01,(96.96,169.68)}},draw=black,line join=bevel,line cap=rect,line width=0.800pt]
  \end{scope}
  \begin{scope}[cm={{1.01,0.0,0.0,1.01,(96.96,169.68)}},draw=black,line join=bevel,line cap=rect,line width=0.800pt]
  \end{scope}
  \begin{scope}[scale=1.010,draw=black,line join=bevel,line cap=rect,line width=0.800pt]
  \end{scope}
  \begin{scope}[cm={{1.01,0.0,0.0,1.01,(6.3158,0.0)}},draw=ca0a0a4,dash pattern=on 0.40pt off 0.80pt,line join=round,line cap=round,line width=0.400pt]
    \path[draw] (130.5000,140.5000) -- (130.5000,128.5000);



  \end{scope}
  \begin{scope}[cm={{1.01,0.0,0.0,1.01,(6.3158,0.0)}},draw=black,line join=round,line cap=round,line width=0.480pt]
    \path[draw] (130.5000,152.5000) -- (130.5000,147.5000);



    \path[draw] (130.5000,128.5000) -- (130.5000,132.5000);



  \end{scope}
  \begin{scope}[scale=1.010,draw=black,line join=bevel,line cap=rect,line width=0.800pt]
  \end{scope}
  \begin{scope}[cm={{1.01,0.0,0.0,1.01,(132.31,169.68)}},draw=black,line join=bevel,line cap=rect,line width=0.800pt]
  \end{scope}
  \begin{scope}[cm={{1.01,0.0,0.0,1.01,(132.31,169.68)}},draw=black,line join=bevel,line cap=rect,line width=0.800pt]
  \end{scope}
  \begin{scope}[cm={{1.01,0.0,0.0,1.01,(132.31,169.68)}},draw=black,line join=bevel,line cap=rect,line width=0.800pt]
  \end{scope}
  \begin{scope}[cm={{1.01,0.0,0.0,1.01,(132.31,169.68)}},draw=black,line join=bevel,line cap=rect,line width=0.800pt]
  \end{scope}
  \begin{scope}[cm={{1.01,0.0,0.0,1.01,(132.31,169.68)}},draw=black,line join=bevel,line cap=rect,line width=0.800pt]
  \end{scope}
  \begin{scope}[cm={{1.01,0.0,0.0,1.01,(132.31,169.68)}},draw=black,line join=bevel,line cap=rect,line width=0.800pt]
  \end{scope}
  \begin{scope}[scale=1.010,draw=black,line join=bevel,line cap=rect,line width=0.800pt]
  \end{scope}
  \begin{scope}[cm={{1.01,0.0,0.0,1.01,(6.3158,0.0)}},draw=black,line join=round,line cap=round,line width=0.480pt]
    \path[draw] (25.5000,128.5000) -- (25.5000,152.5000) -- (142.5000,152.5000) -- (142.5000,128.5000) -- (25.5000,128.5000);



  \end{scope}
  \begin{scope}[scale=1.010,draw=black,line join=bevel,line cap=rect,line width=0.800pt]
  \end{scope}
  \begin{scope}[cm={{1.01,0.0,0.0,1.01,(110.09,153.52)}},draw=black,line join=bevel,line cap=rect,line width=0.800pt]
  \end{scope}
  \begin{scope}[cm={{1.01,0.0,0.0,1.01,(110.09,153.52)}},draw=black,line join=bevel,line cap=rect,line width=0.800pt]
  \end{scope}
  \begin{scope}[cm={{1.01,0.0,0.0,1.01,(110.09,153.52)}},draw=black,line join=bevel,line cap=rect,line width=0.800pt]
  \end{scope}
  \begin{scope}[cm={{1.01,0.0,0.0,1.01,(110.09,153.52)}},draw=black,line join=bevel,line cap=rect,line width=0.800pt]
  \end{scope}
  \begin{scope}[cm={{1.01,0.0,0.0,1.01,(110.09,153.52)}},draw=black,line join=bevel,line cap=rect,line width=0.800pt]
  \end{scope}
  \begin{scope}[cm={{1.01,0.0,0.0,1.01,(116.4058,153.52)}},draw=black,line join=bevel,line cap=rect,line width=0.800pt]
    \path[fill=black] (3.8674,-1.6114) node[above right] (text1264) {\scriptsize $\alpha_3$};



  \end{scope}
  \begin{scope}[cm={{1.01,0.0,0.0,1.01,(110.09,153.52)}},draw=black,line join=bevel,line cap=rect,line width=0.800pt]
  \end{scope}
  \begin{scope}[scale=1.010,draw=black,line join=bevel,line cap=rect,line width=0.800pt]
  \end{scope}
  \begin{scope}[cm={{1.01,0.0,0.0,1.01,(6.3158,0.0)}},draw=black,line join=round,line cap=round,line width=0.480pt]
    \path[draw,even odd rule] (123.5000,148.5000) -- (132.5000,148.5000);



  \end{scope}
  \begin{scope}[scale=1.010,draw=black,line join=bevel,line cap=rect,line width=0.800pt]
  \end{scope}
  \begin{scope}[scale=1.010,draw=black,line join=bevel,line cap=rect,line width=0.800pt]
  \end{scope}
  \begin{scope}[scale=1.010,draw=black,line join=bevel,line cap=rect,line width=0.800pt]
  \end{scope}
  \begin{scope}[scale=1.010,draw=black,line join=bevel,line cap=rect,line width=0.800pt]
  \end{scope}
  \begin{scope}[cm={{1.01,0.0,0.0,1.01,(6.3158,0.0)}},draw=black,line join=round,line cap=round,line width=0.480pt]
    \path[draw] (25.8000,136.6000) -- (25.8000,136.6000) -- (26.2000,137.3000) -- (26.7000,137.7000) -- (27.1000,136.6000) -- (27.5000,137.5000) -- (27.9000,137.0000) -- (28.3000,139.3000) -- (28.8000,132.8000) -- (29.2000,140.7000) -- (29.6000,139.7000) -- (30.0000,133.1000) -- (30.5000,135.4000) -- (30.9000,140.8000) -- (31.3000,140.7000) -- (31.7000,136.2000) -- (32.2000,133.3000) -- (32.6000,134.6000) -- (33.0000,138.2000) -- (33.4000,140.7000) -- (33.8000,140.6000) -- (34.3000,138.4000) -- (34.7000,135.8000) -- (35.1000,134.3000) -- (35.5000,134.4000) -- (36.0000,136.0000) -- (36.4000,138.0000) -- (36.8000,139.6000) -- (37.2000,140.3000) -- (37.6000,140.0000) -- (38.1000,138.8000) -- (38.5000,137.3000) -- (38.9000,135.9000) -- (39.3000,135.0000) -- (39.8000,134.7000) -- (40.2000,135.0000) -- (40.6000,135.8000) -- (41.0000,136.8000) -- (41.5000,137.9000) -- (41.9000,138.8000) -- (42.3000,139.4000) -- (42.7000,139.6000) -- (43.1000,139.4000) -- (43.6000,139.0000) -- (44.0000,138.3000) -- (44.4000,137.4000) -- (44.8000,136.6000) -- (45.3000,136.0000) -- (45.7000,135.5000) -- (46.1000,135.2000) -- (46.5000,135.2000) -- (47.0000,135.5000) -- (47.4000,135.9000) -- (47.8000,136.4000) -- (48.2000,137.0000) -- (48.6000,137.6000) -- (49.1000,138.1000) -- (49.5000,138.5000) -- (49.9000,138.8000) -- (50.3000,138.9000) -- (50.8000,138.8000) -- (51.2000,138.5000) -- (51.6000,138.1000) -- (52.0000,137.6000) -- (52.4000,137.1000) -- (52.9000,136.5000) -- (53.3000,136.1000) -- (53.7000,135.7000) -- (54.1000,135.6000) -- (54.6000,135.6000) -- (55.0000,135.8000) -- (55.4000,136.1000) -- (55.8000,136.6000) -- (56.3000,137.1000) -- (56.7000,137.6000) -- (57.1000,138.0000) -- (57.5000,138.3000) -- (57.9000,138.6000) -- (58.4000,138.7000) -- (58.8000,138.6000) -- (59.2000,138.4000) -- (59.6000,138.1000) -- (60.1000,137.8000) -- (60.5000,137.3000) -- (60.9000,136.9000) -- (61.3000,136.6000) -- (61.8000,136.4000) -- (62.2000,136.3000) -- (62.6000,136.4000) -- (63.0000,136.5000) -- (63.4000,136.7000) -- (63.9000,137.0000) -- (64.3000,137.4000) -- (64.7000,137.7000) -- (65.1000,138.0000) -- (65.6000,138.2000) -- (66.0000,138.3000) -- (66.4000,138.2000) -- (66.8000,138.1000) -- (67.3000,137.9000) -- (67.7000,137.7000) -- (68.1000,137.4000) -- (68.5000,137.1000) -- (68.9000,136.8000) -- (69.4000,136.6000) -- (69.8000,136.5000) -- (70.2000,136.5000) -- (70.6000,136.6000) -- (71.1000,136.7000) -- (71.5000,136.9000) -- (71.9000,137.0000) -- (72.3000,137.2000) -- (72.7000,137.4000) -- (73.2000,137.6000) -- (73.6000,137.8000) -- (74.0000,137.9000) -- (74.4000,138.0000) -- (74.9000,138.0000) -- (75.3000,137.9000) -- (75.7000,137.8000) -- (76.1000,137.7000) -- (76.6000,137.5000) -- (77.0000,137.3000) -- (77.4000,137.2000) -- (77.8000,137.1000) -- (78.2000,136.9000) -- (78.7000,136.8000) -- (79.1000,136.8000) -- (79.5000,136.8000) -- (79.9000,136.8000) -- (80.4000,136.8000) -- (80.8000,136.9000) -- (81.2000,137.0000) -- (81.6000,137.1000) -- (82.1000,137.3000) -- (82.5000,137.4000) -- (82.9000,137.5000) -- (83.3000,137.6000) -- (83.7000,137.6000) -- (84.2000,137.6000) -- (84.6000,137.6000) -- (85.0000,137.6000) -- (85.4000,137.5000) -- (85.9000,137.4000) -- (86.3000,137.3000) -- (86.7000,137.2000) -- (87.1000,137.1000) -- (87.6000,137.0000) -- (88.0000,136.9000) -- (88.4000,136.9000) -- (88.8000,137.0000) -- (89.2000,137.0000) -- (89.7000,137.1000) -- (90.1000,137.2000) -- (90.5000,137.3000) -- (90.9000,137.4000) -- (91.4000,137.5000) -- (91.8000,137.5000) -- (92.2000,137.6000) -- (92.6000,137.6000) -- (93.0000,137.6000) -- (93.5000,137.5000) -- (93.9000,137.5000) -- (94.3000,137.4000) -- (94.7000,137.3000) -- (95.2000,137.2000) -- (95.6000,137.1000) -- (96.0000,137.1000) -- (96.4000,137.0000) -- (96.9000,137.1000) -- (97.3000,137.1000) -- (97.7000,137.1000) -- (98.1000,137.2000) -- (98.5000,137.3000) -- (99.0000,137.3000) -- (99.4000,137.4000) -- (99.8000,137.4000) -- (100.2000,137.5000) -- (100.7000,137.5000) -- (101.1000,137.6000) -- (101.5000,137.5000) -- (101.9000,137.4000) -- (102.3000,137.3000) -- (102.8000,137.3000) -- (103.2000,137.2000) -- (103.6000,137.1000) -- (104.0000,137.1000) -- (104.5000,137.0000) -- (104.9000,137.0000) -- (105.3000,137.0000) -- (105.7000,137.1000) -- (106.2000,137.1000) -- (106.6000,137.2000) -- (107.0000,137.3000) -- (107.4000,137.3000) -- (107.8000,137.4000) -- (108.3000,137.5000) -- (108.7000,137.6000) -- (109.1000,137.6000) -- (109.5000,137.7000) -- (110.0000,137.6000) -- (110.4000,137.6000) -- (110.8000,137.5000) -- (111.2000,137.4000) -- (111.7000,137.3000) -- (112.1000,137.2000) -- (112.5000,137.1000) -- (112.9000,137.0000) -- (113.3000,137.0000) -- (113.8000,136.9000) -- (114.2000,136.9000) -- (114.6000,136.9000) -- (115.0000,136.9000) -- (115.5000,137.0000) -- (115.9000,137.2000) -- (116.3000,137.3000) -- (116.7000,137.4000) -- (117.2000,137.5000) -- (117.6000,137.6000) -- (118.0000,137.7000) -- (118.4000,137.7000) -- (118.8000,137.7000) -- (119.3000,137.7000) -- (119.7000,137.6000) -- (120.1000,137.5000) -- (120.5000,137.4000) -- (121.0000,137.2000) -- (121.4000,137.1000) -- (121.8000,137.0000) -- (122.2000,136.8000) -- (122.6000,136.8000) -- (123.1000,136.8000) -- (123.5000,136.8000) -- (123.9000,136.9000) -- (124.3000,137.0000) -- (124.8000,137.2000) -- (125.2000,137.3000) -- (125.6000,137.4000) -- (126.0000,137.6000) -- (126.5000,137.7000) -- (126.9000,137.7000) -- (127.3000,137.8000) -- (127.7000,137.8000) -- (128.1000,137.8000) -- (128.6000,137.7000) -- (129.0000,137.5000) -- (129.4000,137.4000) -- (129.8000,137.3000) -- (130.3000,137.1000) -- (130.7000,137.0000) -- (131.1000,136.9000) -- (131.5000,136.9000) -- (131.9000,136.8000) -- (132.4000,136.8000) -- (132.8000,136.9000) -- (133.2000,137.0000) -- (133.6000,137.1000) -- (134.1000,137.2000) -- (134.5000,137.3000) -- (134.9000,137.4000) -- (135.3000,137.6000) -- (135.8000,137.7000) -- (136.2000,137.8000) -- (136.6000,137.8000) -- (137.0000,137.8000) -- (137.4000,137.7000) -- (137.9000,137.6000) -- (138.3000,137.5000) -- (138.7000,137.4000) -- (139.1000,137.3000) -- (139.6000,137.1000) -- (140.0000,137.0000) -- (140.4000,136.9000) -- (140.8000,136.8000) -- (141.3000,136.8000) -- (141.7000,136.8000) -- (142.1000,136.9000) -- (142.3000,136.9000);



  \end{scope}
  \begin{scope}[scale=1.010,draw=black,line join=bevel,line cap=rect,line width=0.800pt]
  \end{scope}
  \begin{scope}[scale=1.010,draw=black,line join=bevel,line cap=rect,line width=0.800pt]
  \end{scope}
  \begin{scope}[cm={{1.01,0.0,0.0,1.01,(6.3158,0.0)}},draw=black,line join=round,line cap=round,line width=0.480pt]
    \path[draw] (25.5000,128.5000) -- (25.5000,152.5000) -- (142.5000,152.5000) -- (142.5000,128.5000) -- (25.5000,128.5000);



  \end{scope}
  \begin{scope}[cm={{1.01,0.0,0.0,1.01,(6.3158,0.0)}},draw=ca0a0a4,dash pattern=on 0.40pt off 0.80pt,line join=round,line cap=round,line width=0.400pt]
    \path[draw] (25.5000,168.5000) -- (108.5000,168.5000);



    \path[draw] (137.5000,168.5000) -- (142.5000,168.5000);



  \end{scope}
  \begin{scope}[cm={{1.01,0.0,0.0,1.01,(6.3158,0.0)}},draw=black,line join=round,line cap=round,line width=0.480pt]
    \path[draw] (25.5000,168.5000) -- (28.5000,168.5000);



    \path[draw] (142.5000,168.5000) -- (139.5000,168.5000);



  \end{scope}
  \begin{scope}[scale=1.010,draw=black,line join=bevel,line cap=rect,line width=0.800pt]
  \end{scope}
  \begin{scope}[cm={{1.01,0.0,0.0,1.01,(5.05,174.73)}},draw=black,line join=bevel,line cap=rect,line width=0.800pt]
  \end{scope}
  \begin{scope}[cm={{1.01,0.0,0.0,1.01,(5.05,174.73)}},draw=black,line join=bevel,line cap=rect,line width=0.800pt]
  \end{scope}
  \begin{scope}[cm={{1.01,0.0,0.0,1.01,(5.05,174.73)}},draw=black,line join=bevel,line cap=rect,line width=0.800pt]
  \end{scope}
  \begin{scope}[cm={{1.01,0.0,0.0,1.01,(5.05,174.73)}},draw=black,line join=bevel,line cap=rect,line width=0.800pt]
  \end{scope}
  \begin{scope}[cm={{1.01,0.0,0.0,1.01,(5.05,174.73)}},draw=black,line join=bevel,line cap=rect,line width=0.800pt]
  \end{scope}
  \begin{scope}[cm={{1.01,0.0,0.0,1.01,(11.3658,174.73)}},draw=black,line join=bevel,line cap=rect,line width=0.800pt]
    \path[fill=black] (0.0000,0.0000) node[above right] (text1320) {-20};



  \end{scope}
  \begin{scope}[cm={{1.01,0.0,0.0,1.01,(5.05,174.73)}},draw=black,line join=bevel,line cap=rect,line width=0.800pt]
  \end{scope}
  \begin{scope}[scale=1.010,draw=black,line join=bevel,line cap=rect,line width=0.800pt]
  \end{scope}
  \begin{scope}[cm={{1.01,0.0,0.0,1.01,(6.3158,0.0)}},draw=ca0a0a4,dash pattern=on 0.40pt off 0.80pt,line join=round,line cap=round,line width=0.400pt]
    \path[draw] (25.5000,153.5000) -- (142.5000,153.5000);



  \end{scope}
  \begin{scope}[cm={{1.01,0.0,0.0,1.01,(6.3158,0.0)}},draw=black,line join=round,line cap=round,line width=0.480pt]
    \path[draw] (25.5000,153.5000) -- (28.5000,153.5000);



    \path[draw] (142.5000,153.5000) -- (139.5000,153.5000);



  \end{scope}
  \begin{scope}[scale=1.010,draw=black,line join=bevel,line cap=rect,line width=0.800pt]
  \end{scope}
  \begin{scope}[cm={{1.01,0.0,0.0,1.01,(9.09,159.58)}},draw=black,line join=bevel,line cap=rect,line width=0.800pt]
  \end{scope}
  \begin{scope}[cm={{1.01,0.0,0.0,1.01,(9.09,159.58)}},draw=black,line join=bevel,line cap=rect,line width=0.800pt]
  \end{scope}
  \begin{scope}[cm={{1.01,0.0,0.0,1.01,(9.09,159.58)}},draw=black,line join=bevel,line cap=rect,line width=0.800pt]
  \end{scope}
  \begin{scope}[cm={{1.01,0.0,0.0,1.01,(9.09,159.58)}},draw=black,line join=bevel,line cap=rect,line width=0.800pt]
  \end{scope}
  \begin{scope}[cm={{1.01,0.0,0.0,1.01,(9.09,159.58)}},draw=black,line join=bevel,line cap=rect,line width=0.800pt]
  \end{scope}
  \begin{scope}[cm={{1.01,0.0,0.0,1.01,(15.4058,159.58)}},draw=black,line join=bevel,line cap=rect,line width=0.800pt]
    \path[fill=black] (0.0000,0.0000) node[above right] (text1350) {20};



  \end{scope}
  \begin{scope}[cm={{1.01,0.0,0.0,1.01,(9.09,159.58)}},draw=black,line join=bevel,line cap=rect,line width=0.800pt]
  \end{scope}
  \begin{scope}[scale=1.010,draw=black,line join=bevel,line cap=rect,line width=0.800pt]
  \end{scope}
  \begin{scope}[cm={{1.01,0.0,0.0,1.01,(6.3158,0.0)}},draw=ca0a0a4,dash pattern=on 0.40pt off 0.80pt,line join=round,line cap=round,line width=0.400pt]
    \path[draw] (25.5000,176.5000) -- (25.5000,152.5000);



  \end{scope}
  \begin{scope}[cm={{1.01,0.0,0.0,1.01,(6.3158,0.0)}},draw=black,line join=round,line cap=round,line width=0.480pt]
    \path[draw] (25.5000,176.5000) -- (25.5000,171.5000);



    \path[draw] (25.5000,152.5000) -- (25.5000,156.5000);



  \end{scope}
  \begin{scope}[scale=1.010,draw=black,line join=bevel,line cap=rect,line width=0.800pt]
  \end{scope}
  \begin{scope}[cm={{1.01,0.0,0.0,1.01,(23.23,192.91)}},draw=black,line join=bevel,line cap=rect,line width=0.800pt]
  \end{scope}
  \begin{scope}[cm={{1.01,0.0,0.0,1.01,(23.23,192.91)}},draw=black,line join=bevel,line cap=rect,line width=0.800pt]
  \end{scope}
  \begin{scope}[cm={{1.01,0.0,0.0,1.01,(23.23,192.91)}},draw=black,line join=bevel,line cap=rect,line width=0.800pt]
  \end{scope}
  \begin{scope}[cm={{1.01,0.0,0.0,1.01,(23.23,192.91)}},draw=black,line join=bevel,line cap=rect,line width=0.800pt]
  \end{scope}
  \begin{scope}[cm={{1.01,0.0,0.0,1.01,(23.23,192.91)}},draw=black,line join=bevel,line cap=rect,line width=0.800pt]
  \end{scope}
  \begin{scope}[cm={{1.01,0.0,0.0,1.01,(29.5458,194.48895)}},draw=black,line join=bevel,line cap=rect,line width=0.800pt]
    \path[fill=black] (0.0000,0.0000) node[above right] (text1380) {0};



  \end{scope}
  \begin{scope}[cm={{1.01,0.0,0.0,1.01,(23.23,192.91)}},draw=black,line join=bevel,line cap=rect,line width=0.800pt]
  \end{scope}
  \begin{scope}[scale=1.010,draw=black,line join=bevel,line cap=rect,line width=0.800pt]
  \end{scope}
  \begin{scope}[cm={{1.01,0.0,0.0,1.01,(6.3158,0.0)}},draw=ca0a0a4,dash pattern=on 0.40pt off 0.80pt,line join=round,line cap=round,line width=0.400pt]
    \path[draw] (60.5000,176.5000) -- (60.5000,152.5000);



  \end{scope}
  \begin{scope}[cm={{1.01,0.0,0.0,1.01,(6.3158,0.0)}},draw=black,line join=round,line cap=round,line width=0.480pt]
    \path[draw] (60.5000,176.5000) -- (60.5000,171.5000);



    \path[draw] (60.5000,152.5000) -- (60.5000,156.5000);



  \end{scope}
  \begin{scope}[scale=1.010,draw=black,line join=bevel,line cap=rect,line width=0.800pt]
  \end{scope}
  \begin{scope}[cm={{1.01,0.0,0.0,1.01,(58.58,192.91)}},draw=black,line join=bevel,line cap=rect,line width=0.800pt]
  \end{scope}
  \begin{scope}[cm={{1.01,0.0,0.0,1.01,(58.58,192.91)}},draw=black,line join=bevel,line cap=rect,line width=0.800pt]
  \end{scope}
  \begin{scope}[cm={{1.01,0.0,0.0,1.01,(58.58,192.91)}},draw=black,line join=bevel,line cap=rect,line width=0.800pt]
  \end{scope}
  \begin{scope}[cm={{1.01,0.0,0.0,1.01,(58.58,192.91)}},draw=black,line join=bevel,line cap=rect,line width=0.800pt]
  \end{scope}
  \begin{scope}[cm={{1.01,0.0,0.0,1.01,(58.58,192.91)}},draw=black,line join=bevel,line cap=rect,line width=0.800pt]
  \end{scope}
  \begin{scope}[cm={{1.01,0.0,0.0,1.01,(64.8958,194.48895)}},draw=black,line join=bevel,line cap=rect,line width=0.800pt]
    \path[fill=black] (0.0000,0.0000) node[above right] (text1410) {1};



  \end{scope}
  \begin{scope}[cm={{1.01,0.0,0.0,1.01,(58.58,192.91)}},draw=black,line join=bevel,line cap=rect,line width=0.800pt]
  \end{scope}
  \begin{scope}[scale=1.010,draw=black,line join=bevel,line cap=rect,line width=0.800pt]
  \end{scope}
  \begin{scope}[cm={{1.01,0.0,0.0,1.01,(6.3158,0.0)}},draw=ca0a0a4,dash pattern=on 0.40pt off 0.80pt,line join=round,line cap=round,line width=0.400pt]
    \path[draw] (95.5000,176.5000) -- (95.5000,152.5000);



  \end{scope}
  \begin{scope}[cm={{1.01,0.0,0.0,1.01,(6.3158,0.0)}},draw=black,line join=round,line cap=round,line width=0.480pt]
    \path[draw] (95.5000,176.5000) -- (95.5000,171.5000);



    \path[draw] (95.5000,152.5000) -- (95.5000,156.5000);



  \end{scope}
  \begin{scope}[scale=1.010,draw=black,line join=bevel,line cap=rect,line width=0.800pt]
  \end{scope}
  \begin{scope}[cm={{1.01,0.0,0.0,1.01,(93.93,192.91)}},draw=black,line join=bevel,line cap=rect,line width=0.800pt]
  \end{scope}
  \begin{scope}[cm={{1.01,0.0,0.0,1.01,(93.93,192.91)}},draw=black,line join=bevel,line cap=rect,line width=0.800pt]
  \end{scope}
  \begin{scope}[cm={{1.01,0.0,0.0,1.01,(93.93,192.91)}},draw=black,line join=bevel,line cap=rect,line width=0.800pt]
  \end{scope}
  \begin{scope}[cm={{1.01,0.0,0.0,1.01,(93.93,192.91)}},draw=black,line join=bevel,line cap=rect,line width=0.800pt]
  \end{scope}
  \begin{scope}[cm={{1.01,0.0,0.0,1.01,(93.93,192.91)}},draw=black,line join=bevel,line cap=rect,line width=0.800pt]
  \end{scope}
  \begin{scope}[cm={{1.01,0.0,0.0,1.01,(100.2458,194.48895)}},draw=black,line join=bevel,line cap=rect,line width=0.800pt]
    \path[fill=black] (0.0000,0.0000) node[above right] (text1440) {2};



  \end{scope}
  \begin{scope}[cm={{1.01,0.0,0.0,1.01,(93.93,192.91)}},draw=black,line join=bevel,line cap=rect,line width=0.800pt]
  \end{scope}
  \begin{scope}[scale=1.010,draw=black,line join=bevel,line cap=rect,line width=0.800pt]
  \end{scope}
  \begin{scope}[cm={{1.01,0.0,0.0,1.01,(6.3158,0.0)}},draw=ca0a0a4,dash pattern=on 0.40pt off 0.80pt,line join=round,line cap=round,line width=0.400pt]
    \path[draw] (130.5000,164.5000) -- (130.5000,152.5000);



  \end{scope}
  \begin{scope}[cm={{1.01,0.0,0.0,1.01,(6.3158,0.0)}},draw=black,line join=round,line cap=round,line width=0.480pt]
    \path[draw] (130.5000,176.5000) -- (130.5000,171.5000);



    \path[draw] (130.5000,152.5000) -- (130.5000,156.5000);



  \end{scope}
  \begin{scope}[scale=1.010,draw=black,line join=bevel,line cap=rect,line width=0.800pt]
  \end{scope}
  \begin{scope}[cm={{1.01,0.0,0.0,1.01,(129.785,192.91)}},draw=black,line join=bevel,line cap=rect,line width=0.800pt]
  \end{scope}
  \begin{scope}[cm={{1.01,0.0,0.0,1.01,(129.785,192.91)}},draw=black,line join=bevel,line cap=rect,line width=0.800pt]
  \end{scope}
  \begin{scope}[cm={{1.01,0.0,0.0,1.01,(129.785,192.91)}},draw=black,line join=bevel,line cap=rect,line width=0.800pt]
  \end{scope}
  \begin{scope}[cm={{1.01,0.0,0.0,1.01,(129.785,192.91)}},draw=black,line join=bevel,line cap=rect,line width=0.800pt]
  \end{scope}
  \begin{scope}[cm={{1.01,0.0,0.0,1.01,(129.785,192.91)}},draw=black,line join=bevel,line cap=rect,line width=0.800pt]
  \end{scope}
  \begin{scope}[cm={{1.01,0.0,0.0,1.01,(136.1008,194.48895)}},draw=black,line join=bevel,line cap=rect,line width=0.800pt]
    \path[fill=black] (0.0000,0.0000) node[above right] (text1470) {3};



  \end{scope}
  \begin{scope}[cm={{1.01,0.0,0.0,1.01,(129.785,192.91)}},draw=black,line join=bevel,line cap=rect,line width=0.800pt]
  \end{scope}
  \begin{scope}[scale=1.010,draw=black,line join=bevel,line cap=rect,line width=0.800pt]
  \end{scope}
  \begin{scope}[cm={{1.01,0.0,0.0,1.01,(6.3158,0.0)}},draw=black,line join=round,line cap=round,line width=0.480pt]
    \path[draw] (25.5000,152.5000) -- (25.5000,176.5000) -- (142.5000,176.5000) -- (142.5000,152.5000) -- (25.5000,152.5000);



  \end{scope}
  \begin{scope}[scale=1.010,draw=black,line join=bevel,line cap=rect,line width=0.800pt]
  \end{scope}
  \begin{scope}[cm={{1.01,0.0,0.0,1.01,(109.08,177.76)}},draw=black,line join=bevel,line cap=rect,line width=0.800pt]
  \end{scope}
  \begin{scope}[cm={{1.01,0.0,0.0,1.01,(109.08,177.76)}},draw=black,line join=bevel,line cap=rect,line width=0.800pt]
  \end{scope}
  \begin{scope}[cm={{1.01,0.0,0.0,1.01,(109.08,177.76)}},draw=black,line join=bevel,line cap=rect,line width=0.800pt]
  \end{scope}
  \begin{scope}[cm={{1.01,0.0,0.0,1.01,(109.08,177.76)}},draw=black,line join=bevel,line cap=rect,line width=0.800pt]
  \end{scope}
  \begin{scope}[cm={{1.01,0.0,0.0,1.01,(109.08,177.76)}},draw=black,line join=bevel,line cap=rect,line width=0.800pt]
  \end{scope}
  \begin{scope}[cm={{1.01,0.0,0.0,1.01,(119.30192,176.13245)}},draw=black,line join=bevel,line cap=rect,line width=0.800pt]
    \path[fill=black] (0.0000,0.0000) node[above right] (text1494) {\scriptsize $\beta_3$};



  \end{scope}
  \begin{scope}[cm={{1.01,0.0,0.0,1.01,(109.08,177.76)}},draw=black,line join=bevel,line cap=rect,line width=0.800pt]
  \end{scope}
  \begin{scope}[scale=1.010,draw=black,line join=bevel,line cap=rect,line width=0.800pt]
  \end{scope}
  \begin{scope}[cm={{1.01,0.0,0.0,1.01,(6.3158,0.0)}},draw=black,line join=round,line cap=round,line width=0.480pt]
    \path[draw,even odd rule] (123.5000,172.5000) -- (132.5000,172.5000);



  \end{scope}
  \begin{scope}[scale=1.010,draw=black,line join=bevel,line cap=rect,line width=0.800pt]
  \end{scope}
  \begin{scope}[scale=1.010,draw=black,line join=bevel,line cap=rect,line width=0.800pt]
  \end{scope}
  \begin{scope}[scale=1.010,draw=black,line join=bevel,line cap=rect,line width=0.800pt]
  \end{scope}
  \begin{scope}[scale=1.010,draw=black,line join=bevel,line cap=rect,line width=0.800pt]
  \end{scope}
  \begin{scope}[cm={{1.01,0.0,0.0,1.01,(6.3158,0.0)}},draw=black,line join=round,line cap=round,line width=0.480pt]
    \path[draw] (25.8000,162.1000) -- (25.8000,162.1000) -- (26.2000,161.2000) -- (26.7000,161.1000) -- (27.1000,162.2000) -- (27.5000,159.4000) -- (27.9000,164.7000) -- (28.3000,157.2000) -- (28.8000,163.3000) -- (29.2000,164.4000) -- (29.6000,156.6000) -- (30.0000,159.9000) -- (30.5000,165.6000) -- (30.9000,163.5000) -- (31.3000,158.2000) -- (31.7000,157.3000) -- (32.2000,161.0000) -- (32.6000,164.7000) -- (33.0000,165.0000) -- (33.4000,162.4000) -- (33.8000,159.2000) -- (34.3000,157.7000) -- (34.7000,158.4000) -- (35.1000,160.6000) -- (35.5000,162.9000) -- (36.0000,164.3000) -- (36.4000,164.2000) -- (36.8000,163.0000) -- (37.2000,161.2000) -- (37.6000,159.5000) -- (38.1000,158.5000) -- (38.5000,158.3000) -- (38.9000,159.0000) -- (39.3000,160.2000) -- (39.8000,161.6000) -- (40.2000,162.8000) -- (40.6000,163.6000) -- (41.0000,163.9000) -- (41.5000,163.6000) -- (41.9000,163.0000) -- (42.3000,162.1000) -- (42.7000,161.1000) -- (43.1000,160.2000) -- (43.6000,159.5000) -- (44.0000,159.1000) -- (44.4000,159.0000) -- (44.8000,159.2000) -- (45.3000,159.7000) -- (45.7000,160.4000) -- (46.1000,161.1000) -- (46.5000,161.8000) -- (47.0000,162.5000) -- (47.4000,163.0000) -- (47.8000,163.2000) -- (48.2000,163.3000) -- (48.6000,163.1000) -- (49.1000,162.8000) -- (49.5000,162.4000) -- (49.9000,161.8000) -- (50.3000,161.2000) -- (50.8000,160.7000) -- (51.2000,160.2000) -- (51.6000,159.9000) -- (52.0000,159.7000) -- (52.4000,159.7000) -- (52.9000,159.9000) -- (53.3000,160.3000) -- (53.7000,160.8000) -- (54.1000,161.3000) -- (54.6000,161.9000) -- (55.0000,162.4000) -- (55.4000,162.7000) -- (55.8000,163.0000) -- (56.3000,163.0000) -- (56.7000,162.8000) -- (57.1000,162.6000) -- (57.5000,162.2000) -- (57.9000,161.7000) -- (58.4000,161.2000) -- (58.8000,160.8000) -- (59.2000,160.4000) -- (59.6000,160.1000) -- (60.1000,160.0000) -- (60.5000,160.0000) -- (60.9000,160.1000) -- (61.3000,160.4000) -- (61.8000,160.7000) -- (62.2000,161.1000) -- (62.6000,161.5000) -- (63.0000,161.8000) -- (63.4000,162.1000) -- (63.9000,162.2000) -- (64.3000,162.2000) -- (64.7000,162.1000) -- (65.1000,161.9000) -- (65.6000,161.6000) -- (66.0000,161.3000) -- (66.4000,161.0000) -- (66.8000,160.7000) -- (67.3000,160.5000) -- (67.7000,160.4000) -- (68.1000,160.3000) -- (68.5000,160.4000) -- (68.9000,160.6000) -- (69.4000,160.8000) -- (69.8000,161.1000) -- (70.2000,161.4000) -- (70.6000,161.6000) -- (71.1000,161.8000) -- (71.5000,162.0000) -- (71.9000,162.0000) -- (72.3000,162.0000) -- (72.7000,162.0000) -- (73.2000,161.9000) -- (73.6000,161.8000) -- (74.0000,161.6000) -- (74.4000,161.5000) -- (74.9000,161.2000) -- (75.3000,161.0000) -- (75.7000,160.8000) -- (76.1000,160.7000) -- (76.6000,160.7000) -- (77.0000,160.7000) -- (77.4000,160.7000) -- (77.8000,160.8000) -- (78.2000,160.9000) -- (78.7000,161.0000) -- (79.1000,161.2000) -- (79.5000,161.3000) -- (79.9000,161.5000) -- (80.4000,161.6000) -- (80.8000,161.7000) -- (81.2000,161.8000) -- (81.6000,161.8000) -- (82.1000,161.8000) -- (82.5000,161.8000) -- (82.9000,161.7000) -- (83.3000,161.6000) -- (83.7000,161.5000) -- (84.2000,161.3000) -- (84.6000,161.2000) -- (85.0000,161.1000) -- (85.4000,161.0000) -- (85.9000,161.0000) -- (86.3000,161.0000) -- (86.7000,161.0000) -- (87.1000,161.0000) -- (87.6000,161.1000) -- (88.0000,161.1000) -- (88.4000,161.3000) -- (88.8000,161.4000) -- (89.2000,161.5000) -- (89.7000,161.6000) -- (90.1000,161.6000) -- (90.5000,161.6000) -- (90.9000,161.6000) -- (91.4000,161.6000) -- (91.8000,161.5000) -- (92.2000,161.4000) -- (92.6000,161.3000) -- (93.0000,161.2000) -- (93.5000,161.1000) -- (93.9000,161.1000) -- (94.3000,161.0000) -- (94.7000,161.0000) -- (95.2000,161.0000) -- (95.6000,161.0000) -- (96.0000,161.1000) -- (96.4000,161.2000) -- (96.9000,161.3000) -- (97.3000,161.4000) -- (97.7000,161.5000) -- (98.1000,161.5000) -- (98.5000,161.5000) -- (99.0000,161.5000) -- (99.4000,161.5000) -- (99.8000,161.4000) -- (100.2000,161.4000) -- (100.7000,161.3000) -- (101.1000,161.3000) -- (101.5000,161.2000) -- (101.9000,161.1000) -- (102.3000,161.1000) -- (102.8000,161.0000) -- (103.2000,161.1000) -- (103.6000,161.1000) -- (104.0000,161.2000) -- (104.5000,161.2000) -- (104.9000,161.3000) -- (105.3000,161.4000) -- (105.7000,161.4000) -- (106.2000,161.5000) -- (106.6000,161.5000) -- (107.0000,161.6000) -- (107.4000,161.6000) -- (107.8000,161.5000) -- (108.3000,161.5000) -- (108.7000,161.4000) -- (109.1000,161.4000) -- (109.5000,161.3000) -- (110.0000,161.2000) -- (110.4000,161.1000) -- (110.8000,161.0000) -- (111.2000,161.0000) -- (111.7000,160.9000) -- (112.1000,160.9000) -- (112.5000,161.0000) -- (112.9000,161.0000) -- (113.3000,161.1000) -- (113.8000,161.2000) -- (114.2000,161.3000) -- (114.6000,161.4000) -- (115.0000,161.5000) -- (115.5000,161.6000) -- (115.9000,161.7000) -- (116.3000,161.7000) -- (116.7000,161.7000) -- (117.2000,161.6000) -- (117.6000,161.6000) -- (118.0000,161.5000) -- (118.4000,161.3000) -- (118.8000,161.2000) -- (119.3000,161.1000) -- (119.7000,161.0000) -- (120.1000,160.9000) -- (120.5000,160.8000) -- (121.0000,160.8000) -- (121.4000,160.9000) -- (121.8000,160.9000) -- (122.2000,161.0000) -- (122.6000,161.2000) -- (123.1000,161.3000) -- (123.5000,161.5000) -- (123.9000,161.6000) -- (124.3000,161.7000) -- (124.8000,161.8000) -- (125.2000,161.8000) -- (125.6000,161.8000) -- (126.0000,161.7000) -- (126.5000,161.6000) -- (126.9000,161.5000) -- (127.3000,161.4000) -- (127.7000,161.3000) -- (128.1000,161.1000) -- (128.6000,160.9000) -- (129.0000,160.8000) -- (129.4000,160.8000) -- (129.8000,160.8000) -- (130.3000,160.8000) -- (130.7000,160.9000) -- (131.1000,161.0000) -- (131.5000,161.1000) -- (131.9000,161.3000) -- (132.4000,161.4000) -- (132.8000,161.5000) -- (133.2000,161.6000) -- (133.6000,161.7000) -- (134.1000,161.7000) -- (134.5000,161.8000) -- (134.9000,161.7000) -- (135.3000,161.7000) -- (135.8000,161.6000) -- (136.2000,161.5000) -- (136.6000,161.3000) -- (137.0000,161.2000) -- (137.4000,161.0000) -- (137.9000,160.9000) -- (138.3000,160.9000) -- (138.7000,160.8000) -- (139.1000,160.8000) -- (139.6000,160.9000) -- (140.0000,160.9000) -- (140.4000,161.0000) -- (140.8000,161.1000) -- (141.3000,161.3000) -- (141.7000,161.4000) -- (142.1000,161.6000) -- (142.3000,161.6000);



  \end{scope}
  \begin{scope}[scale=1.010,draw=black,line join=bevel,line cap=rect,line width=0.800pt]
  \end{scope}
  \begin{scope}[draw=black,line join=bevel,line cap=rect,line width=0.800pt]
  \end{scope}
\end{scope}

\end{tikzpicture}


%  \caption[Energy estimation and evolution of the state]{Energy estimation for the first 6 seconds on the left side, the evolution of the state $\mathbf{q}$ on the right.}
%  \label{fig:model}
%\end{figure}

%We implemented a realistic simulator to simulate the empirical data of a given plan and atmospheric conditions. We show the effects of different conditions in Fig.~\ref{fig:trajs}. 

Fig{\color{black}s}.~{\color{black}\hyperref[fig:stat]{6a}--\hyperref[fig:dyn]{7a}} 
illustrate the same plan $\Gamma$ under different conditions. Flights \hyperref[fig:trajs-I-static]{I}--\hyperref[fig:trajs-dyn-i]{i} have a constant wind speed of five meters per second, a wind direction of zero degrees, and initial parameters $c_{i,1},c_{i,2}$ values of zero and ten (i.e., full $r_2$ and detection). Flights \hyperref[fig:trajs-II-static]{II}--\hyperref[fig:trajs-dyn-ii]{ii} (see added gray background for clarity) are the same but a wind direction of 90 degrees and the initial parameters values of -1000 and two (i.e., minimum $r_2$ and detection). 
{\color{black} The initial values of path and computation parameters are chosen to represent the highest and lowest configurations in the search space in \hyperref[fig:trajs-I-static]{I}--\hyperref[fig:trajs-dyn-i]{i} and \hyperref[fig:trajs-II-static]{II}--\hyperref[fig:trajs-dyn-ii]{ii} respectively, modeling the behavior of the best- and worst-case scenarios.
Different search strategies are possible by, e.g., running an ideal instance of planning-scheduling prior to the flight.}

Fig{\color{black}s}.~%\ref{fig:ener}
{\color{black}\hyperref[fig:stat]{6b}--\hyperref[fig:dyn]{7c}} %~\hyperref[fig:ener:static-I]{6.I}--\hyperref[fig:ener:static-II]{6.II} 
illustrates first the power ($\Upsilon$ on Line~\ref{alg:klm1} in Algorithm~\ref{alg}), and then the energy model ($y$ on Line~\ref{alg:evol}).
{\color{black} Fig.~\hyperref[fig:stat]{6b}} %further 
details then the energy model's estimate (%see detail view for \hyperref[fig:ener:static-I]{I}--\hyperref[fig:ener:static-II]{II}
{\color{black}${\hat{y}}$}) on an initial slice%of the model (${\hat{y}}$)
, power ($\Upsilon$), and period ($T$).
{\color{black} Fig.~\hyperref[fig:stat]{6c}} illustrates the evolutions of the state $\mathbf{q}$ in time {\color{black}for \hyperref[fig:ener:static-I]{I}}, concluding that approximately two periods suffice for a consistent %state 
estimate. 

Flight~\hyperref[fig:ener-dyn-i]{i} simulates a battery ({\color{black}see} green line {\color{black}in Fig.~\hyperref[fig:dyn]{7c}}, the battery behavior $b_0$) drop at approximately one minute and a half and four minutes and a half. Planner-scheduler optimizes the path in the proximity of the drops to ensure that the flight is completed, whereas it maximizes the parameter $c_{i,2}$ {\color{black}(see Fig.~\hyperref[fig:dyn]{7b})} when the battery is discharging, respecting the output constraint% (Eq.~(\ref{eq:output-const}))
. Flight~\hyperref[fig:ener-dyn-ii]{ii} simulates the opposite scenario: the lowest configuration of parameters and no battery defects. The path parameter increases as soon as the algorithm has estimated enough data (two periods $T$) 
and the computation parameter decreases matching the battery discharge rate. 

{\color{black}
The performance metric is $\Sigma_{t\in\mathcal{T}}{\small (w_1c_{i,1}(t)+w_2c_{i,2}(t))}/$ $\small {\small |\mathcal{T}|\text{SoC}(t_f)}$. If the initial battery SoC %at $t_0$ 
is seventy percent and both the parameters are weighted equally, i.e., ${\small w_1=w_2=}$ one half, \hyperref[fig:trajs-I-static]{I} would not be able to complete the flight, and \hyperref[fig:trajs-II-static]{II} has a performance metric of zero (i.e., the lowest configuration of parameters throughout the flight).
Nonetheless, performance metrics of \hyperref[fig:trajs-dyn-i]{i} and \hyperref[fig:trajs-dyn-ii]{ii} are 13.05 and 2.24, whereas the average detection and coverage quality is approx. 45 and 35 percent for \hyperref[fig:trajs-dyn-i]{i}, and 62 and 87 percent for \hyperref[fig:trajs-dyn-ii]{ii}.}
For both cases, scaling factors are derived empirically %, 
{\color{black}similarly to $\delta_i$ set to two hundred fifty}, 
the horizon $N$ is set to six seconds {\color{black} as in} relevant literature~\cite{gavilan2015iterative,%kang2009linear,
stastny2018nonlinear%,chao2011collision
}, order $r$ is three% (see Fig.~\ref{fig:il-abs})
, and the matrices $Q,R,Q_f$ are chosen such that the cost is merely 
squared control. {\color{black} $h$ is set to one-hundredth of a second and to one second for $\mathcal{K}$ and $\mathcal{T}$ respectively to allow sufficient precision and re-planning online.}
%\subsection{Algorithm evaluation}
%\subsubsection*{Periodic energy model}
%The figure
%{\color{black} Fig.~\hyperref[fig:stat]{6c}} %further 
%details the energy model's estimate (see detail view for \hyperref[fig:ener:static-I]{I}--\hyperref[fig:ener:static-II]{II}) on an initial slice of the model (${\hat{y}}$), power ($\Upsilon$), and period ($T$). % We motivate this choice again with Fig.~\ref{fig:il-abs}, where the power spectrum subfigure shows that 3 frequencies are adequate. 
%The {\color{black} right} detail of \hyperref[fig:ener:static-I]{I} illustrates the evolutions of the state $\mathbf{q}$ in time, concluding that approximately two periods suffice for a consistent %state 
%estimate. %With non-periodic signals, we observed that the estimator estimates primarily the first state $\alpha_0$ and it neglects the others. It hence approximates the non-periodicity with a linear model.

{\color{black}
Additional results are reported~\cite{seewaldphdthesis} utilizing simulation capabilities of the Paparazzi flight controller. Data are split into two sets of four flights each, one similar to \hyperref[fig:trajs-dyn-i]{i} and the other to \hyperref[fig:trajs-dyn-ii]{ii}, i.e., initial parameters are at boundary configurations.
These results have an average performance metric of 1.81 and 1.24 for flights similar to \hyperref[fig:trajs-dyn-i]{i} and \hyperref[fig:trajs-dyn-i]{ii} respectively.}

Output MPC on Line~\ref{alg:mpc} relies on a software framework for nonlinear optimization called CasADi~\cite{andersson2012casadi%,andersson2012bcasadi,andersson2019casadi
}, and the popular NLP solver IPOPT~\cite{wachter2006implementation}; both are open-source.

%\subsubsection*{Implementation of the planning strategy}

%The practical implementation is based on observations of different variations of paths and computations. A variation of path alters the overall flying time, which we reflect in the factors $\nu_{1,1},\tau_{1,1}$ from Eq.~(\ref{eq:scale-traj}). We compare the remaining flight time with the time needed to completely deplete the battery from Eq.~(\ref{eq:bat}). We reduce or increase the parameter to optimize the battery time. The path parameter $c_{1,1}$ is equal for all the stages and it changes the radius of the first circle in the current period and therefore shifts the other paths accordingly (the change is illustrated in Fig.~\ref{fig:tee1}). It results in a shorter or longer distance between the survey lines and in an increment or reduction of the flying time respectively. The path constraint set is set to $\underline{c}_{1,1}=$ -1000 and $\overline{c}_{1,1}=$ 0 equal for all the stages.

%A variation of computations affects directly the power. We thus select the highest computation which satisfies the constraints from Definition~\ref{def:const} in the line~\ref{alg:mpc} of the algorithm. We observed a low effect on the power of the communication ROS node. Nevertheless, the detection node varies between 5 and 10 watts for the lowest and highest fps. We implemented fps rate parameter $c_{1,2}$ with factors  $\nu_{1,2},\tau_{1,2}$ mapping $c_{1,2}$ to the data from \powprof{}. The computation constraint set is set to $\underline{c}_{1,2}=$ 2 and $\overline{c}_{1,2}=$ 10 equal for all the stages.

%\subsubsection*{Dynamic adaptation of the path and computation parameters}

%Subfigures \hyperref[fig:trajs-dyn-i]{5.i}--\hyperref[fig:trajs-dyn-ii]{5.ii} in Fig.~\ref{fig:trajs} path-wise, and \hyperref[fig:ener-dyn-i]{6.i} and \hyperref[fig:ener-dyn-ii]{6.ii} of Fig.~\ref{fig:ener} energy-wise. For the first path (Subfigure \hyperref[fig:trajs-I-static]{5.I}) the plan starts at the highest configuration of parameters. 


%%%%%%%%%%%%%%%%%%%%%%%%%%%%%%%%%%%%%%%%%%%%%%
\section{Conclusions and Future Directions}  %
\label{sec:conclusion}                       %
                                             %
This %letter 
paper provides a planning-scheduling approach for autonomous aerial robots. % powered by a limited power source, extending past literature. It proposes a novel coverage motion for variable CPP robust to aerial robots constraints such as the turning radius of fixed wings. Energy modeling in the letter exploits collected empirical data of the fixed-wing aerial robot flying static CPP and further incorporates the energy of the computing hardware via the \powprof{} tool. 
The approach compromises two algorithms: one derives a static coverage plan, %whereas 
the other re-plans-schedules the plan on a finite horizon via MPC {\color{black} and a greedy approach}. It evolves the state of the energy model while optimizing battery usage and remedying possible defects. The plan compromise multiple stages, where at each stage the aerial robot flies a path and runs the computations, allowing %further 
extensibility in terms of constructs and approaches.

{\color{black}To enable physical experiments}, we are currently extending the results to a standard flight controller. %The guidance on the coverage, coverage with variable altitude, %(for, e.g., infrastructure inspections), 
%and distributed planning-scheduling merit additional investigation, as well as 
The study of the implications of planning-scheduling on other energy-critical mobile robots {\color{black}merits additional investigation}. Here, our preliminary study led to possible savings~\cite{seewald2020beyond}, in line with relevant literature~\cite{ondruska2015scheduled,lahijanian2018resource}.
Further directions include {\color{black}the use of a purely optimization-based technique, %, e.g., MPC derives both the path and computation parameters trajectories and}
the study of different energy models, and multi-agent planning-scheduling.% Amongst others, these include aperiodic energy models, different linear combinations of the variations of parameters, and stage-dependent energy models.% In an unconventional setting with, e.g., multiple agents utilized for overall coverage, an agent-dependent model might be employed to achieve energy awareness.


%%%%%%%%%%%%%%%%%%%%%%%%%%%%%%%%%%%%%%
{\small\bibliographystyle{IEEEtran}  %
%\bibliography{../../books/phd-thesis/backmatter/references}}
\bibliography{energy-planning}}             %
%  
                                    
%\leavevmode\thispagestyle{empty}\newpage

%\appendices

% appendix, title all on one line
%\renewcommand{\thesectiondis}[2]{\Alph{section}:}

%\vspace*{-1ex}
%\section{Proof of Lem.~\ref{lem:eqv}}
%\label{app:proof-eqv}{\small

%The proof justifies the items of $A, C,$ and $\mathbf{q}_0$ in Lem.~(\ref{lem:eqv}), s.t. the coefficients of the series $a_0,\dots,b_r$ equal the coefficients $\alpha_0,\dots,\beta_r$ of the state $\mathbf{q}$.

%Firstly, it is convenient to re-write Eq.~(\ref{eq:fourier}) in complex form, with, e.g., $e^{it}=\cos{t}+i\sin{t}$. Given $t=\omega jt$, $\cos{\omega jt}=(e^{i\omega jt}+e^{-i\omega jt})/2$ and $\sin{\omega jt}=(e^{i\omega jt}-e^{-i\omega jt})/(2i)$ by substitution of $\sin{\omega jt}$ and $\cos{\omega jt}$ respectively~\cite{kuo1967automatic}. Then
%\vspace*{-.8ex}\begin{equation}\label{eq:proof-complex}\vspace*{-.8ex}
%  h(t)\hspace*{-.4ex}=\hspace*{-.4ex}a_0/T\hspace*{-.4ex}+\hspace*{-.4ex}(1/T)\sum_{j=1}^{r}{e^{i\omega jt}\hspace*{-.3ex}(a_j\hspace*{-.4ex}-\hspace*{-.4ex}ib_j)}\hspace*{-.4ex}+\hspace*{-.4ex}(1/T)\sum_{j=1}^{r}{e^{-i\omega jt}\hspace*{-.3ex}(a_j\hspace*{-.4ex}+\hspace*{-.4ex}ib_j)},\vspace*{-.8ex}
%\end{equation}
%where $i$ is this time the imaginary unit. 

%Secondly, the solution at $t$ of Eq.~(\ref{eq:state-perf}) can be expressed $\mathbf{q}=e^{At}\mathbf{q}_0$~\cite{ogata2002modern}. %Both the solution and the system in Eq.~(\ref{eq:state-perf}) are well established expressions derived using standard textbooks~\cite{kuo1967automatic, ogata2002modern}.
%Here, a method to solve the matrix exponential $e^{At}$ is the eigenvectors matrix decomposition method~\cite{moler2003nineteen} by exploiting the similarity transformation $A=VDV^{-1}$. The power series definition of $e^{At}$ implies $e^{At}=Ve^{Dt}V^{-1}$~\cite{moler2003nineteen}. 

%Within this expression, we consider the non-singular matrix $V$, whose columns are eigenvectors of $A$, $V:=[\begin{matrix}v_0 & v_1^0 & v_1^1 & \dots\end{matrix}$ $\begin{matrix}v_r^0 & v_r^1\end{matrix}]$, and the diagonal matrix of eigenvalues, $D=\mathrm{diag}(\lambda_0,\lambda_1^0,$ $\lambda_1^1,\dots,\lambda_r^0,\lambda_r^1)$. $V$ is built s.t. $\lambda_0$ is the eigenvalue associated with the first item of $A$. $\lambda_j^0,\lambda_j^1$ are the two eigenvalues associated with the block $A_j$. $Av_j=\lambda_jv_j\,,\forall j\in[m]_{>0}$, and so $AV=VD$.

%The approach in terms of Eq.~(\ref{eq:state-perf}) %, under the assumptions made in the lemma (the control is a zero vector);
%is $\dot{\mathbf{q}}=A\mathbf{q}$.
%The linear combination of the initial guess in Lem.~\ref{lem:new} and the generic solution of Eq.~(\ref{eq:state-perf})
%\vspace*{-1.6ex}\begin{subequations}\begin{align}
%  F\mathbf{q}(0)&=\gamma_0 v_0+\sum_{k=0}^{1}{\sum_{j=1}^{r}{\gamma_j v_j^k}},\\\vspace*{-.8ex}
%  F\mathbf{q}(t)&=\gamma_0 e^{\lambda_0 t} v_0+\sum_{k=0}^{1}{\sum_{j=1}^{r}{\gamma_j e^{\lambda_j t} v_j^k}}\label{eq:proof-comb},\vspace*{-.8ex}
%\end{align}\end{subequations}
%where $F:=\begin{bmatrix}1 & \cdots & 1\end{bmatrix}$ is simply a properly sized vector of ones. 

%Eq.~(\ref{eq:proof-comb}) is the linear combination of all the coefficients of the state at time $t$. By dividing the expression with the period
%\vspace*{-1.6ex}\begin{equation}\label{eq:proof-output}
%  F\mathbf{q}(t)/T\hspace*{-.4ex}=\hspace*{-.4ex}\gamma_0 e^{\lambda_0t}v_0/T+(1/T)\sum_{j=1}^r{\gamma_j e^{\lambda_j^0t}v_j^0}+(1/T)\sum_{j=1}^r{\gamma_j e^{\lambda_j^1t}v_j^1}.
%\end{equation}

%Finally, we prove that the eigenvalues $\lambda_0,\lambda_1^0,\lambda_1^1,\dots$ and eigenvectors in $v_0,v_1^0,v_1^1,\dots$ are s.t. Eq.~(\ref{eq:proof-output}) is equivalent to Eq.~(\ref{eq:proof-complex}).

%The matrix $A$ is a block diagonal matrix: its determinant is the multiplication of the determinants of its blocks $\det{(A)}=\det{(0)}\times\det{(A_1)}\times\cdots\times\det{(A_r)}$.

%The first terms of the Eq.~(\ref{eq:proof-complex}) and~(\ref{eq:proof-output}) match. The eigenvalue from $\det(0)=0$ is $\lambda_0=0$. The corresponding eigenvector can be chosen arbitrarily $(0-\lambda_0)v_0=\begin{bmatrix} 0 & \cdots & 0 \end{bmatrix}\,\,\,\forall v_0$, e.g., $v_0=\begin{bmatrix}1 & 0 & \cdots & 0\end{bmatrix}$. We find the value $\gamma_0$ of the vector $\gamma$ so that the terms are equal, e.g., $\gamma_0=\begin{bmatrix}a_0 & 0 & \cdots & 0\end{bmatrix}$. 

%All the terms in the sum of both the Eq.~(\ref{eq:proof-complex}) and~(\ref{eq:proof-output}) match. For the first block $A_1$, the eigenvalues are found by $\det(A_1-\lambda I)=0$. The polynomial $\lambda^2+\omega^2$, gives two complex roots, the two eigenvalues $\lambda_1^0=i\omega$ and $\lambda_1^1=-i\omega$. The eigenvector associated with the eigenvalue $\lambda_1^0$ is $v_1^0=\begin{bmatrix}0 & -i&1&0&\cdots&0\end{bmatrix}'$. The eigenvector associated with the eigenvalue $\lambda_1^1$ is $v_1^1=\begin{bmatrix}0&i&1&0&\cdots&0\end{bmatrix}'$. Again, we find the values $\gamma_1$ of the vector $\gamma$ such that the equivalences 
%\begin{equation}\begin{cases}
%  e^{i\omega t}(a_1-ib_1)&=\gamma_1 e^{i\omega t}v_1^0\\
%  e^{-i\omega t}(a_1+ib_1)&=\gamma_1 e^{i\omega t}v_1^1
%\end{cases},\end{equation}
%hold, e.g., $\gamma_1=\begin{bmatrix}b_1&a_1\end{bmatrix}$. 

%The proof for the remaining $r-1$ blocks is equivalent.

%The initial guess is constructed s.t. the sum of the coefficients is the same in both Eq.~(\ref{eq:proof-output}) and~(\ref{eq:proof-complex}). In the output matrix, the frequency $1/T$ accounts for the period% in Eq.~(\ref{eq:proof-complex}) and~(\ref{eq:proof-output}) and~(\ref{eq:fourier})
%. At time instant zero, the coefficients $b_j$ are not present, and the coefficients $a_j$ are doubled for each $j=[r]_{>0}$ (thus we multiply by one-half the corresponding coefficients in $\mathbf{q}_0$). To match the outputs $h(t)=y(t)$, or equivalently $F\mathbf{q}(t)/T=C\mathbf{q}(t)$, $C=(1/T)\begin{bmatrix}1 & 1 & 0 & \cdots & 1 & 0\end{bmatrix}$. Eq.~(\ref{eq:proof-output}) and~(\ref{eq:proof-complex}) are thus equal. Eq.~(\ref{eq:proof-complex}) is merely the complex form of Eq.~(\ref{eq:fourier}).

%\qed
%Periodicity is preserved with other linear combination of coefficients (for instance, $C=d\begin{bmatrix}1 & 0 & 1 & \cdots & 0 & 1\end{bmatrix}$, or $d\begin{bmatrix}1 & \cdots & 1\end{bmatrix}$ for a constant value $d\in\mathbb{R}$).
%}

%\vspace*{-1ex}
%\section{Proof of Lem.~\ref{lem:new}}
%\label{app:proof-new}{\small

%A change in computations parameters in Lem.~\ref{lem:new} results in different schedules on the computing hardware, which we assumed in Sec.~\ref{sec:intro} affects the instantaneous energy, i.e., computations are energy expensive computational tasks.

%A change in the path parameters in Lem.~\ref{lem:new} alters the flight time, and consequently, the energy. The proof quantifies this time: consider the plan $\overline{\Gamma}$ in Fig.~\ref{fig:zambo}. It is composed of four primitive paths and the highest configuration of path parameter $c_{i,1}$. We assume that the time to travel $\varphi_4(\overline{c}_{i,1})$ is $t_3\in\mathbb{R}_{>0}$, $\varphi_1,\varphi_3$, the two lines, is $2t_1\in\mathbb{R}_{>0}$, and $\varphi_2$ is $t_2\in\mathbb{R}_{>0}$. Further, assume that $v$ is covered merely by the set of primitive paths in Fig.~\ref{fig:zambo}. The coverage time is $t_{\overline{\Gamma}}=7(2t_1+t_2+t_3)+t_1$. 

%Conversely, assume the time needed to travel $\varphi_4(\underline{c}_{i,1})$ is $t_4\in\mathbb{R}_{>0}$. Then $t_{\underline{\Gamma}}=3(2t_1+t_2+t_4)+t_1$, and $t_{\overline{\Gamma}}>t_{\underline{\Gamma}}$. %even under the unrealistic assumption $t_3\approxeq t_4$.
%If $c_{i,1}(t_j),{c}_{i,1}(t_{j+1})$ are $\overline{c}_{i,1},\underline{c}_{i,1}$ in an arbitrary order, Lem.~\ref{lem:new} is satisfied. Furthermore, for any combination $c_{i,1}(t_j)\neq{c}_{i,1}(t_{j+1})$ Lem.~\ref{lem:new} is satisfied, i.e., each variation of $c_{i,1}$ alters $r_2$ in Eq.~(\ref{eq:r2}) and the length of the cover with the Zamboni-like motion.

%\qed
%}

%\vspace*{-1ex}
%\section{Proof of Lem.~\ref{lem:bat}}
%\label{app:proof-bat}{\small

%The proof justifies the expression in Lem.~\ref{lem:bat}.

 

%\qed

%}

%\leavevmode\thispagestyle{empty}\newpage
%\leavevmode\thispagestyle{empty}\newpage


\end{document}


%'\>o-O</'

